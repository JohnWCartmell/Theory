The following definition and its accompanying lemma are used when reasoning about the interpretations of theories. 
\begin{definition}
If $x$ is any object of a contextual category \catcw and if $1 \base y_1 ... \base y_n$ in \catc, for some $n \ge 1$, 
then define a \term{cascade} from $x$ to $y_n$ to consist of an n-tuple of sections $\fn$ of \catc, such that \foreachi, 
$f_i \in Sect(\fipvectorstar(\crossx{x}{y_i}{1}))$. We shall also define the empty tuple to be a cascade from $x$ to the root object $1$. 
\end{definition}
As we show in the next lemma,  in such a  cascade, for each $i$, 
$\fipvectorstar(\crossx{x}{y_i}{1}) \in Cover(a)$ and so in such a cascade there are the following objects
and morphisms:

\begin{displaymath}
\begin{array}{ c p{0.4cm} c p{0.2cm} c p {0.2cm} c } 
\Rnode{fntarget}{\fnonestar...\ftwostar\fonestar(\crossx{x}{y_n}{1})}
&&\Rnode{f3target}{\ftwostar\fonestar(\crossx{x}{y_3}{1})}
&&\Rnode{f2target}{\fonestar(\crossx{x}{y_2}{1})}  
&& \Rnode{ab1}{\crossx{x}{y_1}{\Rnode{f1target}{1}}}     \\[2cm]
      &&     &&   \ovalnode[linestyle=none]{x}{x}     &&            
\makebox[0cm]{
\ncarc[arcangle=-5,nodesepA=15pt,offsetA=-2pt,nodesepB=3pt,offsetB=-5pt]{->}{x}{f1target}
\blabel{f_1}[0.6]
\ncarc[arcangle=10,nodesepA=15pt,offsetA=1pt,nodesepB=2pt,offsetB=2pt]{->}{x}{f2target}
\alabel{f_2}[0.4]
\ncarc[arcangle=10,nodesepA=15pt,offsetA=1pt,nodesepB=2pt,offsetB=2pt]{->}{x}{f3target}
\alabel{f_3}[0.65]
\ncarc[arcangle=7,nodesepA=15pt,offsetA=1pt,nodesepB=2pt,offsetB=2pt]{->}{x}{fntarget}
\alabel{f_n}[0.75][0]
\ncdotdotdot{fntarget}{f3target}
\setlength{\sarnodesepB}{10pt}
\ncsar{fntarget}{x}
\ncsar{f3target}{x}
\ncsar{f2target}{x}
\ncsar{f1target}{x}
\sarreset
}
\end{array}
\hspace{2cm}
\begin{array}{c}
\Rnode{bn}{y_n}             \\[1.0cm]
\Rnode{b2}{y_2}             \\[0.6cm]
\Rnode{b1}{y_1}             \\[0.6cm]
\Rnode{abs}{1}              \\
\makebox[0cm]{
\ncdotdotdot{bn}{b2}
\ncsar{b2}{b1}
\ncsar{b1}{abs}
}
\end{array}
\end{displaymath}



\begin{lemma}
\llabel{cascadelemma}
If $x$ is an object of a contextual category \catc, if $1 \base y_1 ... \base y_n$ in \catcw and if $f_1,...f_n$ is a cascade from $x$ to $y_n$ in  \catcw then \foreachi, $a \base \fipvectorstar(\crossx{x}{y_i}{1})$ in \catc.
Additionally if $y$ is some object of \catcw such that $y_n \base y$ in \catcw then 
 $a \base \fnvectorstar(\crossx{x}{y}{1})$ in \catcw and
if $g$ is a section of $y$ then $\fnvectorstar(\crossx{x}{g}{1})$ is a section of $\fnvectorstar(\crossx{x}{y}{1})$.
\end{lemma}
\begin{proof}
From the definition of $\crossx{}{}{1}$ it follows that $a \base \crossx{x}{y_1}{1} \base \crossx{x}{y_2}{1} ... \base \crossx{x}{y_n}{1} \base \crossx{x}{y}{1}$ in \catc. Now since $f_1 \in Sect(\crossx{x}{y_1}{1})$ it follows from the definition of the extended $^*$ notation
that $a \base f_1^*(\crossx{x}{y_2}{1})  ... \base f_1^*(\crossx{x}{y_n}{1}) \base f_1^*(\crossx{x}{y}{1})$ in \catc.

Similarly, since $f_2 \in Sect(f_1^*(\crossx{x}{y_2}{1}))$ it follows 
that $a \base f_2^*f_1^*(\crossx{x}{y_3}{1})  ... \base f_2^*f_1^*(\crossx{x}{y_n}{1}) \base f_2^*f_1^*(\crossx{x}{y}{1})$ in \catc.

If we continue in this way we see (by induction,  if we were to be formal about it) that \foreachi, $f_i \in Sect(\fipvectorstar(\crossx{x}{y_i}{1}))$
and that $a \base \fnvectorstar(\crossx{x}{y}{1})$ in \catc. In fact we see that
we have the following objects and morphisms in \catc:

\begin{displaymath}
\begin{array}{c  c p{0.4cm} c p{0.2cm} c p {0.2cm} c  p{0.5cm} c}
&&&                                               &&                                           && \Rnode{ab}{\crossx{x}{y}{1}}    &&                \\[1.2cm]
&&&                                               &&  \Rnode{f1ab}{\fonestar(\crossx{x}{y}{1})}
%\rule[-1cm]{3pt}{1pt}
&& \Rnode{abn}{\crossx{x}{y_n}{1}} &&                \\[1.2cm]
&&&                                               &&  \Rnode{f1abn}{\fonestar(\crossx{x}{y_n}{1})}&&                              &&                \\[0.1cm]
&&&                                               &&                                           && \Rnode{ab3}{\crossx{x}{y_3}{1}} &&                \\[1.2cm]
&\Rnode{fn1axb}{\fnonestar...\ftwostar\fonestar(\crossx{x}{y}{1})}&& &&\Rnode{f1axb3}{\fonestar(\crossx{x}{y_3}{1})}  && \Rnode{ab2}{\crossx{x}{y_2}{1}}  &&           \\[1.2cm]
\Rnode{ftarget}{\fnstar...\ftwostar\fonestar(\crossx{x}{y}{1})}\ \ &\Rnode{fntarget}{\fnonestar...\ftwostar\fonestar(\crossx{x}{y_n}{1})}&&
\Rnode{f3target}{\ftwostar\fonestar(\crossx{x}{y_3}{1})} &&\Rnode{f2target}{\fonestar(\crossx{x}{y_2}{1})}  && \Rnode{ab1}{\crossx{x}{y_1}{\Rnode{f1target}{1}}}     \\[1.2cm]
&&&                                               &&                                           &&                                                       \\[-6.4cm] %%% HEE HEE HE
&&&																								&&                                           &&                         && \Rnode{y}{y}                \\[1.2cm]
&&&																								&&                                           &&                         && \Rnode{bn}{y_n}             \\[0.3cm]
&&&                                               &&                                           &&                         &&                             \\[0.3cm]
&&&																								&&                                           &&                         && \Rnode{b3}{y_3}             \\[1.2cm]
&&&																								&&                                           &&                         && \Rnode{b2}{y_2}             \\[1.2cm]
&&&																								&&                                           &&                         && \Rnode{b1}{y_1}             \\[0.3cm]
&&&		\ovalnode[linestyle=none]{x}{x}					    &&                                           &&                         &&                             \\[1.1cm]
&&&                                               &&                                           && \Rnode{abs}{1} \ \ \ \ \ \ \ \ &&                      \\           
\makebox[0cm]{
\ncarr{ab}{y}
\ncarr{abn}{bn}
\ncarr{f1ab}{ab}
\ncarr{f1abn}{abn}
\ncarr{ab3}{b3}
\ncarr{ab2}{b2}
\ncarr{ab1}{b1}
\ncarr{f1axb3}{ab3}
\ncarr{f2target}{ab2}
\ncarr{f3target}{f1axb3}
\ncarr{ftarget}{fn1axb}
\ncdotdotdot{fn1axb}{f1ab} 
\ncdotdotdot{fntarget}{f1abn}
\ncdotdotdot{fntarget}{f3target}
%
\ncarc[arcangle=-20,nodesepA=5pt,offsetA=-3pt,nodesepB=3pt,offsetB=-4pt]{->}{bn}{y}
\blabel{g}[0.6]
\ncarc[arcangle=-5,nodesepA=15pt,offsetA=-2pt,nodesepB=3pt,offsetB=-5pt]{->}{x}{f1target}
\blabel{f_1}[0.6]
\ncarc[arcangle=10,nodesepA=15pt,offsetA=1pt,nodesepB=2pt,offsetB=2pt]{->}{x}{f2target}
\alabel{f_2}[0.4]
\ncarc[arcangle=10,nodesepA=15pt,offsetA=1pt,nodesepB=2pt,offsetB=2pt]{->}{x}{f3target}
\alabel{f_3}[0.6]
\ncarc[arcangle=7,nodesepA=15pt,offsetA=1pt,nodesepB=2pt,offsetB=2pt]{->}{x}{fntarget}
\alabel{f_n}[0.75][0]
\ncarc[arcangle=7, nodesepA=15pt,offsetA=1pt,nodesepB=2pt,offsetB=2pt]{->}{x}{ftarget}
\alabel{\fnvectorstar g}[0.6]
\setlength{\sarnodesepB}{10pt}
\ncsar{fntarget}{x}
\ncsar{ftarget}{x}
\ncsar{f3target}{x}
\ncsar{f2target}{x}
\ncsar{f1target}{x}
\sarreset
\ncsar{fn1axb}{fntarget}
%left but two tower
\ncsar{f1ab}{f1abn}
\ncdotdotdot {f1abn}{f1axb3}
\ncsar{f1axb3}{f2target}
%left but one tower
\ncsar{ab}{abn}
\ncdotdotdot{abn}{ab3}
\ncsar{ab3}{ab2}
\ncsar{ab2}{ab1}
%left tower
\ncsar{y}{bn}
\ncdotdotdot{bn}{b3}
\ncsar{b3}{b2}
\ncsar{b2}{b1}
\ncsar{b1}{abs}
\nccdar{x}{abs}
}
\end{array}
\end{displaymath}
\end{proof}

\begin{lemma}
\llabel{starcrosssublemma}
If $1 \base x \base y_1 ... \base y_n$ and $1 \base z$ is  a contextual category \catcw and if $f \in Sect(y_1)$ then
\begin{enumerate}[(i)]
\item
\begin{equation*}
f^*(\crossx{y_1}{z}{1}) = \crossx{x}{z}{1}
\end{equation*}
\item and for  $i > 1$ 
\begin{equation*}
f^*(\crossx{y_i}{z}{1}) = \crossx{(f^*y_i)}{z}{1}.
\end{equation*}
\end{enumerate}
\end{lemma}

\begin{lemma}
\llabel{starcrosslemma}
If $1 \base x \base y_1 ... \base y_n$ and $1 \base z$ is  a contextual category \catcw and if 
\commentary{... there are sections $f_1,...f_n$ as shown here DIAGRAM}
\begin{align*}
f_1 &\in Sect(y_1)                \\
f_2 &\in Sect(\fonestar y_2)      \\
\vdots                            \\
f_n &\in Sect(\fnvectorstar y_n)
\end{align*}
then
\begin{equation*}
\fnvectorstar(\crossx{y_n}{z}{1}) = \crossx{x}{z}{1}.
\end{equation*}
\end{lemma}

\begin{lemma}
\llabel{cascadedpullbackscohere}
If $x$ is any object of a contextual category \catcw, if $1 \base y_1 ... \base y_n$ and $1 \base z_1 ... \base z_m$ in \catc, for some $n,m \ge 1$, 
if $\tuple{\fn}$ is a \term{cascade} from $x$ to $y_n$ and
if $\tuple{\gm}$ is a \term{cascade} from $y_n$ to $z_m$ and if  $z_m \base z$ in \catcw then
%\newcommand{\clausethreelhs}{\big(\fmvectorstar (\crossx{x}{g_n}{1})\big)^* ... \big(\fmvectorstar (\crossx{x}{g_1}{1})\big) ^* \big(\crossx{y_n}{z}{1}\big)}
%\newcommand{\clausethreerhs}{\fmvectorstar \big(  \crossx{x}{(\gnvectorstar(\crossx{y_n}{z}{1}))}{1} \big)}
\begin{equation}
\label{cascadedpullbackscohereonobjects}
\big(\fnvectorstar (\crossx{x}{g_m}{1})\big)^* ... \big(\fnvectorstar (\crossx{x}{g_1}{1})\big) ^* \big(\crossx{x}{z}{1}\big) 
= \fnvectorstar \big(  \crossx{x}{(\gmvectorstar(\crossx{y_n}{z}{1}))}{1} \big)                                    
\end{equation}
and if  $h$ is any section of $z$ then 
\begin{equation}
\label{cascadedpullbackscohereonsections}
\big(\fnvectorstar (\crossx{x}{g_m}{1})\big)^* ... \big(\fnvectorstar (\crossx{x}{g_1}{1})\big) ^* \big(\crossx{x}{h}{1}\big) 
= \fnvectorstar \big(  \crossx{x}{(\gmvectorstar(\crossx{y_n}{h}{1}))}{1} \big)                                    
\end{equation}
\end{lemma}
\begin{proof}
The first identity is proved as follows
\begin{align*}
\big(\fnvectorstar (\crossx{x}{g_m}{1})\big)^* ... \big(\fnvectorstar (\crossx{x}{g_1}{1})\big) ^* \big(\crossx{x}{z}{1}\big) \hspace{-4cm}\\
   &= \big(\fnvectorstar (\crossx{x}{g_m}{1})\big)^* ... \big(\fnvectorstar (\crossx{x}{g_1}{1})\big) ^* 
         \big( \fnvectorstar (\crossx{(\crossx{x}{y_m}{1})} {z}{1}) \big)               &&\mbox{by lemma \lref{starcrosslemma}} \\
   &= \fnvectorstar \big(  (\crossx{x}{g_m}{1}) ^* ... (\crossx{x}{g_1}{1}) ^* (\crossx{(\crossx{x}{y_m}{1})} {z}{1}) \big) 
                                                                                        &&\mbox{by repeated use of lemma \lref{stardistributesonsections} (i)} \\
   &= \fnvectorstar \big(  (\crossx{x}{g_m}{1}) ^* ... (\crossx{x}{g_1}{1}) ^* (\crossx{x} {(\crossx{y_m}{z}{1})}{1}) \big) 
                                                                                        &&\mbox{by  lemma \lref{crossassociativitylemma} (i)} \\
   &= \fnvectorstar \big(  \crossx{x}{(\gmvectorstar(\crossx{y_m}{z}{1}))}{1} \big)     &&\mbox{by repeated use of lemma \lref{crossstardistributivitylemma} (i)}
\end{align*}
\end{proof}


\highlight{The previously missing lemma:}
\begin{lemma}
\llabel{cascadeprojectionlemma}
If $x$ is any object of a contextual category \catcw, if $1 \base y_1 ... \base y_n$ in \catc, for some $n \ge 1$, 
if $\tuple{\fn}$ is a \term{cascade} from $x$ to $y_n$ then \foreachi,

\begin{equation}
\fnvectorstar(\crossx{x}{s(p_{y_n,y_i})}{1})=f_i.                                    
\end{equation}
\end{lemma}

\begin{proof}
\begin{align*}
\fnvectorstar\big(\xsynyi\big)
    &=\fnvectorstar\big(\sxynxyi\big)                       &&\mbox{by lemma \lref{missingsublemma8},}  \\
    &=\fnstar...\ftwostar\big(s(p_{\fonestar \crossx{x}{y_n}{1},\fonestar\crossx{x}{y_i}{1}})\big) 
                                                    &&\mbox{by lemma \lref{missingsublemma7},}\\
    &=\fnstar...\fistar\big(s(p_{\fitarget,\fitarget})\big)
                                                    &&\mbox{by further repeated use of lemma \lref{missingsublemma7},}   \\
    &=\fnstar...\fiistar\Big(\crossx{\big(\fivectorstar(\xyn)\big)}{f_i}{x}\Big)
                                                    &&\mbox{by lemma \lref{missingsublemma3},}   \\
    &=\fnstar...\fiiistar\Big(\crossx{\big(\fiivectorstar(\xyn)\big)}{f_i}{x}\Big)  
                                                    &&\mbox{by lemma \lref{missingsublemma2},}  \\
%   &= \crossx{\big(\fnvectorstar(\xyn)\big)}{f_i}{x} &&\mbox{by lemma \lref{missingsublemma3},}  \\
    &= \fnstar \Big( \crossx{\big(\fnpvectorstar(\xyn)\big)}{f_i}{x} \Big)
                                                    &&\mbox{by further repeated use of lemma \lref{missingsublemma2} ,}  \\
    &= f_i                                          &&\mbox{by lemma \lref{missingsublemma1}.}
\end{align*}
\iffalse

These diagrams relate to lemma \lref{cascadeprojectionlemma} which expresses
that
\begin{equation}
\fnvectorstar(\crossx{x}{s(p_{y_n,y_i})}{1})=f_i.                                    
\end{equation}

\newpage
\iffalse
%FIRST ALIGN
1.
\begin{align*}
\crossx{x}{s(p_{y_n,y_i})}{1} 
      &= s(p_{\crossx{x}{y_n}{1},\crossx{x}{y_i}{1}}) && \mbox{ by lemma \lref{missingsublemma8}.} \\
\end{align*}
2.
%SECOND ALIGN
\begin{align*}
\fonestar (\crossx{x}{s(p_{y_n,y_i})}{1})
      &= \fonestar (s(p_{\crossx{x}{y_n}{1},\crossx{x}{y_i}{1}})) && \mbox{ by lemma \lref{missingsublemma8}.} \\
      &= s(p_{\fonestar \crossx{x}{y_n}{1},\fonestar\crossx{x}{y_i}{1}}) && \mbox{ by lemma \lref{missingsublemma7}.} 
\end{align*}
3.
%THIRD ALIGN
\begin{align*}
\fipvectorstar(\crossx{x}{s(p_{y_n,y_i})}{1})
      &= s(p_{\fitarget,\fitarget}) && \mbox{ by lemma \lref{missingsublemma7}.} 
\end{align*}
4.
%FOURTH ALIGN
Now  we have that
\begin{align*}
\fistar(\fipvectorstar(\xsynyi)) 
             &= \fistar(s(p_{\fipvectorstar(\xyn),\fipvectorstar(\xyi)})) && \mbox{ as shown above}  \\
             &= \crossx{\big(\fivectorstar(\xyn)\big)}{f_i}{x}            && \mbox{ by lemma \lref{missingsublemma3}}
\end{align*}
5.
%FIFTH ALIGN
From which by missing sublemma \lref{missingsublemma2}
\begin{align*}
\fiistar (\fivectorstar(\xsynyi)) 
    &= \fiistar (\crossx{\big(\fivectorstar(\xyn)\big)}{f_i}{x}) && \mbox{ as shown above} \\
    &= \crossx{\big(\fiivectorstar(\xyn)\big)}{f_i}{x}           && \mbox{ by lemma \lref{missingsublemma2}}
\end{align*}
\fi

\begin{displaymath}
\begin{array}{p{1.5cm}  c p{0.1cm} c c p {0.2cm} c  p{0.5cm} c}
&&&                                               &                     && \Rnode{ab}{\kern-0.9cm\crossx{x}{(\ynyi)}{1}}         &&                \\[2.0cm]
&&&                                               &  \Rnode{f1ab}{\kern-1cm\fonestar(\crossx{x}{(\ynyi)}{1})}&& \Rnode{abn}{\crossx{x}{y_n}{1}} && \\[2.0cm]
&\Rnode{fipxstarget}{\fipxstarget}&&                 &  \Rnode{f1abn}{\fonestar(\crossx{x}{y_n}{1})}&&                              &&                \\[2.0cm]
&\Rnode{fipxspred}{\fipxspred}&&                 &                              && \Rnode{ab3}{\crossx{x}{y_3}{1}} &&                \\[1.2cm]
&
%\Rnode{fn1axb}{\kern-1.3cm\fipvectorstar(\crossx{x}{(\yiyi)}{1})}
&& &\Rnode{f1axb3}{\fonestar(\crossx{x}{y_3}{1})}  && \Rnode{ab2}{\crossx{x}{y_2}{1}}  &&           \\[1.2cm]
%\Rnode{ftarget}{\fivectorstar(\crossx{x}{(\yiyi)}{1})}\ \ 
&\Rnode{fitarget}{\fitarget}&&&\Rnode{f2target}{\fonestar(\crossx{x}{y_2}{1})}  && \Rnode{ab1}{\crossx{x}{y_1}{\Rnode{f1target}{1}}}     \\[1.2cm]
&&&                                &&                          &                                                        \\[-12cm] %was -6.4
&&&                                &&                          &           && \Rnode{y}{\ynyi}                          \\[2cm]
&&&                                &&                          &           && \Rnode{bn}{y_n}                           \\[0.3cm]
&&&                                &&                          &           &&                                           \\[3.9cm] %was 0.3
&&&                                &&                          &           && \Rnode{b3}{y_3}                           \\[1.2cm]
&&&                                &&                          &           && \Rnode{b2}{y_2}                           \\[1.2cm]
&&&                                &&                          &           && \Rnode{b1}{y_1}                           \\[0.9cm]  %was 0.3
&&& \ovalnode[linestyle=none]{x}{x}&&                          &           &&                                           \\[1.1cm]
&&&                                &&  \Rnode{abs}{1} \ \ \ \ \ \ \ \ &    &&                                           \\        
\makebox[0cm]{
\ncarr{ab}{y}
\ncarr{abn}{bn}
\ncarr{f1ab}{ab}
\ncarr{f1abn}{abn}
\ncarr{ab3}{b3}
\ncarr{ab2}{b2}
\ncarr{ab1}{b1}
\ncarr{f1axb3}{ab3}
\ncarr{f2target}{ab2}
%\ncarr{f3target}{f1axb3}
\ncarr{ftarget}{fn1axb}
\ncdotdotdot{fn1axb}{f1ab} 
\ncdotdotdot{fitarget}{f3target}
% 
\ncarc[arcangle=-20,nodesepA=5pt,offsetA=-3pt,nodesepB=3pt,offsetB=-3pt]{->}{bn}{y}
\blabel{s(p_{y_n,y_i})}[0.4]
%FIRST ALIGN
\ncarc[arcangle=-20,nodesepA=5pt,offsetA=-3pt,nodesepB=3pt,offsetB=-3pt]{->}{abn}{ab}
\blabel{\crossx{x}{s(p_{y_n,y_i})}{1}} [0.5]
\blabel{=s(p_{\crossx{x}{y_n}{1},\crossx{x}{y_i}{1}})} [0.3]
%END
%SECOND ALIGN
\ncarc[arcangle=20,nodesepA=5pt,offsetA=3pt,nodesepB=3pt,offsetB=3pt]{->}{f1abn}{f1ab}
\alabel{\fonestar (\crossx{x}{s(p_{y_n,y_i})}{1}) } [0.73]
\alabel{=\fonestar (s(p_{\crossx{x}{y_n}{1},\crossx{x}{y_i}{1}})) } [0.50]
\alabel{=s(p_{\fonestar \crossx{x}{y_n}{1},\fonestar\crossx{x}{y_i}{1}})}[0.27]
%END
\ncarc[arcangle=-5,nodesepA=15pt,offsetA=-2pt,nodesepB=3pt,offsetB=-5pt]{->}{x}{f1target}
\blabel{f_1}[0.6]
\ncarc[arcangle=10,nodesepA=15pt,offsetA=1pt,nodesepB=2pt,offsetB=2pt]{->}{x}{f2target}
\alabel{f_2}[0.4]
%\ncarc[arcangle=10,nodesepA=15pt,offsetA=1pt,nodesepB=2pt,offsetB=2pt]{->}{x}{f3target}
%\alabel{f_3}[0.6]
\ncarc[arcangle=7,nodesepA=15pt,offsetA=1pt,nodesepB=2pt,offsetB=2pt]{->}{x}{fitarget}
\alabel{f_i}[0.75][0]
\ncarc[arcangle=7, nodesepA=15pt,offsetA=1pt,nodesepB=2pt,offsetB=2pt]{->}{x}{ftarget}
\alabel{\fivectorstar (\crossx{x}{s(p_{y_n,y_i})}{1}) } [0.6]
\setlength{\sarnodesepB}{10pt}
\ncsar{fitarget}{x}
\ncsar{ftarget}{x}
%\ncsar{f3target}{x}
\ncsar{f2target}{x}
\ncsar{f1target}{x}
\sarreset
\ncsar{fn1axb}{fntarget}
%left tower
\ncsar{fipxstarget}{fipxspred}
%THIRD ALIGN
\ncarc[arcangle=20,nodesepA=5pt,offsetA=3pt,nodesepB=3pt,offsetB=3pt]{->}{fipxspred}{fipxstarget}  
\alabel{\fipvectorstar(\crossx{x}{s(p_{y_n,y_i})}{1})} [0.73]
\alabel{=s(p_{\fitarget,\fitarget})} [0.50]
%END
\ncdotdotdot{fipxspred}{fitarget}
%right but two tower
\ncsar{f1ab}{f1abn}
\ncdotdotdot {f1abn}{f1axb3}
\ncsar{f1axb3}{f2target}
%right but one tower
\ncsar{ab}{abn}
\ncdotdotdot{abn}{ab3}
\ncsar{ab3}{ab2}
\ncsar{ab2}{ab1}
%right tower
\ncsar{y}{bn}
\ncdotdotdot{bn}{b3}
\ncsar{b3}{b2}
\ncsar{b2}{b1}
\ncsar{b1}{abs}
\nccdar{x}{abs}
}
\end{array}
\end{displaymath}

\hrulefill

\begin{displaymath}
\begin{array}{c}
\fippvectorstar (\xynyi)=  \\
\fippvectorstar \big(\xynxyi \big)= \\
\Rnode{u1}{\crossx{\fippvectorstar(\xyn)}{\kern-0.2cm\fippvectorstar(\xyi)}{\kern-0.15cm\fippvectorstar(\xyip)}} \\[2.6cm]
\Rnode{u2}{\fippvectorstar(\xyn)}     \\[2cm]
\Rnode{s3}{\fippvectorstar(\xyi)}     \\[1.2cm]
\Rnode{s2}{\fippvectorstar(\xyip)}    \\[1.2cm]
\Rnode{s1}{x} \\
\makebox[0cm]{
\ncsar{u1}{u2}
\ncarc[arcangle=-20,nodesepA=5pt,offsetA=-3pt,nodesepB=3pt,offsetB=3pt]{->}{u2}{u1}
\blabel{\fippvectorstar (\xsynyi)=}[0.7]
\blabel{\fippvectorstar (\sxynxyi)=}[0.5]
\blabel{s(p_{\fippvectorstar(\xyn),\fippvectorstar(\xyi)})}[0.3]
\ncdotdotdot{u2}{s3}
\ncsar{s3}{s2}
\ncsar{s2}{s1}
\ncarc[arcangle=-20,nodesepA=5pt,offsetA=-3pt,nodesepB=3pt,offsetB=0pt]{->}{s1}{s2}
\blabel{f_{i-1}} [0.5]
}
\end{array}
\end{displaymath}
\hrulefill
\begin{displaymath}
\begin{array}{c}
\fipvectorstar (\xynyi)=  \\
\fipvectorstar \big( \xynxyi \big)= \\
\Rnode{u1}{\crossx{\fipvectorstar(\xyn)}{\kern-0.2cm\fipvectorstar(\xyi)}{\kern-0.2cm\fipvectorstar(\xyip)}} \\[2.6cm]
\Rnode{u2}{\fipvectorstar(\xyn)}     \\[2cm]
\Rnode{s2}{\fipvectorstar(\xyi)}     \\[1.2cm]
\Rnode{s1}{x} \\
\makebox[0cm]{
\ncsar{u1}{u2}
\ncarc[arcangle=-20,nodesepA=5pt,offsetA=-3pt,nodesepB=3pt,offsetB=3pt]{->}{u2}{u1}
\blabel{\fipvectorstar(\xsynyi)=}[0.7]
\blabel{\fipvectorstar(\sxynxyi)=}[0.5]
\blabel{s(p_{\fipvectorstar(\xyn),\fipvectorstar(\xyi)})}[0.3]
\ncdotdotdot{u2}{s2}
\ncsar{s2}{s1}
\ncarc[arcangle=-20,nodesepA=5pt,offsetA=-3pt,nodesepB=3pt,offsetB=0pt]{->}{s1}{s2}
\blabel{f_i} [0.5]
}
\end{array}
\end{displaymath}

\hrulefill
\begin{displaymath}
\begin{array}{c}
\fivectorstar (\xynyi)=  \\
\fivectorstar \big( \xynxyi \big)= \\
\Rnode{u1}{\crossx{\big(\fivectorstar(\xyn)\big)}{\big(\fivectorstar(\xyi)\big)}{x}} \\[2.6cm]
\Rnode{u2}{\fivectorstar(\xyn)}     \\[2cm]
\Rnode{s2}{\fivectorstar(\xyi)}     \\[1.2cm]
\Rnode{s1}{x} \\
\makebox[0cm]{
\ncsar{u1}{u2}
\ncarc[arcangle=-20,nodesepA=5pt,offsetA=-3pt,nodesepB=3pt,offsetB=3pt]{->}{u2}{u1}
\blabel{\fivectorstar(\xsynyi)=} [0.7]
\blabel{\fivectorstar(\sxynxyi)=}[0.5]
\blabel{\crossx{\big(\fivectorstar(\xyn)\big)}{f_i}{x}}[0.3]
\ncdotdotdot{u2}{s2}
\ncsar{s2}{s1}
\ncarc[arcangle=-20,nodesepA=5pt,offsetA=-3pt,nodesepB=3pt,offsetB=0pt]{->}{s1}{s2}
\blabel{f_{i+1}} [0.5]
}
\end{array}
\end{displaymath}
From which by missing sublemma \lref{missingsublemma2}
\begin{align*}
\fiistar (\fivectorstar(\xsynyi)) 
    &= \fiistar (\crossx{\big(\fivectorstar(\xyn)\big)}{f_i}{x}) && \mbox{ as shown above} \\
    &= \crossx{\big(\fiivectorstar(\xyn)\big)}{f_i}{x}           && \mbox{ by lemma \lref{missingsublemma2}}
\end{align*}

\hrulefill \\
By further repeated use of missing sublemma \lref{missingsublemma2}
\begin{displaymath}
\begin{array}{c}
\fnpvectorstar (\xynyi)=  \\
\fnpvectorstar \big( \xynxyi \big)= \\
\Rnode{u1}{\crossx{\big(\fnpvectorstar(\xyn)\big)}{\big(\fnpvectorstar(\xyi)\big)}{x}} \\[2.6cm]
\Rnode{u2}{\fnpvectorstar(\xyn)}     \\[1.2cm]
\Rnode{s1}{x} \\
\makebox[0cm]{
\ncsar{u1}{u2}
\ncarc[arcangle=-20,nodesepA=5pt,offsetA=-3pt,nodesepB=3pt,offsetB=3pt]{->}{u2}{u1}
\blabel{\fnpvectorstar(\xsynyi)=} [0.7]
\blabel{\fnpvectorstar(\sxynxyi)=}[0.5]
\blabel{\crossx{\big(\fnpvectorstar(\xyn)\big)}{f_i}{x}}[0.3]
\ncdotdotdot{u2}{s2}
\ncsar{u2}{s1}
\ncarc[arcangle=-20,nodesepA=5pt,offsetA=-3pt,nodesepB=3pt,offsetB=0pt]{->}{s1}{u2}
\blabel{f_{n}} [0.5]
}
\end{array}
\end{displaymath}

\newcommand{\xnz}{\crossx{x_n}{z}{w}}
\newcommand{\xng}{\crossx{x_n}{g}{w}}
\begin{displaymath}
\begin{array}{ c p{0.75cm} c p{2.25cm} c p{0.1cm} c p{1.0cm} c} 
                    &&                                                    &&                && \Rnode{xnz}{\xnz} &&                       \\ [1.2cm]
                    &&                                                    &&                && \Rnode{xn}{x_n}   &&                       \\ [2.0cm]
                    && \fnpvectorstar(\xnz)                               &&                &&                   &&                       \\ 
                    && \Rnode{fnpz}{=\crossx{(\fnpvectorstar x_n)}{z}{w}} &&                && \Rnode{x2}{x_2}   &&             \\ [2.0cm]
\Rnode{fnxnz}{\fnvectorstar (\xnz)=z} && \Rnode{fnpxnz}{\fnpvectorstar x_n} && \Rnode{f1x2}{f_1 ^* x_2} && \Rnode{x1}{x_1}  && \Rnode{z}{z} \\ [2.4cm]
                                    &&                                    && \circlenode[framesep=1cm, linestyle=none]{w}{w} \ \ \ \ \ \ \ \ \ \ &&   &&   
\makebox[0cm]{
\ncdotdotdot{fnpxnz}{f1x2}
\ncsar{fnpz}{fnpxnz}
\ncarc[arcangle=-10,nodesepA=5pt,offsetA=-4pt,nodesepB=3pt,offsetB=-3pt]{->}{fnpxnz}{fnpz}
\blabel{\fnpvectorstar(\xng)}[0.65]
\blabel{=\crossx{(\fnpvectorstar x_n)}{g}{w}}[0.4]
\ncsar{xnz}{xn}
\ncdotdotdot{xn}{x2}
\ncsar{x2}{x1}
\ncsar{fnxnz}{w}
\ncarc[arcangle=10,nodesepA=5pt,offsetA=-2pt,nodesepB=3pt,offsetB=0pt]{->}{w}{fnxnz}
\alabel{\fnvectorstar(\xng)=g} [0.5]
\ncsar{fnpxnz}{w}
\ncarc[arcangle=10,nodesepA=5pt,offsetA=-2pt,nodesepB=3pt,offsetB=2pt]{->}{w}{fnpxnz}
\alabel{f_n} [0.6]
\ncsar{f1x2}{w}
\ncarc[arcangle=10,nodesepA=5pt,offsetA=-2pt,nodesepB=3pt,offsetB=2pt]{->}{w}{f1x2}
\alabel{f_2} [0.5]
\ncsar{x1}{w}
\ncarc[arcangle=10,nodesepA=5pt,offsetA=-2pt,nodesepB=3pt,offsetB=2pt]{->}{w}{x1}
\alabel{f_1} [0.4]
\ncsar{z}{w}
\ncarc[arcangle=-10,nodesepA=5pt,offsetA=0pt,nodesepB=3pt,offsetB=-2pt]{->}{w}{z}
\blabel{g} [0.4]
}
\end{array}
\end{displaymath}

\hrulefill
\newpage
\fi
\end{proof}
