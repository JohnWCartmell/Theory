
If $x$ and $y$ are objects of a contextual category \catcw then the diagram 
$$
\begin{array} {c p{1.5cm} c}
\Rnode{xy}{\crossx{x}{y}{1}} && \Rnode{y}{y} \\ [1.7cm]
\Rnode{x}{x}
\end{array}
\mbox{
\ncarr{xy}{y}
\alabel{q(p_{x,1},y)}
\nccdar{xy}{x}
\blabel{p_{\crossx{x}{y}{1},x}}
}
$$
is a product diagram since
\begin{displaymath}
\ccsquareoutline{3cm}{1.7cm}{\crossx{x}{y}{1}}{y}{x}{1} 
%\ccsquareacross{q(p_{x,1},y)}{p_{x,1}}
\begin{arrows}
% composition
\nccdar{TL}{BL}
\blabel{p_{\crossx{x}{y}{1},x}}
\nccdar{TR}{BR}
\alabel{p_{y,1}}
% reference
\ncarr{TL}{TR}
\alabel{q(p_{x,1},y)}
\nccdar{BL}{BR}
\blabel{p_{x,1}}
\end{arrows}
\end{displaymath}
is a pullback diagram.

If $f: w \morph x$ and $g: w \morph y$ then by we shall write $\tuple{f,g}$ 
for the the unique morphism $\tuple{f,g}: w \morph \crossx{x}{y}{1}$ such that
\begin{equation}
\tuple{f,g} \circ p_{\crossx{x}{y}{1},x} = f
\end{equation}
and
\begin{equation}
\tuple{f,g} \circ q(p_{x,1},y) = g
\end{equation}

Note that the product operation $\crossx{}{}{1}$ is far from symmetric 
because if, for example, $1 \base x$ and $1 \base y$ then $x \base \crossx{x}{y}{1}$ and $y \base \crossx{y}{x}{1}$. 

If $x$ is an object of a contextual category \catcw then we shall write $x^n$ for the cannonical n-fold product $x$ with itself so that we have
\begin{equation*}
x^{n+1}=\crossx{x^n}{x}{1}.
\end{equation*} 
If for the moment we denote the i'th projection as $p_i: x^n \morph x$ then if $w$ is some other object and if \foreachi, $f_i:w \morph x$ then we
will denote by $\tuple{f_1,...f_n}:w \morph x^n$ the unique morphism such that \foreachi, $\tuple{f_1,...f_n} \circ p_i = f_i$ and so we will have that
\begin{equation*}
\tuple{f_1,...f_n} = \tuple{\tuple{f_1,...f_{n-1}},f_n}.
\end{equation*}
Note that if $1 \base x$ in \catcw then $1 \base x \base x^2 ... \base x^n $ in \catc.