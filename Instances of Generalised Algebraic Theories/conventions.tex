\newpage

\newpage
\subsubsection{Naming of objects and morphisms}
\label{projectionnaming}
It is convenient to adopt some naming conventions when discussing the structure in a contextual category $\catcw$ into which an interpretation $I$ maps a theory $U$. Firstly it is convenient that the object $I(X)$ which is the interpretation of some sort $X$ of the theory is simply referred to as $X$.
Secondly if $X$ is a sort symbol such that $1 \base X$ in $\catcw$ then it is conventient to adopt some naming conventions for
the projection morphisms: the two canninical projection morphims $X^2 \morph X$ with be referred to as $x_1$ and $x_2$. The three cannonical projection morphisms $X^3 \morph X$ will be referred to as $y_1, y_2$ and $y_3$ and the four projection morphisms $X^4 \morph X$ will be referred to as 
$z_1, z_2, z_3$ and $z_4$. For sake of  uniformity  and considering that  $X^1$ is $X$ with product diagram  $id_X: X \morph X$  in the definitions
below we refer to $id_X$ as $w_1$.
Now in a contextual category, for any object $X$ such that $1 \base X$, $id_X = q(id_1,X) = q(t_1,X)$. Therefore we can define our uses of
$w_1$, $x_1,x_2,y_1,y_2,y_3,z_1,z_2,z_3$ and $z_4$ with respect to powers of an object $X$ by: \\
\begin{tabular} {l p{1cm} l p{1cm} l p{1cm} l}
$w_1 = q(t_1,X)$ &&  $x_1 = p_{X^2} \circ w_1$ && $y_1 = p_{X^3} \circ x_1$  && $z_1 = p_{X^4} \circ y_1$ \\
                 &&  $x_2 = q(t_X,X)$          && $y_2 = p_{X^3} \circ x_2$  && $z_2 = p_{X^4} \circ y_2$ \\
                 &&                            && $y_3 = q(t_{X^2},X)$       && $z_3 = p_{X^4} \circ y_3$ \\
                 &&                            &&                            && $z_4 = q(t_{X^3},X)$
\end{tabular}


Finally, we will see in the examples that it aids  readability to shorten  $p_A \circ f$ to $\dot{f}$, $p_B \circ p_A \circ f$ to $\ddot{f}$ and so on.
Therefore the above definitions can be rewritten as: \\
\begin{tabular} {l p{1cm} l p{1cm} l p{1cm} l}
$w_1 = q(t_1,X)$ &&  $x_1 = \dot{w_1}$ && $y_1 = \dot{x_1}$  && $z_1 = \dot{y_1}$ \\
                 &&  $x_2 = q(t_X,X)$          && $y_2 = \dot{x_2}$  && $z_2 = \dot{y_2}$ \\
                 &&                            && $y_3 = q(t_{X^2},X)$       && $z_3 = \dot{y_3}$ \\
                 &&                            &&                            && $z_4 = q(t_{X^3},X)$
\end{tabular}













 
  