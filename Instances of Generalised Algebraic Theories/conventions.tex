\subsubsection{Finite products and projection morphisms in contextual categories.}
If $A$ and $B$ are objects of a contextual category \catcw then the diagram

$$
\begin{array} {c p{1.5cm} c}
\Rnode{AB}{\crossx{A}{B}{1}} && \Rnode{B}{B} \\ [1cm]
\Rnode{A}{A}
\end{array}
\mbox{
\ncarr{AB}{B}
\alabel{q(t_A,B)}
\ncarr{AB}{A}
\alabel{p_{\crossx{A}{B}{1}}}
}
$$
is a product diagram since

$$
\ccsquareanddroppers{3cm}{1.2cm}{\crossx{A}{B}{1}}{B}{A}{1}
\ccsquareacross{q(t_A,B)}{t_A}
%\ccsquaredown{p_{\crossx{A}{B}{1}}}{t_B}
$$
is a pullback diagram.

Consider the usual conventions for the construction of cannonical products for sequences of objects  in a category
\catcw with specified product diagrams for pairs of objects. Consider how we can choose these cannonical products of sequences of objects 
so that if $X$ is an object
and the n-fold product of $X$ with itself is $X^n$ with cannonically chosen projections $p_i: X^n \morph X$ 
then the chosen $n+1$-fold product of $X$  has projections $p \circ p_1, ... p \circ p_n, q$ where 
$$
\begin{array} {c p{1.5cm} c}
\Rnode{XnX}{X^n \times X} && \Rnode{X}{X} \\ [1cm]
\Rnode{Xn}{X^n}
\end{array}
\mbox{
\ncarr{XnX}{X}
\alabel{q}
\ncarr{XnX}{Xn}
\alabel{p}
}
$$
is the given product diagram for the pair of objects $X^n$ and $X$.

Therefore we can construct a cannonical product diagrams for all sequences of objects $X_1,...X_n$ in a contextual category \catcw . We can do this in such a way that, if $X$ is an object of \catcw and $n >1 $ then $1 \base X \base X^2 ... \base X_n \base x^{n+1}$ in \catcw and if the
$n$ projection morphisms of $x^n$ are $p_1,...p_n$ then the $n+1$ projection morphisms of $X^{n+1}$ are 
$p_{X_{n+1}} \kern-3pt\circ p_1,\ p_{X_{n+1}}\kern-3pt \circ p_2,\  ...\ p_{X_{n+1}}\kern-3pt \circ p_n,\  q(t_{X^n},X) $

\subsubsection{Naming of objects and morphisms}
\label{projectionnaming}
It is convenient to adopt some naming conventions when discussing the structure in a contextual category $\catcw$ into which an interpretation $I$ maps a theory $U$. Firstly it is convenient that the object $I(X)$ which is the interpretation of some sort $X$ of the theory is simply referred to as $X$.
Secondly if $X$ is a sort symbol such that $1 \base X$ in $\catcw$ then it is conventient to adopt some naming conventions for
the projection morphisms: the two canninical projection morphims $X^2 \morph X$ with be referred to as $x_1$ and $x_2$. The three cannonical projection morphisms $X^3 \morph X$ will be referred to as $y_1, y_2$ and $y_3$ and the four projection morphisms $X^4 \morph X$ will be referred to as 
$z_1, z_2, z_3$ and $z_4$. For sake of  uniformity  and considering that  $X^1$ is $X$ with product diagram  $id_X: X \morph X$  in the definitions
below we refer to $id_X$ as $w_1$.
Now in a contextual category, for any object $X$ such that $1 \base X$, $id_X = q(id_1,X) = q(t_1,X)$. Therefore we can define our uses of
$w_1$, $x_1,x_2,y_1,y_2,y_3,z_1,z_2,z_3$ and $z_4$ with respect to powers of an object $X$ by: \\
\begin{tabular} {l p{1cm} l p{1cm} l p{1cm} l}
$w_1 = q(t_1,X)$ &&  $x_1 = p_{X^2} \circ w_1$ && $y_1 = p_{X^3} \circ x_1$  && $z_1 = p_{X^4} \circ y_1$ \\
                 &&  $x_2 = q(t_X,X)$          && $y_2 = p_{X^3} \circ x_2$  && $z_2 = p_{X^4} \circ y_2$ \\
                 &&                            && $y_3 = q(t_{X^2},X)$       && $z_3 = p_{X^4} \circ y_3$ \\
                 &&                            &&                            && $z_4 = q(t_{X^3},X)$
\end{tabular}


Finally, we will see in the examples that it aids  readability to shorten  $p_A \circ f$ to $\dot{f}$, $p_B \circ p_A \circ f$ to $\ddot{f}$ and so on.
Therefore the above definitions can be rewritten as: \\
\begin{tabular} {l p{1cm} l p{1cm} l p{1cm} l}
$w_1 = q(t_1,X)$ &&  $x_1 = \dot{w_1}$ && $y_1 = \dot{x_1}$  && $z_1 = \dot{y_1}$ \\
                 &&  $x_2 = q(t_X,X)$          && $y_2 = \dot{x_2}$  && $z_2 = \dot{y_2}$ \\
                 &&                            && $y_3 = q(t_{X^2},X)$       && $z_3 = \dot{y_3}$ \\
                 &&                            &&                            && $z_4 = q(t_{X^3},X)$
\end{tabular}













 
  