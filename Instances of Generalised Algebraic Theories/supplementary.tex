
\begin{lemma}
\llabel{supplementarylemma}
If $I$ is an instance of the generalised algebraic theory $U$ in a contextual category \catcw then
if  $Q$ and $\encyOmega{m}$ are contexts, for some $m \geq 1$, 
and if $\tuple{\sm}$ is a realisation of $\encyOmega{m}$ wrt $Q$ so that
 \foreachj, the rule \IsOmega{j} which we denote $r_{s_j}$ is a derived rule of $U$, then
\begin{enumerate}[(i)]
\item
if the rule \ZOmega which we denote $r_\Omega$ is a derived rule of $U$ then
$$\displaystyle\Imappedrule{Q}{\isT{\Omega[s_1|y_1...s_m|y_m]}}=\duple{I(r_{s_1}),...I(r_{s_m})}^*I(r_\Omega)$$
\item if \ZsOmega which we denote $r_s$ is a derived rule of $U$ then 
$$\displaystyle\Imappedrule{Q}{\ofT{s[s_1|y_1...s_m|y_m]}{\Omega[s_1|y_1...s_m|y_m]}}=\duple{I(r_{s_1}),...I(r_{s_m})}^*I(r_s)$$
\end{enumerate}
\end{lemma}
\begin{proof}
(i) follows because $r_\Omega$ being a derived rule must be consistently interpreted by $I$ and so can use clause (i)(c) of definition \lref{consistentinterpretation} and then simplify using (d3a). (ii) follows likewise using clause (ii)(c) of definition \lref{consistentinterpretation} and then
(d3b).
\end{proof}


\iffalse
\begin{lemma}
\llabel{supplementarytuplelemma}
\highlight{Live without this perhaps because of the ambiguity of the tuple notation and the complexity of making it more precise?}
If $I$ is an instance of the generalised algebraic theory $U$ in a contextual category \catcw then
if  $Q$ and $\encyOmega{m}$ are contexts, for some $m \geq 1$, 
and if $\tuple{\sm}$ is a realisation of $\encyOmega{m}$ wrt $Q$ so that
 \foreachj, the rule \IsOmega{j} which we denote $r_{s_j}$ is a derived rule of $U$, then
 if \foreachj, there is a morphism $f_j$ such that $I(r_{s_j})=s(f_j)$ then \commentary{check that we do not need to specify the domain of each $f_j$.}
\begin{enumerate}[(i)]
\item
if the rule \ZOmega which we denote $r_\Omega$ is a derived rule of $U$ then
$$\displaystyle\Imappedrule{Q}{\isT{\Omega[s_1|y_1...s_m|y_m]}}=\tuple{f_1,...f_m}^*I(r_\Omega)$$
\item if \ZsOmega which we denote $r_s$ is a derived rule of $U$ then \commentary{unfortunate double use of $s$.}
$$\displaystyle\Imappedrule{Q}{\ofT{s[s_1|y_1...s_m|y_m]}{\Omega[s_1|y_1...s_m|y_m]}}=\tuple{f_1,...f_m}^*I(r_s)$$
\end{enumerate}
\end{lemma}
\begin{proof}
\tbd
\end{proof}
\fi


\begin{lemma}
\llabel{supplementaryweakeninglemma}
If $I$ is an instance of the generalised algebraic theory $U$ in a contextual category \catcw then
if  $Q$ and $\encyOmega{m}$ are contexts, for some $m \geq 0$,  then
\begin{enumerate}[(i)]
\item if the rule \ZOmega which we denote $r_\Omega$ is a derived rule of $U$ then
$$\displaystyle\Imappedrule{Q,\, \yOmega{m}}{\isT{\Omega}}=\crossx{I(Q)}{I(r_\Omega)}{1}$$
\item if \ZsOmega which we denote $r_s$ is a derived rule of $U$ then 
$$\displaystyle\Imappedrule{Q,\, \yOmega{m}}{\ofT{s}{\Omega}}=\crossx{I(Q)}{I(r_s)}{1}$$
\end{enumerate}
\end{lemma}
\begin{proof}
The  $m=0$ case was shown in lemma \lref{typeweakeninglemma}.
For $m \geq 1$ we can assume, inductively, that the result holds for all j,  $j \leq m$, so that 
we can assume \foreachj, the rule \gatdisplayrule{Q,\, \yOmega{j-1}}{\isT{\Omega_j}} is mapped by $I$ to
$\crossx{I(Q)}{I(r_{\Omega_j})}{1}$.

\newcommand{\IofyweakenedbyQ}[1]{I(s(p_{\crossx{I(Q)}{I(r_{\Omega_m})}{1}, \crossx{I(Q)}{I(r_{\Omega_#1})}{1}}))}
With this assumption, because $I$ is an instance then from clause (ii)(d) of definition \lref{consistentinterpretation} 
we have that, \foreachj,
\begin{equation}
\label{weakendedyinterpretation}
I(r_{y_j})=\IofyweakenedbyQ{j} 
\end{equation}
where $r_{y_j}$  is the rule \gatdisplayrule{Q,\, \yOmega{m}}{\ofT{y_j}{\Omega_j}}.

Now we  prove (i) as follows:
\begin{align*}
\displaystyle\Imappedrule{Q,\, \yOmega{m}}{\isT{\Omega}}  \kern-2cm   \\
          &= \duple{I(r_{y_1}),... I(r_{y_n})} ^* I(r_\Omega)                   && \mbox{By lemma \lref{supplementarylemma},}     \\ 
          &= \duple{\IofyweakenedbyQ{1},... \IofyweakenedbyQ{m}} ^* I(r_\Omega) && \mbox{using (\ref{weakendedyinterpretation}),}                         \\
          &= q(p_{I(Q),1},I(r_{\Omega_m})) ^* I(r_\Omega)                       && \mbox{by \lref{duplesofplemma},}               \\
          &= {p_{I(Q),1}} ^*  I(r_\Omega)                                         && \mbox{by (Q6),}                                \\
          &= \crossx{I(Q)}{I(r_\Omega)}{1}                                      && \mbox{by definition of $\crossx{}{}{1}$.}  
\end{align*} 
(ii) follows in like manner.

\end{proof}