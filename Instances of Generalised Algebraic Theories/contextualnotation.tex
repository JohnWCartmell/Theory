%contextualnotation
\label{contextualnotation}

\note
The original definition of contextual category given  in [1] and [2], a contextual category is defined to be a tree-structured category 
\cat{C} with the following additional structure:

\noindent 
(i) whenever
$
\begin{array}{cp{.9cm}c}
            & & \Rnode{z}{z} \\ [1.2cm]
\Rnode{x}{x}& & \Rnode{y}{y} \\ [0.5cm]
\end{array}
$
\jcbarr{f}{x}{y}
\ncasar{p_z}{z}{y}

in \cat{C}, an object $f \sub z$ such that $x \base f \sub z$, a morphism $q(f,z): f \sub z \rightarrow z$ such that

\begin{axiom}{q1}
q(f,z) \circ p_z = p_{f \sub z} \circ f
\end{axiom}

i.e. such that the diagram: 
$$
\ccsquareoutline{0.9cm}{1.2cm}{f^*z}{z}{x}{y}
\ccsquareacross{q(f,z)}{f}
\ccsquaredown{p_{f \sub z}}{p_z}
$$
commutes, 

\noindent
and, (ii), so that each such diagram is a pullback diagram, that is: for all objects $w$ of \cat{C}, and for all
morphisms $h_1: w \rightarrow x$ and $h_2: w \rightarrow z$ (see diagram \ref{pullback} below) such that
$h_1 \circ f = h_2 \circ p_z$ 
there exists a unique $h:w \rightarrow f \sub z$ in \cat{C} such that
$h \circ p_{f \sub z} = h_1$ and $h \circ q(f,z) = h_2$, as shown here:

\vspace{3mm}
\begin{center}
\begin{equation}
\label{pullback}
\begin{array}{cp{0.5cm}cp{1.2cm}c}
\Rnode{w}{w} &&                     &&           \\ [0.7cm]
             &&\Rnode{fstarz}{f^*z} && \Rnode{z}{z}\\ [1.2cm]
             &&\Rnode{x}{x}         && \Rnode{y}{y}
\end{array}
\end{equation}
\ncbsar{p_{f \sub z}}{fstarz}{x}
\jcbarr{f}{x}{y}
\ncaarr{q(f,z)}{fstarz}{z}
\ncasar{p_z}{z}{y}
\setlength{\arrnodesepA}{3pt}
\jcbarr[-35]{h_1}{w}{x}
\ncaarr[35]{h_2}{w}{z}
\psset{linestyle=dashed}
\ncaarr{h}{w}{fstarz}
\end{center}

\vspace {0.25cm}
\noindent and so that (iii) whenever $x \base y$ in \cat{C}, 
\begin{axiom}{q2}
id_x^*y=y
\end{axiom}

and

\begin{axiom}{q3}
q(id_x,y) = id_y
\end{axiom}



\noindent and (iv) whenever 
$
\begin{array}{c p{.9cm} c p{.9cm} c}
             &   &             &   & \Rnode{z}{z} \\ [1.2cm]
\Rnode{w}{w} &   &\Rnode{x}{x} &   & \Rnode{y}{y} \\ [0.5cm]
\end{array}
$
\jcbarr{f}{w}{x}
\jcbarr{g}{x}{y}
\ncasar{c}{z}{y}
in \cat{C}, 

then

\begin{axiom}{q4}
(f \circ g)^*z =  f^* (g ^* z)
\end{axiom}

and 
\begin{axiom}{q5}
q(f \circ g,z) = q(f,g^*z) \circ q(g,z)
\end{axiom}



\note
Following Voevodsky we may replace the pullback condition of the original definition by an 
`s' operator along with axioms as follows:

\noindent (ii') for all morphisms $f: x \rightarrow y$, a morphism $s(f) : x \rightarrow f \sub p_y \sub y$ such that both:

\begin{axiom}{s1}
s(f) \circ p_{f\sub p_y \sub y}=id_x
\end{axiom}

\noindent and

\begin{axiom}{s2}
s(f) \circ q( f \circ p_y     ,y)=f
\end{axiom}	

\noindent i.e. such that the following diagrams commute:
\begin{center}
\begin{displaymath}
\begin{array}{cccp{1.cm} cp{.9cm}c}
&\Rnode{fXyyM}{f\sub p_y \sub y}&  & &  \Rnode{fXyy}{f\sub p_y \sub y} & & \Rnode{yXy}{p_y \sub y}\\ [1.2cm]
\Rnode{xL}{x} & &\Rnode{xR}{x} & &\Rnode{x}{x}         & & \Rnode{y}{y}
\end{array}
\end{displaymath}
\ncasar{p_{f\sub p_y \sub y}}{fXyy}{x}
\jcbarr{f}{x}{y}
\ncaarr{q(f,p_y \sub y)}{fXyy}{yXy}
\ncasar{p_{p_y \sub y}}{yXy}{y}
\ncaarr{s(f)}{xL}{fXyyM}
\ncasar{p_{f\sub p_y \sub y}}{fXyyM}{xR}
\jcbarr{id_x}{xL}{xR}
\end{center}

\noindent
and such that whenever

\begin{center}
\begin{displaymath}
\begin{array}{c p{.9cm} c p{.9cm} c}
\Rnode{w}{w}&& \Rnode{g*z}{g \sub z} && \Rnode{z}{z} \\ [1.2cm]
            && \Rnode{x}{x}  && \Rnode{y}{y} \\ [0.2cm]
\end{array}
\end{displaymath}
\jcbarr{f}{w}{g*z}
\jcbarr{g}{x}{y}
\ncaarr{q(g,z)}{g*z}{z}
\ncasar{}{g*z}{x}
\ncasar{}{z}{y}
\end{center}

\noindent in \cat{C} then

\begin{axiom}{s3}
s(f \circ q(g,z))=s(f)
\end{axiom}



\note
The contextual category structure supplies us with pullbacks for any $\smorph$ morphism, 
these given pullbacks can be pieced together to obtain a pullback for
any $\dmorph$ morphism  along any morphism with the same codomain. 

In general, whenever $y \leq z$ in $C$ and whenever $f:x \morph y$ in $C$, then we have
the following canonical pullback for the morphism $p_{z, y}$ along $f$, where
$w_1, ... w_n$ is the unique sequence of objects of $C$ such that 
$y \base w_1 \base ... \base w_n \base z$:

\vspace{3mm}
\begin{center}
\begin{equation}
\label{compositepullbackdefinition}
\begin{array}{cp{2.9cm}c}
\Rnode{TOPL}{q(...q(f, w_1)...w_n)^* z} & & \Rnode{TOPR}{z}\\ [1.2cm]
\Rnode{zOTTOML}{x}         & & \Rnode{zOTTOMR}{y}
\end{array}
\end{equation}
\jcbarr{f}{zOTTOML}{zOTTOMR}
\ncaarr{q(q(...q(f,w_1)...w_n),z)}{TOPL}{TOPR}
\nccdar{TOPL}{zOTTOML}
\blabel{p_{q(...q(f, w_1)...w_n)^* z,x}}
\nccdar{TOPR}{zOTTOMR}
\alabel{p_{z,y}}
\end{center}

Since these constructed pullbacks form an important part of contextual
category structure we would like a simpler notation for them. As no confusion is
likely, we extend the $^*$ and $q$ notation to cover these new pullback diagrams.
From now on if $f:x \morph y$ in $C$ and $y \leq z$ in $C$, then the diagram

\vspace{3mm}
\begin{center}
\begin{equation}
\label{compositepullbackout}
\begin{array}{cp{.9cm}c}
\Rnode{fstarz}{f^*z} & & \Rnode{z}{z}\\ [1.2cm]
\Rnode{x}{x}         & & \Rnode{y}{y}
\end{array}
\end{equation}
\nccdar{fstarz}{x}
\blabel{p_{f \sub z},x}
\jcbarr{f}{x}{y}
\ncaarr{q(f,z)}{fstarz}{z}
\nccdar{z}{y}
\alabel{p_{z,y}}
\end{center}
is the canonical pullback diagram as defined in (\ref{compositepullbackdefinition}) above. 
\note
The following observation
follows from the way the extended pullback diagrams are constructed. 
In the extended notation, if $f: x \morph y$ 
and $y \leq z \leq zz$ in the contextual category C, then
\begin{equation}
f^*zz = q(f, z)^*zz
\end{equation}
 and 
\begin{equation}
q(f, zz) = q(q(f, z), zz)
\end{equation}
and so the outer diagram in
\renewcommand{\pc}[2]{p_{#1,#2}}  % as \pc defined in ccategories macros differently to this
$
\begin{array}{ccp{.9cm}c}
\\[0.25cm]
&\Rnode{TL}{q(f,z)^*zz} & & \Rnode{TR}{zz}\\ [1.2cm]
&\Rnode{ML}{f^*z} & & \Rnode{MR}{z}\\ [1.2cm]
&\Rnode{BL}{x}         & & \Rnode{BR}{y} \\[1.0cm]
\end{array}
$
%composition
\makebox[0.2cm]{   % This make box prevents white space pushing out to the right
                   % cannot see where this white space is comin from. To investigate
									 % change the \makebox[0.2cm] to \fbox and you will see the problem.
\nccdar{TL}{ML}\blabel{P_{q(f,z)^*zz,f^*z}}\nccdar{ML}{BL}\blabel{p_{f \sub z,x}}\nccdar{TR}{MR}\alabel{p_z}
\nccdar{MR}{BR}
}
\alabel{p_z}
%reference
\ncarr{TL}{TR}
\alabel{q(q(f,z),zz)}
\ncarr{ML}{MR}
\alabel{q(f,z)}
\ncarr{BL}{BR}
\blabel{f}
is diagram (\ref{compositepullbackout}). 
On such diagrams as shown here the $p$ labels on $\dmorph$ morphisms (inclusive of their indices) are entirely predictable  and so in diagrams that follows we may omit them.


\note
If we write $\crossx{y}{z}{x}$ in place of ${p_{y,x}}^*z$, for $x < y$, $x < z$  in \ccat then
$\crossx{y}{z}{x}$  represents  in the syntax the `weakening' of a rule of the form
\begin{displaymath}
x, w1,...w_n \tstyle \isT{z}
\end{displaymath}
from a rule with context $x, w_1,...w_n$ to a rule with broader context $y, w_1, ... w_n$: 
\begin{displaymath} 
y, w_1,...w_n \tstyle \isT{z}.
\end{displaymath}

Within the contextual category I think of $\crossx{y}{z}{x}$  as a local cartesian product but of course categorically it is a filtered product i.e. a pullback. If $w < x$ and $w < y$  then 
\genericcrossxproductdiagram % defined in 'paper.tex'
is a pullback diagram in \ccat.

\note
We can extend the $\crossx{}{}{w}$ notation to morphisms. If $f:x \morph x'$ and $g: y \morph y'$ in a contextual
category $\ccat[C]$ and if $w$ is an object such that $w < x$, $w <x'$, $w < y$ and $w < y'$ then 
define $\crossx{f}{g}{w}:\crossx{x}{y}{w} \morph \crossx{x}{y}{w}$ in $\ccat[C]$ by
\begin{equation}
\crossx{f}{g}{w} = \tuple{p_{\crossx{x}{y}{w}, x} \circ f,q(p_{x,w},y) \circ g}
\end{equation}  

\note
In the case special case that $w$ is the terminal object $1$ then the pullback  specialises to give a product diagram:

\begin{displaymath}
\begin{array}{ccccc}
\Rnode{xy}{\crossx{x}{y}{1}} &&               &&               \\[1.3cm]
\Rnode{x}{x}                 &&               && \Rnode{y}{y}  \\                                    
\end{array}
\mbox{\ncsar{xy}{x}
\blabel{p_{\crossx{x}{y}{1},x}}
\ncaarr{q(p_{x,1},y)}{xy}{y}}
\end{displaymath}

In this special case the $\tuple{}$ operation defined earlier is the pairing operation for if
$f: w \morph x$ and $g: \morph y$ then $\tuple{f,g}: w \morph \crossx{x}{y}{1}$ 
and 
\begin{equation}
\tuple{f,g} \circ p_{\crossx{x}{y}{1},x} = f
\end{equation}
and
\begin{equation}
\tuple{f,g} \circ q(p_{x,1},y) = g
\end{equation}

\note 
Note that the product operation $\crossx{}{}{1}$ is far from symmetric 
because if, for example, $1 \base x$ and $1 \base y$ then $x \base \crossx{x}{y}{1}$ and $y \base \crossx{y}{x}{1}$ but we can define 
a swap operation $sw_{x,y} : \crossx{x}{y}{1} \morph \crossx{y}{x}{1}$ by
\begin{equation}
sw_{x,y} = \tuple{q_{p_x,y}, p_{\crossx{x}{y}{1},x}}
\end{equation}

\note
Associativity of $\crossx{}{}w$  follows from the coherence property of the pullbacks in the contextual category. 
For example if $w < x$, $w < y$, $w < z$ in a \ccat then from coherence of pullbacks in \ccat we have:
$\crossx{x}{(\crossx{y}{z}{w})}{w} = \crossx{(\crossx{x}{y}{w})}{z}{w}$ as shown here in this diagram:
 
\begin{displaymath}
\begin{array}{cp{1.0cm}cp{1.0cm}c}
\Rnode{J1}{}\Rnode{D1} {\crossx{(\crossx{x}{y}{w})}{z}{w}}\Rnode{J2}{} \ \ \ \ \   &&  &&  \\ 
= && && \\
\Rnode{D2} {\crossx{x}{(\crossx{y}{z}{w})}{w}}    &&  &&                        \\ [1.3cm]
\Rnode{xy}{\crossx{x}{y}{w}}&& \Rnode{yz}{\crossx{y}{z}{w}} &&                      \\[1.3cm]
\Rnode{x}{x}&& \Rnode{y}{y} && \ \ \ \ \ \ \ \ \ \ \ \ \ \Rnode{z}{z}                                        \\[1.3cm]
             && \Rnode{w}{w} &&                                                     
\end{array}
\end{displaymath}

\ncaarr[50]{q(\pc{\crossx{x}{y}{w}}{w},z)}{J2}{z}
\ncsar{D2}{xy}
\ncsar{xy}{x}
\ncsar{yz}{y}
\ncsar{x}{w}
\ncsar{y}{w} 
\ncsar{z}{w}
\ncaarr{q(\pc{x}{w},y)}{xy}{y}
\ncaarr{q(\pc{y}{w},z)}{yz}{z}
\ncaarr{q(\pc{x}{w},\crossx{y}{z}{w})}{D2}{yz}



\note A number of minor lemmas:

\begin{lemma}
\llabel{footandstactic}
If $f: A \morph B$ and $f':A \morph B$ in a contextual category \catcw then if 
$f \circ p_B$ = $f' \circ p_B$ and $s(f) = s(f')$ then $f=f'$.
\end{lemma}
\begin{proof}
Follows by axiom (s2) since we have:
$f = s(f) \circ q(f \circ p_B,B)  = s(f') \circ q(f' \circ p_B,B) = f'$.
\end{proof}

From which follows:
\begin{lemma}
\llabel{stactic}
If $A$ is any object of a contextual category \catcw and if $B$ is an object such that $1 \base B$ in \catcw then
if $f: A \morph B$ and $f':A \morph B$ in a contextual category \catcw then $f=f'$ iff $s(f) = s(f')$.
\end{lemma}

\begin{lemma}
\llabel{crosssectionlemma}
If 
%\begin{equation*}
$
\begin{array}{ c c c}
\Rnode{B}{B} &              & \Rnode{Bp}{B'} \\[1cm]
             & \Rnode{A}{A} &     
\mbox{\ncsar{B}{A}
\ncsar{Bp}{A}
%\ncarr[-30]{A}{Bp}
\ncrightsimplesection{A}{Bp}
\blabel{g}}
\end{array}
$
%\end{equation*}
in a contextual category \catcw and if $g$ is a section of $B'$ (i.e. if $g \circ p_{B'}= id_A$) so that we have 
\begin{equation*}
\begin{array}{ c c c}
\Rnode{BBp}{\crossx{B}{B'}{A}} \\[1.3cm]
\Rnode{B}{B} &              & \Rnode{Bp}{B'} \\[1.1cm]
             & \Rnode{A}{A} &
\mbox{
\ncsar{BBp}{B}
\ncrightcrosssection{B}{BBp}
\blabel{\crossx{B}{g}{A}}
\ncsar{B}{A}
\blabel{p_B}
\ncsar{Bp}{A}
\ncrightsimplesection{A}{Bp}
\blabel{g}
}														
\end{array}
\end{equation*}
in \catcw,  then
\begin{equation}
\label{crosssectionlemmatarget}
\crossx{B}{g}{A} = s(p_B \circ g).
\end{equation} 
\end{lemma}
\begin{proof}
$\crossx{B}{g}{A}$ is defined to be the unique section of $\crossx{B}{B'}{A}$ such that $(\crossx{B}{g}{A}) \circ q( p_B,B') = p_B \circ g$.

(\ref{crosssectionlemmatarget}) follows because $s(p_B \circ g)$ is also such a section, since it is defined to be the unique section of $(p_B \circ g \circ p_{B'}) ^* B'$
such that $s(p_B \circ g) \circ q( p_B \circ g \circ p_{B'}, B') = p_B \circ g$ and this simplifies,
because $g \circ p_{B'} =id_A$, to
 $s(p_B \circ g)$ being a section of ${p_B} ^* B'$ (i.e of $\crossx{B}{B'}{A}$) satisfying $s(p_B \circ g) \circ q( p_B,B') = p_B \circ g$. 
\end{proof}

% *************************************************************************************
% sfglemma ****************************************************************************
\begin{lemma}
\llabel{sfglemma}
If $f:A \morph B$ and $g:B\morph C$ in a contextual category \catcw and if $C \in Cover(B)$ then
\begin{equation}
\label{sgflemmagoalone}
ft(f\circ g)^*C = f^*(ft(g)^*C)
\end{equation}
and 
\begin{equation}
\label{sgflemmagoaltwo}
S(f\circ g)=f^*S(g)
\end{equation}
where $ft(g) = g \circ p_B$ so that we have
\begin{displaymath}
\begin{array}{ccp{1.7cm}cp{1.7cm}c}
ft(f\circ g)^*C=\kern-10pt&\Rnode{TL}{f^*(ft(g)^*C)} & & \Rnode{TC}{ft(g)^*C}          \\ [1.7cm]
&\Rnode{BL}{A}         & & \Rnode{BC}{B} && \Rnode{BR}{C}
\end{array}
\mbox{
\ncsar{TL}{BL}
\ncsar{TC}{BC}
\ncarr{BL}{BC}
\blabel{f}
\ncarr{BC}{BR}
\blabel{g}
\ncarr{TL}{TC}
\alabel{q(f,ft(g)^*C)}
\ncarr{TC}{BR}
\alabel{q(ft(g),C)}
\ncleftsimplesection{BL}{TL}
\alabel{s(f\circ g)=f^*s(g)}
\ncleftsimplesection{BC}{TC}
\alabel{s(g)}
}
\end{displaymath}
in \catc.
\end{lemma}
\begin{proof}
That (\ref{sgflemmagoalone}) holds follows by axiom (q4).

To show that (\ref{sgflemmagoaltwo}) holds remember that $s(f \circ g)$is the unique section of $ft(f \circ g)*C$
such that $s(f \circ g) \circ q(f \circ g \circ p_C,C) = f\circ g$. Therefore it suffices to show that
$f^*s(g) \circ q(f \circ g \circ p_C,C) = f\circ g$ and this we can show as follows:
\begin{align*}
(f^*s(g)) \circ q(f \circ g \circ p_C,C) &= (f^*s(g)) \circ q(f ,(g \circ p_C)^*C) \circ q(g \circ p_C ,C) &&\mbox{by axiom (qf)} \\
                             &= f \circ s(g) \circ q(g \circ p_C ,C)                   &&\mbox {from defn. of $f^*s(g)$}\\
														 &= f \circ g                                              &&\mbox {by axiom (s2)}
\end{align*}
\end{proof}

\newcommand{\duplesone}{{\duple{s_1}_{B_1}}}
\newcommand{\duplestwo}{{\duple{s_1,s_2}_{B_2}}}
\newcommand{\duplesn}{\duple{s_1,...s_n}_{B_n}}
\newcommand{\duplesi}{{\duple{s_1,...s_i}_{B_i}}}
\newcommand{\duplesilessone}{\duple{s_1,...s_{i-1}}_{B_{i-1}}}
\newcommand{\duplesj}{{\duple{s_1,...s_j}_{B_j}}}
\newcommand{\duplesjlessone}{\duple{s_1,...s_{j-1}}_{B_{j-1}}}
\newcommand{\duplesisucc}{{\duple{s_1,...s_{i+1}}_{B_{i+1}}}}
\newcommand{\duplesnlessone}{{\duple{s_1,...s_{n-1}}_{B_{n-1}}}}

\newcommand {\sonesub}{{s_1}^*}
\newcommand {\stwosub}{{s_2}^*}
\newcommand {\stwocascade}{\stwosub\sonesub}
\newcommand {\sisub}{{s_i}^*}
\newcommand {\sicascade}{\sisub...\sonesub}
\newcommand {\sisuccsub}{{s_{i+1}}^*}
\newcommand {\sisucccascade}{\sisuccsub...\sonesub}
\newcommand {\snlessonesub}{{s_{n-1}}^*}
\newcommand {\snlessonecascade}{\snlessonesub...\sonesub}
\newcommand {\snsub}{{s_n}^*}
\newcommand {\sncascade}{\snsub...\sonesub}

\note If $A$ is an object of contextual category \catc, if $1 \base B_1 ... \base B_n$ in \catcw and if
\begin{equation*}
\begin{array}{l}
s_1 \in Sect(\crossx{A}{B_1}{1}),                  \\
s_2 \in Sect(\sonesub (\crossx{A}{B_2}{1})),         \\
s_3 \in Sect(\stwocascade (\crossx{A}{B_3}{1})),     \\
\multicolumn{1}{c}{\vdots}                           \\
s_n \in Sect(\snlessonecascade (\crossx{A}{B_n}{1})) \\
\end{array}
\end{equation*}
\mbox{ in \catc},
then  we can define a morphism
$\duplesn:A \morph B_n$ in \catcw such that 
\begin{enumerate}[(i)]
\item $s(\duplesn) = s_n$,
\item for $n> 1$, $\duplesn \circ p_{B_n} = \duplesnlessone$, and 
\item for all objects $B$ of \catcw such that $B_n < B$, 
$\sncascade (\crossx{A}{B}{1}) = \duplesn ^* B$, \\
and for all sections $s$ of $B$,
$\sncascade (\crossx{A}{s}{1}) = \duplesn ^* s$.
\end{enumerate}

The definition of $\duplesn$ proceeds by induction. 
Define $\duplesone= s_1 \circ q(p_{A,1},B_1)$.
By axiom (s1), it follows immediately that $s(\duplesone)=s_1$.

If $B_1 <B$ in \catcw then we have $\sonesub (\crossx{A}{B}{1})=\duplesone ^* B$ because
\begin{align*}
\sonesub (\crossx{A}{B}{1})&= \sonesub q(p_{A,1},B_1)^*B     && \mbox{by definition of $\crossx{1}{}{}$,}\\
                         &= (s_1 \circ q(p_{A,1},B_1))^*B   && \mbox{by pullback coherence axiom (q5),}\\
                         &= \duplesone ^* B                   && \mbox{by definition of $\duplesone ^* B$.}
\end{align*}
Also, if $g$ is a section of $B$ then we can show, by a similar argument, 
that $\sonesub (\crossx{A}{g}{1})=\duplesone ^* g$.

Now assume that $\duplesi$ is defined and satisfies (i) to (iii) above. 
In particular we have  $\sicascade B_{i+1} = \duplesi ^* B_{i+1}$, and therefore that
$s_{i+1} \in Sect(\duplesi ^* B_{i+1})$, and this then allows us to define $\duplesisucc$ by 
\begin{equation*}
\duplesisucc = s_{i+1} \circ q(\duplesi, B_{i+1}).
\end{equation*} 
Immediately by axiom s3
we have that $s(\duplesisucc)=s_{i+1}$.
We have that $\duplesisucc \circ p_{B_{i+1}}= \duplesi$ because
\begin{align*}
\duplesisucc \circ p_{B_{i+1}} &=s_{i+1} \circ q(\duplesi, B_{i+1}) \circ p_{B_{i+1}} && \mbox{by definition of $\duplesisucc$,} \\
                               &=s_{i+1} \circ p_{\duplesi ^* B_{i+1}} \circ \duplesi && \mbox{because the pullback diagram commutes,} \\
															 &= \duplesi                       && \mbox{because $s_{i+1}$ is a section.}
\end{align*}
To establish (iii), suppose $B$ is some object such that $B_{i+1} < B$ in \catcw then we can show that $\sisucccascade (\crossx{A}{B}{1})=\duplesisucc ^* B$ as follows:
\begin{align*}
\sisucccascade (\crossx{A}{B}{1}) 
              &= \sisuccsub \duplesi ^* B && \mbox{by inductive hypothesis,} \\
                         &= \sisuccsub q(\duplesi,B_{i+1})^*B  && \mbox{by definition of extended $^*$,}\\
                         &= (s_{i+1} \circ q(\duplesi,B_{i+1}))^*B   && \mbox{by pullback coherence axiom,}\\
                         &= \duplesisucc ^* B                   && \mbox{by definition of $\duplesisucc$.}
\end{align*}
A similar argument shows that if $g$ is a section of $B$ then $\sisucccascade (\crossx{A}{g}{1})=\duplesisucc ^* g$.


\begin{lemma}
\label{dupledestructionlemma}
If $\duplesn : A \morph B_n$ in a contextual category $\catcw$ then \foreachi, 
\begin{equation}
\duplesn \circ p_{B_n,B_i} = \duplesi
\end{equation} 
\end{lemma}
\begin{proof}
Follows because for each $j$, $i < j \leq n$, $\duplesj \circ p_{B_j} = \duplesjlessone$
and $p_{B_j,B_i} = p_{B_j} \circ p_{B_{j-1},B_i}$.
\end{proof}

\newcommand{\dupletuplerhs}{\bigtuple{\duplesnlessone,g}_{p_{B_{n-1},B_i},C}}
\begin{lemma}
\label{thedupletuplelemma}
If $A$ is an object of contextual category \catc, if $1 \base B_1 ... \base B_n$ in \catcw and if
\begin{equation*}
\begin{array}{l}
s_1 \in Sect(\crossx{A}{B_1}{1}),                  \\
s_2 \in Sect(\sonesub (\crossx{A}{B_2}{1}))=Sect(\duplesone^*B_2),         \\
s_3 \in Sect(\stwocascade (\crossx{A}{B_3}{1}))=Sect(\duplestwo^*B_3),     \\
\multicolumn{1}{c}{\vdots}                           \\
s_n \in Sect(\snlessonecascade (\crossx{A}{B_n}{1})) =Sect(\duplesnlessone^*B_n)\\
\end{array}
\end{equation*}
\mbox{ in \catc}, and if $B_n$ is $\crossx{B_{n-1}}{C}{B_i}$, for some $i$, $0 \leq i < n$, 
(where in the case of $n$ equal to $0$ then 
by $B_0$ we mean the terminal object $1$ of \catc), if $g: A \morph C$ in \catcw and 
$g \circ p_C = \duplesi$, so that
$s(g) \in Sect(\duplesnlessone ^* B_n)$
\footnote {Because by definition of $s(g)$, $s(g) \in Sect((g \circ p_C) ^* C)$ and if 
$g \circ p_C =  \duplesi$ then 
\begin{align*}
(g \circ p_C) ^* C &= \duplesi ^* C  \\
                  &= (\duplesnlessone \circ {p_{B_{n-1},B_i}})^* C\\
									&=\duplesnlessone ^* B_n
\end{align*}
},  then if $s(g)=s_n$ then 


\begin{equation}
\label{dupletuplegoal}
\duplesn = \dupletuplerhs\,.
\end{equation}
\end{lemma}
\begin{proof}


To show that $\dupletuplerhs$ (rhs of (\ref{dupletuplegoal})) is defined we need show that 
\begin{equation}
g \circ p_C = \duplesnlessone \circ p_{B_{n-1},B_i}
\end{equation}
This follows from the assumption that $g \circ p_C = \duplesi$ and from lemma \ref{dupledestructionlemma}.

Now $\dupletuplerhs$ is therefore defined and is the unique morphism such that
\begin{equation}
\dupletuplerhs \circ p_{\crossx{B_{n-1}}{C}{1}} = \duplesnlessone
\end{equation}
and
\begin{equation}
\dupletuplerhs \circ q(p_{B_{n-1},B_i},C) = g
\end{equation}

as shown here

\begin{equation}
\begin{array}{c p{4cm} c p{3cm} c }
\\[1.75cm]
\Rnode{A}{A} && \Rnode{Bn1C}{\crossx{B_{n-1}}{C}{1}} &&                            \\[1.5cm]
						 &&                                      &&\Rnode{C}{C}                \\[0.5cm]
             && \Rnode{Bn1}{B_{n-1}}                 &&                            \\[1.5cm]
						 &&                                      &&\Rnode{Bi}{B_i}             \\[0.5cm]
\end{array}
\mbox{\ncdarr{A}{Bn1C}
\alabel{ \dupletuplerhs}
\ncarr{A}{Bn1}
\blabel{\duplesnlessone}
\ncsar{Bn1C}{Bn1}
\blabel{p_{\crossx{B_{n-1}}{C}{1}}}
\nccdar{Bn1}{Bi}
\blabel{p_{B_{n-1},B_i}}
\ncarr{Bn1C}{C}
\alabel{q(p_{B_{n-1},B_i},C)}[0.3]
\ncsar{C}{Bi}
\alabel{p_C}
\ncarr[60]{A}{C}
\alabel{g}
}
\end{equation}

Therefore to show (\ref{dupletuplegoal}), as we are required,  it suffices  to show that 

\begin{equation}
\label{dupletuplesubgoalone}
\duplesn \circ p_{\crossx{B_{n-1}}{C}{B_i}} = \duplesnlessone
\end{equation}
and
\begin{equation}
\label{dupletuplesubgoaltwo}
\duplesn \circ q(p_{B_{n-1},B_i},C) = g
\end{equation}

(\ref{dupletuplesubgoalone}) holds from the definition of $\duple{}_{B_n}$ because we have initially assumed that $B_n=\crossx{B_{n-1}}{C}{B_i}$.

We show that (\ref{dupletuplesubgoaltwo}) holds by using lemma \ref{footandstactic} and showing that
\begin{equation}
\label{dupletuplesubgoaltwoone}
\duplesn \circ q(p_{B_{n-1},B_i},C) \circ p_C = g \circ p_C
\end{equation}
and
\begin{equation}
\label{dupletuplesubgoaltwotwo}
s(\duplesn \circ q(p_{B_{n-1},B_i},C)) = s(g)
\end{equation}

(\ref{dupletuplesubgoaltwotwo}) follows immediately because by axiom \highlight{(s3)} \commentary{check use of parenthesis in reference to (q1),...(s3)} etc.
$s(\duplesn \circ q(p_{B_{n-1},B_i},C))=s(\duplesn)=s_n$ and from the initial assumption 
$s(g)=s_n$.

We are left with proving (\ref{dupletuplesubgoaltwoone}) which we can do as follows:
\begin{align*}
\duplesn \circ q(p_{B_{n-1},B_i},C) \circ p_C 
              &=  \duplesn \circ p_{\crossx{B_{n-1}}{C}{B_i}} \circ p_{B_{n-1},B_i} 
                                               && \mbox{commutivity of pullback square},               \\
							&=\duplesn \circ p_{\crossx{B_{n-1}}{C}{B_i},B_i} && \mbox{definition of extended $p$,}  \\
							&=\duplesi                                        && \mbox{lemma \ref{dupledestructionlemma},} \\
							&= g \circ p_C                                    && \mbox{from the initial assumption.}
\end{align*}
\end{proof}


