
For any contextual category \catc, the objects and sections with  operations $^*$,  $\crossx{}{}{}$ 
and $\delta$\footnote{For any object $x$ in a contextual category the morphism $delta_x$ is defined to be  $s(id_x)$. If $x_p \base x$ in 
a contextual category \catcw then the moorphism $\delta_x$ is a section of $\crossx{x}{x}{x_p}$} constitute a structure
that in \cite{CartmellMetaTheory} is referred to as a Meta-GAT algebra\footnote{I might call these concept instance algebras in the future}. 
All of the structure of a contextual category  can recovered from its meta-GAT algebra.

Objects, sections and operations $^*$ and  $\crossx{}{}{}$ of a contextual catregory play a big part of the definitions and lemmas that are to come in the next section
and these operations satisfy the Meta-GAT axioms specified in  \cite{CartmellMetaTheory}.
Here we do not give the proofs\footnote{My goal is, in future, to create and document such proofs as an exercise in computer aided algebra.}.

The identities that we will have need of are as follows. 

\begin{equation}
\label{metagattriplestar}
(f^*g)^*(f^*z)=f^*(g^*z)
\end{equation}
and
\begin{equation}
\label{metagatcrossstarcross}
(\crossx{x}{g}{})^*(\crossx{x}{z}{})=\crossx{x}{(g^*z)}{}
\end{equation}