

\newcommand{\Imappedrule}[2]  {I\left(\gatrawdisplayrule{#1}{#2}\right)}
\newcommand{\sectionsofImappedrule}[2] { Sect \left( \Imappedrule{#1}{#2} \right)}

\newcommand{\Imap}[1]{\setlength{\fboxsep}{1pt}       
\setlength{\fboxrule}{0.2pt}I\left(\mbox{#1}\right)}


\newcommand{\iI}{\scalebox{1.1}{$\boldsymbol{i}$}}
\newcommand{\Ibar}{\mkern 2.5mu\overline{\mkern-2.5mu \iI\mkern1.5mu}\mkern -1.5mu}

\newcommand{\ibarmappedrule}[2]  {\Ibar\left(\gatrawdisplayrule{#1}{#2}\right)}


If $I$ is a  mapping of derived T-rules and $\in$-rules of a theory $U$ to objects, respectively sections of a contextual category \catcw then
for any non-empty context $\xDelta{n}$ of $U$ by the mapping under $I$ of this context we shall mean the mapping under $I$ of the derived rule
\IDelta{n}.  By the mapping under $I$ of the empty context we shall mean the terminal object $1$ of \catc.

\newcommand{\sjconclusion}{\ofT{s_j}{\Omega_j[s_1|y_1,...s_{j-1}|y_{j-1}]}}  % xxxxx USED in an Imapped rule
\newcommand{\sjpconclusion}{\ofT{s'_j}{\Omega'_j[s_1|y_1,...s_{j-1}|y_{j-1}]}}
%\newcommand{\SUBsFORy{m}}{s_1|y_1,...s_m|y_m}                                  % used in Imapped rules

\newcommand{\IfIpartialmappingUtoC}{If $U$ is a generalised algebraic theory and if \catcw is a contextual category 
and if $I$ is a partial mapping of derived T-rules and $\in$-rules of the theory $U$ to objects, respectively sections of the contextual category \catc}
\newcommand{\IfIpartialmappingUtoCw}{\IfIpartialmappingUtoC\ }
\begin{definition}
\IfIpartialmappingUtoCw
then we define what it is for any particular derived rule $r$ of $U$ to be \term{consistently interpreted} by $I$: \\
\begin{enumerate}[(i)]
%\setlength\itemindent{2cm}
\item \underline{\textbf{T-rule}} 
Let $r_\Omega$ be any  derived T-rule, assume it to be of the form \ZOmega, for some $m \geq 0$, and let $r_m$ be the rule \IOmega{m}
then define $r_\Omega$ to be consistently interpreted by $I$ iff
\begin{enumerate}[(a)] 
\item $I(r_\Omega)$ and $I(r_m)$ are defined and $I(r_m) \base\, I(r_\Omega)$ in \catcw and 
\item $r_m$ is consistently interpreted by $I$  and
\item
for all contexts $Q$ and for all realisations $\tuple{\sm}$ of $\yOmega{m}$ wrt $Q$, 
if \foreachj, $I(r_{s_j})$ is defined 
and $I(r_{s_j}) \in Sect(I(r_{s_{j-1}})^*...I(r_{s_1})^*(\crossx{I(Q)}{I(r_{\Omega_j})}{1}))$,
where $r_{s_j}$ is the rule \IsOmega{j} 
and $r_{\Omega_j}$ is the rule \IOmega{j},
then
$$ \Imappedrule{Q}{\isT{\Omega[\SUBsFORy{m}]}} 
= I(r_{s_m})^*...I(r_{s_1})^*(\crossx{I(Q)}{I(r_\Omega)}{1})$$
. 
\end {enumerate}
\item \underline{\textbf{$\in$-rule}} 
Let $r_s$ be any derived $\in$-rule, assume it to be of the form \ZsOmega, for some $m \geq 0$, then
$r_s$ is defined to be consistently interpreted by $I$ \commentary{Do we need that $I(r_{\Omega_m}) \base I(r_\Omega)$?}
iff $I(r_{\Omega_m})$ is defined and $I(r_\Omega)$ is defined and $\displaystyle I(r_s) \in Sect(I(r_\Omega))$, where
$r_\Omega$ is the rule \ZOmega and $r_{\Omega_m}$ is the rule \IOmega{m}, and provided that
\begin{enumerate}[(a)]
\item in the case that $s$ is the variable $y_j$, \forsomej, so that $r_s$ is the rule \gatdisplayrule{\yOmega{m}}{\ofT{y_j}{\Omega}},
$$I(r_s) = s(p_{b_m,b_j})$$
Note that $s(p_{b_m,b_j})$ is a section of the object $\crossx{b_m}{b_j}{b_{j-1}}$
\footnote{
Implied, by lemma \lref{sofplemma}, that $s(p_{b_m,b_j}): b_m \morph \crossx{b_m}{b_j}{b_{j-1}}$ in \catc.
} % end footnote  
\commentary{By implication $I(r_\Omega)=\crossx{b_m}{b_j}{b_{j-1}}$}
where \foreachj, $b_j$ is the object
$\displaystyle\Imappedrule{\yOmega{j-1}}{\isT{\Omega_j}}$.  \commentary{and, btw: $s(id_{b_m})=\delta_{b_m}$}
Note that $p_{b_m,b_m}$ is defined to be $id_{b_m}$ and so in the case of $j=m$, $r_s$
is mapped to   $s(id_{b_m})$, and

\item in the case that $s$ is not simply a variable, 
for all contexts $Q$ and for all realisations $\tuple{\sm}$ of $\yOmega{m}$ wrt $Q$, 
such that \foreachj, $I(r_{s_j})$ is defined 
and $I(r_{s_j}) \in Sect(I(r_{s_{j-1}})^*...I(r_{s_1})^*(\crossx{I(Q)}{I(r_{\Omega_j})}{1}))$, 
where $r_{s_j}$ and $r_{\Omega_j}$ defined as above (in clause (i),
$$ \Imappedrule{Q}{\ofT{s[\SUBsFORy{m}]}{\Omega[\SUBsFORy{m}]}} = I(r_{s_m})^*...I(r_{s_1})^*(\crossx{I(Q)}{I(r_s)}{1}).$$
\end{enumerate}

\item \underline{\textbf{T=-rules}} 
If $r$ is the rule  \gatdisplayrule{\xDelta{n}}{\Delta = \Delta'}, for some $n \geq 0$, 
then $r$ is interpreted consistently by $I$ iff
both of the rules \ZDelta and \ZDeltap
are consistently interpreted by $I$ and
$$
\Imappedrule{\xDelta{n}}{\isT{\Delta}} = \Imappedrule{\xDelta{n}}{\isT{\Delta}}
$$
 
\item \underline{\textbf{$\in=$-rules}} 
If $r$ is the rule  \gatdisplayrule{\xDelta{n}}{t = t' \in \Delta}, for some $n \geq 0$, 
then $r$ is interpreted consistently by $I$ iff
both of the rules \ZtDelta and \gatdisplayrule{\xDelta{n}}{\ofT{t}{\Delta'}}
are consistently interpreted by $I$ and
$$
\Imappedrule{\xDelta{n}}{\ofT{t}{\Delta'}} = \Imappedrule{\xDelta{n}}{\ofT{t'}{\Delta'}}
$$
\end{enumerate}
\end{definition}
\highlight{END of DEFINITION OF consistently interpreted}

Since the above definition of a rule being consistently interpreted relies on whether other rules are consistently interpreted we need to check that there are no circularities:
\begin{lemma}
The above property of a rule being consistently interpreted by a mapping of rules is well-defined. 
\end{lemma}
\begin{proof}
The definition in the case of T-rules relies on the definition of other T-rules being consistently interpreted but these other rules are of lower rank and therefore there is no circularity 
(the definition terminates).
Neither are there circularities in the other cases because either they make no recursive references to consistent interpretation (the $\in$-rules case)
\commentary{watch out for $\in$-rule case
 changing}
or they make recursive references to T-rules (the T=-rule case) or to $\in$-rules (the $\in$=-rules case).
\end{proof}

\begin{definition}
An \term{instance} of a generalised algebraic theory $U$ in a contextual category \catcw is  any mapping 
of derived T-rules and $\in$-rules of the theory $U$ to objects, respectively sections of the contextual category \catcw that
consistently interprets every derived rule of $U$.
\end{definition}

\begin{lemma}
\llabel{towerlemma}
\IfIpartialmappingUtoC, if $r$ is a derived rule \ZDelta of $U$ and if
$r$ is consistently interpreted by $I$ 
then \foreachi, $r_{\Delta_i}$ is consistently interpreted by $I$ and 
$1 \base I(r_{\Delta_1}) \base ... \base I(r_{\Delta_1}) \base I(r)$ in \catc,
where, for each $i$, $r_{\Delta_i}$ is the rule \IDelta{i}.
\end{lemma}
\begin{proof}
Follows from clause (i)(b) of the definition of consistent interpretation.
\end{proof}

\begin{lemma}
\llabel{substitutionsublemma}
\IfIpartialmappingUtoC,
then if $r_s$ is the $\in$-rule \ZsOmega and 
$r_s$ is consistently interpreted by $I$ then
for all contexts $Q$ and for all realisations $\tuple{\sm}$ of $\yOmega{m}$ wrt $Q$  
such that \foreachj, $I(r_{s_j})$ is defined 
and $I(r_{s_j}) \in Sect(I(r_{s_{j-1}})^*...I(r_{s_1})^*(\crossx{I(Q)}{I(r_{\Omega_j})}{1}))$,
$$ \Imappedrule{Q}{\ofT{s[\SUBsFORy{m}]}{\Omega[\SUBsFORy{m}]}} 
= I(r_{s_m})^*...I(r_{s_1})^*(\crossx{I(Q)}{I(r_s)}{1})
\in Sect (I(r_{s_m})^*...I(r_{s_1})^*(\crossx{I(Q)}{I(r_\Omega)}{1})).$$
\end{lemma}
\begin{proof}
The definition of consistent interpretation directly requires this for non-variables. 
It remains to show  this in the case when $s$ is a variable. Suppose $s$ is the variable
$y_j$. Since in this case $s[\SUBsFORy{m}]$ is $s_j$ what we have to show is that
$I(s_j) = I(r_{s_m})^*...I(r_{s_1})^*(\crossx{I(Q)}{s(p_{b_m,b_j}}{1}))$. This is given by lemma \lref{cascadeprojectionlemma}.
\end{proof}


\begin{lemma}
\llabel{typeweakeninglemma}
\IfIpartialmappingUtoC,
if \ZOmega is a derived rule of $U$  which  is consistently interpreted by $I$ 
and if $Q$ is a context and $I(Q)$ is defined 
then $\Imappedrule{Q}{\isT{\Omega_1}}=\crossx{I(Q)}{\Imappedrule{}{\isT{\Omega_1}}}{1}$.
\end{lemma}
\begin{proof}
Since \ZOmega is consistently interpreted by $I$ then it follows by lemma \lref{towerlemma} that the rule 
\gatdisplayrule{}{\isT{\Omega}} (the rule with empty context and asserting $\Omega_1$ to be a type) is interpreted consistently
by $I$. Since $Q$ is a context and since the empty tuple $\tuple{}$ is a realisation of the empty context with respect to $Q$ then it follows from the definition of what it means for  \gatdisplayrule{}{\isT{\Omega}} to be consistently interpreted by $I$ that
$\Imappedrule{Q}{\isT{\Omega_1}}=\crossx{I(Q)}{\Imappedrule{}{\isT{\Omega_1}}}{1}$, as required.
\end{proof}
The following lemma establishes that if all the elements of a realisation are consistently interpreted by a mapping then the elements are 
mapped to a cacscade. 
\begin{lemma}
\llabel{realisationmapstocascade}
\commentary{use to be labelled substitutionpropertyvariant}
\newcommand {\forceSOURCEwidth}{\rule{5cm}{0pt}}  % so as to line up three different arrays
\newcommand {\forceTARGETwidth}{\rule{2.2cm}{0pt}}
If $I$ is a partial mapping of T-rules and $\in$-rules of a theory $U$ to objects, respectively sections of a contextual category \catcw,
 if $Q$  and $\encyOmega{m}$ are contexts, for some $m \geq 1$,  and if $\tuple{\sm}$ is a realisation of $\tuple{\yOmega{m}}$ with respect to $Q$,
 if \foreachj, the rule \IsOmega{j} is consistently interpreted by $I$ 
 and if the rule \IOmega{m} is consistently interpreted by $I$
then $\tuple{I(r_{s_1},...I(r_{s_m}}$ is a cacscade from $I(Q)$ to $I(r_{\Omega_m})$ in \catc,
where $r_{s_j}$ is the rule \IsOmega{j} and $r_{\Omega_j}$ is the rule \IOmega{j}.
\foreachj,  $$I(r_{s_j}) \in Sect(I(r_{s_{j-1}})^*...I(r_{s_1})^*(\crossx{I(Q)}{I(r_{\Omega_j})}{1}))$$
 where $r_{s_j}$ is the rule \IsOmega{j} and $r_{\Omega_j}$ is the rule \IOmega{j}.
\end{lemma}
\begin{proof}
It follows from the definition of cascade that we need to establish, \foreachj, that
 $$I(r_{s_j}) \in Sect(I(r_{s_{j-1}})^*...I(r_{s_1})^*(\crossx{I(Q)}{I(r_{\Omega_j})}{1})).$$
To prove this in the case of $m=1$ we need simply show that 
 $$I(r_{s_1}) \in Sect(\crossx{I(Q)}{I(r_{\Omega_1})}{1})).$$
 This follows because from the initial assumption that $r_{s_1}$ is consistently interpreted by $I$ then by definition
 I(Q) is defined and  $I(r_{s_1}) \in Sect(I\big(\gatdisplayrule{Q}{\isT{\Omega_1}}\big))$ 
 and by lemma \lref{typeweakeninglemma} $I(\gatdisplayrule{Q}{\isT{\Omega_1}}) = \crossx{I(Q)}{I(r_{\Omega_1})}{1})$.

Next we complete the proof by showing that the proposition that the lemma holds at a given $m$ follows from
the assumption that it holds at $m-1$. So assume that $Q$  and $\encyOmega{m}$ are contexts, 
that $\tuple{\sm}$ is a realisation of $\tuple{\yOmega{m}}$ with respect to $Q$,
assume that \foreachj, the rule \IsOmega{j} is consistently interpreted by $I$ 
and that the rule $r_{\Omega_m}$ is consistently interpreted by $I$. 
From these assumptions it follows
that  $\tuple{\sm[-1]}$ is a realisation of $\tuple{\yOmega{m-1}}$ with respect to $Q$.
Now since $r_{\Omega_m}$ is consistently interpreted by $I$ it follows, by definition,  that $r_{\Omega_{m-1}}$ is consistently interpreted by $I$. This means that we can use the inductive hypothesis to establish that \foreachj[m-1],  
$$I(r_{s_j}) \in Sect(I(r_{s_{j-1}})^*...I(r_{s_1})^*(\crossx{I(Q)}{I(r_{\Omega_j})}{1})).$$

Now it only remains to show that
$$I(r_{s_m}) \in Sect(I(r_{s_{m-1}})^*...I(r_{s_1})^*(\crossx{I(Q)}{I(r_{\Omega_m})}{1})).$$

This we can do because
$$I(r_{s_m}) \in Sect(\Imappedrule{Q}{\isT{\Omega_m[\SUBsFORy{m-1}]}},$$ since we are given that $r_{s_m}$ is consistently interpreted by $I$
and because, by invoking clause (c) of the definition of what it means for $r_{\Omega_m}$ to be consistently interpreted by $I$
we establish that
 $$\Imappedrule{Q}{\isT{\Omega_m[\SUBsFORy{m-1}]}}= I(r_{s_{m-1}})^*...I(r_{s_1})^*(\crossx{I(Q)}{I(r_{\Omega_m})}{1})$$
 since we have that $\tuple{\sm[-1]}$ is a realisation of $\tuple{\yOmega{m-1}}$ with respect to $Q$
 and since, as required for use of clause (c), we have already established that \foreachj[m-1],  
$$I(r_{s_j}) \in Sect\big(I(r_{s_{j-1}})^*...I(r_{s_1})^*\big(\crossx{I(Q)}{I(r_{\Omega_j})}{1}\big)\big).$$
 \end{proof}

\begin{lemma}
\llabel{substitutionlemma}
\IfIpartialmappingUtoC,
if we suppose that $Q$ and $\encyOmega{m}$ are contexts, for some $m >1$, and that $\tuple{\sm}$ is a realisation of $\encyOmega{m}$ wrt $Q$ and
suppose that \foreachj, the rule
\IsOmega{j} is consistently interpreted by $I$ then
\begin{enumerate}[(i)]
\item if \ZOmega is a derived rule which is consistently interpreted by $I$
then the derived rule 
\ZOmegaSUBsmFORym is consistently interpreted by $I$,
\item if \ZsOmega is a derived rule which is consistently interpreted by $I$
then the derived rule 
\ZsOmegaSUBsmFORym is consistently interpreted by $I$.
\end{enumerate}
\end{lemma}
\begin{proof}
Let $Q$ be the context $\xDelta{n}$. Let $r_{\Delta_i}$, \foreachi, be the rule \IDelta{i}.
\newcommand{\targetruleone}{\gatdisplayrule{\xDelta{n}}{\isT{\Omega[\SUBsFORy{m}]}}}
\newcommand{\Qtargetruleone}{\gatdisplayrule{\xDelta{n}}{\isT{\Omega[\SUBsFORy{m}]}}}
The rule \IDelta{n} is a derived rule because we have assumed that $Q$ is a context. 
\begin{enumerate}[(i)]
\item
First of all we show the required condition (a) that $I(r_{\Delta_n})$ 
and $\Imap{\Qtargetruleone}$ are defined and that 
$I(r_{\Delta_n}) \base \Imap{\Qtargetruleone}$ in \catc. 
First, $I(r_{\Delta_n})$ is defined because rule \gatdisplayrule{\xDelta{n}}{\ofT{s_1}{\Omega_1}} is consistently interpreted by $I$. 
By  lemma \lref{towerlemma} we have 
$I(r_{\Delta_1}) \base ... I(r_{\Delta_{n-1}}) \base I(r_{\Delta_n})$ in \catc.
Also because \foreachj,  we assume that the rule \gatdisplayrule{\xDelta{n}}{\ofT{s_j}{\Omega_j[\SUBsFORy{j-1}]}} is consistently interpreted by $I$, 
we have that $I(r_{s_j}) \in Sect(I(r_{s_{j-1}})^*...I(r_{s_1})^*(\crossx{I(r_{\Delta_n})}{I(r_{\Omega_j})}{1}))$, by lemma \lref{realisationmapstocascade}. Hence 
from the assumption that $r_\Omega$ is consistently interpreted it follows that $\Imap{\Qtargetruleone}$ is defined and 
\begin{equation}
\label{omegaSubsmapping}
\Imappedrule{\xDelta{n}}{\isT{\Omega[\SUBsFORy{m}]}} 
= I(r_{s_m})^*...I(r_{s_1})^*(\crossx{I(r_{\Delta_n})}{I(r_\Omega)}{1})
\end{equation}
and thus $I(r_{\Delta_n}) \base \Imap{\Qtargetruleone}$ in \catcw as required by clause (a).

Next, for clause (b), we are required to show that $r_{\Delta_n}$ is consistently interpreted by $I$. 
This follows from the fact that 
the rule \gatdisplayrule{\xDelta{n}}{\ofT{s_1}{\Omega_1}}  is consistently interpreted by $I$ 
and from the definition of consistently interpreted by $I$.

Finally, for clause (c), assume that $P$ is a context,
assume $\tuple{\tn}$ is a realisation of $\xDelta{n}$ with respect to $P$, 
assume \foreachi, $I(r_{t_i})$ is defined and $I(r_{t_i}) \in Sect(I(r_{t_{i-1}})^* ... I(r_{t_1})^*\crossx{I(P)}{I(r_{\Delta_i})}{1}$ 
where $r_{t_i}$ is the rule $\tDelta{i}$, 
we must show that
$$\Imappedrule{\xDelta{n}}{\isT{\Omega[\SUBsFORy{m}][\SUBtFORx{n}]}}
                =I(r_{t_n})^* ... I(r_{t_1})^*\big(\crossx{I(P)}{\Imappedrule{\xDelta{n}}{\isT{\Omega[\SUBsFORy{m}]}}}{1}\big)$$
By rearrangement of the lhs and by use (\ref{omegaSubsmapping}) this means that we need to show 
\begin{equation}
\label{ctarget}
\Imappedrule{\xDelta{n}}{\isT{\Omega[s_1[\SUBtFORx{n}]|y_1,... s_m[\SUBtFORx{n}]|y_m]}}
                =I(r_{t_n})^* ... I(r_{t_1})^*\big(\crossx{I(P)}{I(r_{s_m})^*...I(r_{s_1})^*(\crossx{I(r_{\Delta_n})}{I(r)}{1})}{1}\big)
\end{equation}

\newcommand{\IOmegaDoublySubstituted}[1]{\Omega_#1[\SUBsFORy{#1-1}][\SUBtFORx{n}]}
Each $r_{s_j}$ is consistently interpreted by $I$ and $\tuple{\tn}$ is a suitably mapped realisation
of the context $\xDelta{n}$ with respect to $P$ and therefore
$$\Imappedrule{P}{\ofT{s_j[\SUBtFORx{n}]}{\IOmegaDoublySubstituted{j}}}
           = I(r_{t_n})^* ... I(r_{t_1})^*\big(\crossx{I(P)}{I(r_{s_j})} {1}\big)$$


$$           \in Sect\Big(I(r_{t_n})^* ... I(r_{t_1})^*\big(\crossx{I(P)}{\Imappedrule{\xDelta{n}}{\isT{\Omega_j[\SUBsFORy{j-1}]}}} {1}\big)\Big)$$
$$           = Sect(I(r_{t_n})^* ... I(r_{t_1})^*\big(\crossx{I(P)}{ I(r_{s_{j-1}})^*...I(r_{s_1})^*(\crossx{I(Q)}{I(r_{\Omega_j})}{1})} {1}\big)), \mbox{by a $j$'th variant of (\lref{omegaSubsmapping})}$$
$$           =  Sect\Big(
               \big(I(r_{t_n})^*...I(r_{t_1})^*(\crossx{I(r_{\Delta_n})}{I(r_{s_{j-1}})}{1})\big)
               ^* ... 
              \big(I(r_{t_n})^*...I(r_{t_1})^*(\crossx{I(r_{\Delta_n})}{I(r_{s_1})}{1})\big)
               ^*\big(\crossx{I(P)}{I(r_{\Omega_j})}{1}\big)
               \Big) \mbox{by lemma \lref{cascadedpullbackscohere} }
$$
and so 
$\tuple{s_1[\SUBtFORx{n}],...s_m[\SUBtFORx{n}]}$ is a suitably mapped realisation of $\yOmega{m}$ with repect to $P$
and we can use the fact that $r_\Omega$is consistently interpreted by $I$ 
to establish that
\begin{multline}
     \Imappedrule{\xDelta{n}}{\isT{\Omega\big[s_1[\SUBtFORx{n}]|y_1,... s_m[\SUBtFORx{n}]|y_m\big]}} \\
           =  \big(I(r_{t_n})^*...I(r_{t_1})^*(\crossx{I(r_{\Delta_n})}{I(r_{s_m})}{1})\big)
               ^* ... 
              \big(I(r_{t_n})^*...I(r_{t_1})^*(\crossx{I(r_{\Delta_n})}{I(r_{s_1})}{1})\big)
               ^*\big(\crossx{I(P)}{I(r_\Omega)}{1}\big)
\end{multline}
which, by lemma \lref{cascadedpullbackscohere}, means that we have established
equation (\ref{ctarget}), as required.

and so clause (iii) follows. \commentary{clause (iii) of what}\commentary{enumerate here with only item (i)}
\item \tbd
\end{enumerate}
\end{proof}
We can show that an instance $I$ of a generalised algebraic theory $U$ in a contextual category \catcw is
completely determined by its mapping of the introductory rules of sort symbols and operator symbols to
objects, respectively, sections of \catc. In order to show this we start with this definition:

\begin{definition}
If $U$ is a generalised algebraic theory  and if \catcw is a contextual category then
a \term{interpretation} $\iI$ of  $U$ in \catcw consists of a pair :
\begin{itemize}
\item a mapping $\Isort$ that maps each sort symbol of $U$ to  an object of \catc,
\item a mapping $\Iop$ that maps each operator symbol of $U$ to a section of \catcw (i.e. to a morphism $f: A \morph B$ for some 
$A \base B$ in \catcw such that $f \circ p_B=id_A$).
\end{itemize}
\end{definition}


We will say that  an interpretation $\iI$ of $U$ in \catcw \term{determines} an  instance $J$ of $U$ in \catcw to mean that for each sort symbol $A$ of $U$,
$J(r_A) = I_{sort}(A)$, where $r_A$ is the introductory rule for $A$ and that for each operator symbol
$f$ of $U$,   $J(r_f) = I_{op}(f)$, where $r_f$ is the introductory rule for $f$.


\begin{definition} [\highlight{Definition of  $\Ibar$}]
If $\iI$ is an interpretation of generalised algebraic theory $U$ in contextual catgeory \catcw
then define a
partial mapping $\Ibar$  of T-rules and $\in$-rules to objects, respectively sections, of \catcw
as follows
\begin{enumerate}[(i)] 
\item \underline{\textbf{T-rules}} 
The derivation lemma (lemma \ref{derivationlemma}) tells us that if $r_\Delta$ is a derived T-rule of $U$  then it is of the form \gatdisplayrule{P}{\isT{A(t_1,...t_n)}} for some premise $P$, for some $n \geq 0$ and for some sort symbol $A$ with introductory rule $r_A$ of the form \gatdisplayrule{\xDelta{n}}{\isT{A(x_1,...x_n)}} where \foreachi, the rule 
\ItDelta[P][,]{i} which we will call $r_{t_i}$, is a derived rule of $U$. 
We define $\Ibar(r_\Delta)$ to be undefined unless the following preconditions are met:
\begin{enumerate}[(a)]
\item
the context  $P$ is mapped by $\Ibar$ to some object $a$ of \catcw and 
\item
\foreachi, the context $\xDelta{i}$ is mapped by $\Ibar$ to some object $b_i$ of \catcw
and $1 \base b_1,...\base b_n \base I(A)$ in \catcw and
\item
$\Ibar(r_{t_i})$ is defined and $\Ibar(r_{t_i}) \in Sect(\Ibar(r_{t_{i-1}})^*...\Ibar(r_{t_1})^*(\crossx{a}{b_i}{1})$, \foreachi,
\end{enumerate}
in which case we define $\Ibar(r_\Delta)$ to be $\Ibar(r_{t_n})^*...\Ibar(r_{t_1})^*(\crossx{a}{I(A)}{1})$. 
\item \underline{\textbf{$\boldsymbol {\in}$-rules}} 
The derivation lemma also tells us that if $r_t$ is an $\in$-rule then it is  
either \highlight{(1)} of the form \gatdisplayrule{\xDelta{n}}{\ofT{x_i}{\Omega}} for some $n \ge 1$, for some $i$, $1 \leq i \leq n$, 
and for some $\Omega$ such that \gatdisplayrule{\xDelta{n}}{\Delta_i=\Omega} is a derived rule of $U$
or \highlight{(2)} it is of the form \gatdisplayrule{P}{\ofT{f(t_1,...t_n)}{\Omega}} for some premise $P$, for some $n \geq 0$ and for some operator symbol $f$ with introductory rule $r_f$ of the form \gatdisplayrule{\xDelta{n}}{\ofT{f(x_1,...x_n)}{\Delta}} where \foreachi, the rule 
\ItDelta[P]{i}, which we will call $r_{t_i}$, is a derived rule of $U$. 

In  case (1) we define $\Ibar(r_t)$ to be undefined unless  the following preconditions are met:
\begin{enumerate}
\item
\foreachi, the context $\xDelta{i}$ is mapped by $\Ibar$ to some object $a_i$ of \catcw such
that $1 \base a_1,...\base a_i$ and
\item the rule \gatdisplayrule{\xDelta{n}}{\isT{\Omega}}, which we call $r_\Omega$, is mapped by $\Ibar$ to the object $\crossx{a_n}{a_i}{a_{i-1}}$
\end{enumerate}
in which case we define $\Ibar(r_t)$ to be $s(p_{a_n,a_i})$ . 

In  case (2) 
we define $\Ibar(r_t)$ to be undefined unless  unless the following preconditions are met:
\begin{enumerate}
\item
the context  $P$ is mapped by $\Ibar$ to some object $a$ of \catcw and 
\item
\foreachi, the context $\xDelta{i}$ is mapped by $\Ibar$ to some object $b_i$ of \catcw
and 
\item the rule \ZDelta is mapped by $\Ibar$ to some object $b$ of \catcw and
$1 \base b_1,...\base b_n \base b$ in \catcw 
and 
\item
\foreachi, $\Ibar(r_{t_i}) \in Sect(r_{t_{i-1}}^*...r_1^* (\crossx{a}{b}{1}))$
\end{enumerate}
in which case we
define $\Ibar(r_t)=\Ibar(r_{t_n})^*...\Ibar(r_{t_1})^*(\crossx{a}{I(f)}{1})$.
This completes the definition of $\Ibar$.
\end{enumerate}
\highlight{END OF Definition of  $\Ibar$}
\end{definition}


\begin{lemma}
\llabel{Ibartowerlemma}
If $\iI$ is an interpretation of generalised algebraic theory $U$ in contextual category \catcw,
Suppose that  $r_\Omega$ is a derived T-rule \ZOmega of a generalised algebraic theory $U$
and suppose that \foreachj, $r_{\Omega_j}$ is the derived rule
\IOmega{j},
if $\iI$ is an interpretation of $U$ in a contextual category \catc
such that $\Ibar(r_Omega)$ is defined then
$\Ibar(r_j)$ is defined \foreachj, and 
$\Ibar(r_1) \base  ... \base \Ibar(r_m) \base \Ibar(r)$ in \catc.
\end{lemma}
\begin{proof}
It follows from the definition of $\Ibar$ that since $\Ibar(r)$ is defined that $\Ibar(r_m)$ is defined and that
$\Ibar(r_m) \base \Ibar(r)$ in \catc. Now we can repeat and argue that $\Ibar(r_{m-1})$ is defined and that $\Ibar(r_{m-1}) \base \Ibar(r_m)$
in \catc. By induction $\Ibar(r_1) \base  ... \base \Ibar(r_m) \base \Ibar(r)$ in \catc, as required.
\end{proof}

\begin{lemma} 
If $\iI$ is an interpretation of generalised algebraic theory $U$ in contextual catgeory \catcw then then if $\iI$ determines an instance of $U$ then the
mapping $\Ibar$  is total i.e. is defined for all T- and $\in$-rules and the instance determined by $\iI$ is $\Ibar$.
\end{lemma}
\begin{proof}  
Use the lemma regarding the interpretation of substitutions (lemma \ref{substitutioninterpretation})...\commentary{expand}
\end{proof}

The condition under which an interpretation of $U$ in \catcw determines an instance i.e that the rule mapping $\Ibar$ is total and respects the axioms can be summarised as being that the interpretation\commentary{slight reword needed} needs be type correct and to satisfy the axioms of the theory. The definition of exactly what we mean by this has to proceed by induction because, for example, we need an instance of the rule
\ZDelta (as an object $a$ of \catc, say) before we can say whether an interpretation of an operator with introductory rule \genericfintroductoryrule
is well-typed (i.e. to know that it is object $a$ that it is required to be a section of).
Similarly we need to be able to interpret both sides of an axiom before we can say whether it is respected
by the interpretation. For this induction to proceed we need an intermediate definition:

\def\restrict{\mathbin{\restriction}}
\newcommand{\predInstance}{\overline{I \restrict U_p}}
\newcommand{\Uincrement}{U \setminus\kern-2pt U_p}

\begin{definition}
 Suppose that $\iI$ is an interpretation of $U$  
define interpretation $\iI$ to be \term{valid}  iff 

\begin{enumerate}[(i)]
\item
for all sort symbols $A$ of $U$ if $A$ has introductory rule $r_A$ then $\Ibar(r_A)$ is defined,
\item  
for all operator symbols $f$ of $U$ if $f$ has introductory rule $r_f$ then $\Ibar(r_f)$ is defined,

\item \underline{\textbf{T=-axioms}} 
for all axiomsof the form
 \gatdisplayrule{\xDelta{n}}{\Delta = \Delta'},
$\Ibar(r)$ is defined and $\Ibar(r')$ is defined and
$\Ibar(r) = \Ibar(r')$ where $r$ is the rule
\ZDelta and  
and $r'$ is the rule \ZDeltap

\item \underline{\textbf{$\boldsymbol{\in=}$-axioms}} 
for all axioms  of the form
\gatdisplayrule{\xDelta{n}}{t = t' \in \Delta},
$\Ibar(r_t)$ is defined and  $\Ibar(r'_t)$ is defined and
$\Ibar(r_t) = \Ibar(r'_t)$ where $r_t$ is the rule
\ZtDelta and  
and $r'_t$ is the rule \gatdisplayrule{\xDelta{n}}{\ofT{t'}{\Delta'}}.
\end{enumerate}
\end{definition}

Now for the main lemma:
\begin{lemma}
\llabel{avalidinterpretationisaninstance}
Suppose that $\iI$ is an interpretation of $U$  then if $\iI$ is valid  then $\Ibar$ is an instance.
\end{lemma}
\begin{proof} 
\newcommand {\forceSOURCEwidth}{\rule{5cm}{0pt}}  % so as to line up three different arrays
\newcommand {\forceTARGETwidth}{\rule{2.2cm}{0pt}}
We need to show that for every derived rule $r$ of $U$, $\Ibar(r)$ is defined and $r$ is consistently interpreted\footnote{To be formally correct we need a definition of \textit{consistently interpreted} for partial mappings of rules not just total mappings -- since here at the point we use the definition we have not yet proved $\Ibar$ to be total.} by $\Ibar$. 
We prove this by induction on the derivation of rules in $U$. We examine each of the principles of derivation in turn
and show that given the stated assumptions then from rules for which $\Ibar$ is defined and which are consistently interpreted by $\Ibar$ 
it is only possible to derive rules for which $\Ibar$ is defined and which themselves are consistently interpreted by $\Ibar$.
The principles of derivation (see \cite{Cartmell86}) are LI1, ... LI7, T1, CF1, CF2(a) and CF2(b), SI1 and SI2. 
The proof of this  in each the cases  LI1,...LI6 is quite trivial. We consider each of the remaining principles in turn. 
 \\
\underline{LI7} This is not quite trivial but related to principle T1. \highlight{Think about this.} \\


\underline{T1}
By this principle from \gatdisplayrule{P}{\Delta=\Delta'} and \gatdisplayrule{P}{\ofT{t}{\Delta}} we can derive \gatdisplayrule[.]{P}{\ofT{t}{\Delta'}}
By the inductive hypothesis we assume that rule  \gatdisplayrule{P}{\Delta=\Delta'} is consistently interpreted by $\Ibar$ i.e. that
$\Ibar$ is defined at both $r_\Delta$ and $r_{\Delta'}$  and that $\Ibar(r_\Delta) = \Ibar(r_{\Delta'})$
and we also assume that $\Ibar$ is defined at $r_{t\Delta}$ and that this rule
is consistently interpreted by $\Ibar$ i.e. that $\Ibar(r_{t\Delta}) \in Sect(I(r_{\Delta}))$.
We have to show that $\Ibar$ is defined at the rule $r_{t\Delta'}$ and that this rule, $r_{t\Delta'}$, is consistently interpreted 
by $\Ibar$ i.e. that $\Ibar(r_{t\Delta'}) \in Sect(I(r_{t\Delta'}))$. 
In addition, in the case that  $t$ is simply a variable we are required to show  that $\Ibar(r_{t\Delta'})$ has an appropriate value.

Now from the definition of $\Ibar$ it can be seen that providing that the preconditions are met then  the value of $\Ibar$ at $r_{t\Delta'}$  is the same
as the value at $r_{t\Delta}$ (since the definition of the value is independent of the type -- $\Delta$ or $\Delta'$) and therefore that the condition(s) for the consistent interpretation of $r_{t\Delta'}$ are met 
since we have that $Sect(I(r_{t\Delta'}))=Sect(I(r_{t\Delta}))$ as we have assumed that $I(r_{t\Delta'})=I(r_{t\Delta})$. 
Therefore we simply need to show that the preconditions
for $\Ibar$ being defined at $r_{t\Delta'}$ hold.
From examination of the preconditions for $\Ibar$ to be defined at $r_{t\Delta'}$ we see that, whether or not $t$ is simply a variable, 
each precondition is either independent of the type (i.e. $\Delta'$) in which case must hold,  as we have assumed that 
$\Ibar(r_{t\Delta})$ be defined,
 or  else  follows directly from our assumption that 
$\Ibar$ is defined both at $r_\Delta$  and $r_{\Delta'}$ and that $\Ibar(r_\Delta) = \Ibar(r_{\Delta'})$.\\


\underline{CF1} According to the principle, whenever a rule of the form \gatdisplayrule{\xDelta{n}}{\isT{\Delta_{n+1}}} is a derived rule of $U$ 
then so to is the rule \gatdisplayrule{\xDelta{n+1}}{\ofT{x_i}{\Delta_{i}}}, \foreachi[n+1], where $x_{n+1}$ is any variable distinct from each of the variables $\xn$.
We have to show that if $\Ibar$ is defined for and consistently interprets any such T-rule, which we will denote, $r_{\Delta_{n+1}}$, 
then so to is $\Ibar$ defined for and consistently interprets each associated rule $r_{x_i}$. 
We show that  if it is the case for $r_{x_j}$ all $j <i$ 
then it follows that it is the case for each rule \gatdisplayrule{\xDelta{n+1}}{\ofT{x_i}{\Delta_{i}}}, which we shall denote  $r_{x_i}$. 
In so doing we establish that it holds for all $i \leq n +1 $, as required. \commentary{pull through new conditions for consistent interpretation}

So, suppose that for each $j$, $j < i$, that $\Ibar$ is defined for and consistently interprets the rule $r_{x_j}$ i.e. suppose that  $\Ibar(r_{x_j})$ is defined and that
$\Ibar(r_{x_j})=s(p_{a_{n_1},a_j})$. 
We shall now show that it follows that 
$\Ibar$ is defined at the rule $r_{x_i}$ and that this rule is consistently interpreted by $\Ibar$ by showing that
$\Ibar(r_{x_i})=s(p_{a_{n_1},a_i})$. 

%Denote the former rule $r_{\Delta_{n+1}}$ and the latter rule $r_{x_i}$. 
First we have to show that the preconditions for $\Ibar$ to be defined at the rule $r_{x_i}$ hold. These are:

\begin{enumerate}
\item That \foreachi, the context $\xDelta{i}$ is mapped by $\Ibar$ to some object $a_i$ of \catcw such
that $1 \base a_1,...\base a_i$. This follows from lemma \lref{Ibartowerlemma}. \commentary{\highlight{CHECK}}

\item That the rule \DDelta{n+1}{i} is mapped by $\Ibar$ to the object $\crossx{a_{n+1}}{a_i}{a_{i-1}}$.

This we can show as follows.
\newcommand{\deltaimapped}{\crossx{a_{n+1}}{a_i}{a_{i-1}}}
\newcommand{\deltaimappedlong}{s(p_{a_{n+1},a_{i-1}})^*...s(p_{a_{n+1},a_1})^*(\crossx{a_{n+1}}{a_i}{1})}
Let $r_{\Delta_i}$ be the rule \IDelta{i}.



Because $\Ibar(r_{\Delta_i}) = a_i$,
because  we have that
$\tuple{x_1,...x_{i-1}}$ is a realisation of context $\xDelta{i-1}$ with respect to context $\Delta_{n+1}$ 
and since we have assumed for each $j<i$ that $r_{x_j}$ is consistently interpreted by $\Ibar$ and in particular
that
\begin{equation*}
\begin{array}{c c c }
\forceSOURCEwidth & & \forceTARGETwidth \\ [-0.1cm]
\gatdisplayrule{\xDelta{n+1}}{\ofT{x_j}{\Delta_j}}  & \Imapsto & s(p_{a_{n+1},a_j}) \\ [0.4cm]
\end{array}
\end{equation*}
and since we have assumed that $r_{\Delta_i}$ is consistently interpreted by $\Ibar$ and in particular that clause (ii)(b) of the definition of consistently interpreted holds
then we have that
\begin{equation*}
\begin{array}{c c c}
\forceSOURCEwidth & & \forceTARGETwidth \\ [-0.1cm]
\gatdisplayrule{\xDelta{n+1}}{\isT{\Delta_i[x_1|x_1,...x_{i-1}|x_{i-1}]}}  & \Imapsto & \deltaimappedlong \\ [0.4cm]
\end{array}
\end{equation*}
i.e.
\begin{equation*}
\begin{array}{c c c}
\forceSOURCEwidth & & \forceTARGETwidth \\ [-0.1cm]
\DDelta{n+1}{i}  & \Imapsto & \deltaimappedlong.\\ [0.4cm]
\end{array}
\end{equation*}
So we are done if we can prove that

\begin{equation*}
\deltaimappedlong = \deltaimapped
\end{equation*}  
This is given by lemma \lref{sofpsubstitutionlemma}
\end{enumerate}
Next we have to show that 
By showing that the reklevant preconditions hold we have shown that $I(r_{x_i})$ is defined. 
To show that the rule $r_{x_i}$ is consistently interpreted by $\Ibar$ we just have to show that
 $\Ibar(r_{x_i})=s(p_{a_{n+1},a_i})$. But this is exactly the definition of $\Ibar$ at $r_{x_i}$. 
\begin{newtt}and that $\Ibar(r_{\Delta_i})=\crossx{a_n}{a_i}{1}$ which we have shown in 1. above.\end{newtt} \\
\underline{CF2(a)}
By this principle, from a sort symbol $A$ with introductory rule $r_A$ of the form \gatdisplayrule[,]{\xDelta{n}}{\isT{A(\xn)}} for some $n \geq 0$, if
$P$ is a context and if in particular the rule $r_P$ asserting that $P$ is a context is a derived rule of $U$
and from derived rules $r_{t_i}$ of the form \ItDelta[P]{i}, \foreachi, we may deduce
the rule \gatdisplayrule{P}{\isT{A(t_1,...t_n)}}, which we shall denote by $r$, is a derived rule of $U$. \\

\noindent Assume as the inductive hypothesis that $\Ibar(r_{t_i})$ is defined \foreachi. We can show that $\Ibar(r)$ is defined providing we can show that conditions (a), (b) and (c) of part (i) of the definition of $\Ibar$ are met. \\
\noindent Precondition (a), that $P$ is mapped by $\Ibar$ to some object $b$ \highlight{$b_m$} of \catc, follows from the inductive hypothesis applied to  rule $r_P$ asserting that $P$ is a context.  \\
\noindent Precondition (b) follows from the assumption that $\Ibar$ is relatively well-typed because rule $r$ is well-typed relative to the theory
$U_p$ because $U$ is a generalised algebraic theory. 
\noindent Therefore $r_{\Delta_n}$ is a derived rule of $U_p$\\ and so $\predInstance(r_{\Delta_n})$ is defined  because $\predInstance$ is assumed to be total. 
Therefore $\Ibar(r_{\Delta_n})$ is defined since
$\Ibar$ extends $\predInstance$ since $\iI$ extends $I \restrict U_p$. 
\noindent  Precondition (c) follows firstly, because that \foreachi, $\Ibar(r_{t_i})$ is defined, is given by the inductive hypothesis and 
secondly, that $\Ibar(r_{t_i}) \in Sect(\Ibar(r_{t_{i-1}})^*...^*\Ibar(r_{t_{1}})^*(\crossx{b_m}{a_i}{1}))$ follows  from 
the inductive hypothesis {CHECK}.\\

\begin{newtt}
\highlight{Next show the substitution operates as it should.} 
                                                              \commentary{ Need unify the above and the below.} \\

\newcommand{\clausethreelhs}{(\gmvectorstar (\crossx{c}{f_n}{1}))^* ... (\gmvectorstar (\crossx{c}{f_1}{1})) ^* (\crossx{b_m}{I_A}{1})}
\newcommand{\clausethreerhs}{\gmvectorstar (  \crossx{c}{(\fnvectorstar(\crossx{b_m}{I_A}{1}))}{1} )}
\newcommand{\tirule}{\gatdisplayrule{\yOmega{m}}{\ofT{t_i}{\Delta_i[t_1 | x_1,...t_{i-1}|x_{i-1}]}}}
Now assume that $P$ is the context $\yOmega{m}$, for some $m \geq 0$, so that $r$ is the rule \gatdisplayrule{\yOmega{m}}{\isT{A(\tn)}}
and suppose $r$  is mapped by $\Ibar$ to some object $b$ of \catcw such that $1 \base b_1 ... \base b_m \base b$. 

Now there are sections  $f_1,..f_n$  of \catcw such that, \foreachi, $\Ibar$ maps
\begin{equation}
\label{timapping}
\begin{array}{c c c}
\forceSOURCEwidth & & \forceTARGETwidth \\ [-0.1cm]
\tirule    & \Imapsto & f_i \\ [0.4cm]
\end{array}
\end{equation} 
Now from the definition of $\Ibar$, we have that $\Ibar$ maps
\begin{equation*}
\begin{array}{c c c}
\forceSOURCEwidth & & \forceTARGETwidth \\ [-0.1cm]
\gatdisplayrule{\yOmega{m}}{\isT{A(\tn)}}   & \Imapsto & \fnvectorstar(\crossx{b_m}{I_A}{1})\\ [0.4cm]
\end{array}
\end{equation*} 

Assume also  that $Q$ is a context and that $\tuple{\sm}$ is a realisation of $\tuple{\yOmega{m}}$ with respect to $Q$
and that there is the following mapping by $\Ibar$ into the objects and sections of \catc:
\begin{equation}
\label{sjmapping}
\begin{array}{c c c}
\forceSOURCEwidth & & \forceTARGETwidth \\ [-0.1cm]
Q          & \Imapsto & c   \\ [0.4cm]
\IsOmega{j}    & \Imapsto & g_j \\ [0.4cm]
\end{array}
\end{equation}
where $g_j \in Sect(g_{j-1}^*...g_i^*(\crossx{c}{b_j}{1}))$ in \catc.

We have to establish that $\Ibar$ maps
\begin{equation}
\label{clauseiiirequiredmapping}
\begin{array}{c c c}
\forceSOURCEwidth & & \forceTARGETwidth \\ [-0.1cm]
\gatdisplayrule{Q}{\isT{A(t_1[\SUBsFORy{m}],...t_n[\SUBsFORy{m}])}}   & \Imapsto & \gmvectorstar (\crossx{c}{\fnvectorstar(\crossx{b_m}{I_A}{1})}{1}) \\ [0.4cm]
\end{array}
\end{equation}
Now, we deduce from \ the inductive hypothesis that $r_{t_i}$ is consistently interpreted and 
from lemma \lref{substitutionsublemma} from (\ref{timapping}) and from (\ref{sjmapping}) that \foreachi, $\Ibar$ maps
\begin{equation}
\begin{array}{c c c}
\forceSOURCEwidth & & \forceTARGETwidth \\ [-0.1cm]
\gatdisplayrule{Q}{\ofT{t_i[\SUBsFORy{m}]}{\Delta_i[t_1 | x_1,...t_{i-1}|x_{i-1}][\SUBsFORy{m}]}}  & \Imapsto & \gmvectorstar (\crossx{c}{f_i}{1}) \\ [0.4cm]
\end{array}
\end{equation}
and therefore from the definition of $\Ibar$ that it maps
\begin{equation*}
\begin{array}{c c c}
\forceSOURCEwidth & & \forceTARGETwidth \\ [-0.1cm]
\gatdisplayrule{Q}{\isT{A(t_1[\SUBsFORy{m}],...t_n[\SUBsFORy{m}])}}  & \Imapsto & \clausethreelhs \\ [0.4cm]
\end{array}
\end{equation*}
and therefore to establish (\ref{clauseiiirequiredmapping}) we need to show that
\begin{equation*}
\clausethreelhs = \clausethreerhs										
\end{equation*}
This we can do by repeated application of identities (\ref{metagattriplestar})
and (\ref{metagatcrossstarcross}), which are two of the meta-GAT axioms described earlier.\\
\end{newtt}

\underline{CF2(b)???} \\

\underline{SI1} 
This principle states that if \gatdisplayrule{\yOmega{m}}{\Omega=\Omega'} is a derived rule, if $Q$ is a context and if  $\sm$ and $\smp$ are expressions such that
\foreachj, \gatdisplayrule{Q}{s_j=s'_j \in \Omega\SUBsFORy{j-1}} is a derived rule then we may derive the rule
\gatdisplayrule{Q}{\Omega[\SUBsFORy{m}]=\Omega'[s'_1|y_1...s'_m|y_m]}. \\

We have to show that this latter rule is consistently intepreted by $\Ibar$ \highlight{CHECK} from the assumption that each of the rules \gatdisplayrule{\yOmega{m}}{\Omega=\Omega'} and
\gatdisplayrule{Q}{s_j=s'_j \in \Omega\SUBsFORy{j-1}}, \foreachj, are consistently intepreted by $\Ibar$.

To do this we have to show that the rules \ZOmegaSUBsmFORym and \gatdisplayrule{Q}{\isT{\Omega'[s'_1|y_1...s'_m|y_m]}}
are consistently interpreted by $\Ibar$ and that
$$
\ibarmappedrule{Q}{\isT{\Omega[\SUBsFORy{m}]}} = \ibarmappedrule{Q}{\isT{\Omega'[s'_1|y_1...s'_m|y_m]}}.
$$
From the assumptions that we have made and from the definition of what it means for a rule to be consistently interpreted it follows that
each of the rules
\ZOmega
and
\gatdisplayrule{\yOmega{m}}{\isT{\Omega'}}
and each of the rules
\IsOmega{j}
and
\gatdisplayrule[,]{Q}{\sjpconclusion}
\foreachj, are consistently intepreted by $\Ibar$. Therefore from lemma \lref{substitutionlemma} 
the first part of what we have to prove, that the rules \ZOmegaSUBsmFORym and \gatdisplayrule{Q}{\isT{\Omega'[s'_1|y_1...s'_m|y_m]}}
are consistently interpreted by $\Ibar$, follows.
Finally we can show that
$$
\ibarmappedrule{Q}{\isT{\Omega[\SUBsFORy{m}]}} = \ibarmappedrule{Q}{\isT{\Omega'[s'_1|y_1...s'_m|y_m]}}.
$$.

\begin{newtt}
$$ \gmstar...\gonestar(\crossx{a}{\ibarmappedrule{\yOmega{m}}{\ofT{s}{\Omega}}}{1})$$
where \foreachj, $g_j$ is the section
$\displaystyle\Imappedrule{Q}{ \ofT{s_j}{\Omega_j \SUBsFORy{j-1}}}$
and where $a$ is the object to which $I$ maps context $Q$.

\end{newtt}

because  from the the fact  that \ZOmegaSUBsmFORym is consistently interpreted by $\Ibar$ we have that
$$
\Imappedrule{Q}{\isT{\Omega[\SUBsFORy{m}]}} = \gmstar...\gonestar(\crossx{a}{\Imappedrule{\yOmega{m}}{\isT{\Omega}}}{1})
$$
where \foreachj, $g_j$ is the section
$\displaystyle\Imappedrule{Q}{ \ofT{s_j}{\Omega_j \SUBsFORy{j-1}}}$
and where $a$ is the object to which $\Ibar$ maps context $Q$,
and similarly we have that
$$
\ibarmappedrule{Q}{\isT{\Omega'[s'_1|y_1...s'_m|y_m]}} = \gprimemstar...\gprimeonestar(\crossx{a}{\Imappedrule{\yOmega{m}}{\isT{\Omega'}}}{1})
$$

where, \foreachj,  
$g_j$ is the section 
and $g'_j$ is the section 
and because \foreachj, $g_j=g'_j$ from the assumption
that rule \gatdisplayrule{Q}{s_j=s'_j \in \Omega\SUBsFORy{j-1}} is consistently interpreted by $\Ibar$
and since
$$
\ibarmappedrule{\yOmega{m}}{\isT{\Omega}} = \ibarmappedrule{\yOmega{m}}{\isT{\Omega'}}
$$
from the assumption that \gatdisplayrule{\yOmega{m}}{\Omega=\Omega'} is consistently interpreted by $\Ibar$.


\vspace{1cm}
\underline{SI2} 
This principle states that if \gatdisplayrule{\yOmega{m}}{s = s' \in \Omega} is a derived rule, if $Q$ is a context and if  $\sm$ and $\smp$ are expressions such that
\foreachj, \gatdisplayrule{Q}{s_j=s'_j \in \Omega\SUBsFORy{j-1}} is a derived rule then we may derive the rule
\gatdisplayrule{Q}{s[\SUBsFORy{m}]=s'[s'_1|y_1...s'_m|y_m] \in \Omega[\SUBsFORy{m}]}. \\
\vspace{1cm}

\underline{A1} 
This principle ensures that an T=axiom is a derived rule providing it is well-typed.
 
It states that from an axiom \gatdisplayrule{\xDelta{n}}{\Delta=\Delta'} and from derived rules
 \ZDelta and \ZDeltap we may derive
\gatdisplayrule{\xDelta{n}}{\Delta=\Delta'}.

\vspace{1cm}
Follows immeadiately from the fact fact that $\Ibar$ is  valid.\\

\underline{A2} 
This principle ensures that an $\in$=axiom is a derived rule providing it is well-typed.
It states that from an axiom \gatdisplayrule{\xDelta{n}}{t=t' \in \Delta} and from derived rules
 \ZtDelta and \ZtpDelta we may derive
\gatdisplayrule{\xDelta{n}}{t=t' \in \Delta}.

\vspace{1cm}
Follows immeadiately from the fact that $\Ibar$ is  valid. \\

\end{proof}

\begin{lemma}
\llabel{Xnlemma}
If $\iI$ is an interpretation of a generalised algebraic theory $U$ in a contextual category \catcw and if $X$ is an absolute sort symbol of $U$ which is mapped 
by $\iI$ to an object $X$ of \catcw (so that $1 \base X$ in \catc) then for any $n \geq 1$ 
\begin{enumerate}[(i)]
\item
The context $\tuple{\ofT{x_1}{X},...\ofT{x_n}{X}}$ is mapped by $\iI$ to the object $X^n$ of \catc,
\item the rule 
\gatdisplayrule{\ofT{x_1}{X},... \ofT{x_n}{X}}{\ofT{x_i}{X}} is mapped by $\iI$ to the section $s(p_i)$ of $X^{n+1}$, where $p_i$ is the $i$'th projection morphism, $p_i: X^n \morph X$,
\item if $P$ is a context of $U$ that extends the context $\tuple{\ofT{x_1}{X},...\ofT{x_n}{X}}$ and if $P$ is mapped by $\iI$ to
the object $Y$ of \catcw (so that $X < Y$ in \catc) then the rule 
\gatdisplayrule{P}{\ofT{x_i}{X}} is mapped by $\iI$ to the section $s(p_{Y,X^n}\circ p_i)$ of object $\crossx{Y}{X}{1}$, where $p_i$ is the $i$'th projection morphism, $p_i: X^n \morph X$.
\end{enumerate}
\end{lemma}
\begin{proof}
See day book -- 20 July 2021.
\end{proof}

