

\begin{definition}
If $U$ is a generalised algebraic theory and if \catcw is a contextual category then an \term{instance} $I$ of $U$ in \catcw is a  mapping 
of derived T- and $\in$- rules of $U$ to objects, respectively sections, of $U$ that satisfies the following:
\begin{enumerate}[(i)]
\setlength\itemindent{2cm}
\item \underline{\textbf{T-rules}} 
Suppose that  the rule
\gatdisplayrule{\xDelta{n}}{\isT{\Delta}} is a derived rule of $U$ which is mapped by $I$ to an object $a$ of \catc. Denote this rule $r$. Recall that because $r$ is a derived rule then it follows  that for each $i$, 
$1 \leq i \leq n$, the rule \gatdisplayrule{\xDelta{i-1}}{\isT{\Delta_i}} is a derived rule of $U$. Let $r_i$ denote this rule.
Suppose that $I$ maps each rule $r_i$ to an object $a_i$ of \catcw.
\begin{enumerate}[(a)]
\item 
Suppose that $I$ maps each rule $r_i$ to an object $a_i$ of \catcw.
It is required that $1 \base a_1 \base ... \base a_n \base a$ in \catc.
\item Suppose that the  expression $\Delta$ is exactly the expression $\Delta_i$, for some $i$, $1 \leq i \leq n$. In this special case we require that the rule $r$  is mapped by $I$ to the object 
$\crossx{a_n}{a_i}{a_{i-1}}$.
\end{enumerate} 

\item \underline{\textbf{$\boldsymbol {\in}$-rules}} 
In addition to the assumptions made in (i),  suppose that the rule
\gatdisplayrule{\xDelta{n}}{\ofT{t}{\Delta}} is a  derived rule of $U$. 
Denote this rule $r_t$. 
\begin{enumerate}[(a)]
\item 
It is required that $I$ maps the rule $r_t$ to a section
 $s:a_n \morph a$ in \catcw i.e. to a morphism $s:a_n \morph a$ such that $s \circ p_a = id_{a_n}$. 
\item Suppose that the  expression $\Delta$ is exactly the expression $\Delta_i$, for some $i$, $1 \leq i \leq n$ and that the expression $t_i$ is simply the variable $x_i$. 
In this special case we require that the rule $r_t$  is mapped by $I$ to the section\footnote{
With these assumptions, $s(p_{a_n,a_i}): a_n \morph \crossx{a_n}{a_i}{a_{i-1}}$ in \catcw because by definition  $s(p_{a_n,a_i}): a_n  \morph (p_{a_n,a_i} \circ p_{a_i})^*a_i$,
and we have 
\begin{align*}
(p_{a_n,a_i} \circ p_{a_i})^*a_i &= {p_{a_n,a_{i-1}}} ^* a_i  && \mbox{ because $p_{a_n,a_i} \circ p_{a_i}=p_{a_n,a_{i-1}}$,} \\
                                 &= \crossx{a_n}{a_i}{a_{i-1}} && \mbox{ by definition of $\crossx{}{}{w}$}.
\end{align*}
} % end footnote
$s(p_{a_n,a_i})$ of the object $\crossx{a_n}{a_i}{a_{i-1}}$. Note that $p_{a_n,a_n}$ is defined to be $id{a_n}$ and so in the case of $i=n$ $r_t$
is mapped to   $s(id_{a_n})$. \commentary{PROOFREAD}
\end{enumerate}
\item \underline{\textbf{T=-rules}} 
In addition to the assumptions made in (i), suppose that  
the rule \gatdisplayrule{\xDelta{n}}{\Delta = \Delta'} is a derived rule of $U$. 
By the Well-Typedness Lemma, we may deduce that the
\gatdisplayrule{\xDelta{n}}{\isT{\Delta'}} is a derived rule of $U$. Denote this latter rule $r'$.
It is required that the rule $r'$ is mapped by $I$ to the same object $a$ of \catcw that $r$ is mapped to.

\item \underline{\textbf{$\boldsymbol{\in=}$-rules}} 
In addition to the assumptions made in (ii),  suppose that the rule
\gatdisplayrule{\xDelta{n}}{t = t' \in \Delta}
is a derived rule of $U$. 
By the Well-Typedness Lemma, we may deduce that the rule
\gatdisplayrule{\xDelta{n}}{\ofT{t'}{\Delta}} is a  derived rule of $U$. 
Denote this latter rule $r'_t$.
It is required that the rule $r'_t$ is mapped by $I$ to the same section $s$ of $a$ that $r_t$ is mapped to.

\item \underline{\textbf{weakening T-rules}} 
Suppose now that $Q$ is any context of $U$ and that it is mapped by $I$ to object $a$ and suppose that the rule 
\gatdisplayrule{\yOmega{m}}{\isT{\Omega}} is a derived rule of $U$ for some $m \geq 0$. Denote this rule $r_\Omega$. 
It follows that \foreachj, the rule   \gatdisplayrule{\yOmega{j-1}}{\isT{\Omega_j}} is a derived rule of $U$. Denote this rule $r_{\Omega_j}$.
Suppose that each rule $r_{\Omega_j}$ is mapped by $I$ to an object $b_j$ of \catcw and that rule $r_\Omega$ is mapped by $I$ to an object $b$ of \catcw so that
we have that $1 \base b_1 \base ... \base b_m \base b$ in \catc.

By the simple weakening lemma it follows that the rules
\gatdisplayrule{Q,\,  \yOmega{j-1}}{\isT{\Omega_j}} \kern-6pt, \foreachj, and 
\gatdisplayrule{Q,\, \yOmega{m}}{\isT{\Omega}} are  derived rules of $U$. It is required that these rules are mapped by $I$ to the objects
$\crossx{a}{b_1}{1},...\crossx{a}{b_m}{1}$ and to $\crossx{a}{b}{1}$, respectively. 

\item \underline{\textbf{weakening $\boldsymbol {\in}$-rules}} 
Suppose in addition to the assumptions in (v) that the rule \gatdisplayrule{\yOmega{m}}{\ofT{s}{\Omega}} is a derived rule of $U$ 
and that this rule is mapped by $I$ to a section $g$ of object $b$ in \catc.
By the simple weakening lemma it follows that the rule \gatdisplayrule{Q,\, \yOmega{m}}{\ofT{s}{\Omega}}
is a derived rule of $U$. It is required that this rule is mapped by $I$ to the section $\crossx{a}{s}{1}$
of object $\crossx{a}{b}{1}$ of \catc.


\item \underline{\textbf{substituting in T-rules}} 
Suppose that, as in (v), above, the rule 
\gatdisplayrule{\yOmega{m}}{\isT{\Omega}} is a derived rule of $U$.
We may deduce that \foreachj, the rule   \gatdisplayrule{\yOmega{j-1}}{\isT{\Omega_j}} is a derived rule of $U$. 
Suppose that, as in (v), these rules are mapped by $I$ to objects $b_1,...b_n$ and $b$ so that
we have  $1 \base b_1 \base ... \base b_m \base b$ in \catc. Now suppose that for some $j$, $1 \leq j \leq m$, the rule
\gatdisplayrule{\yOmega{j-1}}{\ofT{t}{\Omega_j}} is a derived rule of $U$. 
Now it follows by the substitution lemma that for each $j'$, $j < j' \leq m$ the rule

\gatdisplayrule{\yOmega{j-1}, y_{j+1}\in \Omega_{j+1}[t|y_j],... y_{j'-1} \in \Omega_{j'-1}[t|y_j] }{\isT{\Omega_j[t|y_j]}} is a derived rule of $U$ and that likewise the rule

\gatdisplayrule{\yOmega{j-1}, y_{j+1}\in \Omega_{j+1}[t|y_j],... y_m \in \Omega_m[t|y_j] }{\isT{\Omega}[t|y_j]} is a derived rule of $U$.
It is required that these rules are mapped by $I$ to objects $f^*b_{j+1},...f^*b_m$ and $f^*b$, respectively.
Note that as required we have that $f^*b_{j+1}\base ... \base f^*b_m \base f^*b$ in \catc.

\item \underline{\textbf{substituting in $\boldsymbol {\in}$-rules}} 
Suppose that in addition to the situation in (vii), above, the rule
\gatdisplayrule{\yOmega{m}}{\ofT{s}{\Omega}}
is a derived rule of $U$ and suppose that this rule is mapped to a section $g$ of object $b$ of \catc.
Now it follows by the substitution lemma that the rule
\gatdisplayrule{\yOmega{j-1}, y_{j+1}\in \Omega_{j+1}[t|y_j],... y_m \in \Omega_m[t|y_j]}{\ofT{s[t|y_j]}{\Omega[t|y_j]}} 
is a derived rule of $U$.
It is a requirement that this rule is mapped by $I$ to the section $f^*g$ of object $f^*b$ of \catc.
\end{enumerate}
\end{definition}

{ % create a scope for two width forcing commands
\newcommand {\forceSOURCEwidth}{\rule{5cm}{0pt}}  % so as to line up three different arrays
\newcommand {\forceTARGETwidth}{\rule{2.2cm}{0pt}}
\begin{lemma}
\llabel{substitutioninterpretation}

\newcommand{\sjrule}   {\gatdisplayrule{Q}         {\ofT{s_j}{\Omega_j[s_1|y_1,...s_{j-1}|y_{j-1}]}}}
\newcommand{\omegarule}{\gatdisplayrule{\yOmega{m}}{\isT{\Omega}}}
\newcommand{\srule}    {\gatdisplayrule{\yOmega{m}}{\ofT{s}{\Omega}}}
\newcommand{\omegarulesubstituted}{\gatdisplayrule{Q}{\isT{\Omega[s_1|y_1...s_m|y_m]}} }
\newcommand{\srulesubstituted}{\gatdisplayrule{Q}{\ofT{s[s_1|y_1...s_m|y_m]}{\Omega[s_1|y_1...s_m|y_m]}} }

Suppose that $I$ is an instance of the generalised algebraic theory $U$ in the contextual category \catc,
suppose that for some $m \geq 0$ the rules  \omegarule  and \srule are derived rules of $U$
and that these rules are mapped by $I$ into the objects and sections of \catcw as follows


\begin{equation*}
\begin{array}{c c c}
\forceSOURCEwidth & & \forceTARGETwidth \\ [-0.1cm]
\omegarule & \Imapsto & b   \\ [0.4cm]
\srule     & \Imapsto & g
\end{array}
\end{equation*}
suppose that each context $\encyOmega{j}$  is mapped by $I$ to an object $b_j$ so that we have 
$1 \base b_1 ... \base b_m$ in \catc,
suppose  that $Q$ is some other context and $\tuple{\sm}$ is a realisation of $\tuple{\yOmega{m}}$ with respect to $Q$
and that we have the following mappings by $I$ into the objects and sections of \catc:
\begin{equation*}
\begin{array}{c c c}
\forceSOURCEwidth & & \forceTARGETwidth \\ [-0.1cm]
Q          & \Imapsto & a   \\ [0.4cm]
\sjrule    & \Imapsto & f_j \\ [0.4cm]
\end{array}
\end{equation*}
then the rules \omegarulesubstituted and  \srulesubstituted, which are derived rules of $U$ by virtue of the substitution lemma, are mapped by $I$ as follows:
\begin{equation*}
\begin{array}{c c c}
\forceSOURCEwidth & & \forceTARGETwidth \\ [-0.1cm]
\omegarulesubstituted  & \Imapsto & \fmstar...\fonestar (\crossx{a}{b}{1})   \\ [0.4cm]
\srulesubstituted      & \Imapsto & \fmstar...\fonestar (\crossx{a}{g}{1}).
\end{array}
\end{equation*}
and  that we have the following diagram in \catc

\newcommand{\ncdotdotdot}[2]
{\ncline[linestyle=none]{#1}{#2} 
 \ncput[nrot=:U]{\Large$ \hdots$}
}
\begin{displaymath}
\begin{array}{c  c p{0.4cm} c p{0.2cm} c p {0.2cm} c  p{0.5cm} c}
&&&                                               &&                                           && \Rnode{ab}{\crossx{a}{b}{1}}    &&                \\[1.2cm]
&&&                                               &&  \Rnode{f1ab}{\fonestar(\crossx{a}{b}{1})}
%\rule[-1cm]{3pt}{1pt}
&& \Rnode{abm}{\crossx{a}{b_m}{1}} &&                \\[1.2cm]
&&&                                               &&  \Rnode{f1abm}{\fonestar(\crossx{a}{b_m}{1})}&&                              &&                \\[0.1cm]
&&&                                               &&                                           && \Rnode{ab3}{\crossx{a}{b_3}{1}} &&                \\[1.2cm]
&\Rnode{fm1axb}{\fmonestar...\ftwostar\fonestar(\crossx{a}{b}{1})}&& &&\Rnode{f1axb3}{\fonestar(\crossx{a}{b_3}{1})}  && \Rnode{ab2}{\crossx{a}{b_2}{1}}  &&           \\[1.2cm]
\Rnode{ftarget}{\fmstar...\ftwostar\fonestar(\crossx{a}{b}{1})}&\Rnode{fmtarget}{\fmonestar...\ftwostar\fonestar(\crossx{a}{b_m}{1})}&&
\Rnode{f3target}{\ftwostar\fonestar(\crossx{a}{b_3}{1})} &&\Rnode{f2target}{\fonestar(\crossx{a}{b_2}{1})}  && \Rnode{ab1}{\crossx{a}{b_1}{\Rnode{f1target}{1}}}     \\[1.2cm]
&&&                                               &&                                           &&                                                       \\[-6.4cm] %%% HEE HEE HE
&&&																								&&                                           &&                         && \Rnode{b}{b}                \\[1.2cm]
&&&																								&&                                           &&                         && \Rnode{bm}{b_m}             \\[0.3cm]
&&&                                               &&                                           &&                         &&                             \\[0.3cm]
&&&																								&&                                           &&                         && \Rnode{b3}{b_3}             \\[1.2cm]
&&&																								&&                                           &&                         && \Rnode{b2}{b_2}             \\[1.2cm]
&&&																								&&                                           &&                         && \Rnode{b1}{b_1}             \\[0.3cm]
&&&		\ovalnode[linestyle=none]{a}{a}					    &&                                           &&                         &&                             \\[1.1cm]
&&&                                               &&                                           && \Rnode{abs}{1} \ \ \ \ \ \ \ \ &&                      \\           
\makebox[0cm]{
\ncarr{ab}{b}
\ncarr{abm}{bm}
\ncarr{f1ab}{ab}
\ncarr{f1abm}{abm}
\ncarr{ab3}{b3}
\ncarr{ab2}{b2}
\ncarr{ab1}{b1}
\ncarr{f1axb3}{ab3}
\ncarr{f2target}{ab2}
\ncarr{f3target}{f1axb3}
\ncarr{ftarget}{fm1axb}
\ncdotdotdot{fm1axb}{f1ab} 
\ncdotdotdot{fmtarget}{f1abm}
\ncdotdotdot{fmtarget}{f3target}
%
\ncarc[arcangle=-5,nodesepA=15pt,offsetA=-2pt,nodesepB=3pt,offsetB=-5pt]{->}{a}{f1target}
\blabel{f_1}[0.6]
\ncarc[arcangle=10,nodesepA=15pt,offsetA=1pt,nodesepB=2pt,offsetB=2pt]{->}{a}{f2target}
\alabel{f_2}
\ncarc[arcangle=10,nodesepA=15pt,offsetA=1pt,nodesepB=2pt,offsetB=2pt]{->}{a}{f3target}
\alabel{f_3}
\ncarc[arcangle=7,nodesepA=15pt,offsetA=1pt,nodesepB=2pt,offsetB=2pt]{->}{a}{fmtarget}
\alabel{f_m}
\ncarc[arcangle=7, nodesepA=15pt,offsetA=1pt,nodesepB=2pt,offsetB=2pt]{->}{a}{ftarget}
\alabel{f}

\setlength{\sarnodesepB}{10pt}
\ncsar{fmtarget}{a}
\ncsar{ftarget}{a}
\ncsar{f3target}{a}
\ncsar{f2target}{a}
\ncsar{f1target}{a}
\sarreset
\ncsar{fm1axb}{fmtarget}

%left but two tower
\ncsar{f1ab}{f1abm}
\ncdotdotdot {f1abm}{f1axb3}
\ncsar{f1axb3}{f2target}
%left but one tower
\ncsar{ab}{abm}
\ncdotdotdot{abm}{ab3}
\ncsar{ab3}{ab2}
\ncsar{ab2}{ab1}
%left tower
\ncsar{b}{bm}
\ncdotdotdot{bm}{b3}
\ncsar{b3}{b2}
\ncsar{b2}{b1}
\ncsar{b1}{abs}
\nccdar{a}{abs}
}
\end{array}
\end{displaymath}


\end{lemma}
\begin{proof}

\newcommand{\sonerule} {\gatdisplayrule{Q}         {\ofT{s_1}{\Omega_1}}}
\newcommand{\stworule}  {\gatdisplayrule{Q}       {\ofT{s_2}{\Omega_2[s_1|y_1]}}}
\newcommand{\weakenedOmegarule}{\gatdisplayrule{Q,\, \yOmega{m}} {\isT{\Omega}} }
\newcommand{\weakenedsrule}    {\gatdisplayrule{Q,\, \yOmega{m}} {\ofT{s}{\Omega}} }
\newcommand{\weakenedOmegaruleFirstsubstitution}{\gatdisplayrule{Q,\, \ofT{y_2}{\Omega_2[s_1|y_1]},\,...\,\ofT{y_m}{\Omega_m[s_1|y_1]}}{\isT{\Omega[s_1|y_1]}} }
\newcommand{\weakenedsruleFirstsubstitution}{\gatdisplayrule{Q,\, \ofT{y_2}{\Omega_2[s_1|y_1]},\,...\,\ofT{y_m}{\Omega_m[s_1|y_1]}}{\ofT{s[s_1|y_1]}{\Omega[s_1|y_1]}} }
\newcommand{\weakenedOmegaruleSecondsubstitution}{\gatdisplayrule{Q,\, \ofT{y_3}{\Omega_2[s_1|y_1, s_2|y_2]},\,...\,\ofT{y_m}{\Omega_m[s_1|y_1, s_2|y_2]}}{\isT{\Omega[s_1|y_1, s_2|y_2]}} }
\newcommand{\weakenedsruleSecondsubstitution}{\gatdisplayrule{Q,\, \ofT{y_2}{\Omega_2[s_1|y_1, s_2|y_2]},\,...\,\ofT{y_m}{\Omega_m[s_1|y_1, s_2|y_2]}}{\ofT{s[s_1|y_1, s_2|y_2]}{\Omega[s_1|y_1, s_2|y_2]}} }
First note that we have $a \base\, \crossx{a}{b_1}{1}\, \base\, ...\, \base\, \crossx{a}{b_m}{1}\, \base\, \crossx{a}{b}{1}$ in \catcw
and  that application of rule (v) of the main definition determines that the rules \weakenedOmegarule and \weakenedsrule must be mapped by $I$ as follows
\begin{equation*}
\begin{array}{c c c}
\forceSOURCEwidth & & \forceTARGETwidth \\ [-0.1cm]
\weakenedOmegarule  & \Imapsto & \crossx{a}{b}{1}   \\ [0.4cm]
\weakenedsrule      & \Imapsto & \crossx{a}{g}{1}.
\end{array}
\end{equation*}

Given these mapping and since  \sonerule is a derived rule we can now apply clauses (vii) and (viii) of the main definition
to determine that $I$ must map rules as follows
\begin{equation*}
\begin{array}{c c c}
\forceSOURCEwidth & & \forceTARGETwidth \\ [-0.1cm]
\weakenedOmegaruleFirstsubstitution  & \Imapsto & \fonestar(\crossx{a}{b}{1})   \\ [0.4cm]
\weakenedsruleFirstsubstitution      & \Imapsto & \fonestar(\crossx{a}{g}{1}).
\end{array}
\end{equation*}

From these mapping and since  \stworule is a derived rule we can  apply clauses (vii) and (viii) again,
this time substituting $s_2$ into the $s_1$ substituted rules, to determine that $I$ must map the further substituted rules as follows
\begin{equation*}
\begin{array}{c c c}
\forceSOURCEwidth & & \forceTARGETwidth \\ [-0.1cm]
\weakenedOmegaruleSecondsubstitution  & \Imapsto & \ftwostar \fonestar(\crossx{a}{b}{1})   \\ [0.4cm]
\weakenedsruleSecondsubstitution      & \Imapsto & \ftwostar \fonestar(\crossx{a}{g}{1}).
\end{array}
\end{equation*}

We see that the mapping that we are required to show follows from $m$ successive applications of clauses (vii) and (viii) of the main definition.
This completes the proof.
\end{proof}
} % end scope of force width commands
%\newpage
\note
We can show that an instance $I$ of a contextual category $U$ in a contextual category \catcw is
completely determined by its mapping of the introductory rules of sort symbols and operator symbols to
objects, respectively, sections of \catc. In order to show this first define a preinstance as follows:
\begin{definition}
If $U$ is a generalised algebraic theory  and if \catcw is a contextual category then
a \term{preinstance} $I$ of  $U$ in \catcw consists of a pair :
\begin{itemize}
\item a mapping $\Isort$ that maps each sort symbol of $U$ to  an object of \catc,
\item a mapping $\Iop$ that maps each operator symbol of $U$ to a section of \catcw (i.e. to a morphism $f: A \morph B$ for some 
$A \base B$ in \catcw such that $f \circ p_B=id_A$).
\end{itemize}
\end{definition}

Note that I will say that an instance $I$ of $U$ in \catcw extends a preinstance $P$ of $U$ in \catcw to mean that for each sort symbol $A$ of $U$,
$I(r_A) = P_{sort}(A)$, where $r_A$ is the introductory rule for $A$ and that for each operator symbol
$f$ of $U$,   $I(r_f) = P_{op}(f)$, where $r_f$ is the introductory rule for $f$.

\note
 For a preinstance to extend to an instance (we will say that it *is* an instance) 
it will need be type correct and to satisfy the axioms of the theory. The definition of exactly what we mean by this has to proceed by induction because, for example, we need an instance of the rule
\gatdisplayrule{\xDelta{n}}{\isT{\Delta}} (as an object $a$ of \catc, say) before we can say whether a preinstance of an operator with introductory rule \genericfintroductoryrule
is well-typed (i.e. to know that it is object $a$ that it is required to be a section of).
Similarly we need to be able to interpret both sides of an axiom before we can say whether it is respected
by the preinstance. 

\note Suppose $P$ be a preinstance of $U$ in contextual category \catc.
Let $U_0 \subseteq $U$_1 \subseteq $U$_2 \subseteq ...$ be the stratification of $U$ given by the Stratification Lemma. 
Let $P_i$ be the restriction of $P$ to $U_i$. 
We define $P$ to be well-typed and to respect all axioms 
providing that  each  $P_i$ is well-typed and respects all axioms of $U_i$ and 
what we mean by this we define by induction. 
At the same time we prove that if preinstance $P_i$ is well-typed and respects all axioms of $U_i$ then $P_i$
extends uniquely to an instance $I_i$ of $U_i$.  

For the inductive step we require:
\begin{lemmastar}
\item $P_{i+1}$ extends uniquely to an instance $I_{i+1}$ of $U_{i+1}$ iff  $P_i$ extends uniquely to an instance of $U_{i}$ and additionally:
\begin{enumerate}[(i)]
\item
$P_{i+1}$ is well-typed on sort symbol $A$ of $U_{i+1}$. This we define to mean that
for all sort symbols $A$ in $U_{i+1}$, if $A$ has introductory rule 
\genericAintroductoryrule and if this rule is mapped by $P_{i+1}$
to object $a$ of \catcw then $I_i(r_n) \base a$ in \catc, where $r_n$ is the rule 
\gatdisplayrule{\xDelta{n-1}}{\isT{\Delta_n}} (which, as required and due to the stratification, is a derived rule of $U_i$).
\item  $P_{i+1}$ is well-typed on operator symbols  of $U_{i+1}$. This we define to mean that
for all sort symbols $f$ in $U_{i+1}$, if $f$ has introductory rule 
\genericfintroductoryrule then $P_{i+1}$ maps this rule to a section 
of $I_i(r)$ where $r$ is the rule
\gatdisplayrule{\xDelta{n}}{\isT{\Delta}} (which, as required and due to the stratification, is a derived rule of $U_i$). 
\item
 $I_i$ respects the axioms of $U_{i+1}$. By this we mean that 
\begin{enumerate}[(i)]
\item \underline{\textbf{T=-axioms}} 
for all axioms of $U_{i+1}$ of the form
 \gatdisplayrule{\xDelta{n}}{\Delta = \Delta'},
$I_i(r) = I_i(r')$ where $r$ is the rule
\gatdisplayrule{\xDelta{n}}{\isT{\Delta}} and  
and $r'$ is the rule \gatdisplayrule{\xDelta{n}}{\isT{\Delta'}}
\item \underline{\textbf{$\boldsymbol{\in=}$-axioms}} 
for all axioms of $U_{i+1}$ of the form
\gatdisplayrule{\xDelta{n}}{t = t' \in \Delta}
$I_i(r_t) = I_i(r'_t)$ where $r_t$ is the rule
\gatdisplayrule{\xDelta{n}}{\ofT{r_t}{\Delta}} and  
and $r'_t$ is the rule \gatdisplayrule{\xDelta{n}}{\ofT{t'}{\Delta'}}.
\end{enumerate}
\end{enumerate}
\end{lemmastar}
\begin{proof} 
\tbd
\end{proof}

\begin{lemma}
\llabel{Xnlemma}
If $I$ is an interpretation of a generalised algebraic theory $U$ in a contextual category \catcw and if $X$ is an absolute sort symbol of $U$ which is mapped 
by $I$ to an object $X$ of \catcw (so that $1 \base X$ in \catc) then for any $n \geq 1$ 
\begin{enumerate}[(i)]
\item
The context $\tuple{\ofT{x_1}{X},...\ofT{x_n}{X}}$ is mapped by $I$ to the object $X^n$ of \catc,
\item the rule 
\gatdisplayrule{\ofT{x_1}{X},... \ofT{x_n}{X}}{\ofT{x_i}{X}} is mapped by $I$ to the section $s(p_i)$ of $X^{n+1}$, where $p_i$ is the $i$'th projection morphism, $p_i: X^n \morph X$,
\item if $P$ is a context of $U$ that extends the context $\tuple{\ofT{x_1}{X},...\ofT{x_n}{X}}$ and if $P$ is mapped by $I$ to
the object $Y$ of \catcw (so that $X < Y$ in \catc) then the rule 
\gatdisplayrule{P}{\ofT{x_i}{X}} is mapped by $I$ to the section $s(p_{Y,X^n}\circ p_i)$ of object $\crossx{Y}{X}{1}$, where $p_i$ is the $i$'th projection morphism, $p_i: X^n \morph X$.
\end{enumerate}
\end{lemma}
\begin{proof}
See day book -- 20 July 2021.
\end{proof}