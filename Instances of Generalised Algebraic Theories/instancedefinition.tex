
\newcommand{\clause}[1]{clause (#1) of definition \lref{consistentinterpretation}}
\newcommand{\condition}[2]{condition (#2) of \clause{#1}}

In this section, we will define an instance of a generalised algebraic theory $U$ in a contextual category \catcw to be a consistent mapping 
of derived T-rules and $\in$-rules of  $U$ to objects, respectively sections of \catc. 
We will show that such instances are totally defined by their mapping of the introductory rules of the theory and we will define how to extend 
consistent and valid interpretations $\iI$ of the introductory rules   to instances $\Ibar$ that consistently map all the derived  T-rules and $\in$-rules.
To make this work, we need define $\Ibar$ as a partial mapping which, when $\iI$ is consistent and valid, we subsequently show to be total. For this reason we need definitions that apply to partial mappings of rules.
\begin{numbereddefinition}
\llabel{contextmapping}
If $I$ is a  possibly partial mapping of derived T-rules and $\in$-rules of a theory $U$ to objects, respectively sections of a contextual category \catcw then
\begin{enumerate}[(i)]
\item
For any non-empty context $\xDelta{n}$ of $U$, by the mapping under $I$ of this context we shall mean the mapping under $I$ of the derived rule
\IDelta{n}.  
\item
By the mapping under $I$ of the empty context we shall mean the terminal object $1$ of \catc.
\item
For $m \geq 0$, if  $Q$ and $\yOmega{m}$ are contexts and if $\tuple{\sm}$ is a realisation of $\yOmega{m}$ wrt $Q$ in $U$
then we will define the mapping of  $\tuple{\sm}$ under $I$ to be the tuple of sections
$\tuple{I(r_{s_1}),...I(r_{s_m})}$, 
where \foreachj, $r_{s_j}$ is the derived rule \IsOmega{j}.
We will say that the realisation $\tuple{\sm}$ is mapped to a cascade iff $I(Q)$ is defined and
 $I(r_{s_j})$ is defined, \foreachj, and 
 $\tuple{I(r_{s_1}),...I(r_{s_m})}$ is a cascade of sections i.e. \foreachj, 
 $I(r_{s_j}) \in Sect(I(r_{s_{j-1}})^*...I(r_{s_1})^*(\crossx{I(Q)}{I(r_{\Omega_j})}{1}))$,
where $r_{\Omega_j}$ is the derived rule \IOmega{j}. Note that according to this description 
the empty realisation considered as a realisation of the empty context wrt a context $Q$ is deemed to map to a cascade under $I$ provided that $I(Q)$
is defined.  
\end{enumerate}
\end{numbereddefinition}
\newcommand{\smMappedToCacscade}{
for all contexts $Q$ and for all realisations $\tuple{\sm}$ of $\yOmega{m}$ wrt $Q$ 
which map to a cascade under $I$ ,}
\newcommand{\sjpconclusion}{\ofT{s'_j}{\Omega'_j[s_1|y_1,...s_{j-1}|y_{j-1}]}}
\newcommand{\IfIpartialmappingUtoC}{If $U$ is a generalised algebraic theory and \catcw is a contextual category 
and if $I$ is a partial mapping of derived T-rules and $\in$-rules of the theory $U$ to objects, respectively sections of the contextual category \catc}
\newcommand{\IfIpartialmappingUtoCw}{\IfIpartialmappingUtoC\ }

\begin{numbereddefinition}
\llabel{consistentinterpretation}
\IfIpartialmappingUtoCw
then we define what it is for any particular derived rule $r$ of $U$ to be \term{consistently interpreted} by $I$: \\
\begin{enumerate}[(i)]
%\setlength\itemindent{2cm}
\item \underline{\textbf{T-rule}} 
Let $r_\Omega$ be any  derived T-rule of $U$, assume it to be of the form \ZOmega, for some $m \geq 0$, 
and let $r_{\Omega_m}$ be the rule \IOmega{m}
then define $r_\Omega$ to be consistently interpreted by $I$ iff
\begin{enumerate}[(a)] 
\item both $I(r_\Omega)$ and $I(r_{\Omega_m})$ are defined and $I(r_{\Omega_m}) \base\, I(r_\Omega)$ in \catcw and 
\item $r_{\Omega_m}$ is consistently interpreted by $I$  and
\item 
\smMappedToCacscade
$$ \Imappedrule{Q}{\isT{\Omega[\SUBsFORy{m}]}} 
= I(r_{s_m})^*...I(r_{s_1})^*(\crossx{I(Q)}{I(r_\Omega)}{1}),$$
where $r_{s_j}$ is the rule \IsOmega{j} 
and $r_{\Omega_j}$ is the rule \IOmega{j}.
\end {enumerate}
\item \underline{\textbf{$\in$-rule}} 
Let $r_s$ be any derived $\in$-rule, assume it to be of the form \ZsOmega, for some $m \geq 0$, then
$r_s$ is defined to be consistently interpreted by $I$ 
iff  
\begin{enumerate}[(a)]
\item $I(r_{\Omega_m})$ is defined and $I(r_\Omega)$ is defined and $\displaystyle I(r_s) \in Sect(I(r_\Omega))$, where
$r_\Omega$ is the rule \ZOmega and $r_{\Omega_m}$ is the rule \IOmega{m}, and
\item
$r_{\Omega}$ is consistently interpreted by $I$ (and so by implication of condition (b) of clause (i),  $r_{\Omega_m}$ is also consistently interpreted by $I$) and,
\item  
\smMappedToCacscade
$$ \Imappedrule{Q}{\ofT{s[\SUBsFORy{m}]}{\Omega[\SUBsFORy{m}]}} = I(r_{s_m})^*...I(r_{s_1})^*(\crossx{I(Q)}{I(r_s)}{1}),$$
where $r_{s_j}$ and $r_{\Omega_j}$ defined as above (in clause (i)), or
\item in the case that $s$ is the variable $y_j$, \forsomej, so that $r_s$ is the rule \gatdisplayrule{\yOmega{m}}{\ofT{y_j}{\Omega}},
$$I(r_s) = s(p_{r_{\Omega_m},r_{\Omega_j}})$$
By lemma \lref{sofplemma}, $s(p_{r_{\Omega_m},r_{\Omega_j}}))$ is a section of the object $\crossx{r_{\Omega_m}}{r_{\Omega_j}}{r_{\Omega_{j-1}}}$.
Note that $p_{x,x}$ is defined to be $id_{x}$ for any object $x$ of \catcw and so in the case of $j=m$, $r_s$
is mapped to   $s(id_{r_{\Omega_m}})$.
By implication $I(r_\Omega)=\crossx{r_{\Omega_m}}{r_{\Omega_j}}{r_{\Omega_{j-1}}}$. 
\end{enumerate}

\item \underline{\textbf{T=-rules}} 
If $r$ is the rule  \gatdisplayrule{\xDelta{n}}{\Delta = \Delta'}, for some $n \geq 0$, 
then $r$ is interpreted consistently by $I$ iff
both of the rules \ZDelta and \ZDeltap
are consistently interpreted by $I$ and
$$
\Imappedrule{\xDelta{n}}{\isT{\Delta}} = \Imappedrule{\xDelta{n}}{\isT{\Delta}}
$$
 
\item \underline{\textbf{$\in=$-rules}} 
If $r$ is the rule  \gatdisplayrule{\xDelta{n}}{t = t' \in \Delta}, for some $n \geq 0$, 
then $r$ is interpreted consistently by $I$ iff
both of the rules \ZtDelta and \gatdisplayrule{\xDelta{n}}{\ofT{t}{\Delta'}}
are consistently interpreted by $I$ and
$$
\Imappedrule{\xDelta{n}}{\ofT{t}{\Delta'}} = \Imappedrule{\xDelta{n}}{\ofT{t'}{\Delta'}}
$$
\end{enumerate}
\end{numbereddefinition}
\highlight{END of DEFINITION OF consistently interpreted}


Since the above definition of a rule being consistently interpreted relies on whether other rules are consistently interpreted we need to check that there are no circularities:
\begin{lemma}
The above property of a rule being consistently interpreted by a mapping of rules is well-defined. 
\end{lemma}
\begin{proof}
The definition in the case of T-rules relies on the definition of other T-rules being consistently interpreted but these other rules are of lower rank and therefore there is no circularity 
(the definition terminates).
Neither are there circularities in the other cases because either they make no recursive references to consistent interpretation (the $\in$-rules case)
or they make recursive references to T-rules (the T=-rule case) or to $\in$-rules (the $\in$=-rules case).
\end{proof}



\begin{definition}
An \term{instance} of a generalised algebraic theory $U$ in a contextual category \catcw is  any mapping 
of derived T-rules and $\in$-rules of the theory $U$ to objects, respectively sections of the contextual category \catcw that
consistently interprets every derived rule of $U$.
\end{definition}

We will show that an instance $I$ of a generalised algebraic theory $U$ in a contextual category \catcw is
completely determined by its mapping of the introductory rules of sort symbols and operator symbols to
objects, respectively, sections of \catc. To state this precisely  we require a number of defintions.

\begin{definition}
If $U$ is a generalised algebraic theory  and if \catcw is a contextual category then
a \term{interpretation} $\iI$ of  $U$ in \catcw consists of a pair :
\begin{itemize}
\item a mapping $\Isort$ that maps each sort symbol of $U$ to  an object of \catc,
\item a mapping $\Iop$ that maps each operator symbol of $U$ to a section of \catcw (i.e. to a morphism $f: A \morph B$ for some 
$A \base B$ in \catcw such that $f \circ p_B=id_A$.
\end{itemize}
\end{definition}

We will say that  an interpretation $\iI$ of $U$ in \catcw \term{determines} an  instance $I$ of $U$ in \catcw to mean that for each sort symbol $A$ of $U$,
$I(r_A) = \iI_{sort}(A)$, where $r_A$ is the introductory rule for $A$ and that for each operator symbol
$f$ of $U$,   $I(r_f) = \iI_{op}(f)$, where $r_f$ is the introductory rule for $f$.

We will show (lemma \lref{uniquenessofinstancedeterminedbyaninterpretation}) that 
if $\iI$ is an interpretation of a generalised algebraic theory $U$ in a contextual catgeory \catcw then
there is at most one  instance determined by $\iI$. This instance will be denoted $\Ibar$ and at the outset 
it is defined to be a partial mapping by the following definition.

\begin{definition} [\highlight{Definition of  $\Ibar$}]
If $\iI$ is an interpretation of generalised algebraic theory $U$ in a contextual catgeory \catcw
then define a
partial mapping $\Ibar$  of T-rules and $\in$-rules to objects, respectively sections, of \catcw
as follows
\begin{enumerate}[(i)] 
\item \underline{\textbf{T-rules}} 
The derivation lemma (lemma \ref{derivationlemma}) tells us that if $r_\Delta$ is a derived T-rule of $U$  then it is of the form \gatdisplayrule{P}{\isT{A(t_1,...t_n)}} for some premise $P$, for some $n \geq 0$ and for some sort symbol $A$ with introductory rule $r_A$ of the form \gatdisplayrule{\xDelta{n}}{\isT{A(x_1,...x_n)}}, where \foreachi, the rule 
\ItDelta[P]{i}  is a derived rule of $U$. 
We define $\Ibar(r_\Delta)$ to be undefined unless the following preconditions are met:
\begin{enumerate}[(a)]
\item
the context  $P$ is mapped by $\Ibar$ to some object $\Ibar(P)$ of \catcw and 
\item
\foreachi, the rule $\IDelta{i}$, which we denote $r_{\Delta_i}$, is mapped by $\Ibar$ to some object $\Ibar(r_{\Delta_i})$ of \catcw
and $1 \base \Ibar(r_{\Delta_1}),...\Ibar(r_{\Delta_n}) \base \iI_{sort}(A)$ in \catcw 
\item
$\tuple{\tn}$ is mapped to a cascade by $\Ibar$
\end{enumerate}
in which case we define $\Ibar(r_\Delta)$ to be $\Ibar(r_{t_n})^*...\Ibar(r_{t_1})^*(\crossx{\Ibar(P)}{\Ibar(A)}{1})$,
where $r_{t_i}$ is the rule \ItDelta[P][.]{i}
\item \underline{\textbf{$\boldsymbol {\in}$-rules}} 
The derivation lemma also tells us that if $r_t$ is an $\in$-rule then it is  
either (1) of the form \gatdisplayrule{\xDelta{n}}{\ofT{x_i}{\Omega}} for some $n \ge 1$, for some $i$, $1 \leq i \leq n$, 
and for some $\Omega$ such that \gatdisplayrule{\xDelta{n}}{\Delta_i=\Omega} is a derived rule of $U$
or (2) it is of the form \gatdisplayrule{P}{\ofT{f(t_1,...t_n)}{\Omega}} for some premise $P$, for some $n \geq 0$ and for some operator symbol $f$ with introductory rule $r_f$ of the form \gatdisplayrule{\xDelta{n}}{\ofT{f(x_1,...x_n)}{\Delta}} where \foreachi, the rule 
\ItDelta[P]{i}, which we will call $r_{t_i}$, is a derived rule of $U$. 

In  case (1) we define $\Ibar(r_t)$ to be undefined unless  the following preconditions are met:
\begin{enumerate}
\item
\foreachi, the rule $\IDelta{i}$, which we denote $r_{\Delta_i}$, is mapped by $\Ibar$ to some object $\Ibar(r_{\Delta_i})$ of \catcw
and $1 \base \Ibar(r_{\Delta_1}),...\Ibar(r_{\Delta_n})$ in \catcw 
\item the rule \gatdisplayrule{\xDelta{n}}{\isT{\Omega}}, which we call $r_\Omega$, 
is mapped by $\Ibar$ to the object $\crossx{\Ibar(r_{\Delta_n})}{\Ibar(r_{\Delta_i})}{\kern-4pt\Ibar(r_{\Delta_{i-1}})\kern-12pt}$
\end{enumerate}
in which case we define $\Ibar(r_t)$ to be $s(p_{\Ibar(r_{\Delta_n}),\Ibar(r_{\Delta_i})})$ . 

In  case (2) 
we define $\Ibar(r_t)$ to be undefined unless  unless the following preconditions are met:
\begin{enumerate}
\item
the context  $P$ is mapped by $\Ibar$ to some object $\Ibar(P)$ of \catcw and 
\item
\foreachi, the rule $\IDelta{i}$, which we denote $r_{\Delta_i}$, is mapped by $\Ibar$ to some object $\Ibar(r_{\Delta_i})$ of \catcw
and  
%\foreachi, the context $\xDelta{i}$ is mapped by $\Ibar$ to some object $b_i$ of \catcw
and 
\item the rule \ZDelta, which we shall call $r_\Delta$, is mapped by $\Ibar$ to some object $\Ibar(r_\Delta)$ of \catcw and
$1 \base \Ibar(r_{\Delta_1}),...\Ibar(r_{\Delta_n}) \base \Ibar(r_\Delta)$ in \catcw 
and 
\item
%\foreachi, $\Ibar(r_{t_i}) \in Sect(r_{t_{i-1}}^*...r_1^* (\crossx{a}{b}{1}))$
$\tuple{\tn}$ is mapped to a cascade by $\Ibar$,
\end{enumerate}
in which case we
define $\Ibar(r_t)=\Ibar(r_{t_n})^*...\Ibar(r_{t_1})^*(\crossx{\Ibar(P)}{\iI_{op}(f)}{1})$.
This completes the definition of $\Ibar$.
\end{enumerate}
\highlight{END OF Definition of  $\Ibar$}
\end{definition}


\def\restrict{\mathbin{\restriction}}
\newcommand{\predInstance}{\overline{I \restrict U_p}}
\newcommand{\Uincrement}{U \setminus\kern-2pt U_p}

\begin{definition}
 Suppose that $\iI$ is an interpretation of $U$  
define interpretation $\iI$ to be \term{valid}  iff 

\begin{enumerate}[(i)]
\item
for all sort symbols $A$ of $U$ if $A$ has introductory rule $r_A$ then $\Ibar(r_A)$ is defined,
\item  
for all operator symbols $f$ of $U$ if $f$ has introductory rule $r_f$ then $\Ibar(r_f)$ is defined,

\item \underline{\textbf{T=-axioms}} 
for all axioms of the form
 \gatdisplayrule{\xDelta{n}}{\Delta = \Delta'},
$\Ibar(r)$ is defined and $\Ibar(r')$ is defined and
$\Ibar(r) = \Ibar(r')$, where $r$ is the rule
\ZDelta and  
and $r'$ is the rule \ZDeltap[,]

\item \underline{\textbf{$\boldsymbol{\in=}$-axioms}} 
for all axioms  of the form
\gatdisplayrule{\xDelta{n}}{t = t' \in \Delta},
$\Ibar(r_t)$ is defined and  $\Ibar(r'_t)$ is defined and
$\Ibar(r_t) = \Ibar(r'_t)$, where $r_t$ is the rule
\ZtDelta and  
and $r'_t$ is the rule \gatdisplayrule{\xDelta{n}}{\ofT{t'}{\Delta'}}.
\end{enumerate}
\end{definition}

Now we can state precisely how an interpretation $\iI$ determines an instance: 


\begin{proposition}
Suppose that $\iI$ is an interpretation of $U$  then $\iI$ is valid  iff $\iI$ determines an instance. 
If $\iI$ is valid then the instance determined by $\iI$ is $\Ibar$.
\end{proposition}
\begin{proof}
The proof is given in lemmas \lref{uniquenessofinstancedeterminedbyaninterpretation} and \lref{avalidinterpretationisaninstance}.
The proof relies on various algebraic identities regarding the operators $^*$, $\crossx{}{}{}$ and $s$ and these we give in the next section.
\end{proof}



