\begin{lemma}
\llabel{towerlemma}
\IfIpartialmappingUtoC, if $r$ is a derived rule \ZDelta of $U$ and if
$r$ is consistently interpreted by $I$ 
then \foreachi, $r_{\Delta_i}$ is consistently interpreted by $I$ and 
$1 \base I(r_{\Delta_1}) \base ... \base I(r_{\Delta_n}) \base I(r)$ in \catc,
where, for each $i$, $r_{\Delta_i}$ is the rule \IDelta{i}.
\end{lemma}
\begin{proof}
Follows from clause (i)(b) of the definition of consistent interpretation.
\end{proof}

\iffalse
\begin{lemma}
\llabel{substitutionsublemma}
\IfIpartialmappingUtoC,
then if $r_s$ is the $\in$-rule \ZsOmega and 
$r_s$ is consistently interpreted by $I$ then
for all contexts $Q$ and for all realisations $\tuple{\sm}$ of $\yOmega{m}$ wrt $Q$  
which are mapped to a cascade by $I$,
$$ \Imappedrule{Q}{\ofT{s[\SUBsFORy{m}]}{\Omega[\SUBsFORy{m}]}} 
= I(r_{s_m})^*...I(r_{s_1})^*(\crossx{I(Q)}{I(r_s)}{1})
\in Sect (I(r_{s_m})^*...I(r_{s_1})^*(\crossx{I(Q)}{I(r_\Omega)}{1})),$$
where \foreachj, $r_{s_j}$ is the rule \IsOmega{j} and where
$r_\Omega$ is the rule \ZOmega.
\end{lemma}
\begin{proof}
The definition of consistent interpretation directly requires this for non-variables. 
It remains to show  this in the case when $s$ is a variable. Suppose $s$ is the variable
$y_j$. Since in this case $s[\SUBsFORy{m}]$ is $s_j$ what we have to show is that
$I(s_j) = I(r_{s_m})^*...I(r_{s_1})^*(\crossx{I(Q)}{s(p_{I(r_{\Omega_m}),I(r_{\Omega_j})}}{1}))$,
where \foreachj, $r_{\Omega_j}$ is the rule \IOmega{j}. This is given by lemma \lref{cascadeprojectionlemma}.
\end{proof}
\fi

\begin{lemma}
\llabel{typeweakeninglemma}
\IfIpartialmappingUtoC,
if \ZOmega is a derived rule of $U$  which  is consistently interpreted by $I$ 
and if $Q$ is a context and $I(Q)$ is defined 
then $\Imappedrule{Q}{\isT{\Omega_1}}=\crossx{I(Q)}{\Imappedrule{}{\isT{\Omega_1}}}{1}$.
\end{lemma}
\begin{proof}
Since \ZOmega is consistently interpreted by $I$ then it follows by lemma \lref{towerlemma} that the rule 
\gatdisplayrule{}{\isT{\Omega_1}} (the rule with empty context and asserting $\Omega_1$ to be a type) is interpreted consistently
by $I$. Since $Q$ is a context and since the empty tuple $\tuple{}$ is a realisation of the empty context with respect to $Q$ that, by definition, maps to a cascade then it follows from the definition of what it means for  \gatdisplayrule{}{\isT{\Omega}} to be consistently interpreted by $I$ that
$\Imappedrule{Q}{\isT{\Omega_1}}=\crossx{I(Q)}{\Imappedrule{}{\isT{\Omega_1}}}{1}$, as required.
\end{proof}
The following lemma establishes that if all the elements of a realisation are consistently interpreted by a mapping then the elements are 
mapped to a cascade. 
\begin{lemma}
\llabel{realisationmapstocascade}
\newcommand {\forceSOURCEwidth}{\rule{5cm}{0pt}}  % so as to line up three different arrays
\newcommand {\forceTARGETwidth}{\rule{2.2cm}{0pt}}
If $I$ is a partial mapping of T-rules and $\in$-rules of a theory $U$ to objects, respectively sections of a contextual category \catc,
 if $Q$  and $\encyOmega{m}$ are contexts, for some $m \geq 1$,  and if $\tuple{\sm}$ is a realisation of $\tuple{\yOmega{m}}$ with respect to $Q$,
 if \foreachj, the rule \IsOmega{j} is consistently interpreted by $I$ 
 and if the rule \IOmega{m} is consistently interpreted by $I$
then the realisation $\tuple{s_1,...s_m}$ is mapped by $I$ to a cascade.
\end{lemma}
\begin{proof}
It follows from the definition of cascade that we need to establish, \foreachj, that
 $$I(r_{s_j}) \in Sect(I(r_{s_{j-1}})^*...I(r_{s_1})^*(\crossx{I(Q)}{I(r_{\Omega_j})}{1})) ,$$
  where $r_{s_j}$ is the rule \IsOmega{j} and $r_{\Omega_j}$ is the rule \IOmega{j}.
To prove this in the case of $m=1$ we need simply show that 
 $$I(r_{s_1}) \in Sect(\crossx{I(Q)}{I(r_{\Omega_1})}{1})).$$
 This follows because from the initial assumption that $r_{s_1}$ is consistently interpreted by $I$ and so by definition \lref{consistentinterpretation}, clause (ii),
 I(Q) is defined and  $I(r_{s_1}) \in Sect(I\big(\gatdisplayrule{Q}{\isT{\Omega_1}}\big))$ 
 and because \IOmega{m} is consistently interpreted by $I$ we can use lemma \lref{typeweakeninglemma} to establish $I(\gatdisplayrule{Q}{\isT{\Omega_1}}) = \crossx{I(Q)}{I(r_{\Omega_1})}{1})$.

Next we complete the proof by showing that the proposition that the lemma holds at a given $m$ follows from
the assumption that it holds at $m-1$. So assume that $Q$  and $\encyOmega{m}$ are contexts, 
that $\tuple{\sm}$ is a realisation of $\tuple{\yOmega{m}}$ with respect to $Q$,
assume that \foreachj, the rule \IsOmega{j} is consistently interpreted by $I$ 
and that the rule $r_{\Omega_m}$ is consistently interpreted by $I$. 
From these assumptions it follows
that  $\tuple{\sm[-1]}$ is a realisation of $\tuple{\yOmega{m-1}}$ with respect to $Q$.
Now since $r_{\Omega_m}$ is consistently interpreted by $I$ it follows, by definition,  that $r_{\Omega_{m-1}}$ is consistently interpreted by $I$. This means that we can use the inductive hypothesis to establish that $\tuple{\sm[-1]}$ maps to a cascade which is to say that \foreachj[m-1],  
$$I(r_{s_j}) \in Sect(I(r_{s_{j-1}})^*...I(r_{s_1})^*(\crossx{I(Q)}{I(r_{\Omega_j})}{1})).$$

It only remains to show that
$$I(r_{s_m}) \in Sect(I(r_{s_{m-1}})^*...I(r_{s_1})^*(\crossx{I(Q)}{I(r_{\Omega_m})}{1})).$$ \

This we can do because first of all, since we are given that $r_{s_m}$ is consistently interpreted by $I$ and using definition \lref{consistentinterpretation}, clause (ii), we establish that
$$I(r_{s_m}) \in Sect(\Imappedrule{Q}{\isT{\Omega_m[\SUBsFORy{m-1}]}}.$$ 
Next, because $r_{\Omega_m}$ is consistently interpreted by $I$ then using definition \lref{consistentinterpretation}, clause (i)(c), 
and since we have from the inductive hypothesis that $\tuple{\sm[-1]}$ is a realisation of $\tuple{\yOmega{m-1}}$ with respect to $Q$ that maps to a cascade,
we establish that
 $$\Imappedrule{Q}{\isT{\Omega_m[\SUBsFORy{m-1}]}}= I(r_{s_{m-1}})^*...I(r_{s_1})^*(\crossx{I(Q)}{I(r_{\Omega_m})}{1}).$$
 \end{proof}
Because every initial subtuple of a realisation is a realisation and every initial subtuple of a cascade is a cascade we have the following corollary:
 \begin{corollary}
 \llabel{subrealisationmapstocascade}
 In the situation envisaged in the previous  lemma (lemma \lref{realisationmapstocascade}), \foreachj,
 the realisation $\tuple{s_1,...s_j}$ is mapped by $I$ to a cascade. 
 \end{corollary}

\begin{lemma}
\llabel{substitutioninterpretationlemma}
\IfIpartialmappingUtoC,
if we suppose that $Q$ and $\encyOmega{m}$ are contexts, for some $m \geq 1$, 
and that $\tuple{\sm}$ is a realisation of $\encyOmega{m}$ wrt $Q$ and
if we suppose that \foreachj, the rule \IsOmega{j} is consistently interpreted by $I$ then
\begin{enumerate}[(i)]
\item if \ZOmega is a derived rule which is consistently interpreted by $I$ then \\
the derived rule \ZOmegaSUBsmFORym is consistently interpreted by $I$,
\item if \ZsOmega is a derived rule which is consistently interpreted by $I$ then \\
the derived rule \ZsOmegaSUBsmFORym is consistently interpreted by $I$.
\end{enumerate}
\end{lemma}
\begin{proof}
Suppose  $Q$ to be the context $\xDelta{n}$, for some $n \geq 0$. Let $r_{\Delta_i}$, \foreachi, be the rule \IDelta{i}.

The rule \IDelta{n} is a derived rule of $U$ because we have assumed that $Q$ is a context. 
\begin{enumerate}[(i)]
\item
\newcommand{\targetruleone}{\gatdisplayrule{\xDelta{n}}{\isT{\Omega[\SUBsFORy{m}]}}}
Let $r_\Omega$ be the rule \ZOmega and assume it to be consistently interpreted by $I$.
To show that the rule \ZOmegaSUBsmFORym is consistently interpreted by $I$ we need show that conditions (a), (b) and (c)
of clause (i) of definition \lref{consistentinterpretation} hold for this rule.
To  show  that condition (a) holds we have to show that $I(r_{\Delta_n})$ is defined 
and $\Imap{\targetruleone}$ is defined and that  $I(r_{\Delta_n}) \base \Imap{\targetruleone}$ in \catc. 
That $I(r_{\Delta_n})$ is defined follows from the initial assumption
that the rule \gatdisplayrule{\xDelta{n}}{\ofT{s_1}{\Omega_1}} is consistently interpreted by $I$
and from condition (ii)(b) of definition \lref{consistentinterpretation}. 
By  lemma \lref{towerlemma} it then follows that 
$1 \base I(r_{\Delta_1}) \base ... I(r_{\Delta_{n-1}}) \base I(r_{\Delta_n})$ in \catc.
Also because \foreachj,  we assume that the rule \gatdisplayrule{\xDelta{n}}{\ofT{s_j}{\Omega_j[\SUBsFORy{j-1}]}} is consistently interpreted by $I$, 
we deduce that $\tuple{\sm}$ is mapped to a cascade by $I$, by lemma \lref{realisationmapstocascade}. Hence 
from condition (i)(a) of definition \lref{consistentinterpretation} and from the assumption that $r_\Omega$ is consistently interpreted it follows that $\Imap{\targetruleone}$ is defined and 
\begin{equation}
\label{omegaSubsmapping}
\Imap{\targetruleone} = I(r_{s_m})^*...I(r_{s_1})^*(\crossx{I(r_{\Delta_n})}{I(r_\Omega)}{1})
\end{equation}
and thus, by lemma \lref{cascadelemma}, that $I(r_{\Delta_n}) \base \Imap{\targetruleone}$ in \catcw as required by clause (a).

In fact by the same argument and from the fact that by corollary \lref{subrealisationmapstocascade}, \foreachj,  
the realisation $\tuple{s_1,...s_{j-1}}$ is mapped to a cascade by $I$ and from lemma \lref{towerlemma} that tells us that 
$r_{\Omega_j}$ is consistently interpreted by $I$ we have that
$\displaystyle\Imappedrule{\xDelta{n}}{\isT{\Omega_j[\SUBsFORy{j-1}]}}$ is defined and 
\begin{equation}
\label{omegajSubsmapping}
\Imappedrule{\xDelta{n}}{\isT{\Omega_j[\SUBsFORy{j-1}]}} 
= I(r_{s_{j-1}})^*...I(r_{s_1})^*(\crossx{I(r_{\Delta_j})}{I(r_{\Omega_j})}{1})
\end{equation}
We use this fact in a moment in the demonstration that clause (c) is satisfied.

Next, for clause (b), we are required to show that $r_{\Delta_n}$ is consistently interpreted by $I$. 
This follows from the fact that 
the rule \gatdisplayrule{\xDelta{n}}{\ofT{s_1}{\Omega_1}}  is consistently interpreted by $I$ 
and from condition (ii)(b) of definition \lref{consistentinterpretation}.

Finally, for clause (c), assume that $P$ is a context,
assume $\tuple{\tn}$ is a realisation of $\xDelta{n}$ with respect to $P$ that is mapped to a cascade by $I$, 
we must show that
$$\Imappedrule{\xDelta{n}}{\isT{\Omega[\SUBsFORy{m}][\SUBtFORx{n}]}}
                =I(r_{t_n})^* ... I(r_{t_1})^*\big(\crossx{I(P)}{\Imappedrule{\xDelta{n}}{\isT{\Omega[\SUBsFORy{m}]}}}{1}\big),$$
where each $r_{t_i}$ is the rule $\ItDelta{i}$.
By rearrangement of the lhs and by use of (\ref{omegaSubsmapping}) this means that we need to show 
\begin{multline}
\label{ctarget}
\Imappedrule{\xDelta{n}}{\isT{\Omega[s_1[\SUBtFORx{n}]|y_1,... s_m[\SUBtFORx{n}]|y_m]}} \\
            =I(r_{t_n})^* ... I(r_{t_1})^*\Big(\crossx{I(P)}{ \big( I(r_{s_m})^*...I(r_{s_1})^*(\crossx{I(r_{\Delta_n})}{I(r_\Omega)}{1}) \big) }{1}\Big)
\end{multline}

\newcommand{\IOmegaDoublySubstituted}[1]{\Omega_#1[\SUBsFORy{#1-1}][\SUBtFORx{n}]}
Now, each $r_{s_j}$ is consistently interpreted by $I$ and $\tuple{\tn}$ is a realisation
of the context $\xDelta{n}$ with respect to $P$ that is mapped to a cascade by $I$ and therefore using 
by condition (c) of clause (ii) of definition \lref{consistentinterpretation} we have 
\begin{multline*}
\Imappedrule{P}{\ofT{s_j[\SUBtFORx{n}]}{\IOmegaDoublySubstituted{j}}}
           = I(r_{t_n})^* ... I(r_{t_1})^*\big(\crossx{I(P)}{I(r_{s_j})} {1}\big) \\
           \in Sect\Big(I(r_{t_n})^* ... I(r_{t_1})^*\Big(\crossx{I(P)}{\Imappedrule{\xDelta{n}}{\isT{\Omega_j[\SUBsFORy{j-1}]}}} {1}\Big)\Big)
\end{multline*}

and since
\begin{align*}
I(r_{t_n})^* ... I(r_{t_1})^*\Big(\crossx{I(P)}{\Imappedrule{\xDelta{n}}{\isT{\Omega_j[\SUBsFORy{j-1}]}}} {1}\Big)\kern-8.5cm\\
        & = I(r_{t_n})^* ... I(r_{t_1})^*\Big(\crossx{I(P)}{ I(r_{s_{j-1}})^*...I(r_{s_1})^*\big(\crossx{I(Q)}{I(r_{\Omega_j})}{1}\big)} {1}\Big) 
                                                               &&\mbox{by (\lref{omegajSubsmapping}),}\\
        & =  \Big(I(r_{t_n})^*...I(r_{t_1})^*(\crossx{I(r_{\Delta_n})}{I(r_{s_{j-1}})}{1})\Big)
               ^* ... 
             \Big(I(r_{t_n})^*...I(r_{t_1})^*(\crossx{I(r_{\Delta_n})}{I(r_{s_1})}{1})\Big)
               ^*\Big(\crossx{I(P)}{I(r_{\Omega_j})}{1}\Big)
                                                            &&\mbox{by lemma \lref{cascadedpullbackscohere}}
\end{align*}
then we have
\begin{multline*}
\Imappedrule{P}{\ofT{s_j[\SUBtFORx{n}]}{\IOmegaDoublySubstituted{j}}}
           = I(r_{t_n})^* ... I(r_{t_1})^*\big(\crossx{I(P)}{I(r_{s_j})} {1}\big) \\
\in Sect\bigg( \Big(I(r_{t_n})^*...I(r_{t_1})^*(\crossx{I(r_{\Delta_n})}{I(r_{s_{j-1}})}{1})\Big)
               ^* ... 
             \Big(I(r_{t_n})^*...I(r_{t_1})^*(\crossx{I(r_{\Delta_n})}{I(r_{s_1})}{1})\Big)
               ^*\Big(\crossx{I(P)}{I(r_{\Omega_j})}{1}\Big) 
        \bigg)
\end{multline*}

and so 
$\tuple{s_1[\SUBtFORx{n}],...s_m[\SUBtFORx{n}]}$ is a realisation of $\yOmega{m}$ with repect to $P$
that is mapped to a cascade by $I$
and we can use the initial assumption that $r_\Omega$ is consistently interpreted by $I$ 
to establish that
\begin{multline}
     \Imappedrule{\xDelta{n}}{\isT{\Omega\big[s_1[\SUBtFORx{n}]|y_1,... s_m[\SUBtFORx{n}]|y_m\big]}} \\
           =  \Big(I(r_{t_n})^*...I(r_{t_1})^*\big(\crossx{I(P)}{I(r_{s_m})}{1}\big)\Big)
               ^* ... 
              \Big(I(r_{t_n})^*...I(r_{t_1})^*\big(\crossx{I(P)}{I(r_{s_1})}{1}\big)\Big)
               ^*\Big(\crossx{I(P)}{I(r_\Omega)}{1}\Big)
\end{multline}
the rhs of which can be rearranged using equation (\lref{cascadedpullbackscohereonobjects}) of lemma \ref{cascadedpullbackscohere}   so as to establish
equation (\ref{ctarget}) as required to show that condition (c) holds.

\item 
\newcommand{\targetruletwo}{\gatdisplayrule{\xDelta{n}}{\ofT{s[\SUBsFORy{m}]}{\Omega[\SUBsFORy{m}]}}}
Now let $r_s$ be the rule \ZsOmega and assume this rule to be consistently interpreted by $I$ as defined
in clause (ii) of definition \lref{consistentinterpretation}. Condition (b) of the definition tells us that
\targetruleone must be consistently interpreted by $I$ and therefore in this part (ii) of the proof we can use
anything or everything established in part (i) of this proof.

We need to show that the rule \ZsOmegaSUBsmFORym is consistently interpreted by $I$ and for this we need show that conditions (a), (b), (c) and (d)
of clause (ii) of definition \lref{consistentinterpretation} hold for this rule.

To show that condition (a) holds we need show that $I(r_{\Delta_n})$ and $\Imap{\targetruleone}$ are defined, which we have established already in part (i),
and that $$\Imap{\targetruletwo} \in Sect\big(\Imap{\targetruleone}\big).$$
This follows from the initial assumption that \ZsOmega is consistently interpreted by $I$ 
and because we have already established that $\tuple{\sm}$ is a realisation which is mapped to a cascade by $I$
and from which we can use condition (c) of clause (ii) of lemma \ref{consistentinterpretation} to deduce that
\begin{multline}
\label{targetrule2mapping}
\Imap{\targetruletwo} = I(r_{s_m})^*...I(r_{s_1})^*(\crossx{I(r_{\Delta_n})}{I(r_s)}{1})\\
                      \in Sect\big(\Imap{\targetruleone}\big)
\end{multline}
To show that condition (b) holds we have to show that the rule \targetruleone is consistently interpreted by $I$. 
This is established in part (i) of this proof.

To show that condition (c) holds assume that $P$ is a context and $\tuple{\tn}$ is a realisation as described in part (i) of this proof.
We need to show that 
\begin{multline}
\Imappedrule{\xDelta{n}}{\ofT{s[\SUBsFORy{m}][\SUBtFORx{n}]}{\Omega[\SUBsFORy{m}][\SUBtFORx{n}]}} \\
                =I(r_{t_n})^* ... I(r_{t_1})^*\big(\crossx{I(P)}{\Imappedrule{\xDelta{n}}{\ofT{s[\SUBsFORy{m}]}{\Omega[\SUBsFORy{m}]}}}{1}\big),
\end{multline}

which after rearangement of the lhs and use of (\ref{targetrule2mapping}) means that we need to show that

\begin{multline}
\label{ctargettwo}
\Imappedrule{\xDelta{n}}{\ofT{s[s_1[\SUBtFORx{n}]|y_1,... s_m[\SUBtFORx{n}]|y_m]}{\Omega[s_1[\SUBtFORx{n}]|y_1,... s_m[\SUBtFORx{n}]|y_m]}}\\
                =I(r_{t_n})^* ... I(r_{t_1})^*\big(\crossx{I(P)}{I(r_{s_m})^*...I(r_{s_1})^*(\crossx{I(r_{\Delta_n})}{I(r_s)}{1})}{1}\big)
\end{multline}
Similarly to part (i) of this proof, 
this follows because $\tuple{s[s_1[\SUBtFORx{n}]|y_1,... s_m[\SUBtFORx{n}]|y_m]}$
is a realisation which is mapped by $I$ to a cascade,
from the initial assumption that 
\ZsOmega is a derived rule which is consistently interpreted by $I$
and by use of equation (\lref{cascadedpullbackscohereonsections}) of lemma \ref{cascadedpullbackscohere}.

Finally, we need show that condition (d) holds in the case that the expression $s[\SUBsFORy{m}]$ is simply a variable. For this to be the case the
expression $s$ must be a variable, $y_j$, say, and the expression $s_j$ must be a variable $x_i$, say,  
so that the expression $s[\SUBsFORy{m}]$ is simply the variable $x_i$. To show that condition (d) holds we need to show that
$$\Imap{\gatdisplayrule{\xDelta{n}}{\ofT{x_i}{\Omega[\SUBsFORy{m}]}}}=s(p_{r_{\Delta_n},r_{\Delta_i}}).$$
This is established as follows:
\begin{align*}
\Imap{\gatdisplayrule{\xDelta{n}}{\ofT{x_i}{\Omega[\SUBsFORy{m}]}}}\kern-3cm\\
&=I(r_{s_m})^*...I(r_{s_1})^*\Big(\crossx{I(r_{\Delta_n})}{\Imap{\gatdisplayrule{\yOmega{m}}{\ofT{y_j}{\Omega}}}} {1}  \Big) \kern-1.5cm
                                                                               && \kern1.5cm \parbox[t]{4.5cm}{by condition (c) of clause (ii) of definition \lref{consistentinterpretation}, }\\
&=I(r_{s_m})^*...I(r_{s_1})^*\Big(\crossx{I(r_{\Delta_n})} {s(p_{r_{\Omega_m},r_{\Omega_j}})} {1}  \Big) 
                                                                               && \mbox{because $r_s$ is consistently interpreted by $I$, } \\
&=I(r_{s_j})                                                                   && \mbox{by lemma \lref{cascadeprojectionlemma}}, \\
&=s(p_{\Delta_n,\Delta_i})                  && \mbox{because $r_{s_j}$ is  \gatdisplayrule{\xDelta{n}}{\ofT{x_i}{\Omega_j[\SUBsFORy{j-1}]}} }\\
&                                                                              &&  \mbox{and this is consistently interpreted by $I$.} \\
\end{align*}
\end{enumerate}
\end{proof}

\begin{lemma}
\llabel{Ibartowerlemma}
Suppose $\iI$ is an interpretation of generalised algebraic theory $U$ in contextual category \catc,
that  $r_\Omega$ is a derived T-rule \ZOmega of  $U$
and that \foreachj, $r_{\Omega_j}$ is the derived rule
\IOmega{j},
if  $\Ibar(r_\Omega)$ is defined then
$\Ibar(r_{\Omega_j})$ is defined \foreachj, and 
$1 \base \Ibar(r_{\Omega_1}) \base  ... \base \Ibar(r_{\Omega_m}) \base \Ibar(r_\Omega)$ in \catc.
\end{lemma}
\begin{proof}
It follows from the definition of $\Ibar$ that since $\Ibar(r_\Omega)$ is defined that $\Ibar(r_{\Omega_m})$ is defined and that
$\Ibar(r_{\Omega_m}) \base \Ibar(r_\Omega)$ in \catc. Now we can repeat and argue that $\Ibar(r_{\Omega_{m-1}})$ is defined and that $\Ibar(r_{\Omega_{m-1}}) \base \Ibar(r_{\Omega_m})$
in \catc. By induction $\Ibar(r_{\Omega_1}) \base  ... \base \Ibar(r_{\Omega_m}) \base \Ibar(r_\Omega)$ in \catc, as required. 
That $1 \base \Ibar(r_{\Omega_1})$ follows because $\Ibar$ is defined to map the empty context to the object $1$.
\end{proof}


\begin{lemma}
\llabel{uniquenesssublemma}
If $I$ and $I'$ are instances of a generalised algebraic theory $U$ in a contextual category \catcw and if \gatdisplayrule{P}{\ofT{t}{\Delta}}
is a derived rule of $U$ such that $\Imap{\gatdisplayrule{P}{\ofT{t}{\Delta}}}=\Ipmap{\gatdisplayrule{P}{\ofT{t}{\Delta}}}$
then
\begin{enumerate}[(i)]
\item
$\Imap{\gatdisplayrule{P}{\isT{\Delta}}}=\Ipmap{\gatdisplayrule{P}{\isT{\Delta}}}$
\item
$I(P)=I'(P)$.
\end{enumerate}
\end{lemma}
\begin{proof}
\begin{enumerate}[(i)]
\item
The rule \gatdisplayrule{P}{\ofT{t}{\Delta}} is a derived rule of $U$ and so 
it is consistently interpreted by $I$ since $I$ is an instance and therefore 
$\Imap{\gatdisplayrule{P}{\ofT{t}{\Delta}}} \in Sect \big(\Imap{\gatdisplayrule{P}{\isT{\Delta}}})$ by condition (a) of clause (ii) of definition \lref{consistentinterpretation}.
Likewise it follows that $\Ipmap{\gatdisplayrule{P}{\ofT{t}{\Delta}}} \in Sect \big(\Ipmap{\gatdisplayrule{P}{\isT{\Delta}}})$.
But $\Imap{\gatdisplayrule{P}{\ofT{t}{\Delta}}} = \Ipmap{\gatdisplayrule{P}{\ofT{t}{\Delta}}}$ and therefore
$Sect \big(\Imap{\gatdisplayrule{P}{\isT{\Delta}}})=Sect \big(\Ipmap{\gatdisplayrule{P}{\isT{\Delta}}})$
and therefore $\Imap{\gatdisplayrule{P}{\isT{\Delta}}}=\Ipmap{\gatdisplayrule{P}{\isT{\Delta}}}$.
\item
The rule \gatdisplayrule{P}{\isT{\Delta}} is a derived rule of $U$ and so 
it is consistently interpreted by $I$ since $I$ is an instance and therefore 
$I(P) \base \Imap{\gatdisplayrule{P}{\isT{\Delta}}}$ in \catcw by condition (a) of clause (i) of definition \lref{consistentinterpretation} . 
Likewise it follows that  $I'(P) \base \Ipmap{\gatdisplayrule{P}{\isT{\Delta}}}$ in \catc. 
But $\Imap{\gatdisplayrule{P}{\isT{\Delta}}} = \Ipmap{\gatdisplayrule{P}{\isT{\Delta}}}$ and therefore $I(P)=I'(P)$. 
\end{enumerate}
\end{proof}
\begin{lemma} 
\llabel{uniquenessofinstancedeterminedbyaninterpretation}
If $\iI$ is an interpretation of generalised algebraic theory $U$ in contextual catgeory \catcw that determines instances $I$ and $I'$ of $U$ then
$I=I'$. 
\end{lemma}
\begin{proof}
We show that for all derived  T- and $\in$-rules of $U$, $I(r)=I'(r)$.  
The proof proceeds by induction on the derivation of $r$. 
We need only consider principles T1, CF1, CF2(a) and CF2(b), for these are the only principles by which  T-rules and $\in$-rules may be derived.\\
\underline{T1}
By this principle from \gatdisplayrule{P}{\Delta=\Delta'} and \gatdisplayrule{P}{\ofT{t}{\Delta}} we can derive \gatdisplayrule[.]{P}{\ofT{t}{\Delta'}}
Assume the inductive hypothesis that $\Imap{\gatdisplayrule{P}{\ofT{t}{\Delta}}} = \Ipmap{\gatdisplayrule{P}{\ofT{t}{\Delta}}}$.
We need show that $\Imap{\gatdisplayrule{P}{\ofT{t}{\Delta'}}} = \Ipmap{\gatdisplayrule{P}{\ofT{t}{\Delta'}}}$.
This follows because from the definition of instance we have that $\Imap{\gatdisplayrule{P}{\ofT{t}{\Delta}}} = \Imap{\gatdisplayrule{P}{\ofT{t}{\Delta'}}}$
and $\Ipmap{\gatdisplayrule{P}{\ofT{t}{\Delta}}} = \Ipmap{\gatdisplayrule{P}{\ofT{t}{\Delta'}}}$.
\\

\underline{CF1} According to this principle, 
whenever a rule of the form \gatdisplayrule{\xDelta{n}}{\isT{\Delta_{n+1}}} is a derived rule of $U$ for some $n \geq 0$
then so to is the rule \gatdisplayrule{\xDelta{n+1}}{\ofT{x_i}{\Delta_{i}}}, \foreachi[n+1], where $x_{n+1}$ is any variable distinct from each of the variables $\xn$. For each $i$, $1 \leq i \leq n+1$, let $r_{\Delta_i}$ be the rule \IDelta{i}. Assume the inductive hypothesis that
$I(r_{\Delta_{n+1}})=I'(r_{\Delta_{n+1}})$. Because $I$ and $I'$ are instances they both consistently interpret the rule $r_{\Delta_{n+1}}$
and so it follows by lemma \lref{towerlemma} that
$I(r_{\Delta_i})=I'(r_{\Delta_i})$, \foreachi. Now, since \gatdisplayrule{\xDelta{n+1}}{\ofT{x_i}{\Delta_{i}}} is a derived rule
of $U$ and is therefore consistently interpreted by both $I$ and $I'$
it follows by condition (d) of clause (ii) of defintion \lref{consistentinterpretation} that
$
\Imap{\gatdisplayrule{\xDelta{n+1}}{\ofT{x_i}{\Delta_{i}}}}
=s(p_{I(r_{\Delta_{n+1}}),I(r_{\Delta_{i}})})
=s(p_{I'(r_{\Delta_{n+1}}),I'(r_{\Delta_{i}})})
=\Ipmap{\gatdisplayrule{\xDelta{n+1}}{\ofT{x_i}{\Delta_{i}}}}
$, as required.

\underline{CF2(a)}\ By this principle, from a sort symbol $A$ with introductory rule $r_A$ of the form \gatdisplayrule[,]{\xDelta{n}}{\isT{A(\xn)}} for  $n \geq 0$, 
if the rule \gatdisplayrule[,]{\xDelta{n-1}}{\isT{\Delta_n}} is a derived rule of $U$,
if $P$ is a context and if in particular the rule $r_P$ asserting that $P$ is a context is a derived rule of $U$
and from derived rules $r_{t_i}$ of the form \ItDelta[P]{i}, \foreachi, we may deduce
the rule \gatdisplayrule{P}{\isT{A(t_1,...t_n)}}, which we shall denote by $r$, is a derived rule of $U$. 
Because by lemma \lref{everyintroruleisderived} the introductory rule for $A$ is a derived rule of $U$  and is therefore consistently interpreted by $I$
since $I$ is an instance, and since  each rule $r_{t_i}$ is consistently interpreted by $I$ we have by lemma \lref{realisationmapstocascade}
that $\tuple{\tn}$ is mapped to a cascade by $I$ and hence by \condition{i}{c}
$I(r_{t_n})^* ... I(r_{t_1})^*\big(\crossx{I(P)}{I(r_A)}{1}\big)$, where $r_A$ is the introductory rule for $A$.
Similarly $r$ is mapped by $I'$ to
$I'(t_n)^* ... I'(t_1)^*\big(\crossx{I'(P)}{I'(r_A)}{1}\big)$.
By the inductive hypothesis we have both that $I(r_{t_i})=I'(r_{t_i})$, for each $i$, and that $I(P)=I'(P)$,  and 
because both $I$ and $I'$ are instances which extend interpretation $\iI$
we have that $I(r_A)=\iI(f)=I'(r_A)$. Therefore $I(r)=I'(r)$.
\\

\underline{CF2(b)}\ By this principle, from a operator symbol $f$ with introductory rule $r_f$ of the form \gatdisplayrule[,]{\xDelta{n}}{\ofT{f(\xn)}{\Delta}} for some $n \geq 0$, if the rule \gatdisplayrule[,]{\xDelta{n}}{\isT{\Delta}} is a derived rule of $U$,
if $P$ is a context and if in particular the rule $r_P$ asserting that $P$ is a context is a derived rule of $U$
and from derived rules $r_{t_i}$ of the form \ItDelta[P]{i}, \foreachi, we may deduce
the rule \gatdisplayrule{P}{\ofT{f(t_1,...t_n)}{\Delta[\SUBtFORx{n}]}}, which we shall denote by $r_t$, is a derived rule of $U$. 
Because by lemma \lref{everyintroruleisderived}the introductory rule for $f$ is a derived rule of $U$ and therefore consistently interpreted by $I$
since $I$ is an instance, and because, as above,  $\tuple{\tn}$ is mapped to a cascade by $I$,
by condition (c) of clause (ii) of definition \lref{consistentinterpretation} the rule $r_t$ is mapped  by $I$ to
$I(r_{t_n})^* ... I(r_{t_1})^*\big(\crossx{I(P)}{I(r_f)}{1}\big)$, where $r_f$ is the introductory rule for $f$.
Similarly $r_t$ is mapped by $I'$ to
$I'(t_n)^* ... I'(t_1)^*\big(\crossx{I'(P)}{I'(r_f)}{1}\big)$.
By the inductive hypothesis we have both that $I(r_{t_i})=I'(r_{t_i})$, for each $i$, and that $I(P)=I'(P)$,  and 
because both $I$ and $I'$ are instances which extend interpretation $\iI$
we have that $I(r_f)=\iI(f)=I'(r_f)$. Therefore $I(r_t)=I'(r_t)$.
\end{proof}

\begin{definition}
We will say that a partial mapping I  
of derived T-rules and $\in$-rules of the theory $U$ to objects, respectively sections of the contextual category \catcw
is \term{type independent} on   $\in$-rules iff whenever
\gatdisplayrule{P}{\ofT{t}{\Delta}} and \gatdisplayrule{P}{\ofT{t}{\Delta'}} are derived rules of $U$ then 
$\Imap{\gatdisplayrule{P}{\ofT{t}{\Delta}}}=\Imap{\gatdisplayrule{P}{\ofT{t}{\Delta'}}}$.
\end{definition}

\begin{lemma}
\llabel{T1preservesconsistentinterpretation}
If $I$ is a partial mapping that is type independent on $\in$-rules,
if \gatdisplayrule{P}{\Delta=\Delta'} and \gatdisplayrule{P}{\ofT{t}{\Delta}} are derived rules of $U$ 
which are consistently interpreted by $I$ then the rule \gatdisplayrule{P}{\ofT{t}{\Delta'}}
is a derived rule of $U$ and is consistently interpreted by $I$.
\end{lemma}
\begin{proof}
That the rule \gatdisplayrule{P}{\ofT{t}{\Delta'}} is derived follows by application of principle T1 to the given derived rules.

Now, because \gatdisplayrule{P}{\Delta=\Delta'} is consistently interpreted by $I$ we know by \clause{iii} that
\begin{equation}
\label{T1lemmaEq1}
\Imap{\gatdisplayrule{P}{\isT{\Delta}}} = \Imap{\gatdisplayrule{P}{\isT{\Delta'}}}.
\end{equation}
Because \gatdisplayrule{P}{\ofT{t}{\Delta}} is consistently interpreted by $I$ we know by \condition {ii}{b} that
\begin{equation}
\label{T1lemmaEq2}
\gatdisplayrule{P}{\ofT{t}{\Delta}} \in Sect(\Imap{\gatdisplayrule{P}{\isT{\Delta}}}).
\end{equation}
Because $I$ is type independent on $\in$-rules we know, by defintion,  that
\begin{equation}
\label{T1lemmaEq3}
\Imap{\gatdisplayrule{P}{\ofT{t}{\Delta}}} = \Imap{\gatdisplayrule{P}{\ofT{t}{\Delta'}}}.
\end{equation}

We have to show that conditions (a), (b), (c) and (d) of \clause{ii} hold. We have to show
\begin{enumerate}[(a)]
\item $\Imap{\gatdisplayrule{P}{\ofT{t}{\Delta'}}}$ is defined
and that $\Imap{\gatdisplayrule{P}{\ofT{t}{\Delta'}}} \in Sect(\Imap{\gatdisplayrule{P}{\isT{\Delta'}}})$.
That  $\Imap{\gatdisplayrule{P}{\ofT{t}{\Delta'}}}$ is defined follows because
$I$ is type independent on $\in$-rules.
 That $\Imap{\gatdisplayrule{P}{\ofT{t}{\Delta'}}} \in Sect(\Imap{\gatdisplayrule{P}{\isT{\Delta'}}})$
  follows from (\ref{T1lemmaEq1}), (\ref{T1lemmaEq2}) and (\ref{T1lemmaEq3}).\\
\item
\gatdisplayrule{P}{\isT{\Delta'}} is consistently interpreted by $I$. This  follows from clause (iii) of
definition \lref{consistentinterpretation} because \gatdisplayrule{P}{\Delta=\Delta'} is consistently interpreted by $I$.

\item
That for certain appropriate realisations $\tuple{\sm}$,
$$ \Imappedrule{Q}{\ofT{t[\SUBsFORy{m}]}{\Delta'[\SUBsFORy{m}]}} = I(r_{s_m})^*...I(r_{s_1})^*(\crossx{I(Q)}{I(r_{t\Delta'})}{1}),$$
where  $r_{t\Delta'}$ is \gatdisplayrule{P}{\ofT{t}{\Delta'}}.
This follows because under these same conditions we know that
$$ \Imappedrule{Q}{\ofT{t[\SUBsFORy{m}]}{\Delta[\SUBsFORy{m}]}} = I(r_{s_m})^*...I(r_{s_1})^*(\crossx{I(Q)}{I(r_{t\Delta})}{1})$$
where  $r_{t\Delta}$ is \gatdisplayrule{P}{\ofT{t}{\Delta}},
and because both $I(r_{t\Delta})=I(r_{t\Delta'})$ and 
$$\Imappedrule{Q}{\ofT{t[\SUBsFORy{m}]}{\Delta'[\SUBsFORy{m}]}} = \Imappedrule{Q}{\ofT{t[\SUBsFORy{m}]}{\Delta'[\SUBsFORy{m}]}}$$
because $I$ is type independent on $\in$-rules.
\item
In the case that $t$ is a variable $y_j$ then
$$I(r_{t_\Delta'})=s(p_{r_{\Omega_m},r_{\Omega_j}}).$$ This too follows because $I$ is type independent on $\in$-rules.
\end{enumerate}
\end{proof}

\begin{lemma}
\llabel{Ibarintro1}
Suppose that $\iI$ is an interpretation of $U$  and that $\iI$ is valid  then 
\begin{enumerate}[(a)]
\item for every sort symbol $A$ of $U$ with introductory rule $r_A$ of the form \gatdisplayrule[,]{\xDelta{n}}{\isT{A(\xn)}} for some $n \geq 0$,
$\Ibar(r_A)$ is defined and $\Ibar(r_A)=\iI_{sort}(A)$,
\item for every operator symbol $f$ of $U$ with introductory rule $r_f$ of the form \gatdisplayrule[,]{\xDelta{n}}{\ofT{f(\xn)}{\Delta}} for some $n \geq 0$,
$\Ibar(r_f)$ is defined $\Ibar(r_f)=\iI_{op}(f)$.
\end{enumerate}
\end{lemma}
\begin{proof}
\begin{enumerate}[(i)]
\item
Assume $A$ is such a sort symbol of theory $U$.
For each $i$, $1 \leq i \leq n$, let $r_{x_i}$ be the rule \gatdisplayrule{\xDelta{n}}{\ofT{x_i}{\Delta_i}} and let $r_{\Delta_i}$ by the rule \IDelta{i}.
From the assumption that $\iI$ is valid, and from what it means for an interpretation to be valid, it follows that  $\Ibar(r_A)$ is defined.

From the definition of $\Ibar$ it follows therefore that $\Ibar(r_{x_i})$ is defined, \foreachi, and we have
\begin{align*}
\Ibar(r_A)
   &=\Ibar(r_{x_i})^*...\Ibar(r_{x_i})^*\big(\crossx{\Ibar(r_{\Delta_n})}{\iI_{sort}(A)}{1}\big)
   &&\mbox{ by clause (i) of the definition of $\Ibar$,}\\
   &=s(p_{\Ibar(r_{\Delta_n}),\Ibar(r_{\Delta_n})})^*...s(p_{\Ibar(r_{\Delta_n}),\Ibar(r_{\Delta_1})})^*\big(\crossx{\Ibar(r_{\Delta_n})}{\iI_{sort}(A)}{1}\big)
   &&\mbox{by clause (ii) of the definition of $\Ibar$,} \\
   &={p_{\Ibar(r_{\Delta_n}),\Ibar(r_{\Delta_n})}}^* \iI_{sort}(A)
   &&\mbox{by identity (a) of lemma \lref{sofpsubstitutionlemma}, } \\
   &={id_{\Ibar(r_{\Delta_n})}}^* \iI_{sort}(A)
   &&\mbox{by definition of $p_{\Ibar(r_{\Delta_n}),\Ibar(r_{\Delta_n})}$, } \\
   &=\iI_{sort}(A)
   &&\mbox{by (q3), as required.}
\end{align*} 
\item For an operator symbol $f$ the proof is similar to the above, but in this case using clause (b) of lemma \lref{sofpsubstitutionlemma}.
\end{enumerate}
\end{proof}

\begin{lemma}
\llabel{Ibarconsistentonintros}
Suppose that $\iI$ is an interpretation of $U$  and that $\iI$ is valid  then 
\begin{enumerate}[(i)]
\item for every sort symbol $A$ of $U$ with introductory rule $r_A$ of the form \gatdisplayrule[,]{\xDelta{n}}{\isT{A(\xn)}} for some $n \geq 0$,
if the rule $r_{\Delta_n}$ is consistently interpreted by $\Ibar$ then the rule $r_A$ is consistently interpreted by $\Ibar$,
\item for every operator symbol $f$ of $U$ with introductory rule $r_f$ of the form \gatdisplayrule[,]{\xDelta{n}}{\ofT{f(\xn)}{\Delta}} for some $n \geq 0$,
if the rule $r_{\Delta}$ is consistently interpreted by $\Ibar$ then the rule $r_f$ is consistently interpreted by $\Ibar$.
\end{enumerate}
\end{lemma}
\begin{proof}
\begin{enumerate}[(i) ]
    \item We have to show that conditions (a), (b) and (c) of \clause{i} hold.
    \begin{enumerate}[(a)]
        \item We need show that $\Ibar(r_A)$ is defined and that $\Ibar(r_{\Delta_n})$ is defined and that $\Ibar(r_{\Delta_n}) \base \Ibar(r_A)$ in \catc.
        That $\Ibar(r_A)$ is defined is shown in lemma \lref{Ibarintro1}. That $\Ibar(r_{\Delta_n}) \base \Ibar(r_A)$ in \catcw follows from 
        the fact that $\Ibar(r_A)$ is defined and from the implication that precondition (b) of clause (i) of the definition of $\Ibar$ must therefore be met.
        \item We need show that $r_{\Delta_n}$ is consistently interpreted by $\Ibar$. This we have as an assumption.
        \item we need show that for all contexts $Q$ and for any realisation $\tuple{s_1,...s_n}$ of $\xDelta{n}$ wrt $Q$
        which is mapped to a cascade by $\Ibar$,
        $$\ibarmappedrule{Q}{A(s_1,...s_n)}=I(r_{s_n})^*I(r_{s_1})^*\big(\crossx{I(Q)}{I(r_A)}{1}\big)$$
        where $r_{s_i}$, \foreachi, is the rule \gatdisplayrule{Q}{\ofT{s_i}{\Delta_i[s_{i-1}|x_{i-1},...s_1|x_1]}}.
        This follows from the definition of $\Ibar$ provided that we can show that the preconditions (a), (b) and (c). 
        We can do this as follows.
        \begin{enumerate}[(a)]
            \item That $\Ibar$ is defined at the context $Q$ follows from the definition of realisation $\sntuple$ being mapped to a cascade.
            \item That \foreachi, the rule $\IDelta{i}$, which we denote $r_{\Delta_i}$, is mapped by $\Ibar$ to some object $\Ibar(r_{\Delta_i})$ of \catcw
            and $1 \base \Ibar(r_{\Delta_1}),...\Ibar(r_{\Delta_n}) \base \iI_{sort}(A)$ in \catcw. 
            This follows from the definition of $\Ibar$ because we have shown that $\Ibar(r_a)$ is defined
            and that $\Ibar(r_A)= \iI_{sort}(A)$. 
           \item That $\tuple{s_1,...s_n}$ is mapped to a cascade by $\Ibar$. This is one of the assumptions.
        \end{enumerate}
    \end{enumerate}
    \item Very similar to the above.
\end{enumerate}
\end{proof}


Now for the main lemma:
\begin{lemma}
\llabel{avalidinterpretationisaninstance}
Suppose that $\iI$ is an interpretation of $U$  and that $\iI$ is valid  then $\Ibar$ is an instance determined by $\iI$.
\end{lemma}
\begin{proof} 
\newcommand {\forceSOURCEwidth}{\rule{5cm}{0pt}}  % so as to line up three different arrays
\newcommand {\forceTARGETwidth}{\rule{2.2cm}{0pt}}
First note that by examining clause (ii) of the definition of $\Ibar$ we can see that $\Ibar$ is  type independent 
on $\in$-rules. 

We need to show that for every derived rule $r$ of $U$, $\Ibar(r)$ is defined and $r$ 
is consistently interpreted\footnote{The reason that we have defined
\textit{consistently interpreted} for partial mappings of rules and not just for total mappings is because  here -- at the point that we use the definition -- we have not yet proved $\Ibar$ to be total.} by $\Ibar$. 
We prove this by induction on the derivation of rules in $U$. We examine each of the principles of derivation in turn
and show that from rules for which $\Ibar$ is defined and which are consistently interpreted by $\Ibar$ 
it is only possible to derive rules for which $\Ibar$ is defined and which themselves are consistently interpreted by $\Ibar$.
The principles of derivation (see \cite{Cartmell86}) are LI1, ... LI7, T1, CF1, CF2(a) and CF2(b), SI1 and SI2. 
The proof of this  in each the cases  LI1,...LI6 is quite trivial. We consider each of the remaining principles in turn. 
 \\
\underline{LI7} 
By this principle from \gatdisplayrule{P}{t_1 = t_2 \in \Delta_1} and \gatdisplayrule{P}{\Delta_1=\Delta_2} derive
\gatdisplayrule{P}{t_1 = t_2 \in \Delta_2}. 
Assume the inductive hypothesis that \gatdisplayrule{P}{t_1 = t_2 \in \Delta_1} is consistently interpreted by $\iI$,
so that
\gatdisplayrule{P}{\ofT{t_1}{\Delta_1}} and
\gatdisplayrule{P}{\ofT{t_2}{\Delta_1}} are consistently interpreted by $\Ibar$
and  
\begin{equation}
\label{LI7t1t2delta1}
\ibarmappedrule{P}{\ofT{t_1}{\Delta_1}}=\ibarmappedrule{P}{\ofT{t_2}{\Delta_1}}
\end{equation}
and that \gatdisplayrule{P}{\Delta_1=\Delta_2} is consistently interpreted by $\Ibar$.

To show that the conclusion, \gatdisplayrule{P}{t_1 = t_2 \in \Delta_2}, is consistently interpreted by $\Ibar$ we need show that the derived rules 
\gatdisplayrule{P}{\ofT{t_1}{\Delta_2}} and
\gatdisplayrule{P}{\ofT{t_2}{\Delta_2}} are consistently interpreted by $\Ibar$
and that 
\begin{equation}
\label{LI7t1t2delta2}
\ibarmappedrule{P}{\ofT{t_1}{\Delta_2}}=\ibarmappedrule{P}{\ofT{t_2}{\Delta_2}}.
\end{equation}
 
That  \gatdisplayrule{P}{\ofT{t_1}{\Delta_2}}  is consistently interpreted by $\Ibar$ follows 
from the fact that \gatdisplayrule{P}{\ofT{t_1}{\Delta_1}}  and \gatdisplayrule{P}{\Delta_1=\Delta_2} 
are consistently interpreted by $\Ibar$ by lemma \lref{T1preservesconsistentinterpretation}.
That \gatdisplayrule{P}{\ofT{t_2}{\Delta_2}}   is consistently interpreted by $\Ibar$ follows in the same way.

Equation (\ref{LI7t1t2delta2}) follows from (\ref{LI7t1t2delta1}) from the earlier observation that $\Ibar$ is type independent on $\in$-rules. \\
\underline{T1}
By this principle from \gatdisplayrule{P}{\Delta=\Delta'} and \gatdisplayrule{P}{\ofT{t}{\Delta}} we can derive \gatdisplayrule[.]{P}{\ofT{t}{\Delta'}}
By the inductive hypothesis we assume that the rules   \gatdisplayrule{P}{\Delta=\Delta'} and \gatdisplayrule{P}{\ofT{t}{\Delta}} are
consistently interpreted by $\Ibar$. We have to show that the rule \gatdisplayrule[.]{P}{\ofT{t}{\Delta'}} is consistently interpreted by $\Ibar$.
This is proven in lemma \lref{T1preservesconsistentinterpretation}. \\
%
%
\underline{CF1} According to the principle, 
whenever a rule of the form \gatdisplayrule{\xDelta{n}}{\isT{\Delta_{n+1}}} is a derived rule of $U$ 
then so to is the rule \gatdisplayrule{\xDelta{n+1}}{\ofT{x_i}{\Delta_{i}}}, \foreachi[n+1], 
where $x_{n+1}$ is any variable distinct from each of the variables $\xn$.
Let $r_{x_i}$ be the rule \gatdisplayrule{\xDelta{n+1}}{\ofT{x_i}{\Delta_{i}}}.
We have to show that if $\Ibar$ is defined for and consistently interprets any such derived T-rule
\gatdisplayrule{\xDelta{n}}{\isT{\Delta_{n+1}}}, 
which we will denote, $r_{\Delta_{n+1}}$, 
then so to is $\Ibar$ defined for and consistently interprets each associated rule $r_{x_i}$. 
Since $\Ibar$ is defined for $r_{\Delta_{n+1}}$ it follows by lemma \lref{Ibartowerlemma} that 
$\Ibar(r_{\Delta_i})$ is defined \foreachi[n+1], where $r_{\Delta_i}$ is the rule \gatdisplayrule{\xDelta{i-1}}{\isT{\Delta_{i}}}.
Denote by $a_i$ the object  $\Ibar(r_{\Delta_i})$ \foreachi[n+1]. By lemma \lref{Ibartowerlemma}
it follows that $1 \base a_1 ... \base a_n \base a_{n+1}$ in \catc.  
It suffices to show show that \foreachi[n+1],
if it is the case that $\Ibar$ is defined at and consistently interprets every rule $r_{x_j}$ all $j <i$ 
then it follows that $\Ibar$ is defined at and consistently interprets the rule $r_{x_i}$. 

So, suppose such an $i$ and suppose that for each $j$, $j < i$, 
that $\Ibar$ is defined for and consistently interprets the rule $r_{x_j}$. 
We shall now show that it follows that 
$\Ibar$ is defined at the rule $r_{x_i}$ and that this rule is consistently interpreted by $\Ibar$ by showing that
$\Ibar(r_{x_i})=s(p_{a_{n+1},a_i})$. 

%Denote the former rule $r_{\Delta_{n+1}}$ and the latter rule $r_{x_i}$. 
First we have to show that the preconditions for $\Ibar$ to be defined at the rule $r_{x_i}$ hold. These are:

\begin{enumerate}
\item The first precondition is that \foreachi[n+1], the context $\xDelta{i}$ is mapped by $\Ibar$ to some object $a_i$ of \catcw such
that $1 \base a_1,...\base a_i$. We have established this above. 

\item The second precondition is that the rule \DDelta{n+1}{i} be mapped by $\Ibar$ to the object $\crossx{a_{n+1}}{a_i}{a_{i-1}}$.

This we can show as follows.
\newcommand{\deltaimapped}{\crossx{a_{n+1}}{a_i}{a_{i-1}}}
\newcommand{\deltaimappedlong}{s(p_{a_{n+1},a_{i-1}})^*...s(p_{a_{n+1},a_1})^*(\crossx{a_{n+1}}{a_i}{1})}
Let $r_{\Delta_i}$ be the rule \IDelta{i}.


Because $\Ibar(r_{\Delta_i}) = a_i$,
because  we have that
$\tuple{x_1,...x_{i-1}}$ is a realisation of context $\xDelta{i-1}$ with respect to context $\xDelta{n+1}$ 
and since we have assumed for each $j<i$ that $\Ibar$ is defined at and consistently interprets $r_{x_j}$  and in particular
that
\begin{equation*}
\begin{array}{c c c }
\forceSOURCEwidth & & \forceTARGETwidth \\ [-0.1cm]
\gatdisplayrule{\xDelta{n+1}}{\ofT{x_j}{\Delta_j}}  & \Imapsto & s(p_{a_{n+1},a_j}) \\ [0.4cm]
\end{array}
\end{equation*}
and since we have assumed that $r_{\Delta_i}$ is consistently interpreted by $\Ibar$ and in particular that clause (ii)(b) of the definition of consistently interpreted holds
then we have that
\begin{equation*}
\begin{array}{c c c}
\forceSOURCEwidth & & \forceTARGETwidth \\ [-0.1cm]
\gatdisplayrule{\xDelta{n+1}}{\isT{\Delta_i[x_1|x_1,...x_{i-1}|x_{i-1}]}}  & \Imapsto & \deltaimappedlong \\ [0.4cm]
\end{array}
\end{equation*}
i.e.
\begin{equation*}
\begin{array}{c c c}
\forceSOURCEwidth & & \forceTARGETwidth \\ [-0.1cm]
\DDelta{n+1}{i}  & \Imapsto & \deltaimappedlong.\\ [0.4cm]
\end{array}
\end{equation*}
but
\begin{equation*}
\deltaimappedlong = \deltaimapped
\end{equation*}  
 by lemma \lref{sofpsubstitutionlemma}.
 Therefore \begin{equation*}
\begin{array}{c c c}
\forceSOURCEwidth & & \forceTARGETwidth \\ [-0.1cm]
\DDelta{n+1}{i}  & \Imapsto & \deltaimapped\\ [0.4cm]
\end{array}
\end{equation*}
as required for this precondition.
\end{enumerate}
Since we have shown the preconditions hold then it follows from the definition of $\Ibar$ that 
 $\Ibar(r_{x_i})=s(p_{a_{n+1},a_i}) \in Sect(\crossx{a_{n+1}}{a_i}{a_{i-1}})$.

In fact, along the way in this proof that $\Ibar$ is defined at the rule $r_{x_i}$ we have shown that each of the required conditions 
 (a), (b) and (d) of clause (ii) of definition \lref{consistentinterpretation} holds. This establishes, as required,
that the rule  $r_{x_i}$ is consistently interpreted by $\Ibar$. \\
\underline{CF2(a)}\ By this principle, from a sort symbol $A$ with introductory rule $r_A$ of the form \gatdisplayrule[,]{\xDelta{n}}{\isT{A(\xn)}} for some $n \geq 0$, if the rule \gatdisplayrule[,]{\xDelta{n-1}}{\isT{\Delta_n}} is a derived rule of $U$,
if $P$ is a context and if in particular the rule $r_P$ asserting that $P$ is a context is a derived rule of $U$
and from derived rules $r_{t_i}$ of the form \ItDelta[P]{i}, \foreachi, we may deduce
the rule \gatdisplayrule{P}{\isT{A(t_1,...t_n)}}, which we shall denote by $r$, is a derived rule of $U$. \\

Assume as the inductive hypotheses that the rule $r_P$ is consistently interpreted by $\Ibar$ and that 
$\Ibar(r_{t_i})$ is defined \foreachi and that each $r_{t_i}$ is consistently
interpreted by $\Ibar$.
Now, by lemma \lref{Ibarconsistentonintros} the introductory rule for $A$ is consistently interpreted by $\Ibar$
and so we can use lemma \lref{substitutioninterpretationlemma}
to establish that the rule \gatdisplayrule{P}{\isT{A(t_1,...t_n)}} is consistently interpreted by $\Ibar$, as required. \\
\underline{CF2(b)}\ By this principle, from a operator symbol $f$ with introductory rule $r_f$ of the form \gatdisplayrule[,]{\xDelta{n}}{\ofT{f(\xn)}{\Delta}} for some $n \geq 0$, if the rule \gatdisplayrule[,]{\xDelta{n}}{\isT{\Delta}} is a derived rule of $U$,
if $P$ is a context and if in particular the rule $r_P$ asserting that $P$ is a context is a derived rule of $U$
and from derived rules $r_{t_i}$ of the form \ItDelta[P]{i}, \foreachi, we may deduce
the rule \gatdisplayrule{P}{\ofT{f(t_1,...t_n)}{\Delta[\SUBtFORx{n}]}}, which we shall denote by $r$, is a derived rule of $U$. 

Assume as the inductive hypotheses  that the rule $r_P$ is consistently interpreted by $\Ibar$ and that 
the rule \gatdisplayrule{\xDelta{n}}{\isT{\Delta}} is consistently interpreted by $\Ibar$
and that $\Ibar(r_{t_i})$ is defined \foreachi and that each $r_{t_i}$ is consistently
interpreted by $\Ibar$.
With these assumptions we can use lemma \lref{Ibarconsistentonintros} to establish that the introductory rule for $f$ is consistently interpreted by $\Ibar$.
Next we can use lemma \lref{substitutioninterpretationlemma}
to establish that the rule \gatdisplayrule{P}{\ofT{f(t_1,...t_n)}{\Delta[\SUBtFORx{n}]}} is consistently interpreted by $\Ibar$, as required. \\
%
\underline{SI1} 
\newcommand{\SIonesourcelhs}{\gatdisplayrule{\yOmega{m}}{\isT{\Omega}}}
\newcommand{\SIonesourcerhs}{\gatdisplayrule{\yOmega{m}}{\isT{\Omega'}}}
\newcommand{\SIoneconclusion}{\gatdisplayrule{Q}{\Omega[\SUBsFORy{m}]=\Omega'[\SUBspFORy{m}]}}
\newcommand{\SIoneconclusionlhs}{\ZOmegaSUBsmFORym}
\newcommand{\SIoneconclusionrhs}{\gatdisplayrule{Q}{\isT{\Omega'[\SUBspFORy{m}]}}}

This principle states that if \gatdisplayrule{\yOmega{m}}{\Omega=\Omega'} is a derived rule, if $Q$ is a context and if  $\sm$ and $\smp$ are expressions such that \foreachj, \gatdisplayrule{Q}{s_j=s'_j \in \Omega[\SUBsFORy{j-1}]} is a derived rule then we may derive the rule \SIoneconclusion. 

We have to show that the rule \SIoneconclusion is consistently intepreted by $\Ibar$
from the inductive hypothesis  that the rule \gatdisplayrule{\yOmega{m}}{\Omega=\Omega'}
and for $1 \leq j \leq m$ the rules \gatdisplayrule{Q}{s_j=s'_j \in \Omega[\SUBsFORy{j-1}]},  are consistently intepreted by $\Ibar$.

To show that \SIoneconclusion is consistently interpreted by $\Ibar$
we have to show that each of the rules \SIoneconclusionlhs and \SIoneconclusionrhs
are consistently interpreted by $\Ibar$ and that
\begin{equation}
\label{SI1inductivetarget2}
\ibarmappedrule{Q}{\isT{\Omega[\SUBsFORy{m}]}} = \ibarmappedrule{Q}{\isT{\Omega'[\SUBspFORy{m}]}}.
\end{equation}

From the  inductive hypothesis  and from clauses (iii) and (iv) of definition \lref{consistentinterpretation} it follows that
each of the rules
\SIonesourcelhs
and
\SIonesourcerhs
and each of the rules
\IsOmega{j} \\
and
\gatdisplayrule[,]{Q}{\ofT{s'_j}{\Omega_j[\SUBspFORy{j-1}]}}
\wherej, are consistently intepreted by $\Ibar$. 
Therefore from lemma \lref{substitutioninterpretationlemma} it follows that
 the rules \SIoneconclusionlhs and \SIoneconclusionrhs
are consistently interpreted by $\Ibar$, as required.

Finally we are required to show  (\ref{SI1inductivetarget2}).
Now further to what we have said above, note that from the assumptions in the inductive hypothesis 
and from clauses (iii) and (iv) of definition \lref{consistentinterpretation} we know that
\begin{equation}
\label{SI1inductiveequation1}
\Ibar(r_\Omega)=\Ibar(r_{\Omega'})
\end{equation}

where $r_\Omega$ is the rule \SIonesourcelhs and $r_{\Omega'}$ is the rule \SIonesourcerhs
and that, \foreachj,
\begin{equation}
\label{SI1inductiveequation2}
\Ibar(r_{s_j})=\Ibar(r_{s'_j})
\end{equation}
where  $r_{s_j}$ is the rule \gatdisplayrule{Q}{\ofT{s_j}{\Omega_j[\SUBsFORy{j-1}]}}
and $r_{s'_j}$ is the rule \gatdisplayrule{Q}{\ofT{s'_j}{\Omega_j[\SUBspFORy{j-1}]}}, \foreachj.
This means that (\ref{SI1inductivetarget2}) can be shown as follows:
\begin{align*}
\ibarmappedrule{Q}{\isT{\Omega[\SUBsFORy{m}]}} 
    &=\Ibar(r_{s_m})^* ... \Ibar(r_{s_1})^*\big( \crossx{\Ibar(Q)}{\Ibar(r_\Omega)}{1} \big)
    &&\mbox{ because $r_\Omega$ consistently interpreted by $\Ibar$,} \\
    &=\Ibar(r_{s'_m})^* ... \Ibar(r_{s'_1})^*\big( \crossx{\Ibar(Q)}{\Ibar(r_{\Omega'})}{1} \big)
    &&\mbox{ by (\ref{SI1inductiveequation1}) and (\ref{SI1inductiveequation2}),} \\
    &=\ibarmappedrule{Q}{\isT{\Omega'[\SUBspFORy{m}]}}
%    &&\parbox[t]{5.5cm}{ as required, because $r_{\Omega'}$ is consistently interpreted by $\Ibar$.}
    &&\mbox{ because $r_\Omega'$ consistently interpreted by $\Ibar$.} 
\end{align*}
\newpage
\underline{SI2} 
\newcommand{\SItwosourcelhs}{\gatdisplayrule{\yOmega{m}}{\ofT{s}{\Omega}}}
\newcommand{\SItwosourcerhs}{\gatdisplayrule{\yOmega{m}}{\ofT{s'}{\Omega'}}}
\newcommand{\SItwoconclusion}{\gatdisplayrule{Q}{s[\SUBsFORy{m}]=s'[\SUBspFORy{m}] \in \Omega[\SUBsFORy{m}]}}
\newcommand{\SItwoconclusionlhs}{\gatdisplayrule{Q}{\ofT{s[\SUBsFORy{m}]}{\Omega[\SUBsFORy{m}]}}}
\newcommand{\SItwoconclusionrhs}{\gatdisplayrule{Q}{\ofT{s'[\SUBspFORy{m}]}{\Omega[\SUBsFORy{m}]}}}
This principle states that if \gatdisplayrule{\yOmega{m}}{s = s' \in \Omega} is a derived rule, 
if $Q$ is a context and if  $\sm$ and $\smp$ are expressions such that
\foreachj, \gatdisplayrule{Q}{s_j=s'_j \in \Omega[\SUBsFORy{j-1}]} is a derived rule then we may derive the rule \SItwoconclusion. 

The proof in this case is similar to the case SI1 above. 

We have to show that the rule \SItwoconclusion is consistently intepreted by $\Ibar$
from the inductive hypothesis  that the rule \gatdisplayrule{\yOmega{m}}{s = s' \in \Omega}
and \foreachj  and each of the rules \gatdisplayrule{Q}{s_j=s'_j \in \Omega[\SUBsFORy{j-1}]}, wherej,  are consistently intepreted by $\Ibar$.

To show that \SItwoconclusion is consistently interpreted by $\Ibar$
we have to show that each of the rules \SItwoconclusionlhs and \SItwoconclusionrhs\ 
are consistently interpreted by $\Ibar$ and that
\begin{equation}
\label{SI2inductivetarget2}
\ibarmappedrule{Q}{\ofT{s[\SUBsFORy{m}]}{\Omega[\SUBsFORy{m}]}} = \ibarmappedrule{Q}{\ofT{s'[\SUBspFORy{m}]}{\Omega'[\SUBspFORy{m}]}}.
\end{equation}
From the  inductive hypothesis  and from clauses (iii) and (iv) of definition \lref{consistentinterpretation} it follows that
each of the rules
\SItwosourcelhs
and
\SItwosourcerhs
and each of the rules
\IsOmega{j} \\
and
\gatdisplayrule[,]{Q}{\ofT{s'_j}{\Omega_j[\SUBspFORy{j-1}]}}
\wherej, are consistently intepreted by $\Ibar$. 
Therefore from lemma \ref{substitutioninterpretationlemma} it follows that
 the rules \SItwoconclusionlhs and \SItwoconclusionrhs
are consistently interpreted by $\Ibar$, as required.

Finally we are required to show (\ref{SI2inductivetarget2}).
Now further to what we have said above, note that from the assumptions in the inductive hypothesis 
and from clauses (iii) and (iv) of definition \lref{consistentinterpretation} we know that
\begin{equation}
\label{SI2inductiveequation1}
\Ibar(r_s)=\Ibar(r_{s'})
\end{equation}
where $r_s$ is the rule \SItwosourcelhs and $r_{s'}$ is the rule \SItwosourcerhs 
and that, \foreachj,
\begin{equation}
\label{SI2inductiveequation2}
\Ibar(r_{s_j})=\Ibar(r_{s'_j})
\end{equation}
where  $r_{s_j}$ is the rule \gatdisplayrule{Q}{\ofT{s_j}{\Omega_j[\SUBsFORy{j-1}]}}
and $r_{s'_j}$ is the rule \gatdisplayrule{Q}{\ofT{s'_j}{\Omega_j[\SUBspFORy{j-1}]}}, \foreachj.
This means that (\ref{SI2inductivetarget2}) can be shown as follows:
\begin{align*}
\ibarmappedrule{Q}{\ofT{s[\SUBsFORy{m}]}{\Omega[\SUBsFORy{m}]}} \kern-3cm  \\
    &=\Ibar(r_{s_m})^* ... \Ibar(r_{s_1})^*\big( \crossx{\Ibar(Q)}{\Ibar(r_s)}{1} \big)
              &&\mbox{ because $r_\Omega$ consistently interpreted by $\Ibar$,} \\
    &=\Ibar(r_{s'_m})^* ... \Ibar(r_{s'_1})^*\big( \crossx{\Ibar(Q)}{\Ibar(r_{s'})}{1} \big)
              &&\mbox{ by (\ref{SI2inductiveequation1}) and (\ref{SI2inductiveequation2}),} \\
    &=\ibarmappedrule{Q}{\ofT{s'[\SUBspFORy{m}]}{\Omega'[\SUBspFORy{m}]}}
%             &&\parbox[t]{5.5cm}{ as required, because $r_{\Omega'}$ is consistently interpreted by $\Ibar$.}
              &&\mbox{ because $r_\Omega'$ consistently interpreted by $\Ibar$.} 
\end{align*}

\underline{A1} 
This principle ensures that an T=axiom is a derived rule providing it is well-typed.
It states that from an axiom \gatdisplayrule{\xDelta{n}}{\Delta=\Delta'} and from derived rules
 \ZDelta and \ZDeltap we may derive
\gatdisplayrule{\xDelta{n}}{\Delta=\Delta'}.

\vspace{1cm}
Follows immeadiately from the fact fact that $\Ibar$ is  valid.\\

\underline{A2} 
This principle ensures that an $\in$=axiom is a derived rule providing it is well-typed.
It states that from an axiom \gatdisplayrule{\xDelta{n}}{t=t' \in \Delta} and from derived rules
 \ZtDelta and \ZtpDelta we may derive
\gatdisplayrule{\xDelta{n}}{t=t' \in \Delta}.

Follows immeadiately from the fact that $\Ibar$ is  valid. 
\end{proof}

