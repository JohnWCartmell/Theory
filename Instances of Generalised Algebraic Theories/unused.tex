


\note The Inititiality Conjecture is described in \cite{VoevodskyInitialityConjecture}.
\begin{tightquote}
A C-system equipped with additional
operations corresponding to the inference rules of a type theory is called a
model or a C-system model of these rules or of this type theory.
\end{tightquote}
and
\begin{tightquote}
The model whose underlying
C-system is the term C-system is called the term model... for a particular
class of inference rules the term model is an initial object in the category of models.
This is known as the Inititiality Conjecture.
\end{tightquote} 
\ \\
\note Whereas from nLab (\url{https://ncatlab.org/nlab/show/Initiality+Project}) I read
\begin{tightquote}
The Initiality Project is a communal effort to prove an initiality theorem for a dependent type theory: that the categorical structure constructed out of the syntax is the initial object in some category of structured categorical objects.
\end{tightquote}

\note From my 1986 paper (\cite{Cartmell86})based on my thesis (\cite{Cartmell78}):
\begin{tightquote}
An algebraic semantics is witnessed by an equivalence between a category of
theories and a category of structures. In most instances of algebraic semantics
there is a further equivalence in that the usual definition of model of a theory can
be replaced by a definition which uses only the notion of structure. Lawvere has
used the term Functorial Semantics in describing this kind of semantics.
Functorial semantics depends on an equivalence between the category of models
of a theory U and the category of structure preserving morphisms from the
structure $C(U)$ corresponding to $U$ to a special canonical structure (the world
structure?). In the case of algebraic theories (Lawvere \cite{LawvereAlgebraicTheories}) the canonical structure
is taken to be the category of sets Set while in the case of classical proposition
theories the canonical structure is taken to be the Boolean Algebra $\set{0, 1}$.
The present situation is as well-behaved as any if the canonical structure is
taken to be the contextual category $Fam$.
If U is a generalised algebraic theory, then the category of models of U is
equivalent to the category which has contextual functors $C(U)$ to $Fam$ as objects
and natural transformations as morphisms. Thus we can assert
\begin{equation*}
\Ualg \cong ConFunc(C(U), Fam).
\end{equation*}
The inductive construction of $C(U)$ from $U$ has enabled us to replace the usual 
inductive definition of model of $U$ by the definition "a model of $U$ is a contextual
functor $M: C(U): \morph Fam$".
end{tightquote}

\begin{oldtt}
\note The definition of $\Ihat$ from $\Isort$ and $\Iop$ whenever $I$ is an interpretation proceeds by induction 
on the derivation of rules in  \gatUw 
as described in the principles of derivation in Definition 2(b) of \cite{Cartmell86}. 
The only non-trivial parts of this definition relate to the rules
identified as CF1, CF2(a) and CF2(b)\footnote{So identified, by the way, as a mnemonic for cut-free.}. We consider each of these rules in turn.

\begin{point}
Rule CF1 states that for $n \geq 0$, for $1 \leq i \leq n+1$, from the derived rule 
$\frac{\xDelta{n}}{\isT{\Delta_{n+1}}}$ which we shall denote $R$ 
we may derive the rule
$\frac{\xDelta{n+1}}{\ofT{x_i}{\Delta_i}}$ which, in turn, we shall denote $R_{x_i}$.
Define $\Ihat(R_{x_i}) :  \Ihat(R) \morph \crossx{\Ihat(R)}{\Ihat(R_i)}{\Ihat(R_{i-1})}$
to be $\tuple{p_{\Ihat(R),\Ihat(R_{i-1})},p_{\Ihat(R),\Ihat(R_{i})}}$. 

This presumably is $s(p_{\Ihat(R),\Ihat(R_{i-1})})$ where $s$ is Vladimir's s-operator.
\end{point}
\begin{point}
CF2(a) states that if $A$ is a sort symbol introduced by
$\frac{\xDelta{n}}{\isT{A(\xn)}}$ 
and if $P$ is a context and $\tn$ are expressions then from the following rules, which we shall denote $R_{t_1}$,..$R_{t_n}$,
$\frac{P}{\ofT{t_1}{\Delta_1}}$,
$\frac{P}{\ofT{t_2}{\Delta_2[t_1|x_1]}}$,
... and 
$\frac{P}{\ofT{t_n}{\Delta_n[t_1|x_1,...t_{n-1}|x_{n-1}]}}$
we may derive the rule
$\frac{P}{\isT{A(t_1,...t_n)}}$ which we denote as $R$. 
Define $I(R)$ to be $\Ihat(R_n)^*...\Ihat(R_1)^*\crossx{\Ihat(R_n)}{I(A)}{1}$.\commentary{check this}
\end{point}
\begin{point}
CF2(b) \highlight{fill this in}
\end{point}
\end{oldtt}

\begin{oldtt}
\begin{displaymath}
\begin{array}{c}
\crossx{a_n}{a_i}{\Rnode{cross}{a_{i-1}}} \\[0.9cm]
\Rnode{an}{a_n}\\[0.7cm]
%\Rnode{highervdots}{\vdots}\\
\Rnode{ai}{\begin{array}{c}
\vdots\\
a_i\\
\vdots
\end{array}} \\[1.1cm]
%\Rnode{lowervdots}{\vdots}\\[0.4cm]
\Rnode{a1}{a_1}\\[0.7cm]
\Rnode{abs}{1}
\end{array}
\end{displaymath}
\ncsar{cross}{an}
%\ncsar{an}{highervdots}
%\ncsar{lowervdots}{a1}
\ncsar{an}{ai}
\ncsar{ai}{a1}
\ncsar{a1}{abs}
\ncarc[arcangle=30,nodesepA=5pt,offsetA=2pt,nodesepB=2pt,offsetB=2pt]{->}{an}{cross}
\alabel{s(p_{a_n,a_i})}
\end{oldtt}
\begin{oldtt}
\note 3
\label{omegarealisationwrtQ}
 Suppose also that $\encyOmega{m}$ is a context of generalised algebraic theory \gatUw and suppose that $Q$ is some other context and that for some $m \geq 1$,
 \foreachj, \gatdisplayrule{Q}{\ofT{s_j}{\Omega_j[s_1|y_1,...s_{j-1}|y_{j-1}]}} is a derived rule of \gatU\footnote{Recall that such an m-tuple $\tuple{\sm}$ is said to be a realisation of 
$\encyOmega{m}$ wrt $Q$.}.  Suppose that an interpretation $I$ of \gatUw in a contextual category \catcw maps the context $Q$ to an object $a$  of $\catc$ and maps
the context $\encyOmega{j}$ to an object $b_j$ of \catc, \foreachj, so that $1 \base b_1 ... \base b_m$ in \catc. In this situation the rule 
\gatdisplayrule{Q}{\ofT{s_1}{\Omega_1}} will be mapped by $I$ to some section $f_1:a \morph \crossx{a}{b_1}{1}$. The j'th rule,
\gatdisplayrule{Q}{\ofT{s_j}{\Omega_j[s_1|y_1,...s_{j-1}|y_{j-1}]}}, will be mapped to a section $f_j:a \morph \fjpstar ... \fonestar\crossx{a}{b_j}{1}$.
In the case that $m=3$  then in \catcw we will have objects and morphisms as follows:
\begin{displaymath}
\begin{array}{c p{1cm} c p {1cm} c  p{1cm} c}
                                                &&                                           && \Rnode{ab3}{\crossx{a}{b_3}{1}}                       \\[1.2cm]
                                                &&\Rnode{f1axb3}{\fonestar\crossx{a}{b_3}{1}}  && \Rnode{ab2}{\crossx{a}{b_2}{1}}                       \\[1.2cm]
 \Rnode{f3target}{\ftwostar\fonestar\crossx{a}{b_3}{1}} &&\Rnode{f2target}{\fonestar\crossx{a}{b_2}{1}}  && \Rnode{ab1}{\crossx{a}{b_1}{\Rnode{f1target}{1}}}     \\[1.2cm]
                                                &&\Rnode{a}{a}                               &&                                                       \\[-3.0cm]
																								&&                                           &&                         && \Rnode{b3}{b_3}             \\[1.2cm]
																								&&                                           &&                         && \Rnode{b2}{b_2}             \\[1.2cm]
																								&&                                           &&                         && \Rnode{b1}{b_1}             \\[1.1cm]
																								&&                                           && \Rnode{abs}{1} \ \ \ \ \ \ \ \ &&    
\end{array}
\end{displaymath}
\ncarr{ab3}{b3}
\ncarr{ab2}{b2}
\ncarr{ab1}{b1}
\ncarr{f1axb3}{ab3}
\ncarr{f2target}{ab2}
\ncarr{f3target}{f1axb3}
\ncarc[arcangle=10,nodesepA=5pt,offsetA=2pt,nodesepB=2pt,offsetB=2pt]{->}{a}{f1target}
\alabel{f_1}[0.25]
\ncarc[arcangle=15,nodesepA=5pt,offsetA=2pt,nodesepB=2pt,offsetB=2pt]{->}{a}{f2target}
\alabel{f_2}
\ncarc[arcangle=10,nodesepA=5pt,offsetA=2pt,nodesepB=2pt,offsetB=2pt]{->}{a}{f3target}
\alabel{f_3}
\ncsar{f3target}{a}
\ncsar{f2target}{a}
\ncsar{f1target}{a}
\ncsar{ab2}{ab1}
\ncsar{ab3}{ab2}
\ncsar{f1axb3}{f2target}
\ncsar{b3}{b2}
\ncsar{b2}{b1}
\ncsar{b1}{abs}
\nccdar{a}{abs}

\note 
How suppose that additional to the situation of para. \ref{omegarealisationwrtQ} the rules 
\gatdisplayrule{\yOmega{m}}{\isT{\Delta}} and  \gatdisplayrule{\yOmega{m}}{\ofT{t}{\Delta}} are derived rules of \gatUw. 
Let us denote these rules $r$ and $r_t$, respectively. Suppose that $I$ is an interpretation of $\gatUw$ in \catcw as mapping rules and contexts as described in para. \ref{omegarealisationwrtQ}.
From what we have said in para \ref{omegarealisationwrtQ}, the interpretation  $I$ will map rule $r$ to an object $b$ such that ${b_m \base b}$ in \catcw and it will map the rule $r_t$ to a section $g:b_m \morph b$.  

Now it follows by the substitution lemma (see \cite{Cartmell86})
that the substituted $r$ and $r_t$ rules: 
\gatdisplayrule{Q}{\isT{\Delta[s_1|y_1...s_m|y_m]}} 
and  \gatdisplayrule{Q}{\ofT{t[s_1|y_1...s_m|y_m]}{\Delta[s_1|y_1...s_m|y_m]}} are derived rules of \gatU. 

\highlight{We require that}
the substituted $r$ rule will be mapped by $I$ to the object $\smstar...\sonestar\crossx{a}{b}{1}$ and the substituted $r_t$ rule will
be mapped by $I$ to the morphism  $\smstar...\sonestar\crossx{a}{g}{1}$ (which is defined since $g$ is a section).
\end{oldtt}

\begin{oldtt}

\note There is a large 2-category $\catofccs$ of contextual categories, contextual functors and natural transformations. \\

\note
If \isagat[U] then $\CofU$ is a contextual category. 
$\CofU$ is the structured assembly of contexts and realisations of $\gat[U]$.
There is an interpretation $I_0$ of theory $\gat[U]$ in contextual category
$\CofU$\footnote{
In the context of a type theory I think that this is what Vladimir refers to as the term model though it is safer to think of this as the abstract-syntax model to distinguish it from another model to be described later. It is confusing because both this model and the later model are initial in some category.}.\\

\note If $F : \ccat[C] \morph \ccat[C']$ is a contextual functor then for any $\gat[U]$, 
$F$ induces a mapping of interpretations of $\gat[U]$ in $\ccat[C]$ to interpretations of $\gat[U]$ in $\ccat[C']$. Denote this mapping $\phi_F$. \\

\note
If \isagat[U] then a $\gat[U]$-algebra $A$ is a contextual functor $A: \CofU \morph \Fam$. \\

\note 
Let $tcc$ denote the gat of contextual categories. Then
\begin{point}
to every contextual category $\ccat[C]$ there corresponds a \tccalgebra 
which we shall denote $\alg{C}$  i.e. there is a contextual functor $\alg{C} :\ccat[C](tcc) \morph \Fam$,
\end{point}
\begin{point}
to every a \tccalgebra i.e. to every contextual functor $A :\ccat[C](tcc) \morph \Fam$ there is a contextual category $\cc{A}$
\end{point}
\begin{point}
for all contextual categories $\ccat[C]$,
\begin{equation}
\cc{\alg{\ccat[C]}}= \ccat[C],
\end{equation}
\end{point}
\begin{point}
for all \tccalgebras $A$,
\begin{equation}
\alg{\cc{A}} = A.
\end{equation}
\end{point}

\note
From the previous it follows 
\begin{pointeq}
\label{cualg}
for every gat \gat[U], to the contextual category $\CofU$ corresponds a contextual functor
   $\alg{\CofU} :\ccat[C](tcc) \morph \Fam$. \\
\end{pointeq} 

\note Particularising (\ref{cualg}) to the theory $tcc$ it follows that
\begin{pointeq}
  $\alg{\ccat[C](tcc)}$ is a contextual functor   $\alg{\ccat[C](tcc)} :\ccat[C](tcc) \morph \Fam$.
\end{pointeq}

\note
Let $\Fam$ be the (large) contextual category of sets, indexed families of sets, indexed families of families of sets and so on and
let $\FAM$ be the (larger still) contextual category of large sets, indexed families of large sets, indexed families of families of large sets and so on.
Particularising (\ref{cualg}) to the category $\Fam$ we have
\begin{pointeq}
  \label{inducedalgebra}
  $\alg{\Fam}$ is a contextual functor   $\alg{Fam} :\ccat[C](tcc) \morph \FAM$. 
\end{pointeq}
 
\note
Aside: Assume now that $\catofccs$ is the category of large contextual categories so that $\Fam$ is an object of $\catofccs$. 
Let $\catoflargerccs$ be the category of larger contextual categories. \\



\note Suppose we add a sum type to generalised algebraic theories so that from
types $\Delta$ and $\Delta'$ in context $Delta_n$ we can construct a type $\Delta + \Delta'$
along with inclusion operations and that we can construct $t | t'$. 

\note Suppose we have a gat+ $U$. Then can we construct a term model which is a contextual category?
In the term model is $[\Delta + \Delta']$ actually the coproduct of $[\Delta]$ and $[\
Delta'] $
\end{oldtt}

\begin{oldtt}
\note 
\begin{lemmastar}
\label{finiteinterpretationlemma}
If $F$ is a finite generalised algebraic theory and if $P$ is a preinterpretation of $F$ in a contextual category \catcw then there is at most one interpretation $I$ of $F$ in \catcw that is consistent with $P$.
\end{lemmastar}
\begin{proof}
By induction on the theory $F$. True for the empty theory. Then provably true as sort symbols and operator symbols are added.
\end{proof} 
\begin{lemmastar}
If \gatUw is any finite generalised algebraic theory and if $P$ is a preinterpretation of \gatUw in a contextual category \catcw then there is at most one interpretation $I$ of \gatUw in \catcw that is consistent with $P$.
\end{lemmastar}
\begin{proof}
Use lemma \ref{finiteinterpretationlemma} as follows.
Suppose $I$ and $I'$ are interpretations of \gatUw in \catc.
We aim to show that for all derived T-rules or $\in$-rules $r$ of \gatU, $I(r)=I'(r)$.
Suppose then that $r$ is a derived T-rule or $\in$-rule of \gatU. Since $r$ is a derived rule of \gatUw then by lemma \lref{stratification lemma} it is a derived rule of some finite subtheory $F \subseteq \gatU$. 
Now $I \restriction F$ and $I' \restriction F$ are interpretations
of $F$ that both extend the preinterpretation $P \restriction F$. Therefore $I \restriction F$ = $I' \restriction F$
from which we may derive $I(r) = (I \restriction F) (r) = (I' \restriction F)(r) = I'(r)$, as required. 
\end{proof}
\end{oldtt}
