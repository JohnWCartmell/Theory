\newcommand{\duplesone}{{\duple{s_1}_{y_1}}}
\newcommand{\duplestwo}{{\duple{s_1,s_2}_{y_2}}}

\newcommand{\duplesn}{\duple{s_1,...s_n}_{y_n}}
\newcommand{\duplesi}{{\duple{s_1,...s_i}_{y_i}}}
\newcommand{\duplesilessone}{\duple{s_1,...s_{i-1}}_{y_{i-1}}}
\newcommand{\duplesj}{{\duple{s_1,...s_j}_{y_j}}}
\newcommand{\duplesjlessone}{\duple{s_1,...s_{j-1}}_{y_{j-1}}}
\newcommand{\duplesisucc}{{\duple{s_1,...s_{i+1}}_{y_{i+1}}}}
\newcommand{\duplesnlessone}{{\duple{s_1,...s_{n-1}}_{y_{n-1}}}}
\newcommand{\ynz}{\crossx{y_n}{z}{y_i}}


\newcommand {\sonesub}{{s_1}^*}
\newcommand {\stwosub}{{s_2}^*}
\newcommand {\stwocascade}{\stwosub\sonesub}
\newcommand {\sisub}{{s_i}^*}
\newcommand {\sicascade}{\sisub...\sonesub}
\newcommand {\sisuccsub}{{s_{i+1}}^*}
\newcommand {\sisucccascade}{\sisuccsub...\sonesub}
\newcommand {\snlessonesub}{{s_{n-1}}^*}
\newcommand {\snlessonecascade}{\snlessonesub...\sonesub}
\newcommand {\snsub}{{s_n}^*}
\newcommand {\sncascade}{\snsub...\sonesub}

If $x$ is an object of contextual category \catc, if $1 \base y_1 ... \base y_n$ in \catcw and if
$\sntuple$ is a cascade from $x$ to $y_n$ in \catcw
then  we can define a morphism
$\duplesn:x \morph y_n$ in \catcw such that 
\begin{axiom}{d1}
s(\duplesn) = s_n,
\end{axiom}
for $n> 1$, 
\begin{axiom}{d2}
\duplesn \circ p_{y_n} = \duplesnlessone, 
\end{axiom}
and for all objects $y \in Cover(y_n)$, 
\begin{axiom}{d3a}
{\duplesn} ^*y = \sncascade (\crossx{x}{y}{1}),
\end{axiom}
and for all sections $g$ of $y$,
\begin{axiom}{d3b}
{\duplesn} ^* g = \sncascade (\crossx{x}{g}{1}).
\end{axiom}


The definition of $\duplesn$ proceeds by induction. 
Define $\duplesone= s_1 \circ q(p_{x,1},y_1)$.
By axiom (s1), we directly establish, (d1), that $s(\duplesone)=s_1$.

If $y_1 <y$ in \catcw then we establish (d3a), that $\sonesub (\crossx{x}{y}{1})=\duplesone ^*y$ as follows
\begin{align*}
\sonesub (\crossx{x}{y}{1})&= \sonesub q(p_{x,1},y_1)^*y     && \mbox{by definition of $\crossx{}{}{1}$,}\\
                         &= (s_1 \circ q(p_{x,1},y_1))^*y   && \mbox{by pullback coherence axiom (q5),}\\
                         &= \duplesone ^*y                   && \mbox{by definition of $\duplesone ^*y$.}
\end{align*}
Also, if $g$ is a section of $y$ then we can show (d3b),  
that $\sonesub (\crossx{x}{g}{1})=\duplesone ^* g$, by a similar argument. 
\commentary{\highlight{Write this out.}}

Now assume that $\duplesi$ is defined and satisfies (\ref{d1}) to (\ref{d3b}) above. 
In particular we have  $\sicascade y_{i+1} = \duplesi ^*y_{i+1}$, and therefore that
$s_{i+1} \in Sect(\duplesi ^*y_{i+1})$. This allows us to define $\duplesisucc$ by 
\begin{equation*}
\duplesisucc = s_{i+1} \circ q(\duplesi, y_{i+1}).
\end{equation*} 
Immediately by axiom (s3)
we establish (d1), that $s(\duplesisucc)=s_{i+1}$.
We establish (d2), that $\duplesisucc \circ p_{y_{i+1}}= \duplesi$, as follows
\begin{align*}
\duplesisucc \circ p_{y_{i+1}} &=s_{i+1} \circ q(\duplesi, y_{i+1}) \circ p_{y_{i+1}} && \mbox{by definition of $\duplesisucc$,} \\
                               &=s_{i+1} \circ p_{\duplesi ^*y_{i+1}} \circ \duplesi && \mbox{by (Q1),} \\
															 &= \duplesi                       && \mbox{because $s_{i+1}$ is a section.}
\end{align*}
To establish (\ref{d3a}), suppose $y$ is some object such that $y_{i+1} < y$ in \catcw then we can show that $\sisucccascade (\crossx{x}{y}{1})=\duplesisucc ^*y$ as follows:
\begin{align*}
\sisucccascade (\crossx{x}{y}{1}) 
              &= \sisuccsub \duplesi ^*y && \mbox{by inductive hypothesis,} \\
                         &= \sisuccsub q(\duplesi,y_{i+1})^*y  && \mbox{by (Q6),}\\
                         &= (s_{i+1} \circ q(\duplesi,y_{i+1}))^*y   && \mbox{by (Q4),}\\
                         &= \duplesisucc ^*y                   && \mbox{by definition of $\duplesisucc$.}
\end{align*}
A similar argument \commentary{\highlight{Write this out.}} shows that if $g$ is a section of $y$ then $\sisucccascade (\crossx{x}{g}{1})=\duplesisucc ^* g$ to establish (\ref{d3b}).


\begin{lemma}
\llabel{dupledestructionlemma}
If $\duplesn : x \morph y_n$ in a contextual category $\catcw$ then \foreachi, 
\begin{equation}
\duplesn \circ p_{y_n,y_i} = \duplesi
\end{equation} 
\end{lemma}
\begin{proof}
Follows because for each $j$, $i < j \leq n$, $\duplesj \circ p_{y_j} = \duplesjlessone$
and $p_{y_j,y_i} = p_{y_j} \circ p_{y_{j-1},y_i}$.
\end{proof}
\begin{lemma}
\llabel{dupleofslemma}
If $x$ and $y$ are objects of a contextual category \catcw such that $1 \base y$ and if $g: x \morph y$ is a morphism then
\begin{equation*}
\duple{s(g)} = g
\end{equation*}
\end{lemma}
\begin{proof}
By definition of $s$, $s(g):x \morph \crossx{x}{y}{1}$ in \catc. 
Therefore,by definition of $\duple{}$,  $\duple{s(g)}$ is defined,  $\duple{s(g)}:x \morph y$ in \catcw and
satisfies (d1) i.e. that $s(\duple{s(g)}) = s(g)$ 
and therefore that, $by lemma \lref{stactic}, \duple{s(g)}=g$.
\end{proof}
%
%
%
{ % BEGIN   {thegeneraldupletuplelemma} and proof
\newcommand{\tuplesnsg}{\tuple{s_1,...s_n,s(g)}}
\newcommand{\duplesnsg}{\duple{s_1,...s_n, s(g)}_{f^*z}}
\newcommand{\dupletuplerhs}{\bigtuple{\duplesn,g}}
\begin{lemma}
\llabel{thegeneraldupletuplelemma} 
If $x$, $y_1$,...$y_n$, $z_p$ and $z$ are objects of a contextual category \catcw 
such that $1 \base y_1 ... \base y_n$ and $z_p \base z$ in \catc, 
if $f: y_n \morph z_p$, so that there is this pullback diagram 
\begin{displaymath}
\begin{array} {c p{3cm} c p{2cm} c}
              && \Rnode{TL}{f^*z}  && \Rnode{TR}{z}  \\[1.2cm]
              && \Rnode{BL}{y_n}   && \Rnode{BR}{z_p}
\end{array}
\begin{arrows}
\ncsar{TL}{BL}
\ncsar{TR}{BR}
\ncarr{TL}{TR}
\alabel{q(f,z)}
\ncarr{BL}{BR}
\blabel{f}
\end{arrows}
\end{displaymath}
in \catc, if $\tuplesnsg$ is a cascade from $x$ to $f^*z$ and $g:x \morph z$ in \catcw 
such that
\begin{equation} \label{generaldupletuplegiven}
g \circ p_z = \duplesn \circ f
\end{equation} 
then
\begin{equation}
\label{generaldupletuplegoaltwo}
\duplesnsg = \dupletuplerhs
\end{equation}
where $\dupletuplerhs$ is the unique morphism $\dupletuplerhs:x \morph f^*z$ such that
\begin{equation}
\label{generaldupletupledefone}
\dupletuplerhs \circ q(f,z) =g
\end{equation} 
and 
\begin{equation}
\dupletuplerhs \circ p_{f^*z} = \duplesn
\end{equation}
 as shown in this diagram
\begin{displaymath}
\begin{array} {c p{3cm} c p{2cm} c}
% FL is Far Left !!
\Rnode{FL}{x} &&                   &&                \\[1.0cm]
              && \Rnode{TL}{f^*z}  && \Rnode{TR}{z}  \\[1.2cm]
              && \Rnode{BL}{y_n}   && \Rnode{BR}{z_p}
\end{array}
\begin{arrows}
\ncsar{TL}{BL}
\ncsar{TR}{BR}
\ncarr[30]{FL}{TR}
\alabel{g}
\ncarr{FL}{TL}
\alabel{\dupletuplerhs}[0.7][1]
\ncarr[-20]{FL}{BL}
\blabel{\duplesn}[0.5][0]
\ncarr{TL}{TR}
\alabel{q(f,z)}
\ncarr{BL}{BR}
\blabel{f}
\end{arrows}
\end{displaymath}
\end{lemma}
\begin{proof}
To show (\ref{generaldupletuplegoaltwo})
we need just show that
\begin{equation}
\label{generaldupletuplesubgoalone}
\duplesnsg \circ q(f,z) =g
\end{equation} 
and 
\begin{equation}
\label{generaldupletuplesubgoaltwo}
\duplesnsg \circ p_{f^*z} = \duplesn
\end{equation}
(\ref{generaldupletuplesubgoalone}) follows because by lemma \lref{footandstactic}
 it suffices to show that
\begin{equation}
\label{generaldupletuplesubgoaloneone}
\duplesnsg \circ q(f,z) \circ p_z =g \circ p_z,
\end{equation}
which follows directly from (\ref{generaldupletupledefone}), and
\begin{equation}
\label{generaldupletuplesubgoalonetwo}
s(\duplesnsg \circ q(f,z)) =s(g)
\end{equation}
which we prove as follows:
\begin{align*}
s(\duplesnsg \circ q(f,z)) &=s(\duplesnsg  && \mbox{ by (s3),} \\
                          &= s(g)                       && \mbox{ by (d1).}
\end{align*}
whereas (\ref{generaldupletuplesubgoaltwo}) is an instance of clause (d2) of the definition of $\duple{}$.
\end{proof}
} % END   {thegeneraldupletuplelemma} and proof
%
%
%

{ % BEGIN {thedupletuplelemma} and proof and the {absolutedupletuplesublemma} and proof
\newcommand{\tuplesnsg}{\tuple{s_1,...s_n, s(g)}} 
\newcommand{\duplesnsg}{\duple{s_1,...s_n, s(g)}_{\ynz}}
\newcommand{\dupletuplerhs}{\bigtuple{\duplesn,g} }
\begin{lemma}
\llabel{thedupletuplelemma} 
If $x$, $y_1$,...$y_n$ and $z$ are objects of a contextual category \catcw such that $1 \base y_1 ... \base y_n$ in \catcw and
$y_i \base z$, for some $i$, $1 \leq i \leq n$,
if $\sntuple$ is a cascade from $x$ to $y_n$ in \catcw 
and if $g: x \morph z$ in \catcw such that
$g \circ p_z = \duplesi$, 
then 
$\tuplesnsg$ is a cascade from $x$ to $\ynz$ in \catcw
and
\begin{equation}
\label{dupletuplegoal}
\duplesnsg = \dupletuplerhs\,.
\end{equation}
\end{lemma}
\begin{proof}

To show that $\tuplesnsg$ is a cascade from $x$ to $\ynz$ in \catcw we need show that
$s(g) \in Sect({s_n}^* ... {s_1}^* (\crossx{x}{(\ynz)}{1})$.
This follows because by definition of $s(g)$, $s(g) \in Sect((g \circ p_z) ^*z)$ and because
\begin{align*}
(g \circ p_z) ^*z &= \duplesi ^*z                                && \mbox{from initial assumption that $g \circ p_z =\duplesi$,}\\
                  &= (\duplesn \circ {p_{y_n,y_i}})^*z           && \mbox{by lemma \lref{dupledestructionlemma},} \\
                  &= {\duplesn} ^* {p_{y_n,y_i}}^*z              && \mbox{by (q4),} \\
                  &= {\duplesn} ^* (\ynz)                        && \mbox{by definition of $\crossx{}{}{}$,} \\
                  &= {s_n}^* ... {s_1}^* (\crossx{x}{(\ynz)}{1}) && \mbox{by (d3a).}
\end{align*}

Now (\ref{dupletuplegoal}) follows as a special case of lemma \lref{thegeneraldupletuplelemma} with $z_p$ being $y_i$ and $f$ being $p_{y_n,y_i}$
provided that we can show that 
\begin{equation}
g \circ p_z = \duplesn \circ p_{y_n,y_i}.
\end{equation}
This follows from the assumption that $g \circ p_z = \duplesi$ and from lemma \ref{dupledestructionlemma}.
\end{proof}
% END of {thedupletuplelemma} and proof%
%

\begin{lemma}
\llabel{absolutedupletuplesublemma}
If $x$, $y_1$,...$y_n$ and $z$ are objects of a contextual category \catcw such that $1 \base y_1 ... \base y_n$ in \catcw and
$1 \base z$, 
if $\sntuple$ is a cascade from $x$ to $y_n$ in \catcw 
and if $g: x \morph z$ in \catcw 
then $\tuplesnsg$ is a cascade from $x$ to $\ynz$ in \catcw
and
\begin{equation}
\label{dupletuplegoalx}
\duplesnsg = \dupletuplerhs\,.
\end{equation}
\end{lemma}
\begin{proof}

To show that $\tuplesnsg$ is a cascade from $x$ to $\ynz$ in \catcw we need show that
$s(g) \in Sect({s_n}^* ... {s_1}^* (\crossx{x}{(\ynz)}{1})$.
This follows because by definition of $s(g)$, $s(g) \in Sect((g \circ p_z) ^*z)$ and because
\begin{align*}
(g \circ p_z) ^*z  &= (\duplesn \circ {p_{y_n,1}})^*z           && \mbox{because $1$ is terminal,} \\
                  &= {\duplesn} ^* {p_{y_n,1}}^*z               && \mbox{by (q4),} \\
                  &= {\duplesn} ^* (\ynz)                       && \mbox{by definition of $\crossx{}{}{}$,} \\
                  &= {s_n}^* ... {s_1}^* (\crossx{x}{(\ynz)}{1} && \mbox{by (d3a).}
\end{align*}

Now (\ref{dupletuplegoalx}) follows as a special case of lemma \lref{thegeneraldupletuplelemma} with $z_p$ being $1$ and $f$ being $p_{y_n,1}$
provided that we can show that 
\begin{equation}
g \circ p_z = \duplesn \circ p_{y_n,1}.
\end{equation}
This holds because $1$ is terminal.
\end{proof}
}  %end scope for two lemmas



\begin{lemma}
\llabel{absolutedupletuplelemma}
For $n \geq 1$, if $x$ and $y_1,...y_n$ are objects of a contextual category \catcw such that \foreachi, $1 \base y_i$ and if \foreachi, $f_i: x \morph y_i$ then
\begin{equation*}
\duple{s(f_1),...s(f_n)}=\tuple{\fn}
\end{equation*}
\end{lemma}
\begin{proof}
\begin{align*}
\duple{s(f_1),...s(f_n)} &= \tuple{\duple{s(f_1),...s(f_{n-1})},f_n} &&\mbox{by lemma \lref{absolutedupletuplesublemma},}\\
                         &= \tuple{\tuple{f_1,...f_{n-1}},f_n}       &&\mbox{by the inductive hypothesis,}  \\
                         &= \tuple{\fn}                              && \mbox{by definition of $\tuple{}$.}
\end{align*}
\end{proof}

\begin{lemma}
\llabel{absolutedupletuplelemma}
For $n > 1$, if $x$ and $y_1,...y_n$ are objects of a contextual category \catcw such that \foreachi, $1 \base y_i$ and if \foreachi, $f_i: x \morph y_i$ then
\begin{equation*}
\tuple{\fn}=\duple{s(f_1),...s(f_n)}
\end{equation*}
\end{lemma}
\begin{proof}
\begin{align*}
\duple{s(f_1),...s(f_n)} &= \tuple{\duple{s(f_1),...s(f_{n-1})},f_n} &&\mbox{by lemma \lref{thegeneraldupletuplelemma},}\\
                         &= \tuple{\tuple{f_1,...f_{n-1}},f_n}       &&\mbox{by the inductive hypothesis,}  \\
                         &= \tuple{\fn}                              && \mbox{by definition of $\tuple{}$.}
\end{align*}
\end{proof}

\begin{lemma}
\llabel{duplesofplemma}
If $x$,$y_1$,...$y_m$ are objects of a contextual category \catcw such that $x \base y_1...\base y_m$ in \catcw then
\newcommand{\xyj}[1]{\crossx{x}{y_{#1}}{1}}
\begin{equation*}
\duple{s(p_{\xyj{m},\xyj{1}},...s(p_{\xyj{m},\xyj{m}})} = q(p_{x,1},y_m)
\end{equation*}
\end{lemma}
\begin{proof}
\tbd
\end{proof}

