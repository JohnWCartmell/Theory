\newcommand{\duplesone}{{\duple{s_1}_{B_1}}}
\newcommand{\duplestwo}{{\duple{s_1,s_2}_{B_2}}}
\newcommand{\duplesn}{\duple{s_1,...s_n}_{B_n}}
\newcommand{\duplesi}{{\duple{s_1,...s_i}_{B_i}}}
\newcommand{\duplesilessone}{\duple{s_1,...s_{i-1}}_{B_{i-1}}}
\newcommand{\duplesj}{{\duple{s_1,...s_j}_{B_j}}}
\newcommand{\duplesjlessone}{\duple{s_1,...s_{j-1}}_{B_{j-1}}}
\newcommand{\duplesisucc}{{\duple{s_1,...s_{i+1}}_{B_{i+1}}}}
\newcommand{\duplesnlessone}{{\duple{s_1,...s_{n-1}}_{B_{n-1}}}}

\newcommand {\sonesub}{{s_1}^*}
\newcommand {\stwosub}{{s_2}^*}
\newcommand {\stwocascade}{\stwosub\sonesub}
\newcommand {\sisub}{{s_i}^*}
\newcommand {\sicascade}{\sisub...\sonesub}
\newcommand {\sisuccsub}{{s_{i+1}}^*}
\newcommand {\sisucccascade}{\sisuccsub...\sonesub}
\newcommand {\snlessonesub}{{s_{n-1}}^*}
\newcommand {\snlessonecascade}{\snlessonesub...\sonesub}
\newcommand {\snsub}{{s_n}^*}
\newcommand {\sncascade}{\snsub...\sonesub}

\note The duple construction. If $A$ is an object of contextual category \catc, if $1 \base B_1 ... \base B_n$ in \catcw and if
\begin{equation*}
\begin{array}{l}
s_1 \in Sect(\crossx{A}{B_1}{1}),                  \\
s_2 \in Sect(\sonesub (\crossx{A}{B_2}{1})),         \\
s_3 \in Sect(\stwocascade (\crossx{A}{B_3}{1})),     \\
\multicolumn{1}{c}{\vdots}                           \\
s_n \in Sect(\snlessonecascade (\crossx{A}{B_n}{1})) \\
\end{array}
\end{equation*}
\mbox{ in \catc},
then  we can define a morphism
$\duplesn:A \morph B_n$ in \catcw such that 
\begin{enumerate}[(i)]
\item $s(\duplesn) = s_n$,
\item for $n> 1$, $\duplesn \circ p_{B_n} = \duplesnlessone$, and 
\item for all objects $B$ of \catcw such that $B_n < B$, 
$\sncascade \crossx{A}{B}{1} = \duplesn ^* B$, \\
and for all sections $s$ of $B$,
$\sncascade \crossx{A}{s}{1} = \duplesn ^* s$.
\end{enumerate}

The definition of $\duplesn$ proceeds by induction. 
Define $\duplesone= s_1 \circ q(p_{A,1},B_1)$.
By axiom (s1), it follows immediately that $s(\duplesone)=s_1$.

If $B_1 <B$ in \catcw then we have $\sonesub \crossx{A}{B}{1}=\duplesone ^* B$ because
\begin{align*}
\sonesub (\crossx{A}{B}{1})&= \sonesub q(p_{A,1},B_1)^*B     && \mbox{by definition of $\crossx{}{}{}$,}\\
                         &= (s_1 \circ q(p_{A,1},B_1))^*B   && \mbox{by pullback coherence axiom,}\\
                         &= \duplesone ^* B                   && \mbox{by definition of $\duplesone ^* B$.}
\end{align*}
Also, if $g$ is a section of $B$ then we can show, by a similar argument, 
that $\sonesub (\crossx{A}{g}{1})=\duplesone ^* g$.

Now assume that $\duplesi$ is defined and satisfies (i) to (iii) above. 
In particular we have then that $\sicascade B_{i+1} = \duplesi ^* B_{i+1}$, and therefore that
$s_{i+1} \in Sect(\duplesi ^* B_{i+1})$, and this then allows us to define $\duplesisucc$ by 
\begin{equation*}
\duplesisucc = s_{i+1} \circ q(\duplesi, B_{i+1}).
\end{equation*} 
Immediately by axiom s3
we have that $s(\duplesisucc)=s_{i+1}$.
We have that $\duplesisucc \circ p_{B_{i+1}}= \duplesi$ because
\begin{align*}
\duplesisucc \circ p_{B_{i+1}} &=s_{i+1} \circ q(\duplesi, B_{i+1}) \circ p_{B_{i+1}} && \mbox{by definition of $\duplesisucc$,} \\
                               &=s_{i+1} \circ p_{\duplesi ^* B_{i+1}} \circ \duplesi && \mbox{because pullback commutes,} \\
															 &= \duplesi                       && \mbox{because $s_{i+1}$ is a section.}
\end{align*}
When $B$ is some object such that $B_{i+1} < B$ in \catcw then we can show that$\sisucccascade (\crossx{A}{B}{1})=\duplesisucc ^* B$ as follows:
\begin{align*}
\sisucccascade (\crossx{A}{B}{1}) 
              &= \sisuccsub \duplesi ^* B && \mbox{by inductive hypothesis,} \\
                         &= \sisuccsub q(\duplesi,B_{i+1})^*B  && \mbox{by definition of extended $^*$,}\\
                         &= (s_{i+1} \circ q(\duplesi,B_{i+1}))^*B   && \mbox{by pullback coherence axiom,}\\
                         &= \duplesisucc ^* B                   && \mbox{by definition of $\duplesisucc$.}
\end{align*}
A similar argument shows that if $g$ is a section of $B$ then $\sisucccascade (\crossx{A}{g}{1})=\duplesisucc ^* g$.


\begin{lemma}
\label{dupledestructionlemma}
If $\duplesn : A \morph B_n$ in a contextual category $\catcw$ then \foreachi, 
\begin{equation}
\duplesn \circ p_{B_n,B_i} = \duplesi
\end{equation} 
\end{lemma}
\begin{proof}
Follows because for each $j$, $i < j \leq n$, $\duplesj \circ p_{B_j} = \duplesjlessone$
and $p_{B_j,B_i} = p_{B_j} \circ p_{B_{j-1},B_i}$.
\end{proof}

\newcommand{\dupletuplerhs}{\bigtuple{\duplesnlessone,g}_{p_{B_{n-1},B_i},C}}
\begin{lemma}
\label{thedupletuplelemma}
If $A$ is an object of contextual category \catc, if $1 \base B_1 ... \base B_n$ in \catcw and if
\begin{equation*}
\begin{array}{l}
s_1 \in Sect(\crossx{A}{B_1}{1}),                  \\
s_2 \in Sect(\sonesub (\crossx{A}{B_2}{1}))=Sect(\duplesone^*B_2),         \\
s_3 \in Sect(\stwocascade (\crossx{A}{B_3}{1}))=Sect(\duplestwo^*B_3),     \\
\multicolumn{1}{c}{\vdots}                           \\
s_n \in Sect(\snlessonecascade (\crossx{A}{B_n}{1})) =Sect(\duplesnlessone^*B_n)\\
\end{array}
\end{equation*}
\mbox{ in \catc}, and if $B_n$ is $\crossx{B_{n-1}}{C}{B_i}$, for some $i$, $0 \leq i < n$, 
(where in the case of $n$ equal to $0$ then 
by $B_0$ we mean the terminal object $1$ of \catc), if $g: A \morph C$ in \catcw and 
$g \circ p_C = \duplesi$, so that
$s(g) \in Sect(\duplesnlessone ^* B_n)$
\footnote {Because by definition of $s(g)$, $s(g) \in Sect((g \circ p_C) ^* C)$ and if 
$g \circ p_C =  \duplesi$ then 
\begin{align*}
(g \circ p_C) ^* C &= \duplesi ^* C  \\
                  &= (\duplesnlessone \circ {p_{B_{n-1},B_i}})^* C\\
									&=\duplesnlessone ^* B_n
\end{align*}
},  then if $s(g)=s_n$ then 


\begin{equation}
\label{dupletuplegoal}
\duplesn = \dupletuplerhs\,.
\end{equation}
\end{lemma}
\begin{proof}


To show that $\dupletuplerhs$ (rhs of (\ref{dupletuplegoal})) is defined we need show that 
\begin{equation}
g \circ p_C = \duplesnlessone \circ p_{B_{n-1},B_i}
\end{equation}
This follows from the assumption that $g \circ p_C = \duplesi$ and from lemma \ref{dupledestructionlemma}.

Now $\dupletuplerhs$ is therefore defined and is the unique morphism such that
\begin{equation}
\dupletuplerhs \circ p_{\crossx{B_{n-1}}{C}{1}} = \duplesnlessone
\end{equation}
and
\begin{equation}
\dupletuplerhs \circ q(p_{B_{n-1},B_i},C) = g
\end{equation}

as shown here

\begin{equation}
\begin{array}{c p{4cm} c p{3cm} c }
\\[1.75cm]
\Rnode{A}{A} && \Rnode{Bn1C}{\crossx{B_{n-1}}{C}{1}} &&                            \\[1.5cm]
						 &&                                      &&\Rnode{C}{C}                \\[0.5cm]
             && \Rnode{Bn1}{B_{n-1}}                 &&                            \\[1.5cm]
						 &&                                      &&\Rnode{Bi}{B_i}             \\[0.5cm]
\end{array}
\mbox{\ncdarr{A}{Bn1C}
\alabel{ \dupletuplerhs}
\ncarr{A}{Bn1}
\blabel{\duplesnlessone}
\ncsar{Bn1C}{Bn1}
\blabel{p_{\crossx{B_{n-1}}{C}{1}}}
\nccdar{Bn1}{Bi}
\blabel{p_{B_{n-1},B_i}}
\ncarr{Bn1C}{C}
\alabel{q(p_{B_{n-1},B_i},C)}[0.3]
\ncsar{C}{Bi}
\alabel{p_C}
\ncarr[60]{A}{C}
\alabel{g}
}
\end{equation}

Therefore to show (\ref{dupletuplegoal}), as we are required,  it suffices  to show that 

\begin{equation}
\label{dupletuplesubgoalone}
\duplesn \circ p_{\crossx{B_{n-1}}{C}{B_i}} = \duplesnlessone
\end{equation}
and
\begin{equation}
\label{dupletuplesubgoaltwo}
\duplesn \circ q(p_{B_{n-1},B_i},C) = g
\end{equation}

(\ref{dupletuplesubgoalone}) holds from the definition of $\duple{}_{B_n}$ because, remember, $B_n=\crossx{B_{n-1}}{C}{B_i}$.

We show that (\ref{dupletuplesubgoaltwo}) holds by using lemma \ref{footandstactic} and showing that
\begin{equation}
\label{dupletuplesubgoaltwoone}
\duplesn \circ q(p_{B_{n-1},B_i},C) \circ p_C = g \circ p_C
\end{equation}
and
\begin{equation}
\label{dupletuplesubgoaltwoone}
s(\duplesn \circ q(p_{B_{n-1},B_i},C)) = s(g)
\end{equation}

(\ref{dupletuplesubgoaltwoone}) follows immediately because by axiom (s3) 
$s(\duplesn \circ q(p_{B_{n-1},B_i},C))=s(\duplesn)=s_n$ and from the initial assumption that
$s(g)=s_n$.

We are left with proving (\ref{dupletuplesubgoaltwoone}) which we can do as follows:
\begin{align*}
\duplesn \circ q(p_{B_{n-1},B_i},C) \circ p_C 
              &=  \duplesn \circ p_{\crossx{B_{n-1}}{C}{B_i}} \circ p_{B_{n-1},B_i} 
                                               && \mbox{commutivity of pullback square},               \\
							&=\duplesn \circ p_{\crossx{B_{n-1}}{C}{B_i},B_i} && \mbox{definition of extended $p$,}  \\
							&=\duplesi                                        && \mbox{lemma \ref{dupledestructionlemma},} \\
							&= g                                              && \mbox{from the initial assumption.}
\end{align*}
\end{proof}