
\label{contextualnotationpartone}



\note 
Terminology: By  the generic term \term{tree} is meant a partially ordered set (poset) $(T, <)$ such that for each $t \in T$, the set $\set{s \in T : s < t}$ is well-ordered by the relation $<$.
In this discussion we restrict ourselves to rooted $\omega$-trees i.e. trees for which the set $\set{s \in T : s < t}$
is finite for all $t \in T$ and for which there is a least element in the partial ordering. 

With respect to a partial ordering $<$, we say that an element $y$ \textit{covers}  an element $x$ in  iff $x<y$ and there does not exist $w$ such that $x < w$ and $w < y$.
If object $y$ covers object $x$ in the partial ordering 
then we write $x \base y$ (we use this in preference to the more usual $x \lessdot y$).


\note We define the rank (sometimes called the grade) of an element $t \in T$ to be the cardinality
of the set $\setsuchthat{s \in T}{s < t}$. If we define the set $T_i$ to be the set of elements of a tree
of rank $i$ then we have that $T= \bigcup_{i \in N}T_i$. 

\note In the  definition of contextual categories (\cite{Cartmell78,Cartmell86}) there is defined to be such a tree-structure on the objects of the category. In a contextual category the root of the tree of objects is also  terminal object $1$
of the category. For $x$ an object of the category we define the set of objects  $Cover(x)$ to be the set of objects covering $x$.

\note
By a \term{tree-structured category}\footnote{Is this what Jacopo refered to as a `stratified category'? If so does this term appear in the lierature somewhere. Why is that adjective used in preference to a choice of either `ranked' or `graded'?} we mean (i) a category with a tree-structure defined on its objects such that the tree of objects has a unique root object and (ii) for every $x \base y$ in the tree of objects  a canonical morphism $p_y:y \rightarrow x$.  I shall shall say morphisms of this form  are \term{direct dependency morphisms} and they will
be distinguished in diagrams by an arrow with  a triangular head so:
\begin{center}
$
\begin{array}{p{2cm}}
\Rnode{y}{y}\\ [1.4cm]
\Rnode{x}{x} \\
\mbox{\ncbsar{p_y}{y}{x}}
\end{array}
$
\end{center}

\note
If $x$ is an object of a tree-structured category \catcw and if $y \in Cover(x)$ in \catcw then we define 
the set  of sections of $y$, denoted $Sect(y)$, to be the set of morphisms $s: x \morph y$ in \catc  such that $s \circ p_y = id_x$. So that if $x$ is a section of $y$ and $y$ covers $x$ then
we have\ \ \ 
\begin{tabular}{cccc}
$
\begin{array}{p{2cm}}
\Rnode{y}{y} \\ [1.4cm]
\Rnode{x}{x} \\
\mbox{
\ncsar{y}{x}
\alabel{p_y}
\ncarrZZ[30]{x}{y} 
\alabel{s}}
\end{array}
$  & so that &
$
\begin{array}{c p{0.5cm}c p{0.5cm}c}
              && \Rnode{y}{y}&&                \\ [1.4cm]
\Rnode{x1}{x} &&             &&   \Rnode{x2}{x}\\
\mbox{
\ncsar{y}{x2}
\alabel{p_y}
\ncarr{x1}{y} 
\alabel{s}
\ncarr{x1}{x2} 
\blabel{id_x}
}
\end{array}
$& commutes
\end{tabular}

\note
The original definition of contextual category given  in [1] and [2], a contextual category is defined to be a tree-structured category 
\cat{C} with the following additional structure:

\noindent 
(i) whenever
$
\begin{array}{cp{.9cm}c}
            & & \Rnode{z}{z} \\ [1.2cm]
\Rnode{x}{x}& & \Rnode{y}{y} \\ [0.5cm]
\end{array}
$
\jcbarr{f}{x}{y}
\ncasar{p_z}{z}{y}

in \cat{C}, an object $f \sub z$ such that $x \base f \sub z$, a morphism $q(f,z): f \sub z \rightarrow z$ such that

\begin{axiom}{q1}
q(f,z) \circ p_z = p_{f \sub z} \circ f
\end{axiom}

i.e. such that the diagram: 
$$
\ccsquareoutline{0.9cm}{1.2cm}{f^*z}{z}{x}{y}
\ccsquareacross{q(f,z)}{f}
\ccsquaredown{p_{f \sub z}}{p_z}
$$
commutes, 

\noindent
and, (ii), so that each such diagram is a pullback diagram, that is: for all objects $w$ of \cat{C}, and for all
morphisms $h_1: w \rightarrow x$ and $h_2: w \rightarrow z$ (see diagram \ref{pullback} below) such that
$h_1 \circ f = h_2 \circ p_z$ 
there exists a unique $h:w \rightarrow f \sub z$ in \cat{C} such that
$h \circ p_{f \sub z} = h_1$ and $h \circ q(f,z) = h_2$, as shown here:

\vspace{3mm}
\begin{center}
\begin{equation}
\label{pullback}
\begin{array}{cp{0.5cm}cp{1.2cm}c}
\Rnode{w}{w} &&                     &&           \\ [0.7cm]
             &&\Rnode{fstarz}{f^*z} && \Rnode{z}{z}\\ [1.2cm]
             &&\Rnode{x}{x}         && \Rnode{y}{y}
\end{array}
\end{equation}
\ncbsar{p_{f \sub z}}{fstarz}{x}
\jcbarr{f}{x}{y}
\ncaarr{q(f,z)}{fstarz}{z}
\ncasar{p_z}{z}{y}
\setlength{\arrnodesepA}{3pt}
\jcbarr[-35]{h_1}{w}{x}
\ncaarr[35]{h_2}{w}{z}
\psset{linestyle=dashed}
\ncaarr{h}{w}{fstarz}
\end{center}

\vspace {0.25cm}
\noindent and so that (iii) whenever $x \base y$ in \cat{C}, 
\begin{axiom}{q2}
id_x^*y=y
\end{axiom}

and

\begin{axiom}{q3}
q(id_x,y) = id_y
\end{axiom}



\noindent and (iv) whenever 
$
\begin{array}{c p{.9cm} c p{.9cm} c}
             &   &             &   & \Rnode{z}{z} \\ [1.2cm]
\Rnode{w}{w} &   &\Rnode{x}{x} &   & \Rnode{y}{y} \\ [0.5cm]
\end{array}
$
\jcbarr{f}{w}{x}
\jcbarr{g}{x}{y}
\ncasar{c}{z}{y}
in \cat{C}, 

then

\begin{axiom}{q4}
(f \circ g)^*z =  f^* (g ^* z)
\end{axiom}

and 
\begin{axiom}{q5}
q(f \circ g,z) = q(f,g^*z) \circ q(g,z)
\end{axiom}



\note
Following Voevodsky we may replace the pullback condition of the original definition by an 
`s' operator along with axioms as follows:

\noindent (ii') for all morphisms $f: x \rightarrow y$, a morphism $s(f) : x \rightarrow f \sub p_y \sub y$ such that both:

\begin{axiom}{s1}
s(f) \circ p_{f\sub p_y \sub y}=id_x
\end{axiom}

\noindent and

\begin{axiom}{s2}
s(f) \circ q( f \circ p_y     ,y)=f
\end{axiom}	

\noindent i.e. such that the following diagrams commute:
\begin{center}
\begin{displaymath}
\begin{array}{cccp{1.cm} cp{.9cm}c}
&\Rnode{fXyyM}{f\sub p_y \sub y}&  & &  \Rnode{fXyy}{f\sub p_y \sub y} & & \Rnode{yXy}{p_y \sub y}\\ [1.2cm]
\Rnode{xL}{x} & &\Rnode{xR}{x} & &\Rnode{x}{x}         & & \Rnode{y}{y}
\end{array}
\end{displaymath}
\ncasar{p_{f\sub p_y \sub y}}{fXyy}{x}
\jcbarr{f}{x}{y}
\ncaarr{q(f,p_y \sub y)}{fXyy}{yXy}
\ncasar{p_{p_y \sub y}}{yXy}{y}
\ncaarr{s(f)}{xL}{fXyyM}
\ncasar{p_{f\sub p_y \sub y}}{fXyyM}{xR}
\jcbarr{id_x}{xL}{xR}
\end{center}

\noindent
and such that whenever

\begin{center}
\begin{displaymath}
\begin{array}{c p{.9cm} c p{.9cm} c}
\Rnode{w}{w}&& \Rnode{g*z}{g \sub z} && \Rnode{z}{z} \\ [1.2cm]
            && \Rnode{x}{x}  && \Rnode{y}{y} \\ [0.2cm]
\end{array}
\end{displaymath}
\jcbarr{f}{w}{g*z}
\jcbarr{g}{x}{y}
\ncaarr{q(g,z)}{g*z}{z}
\ncasar{}{g*z}{x}
\ncasar{}{z}{y}
\end{center}

\noindent in \cat{C} then

\begin{axiom}{s3}
s(f \circ q(g,z))=s(f)
\end{axiom}

\iffalse
\begin{oldtt}
\note
Now consider the pullbacks in the original definition of contextual categories.
According to the definition whenever
$
\begin{array}{cp{.9cm}c}
            & & \Rnode{z}{z} \\ [1.2cm]
\Rnode{x}{x}& & \Rnode{y}{y} \\ [0.5cm]
\mbox{\jcbarr{f}{x}{y}
\ncasar{p_z}{z}{y}}
\end{array}
$
in \catcw then there is a pullback diagram: \ \ 
$
\ccsquareoutline{0.9cm}{1.2cm}{f^*z}{z}{x}{y}
\ccsquareacross{q(f,z)}{f}
\ccsquaredown{p_{f \sub z}}{p_z}
$
in \catcw i.e. such objects and morphisms so that for all objects $w$ of \catc, and for all
morphisms $h_1: w \rightarrow x$ and $h_2: w \rightarrow z$  such that
$h_1 \circ f = h_2 \circ p_z$ 
there exists a unique $h:w \rightarrow f \sub z$ in \catcw such that
$h \circ p_{f \sub z} = h_1$ and $h \circ q(f,z) = h_2$, as shown is diagram (\ref{pullback}).

After reading Peter Dybjer's axioms for categories with families (\cite{dybjer96}) I have been wondering whether I might not use the notation $\tuple{h_1,h_2}_{x,y,f,z}$  for the unique morphism   
$h:w \rightarrow f \sub z$ in \cat{C} such that
$h \circ p_{f \sub z} = h_1$ and $h \circ q(f,z) = h_2$.\\

$\tuple{h_1,h_2}_{x,y,f,z}$ may be be safely elided to $\tuple{h_1,h_2}_{f,z}$ and, rather less safely, to $\tuple{h_1,h_2}$.
In fully elided form we then  have\footnote{Strikingly similar in appearance to Dybjer's axioms but, I think, not like for like i.e. not inter-translatable.} 
\begin{equation}
\mbox{$\tuple{h_1,h_2} \circ p_{f^*z} = h_1$}
\end{equation}
and
\begin{equation}
\mbox{$\tuple{h_1,h_2} \circ q(f,z) = h_2$}
\end{equation}

I will use this notation in the detailed examples that follow.
\end{oldtt}
\fi

\note I use several other notational conveniences when working in contextual categories. 
If $x < y$ in the contextual category \catc, then define the morphism $p_{y,x}:y \morph  x$ in \catc, \\

\begin{tabular}{c c c  c  c  c c}
by defining
& %2 c
$
\begin{array} {c}
\Rnode{midy}{y} \\[2.0cm]
\Rnode{midx}{x}  \\ 
\end{array}
\mbox{\ncarr{midy}{midx}
      \blabel{p_{y,x}}[0.2]
		 }
$
& %3 c
(drawn also  as
& %4 c
$
\begin{array} {c}
\Rnode{lhsy}{y} \\[2.0cm]
\Rnode{lhsx}{x} 
\end{array})
\makebox[0.1cm]{\nccdar{lhsy}{lhsx}
      \blabel{p_{y,x}}[0.275]
		}
$
& %5
 as the composition 
& %6 c
$
\begin{array}{c}
%\Rnode{b}{B}&&\Rnode{xn}{w_n}&&\Rnode{xn1}{w_{n-1}}&&\Rnode{dots}{\ ...\ }&&\Rnode{x1}{w_1}&&\Rnode{a}{A} 
\Rnode{b}{y}\\[0.7cm]
\Rnode{xn}{w_n}\\[0.7cm]
\Rnode{xn1}{w_{n-1}}\\[0.1cm]
\Rnode{dots}{\vdots}\\[0.1cm]
\Rnode{x1}{w_1}\\[0.7cm]
\Rnode{a}{x} 
\end{array}
,
\makebox[0.1cm]{
\ncsar{b}{xn}
\alabel{p_y}
\ncsar{xn}{xn1}
\alabel{p_{w_n}}
\ncsar{xn1}{e1}
\ncline[linestyle=dotted,dotsep=4pt]{e1}{e2}
\ncsar{e2}{x1}
\ncsar{x1}{a}
\alabel{p_{w_1}}}
$ 
& %7 c
,
\end{tabular}

where
$w_1, ... w_n$ is the unique sequence of objects of $C$ such that 
$x \base w_1 \base ... \base w_n \base y$. If $x = y$, then define $p(y, x) = id_x$.
We say that the morphism  $p_{y,x}$ is a dependency morphism. 


\note
The contextual category structure supplies us with pullbacks for any $\smorph$ morphism, 
these given pullbacks can be pieced together to obtain a pullback for
any $\dmorph$ morphism  along any morphism with the same codomain. 

In general, whenever $y \leq z$ in $C$ and whenever $f:x \morph y$ in $C$, then we have
the following canonical pullback for the morphism $p_{z, y}$ along $f$, where
$w_1, ... w_n$ is the unique sequence of objects of $C$ such that 
$y \base w_1 \base ... \base w_n \base z$:

\vspace{3mm}
\begin{center}
\begin{equation}
\label{compositepullbackdefinition}
\begin{array}{cp{2.9cm}c}
\Rnode{TOPL}{q(...q(f, w_1)...w_n)^* z} & & \Rnode{TOPR}{z}\\ [1.2cm]
\Rnode{zOTTOML}{x}         & & \Rnode{zOTTOMR}{y}
\end{array}
\end{equation}
\jcbarr{f}{zOTTOML}{zOTTOMR}
\ncaarr{q(q(...q(f,w_1)...w_n),z)}{TOPL}{TOPR}
\nccdar{TOPL}{zOTTOML}
\blabel{p_{q(...q(f, w_1)...w_n)^* z,x}}
\nccdar{TOPR}{zOTTOMR}
\alabel{p_{z,y}}
\end{center}

Since these constructed pullbacks form an important part of contextual
category structure we would like a simpler notation for them. As no confusion is
likely, we extend the $^*$ and $q$ notation to cover these new pullback diagrams.
From now on if $f:x \morph y$ in $C$ and $y \leq z$ in $C$, then the diagram

\vspace{3mm}
\begin{center}
\begin{equation}
\label{compositepullbackout}
\begin{array}{cp{.9cm}c}
\Rnode{fstarz}{f^*z} & & \Rnode{z}{z}\\ [1.2cm]
\Rnode{x}{x}         & & \Rnode{y}{y}
\end{array}
\end{equation}
\nccdar{fstarz}{x}
\blabel{p_{f \sub z},x}
\jcbarr{f}{x}{y}
\ncaarr{q(f,z)}{fstarz}{z}
\nccdar{z}{y}
\alabel{p_{z,y}}
\end{center}
is the canonical pullback diagram as defined in (\ref{compositepullbackdefinition}) above. 
\note
The following observation
follows from the way the extended pullback diagrams are constructed. 
In the extended notation, if $f: x \morph y$ 
and $y \leq z \leq zz$ in the contextual category C, then
\begin{equation}
f^*zz = q(f, z)^*zz
\end{equation}
 and 
\begin{equation}
q(f, zz) = q(q(f, z), zz)
\end{equation}
and so the outer diagram in
\renewcommand{\pc}[2]{p_{#1,#2}}  % as \pc defined in ccategories macros differently to this
$
\begin{array}{ccp{.9cm}c}
\\[0.25cm]
&\Rnode{TL}{q(f,z)^*zz} & & \Rnode{TR}{zz}\\ [1.2cm]
&\Rnode{ML}{f^*z} & & \Rnode{MR}{z}\\ [1.2cm]
&\Rnode{BL}{x}         & & \Rnode{BR}{y} \\[1.0cm]
\end{array}
$
%composition
\makebox[0.2cm]{   % This make box prevents white space pushing out to the right
                   % cannot see where this white space is comin from. To investigate
									 % change the \makebox[0.2cm] to \fbox and you will see the problem.
\nccdar{TL}{ML}\blabel{P_{q(f,z)^*zz,f^*z}}\nccdar{ML}{BL}\blabel{p_{f \sub z,x}}\nccdar{TR}{MR}\alabel{p_z}
\nccdar{MR}{BR}
}
\alabel{p_z}
%reference
\ncarr{TL}{TR}
\alabel{q(q(f,z),zz)}
\ncarr{ML}{MR}
\alabel{q(f,z)}
\ncarr{BL}{BR}
\blabel{f}
is diagram (\ref{compositepullbackout}). 
On such diagrams as shown here the $p$ labels on $\dmorph$ morphisms (inclusive of their indices) are entirely predictable  and so in diagrams that follows we may omit them.


\note
If we write $\crossx{y}{z}{x}$ in place of ${p_{y,x}}^*z$, for $x < y$, $x < z$  in \ccat then
$\crossx{y}{z}{x}$  represents  in the syntax the `weakening' of a rule of the form
\begin{displaymath}
x, w1,...w_n \tstyle \isT{z}
\end{displaymath}
from a rule with context $x, w_1,...w_n$ to a rule with broader context $y, w_1, ... w_n$: 
\begin{displaymath} 
y, w_1,...w_n \tstyle \isT{z}.
\end{displaymath}

Within the contextual category I think of $\crossx{y}{z}{x}$  as a local cartesian product but of course categorically it is a filtered product i.e. a pullback. If $w < x$ and $w < y$  then 
\genericcrossxproductdiagram % defined in 'paper.tex'
is a pullback diagram in \ccat.

\note
We can extend the $\crossx{}{}{w}$ notation to morphisms. If $f:x \morph x'$ and $g: y \morph y'$ in a contextual
category $\ccat[C]$ and if $w$ is an object such that $w < x$, $w <x'$, $w < y$ and $w < y'$ then 
define $\crossx{f}{g}{w}:\crossx{x}{y}{w} \morph \crossx{x}{y}{w}$ in $\ccat[C]$ by
\begin{equation}
\crossx{f}{g}{w} = \tuple{p_{\crossx{x}{y}{w}, x} \circ f,q(p_{x,w},y) \circ g}
\end{equation}  

\note
In the case special case that $w$ is the terminal object $1$ then the pullback  specialises to give a product diagram:

\begin{displaymath}
\begin{array}{ccccc}
\Rnode{xy}{\crossx{x}{y}{1}} &&               &&               \\[1.3cm]
\Rnode{x}{x}                 &&               && \Rnode{y}{y}  \\                                    
\end{array}
\mbox{\ncsar{xy}{x}
\blabel{p_{\crossx{x}{y}{1},x}}
\ncaarr{q(p_{x,1},y)}{xy}{y}}
\end{displaymath}

In this special case the $\tuple{}$ operation defined earlier is the pairing operation for if
$f: w \morph x$ and $g: \morph y$ then $\tuple{f,g}: w \morph \crossx{x}{y}{1}$ 
and 
\begin{equation}
\tuple{f,g} \circ p_{\crossx{x}{y}{1},x} = f
\end{equation}
and
\begin{equation}
\tuple{f,g} \circ q(p_{x,1},y) = g
\end{equation}

\note 
Note that the product operation $\crossx{}{}{1}$ is far from symmetric 
because if, for example, $1 \base x$ and $1 \base y$ then $x \base \crossx{x}{y}{1}$ and $y \base \crossx{y}{x}{1}$ but we can define 
a swap operation $sw_{x,y} : \crossx{x}{y}{1} \morph \crossx{y}{x}{1}$ by
\begin{equation}
sw_{x,y} = \tuple{q_{p_x,y}, p_{\crossx{x}{y}{1},x}}
\end{equation}

\note
Associativity of $\crossx{}{}w$  follows from the coherence property of the pullbacks in the contextual category. 
For example if $w < x$, $w < y$, $w < z$ in a \ccat then from coherence of pullbacks in \ccat we have:
$\crossx{x}{(\crossx{y}{z}{w})}{w} = \crossx{(\crossx{x}{y}{w})}{z}{w}$ as shown here in this diagram:
 
\begin{displaymath}
\begin{array}{cp{1.0cm}cp{1.0cm}c}
\Rnode{J1}{}\Rnode{D1} {\crossx{(\crossx{x}{y}{w})}{z}{w}}\Rnode{J2}{} \ \ \ \ \   &&  &&  \\ 
= && && \\
\Rnode{D2} {\crossx{x}{(\crossx{y}{z}{w})}{w}}    &&  &&                        \\ [1.3cm]
\Rnode{xy}{\crossx{x}{y}{w}}&& \Rnode{yz}{\crossx{y}{z}{w}} &&                      \\[1.3cm]
\Rnode{x}{x}&& \Rnode{y}{y} && \ \ \ \ \ \ \ \ \ \ \ \ \ \Rnode{z}{z}                                        \\[1.3cm]
             && \Rnode{w}{w} &&                                                     
\end{array}
\end{displaymath}

\ncaarr[50]{q(\pc{\crossx{x}{y}{w}}{w},z)}{J2}{z}
\ncsar{D2}{xy}
\ncsar{xy}{x}
\ncsar{yz}{y}
\ncsar{x}{w}
\ncsar{y}{w} 
\ncsar{z}{w}
\ncaarr{q(\pc{x}{w},y)}{xy}{y}
\ncaarr{q(\pc{y}{w},z)}{yz}{z}
\ncaarr{q(\pc{x}{w},\crossx{y}{z}{w})}{D2}{yz}



\note A number of minor lemmas:

\begin{lemma}
\label{footandstactic}
If $f: A \morph B$ and $f':A \morph B$ in a contextual category \catcw then if 
$f \circ p_B$ = $f' \circ p_B$ and $s(f) = s(f')$ then $f=f'$.
\end{lemma}
\begin{proof}
Follows by axiom (s2) since we have:
$f = s(f) \circ q(f \circ p_B,B)  = s(f') \circ q(f' \circ p_B,B) = f'$.
\end{proof}

From which follows:
\begin{lemma}
\label{stactic}
If $A$ is any object of a contextual category \catcw and if $B$ is an object such that $1 \base B$ in \catcw then
if $f: A \morph B$ and $f':A \morph B$ in a contextual category \catcw then $f=f'$ iff $s(f) = s(f')$.
\end{lemma}

\begin{lemma}
\label{crosssectionlemma}
If 
%\begin{equation*}
$
\begin{array}{ c c c}
\Rnode{B}{B} &              & \Rnode{Bp}{B'} \\[1cm]
             & \Rnode{A}{A} &     
\mbox{\ncsar{B}{A}
\ncsar{Bp}{A}
%\ncarr[-30]{A}{Bp}
\ncrightsimplesection{A}{Bp}
\blabel{g}}
\end{array}
$
%\end{equation*}
in a contextual category \catcw and if $g$ is a section of $B'$ (i.e. if $g \circ p_{B'}= id_A$) so that we have 
\begin{equation*}
\begin{array}{ c c c}
\Rnode{BBp}{\crossx{B}{B'}{A}} \\[1.3cm]
\Rnode{B}{B} &              & \Rnode{Bp}{B'} \\[1.1cm]
             & \Rnode{A}{A} &
\mbox{
\ncsar{BBp}{B}
\ncrightcrosssection{B}{BBp}
\blabel{\crossx{B}{g}{A}}
\ncsar{B}{A}
\blabel{p_B}
\ncsar{Bp}{A}
\ncrightsimplesection{A}{Bp}
\blabel{g}
}														
\end{array}
\end{equation*}
in \catcw,  then
\begin{equation}
\label{crosssectionlemmatarget}
\crossx{B}{g}{A} = s(p_B \circ g).
\end{equation} 
\end{lemma}
\begin{proof}
$\crossx{B}{g}{A}$ is defined to be the unique section of $\crossx{B}{B'}{A}$ such that $(\crossx{B}{g}{A}) \circ q( p_B,B') = p_B \circ g$.

(\ref{crosssectionlemmatarget}) follows because $s(p_B \circ g)$ is also such a section, since it is defined to be the unique section of $(p_B \circ g \circ p_{B'}) ^* B'$
such that $s(p_B \circ g) \circ q( p_B \circ g \circ p_{B'}, B') = p_B \circ g$ and this simplifies,
because $g \circ p_{B'} =id_A$, to
 $s(p_B \circ g)$ being a section of ${p_B} ^* B'$ (i.e of $\crossx{B}{B'}{A}$) satisfying $s(p_B \circ g) \circ q( p_B,B') = p_B \circ g$. 
\end{proof}

% *************************************************************************************
% sfglemma ****************************************************************************
\begin{lemma}
\label{sfglemma}
If $f:A \morph B$ and $g:B\morph C$ in a contextual category \catcw and if $C \in Cover(B)$ then
\begin{equation}
\label{sgflemmagoalone}
ft(f\circ g)^*C = f^*(ft(g)^*C)
\end{equation}
and 
\begin{equation}
\label{sgflemmagoaltwo}
s(f\circ g)=f^*s(g)
\end{equation}
where $ft(g) = g \circ p_B$ so that we have
\begin{displaymath}
\begin{array}{ccp{1.7cm}cp{1.7cm}c}
ft(f\circ g)^*C=\kern-10pt&\Rnode{TL}{f^*(ft(g)^*C)} & & \Rnode{TC}{ft(g)^*C}          \\ [1.7cm]
&\Rnode{BL}{A}         & & \Rnode{BC}{B} && \Rnode{BR}{C}
\end{array}
\mbox{
\ncsar{TL}{BL}
\ncsar{TC}{BC}
\ncarr{BL}{BC}
\blabel{f}
\ncarr{BC}{BR}
\blabel{g}
\ncarr{TL}{TC}
\alabel{q(f,ft(g)^*C)}
\ncarr{TC}{BR}
\alabel{q(ft(g),C)}
\ncleftsimplesection{BL}{TL}
\alabel{s(f\circ g)=f^*s(g)}
\ncleftsimplesection{BC}{TC}
\alabel{s(g)}
}
\end{displaymath}
in \catc.
\end{lemma}
\begin{proof}
That (\ref{sgflemmagoalone}) holds follows by axiom (q4).

To show that (\ref{sgflemmagoaltwo}) holds remember that $s(f \circ g)$is the unique section of $ft(f \circ g)*C$
such that $s(f \circ g) \circ q(f \circ g \circ p_C,C) = f\circ g$. Therefore it suffices to show that
$f^*s(g) \circ q(f \circ g \circ p_C,C) = f\circ g$ and this we can show as follows:
\begin{align*}
(f^*s(g)) \circ q(f \circ g \circ p_C,C) &= (f^*s(g)) \circ q(f ,(g \circ p_C)^*C) \circ q(g \circ p_C ,C) &&\mbox{by axiom (qf)} \\
                             &= f \circ s(g) \circ q(g \circ p_C ,C)                   &&\mbox {from defn. of $f^*s(g)$}\\
														 &= f \circ g                                              &&\mbox {by axiom (s2)}
\end{align*}
\end{proof}

\iffalse
\newcommand{\duplesone}{{\duple{s_1}_{y_1}}}
\newcommand{\duplestwo}{{\duple{s_1,s_2}_{y_2}}}

\newcommand{\duplesn}{\duple{s_1,...s_n}_{y_n}}
\newcommand{\duplesi}{{\duple{s_1,...s_i}_{y_i}}}
\newcommand{\duplesilessone}{\duple{s_1,...s_{i-1}}_{y_{i-1}}}
\newcommand{\duplesj}{{\duple{s_1,...s_j}_{y_j}}}
\newcommand{\duplesjlessone}{\duple{s_1,...s_{j-1}}_{y_{j-1}}}
\newcommand{\duplesisucc}{{\duple{s_1,...s_{i+1}}_{y_{i+1}}}}
\newcommand{\duplesnlessone}{{\duple{s_1,...s_{n-1}}_{y_{n-1}}}}
\newcommand{\ynz}{\crossx{y_n}{z}{y_i}}


\newcommand {\sonesub}{{s_1}^*}
\newcommand {\stwosub}{{s_2}^*}
\newcommand {\stwocascade}{\stwosub\sonesub}
\newcommand {\sisub}{{s_i}^*}
\newcommand {\sicascade}{\sisub...\sonesub}
\newcommand {\sisuccsub}{{s_{i+1}}^*}
\newcommand {\sisucccascade}{\sisuccsub...\sonesub}
\newcommand {\snlessonesub}{{s_{n-1}}^*}
\newcommand {\snlessonecascade}{\snlessonesub...\sonesub}
\newcommand {\snsub}{{s_n}^*}
\newcommand {\sncascade}{\snsub...\sonesub}

If $x$ is an object of contextual category \catc, if $1 \base y_1 ... \base y_n$ in \catcw and if
$\sntuple$ is a cascade from $x$ to $y_n$ in \catcw
then  we can define a morphism
$\duplesn:x \morph y_n$ in \catcw such that 
\begin{axiom}{d1}
s(\duplesn) = s_n,
\end{axiom}
for $n> 1$, 
\begin{axiom}{d2}
\duplesn \circ p_{y_n} = \duplesnlessone, 
\end{axiom}
and for all objects $y \in Cover(y_n)$, 
\begin{axiom}{d3a}
{\duplesn} ^*y = \sncascade (\crossx{x}{y}{1}),
\end{axiom}
and for all sections $g$ of $y$,
\begin{axiom}{d3b}
{\duplesn} ^* g = \sncascade (\crossx{x}{g}{1}).
\end{axiom}


The definition of $\duplesn$ proceeds by induction. 
Define $\duplesone= s_1 \circ q(p_{x,1},y_1)$.
By axiom (s1), we directly establish, (d1), that $s(\duplesone)=s_1$.

If $y_1 <y$ in \catcw then we establish (d3a), that $\sonesub (\crossx{x}{y}{1})=\duplesone ^*y$ as follows
\begin{align*}
\sonesub (\crossx{x}{y}{1})&= \sonesub q(p_{x,1},y_1)^*y     && \mbox{by definition of $\crossx{}{}{1}$,}\\
                         &= (s_1 \circ q(p_{x,1},y_1))^*y   && \mbox{by pullback coherence axiom (q5),}\\
                         &= \duplesone ^*y                   && \mbox{by definition of $\duplesone ^*y$.}
\end{align*}
Also, if $g$ is a section of $y$ then we can show (d3b),  
that $\sonesub (\crossx{x}{g}{1})=\duplesone ^* g$, by a similar argument. 

Now assume that $\duplesi$ is defined and satisfies (\ref{d1}) to (\ref{d3b}) above. 
In particular we have  $\sicascade y_{i+1} = \duplesi ^*y_{i+1}$, and therefore that
$s_{i+1} \in Sect(\duplesi ^*y_{i+1})$. This allows us to define $\duplesisucc$ by 
\begin{equation*}
\duplesisucc = s_{i+1} \circ q(\duplesi, y_{i+1}).
\end{equation*} 
Immediately by axiom (s3)
we establish (d1), that $s(\duplesisucc)=s_{i+1}$.
We establish (d2), that $\duplesisucc \circ p_{y_{i+1}}= \duplesi$, as follows
\begin{align*}
\duplesisucc \circ p_{y_{i+1}} &=s_{i+1} \circ q(\duplesi, y_{i+1}) \circ p_{y_{i+1}} && \mbox{by definition of $\duplesisucc$,} \\
                               &=s_{i+1} \circ p_{\duplesi ^*y_{i+1}} \circ \duplesi && \mbox{by (Q1),} \\
															 &= \duplesi                       && \mbox{because $s_{i+1}$ is a section.}
\end{align*}
To establish (\ref{d3a}), suppose $y$ is some object such that $y_{i+1} < y$ in \catcw then we can show that $\sisucccascade (\crossx{x}{y}{1})=\duplesisucc ^*y$ as follows:
\begin{align*}
\sisucccascade (\crossx{x}{y}{1}) 
              &= \sisuccsub \duplesi ^*y && \mbox{by inductive hypothesis,} \\
                         &= \sisuccsub q(\duplesi,y_{i+1})^*y  && \mbox{by (Q6),}\\
                         &= (s_{i+1} \circ q(\duplesi,y_{i+1}))^*y   && \mbox{by (Q4),}\\
                         &= \duplesisucc ^*y                   && \mbox{by definition of $\duplesisucc$.}
\end{align*}
A similar argument  shows that if $g$ is a section of $y$ then $\sisucccascade (\crossx{x}{g}{1})=\duplesisucc ^* g$ to establish (\ref{d3b}).


\begin{lemma}
\llabel{dupledestructionlemma}
If $\duplesn : x \morph y_n$ in a contextual category $\catcw$ then \foreachi, 
\begin{equation}
\duplesn \circ p_{y_n,y_i} = \duplesi
\end{equation} 
\end{lemma}
\begin{proof}
Follows because for each $j$, $i < j \leq n$, $\duplesj \circ p_{y_j} = \duplesjlessone$
and $p_{y_j,y_i} = p_{y_j} \circ p_{y_{j-1},y_i}$.
\end{proof}
\begin{lemma}
\llabel{dupleofslemma}
If $x$ and $y$ are objects of a contextual category \catcw such that $1 \base y$ and if $g: x \morph y$ is a morphism then
\begin{equation*}
\duple{s(g)} = g
\end{equation*}
\end{lemma}
\begin{proof}
By definition of $s$, $s(g):x \morph \crossx{x}{y}{1}$ in \catc. 
Therefore,by definition of $\duple{}$,  $\duple{s(g)}$ is defined,  $\duple{s(g)}:x \morph y$ in \catcw and
satisfies (d1) i.e. that $s(\duple{s(g)}) = s(g)$ 
and therefore that, $by lemma \lref{stactic}, \duple{s(g)}=g$.
\end{proof}
%
%
%
{ % BEGIN   {thegeneraldupletuplelemma} and proof
\newcommand{\tuplesnsg}{\tuple{s_1,...s_n,s(g)}}
\newcommand{\duplesnsg}{\duple{s_1,...s_n, s(g)}_{f^*z}}
\newcommand{\dupletuplerhs}{\bigtuple{\duplesn,g}}
\begin{lemma}
\llabel{thegeneraldupletuplelemma} 
If $x$, $y_1$,...$y_n$, $z_p$ and $z$ are objects of a contextual category \catcw 
such that $1 \base y_1 ... \base y_n$ and $z_p \base z$ in \catc, 
if $f: y_n \morph z_p$, so that there is this pullback diagram 
\begin{displaymath}
\begin{array} {c p{3cm} c p{2cm} c}
              && \Rnode{TL}{f^*z}  && \Rnode{TR}{z}  \\[1.2cm]
              && \Rnode{BL}{y_n}   && \Rnode{BR}{z_p}
\end{array}
\begin{arrows}
\ncsar{TL}{BL}
\ncsar{TR}{BR}
\ncarr{TL}{TR}
\alabel{q(f,z)}
\ncarr{BL}{BR}
\blabel{f}
\end{arrows}
\end{displaymath}
in \catc, if $\tuplesnsg$ is a cascade from $x$ to $f^*z$ and $g:x \morph z$ in \catcw 
such that
\begin{equation} \label{generaldupletuplegiven}
g \circ p_z = \duplesn \circ f
\end{equation} 
then
\begin{equation}
\label{generaldupletuplegoaltwo}
\duplesnsg = \dupletuplerhs
\end{equation}
where $\dupletuplerhs$ is the unique morphism $\dupletuplerhs:x \morph f^*z$ such that
\begin{equation}
\label{generaldupletupledefone}
\dupletuplerhs \circ q(f,z) =g
\end{equation} 
and 
\begin{equation}
\dupletuplerhs \circ p_{f^*z} = \duplesn
\end{equation}
 as shown in this diagram
\begin{displaymath}
\begin{array} {c p{3cm} c p{2cm} c}
% FL is Far Left !!
\Rnode{FL}{x} &&                   &&                \\[1.0cm]
              && \Rnode{TL}{f^*z}  && \Rnode{TR}{z}  \\[1.2cm]
              && \Rnode{BL}{y_n}   && \Rnode{BR}{z_p}
\end{array}
\begin{arrows}
\ncsar{TL}{BL}
\ncsar{TR}{BR}
\ncarr[30]{FL}{TR}
\alabel{g}
\ncarr{FL}{TL}
\alabel{\dupletuplerhs}[0.7][1]
\ncarr[-20]{FL}{BL}
\blabel{\duplesn}[0.5][0]
\ncarr{TL}{TR}
\alabel{q(f,z)}
\ncarr{BL}{BR}
\blabel{f}
\end{arrows}
\end{displaymath}
\end{lemma}
\begin{proof}
To show (\ref{generaldupletuplegoaltwo})
we need just show that
\begin{equation}
\label{generaldupletuplesubgoalone}
\duplesnsg \circ q(f,z) =g
\end{equation} 
and 
\begin{equation}
\label{generaldupletuplesubgoaltwo}
\duplesnsg \circ p_{f^*z} = \duplesn
\end{equation}
(\ref{generaldupletuplesubgoalone}) follows because by lemma \lref{footandstactic}
 it suffices to show that
\begin{equation}
\label{generaldupletuplesubgoaloneone}
\duplesnsg \circ q(f,z) \circ p_z =g \circ p_z,
\end{equation}
which follows directly from (\ref{generaldupletupledefone}), and
\begin{equation}
\label{generaldupletuplesubgoalonetwo}
s(\duplesnsg \circ q(f,z)) =s(g)
\end{equation}
which we prove as follows:
\begin{align*}
s(\duplesnsg \circ q(f,z)) &=s(\duplesnsg  && \mbox{ by (s3),} \\
                          &= s(g)                       && \mbox{ by (d1).}
\end{align*}
whereas (\ref{generaldupletuplesubgoaltwo}) is an instance of clause (d2) of the definition of $\duple{}$.
\end{proof}
} % END   {thegeneraldupletuplelemma} and proof
%
%
%

{ % BEGIN {thedupletuplelemma} and proof and the {absolutedupletuplesublemma} and proof
\newcommand{\tuplesnsg}{\tuple{s_1,...s_n, s(g)}} 
\newcommand{\duplesnsg}{\duple{s_1,...s_n, s(g)}_{\ynz}}
\newcommand{\dupletuplerhs}{\bigtuple{\duplesn,g} }
\begin{lemma}
\llabel{thedupletuplelemma} 
If $x$, $y_1$,...$y_n$ and $z$ are objects of a contextual category \catcw such that $1 \base y_1 ... \base y_n$ in \catcw and
$y_i \base z$, for some $i$, $1 \leq i \leq n$,
if $\sntuple$ is a cascade from $x$ to $y_n$ in \catcw 
and if $g: x \morph z$ in \catcw such that
$g \circ p_z = \duplesi$, 
then 
$\tuplesnsg$ is a cascade from $x$ to $\ynz$ in \catcw
and
\begin{equation}
\label{dupletuplegoal}
\duplesnsg = \dupletuplerhs\,.
\end{equation}
\end{lemma}
\begin{proof}

To show that $\tuplesnsg$ is a cascade from $x$ to $\ynz$ in \catcw we need show that
$s(g) \in Sect({s_n}^* ... {s_1}^* (\crossx{x}{(\ynz)}{1})$.
This follows because by definition of $s(g)$, $s(g) \in Sect((g \circ p_z) ^*z)$ and because
\begin{align*}
(g \circ p_z) ^*z &= \duplesi ^*z                                && \mbox{from initial assumption that $g \circ p_z =\duplesi$,}\\
                  &= (\duplesn \circ {p_{y_n,y_i}})^*z           && \mbox{by lemma \lref{dupledestructionlemma},} \\
                  &= {\duplesn} ^* {p_{y_n,y_i}}^*z              && \mbox{by (q4),} \\
                  &= {\duplesn} ^* (\ynz)                        && \mbox{by definition of $\crossx{}{}{}$,} \\
                  &= {s_n}^* ... {s_1}^* (\crossx{x}{(\ynz)}{1}) && \mbox{by (d3a).}
\end{align*}

Now (\ref{dupletuplegoal}) follows as a special case of lemma \lref{thegeneraldupletuplelemma} with $z_p$ being $y_i$ and $f$ being $p_{y_n,y_i}$
provided that we can show that 
\begin{equation}
g \circ p_z = \duplesn \circ p_{y_n,y_i}.
\end{equation}
This follows from the assumption that $g \circ p_z = \duplesi$ and from lemma \ref{dupledestructionlemma}.
\end{proof}
% END of {thedupletuplelemma} and proof%
%

\begin{lemma}
\llabel{absolutedupletuplesublemma}
If $x$, $y_1$,...$y_n$ and $z$ are objects of a contextual category \catcw such that $1 \base y_1 ... \base y_n$ in \catcw and
$1 \base z$, 
if $\sntuple$ is a cascade from $x$ to $y_n$ in \catcw 
and if $g: x \morph z$ in \catcw 
then $\tuplesnsg$ is a cascade from $x$ to $\ynz$ in \catcw
and
\begin{equation}
\label{dupletuplegoalx}
\duplesnsg = \dupletuplerhs\,.
\end{equation}
\end{lemma}
\begin{proof}

To show that $\tuplesnsg$ is a cascade from $x$ to $\ynz$ in \catcw we need show that
$s(g) \in Sect({s_n}^* ... {s_1}^* (\crossx{x}{(\ynz)}{1})$.
This follows because by definition of $s(g)$, $s(g) \in Sect((g \circ p_z) ^*z)$ and because
\begin{align*}
(g \circ p_z) ^*z  &= (\duplesn \circ {p_{y_n,1}})^*z           && \mbox{because $1$ is terminal,} \\
                  &= {\duplesn} ^* {p_{y_n,1}}^*z               && \mbox{by (q4),} \\
                  &= {\duplesn} ^* (\ynz)                       && \mbox{by definition of $\crossx{}{}{}$,} \\
                  &= {s_n}^* ... {s_1}^* (\crossx{x}{(\ynz)}{1} && \mbox{by (d3a).}
\end{align*}

Now (\ref{dupletuplegoalx}) follows as a special case of lemma \lref{thegeneraldupletuplelemma} with $z_p$ being $1$ and $f$ being $p_{y_n,1}$
provided that we can show that 
\begin{equation}
g \circ p_z = \duplesn \circ p_{y_n,1}.
\end{equation}
This holds because $1$ is terminal.
\end{proof}
}  %end scope for two lemmas



\begin{lemma}
\llabel{absolutedupletuplelemma}
For $n \geq 1$, if $x$ and $y_1,...y_n$ are objects of a contextual category \catcw such that \foreachi, $1 \base y_i$ and if \foreachi, $f_i: x \morph y_i$ then
\begin{equation*}
\duple{s(f_1),...s(f_n)}=\tuple{\fn}
\end{equation*}
\end{lemma}
\begin{proof}
\begin{align*}
\duple{s(f_1),...s(f_n)} &= \tuple{\duple{s(f_1),...s(f_{n-1})},f_n} &&\mbox{by lemma \lref{absolutedupletuplesublemma},}\\
                         &= \tuple{\tuple{f_1,...f_{n-1}},f_n}       &&\mbox{by the inductive hypothesis,}  \\
                         &= \tuple{\fn}                              && \mbox{by definition of $\tuple{}$.}
\end{align*}
\end{proof}

\begin{lemma}
\llabel{absolutedupletuplelemma}
For $n > 1$, if $x$ and $y_1,...y_n$ are objects of a contextual category \catcw such that \foreachi, $1 \base y_i$ and if \foreachi, $f_i: x \morph y_i$ then
\begin{equation*}
\tuple{\fn}=\duple{s(f_1),...s(f_n)}
\end{equation*}
\end{lemma}
\begin{proof}
\begin{align*}
\duple{s(f_1),...s(f_n)} &= \tuple{\duple{s(f_1),...s(f_{n-1})},f_n} &&\mbox{by lemma \lref{thegeneraldupletuplelemma},}\\
                         &= \tuple{\tuple{f_1,...f_{n-1}},f_n}       &&\mbox{by the inductive hypothesis,}  \\
                         &= \tuple{\fn}                              && \mbox{by definition of $\tuple{}$.}
\end{align*}
\end{proof}

\begin{lemma}
\llabel{duplesofplemma}
If $x$,$y_1$,...$y_m$ are objects of a contextual category \catcw such that $x \base y_1...\base y_m$ in \catcw then
\newcommand{\xyj}[1]{\crossx{x}{y_{#1}}{1}}
\begin{equation*}
\duple{s(p_{\xyj{m},\xyj{1}},...s(p_{\xyj{m},\xyj{m}})} = q(p_{x,1},y_m)
\end{equation*}
\end{lemma}
\begin{proof}
\tbd
\end{proof}


\fi

