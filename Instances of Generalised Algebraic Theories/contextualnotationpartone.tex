

\label{contextualnotationpartone}
\subsection{Definition}
Terminology: By  the generic term \term{tree} is meant a partially ordered set (poset) $(T, <)$ such that for each $t \in T$, the set $\set{s \in T : s < t}$ is well-ordered by the relation $<$.
In this discussion we restrict ourselves to rooted $\omega$-trees i.e. trees for which the set $\set{s \in T : s < t}$
is finite for all $t \in T$ and for which there is a least element in the partial ordering. 

With respect to a partial ordering $<$, we say that an element $y$ \textit{covers}  an element $x$  iff $x<y$ and there does not exist $w$ such that $x < w$ and $w < y$.
If object $y$ covers object $x$ in the partial ordering 
then we write $x \base y$ (we use this in preference to the more usual $x \lessdot y$).
We shall write $Cover(x)$ for the set of elements that cover an element $x$.

We define the \term{rank} (sometimes this is called the grade) of an element $t \in T$ to be the cardinality
of the set $\setsuchthat{s \in T}{s < t}$. If we define the set $T_i$ to be the set of elements of a tree
of rank $i$ then we have that $T= \bigcup_{i \in N}T_i$. 


By a \term{tree-structured category} we mean (i) a category with a tree-structure defined on its objects such that the tree of objects has a unique root object 
that is also a terminal object of the category and (ii) for every $x \base y$ in the tree of objects  a canonical morphism $p_y:y \rightarrow x$. 
I shall say morphisms of this form  are \term{direct dependency morphisms} and they will
be distinguished in diagrams by an arrow with  a triangular head so:
\begin{center}
$
\begin{array}{p{2cm}}
\Rnode{y}{y}\\ [1.4cm]
\Rnode{x}{x} \\
\mbox{\ncbsar{p_y}{y}{x}}
\end{array}
$
\end{center}

If $x$ is an object of a tree-structured category \catcw and if $y \in Cover(x)$ in \catcw then we define 
the set  of sections of $y$, denoted $Sect(y)$, to be the set of morphisms $s: x \morph y$ in \catcw  such that $s \circ p_y = id_x$. We have therefore that if $s$ is a section of $y$ and $y$ covers $x$ then
\ \ \ \ 
\begin{tabular}{cccc}
$
\begin{array}{p{0.5cm}}
\Rnode{y}{y} \\ [1.4cm]
\Rnode{x}{x} \\
\mbox{
\ncsar{y}{x}
\alabel{p_y}
\ncarrZZ[30]{x}{y} 
\alabel{s}}
\end{array}
$  & in \catcw and &
$
\begin{array}{c p{0.5cm}c p{0.5cm}c}
              && \Rnode{y}{y}&&                \\ [1.4cm]
\Rnode{x1}{x} &&             &&   \Rnode{x2}{x}\\
\mbox{
\ncsar{y}{x2}
\alabel{p_y}
\ncarr{x1}{y} 
\alabel{s}
\ncarr{x1}{x2} 
\blabel{id_x}
}
\end{array}
$& commutes.
\end{tabular}

The following is the original  definition of contextual category given  in [1] and [2], 
\begin{definition}
\llabel{contextualcategory}
A \term{contextual category} is defined to be a tree-structured category 
\cat{C} with the following additional structure:

\noindent 
(i) whenever
$
\begin{array}{cp{.9cm}c}
            & & \Rnode{z}{z} \\ [1.2cm]
\Rnode{x}{x}& & \Rnode{y}{y} \\ [0.5cm]
\end{array}
$
\jcbarr{f}{x}{y}
\ncasar{p_z}{z}{y}
in \cat{C}, an object $f \sub z$ such that $x \base f \sub z$, a morphism $q(f,z): f \sub z \rightarrow z$ such that

\begin{axiom}{q1}
q(f,z) \circ p_z = p_{f \sub z} \circ f
\end{axiom}

\noindent i.e. such that the diagram: \ \ \ 
$
\ccsquareoutline{0.9cm}{1.2cm}{f^*z}{z}{x}{y}
\ccsquareacross{q(f,z)}{f}
\ccsquaredown{p_{f \sub z}}{p_z}
$
commutes and, (ii), is a pullback diagram, that is: \\
\hspace{0.2cm}

\noindent for all objects $w$ of \cat{C}, and for all
morphisms $h_1: w \rightarrow x$ and $h_2: w \rightarrow z$ such that
$h_1 \circ f = h_2 \circ p_z$ 
there exists a unique $h:w \rightarrow f \sub z$ in \cat{C} such that
$h \circ p_{f \sub z} = h_1$ and $h \circ q(f,z) = h_2$, as shown here:

\vspace{3mm}
\begin{center}
\begin{equation}
\label{pullback}
\begin{array}{cp{0.5cm}cp{1.2cm}c}
\Rnode{w}{w} &&                     &&           \\ [0.7cm]
             &&\Rnode{fstarz}{f^*z} && \Rnode{z}{z}\\ [1.2cm]
             &&\Rnode{x}{x}         && \Rnode{y}{y}
\end{array}
\end{equation}
\ncbsar{p_{f \sub z}}{fstarz}{x}
\jcbarr{f}{x}{y}
\ncaarr{q(f,z)}{fstarz}{z}
\ncasar{p_z}{z}{y}
\setlength{\arrnodesepA}{3pt}
\jcbarr[-35]{h_1}{w}{x}
\ncaarr[35]{h_2}{w}{z}
\psset{linestyle=dashed}
\ncaarr{h}{w}{fstarz}
\end{center}

\vspace {0.25cm}
\noindent and so that (iii) whenever $x \base y$ in \cat{C}, 
\begin{axiom}{q2}
id_x^*y=y
\end{axiom}

and

\begin{axiom}{q3}
q(id_x,y) = id_y
\end{axiom}

\noindent and (iv) whenever 
$
\begin{array}{c p{.9cm} c p{.9cm} c}
             &   &             &   & \Rnode{z}{z} \\ [1.2cm]
\Rnode{w}{w} &   &\Rnode{x}{x} &   & \Rnode{y}{y} \\ [0.5cm]
\end{array}
$
\jcbarr{f}{w}{x}
\jcbarr{g}{x}{y}
\ncasar{c}{z}{y}
in \cat{C}, 

then

\begin{axiom}{q4}
(f \circ g)^*z =  f^* (g ^* z)
\end{axiom}

and 
\begin{axiom}{q5}
q(f \circ g,z) = q(f,g^*z) \circ q(g,z)
\end{axiom}

\end{definition}



\subsection{Extended $p$, $q$ and $^*$ notation}
We can introduce several additional notations for use in contextual categories. 
If $x < y$ in the contextual category \catc, then define the morphism $p_{y,x}:y \morph  x$ in \catc, \\

\begin{tabular}{c c c  c  c  c c}
by defining
& %2 c
$
\begin{array} {c}
\Rnode{midy}{y} \\[2.0cm]
\Rnode{midx}{x}  \\ 
\end{array}
\mbox{\ncarr{midy}{midx}
      \blabel{p_{y,x}}[0.2]
		 }
$
& %3 c
(drawn also  as
& %4 c
$
\begin{array} {c}
\Rnode{lhsy}{y} \\[2.0cm]
\Rnode{lhsx}{x} 
\end{array})
\makebox[0.1cm]{\nccdar{lhsy}{lhsx}
      \blabel{p_{y,x}}[0.275]
		}
$
& %5
as the composition 
& %6 c
$
\begin{array}{c}
%\Rnode{b}{B}&&\Rnode{xn}{w_n}&&\Rnode{xn1}{w_{n-1}}&&\Rnode{dots}{\ ...\ }&&\Rnode{x1}{w_1}&&\Rnode{a}{A} 
\Rnode{b}{y}\\[0.7cm]
\Rnode{xn}{w_n}\\[0.7cm]
\Rnode{xn1}{w_{n-1}}\\[0.1cm]
\Rnode{dots}{\vdots}\\[0.1cm]
\Rnode{x1}{w_1}\\[0.7cm]
\Rnode{a}{x} 
\end{array}
,
\makebox[0.1cm]{
\ncsar{b}{xn}
\alabel{p_y}
\ncsar{xn}{xn1}
\alabel{p_{w_n}}
\ncsar{xn1}{e1}
\ncline[linestyle=dotted,dotsep=4pt]{e1}{e2}
\ncsar{e2}{x1}
\ncsar{x1}{a}
\alabel{p_{w_1}}}
$ 
& %7 c
\end{tabular}

\noindent where $w_1, ... w_n$ is the unique sequence of objects of $C$ such that 
$x \base w_1 \base ... \base w_n \base y$. It is also useful to have a definition of $p_{x,x}$ as the identity morphism
$id_x$, for any object $x$, so that $p_{y,x}$ is defined for any pair of objects $x$ and $y$ such that $x \leq y$.
We say that any morphism  of the form $p_{y,x}$ is a dependency morphism. 

The contextual category structure supplies us with pullbacks for direct dependency morpisms.
These can be pieced together to obtain  a pullback for any indirect dependency morphism   along any morphism with common codomain
as  whenever $y < z$ in $C$ and whenever $f:x \morph y$ in $C$, then we have
the following canonical pullback for the morphism $p_{z, y}$ along $f$, where
$w_1, ... w_n$ is the unique sequence of objects of $C$ such that 
$y \base w_1 \base ... \base w_n \base z$:

\vspace{3mm}
\begin{center}
\begin{equation}
\label{compositepullbackdefinition}
\begin{array}{cp{2.9cm}c}
\Rnode{TOPL}{q(...q(f, w_1)...w_n)^* z} & & \Rnode{TOPR}{z}\\ [1.2cm]
\Rnode{zOTTOML}{x}         & & \Rnode{zOTTOMR}{y}
\end{array}
\end{equation}
\jcbarr{f}{zOTTOML}{zOTTOMR}
\ncaarr{q(q(...q(f,w_1)...w_n),z)}{TOPL}{TOPR}
\nccdar{TOPL}{zOTTOML}
\blabel{p_{q(...q(f, w_1)...w_n)^* z,x}}
\nccdar{TOPR}{zOTTOMR}
\alabel{p_{z,y}}
\end{center}

  

Since these constructed pullbacks form an important part of contextual
category structure we would like a simpler notation for them. As no confusion is
likely, we extend the $^*$ and $q$ notation to cover these new pullback diagrams.
From now on if $f:x \morph y$ in $C$ and $y \leq z$ in $C$, then $f^*z$ 
is defined to be $q(...q(f, w_1)...w_n)^* z$ as shown in diagram (\ref{compositepullbackdefinition}) 
and $q(f,z)$ is defined as the secondary projection shown in (\ref{compositepullbackdefinition}), i.e. as 
$q(q(...q(f,w_1)...w_n),z)$,
so that (\ref{compositepullbackdefinition}) can be rewritten as:

\vspace{3mm}
\begin{center}
\begin{equation}
\label{compositepullbackout}
\begin{array}{cp{.9cm}c}
\Rnode{fstarz}{f^*z} & & \Rnode{z}{z}\\ [1.2cm]
\Rnode{x}{x}         & & \Rnode{y}{y}
\end{array}
\end{equation}
\nccdar{fstarz}{x}
\blabel{p_{f \sub z},x}
\jcbarr{f}{x}{y}
\ncaarr{q(f,z)}{fstarz}{z}
\nccdar{z}{y}
\alabel{p_{z,y}}
\end{center}

In addition whenever $f:x \morph y$ in \catcw then we define $f^*y=y$ and $q(f,y)=f$ so that $f^*z$ and $q(f,z)$ are defined
and give a pullback diagram whenever $y \leq z$ in \catc.

\newcommand{\pbpair}[2]{\tuple{#2}_{#1}}

The following observation
follows from the way the extended pullback diagrams are constructed. 
\begin{lemma}
\llabel{extendedstarcoheres}
In the extended notation, if $f: x \morph y$ 
and $y \leq z \leq zz$ in the contextual category C, then
\begin{equation}
f^*zz = q(f, z)^*zz
\end{equation}
 and 
\begin{equation}
q(f, zz) = q(q(f, z), zz)
\end{equation}
and so the outer diagram in
%\renewcommand{\pc}[2]{p_{#1,#2}}  % as \pc defined in ccategories macros differently to this   %UNUSED
$
\begin{array}{ccp{.9cm}c}
\\[0.25cm]
&\Rnode{TL}{q(f,z)^*zz} & & \Rnode{TR}{zz}\\ [1.2cm]
&\Rnode{ML}{f^*z} & & \Rnode{MR}{z}\\ [1.2cm]
&\Rnode{BL}{x}         & & \Rnode{BR}{y} 
\end{array}
\begin{arrows} 
%composition  
\nccdar{TL}{ML}
\blabel{p_{q(f,z)^*zz,f^*z}}
%
\nccdar{ML}{BL}
\blabel{p_{f \sub z,x}}
%
\nccdar{TR}{MR}
\alabel{p_{zz,z}}
%
\nccdar{MR}{BR}
\alabel{p_{z,y}}
%reference
\ncarr{TL}{TR}
\alabel{q(q(f,z),zz)}
%
\ncarr{ML}{MR}
\alabel{q(f,z)}
%
\ncarr{BL}{BR}
\blabel{f}
\end{arrows}
$
is diagram (\ref{compositepullbackout}). 
\end{lemma}

\subsection{Axiomatisation of the extended $p$, $q$ and $^*$ notation}
\newcommand{\ssub}{\kern-2pt^*\kern-1pt}
\renewcommand{\sub}{^*\kern-1pt}
\newcommand{\hash}{^\#}
\newcommand{\byaxiom}[1]{by axiom (\ref{#1})}
%
%\newcommand{\pbar}{\bar{p}}
%\newcommand{\pp}[2]{\pbar(#1,#2)}
\newcommand{\pp}[2]{p_{#1,#2}}

%
%\newcommand{\qbar}{\bar{q}}
%\newcommand{\extq}[2]{\qbar(#1,#2)}   %sadly I have previously used \qq for quine quote
\newcommand{\extq}[2]{q(#1,#2)}   
%
\newcommand{\dsub}{^{\ast \kern-1pt \ast}\kern-2pt}

\newcommand{\pbartag}[1]{$P#1$}

We present the basic properties of the extended $p$, $q$ and $^*$ notation in the form of a  revised definition of contextual category. This revised definition is easily be shown to be  equivalent to the original definition\footnote{Note that (\ref{qq1}) to (\ref{qq5}) in this definition are extended forms of  (q1) to (q5) from the original definition.}.

\begin{definition}
A  \term{contextual category} is a category \catcw with a tree-structure defined on its objects 
such that the tree of objects has a unique root object
and  with  the following additional structure
\begin{itemize}
\item  for every $x \leq y$ in the tree of objects  a cannonical morphism $\pp{y}{x} :y \rightarrow x$
satisfying
\item for any object $x$ of $\catcw$,
\begin{axiomtagged}{pp1}{$\pbartag{1}$}
\pp{x}{x}=id_x
\end{axiomtagged}
\item for all objects $x$, $y$ and $z$ of \catcw such that $x \leq y \leq z$
\begin{axiomtagged}{pp2}{$\pbartag{2}$}
\pp{z}{y} \circ \pp{y}{x} = \pp{z}{x}
\end{axiomtagged}
\end{itemize}


%\newcommand{\qbartag}[1]{$\qbar \textsl{#1}$}
%\newcommand{\qbartag}[1]{$\qbar \textsl{#1}$}
\newcommand{\qbartag}[1]{$Q \textsl{#1}$}
\begin{itemize}
\item
for all morphisms $f: x \morph y$ in \catcw and for all objects $z$ such that $y \leq z$ in \catc, 
an object $f \dsub z$ such that $x \leq f \dsub z$ and
\begin{itemize}
\item such that for all morphisms $f:x \morph y$ in \catcw 
\begin{axiomtagged}{qq0}{\qbartag{0}}
f \dsub y = x
\end{axiomtagged}
\item such that if $w$, $x$, $y$ and $z$ are objects of \catc, if $f:w \morph x$ in \catcw and $x \leq y \base z$ then $f \dsub y \base f \dsub z$ in \catc,
\end{itemize}
\item
for all morphisms $f: x \morph y$ in \catcw and for all objects $z$ such that $y \leq z$ in \catc, 
a morphism $\extq{f}{z}: f \dsub z \morph z$ such that
\begin{axiomtagged}{qq1}{\qbartag{1}}
\extq{f}{z} \circ \pp{z}{y} = \pp{f \dsub z}{x} \circ f
\end{axiomtagged}
and such that the diagram
\begin{displaymath}
\begin{array}{cp{.9cm}c}
\Rnode{fstarz}{f\dsub z} & & \Rnode{z}{z}\\ [1.2cm]
\Rnode{x}{x}         & & \Rnode{y}{y}
\end{array}
\begin{arrows}
\nccdar{fstarz}{x}
\blabel{\pp{f \dsub z}{x}}
\ncarr{x}{y}
\blabel{f}
\ncarr{fstarz}{z}
\alabel{\extq{f}{z}}
\nccdar{z}{y}
\alabel{\pp{z}{x}}
\end{arrows}
\end{displaymath}
 whose commutivity is given \byaxiom{qq1} is a pullback diagram in \catc,
\item such that  $x$ and $y$ are objects of \catcw such that $x \leq y$ then
\begin{axiomtagged}{qq2}{\qbartag{2}}
{id_x} \dsub y = y
\end{axiomtagged}
and
\begin{axiomtagged}{qq3}{\qbartag{3}}
\extq{id_x}{y} = id_y
\end{axiomtagged}
and 
\item
such that for all objects
$w$, $x$, $y$ and $z$ of \catcw such that $y \leq z$, if $f: w \morph x$ and $g:x \morph w$
in \catc, then
\begin{axiomtagged}{qq4}{\qbartag{4}}
(f \circ g)\dsub z =  f \dsub (g \dsub z)
\end{axiomtagged}
and 
\begin{axiomtagged}{qq5}{\qbartag{5}}
\bar{q}(f \circ g,z) = \bar{q}(f,g \dsub z) \circ \bar{q}(g,z)
\end{axiomtagged}
as shown in this diagram
\begin{displaymath}
\begin{array}{c p{1.3cm} c p{2.0cm} c}
                                                   &&                        &&           \\ [0.5cm] % need space for top most arc and label
\Rnode{UTL}{(f \circ g) \dsub z} \ \ \ \           &&                        &&           \\ [0.05cm]
\Rnode{TL}{=f \dsub (g \dsub z)}  && \Rnode{TC}{g \dsub z}  &   & \Rnode{TR}{z}           \\ [1.2cm]
\Rnode{BL}{w} &&\Rnode{BC}{x} &   & \Rnode{BR}{y}                                         \\ [0.5cm]
\end{array}
\begin{arrows}
% composition
\nccdar{TL}{BL}
%
\nccdar{TC}{BC}
%
\nccdar{TR}{BR}
\alabel{\pp{z}{y}}
%
% reference
\ncarr{BL}{BC}
\blabel{f}
%
\ncarr{BC}{BR}
\blabel{g}
%
\ncarr{TL}{TC}
\alabel{\extq{f}{g \dsub z}}
%
\ncarr{TC}{TR}
\alabel{\extq{g}{z}}
%
\ncarr[30]{UTL}{TR}
\alabel{\bar{q}(f \circ g,z)}
\end{arrows}
\end{displaymath}

\item such that if $w$, $x$, $y$ and $z$ are objects of \catc, if $f:w \morph x$ in \catcw and $x \leq y \leq z$ then
\begin{axiomtagged}{qq6}{\qbartag{6}}
f\dsub z = \extq{f}{y} \dsub z
\end{axiomtagged}
 and 
\begin{axiomtagged}{qq7}{\qbartag{7}}
\extq{f}{z} = \extq{\extq{f}{y}}{z}
\end{axiomtagged}
as shown in this diagram
\begin{displaymath}
\begin{array}{ccp{4.0cm}c}
%&                             &&              \\ [0.1cm]
&f\dsub z = \ \ \ \ \ \ \     &&              \\
&\Rnode{TL}{\extq{f}{y}\dsub z} && \Rnode{TR}{z}\\ [1.2cm]
&\Rnode{ML}{f \dsub y}        && \Rnode{MR}{y}\\ [1.2cm]
&\Rnode{BL}{w}                && \Rnode{BR}{x} 
\end{array}
\begin{arrows} 
%composition  
\nccdar{TL}{ML}
%
\nccdar{ML}{BL}
\nccdar{TR}{MR}
%
\nccdar{MR}{BR}
%reference
\ncarr{TL}{TR}
\alabel{\extq{f}{z}=\extq{\extq{f}{y}}{z}}
%
\ncarr{ML}{MR}
\alabel{\extq{f}{y}}
%
\ncarr{BL}{BR}
\blabel{f}
\end{arrows}
\end{displaymath}
\end{itemize}
\end{definition}

\begin{lemma}
If $f:x \morph y$ in a contextual category \catcw then $\extq{f}{y}=f$.
\end{lemma}
\begin{proof}
Use (\ref{qq0}) and (\ref{qq1}).
\end{proof}.


\subsection{Extending $^*$ to morphisms}
Now we extend the $^*$ notation further so that if $x,y,z_1$ and $z_2$ are objects of a contextual catgeory \catcw 
and if $y \leq z_1$ and $y \leq z_2$, and if  $f:x\morph y$ and if $g: z_1 \morph z_2$ such that $g \circ p_{z_2,y} = p_{z_1,y}$
then $f^*g:f^*z_1 \morph f^*z_2$.

$f^*g$ is defined
as the unique morphism $h:f^*z_1 \morph f^*z_2$ such that
$h \circ p_{f^*z_2,x} = p_{f^*z_1,x}$ and $h \circ q(f,z_2) = q(f,z_1)\circ g$
so that we have the following diagram in \catc
$$
\begin{array}{cccp{.9cm}ccc}
                   &               &\Rnode{TL}{f^*z_2} & &                &                & \ \ \Rnode{TR}{z_2}\ \\ [0.5cm]
\Rnode{ML}{f^*z_1} &               &                   & &\ \ \Rnode{MR}{z_1}\ \ &                &               \\ [1.2cm]
                   & \Rnode{BL}{x} &                   & &                & \Rnode{BR}{y}  &
\end{array}
$$
\makebox[0.2cm]{   % This make box prevents white space pushing out to the right
                   % cannot see where this white space is comin from. To investigate
                                                       % change the \makebox[0.2cm] to \fbox and you will see the problem.
\nccdar{TL}{BL}
\nccdar{ML}{BL}
\nccdar{TR}{BR}
\nccdar{MR}{BR}
\ncarr{TL}{TR}\alabel{q(f,z_2)}
\ncarr{ML}{MR}\alabel{q(f,z_1)}[0.6]
\ncarr{BL}{BR}\blabel{f}
\ncarr{ML}{TL}\alabel{f^*g}
\ncarr{MR}{TR}\alabel{g}[0.3]
}

\newcommand{\lhsdom}{(f^*g)^*(f^*z_1)}
\newcommand{\lhscod}{(f^*g)^*(f^*z_2)}
\newcommand{\lhs}{(f^*g)^*(f^*h)}
\newcommand{\fghdom}{f^*(g^*z_1)}
\newcommand{\fghcod}{f^*(g^*z_2)}
\newcommand{\fgh}{f^*(g^*h)}
\newcommand{\fhdom}{f^*z_1}
\newcommand{\fhcod}{f^*z_2}
\newcommand{\fh}{f^*h}
\newcommand{\ghdom}{g^*z_1}
\newcommand{\ghcod}{g^*z_2}
\newcommand{\gh}{g^*h}
\newcommand{\fgdom}{f^*y_1}
\newcommand{\fgcod}{f^*y_2}
\newcommand{\fg}{f^*g}
\newcommand{\fdom}{x_1}
\newcommand{\fcod}{x_2}
\newcommand{\f}{f}
\newcommand{\gdom}{y_1}
\newcommand{\gcod}{y_2}
\newcommand{\g}{g}
\newcommand{\hdom}{z_1}
\newcommand{\hcod}{z_2}
\newcommand{\h}{h}

\


\subsection {Voevodsky's axiomatisation}
Following Voevodsky \cite{Voevodsky14C} we may replace the pullback condition (condition (ii) of definition \lref{contextualcategory}) by introducing and axiomatising an 
`s' operator  as follows:

\noindent (ii') for all morphisms $f: x \rightarrow y$, a morphism $s(f) : x \rightarrow f \sub p_y \sub y$ such that both:

\begin{axiom}{s1}
s(f) \circ p_{f\sub p_y \sub y}=id_x
\end{axiom}

\noindent and

\begin{axiom}{s2}
s(f) \circ q( f \circ p_y     ,y)=f
\end{axiom} 

\noindent i.e. such that the following diagrams commute:
\begin{center}
\begin{displaymath}
\begin{array}{cccp{1.cm} cp{.9cm}c}
&\Rnode{fXyyM}{f\sub p_y \sub y}&  & &  \Rnode{fXyy}{f\sub p_y \sub y} & & \Rnode{yXy}{p_y \sub y}\\ [1.2cm]
\Rnode{xL}{x} & &\Rnode{xR}{x} & &\Rnode{x}{x}         & & \Rnode{y}{y}
\end{array}
\end{displaymath}
\ncasar{p_{f\sub p_y \sub y}}{fXyy}{x}
\jcbarr{f}{x}{y}
\ncaarr{q(f,p_y \sub y)}{fXyy}{yXy}
\ncasar{p_{p_y \sub y}}{yXy}{y}
\ncaarr{s(f)}{xL}{fXyyM}
\ncasar{p_{f\sub p_y \sub y}}{fXyyM}{xR}
\jcbarr{id_x}{xL}{xR}
\end{center}

\noindent
and such that whenever
\begin{center}
\begin{displaymath}
\begin{array}{c p{.9cm} c p{.9cm} c}
\Rnode{w}{w}&& \Rnode{g*z}{g \sub z} && \Rnode{z}{z} \\ [1.2cm]
            && \Rnode{x}{x}  && \Rnode{y}{y} \\ [0.2cm]
\end{array}
\end{displaymath}
\jcbarr{f}{w}{g*z}
\jcbarr{g}{x}{y}
\ncaarr{q(g,z)}{g*z}{z}
\ncasar{}{g*z}{x}
\ncasar{}{z}{y}
\end{center}

\noindent in \cat{C} then

\begin{axiom}{s3}
s(f \circ q(g,z))=s(f)
\end{axiom}


% end of the 's' subsection

\subsection{The $\crossx{}{}{}$ notation}

We shall write $\crossx{y}{z}{x}$ in place of ${p_{y,x}}^*z$, for $x \leq y$, $x \leq z$  in \catc. 
Note that
$\crossx{y}{z}{x}$  represents what in the generalised algebraic syntax  is the `weakening' of a rule of the form
\begin{displaymath}
x,\, w_1,...w_n \tstyle \isT{z}
\end{displaymath}
from a rule with context $x, w_1,...w_n$ to a rule with broader context $y, w_1, ... w_n$: 
\begin{displaymath} 
y,\, w_1,...w_n \tstyle \isT{z}.
\end{displaymath}

\noindent Within the contextual category we can think of $\crossx{y}{z}{x}$  as a local cartesian product but of course categorically it is a filtered product i.e. a pullback --- if $w < x$ and $w < y$  then 
\genericcrossxproductdiagram % defined in 'paper.tex'
is a pullback diagram in \ccat.


We can extend the $\crossx{}{}{w}$ notation to morphisms. If $f:x \morph x'$ and $g: y \morph y'$ in a contextual
category $\ccat[C]$ and if $w$ is an object such that $w < x$, $w <x'$, $w < y$ and $w < y'$ then 
define $\crossx{f}{g}{w}:\crossx{x}{y}{w} \morph \crossx{x'}{y'}{w}$ in $\ccat[C]$ by
\begin{equation}
\crossx{f}{g}{w} = \tuple{p_{\crossx{x}{y}{w}, x} \circ f,q(p_{x,w},y) \circ g}
\end{equation}
as shown here:  
\begin{equation*}
\begin{array}{c p{1.2cm} c  p{0.2cm} c p{0.2cm} c p{1.8cm} c}
\Rnode{xy}{\crossx{x}{y}{w}} && \Rnode{xpyp}{\crossx{x'}{y'}{w}} \\[1.4cm]
\Rnode{x}{x}                 && \Rnode{xp}{x'}                   &&&& \Rnode{y}{y} && \Rnode{yp}{y'} \\[1.8cm]
                             &&                         && \Rnode{w}{w}  &&    &&      
\makebox[0cm]{
\nccdar{xy}{x}
\nccdar{xpyp}{xp}
\nccdar{x}{w}
\nccdar{xp}{w}
\nccdar{y}{w}
\nccdar{yp}{w}
\ncarr{x}{xp}
\alabel{f}
\ncarr{y}{yp}
\alabel{g}[0.35]
\ncarr{xy}{y}
\alabel{q(p_{x,w},y)}[0.7][0.2]
\ncarr{xpyp}{yp}
\alabel{q(p_{x',w},y')}[0.5][0.2]
\ncdarr[10]{xy}{xpyp}
%\alabel{\tuple{p_{\crossx{x}{y}{w}, x} \circ f,q(p_{x,w},y) \circ g}}[0.5][2]
\alabel{\crossx{f}{g}{w}}%[-0.1][15]
}
\end{array}
\end{equation*}

In the special case that $x=x'$ and $f=id_x$ then we get  $\crossx{id_x}{g}{w} : \crossx{x}{y}{w} \morph \crossx{x}{y'}{w}$. 
This morphism we abbreviate as $\crossx{x}{g}{w}$. 
