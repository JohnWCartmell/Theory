




% Sublime Text: Ctrl-B -- to build
%               Shift-Ctrl-D -- duplicate line
 

\documentclass[10pt,a4paper]{article}
%\documentclass[10pt,a3paper]{article}
%

%ccategories.macros.tex 

% Macros for diagrams in contextual categories and related categories

\usepackage{twoopt}
\usepackage{scalerel} 
\usepackage{xargs}

%\usepackage{mathabx}  %Caused font problems
%\usepackage{MnSymbol}  % caused font problems

\newcommand{\conu}
{\mathbf{C}(U)}

\newcommand{\depu}
{\mathbf{D}(U)}

\newcommand{\cat}[1]{\textbf{#1}}
\newcommand{\obj}[1]{\ensuremath{|\cat{#1}|}}
\newcommand{\ccat}[1][C]{\ensuremath{\mathbb{#1}} }
\newcommand{\ccatc}{contextual category \ccat}
\newcommand{\cobj}[2][]{\ensuremath{|\ccat[#2]|_{#1}}}
\newcommand{\cslice}[2]{\ensuremath{\ccat[#1]_{#2}}}
\newcommand{\csliceobj}[3][]{\ensuremath{|\mathbb{#2}_{#3}|_{#1} }}
\newcommand{\varset}[1][]{\ensuremath{V_{#1} }}
\newcommand{\localvarsets}{\ensuremath{\mathcal{V} }}
\newcommand{\Fam}{\ensuremath{\mathbb{F\mathrm{am}} }}
\newcommand{\Famslice}[1]{\ensuremath{\mathbb{F\mathrm{am}}_{#1} }}
\newcommand{\Famobj}[1][]{\ensuremath{|\mathbb{F\mathrm{am}}|_{#1} }}
\newcommand{\Famsliceobj}[2][]{\ensuremath{|\mathbb{F\mathrm{am}}_{#2}|_{#1} }}
\newcommand{\morph}{\rightarrow}
\newcommand{\epi}{\twoheadrightarrow}
\newcommand{\base}{\triangleleft}
\newcommand{\comp}{\circ}
\newcommand{\cross}{\otimes}
\newcommand{\pc}[2]{d^{#1}_{#2}}
\newcommand{\sub}{^*}
\newcommand{\diag}{\delta}
\newcommand{\pbase}[1]{\tilde{#1}}

\newcommand{\tuple}[1]{\langle#1\rangle}
\newcommand{\ndidly}{\ensuremath{\Join_n}}
\newcommand{\ndidlycospan}{quotiented n-cospan}

\newcommand{\crossx}[3]{#1 \underset{#3}{\cross} #2}
\newcommand{\fibrex}[3]{#1 \underset{#3}{\Join} #2}
\newcommand{\powerset}{\mathcal{P}}
\newcommand{\primeds}[1]{
\ensuremath{\mathcal{P}(#1)} }
\newcommand{\compset}{\ \dot{\circ}\, }

% darrow
%\newcommand{\darrow}{\rightarrowtriangle} %use \smorph instead
\newcommand{\smorph}{\rightarrowtriangle}

 

\newcommand\dhead{\scaleobj{0.6}{\triangleright}}
\newcommand{\dmorph}{\, \mbox{---} \! \cdot \! \raisebox{1.1pt}{\dhead}}

% projection tree
%\newcommand{\proj}[2]{proj_{#2}(#1)}

\newcommand{\proj}[2]{
\ensuremath{\mathcal{P}_{#2}(#1)} }

%pstrick supplements for arrows

\newlength{\arrnodesepA}
\newlength{\arrnodesepB}
\newlength{\arroffsetA}
\newlength{\arroffsetB}

%Modified to 2pt from 0pt on 23 July 2018
\newcommand{\arreset}{
\setlength{\arrnodesepA}{2pt}
\setlength{\arrnodesepB}{2pt}
\setlength{\arroffsetA}{0pt}
\setlength{\arroffsetB}{0pt}
}
\arreset

\newcommand{\ncarr}[3][0]{\ncarc[arcangle=#1,nodesepA=\arrnodesepA,nodesepB=\arrnodesepB,offsetA=\arroffsetA,offsetB=\arroffsetB,arrowsize=5pt,arrowinset=0.7]{->}{#2}{#3}}
\newcommand{\jcbarr}[4][0]{ % ncbarr is defined in some thridy party package so do not use!\emph{}
\ncarr[#1]{#3}{#4}
\nbput[labelsep=2pt]{\footnotesize $#2$}
}

\newcommand{\ncaarr}[4][0]{
\ncarr[#1]{#3}{#4}
\naput[labelsep=2pt]{\footnotesize $#2$}
}

% \alabel{label}[npos][labelsep_pts]
\newcommandx*\alabel[3][2=0.5,3=2,usedefault]{\naput[labelsep=#3pt,npos=#2]{\footnotesize $#1$}}
% \blabel{label}[npos][labelsep_pts]
\newcommandx*\blabel[3][2=0.5,3=2,usedefault]{\nbput[labelsep=#3pt,npos=#2]{\footnotesize $#1$}}

% \idcomp mark an arrow as one component of an identifier
\newcommand{\idcomp}{\ncput[npos=0, nrot=:U]{\psline(0.2,-0.075)(0.2,0.075)}}  %add a bar to a node connection arrow
% pstrick supplements for s-arrows (previous name for d-arrow - should convert}

\newlength{\sarnodesepA}
\newlength{\sarnodesepB}
\newlength{\saroffsetA}
\newlength{\saroffsetB}
\newlength{\sarnodesepAsav}
\newlength{\sarnodesepBsav}

\newcommand{\sarreset}{
\setlength{\sarnodesepA}{0pt}
\setlength{\sarnodesepB}{0pt}
\setlength{\saroffsetA}{0pt}
\setlength{\saroffsetB}{0pt}
}

\sarreset

% sar - S-arrow
\newcommand{\ncsar}[3][0]{
\setlength{\sarnodesepAsav}{\sarnodesepA}
\setlength{\sarnodesepBsav}{\sarnodesepB}
\addtolength{\sarnodesepA}{3pt}
\addtolength{\sarnodesepB}{7pt}
\ncarc[nodesepA=\sarnodesepA,nodesepB=\sarnodesepB,offsetA=\saroffsetA,offsetB=\saroffsetB,arcangle=#1]{-}{#2}{#3}
\ncput[nrot=:R,npos=1]{\pstriangle(0,0)(.2,.2)}
\setlength{\sarnodesepA}{\sarnodesepAsav}
\setlength{\sarnodesepB}{\sarnodesepBsav}
}


% bsar - below labelled S-arrow
\newcommand{\ncbsar}[4][0]{
\ncsar[#1]{#3}{#4}
\nbput[labelsep=2pt]{\footnotesize $#2$}
}
% asar - above labelled S-arrow
\newcommand{\ncasar}[4][0]{
\ncsar[#1]{#3}{#4}
\naput[labelsep=2pt]{\footnotesize $#2$}
}

% cdar - composite dependency arrow
\newcommand{\nccdar}[3][0]{
\setlength{\sarnodesepAsav}{\sarnodesepA}
\setlength{\sarnodesepBsav}{\sarnodesepB}
\addtolength{\sarnodesepA}{3pt}
\addtolength{\sarnodesepB}{11pt}
\ncarc[nodesepA=\sarnodesepA,nodesepB=\sarnodesepB,offsetA=\saroffsetA,offsetB=\saroffsetB,arcangle=#1]{-}{#2}{#3}
\ncput[nrot=:R,npos=1]{\pstriangle(0,0.1)(.2,.2)}
\ncput[nrot=:R,npos=1]{\psdot[dotsize=1pt](-0.0075,0.05)}   %!!
\setlength{\sarnodesepA}{\sarnodesepAsav}
\setlength{\sarnodesepB}{\sarnodesepBsav}
}


% bcdar - below labelled composite dependency arrow
\newcommand{\ncbcdar}[4][0]{
\nccdar[#1]{#3}{#4}
\nbput[labelsep=2pt]{\footnotesize $#2$}
}
% acdar - above labelled composite dependency arrow
\newcommand{\ncacdar}[4][0]{
\nccdar[#1]{#3}{#4}
\naput[labelsep=2pt]{\footnotesize $#2$}
}


% rsar - recursive S-arrow
\newcommand{\ncrsar}[2]{
\setlength{\sarnodesepAsav}{\sarnodesepA}
\setlength{\sarnodesepBsav}{\sarnodesepB}
\addtolength{\sarnodesepA}{3pt}
\addtolength{\sarnodesepB}{7pt}
\ncloop[nodesepA=\sarnodesepA,nodesepB=\sarnodesepB,
        offsetA=\saroffsetA,offsetB=\saroffsetB,
        armA=0.7cm,armB=0.6cm,angleA=90,angleB=-90,loopsize=-1,linearc=0.4
				]{-}{#1}{#2}
\ncput[nrot=:R,npos=5]{\pstriangle(0,0)(.2,.2)}
\setlength{\sarnodesepA}{\sarnodesepAsav}
\setlength{\sarnodesepB}{\sarnodesepBsav}
}

% pstrick supplements for multi-arrows

\newlength{\marnodesepA}
\newlength{\marnodesepB}
\newlength{\maroffsetB}
\newlength{\marnodesepBsav}

\newcommand{\marreset}{
\setlength{\marnodesepA}{0pt}
\setlength{\marnodesepB}{0pt}
\setlength{\maroffsetB}{0pt}
}

\marreset

%ncmarr[#1 arcangle1][#2 arcangle2]{#3 name}{#4 domain1}{#5 domain2}{#6 junction}{#7 codomain}
\newcommandtwoopt{\ncmarr}[6][8][8]{%
\ncarc[nodesepA=\marnodesepA,nodesepB=0,arcangle=#1]{-}{#3}{#5}
\ncarc[nodesepB=0,arcangle=-#1]{-}{#4}{#5}
\ncarc[arcangle=#2,nodesepB=\marnodesepB,offsetB=\maroffsetB]{->}{#5}{#6}
}%


\newcommandtwoopt{\nchmarr}[6][8][8]{%
\ncarc[nodesepA=\marnodesepA,nodesepB=0,arcangle=#1]{-}{#3}{#5}
\ncarc[nodesepB=0,arcangle=#1]{-}{#4}{#5}
\ncarc[arcangle=#2,nodesepB=\marnodesepB,offsetB=\maroffsetB]{->}{#5}{#6}
}%

\newcommandtwoopt{\ncamarr}[7][8][8]{%
\ncmarr[#1][#2]{#4}{#5}{#6}{#7}
\naput[npos=.05]{$#3$}
}%
\newcommandtwoopt{\ncbmarr}[7][8][8]{%
\ncmarr[#1][#2]{#4}{#5}{#6}{#7}
\nbput[npos=.05]{$#3$}
}%

\newcommandtwoopt{\ncbhmarr}[7][8][8]{%
\nchmarr[#1][#2]{#4}{#5}{#6}{#7}
\nbput[npos=.05]{$#3$}
}%

\newcommandtwoopt{\ncmarrr}[7][8][8]{
\ncarc[nodesepB=0,arcangle=#1]{-}{#3}{#6}
\ncline[nodesepB=0]{-}{#4}{#6}
\ncarc[nodesepB=0,arcangle=-#1]{-}{#5}{#6}
\ncarc[nodesepA=0,arcangle=#2]{->}{#6}{#7}
}

\newcommandtwoopt{\ncamarrr}[8][8][8]{
\ncmarrr[#1][#2]{#4}{#5}{#6}{#7}{#8}
\naput[npos=.05]{$#3$}
}
\newcommandtwoopt{\ncbmarrr}[8][8][8]{
\ncmarrr[#1][#2]{#4}{#5}{#6}{#7}{#8}
\nbput[npos=.05]{$#3$}
}

\usepackage[margin=4.0cm]{geometry} %was 3cm
\usepackage{mathptmx}
\usepackage{amsfonts}
\usepackage{array}
\usepackage{pstricks}
\usepackage{pst-tree}
\usepackage{pst-plot}
\usepackage{pst-node}
\usepackage{stmaryrd}
\usepackage{amsmath}
\usepackage{verbatim}
\usepackage{graphicx}  
\usepackage{calc}
\usepackage{xifthen}
\usepackage{xcolor}
\usepackage{color}
\usepackage{stringstrings}
%\usepackage[small,bf,margin=3pt,format=hang, labelsep=endash,singlelinecheck=false]{caption} %prevuiously justification=justified
%\usepackage{enumerate}
%\usepackage{enumitem}
\usepackage{enumerate}
\usepackage[shortlabels]{enumitem}
\usepackage{float}
\usepackage[section]{placeins}
%\setlength{\captionmargin}{5pt}
\usepackage{environ}
\usepackage{multirow}
\usepackage{rotating}
\usepackage{longtable}
\usepackage{afterpage}
\usepackage{needspace}


%DEFINE ENVIRONMENT BLOCK
% Riddle
\newsavebox{\riddlebox}

\newenvironment{erexample}
{\newcommand\colboxcolor{F0F0F0}%was F8F8F8
\begin{lrbox}{\riddlebox}
\begin{minipage}{\dimexpr\columnwidth-2\fboxsep\relax} \textbf{} \\ \itshape}
{\end{minipage}\end{lrbox}%
%\begin{center}
\colorbox[HTML]{\colboxcolor}{\usebox{\riddlebox}}
%\end{center}
}

\newenvironment{erbox}
{\newcommand\colboxcolor{F0F0F0}%was F8F8F8
\begin{lrbox}{\riddlebox}%
\begin{minipage}{\dimexpr\columnwidth-2\fboxsep\relax} }
{\end{minipage}\end{lrbox}%
%\begin{center}
\colorbox[HTML]{\colboxcolor}{\usebox{\riddlebox}}
%\end{center}
}

%\begin{erboxedFigure}{#1 FigureParam}{#2 Label}{#3 Caption}
\NewEnviron{erboxedFigure}[3]{%
\begin{figure}[#1]
\begin{erexample}
\begin{center}
\BODY
\end{center}
\vspace{-0.5cm}
\caption{#3}
\label{#2}
\end{erexample}
\end{figure}
}

\newcommand{\erpictureFolder}[0]{../SharedPictures}

\newcommand{\ercenterPicture}[1]{
\begin{center}
\input{\erpictureFolder/#1}
\end{center}
}


\newlength{\erhalfHt}

%\erinlinePicture{#1 pictureFilename}{#2 pictureHeight}
\newcommand{\erinlinePicture}[2]{
\setlength{\erhalfHt}{#2cm * \real{0.5}}
\raisebox{-\erhalfHt}[\erhalfHt + 0.5cm][\erhalfHt + 0.5cm]{
\input{\erpictureFolder/#1}
} 
}

%\erplainFig{#1 pictureFilename}{#2 figureParam}{#3Caption}
\newcommand{\erplainFig}[3]{
\begin{figure}[#2]
\begin{center}
\input{\erpictureFolder/#1}
\end{center}
\caption{#3}
\label{#1}
\end{figure}
}

%\erboxedFigPicture{#1 pictureFilename}{#2 figureParam}{#3Caption}
\newcommand{\erboxedFigPicture}[3]{
\begin{figure}[#2]
\begin{erexample}
\vspace{-0.5cm}
\begin{center}
\input{\erpictureFolder/#1}
\end{center}
\caption{#3}
\label{#1}
\end{erexample}
\end{figure}
}

%\erLeftSideFig{#1 pictureFilename}{#2 figureParam}{#3Caption}
\newcommand{\erLeftSideFig}[3]{
\begin{figure}[#2]
\begin{erexample}
  \begin{minipage}[c]{0.4\textwidth}
    \caption{#3}
    \label{#1}
  \end{minipage}
  \begin{minipage}[c]{0.5\textwidth}
    \input{\erpictureFolder/#1}
  \end{minipage}
\end{erexample}
\end{figure}
}

%\erbulletedFig{#1 pictureFilename}{#2 figureParam}{#3Caption}
\NewEnviron{erbulletedFig}[3]{%
\begin{figure}[#2]
\begin{erexample}
\vspace{-0.5cm}
\begin{center}
$
\begin{array}{c m{0.25cm} | m{6cm}}
\raisebox{-2.0cm}{
\input{\erpictureFolder/#1}}& & \text{\parbox{6cm}{\raggedright{\footnotesize{
\begin{enumerate}[(i)]
\BODY
\end{enumerate}}}}} \\
\end{array}
$
\end{center}
\caption{#3}
\label{#1}
\end{erexample}
\end{figure} 
}


%\begin{erbulletedDimFig}{#1 pictureFilename}{#2figureParam} {#3Caption} {#4PictureHeight}{#5TextWidth}

\NewEnviron{erbulletedDimFig}[5]{%
\begin{figure}[#2]
\begin{erexample}
\vspace{-0.5cm}
\begin{center}
$
\begin{array}{c m{0.25cm} |  m{#5cm}}
\setlength{\erhalfHt}{#4cm * \real{0.5}}
\raisebox{-\erhalfHt}{
\input{\erpictureFolder/#1}}& & \text{\parbox{#5cm}{\raggedright{\footnotesize{
\begin{enumerate}[(i)]
\BODY
\end{enumerate}}}}} \\
\end{array}
$
\end{center}
\caption{#3}
\label{#1}
\end{erexample}
\end{figure} 
}

%\begin{ernotedModel}{#1 pictureFilename}{#2PictureHeight}{#3PictureWidth}{#4TextWidth}

\NewEnviron{ernotedModel}[4]{%
\begin{center}
$
\begin{array}{m{#3cm} m{1cm} | c m{#4cm}}
\setlength{\erhalfHt}{#2cm * \real{0.5}}
\raisebox{-\erhalfHt}{
\input{\erpictureFolder/#1}}& & & \text{\parbox{#4cm}{\raggedright{\footnotesize{
\BODY
}}}} \\
\end{array}
$
\end{center} 
}

%\begin{ermodelText}{#1 pictureFilename}{#2PictureHeight}{#3PictureWidth}{#4TextWidth}

\NewEnviron{ermodelText}[4]{%
\begin{center}
\begin{tabular}{m{#3cm} m{1cm}  c m{#4cm}}
\setlength{\erhalfHt}{#2cm * \real{0.5}}
\raisebox{-\erhalfHt}{
\input{\erpictureFolder/#1}}& & & \text{\parbox{#4cm}{\raggedright{\small{
\BODY
}}}} \\
\end{tabular}
\end{center} 
}


%\erbulletedModel{#1 pictureFilename}{#2PictureHeight}{#3PictureWidth}{#4TextWidth}

\NewEnviron{erbulletedModel}[4]{%
\begin{center}
$
\begin{array}{m{#3cm} m{1cm} | c m{#4cm}}
\setlength{\erhalfHt}{2cm * \real{0.5}}
\raisebox{-\erhalfHt}{
\input{\erpictureFolder/#1}}& & & \text{\parbox{#4cm}{\raggedright{\footnotesize{
\begin{enumerate}[(i)]
\BODY
\end{enumerate}}}}} \\
\end{array}
$
\end{center} 
}



%\ernotedDimFig{#1 pictureFilename}{#2 figureParam}{#3Caption}{#4PictureHeight}{#5TextWidth}
\NewEnviron{ernotedDimFig}[5]{%
\begin{figure}[#2]
\begin{erexample}
\vspace{-0.5cm}
\begin{center}
$
\begin{array}{c m{0.25cm} | c m{#5cm}}
\setlength{\erhalfHt}{#4cm * \real{0.5}}
\raisebox{-\erhalfHt}{
\input{\erpictureFolder/#1}}& & & \text{\parbox{#5cm}{\raggedright{\footnotesize{
\BODY }}}}\\
\end{array}
$
\end{center}
\caption{#3}
\label{#1}
\end{erexample}
\end{figure} 
}
%\begin{ernotedDimFigPW}{#1 pictureFilename}{#2 figureParam}{#3Caption}{#4PictureHeight}{#5PictureWidth}{#6TextWidth}
\NewEnviron{ernotedDimFigPW}[6]{%
\begin{figure}[#2]
\begin{erexample}
\vspace{-0.5cm}
\begin{center}
$
\begin{array}{>{\centering}m{#5cm} m{0.5cm} | c m{#6cm}}
\setlength{\erhalfHt}{#4cm * \real{0.5}}
\raisebox{-\erhalfHt}{
\centering \input{\erpictureFolder/#1}
}& & & \text{\parbox{#6cm - 0.5cm}{\raggedright{\footnotesize{
\BODY }}}}\\
\end{array}
$ \\
\vspace {0.2cm}
\end{center}
\caption{#3}
\label{#1}
\end{erexample}
\end{figure}
}



\newenvironment{erquote}
{\begin{quote}\itshape}
{\end{quote}}


%
%  erdiag
%
  
%\begin{erdiagram}{#1 height}{#2 width} 
% ....
% ....
%\end{erdiagram}
\newenvironment{erdiagram}[2]
{%\pspicture*(-#1,0)(#2,0)
\pspicture*(0,-#1)(#2,0)
%\psgrid
}
{\endpspicture}

\definecolor{lightyellow}{cmyk}{0,0,0.3,0}

% \eret{#1 x0} {#2 y0} {#3 x1} {#4 y1} {#5 corner radius} {#6 fill}
\newcommand {\eret}[6]
{ 
\ifthenelse{\equal{#6}{1}}
{\psframe[framearc=#5,fillstyle=solid,fillcolor=lightyellow](#1,#2)(#3,#4)}
{\psframe[framearc=#5,fillstyle=solid,fillcolor=white](#1,#2)(#3,#4)}
}

% et top 
\newcommand {\erettop}[4]
{
%\psframe[linestyle=none,linearc=2pt,cornersize=absolute,fillstyle=solid,fillcolor=lightyellow](#1,#2)(#3,#4)
\psline[linearc=2pt,fillstyle=none,fillcolor=lightyellow](#1,#4)(#1,#2)(#3,#2)(#3,#4)
}

% et bottom 
\newcommand {\eretbtm}[4]
{
%\psframe[linestyle=none,linearc=2pt,cornersize=absolute,fillstyle=solid,fillcolor=lightyellow](#1,#2)(#3,#4)
\psline[linearc=2pt,fillstyle=none,fillcolor=lightyellow](#1,#2)(#1,#4)(#3,#4)(#3,#2)
}

% et bottom left
\newcommand {\eretbl}[4]
{
%\psframe[linestyle=none,linearc=2pt,cornersize=absolute,fillstyle=solid,fillcolor=lightyellow](#1,#2)(#3,#4)
\psline[linearc=2pt,fillstyle=none,fillcolor=lightyellow](#1,#4)(#3,#4)(#3,#2)
}

% et middle left
\newcommand {\eretml}[4]
{
%\psframe[linestyle=none,linearc=2pt,cornersize=absolute,fillstyle=solid,fillcolor=lightyellow](#1,#2)(#3,#4)
\psline[linearc=2pt,fillstyle=none,fillcolor=lightyellow](#1,#2)(#3,#2)(#3,#4)(#1,#4)
}

% et top left
\newcommand {\erettl}[4]
{
%\psframe[linestyle=none,linearc=2pt,cornersize=absolute,fillstyle=solid,fillcolor=lightyellow](#1,#2)(#3,#4)
\psline[linearc=2pt,fillstyle=none,fillcolor=lightyellow](#1,#2)(#3,#2)(#3,#4)
}

% et bottom right
\newcommand {\eretbr}[4]
{
%\psframe[linestyle=none,linearc=2pt,cornersize=absolute,fillstyle=solid,fillcolor=lightyellow](#1,#2)(#3,#4)
\psline[linearc=2pt,fillstyle=none,fillcolor=lightyellow](#1,#2)(#1,#4)(#3,#4)
}

% et middle right
\newcommand {\eretmr}[4]
{
%\psframe[linestyle=none,linearc=2pt,cornersize=absolute,fillstyle=solid,fillcolor=lightyellow](#1,#2)(#3,#4)
\psline[linearc=2pt,fillstyle=none,fillcolor=lightyellow](#3,#4)(#1,#4)(#1,#2)(#3,#2)
}

% et top right
\newcommand {\erettr}[4]
{
%\psframe[linestyle=none,linearc=2pt,cornersize=absolute,fillstyle=solid,fillcolor=lightyellow](#1,#2)(#3,#4)
\psline[linearc=2pt,fillstyle=none,fillcolor=lightyellow](#1,#4)(#1,#2)(#3,#2)
}

% \ergrp{#1 x0} {#2 y0} {#3 x1} {#4 y1} {#5 corner radius} {#6 fill}
% #5 corner radius is unused!
\newcommand {\ergrp}[6]
{ 
\ifthenelse{\equal{#6}{1}}
{\psframe[fillstyle=solid,fillcolor=lightgray](#1,#2)(#3,#4)}
{\psframe[fillstyle=solid,fillcolor=white](#1,#2)(#3,#4)}
}

% \eretname {#1 x left of text} {#2 y top of text} {#3 text}
\newcommand {\eretname}[3]
{
%shift down 0.1 for height of text the anchor at baseline (B)
\rput[bl]{0}(0,-0.1){\rput[Bl]{0}(#1,#2){\footnotesize \textit{#3}}}
}

% \errelarm {#1 x0} {#2 y0} {#3 x1} {#4 y1} {#5 ismandatory} {#6 isconstructed}
\newcommand {\errelarm}[6]
{
\ifthenelse{\equal{#6}{1}}
{
%%\psline[linewidth=0.5pt,linearc=.05,linestyle=dashed,dash=6pt 6pt]{-}(#1,#2)(#3,#4)}
\ifthenelse{\equal{#5}{1}}
{\psline[linewidth=1.5pt,linearc=.05,linecolor=lightgray]{-}(#1,#2)(#3,#4)}
{\psline[linewidth=1.5pt,linearc=.05,linecolor=lightgray,linestyle=dashed,dash=2pt 2pt]{-}(#1,#2)(#3,#4)}
}
{
\ifthenelse{\equal{#5}{1}}
{\psline[linewidth=0.9pt,linearc=.05]{-}(#1,#2)(#3,#4)}
{\psline[linewidth=0.9pt,linearc=.05,linestyle=dashed,dash=2pt 2pt]{-}(#1,#2)(#3,#4)}
}
}

% \errelangle {#1 x0} {#2 y0} {#3 x1} {#4 y1} {#5 x2} {#6 y2} {#7 ismandatory} {#8 isocnstructed}
\newcommand {\errelangle}[8]
{
\ifthenelse{\equal{#8}{1}}
{
%\psline[linewidth=0.5pt,linearc=.1,linestyle=dashed,dash=6pt 6pt]{-}(#1,#2)(#3,#4)(#5,#6)}
\ifthenelse{\equal{#7}{1}}
{\psline[linewidth=1.5pt,linearc=.05,linecolor=lightgray]{-}(#1,#2)(#3,#4)(#5,#6)}
{\psline[linewidth=1.5pt,linearc=.1,linecolor=lightgray,linestyle=dashed,dash=2pt 2pt]{-}(#1,#2)(#3,#4)(#5,#6)}
}
{
\ifthenelse{\equal{#7}{1}}
{\psline[linewidth=0.9pt,linearc=.1]{-}(#1,#2)(#3,#4)(#5,#6)}
{\psline[linewidth=0.9pt,linearc=.1,linestyle=dashed,dash=2pt 2pt]{-}(#1,#2)(#3,#4)(#5,#6)}
}
}

% \ercrowfoot {#1 x0} {#2 y0} {#3 x11} {#4 y11} {#5 x12} {#6 y12} {#7 x13} {#8 y13} {#9 isconstructed}
\newcommand {\ercrowfoot}[9]
{
\ifthenelse{\equal{#9}{1}}
{
\psline[linewidth=1.5pt,linearc=.05,linecolor=lightgray]{-}(#1,#2)(#3,#4)
\psline[linewidth=1.5pt,linearc=.05,linecolor=lightgray]{-}(#1,#2)(#5,#6)
\psline[linewidth=1.5pt,linearc=.05,linecolor=lightgray]{-}(#1,#2)(#7,#8)
}{
\psline[linewidth=0.9pt,linearc=.05]{-}(#1,#2)(#3,#4)
\psline[linewidth=0.9pt,linearc=.05]{-}(#1,#2)(#5,#6)
\psline[linewidth=0.9pt,linearc=.05]{-}(#1,#2)(#7,#8)
}
}


% \eridcomprel{#1 x1}{#2 x2}{#3 y1}{#4 ymid}{#5 y2}
\newcommand {\eridcomprel}[5]
{
\psline[linewidth=0.9pt](#1,#3)(#1,#5)
\psline[linewidth=0.9pt](#2,#3)(#2,#5)
\psline[linewidth=0.9pt](#1,#4)(#2,#4)
}

% \eridrefrel{#1 x1}{#2 xmid}{#3 x2}{#4 y1}{#5 y2}
\newcommand {\eridrefrel}[5]
{
\psline[linewidth=0.9pt](#1,#4)(#3,#4)
\psline[linewidth=0.9pt](#1,#5)(#3,#5)
\psline[linewidth=0.9pt](#2,#4)(#2,#5)
}


% \errelname {#1 x} {#2 y} {#3 text}
\newcommand {\errelname}[3]
{
\rput[l]{0}(#1,#2){\textit{#3}}
}
% \errelseq {#1 x} {#2 y}
\newcommand {\erelseq}[2]
{
}
% \erattr {#1 x} {#2 y} {#3 ismandatory}{#4 idenitfying} {#5 text}
\newcommand {\erattr}[5]
{
\ifthenelse{\equal{#3}{1}}
{\rput[l]{0}(#1,#2){{\tiny $\square$} {\footnotesize \textit{\ifthenelse{\equal{#4}{0}}{\underline{#5}}{#5}}}}}
{\rput[l]{0}(#1,#2){\footnotesize $\circ$ \textit{\ifthenelse{\equal{#4}{0}}{\underline{#5}}{#5}}}}
}

%\ifthenelse{\equal{#4}{1}}
% \ertext {#1 x} {#2 y} {#3 text anchor} {#4 text}
%{\rput[l]{0}(#1,#2){\footnotesize $\circ$ \underline{\textit{#5}}}}
\newcommand {\ertext}[4]
{
\rput[B#3]{0}(#1,#2){{\footnotesize #4}}
}
% \erarc {#1 x0} {#2 y0} {#3 x1} {#4 y1} {#5 x2} {#6 y2} {#7 x3} {#8 y3}
\newcommand {\erarc}[8]
{
\psbezier[showpoints=false]{-}(#1,#2) (#3, #4)(#5,#6) (#7, #8)
}

% \erarc {#1 x0} {#2 y0} {#3 x1} {#4 y1} {#5 x2} {#6 y2} {#7 x3} {#8 y3}
\newcommand {\errelseq}[8]
{
\psbezier[showpoints=false]{-}(#1,#2) (#3, #4)(#5,#6) (#7, #8)
}
% \ertrace {#1 trace}   
\newcommand {\ertrace}[1]
{
}

\usepackage{amsthm} % added 7th April 2018
% theorems.macros.tex

\newtheorem{theorem}{Theorem}[section]
\newtheorem{observation}[theorem]{Observation}
\newtheorem{lemma}[theorem]{Lemma}

\newtheorem{proposition}[theorem]{Proposition}
\newtheorem{corollary}[theorem]{Corollary}
\newtheorem{conjecture}[theorem]{Conjecture}
\newtheorem{numbereddefinition}[theorem]{Definition}

\newenvironment{definition}[1][Definition]{\begin{trivlist}
\item[\hskip \labelsep {\bfseries #1}]}{\end{trivlist}}
\newenvironment{examples}[1][Examples]{\begin{trivlist}
\item[\hskip \labelsep {\bfseries #1}]}{\end{trivlist}}
\newenvironment{example}[1][Example]{\begin{trivlist}
\item[\hskip \labelsep {\bfseries #1}]}{\end{trivlist}}
\newenvironment{remark}[1][Remark]{\begin{trivlist}
\item[\hskip \labelsep {\bfseries #1}]}{\end{trivlist}}

\newenvironment{tageqn}[1]
{
\begin{equation}
\stepcounter{equation}
\label{#1}
\tag{\theequation --#1}
}
{
\end{equation}
}

\newenvironment{axiom}[1]
{
\begin{equation}
\label{#1}
\tag{#1}
}
{
\end{equation}
}

% when the tag is required different from the label eg when has math symbols can use:
\newenvironment{axiomtagged}[2]
{
\begin{equation}
\label{#1}
\tag{#2}
}
{
\end{equation}
}

%visible label
\newcommand{\vlabel}[2][]{\label{#2}#1(\textit{#2}):}





\usepackage{mathptmx}  % This changes font to roman
\usepackage{anyfontsize}
\usepackage{mathtools}  % why have we got this?
\usepackage{alltt}    
\usepackage{mnsymbol} %used for rightpitchfork
\usepackage{cmll}
\usepackage{ulem}
\renewcommand{\ttdefault}{txtt}
\usepackage[left=1.5cm, right=4cm, marginparwidth=3cm, top=2cm, bottom=1.5cm]{geometry}
\usepackage{framed}
\usepackage[font=small]{caption}
\setlength{\captionmargin}{2cm}
\newcommand{\commentary}[1]{\marginpar{\footnotesize #1}}

\renewcommand{\erpictureFolder}[0]{../SharedPictures}

\newenvironment{categoricalaside}
{\begin{framed}
\textbf{Categorical Aside}
}
{
\end{framed}
}

\newenvironment{noteforfuture}
{\begin{framed}
\textbf{Note For Future}
}
{
\end{framed}
}

\newenvironment{problem}
{\begin{framed}
\textbf{Problem}
}
{
\end{framed}
}


%from berkley
\newcommand{\langl}{\begin{picture}(4.5,7)
\put(1.1,2.5){\rotatebox{60}{\line(1,0){5.5}}}
\put(1.1,2.5){\rotatebox{300}{\line(1,0){5.5}}}
\end{picture}}
\newcommand{\rangl}{\begin{picture}(4.5,7)
\put(.9,2.5){\rotatebox{120}{\line(1,0){5.5}}}
\put(.9,2.5){\rotatebox{240}{\line(1,0){5.5}}}
\end{picture}}
\newcommand{\lang}{\begin{picture}(5,7)\put(1.1,2.5){\rotatebox{45}{\line(1,0){6.0}}}\put(1.1,2.5){\rotatebox{315}{\line(1,0){6.0}}}\end{picture}}
\newcommand{\rang}{\begin{picture}(5,7)\put(.1,2.5){\rotatebox{135}{\line(1,0){6.0}}}\put(.1,2.5){\rotatebox{225}{\line(1,0){6.0}}}\end{picture}}
%Try sharper tuple brackets -- except gives errors nested in captions so comment out
%\renewcommand{\tuple}[1]{\lang #1 \rang}

\newcommand{\setsuchthat}[2]{\left\{#1 \ \middle|\ #2\right\}}
\newcommand{\set}[1]{\left\{#1\right\}} 

\newcommand{\genericmodel}{\mathcal{M}}  %PREVIOUSLY
\renewcommand{\genericmodel}{{m}}        %PREVIOUSLY
\renewcommand{\genericmodel}{\gamma}     % TRY THIS FOR A WHILE except texworks isnt happy with greek
%\renewcommand{\genericmodel}{M}  %while debugging
\newcommand{\chiZero}{\mathcal{X}_0}
\newcommand{\chiZeroM}{\chiZero(\genericmodel)}
\newcommand{\chiOne}{\mathcal{X}_1}
\newcommand{\chiOneM}{\chiOne(\genericmodel)}
\newcommand{\chiM}{\mathcal{X}(\genericmodel)}
\newcommand{\veee}{v}
\newcommand{\Veee}{V}
\newcommand{\et}[1][\genericmodel]{et_{#1}}
\newcommand{\edge}[3][\genericmodel]{Edge_{#1}(#2,#3)}
\newcommand{\iedge}[3][\genericmodel]{IEdge_{#1}(#2,#3)}
\newcommand{\path}[3][\genericmodel]{Path_{#1}(#2,#3)}
\newcommand{\ipath}[3][\genericmodel]{IPath_{#1}(#2,#3)}
\newcommand{\attr}[2] [\genericmodel]{attr_{#1}(#2)}
\newcommand{\iattr}[2] [\genericmodel]{IAttr_{#1}(#2)}
\newcommand{\rel}[3][\genericmodel]{rel_{#1}(#2,#3)}
\newcommand{\irel}[3][\genericmodel]{IRel_{#1}(#2,#3)}
\newcommand{\iedges}[2] [\genericmodel]{i_{#1}(#2)}
\newcommand{\pk}[2] [\genericmodel]{pk_{#1}(#2)}
\newcommand{\fk}[2] [\genericmodel]{fk_{#1}(#2)}
\newcommand{\fkp}[2] [\genericmodel]{fk'_{#1}(#2)}
\newcommand{\fkpp}[2] [\genericmodel]{fk''_{#1}(#2)}

%functional dependencies
\newcommand{\sfd}[2]{\ensuremath{\set{#1} \morph #2}}  %singleton
\newcommand{\fd}[2]{\ensuremath{\sfd{#1}{\set{#2}}}}

\newcommand{\simplepath}[2]{
\ncline[linestyle=none,linewidth=0.1pt]{#1}{#2}   %was linestyle=dotted
\ncput[npos=0.05]{\pnode{dot#21}}
\ncput[npos=0.27]{\dotnode[dotsize=1pt]{dot#22}}
\ncput[npos=0.50]{\dotnode[dotsize=1pt]{dot#23}}
\ncput[npos=0.80]{\dotnode[dotsize=1pt]{dot#24}}
\ncput[npos=0.975]{\pnode{dot#25}}
\ncline[nodesep=2pt]{->}{dot#21}{dot#22}
\ncline[nodesep=2pt]{->}{dot#22}{dot#23}
\ncline[nodesep=2pt]{->}{dot#24}{dot#25}
\ncline[linestyle=dotted,nodesep=8pt]{dot#23}{dot#24} %was 10pt
}

\newcommand{\simplepatha}[3]{
\simplepath{#2}{#3}
\naput[labelsep=1pt]{#1}
}

\newcommand{\simplepathb}[3]{
\simplepath{#2}{#3}
\nbput[labelsep=1pt]{#1}
}
\newcommand{\term}[1]{\textit{{#1}}}
\newcommand{\logtophys}{\mathcal{X}}
\newcommand{\chen}{\mathcal{X}_0}
\newcommand{\chengenericmodel}{\chen(\genericmodel)}
\newcommand{\chigenericmodel}{\logtophys(\genericmodel)}
\newcommand{\phys}[1]{\overline{#1}}
\newcommand{\genericphysical}{\logtophys(\genericmodel)}

\newcommand{\inc}{\subseteq}
\newcommand{\incd}[4]{#1\left[#2\right]\inc#3\left[#4\right]}

\newcommand{\ntuple}[1]{\tuple{#1_1,...#1_n}}
\newcommand{\mtuple}[1]{\tuple{#1_1,...#1_m}}

\newcommand {\bntuple}{\ensuremath{\ntuple{b}}}
\newcommand {\fntuple}{\ensuremath{\ntuple{f}}}
\newcommand {\pntuple}{\ensuremath{\ntuple{p}}}
\newcommand {\qntuple}{\ensuremath{\ntuple{q}}}
\newcommand {\qmtuple}{\ensuremath{\mtuple{q}}}
\newcommand {\xntuple}{\ensuremath{\ntuple{x}}}
\newcommand {\ymtuple}{\ensuremath{\mtuple{y}}}
\newcommand{\foreachi}[1][n]{for each $i$, $1 \leq i \leq #1$}
\newcommand{\foreachj}[1][m]{for each $j$, $1 \leq j \leq #1$}
\newcommand{\foreachk}[1][l]{for each $k$, $1 \leq k \leq #1$}
\newcommand{\fdfactoring}{fd factoring}
\newcommand{\attributelike}{attribute-like}



%ccategories.macros.tex 

% Macros for diagrams in contextual categories and related categories

\usepackage{twoopt}
\usepackage{scalerel} 
\usepackage{xargs}

%\usepackage{mathabx}  %Caused font problems
%\usepackage{MnSymbol}  % caused font problems

\newcommand{\conu}
{\mathbf{C}(U)}

\newcommand{\depu}
{\mathbf{D}(U)}

\newcommand{\cat}[1]{\textbf{#1}}
\newcommand{\obj}[1]{\ensuremath{|\cat{#1}|}}
\newcommand{\ccat}[1][C]{\ensuremath{\mathbb{#1}} }
\newcommand{\ccatc}{contextual category \ccat}
\newcommand{\cobj}[2][]{\ensuremath{|\ccat[#2]|_{#1}}}
\newcommand{\cslice}[2]{\ensuremath{\ccat[#1]_{#2}}}
\newcommand{\csliceobj}[3][]{\ensuremath{|\mathbb{#2}_{#3}|_{#1} }}
\newcommand{\varset}[1][]{\ensuremath{V_{#1} }}
\newcommand{\localvarsets}{\ensuremath{\mathcal{V} }}
\newcommand{\Fam}{\ensuremath{\mathbb{F\mathrm{am}} }}
\newcommand{\Famslice}[1]{\ensuremath{\mathbb{F\mathrm{am}}_{#1} }}
\newcommand{\Famobj}[1][]{\ensuremath{|\mathbb{F\mathrm{am}}|_{#1} }}
\newcommand{\Famsliceobj}[2][]{\ensuremath{|\mathbb{F\mathrm{am}}_{#2}|_{#1} }}
\newcommand{\morph}{\rightarrow}
\newcommand{\epi}{\twoheadrightarrow}
\newcommand{\base}{\triangleleft}
\newcommand{\comp}{\circ}
\newcommand{\cross}{\otimes}
\newcommand{\pc}[2]{d^{#1}_{#2}}
\newcommand{\sub}{^*}
\newcommand{\diag}{\delta}
\newcommand{\pbase}[1]{\tilde{#1}}

\newcommand{\tuple}[1]{\langle#1\rangle}
\newcommand{\ndidly}{\ensuremath{\Join_n}}
\newcommand{\ndidlycospan}{quotiented n-cospan}

\newcommand{\crossx}[3]{#1 \underset{#3}{\cross} #2}
\newcommand{\fibrex}[3]{#1 \underset{#3}{\Join} #2}
\newcommand{\powerset}{\mathcal{P}}
\newcommand{\primeds}[1]{
\ensuremath{\mathcal{P}(#1)} }
\newcommand{\compset}{\ \dot{\circ}\, }

% darrow
%\newcommand{\darrow}{\rightarrowtriangle} %use \smorph instead
\newcommand{\smorph}{\rightarrowtriangle}

 

\newcommand\dhead{\scaleobj{0.6}{\triangleright}}
\newcommand{\dmorph}{\, \mbox{---} \! \cdot \! \raisebox{1.1pt}{\dhead}}

% projection tree
%\newcommand{\proj}[2]{proj_{#2}(#1)}

\newcommand{\proj}[2]{
\ensuremath{\mathcal{P}_{#2}(#1)} }

%pstrick supplements for arrows

\newlength{\arrnodesepA}
\newlength{\arrnodesepB}
\newlength{\arroffsetA}
\newlength{\arroffsetB}

%Modified to 2pt from 0pt on 23 July 2018
\newcommand{\arreset}{
\setlength{\arrnodesepA}{2pt}
\setlength{\arrnodesepB}{2pt}
\setlength{\arroffsetA}{0pt}
\setlength{\arroffsetB}{0pt}
}
\arreset

\newcommand{\ncarr}[3][0]{\ncarc[arcangle=#1,nodesepA=\arrnodesepA,nodesepB=\arrnodesepB,offsetA=\arroffsetA,offsetB=\arroffsetB,arrowsize=5pt,arrowinset=0.7]{->}{#2}{#3}}
\newcommand{\jcbarr}[4][0]{ % ncbarr is defined in some thridy party package so do not use!\emph{}
\ncarr[#1]{#3}{#4}
\nbput[labelsep=2pt]{\footnotesize $#2$}
}

\newcommand{\ncaarr}[4][0]{
\ncarr[#1]{#3}{#4}
\naput[labelsep=2pt]{\footnotesize $#2$}
}

% \alabel{label}[npos][labelsep_pts]
\newcommandx*\alabel[3][2=0.5,3=2,usedefault]{\naput[labelsep=#3pt,npos=#2]{\footnotesize $#1$}}
% \blabel{label}[npos][labelsep_pts]
\newcommandx*\blabel[3][2=0.5,3=2,usedefault]{\nbput[labelsep=#3pt,npos=#2]{\footnotesize $#1$}}

% \idcomp mark an arrow as one component of an identifier
\newcommand{\idcomp}{\ncput[npos=0, nrot=:U]{\psline(0.2,-0.075)(0.2,0.075)}}  %add a bar to a node connection arrow
% pstrick supplements for s-arrows (previous name for d-arrow - should convert}

\newlength{\sarnodesepA}
\newlength{\sarnodesepB}
\newlength{\saroffsetA}
\newlength{\saroffsetB}
\newlength{\sarnodesepAsav}
\newlength{\sarnodesepBsav}

\newcommand{\sarreset}{
\setlength{\sarnodesepA}{0pt}
\setlength{\sarnodesepB}{0pt}
\setlength{\saroffsetA}{0pt}
\setlength{\saroffsetB}{0pt}
}

\sarreset

% sar - S-arrow
\newcommand{\ncsar}[3][0]{
\setlength{\sarnodesepAsav}{\sarnodesepA}
\setlength{\sarnodesepBsav}{\sarnodesepB}
\addtolength{\sarnodesepA}{3pt}
\addtolength{\sarnodesepB}{7pt}
\ncarc[nodesepA=\sarnodesepA,nodesepB=\sarnodesepB,offsetA=\saroffsetA,offsetB=\saroffsetB,arcangle=#1]{-}{#2}{#3}
\ncput[nrot=:R,npos=1]{\pstriangle(0,0)(.2,.2)}
\setlength{\sarnodesepA}{\sarnodesepAsav}
\setlength{\sarnodesepB}{\sarnodesepBsav}
}


% bsar - below labelled S-arrow
\newcommand{\ncbsar}[4][0]{
\ncsar[#1]{#3}{#4}
\nbput[labelsep=2pt]{\footnotesize $#2$}
}
% asar - above labelled S-arrow
\newcommand{\ncasar}[4][0]{
\ncsar[#1]{#3}{#4}
\naput[labelsep=2pt]{\footnotesize $#2$}
}

% cdar - composite dependency arrow
\newcommand{\nccdar}[3][0]{
\setlength{\sarnodesepAsav}{\sarnodesepA}
\setlength{\sarnodesepBsav}{\sarnodesepB}
\addtolength{\sarnodesepA}{3pt}
\addtolength{\sarnodesepB}{11pt}
\ncarc[nodesepA=\sarnodesepA,nodesepB=\sarnodesepB,offsetA=\saroffsetA,offsetB=\saroffsetB,arcangle=#1]{-}{#2}{#3}
\ncput[nrot=:R,npos=1]{\pstriangle(0,0.1)(.2,.2)}
\ncput[nrot=:R,npos=1]{\psdot[dotsize=1pt](-0.0075,0.05)}   %!!
\setlength{\sarnodesepA}{\sarnodesepAsav}
\setlength{\sarnodesepB}{\sarnodesepBsav}
}


% bcdar - below labelled composite dependency arrow
\newcommand{\ncbcdar}[4][0]{
\nccdar[#1]{#3}{#4}
\nbput[labelsep=2pt]{\footnotesize $#2$}
}
% acdar - above labelled composite dependency arrow
\newcommand{\ncacdar}[4][0]{
\nccdar[#1]{#3}{#4}
\naput[labelsep=2pt]{\footnotesize $#2$}
}


% rsar - recursive S-arrow
\newcommand{\ncrsar}[2]{
\setlength{\sarnodesepAsav}{\sarnodesepA}
\setlength{\sarnodesepBsav}{\sarnodesepB}
\addtolength{\sarnodesepA}{3pt}
\addtolength{\sarnodesepB}{7pt}
\ncloop[nodesepA=\sarnodesepA,nodesepB=\sarnodesepB,
        offsetA=\saroffsetA,offsetB=\saroffsetB,
        armA=0.7cm,armB=0.6cm,angleA=90,angleB=-90,loopsize=-1,linearc=0.4
				]{-}{#1}{#2}
\ncput[nrot=:R,npos=5]{\pstriangle(0,0)(.2,.2)}
\setlength{\sarnodesepA}{\sarnodesepAsav}
\setlength{\sarnodesepB}{\sarnodesepBsav}
}

% pstrick supplements for multi-arrows

\newlength{\marnodesepA}
\newlength{\marnodesepB}
\newlength{\maroffsetB}
\newlength{\marnodesepBsav}

\newcommand{\marreset}{
\setlength{\marnodesepA}{0pt}
\setlength{\marnodesepB}{0pt}
\setlength{\maroffsetB}{0pt}
}

\marreset

%ncmarr[#1 arcangle1][#2 arcangle2]{#3 name}{#4 domain1}{#5 domain2}{#6 junction}{#7 codomain}
\newcommandtwoopt{\ncmarr}[6][8][8]{%
\ncarc[nodesepA=\marnodesepA,nodesepB=0,arcangle=#1]{-}{#3}{#5}
\ncarc[nodesepB=0,arcangle=-#1]{-}{#4}{#5}
\ncarc[arcangle=#2,nodesepB=\marnodesepB,offsetB=\maroffsetB]{->}{#5}{#6}
}%


\newcommandtwoopt{\nchmarr}[6][8][8]{%
\ncarc[nodesepA=\marnodesepA,nodesepB=0,arcangle=#1]{-}{#3}{#5}
\ncarc[nodesepB=0,arcangle=#1]{-}{#4}{#5}
\ncarc[arcangle=#2,nodesepB=\marnodesepB,offsetB=\maroffsetB]{->}{#5}{#6}
}%

\newcommandtwoopt{\ncamarr}[7][8][8]{%
\ncmarr[#1][#2]{#4}{#5}{#6}{#7}
\naput[npos=.05]{$#3$}
}%
\newcommandtwoopt{\ncbmarr}[7][8][8]{%
\ncmarr[#1][#2]{#4}{#5}{#6}{#7}
\nbput[npos=.05]{$#3$}
}%

\newcommandtwoopt{\ncbhmarr}[7][8][8]{%
\nchmarr[#1][#2]{#4}{#5}{#6}{#7}
\nbput[npos=.05]{$#3$}
}%

\newcommandtwoopt{\ncmarrr}[7][8][8]{
\ncarc[nodesepB=0,arcangle=#1]{-}{#3}{#6}
\ncline[nodesepB=0]{-}{#4}{#6}
\ncarc[nodesepB=0,arcangle=-#1]{-}{#5}{#6}
\ncarc[nodesepA=0,arcangle=#2]{->}{#6}{#7}
}

\newcommandtwoopt{\ncamarrr}[8][8][8]{
\ncmarrr[#1][#2]{#4}{#5}{#6}{#7}{#8}
\naput[npos=.05]{$#3$}
}
\newcommandtwoopt{\ncbmarrr}[8][8][8]{
\ncmarrr[#1][#2]{#4}{#5}{#6}{#7}{#8}
\nbput[npos=.05]{$#3$}
}

%gats.macros.tex

\usepackage{environ}    % also used in ermacros % here used for \NewEnvrion

\newcommand{\gat}[1][U]{
\ensuremath{\mathcal{#1}}}  % used to hav a space in here
\newcommand{\gatw}[1][U]{\gat[#1]\ }  % use this if need trailing space
\newcommand{\ingat}[1][U]{in \gat[#1]}
\newcommand{\isagat}[1][U]{\gat[#1] is a g.a.t.}
\newcommand{\inagat}{in a g.a.t. }

% macro for a generic theory
%\newcommand{\theory}
%{\textit{U}}

\newcommand{\intheory}
{is a derived rule of \gat[U]}

% Macros for GAT rules

\newcommand{\isT}[1]
{#1\mbox{ is a type}}

\newcommand{\ofT}[2]
{#1 \in #2
}

% Macros for GAT rules   <!-- new old -->
\newcommand{\istype}[1]
{#1\mbox{ is a type}}

\newcommand{\oftype}[2]
{#1 \in #2
}

%\context{x}{\Delta}{n}
\newcommand{\context}[3]
{\ofT{#1_1}{#2_1},... \ofT{#1_{#3}}{#2_{#3}(#1_1,...#1_{#3-1})}
}

%\subcontext{x}{\Delta}{i}{k}
\newcommand{\subcontext}[4]
{\ofT{#1_{#3_1}}{#2_{#3_1}},... \ofT{#1_{#3_#4}}{#2_{#3_#4}(#1_1,...#1_{#3_#4-1})}
}

% #schematic context
\newcommand{\schmcon}[3]
{\ofT{#1_1}{#2_1},... \ofT{#1_{#3}}{#2_{#3}}
}
% abbreviated to
\newcommand{\con}[3]
{\schmcon{#1}{#2}{#3}}

% schematic subcontext
%\subcon{x}{\Delta}{i}{k}
\newcommand{\subcon}[4]
{\ofT{#1_{#3_1}}{#2_{#3_1}},... \ofT{#1_{#3_#4}}{#2_{#3_#4}}
}

% permuted context
%\permcon{x}{\Delta}{n}{\sigma}
\newcommand{\permcon}[4]
{\ofT{#1_{#4(1)}}{#2_{#4(1)}},... \ofT{#1_{#4(#3)}}{#2_{#4(#3)}}
}
% permuted term
%\permterm{t}{n}{\sigma}
\newcommand{\permterm}[3]
{
#1_{#3(1)},...#1_{#3(#2)}
}


% Idioms
\newcommand{\xDelta}[1]{\con{x}{\Delta}{#1}}
\newcommand{\xDeltap}[1]{\con{x}{\Delta'}{#1}}
\newcommand{\xOmega}[1]{\con{x}{\Omega}{#1}}
\newcommand{\xOmegap}[1]{\con{x}{\Omega'}{#1}}
\newcommand{\yOmega}[1]{\con{y}{\Omega}{#1}}
\newcommand{\yOmegap}[1]{\con{y}{\Omega'}{#1}}

\newcommand{\xDeltasigma}[1]{\permcon{x}{\Delta}{#1}{\sigma}}
\newcommand{\xDeltapsigma}[1]{\permcon{x}{\Delta'}{#1}{\sigma}}
\newcommand{\xOmegasigma}[1]{\permcon{x}{\Omega}{#1}{\sigma}}
\newcommand{\xOmegapsigma}[1]{\permcon{x}{\Omega'}{#1}{\sigma}}
\newcommand{\yOmegasigma}[1]{\permcon{y}{\Omega}{#1}{\sigma}}
\newcommand{\yOmegapsigma}[1]{\permcon{y}{\Omega'}{#1}{\sigma}}

\newcommand{\xDeltainvsigma}[1]{\permcon{x}{\Delta}{#1}{\sigma^{-1}}}
\newcommand{\xDeltapinvsigma}[1]{\permcon{x}{\Delta'}{#1}{\sigma^{-1}}}
\newcommand{\xOmegainvsigma}[1]{\permcon{x}{\Omega}{#1}{\sigma^{-1}}}
\newcommand{\xOmegapinvsigma}[1]{\permcon{x}{\Omega'}{#1}{\sigma^{-1}}}
\newcommand{\yOmegainvsigma}[1]{\permcon{y}{\Omega}{#1}{\sigma^{-1}}}
\newcommand{\yOmegapinvsigma}[1]{\permcon{y}{\Omega'}{#1}{\sigma^{-1}}}

%Idioms enclosed as tuples
\newcommand{\encxDelta}[1]{\tuple{\con{x}{\Delta}{#1}}}
\newcommand{\encxDeltap}[1]{\tuple{\con{x}{\Delta'}{#1}}}
\newcommand{\encxOmega}[1]{\tuple{\con{x}{\Omega}{#1}}}
\newcommand{\encxOmegap}[1]{\tuple{\con{x}{\Omega'}{#1}}}
\newcommand{\encyOmega}[1]{\tuple{\con{y}{\Omega}{#1}}}
\newcommand{\encyOmegap}[1]{\tuple{\con{y}{\Omega'}{#1}}}

\newcommand{\encxDeltasigma}[1]{\tuple{\permcon{x}{\Delta}{#1}{\sigma}}}
\newcommand{\encxDeltapsigma}[1]{\tuple{\permcon{x}{\Delta'}{#1}{\sigma}}}
\newcommand{\encxOmegasigma}[1]{\tuple{\permcon{x}{\Omega}{#1}{\sigma}}}
\newcommand{\encxOmegapsigma}[1]{\tuple{\permcon{x}{\Omega'}{#1}{\sigma}}}
\newcommand{\encyOmegasigma}[1]{\tuple{\permcon{y}{\Omega}{#1}{\sigma}}}
\newcommand{\encyOmegapsigma}[1]{\tuple{\permcon{y}{\Omega'}{#1}{\sigma}}}

\newcommand{\encxDeltainvsigma}[1]{\tuple{\permcon{x}{\Delta}{#1}{\sigma^{-1}}}}
\newcommand{\encxDeltapinvsigma}[1]{\tuple{\permcon{x}{\Delta'}{#1}{\sigma^{-1}}}}
\newcommand{\encxOmegainvsigma}[1]{\tuple{\permcon{x}{\Omega}{#1}{\sigma^{-1}}}}
\newcommand{\encxOmegapinvsigma}[1]{\tuple{\permcon{x}{\Omega'}{#1}{\sigma^{-1}}}}
\newcommand{\encyOmegainvsigma}[1]{\tuple{\permcon{y}{\Omega}{#1}{\sigma^{-1}}}}
\newcommand{\encyOmegapinvsigma}[1]{\tuple{\permcon{y}{\Omega'}{#1}{\sigma^{-1}}}}

\newcommand{\tstyle}{\vdash}

% gat macros developed for cwf paper

% Expressing gats
\newenvironment{gatrules}
{
$$
\begin{array}{l l}
}
{
\end{array}
$$
}
\newcommand{\gatintros}
{
\textbf{Symbol} & \textbf{Introductory\ Rule}                      \\}

\newcommand{\gataxioms}
{\textbf{Axioms}\\}
\newcommand{\gatintro}[3]{\ #1 & #2 \tstyle #3 \\}
\newcommand{\gatlocalintro}[3]{\ #1 & #2 \dashv }
\newcommand{\gataxiom}[2]{\multicolumn{2}{l}{\ \ #1\mbox{,  whenever\ } #2} \\}
\newcommand{\noleft}{\left.\kern-\nulldelimiterspace} % so that no space taken by absent left brace


\newcommand{\gatmultiaxiom}[2]
{\multicolumn{2}{l}{
  \noleft
    \begin{array}{l}
		#1
    \end{array} 
  \right\} \mbox{whenever\ } 	#2 
	}\\}
	
	\newcommand{\axid}[1]{\text{#1}.\ }	

%New context sharing macros
\newcommand{\gatintroducing}[1]{
{\arraycolsep=0pt
  \begin{array}{l}
          #1
  \end{array}} &
}

%*********************************
% \begin{\gatgroup}{context}
%    rules
%  \end{\gatgroup}
%*********************************
\NewEnviron{gatgroup}[1]{%
  \noleft
  {\arraycolsep=0pt
   \begin{array}{l}
\BODY
    \end{array} 
   }
   \ \right\} 
	%\mbox{\ whenever\ } 
	#1
	\vspace{0.1cm} 
}
%*********************************

%*********************************
% \begin{\gatgroupnoshared}
%    rule
%  \end{\gatgroupnoshared}
%*********************************
\NewEnviron{gatgroupnoshared}{%
  {\arraycolsep=0pt
   \begin{array}{l}
\BODY
    \end{array} 
   }
   \ 
	\vspace{0.1cm} 
}
%*********************************

% \gatsingular[width]{context}{conclusion}
\newcommand{\gatsingular}[3][4cm]{
\begin{gatgroupnoshared}
\gatleaf[#1]{#2}{#3} 
\end{gatgroupnoshared}
}

%*********************************
% \gatleaf}[width]{context}{assertion}
%*********************************
\newcommand{\gatleaf}[3][4cm]{%
\makebox[#1]{$#3$ \dotfill} \dotfill \  #2
}
%*********************************
%*********************************
% \gatstandalonesingle}{context}{assertion}
%*********************************
\newcommand{\gatstandalonesingle}[2]{%
#2 \makebox[2.5cm]{\dotfill} \  #1
}
%*********************************

% \gataxiomno{axiomno}
\newcommand{\gataxiomno}[1]{\makebox[0.5cm]{} \axid{#1}}


% metagat.macros.tex

%Meta-theories

%\newcommand{\typ}{\triangleright}
\newcommand{\typ}{\nabla}
\newcommand{\trm}{\tau}
\newcommand{\cross}{\otimes}
\newcommand{\sub}{^*}
\newcommand{\diag}{\delta}

\newcommand{\typeseq}[2]
{\ofT{#1_1}{\typ},... \ofT{#1_{#2}}{\typ(#1_{#2-1})}}

\newcommand{\typeseqcont}[3]
{\ofT{#1_1}{\typ({#2})},... \ofT{#1_{#3}}{\typ(#1_{#3-1})}}

\newcommand{\Ob}{Ob}
\newcommand{\obj}{Ob} % <!-- new old --<
\newcommand{\Hom}{Hom}
\newcommand{\objseq}[2]
{\ofT{#1_1}{\obj},... \ofT{#1_{#2}}{\obj(#1_{#2-1})}}


\def\dottededge{\ncline[linestyle=dotted, nodesep=0.3cm]}
\def\noedge{\ncline[linestyle=none]}
\def\thinedge{\ncline[linewidth=0.4pt]}

\newcommand{\member}[1]
{\ncarc[arcangle=-30,nodesepB=0.03]{->}{\pspred}{\pssucc}
\nbput[labelsep=0.1]{#1}}

\newcommand{\loweraccutemember}[1]
{\ncarc[arcangle=-15,nodesepB=0.03]{->}{\pspred}{\pssucc}
\nbput[labelsep=0.05,npos=0.85]{#1}}

\newcommand{\uppermember}[1]
{\ncarc[arcangle=30,nodesepB=0.03]{->}{\pspred}{\pssucc}\naput{#1}}

\newcommand{\upperaccutemember}[1]
{\ncarc[arcangle=10,nodesepB=0.03]{->}{\pspred}{\pssucc}\naput[npos=0.85]{#1}}

% flexbranch 
% #1 node label
% #2 thislevelsep
% #3 next level sep
% #4 variable (eg x)
% #5 index leter (eg n)
% #6 close parenthesis
% #7 continuation branches
\newcommand{\flexbranch}[7]
{
\pstree[thislevelsep=*#2,nodesep=0.05]
		{\Rnode{#1 1}{\Tr{#4_1 #6}}}
	  {\pstree[thislevelsep=#3]  
				   {\Rnode{#1 2}{\Tr[edge=\dottededge]{#4_{#5} #6}}}
					 {#7}
		}
}

\newcommand{\flexbranchplusleaf}[6]
{
\flexbranch{#1}{#2}{#3}{#4} {#5} {#6}
  {
   %\Rnode{#1 3}{\Tr{#4 #6}}
	 \Tr{\Rnode{#1 3}{#4 #6}}
  }
}

\newcommand{\flexbranchplusarc}[7]
{
\flexbranch{#1}{#2}{#3}{#4} {#5} {#6}
  {
   %\Rnode{#1 3}{\Tr{#4 #6}\member{#7}}
	 \Tr{\Rnode{#1 3}{#4 #6}}\member{#7}
  }
}

\newcommand{\flexbranchinitialarc}[9]
{
\pstree[thislevelsep=*#2,nodesep=0.05]
		{\Rnode{#1 1}{\Tr{#4_#8 #6}}#9}
	  {\pstree[thislevelsep=#3]  
				   {\Rnode{#1 2}{\Tr[edge=\dottededge]{#4_{#5} #6}}}
					 {#7}
		}
}

\newcommand{\equality}[2]
{
\ncline [doubleline=true, nodesep=0.2cm]{#1}{#2}
}
\newcommand{\equalityarc}[2]
{
\ncarc [arcangleA=-30, arcangleB=-20, doubleline=true, nodesep=0.1cm]{#1}{#2}
}

\usepackage[margin=4.0cm]{geometry} %was 3cm
\usepackage{mathptmx}
\usepackage{amsfonts}
\usepackage{array}
\usepackage{pstricks}
\usepackage{pst-tree}
\usepackage{pst-plot}
\usepackage{pst-node}
\usepackage{stmaryrd}
\usepackage{amsmath}
\usepackage{verbatim}
\usepackage{graphicx}  
\usepackage{calc}
\usepackage{xifthen}
\usepackage{xcolor}
\usepackage{color}
\usepackage{stringstrings}
%\usepackage[small,bf,margin=3pt,format=hang, labelsep=endash,singlelinecheck=false]{caption} %prevuiously justification=justified
%\usepackage{enumerate}
%\usepackage{enumitem}
\usepackage{enumerate}
\usepackage[shortlabels]{enumitem}
\usepackage{float}
\usepackage[section]{placeins}
%\setlength{\captionmargin}{5pt}
\usepackage{environ}
\usepackage{multirow}
\usepackage{rotating}
\usepackage{longtable}
\usepackage{afterpage}
\usepackage{needspace}


%DEFINE ENVIRONMENT BLOCK
% Riddle
\newsavebox{\riddlebox}

\newenvironment{erexample}
{\newcommand\colboxcolor{F0F0F0}%was F8F8F8
\begin{lrbox}{\riddlebox}
\begin{minipage}{\dimexpr\columnwidth-2\fboxsep\relax} \textbf{} \\ \itshape}
{\end{minipage}\end{lrbox}%
%\begin{center}
\colorbox[HTML]{\colboxcolor}{\usebox{\riddlebox}}
%\end{center}
}

\newenvironment{erbox}
{\newcommand\colboxcolor{F0F0F0}%was F8F8F8
\begin{lrbox}{\riddlebox}%
\begin{minipage}{\dimexpr\columnwidth-2\fboxsep\relax} }
{\end{minipage}\end{lrbox}%
%\begin{center}
\colorbox[HTML]{\colboxcolor}{\usebox{\riddlebox}}
%\end{center}
}

%\begin{erboxedFigure}{#1 FigureParam}{#2 Label}{#3 Caption}
\NewEnviron{erboxedFigure}[3]{%
\begin{figure}[#1]
\begin{erexample}
\begin{center}
\BODY
\end{center}
\vspace{-0.5cm}
\caption{#3}
\label{#2}
\end{erexample}
\end{figure}
}

\newcommand{\erpictureFolder}[0]{../SharedPictures}

\newcommand{\ercenterPicture}[1]{
\begin{center}
\input{\erpictureFolder/#1}
\end{center}
}


\newlength{\erhalfHt}

%\erinlinePicture{#1 pictureFilename}{#2 pictureHeight}
\newcommand{\erinlinePicture}[2]{
\setlength{\erhalfHt}{#2cm * \real{0.5}}
\raisebox{-\erhalfHt}[\erhalfHt + 0.5cm][\erhalfHt + 0.5cm]{
\input{\erpictureFolder/#1}
} 
}

%\erplainFig{#1 pictureFilename}{#2 figureParam}{#3Caption}
\newcommand{\erplainFig}[3]{
\begin{figure}[#2]
\begin{center}
\input{\erpictureFolder/#1}
\end{center}
\caption{#3}
\label{#1}
\end{figure}
}

%\erboxedFigPicture{#1 pictureFilename}{#2 figureParam}{#3Caption}
\newcommand{\erboxedFigPicture}[3]{
\begin{figure}[#2]
\begin{erexample}
\vspace{-0.5cm}
\begin{center}
\input{\erpictureFolder/#1}
\end{center}
\caption{#3}
\label{#1}
\end{erexample}
\end{figure}
}

%\erLeftSideFig{#1 pictureFilename}{#2 figureParam}{#3Caption}
\newcommand{\erLeftSideFig}[3]{
\begin{figure}[#2]
\begin{erexample}
  \begin{minipage}[c]{0.4\textwidth}
    \caption{#3}
    \label{#1}
  \end{minipage}
  \begin{minipage}[c]{0.5\textwidth}
    \input{\erpictureFolder/#1}
  \end{minipage}
\end{erexample}
\end{figure}
}

%\erbulletedFig{#1 pictureFilename}{#2 figureParam}{#3Caption}
\NewEnviron{erbulletedFig}[3]{%
\begin{figure}[#2]
\begin{erexample}
\vspace{-0.5cm}
\begin{center}
$
\begin{array}{c m{0.25cm} | m{6cm}}
\raisebox{-2.0cm}{
\input{\erpictureFolder/#1}}& & \text{\parbox{6cm}{\raggedright{\footnotesize{
\begin{enumerate}[(i)]
\BODY
\end{enumerate}}}}} \\
\end{array}
$
\end{center}
\caption{#3}
\label{#1}
\end{erexample}
\end{figure} 
}


%\begin{erbulletedDimFig}{#1 pictureFilename}{#2figureParam} {#3Caption} {#4PictureHeight}{#5TextWidth}

\NewEnviron{erbulletedDimFig}[5]{%
\begin{figure}[#2]
\begin{erexample}
\vspace{-0.5cm}
\begin{center}
$
\begin{array}{c m{0.25cm} |  m{#5cm}}
\setlength{\erhalfHt}{#4cm * \real{0.5}}
\raisebox{-\erhalfHt}{
\input{\erpictureFolder/#1}}& & \text{\parbox{#5cm}{\raggedright{\footnotesize{
\begin{enumerate}[(i)]
\BODY
\end{enumerate}}}}} \\
\end{array}
$
\end{center}
\caption{#3}
\label{#1}
\end{erexample}
\end{figure} 
}

%\begin{ernotedModel}{#1 pictureFilename}{#2PictureHeight}{#3PictureWidth}{#4TextWidth}

\NewEnviron{ernotedModel}[4]{%
\begin{center}
$
\begin{array}{m{#3cm} m{1cm} | c m{#4cm}}
\setlength{\erhalfHt}{#2cm * \real{0.5}}
\raisebox{-\erhalfHt}{
\input{\erpictureFolder/#1}}& & & \text{\parbox{#4cm}{\raggedright{\footnotesize{
\BODY
}}}} \\
\end{array}
$
\end{center} 
}

%\begin{ermodelText}{#1 pictureFilename}{#2PictureHeight}{#3PictureWidth}{#4TextWidth}

\NewEnviron{ermodelText}[4]{%
\begin{center}
\begin{tabular}{m{#3cm} m{1cm}  c m{#4cm}}
\setlength{\erhalfHt}{#2cm * \real{0.5}}
\raisebox{-\erhalfHt}{
\input{\erpictureFolder/#1}}& & & \text{\parbox{#4cm}{\raggedright{\small{
\BODY
}}}} \\
\end{tabular}
\end{center} 
}


%\erbulletedModel{#1 pictureFilename}{#2PictureHeight}{#3PictureWidth}{#4TextWidth}

\NewEnviron{erbulletedModel}[4]{%
\begin{center}
$
\begin{array}{m{#3cm} m{1cm} | c m{#4cm}}
\setlength{\erhalfHt}{2cm * \real{0.5}}
\raisebox{-\erhalfHt}{
\input{\erpictureFolder/#1}}& & & \text{\parbox{#4cm}{\raggedright{\footnotesize{
\begin{enumerate}[(i)]
\BODY
\end{enumerate}}}}} \\
\end{array}
$
\end{center} 
}



%\ernotedDimFig{#1 pictureFilename}{#2 figureParam}{#3Caption}{#4PictureHeight}{#5TextWidth}
\NewEnviron{ernotedDimFig}[5]{%
\begin{figure}[#2]
\begin{erexample}
\vspace{-0.5cm}
\begin{center}
$
\begin{array}{c m{0.25cm} | c m{#5cm}}
\setlength{\erhalfHt}{#4cm * \real{0.5}}
\raisebox{-\erhalfHt}{
\input{\erpictureFolder/#1}}& & & \text{\parbox{#5cm}{\raggedright{\footnotesize{
\BODY }}}}\\
\end{array}
$
\end{center}
\caption{#3}
\label{#1}
\end{erexample}
\end{figure} 
}
%\begin{ernotedDimFigPW}{#1 pictureFilename}{#2 figureParam}{#3Caption}{#4PictureHeight}{#5PictureWidth}{#6TextWidth}
\NewEnviron{ernotedDimFigPW}[6]{%
\begin{figure}[#2]
\begin{erexample}
\vspace{-0.5cm}
\begin{center}
$
\begin{array}{>{\centering}m{#5cm} m{0.5cm} | c m{#6cm}}
\setlength{\erhalfHt}{#4cm * \real{0.5}}
\raisebox{-\erhalfHt}{
\centering \input{\erpictureFolder/#1}
}& & & \text{\parbox{#6cm - 0.5cm}{\raggedright{\footnotesize{
\BODY }}}}\\
\end{array}
$ \\
\vspace {0.2cm}
\end{center}
\caption{#3}
\label{#1}
\end{erexample}
\end{figure}
}



\newenvironment{erquote}
{\begin{quote}\itshape}
{\end{quote}}


%
%  erdiag
%
  
%\begin{erdiagram}{#1 height}{#2 width} 
% ....
% ....
%\end{erdiagram}
\newenvironment{erdiagram}[2]
{%\pspicture*(-#1,0)(#2,0)
\pspicture*(0,-#1)(#2,0)
%\psgrid
}
{\endpspicture}

\definecolor{lightyellow}{cmyk}{0,0,0.3,0}

% \eret{#1 x0} {#2 y0} {#3 x1} {#4 y1} {#5 corner radius} {#6 fill}
\newcommand {\eret}[6]
{ 
\ifthenelse{\equal{#6}{1}}
{\psframe[framearc=#5,fillstyle=solid,fillcolor=lightyellow](#1,#2)(#3,#4)}
{\psframe[framearc=#5,fillstyle=solid,fillcolor=white](#1,#2)(#3,#4)}
}

% et top 
\newcommand {\erettop}[4]
{
%\psframe[linestyle=none,linearc=2pt,cornersize=absolute,fillstyle=solid,fillcolor=lightyellow](#1,#2)(#3,#4)
\psline[linearc=2pt,fillstyle=none,fillcolor=lightyellow](#1,#4)(#1,#2)(#3,#2)(#3,#4)
}

% et bottom 
\newcommand {\eretbtm}[4]
{
%\psframe[linestyle=none,linearc=2pt,cornersize=absolute,fillstyle=solid,fillcolor=lightyellow](#1,#2)(#3,#4)
\psline[linearc=2pt,fillstyle=none,fillcolor=lightyellow](#1,#2)(#1,#4)(#3,#4)(#3,#2)
}

% et bottom left
\newcommand {\eretbl}[4]
{
%\psframe[linestyle=none,linearc=2pt,cornersize=absolute,fillstyle=solid,fillcolor=lightyellow](#1,#2)(#3,#4)
\psline[linearc=2pt,fillstyle=none,fillcolor=lightyellow](#1,#4)(#3,#4)(#3,#2)
}

% et middle left
\newcommand {\eretml}[4]
{
%\psframe[linestyle=none,linearc=2pt,cornersize=absolute,fillstyle=solid,fillcolor=lightyellow](#1,#2)(#3,#4)
\psline[linearc=2pt,fillstyle=none,fillcolor=lightyellow](#1,#2)(#3,#2)(#3,#4)(#1,#4)
}

% et top left
\newcommand {\erettl}[4]
{
%\psframe[linestyle=none,linearc=2pt,cornersize=absolute,fillstyle=solid,fillcolor=lightyellow](#1,#2)(#3,#4)
\psline[linearc=2pt,fillstyle=none,fillcolor=lightyellow](#1,#2)(#3,#2)(#3,#4)
}

% et bottom right
\newcommand {\eretbr}[4]
{
%\psframe[linestyle=none,linearc=2pt,cornersize=absolute,fillstyle=solid,fillcolor=lightyellow](#1,#2)(#3,#4)
\psline[linearc=2pt,fillstyle=none,fillcolor=lightyellow](#1,#2)(#1,#4)(#3,#4)
}

% et middle right
\newcommand {\eretmr}[4]
{
%\psframe[linestyle=none,linearc=2pt,cornersize=absolute,fillstyle=solid,fillcolor=lightyellow](#1,#2)(#3,#4)
\psline[linearc=2pt,fillstyle=none,fillcolor=lightyellow](#3,#4)(#1,#4)(#1,#2)(#3,#2)
}

% et top right
\newcommand {\erettr}[4]
{
%\psframe[linestyle=none,linearc=2pt,cornersize=absolute,fillstyle=solid,fillcolor=lightyellow](#1,#2)(#3,#4)
\psline[linearc=2pt,fillstyle=none,fillcolor=lightyellow](#1,#4)(#1,#2)(#3,#2)
}

% \ergrp{#1 x0} {#2 y0} {#3 x1} {#4 y1} {#5 corner radius} {#6 fill}
% #5 corner radius is unused!
\newcommand {\ergrp}[6]
{ 
\ifthenelse{\equal{#6}{1}}
{\psframe[fillstyle=solid,fillcolor=lightgray](#1,#2)(#3,#4)}
{\psframe[fillstyle=solid,fillcolor=white](#1,#2)(#3,#4)}
}

% \eretname {#1 x left of text} {#2 y top of text} {#3 text}
\newcommand {\eretname}[3]
{
%shift down 0.1 for height of text the anchor at baseline (B)
\rput[bl]{0}(0,-0.1){\rput[Bl]{0}(#1,#2){\footnotesize \textit{#3}}}
}

% \errelarm {#1 x0} {#2 y0} {#3 x1} {#4 y1} {#5 ismandatory} {#6 isconstructed}
\newcommand {\errelarm}[6]
{
\ifthenelse{\equal{#6}{1}}
{
%%\psline[linewidth=0.5pt,linearc=.05,linestyle=dashed,dash=6pt 6pt]{-}(#1,#2)(#3,#4)}
\ifthenelse{\equal{#5}{1}}
{\psline[linewidth=1.5pt,linearc=.05,linecolor=lightgray]{-}(#1,#2)(#3,#4)}
{\psline[linewidth=1.5pt,linearc=.05,linecolor=lightgray,linestyle=dashed,dash=2pt 2pt]{-}(#1,#2)(#3,#4)}
}
{
\ifthenelse{\equal{#5}{1}}
{\psline[linewidth=0.9pt,linearc=.05]{-}(#1,#2)(#3,#4)}
{\psline[linewidth=0.9pt,linearc=.05,linestyle=dashed,dash=2pt 2pt]{-}(#1,#2)(#3,#4)}
}
}

% \errelangle {#1 x0} {#2 y0} {#3 x1} {#4 y1} {#5 x2} {#6 y2} {#7 ismandatory} {#8 isocnstructed}
\newcommand {\errelangle}[8]
{
\ifthenelse{\equal{#8}{1}}
{
%\psline[linewidth=0.5pt,linearc=.1,linestyle=dashed,dash=6pt 6pt]{-}(#1,#2)(#3,#4)(#5,#6)}
\ifthenelse{\equal{#7}{1}}
{\psline[linewidth=1.5pt,linearc=.05,linecolor=lightgray]{-}(#1,#2)(#3,#4)(#5,#6)}
{\psline[linewidth=1.5pt,linearc=.1,linecolor=lightgray,linestyle=dashed,dash=2pt 2pt]{-}(#1,#2)(#3,#4)(#5,#6)}
}
{
\ifthenelse{\equal{#7}{1}}
{\psline[linewidth=0.9pt,linearc=.1]{-}(#1,#2)(#3,#4)(#5,#6)}
{\psline[linewidth=0.9pt,linearc=.1,linestyle=dashed,dash=2pt 2pt]{-}(#1,#2)(#3,#4)(#5,#6)}
}
}

% \ercrowfoot {#1 x0} {#2 y0} {#3 x11} {#4 y11} {#5 x12} {#6 y12} {#7 x13} {#8 y13} {#9 isconstructed}
\newcommand {\ercrowfoot}[9]
{
\ifthenelse{\equal{#9}{1}}
{
\psline[linewidth=1.5pt,linearc=.05,linecolor=lightgray]{-}(#1,#2)(#3,#4)
\psline[linewidth=1.5pt,linearc=.05,linecolor=lightgray]{-}(#1,#2)(#5,#6)
\psline[linewidth=1.5pt,linearc=.05,linecolor=lightgray]{-}(#1,#2)(#7,#8)
}{
\psline[linewidth=0.9pt,linearc=.05]{-}(#1,#2)(#3,#4)
\psline[linewidth=0.9pt,linearc=.05]{-}(#1,#2)(#5,#6)
\psline[linewidth=0.9pt,linearc=.05]{-}(#1,#2)(#7,#8)
}
}


% \eridcomprel{#1 x1}{#2 x2}{#3 y1}{#4 ymid}{#5 y2}
\newcommand {\eridcomprel}[5]
{
\psline[linewidth=0.9pt](#1,#3)(#1,#5)
\psline[linewidth=0.9pt](#2,#3)(#2,#5)
\psline[linewidth=0.9pt](#1,#4)(#2,#4)
}

% \eridrefrel{#1 x1}{#2 xmid}{#3 x2}{#4 y1}{#5 y2}
\newcommand {\eridrefrel}[5]
{
\psline[linewidth=0.9pt](#1,#4)(#3,#4)
\psline[linewidth=0.9pt](#1,#5)(#3,#5)
\psline[linewidth=0.9pt](#2,#4)(#2,#5)
}


% \errelname {#1 x} {#2 y} {#3 text}
\newcommand {\errelname}[3]
{
\rput[l]{0}(#1,#2){\textit{#3}}
}
% \errelseq {#1 x} {#2 y}
\newcommand {\erelseq}[2]
{
}
% \erattr {#1 x} {#2 y} {#3 ismandatory}{#4 idenitfying} {#5 text}
\newcommand {\erattr}[5]
{
\ifthenelse{\equal{#3}{1}}
{\rput[l]{0}(#1,#2){{\tiny $\square$} {\footnotesize \textit{\ifthenelse{\equal{#4}{0}}{\underline{#5}}{#5}}}}}
{\rput[l]{0}(#1,#2){\footnotesize $\circ$ \textit{\ifthenelse{\equal{#4}{0}}{\underline{#5}}{#5}}}}
}

%\ifthenelse{\equal{#4}{1}}
% \ertext {#1 x} {#2 y} {#3 text anchor} {#4 text}
%{\rput[l]{0}(#1,#2){\footnotesize $\circ$ \underline{\textit{#5}}}}
\newcommand {\ertext}[4]
{
\rput[B#3]{0}(#1,#2){{\footnotesize #4}}
}
% \erarc {#1 x0} {#2 y0} {#3 x1} {#4 y1} {#5 x2} {#6 y2} {#7 x3} {#8 y3}
\newcommand {\erarc}[8]
{
\psbezier[showpoints=false]{-}(#1,#2) (#3, #4)(#5,#6) (#7, #8)
}

% \erarc {#1 x0} {#2 y0} {#3 x1} {#4 y1} {#5 x2} {#6 y2} {#7 x3} {#8 y3}
\newcommand {\errelseq}[8]
{
\psbezier[showpoints=false]{-}(#1,#2) (#3, #4)(#5,#6) (#7, #8)
}
% \ertrace {#1 trace}   
\newcommand {\ertrace}[1]
{
}

\usepackage{amsthm} % added 7th April 2018
% theorems.macros.tex

\newtheorem{theorem}{Theorem}[section]
\newtheorem{observation}[theorem]{Observation}
\newtheorem{lemma}[theorem]{Lemma}

\newtheorem{proposition}[theorem]{Proposition}
\newtheorem{corollary}[theorem]{Corollary}
\newtheorem{conjecture}[theorem]{Conjecture}
\newtheorem{numbereddefinition}[theorem]{Definition}

\newenvironment{definition}[1][Definition]{\begin{trivlist}
\item[\hskip \labelsep {\bfseries #1}]}{\end{trivlist}}
\newenvironment{examples}[1][Examples]{\begin{trivlist}
\item[\hskip \labelsep {\bfseries #1}]}{\end{trivlist}}
\newenvironment{example}[1][Example]{\begin{trivlist}
\item[\hskip \labelsep {\bfseries #1}]}{\end{trivlist}}
\newenvironment{remark}[1][Remark]{\begin{trivlist}
\item[\hskip \labelsep {\bfseries #1}]}{\end{trivlist}}

\newenvironment{tageqn}[1]
{
\begin{equation}
\stepcounter{equation}
\label{#1}
\tag{\theequation --#1}
}
{
\end{equation}
}

\newenvironment{axiom}[1]
{
\begin{equation}
\label{#1}
\tag{#1}
}
{
\end{equation}
}

% when the tag is required different from the label eg when has math symbols can use:
\newenvironment{axiomtagged}[2]
{
\begin{equation}
\label{#1}
\tag{#2}
}
{
\end{equation}
}

%visible label
\newcommand{\vlabel}[2][]{\label{#2}#1(\textit{#2}):}





\usepackage{imakeidx}
\makeindex[name=definitions, title=Index of Definitions]
\makeindex[name=lemmas, title=Index of Lemmas]



\newcommand{\commentary}[1]{\marginpar{\footnotesize #1}}
\newcommand{\highlight}[1]{\colorbox{orange}{#1}}
\newcommand{\term}[1]{\textit{#1}\commentary{\colorbox{lightgray}{\textit{#1}}}\index[definitions]{#1}}
\newcommand{\llabel}[1]{\label{#1}\commentary{\colorbox{pink}{\scriptsize{#1}}}\index[lemmas]{#1}}
\newcommand{\lref}[1]{\ref{#1}\colorbox{pink}{\scriptsize{#1}}\index[lemmas]{#1!use of}}

\newcommand{\newt}[1]{\colorbox{yellow}{#1}}
\newenvironment{newtt}
{  \colorbox{yellow}{$[$ ...} 
}
{  \colorbox{yellow}{... $]$}
}
\newcommand{\oldt}[1]{\colorbox{yellow}{\sout{#1}}}
\newenvironment{oldtt}
{  \colorbox{red}{$[$ ...} 
}
{  \colorbox{red}{... $]$}
}

\newcommand{\reinstatet}[1]{\colorbox{lime}{#1}}
\newenvironment{reinstatett}
{  \colorbox{lime}{$[$ ...}
}
{  \colorbox{lime}{... $]$}
}

\newcommand{\tbd}{\highlight{TBD}}

%ithprojection function
\newcommand{\proji}[1]{\pi_#1}



\newenvironment{categoricalaside}
{\begin{framed}
\textbf{Categorical Aside}
}
{
\end{framed}
}

\newenvironment{noteforfuture}
{\begin{framed}
\textbf{Note For Future}
}
{
\end{framed}
}

\newenvironment{problem}
{\begin{framed}
\textbf{Problem}
}
{
\end{framed}
}

%quine quote
\newcommand{\qq}[1]{
\left\ulcorner#1\right\urcorner
}

%single quote
\newcommand{\sq}[1]{
\textnormal{\textquotesingle}#1\textnormal{\textquotesingle}
}

%lower quine quote
\newcommand{\lqq}[1]{
\left\llcorner #1\right\lrcorner
}


%from berkley
\newcommand{\langl}{\begin{picture}(4.5,7)
\put(1.1,2.5){\rotatebox{60}{\line(1,0){5.5}}}
\put(1.1,2.5){\rotatebox{300}{\line(1,0){5.5}}}
\end{picture}}
\newcommand{\rangl}{\begin{picture}(4.5,7)
\put(.9,2.5){\rotatebox{120}{\line(1,0){5.5}}}
\put(.9,2.5){\rotatebox{240}{\line(1,0){5.5}}}
\end{picture}}
\newcommand{\lang}{\begin{picture}(5,7)\put(1.1,2.5){\rotatebox{45}{\line(1,0){6.0}}}\put(1.1,2.5){\rotatebox{315}{\line(1,0){6.0}}}\end{picture}}
\newcommand{\rang}{\begin{picture}(5,7)\put(.1,2.5){\rotatebox{135}{\line(1,0){6.0}}}\put(.1,2.5){\rotatebox{225}{\line(1,0){6.0}}}\end{picture}}
%Try sharper tuple brackets -- except gives errors nested in captions so comment out
%\renewcommand{\tuple}[1]{\lang #1 \rang}

\newcommand{\setsuchthat}[2]{\left\{#1 \ \middle|\ #2\right\}}
\newcommand{\set}[1]{\left\{#1\right\}} 

% one to n - wanton
\newcommand{\wanton}[1]{#1_1,...#1_n}
\newcommand{\fn}{\wanton{f}}
\newcommand{\pn}{\wanton{p}}
\newcommand{\qn}{\wanton{q}}
\newcommand{\qnprime}{\wanton{q'}}
\newcommand{\xn}{\wanton{x}}
\newcommand{\xnp}{\wanton{x'}}
\newcommand{\yn}{\wanton{y}}
\newcommand{\ntuple}[1]{\tuple{\wanton{#1}}}
\newcommand{\wantom}[1]{#1_1,...#1_m}
\newcommand{\mtuple}[1]{\tuple{#1_1,...#1_m}}
\newcommand{\qm}{\wantom{q}}
\newcommand{\ym}{\wantom{y}}
\newcommand {\bntuple}{\ensuremath{\ntuple{b}}}
\newcommand {\fntuple}{\ensuremath{\ntuple{f}}}
\newcommand {\fnptuple}{\ensuremath{\ntuple{f}}}
\newcommand {\pntuple}{\ensuremath{\ntuple{p}}}
\newcommand {\qntuple}{\ensuremath{\ntuple{q}}}
\newcommand {\qnptuple}{\ensuremath{\ntuple{q'}}}
\newcommand {\qmtuple}{\ensuremath{\mtuple{q}}}
\newcommand {\sntuple}{\ensuremath{\ntuple{s}}}
\newcommand {\xntuple}{\ensuremath{\ntuple{x}}}
\newcommand {\xnptuple}{\ensuremath{\ntuple{x'}}}
\newcommand {\ymtuple}{\ensuremath{\mtuple{y}}}
\newcommand{\foreachi}[1][n]{for each $i$, $1 \leq i \leq #1$}
\newcommand{\foreachj}[1][m]{for each $j$, $1 \leq j \leq #1$}
\newcommand{\foreachk}[1][l]{for each $k$, $1 \leq k \leq #1$}


\newcommand{\ncarrNEGZZ}[3][0]{\ncarc[arcangle=#1,nodesepA=2pt,nodesepB=2pt,offsetA=-2pt,offsetB=-2pt,arrowsize=5pt,arrowinset=0.7]{->}{#2}{#3}}
\newcommand{\ncarrZ}[3][0]{\ncarc[arcangle=#1,nodesepA=2pt,nodesepB=2pt,offsetA=0pt,offsetB=0pt,arrowsize=5pt,arrowinset=0.7]{->}{#2}{#3}}
\newcommand{\ncarrZZ}[3][0]{\ncarc[arcangle=#1,nodesepA=2pt,nodesepB=2pt,offsetA=2pt,offsetB=2pt,arrowsize=5pt,arrowinset=0.7]{->}{#2}{#3}}
\newcommand{\ncarrZZZ}[3][0]{\ncarc[arcangle=#1,nodesepA=2pt,nodesepB=2pt,offsetA=4pt,offsetB=4pt,arrowsize=5pt,arrowinset=0.7]{->}{#2}{#3}}
\newcommand{\ncarrZZZZ}[3][0]{\ncarc[arcangle=#1,nodesepA=2pt,nodesepB=2pt,offsetA=6pt,offsetB=6pt,arrowsize=5pt,arrowinset=0.7]{->}{#2}{#3}}


\newcommand{\ccsquareoutline}[6]
{\begin{array}{c p{#1} c}
\Rnode{TL}{#3}  &\ &  \Rnode{TR}{#4}\\[#2]
\Rnode{BL}{#5}  &\ &  \Rnode{BR}{#6}
\end{array}
}
\newcommand{\ccsquareacross}[2]
{\mbox{\ncarr{TL}{TR}
\alabel{#1}
\ncarr{BL}{BR}
\blabel{#2}}
}
\newcommand{\ccsquaredown}[2]
{\mbox{\ncsar{TL}{BL}
\blabel{#1}
\ncsar{TR}{BR}
\alabel{#2}}
}

\newcommand{\ccsquareup}[2]
{\mbox{\ncsar{BL}{TL}
\blabel{#1}
\ncsar{BR}{TR}
\alabel{#2}}
}

\newcommand{\ccsquareanddroppers}[6]
{\ccsquareoutline{#1}{#2}{#3}{#4}{#5}{#6}
\ccsquaredown{p_{#3}}{p_{#4}}
}
\newcommand{\ccsquareprimitivedroppers}
{
\begin{arrows}
\ncsar{TL}{BL}
\ncsar{TR}{BR}
\end{arrows}
}

% ccprimitivepullbacksquare{width}{height}{TR}{BL}{BR}{f}
\newcommand{\ccprimitivepullbacksquare}[6]
{\ccsquareoutline{#1}{#2}{#6^*#3}{#3}{#4}{#5}
\ccsquareprimitivedroppers
\ccsquareacross{q(#6,#3)}{#6}
}

 
\NewEnviron{coneoutline}[3]{% width depth source
\begin{array} {c p{#1} c}
                      \\[0.5cm]
\Rnode{OTL}{#3} &\ &  \\[#2]
&&
\BODY
\end{array}
}

\newcommand{\conearrows}[3]
{\begin{arrows}
\ncarr[-15]{OTL}{BL}
\blabel{#1}
\ncarr[20]{OTL}{TR}
\alabel{#2}
\ncline{OTL}{TL}  % want \ncdarr for dashed I think
\trput[tpos=0.2,labelsep=6pt]{\footnotesize{$#3$}}
%\salabel{sa label #3}[0.5]
\end{arrows}
}

\newcommand{\conearrowsdefiningtuple}[2]
{\conearrows{#1}{#2}{\tuple{#1,#2}}
}

%\scopeTriangle{subject}{domain}{codomain}{apex}{diagonal}{riser}
\newcommand{\scopeTriangle}[6]
{
 \begin{array}{c  c  c}
                & \Rnode{apex}{#4} &                \\[1cm]
\Rnode{dom}{#2} &                  & \Rnode{cod}{#3} 
\end{array}
\begin{arrows}
\ncarr{dom}{cod}
\blabel{#1}
\ncsar{dom}{apex}
\alabel{#5}[0.2]
\ncsar{cod}{apex}
\blabel{#6}[0.2]
\end{arrows} 
}


% \composeSevenShaped[nodesize]{A}{B}{C}{f}{g}{h}
\newcommand{\composeSevenShaped}[7][1cm]
{\begin{array}{c c c}
\makebox[#1][c]{\nudgeup{0.5cm}\Rnode{A}{#2}} && \makebox[#1][c]{\Rnode{B}{#3}} \\[0.5cm]
           &    \makebox[#1][c]{\Rnode{C}{#4}}    &
\end{array}
\begin{arrows}
\ncarr{A}{B}
\alabel{#5}
\ncarr{B}{C}
\alabel{#6}[0.2][0.05]
\ncarr{A}{C}
\blabel{#7}[0.2][0.05]
\end{arrows}
}



\usepackage{mathptmx}  % This changes font to roman
\usepackage{anyfontsize}
\usepackage{mathtools}  % why have we got this?
\usepackage{alltt} 
\usepackage{cmll}
\usepackage{ulem}
\renewcommand{\ttdefault}{txtt}
\usepackage[left=1.5cm, right=4cm, marginparwidth=3cm, top=2cm, bottom=2.0cm]{geometry}
\usepackage{framed}
\usepackage[font=small]{caption}
\usepackage{changepage} % used for adjustwidth

\usepackage[hidelinks=true]{hyperref}   % I have become dependent on this get error otherwise... but i don't know why

%\usepackage{enumitem}

\setlength{\captionmargin}{2cm}
\theoremstyle{remark}
\newtheorem*{lemma*}{Lemma}

% have boxed figures
\usepackage{float}
\floatstyle{ruled} 
\restylefloat{figure}
\restylefloat{table}

\usepackage{colortbl}

%\usepackage{mathabox} % wanted this for \lcorners \rcorners but sadly error: mathabox.sty not found

\newtheorem*{lemmastar}{Lemma}

\NewEnviron{tightquote} %italic text indented left and right hand side
{\begin{adjustwidth}{1.5cm}{1.5cm}
\textit{
\BODY
}
\end{adjustwidth}
}

\usepackage{amsmath}
\mathchardef\mhyphen="2D % Define a "math hyphen"

\newcommand{\Ualg}{U \mhyphen alg}
\newcommand{\Upalg}{U' \mhyphen alg}
\newcommand{\Ialg}{I \mhyphen alg}
\newcommand{\catofccs}{\mathbf{Concat}}
\newcommand{\catoflargerccs}{\mathbf{CONCAT}}
%\newcommand{\alg}[1]{{#1}_{alg}}
\newcommand{\alg}[1]{alg(#1)}
\newcommand{\cc}[1]{cc(#1)}
\newcommand{\tccalgebra}{$tcc$-algebra\ }
\newcommand{\tccalgebras}{$tcc$-algebras\ }
\newcommand{\FAM}{\ensuremath{\mathbb{F\mathrm{AM}} }}

\def\therefore{\boldsymbol{\text{ }
\leavevmode
\lower0.4ex\hbox{$\cdot$}
\kern-.5em\raise0.7ex\hbox{$\cdot$}
\kern-0.55em\lower0.4ex\hbox{$\cdot$}
\thinspace\text{ }}}


\NewEnviron{point}
{\begin{adjustwidth}{1.5cm}{1.5cm}
\textbullet\ 
\BODY
\end{adjustwidth}
}

\begin{document}

\title{Instances (Models) of Generalised Algebraic Theories\\
DRAFT }

\author{John Cartmell}

\maketitle

\newcommand{\gatU}{\gat[U]}
\newcommand{\gatUw}{\gatU\ }
\newcommand{\CofU}{\ccat[C](U)}
\newcommand{\KU}{K_{U}}
\newcommand{\KUp}{K_{U'}}
\newcommand{\catCon}{\cat{Con}}
\newcommand{\catGAT}{\cat{GAT}}

\newcommand{\gatrule}[2]{$#1 \tstyle #2$}

\newcommand{\inlinedisplay}[1]
{
\setlength{\fboxsep}{1.5pt}
\setlength{\fboxrule}{0pt}
\fbox{$\displaystyle #1$}
}

\newcommand{\Imappedrule}[2]  {I\left(\gatrawdisplayrule{#1}{#2}\right)}
\newcommand{\sectionsofImappedrule}[2] { Sect \left( \Imappedrule{#1}{#2} \right)}

\newcommand{\Imap}[1]{\setlength{\fboxsep}{1pt}       
\setlength{\fboxrule}{0.2pt}I\left(\mbox{#1}\right)}

\newcommand{\Ipmap}[1]{\setlength{\fboxsep}{1pt}       
\setlength{\fboxrule}{0.2pt}I'\left(\mbox{#1}\right)}

\newcommand{\iI}{\scalebox{1.1}{$\boldsymbol{a}$}}
\newcommand{\Ibar}{\mkern 2.5mu\overline{\mkern-2.5mu \iI\mkern1.5mu}\mkern -1.5mu}

\newcommand{\ibarmappedrule}[2]  {\Ibar\left(\gatrawdisplayrule{#1}{#2}\right)}
\newcommand{\Isort}{\iI_{sort}}
\newcommand{\Iop}{\iI_{op}}

\newcommand{\hatU}{\rule{0pt}{11pt}\widehat {U}}

% workaround when used with Rnode for size of font used under the cross
%XXXXXXXXXXXXXXXXXXXXXXXXXXXXXXXXXXXXXXXXXXXXXXXXXXXXXXXXXXXXXXXXXXXXXXXXXXXX!!!!!!!!!!!!!!!!!!!!!!!!!!!!!!!
\renewcommand{\crossx}[3]{#1 \underset{\tiny #3}{\cross} #2}
\renewcommand{\crossx}[3]{#1 \underset{#3}{\cross} #2}  %Live without for a while to shut up warning messages.
%XXXXXXXXXXXXXXXXXXXXXXXXXXXXXXXXXXXXXXXXXXXXXXXXXXXXXXXXXXXXXXXXXXXXXXXXXXXX!!!!!!!!!!!!!!!!!!!!!!!!!!!!!!!

% macros for closing up * for ease iof readability.
\newcommand{\onestar}   {{_1}\kern-.15em^*}
\newcommand{\twostar}   {{_2}\kern-.15em^*}
\newcommand{\ipstar}    {{_{i-1}}\kern-.2em^*}
\newcommand{\istar}     {{_i}\kern-.2em^*}
\newcommand{\jstar}     {{_j}\kern-.2em^*}
\newcommand{\jpstar}    {{_{j-1}}\kern-.25em^*}
\newcommand{\monestar}{{_{m-1}}\kern-.15em^*}
\newcommand{\mstar}{{_m}\kern-.25em^*}
\newcommand{\nonestar}{{_{n-1}}\kern-.1em^*}
\newcommand{\nstar}{{_n}\kern-.2em^*}
\newcommand{\fonestar}   {f\onestar}             
\newcommand{\ftwostar}   {f\twostar}             
\newcommand{\fipstar}     {f\ipstar}
\newcommand{\fistar}     {f\istar}
\newcommand{\fjstar}     {f\jstar}              
\newcommand{\fjpstar}    {f\jpstar}              
\newcommand{\fnstar}     {f\nstar}
\newcommand{\fnonestar}  {f\nonestar} 
\newcommand{\fmonestar}  {f\monestar}        
\newcommand{\fmstar}     {f\mstar}        
\newcommand{\smstar}     {s\mstar}    
\newcommand{\sonestar}   {s\onestar}    
\newcommand{\gonestar}   {g\onestar}
\newcommand{\gprimeonestar}   {g'\onestar}
\newcommand{\gtwostar}   {g\twostar}         
\newcommand{\gjstar}     {g\jstar}  
\newcommand{\gprimejstar}     {g'\jstar}  
\newcommand{\gjpstar}    {g\jpstar}
\newcommand{\gmonestar}  {g\monestar} 
\newcommand{\gmstar}     {g\mstar} 
\newcommand{\gprimemstar}     {g'\mstar} 
\newcommand{\gnstar}     {g\nstar}   

\newcommand{\fipvectorstar}{\fipstar...\fonestar}
\newcommand{\fnvectorstar}{\fnstar...\fonestar}
\newcommand{\fmvectorstar}{\fmstar...\fonestar}
\newcommand{\gnvectorstar}{\gnstar...\gonestar}
\newcommand{\gmvectorstar}{\gmstar...\gonestar}

\newcommand{\Trule} {T-rule\ }
\newcommand{\trule} {$\in$-rule\ }
\newcommand{\Trules} {T-rules\ }
\newcommand{\trules} {$\in$-rules\ }
\newcommand{\Teqrule} {T=-rule\ }
\newcommand{\teqrule} {$\in=$-rule\ }

\newcommand{\Imapsto}{\scaleto{\mapsto}{8pt}}

\definecolor{lightergrey}{rgb}{0.9,0.9,0.9}
%\gatinterpretationdetail{label}{rulepremise}{ruleconclusion}{mapping}{justification}
\newcommand{\gatinterpretationdetail}[5]{
\refstepcounter{equation}(\theequation)\label{#1}&\gatrule{#2}{#3}&$\mapsto$&&$#4$&#5}

\newcommand{\gatinterpretationdetailcontinuation}[2]{&&&&$#1$&#2}

\newcommand{\gatinterpretationintro}[5]{
\refstepcounter{equation}(\theequation)\label{#1}& \gatrule{#2}{#3}&$\mapsto$&&\cellcolor{lightergrey}$#4$&#5}

%\gatinterpretationmapeqv{equivalentmapping}{justificaton}
\newcommand{\gatinterpretationmapeqv}[2]{&&&=&\ \ $#1$&#2}
\newcommand{\gatinterpretationmapeqvsingle}[1]{&&&=&\multicolumn{2}{l}{\ \ $#1$}}

\newcommand{\gatinterpretationaxcond}[5]{
\refstepcounter{equation}(\theequation)\label{#1}& \gatrule{#2}{#3}&$\scriptstyle iff$&&\cellcolor{lightergrey}$#4$&#5}

\newcommand{\gatinterpretationaxcondrhscontinuation}[2]{ &&&& \cellcolor{lightergrey}\hspace{-0.2
cm} $#1$ &{#2}}

%\gatinterpretationaxeqv{equivalentcondition}{justificatoI(r_{s_j})n}
\newcommand{\gatinterpretationaxeqv}[2]{&&$\scriptstyle iff$&&$#1$&#2}


\newcommand{\bigtuple}[1]{\big \langle #1 \big \rangle}


\newcommand{\highlightpara}[1]{\colorbox{highlight}{%
    \parbox{\dimexpr\linewidth-2\fboxsep}% a box with line-breaks that's just wide enough
        {#1}}
}

\newcommand{\genericcrossxproductdiagram}{
$
\begin{array}{ccccc}
\Rnode{xy}{\crossx{x}{y}{w}} &&               &&               \\[1.3cm]
\Rnode{x}{x}                 &&               && \Rnode{y}{y}  \\[1.3cm]
                             && \Rnode{w}{w}  &&                                                   
\end{array}
$
\makebox[0.2cm]{% technique to avoid throwing white space
\nccdar{xy}{x}
\blabel{p_{\crossx{x}{y}{w},x}}
\nccdar{x}{w}
\blabel{p_{x,w}}
\nccdar{y}{w}
\alabel{p_{y,w}} 
\ncaarr{q(p_{x,w},y)}{xy}{y}
}
}
\iffalse   %     IF FALSE  %%%%%%%%%%%%%%%%%%%%%%%%%%%%%%%%%%%%%%%%%%%%%%%
\begin{abstract}
The notion of 
an \textit{instance of  a generalised algebraic theory $U$ in a contextual category \catc} 
or, equivalently, of an \textit{internal $U$-structure in the contextual category \catc}, is defined. 
From inspection of the definition it is clear  
that to every generalised algebraic theory $U$ there is generalised algebraic theory $\hatU$ of internal $U$-structures.
The theory $\hatU$ is an extension of the generalised algebraic theory of contextual categories
by a set of rules (introductory rules and axioms) that have  the empty context as premise -- as such it is an extension
by constants and equational identitites between closed terms. Furthermore, every such extension of
the theory of contextual categories arises in this way as a $\hatU$  to some generalised algebraic theory  $U$. 

For $U$ a generalised algebraic theory, a $U$-algebra (previously defined algebraically in \cite{Cartmell78} and published in \cite{Cartmell86} ) can be defined to be any instance of the theory $U$ in the contextual category $\Fam$ of sets,
families of sets, families of families of sets and so on. 

Following the line taken in \cite{BCDEpaper} we study the theory of internal monoids and the theory of internal categories as worked examples. 
The theory of internal monoids is, as to be expected, in agreement with other descriptions such as in the book by Barr and Wells \cite{BarrandWells}. 
\end{abstract}

\section{Introduction}
\subsection{Background}
In metamathematics there are well established paradigms for use of the terms
\term{theory}, \term{signature}, \term{interpretation} and \term{model};
in this paper  we will follow these paradigms as closely as possible except that the term \term{instance} will substituted  for that of model. 
This substitution is made so as to reserve use of the term \term{model} for  use not as metamathematicians 
use it but as other theoreticians do and so as to be able to speak of theories modelling 
the world rather than of theories having models. For a discussion of the antithetical uses of the term 
\term{model} in metamathematics versus other disciplines see \cite{HodgesModelTheory}.

Traditionally, the notion of a theory, or of a certain class of theories, is  defined syntactically.
A metamathematically important class of theories  is the class of elementary theories. 
This is the class of theories written in first-order predicate logic with equality. 
Such a theory is defined to consist of a signature (see \cite{HodgesModelTheory}, for example) plus a set of axioms: 
the signature is required to table a set of predicate symbols and functions symbols and to assign arities to each, 
the axioms are required  to be a set of closed \term{well formed formulas} (wffs) written
in the language of the first-order predicate calculus (with equality)
using the tabled  symbols  consistently with the arities defined for them. 

In this tradition (see \cite{Mendelson}, for example), an interpretation of an elementary theory is defined to be 
a mapping of the symbols defined in the signature  of the theory 
to actual predicates and actual functions over some domain 
and subject to the requirement that n-ary predicate, respectively function, symbols be mapped to n-ary predicates, respectively functions.
As defined by Tarski in the so called satisfaction definition
such an interpretation induces an interpretation of all
closed well formed formulas of the signature as truth values. 
A model of a theory is defined to be an interpretation of the theory such that all axioms of the theory are mapped to true (i.e. are satisfied by the interpretation). 

Note that the distinction between signature and theory is significant because it makes possible a definition of model 
via a definition of interpretation: an interpretation is an interpretation of a signature, 
a model is an interpretation in which all axioms are satisfied i.e. have truth value `true'.   

As an aside we should mention that the term theory can be used somewhat differently. 
For example in the case of the theory of groups two different axiomatisations may be given 
-- one in which the inverse to a given element is axiomatised as a left inverse, and one in which the inverse to a given element is axiomatised as a right inverse --
the net effect of either axiomatisation being the same.
This usage in which  theory refers to the net effect of a signature plus associated axioms
can be formalised by describing a 
theory not as we have outlined above but as any set of sentences
 (of a given signature) that is closed under logical deduction\footnote{One such theory in this sense is the total set of true sentences of a model and  
this is called the diagram of the model. I have a  memory of a conference in the 1970's and of hearing a German set theorist
 remark that "God doesn't need Logic -- he has the diagram of the universe". He meant, of course, diagram in this sense of set of all true sentences.}. 
If we adopt this other useage in which a theory is a set of sentences closed under deduction then a signature plus axioms is referred to not as a theory but as a presentation of a theory. 

A most significant subclass of the class of elementary theories is the class of algebraic theories. 
What is meant by a algebraic theory, prior to Lawvere at least, is an elementary theory in a signature having only function symbols, 
i.e. having no predicate symbols, 
and in which every axiom is an equational identity between open terms with the understanding that
all variables are universally quantified. 
Because algebraic theories are special kinds of elementary theory the definitions of interpretation and model  specialise to yield definitions of 
interpretation and model for algebraic theories; in this special case, though, a model of a theory $U$ is usually spoken of as an algebra of $U$.

In the traditional view, the domain of interpretation of an algebraic theory may be any set $A$ 
and each n-ary function symbol is to be interpreted by a  function $f:A^n \morph A$.
After Lawvere it becomes usual to consider that an algebraic theory can be interpreted in any category with finite products
 so that $A$ may be any object of the category and  $f$ may be any morphism $f:A^n \morph A$. 
For $U$ an algebraic theory, the term `model of $U$' therefore may take two meanings:
\begin{enumerate}[(i)]
\item in the most general sense of `model in any category  with finite products',
 such models of $U$ are sometimes said to be internal $U$-objects (this terminology is then shortened further so that internal monoid-object becomes internal monoid and so on). 
\item in the restricted sense in which the interpretation is by sets and functions, such models are sometimes said to be algebras or
$U$-algebras. The category of $U$-algebras is denoted $U$-$alg$. 
\end{enumerate} 
Model in sense (ii) is a special case of model in sense (i) ---  the case in which the category with finite products is taken to be the category of sets and functions.

For completeness I must mention a different but related use of the term `interpretation' to describe 
mappings of the syntax of one theory to the syntax of a second
 determined by a mapping of the symbols of the first into the terms and well formed formulae of the second. 
Such an interpretation is said to be \term{valid} provided that the axioms of the first are mapped into provable well formed formulae of the second.  Such syntactic interpretations compose and therefore for each class of theory there is a category of theories and interpretations and accordingly such 
interpretations between theories may be said to be theory morphisms.
As an example, the two different presentations of the theory of groups mentioned above are different but isomorphic objects in the category of algebraic theories. 

\subsection{Generalised Algebraic Theories 	and Contextual Categories}

In the case of generalised algebraic theories then much of the above carries through except that
it is not possible to define an independent notion of `signature of a generalised algebraic theory' and
the approach to defining what a generalised algebraic theory is cannot simply be via a definition of what a signature is followed by a definition 
of a theory  as a signature plus axioms. This is because the rules for introducing symbols need be well-typed and to know that
they are well-typed we already need knowledge of the theory -- in other words the notions of theory and signature are interdependent. 
The  definition of generalised algebraic theory (\cite{Cartmell78},\cite{Cartmell86})  works around this difficulty and is by way of 
a definition of pretheory, 
followed by a definition of a theory as a well-typed{\footnote{In this paper I use the term `well-typed' in place of the term `well-formed' used in \cite{Cartmell78} and \cite{Cartmell86}.} pretheory. 

Consequential to the difficulty defining the notion of `signature
(to be used in a generalised algebraic theory)' prior to defining
what a (generalised algebraic) theory consists of, is  a difficulty defining the notion of `interpretation of a signature (for a generalised algebraic theory)' prior to 
defining what a `model of a generalised algebraic theory' consists of.  In \cite{Cartmell78} (published in \cite{Cartmell86}) this consequential difficulty is avoided by stepping over into algebra.
First  it is established \footnote{
The proof that categories $\catGAT$ and $\catCon$ are equivalent  is entirely trivial but runs to more than 50 pages. I have always interpreted this equivalence as meaning that generalised algebraic theories and contextual categories are more or less the same thing but if this is considered from the point of view of foundations then we have to tread carefully.} 
that  generalised algebraic theories and contextual categories are equivalent in the following sense: 
\noindent \label{ccgatequivalence}
\begin{point}
there is a category $\catGAT$ of generalised algebraic theories and interpretations,
\end{point}
\begin{point}
there is a category $\catCon$ of contextual categories,
\end{point}
\begin{point}
there is a functor $\ccat[C]: \catGAT \morph \catCon$  (for $U$ a generalised algebraic theory, the category $\ccat[C](U)$ 
has as objects equivalence classes of contexts and realisations, as defined 
in \cite{Cartmell78} and  \cite{Cartmell86}), 
\end{point}
\begin{point}
there is a functor $\gat[U]:\catCon \morph \catGAT$,
\end{point}
\begin{point}
the functor $\ccat[C]$ is an equivalence with inverse $\gat[U]$.
\end{point}
Next there is a description of
a contextual category $\Fam$ of sets, families of sets, families of families of sets and so on. 

This makes possible an  algebraic, post-Lawvere
style,  definition of model in either the general or the more restrictive sense so that in 
\cite{Cartmell78} and  \cite{Cartmell86}) we find,
for $U$ a generalised algebraic theory, 
\begin{point}
 a model in the restricted sense of a U-algebra is defined to be a functor $A: \ccat[C](U) \morph Fam$, 
\end{point}
\begin{point}
a model in the more general sense of an internal $U$-structure\footnote{The  term `internal $U$-structure is chosen over `internal $U$-object' because the latter
is no longer appropriate in the broader context in which, generally, theories require multiplicities of objects rather than single objects for their domain of interpretation.} in a contextual category
$C$ is defined to be a contextual functor $A: \ccat[C](U) \morph C$.
\end{point}
The category of internal $U$-structures is defined to be the category whose objects
are pairs $\tuple{\catc,A}$ 
where $\catcw$ is a contextual 
category and $A$ is an internal 
$U$-structure in
 \catcw and whose morphisms between $\tuple{\catc,A}$ and $\tuple{\catc',A'}$ are pairs $\tuple{F, \eta}$ where
$F: \catc \morph \catc'$ is a contextual functor and $\eta: A  \morph  F \circ A'$ is a natural transformation.
In other words the category of internal $U$-structures
is  the coslice category
$\CofU \downarrow \catCon$. Needless to say this category has an initial object
--- the identity functor on  $\CofU$.
 
Specialising to the case of $U$-algebras,  
an homomorphism between $U$-algebras $A$ and $A'$ is defined to be a 
natural transformation $\eta: A \morph A'$ and the category of $U$-algebras is a full subcategory of the 
functor category $\Fam^{\CofU}$. 

In \cite{Cartmell78} and  \cite{Cartmell86}) there is also a description of initial $U$-algebras but details are omitted:
\begin{tightquote}
Consider for a moment. Every theory $U$ has a minimal model denoted $\KU$ built out of the closed terms of $U$. Alternatively this minimal model is described just in terms of the structure $\CofU$. For example
if $1 \base A$ in $\CofU$ then 
$\KU(A)=Hom(1,A)$, otherwise if $1 \base A_1 \base ... \base A_n \base A$ in $\CofU$
then if $a_1 \in \KU(A_1)$, ... if $a_n \in \KU(A_n)(a_1,...a_{n-1})$ then 
$\KU(A)(a_1,...a_n)=\setsuchthat{a\in Hom_{\CofU}(1,A)}{a \circ p_A = a_n}$. \\
\end{tightquote} 

Followed by :
\begin{tightquote}
Now, the free $U$-algebras are the algebras $I$-$alg(\KUp)$ for $I: U \morph U'$ an extension of $U$ by constants alone. The finitely generated free $U$-algebras are those algebras where $U'$ is an extension by finitely many constants. \\
\end{tightquote}

\subsection{Interpreting Generalised Algebraic Theories}

There is an indirectness in the algebraic-style definitions of model given in \cite{Cartmell78} and  \cite{Cartmell86} and that we have summarised above. 
As a consequence, though they are  useful for a 
number of metamathematical purposes they  don't really
help one reason directly about what constitutes a model in the case of any particular generalised algebraic theory --
although this is something we can have a good intuition about. 
The definitions we shall give in this paper rectify this and directly formalise the intuition.
As a bonus, from an examination of these definitions it can be seen  that to every generalised algebraic theory $U$ there is a generalised algebraic theory 
$\hat{U}$ which is the theory of internal $U$-structures\footnote{This is reliant on their being a generalised algebraic theory of contextual categories. There are nuances here and so at some point we need to discuss the extent or the manner to which this is true.}.

In this paper, in section \ref{sectioninwhichinstanceisdefined},  we define the missing definition, 
so to speak,  of an interpretation of  a generalised algebraic theory $U$ in  a contextual category \catcw and we say what it is for such an interpretation to be valid. 
In essence such an interpretation $I$ consists of a \textit{consistent} mapping

\begin{center}
\begin{tabular}{c p{1cm} c}
derived \Trules of $U$           & \raisebox{-0.07cm}{$\Imapsto$} & objects of \catc \\ [0.1cm]
derived \trules of $U$    & \raisebox{-0.07cm}{$\Imapsto$} & sections of \catc \\ [0.1cm]
\end{tabular}
\end{center}
so that derivable equalities in $U$ map to identical objects, respectively, sections of \catc.
There is a fair amount of detail to what is meant by  `consistent mapping' but what is fundamental is that this detail implies that 
interpretations $I$ of $U$ in \catcw are completely
determined by their mapping of the introductory rules of $U$. 
This is the equivalent, in the generalised algebraic case, of 
 the fact that, in regard to algebraic or first-order  theories, interpretations
are determined by a consistent mapping of the symbols withrespect to their signatures.

The definition that we give is a definition  of model in the sense of an internal $U$-structure in the contextual category $\catc$:
an internal $U$-structure in $C$ is exactly 
a valid interpretation of $U$ in $C$. This definition may of course be specialised to a  definition of model in the sense of
$U$-algebra by particularising to the case where $C$ is the contextual category $\Fam$.


From the details in
section \ref{sectioninwhichinstanceisdefined} 
of the definition of interpretation
it turns out that 
to every gat $U$ there is a theory of internal $U$-structures. We shall denote this theory as $\hatU$.

Every such theory $\hatU$ is an extension of the generalised algebraic theory of contextual categories
by a set of rules (introductory rules and axioms) that have  the empty context as premise
 --- as such it is an extension
by constants and equational identities between closed terms.  Conversely every  extension of
the theory of contextual categories by just constants and equational identities between closed terms can be interpreted as being a description of a generalised algebraic theory. This is expoited in \cite{BCDEpaper}.


The instances of $\hatU$  in $\Fam$ consist of  internal $U$-structures  i.e. they consist of contextual categories \catcw along with particular instances $I$ of
the theory $U$ in the contextual category \catc. \\
The category of $\hatU$-algebras is (isomorphic to) the category of internal $U$-structures.


\label{termmodelEQfreealgebra}For any generalised algebraic theory $\gat[U]$ we have two different 
 descriptions of the initial object of the category of internal $\gatU$-structures:
one as the free algebra $K_{\hatU}$ of the theory $\hatU$ as summarised above and one  as the pair $\bigtuple{\CofU, I_{triv}}$ 
where $I_{triv}$ is the trivial instance of $\gatU$ in $\CofU$. We have therefore that
\begin{equation*}
K_{\hat{U}} \cong \bigtuple{\CofU, I_{triv}}
\end{equation*}
in the category of internal $U$-structures.


\subsection{Outline}
\highlight{The following must be revised following the reorganisation}
To pave the way for the main definition,
the definition of generalised algebraic theories is reviewed 
in section \ref{generalisedalgrbraictheories}, and
a summary of some housekeeping lemmas from \cite{Cartmell78} is given. 
Various  extensions to the notation of contextual categories are described in section \ref{contextualnotationpartone} 
including a description of Voevodsky's `s' operator and of a local product notation 
that uses an operator $\crossx{}{}{}$ as a special case of pullback. Various
identities that we require are then detailed. 
The main definition is in section \ref{sectioninwhichinstanceisdefined}.
In section
 \ref{contextualnotationparttwo} introduce a further extension to the notation of contextual categories and present and prove various identities that 
support the presentation and simplication of examples.

We show in lemma \ref{internalmonoidlemma}  that 
the generalised algebraic theory of internal monoids can be expressed  as 
the theory of contextual categories plus:

\begin{gatrules}
\gatintros
\gatintroducing{M}
\ofT{M}{Ob} \\
\gatintroducing{unit}
\ofT{unit}{Hom(1,M)} \\
\gatintroducing{mult}
\ofT{mult}{Hom(M \times M,M)} \\
\gataxioms
\gatintroducing{ \gataxiomno{1} }
\tuple{p_M \circ unit,id_M} \circ mult =id_M \\
\gatintroducing{ \gataxiomno{2} }
\tuple{id_M,p_M \circ unit} \circ mult =id_M \\
\gatintroducing{ \gataxiomno{3} }
(mult \times id_M) \circ mult = (id_M \times mult) \circ mult
\end{gatrules}

This is in agreement with Barr and Wells \cite{BarrandWells}, page 232, where they describe
monoids internal to  a category\footnote{Generally we would be thinking of such a monoid internal to a category with finite products.}
as an example of a finite product (FP) sketch.

As a second worked example (in lemma \ref{internalcategorylemma})  we derive the generalised algebraic theory of internal categories.

\subsection{Acknowledgement}
I have written this paper after studying a draft of a paper (\cite{BCDEpaper}) sent to me by 
 one of its authors, Peter Dybjer. I hope the material presented here sheds some light on that paper. 
The two examples that I give are the same two presented there.

%\fi
\section{Background regarding Generalised Algebraic Theories}
\label{generalisedalgrbraictheories}

In this section we recap the definition  of the class of generalised algebraic theories 
as described in detail in \cite{Cartmell78}  and published in summary in \cite{Cartmell86}.
We shall adopt a few changes of terminology.

A generalised algebraic theory consists of a set of symbols each with an introductory rule and a set of axioms. 
Each axiom is an equality rule. We can consider the symbols plus introductory rules within a generalised algebraic theory to be analogous to the signature of an elementary theory in that
the introductory rule for a symbol defines how that symbol may properly be used, as does, within the context of an elementary theory, the arity of a function or predicate symbol within a signature.

From the given rules of a generalised algebraic theory, 
principles of derivation given in \cite{Cartmell86} enable further rules
to be derived and in this way for any generalised algebraic theory 
there is defined the set of all derived rules of the theory. 
This set of derived rules of a generalised algebraic theory is analogous to the set of provable sentences of an elementary theory.
  
Each rule is expressed as the combination of a  \term{Premise} and a \term{Conclusion} 
and these may be arranged on the page in a number of ways as suits the occasion or as befits personal preference. 
In the metamathematical arguments we prefer to write the premise above the conclusion, 
so \gatdisplayrule{Premise}{Conclusion}, but often, in examples, we write  the rule on a single line. 
If we write the premise before the conclusion then we separate by a turnstyle ($\tstyle$). 
It is also possible to foreground the conclusion by writing it before the premise and to group rules having some shared context --- the 
examples in section \ref{examples} are presented in this way.
Of course, in any informal presentation all that is important is that the formal structure shows through.  

In the syntax that we use,  that an expression $t$ represents an instance of a type represented by expression $\Delta$ is written as $\ofT{t}{\Delta}$. Other authors have prefered to use colon at this point in the syntax  and to instead write $t:\Delta$ and such a choice has the benefit of overlap with the syntax
of programming languages, which admittedly does seem appropriate. In our chosen  syntax 
the premise of a rule  takes the form $\xDelta{n}$, for $n \geq 0$, where $\xn$ is a sequence of distinct variables and each $\Delta_i$ is an expression.


Rules can be of four different forms. They either:

\begin{enumerate}[(i)]
\item Assert that an expression $\Delta$ represents a type within the context provided by the premise. For such a rule we write
\gatdisplayrule{\xDelta{n}}{\isT{\Delta}}. Such rules we have referred to as  \Trules in the past but other authors refer to them as \term{type judgements}.

\item Assert that an expression $t$ represents an instance of a type represented by an expression $\Delta$ within the context provided by the premise. Such a rule is written as
\gatdisplayrule{\xDelta{n}}{\ofT{t}{\Delta}}. Such a rule we refer to as an \trule.

\item Assert that two expressions, $\Delta$ and $\Delta'$ represents identical types within the context provided by the premise. Such a rule is written as 
\gatdisplayrule{\xDelta{n}}{\Delta=\Delta'} and we refer to as a \Teqrule.

\item Assert that two expressions, $t$ and $t'$ represent identical instances of a type represented by an expression $\Delta$  within the context provided by the premise. Such a rule is written as 
\gatdisplayrule{\xDelta{n}}{t=\ofT{t'}{\Delta}} and we  refer to it as an \teqrule.
\end{enumerate}


A \term{pretheory} is defined as a collection of symbols each with an introductory rule and a set of axioms. A pretheory is defined to be \term{well-typed}\footnote{In \cite{Cartmell86}, I define what it is for a rule written in the alphabet of a generalised algebraic theory to be \textit{well-formed}. Here I will use the adjective \textit{well-typed} instead.} iff all its introductory rules and axioms are. A \term{generalised algebraic theory} is defined to be a well-typed pretheory. 
With this revised terminology, definition 2(a) from \cite{Cartmell86}, states that in  a generalised algebraic theory $U$:
\begin{enumerate} [(i)]
\item 
a \Trule \gatdisplayrule{\xDelta{n}}{\isT{\Delta}} is well-typed  iff 
\gatdisplayrule{\xDelta{n-1}}{\isT{\Delta_n}} is a derived rule of $U$ and $x_n$ is distinct from all of $x_1,...x_{n-1}$, 
\item 
an \trule \gatdisplayrule{\xDelta{n}}{\ofT{t}{\Delta}} is well-typed iff
the rule \gatdisplayrule{\xDelta{n}}{\isT{\Delta}} is a derived rule of $U$,
\item 
a \Teqrule \gatdisplayrule{\xDelta{n}}{\Delta=\Delta'} is well-typed iff
both \gatdisplayrule{\xDelta{n}}{\isT{\Delta}} and \gatdisplayrule{\xDelta{n}}{\isT{\Delta'}} are derived rules
of $U$,
\item 
an \teqrule \gatdisplayrule{\xDelta{n}}{t=t' \in \Delta} is well-typed iff
both \gatdisplayrule{\xDelta{n}}{\ofT{t}{\Delta}} and \gatdisplayrule{\xDelta{n}}{\ofT{t'}{\Delta}} are derived rules
of $U$.
\end{enumerate}



In a theory $U$, we say that a compound expression of the form $\xDelta{n}$, where $xn$ are variables, is a \term{context} (within theory $U$)  iff
the rule \gatdisplayrule{\xDelta{n-1}}{\isT{\Delta_n}} is a derived rule of $U$. It follows (lemma \ref{contextlemma}, below)  that  $\xDelta{n}$ is a context iff
it is the premise of some derived rule of $U$.

A \term{realisation} of one context from another is defined as follows: if $\xDelta{n}$ and $\yOmega{m}$ are contexts of a generalised algebraic theory $U$  then a \term{realisation} of  $\yOmega{m}$ with respect to $\xDelta{n}$ is an $m$-tuple of expressions $\tuple{s_1,...s_m}$
such that \foreachj, the rule \gatdisplayrule{Q}{\ofT{s_j}{\Omega_j[s_1|y_1,...s_{j-1}|y_{j-1}]}} is a derived rule of $U$.

There are a number of housekeeping lemmas \commentary{for what purpose?} which are required. Some of these lemmas are summarised in \cite{Cartmell86}. 
Here it is appropriate to expand on that summary and to name some lemmas that were previously unnamed. The proofs of all of these lemmas 
	are given in \cite{Cartmell78}; proofs of some of  lemmas that we give in section \ref{sectioninwhichinstanceisdefined} structurally enrich these proofs.
In the following, assume that $U$ is a generalised algebraic theory.


\begin{lemma}[Context Lemma]
\llabel{contextlemma}
\begin{enumerate}[(i)]
\item the premise of a derived rule of $U$ is a context in $U$.
\item if $\xDelta{n}$ is a context, then \foreachi, the rule \gatdisplayrule{\xDelta{i-1}}{\isT{\Delta_i}} is a derived rule of $U$ i.e.
$\xDelta{i}$ is a context.
\end{enumerate}
\end{lemma}
\begin{proof}
(i) is proved by induction on the derivation of the rule by examining each principle of induction in turn. (ii) follows from repeated use of (i).
\end{proof}

The Substitution Lemma states that every type-correct substitutional instance of a derived rule is a derived rule:
\begin{lemma}[Substitution Lemma]
\llabel{gatsubstitutionlemma}  %label changed to make it unqiue 2 Nov 2021
For $m \geq 1$, if \gatdisplayrule{\yOmega{m}}{Conclusion} is a derived rule of $U$
and  if $\tuple{\sm}$ is a realisation of the context $\yOmega{m}$ wrt some context $Q$ 
then \gatdisplayrule{Q}{Conclusion[s_1|y_1,...s_m|y_m]} is a derived rule of $U$, where
for any expression $e$ we write
$e[s_1|y_1,...s_m|y_m]$ to mean
the result of substituting each instance of variable $y_j$ in $e$ by $s_j$, \foreachj.
\end{lemma}

The following lemma may be found on page 1-33 of \cite{Cartmell78}:
\begin{lemma}[The Well-Typedness Lemma]
\llabel{welltypednesslemma}
Every derived rule of a generalised algebraic theory $U$ is well-typed.
\end{lemma} 
\noindent and this is followed by:
\begin{lemma}[The Derivation Lemma]
\llabel{derivationlemma} 
\begin{enumerate}[(i)]
\item Every derived \Trule of a generalised algebraic theory $U$ is of the form
\gatdisplayrule{\yOmega{m}}{\isT{A(\tn)}} for some sort symbol $A$ of $U$ with introductory rule of the form
\gatdisplayrule{\xDelta{n}}{\isT{A(\xn)}} and for some expressions $\tn$ such that \foreachi, the rule
\gatdisplayrule{\yOmega{m}}{\ofT{t_i}{\Delta_i[t_1|x_1,...t_{i-1}|x_{i-1}]}} is a derived rule of $U$.
\item Every derived \trule of $U$ is 
either of the form \gatdisplayrule{\xDelta{n}}{\ofT{x_i}{\Omega}} for some $n \ge 1$, for some $i$, $1 \leq i \leq n$, 
and for some $\Omega$ such that \gatdisplayrule{\xDelta{n}}{\Delta_i=\Omega} is a derived rule of $U$
or is of the form
\gatdisplayrule{\yOmega{m}}{\ofT{f(\tn)}{\Omega}} for some operator symbol $f$ of $U$ 
with introductory rule of the form
\gatdisplayrule{\xDelta{n}}{\ofT{f(\xn)}{\Delta}} 
and for some expressions $\tn$ such that \foreachi, the rule
\gatdisplayrule{\yOmega{m}}{\ofT{t_i}{\Delta_i[t_1|x_1,...t_{i-1}|x_{i-1}]}} is a derived rule of $U$
and such that
\gatdisplayrule{\yOmega{m}}{\Delta[t_1|x_1,...t_n|x_n]=\Omega} is a derived rule of $U$.
\end{enumerate}
\end{lemma}

Lemma 4 of section 1.7 of \cite{Cartmell78} (page 1.37) may be summarised as saying that if the premise of a derived rule is weakened then the resulting rule is a derived rule. It is expressed as follows:
\begin{lemma}[The Weakening Lemma]
\llabel{weakeninglemma}
If
\gatdisplayrule{\xDelta{n}}{\isT{\Delta}}  and
\gatdisplayrule{\xDelta{n},\,\yOmega{m}}{Conclusion} are both derived rules of $U$ then if $z$ is a variable
distinct from $\xn,\ym$ then
the (weakened) rule \\
\gatdisplayrule{\xDelta{n},\,z \in \Delta,\,\yOmega{m}}{Conclusion} is a derived rule
of $U$.
\end{lemma}

The following lemma suggests an  alternative way of defining the notion of a generalised algebraic theory. 
This is Lemma 3 of section 1.7 of \cite{Cartmell78} (page 1.36):
\begin{lemma}[The Stratification Lemma]
\llabel{stratificationlemma}
 For each generalised algebraic theory $U$  there is a sequence of theories 
$U_0 \subseteq $U$_1 \subseteq $U$_2 \subseteq ...$ such that  \inlinedisplay{$U$ = \bigcup_i $U$_i}
and such that each $U_{i+1}$ is a simple extension of $U_i$ in the sense that each introductory rule and axiom of $U_{i+1}$ is a well-typed  with respect to $U_i$.
\end{lemma}
This tells us that those pretheories that are well-typed (i.e. those that are generalised algebrauc theories) are those that can be expressed as the union of a chain of theories in which each theory is provably well-typed with respect to the previous theory in the chain.  
Expressing generalised algebraic theories by way of such chains would be a good approach to follow
for any environment for managing and machine checking generalised algebraic  rules and derivations.

\begin{example}
If $U$ is a single or multi-sorted algebraic theory considered as an example of a
 generalised algebraic theory then $U$ can be stratified as $U_0 \subseteq U_1 \subseteq U_2=U$
where $U_0$ defines the sort symbol(s), $U_1$ defines the operator symbols and $U_2$ defines the axioms. 
\end{example}

\begin{example}
 The generalised algebraic theory $cc$ of categories stratifies as: $cc_0$ - sort Ob,
$cc_1$ - sort Hom, $cc_2$ - operator symbols $\circ$ and $id$, $cc_3$ - identity and associativity axioms.
\end{example} 

 Finally, note that there is a compactnessl lemma for generalised algebraic theories:
\begin{lemma}[The Compactness Lemma]
\llabel{compactnesslemma}
For each generalised algebraic theory $U$, if r is a derived rule of $U$ then it is a derived rule of some finite subtheory $F \subseteq U$. \
\end{lemma}




%\iffalse
\fi
\section{Contextual Categories }

\label{contextualnotationpartone}
\subsection{The $s$ operator and other operators}
\note 
Terminology: By  the generic term \term{tree} is meant a partially ordered set (poset) $(T, <)$ such that for each $t \in T$, the set $\set{s \in T : s < t}$ is well-ordered by the relation $<$.
In this discussion we restrict ourselves to rooted $\omega$-trees i.e. trees for which the set $\set{s \in T : s < t}$
is finite for all $t \in T$ and for which there is a least element in the partial ordering. 

With respect to a partial ordering $<$, we say that an element $y$ \textit{covers}  an element $x$  iff $x<y$ and there does not exist $w$ such that $x < w$ and $w < y$.
If object $y$ covers object $x$ in the partial ordering 
then we write $x \base y$ (we use this in preference to the more usual $x \lessdot y$).


\note We define the rank (sometimes called the grade) of an element $t \in T$ to be the cardinality
of the set $\setsuchthat{s \in T}{s < t}$. If we define the set $T_i$ to be the set of elements of a tree
of rank $i$ then we have that $T= \bigcup_{i \in N}T_i$. 

\note In the  definition of contextual categories (\cite{Cartmell78,Cartmell86}) there is defined to be such a tree-structure on the objects of the category. In a contextual category the root of the tree of objects is also  a terminal object
of the category and is denoted $1$. For $x$ an object of the category we define the set of objects  $Cover(x)$ to be the set of objects covering $x$.

\note
By a \term{tree-structured category} we mean (i) a category with a tree-structure defined on its objects such that the tree of objects has a unique root object and (ii) for every $x \base y$ in the tree of objects  a canonical morphism $p_y:y \rightarrow x$.  I shall say morphisms of this form  are \term{direct dependency morphisms} and they will
be distinguished in diagrams by an arrow with  a triangular head so:
\begin{center}
$
\begin{array}{p{2cm}}
\Rnode{y}{y}\\ [1.4cm]
\Rnode{x}{x} \\
\mbox{\ncbsar{p_y}{y}{x}}
\end{array}
$
\end{center}

\note
If $x$ is an object of a tree-structured category \catcw and if $y \in Cover(x)$ in \catcw then we define 
the set  of sections of $y$, denoted $Sect(y)$, to be the set of morphisms $s: x \morph y$ in \catcw  such that $s \circ p_y = id_x$. We have therefore that if $x$ is a section of $y$ and $y$ covers $x$ then
\ \ \ \ 
\begin{tabular}{cccc}
$
\begin{array}{p{0.5cm}}
\Rnode{y}{y} \\ [1.4cm]
\Rnode{x}{x} \\
\mbox{
\ncsar{y}{x}
\alabel{p_y}
\ncarrZZ[30]{x}{y} 
\alabel{s}}
\end{array}
$  & in \catcw and &
$
\begin{array}{c p{0.5cm}c p{0.5cm}c}
              && \Rnode{y}{y}&&                \\ [1.4cm]
\Rnode{x1}{x} &&             &&   \Rnode{x2}{x}\\
\mbox{
\ncsar{y}{x2}
\alabel{p_y}
\ncarr{x1}{y} 
\alabel{s}
\ncarr{x1}{x2} 
\blabel{id_x}
}
\end{array}
$& commutes.
\end{tabular}

\note
In the  definition of contextual category given  in [1] and [2], a contextual category is defined to be a tree-structured category 
\cat{C} with the following additional structure:

\noindent 
(i) whenever
$
\begin{array}{cp{.9cm}c}
            & & \Rnode{z}{z} \\ [1.2cm]
\Rnode{x}{x}& & \Rnode{y}{y} \\ [0.5cm]
\end{array}
$
\jcbarr{f}{x}{y}
\ncasar{p_z}{z}{y}
in \cat{C}, an object $f \sub z$ such that $x \base f \sub z$, a morphism $q(f,z): f \sub z \rightarrow z$ such that

\begin{axiom}{q1}
q(f,z) \circ p_z = p_{f \sub z} \circ f
\end{axiom}

\noindent i.e. such that the diagram: \ \ \ 
$
\ccsquareoutline{0.9cm}{1.2cm}{f^*z}{z}{x}{y}
\ccsquareacross{q(f,z)}{f}
\ccsquaredown{p_{f \sub z}}{p_z}
$
commutes and, (ii), is a pullback diagram, that is: \\
\hspace{0.2cm}

\noindent for all objects $w$ of \cat{C}, and for all
morphisms $h_1: w \rightarrow x$ and $h_2: w \rightarrow z$ such that
$h_1 \circ f = h_2 \circ p_z$ 
there exists a unique $h:w \rightarrow f \sub z$ in \cat{C} such that
$h \circ p_{f \sub z} = h_1$ and $h \circ q(f,z) = h_2$, as shown here:

\vspace{3mm}
\begin{center}
\begin{equation}
\label{pullback}
\begin{array}{cp{0.5cm}cp{1.2cm}c}
\Rnode{w}{w} &&                     &&           \\ [0.7cm]
             &&\Rnode{fstarz}{f^*z} && \Rnode{z}{z}\\ [1.2cm]
             &&\Rnode{x}{x}         && \Rnode{y}{y}
\end{array}
\end{equation}
\ncbsar{p_{f \sub z}}{fstarz}{x}
\jcbarr{f}{x}{y}
\ncaarr{q(f,z)}{fstarz}{z}
\ncasar{p_z}{z}{y}
\setlength{\arrnodesepA}{3pt}
\jcbarr[-35]{h_1}{w}{x}
\ncaarr[35]{h_2}{w}{z}
\psset{linestyle=dashed}
\ncaarr{h}{w}{fstarz}
\end{center}

\vspace {0.25cm}
\noindent and so that (iii) whenever $x \base y$ in \cat{C}, 
\begin{axiom}{q2}
id_x^*y=y
\end{axiom}

and

\begin{axiom}{q3}
q(id_x,y) = id_y
\end{axiom}



\noindent and (iv) whenever 
$
\begin{array}{c p{.9cm} c p{.9cm} c}
             &   &             &   & \Rnode{z}{z} \\ [1.2cm]
\Rnode{w}{w} &   &\Rnode{x}{x} &   & \Rnode{y}{y} \\ [0.5cm]
\end{array}
$
\jcbarr{f}{w}{x}
\jcbarr{g}{x}{y}
\ncasar{c}{z}{y}
in \cat{C}, 

then

\begin{axiom}{q4}
(f \circ g)^*z =  f^* (g ^* z)
\end{axiom}

and 
\begin{axiom}{q5}
q(f \circ g,z) = q(f,g^*z) \circ q(g,z)
\end{axiom}



\note
Following Voevodsky we may replace the pullback condition (condition (ii) above) by introducing and axiomatising an 
`s' operator  as follows:

\noindent (ii') for all morphisms $f: x \rightarrow y$, a morphism $s(f) : x \rightarrow f \sub p_y \sub y$ such that both:

\begin{axiom}{s1}
s(f) \circ p_{f\sub p_y \sub y}=id_x
\end{axiom}

\noindent and

\begin{axiom}{s2}
s(f) \circ q( f \circ p_y     ,y)=f
\end{axiom}	

\noindent i.e. such that the following diagrams commute:
\begin{center}
\begin{displaymath}
\begin{array}{cccp{1.cm} cp{.9cm}c}
&\Rnode{fXyyM}{f\sub p_y \sub y}&  & &  \Rnode{fXyy}{f\sub p_y \sub y} & & \Rnode{yXy}{p_y \sub y}\\ [1.2cm]
\Rnode{xL}{x} & &\Rnode{xR}{x} & &\Rnode{x}{x}         & & \Rnode{y}{y}
\end{array}
\end{displaymath}
\ncasar{p_{f\sub p_y \sub y}}{fXyy}{x}
\jcbarr{f}{x}{y}
\ncaarr{q(f,p_y \sub y)}{fXyy}{yXy}
\ncasar{p_{p_y \sub y}}{yXy}{y}
\ncaarr{s(f)}{xL}{fXyyM}
\ncasar{p_{f\sub p_y \sub y}}{fXyyM}{xR}
\jcbarr{id_x}{xL}{xR}
\end{center}

\noindent
and such that whenever

\begin{center}
\begin{displaymath}
\begin{array}{c p{.9cm} c p{.9cm} c}
\Rnode{w}{w}&& \Rnode{g*z}{g \sub z} && \Rnode{z}{z} \\ [1.2cm]
            && \Rnode{x}{x}  && \Rnode{y}{y} \\ [0.2cm]
\end{array}
\end{displaymath}
\jcbarr{f}{w}{g*z}
\jcbarr{g}{x}{y}
\ncaarr{q(g,z)}{g*z}{z}
\ncasar{}{g*z}{x}
\ncasar{}{z}{y}
\end{center}

\noindent in \cat{C} then

\begin{axiom}{s3}
s(f \circ q(g,z))=s(f)
\end{axiom}


\note We can introduce several additional notations for use in in contextual categories. 
If $x < y$ in the contextual category \catc, then define the morphism $p_{y,x}:y \morph  x$ in \catc, \\

\begin{tabular}{c c c  c  c  c c}
by defining
& %2 c
$
\begin{array} {c}
\Rnode{midy}{y} \\[2.0cm]
\Rnode{midx}{x}  \\ 
\end{array}
\mbox{\ncarr{midy}{midx}
      \blabel{p_{y,x}}[0.2]
		 }
$
& %3 c
(drawn also  as
& %4 c
$
\begin{array} {c}
\Rnode{lhsy}{y} \\[2.0cm]
\Rnode{lhsx}{x} 
\end{array})
\makebox[0.1cm]{\nccdar{lhsy}{lhsx}
      \blabel{p_{y,x}}[0.275]
		}
$
& %5
 as the composition 
& %6 c
$
\begin{array}{c}
%\Rnode{b}{B}&&\Rnode{xn}{w_n}&&\Rnode{xn1}{w_{n-1}}&&\Rnode{dots}{\ ...\ }&&\Rnode{x1}{w_1}&&\Rnode{a}{A} 
\Rnode{b}{y}\\[0.7cm]
\Rnode{xn}{w_n}\\[0.7cm]
\Rnode{xn1}{w_{n-1}}\\[0.1cm]
\Rnode{dots}{\vdots}\\[0.1cm]
\Rnode{x1}{w_1}\\[0.7cm]
\Rnode{a}{x} 
\end{array}
,
\makebox[0.1cm]{
\ncsar{b}{xn}
\alabel{p_y}
\ncsar{xn}{xn1}
\alabel{p_{w_n}}
\ncsar{xn1}{e1}
\ncline[linestyle=dotted,dotsep=4pt]{e1}{e2}
\ncsar{e2}{x1}
\ncsar{x1}{a}
\alabel{p_{w_1}}}
$ 
& %7 c

\end{tabular}

\noindent where $w_1, ... w_n$ is the unique sequence of objects of $C$ such that 
$x \base w_1 \base ... \base w_n \base y$. It is also useful to have a defintion of $p_{x,x}$ as the identity morphism
$id_x$, for any object $x$, so that $p_{y,x}$ is defined for any pair of objects $x$ and $y$ such that $x \leq y$.
We say that any morphism  of the form $p_{y,x}$ is a dependency morphism. 

\note
The contextual category structure supplies us with pullbacks for direct dependency morpisms.
These can be pieced together to obtain  a pullback for any indirect dependency morphism   along any morphism with common codomain
as  whenever $y \leq z$ in $C$ and whenever $f:x \morph y$ in $C$, then we have
the following canonical pullback for the morphism $p_{z, y}$ along $f$, where
$w_1, ... w_n$ is the unique sequence of objects of $C$ such that 
$y \base w_1 \base ... \base w_n \base z$:

\vspace{3mm}
\begin{center}
\begin{equation}
\label{compositepullbackdefinition}
\begin{array}{cp{2.9cm}c}
\Rnode{TOPL}{q(...q(f, w_1)...w_n)^* z} & & \Rnode{TOPR}{z}\\ [1.2cm]
\Rnode{zOTTOML}{x}         & & \Rnode{zOTTOMR}{y}
\end{array}
\end{equation}
\jcbarr{f}{zOTTOML}{zOTTOMR}
\ncaarr{q(q(...q(f,w_1)...w_n),z)}{TOPL}{TOPR}
\nccdar{TOPL}{zOTTOML}
\blabel{p_{q(...q(f, w_1)...w_n)^* z,x}}
\nccdar{TOPR}{zOTTOMR}
\alabel{p_{z,y}}
\end{center}

Since these constructed pullbacks form an important part of contextual
category structure we would like a simpler notation for them. As no confusion is
likely, we extend the $^*$ and $q$ notation to cover these new pullback diagrams.
From now on if $f:x \morph y$ in $C$ and $y \leq z$ in $C$, then $f^*z$ 
is defined to be $q(...q(f, w_1)...w_n)^* z$ as shown in diagram (\ref{compositepullbackdefinition}) 
and $q(f,z)$ is defined as the secondary projection shown in (\ref{compositepullbackdefinition}), i.e. as 
$q(q(...q(f,w_1)...w_n),z)$,
so that (\ref{compositepullbackdefinition}) can be rewritten as:

\vspace{3mm}
\begin{center}
\begin{equation}
\label{compositepullbackout}
\begin{array}{cp{.9cm}c}
\Rnode{fstarz}{f^*z} & & \Rnode{z}{z}\\ [1.2cm]
\Rnode{x}{x}         & & \Rnode{y}{y}
\end{array}
\end{equation}
\nccdar{fstarz}{x}
\blabel{p_{f \sub z},x}
\jcbarr{f}{x}{y}
\ncaarr{q(f,z)}{fstarz}{z}
\nccdar{z}{y}
\alabel{p_{z,y}}
\end{center}
\note
With respect to any pullback diagram (\ref{compositepullbackout})
 we use the notation $\tuple{h_1,h_2}_{x,y,f,z}$  for the unique morphism   
$h:w \rightarrow f \sub z$ in \cat{C} such that
$h \circ p_{f \sub z} = h_1$ and $h \circ q(f,z) = h_2$.\\
$\tuple{h_1,h_2}_{x,y,f,z}$ may be be safely elided to $\tuple{h_1,h_2}_{f,z}$ and, rather less safely, to $\tuple{h_1,h_2}$.
In  elided form we  have 
\begin{equation}
\mbox{$\tuple{h_1,h_2}_{f,z} \circ p_{f^*z} = h_1$}
\end{equation}
and
\begin{equation}
\mbox{$\tuple{h_1,h_2}_{f,z} \circ q(f,z) = h_2$}
\end{equation}

\note
The following observation
follows from the way the extended pullback diagrams are constructed. 
In the extended notation, if $f: x \morph y$ 
and $y \leq z \leq zz$ in the contextual category C, then
\begin{equation}
f^*zz = q(f, z)^*zz
\end{equation}
 and 
\begin{equation}
q(f, zz) = q(q(f, z), zz)
\end{equation}
and so the outer diagram in
\renewcommand{\pc}[2]{p_{#1,#2}}  % as \pc defined in ccategories macros differently to this
$
\begin{array}{ccp{.9cm}c}
\\[0.25cm]
&\Rnode{TL}{q(f,z)^*zz} & & \Rnode{TR}{zz}\\ [1.2cm]
&\Rnode{ML}{f^*z} & & \Rnode{MR}{z}\\ [1.2cm]
&\Rnode{BL}{x}         & & \Rnode{BR}{y} \\[1.0cm]
\end{array}
$
%composition
\makebox[0.2cm]{   % This make box prevents white space pushing out to the right
                   % cannot see where this white space is comin from. To investigate
									 % change the \makebox[0.2cm] to \fbox and you will see the problem.
\nccdar{TL}{ML}\blabel{P_{q(f,z)^*zz,f^*z}}\nccdar{ML}{BL}\blabel{p_{f \sub z,x}}\nccdar{TR}{MR}\alabel{p_z}
\nccdar{MR}{BR}
}
\alabel{p_z}
%reference
\ncarr{TL}{TR}
\alabel{q(q(f,z),zz)}
\ncarr{ML}{MR}
\alabel{q(f,z)}
\ncarr{BL}{BR}
\blabel{f}
is diagram (\ref{compositepullbackout}). 

\note
We shall write $\crossx{y}{z}{x}$ in place of ${p_{y,x}}^*z$, for $x < y$, $x < z$  in \catc. 
Note that
$\crossx{y}{z}{x}$  represents what in the generalised algebraic syntax  is the `weakening' of a rule of the form
\begin{displaymath}
x,\, w_1,...w_n \tstyle \isT{z}
\end{displaymath}
from a rule with context $x, w_1,...w_n$ to a rule with broader context $y, w_1, ... w_n$: 
\begin{displaymath} 
y,\, w_1,...w_n \tstyle \isT{z}.
\end{displaymath}

\noindent Within the contextual category we can think of $\crossx{y}{z}{x}$  as a local cartesian product but of course categorically it is a filtered product i.e. a pullback --- if $w < x$ and $w < y$  then 
\genericcrossxproductdiagram % defined in 'paper.tex'
is a pullback diagram in \ccat.

\note
We can extend the $\crossx{}{}{w}$ notation to morphisms. If $f:x \morph x'$ and $g: y \morph y'$ in a contextual
category $\ccat[C]$ and if $w$ is an object such that $w < x$, $w <x'$, $w < y$ and $w < y'$ then 
define $\crossx{f}{g}{w}:\crossx{x}{y}{w} \morph \crossx{x'}{y'}{w}$ in $\ccat[C]$ by
\begin{equation}
\crossx{f}{g}{w} = \tuple{p_{\crossx{x}{y}{w}, x} \circ f,q(p_{x,w},y) \circ g}
\end{equation}
as shown here:  
\begin{equation*}
\begin{array}{c p{1.2cm} c  p{0.2cm} c p{0.2cm} c p{1.8cm} c}
\Rnode{xy}{\crossx{x}{y}{w}} && \Rnode{xpyp}{\crossx{x'}{y'}{w}} \\[1.4cm]
\Rnode{x}{x}                 && \Rnode{xp}{x'}                   &&&& \Rnode{y}{y} && \Rnode{yp}{y'} \\[1.8cm]
                             &&                         && \Rnode{w}{w}  &&    &&      
\makebox[0cm]{
\nccdar{xy}{x}
\nccdar{xpyp}{xp}
\nccdar{x}{w}
\nccdar{xp}{w}
\nccdar{y}{w}
\nccdar{yp}{w}
\ncarr{x}{xp}
\alabel{f}
\ncarr{y}{yp}
\alabel{g}[0.35]
\ncarr{xy}{y}
\alabel{q(p_{x,w},y)}[0.7][0.2]
\ncarr{xpyp}{yp}
\alabel{q(p_{x',w},y')}[0.5][0.2]
\ncdarr[10]{xy}{xpyp}
%\alabel{\tuple{p_{\crossx{x}{y}{w}, x} \circ f,q(p_{x,w},y) \circ g}}[0.5][2]
\alabel{\crossx{f}{g}{w}}%[-0.1][15]
}
\end{array}
\end{equation*}

In the special case that $x=x'$ and $f=id_x$ then we get  $\crossx{id_x}{g}{w} : \crossx{x}{y}{w} \morph \crossx{x}{y'}{w}$. 
This morphism we abbreviate as $\crossx{x}{g}{w}$. 

\note
In the   case that $w$ is the terminal object $1$ then the pullback  specialises to give a product diagram:

\begin{displaymath}
\begin{array}{ccccc}
\Rnode{xy}{\crossx{x}{y}{1}} &&               &&               \\[1.3cm]
\Rnode{x}{x}                 &&               && \Rnode{y}{y}  \\                                    
\end{array}
\mbox{\ncsar{xy}{x}
\blabel{p_{\crossx{x}{y}{1},x}}
\ncaarr{q(p_{x,1},y)}{xy}{y}}
\end{displaymath}

In this special case the $\tuple{}$ operation defined earlier is the pairing operation for if
$f: w \morph x$ and $g: \morph y$ then $\tuple{f,g}: w \morph \crossx{x}{y}{1}$ 
and 
\begin{equation}
\tuple{f,g} \circ p_{\crossx{x}{y}{1},x} = f
\end{equation}
and
\begin{equation}
\tuple{f,g} \circ q(p_{x,1},y) = g
\end{equation}

\note 
Note that the product operation $\crossx{}{}{1}$ is far from symmetric 
because if, for example, $1 \base x$ and $1 \base y$ then $x \base \crossx{x}{y}{1}$ and $y \base \crossx{y}{x}{1}$. We can define 
a swap operation $sw_{x,y} : \crossx{x}{y}{1} \morph \crossx{y}{x}{1}$ by
\begin{equation}
sw_{x,y} = \tuple{q_{p_x,y}, p_{\crossx{x}{y}{1},x}}
\end{equation}

\note
Associativity of $\crossx{}{}w$  follows from the coherence property of the pullbacks in the contextual category. 
For example if $w < x$, $w < y$, $w < z$ in a \ccat then from coherence of pullbacks in \ccat we have:
$\crossx{x}{(\crossx{y}{z}{w})}{w} = \crossx{(\crossx{x}{y}{w})}{z}{w}$ as shown here in this diagram:
 
\begin{displaymath}
\begin{array}{cp{1.0cm}cp{1.0cm}c}
\Rnode{J1}{}\Rnode{D1} {\crossx{(\crossx{x}{y}{w})}{z}{w}}\Rnode{J2}{} \ \ \ \ \   &&  &&  \\ 
= && && \\
\Rnode{D2} {\crossx{x}{(\crossx{y}{z}{w})}{w}}    &&  &&                        \\ [1.3cm]
\Rnode{xy}{\crossx{x}{y}{w}}&& \Rnode{yz}{\crossx{y}{z}{w}} &&                      \\[1.3cm]
\Rnode{x}{x}&& \Rnode{y}{y} && \ \ \ \ \ \ \ \ \ \ \ \ \ \Rnode{z}{z}                                        \\[1.3cm]
             && \Rnode{w}{w} &&                                                     
\end{array}
\end{displaymath}

\ncaarr[50]{q(\pc{\crossx{x}{y}{w}}{w},z)}{J2}{z}
\ncsar{D2}{xy}
\ncsar{xy}{x}
\ncsar{yz}{y}
\ncsar{x}{w}
\ncsar{y}{w} 
\ncsar{z}{w}
\ncaarr{q(\pc{x}{w},y)}{xy}{y}
\ncaarr{q(\pc{y}{w},z)}{yz}{z}
\ncaarr{q(\pc{x}{w},\crossx{y}{z}{w})}{D2}{yz}



\note In the examples we give later we make use of the following lemmas:

\begin{lemma}
\llabel{footandstactic}
If $f: A \morph B$ and $f':A \morph B$ in a contextual category \catcw then if 
$f \circ p_B$ = $f' \circ p_B$ and $s(f) = s(f')$ then $f=f'$.
\end{lemma}
\begin{proof}
Follows by axiom (s2) since we have:
$f = s(f) \circ q(f \circ p_B,B)  = s(f') \circ q(f' \circ p_B,B) = f'$.
\end{proof}

From which follows:
\begin{lemma}
\llabel{stactic}
If $A$ is any object of a contextual category \catcw and if $B$ is an object such that $1 \base B$ in \catcw then
if $f: A \morph B$ and $f':A \morph B$ in  \catcw then $f=f'$ iff $s(f) = s(f')$.
\end{lemma}

\begin{lemma}
\llabel{crosssectionlemma}
If 
%\begin{equation*}
$
\begin{array}{ c c c}
\Rnode{B}{B} &              & \Rnode{Bp}{B'} \\[1cm]
             & \Rnode{A}{A} &     
\mbox{\ncsar{B}{A}
\ncsar{Bp}{A}
%\ncarr[-30]{A}{Bp}
\ncrightsimplesection{A}{Bp}
\blabel{g}}
\end{array}
$
%\end{equation*}
in a contextual category \catcw and if $g$ is a section of $B'$ (i.e. if $g \circ p_{B'}= id_A$) so that we have 
\begin{equation*}
\begin{array}{ c c c}
\Rnode{BBp}{\crossx{B}{B'}{A}} \\[1.3cm]
\Rnode{B}{B} &              & \Rnode{Bp}{B'} \\[1.1cm]
             & \Rnode{A}{A} &
\mbox{
\ncsar{BBp}{B}
\ncrightcrosssection{B}{BBp}
\blabel{\crossx{B}{g}{A}}
\ncsar{B}{A}
\blabel{p_B}
\ncsar{Bp}{A}
\ncrightsimplesection{A}{Bp}
\blabel{g}
\ncarr[30]{BBp}{Bp}
\alabel{q(p_B,B')}
}														
\end{array}
\end{equation*}
in \catcw,  then
\begin{equation}
\label{crosssectionlemmatarget}
\crossx{B}{g}{A} = s(p_B \circ g).
\end{equation} 
\end{lemma}
\begin{proof}
$\crossx{B}{g}{A}$ is defined to be the unique section of $\crossx{B}{B'}{A}$ such that $(\crossx{B}{g}{A}) \circ q( p_B,B') = p_B \circ g$ and
so (\ref{crosssectionlemmatarget}) follows because $s(p_B \circ g)$ issuch a section, since it is defined to be the unique section of $(p_B \circ g \circ p_{B'}) ^* B'$
such that $s(p_B \circ g) \circ q( p_B \circ g \circ p_{B'}, B') = p_B \circ g$ and this simplifies,
because $g \circ p_{B'} =id_A$, to
 $s(p_B \circ g)$ being a section of ${p_B} ^* B'$ (i.e of $\crossx{B}{B'}{A}$) satisfying $s(p_B \circ g) \circ q( p_B,B') = p_B \circ g$. 
\end{proof}

% *************************************************************************************
% sfglemma ****************************************************************************
\begin{lemma}
\llabel{sfglemma}
If $f:A \morph B$ and $g:B\morph C$ in a contextual category \catcw and if $C \in Cover(B)$ then
\begin{equation}
\label{sgflemmagoalone}
ft(f\circ g)^*C = f^*(ft(g)^*C)
\end{equation}
and 
\begin{equation}
\label{sgflemmagoaltwo}
s(f\circ g)=f^*s(g)
\end{equation}
where $ft(g) = g \circ p_B$ and  $ft(f \circ g) = f \circ g \circ p_B$ so that we have
\begin{displaymath}
\begin{array}{ccp{1.7cm}cp{1.7cm}c}
ft(f\circ g)^*C=\kern-10pt&\Rnode{TL}{f^*(ft(g)^*C)} & & \Rnode{TC}{ft(g)^*C}          \\ [1.7cm]
&\Rnode{BL}{A}         & & \Rnode{BC}{B} && \Rnode{BR}{C}
\end{array}
\mbox{
\ncsar{TL}{BL}
\ncsar{TC}{BC}
\ncarr{BL}{BC}
\blabel{f}
\ncarr{BC}{BR}
\blabel{g}
\ncarr{TL}{TC}
\alabel{q(f,ft(g)^*C)}
\ncarr{TC}{BR}
\alabel{q(ft(g),C)}
\ncleftsimplesection{BL}{TL}
\alabel{s(f\circ g)=f^*s(g)}
\ncleftsimplesection{BC}{TC}
\alabel{s(g)}
}
\end{displaymath}
in \catc.
\end{lemma}
\begin{proof}
That (\ref{sgflemmagoalone}) holds follows by axiom (q4).

To show that (\ref{sgflemmagoaltwo}) holds remember that $s(f \circ g)$is the unique section of $ft(f \circ g)^*C$
such that $s(f \circ g) \circ q(f \circ g \circ p_C,C) = f\circ g$. Therefore it suffices to show that
$f^*s(g) \circ q(f \circ g \circ p_C,C) = f\circ g$ and this we can show as follows:
\begin{align*}
(f^*s(g)) \circ q(f \circ g \circ p_C,C) &= (f^*s(g)) \circ q(f ,(g \circ p_C)^*C) \circ q(g \circ p_C ,C) &&\mbox{by axiom (q5)} \\
                             &= f \circ s(g) \circ q(g \circ p_C ,C)                   &&\mbox {from defn. of $f^*s(g)$}\\
														 &= f \circ g                                              &&\mbox {by axiom (s2)}
\end{align*}
\end{proof}

\subsection{Cascades of sections}

We will find the following definition and accompanying lemma useful when reasoning about the interpretations of theories. 
\begin{definition}
If $a$ is any object of a contextual category \catcw and if $1 \base b_1 ... \base b_n$ in \catc, for some $n \ge 1$, 
then define a \term{cascade} from $a$ to $b_n$ to consist of an n-tuple of sections $\fn$ of \catc, such that \foreachi, 
$f_i \in Sect(\fipvectorstar(\crossx{a}{b_i}{1}))$.
\end{definition}

\commentary{Do I need a lemma to say that ^* maps section to sections?}
\begin{lemma}
\llabel{cacscadelemma}
If $a$ is an object of a contextual category \catc, if $1 \base b_1 ... \base b_n$ in \catc and if $f_1,...f_n$ is a cascade from $a$ to $b_n$ in  \catcw then \foreachi, $a \base \fipvectorstar(\crossx{a}{b_i}{1}$ in \catc.
Additionally if $b$ is some object of \catcw such that $b_n \base b$ in \catcw then and $a \base \fnvectorstar(\crossx{a}{b}{1})$ in \catc and
if $g$ is a section of $b$ then $\fnvectorstar(\crossx{a}{g}{1})$ is a section of $\fnvectorstar(\crossx{a}{b}{1})$.
\end{lemma}
\begin{proof}
From the definition of $\crossx{}{}{1}$ it follows that $a \base \crossx{a}{b_1}{1} \base \crossx{a}{b_2}{1} ... \base \crossx{a}{b_n}{1} \base \crossx{a}{b}{1}$ in \catc. Now since $f_1 \in Sect(\crossx{a}{b_1}{1})$ it follows from the definition of the extended $^*$ notation
that $a \base f_1^*(\crossx{a}{b_2}{1})  ... \base f_1^*(\crossx{a}{b_n}{1}) \base f_1^*(\crossx{a}{b}{1})$ in \catc.

Similarly, since $f_2 \in Sect(f_1^*(\crossx{a}{b_2}{1}))$ it follows 
that $a \base f_2^*f_1^*(\crossx{a}{b_3}{1})  ... \base f_2^*f_1^*(\crossx{a}{b_n}{1}) \base f_2^*f_1^*(\crossx{a}{b}{1})$ in \catc.

We see (by induction,  if we were to be formal about it) that \foreachi, $f_i \in Sect(\fipvectorstar(\crossx{a}{b_i}{1}))$
and that $a \base \fnvectorstar(\crossx{a}{b}{1})$ in \catc. In fact we see that
we have the following objects and morphisms in \catc:
\newcommand{\ncdotdotdot}[2]
{\ncline[linestyle=none]{#1}{#2} 
 \ncput[nrot=:U]{\Large$ \hdots$}
}
\begin{displaymath}
\begin{array}{c  c p{0.4cm} c p{0.2cm} c p {0.2cm} c  p{0.5cm} c}
&&&                                               &&                                           && \Rnode{ab}{\crossx{a}{b}{1}}    &&                \\[1.2cm]
&&&                                               &&  \Rnode{f1ab}{\fonestar(\crossx{a}{b}{1})}
%\rule[-1cm]{3pt}{1pt}
&& \Rnode{abm}{\crossx{a}{b_m}{1}} &&                \\[1.2cm]
&&&                                               &&  \Rnode{f1abm}{\fonestar(\crossx{a}{b_m}{1})}&&                              &&                \\[0.1cm]
&&&                                               &&                                           && \Rnode{ab3}{\crossx{a}{b_3}{1}} &&                \\[1.2cm]
&\Rnode{fm1axb}{\fmonestar...\ftwostar\fonestar(\crossx{a}{b}{1})}&& &&\Rnode{f1axb3}{\fonestar(\crossx{a}{b_3}{1})}  && \Rnode{ab2}{\crossx{a}{b_2}{1}}  &&           \\[1.2cm]
\Rnode{ftarget}{\fmstar...\ftwostar\fonestar(\crossx{a}{b}{1})}&\Rnode{fmtarget}{\fmonestar...\ftwostar\fonestar(\crossx{a}{b_m}{1})}&&
\Rnode{f3target}{\ftwostar\fonestar(\crossx{a}{b_3}{1})} &&\Rnode{f2target}{\fonestar(\crossx{a}{b_2}{1})}  && \Rnode{ab1}{\crossx{a}{b_1}{\Rnode{f1target}{1}}}     \\[1.2cm]
&&&                                               &&                                           &&                                                       \\[-6.4cm] %%% HEE HEE HE
&&&																								&&                                           &&                         && \Rnode{b}{b}                \\[1.2cm]
&&&																								&&                                           &&                         && \Rnode{bm}{b_m}             \\[0.3cm]
&&&                                               &&                                           &&                         &&                             \\[0.3cm]
&&&																								&&                                           &&                         && \Rnode{b3}{b_3}             \\[1.2cm]
&&&																								&&                                           &&                         && \Rnode{b2}{b_2}             \\[1.2cm]
&&&																								&&                                           &&                         && \Rnode{b1}{b_1}             \\[0.3cm]
&&&		\ovalnode[linestyle=none]{a}{a}					    &&                                           &&                         &&                             \\[1.1cm]
&&&                                               &&                                           && \Rnode{abs}{1} \ \ \ \ \ \ \ \ &&                      \\           
\makebox[0cm]{
\ncarr{ab}{b}
\ncarr{abm}{bm}
\ncarr{f1ab}{ab}
\ncarr{f1abm}{abm}
\ncarr{ab3}{b3}
\ncarr{ab2}{b2}
\ncarr{ab1}{b1}
\ncarr{f1axb3}{ab3}
\ncarr{f2target}{ab2}
\ncarr{f3target}{f1axb3}
\ncarr{ftarget}{fm1axb}
\ncdotdotdot{fm1axb}{f1ab} 
\ncdotdotdot{fmtarget}{f1abm}
\ncdotdotdot{fmtarget}{f3target}
%
\ncarc[arcangle=-5,nodesepA=15pt,offsetA=-2pt,nodesepB=3pt,offsetB=-5pt]{->}{a}{f1target}
\blabel{f_1}[0.6]
\ncarc[arcangle=10,nodesepA=15pt,offsetA=1pt,nodesepB=2pt,offsetB=2pt]{->}{a}{f2target}
\alabel{f_2}
\ncarc[arcangle=10,nodesepA=15pt,offsetA=1pt,nodesepB=2pt,offsetB=2pt]{->}{a}{f3target}
\alabel{f_3}
\ncarc[arcangle=7,nodesepA=15pt,offsetA=1pt,nodesepB=2pt,offsetB=2pt]{->}{a}{fmtarget}
\alabel{f_m}
\ncarc[arcangle=7, nodesepA=15pt,offsetA=1pt,nodesepB=2pt,offsetB=2pt]{->}{a}{ftarget}
\alabel{f}

\setlength{\sarnodesepB}{10pt}
\ncsar{fmtarget}{a}
\ncsar{ftarget}{a}
\ncsar{f3target}{a}
\ncsar{f2target}{a}
\ncsar{f1target}{a}
\sarreset
\ncsar{fm1axb}{fmtarget}

%left but two tower
\ncsar{f1ab}{f1abm}
\ncdotdotdot {f1abm}{f1axb3}
\ncsar{f1axb3}{f2target}
%left but one tower
\ncsar{ab}{abm}
\ncdotdotdot{abm}{ab3}
\ncsar{ab3}{ab2}
\ncsar{ab2}{ab1}
%left tower
\ncsar{b}{bm}
\ncdotdotdot{bm}{b3}
\ncsar{b3}{b2}
\ncsar{b2}{b1}
\ncsar{b1}{abs}
\nccdar{a}{abs}
}
\end{array}
\end{displaymath}

\end{proof}


%\fi
%\iffalse
%\section{Meta-GAT algebras}  THIS MATERIAL MOVED INTO PREVIOUS SECTION
%\input{metaGATalgebras} 
%\fi                         % END IF
\iffalse
\section{Instances of Generalised Algebraic Theories in Contextual Categories}
\label{sectioninwhichinstanceisdefined}

\newcommand{\clause}[1]{clause (#1) of definition \lref{consistentinterpretation}}
\newcommand{\condition}[2]{condition (#2) of \clause{#1}}

In this section, we will define an instance of a generalised algebraic theory $U$ in a contextual category \catcw to be a consistent mapping 
of derived T-rules and $\in$-rules of  $U$ to objects, respectively sections of \catc. 
We will show that such instances are totally defined by their mapping of the introductory rules of the theory and we will define how to extend 
consistent and valid interpretations $\iI$ of the introductory rules   to instances $\Ibar$ that consistently map all the derived  T-rules and $\in$-rules.
To make this work, we need define $\Ibar$ as a partial mapping which, when $\iI$ is consistent and valid, we subsequently show to be total. For this reason we need definitions that apply to partial mappings of rules.

If $I$ is a  possibly partial mapping of derived T-rules and $\in$-rules of a theory $U$ to objects, respectively sections of a contextual category \catcw then
\begin{itemize}
\item
For any non-empty context $\xDelta{n}$ of $U$, by the mapping under $I$ of this context we shall mean the mapping under $I$ of the derived rule
\IDelta{n}.  
\item
By the mapping under $I$ of the empty context we shall mean the terminal object $1$ of \catc.
\item
For $m \geq 0$, if  $Q$ and $\yOmega{m}$ are contexts and if $\tuple{\sm}$ is a realisation of $\yOmega{m}$ wrt $Q$ in $U$
then we will define the mapping of  $\tuple{\sm}$ under $I$ to be the tuple of sections
$\tuple{I(r_{s_1}),...I(r_{s_m})}$, 
where \foreachj, $r_{s_j}$ is the derived rule \IsOmega{j}.
We will say that the realisation $\tuple{\sm}$ is mapped to a cascade iff
 $I(r_{s_j})$ is defined, \foreachj, and 
 $\tuple{I(r_{s_1}),...I(r_{s_m})}$ is a cascade of sections i.e. \foreachj, 
 $I(r_{s_j}) \in Sect(I(r_{s_{j-1}})^*...I(r_{s_1})^*(\crossx{I(Q)}{I(r_{\Omega_j})}{1}))$,
where $r_{\Omega_j}$ is the derived rule \IOmega{j}. Note that according to this description the empty realisation, which is a realisation of the empty context with respect to any other context, is deemed to map to a cascade under $I$.  \commentary{Do we need modify defn. of cascade in line with this?}
\end{itemize}

\newcommand{\smMappedToCacscade}{
for all contexts $Q$ and for all realisations $\tuple{\sm}$ of $\yOmega{m}$ wrt $Q$ 
which map to a cascade under $I$ ,}
\newcommand{\sjpconclusion}{\ofT{s'_j}{\Omega'_j[s_1|y_1,...s_{j-1}|y_{j-1}]}}
\newcommand{\IfIpartialmappingUtoC}{If $U$ is a generalised algebraic theory and \catcw is a contextual category 
and if $I$ is a partial mapping of derived T-rules and $\in$-rules of the theory $U$ to objects, respectively sections of the contextual category \catc}
\newcommand{\IfIpartialmappingUtoCw}{\IfIpartialmappingUtoC\ }

\begin{numbereddefinition}
\llabel{consistentinterpretation}
\IfIpartialmappingUtoCw
then we define what it is for any particular derived rule $r$ of $U$ to be \term{consistently interpreted} by $I$: \\
\begin{enumerate}[(i)]
%\setlength\itemindent{2cm}
\item \underline{\textbf{T-rule}} 
Let $r_\Omega$ be any  derived T-rule of $U$, assume it to be of the form \ZOmega, for some $m \geq 0$, 
and let $r_{\Omega_m}$ be the rule \IOmega{m}
then define $r_\Omega$ to be consistently interpreted by $I$ iff
\begin{enumerate}[(a)] 
\item both $I(r_\Omega)$ and $I(r_{\Omega_m})$ are defined and $I(r_{\Omega_m}) \base\, I(r_\Omega)$ in \catcw and 
\item $r_{\Omega_m}$ is consistently interpreted by $I$  and
\item
\smMappedToCacscade
$$ \Imappedrule{Q}{\isT{\Omega[\SUBsFORy{m}]}} 
= I(r_{s_m})^*...I(r_{s_1})^*(\crossx{I(Q)}{I(r_\Omega)}{1}),$$
where $r_{s_j}$ is the rule \IsOmega{j} 
and $r_{\Omega_j}$ is the rule \IOmega{j}.
\end {enumerate}
\item \underline{\textbf{$\in$-rule}} 
Let $r_s$ be any derived $\in$-rule, assume it to be of the form \ZsOmega, for some $m \geq 0$, then
$r_s$ is defined to be consistently interpreted by $I$ \commentary{Do we need that $I(r_{\Omega_m}) \base I(r_\Omega)$?}
iff  
\begin{enumerate}[(a)]
\item $I(r_{\Omega_m})$ is defined and $I(r_\Omega)$ is defined and $\displaystyle I(r_s) \in Sect(I(r_\Omega))$, where
$r_\Omega$ is the rule \ZOmega and $r_{\Omega_m}$ is the rule \IOmega{m}, and
\item
$r_{\Omega}$ is consistently interpreted by $I$ (and so by implication of condition (b) of clause (i),  $r_{\Omega_m}$ is also consistently interpreted by $I$) and,
\item  
\smMappedToCacscade
$$ \Imappedrule{Q}{\ofT{s[\SUBsFORy{m}]}{\Omega[\SUBsFORy{m}]}} = I(r_{s_m})^*...I(r_{s_1})^*(\crossx{I(Q)}{I(r_s)}{1}),$$
where $r_{s_j}$ and $r_{\Omega_j}$ defined as above (in clause (i)), or
\item in the case that $s$ is the variable $y_j$, \forsomej, so that $r_s$ is the rule \gatdisplayrule{\yOmega{m}}{\ofT{y_j}{\Omega}},
$$I(r_s) = s(p_{r_{\Omega_m},r_{\Omega_j}})$$
By lemma \lref{sofplemma}, $s(p_{r_{\Omega_m},r_{\Omega_j}}))$ is a section of the object $\crossx{r_{\Omega_m}}{r_{\Omega_j}}{r_{\Omega_{j-1}}}$.
\commentary{By implication $I(r_\Omega)=\crossx{r_{\Omega_m}}{r_{\Omega_j}}{r_{\Omega_{j-1}}}$}  
\commentary{and, btw: $s(id_{r_{\Omega_m}})=\delta_{r_{\Omega_m}}$}
Note that $p_{x,x}$ is defined to be $id_{x}$ for any object $x$ of \catcw and so in the case of $j=m$, $r_s$
is mapped to   $s(id_{r_{\Omega_m}})$.
\end{enumerate}

\item \underline{\textbf{T=-rules}} 
If $r$ is the rule  \gatdisplayrule{\xDelta{n}}{\Delta = \Delta'}, for some $n \geq 0$, 
then $r$ is interpreted consistently by $I$ iff
both of the rules \ZDelta and \ZDeltap
are consistently interpreted by $I$ and
$$
\Imappedrule{\xDelta{n}}{\isT{\Delta}} = \Imappedrule{\xDelta{n}}{\isT{\Delta}}
$$
 
\item \underline{\textbf{$\in=$-rules}} 
If $r$ is the rule  \gatdisplayrule{\xDelta{n}}{t = t' \in \Delta}, for some $n \geq 0$, 
then $r$ is interpreted consistently by $I$ iff
both of the rules \ZtDelta and \gatdisplayrule{\xDelta{n}}{\ofT{t}{\Delta'}}
are consistently interpreted by $I$ and
$$
\Imappedrule{\xDelta{n}}{\ofT{t}{\Delta'}} = \Imappedrule{\xDelta{n}}{\ofT{t'}{\Delta'}}
$$
\end{enumerate}
\end{numbereddefinition}
\highlight{END of DEFINITION OF consistently interpreted}


Since the above definition of a rule being consistently interpreted relies on whether other rules are consistently interpreted we need to check that there are no circularities:
\begin{lemma}
The above property of a rule being consistently interpreted by a mapping of rules is well-defined. 
\end{lemma}
\begin{proof}
The definition in the case of T-rules relies on the definition of other T-rules being consistently interpreted but these other rules are of lower rank and therefore there is no circularity 
(the definition terminates).
Neither are there circularities in the other cases because either they make no recursive references to consistent interpretation (the $\in$-rules case)
\commentary{watch out for $\in$-rule case
 changing}
or they make recursive references to T-rules (the T=-rule case) or to $\in$-rules (the $\in$=-rules case).
\end{proof}

\begin{definition}
We will say that a partial mapping I  
of derived T-rules and $\in$-rules of the theory $U$ to objects, respectively sections of the contextual category \catcw
is \term{type independent} on   $\in$-rules iff whenever
\gatdisplayrule{P}{\ofT{t}{\Delta}} and \gatdisplayrule{P}{\ofT{t}{\Delta'}} are derived rules of $U$ then 
$\Imap{\gatdisplayrule{P}{\ofT{t}{\Delta}}}=\Imap{\gatdisplayrule{P}{\ofT{t}{\Delta'}}}$.
\end{definition}

\begin{definition}
An \term{instance} of a generalised algebraic theory $U$ in a contextual category \catcw is  any mapping 
of derived T-rules and $\in$-rules of the theory $U$ to objects, respectively sections of the contextual category \catcw that
is type independent on $\in$-rules and which
consistently interprets every derived rule of $U$.
\end{definition}

We will show that an instance $I$ of a generalised algebraic theory $U$ in a contextual category \catcw is
completely determined by its mapping of the introductory rules of sort symbols and operator symbols to
objects, respectively, sections of \catc. To state this precisely  we require a number of defintions.

\begin{definition}
If $U$ is a generalised algebraic theory  and if \catcw is a contextual category then
a \term{interpretation} $\iI$ of  $U$ in \catcw consists of a pair :
\begin{itemize}
\item a mapping $\Isort$ that maps each sort symbol of $U$ to  an object of \catc,
\item a mapping $\Iop$ that maps each operator symbol of $U$ to a section of \catcw (i.e. to a morphism $f: A \morph B$ for some 
$A \base B$ in \catcw such that $f \circ p_B=id_A$.
\end{itemize}
\end{definition}

We will say that  an interpretation $\iI$ of $U$ in \catcw \term{determines} an  instance $I$ of $U$ in \catcw to mean that for each sort symbol $A$ of $U$,
$I(r_A) = \iI_{sort}(A)$, where $r_A$ is the introductory rule for $A$ and that for each operator symbol
$f$ of $U$,   $I(r_f) = \iI_{op}(f)$, where $r_f$ is the introductory rule for $f$.

We will show (lemma \lref{uniquenessofinstancedeterminedbyaninterpretation}) that 
if $\iI$ is an interpretation of a generalised algebraic theory $U$ in a contextual catgeory \catcw then
there is at most one  instance determined by $\iI$.

\begin{definition} [\highlight{Definition of  $\Ibar$}]
If $\iI$ is an interpretation of generalised algebraic theory $U$ in a contextual catgeory \catcw
then define a
partial mapping $\Ibar$  of T-rules and $\in$-rules to objects, respectively sections, of \catcw
as follows
\begin{enumerate}[(i)] 
\item \underline{\textbf{T-rules}} 
The derivation lemma (lemma \ref{derivationlemma}) tells us that if $r_\Delta$ is a derived T-rule of $U$  then it is of the form \gatdisplayrule{P}{\isT{A(t_1,...t_n)}} for some premise $P$, for some $n \geq 0$ and for some sort symbol $A$ with introductory rule $r_A$ of the form \gatdisplayrule{\xDelta{n}}{\isT{A(x_1,...x_n)}}, where \foreachi, the rule 
\ItDelta[P]{i}  is a derived rule of $U$. 
We define $\Ibar(r_\Delta)$ to be undefined unless the following preconditions are met:
\begin{enumerate}[(a)]
\item
the context  $P$ is mapped by $\Ibar$ to some object $\Ibar(P)$ of \catcw and 
\item
\foreachi, the rule $\IDelta{i}$, which we denote $r_{\Delta_i}$, is mapped by $\Ibar$ to some object $\Ibar(r_{\Delta_i})$ of \catcw
and $1 \base \Ibar(r_{\Delta_1}),...\Ibar(r_{\Delta_n}) \base \iI_{sort}(A)$ in \catcw 
\item
$\tuple{\tn}$ is mapped to a cascade by $\Ibar$
\end{enumerate}
in which case we define $\Ibar(r_\Delta)$ to be $\Ibar(r_{t_n})^*...\Ibar(r_{t_1})^*(\crossx{\Ibar(P)}{\Ibar(A)}{1})$,
where $r_{t_i}$ is the rule \ItDelta[P][.]{i}
\item \underline{\textbf{$\boldsymbol {\in}$-rules}} 
The derivation lemma also tells us that if $r_t$ is an $\in$-rule then it is  
either \highlight{(1)} of the form \gatdisplayrule{\xDelta{n}}{\ofT{x_i}{\Omega}} for some $n \ge 1$, for some $i$, $1 \leq i \leq n$, 
and for some $\Omega$ such that \gatdisplayrule{\xDelta{n}}{\Delta_i=\Omega} is a derived rule of $U$
or \highlight{(2)} it is of the form \gatdisplayrule{P}{\ofT{f(t_1,...t_n)}{\Omega}} for some premise $P$, for some $n \geq 0$ and for some operator symbol $f$ with introductory rule $r_f$ of the form \gatdisplayrule{\xDelta{n}}{\ofT{f(x_1,...x_n)}{\Delta}} where \foreachi, the rule 
\ItDelta[P]{i}, which we will call $r_{t_i}$, is a derived rule of $U$. 

In  case (1) we define $\Ibar(r_t)$ to be undefined unless  the following preconditions are met:
\begin{enumerate}
\item
\foreachi, the rule $\IDelta{i}$, which we denote $r_{\Delta_i}$, is mapped by $\Ibar$ to some object $\Ibar(r_{\Delta_i})$ of \catcw
and $1 \base \Ibar(r_{\Delta_1}),...\Ibar(r_{\Delta_n})$ in \catcw 
\item the rule \gatdisplayrule{\xDelta{n}}{\isT{\Omega}}, which we call $r_\Omega$, 
is mapped by $\Ibar$ to the object $\crossx{\Ibar(r_{\Delta_n})}{\Ibar(r_{\Delta_i})}{\kern-4pt\Ibar(r_{\Delta_{i-1}})\kern-12pt}$
\end{enumerate}
in which case we define $\Ibar(r_t)$ to be $s(p_{\Ibar(r_{\Delta_n}),\Ibar(r_{\Delta_i})})$ . 

In  case (2) 
we define $\Ibar(r_t)$ to be undefined unless  unless the following preconditions are met:
\begin{enumerate}
\item
the context  $P$ is mapped by $\Ibar$ to some object $\Ibar(P)$ of \catcw and 
\item
\foreachi, the rule $\IDelta{i}$, which we denote $r_{\Delta_i}$, is mapped by $\Ibar$ to some object $\Ibar(r_{\Delta_i})$ of \catcw
and  
%\foreachi, the context $\xDelta{i}$ is mapped by $\Ibar$ to some object $b_i$ of \catcw
and 
\item the rule \ZDelta, which we shall call $r_\Delta$, is mapped by $\Ibar$ to some object $\Ibar(r_\Delta)$ of \catcw and
$1 \base \Ibar(r_{\Delta_1}),...\Ibar(r_{\Delta_n}) \base \Ibar(r_\Delta)$ in \catcw 
and 
\item
%\foreachi, $\Ibar(r_{t_i}) \in Sect(r_{t_{i-1}}^*...r_1^* (\crossx{a}{b}{1}))$
$\tuple{\tn}$ is mapped to a cascade by $\Ibar$,
\end{enumerate}
in which case we
define $\Ibar(r_t)=\Ibar(r_{t_n})^*...\Ibar(r_{t_1})^*(\crossx{\Ibar(P)}{\iI_{op}(f)}{1})$.
This completes the definition of $\Ibar$.
\end{enumerate}
\highlight{END OF Definition of  $\Ibar$}
\end{definition}


\def\restrict{\mathbin{\restriction}}
\newcommand{\predInstance}{\overline{I \restrict U_p}}
\newcommand{\Uincrement}{U \setminus\kern-2pt U_p}

\begin{definition}
 Suppose that $\iI$ is an interpretation of $U$  
define interpretation $\iI$ to be \term{valid}  iff 

\begin{enumerate}[(i)]
\item
for all sort symbols $A$ of $U$ if $A$ has introductory rule $r_A$ then $\Ibar(r_A)$ is defined,
\item  
for all operator symbols $f$ of $U$ if $f$ has introductory rule $r_f$ then $\Ibar(r_f)$ is defined,

\item \underline{\textbf{T=-axioms}} 
for all axioms of the form
 \gatdisplayrule{\xDelta{n}}{\Delta = \Delta'},
$\Ibar(r)$ is defined and $\Ibar(r')$ is defined and
$\Ibar(r) = \Ibar(r')$, where $r$ is the rule
\ZDelta and  
and $r'$ is the rule \ZDeltap[,]

\item \underline{\textbf{$\boldsymbol{\in=}$-axioms}} 
for all axioms  of the form
\gatdisplayrule{\xDelta{n}}{t = t' \in \Delta},
$\Ibar(r_t)$ is defined and  $\Ibar(r'_t)$ is defined and
$\Ibar(r_t) = \Ibar(r'_t)$, where $r_t$ is the rule
\ZtDelta and  
and $r'_t$ is the rule \gatdisplayrule{\xDelta{n}}{\ofT{t'}{\Delta'}}.
\end{enumerate}
\end{definition}

Now we can state precisely how an interpretation $\iI$ determines an instance: 
Suppose that $\iI$ is an interpretation of $U$  then $\iI$ is valid  iff $\iI$ determines an instance. 
If $\iI$ does determine an instance then that instance is $\Ibar$.
The remainder of this section leads to a proof of this statement in
lemma  \lref{avalidinterpretationisaninstance} at the end of this section.

\begin{lemma}
\llabel{towerlemma}
\IfIpartialmappingUtoC, if $r$ is a derived rule \ZDelta of $U$ and if
$r$ is consistently interpreted by $I$ 
then \foreachi, $r_{\Delta_i}$ is consistently interpreted by $I$ and 
$1 \base I(r_{\Delta_1}) \base ... \base I(r_{\Delta_n}) \base I(r)$ in \catc,
where, for each $i$, $r_{\Delta_i}$ is the rule \IDelta{i}.
\end{lemma}
\begin{proof}
Follows from clause (i)(b) of the definition of consistent interpretation.
\end{proof}

\iffalse
\begin{lemma}
\llabel{substitutionsublemma}
\IfIpartialmappingUtoC,
then if $r_s$ is the $\in$-rule \ZsOmega and 
$r_s$ is consistently interpreted by $I$ then
for all contexts $Q$ and for all realisations $\tuple{\sm}$ of $\yOmega{m}$ wrt $Q$  
which are mapped to a cascade by $I$,
$$ \Imappedrule{Q}{\ofT{s[\SUBsFORy{m}]}{\Omega[\SUBsFORy{m}]}} 
= I(r_{s_m})^*...I(r_{s_1})^*(\crossx{I(Q)}{I(r_s)}{1})
\in Sect (I(r_{s_m})^*...I(r_{s_1})^*(\crossx{I(Q)}{I(r_\Omega)}{1})),$$
where \foreachj, $r_{s_j}$ is the rule \IsOmega{j} and where
$r_\Omega$ is the rule \ZOmega.
\end{lemma}
\begin{proof}
The definition of consistent interpretation directly requires this for non-variables. 
It remains to show  this in the case when $s$ is a variable. Suppose $s$ is the variable
$y_j$. Since in this case $s[\SUBsFORy{m}]$ is $s_j$ what we have to show is that
$I(s_j) = I(r_{s_m})^*...I(r_{s_1})^*(\crossx{I(Q)}{s(p_{I(r_{\Omega_m}),I(r_{\Omega_j})}}{1}))$,
where \foreachj, $r_{\Omega_j}$ is the rule \IOmega{j}. This is given by lemma \lref{cascadeprojectionlemma}.
\end{proof}
\fi

\begin{lemma}
\llabel{typeweakeninglemma}
\IfIpartialmappingUtoC,
if \ZOmega is a derived rule of $U$  which  is consistently interpreted by $I$ 
and if $Q$ is a context and $I(Q)$ is defined 
then $\Imappedrule{Q}{\isT{\Omega_1}}=\crossx{I(Q)}{\Imappedrule{}{\isT{\Omega_1}}}{1}$.
\end{lemma}
\begin{proof}
Since \ZOmega is consistently interpreted by $I$ then it follows by lemma \lref{towerlemma} that the rule 
\gatdisplayrule{}{\isT{\Omega_1}} (the rule with empty context and asserting $\Omega_1$ to be a type) is interpreted consistently
by $I$. Since $Q$ is a context and since the empty tuple $\tuple{}$ is a realisation of the empty context with respect to $Q$ that, by definition, maps to a cascade then it follows from the definition of what it means for  \gatdisplayrule{}{\isT{\Omega}} to be consistently interpreted by $I$ that
$\Imappedrule{Q}{\isT{\Omega_1}}=\crossx{I(Q)}{\Imappedrule{}{\isT{\Omega_1}}}{1}$, as required.
\end{proof}
The following lemma establishes that if all the elements of a realisation are consistently interpreted by a mapping then the elements are 
mapped to a cascade. 
\begin{lemma}
\llabel{realisationmapstocascade}
\commentary{use to be labelled substitutionpropertyvariant}
\newcommand {\forceSOURCEwidth}{\rule{5cm}{0pt}}  % so as to line up three different arrays
\newcommand {\forceTARGETwidth}{\rule{2.2cm}{0pt}}
If $I$ is a partial mapping of T-rules and $\in$-rules of a theory $U$ to objects, respectively sections of a contextual category \catc,
 if $Q$  and $\encyOmega{m}$ are contexts, for some $m \geq 1$,  and if $\tuple{\sm}$ is a realisation of $\tuple{\yOmega{m}}$ with respect to $Q$,
 if \foreachj, the rule \IsOmega{j} is consistently interpreted by $I$ 
 and if the rule \IOmega{m} is consistently interpreted by $I$
then the realisation $\tuple{s_1,...s_m}$ is mapped by $I$ to a cascade.
\end{lemma}
\begin{proof}
It follows from the definition of cascade that we need to establish, \foreachj, that
 $$I(r_{s_j}) \in Sect(I(r_{s_{j-1}})^*...I(r_{s_1})^*(\crossx{I(Q)}{I(r_{\Omega_j})}{1})) ,$$
  where $r_{s_j}$ is the rule \IsOmega{j} and $r_{\Omega_j}$ is the rule \IOmega{j}.
To prove this in the case of $m=1$ we need simply show that 
 $$I(r_{s_1}) \in Sect(\crossx{I(Q)}{I(r_{\Omega_1})}{1})).$$
 This follows because from the initial assumption that $r_{s_1}$ is consistently interpreted by $I$ and so by definition \lref{consistentinterpretation}, clause (ii),
 I(Q) is defined and  $I(r_{s_1}) \in Sect(I\big(\gatdisplayrule{Q}{\isT{\Omega_1}}\big))$ 
 and because \IOmega{m} is consistently interpreted by $I$ we can use lemma \lref{typeweakeninglemma} to establish $I(\gatdisplayrule{Q}{\isT{\Omega_1}}) = \crossx{I(Q)}{I(r_{\Omega_1})}{1})$.

Next we complete the proof by showing that the proposition that the lemma holds at a given $m$ follows from
the assumption that it holds at $m-1$. So assume that $Q$  and $\encyOmega{m}$ are contexts, 
that $\tuple{\sm}$ is a realisation of $\tuple{\yOmega{m}}$ with respect to $Q$,
assume that \foreachj, the rule \IsOmega{j} is consistently interpreted by $I$ 
and that the rule $r_{\Omega_m}$ is consistently interpreted by $I$. 
From these assumptions it follows
that  $\tuple{\sm[-1]}$ is a realisation of $\tuple{\yOmega{m-1}}$ with respect to $Q$.
Now since $r_{\Omega_m}$ is consistently interpreted by $I$ it follows, by definition,  that $r_{\Omega_{m-1}}$ is consistently interpreted by $I$. This means that we can use the inductive hypothesis to establish that $\tuple{\sm[-1]}$ maps to a cascade which is to say that \foreachj[m-1],  
$$I(r_{s_j}) \in Sect(I(r_{s_{j-1}})^*...I(r_{s_1})^*(\crossx{I(Q)}{I(r_{\Omega_j})}{1})).$$

It only remains to show that
$$I(r_{s_m}) \in Sect(I(r_{s_{m-1}})^*...I(r_{s_1})^*(\crossx{I(Q)}{I(r_{\Omega_m})}{1})).$$ \

This we can do because first of all, since we are given that $r_{s_m}$ is consistently interpreted by $I$ and using definition \lref{consistentinterpretation}, clause (ii), we establish that
$$I(r_{s_m}) \in Sect(\Imappedrule{Q}{\isT{\Omega_m[\SUBsFORy{m-1}]}}.$$ 
Next, because $r_{\Omega_m}$ is consistently interpreted by $I$ then using definition \lref{consistentinterpretation}, clause (i)(c), 
and since we have from the inductive hypothesis that $\tuple{\sm[-1]}$ is a realisation of $\tuple{\yOmega{m-1}}$ with respect to $Q$ that maps to a cascade,
we establish that
 $$\Imappedrule{Q}{\isT{\Omega_m[\SUBsFORy{m-1}]}}= I(r_{s_{m-1}})^*...I(r_{s_1})^*(\crossx{I(Q)}{I(r_{\Omega_m})}{1}).$$
 \end{proof}

\begin{lemma}
\llabel{substitutioninterpretationlemma}
\IfIpartialmappingUtoC,
if we suppose that $Q$ and $\encyOmega{m}$ are contexts, for some $m \geq 1$, 
and that $\tuple{\sm}$ is a realisation of $\encyOmega{m}$ wrt $Q$ and
if we suppose that \foreachj, the rule \IsOmega{j} is consistently interpreted by $I$ then
\begin{enumerate}[(i)]
\item if \ZOmega is a derived rule which is consistently interpreted by $I$
then the derived rule \ZOmegaSUBsmFORym is consistently interpreted by $I$,
\item if \ZsOmega is a derived rule which is consistently interpreted by $I$
then the derived rule \ZsOmegaSUBsmFORym is consistently interpreted by $I$.
\end{enumerate}
\end{lemma}
\begin{proof}
Suppose  $Q$ to be the context $\xDelta{n}$, for some $n \geq 0$. Let $r_{\Delta_i}$, \foreachi, be the rule \IDelta{i}.

The rule \IDelta{n} is a derived rule of $U$ because we have assumed that $Q$ is a context. 
\begin{enumerate}[(i)]
\item
\newcommand{\targetruleone}{\gatdisplayrule{\xDelta{n}}{\isT{\Omega[\SUBsFORy{m}]}}}
Let $r_\Omega$ be the rule \ZOmega and assume it to be consistently interpreted by $I$.
To show that the rule \ZOmegaSUBsmFORym is consistently interpreted by $I$ we need show that conditions (a), (b) and (c)
of clause (i) of definition \lref{consistentinterpretation} hold for this rule.
To  show  that condition (a) holds we have to show that $I(r_{\Delta_n})$ is defined 
and $\Imap{\targetruleone}$ is defined and that  $I(r_{\Delta_n}) \base \Imap{\targetruleone}$ in \catc. 
That $I(r_{\Delta_n})$ is defined follows from the initial assumption
that the rule \gatdisplayrule{\xDelta{n}}{\ofT{s_1}{\Omega_1}} is consistently interpreted by $I$
and from condition (ii)(b) of definition \lref{consistentinterpretation}. 
By  lemma \lref{towerlemma} it then follows that 
$1 \base I(r_{\Delta_1}) \base ... I(r_{\Delta_{n-1}}) \base I(r_{\Delta_n})$ in \catc.
Also because \foreachj,  we assume that the rule \gatdisplayrule{\xDelta{n}}{\ofT{s_j}{\Omega_j[\SUBsFORy{j-1}]}} is consistently interpreted by $I$, 
we deduce that $\tuple{\sm}$ is mapped to a cascade by $I$, by lemma \lref{realisationmapstocascade}. Hence 
from condition (i)(a) of definition \lref{consistentinterpretation} and from the assumption that $r_\Omega$ is consistently interpreted it follows that $\Imap{\targetruleone}$ is defined and 
\begin{equation}
\label{omegaSubsmapping}
\Imap{\targetruleone} = I(r_{s_m})^*...I(r_{s_1})^*(\crossx{I(r_{\Delta_n})}{I(r_\Omega)}{1})
\end{equation}
and thus\commentary{can we reference a lemma here?} that $I(r_{\Delta_n}) \base \Imap{\targetruleone}$ in \catcw as required by clause (a).

In fact by the same argument and from the fact that \foreachj,  
the realisation $\tuple{s_1,...s_{j-1}}$ is mapped to a cascade by $I$ and from lemma \lref{towerlemma} that tells us that 
$r_{\Omega_j}$ is consistently interpreted by $I$ we have that
$\displaystyle\Imappedrule{\xDelta{n}}{\isT{\Omega_j[\SUBsFORy{j-1}]}}$ is defined and 
\begin{equation}
\label{omegajSubsmapping}
\Imappedrule{\xDelta{n}}{\isT{\Omega_j[\SUBsFORy{j-1}]}} 
= I(r_{s_{j-1}})^*...I(r_{s_1})^*(\crossx{I(r_{\Delta_j})}{I(r_{\Omega_j})}{1})
\end{equation}
We use this fact in a moment in proof that clause (c) is satisfied.

Next, for clause (b), we are required to show that $r_{\Delta_n}$ is consistently interpreted by $I$. 
This follows from the fact that 
the rule \gatdisplayrule{\xDelta{n}}{\ofT{s_1}{\Omega_1}}  is consistently interpreted by $I$ 
and from condition (ii)(b) of definition \lref{consistentinterpretation}.

Finally, for clause (c), assume that $P$ is a context,
assume $\tuple{\tn}$ is a realisation of $\xDelta{n}$ with respect to $P$ that is mapped to a cascade by $I$, 
we must show that
$$\Imappedrule{\xDelta{n}}{\isT{\Omega[\SUBsFORy{m}][\SUBtFORx{n}]}}
                =I(r_{t_n})^* ... I(r_{t_1})^*\big(\crossx{I(P)}{\Imappedrule{\xDelta{n}}{\isT{\Omega[\SUBsFORy{m}]}}}{1}\big),$$
where each $r_{t_i}$ is the rule $\ItDelta{i}$.
By rearrangement of the lhs and by use of (\ref{omegaSubsmapping}) this means that we need to show 
\begin{multline}
\label{ctarget}
\Imappedrule{\xDelta{n}}{\isT{\Omega[s_1[\SUBtFORx{n}]|y_1,... s_m[\SUBtFORx{n}]|y_m]}} \\
                =I(r_{t_n})^* ... I(r_{t_1})^*\big(\crossx{I(P)}{I(r_{s_m})^*...I(r_{s_1})^*(\crossx{I(r_{\Delta_n})}{I(r_\Omega)}{1})}{1}\big)
\end{multline}

\newcommand{\IOmegaDoublySubstituted}[1]{\Omega_#1[\SUBsFORy{#1-1}][\SUBtFORx{n}]}
Now, each $r_{s_j}$ is consistently interpreted by $I$ and $\tuple{\tn}$ is a realisation
of the context $\xDelta{n}$ with respect to $P$ that is mapped to a cascade by $I$ and therefore using 
by condition (c) of clause (ii) of definition \lref{consistentinterpretation} we have 
\begin{multline*}
\Imappedrule{P}{\ofT{s_j[\SUBtFORx{n}]}{\IOmegaDoublySubstituted{j}}}
           = I(r_{t_n})^* ... I(r_{t_1})^*\big(\crossx{I(P)}{I(r_{s_j})} {1}\big) \\
           \in Sect\Big(I(r_{t_n})^* ... I(r_{t_1})^*\Big(\crossx{I(P)}{\Imappedrule{\xDelta{n}}{\isT{\Omega_j[\SUBsFORy{j-1}]}}} {1}\Big)\Big)
\end{multline*}

and since
\begin{align*}
I(r_{t_n})^* ... I(r_{t_1})^*\Big(\crossx{I(P)}{\Imappedrule{\xDelta{n}}{\isT{\Omega_j[\SUBsFORy{j-1}]}}} {1}\Big)\kern-8.5cm\\
        & = I(r_{t_n})^* ... I(r_{t_1})^*\Big(\crossx{I(P)}{ I(r_{s_{j-1}})^*...I(r_{s_1})^*\big(\crossx{I(Q)}{I(r_{\Omega_j})}{1}\big)} {1}\Big) 
                                                               &&      \mbox{by  (\lref{omegajSubsmapping})}\\
        & =  \Big(I(r_{t_n})^*...I(r_{t_1})^*(\crossx{I(r_{\Delta_n})}{I(r_{s_{j-1}})}{1})\Big)
               ^* ... 
             \Big(I(r_{t_n})^*...I(r_{t_1})^*(\crossx{I(r_{\Delta_n})}{I(r_{s_1})}{1})\Big)
               ^*\Big(\crossx{I(P)}{I(r_{\Omega_j})}{1}\Big)
                                                            &&  \mbox{by lemma \lref{cascadedpullbackscohere}}
\end{align*}
then we have
\begin{multline*}
\Imappedrule{P}{\ofT{s_j[\SUBtFORx{n}]}{\IOmegaDoublySubstituted{j}}}
           = I(r_{t_n})^* ... I(r_{t_1})^*\big(\crossx{I(P)}{I(r_{s_j})} {1}\big) \\
\in Sect\bigg( \Big(I(r_{t_n})^*...I(r_{t_1})^*(\crossx{I(r_{\Delta_n})}{I(r_{s_{j-1}})}{1})\Big)
               ^* ... 
             \Big(I(r_{t_n})^*...I(r_{t_1})^*(\crossx{I(r_{\Delta_n})}{I(r_{s_1})}{1})\Big)
               ^*\Big(\crossx{I(P)}{I(r_{\Omega_j})}{1}\Big) 
        \bigg)
\end{multline*}

and so 
$\tuple{s_1[\SUBtFORx{n}],...s_m[\SUBtFORx{n}]}$ is a realisation of $\yOmega{m}$ with repect to $P$
that is mapped to a cascade by $I$
and we can use the initial assumption that $r_\Omega$ is consistently interpreted by $I$ 
to establish that
\begin{multline}
     \Imappedrule{\xDelta{n}}{\isT{\Omega\big[s_1[\SUBtFORx{n}]|y_1,... s_m[\SUBtFORx{n}]|y_m\big]}} \\
           =  \Big(I(r_{t_n})^*...I(r_{t_1})^*\big(\crossx{I(r_{\Delta_n})}{I(r_{s_m})}{1}\big)\Big)
               ^* ... 
              \Big(I(r_{t_n})^*...I(r_{t_1})^*\big(\crossx{I(r_{\Delta_n})}{I(r_{s_1})}{1}\big)\Big)
               ^*\Big(\crossx{I(P)}{I(r_\Omega)}{1}\Big)
\end{multline}
the rhs of which can be rearranged using equation (\lref{cascadedpullbackscohereonobjects}) of lemma \ref{cascadedpullbackscohere}   so as to establish
equation (\ref{ctarget}) as required to show that condition (c) holds.

\item 
\newcommand{\targetruletwo}{\gatdisplayrule{\xDelta{n}}{\ofT{s[\SUBsFORy{m}]}{\Omega[\SUBsFORy{m}]}}}
Now let $r_s$ be the rule \ZsOmega and assume this rule to be consistently interpreted by $I$ as defined
in clause (ii) of definition \lref{consistentinterpretation}. Condition (b) of the definition tells us that
\targetruleone must be consistently interpreted by $I$ and therefore in this part (ii) of the proof we can use
anything or everything established in part (i) of this proof.

We need to show that the rule \ZsOmegaSUBsmFORym is consistently interpreted by $I$ and for this we need show that conditions (a), (b), (c) and (d)
of clause (ii) of definition \lref{consistentinterpretation} hold for this rule.

To show that condition (a) holds we need show that $I(r_{\Delta_n})$ and $\Imap{\targetruleone}$ are defined, which we have established already in part (i),
and that $$\Imap{\targetruletwo} \in Sect\big(\Imap{\targetruleone}\big).$$
This follows from the initial assumption that \ZsOmega is consistently interpreted by $I$ 
and because we have already established that $\tuple{\sm}$ is a realisation which is mapped to a cascade by $I$
and from which we can use condition (c) of clause (ii) of lemma \ref{consistentinterpretation} to deduce that
\begin{multline}
\label{targetrule2mapping}
\Imap{\targetruletwo} = I(r_{s_m})^*...I(r_{s_1})^*(\crossx{I(r_{\Delta_n})}{I(r_s)}{1})\\
                      \in Sect\big(\Imap{\targetruleone}\big)
\end{multline}
To show that condition (b) holds we have to show that the rule \targetruleone is consistently interpreted by $I$. 
This is established in part (i) of this proof.

To show that condition (c) holds assume that $P$ is a context and $\tuple{\tn}$ is a realisation as described in part (i) of this proof.
We need to show that 
\begin{multline}
\Imappedrule{\xDelta{n}}{\ofT{s[\SUBsFORy{m}][\SUBtFORx{n}]}{\Omega[\SUBsFORy{m}][\SUBtFORx{n}]}} \\
                =I(r_{t_n})^* ... I(r_{t_1})^*\big(\crossx{I(P)}{\Imappedrule{\xDelta{n}}{\ofT{s[\SUBsFORy{m}]}{\Omega[\SUBsFORy{m}]}}}{1}\big),
\end{multline}

which after rearangement of the lhs and use of (\ref{targetrule2mapping}) means that we need to show that

\begin{multline}
\label{ctargettwo}
\Imappedrule{\xDelta{n}}{\ofT{s[s_1[\SUBtFORx{n}]|y_1,... s_m[\SUBtFORx{n}]|y_m]}{\Omega[s_1[\SUBtFORx{n}]|y_1,... s_m[\SUBtFORx{n}]|y_m]}}\\
                =I(r_{t_n})^* ... I(r_{t_1})^*\big(\crossx{I(P)}{I(r_{s_m})^*...I(r_{s_1})^*(\crossx{I(r_{\Delta_n})}{I(r_s)}{1})}{1}\big)
\end{multline}
Similarly to part (i) of this proof, 
this follows because $\tuple{s[s_1[\SUBtFORx{n}]|y_1,... s_m[\SUBtFORx{n}]|y_m]}$
is a realisation which is mapped by $I$ to a cascade,
from the initial assumption that 
\ZsOmega is a derived rule which is consistently interpreted by $I$
and by use of equation (\lref{cascadedpullbackscohereonsections}) of lemma \ref{cascadedpullbackscohere}.

Finally, we need show that condition (d) holds in the case that the expression $s[\SUBsFORy{m}]$ is simply a variable. For this to be the case the
expression $s$ must be a variable, $y_j$, say, and the expression $s_j$ must be a variable $x_i$, say,  
so that the expression $s[\SUBsFORy{m}]$ is simply the variable $x_i$. To show that condition (d) holds we need to show that
$$\Imap{\gatdisplayrule{\xDelta{n}}{\ofT{x_i}{\Omega[\SUBsFORy{m}]}}}=s(p_{r_{\Delta_n},r_{\Delta_i}}).$$
This is established as follows:
\begin{align*}
\Imap{\gatdisplayrule{\xDelta{n}}{\ofT{x_i}{\Omega[\SUBsFORy{m}]}}}\kern-3cm\\
&=I(r_{s_m})^*...I(r_{s_1})^*\Big(\crossx{I(r_{\Delta_n})}{\Imap{\gatdisplayrule{\yOmega{m}}{\ofT{y_j}{\Omega}}}} {1}  \Big) 
                                                                               && \mbox{by condition (c) of clause (ii) of definition \lref{consistentinterpretation}, }\\
&=I(r_{s_m})^*...I(r_{s_1})^*\Big(\crossx{I(r_{\Delta_n})} {s(p_{r_{\Omega_m},r_{\Omega_j}})} {1}  \Big) 
                                                                               && \mbox{because $r_s$ is consistently interpreted by $I$, } \\
&=I(r_{s_j})                                                                   && \mbox{by lemma \lref{cascadeprojectionlemma}}, \\
&=s(p_{\Delta_n,\Delta_i})                  && \mbox{because $r_{s_j}$ is the rule \gatdisplayrule{\xDelta{n}}{\ofT{x_i}{\Omega_j[\SUBsFORy{j-1}]}} }\\
&                                                                              &&  \mbox{and it is consistently interpreted by $I$.} \\
\end{align*}
\end{enumerate}
\end{proof}

\begin{lemma}
\llabel{Ibartowerlemma}
If $\iI$ is an interpretation of generalised algebraic theory $U$ in contextual category \catc,
Suppose that  $r_\Omega$ is a derived T-rule \ZOmega of a generalised algebraic theory $U$
and suppose that \foreachj, $r_{\Omega_j}$ is the derived rule
\IOmega{j},
if $\iI$ is an interpretation of $U$ in a contextual category \catcw
such that $\Ibar(r_\Omega)$ is defined then
$\Ibar(r_j)$ is defined \foreachj, and 
$1 \base Ibar(r_1) \base  ... \base \Ibar(r_m) \base \Ibar(r)$ in \catc.
\end{lemma}
\begin{proof}
It follows from the definition of $\Ibar$ that since $\Ibar(r)$ is defined that $\Ibar(r_m)$ is defined and that
$\Ibar(r_m) \base \Ibar(r)$ in \catc. Now we can repeat and argue that $\Ibar(r_{m-1})$ is defined and that $\Ibar(r_{m-1}) \base \Ibar(r_m)$
in \catc. By induction $1 \base \Ibar(r_1) \base  ... \base \Ibar(r_m) \base \Ibar(r)$ in \catc, as required. 
\commentary{Have we proved $1 \base r_1$ ??}
\end{proof}


\begin{lemma}
\llabel{uniquenesssublemma}
If $I$ and $I'$ are instances of a generalised algebraic theory $U$ in a contextual category \catcw and if \gatdisplayrule{P}{\ofT{t}{\Delta}}
is a derived rule of $U$ such that $\Imap{\gatdisplayrule{P}{\ofT{t}{\Delta}}}=\Ipmap{\gatdisplayrule{P}{\ofT{t}{\Delta}}}$
then
\begin{enumerate}[(i)]
\item
$\Imap{\gatdisplayrule{P}{\isT{\Delta}}}=\Ipmap{\gatdisplayrule{P}{\isT{\Delta}}}$
\item
$I(P)=I'(P)$.
\end{enumerate}
\end{lemma}
\begin{proof}
\begin{enumerate}[(i)]
\item
The rule \gatdisplayrule{P}{\ofT{t}{\Delta}} is a derived rule of $U$ and so 
it is consistently interpreted by $I$ since $I$ is an instance and therefore 
$\Imap{\gatdisplayrule{P}{\ofT{t}{\Delta}}} \in Sect \big(\Imap{\gatdisplayrule{P}{\isT{\Delta}}})$ by condition (a) of clause (ii) of definition \lref{consistentinterpretation}.
Likewise it follows that $\Ipmap{\gatdisplayrule{P}{\ofT{t}{\Delta}}} \in Sect \big(\Ipmap{\gatdisplayrule{P}{\isT{\Delta}}})$.
But $\Imap{\gatdisplayrule{P}{\ofT{t}{\Delta}}} = \Ipmap{\gatdisplayrule{P}{\ofT{t}{\Delta}}}$ and therefore
$Sect \big(\Imap{\gatdisplayrule{P}{\isT{\Delta}}})=Sect \big(\Ipmap{\gatdisplayrule{P}{\isT{\Delta}}})$
and therefore $\Imap{\gatdisplayrule{P}{\isT{\Delta}}}=\Ipmap{\gatdisplayrule{P}{\isT{\Delta}}}$.
\item
The rule \gatdisplayrule{P}{\isT{\Delta}} is a derived rule of $U$ and so 
it is consistently interpreted by $I$ since $I$ is an instance and therefore 
$I(P) \base \Imap{\gatdisplayrule{P}{\isT{\Delta}}}$ in \catcw by condition (a) of clause (i) of definition \lref{consistentinterpretation} . 
Likewise it follows that  $I'(P) \base \Ipmap{\gatdisplayrule{P}{\isT{\Delta}}}$ in \catc. 
But $\Imap{\gatdisplayrule{P}{\isT{\Delta}}} = \Ipmap{\gatdisplayrule{P}{\isT{\Delta}}}$ and therefore $I(P)=I'(P)$. 
\end{enumerate}
\end{proof}
\begin{lemma} 
\llabel{uniquenessofinstancedeterminedbyaninterpretation}
If $\iI$ is an interpretation of generalised algebraic theory $U$ in contextual catgeory \catcw that determines instances $I$ and $I'$ of $U$ then
$I=I'$. 
\end{lemma}
\begin{proof}
We show that for all derived  T- and $\in$-rules of $U$, $I(r)=I'(r)$.  
The proof proceeds by induction on the derivation of $r$. 
We need only consider principles T1, CF1, CF2(a) and CF2(b), for these are the only principles by which  T-rules and $\in$-rules may be derived.\\
\underline{T1}
By this principle from \gatdisplayrule{P}{\Delta=\Delta'} and \gatdisplayrule{P}{\ofT{t}{\Delta}} we can derive \gatdisplayrule[.]{P}{\ofT{t}{\Delta'}}
Assume the inductive hypothesis that $\Imap{\gatdisplayrule{P}{\ofT{t}{\Delta}}} = \Ipmap{\gatdisplayrule{P}{\ofT{t}{\Delta}}}$.
We need show that $\Imap{\gatdisplayrule{P}{\ofT{t}{\Delta'}}} = \Ipmap{\gatdisplayrule{P}{\ofT{t}{\Delta'}}}$.
This follows because from the definition of instance we have that $\Imap{\gatdisplayrule{P}{\ofT{t}{\Delta}}} = \Imap{\gatdisplayrule{P}{\ofT{t}{\Delta'}}}$
and $\Ipmap{\gatdisplayrule{P}{\ofT{t}{\Delta}}} = \Ipmap{\gatdisplayrule{P}{\ofT{t}{\Delta'}}}$.
\\

\underline{CF1} According to this principle, 
whenever a rule of the form \gatdisplayrule{\xDelta{n}}{\isT{\Delta_{n+1}}} is a derived rule of $U$ for some $n \geq 0$
then so to is the rule \gatdisplayrule{\xDelta{n+1}}{\ofT{x_i}{\Delta_{i}}}, \foreachi[n+1], where $x_{n+1}$ is any variable distinct from each of the variables $\xn$. For each $i$, $1 \leq i \leq n+1$, let $r_{\Delta_i}$ be the rule \IDelta{i}. Assume the inductive hypothesis that
$I(r_{\Delta_{n+1}})=I'(r_{\Delta_{n+1}})$. Because $I$ and $I'$ are instances they both consistently interpret the rule $r_{\Delta_{n+1}}$
and so it follows by lemma \lref{towerlemma} that
$I(r_{\Delta_i})=I'(r_{\Delta_i})$, \foreachi. Now, since \gatdisplayrule{\xDelta{n+1}}{\ofT{x_i}{\Delta_{i}}} is a derived rule
of $U$ and is therefore consistently interpreted by both $I$ and $I'$
it follows by condition (d) of clause (ii) of defintion \lref{consistentinterpretation} that
$
\Imap{\gatdisplayrule{\xDelta{n+1}}{\ofT{x_i}{\Delta_{i}}}}
=s(p_{I(r_{\Delta_{n+1}}),I(r_{\Delta_{i}})})
=s(p_{I'(r_{\Delta_{n+1}}),I'(r_{\Delta_{i}})})
=\Ipmap{\gatdisplayrule{\xDelta{n+1}}{\ofT{x_i}{\Delta_{i}}}}
$, as required.



\underline{CF2(a)} 
By this principle, from a sort symbol $A$ with introductory rule $r_A$ of the form \gatdisplayrule[,]{\xDelta{n}}{\isT{A(\xn)}} for  $n \geq 0$, 
if the rule \gatdisplayrule[,]{\xDelta{n-1}}{\isT{\Delta_n}} is a derived rule of $U$,
if $P$ is a context and if in particular the rule $r_P$ asserting that $P$ is a context is a derived rule of $U$
and from derived rules $r_{t_i}$ of the form \ItDelta[P]{i}, \foreachi, we may deduce
the rule \gatdisplayrule{P}{\isT{A(t_1,...t_n)}}, which we shall denote by $r$, is a derived rule of $U$. 
Because the introductory rule for $A$ is a derived rule of $U$ \footnote{Should really have a lemma for this} and is therefore consistently interpreted by $I$
since $I$ is an instance, and since  each rule $r_{t_i}$ is consistently interpreted by $I$ we have by lemma \lref{realisationmapstocascade}
that $\tuple{\tn}$ is mapped to a cascade by $I$ and hence by \condition{i}{c}
$I(r_{t_n})^* ... I(r_{t_1})^*\big(\crossx{I(P)}{I(r_A)}{1}\big)$, where $r_A$ is the introductory rule for $A$.
Similarly $r$ is mapped by $I'$ to
$I'(t_n)^* ... I'(t_1)^*\big(\crossx{I'(P)}{I'(r_A)}{1}\big)$.
By the inductive hypothesis we have both that $I(r_{t_i})=I'(r_{t_i})$, for each $i$, and that $I(P)=I'(P)$,  and 
because both $I$ and $I'$ are instances which extend interpretation $\iI$
we have that $I(r_A)=\iI(f)=I'(r_A)$. Therefore $I(r)=I'(r)$.
\\

\underline{CF2(b)} By this principle, from a operator symbol $f$ with introductory rule $r_f$ of the form \gatdisplayrule[,]{\xDelta{n}}{\ofT{f(\xn)}{\Delta}} for some $n \geq 0$, if the rule \gatdisplayrule[,]{\xDelta{n}}{\isT{\Delta}} is a derived rule of $U$,
if $P$ is a context and if in particular the rule $r_P$ asserting that $P$ is a context is a derived rule of $U$
and from derived rules $r_{t_i}$ of the form \ItDelta[P]{i}, \foreachi, we may deduce
the rule \gatdisplayrule{P}{\ofT{f(t_1,...t_n)}{\Delta[\SUBtFORx{n}]}}, which we shall denote by $r_t$, is a derived rule of $U$. 
Because the introductory rule for $f$ is a derived rule of $U$ \footnote{Should really have a lemma for this} and therefore consistently interpreted by $I$
since $I$ is an instance, and because, as above,  $\tuple{\tn}$ is mapped to a cascade by $I$,
by condition (c) of clause (ii) of definition \lref{consistentinterpretation} the rule $r_t$ is mapped  by $I$ to
$I(r_{t_n})^* ... I(r_{t_1})^*\big(\crossx{I(P)}{I(r_f)}{1}\big)$, where $r_f$ is the introductory rule for $f$.
Similarly $r_t$ is mapped by $I'$ to
$I'(t_n)^* ... I'(t_1)^*\big(\crossx{I'(P)}{I'(r_f)}{1}\big)$.
By the inductive hypothesis we have both that $I(r_{t_i})=I'(r_{t_i})$, for each $i$, and that $I(P)=I'(P)$,  and 
because both $I$ and $I'$ are instances which extend interpretation $\iI$
we have that $I(r_f)=\iI(f)=I'(r_f)$. Therefore $I(r_t)=I'(r_t)$.
\end{proof}

\begin{lemma}
\llabel{T1preservesconsistentinterpretation}
If $I$ is a partial mapping that is type independent on $\in$-rules,
if \gatdisplayrule{P}{\Delta=\Delta'} and \gatdisplayrule{P}{\ofT{t}{\Delta}} are derived rules of $U$ 
which are consistently interpreted by $I$ then the rule \gatdisplayrule{P}{\ofT{t}{\Delta'}}
is a derived rule of $U$ and is consistently interpreted by $I$.
\end{lemma}
\begin{proof}
That the rule \gatdisplayrule{P}{\ofT{t}{\Delta'}} is derived follows by application of principle T1 to the given derived rules.

Now, because \gatdisplayrule{P}{\Delta=\Delta'} is consistently interpreted by $I$ we know by \clause{iii} that
\begin{equation}
\label{T1lemmaEq1}
\Imap{\gatdisplayrule{P}{\isT{\Delta}}} = \Imap{\gatdisplayrule{P}{\isT{\Delta'}}}.
\end{equation}
Because \gatdisplayrule{P}{\ofT{t}{\Delta}} is consistently interpreted by $I$ we know by \condition {ii}{b} that
\begin{equation}
\label{T1lemmaEq2}
\gatdisplayrule{P}{\ofT{t}{\Delta}} \in Sect(\Imap{\gatdisplayrule{P}{\isT{\Delta}}}).
\end{equation}
Because $I$ is type independent on $\in$-rules we know, by defintion,  that
\begin{equation}
\label{T1lemmaEq3}
\Imap{\gatdisplayrule{P}{\ofT{t}{\Delta}}} = \Imap{\gatdisplayrule{P}{\ofT{t}{\Delta'}}}.
\end{equation}

We have to show that conditions (a), (b), (c) and (d) of \clause{ii} hold. We have to show
\begin{enumerate}[(a)]
\item $\Imap{\gatdisplayrule{P}{\ofT{t}{\Delta'}}}$ is defined
and that $\Imap{\gatdisplayrule{P}{\ofT{t}{\Delta'}}} \in Sect(\Imap{\gatdisplayrule{P}{\isT{\Delta'}}})$.
That  $\Imap{\gatdisplayrule{P}{\ofT{t}{\Delta'}}}$ is defined follows because
$I$ is type independent on $\in$-rules.
 That $\Imap{\gatdisplayrule{P}{\ofT{t}{\Delta'}}} \in Sect(\Imap{\gatdisplayrule{P}{\isT{\Delta'}}})$
  follows from (\ref{T1lemmaEq1}), (\ref{T1lemmaEq2}) and (\ref{T1lemmaEq3}).\\
\item
\gatdisplayrule{P}{\isT{\Delta'}} is consistently interpreted by $I$. This  follows from clause (iii) of
definition \lref{consistentinterpretation} because \gatdisplayrule{P}{\Delta=\Delta'} is consistently interpreted by $I$.

\item
That for certain appropriate realisations $\tuple{\sm}$,
$$ \Imappedrule{Q}{\ofT{t[\SUBsFORy{m}]}{\Delta'[\SUBsFORy{m}]}} = I(r_{s_m})^*...I(r_{s_1})^*(\crossx{I(Q)}{I(r_{t\Delta'})}{1}),$$
where  $r_{t\Delta'}$ is \gatdisplayrule{P}{\ofT{t}{\Delta'}}.
This follows because under these same conditions we know that
$$ \Imappedrule{Q}{\ofT{t[\SUBsFORy{m}]}{\Delta[\SUBsFORy{m}]}} = I(r_{s_m})^*...I(r_{s_1})^*(\crossx{I(Q)}{I(r_{t\Delta})}{1})$$
where  $r_{t\Delta}$ is \gatdisplayrule{P}{\ofT{t}{\Delta}},
and because both $I(r_{t\Delta})=I(r_{t\Delta'})$ and 
$$\Imappedrule{Q}{\ofT{t[\SUBsFORy{m}]}{\Delta'[\SUBsFORy{m}]}} = \Imappedrule{Q}{\ofT{t[\SUBsFORy{m}]}{\Delta'[\SUBsFORy{m}]}}$$
because $I$ is type independent on $\in$-rules.
\item
In the case that $t$ is a variable $y_j$ then
$$I(r_{t_\Delta'})=s(p_{r_{\Omega_m},r_{\Omega_j}}).$$ This too follows because $I$ is type independent on $\in$-rules.
\end{enumerate}
\end{proof}

\begin{lemma}
\llabel{Ibarintro1}
Suppose that $\iI$ is an interpretation of $U$  and that $\iI$ is valid  then 
\begin{enumerate}[(a)]
\item for every sort symbol $A$ of $U$ with introductory rule $r_A$ of the form \gatdisplayrule[,]{\xDelta{n}}{\isT{A(\xn)}} for some $n \geq 0$,
$\Ibar(r_A)$ is defined and $\Ibar(r_A)=\iI_{sort}(A)$,
\item for every operator symbol $f$ of $U$ with introductory rule $r_f$ of the form \gatdisplayrule[,]{\xDelta{n}}{\ofT{f(\xn)}{\Delta}} for some $n \geq 0$,
$\Ibar(r_f)$ is defined $\Ibar(r_f)=\iI_{op}(f)$,.
\end{enumerate}
\end{lemma}
\begin{proof}
\begin{enumerate}[(i)]
\item
Assume $A$ is such a sort symbol of theory $U$.
For each $i$, $1 \leq i \leq n$, let $r_{x_i}$ be the rule \gatdisplayrule{\xDelta{n}}{\ofT{x_i}{\Delta_i}} and let $r_{\Delta_i}$ by the rule \IDelta{i}.
From the assumption that $\iI$ is valid, and from what it means for an interpretation to be valid, it follows that  $\Ibar(r_A)$ is defined.

From the definition of $\Ibar$ it follows therefore that $\Ibar(r_{x_i})$ is defined, \foreachi, and we have
\begin{align*}
\Ibar(r_A)
   &=\Ibar(r_{x_i})^*...\Ibar(r_{x_i})^*\big(\crossx{\Ibar(r_{\Delta_n})}{\iI_{sort}(A)}{1}\big)
   &&\mbox{ by clause (i) of the definition of $\Ibar$,}\\
   &=s(p_{\Ibar(r_{\Delta_n}),\Ibar(r_{\Delta_n})})^*...s(p_{\Ibar(r_{\Delta_n}),\Ibar(r_{\Delta_1})})^*\big(\crossx{\Ibar(r_{\Delta_n})}{\iI_{sort}(A)}{1}\big)
   &&\mbox{by clause (ii) of the definition of $\Ibar$,} \\
   &={p_{\Ibar(r_{\Delta_n}),\Ibar(r_{\Delta_n})}}^* \iI_{sort}(A)
   &&\mbox{by identity (a) of lemma \lref{sofpsubstitutionlemma}, } \\
   &={id_{\Ibar(r_{\Delta_n})}}^* \iI_{sort}(A)
   &&\mbox{by definition of $p_{\Ibar(r_{\Delta_n}),\Ibar(r_{\Delta_n})}$, } \\
   &=\iI_{sort}(A)
   &&\mbox{by (q3), as required.}
\end{align*} 
\item For an operator symbol $f$ the proof is similar to the above, but in this case using clause (b) of lemma \lref{sofpsubstitutionlemma}.
\end{enumerate}
\end{proof}

\begin{lemma}
\llabel{Ibarconsistentonintros}
Suppose that $\iI$ is an interpretation of $U$  and that $\iI$ is valid  then 
\begin{enumerate}[(i)]
\item for every sort symbol $A$ of $U$ with introductory rule $r_A$ of the form \gatdisplayrule[,]{\xDelta{n}}{\isT{A(\xn)}} for some $n \geq 0$,
if the rule $r_{\Delta_n}$ is consistently interpreted by $\Ibar$ then the rule $r_A$ is consistently interpreted by $\Ibar$,
\item for every operator symbol $f$ of $U$ with introductory rule $r_f$ of the form \gatdisplayrule[,]{\xDelta{n}}{\ofT{f(\xn)}{\Delta}} for some $n \geq 0$,
if the rule $r_{\Delta}$ is consistently interpreted by $\Ibar$ then the rule $r_f$ is consistently interpreted by $\Ibar$.
\end{enumerate}
\end{lemma}
\begin{proof}
\begin{enumerate}[(i) ]
    \item We have to show that conditions (a), (b) and (c) of \clause{i} hold.
    \begin{enumerate}[(a)]
        \item We need show that $\Ibar(r_A)$ is defined and that $\Ibar(r_{\Delta_n})$ is defined and that $\Ibar(r_{\Delta_n}) \base \Ibar(r_A)$ in \catc.
        That $\Ibar(r_A)$ is defined is shown in lemma \lref{Ibarintro1}. That $\Ibar(r_{\Delta_n}) \base \Ibar(r_A)$ in \catc follows from 
        the fact that $\Ibar(r_A)$ is defined and the implication that precondition (b) of clause (i) of the definition of $\Ibar$ must therefore be met.
        \item We need show that $r_{\Delta_n}$ is consistently interpreted by $\Ibar$. This we have as an assumption.
        \item we need show that for all contexts $Q$ and for any realisation $\tuple{s_1,...s_n}$ of $\xDelta{n}$ wrt $Q$
        which is mapped to a cascade by $\Ibar$,
        $$\ibarmappedrule{Q}{A(s_1,...s_n)}=I(r_{s_n})^*I(r_{s_1})^*\big(\crossx{I(Q)}{I(r_A)}{1}\big)$$
        where $r_{s_i}$, \foreachi, is the rule \gatdisplayrule{Q}{\ofT{s_i}{\Delta_i[s_{i-1}|x_{i-1},...s_1|x_1]}}.
        Now this follows from the definition of $\Ibar$ provided that we can show that the preconditions (a), (b) and (c). 
        We can do this as follows.
        \begin{enumerate}[(a)]
            \item That $\Ibar$ is defined at the context $Q$. 
            This follows because $\Ibar$ is defined at the rule $r_{s_1}$. 
            \highlight{Assumption} \highlight{ hat $n \geq 1$!!!}\commentary{patch required to definitions maybe}
            \item That \foreachi, the rule $\IDelta{i}$, which we denote $r_{\Delta_i}$, is mapped by $\Ibar$ to some object $\Ibar(r_{\Delta_i})$ of \catcw
            and $1 \base \Ibar(r_{\Delta_1}),...\Ibar(r_{\Delta_n}) \base \iI_{sort}(A)$ in \catcw. 
            This follows because we have shown that $\Ibar(r_a)$ is defined
            and that $\Ibar(r_a)= \iI_{sort}(A)$. \highlight{CHECK}
           \item That $\tuple{s_1,...s_n}$ is mapped to a cascade by $\Ibar$. This is one of the assumptions.
        \end{enumerate}
    \end{enumerate}
    \item Very similar to the above.
\end{enumerate}
\end{proof}


Now for the main lemma:
\begin{lemma}
\llabel{avalidinterpretationisaninstance}
Suppose that $\iI$ is an interpretation of $U$  and that $\iI$ is valid  then $\Ibar$ is an instance determined by $\iI$.
\end{lemma}
\begin{proof} 
\newcommand {\forceSOURCEwidth}{\rule{5cm}{0pt}}  % so as to line up three different arrays
\newcommand {\forceTARGETwidth}{\rule{2.2cm}{0pt}}
First note that by examining clause (ii) of the definition of $\Ibar$ we can see that $\Ibar$ is  type independent 
on $\in$-rules. 

We need to show that for every derived rule $r$ of $U$, $\Ibar(r)$ is defined and $r$ is consistently interpreted\footnote{To be formally correct we need a definition of \textit{consistently interpreted} for partial mappings of rules not just total mappings -- since here at the point we use the definition we have not yet proved $\Ibar$ to be total.} by $\Ibar$. 
We prove this by induction on the derivation of rules in $U$. We examine each of the principles of derivation in turn
and show that given the stated assumptions then from rules for which $\Ibar$ is defined and which are consistently interpreted by $\Ibar$ 
it is only possible to derive rules for which $\Ibar$ is defined and which themselves are consistently interpreted by $\Ibar$.
The principles of derivation (see \cite{Cartmell86}) are LI1, ... LI7, T1, CF1, CF2(a) and CF2(b), SI1 and SI2. 
The proof of this  in each the cases  LI1,...LI6 is quite trivial. We consider each of the remaining principles in turn. 
 \\
\underline{LI7} 
By this principle from \gatdisplayrule{P}{t_1 = t_2 \in \Delta_1} and \gatdisplayrule{P}{\Delta_1=\Delta_2} derive
\gatdisplayrule{P}{t_1 = t_2 \in \Delta_2}. 
Assume the inductive hypothesis that \gatdisplayrule{P}{t_1 = t_2 \in \Delta_1} is consistently interpreted by $\iI$,
so that
\gatdisplayrule{P}{\ofT{t_1}{\Delta_1}} and
\gatdisplayrule{P}{\ofT{t_2}{\Delta_1}} are consistently interpreted by $\Ibar$
and  
\begin{equation}
\label{LI7t1t2delta1}
\ibarmappedrule{P}{\ofT{t_1}{\Delta_1}}=\ibarmappedrule{P}{\ofT{t_2}{\Delta_1}}
\end{equation}
and that \gatdisplayrule{P}{\Delta_1=\Delta_2} is consistently interpreted by $\Ibar$.

To show that the conclusion, \gatdisplayrule{P}{t_1 = t_2 \in \Delta_2}, is consistently interpreted by $\Ibar$ we need show that the derived rules 
\gatdisplayrule{P}{\ofT{t_1}{\Delta_2}} and
\gatdisplayrule{P}{\ofT{t_2}{\Delta_2}} are consistently interpreted by $\Ibar$
and that 
\begin{equation}
\label{LI7t1t2delta2}
\ibarmappedrule{P}{\ofT{t_1}{\Delta_2}}=\ibarmappedrule{P}{\ofT{t_2}{\Delta_2}}.
\end{equation}
 
That  \gatdisplayrule{P}{\ofT{t_1}{\Delta_2}}  is consistently interpreted by $I$ follows 
from the fact that \gatdisplayrule{P}{\ofT{t_1}{\Delta_1}}  and \gatdisplayrule{P}{\Delta_1=\Delta_2} 
are consistently interpreted by $I$ by lemma \lref{T1preservesconsistentinterpretation}.
That \gatdisplayrule{P}{\ofT{t_2}{\Delta_2}}   is consistently interpreted by $I$ follows in the same way.

Equation (\ref{LI7t1t2delta2}) follows from (\ref{LI7t1t2delta1}) from the earlier observation that $\Ibar$ is type independent on $\in$-rules. \\
\underline{T1}
By this principle from \gatdisplayrule{P}{\Delta=\Delta'} and \gatdisplayrule{P}{\ofT{t}{\Delta}} we can derive \gatdisplayrule[.]{P}{\ofT{t}{\Delta'}}
By the inductive hypothesis we assume that the rules   \gatdisplayrule{P}{\Delta=\Delta'} and \gatdisplayrule{P}{\ofT{t}{\Delta}} are
consistently interpreted by $\Ibar$. We have to show that the rule \gatdisplayrule[.]{P}{\ofT{t}{\Delta'}} is consistently interpreted by $\Ibar$.
This is proven in lemma \lref{T1preservesconsistentinterpretation}.

\underline{CF1} According to the principle, whenever a rule of the form \gatdisplayrule{\xDelta{n}}{\isT{\Delta_{n+1}}} is a derived rule of $U$ 
then so to is the rule \gatdisplayrule{\xDelta{n+1}}{\ofT{x_i}{\Delta_{i}}}, \foreachi[n+1], where $x_{n+1}$ is any variable distinct from each of the variables $\xn$.
We have to show that if $\Ibar$ is defined for and consistently interprets any such T-rule, which we will denote, $r_{\Delta_{n+1}}$, 
then so to is $\Ibar$ defined for and consistently interprets each associated rule $r_{x_i}$. 
We show that  if it is the case for $r_{x_j}$ all $j <i$ 
then it follows that it is the case for each rule \gatdisplayrule{\xDelta{n+1}}{\ofT{x_i}{\Delta_{i}}}, which we shall denote  $r_{x_i}$. 
In so doing we establish that it holds for all $i \leq n +1 $, as required. \commentary{pull through new conditions for consistent interpretation}

So, suppose that for each $j$, $j < i$, that $\Ibar$ is defined for and consistently interprets the rule $r_{x_j}$ i.e. suppose that  $\Ibar(r_{x_j})$ is defined and that
$\Ibar(r_{x_j})=s(p_{a_{n_1},a_j})$. 
We shall now show that it follows that 
$\Ibar$ is defined at the rule $r_{x_i}$ and that this rule is consistently interpreted by $\Ibar$ by showing that
$\Ibar(r_{x_i})=s(p_{a_{n_1},a_i})$. 

%Denote the former rule $r_{\Delta_{n+1}}$ and the latter rule $r_{x_i}$. 
First we have to show that the preconditions for $\Ibar$ to be defined at the rule $r_{x_i}$ hold. These are:

\begin{enumerate}
\item That \foreachi, the context $\xDelta{i}$ is mapped by $\Ibar$ to some object $a_i$ of \catcw such
that $1 \base a_1,...\base a_i$. This follows from lemma \lref{Ibartowerlemma}. \commentary{\highlight{CHECK}}

\item That the rule \DDelta{n+1}{i} is mapped by $\Ibar$ to the object $\crossx{a_{n+1}}{a_i}{a_{i-1}}$.

This we can show as follows.
\newcommand{\deltaimapped}{\crossx{a_{n+1}}{a_i}{a_{i-1}}}
\newcommand{\deltaimappedlong}{s(p_{a_{n+1},a_{i-1}})^*...s(p_{a_{n+1},a_1})^*(\crossx{a_{n+1}}{a_i}{1})}
Let $r_{\Delta_i}$ be the rule \IDelta{i}.



Because $\Ibar(r_{\Delta_i}) = a_i$,
because  we have that
$\tuple{x_1,...x_{i-1}}$ is a realisation of context $\xDelta{i-1}$ with respect to context $\Delta_{n+1}$ 
and since we have assumed for each $j<i$ that $r_{x_j}$ is consistently interpreted by $\Ibar$ and in particular
that
\begin{equation*}
\begin{array}{c c c }
\forceSOURCEwidth & & \forceTARGETwidth \\ [-0.1cm]
\gatdisplayrule{\xDelta{n+1}}{\ofT{x_j}{\Delta_j}}  & \Imapsto & s(p_{a_{n+1},a_j}) \\ [0.4cm]
\end{array}
\end{equation*}
and since we have assumed that $r_{\Delta_i}$ is consistently interpreted by $\Ibar$ and in particular that clause (ii)(b) of the definition of consistently interpreted holds
then we have that
\begin{equation*}
\begin{array}{c c c}
\forceSOURCEwidth & & \forceTARGETwidth \\ [-0.1cm]
\gatdisplayrule{\xDelta{n+1}}{\isT{\Delta_i[x_1|x_1,...x_{i-1}|x_{i-1}]}}  & \Imapsto & \deltaimappedlong \\ [0.4cm]
\end{array}
\end{equation*}
i.e.
\begin{equation*}
\begin{array}{c c c}
\forceSOURCEwidth & & \forceTARGETwidth \\ [-0.1cm]
\DDelta{n+1}{i}  & \Imapsto & \deltaimappedlong.\\ [0.4cm]
\end{array}
\end{equation*}
So we are done if we can prove that

\begin{equation*}
\deltaimappedlong = \deltaimapped
\end{equation*}  
This is given by lemma \lref{sofpsubstitutionlemma}
\end{enumerate}
Next we have to show that 
By showing that the relevant preconditions hold we have shown that $I(r_{x_i})$ is defined. 
To show that the rule $r_{x_i}$ is consistently interpreted by $\Ibar$ we just have to show that
 $\Ibar(r_{x_i})=s(p_{a_{n+1},a_i})$. But this is exactly the definition of $\Ibar$ at $r_{x_i}$. 
and that $\Ibar(r_{\Delta_i})=\crossx{a_n}{a_i}{1}$ which we have shown in 1. above.\\
\highlight{Some more work to be done above since we have to show conditions (a),(b) and (c) hold as well as condition (d). } \\
\underline{CF2(a)}By this principle, from a sort symbol $A$ with introductory rule $r_A$ of the form \gatdisplayrule[,]{\xDelta{n}}{\isT{A(\xn)}} for some $n \geq 0$, if the rule \gatdisplayrule[,]{\xDelta{n-1}}{\isT{\Delta_n}} is a derived rule of $U$,
if $P$ is a context and if in particular the rule $r_P$ asserting that $P$ is a context is a derived rule of $U$
and from derived rules $r_{t_i}$ of the form \ItDelta[P]{i}, \foreachi, we may deduce
the rule \gatdisplayrule{P}{\isT{A(t_1,...t_n)}}, which we shall denote by $r$, is a derived rule of $U$. \\

Assume as the inductive hypotheses that the rule $r_P$ is consistently interpreted by $\Ibar$ and that 
$\Ibar(r_{t_i})$ is defined \foreachi and that each $r_{t_i}$ is consistently
interpreted by $Ibar$.
Now, by lemma \lref{Ibarconsistentonintros} the introductory rule for $A$ is consistently interpreted by $\Ibar$
and so we can use lemma \lref{substitutioninterpretationlemma}
to establish that the rule \gatdisplayrule{P}{\isT{A(t_1,...t_n)}} is consistently interpreted by $\Ibar$, as required. \\
\underline{CF2(b)} By this principle, from a operator symbol $f$ with introductory rule $r_f$ of the form \gatdisplayrule[,]{\xDelta{n}}{\ofT{f(\xn)}{\Delta}} for some $n \geq 0$, if the rule \gatdisplayrule[,]{\xDelta{n}}{\isT{\Delta}} is a derived rule of $U$,
if $P$ is a context and if in particular the rule $r_P$ asserting that $P$ is a context is a derived rule of $U$
and from derived rules $r_{t_i}$ of the form \ItDelta[P]{i}, \foreachi, we may deduce
the rule \gatdisplayrule{P}{\ofT{f(t_1,...t_n)}{\Delta[\SUBtFORx{n}]}}, which we shall denote by $r$, is a derived rule of $U$. 

Assume as the inductive hypotheses  that the rule $r_P$ is consistently interpreted by $\Ibar$ and that 
the rule \gatdisplayrule{\xDelta{n}}{\isT{\Delta}} is consistently interpreted by $\Ibar$
and that $\Ibar(r_{t_i})$ is defined \foreachi and that each $r_{t_i}$ is consistently
interpreted by $\Ibar$.
With these assumptions we can use lemma \lref{Ibarconsistentonintros} to establish that the introductory rule for $f$ is consistently interpreted by $\Ibar$.
Next we can use lemma \lref{substitutioninterpretationlemma}
to establish that the rule \gatdisplayrule{P}{\ofT{f(t_1,...t_n)}{\Delta[\SUBtFORx{n}]}} is consistently interpreted by $\Ibar$, as required. \\
%
\newpage
\underline{SI1} 
\newcommand{\SIonesourcelhs}{\gatdisplayrule{\yOmega{m}}{\isT{\Omega}}}
\newcommand{\SIonesourcerhs}{\gatdisplayrule{\yOmega{m}}{\isT{\Omega'}}}
\newcommand{\SIoneconclusion}{\gatdisplayrule{Q}{\Omega[\SUBsFORy{m}]=\Omega'[\SUBspFORy{m}]}}
\newcommand{\SIoneconclusionlhs}{\ZOmegaSUBsmFORym}
\newcommand{\SIoneconclusionrhs}{\gatdisplayrule{Q}{\isT{\Omega'[\SUBspFORy{m}]}}}

This principle states that if \gatdisplayrule{\yOmega{m}}{\Omega=\Omega'} is a derived rule, if $Q$ is a context and if  $\sm$ and $\smp$ are expressions such that \foreachj, \gatdisplayrule{Q}{s_j=s'_j \in \Omega[\SUBsFORy{j-1}]} is a derived rule then we may derive the rule \SIoneconclusion. 

We have to show that the rule \SIoneconclusion is consistently intepreted by $\Ibar$
from the inductive hypothesis  that the rule \gatdisplayrule{\yOmega{m}}{\Omega=\Omega'}
and each of the rules \gatdisplayrule{Q}{s_j=s'_j \in \Omega[\SUBsFORy{j-1}]}, \wherej, are consistently intepreted by $\Ibar$.

To show that \SIoneconclusion is consistently interpreted by $\Ibar$
we have to show that each of the rules \SIoneconclusionlhs and \SIoneconclusionrhs
are consistently interpreted by $\Ibar$ and that
\begin{equation}
\label{SI1inductivetarget2}
\ibarmappedrule{Q}{\isT{\Omega[\SUBsFORy{m}]}} = \ibarmappedrule{Q}{\isT{\Omega'[\SUBspFORy{m}]}}.
\end{equation}

From the  inductive hypothesis  and from clauses (iii) and (iv) of definition \lref{consistentinterpretation} it follows that
each of the rules
\SIonesourcelhs
and
\SIonesourcerhs
and each of the rules
\IsOmega{j}
and
\gatdisplayrule[,]{Q}{\ofT{s'_j}{\Omega_j[\SUBspFORy{j-1}]}}
\wherej, are consistently intepreted by $\Ibar$. 
Therefore from lemma \lref{substitutioninterpretationlemma} it follows that
 the rules \SIoneconclusionlhs and \SIoneconclusionrhs
are consistently interpreted by $\Ibar$, as required.

Finally we are required to show  (\ref{SI1inductivetarget2}).
Now further to what we have said above, note that from the assumptions in the inductive hypothesis 
and from clauses (iii) and (iv) of definition \lref{consistentinterpretation} we know that
\begin{equation}
\label{SI1inductiveequation1}
\Ibar(r_\Omega)=\Ibar(r_{\Omega'})
\end{equation}

where $r_\Omega$ is the rule \SIonesourcelhs and $r_{\Omega'}$ is the rule \SIonesourcerhs
and that, \foreachj,
\begin{equation}
\label{SI1inductiveequation2}
\Ibar(r_{s_j})=\Ibar(r_{s'_j})
\end{equation}
where  $r_{s_j}$ is the rule \gatdisplayrule{Q}{\ofT{s_j}{\Omega_j[\SUBsFORy{j-1}]}}
and $r_{s'_j}$ is the rule \gatdisplayrule{Q}{\ofT{s'_j}{\Omega_j[\SUBspFORy{j-1}]}}, \foreachj.
This means that (\ref{SI1inductivetarget2}) can be shown as follows:
\begin{align*}
\ibarmappedrule{Q}{\isT{\Omega[\SUBsFORy{m}]}} 
    &=I(r_{s_m})^* ... I(r_{s_1})^*\big( \crossx{\Ibar(Q)}{\Ibar(r_\Omega)}{1} \big)
    &&\mbox{ because $r_\Omega$ is consistently interpreted by $\Ibar$,} \\
    &=I(r_{s'_m})^* ... I(r_{s'_1})^*\big( \crossx{\Ibar(Q)}{\Ibar(r_{\Omega'})}{1} \big)
    &&\mbox{ by (\ref{SI1inductiveequation1}) and (\ref{SI1inductiveequation2}),} \\
    &=\ibarmappedrule{Q}{\isT{\Omega'[\SUBspFORy{m}]}}
    &&\mbox{ as required, because $r_{\Omega'}$ is consistently interpreted by $\Ibar$.}
\end{align*}
\newpage
\underline{SI2} 
\newcommand{\SItwosourcelhs}{\gatdisplayrule{\yOmega{m}}{\ofT{s}{\Omega}}}
\newcommand{\SItwosourcerhs}{\gatdisplayrule{\yOmega{m}}{\ofT{s'}{\Omega'}}}
\newcommand{\SItwoconclusion}{\gatdisplayrule{Q}{s[\SUBsFORy{m}]=s'[\SUBspFORy{m}] \in \Omega[\SUBsFORy{m}]}}
\newcommand{\SItwoconclusionlhs}{\gatdisplayrule{Q}{\ofT{s[\SUBsFORy{m}]}{\Omega[\SUBsFORy{m}]}}}
\newcommand{\SItwoconclusionrhs}{\gatdisplayrule{Q}{\ofT{s'[\SUBspFORy{m}]}{\Omega[\SUBsFORy{m}]}}}
This principle states that if \gatdisplayrule{\yOmega{m}}{s = s' \in \Omega} is a derived rule, 
if $Q$ is a context and if  $\sm$ and $\smp$ are expressions such that
\foreachj, \gatdisplayrule{Q}{s_j=s'_j \in \Omega[\SUBsFORy{j-1}]} is a derived rule then we may derive the rule \SItwoconclusion. 

The proof in this case is similar to the case SI1 above. 

We have to show that the rule \SItwoconclusion is consistently intepreted by $\Ibar$
from the inductive hypothesis  that the rule \gatdisplayrule{\yOmega{m}}{s = s' \in \Omega}
and each of the rules \gatdisplayrule{Q}{s_j=s'_j \in \Omega[\SUBsFORy{j-1}]}, \wherej, are consistently intepreted by $\Ibar$.

To show that \SItwoconclusion is consistently interpreted by $\Ibar$
we have to show that each of the rules \SItwoconclusionlhs and \SItwoconclusionrhs
are consistently interpreted by $\Ibar$ and that
\begin{equation}
\label{SI2inductivetarget2}
\ibarmappedrule{Q}{\ofT{s[\SUBsFORy{m}]}{\Omega[\SUBsFORy{m}]}} = \ibarmappedrule{Q}{\ofT{s'[\SUBspFORy{m}]}{\Omega'[\SUBspFORy{m}]}}.
\end{equation}
From the  inductive hypothesis  and from clauses (iii) and (iv) of definition \lref{consistentinterpretation} it follows that
each of the rules
\SItwosourcelhs
and
\SItwosourcerhs
and each of the rules
\IsOmega{j}
and
\gatdisplayrule[,]{Q}{\ofT{s'_j}{\Omega_j[\SUBspFORy{j-1}]}}
\wherej, are consistently intepreted by $\Ibar$. 
Therefore from lemma \lref{substitutioninterpretationlemma} it follows that
 the rules \SItwoconclusionlhs and \SItwoconclusionrhs
are consistently interpreted by $\Ibar$, as required.

Finally we are required to show (\ref{SI2inductivetarget2}).
Now further to what we have said above, note that from the assumptions in the inductive hypothesis 
and from clauses (iii) and (iv) of definition \lref{consistentinterpretation} we know that
\begin{equation}
\label{SI2inductiveequation1}
\Ibar(r_s)=\Ibar(r_{s'})
\end{equation}
where $r_s$ is the rule \SItwosourcelhs and $r_{s'}$ is the rule \SItwosourcerhs 
and that, \foreachj,
\begin{equation}
\label{SI2inductiveequation2}
\Ibar(r_{s_j})=\Ibar(r_{s'_j})
\end{equation}
where  $r_{s_j}$ is the rule \gatdisplayrule{Q}{\ofT{s_j}{\Omega_j[\SUBsFORy{j-1}]}}
and $r_{s'_j}$ is the rule \gatdisplayrule{Q}{\ofT{s'_j}{\Omega_j[\SUBspFORy{j-1}]}}, \foreachj.
This means that (\ref{SI2inductivetarget2}) can be shown as follows:
\begin{align*}
\ibarmappedrule{Q}{\ofT{s[\SUBsFORy{m}]}{\Omega[\SUBsFORy{m}]}} 
    &=I(r_{s_m})^* ... I(r_{s_1})^*\big( \crossx{\Ibar(Q)}{\Ibar(r_s)}{1} \big)
    &&\mbox{ because $r_\Omega$ is consistently interpreted by $\Ibar$,} \\
    &=I(r_{s'_m})^* ... I(r_{s'_1})^*\big( \crossx{\Ibar(Q)}{\Ibar(r_{s'})}{1} \big)
    &&\mbox{ by (\ref{SI2inductiveequation1}) and (\ref{SI2inductiveequation2}),} \\
    &=\ibarmappedrule{Q}{\ofT{s'[\SUBspFORy{m}]}{\Omega'[\SUBspFORy{m}]}}
    &&\mbox{ as required, because $r_{\Omega'}$ is consistently interpreted by $\Ibar$.}
\end{align*}

\underline{A1} 
This principle ensures that an T=axiom is a derived rule providing it is well-typed.
 
It states that from an axiom \gatdisplayrule{\xDelta{n}}{\Delta=\Delta'} and from derived rules
 \ZDelta and \ZDeltap we may derive
\gatdisplayrule{\xDelta{n}}{\Delta=\Delta'}.

\vspace{1cm}
Follows immeadiately from the fact fact that $\Ibar$ is  valid.\\

\underline{A2} 
This principle ensures that an $\in$=axiom is a derived rule providing it is well-typed.
It states that from an axiom \gatdisplayrule{\xDelta{n}}{t=t' \in \Delta} and from derived rules
 \ZtDelta and \ZtpDelta we may derive
\gatdisplayrule{\xDelta{n}}{t=t' \in \Delta}.

Follows immeadiately from the fact that $\Ibar$ is  valid. 
\end{proof}

\begin{lemma}
\llabel{Xnlemma}
If $\iI$ is an interpretation of a generalised algebraic theory $U$ in a contextual category \catcw and if $X$ is an absolute sort symbol of $U$ which is mapped 
by $\iI$ to an object $X$ of \catcw (so that $1 \base X$ in \catc) then for any $n \geq 1$ 
\begin{enumerate}[(i)]
\item
The context $\tuple{\ofT{x_1}{X},...\ofT{x_n}{X}}$ is mapped by $\iI$ to the object $X^n$ of \catc,
\item the rule 
\gatdisplayrule{\ofT{x_1}{X},... \ofT{x_n}{X}}{\ofT{x_i}{X}} is mapped by $\iI$ to the section $s(p_i)$ of $X^{n+1}$, where $p_i$ is the $i$'th projection morphism, $p_i: X^n \morph X$,
\item if $P$ is a context of $U$ that extends the context $\tuple{\ofT{x_1}{X},...\ofT{x_n}{X}}$ and if $P$ is mapped by $\iI$ to
the object $Y$ of \catcw (so that $X < Y$ in \catc) then the rule 
\gatdisplayrule{P}{\ofT{x_i}{X}} is mapped by $\iI$ to the section $s(p_{Y,X^n}\circ p_i)$ of object $\crossx{Y}{X}{1}$, where $p_i$ is the $i$'th projection morphism, $p_i: X^n \morph X$.
\end{enumerate}
\end{lemma}
\begin{proof}
See day book -- 20 July 2021.
\end{proof}



%\iffalse

\section{Contextual Categories -- The Duple Construction}
\label{contextualnotationparttwo}
\newcommand{\duplesone}{{\duple{s_1}_{y_1}}}
\newcommand{\duplestwo}{{\duple{s_1,s_2}_{y_2}}}

\newcommand{\duplesn}{\duple{s_1,...s_n}_{y_n}}
\newcommand{\duplesi}{{\duple{s_1,...s_i}_{y_i}}}
\newcommand{\duplesilessone}{\duple{s_1,...s_{i-1}}_{y_{i-1}}}
\newcommand{\duplesj}{{\duple{s_1,...s_j}_{y_j}}}
\newcommand{\duplesjlessone}{\duple{s_1,...s_{j-1}}_{y_{j-1}}}
\newcommand{\duplesisucc}{{\duple{s_1,...s_{i+1}}_{y_{i+1}}}}
\newcommand{\duplesnlessone}{{\duple{s_1,...s_{n-1}}_{y_{n-1}}}}
\newcommand{\ynz}{\crossx{y_n}{z}{y_i}}


\newcommand {\sonesub}{{s_1}^*}
\newcommand {\stwosub}{{s_2}^*}
\newcommand {\stwocascade}{\stwosub\sonesub}
\newcommand {\sisub}{{s_i}^*}
\newcommand {\sicascade}{\sisub...\sonesub}
\newcommand {\sisuccsub}{{s_{i+1}}^*}
\newcommand {\sisucccascade}{\sisuccsub...\sonesub}
\newcommand {\snlessonesub}{{s_{n-1}}^*}
\newcommand {\snlessonecascade}{\snlessonesub...\sonesub}
\newcommand {\snsub}{{s_n}^*}
\newcommand {\sncascade}{\snsub...\sonesub}

If $x$ is an object of contextual category \catc, if $1 \base y_1 ... \base y_n$ in \catcw and if
$\sntuple$ is a cascade from $x$ to $y_n$ in \catcw
then  we can define a morphism
$\duplesn:x \morph y_n$ in \catcw such that 
\begin{axiom}{d1}
s(\duplesn) = s_n,
\end{axiom}
for $n> 1$, 
\begin{axiom}{d2}
\duplesn \circ p_{y_n} = \duplesnlessone, 
\end{axiom}
and for all objects $y \in Cover(y_n)$, 
\begin{axiom}{d3a}
{\duplesn} ^*y = \sncascade (\crossx{x}{y}{1}),
\end{axiom}
and for all sections $g$ of $y$,
\begin{axiom}{d3b}
{\duplesn} ^* g = \sncascade (\crossx{x}{g}{1}).
\end{axiom}


The definition of $\duplesn$ proceeds by induction. 
Define $\duplesone= s_1 \circ q(p_{x,1},y_1)$.
By axiom (s1), we directly establish, (d1), that $s(\duplesone)=s_1$.

If $y_1 <y$ in \catcw then we establish (d3a), that $\sonesub (\crossx{x}{y}{1})=\duplesone ^*y$ as follows
\begin{align*}
\sonesub (\crossx{x}{y}{1})&= \sonesub q(p_{x,1},y_1)^*y     && \mbox{by definition of $\crossx{}{}{1}$,}\\
                         &= (s_1 \circ q(p_{x,1},y_1))^*y   && \mbox{by pullback coherence axiom (q5),}\\
                         &= \duplesone ^*y                   && \mbox{by definition of $\duplesone ^*y$.}
\end{align*}
Also, if $g$ is a section of $y$ then we can show (d3b),  
that $\sonesub (\crossx{x}{g}{1})=\duplesone ^* g$, by a similar argument. 

Now assume that $\duplesi$ is defined and satisfies (\ref{d1}) to (\ref{d3b}) above. 
In particular we have  $\sicascade y_{i+1} = \duplesi ^*y_{i+1}$, and therefore that
$s_{i+1} \in Sect(\duplesi ^*y_{i+1})$. This allows us to define $\duplesisucc$ by 
\begin{equation*}
\duplesisucc = s_{i+1} \circ q(\duplesi, y_{i+1}).
\end{equation*} 
Immediately by axiom (s3)
we establish (d1), that $s(\duplesisucc)=s_{i+1}$.
We establish (d2), that $\duplesisucc \circ p_{y_{i+1}}= \duplesi$, as follows
\begin{align*}
\duplesisucc \circ p_{y_{i+1}} &=s_{i+1} \circ q(\duplesi, y_{i+1}) \circ p_{y_{i+1}} && \mbox{by definition of $\duplesisucc$,} \\
                               &=s_{i+1} \circ p_{\duplesi ^*y_{i+1}} \circ \duplesi && \mbox{by (Q1),} \\
															 &= \duplesi                       && \mbox{because $s_{i+1}$ is a section.}
\end{align*}
To establish (\ref{d3a}), suppose $y$ is some object such that $y_{i+1} < y$ in \catcw then we can show that $\sisucccascade (\crossx{x}{y}{1})=\duplesisucc ^*y$ as follows:
\begin{align*}
\sisucccascade (\crossx{x}{y}{1}) 
              &= \sisuccsub \duplesi ^*y && \mbox{by inductive hypothesis,} \\
                         &= \sisuccsub q(\duplesi,y_{i+1})^*y  && \mbox{by (Q6),}\\
                         &= (s_{i+1} \circ q(\duplesi,y_{i+1}))^*y   && \mbox{by (Q4),}\\
                         &= \duplesisucc ^*y                   && \mbox{by definition of $\duplesisucc$.}
\end{align*}
A similar argument  shows that if $g$ is a section of $y$ then $\sisucccascade (\crossx{x}{g}{1})=\duplesisucc ^* g$ to establish (\ref{d3b}).


\begin{lemma}
\llabel{dupledestructionlemma}
If $\duplesn : x \morph y_n$ in a contextual category $\catcw$ then \foreachi, 
\begin{equation}
\duplesn \circ p_{y_n,y_i} = \duplesi
\end{equation} 
\end{lemma}
\begin{proof}
Follows because for each $j$, $i < j \leq n$, $\duplesj \circ p_{y_j} = \duplesjlessone$
and $p_{y_j,y_i} = p_{y_j} \circ p_{y_{j-1},y_i}$.
\end{proof}
\begin{lemma}
\llabel{dupleofslemma}
If $x$ and $y$ are objects of a contextual category \catcw such that $1 \base y$ and if $g: x \morph y$ is a morphism then
\begin{equation*}
\duple{s(g)} = g
\end{equation*}
\end{lemma}
\begin{proof}
By definition of $s$, $s(g):x \morph \crossx{x}{y}{1}$ in \catc. 
Therefore,by definition of $\duple{}$,  $\duple{s(g)}$ is defined,  $\duple{s(g)}:x \morph y$ in \catcw and
satisfies (d1) i.e. that $s(\duple{s(g)}) = s(g)$ 
and therefore that, $by lemma \lref{stactic}, \duple{s(g)}=g$.
\end{proof}
%
%
%
{ % BEGIN   {thegeneraldupletuplelemma} and proof
\newcommand{\tuplesnsg}{\tuple{s_1,...s_n,s(g)}}
\newcommand{\duplesnsg}{\duple{s_1,...s_n, s(g)}_{f^*z}}
\newcommand{\dupletuplerhs}{\bigtuple{\duplesn,g}}
\begin{lemma}
\llabel{thegeneraldupletuplelemma} 
If $x$, $y_1$,...$y_n$, $z_p$ and $z$ are objects of a contextual category \catcw 
such that $1 \base y_1 ... \base y_n$ and $z_p \base z$ in \catc, 
if $f: y_n \morph z_p$, so that there is this pullback diagram 
\begin{displaymath}
\begin{array} {c p{3cm} c p{2cm} c}
              && \Rnode{TL}{f^*z}  && \Rnode{TR}{z}  \\[1.2cm]
              && \Rnode{BL}{y_n}   && \Rnode{BR}{z_p}
\end{array}
\begin{arrows}
\ncsar{TL}{BL}
\ncsar{TR}{BR}
\ncarr{TL}{TR}
\alabel{q(f,z)}
\ncarr{BL}{BR}
\blabel{f}
\end{arrows}
\end{displaymath}
in \catc, if $\tuplesnsg$ is a cascade from $x$ to $f^*z$ and $g:x \morph z$ in \catcw 
such that
\begin{equation} \label{generaldupletuplegiven}
g \circ p_z = \duplesn \circ f
\end{equation} 
then
\begin{equation}
\label{generaldupletuplegoaltwo}
\duplesnsg = \dupletuplerhs
\end{equation}
where $\dupletuplerhs$ is the unique morphism $\dupletuplerhs:x \morph f^*z$ such that
\begin{equation}
\label{generaldupletupledefone}
\dupletuplerhs \circ q(f,z) =g
\end{equation} 
and 
\begin{equation}
\dupletuplerhs \circ p_{f^*z} = \duplesn
\end{equation}
 as shown in this diagram
\begin{displaymath}
\begin{array} {c p{3cm} c p{2cm} c}
% FL is Far Left !!
\Rnode{FL}{x} &&                   &&                \\[1.0cm]
              && \Rnode{TL}{f^*z}  && \Rnode{TR}{z}  \\[1.2cm]
              && \Rnode{BL}{y_n}   && \Rnode{BR}{z_p}
\end{array}
\begin{arrows}
\ncsar{TL}{BL}
\ncsar{TR}{BR}
\ncarr[30]{FL}{TR}
\alabel{g}
\ncarr{FL}{TL}
\alabel{\dupletuplerhs}[0.7][1]
\ncarr[-20]{FL}{BL}
\blabel{\duplesn}[0.5][0]
\ncarr{TL}{TR}
\alabel{q(f,z)}
\ncarr{BL}{BR}
\blabel{f}
\end{arrows}
\end{displaymath}
\end{lemma}
\begin{proof}
To show (\ref{generaldupletuplegoaltwo})
we need just show that
\begin{equation}
\label{generaldupletuplesubgoalone}
\duplesnsg \circ q(f,z) =g
\end{equation} 
and 
\begin{equation}
\label{generaldupletuplesubgoaltwo}
\duplesnsg \circ p_{f^*z} = \duplesn
\end{equation}
(\ref{generaldupletuplesubgoalone}) follows because by lemma \lref{footandstactic}
 it suffices to show that
\begin{equation}
\label{generaldupletuplesubgoaloneone}
\duplesnsg \circ q(f,z) \circ p_z =g \circ p_z,
\end{equation}
which follows directly from (\ref{generaldupletupledefone}), and
\begin{equation}
\label{generaldupletuplesubgoalonetwo}
s(\duplesnsg \circ q(f,z)) =s(g)
\end{equation}
which we prove as follows:
\begin{align*}
s(\duplesnsg \circ q(f,z)) &=s(\duplesnsg  && \mbox{ by (s3),} \\
                          &= s(g)                       && \mbox{ by (d1).}
\end{align*}
whereas (\ref{generaldupletuplesubgoaltwo}) is an instance of clause (d2) of the definition of $\duple{}$.
\end{proof}
} % END   {thegeneraldupletuplelemma} and proof
%
%
%

{ % BEGIN {thedupletuplelemma} and proof and the {absolutedupletuplesublemma} and proof
\newcommand{\tuplesnsg}{\tuple{s_1,...s_n, s(g)}} 
\newcommand{\duplesnsg}{\duple{s_1,...s_n, s(g)}_{\ynz}}
\newcommand{\dupletuplerhs}{\bigtuple{\duplesn,g} }
\begin{lemma}
\llabel{thedupletuplelemma} 
If $x$, $y_1$,...$y_n$ and $z$ are objects of a contextual category \catcw such that $1 \base y_1 ... \base y_n$ in \catcw and
$y_i \base z$, for some $i$, $1 \leq i \leq n$,
if $\sntuple$ is a cascade from $x$ to $y_n$ in \catcw 
and if $g: x \morph z$ in \catcw such that
$g \circ p_z = \duplesi$, 
then 
$\tuplesnsg$ is a cascade from $x$ to $\ynz$ in \catcw
and
\begin{equation}
\label{dupletuplegoal}
\duplesnsg = \dupletuplerhs\,.
\end{equation}
\end{lemma}
\begin{proof}

To show that $\tuplesnsg$ is a cascade from $x$ to $\ynz$ in \catcw we need show that
$s(g) \in Sect({s_n}^* ... {s_1}^* (\crossx{x}{(\ynz)}{1})$.
This follows because by definition of $s(g)$, $s(g) \in Sect((g \circ p_z) ^*z)$ and because
\begin{align*}
(g \circ p_z) ^*z &= \duplesi ^*z                                && \mbox{from initial assumption that $g \circ p_z =\duplesi$,}\\
                  &= (\duplesn \circ {p_{y_n,y_i}})^*z           && \mbox{by lemma \lref{dupledestructionlemma},} \\
                  &= {\duplesn} ^* {p_{y_n,y_i}}^*z              && \mbox{by (q4),} \\
                  &= {\duplesn} ^* (\ynz)                        && \mbox{by definition of $\crossx{}{}{}$,} \\
                  &= {s_n}^* ... {s_1}^* (\crossx{x}{(\ynz)}{1}) && \mbox{by (d3a).}
\end{align*}

Now (\ref{dupletuplegoal}) follows as a special case of lemma \lref{thegeneraldupletuplelemma} with $z_p$ being $y_i$ and $f$ being $p_{y_n,y_i}$
provided that we can show that 
\begin{equation}
g \circ p_z = \duplesn \circ p_{y_n,y_i}.
\end{equation}
This follows from the assumption that $g \circ p_z = \duplesi$ and from lemma \ref{dupledestructionlemma}.
\end{proof}
% END of {thedupletuplelemma} and proof%
%

\begin{lemma}
\llabel{absolutedupletuplesublemma}
If $x$, $y_1$,...$y_n$ and $z$ are objects of a contextual category \catcw such that $1 \base y_1 ... \base y_n$ in \catcw and
$1 \base z$, 
if $\sntuple$ is a cascade from $x$ to $y_n$ in \catcw 
and if $g: x \morph z$ in \catcw 
then $\tuplesnsg$ is a cascade from $x$ to $\ynz$ in \catcw
and
\begin{equation}
\label{dupletuplegoalx}
\duplesnsg = \dupletuplerhs\,.
\end{equation}
\end{lemma}
\begin{proof}

To show that $\tuplesnsg$ is a cascade from $x$ to $\ynz$ in \catcw we need show that
$s(g) \in Sect({s_n}^* ... {s_1}^* (\crossx{x}{(\ynz)}{1})$.
This follows because by definition of $s(g)$, $s(g) \in Sect((g \circ p_z) ^*z)$ and because
\begin{align*}
(g \circ p_z) ^*z  &= (\duplesn \circ {p_{y_n,1}})^*z           && \mbox{because $1$ is terminal,} \\
                  &= {\duplesn} ^* {p_{y_n,1}}^*z               && \mbox{by (q4),} \\
                  &= {\duplesn} ^* (\ynz)                       && \mbox{by definition of $\crossx{}{}{}$,} \\
                  &= {s_n}^* ... {s_1}^* (\crossx{x}{(\ynz)}{1} && \mbox{by (d3a).}
\end{align*}

Now (\ref{dupletuplegoalx}) follows as a special case of lemma \lref{thegeneraldupletuplelemma} with $z_p$ being $1$ and $f$ being $p_{y_n,1}$
provided that we can show that 
\begin{equation}
g \circ p_z = \duplesn \circ p_{y_n,1}.
\end{equation}
This holds because $1$ is terminal.
\end{proof}
}  %end scope for two lemmas



\begin{lemma}
\llabel{absolutedupletuplelemma}
For $n \geq 1$, if $x$ and $y_1,...y_n$ are objects of a contextual category \catcw such that \foreachi, $1 \base y_i$ and if \foreachi, $f_i: x \morph y_i$ then
\begin{equation*}
\duple{s(f_1),...s(f_n)}=\tuple{\fn}
\end{equation*}
\end{lemma}
\begin{proof}
\begin{align*}
\duple{s(f_1),...s(f_n)} &= \tuple{\duple{s(f_1),...s(f_{n-1})},f_n} &&\mbox{by lemma \lref{absolutedupletuplesublemma},}\\
                         &= \tuple{\tuple{f_1,...f_{n-1}},f_n}       &&\mbox{by the inductive hypothesis,}  \\
                         &= \tuple{\fn}                              && \mbox{by definition of $\tuple{}$.}
\end{align*}
\end{proof}

\begin{lemma}
\llabel{absolutedupletuplelemma}
For $n > 1$, if $x$ and $y_1,...y_n$ are objects of a contextual category \catcw such that \foreachi, $1 \base y_i$ and if \foreachi, $f_i: x \morph y_i$ then
\begin{equation*}
\tuple{\fn}=\duple{s(f_1),...s(f_n)}
\end{equation*}
\end{lemma}
\begin{proof}
\begin{align*}
\duple{s(f_1),...s(f_n)} &= \tuple{\duple{s(f_1),...s(f_{n-1})},f_n} &&\mbox{by lemma \lref{thegeneraldupletuplelemma},}\\
                         &= \tuple{\tuple{f_1,...f_{n-1}},f_n}       &&\mbox{by the inductive hypothesis,}  \\
                         &= \tuple{\fn}                              && \mbox{by definition of $\tuple{}$.}
\end{align*}
\end{proof}

\begin{lemma}
\llabel{duplesofplemma}
If $x$,$y_1$,...$y_m$ are objects of a contextual category \catcw such that $x \base y_1...\base y_m$ in \catcw then
\newcommand{\xyj}[1]{\crossx{x}{y_{#1}}{1}}
\begin{equation*}
\duple{s(p_{\xyj{m},\xyj{1}},...s(p_{\xyj{m},\xyj{m}})} = q(p_{x,1},y_m)
\end{equation*}
\end{lemma}
\begin{proof}
\tbd
\end{proof}



\section{Supplementary Lemmas}
Using the notation just introduced we can give a modified version of lemma \lref{substitutioninterpretationlemma}.

\begin{lemma}
\llabel{supplementarylemma}
If $I$ is an instance of the generalised algebraic theory $U$ in a contextual category \catcw then
if  $Q$ and $\encyOmega{m}$ are contexts, for some $m \geq 1$, 
and if $\tuple{\sm}$ is a realisation of $\encyOmega{m}$ wrt $Q$ so that
 \foreachj, the rule \IsOmega{j} which we denote $r_{s_j}$ is a derived rule of $U$, then
\begin{enumerate}[(i)]
\item
if the rule \ZOmega which we denote $r_\Omega$ is a derived rule of $U$ then
$$\displaystyle\Imappedrule{Q}{\isT{\Omega[s_1|y_1...s_m|y_m]}}=\duple{I(r_{s_1}),...I(r_{s_m})}^*I(r_\Omega)$$
\item if \ZsOmega which we denote $r_s$ is a derived rule of $U$ then 
$$\displaystyle\Imappedrule{Q}{\ofT{s[s_1|y_1...s_m|y_m]}{\Omega[s_1|y_1...s_m|y_m]}}=\duple{I(r_{s_1}),...I(r_{s_m})}^*I(r_s)$$
\end{enumerate}
\end{lemma}
\begin{proof}
\tbd
\end{proof}


\begin{lemma}
\llabel{supplementarytuplelemma}
\highlight{Live without this perhaps because of the ambiguity of the tuple notation and the complexity of making it more precise?}
If $I$ is an instance of the generalised algebraic theory $U$ in a contextual category \catcw then
if  $Q$ and $\encyOmega{m}$ are contexts, for some $m \geq 1$, 
and if $\tuple{\sm}$ is a realisation of $\encyOmega{m}$ wrt $Q$ so that
 \foreachj, the rule \IsOmega{j} which we denote $r_{s_j}$ is a derived rule of $U$, then
 if \foreachj, there is a morphism $f_j$ such that $I(r_{s_j})=s(f_j)$ then \commentary{check that we do not need to specify the domain of each $f_j$.}
\begin{enumerate}[(i)]
\item
if the rule \ZOmega which we denote $r_\Omega$ is a derived rule of $U$ then
$$\displaystyle\Imappedrule{Q}{\isT{\Omega[s_1|y_1...s_m|y_m]}}=\tuple{f_1,...f_m}^*I(r_\Omega)$$
\item if \ZsOmega which we denote $r_s$ is a derived rule of $U$ then \commentary{unfortunate double use of $s$.}
$$\displaystyle\Imappedrule{Q}{\ofT{s[s_1|y_1...s_m|y_m]}{\Omega[s_1|y_1...s_m|y_m]}}=\tuple{f_1,...f_m}^*I(r_s)$$
\end{enumerate}
\end{lemma}
\begin{proof}
\tbd
\end{proof}



\begin{lemma}
\llabel{supplementaryweakeninglemma}
If $I$ is an instance of the generalised algebraic theory $U$ in a contextual category \catcw then
if  $Q$ and $\encyOmega{m}$ are contexts, for some $m \geq 1$,  then
\begin{enumerate}[(i)]
\item if the rule \ZOmega which we denote $r_\Omega$ is a derived rule of $U$ then
$$\displaystyle\Imappedrule{Q,\, \yOmega{m}}{\isT{\Omega}}=\crossx{I(Q)}{I(r_\Omega)}{1}$$
\item if \ZsOmega which we denote $r_s$ is a derived rule of $U$ then 
$$\displaystyle\Imappedrule{Q,\, \yOmega{m}}{\ofT{s}{\Omega}}=\crossx{I(Q)}{I(r_s)}{1}$$
\end{enumerate}
\end{lemma}
\begin{proof}
\tbd
\end{proof}
%\fi
\section{Examples}
\label{examples}
I give two examples to illustrate how the the main definitions may be applied. 
\commentary{Something about \highlight{$\hat tm$} and \highlight{$\hat tc$}?}
\subsection{Approach -- Conventions}

\subsubsection*{Naming of objects and morphisms}
\label{projectionnaming}
It is convenient to adopt some naming conventions when discussing the structure in a contextual category $\catcw$ into which an interpretation $I$ maps a theory $U$. Firstly it is convenient that the object $I(X)$ which is the interpretation of some sort $X$ of the theory is simply referred to as $X$.
Secondly if $X$ is a sort symbol such that $1 \base X$ in $\catcw$ then it is conventient to adopt some naming conventions for
the projection morphisms: the two canninical projection morphims $X^2 \morph X$ with be referred to as $x_1$ and $x_2$. The three cannonical projection morphisms $X^3 \morph X$ will be referred to as $y_1, y_2$ and $y_3$ and the four projection morphisms $X^4 \morph X$ will be referred to as 
$z_1, z_2, z_3$ and $z_4$. For sake of  uniformity  and considering that  $X^1$ is $X$ with product diagram  $id_X: X \morph X$  in the definitions
below we refer to $id_X$ as $w_1$.
Now in a contextual category, for any object $X$ such that $1 \base X$, $id_X = q(id_1,X) = q(t_1,X)$. Therefore we can define our uses of
$w_1$, $x_1,x_2,y_1,y_2,y_3,z_1,z_2,z_3$ and $z_4$ with respect to powers of an object $X$ by: \\
\begin{tabular} {l p{1cm} l p{1cm} l p{1cm} l}
$w_1 = q(t_1,X)$ &&  $x_1 = p_{X^2} \circ w_1$ && $y_1 = p_{X^3} \circ x_1$  && $z_1 = p_{X^4} \circ y_1$ \\
                 &&  $x_2 = q(t_X,X)$          && $y_2 = p_{X^3} \circ x_2$  && $z_2 = p_{X^4} \circ y_2$ \\
                 &&                            && $y_3 = q(t_{X^2},X)$       && $z_3 = p_{X^4} \circ y_3$ \\
                 &&                            &&                            && $z_4 = q(t_{X^3},X)$
\end{tabular}


Finally, we will see in the examples that it aids  readability to shorten  $p_A \circ f$ to $\dot{f}$, $p_B \circ p_A \circ f$ to $\ddot{f}$ and so on.
Therefore the above definitions can be rewritten as: \\
\begin{tabular} {l p{1cm} l p{1cm} l p{1cm} l}
$w_1 = q(t_1,X)$ &&  $x_1 = \dot{w_1}$ && $y_1 = \dot{x_1}$  && $z_1 = \dot{y_1}$ \\
                 &&  $x_2 = q(t_X,X)$          && $y_2 = \dot{x_2}$  && $z_2 = \dot{y_2}$ \\
                 &&                            && $y_3 = q(t_{X^2},X)$       && $z_3 = \dot{y_3}$ \\
                 &&                            &&                            && $z_4 = q(t_{X^3},X)$
\end{tabular}













 
  

\subsubsection{References from tables}

In the workings of these examples references (i), (ii), (iii) \highlight{etc.} refer to clauses in the  definition of instance given in 
section \ref{sectioninwhichinstanceisdefined}.

\subsection{Internal Monoids -- instances of the Theory of Monoids within Contextual Categories}
\label{monoidsinstanceexample} 

Now we consider what constitutes a monoid internal to a contextual category \catc. That is to say 
we consider what constitutes an instance $I$ of the theory of monoids   in a contextual category \catc.
For this purpose we shall write the theory of monoids ($tm$) as a generalised algebraic theory like this: 
\begin{gatrules}
\gatintros
\gatintroducing{M}
\isT{M} \\
\gatintroducing{unit}
\ofT{unit}{M} \\
\gatintroducing{mult}
\gatsingular[7.5cm]{\ofT{x_1,x_2}{M}}{\ofT{mult(x_1,x_2)}{M}} \\
\gataxioms

\gatintroducing{ \gataxiomno{1} \\ \gataxiomno{2} }
\begin{gatgroup}{\ofT{w}{M}}
\gatleaf[7.5cm]{}{mult(unit,w)=w} \\
\gatleaf[7.5cm]{}{mult(w,unit)=w}
\end{gatgroup} \\
\gatintroducing{ \gataxiomno{3} }
\gatsingular[7.5cm]{\ofT{y_1,y_2,y_3}{M}}{mult(mult(y_1,y_2),y_3)=mult(y_1,mult(y_2,y_3))} 
\end{gatrules}

For the sakes of readability we write the interpretation $I(M)$ of the sort $M$ simply as $M$. Similarly we write $I(unit)$ as $unit$. We write $I(mult)$ as $m$. I will ask the reader  to distinguish for themselves 
those uses of `$M$' and `$unit$' in reference to sorts and operators of the theory $tm$ from those uses in reference to the interpretation of these sorts by the instance $I$ in the contextual category \catc. 
  In the  lemma that follows we use the naming convention outlined above in section \ref{projectionnaming} and name the three projection morphisms $M^3 \morph M$ i.e. by $y_1, y_2, y_3$.
As discussed above these are defined by:

\begin{align*}
y_1 &= p_{M^3,M}, \\
y_2 &= p_{M^3,M^2} \circ q(p_{M,1},M), \\
y_3 &= q(p_{M^2,1},M). \hspace{1cm}\\
\end{align*}

\newcommand{\wM}{\ofT{w}{M}}
\newcommand{\xM}{\ofT{x_1, x_2}{M}}
\newcommand{\yM}{\ofT{y_1, y_2, y_3}{M}}
\newcommand{\doubleM}{M^2}                       %{\crossx{M}{M}{1}}
\newcommand{\trebleM}{M^3}                       %{\crossx{\big(\doubleM\big)}{M}{1}}
\newcommand{\quadM}{M^4}                         % {\crossx{\big(\trebleM\big)}{M}{1}}
\newcommand{\spi}{s(p_{M^3,M^i})}
\newcommand{\sptrebleone}{s(p_{M^3,M^1})}
\newcommand{\sptrebletwo}{s(p_{M^3,M^2})}
\newcommand{\sptreblethree}{s(p_{M^3,M^3})}
\newcommand{\fmult}{m}  %macro used for name of section that monoidal multiplication maps to

\begin{lemma}
\label{internalmonoidlemma}
An internal monoid in a contextual category \catcw consists of
\begin{enumerate}[(a)]
\item 
\begin{itemize}
\item An object $M$ of \catcw which is the interpretation of the sort $M$ of theory $tm$,
\item A section $\fmult$ of \catc, $\fmult \in Sect(\trebleM)$, that is the interpretation of the operator $mult$,
\item A section $unit$ of \catc, $unit \in Sect(M)$, that is the interpretation of the operator $unit$,
\end{itemize}
such that
\begin{equation}
\label{internalmonoidrepresentation1axiom1}
\tuple{p_M \circ unit,id_M}^*\fmult=s(id_M),
\end{equation}
\begin{equation}
\label{internalmonoidrepresentation1axiom2}
\tuple{id_M,p_M \circ unit}^*\fmult=s(id_M),
\end{equation}
and
\begin{equation}
\label{internalmonoidrepresentation1axiom3}
\bigtuple{(\tuple{y_1,y_2}^*\fmult)\circ q(p_{M^3,1},M),y_3}^*\fmult
=\bigtuple{y_1,(\tuple{y_2,y_3}^*\fmult)\circ q(p_{M^3,1},M)}^*\fmult
\end{equation}.
\end{enumerate}
Or, equivalently, 
\begin{enumerate}[(b)]
\item
\begin{itemize}
\item An object $M$ of \catc,
\item A morphism $mult$ of \catc, $mult: \doubleM \morph M$ in \catc,
\item A morphism $unit$ of \catc, $unit: 1 \morph M$ in \catc
\end{itemize}
such that
\begin{equation}
\label{internalmonoidrepresentation2axiom1}
\tuple{p_M \circ unit,id_M}\circ mult=id_M,
\end{equation}
\begin{equation}
\label{internalmonoidrepresentation2axiom2}
\tuple{{id_M,p_M \circ unit}}\circ mult=id_M
\end{equation}
and
\begin{equation}
\label{internalmonoidrepresentation2axiom3}
\bigtuple{\tuple{y_1,y_2}\circ mult,y_3}\circ mult
=\bigtuple{y_1,\tuple{y_2,y_3} \circ mult}\circ mult
\end{equation}
\end{enumerate}
Note that equation (\ref{internalmonoidrepresentation2axiom3})  can also be written as 
\begin{equation}
\label{internalmonoidrepresentation2axiom3rewritten}
\tuple{\crossx{mult}{id_M}{1}}\circ mult = \tuple{\crossx{id_M}{mult}{1}}\circ mult.
\end{equation}
\end{lemma}
\begin{proof}
We show the derivation of the first of these representations, representation (a),  in table \ref{internalmonoidtable} below. 

It remains to show that the second representation, representation (b) can be derived from the first representation.

To do this, first of all define $mult: \doubleM \morph M$ in \catcw from $\fmult: \doubleM \morph \trebleM$ by defining
$mult = \fmult \circ q(p_{\doubleM,1} , M)$. Now we show that each of the equations (\ref{internalmonoidrepresentation1axiom1}),
(\ref{internalmonoidrepresentation1axiom2}), and (\ref{internalmonoidrepresentation1axiom3}) hold of $\fmult$ iff
the respective equation (\ref{internalmonoidrepresentation2axiom1}),
(\ref{internalmonoidrepresentation2axiom2}) or (\ref{internalmonoidrepresentation2axiom3}) hold of $mult$.

That (\ref{internalmonoidrepresentation1axiom1}) holds iff (\ref{internalmonoidrepresentation2axiom1}) holds follows
because by lemma \lref{stactic}, (\ref{internalmonoidrepresentation2axiom1}) holds iff
\begin{equation}
\label{equivalenceone}
s(\tuple{p_M \circ unit,id_M} \circ mult) = s(id_M)
\end{equation}
which holds iff equation (\ref{internalmonoidrepresentation1axiom1}) holds because 
\begin{align*}
s(\tuple{p_M \circ unit,id_M}\circ mult) 
             &= s(\tuple{p_M \circ unit,id_M}\circ s(mult)) && \mbox{by lemma \lref{sfsglemma},} \\
             &= \tuple{p_M \circ unit,id_M} ^* s(mult )  && \mbox{by lemma \lref{sfglemma},} \\
             &= \tuple{p_M \circ unit,id_M} ^* s(\fmult \circ q(p_{\doubleM,1} , M)) && \mbox{definition of $mult$,} \\
			       &= \tuple{p_M \circ unit,id_M} ^* \fmult                                &&  \mbox{by axiom (s2).}
\end{align*}

\noindent By the same reasoning, it follows that (\ref{internalmonoidrepresentation1axiom2}) holds iff (\ref{internalmonoidrepresentation2axiom2}) holds.

We shall show that (\ref{internalmonoidrepresentation1axiom3}) holds iff (\ref{internalmonoidrepresentation2axiom3}) holds.
First note, though, that
\begin{equation}
\label{thirdaxiomsubgoal}
 \tuple{y_1,y_2}\circ \fmult \circ q(p_{\doubleM,1} , M) = (\tuple{y_1,y_2}^*\fmult)\circ q(p_{M^3,1},M)
\end{equation}
because
\begin{align*}
lhs &= (\tuple{y_1,y_2}^*\fmult) \circ q(\tuple{y_1,y_2},M^3) \circ q(p_{M^2,1},M)  && \mbox{by definition of $^*$, } \\
    &= (\tuple{y_1,y_2}^*\fmult) \circ q(\tuple{y_1,y_2} \circ p_{M^2,1},M)         && \mbox{by (q5),} \\
    &= (\tuple{y_1,y_2}^*\fmult) \circ q(p_{M^3,1},M)                               && \mbox{because $1$ is terminal. }\\
    &= rhs
\end{align*}

Now that (\ref{internalmonoidrepresentation1axiom3}) holds iff (\ref{internalmonoidrepresentation2axiom3}) holds
follows because by lemma \ref{stactic}, (\ref{internalmonoidrepresentation2axiom3}) holds iff

\begin{equation}
\label{sofinternalmonoidrepresentation2axiom3}
s(\bigtuple{\tuple{y_1,y_2}\circ mult,y_3}\circ mult)
=s(\bigtuple{y_1,\tuple{y_2,y_3} \circ mult}\circ mult)
\end{equation}
and the lhs of (\ref{sofinternalmonoidrepresentation2axiom3}) and the lhs of (\ref{internalmonoidrepresentation1axiom3}) 
are identical because
\begin{align*}
s(\bigtuple{\tuple{y_1,y_2}\circ mult,y_3}\circ mult) 
    &= \bigtuple{\tuple{y_1,y_2}\circ mult,y_3} ^* s(mult) && \mbox{by lemma \lref{sfglemma},} \\
		&= \bigtuple{\tuple{y_1,y_2}\circ mult,y_3} ^* s(\fmult \circ q(p_{\doubleM,1} , M)) && \mbox{definition of $mult$,} \\
		&= \bigtuple{\tuple{y_1,y_2}\circ mult,y_3} ^* \fmult                                &&  \mbox{by axiom (s2),} \\
		&= \bigtuple{\tuple{y_1,y_2}\circ \fmult \circ q(p_{\doubleM,1} , M),y_3} ^* \fmult  
		                                                                    &&  \mbox{definition of $mult$,} \\	
	  &= \bigtuple{(\tuple{y_1,y_2}^*\fmult)\circ q(p_{M^3,1},M),y_3} ^* \fmult 
		                                                                    &&  \mbox{by (\ref{thirdaxiomsubgoal}).}
\end{align*}
Similarly, the rhs of (\ref{sofinternalmonoidrepresentation2axiom3}) and the rhs of (\ref{internalmonoidrepresentation1axiom3}) 
can be shown to be identical.

\begin{table}[H]
\caption{Deriving what constitutes an instance of the theory of monoids $tm$ in a contextual category \catc. \\
Part One - Introductory rules for $M$, $unit$ and $\fmult$. 
The result is shown (highlighted) in rows (\ref{tm1}), (\ref{tm11}) and  (\ref{tm13}). Other rows 
show intermediate steps. Each step other than the first is justified by reference to earlier steps.}
\label{internalmonoidtableA}
\setlength{\arrayrulewidth}{1mm}
\setlength{\tabcolsep}{2pt}
\begin{tabular}{l l  c  p{0cm} l  l}
\multicolumn{2}{l}{Derived Rule} &&& Interpretation by $I$ in \catcw & Reason why\\
\hline
\gatinterpretationintro {tm1}{}{\isT{M}}{M \in Cover(1)}{definitions \ref{contextmapping} (ii) and \ref{consistentinterpretation} (i)(a)} \\
\\[-0.1cm]
\gatinterpretationdetail{tm2}{\wM}{\isT{M}}{\doubleM \in Cover(M)}{\highlight{lemma \ref{supplementaryweakeninglemma} (i)} and (\ref{tm1})} \\[0.3cm]
\gatinterpretationdetail{tm4}{\xM}{\isT{M}}{\trebleM \in Cover(\doubleM)}{\highlight{lemma \ref{supplementaryweakeninglemma} (i)}, (\ref{tm2}) and (\ref{tm1}) } \\[0.3cm]
\gatinterpretationintro {tm11}{}{\ofT{unit}{M}}{unit \in Sect(M)}{definition \ref{consistentinterpretation} (ii)(a) and (\ref{tm1})} \\
\\[-0.1cm]
\gatinterpretationintro{tm13}{\xM}{\ofT{\fmult(x_1,x_2)}{M}}{\fmult \in Sect(\trebleM)}{definition \ref{consistentinterpretation} (ii)(a) and (\ref{tm4})} \\
\\[-0.1cm]
\end{tabular}
\end{table}



\begin{table}[H]
\caption{Deriving what constitutes an instance of the theory of monoids $tm$ in a contextual category \catc.\\
Part Two -- Interpretation of the axioms.
The conditions for the axioms to hold are shown (highlighted) in rows (\ref{tmax1}), (\ref{tmax2}) and (\ref{tmax3}).}
\label{internalmonoidtableB}
\setlength{\arrayrulewidth}{1mm}
\setlength{\tabcolsep}{2pt}
\begin{tabular}{l l  c  p{0cm} l  l}
\multicolumn{2}{l}{Derived Rule} &&& Interpretation by $I$ in \catcw & Reason why\\
\hline
%\gatinterpretationintro {tm1}{}{\isT{M}}{M \in Cover(1)}{definitions \ref{contextmapping} (ii) and \ref{consistentinterpretation} (i)(a)} \\
%\\[-0.1cm]
%\gatinterpretationdetail{tm2}{\wM}{\isT{M}}{\doubleM \in Cover(M)}{\highlight{lemma \ref{supplementaryweakeninglemma} (i)} and (\ref{tm1})} \\[0.3cm]
\gatinterpretationdetail{tm3}{\wM}{\ofT{w}{M}}{s(id_M) \in Sect(\doubleM)}{definition \ref{consistentinterpretation} (ii)(d) and (\ref{tm1})} \\[0.3cm]
%\gatinterpretationdetail{tm4}{\xM}{\isT{M}}{\trebleM \in Cover(\doubleM)}{\highlight{lemma \ref{supplementaryweakeninglemma} (i)}, (\ref{tm2}) and (\ref{tm1}) } \\[0.3cm]
% Not USED
%\gatinterpretationdetail{tm5}{\xM}{\ofT{x_1}{M}}{s(p_{M^2}) \in Sect(\trebleM)}{definition \ref{consistentinterpretation} (ii)(d) and (\ref{tm2}) \highlight{USED?}} \\[0.3cm]
%\gatinterpretationdetail{tm6}{\xM}{\ofT{x_2}{M}}{s(id_{M^2}) \in Sect(\trebleM)}{definition \ref{consistentinterpretation} (ii)(d) and (\ref{tm2})\highlight{USED?}} \\[0.3cm]
\gatinterpretationdetail{tm7}{\yM}{\isT{M}}{\quadM \in Cover(\trebleM)}{\highlight{lemma \ref{supplementaryweakeninglemma} (i)}, (\ref{tm1}) and (\ref{tm4})} \\[0.3cm]
\gatinterpretationdetail{tm8}{\yM}{\ofT{y_1}{M}}{\sptrebleone \in Sect(\quadM)}{definition \ref{consistentinterpretation} (ii)(d) and (\ref{tm4})} \\[0.3cm]
\gatinterpretationmapeqv{s(y_1)} 
												{ definition of $y_1$}\\[0.2cm]
\gatinterpretationdetail{tm9}{\yM}{\ofT{y_2}{M}}{\sptrebletwo \in Sect(\quadM)}{definition \ref{consistentinterpretation} (ii)(d) and (\ref{tm4})} \\[0.3cm]
\gatinterpretationmapeqv{s(p_{M^3,M^2} \circ q(p_{M,1},M)) } 
												{by axiom (s3) }\\[0.2cm]
\gatinterpretationmapeqv{ s(y_2)} 
												{definition of $y_2$ }\\[0.2cm]
\gatinterpretationdetail{tm10}{\yM}{\ofT{y_3}{M}}{s(id_{M^3}) \in Sect(\quadM)}{definition \ref{consistentinterpretation} (ii)(d) and (\ref{tm4})} \\[0.3cm]
\gatinterpretationmapeqv{s(q(p_{M^2,1},M)) } 
												{ by axiom (s3)}\\[0.2cm]
\gatinterpretationmapeqv{s(y_3)} 
												{definition of $y_3$ }\\[0.2cm]
%\gatinterpretationintro {tm11}{}{\ofT{unit}{M}}{unit \in Sect(M)}{definition \ref{consistentinterpretation} (ii)(a) and (\ref{tm1})} \\
%\\[-0.1cm]
\gatinterpretationdetail{tm12}{\wM}{\ofT{unit}{M}}{\crossx{M}{unit}{1} \in Sect(\doubleM)}{\highlight{lemma \ref{supplementaryweakeninglemma} (ii)}, (\ref{tm1}) and (\ref{tm11})} \\[0.3cm]\gatinterpretationmapeqv{s(p_M \circ unit)} 
												{by lemma \ref{crosssectionlemma}}\\[0.2cm]
%\gatinterpretationintro{tm13}{\xM}{\ofT{\fmult(x_1,x_2)}{M}}{\fmult \in Sect(\trebleM)}{definition \ref{consistentinterpretation} (ii)(a) and (\ref{tm4})} \\
%\\[-0.1cm]
\gatinterpretationdetail{tm14}{\wM}
                        {\ofT{\fmult(w,unit)}{M}}
                        {\duple{s(id_M),s(p_M \circ unit)}^*\fmult \in Sect(\doubleM)}                   
												{lemma \ref{supplementarylemma} (ii), (\ref{tm3}), (\ref{tm12}) and(\ref{tm13}) }\\[0.2cm]
\gatinterpretationmapeqv{\tuple{id_M,p_M \circ unit}^*\fmult} 
												{lemma \lref{newlemma}}\\[0.2cm]
\gatinterpretationdetail{tm15}{\wM}
                        {\ofT{\fmult(unit,w)}{M}}
                        {\duple{s(p_M \circ unit),s(id_M)}^*\fmult \in Sect(\doubleM)}
												{lemma \ref{supplementarytuplelemma} (ii), (\ref{tm12}), (\ref{tm3}) and (\ref{tm13}) } \\[0.2cm]
\gatinterpretationmapeqv{\tuple{p_M \circ unit,id_M}^*\fmult}
												{lemma \lref{newlemma} }\\[0.2cm]
\gatinterpretationdetail{tm16}{\yM}
                        {\ofT{\fmult(y_1,y_2)}{M}}
												{\duple{s(y_1),s(y_2)}^*\fmult}
												{lemma \ref{supplementarylemma} (ii), (\ref{tm8}), (\ref{tm9}) and (\ref{tm13})} \\[0.2cm]
\gatinterpretationmapeqv{\tuple{y_1,y_2}^*\fmult}
												{lemma \lref{newlemma} (ii)}                                                     \\[0.2cm]
\gatinterpretationdetail{tm17}{\yM}
                        {\ofT{\fmult(y_2,y_3)}{M}}
												{\duple{s(y_2),s(y_3)}^*\fmult}
												{lemma \ref{supplementarylemma} (ii), (\ref{tm9}), (\ref{tm10}) and (\ref{tm13})}  \\[0.2cm]
\gatinterpretationmapeqv{\tuple{y_2,y_3}^*\fmult} 
												{lemma \lref{newlemma} and axiom (s3)}\\[0.2cm]						
\gatinterpretationdetail{tm18}{\yM}
                        {\fmult(\fmult(y_1,y_2),y_3)}
												{\duple{\tuple{y_1,y_2}^*\fmult,s(y_3)}^*\fmult}
												{lemma \ref{supplementarylemma}, (\ref{tm16}), (\ref{tm10}) and (\ref{tm13})}  \\[0.2cm]
%\gatinterpretationmapeqv{\bigtuple{(\tuple{y_1,y_2}^*\fmult)\circ q(p_{M^3,1},M),y_3}^*\fmult} 
%												{lemma \ref{thedupletuplelemma} and axiom (s3) (twice)} \\[0.2cm]
\gatinterpretationmapeqv{\bigtuple{\duple{\tuple{y_1,y_2}^*\fmult},y_3}^*\fmult} 
												{by definition of $\duple{}$ ??????} \\[0.2cm]
\gatinterpretationmapeqv{\bigtuple{(\tuple{y_1,y_2}^*\fmult)\circ q(p_{M^3,1},M),y_3}^*\fmult}  
												{by definition of $\duple{}$ ????????} \\[0.2cm]
\gatinterpretationdetail{tm19}{\yM}
                        {\fmult(y_1,\fmult(y_2,y_3))}
												{\duple{s(y_1),\tuple{y_2,y_3}^*\fmult}^*\fmult}
												{lemma \ref{supplementarylemma}, (\ref{tm8}), (\ref{tm17}) and (\ref{tm13})} \\[0.2cm]
\gatinterpretationmapeqv{\duple{s(y_1),s(\tuple{y_2,y_3}\circ \fmult)}^*\fmult}
												{by lemma \lref{regardingfstarsection}} \\[0.2cm]
\gatinterpretationmapeqv{\duple{s(y_1),s(\tuple{y_2,y_3}\circ \fmult \circ q(p_{M^2,1},M))}^*\fmult}
												{by (s3)} \\[0.2cm]
\gatinterpretationmapeqv{\bigtuple{y_1,\tuple{y_2,y_3}\circ \fmult \circ q(p_{M^2,1},M)}^*\fmult}
												{by  lemma \lref{duplestuplelemma}} \\[0.2cm]
%\gatinterpretationmapeqv{\bigtuple{y_1, (\tuple{y_2,y_3}^*\fmult) \circ q(p_{M^3,M},M^2) \circ q(p_{M^2,1},M)}^*\fmult}
%												{by definition of $^*$} \\[0.2cm]
\gatinterpretationmapeqv{\bigtuple{y_1,(\tuple{y_2,y_3}^*\fmult)\circ q(p_{M^3,1},M)}^*\fmult} 
												{by definition of $^*$ and (q5)}\\[0.2cm]
\gatinterpretationaxcond{tmax1}{\wM}{\fmult(unit,w)=w}{\tuple{p_M \circ unit,id_M}^*\fmult=s(id_M)}{definition \ref{consistentinterpretation} (iv), (\ref{tm15}) and (\ref{tm3})} \\[0.2cm]
\arrayrulecolor{white}\hline
%\gatinterpretationaxeqv {\tuple{p_M \circ unit,id_M}\comp \duple{\fmult}=id_M}{could push transform back?} \\
%												\rowcolor{lightergrey}
\gatinterpretationaxcond{tmax2}{\wM}{\fmult(w,unit)=w}{\tuple{id_M,p_M \circ unit}^*\fmult=s(id_M)}{definition \ref{consistentinterpretation} (iv), (\ref{tm14}) and (\ref{tm3})} \\[0.2cm]
%\gatinterpretationaxeqv {\tuple{{id_M,p_M \circ unit}}\comp \duple{\fmult}=id_M}{new lemmas}  \\
%												\rowcolor{lightergrey}
\arrayrulecolor{white}\hline
\gatinterpretationaxcond{tmax3}{\yM}{\fmult(\fmult(y_1,y_2),y_3)}
                                     {\bigtuple{(\tuple{y_1,y_2}^*\fmult)\circ q(p_{M^3,1},M),y_3}^*\fmult} \\
																		 &\hspace{2cm}$=\fmult(y_1,\fmult(y_2,y_3))$
																		 &&& \cellcolor{lightergrey}\hspace{0.5cm}
																		    $=\bigtuple{y_1,(\tuple{y_2,y_3}^*\fmult)\circ q(p_{M^3,1},M)}^*\fmult$
																		                           &{definition \ref{consistentinterpretation} (iv), (\ref{tm18}) and (\ref{tm19})} 
\end{tabular}
\end{table}
\end{proof}
\newpage 

\subsection{Internal Categories -- instances of the Theory of Categories within Contextual Categories}
\label{categoriesinstanceexample}
 
\newcommand{\sect}{Sect}
\newcommand{\insect}[2]{#1 \in Sect(#2)}

\newcommand {\OO}{Ob^2}
\newcommand {\OOO}{Ob^3}
\newcommand {\OOOO}{Ob^4}
\newcommand{\HomOb}{\crossx{Hom}{Ob}{1}}
\newcommand{\fid}{\qq{id}}
\newcommand{\fcomp}{\qq{\kern-2pt\circ \kern-2pt}}

\newcommand{\leftidentitylhsterm}{({x_1}^*\qq{id})^*\tuple{x_1,x_1,x_2}^*\fcomp}
\newcommand{\rightidentitylhsterm}{({x_2}^*\qq{id})^*\tuple{x_1,x_2,x_2}^*\fcomp}
\newcommand{\HomHom}{\crossx{Hom}{Hom}{\OO}}

\newcommand {\yOOO}{\ofT{y_1,y_2,y_3}{Ob}}
\newcommand {\yOOOfH}{\yOOO,\,\ofT{f}{Hom(y_1,y_2)}}
\newcommand{\yOOOfHgH}{\yOOOfH,\,\ofT{g}{Hom(y_2,y_3)}}

\newcommand {\yOOOfHmapped}{\tuple{y_1,y_2}^*Hom}
\newcommand {\yOOOfHgHmapped}{\crossx{\yOOOfHmapped}{\tuple{y_2,y_3}^*Hom}{\OOO}}
\newcommand {\yOOOfHgHHmapped}{\crossx{\big(\yOOOfHgHmapped\big)}{{\tuple{y_1,y_3}^*Hom}}{\OOO}}
\newcommand{\gatinterpretationcontext}[1]{&\multicolumn{5}{p{15cm}}{#1}}


%Composition introductory rule
\newcommand{\compfour}{\tuple{y_1,y_2}^*Hom}
\newcommand{\compfive}{\tuple{\dot y_2,\dot y_3}^*Hom}
\newcommand{\compsix}{\tuple{\ddot y_1,\ddot y_3}^*Hom}


% Left identity axiom mapping
\newcommand{\leftidentitymapped}{\tuple{x_1,x_1,x_2,x_1\circ \fid,id_{Hom}}^*\fcomp=s(id_{Hom})}

% Right identity axiom mapping
\newcommand{\rightidentitymapped}{\tuple{x_1,x_1,x_2,id_{Hom},x_2\circ \fid}^*\fcomp=s(id_{Hom})}


%*****************************
% Associativity axiom mapping
%******************************
\newcommand {\zOOOO}{\ofT{z_1,z_2,z_3,z_4}{Ob}}
\newcommand{\associativitypremisepoppop}
       {\zOOOO,\,\ofT{f}{Hom(z_1,z_2)}}	
\newcommand{\associativitypremisepop}		
			{\associativitypremisepoppop,\,\ofT{g}{Hom(z_2,z_3)}}
\newcommand{\associativitypremise}
       {\associativitypremisepop,\,\ofT{h}{Hom(z_3,z_4)}}	
\newcommand{\associativitypremisereversed}
       {\ofT{f}{Hom(z_1,z_2)},\,\ofT{g}{Hom(z_2,z_3)},\,\ofT{h}{Hom(z_3,z_4)},\,
			                \ofT{z_1,z_2,z_3,z_4}{Ob}
                                }																												
\newcommand{\associativitypremisepoppopmapped}{\tuple{z_1,z_2}^*Hom}
\newcommand{\associativitypremisepopmapped}{\tuple{\dot z_2,\dot z_3}^*Hom}											
\newcommand{\associativitypremisemapped}{\tuple{\ddot z_3,\ddot z_4}^*Hom}
\newcommand{\Q}{\associativitypremisemapped}
\newcommand{\Qp}{\associativitypremisepopmapped}
\newcommand{\Qpp}{\associativitypremisepoppopmapped}
\newcommand{\assoczimapped}{s(p_{\Q,Ob^i})}
\newcommand{\assoczimappedintermediary}{s(p_{\Q,Ob^i}\circ q(p_{Ob^{i-1},1},Ob))}
\newcommand{\assocziremapped}{{s(\dddot z_i)}}
\newcommand{\assoctripledotzidefiniens}{p_{\Q,\OOOO}\circ z_i}
\newcommand {\assocfmapped}{s(p_{\Q,\Qpp})}
\newcommand {\assocgmapped}{s(p_{\Q,\Qp})}
\newcommand {\assochmapped}{s(id_{\Q})}
\newcommand {\assocfdefiniens}{p_{\Q,\Qpp}\circ q(\tuple{z_1,z_2},Hom)}
\newcommand {\assocgdefiniens}{p_{\Q,\Qp}\circ q(\tuple{\dot z_2, \dot z_3},Hom)}
\newcommand {\assochdefiniens}{q(\tuple{\ddot z_3, \ddot z_4},Hom)}
\newcommand {\assocfmappedintermediary}{s(\assocfdefiniens)}
\newcommand {\assocgmappedintermediary}{s(\assocgdefiniens)}
\newcommand {\assochmappedintermediary}{s(\assochdefiniens)}
\newcommand {\assocfremapped}{s(f)}
\newcommand {\assocgremapped}{s(g)}
\newcommand {\assochremapped}{s(h)}
\newcommand{\associativitylhstype}{\isT{{Hom(z_1,z_4)}}}
\newcommand{\associativitylhstypemappedinitially}{\duple{s(\dddot z_1),s(\dddot z_4)}^*Hom}
\newcommand{\associativitylhstypemapped}{\tuple{\dddot z_1,\dddot z_4}^*Hom}
\newcommand{\associativitylhstermtyping}{\ofT{(f \circ g) \circ h}{Hom(z_1,z_4)}}
\newcommand{\associativityrhstermtyping}{\ofT{f \circ (g \circ h)}{Hom(z_1,z_4)}}	
\newcommand {\assocfogmapped}{\tuple{\dddot z_1,\dddot z_2,\dddot z_3,f,g}^*\fcomp }
\newcommand {\assoclhsmapped}{\duple{s(\dddot z_1),s(\dddot z_3),s(\dddot z_4),\assocfogmapped,s(h)}^*\fcomp}
\newcommand {\assoclhsremapped}{\tuple{\dddot z_1,\dddot z_3,\dddot z_4,(\assocfogmapped) \circ q(\tuple{\dddot z_1,\dddot z_3},Hom),h}^*\fcomp}
\newcommand {\assocgohmapped}{\duple{s(\dddot z_1),s(\dddot z_2),s(\dddot z_3),s(g),s(h)}^*\fcomp }
\newcommand {\assocgohremapped}{\tuple{\dddot z_1,\dddot z_2,\dddot z_3,g,h}^*\fcomp }
\newcommand {\assocrhsmapped}{\duple{s(\dddot z_1),s(\dddot z_2),s(\dddot z_4),s(f),\assocgohremapped}^*\fcomp}
\newcommand {\assocrhsremapped}{\tuple{\dddot z_1,\dddot z_2,\dddot z_4,f,(\assocgohremapped) \circ q(\tuple{\dddot z_2,\dddot z_4},Hom)}^*\fcomp}

\newcommand{\assocequivalentlhs}{\tuple{\dddot z_1,\dddot z_3,\dddot z_4,\tuple{\dddot z_1,\dddot z_2,\dddot z_3,f,g}\circ \compmorph,h} \circ \compmorph}
\newcommand{\assocequivalentrhs}{\tuple{\dddot z_1,\dddot z_2,\dddot z_4,f,\tuple{\dddot z_1,\dddot z_2,\dddot z_3,g,h}\circ \compmorph} \circ \compmorph}

% remapping
\newcommand{\compmorph}{\text{`$\circ$\kern-2pt'}}%{\odot} %{\llcorner \circ \lrcorner}

% These two should maybe be moved into ccategories shared macros
\newcommand{\ccplaceholder}{\rule[-0.2cm]{0cm}{0.6cm}\kern0.2cm}
\newcommand{\rightend}[1] { \kern-0.2cm\Rnode{#1} {\ccplaceholder} }

\note
In this second example we describe the structure of internal categories by following the main definition and examining
what constitutes a valid interpretation of the (generalised algebraic) theory of categories ($tc$) in some 
 contextual category \catc.

\note The theory of categories ($tc$) that I work with is presented as follows:
\begin{gatrules}
\gatintros
\gatintroducing{Ob}
\isT{Ob} \\
\gatintroducing{Hom}
  \gatsingular{\ofT{x_1,x_2}{Ob}}{\isT{Hom(x_1,x_2)}} \\	
\gatintroducing{id}
  \gatsingular{\ofT{w}{Ob}}{\ofT{id(w)}{Hom(w,w)}} \\	
\gataxioms
\gatintroducing{  \gataxiomno{1} \\   \gataxiomno{2}}
\begin{gatgroup}{\ofT{f}{Hom(x_1,x_2)},\ \ofT{x_1,x_2}{Ob}}
    \gatleaf{}{id_{x_1} \circ f = f} \\
    \gatleaf{}{f \circ id_{x_2} = f}
\end{gatgroup} \\
\gatintroducing{ \gataxiomno{3} }
\gatsingular{\associativitypremisereversed}{(f \circ g) \circ h = f \circ (g \circ h)} 
\end{gatrules}

\note If $I$ is a valid interpretation of $tc$ in a contextual category \catc then the sorts $Ob$ and $Hom$ of $tc$ 
must be mapped by $I$  to objects $I(Ob)$ and  $I(Hom)$ of \catc.
Similarly  the operators symbols
$id$ and $\circ$ must be mapped to sections $I(id)$ and $I(\circ)$ of \catc.

Following the convention described earlier in section \ref{projectionnaming} we simplify  
the description that follows by writing $Ob$ for $I(Ob)$, $Hom$ for $I(Hom)$.
For a further simplification we write $\qq{id}$ for $I(id)$ and   $\qq{\circ}$ for $I(\circ)$.   I will ask the reader  to distinguish for themselves 
those uses of `$Ob$' and `$Hom$' in reference to sorts of the theory $tc$ from those uses in reference to the interpretation of these sorts in the contextual category \catc. 
\note With respect to the projection morphisms of powers of $Ob$, I will use the conventions described earlier in section \ref{projectionnaming}.
\note
In regard to the object $Ob^3$ in \catcw, for $i=1,2,3$, I define
 $y_i: \OOO \morph Ob$ to be the i'th projection morphism  i.e. $y_1 = p_{\OOO,Ob}$, $y_2 = p_{Ob^3}\circ q(t_{Ob},Ob)$,$y_3 = q(t_{Ob^2},Ob)$
This enables us to construct the following pullback
\begin{equation*}
\begin{array}{r  p{4cm} c}
\compfour     \rightend{Qpp} && \Rnode{Hom}{Hom}               \\ [1cm]
\OOO          \rightend{O3}  && \Rnode{O2}{Ob^2}              
\end{array}
\mbox{
\ncsar{Qpp}{O3}
\ncsar{Hom}{O2}
\ncarr{Qpp}{Hom}
\alabel{q(\tuple{y_1,y_2},Hom)}
\ncarr{O3}{O2}
\alabel{\tuple{y_1,y_2}}}
\end{equation*}														

In accordance with a convention mentioned earlier we now define $\dot y_i : \compfour \morph Ob$, for $i = 1,2,3$, 
                                     by $\dot y_i = p_{\compfour}\circ y_i$. \\
																																																												
Next we consider the pullback:

\begin{equation*}
\begin{array}{r  p{4cm} c}
\compfive     \rightend{Qpp} && \Rnode{Hom}{Hom}               \\ [1cm]
\compfour     \rightend{O3}  && \Rnode{O2}{Ob^2}              
\end{array}
\mbox{
\ncsar{Qpp}{O3}
\ncsar{Hom}{O2}
\ncarr{Qpp}{Hom}
\alabel{q(\tuple{\dot y_2,\dot y_3},Hom)}
\ncarr{O3}{O2}
\alabel{\tuple{\dot y_2,\dot y_3}}}
\end{equation*}	

and define   $\ddot y_i : \compfive \morph Ob$, for $i = 1,2,3$, 
                                     by $\ddot y_i = p_{\compfive}\circ \dot y_i$. \\
																		
Finally this enables us to construct  the pullback
\begin{equation*}
\begin{array}{r  p{4cm} c}
\compsix     \rightend{Qpp} && \Rnode{Hom}{Hom}               \\ [1cm]
\compfive     \rightend{O3}  && \Rnode{O2}{Ob^2}              
\end{array}
\mbox{
\ncsar{Qpp}{O3}
\ncsar{Hom}{O2}
\ncarr{Qpp}{Hom}
\alabel{q(\tuple{\ddot y_1,\ddot y_3},Hom)}
\ncarr{O3}{O2}
\alabel{\tuple{\ddot y_1,\ddot y_3}}}
\end{equation*}	
																	
\noindent and this is relevant to us because in the lemma that follows we will show that for an interpretation $I$ to be valid the section $I(\circ)$ must be a section of $\compsix$.																	
	
					
\note Next we turn to $Ob^4$.
Following the earlier convention we define the projection functions 
to be $z_1,z_2,z_3$ and $z_4$ so that for $i = 1, 2,3,4$, $z_i: \OOOO \morph Ob$. \\

Then we proceed to define   $\dot z_i : \associativitypremisepoppopmapped \morph Ob$
                                      by $\dot z_i = p_{\associativitypremisepoppopmapped}\circ z_i$, 
to define  $\ddot z_i : \associativitypremisepopmapped \morph Ob$ 
                                    by $\ddot z_i = p_{\associativitypremisepopmapped, \OOOO}\circ z_i$, 
and, finally, to define $\dddot z_i : \associativitypremisemapped \morph Ob$ 
                                      by $\ddot z_i = p_{\associativitypremisemapped, \OOOO}\circ z_i$ 	
so that for $i = 1, 2,3,4$ we have
\begin{equation*}
\begin{array}{r l p{4cm} c}
\associativitypremisemapped       \rightend{Q}  & \kern-0.2cm\rightend{Qright}                          \\ [1cm]
\associativitypremisepopmapped    \rightend{Qp} &  &&   \\ [1cm]
\associativitypremisepoppopmapped \rightend{Qpp}&  &&   \\ [1cm]
\OOOO                             \rightend{O4} & && \Rnode{Ob}{Ob}              
\end{array}
\mbox{
\ncsar{Q}{Qp}
\ncsar{Qp}{Qpp}
\ncsar{Qpp}{O4}
%\ncarr{Q}{Ob}
\ncarc[nodesepA=5pt,nodesepB=\arrnodesepB,offsetA=\arroffsetA,offsetB=\arroffsetB,arrowsize=5pt,arrowinset=0.7]{->}{Q}{Ob}
\alabel{\dddot z_i}
\ncarr{Qp}{Ob}
\alabel{\ddot z_i}
\ncarr{Qpp}{Ob}
\alabel{\dot z_i}
\ncarr{O4}{Ob}
\alabel{z_i}
}
\end{equation*} in \catcw.

The object $\associativitypremisemapped$ is relevant to us because it is the interopretation under $I$ of the premise
$\associativitypremise$ of the associativity axiom. This helps explain the next three definitions.
We define $f$, $g$ and $h$ by

\begin{align*}
f &= \assocfdefiniens,
g &= \assocgdefiniens,
h &= \assochdefiniens.
\end{align*}

With $f$, $g$ ad $h$ so defined, the following diagrams
\vspace{0.3cm}
\begin{equation*}
\ccsquareoutline{1.4cm}{1.2cm}{\associativitypremisemapped}{Hom}{\OOOO}{\OO}
\mbox{
\nccdar{TL}{BL}
\ncsar{TR}{BR}
\ccsquareacross{f}{\tuple{z_1, z_2}}
\kern -1cm %work around bug with lost arrow space bug
}
\ccsquareoutline{1.4cm}{1.2cm}{\associativitypremisemapped}{Hom}{\OOOO}{\OO}
\mbox{
\nccdar{TL}{BL}
\ncsar{TR}{BR}
\ccsquareacross{g}{\tuple{z_2, z_3}}
\kern -1cm %work around bug with lost arrow space bug
}
\ccsquareoutline{1.4cm}{1.2cm}{\associativitypremisemapped}{Hom}{\OOOO}{\OO}
\mbox{
\nccdar{TL}{BL}
\ncsar{TR}{BR}
\ccsquareacross{h}{\tuple{z_3, z_4}}
}
\end{equation*} commute in \catc.	\\


\begin{lemma}
\llabel{associativitycontextmapping}
If $I$ is an interpretation of the theory $tc$ in a contextual category \catcw then
$I$ maps the context $\tuple{\associativitypremise}$ to the object $\associativitypremisemapped$ in \catc.
\end{lemma}
\begin{proof}
% two width forcing commands
\newcommand {\forceSOURCEwidth}{\rule{5cm}{0pt}}  % so as to line up three different arrays
\newcommand {\forceTARGETwidth}{\rule{2.2cm}{0pt}}

From lemma \ref{Xnlemma} we have the following interpretation by $I$, for each $i$, $1 \leq i \leq 4$:
\begin{equation*}
\begin{array}{c c c}
\forceSOURCEwidth & & \forceTARGETwidth \\ [-0.1cm]
\gatdisplayrule{\zOOOO}{\ofT{z_i}{Ob}} & \Imapsto & s(z_i) 
\end{array}
\end{equation*}

From these mappings it follows by lemma \ref{supplementarylemma2} that 

\begin{equation*}
\begin{array}{c c c}
\forceSOURCEwidth & & \forceTARGETwidth \\ [-0.1cm]
\gatdisplayrule{\zOOOO}{\isT{Hom(z_1,z_2)}} & \Imapsto & \associativitypremisepoppopmapped 
\end{array}
\end{equation*}

The context $\associativitypremisepoppop$ is therefore mapped to $\associativitypremisepoppopmapped$.

Now it follows by lemma \ref{Xnlemma} that 

\begin{equation*}
\begin{array}{c c c}
\forceSOURCEwidth & & \forceTARGETwidth \\ [-0.1cm]
\gatdisplayrule{\associativitypremisepoppop}{\ofT{z_i}{Ob}} & \Imapsto & s(p_{\associativitypremisepoppopmapped} \circ z_i) \\
                                                            & = & s(\dot{z_i})
\end{array}
\end{equation*}
and therefore by  lemma \ref{supplementarylemma2} that 
\begin{equation*}
\begin{array}{c c c}
\forceSOURCEwidth & & \forceTARGETwidth \\ [-0.1cm]
\gatdisplayrule{\associativitypremisepoppop}{\isT{Hom(z_2,z_3)}} & \Imapsto & \associativitypremisepopmapped.
\end{array}
\end{equation*}

The context $\associativitypremisepop$ is therefore mapped to $\associativitypremisepopmapped$.

By application of lemma \ref{Xnlemma} again we establish that 

\begin{equation*}
\begin{array}{c c c}
\forceSOURCEwidth & & \forceTARGETwidth \\ [-0.1cm]
\gatdisplayrule{\associativitypremisepop}{\ofT{z_i}{Ob}} & \Imapsto & s(p_{\associativitypremisepopmapped} \circ z_i) \\
                                                            & = & s(\ddot{z_i})
\end{array}
\end{equation*}
and therefore by  lemma \ref{supplementarylemma2} that 
\begin{equation*}
\begin{array}{c c c}
\forceSOURCEwidth & & \forceTARGETwidth \\ [-0.1cm]
\gatdisplayrule{\associativitypremisepop}{\isT{Hom(z_3,z_4)}} & \Imapsto & \associativitypremisemapped.
\end{array}
\end{equation*}

The context $\associativitypremise$ is therefore mapped to $\associativitypremisemapped$.

\end{proof}



\begin{lemma}
\label{internalcategorylemma}
An internal category in a contextual category \catcw consists of
\begin{itemize}
\item An object $Ob$ of \catc,
\item an object $Hom \in Cover(O^2)$ in \catc,
\item A section $\fid \in Sect(s(id_{Ob})^*Hom)$ in \catc, 
\item A section $\fcomp \in Sect(\compsix) $ of \catc \\ 
													
\end{itemize}
such that
\begin{equation}
\label{leftidentityaxiom}
\leftidentitymapped
\end{equation}
\begin{equation}
\label{rightidentityaxiom}
\rightidentitymapped
\end{equation}
and
\begin{multline}
\label{associativityaxiom}
\assoclhsremapped\\
            = \assocrhsremapped
\end{multline}


Equivalently an internal category in a contextual category \catcw consists of
\item objects $Ob$ and  $Hom$  and a section $\fid$ in \catc,  as above, along with
\begin{itemize}
\item a morphism $\compmorph$ of \catc, $\compmorph: \compfive \morph Hom$ in \catc
\end{itemize}
such that
\begin{equation}
\label{leftidentityrepresentation2}
\tuple{x_1,x_1,x_2,x_1\circ \fid,id_{Hom}} \circ \compmorph =id_{Hom}
\end{equation}
\begin{equation}
\label{rightidentityrepresentation2}
\tuple{x_1,x_1,x_2,id_{Hom},x_2 \circ \fid} \circ \compmorph =id_{Hom}
\end{equation}
and
\begin{equation}
\label{associativityrepresentation2}
\assocequivalentlhs = \assocequivalentrhs
\end{equation}.
\end{lemma}
\begin{proof}
Of these two equivalent representations the first results from a literal reading of the definition of instance given earlier
along with the judicious choice of intermediate definitions made with readability in mind.
This is demonstrated in tables \ref{internalcategorytableone}  - \ref{internalcategorytablefour} below. 

To show that the second representation follows from the first then from $\fcomp$ define $\compmorph$ by defining $\compmorph=\fcomp \circ q(p_{\tuple{\ddot y_1,\ddot y_3}},Hom)$ and then it is easy to show that 
(\ref{leftidentityrepresentation2}) follows from (\ref{leftidentityaxiom}), 
(\ref{rightidentityrepresentation2}) follows from (\ref{rightidentityaxiom}) and
(\ref{associativityrepresentation2}) follows from (\ref{associativityaxiom})

Vice-versa, from the second representation follows the first if we deffine define $\fcomp$ from  $\compmorph$ by defining $\fcomp=s(\compmorph)$.

\begin{table}[H]
\caption{Deriving what constitutes an intepretation of the theory of categories $tc$ in a contextual category \catc.
Part One - Introductory rules for $Ob$, $Hom$ and $id$.
}
\label{internalcategorytableone}
%\setlength{\arrayrulewidth}{1mm}
\setlength{\tabcolsep}{2pt}
\begin{tabular}{l l  c  p{0cm} l  l}
\multicolumn{2}{l}{Derived Rule} &&& Interpretation by $I$ in \catcw & Reason why\\
\hline
\gatinterpretationintro {obintro}{}{\isT{Ob}}{Ob \in Cover(1)}{(i)}                                   \\
\gatinterpretationdetail{homintrohelper}{\ofT{x_1}{Ob}}{\isT{Ob}}{Ob^2 \in Cover(Ob)}
                                                               {(v) and (\ref{obintro})}             \\
\gatinterpretationintro {homintro}{\ofT{x_1}{Ob},\ofT{x_2}{Ob}}{\isT{Hom}}{Hom \in Cover(Ob^2)}
                                                               {(i) and (\ref{homintrohelper})}      \\
\gatinterpretationdetail{idintrohelperhelper}{\ofT{w}{Ob}}{\ofT{w}{Ob}}{s(id_{Ob})}
                                                               {(ii)(b) and (\ref{homintrohelper})}  \\
\gatinterpretationdetail{idintrohelper}{\ofT{w}{Ob}}
                                 {\isT{Hom(w,w)}}{s(id_{Ob})^*Hom }
                                 {(vii) (\ref{homintro}) and (\ref{idintrohelperhelper})}           \\
\gatinterpretationintro {idintro}{\ofT{w}{Ob}}{\ofT{id(w)}{Hom(w,w)}} 
                                 {\fid \in Sect(s(id_{Ob})^*Hom) }
                                 {(ii)and (\ref{idintrohelper})}                                      \\
\end{tabular}
\end{table}


\begin{table}[H]
\caption{Deriving what constitutes an intepretation of the theory of categories $tc$ in a contextual category \catc.
Part Two Introductory rule for $\circ$. Indication of the reasoning is not included due to lack of space. 
The reasoning follows the patterns indicated in accompanying tables \ref{internalcategorytableone} and \ref{internalcategorytablethree}.
}
\label{internalcategorytabletwo}
%\setlength{\arrayrulewidth}{1mm}
\setlength{\tabcolsep}{2pt}
\begin{tabular}{l l  c  p{0cm} l  l}
\multicolumn{2}{l}{Derived Rule} &&& Interpretation by $I$ in \catcw \\
\hline
\gatinterpretationdetail{comp1}{\ofT{x_1,x_2}{Ob}}{\isT{Ob}}{ \OOO \in Cover(\OO) }{}              \\
\gatinterpretationdetail{comp2}{\ofT{y_1,y_2,y_3}{Ob}}{\isT{Hom(y_1,y_2)}}{ \compfour \in Cover(\OOO) }{} \\
\gatinterpretationdetail{comp3}{\ofT{y_1,y_2,y_3}{Ob}, \ofT{f_1}{Hom(y_1,y_2)}}{\isT{Hom(y_2,y_3)}}
                        {  \compfive \in Cover(\compfour) }{} \\
\gatinterpretationdetail{comp4}{\ofT{y_1,y_2,y_3}{Ob}, \ofT{f}{Hom(y_1,y_2)},\ofT{g}{Hom(y_2,y_3)}} {\isT{Hom(y_1,y_3)}}
                        { \compsix \in Cover(\compfive) }{} \\
\gatinterpretationintro {compintro}	{\ofT{y_1,y_2,y_3}{Ob}, \ofT{f}{Hom(y_1,y_2)},\ofT{g}{Hom(y_2,y_3)}} 
                                    {\ofT{f \circ g}{Hom(y_1,y_3)}}
																    {\fcomp \in Sect(\compsix)}
\end{tabular}
\end{table}

\newcommand{\leftidentityidremapped}{s(\dot{x_1}\circ \fid)}
\newcommand{\leftidentityrhsmapped}{s(id_{Hom})}      
\newcommand{\leftidentitylhsmapped}{\duple{s(\dot{x_1}),s(\dot{x_1}),s(p_{Hom,\OO}),\leftidentityidremapped,s(id_{Hom})}^*\fcomp}
\newcommand{\leftidentitylhsremapped}{\tuple{\dot{x_1},\dot{x_1},\dot{x_2},\dot{x_1}\circ \fid,id_{Hom}}^*\fcomp}

\begin{table}[H]
\caption{Deriving what constitutes an intepretation of the theory of categories $tc$ in a contextual category \catc.
Part Three. The left identity axiom.
}
\label{internalcategorytablethree}
%\setlength{\arrayrulewidth}{1mm}
\setlength{\tabcolsep}{2pt}
\begin{tabular}{l l  c  p{0cm} l  l}
\gatinterpretationcontext{Let $P$ be the context $\ofT{x_1}{Ob},\,\ofT{x_2}{Ob},\,\ofT{f}{Hom(x_1,x_2)} $
                                 then from (\ref{homintro}) we have $P \mapsto Hom \in Cover(Ob^2)$.} \\
\hline
\gatinterpretationcontext{	 Define morphisms $\rule[-10pt]{0pt}{30pt}\Rnode{Hom}{Hom} \hspace{1cm} \Rnode{Ob}{Ob}$ 
               \ncline[nodesepA=5pt,nodesepB=5pt,offsetA=3pt,offsetB=3pt,arrowsize=5pt,arrowinset=0.7]{->}{Hom}{Ob}
							 \alabel{\dot{x_1}}
							 \ncline[nodesepA=5pt,nodesepB=5pt,offsetA=-3pt,offsetB=-3pt,arrowsize=5pt,arrowinset=0.7]{->}{Hom}{Ob}
							 \blabel{\dot{x_2}}
							  in \catcw by $\dot{x_1}=p_{Hom} \circ x_1$ and $\dot{x_2}=p_{Hom} \circ x_2$
								where $x_1$ and $x_2$ are the two projection morphisms $x1,x2:Ob^2 \morph Ob$.
							           } \\
\hline
\multicolumn{2}{l}{Derived Rule} &&& Interpretation by $I$ in \catcw & Reason why\\
\hline
\gatinterpretationdetail{rightidentity1}{P}{\isT{Ob}}{ \HomOb \in Cover(Hom) }{(v), (\ref{homintro}) and (\ref{obintro})}              \\
\gatinterpretationdetail{rightidentity2}{P}{\ofT{x_1}{Ob}}{ s(p_{Hom,Ob}) \in Section(\HomOb) }{(ii)(b)}                    \\
\gatinterpretationmapeqv        {s(\dot{x_1})}                                            {defn. of $\dot{x_1}$}             \\
\gatinterpretationdetail{rightidentity3}{P}{\ofT{x_2}{Ob}}{ s(p_{Hom,Ob^2}) \in Section(\HomOb) }{(ii)(b)}                  \\
\gatinterpretationmapeqv        {s(\dot{x_2})}                                            {defn. of $\dot{x_2}$}             \\
\gatinterpretationdetail{rightidentity4}{P}{\isT{Hom(x_1,x_1)}}{\tuple{\dot{x_1},\dot{x_1}}^*Hom \in Cover(Hom)} 
                                                             {lemma \ref{supplementarylemma2}, (\ref{homintro}), (\ref{rightidentity2}) and (\ref{rightidentity3})} \\
\gatinterpretationdetail{rightidentityidmapping}{P}{\ofT{id(x_1)}{Hom(x_1,x_1)}}{{\dot{x_1}}^*\fid \in Sect(\tuple{\dot{x_1},\dot{x_1}}^*Hom)}  
                                                                    {lemma \ref{supplementarylemma2}, (\ref{idintro}) and (\ref{rightidentity2})} \\
\gatinterpretationmapeqv       {\leftidentityidremapped}                                      {lemma \ref{sfglemma} check}     \\
\gatinterpretationdetail{rightidentityrhsmappping}{P}{\ofT{f}{Hom(x_1,x_2)}}{\leftidentityrhsmapped \in Sect(\HomHom) }{(ii)(b)}                         \\
\gatinterpretationdetail{rightidentitylhsmapping}{P}{\ofT{id(x_1) \circ f}{Hom(x_1,x_2)}}{\leftidentitylhsremapped     }
                                     {lemma \ref{supplementarylemma2}, (\ref{rightidentity2}), (\ref{rightidentity3}) and (\ref{rightidentityidmapping})} \\
																																			&&&&\hspace{3.5cm}$\in Sect(\HomHom)$&  and defn. of $\dot{x_2}$  \\
\gatinterpretationaxcond{tcaxiomone}{P}{id(x_1) \circ f = f}
                                       {\leftidentitylhsremapped=\leftidentityrhsmapped}{(\ref{rightidentitylhsmapping}) and (\ref{rightidentityrhsmappping})}    
\end{tabular}
\end{table}



\begin{table}[H]
\caption{Deriving what constitutes an intepretation of the theory of categories $tc$ in a contextual category \catc.
Part Four. Associativity axiom.
}
\label{internalcategorytablefour}
%\setlength{\arrayrulewidth}{1mm}
\setlength{\tabcolsep}{2pt}
\begin{tabular}{l l  c  p{0cm} l  l}
\gatinterpretationcontext{Let $Q$ be the context $\associativitypremise$} \\
\gatinterpretationcontext{then $Q \mapsto \associativitypremisemapped \in Cover(\associativitypremisepopmapped)$ in \catcw by lemma \ref{associativitycontextmapping}.}\\
\hline

\multicolumn{2}{l}{Derived Rule} &&& Interpretation by $I$ in \catcw & Reason why                   \\
\hline \\[-0.4cm]
\gatinterpretationdetail{assoczimapping}{Q}{\ofT{z_i}{Ob},\mbox{ for } i=1,2,3,4}{\assoczimapped}{(ii)(b)}   \\[0.2cm]
\gatinterpretationmapeqv          {\assoczimappedintermediary}                   {axiom (s3)}                \\[0.2cm]
\gatinterpretationmapeqv          {\assocziremapped}                   {by defn. of $\dddot z_i$}  \\[0.2cm]
\gatinterpretationdetail{assocfmapping}{Q}{\ofT{f}{Hom(z_1,z_2)}}{\assocfmapped}{(ii)(b)}             \\[0.2cm]
\gatinterpretationmapeqv          {\assocfmappedintermediary}                   {axiom (s3)}     \\[0.2cm]
\gatinterpretationmapeqv          {\assocfremapped}                             { be defn. of $f$}      \\[0.2cm]
%\gatinterpretationmapeqvsingle    {\assocfremapped \mbox{ where $f$ defined by $f = \assocfdefiniens$}} \\[0.2cm]
\gatinterpretationdetail{assocgmapping}{Q}{\ofT{g}{Hom(z_2,z_s)}}{\assocgmapped}{(ii)(b)}              \\[0.2cm]
\gatinterpretationmapeqv                                  {\assocgmappedintermediary} {axiom (s3)}      \\[0.2cm]
\gatinterpretationmapeqv          {\assocgremapped}                             { be defn. of $g$}      \\[0.2cm]
%\gatinterpretationmapeqvsingle    {\assocgremapped \mbox{ where $g$ defined by $g = \assocgdefiniens$}} \\[0.2cm]
\gatinterpretationdetail{assochmapping}{Q}{\ofT{h}{Hom(z_3,z_4)}}{\assochmapped}{(ii)(b)}               \\[0.2cm]
\gatinterpretationmapeqv                                  {\assochmappedintermediary}  {axiom (s3)}     \\[0.2cm]
%\gatinterpretationmapeqvsingle    {\assochremapped \mbox{ where $h$ defined by $h = \assochdefiniens$}} \\[0.2cm]
\gatinterpretationmapeqv          {\assochremapped}                             { be defn. of $h$}      \\[0.2cm] 

\gatinterpretationdetail{assocfgmapping}{Q}{\ofT{f \circ g}{Hom(z_1,z_3)}}
                                   { \assocfogmapped \in Sect(\tuple{\dddot z_1,\dddot z_3}^*Hom) }
																	 {lemma \ref{supplementarylemma}, (\ref{assocfmapping}) and (\ref{assocgmapping})}                  \\[0.2cm]
\gatinterpretationdetail{assoctypemapping}{Q}{\associativitylhstype}{\associativitylhstypemapped}{lemma \ref{supplementarylemma2}, (\ref{homintro}) and (\ref{assoczimapping})}    \\[0.2cm]
%\gatinterpretationmapeqv                     {\associativitylhstypemapped}{lemma \ref{thedupletuplelemma}}             \\[0.2cm]         
%\gatinterpretationdetail{assoctypemapping}{Q}{\associativitylhstype}{\associativitylhstypemapped}{\highlight{\tbd}}    \\[0.2cm]
\gatinterpretationdetail{assocLHSmapping}{Q}{\associativitylhstermtyping}{\assoclhsmapped}
                                            {lemma \ref{supplementarylemma}, (\ref{assocfgmapping}) and (\ref{assochmapping})}\\[0.2cm]
\gatinterpretationmapeqv                    {\assoclhsremapped}{lemma \ref{thedupletuplelemma} and  (s3)}\\[0.2cm]
\gatinterpretationdetail{assocghmapping}{Q}{\ofT{g \circ h}{Hom(z_2,z_4)}}
                                   { \assocgohremapped \in Sect(\tuple{\dddot z_2,\dddot z_4}^*Hom) }
																	 {lemma \ref{supplementarylemma2}, (\ref{assoczimapping}), (\ref{assocgmapping})}              \\[0.2cm]
\gatinterpretationdetailcontinuation{}{\hspace{2.2cm} and (\ref{assochmapping})}                                                   \\[0.2cm]
%\gatinterpretationmapeqv                    {\assocgohremapped}{lemma \ref{thedupletuplelemma}}\\[0.2cm]
\gatinterpretationdetail{assocRHSmapping}{Q}{\associativityrhstermtyping}
                                            {\assocrhsmapped \iffalse{\in Sect(\associativitylhstypemapped)}\fi}
																						    {lemma \ref{supplementarylemma}, (\ref{assocfmapping}) and (\ref{assocghmapping})} \\ [0.2cm]
\gatinterpretationmapeqv                    {\assocrhsremapped}{lemma \ref{thedupletuplelemma} and (s3)}\\[0.2cm]
\gatinterpretationaxcond{associativity}{Q}{(f \circ g) \circ h = f \circ (g \circ h)}
                                     { \assoclhsremapped  } \\
\gatinterpretationaxcondrhscontinuation{= \assocrhsremapped } {(\ref{assocLHSmapping}) and  (\ref{assocRHSmapping})}\\
\end{tabular}
\end{table}

%\ncarc[arcangle=#1,nodesepA=5pt,nodesepB=5pt,offsetA=#2pt,offsetB=#2pt,arrowsize=5pt,arrowinset=0.7]{->}{#3}{#4}
\iffalse
\begin{equation*}
\begin{array}{c}
\begin{array}{r c p{4cm} c}
\associativitypremisemapped        \rightend{Q}            \\ [1cm]
\associativitypremisepopmapped     \rightend{Qp}           \\ [1cm]
\associativitypremisepoppopmapped  \rightend{Qpp}          \\ [1cm]
\OOOO   \rightend{O4}                                   \\ [1cm]
\OOO    \rightend{O3}      &  & & \Rnode{H}{Hom}            \\ [1cm]
\OO     \rightend{O2}      & & & \Rnode{Hp}{\OO}           \\ [1cm]
Ob      \rightend{O}       & & & \Rnode{Hpp}{Ob}           \\ [1cm]
\Rnode{abs}{1}        \\ 
\end{array} \\
\mbox{ % sadly this is taking up horizontal space and pushing visible diagram to the left 
       % I tried \sbox0 but sadly this stopped the arrows being typeset also
			 % need to try and debug this sometime instead I have put as a row in an outer array 
			 % this will just cause one blank line
			 % change mbox to fbox to see what is happening
\ncsar{Q}{Qp}
\ncsar{Qp}{Qpp}
\ncsar{Qpp}{O4}
\ncsar{O4}{O3}
\ncsar{O3}{O2}
\ncsar{O2}{O}   
\ncsar{O}{abs}
\ncsar{H}{Hp}
\ncsar{Hp}{Hpp}
\ncsar{Hpp}{abs}
\ncarrNEGZZ[-10]{Q}{H}    \alabel{f}
\ncarrZ{Q}{H}             \alabel{g}
\ncarrZZ[10]{Q}{H}        \alabel{h}
\ncarrNEGZZ[-10]{O4}{Hpp} \alabel{z_1}
\ncarrZ{O4}{Hpp}          \alabel{z_2}
\ncarrZZ[10]{O4}{Hpp}     \alabel{z_3}
\ncarrZZZ[20]{O4}{Hpp}    \alabel{z_4}
}
\end{array}
\end{equation*}
\fi

\end{proof}





\fi

%unused
%\section{Unused}
%\note It seems to me that in the syntax of gats, \textit{types and terms} on the one hand and \textit{contexts and realisations} on the other are each mutually dependent. In the world 
of syntax neither can exist without the other.
\begin{center}
$
\begin{array}{ c }
\etype{\Rnode{TandT}{types\ and\ terms}} \\[1.3cm]
\etype{\Rnode{CandR}{contexts\ and\ realisations}}   
\end{array}                     
$
\setlength{\sarnodesepA}{7pt}
\setlength{\sarnodesepB}{7pt}
\setlength{\saroffsetA}{7pt}
\setlength{\saroffsetB}{7pt}
\ncsar[10]{TandT}{CandR}
\ncsar[10]{CandR}{TandT}
\end{center}

\note 
There now follows an aside relevant to the statement of a  condition for a cwf to be contextual. Consider  Lisp-style lists. These  are often described as algebraic data structures but of course such structures are not algebraic in the precise sense that we mean when we speak of many-sorted algebraic theories or generalised algebraic theories rather they require coproducts of types for their definition. In a type theory that is many-sorted algebraic (or generalised algebraic) augmented by coproducts Lisp-style the type of lists over some other type $A$ can be defined as the coproduct of the types of empty lists $Empty$ and non-empty lists $Cons$
 as follows:\\

\begin{minipage}{\textwidth}
\begin{notebox}[Theory of lists of elements of type A]

If the empty list is written as $nil$ and if the LISP list constructor $cons$ is represented
as an infix operator $::$ then the theory of a list of elements of type $A$ can expressed as follows: 
\begin{gatrules}
\gatintros
\gatintro{Empty}{}{\isT{Empty}}   
\gatintro{Cons}{}{\isT{Cons}}     
\gatintro{List}{}{List = Empty + Cons} 
\gatintro{nil}{}{\ofT{nil}{Empty}}    
\gatintro{::}{\ofT{x}{A},\ \ofT{l}{List}}{\ofT{x::l}{List}}
\gatintro{hd}{\ofT{l}{Cons}}{\ofT{hd(l)}{A}} 
\gatintro{tl}{\ofT{l}{Cons}}{\ofT{tl(l)}{List}}   
\gataxioms
\gataxiom{e=nil}{\ofT{e}{Empty}}                            
\gataxiom{hd(x::l)=x}{\ofT{l}{List},\  \ofT{x}{A}}         
\gataxiom{tl(x::l)=l}{\ofT{l}{List},\  \ofT{x}{A}}        
\gataxiom{hd(l)::tl(l)=l}{\ofT{l}{Cons}} 
\end{gatrules}
Expressed as sketch of a category we have a coproduct diagram
expressing $List = Empty + Cons$,
a product diagram expressing the product $A \times List$,
additional morphisms:
$nil: 1 \morph Empty$,
$cons : A \times List \morph Cons$,
$hd: Cons \morph A$ and
$tl: Cons \morph List$.

and identities (commuting diagrams) expressing that
morphisms $nil: 1 \morph Empty$ and $cons : A \times List \morph Cons$ are isomorphims 
with inverses, respectively, of the unique morphism $t_{Empty}: Empty \morph 1$
and the morphism $\tuple{hd,tl}:Cons \morph A \times List$. 

As an entity model:
\begin{center}
\begin{erdiagram}{4.15}{10.5166}

\eret{0.1}{-1.6}{1.433}{-1}{0.2}{1}\eretname{0.767}{-1.35}{}{1}
\eret{0.1}{-3.15}{1.433}{-2.55}{0.2}{1}\eretname{0.767}{-2.9}{}{A}
\eret{2.933}{-3.65}{4.525}{-1.85}{0.2}{1}\eretname{3.729}{-2.2}{}{AxList}
\eret{5.933}{-3.2}{9.717}{-0.5}{0.2}{1}\eretname{6.149}{-0.85}{l}{List}
\eret{7.133}{-1.6}{8.467}{-1}{0.2}{0}\eretname{7.8}{-1.35}{}{Empty}
\eret{7.133}{-2.8}{8.467}{-1.9}{0.2}{0}\eretname{7.8}{-2.25}{}{Cons}

% relationship nil
\errelname{1.583}{-1.15}{l}{nil}\errelname{6.983}{-1.15}{r}{t}\errelarm{1.433}{-1.3}{4.283}{-1.3}{1}{0}\errelarm{4.283}{-1.3}{7.133}{-1.3}{1}{0}
% relationship cons
\errelname{4.675}{-2.15}{l}{cons}\errelname{6.983}{-2.15}{r}{head,tail}\errelarm{4.525}{-2.3}{5.829}{-2.3}{1}{0}\errelarm{5.829}{-2.3}{7.133}{-2.3}{1}{0}
% relationship p2
\errelname{4.675}{-2.69}{l}{p2}\errelarm{4.525}{-2.84}{5.229}{-2.84}{1}{0}\errelarm{5.229}{-2.84}{5.933}{-2.84}{1}{0}\ercrowfoot{4.675}{-2.84}{4.525}{-2.69}{4.525}{-2.84}{4.525}{-2.99}{0}\eridrefrel{4.7748}{-2.7399999999999998}{-2.94}
% relationship p1
\errelname{2.783}{-2.7}{r}{p1}\errelarm{2.933}{-2.85}{2.183}{-2.85}{1}{0}\errelarm{2.183}{-2.85}{1.433}{-2.85}{1}{0}\ercrowfoot{2.783}{-2.85}{2.933}{-2.7}{2.933}{-2.85}{2.933}{-3}{0}\eridrefrel{2.6833}{-2.7499999999999996}{-2.9499999999999997}
\end{erdiagram}
 
\end{center}
\end{notebox}
\end{minipage}

%\bibliographystyle{alpha} 
\bibliographystyle{abbrv}
\bibliography{../SharedBibliography/temp/bibliography}
\end{document}
