


 

% Sublime Text: Ctrl-B -- to build
%               Shift-Ctrl-D -- duplicate line
 

\documentclass[10pt,a4paper]{article}
%\documentclass[10pt,a3paper]{article}
%


%ccategories.macros.tex 

% Macros for diagrams in contextual categories and related categories

\usepackage{twoopt}
\usepackage{scalerel} 
\usepackage{xargs}

%\usepackage{mathabx}  %Caused font problems
%\usepackage{MnSymbol}  % caused font problems

\newcommand{\conu}
{\mathbf{C}(U)}

\newcommand{\depu}
{\mathbf{D}(U)}


\newcommand{\reqt}{\textbf{R}}
\newcommand{\reqtc}[1][\catc]{\reqt_{#1}}
\newcommand{\reqtcp}[1][\catcp]{\reqt_{#1}}



\newcommand{\cat}[1]{\textbf{#1}}

\newcommand{\catc}{\cat{C}}
\newcommand{\catcw}{\cat{C}\ }
\newcommand{\catcp}[1][C]{\textbf{#1}'}
\newcommand{\catcpp}[1][C]{\textbf{#1}''}
\newcommand{\obj}[1]{\ensuremath{|\cat{#1}|}}
\newcommand{\ccat}[1][C]{\ensuremath{\mathbb{#1}} }
\newcommand{\ccatc}{contextual category \ccat}
\newcommand{\cobj}[2][]{\ensuremath{|\ccat[#2]|_{#1}}}
\newcommand{\cslice}[2]{\ensuremath{\ccat[#1]_{#2}}}
\newcommand{\csliceobj}[3][]{\ensuremath{|\mathbb{#2}_{#3}|_{#1} }}
\newcommand{\varset}[1][]{\ensuremath{V_{#1} }}
\newcommand{\localvarsets}{\ensuremath{\mathcal{V} }}
\newcommand{\Fam}{\ensuremath{\mathbb{F\mathrm{am}} }}
\newcommand{\Fin}{\ensuremath{\textbf{Fin}} }
\newcommand{\Finp}{\ensuremath{\textbf{Finp}} }
\newcommand{\Po}{\ensuremath{\textbf{Po}} }
\newcommand{\Famslice}[1]{\ensuremath{\mathbb{F\mathrm{am}}_{#1} }}
\newcommand{\Famobj}[1][]{\ensuremath{|\mathbb{F\mathrm{am}}|_{#1} }}
\newcommand{\Famsliceobj}[2][]{\ensuremath{|\mathbb{F\mathrm{am}}_{#2}|_{#1} }}
\newcommand{\morph}{\rightarrow}
\newcommand{\epi}{\twoheadrightarrow}
\newcommand{\base}{\triangleleft}
\newcommand{\comp}{\circ}
\newcommand{\cross}{\otimes}
\newcommand{\pc}[2]{d^{#1}_{#2}}
\newcommand{\sub}{^*}
\newcommand{\diag}{\delta}
\newcommand{\pbase}[1]{\tilde{#1}}
\newcommand{\tuple}[1]{\langle#1\rangle}
\newcommand{\ndidly}{\ensuremath{\Join_n}}

\newcommand{\product}[1]{\bigtimes_{#1}}
\newcommand{\productn}{\product{n}}
\newcommand{\crossx}[3]{#1 \underset{#3}{\cross} #2}
\newcommand{\fibrex}[3]{#1 \underset{#3}{\Join} #2}
\newcommand{\powerset}{\mathcal{P}}
\newcommand{\primeds}[1]{
\ensuremath{\mathcal{P}(#1)} }
\newcommand{\compset}{\ \dot{\circ}\, }

% darrow
%\newcommand{\darrow}{\rightarrowtriangle} %use \smorph instead
\newcommand{\smorph}{\rightarrowtriangle}

 
\newcommand\dhead{\scaleobj{0.6}{\triangleright}}
%\newcommand{\dmorph}{\, \mbox{---} \! \cdot \! \raisebox{1.1pt}{\dhead}}    % dot style
\newcommand{\dmorph}{\, \mbox{---}\kern-1pt\raisebox{1.1pt}{\dhead\kern-1.75pt\dhead}}\,     % double triangle style

% projection tree
%\newcommand{\proj}[2]{proj_{#2}(#1)}

\newcommand{\proj}[2]{
\ensuremath{\mathcal{P}_{#2}(#1)} }

%pstrick supplements for arrows

\newlength{\arrnodesepA}
\newlength{\arrnodesepB}
\newlength{\arroffsetA}
\newlength{\arroffsetB}

%Modified to 2pt from 0pt on 23 July 2018
\newcommand{\arreset}{
\setlength{\arrnodesepA}{2pt}
\setlength{\arrnodesepB}{2pt}
\setlength{\arroffsetA}{0pt}
\setlength{\arroffsetB}{0pt}
}
\arreset

\newcommand{\ncarr}[3][0]{\ncarc[arcangle=#1,nodesepA=\arrnodesepA,nodesepB=\arrnodesepB,offsetA=\arroffsetA,offsetB=\arroffsetB,arrowsize=5pt,arrowinset=0.7]{->}{#2}{#3}}
\newcommand{\ncdarr}[3][0]{\ncarc[linestyle=dashed,arcangle=#1,nodesepA=\arrnodesepA,nodesepB=\arrnodesepB,offsetA=\arroffsetA,offsetB=\arroffsetB,arrowsize=5pt,arrowinset=0.7]{->}{#2}{#3}}
\newcommand{\jcbarr}[4][0]{ % ncbarr is defined in some thridy party package so do not use!\emph{}
\ncarr[#1]{#3}{#4}
\nbput[labelsep=2pt]{\footnotesize $#2$}
}

\newcommand{\ncaarr}[4][0]{
\ncarr[#1]{#3}{#4}
\naput[labelsep=2pt]{\footnotesize $#2$}
}

% \alabel{label}[npos][labelsep_pts]
\newcommandx*\alabel[3][2=0.5,3=2,usedefault]{\naput[labelsep=#3pt,npos=#2]{\footnotesize $#1$}}
% \blabel{label}[npos][labelsep_pts]
\newcommandx*\blabel[3][2=0.5,3=2,usedefault]{\nbput[labelsep=#3pt,npos=#2]{\footnotesize $#1$}}


\newif \ifbars
% to supress display of bars use \barsfalse to swith them on use \barstrue
\barstrue 
% \idcomp mark an arrow as one component of an identifier
\newcommand{\idcomp}{\ifbars{\ncput[npos=0, nrot=:U]{\psline(0.2,-0.075)(0.2,0.075)}}\fi}  %add a bar to a node connection arrow
% pstrick supplements for s-arrows (previous name for d-arrow - should convert}

\newlength{\sarnodesepA}
\newlength{\sarnodesepB}
\newlength{\saroffsetA}
\newlength{\saroffsetB}
\newlength{\sarnodesepAsav}
\newlength{\sarnodesepBsav}

\newcommand{\sarreset}{
\setlength{\sarnodesepA}{0pt}
\setlength{\sarnodesepB}{0pt}
\setlength{\saroffsetA}{0pt}
\setlength{\saroffsetB}{0pt}
}

\sarreset

% sar - S-arrow
\newcommand{\ncsar}[3][0]{
\setlength{\sarnodesepAsav}{\sarnodesepA}
\setlength{\sarnodesepBsav}{\sarnodesepB}
\addtolength{\sarnodesepA}{3pt}
\addtolength{\sarnodesepB}{7pt}
\ncarc[nodesepA=\sarnodesepA,nodesepB=\sarnodesepB,offsetA=\saroffsetA,offsetB=\saroffsetB,arcangle=#1]{-}{#2}{#3}
\ncput[nrot=:R,npos=1]{\pstriangle(0,0)(.2,.2)}
\setlength{\sarnodesepA}{\sarnodesepAsav}
\setlength{\sarnodesepB}{\sarnodesepBsav}
}


% bsar - below labelled S-arrow
\newcommand{\ncbsar}[4][0]{
\ncsar[#1]{#3}{#4}
\nbput[labelsep=2pt]{\footnotesize $#2$}
}
% asar - above labelled S-arrow
\newcommand{\ncasar}[4][0]{
\ncsar[#1]{#3}{#4}
\naput[labelsep=2pt]{\footnotesize $#2$}
}

% OLD cdar - composite dependency arrow - dot tyle
\iffalse
\newcommand{\nccdar}[3][0]{
\setlength{\sarnodesepAsav}{\sarnodesepA}
\setlength{\sarnodesepBsav}{\sarnodesepB}
\addtolength{\sarnodesepA}{3pt}
\addtolength{\sarnodesepB}{11pt}
\ncarc[nodesepA=\sarnodesepA,nodesepB=\sarnodesepB,offsetA=\saroffsetA,offsetB=\saroffsetB,arcangle=#1]{-}{#2}{#3}
\ncput[nrot=:R,npos=1]{\pstriangle(0,0.1)(.2,.2)}
\ncput[nrot=:R,npos=1]{\psdot[dotsize=1pt](-0.0075,0.05)}   %!!
\setlength{\sarnodesepA}{\sarnodesepAsav}
\setlength{\sarnodesepB}{\sarnodesepBsav}
}
\fi

% cdar - composite dependency arrow Mark II - double trangle style
\newcommand{\nccdar}[3][0]{
\setlength{\sarnodesepAsav}{\sarnodesepA}
\setlength{\sarnodesepBsav}{\sarnodesepB}
\addtolength{\sarnodesepA}{3pt}
\addtolength{\sarnodesepB}{13pt}
\ncarc[nodesepA=\sarnodesepA,nodesepB=\sarnodesepB,offsetA=\saroffsetA,offsetB=\saroffsetB,arcangle=#1]{-}{#2}{#3}
\ncput[nrot=:R,npos=1]{\pstriangle(0,0)(.2,.2)}
\ncput[nrot=:R,npos=1]{\pstriangle(0,0.2)(.2,.2)}
\setlength{\sarnodesepA}{\sarnodesepAsav}
\setlength{\sarnodesepB}{\sarnodesepBsav}
}


% bcdar - below labelled composite dependency arrow
\newcommand{\ncbcdar}[4][0]{
\nccdar[#1]{#3}{#4}
\nbput[labelsep=2pt]{\footnotesize $#2$}
}
% acdar - above labelled composite dependency arrow
\newcommand{\ncacdar}[4][0]{
\nccdar[#1]{#3}{#4}
\naput[labelsep=2pt]{\footnotesize $#2$}
}


% rsar - recursive S-arrow
\newcommand{\ncrsar}[2]{
\setlength{\sarnodesepAsav}{\sarnodesepA}
\setlength{\sarnodesepBsav}{\sarnodesepB}
\addtolength{\sarnodesepA}{3pt}
\addtolength{\sarnodesepB}{7pt}
\ncloop[nodesepA=\sarnodesepA,nodesepB=\sarnodesepB,
        offsetA=\saroffsetA,offsetB=\saroffsetB,
        armA=0.7cm,armB=0.6cm,angleA=90,angleB=-90,loopsize=-1,linearc=0.4
				]{-}{#1}{#2}
\ncput[nrot=:R,npos=5]{\pstriangle(0,0)(.2,.2)}
\setlength{\sarnodesepA}{\sarnodesepAsav}
\setlength{\sarnodesepB}{\sarnodesepBsav}
}

% pstrick supplements for multi-arrows

\newlength{\marnodesepA}
\newlength{\marnodesepB}
\newlength{\maroffsetB}
\newlength{\marnodesepBsav}

\newcommand{\marreset}{
\setlength{\marnodesepA}{0pt}
\setlength{\marnodesepB}{0pt}
\setlength{\maroffsetB}{0pt}
}

\marreset

%ncmarr[#1 arcangle1][#2 arcangle2]{#3 name}{#4 domain1}{#5 domain2}{#6 junction}{#7 codomain}
\newcommandtwoopt{\ncmarr}[6][8][8]{%
\ncarc[nodesepA=\marnodesepA,nodesepB=0,arcangle=#1]{-}{#3}{#5}
\ncarc[nodesepB=0,arcangle=-#1]{-}{#4}{#5}
\ncarc[arcangle=#2,nodesepB=\marnodesepB,offsetB=\maroffsetB]{->}{#5}{#6}
}%


\newcommandtwoopt{\nchmarr}[6][8][8]{%
\ncarc[nodesepA=\marnodesepA,nodesepB=0,arcangle=#1]{-}{#3}{#5}
\ncarc[nodesepB=0,arcangle=#1]{-}{#4}{#5}
\ncarc[arcangle=#2,nodesepB=\marnodesepB,offsetB=\maroffsetB]{->}{#5}{#6}
}%

\newcommandtwoopt{\ncamarr}[7][8][8]{%
\ncmarr[#1][#2]{#4}{#5}{#6}{#7}
\naput[npos=.05]{$#3$}
}%
\newcommandtwoopt{\ncbmarr}[7][8][8]{%
\ncmarr[#1][#2]{#4}{#5}{#6}{#7}
\nbput[npos=.05]{$#3$}
}%

\newcommandtwoopt{\ncbhmarr}[7][8][8]{%
\nchmarr[#1][#2]{#4}{#5}{#6}{#7}
\nbput[npos=.05]{$#3$}
}%

\newcommandtwoopt{\ncmarrr}[7][8][8]{
\ncarc[nodesepB=0,arcangle=#1]{-}{#3}{#6}
\ncline[nodesepB=0]{-}{#4}{#6}
\ncarc[nodesepB=0,arcangle=-#1]{-}{#5}{#6}
\ncarc[nodesepA=0,arcangle=#2]{->}{#6}{#7}
}

\newcommandtwoopt{\ncamarrr}[8][8][8]{
\ncmarrr[#1][#2]{#4}{#5}{#6}{#7}{#8}
\naput[npos=.05]{$#3$}
}
\newcommandtwoopt{\ncbmarrr}[8][8][8]{
\ncmarrr[#1][#2]{#4}{#5}{#6}{#7}{#8}
\nbput[npos=.05]{$#3$}
}


% 6 June 2020
% Edges representing attributes and relationship graphs
%  Ep   - partial
%  Epm  - partial mono
%  Epe  - partial epi
%  Epme - partial mono epi
%  Et   - total
%  Etm  - total mono
%  Ete  - total epi
%  Etme - total mono epi
%  recursive edges (use nccircle)
%  rEp   - partial
%  rEpm  - partial mono
%  rEpe  - partial epi
%  rEpme - partial mono epi
%  rEt   - total
%  rEtm  - total mono
%  rEte  - total epi
%  rEtme - total mono epi

\newcounter{EangleA}
\newcounter{EangleB}
\newcounter{EmidangleA}
\newcounter{EmidangleB}

% Ep - Edge partial
\newcommandtwoopt{\Ep}[4][0][0]{
\crowsfootedEdge{#1}{#2}{#3}{#4}{dashed}{dashed}
}



% Epm - Edge partial mono
\newcommandtwoopt{\Epm}[4][0][0]{
\monoEdge{#1}{#2}{#3}{#4}{dashed}{dashed}
}


% Epe - Edge partial epi
\newcommandtwoopt{\Epe}[4][0][0]{
\crowsfootedEdge{#1}{#2}{#3}{#4}{dashed}{solid}
}

% Epme - Edge partial mono epi
\newcommandtwoopt{\Epme}[4][0][0]{
\monoEdge{#1}{#2}{#3}{#4}{dashed}{solid}
}

% Et - Edge total
\newcommandtwoopt{\Et}[4][0][0]{
\crowsfootedEdge{#1}{#2}{#3}{#4}{solid}{dashed}
}

% Etm - Edge total mono
\newcommandtwoopt{\Etm}[4][0][0]{
\monoEdge{#1}{#2}{#3}{#4}{solid}{dashed}
}

% Ete - Edge total epi
\newcommandtwoopt{\Ete}[4][0][0]{
\crowsfootedEdge{#1}{#2}{#3}{#4}{solid}{solid}
}

% Etme - Edge total mono epi
\newcommandtwoopt{\Etme}[4][0][0]{
\monoEdge{#1}{#2}{#3}{#4}{solid}{solid}
}

% crowsfootedEdge - \crowsfootedEdge[angleA][midpointangle]{startnode}{endnode}[startstyle][endstyle]
\newcommand{\crowsfootedEdge}[6]{
\setlength{\sarnodesepAsav}{\sarnodesepA}
\setlength{\sarnodesepBsav}{\sarnodesepB}
\addtolength{\sarnodesepA}{3pt}
\addtolength{\sarnodesepB}{3pt}
\setcounter{EangleA}{ #1 + #2}
\setcounter{EangleB}{180  - #1 + #2}
\setcounter{EmidangleA}{#2}
\setcounter{EmidangleB}{#2 + 180}
\nccurve[nodesepA=\sarnodesepA,nodesepB=\sarnodesepB,offsetA=\saroffsetA,offsetB=\saroffsetB,angleA=\theEangleA, angleB=\theEangleB,linestyle=none,linewidth=0]{->}{#3}{#4}
\ncput[nrot=:R,npos=0]{\psline(0,.1)(.075,0)}
\ncput[nrot=:R,npos=0]{\psline(0,.1)(-0.075,0)}
\ncput{\pnode(0,0){xxx}}
\nccurve[nodesepA=0,nodesepB=\sarnodesepB,offsetA=0,offsetB=\saroffsetB,angleA=\theEmidangleA, angleB=\theEangleB, linestyle=#6]{->}{xxx}{#4}
%the following provides context for any following label
\nccurve[nodesepA=\sarnodesepA,nodesepB=0,offsetA=\saroffsetA,offsetB=0,angleA=\theEangleA, angleB=\theEmidangleB,linestyle=#5]{-}{#3}{xxx}
\setlength{\sarnodesepA}{\sarnodesepAsav}
\setlength{\sarnodesepB}{\sarnodesepBsav}
}

% monoEdge - \monoEdge[angleA][midpointangle]{startnode}{endnode}[startstyle][endstyle]
\newcommand{\monoEdge}[6]{ 
\setlength{\sarnodesepAsav}{\sarnodesepA}
\setlength{\sarnodesepBsav}{\sarnodesepB}
\addtolength{\sarnodesepA}{3pt}
\addtolength{\sarnodesepB}{3pt}
\setcounter{EangleA}{ #1 + #2}
\setcounter{EangleB}{180  - #1 + #2}
\setcounter{EmidangleA}{#2}
\setcounter{EmidangleB}{#2 + 180}
\nccurve[nodesepA=\sarnodesepA,nodesepB=\sarnodesepB,offsetA=\saroffsetA,offsetB=\saroffsetB,angleA=\theEangleA, angleB=\theEangleB,linestyle=none,linewidth=0]{->}{#3}{#4}
\ncput{\pnode(0,0){xxx}}
\nccurve[nodesepA=0,nodesepB=\sarnodesepB,offsetA=0,offsetB=\saroffsetB,angleA=\theEmidangleA, angleB=\theEangleB, linestyle=#6]{->}{xxx}{#4}
%the following provides context for any following label
\nccurve[nodesepA=\sarnodesepA,nodesepB=0,offsetA=\saroffsetA,offsetB=0,angleA=\theEangleA, angleB=\theEmidangleB,linestyle=#5]{-}{#3}{xxx}
\setlength{\sarnodesepA}{\sarnodesepAsav}
\setlength{\sarnodesepB}{\sarnodesepBsav}
}


\newcounter{EangleGiven}
\newcounter{EangleComplementary}
\newcounter{EangleStartCorrected}
\newcounter{EangleEndCorrected}


%  rEp   - recursive Edge partial
\newcommand{\rEp}[2][0]{
\setcounter{EangleGiven}{#1}
\setcounter{EangleStartCorrected}{#1-10} %correction required because for nccurve unlike nccircle angle measured at boundary not at centre of node
\setcounter{EangleEndCorrected}{#1+180+10} %correction required because angle measured at boundary not at centre of node
\setcounter{EangleComplementary}{#1 + 180}
\nccircle[angleA=\theEangleComplementary, nodesep=0pt, linestyle=none]{-}{#2}{.4cm} % an invisible circle to hang the midpoint from
\ncput{\pnode(0,0){midpoint}}                                         
\nccurve[nodesepA=1pt,nodesepB=0pt,offsetA=0pt,offsetB=0pt,angleA=\theEangleStartCorrected, angleB=\theEangleGiven, ncurv=1.359, linecolor=black, linestyle=dashed]{-}{#2}{midpoint}
\ncput[nrot=:R,npos=0]{\psline(0,.1)(.075,0)}
\ncput[nrot=:R,npos=0]{\psline(0,.1)(-0.075,0)}
\nccurve[nodesepA=0pt,nodesepB=2pt,offsetA=0pt,offsetB=0pt,angleA=\theEangleComplementary, angleB=\theEangleEndCorrected, ncurv=1.359, linestyle=dashed]{-}{midpoint}{#2}
% 1.359 is e/2 happenchance or algorithmically necessary???
% now draw arrowhead -- dont include in the nccurve because this alters the line position - a strange feature of pstruicks
\ncput[npos=0.9]{\pnode(0,0){yyy}}
\ncline{->}{yyy}{#2}
% repeat from earlier to provide context for label that might follow
\nccurve[nodesepA=1pt,nodesepB=0pt,offsetA=0pt,offsetB=0pt,angleA=\theEangleStartCorrected, angleB=\theEangleGiven, ncurv=1.359, linecolor=black, linestyle=dashed]{-}{#2}{midpoint} 
} 

%  rEpm  - recursive Edge partial mono
\newcommand{\rEpm}[2][0]{
\setcounter{EangleGiven}{#1}
\setcounter{EangleStartCorrected}{#1-10} %correction required because for nccurve unlike nccircle angle measured at boundary not at centre of node
\setcounter{EangleEndCorrected}{#1+180+10} %correction required because angle measured at boundary not at centre of node
\setcounter{EangleComplementary}{#1 + 180}
\nccircle[angleA=\theEangleComplementary, nodesep=0pt, linestyle=none]{-}{#2}{.4cm} % an invisible circle to hang the midpoint from
\ncput{\pnode(0,0){midpoint}}   
\nccurve[nodesepA=0pt,nodesepB=2pt,offsetA=0pt,offsetB=0pt,angleA=\theEangleComplementary, angleB=\theEangleEndCorrected, ncurv=1.359, linestyle=dashed]{-}{midpoint}{#2}
% 1.359 is e/2 happenchance or algorithmically necessary???
% now draw arrowhead -- dont include in the nccurve because this alters the line position - a strange feature of pstruicks
\ncput[npos=0.9]{\pnode(0,0){yyy}}
\ncline{->}{yyy}{#2}
% last to provide context for label that might follow
\nccurve[nodesepA=1pt,nodesepB=0pt,offsetA=0pt,offsetB=0pt,angleA=\theEangleStartCorrected, angleB=\theEangleGiven, ncurv=1.359, linecolor=black, linestyle=dashed]{-}{#2}{midpoint} 
}

%  rEpe  - recursive Edge partial epi
\newcommand{\rEpe}[2][0]{
\setcounter{EangleGiven}{#1}
\setcounter{EangleStartCorrected}{#1-10} %correction required because for nccurve unlike nccircle angle measured at boundary not at centre of node
\setcounter{EangleEndCorrected}{#1+180+10} %correction required because angle measured at boundary not at centre of node
\setcounter{EangleComplementary}{#1 + 180}
\nccircle[angleA=\theEangleComplementary, nodesep=0pt, linestyle=none]{-}{#2}{.4cm} % an invisible circle to hang the midpoint from
\ncput{\pnode(0,0){midpoint}}                                         
\nccurve[nodesepA=1pt,nodesepB=0pt,offsetA=0pt,offsetB=0pt,angleA=\theEangleStartCorrected, angleB=\theEangleGiven, ncurv=1.359, linecolor=black, linestyle=dashed]{-}{#2}{midpoint}
\ncput[nrot=:R,npos=0]{\psline(0,.1)(.075,0)}
\ncput[nrot=:R,npos=0]{\psline(0,.1)(-0.075,0)}
\nccurve[nodesepA=0pt,nodesepB=2pt,offsetA=0pt,offsetB=0pt,angleA=\theEangleComplementary, angleB=\theEangleEndCorrected, ncurv=1.359]{-}{midpoint}{#2}
% 1.359 is e/2 happenchance or algorithmically necessary???
% now draw arrowhead -- dont include in the nccurve because this alters the line position - a strange feature of pstruicks
\ncput[npos=0.9]{\pnode(0,0){yyy}}
\ncline{->}{yyy}{#2}
% repeat from earlier to provide context for label that might follow
\nccurve[nodesepA=1pt,nodesepB=0pt,offsetA=0pt,offsetB=0pt,angleA=\theEangleStartCorrected, angleB=\theEangleGiven, ncurv=1.359, linecolor=black, linestyle=dashed]{-}{#2}{midpoint} 
}

%  rEpme - recursive Edge partial mono epi
\newcommand{\rEpme}[2][0]{
\setcounter{EangleGiven}{#1}
\setcounter{EangleStartCorrected}{#1-10} %correction required because for nccurve unlike nccircle angle measured at boundary not at centre of node
\setcounter{EangleEndCorrected}{#1+180+10} %correction required because angle measured at boundary not at centre of node
\setcounter{EangleComplementary}{#1 + 180}
\nccircle[angleA=\theEangleComplementary, nodesep=0pt, linestyle=none]{-}{#2}{.4cm} % an invisible circle to hang the midpoint from
\ncput{\pnode(0,0){midpoint}}                                         
%\nccurve[nodesepA=0pt,nodesepB=0pt,offsetA=0pt,offsetB=0pt,angleA=\theEangleComplementary, angleB=\theEangleEndCorrected, ncurv=1.359, linestyle=dashed]{->}{xxx}{#2}
\nccurve[nodesepA=0pt,nodesepB=2pt,offsetA=0pt,offsetB=0pt,angleA=\theEangleComplementary, angleB=\theEangleEndCorrected, ncurv=1.359]{-}{midpoint}{#2}
% 1.359 is e/2 happenchance or algorithmically necessary???
% now draw arrowhead -- dont include in the nccurve because this alters the line position - a strange feature of pstruicks
\ncput[npos=0.9]{\pnode(0,0){yyy}}
\ncline{->}{yyy}{#2}
% last so that to provide context for label that might follow
\nccurve[nodesepA=1pt,nodesepB=0pt,offsetA=0pt,offsetB=0pt,angleA=\theEangleStartCorrected, angleB=\theEangleGiven, ncurv=1.359, linecolor=black, linestyle=dashed]{-}{#2}{midpoint} 
}

% rEt - recursive Edge total
\newcommand{\rEt}[2][0]{
\setcounter{EangleGiven}{#1}
\setcounter{EangleStartCorrected}{#1-10} %correction required because for nccurve unlike nccircle angle measured at boundary not at centre of node
\setcounter{EangleEndCorrected}{#1+180+10} %correction required because angle measured at boundary not at centre of node
\setcounter{EangleComplementary}{#1 + 180}
\nccircle[angleA=\theEangleComplementary, nodesep=0pt, linestyle=none]{-}{#2}{.4cm} % an invisible circle to hang the midpoint from
\ncput{\pnode(0,0){midpoint}}                                         
\nccurve[nodesepA=1pt,nodesepB=0pt,offsetA=0pt,offsetB=0pt,angleA=\theEangleStartCorrected, angleB=\theEangleGiven, ncurv=1.359, linecolor=black]{-}{#2}{midpoint}
\ncput[nrot=:R,npos=0]{\psline(0,.1)(.075,0)}
\ncput[nrot=:R,npos=0]{\psline(0,.1)(-0.075,0)}
%\nccurve[nodesepA=0pt,nodesepB=0pt,offsetA=0pt,offsetB=0pt,angleA=\theEangleComplementary, angleB=\theEangleEndCorrected, ncurv=1.359, linestyle=dashed]{->}{xxx}{#2}
\nccurve[nodesepA=0pt,nodesepB=2pt,offsetA=0pt,offsetB=0pt,angleA=\theEangleComplementary, angleB=\theEangleEndCorrected, ncurv=1.359, linestyle=dashed]{-}{midpoint}{#2}
% 1.359 is e/2 happenchance or algorithmically necessary???
% now draw arrowhead -- dont include in the nccurve because this alters the line position - a strange feature of pstruicks
\ncput[npos=0.9]{\pnode(0,0){yyy}}
\ncline{->}{yyy}{#2}
% repeat from earlier to provide context for label that might follow
\nccurve[nodesepA=1pt,nodesepB=0pt,offsetA=0pt,offsetB=0pt,angleA=\theEangleStartCorrected, angleB=\theEangleGiven, ncurv=1.359, linecolor=black]{-}{#2}{midpoint} 
}

%  rEtm  - recursive Edge total mono
\newcommand{\rEtm}[2][0]{
\setcounter{EangleGiven}{#1}
\setcounter{EangleStartCorrected}{#1-10} %correction required because for nccurve unlike nccircle angle measured at boundary not at centre of node
\setcounter{EangleEndCorrected}{#1+180+10} %correction required because angle measured at boundary not at centre of node
\setcounter{EangleComplementary}{#1 + 180}
\nccircle[angleA=\theEangleComplementary, nodesep=0pt, linestyle=none]{-}{#2}{.4cm} % an invisible circle to hang the midpoint from
\ncput{\pnode(0,0){midpoint}}     
\nccurve[nodesepA=0pt,nodesepB=2pt,offsetA=0pt,offsetB=0pt,angleA=\theEangleComplementary, angleB=\theEangleEndCorrected, ncurv=1.359, linestyle=dashed]{-}{midpoint}{#2}
% 1.359 is e/2 happenchance or algorithmically necessary???
% now draw arrowhead -- dont include in the nccurve because this alters the line position - a strange feature of pstruicks
\ncput[npos=0.9]{\pnode(0,0){yyy}}
\ncline{->}{yyy}{#2}
% last to provide context for label that might follow
\nccurve[nodesepA=1pt,nodesepB=0pt,offsetA=0pt,offsetB=0pt,angleA=\theEangleStartCorrected, angleB=\theEangleGiven, ncurv=1.359, linecolor=black]{-}{#2}{midpoint} 
}

%  rEte  - total epi
\newcommand{\rEte}[2][0]{
\setcounter{EangleGiven}{#1}
\setcounter{EangleStartCorrected}{#1-10} %correction required because for nccurve unlike nccircle angle measured at boundary not at centre of node
\setcounter{EangleEndCorrected}{#1+180+10} %correction required because angle measured at boundary not at centre of node
\setcounter{EangleComplementary}{#1 + 180}
\nccircle[angleA=\theEangleComplementary, nodesep=0pt, linestyle=none]{-}{#2}{.4cm} % an invisible circle to hang the midpoint from
\ncput{\pnode(0,0){midpoint}}                                         
\nccurve[nodesepA=1pt,nodesepB=0pt,offsetA=0pt,offsetB=0pt,angleA=\theEangleStartCorrected, angleB=\theEangleGiven, ncurv=1.359, linecolor=black]{-}{#2}{midpoint}
\ncput[nrot=:R,npos=0]{\psline(0,.1)(.075,0)}
\ncput[nrot=:R,npos=0]{\psline(0,.1)(-0.075,0)}
%\nccurve[nodesepA=0pt,nodesepB=0pt,offsetA=0pt,offsetB=0pt,angleA=\theEangleComplementary, angleB=\theEangleEndCorrected, ncurv=1.359, linestyle=dashed]{->}{xxx}{#2}
\nccurve[nodesepA=0pt,nodesepB=2pt,offsetA=0pt,offsetB=0pt,angleA=\theEangleComplementary, angleB=\theEangleEndCorrected, ncurv=1.359]{-}{midpoint}{#2}
% 1.359 is e/2 happenchance or algorithmically necessary???
% now draw arrowhead -- dont include in the nccurve because this alters the line position - a strange feature of pstruicks
\ncput[npos=0.9]{\pnode(0,0){yyy}}
\ncline{->}{yyy}{#2}
% repeat from earlier to provide context for label that might follow
\nccurve[nodesepA=1pt,nodesepB=0pt,offsetA=0pt,offsetB=0pt,angleA=\theEangleStartCorrected, angleB=\theEangleGiven, ncurv=1.359, linecolor=black]{-}{#2}{midpoint} 
}

%  rEtme - recursive Edge total mono epi

\newcommand{\rEtme}[2][0]{
\setcounter{EangleGiven}{#1}
\setcounter{EangleStartCorrected}{#1-10} %correction required because for nccurve unlike nccircle angle measured at boundary not at centre of node
\setcounter{EangleEndCorrected}{#1+180+10} %correction required because angle measured at boundary not at centre of node
\setcounter{EangleComplementary}{#1 + 180}
\nccircle[angleA=\theEangleComplementary, nodesep=0pt, linestyle=none]{-}{#2}{.4cm} % an invisible circle to hang the midpoint from
\ncput{\pnode(0,0){midpoint}}     
\nccurve[nodesepA=0pt,nodesepB=2pt,offsetA=0pt,offsetB=0pt,angleA=\theEangleComplementary, angleB=\theEangleEndCorrected, ncurv=1.359]{-}{midpoint}{#2}
% 1.359 is e/2 happenchance or algorithmically necessary???
% now draw arrowhead -- dont include in the nccurve because this alters the line position - a strange feature of pstruicks
\ncput[npos=0.9]{\pnode(0,0){yyy}}
\ncline{->}{yyy}{#2}
% last to provide context for label that might follow
\nccurve[nodesepA=1pt,nodesepB=0pt,offsetA=0pt,offsetB=0pt,angleA=\theEangleStartCorrected, angleB=\theEangleGiven, ncurv=1.359, linecolor=black]{-}{#2}{midpoint} 
}

%The following are stylistic so belong in main document not here.
%\usepackage[margin=4.0cm]{geometry} % This shouldn't be here commented out 17 July 2018
%\usepackage{mathptmx}               % This changes font to roman so doesn't belong here
%
\usepackage{amsfonts}
\usepackage{amssymb} % added 08\02\2019 as an experiment. Needed in some instances for \blacksquare
                     % not needed is class is `beamer' but I don't know why not
\usepackage{array}
\usepackage{pstricks}
\usepackage{pst-tree}
\usepackage{pst-plot}
\usepackage{pst-node}
\usepackage{stmaryrd}
\usepackage{amsmath}
\usepackage{verbatim}
\usepackage{graphicx}  
\usepackage{calc}
\usepackage{xifthen}
%\usepackage{xcolor} investigate with beamer
\usepackage{color}
\usepackage{stringstrings}
%\usepackage[small,bf,margin=3pt,format=hang, labelsep=endash,singlelinecheck=false]{caption} %prevuiously justification=justified
%\usepackage{enumerate}
%\usepackage{enumitem}
\usepackage{enumerate}
%\usepackage[shortlabels]{enumitem} %Removed this 28/01/2019 because interfereing with a beamer presentation. 
\usepackage{float}
\usepackage[section]{placeins}
%\setlength{\captionmargin}{5pt}
\usepackage{environ}
\usepackage{multirow}
\usepackage{rotating}
\usepackage{longtable}
\usepackage{afterpage}
\usepackage{needspace}


%DEFINE ENVIRONMENT BLOCK
% Riddle
\newsavebox{\riddlebox}

\newenvironment{erexample}
{\newcommand\colboxcolor{F0F0F0}%was F8F8F8
\begin{lrbox}{\riddlebox}
\begin{minipage}{\dimexpr\columnwidth-2\fboxsep\relax} \textbf{} \\ \itshape}
{\end{minipage}\end{lrbox}%
%\begin{center}
\colorbox[HTML]{\colboxcolor}{\usebox{\riddlebox}}
%\end{center}
}

\newenvironment{erbox}
{\newcommand\colboxcolor{F0F0F0}%was F8F8F8
\begin{lrbox}{\riddlebox}%
\begin{minipage}{\dimexpr\columnwidth-2\fboxsep\relax} }
{\end{minipage}\end{lrbox}%
%\begin{center}
\colorbox[HTML]{\colboxcolor}{\usebox{\riddlebox}}
%\end{center}
}

%\begin{erboxedFigure}{#1 FigureParam}{#2 Label}{#3 Caption}
\NewEnviron{erboxedFigure}[3]{%
\begin{figure}[#1]
\begin{erexample}
\begin{center}
\BODY
\end{center}
\vspace{-0.5cm}
\caption{#3}
\label{#2}
\end{erexample}
\end{figure}
}

\newcommand{\erpictureFolder}[0]{../SharedPictures}

\newcommand{\ercenterPicture}[1]{
\begin{center}
\input{\erpictureFolder/#1}
\end{center}
}


\newlength{\erhalfHt}

%\erinlinePicture{#1 pictureFilename}{#2 pictureHeight}
\newcommand{\erinlinePicture}[2]{
\setlength{\erhalfHt}{#2cm * \real{0.5}}
\raisebox{-\erhalfHt}[\erhalfHt + 0.5cm][\erhalfHt + 0.5cm]{
\input{\erpictureFolder/#1}
} 
}

%\erplainFig{#1 pictureFilename}{#2 figureParam}{#3Caption}
\newcommand{\erplainFig}[3]{
\begin{figure}[#2]
\begin{center}
\input{\erpictureFolder/#1}
\end{center}
\caption{#3}
\label{#1}
\end{figure}
}

%\erboxedFigPicture{#1 pictureFilename}{#2 figureParam}{#3Caption}
\newcommand{\erboxedFigPicture}[3]{
\begin{figure}[#2]
\begin{erexample}
\vspace{-0.5cm}
\begin{center}
\input{\erpictureFolder/#1}
\end{center}
\caption{#3}
\label{#1}
\end{erexample}
\end{figure}
}

%\erLeftSideFig{#1 pictureFilename}{#2 figureParam}{#3Caption}
\newcommand{\erLeftSideFig}[3]{
\begin{figure}[#2]
\begin{erexample}
  \begin{minipage}[c]{0.4\textwidth}
    \caption{#3}
    \label{#1}
  \end{minipage}
  \begin{minipage}[c]{0.5\textwidth}
    \input{\erpictureFolder/#1}
  \end{minipage}
\end{erexample}
\end{figure}
}

%\erbulletedFig{#1 pictureFilename}{#2 figureParam}{#3Caption}
\NewEnviron{erbulletedFig}[3]{%
\begin{figure}[#2]
\begin{erexample}
\vspace{-0.5cm}
\begin{center}
$
\begin{array}{c m{0.25cm} | m{6cm}}
\raisebox{-2.0cm}{
\input{\erpictureFolder/#1}}& & \text{\parbox{6cm}{\raggedright{\footnotesize{
\begin{enumerate}[(i)]
\BODY
\end{enumerate}}}}} \\
\end{array}
$
\end{center}
\caption{#3}
\label{#1}
\end{erexample}
\end{figure} 
}


%\begin{erbulletedDimFig}{#1 pictureFilename}{#2figureParam} {#3Caption} {#4PictureHeight}{#5TextWidth}

\NewEnviron{erbulletedDimFig}[5]{%
\begin{figure}[#2]
\begin{erexample}
\vspace{-0.5cm}
\begin{center}
$
\begin{array}{c m{0.25cm} |  m{#5cm}}
\setlength{\erhalfHt}{#4cm * \real{0.5}}
\raisebox{-\erhalfHt}{
\input{\erpictureFolder/#1}}& & \text{\parbox{#5cm}{\raggedright{\footnotesize{
\begin{enumerate}[(i)]
\BODY
\end{enumerate}}}}} \\
\end{array}
$
\end{center}
\caption{#3}
\label{#1}
\end{erexample}
\end{figure} 
}

%\begin{ernotedModel}{#1 pictureFilename}{#2PictureHeight}{#3PictureWidth}{#4TextWidth}

\NewEnviron{ernotedModel}[4]{%
\begin{center}
$
\begin{array}{m{#3cm} m{1cm} | c m{#4cm}}
\setlength{\erhalfHt}{#2cm * \real{0.5}}
\raisebox{-\erhalfHt}{
\input{\erpictureFolder/#1}}& & & \text{\parbox{#4cm}{\raggedright{\footnotesize{
\BODY
}}}} \\
\end{array}
$
\end{center} 
}

%\begin{ermodelText}{#1 pictureFilename}{#2PictureHeight}{#3PictureWidth}{#4TextWidth}

\NewEnviron{ermodelText}[4]{%
\begin{center}
\begin{tabular}{m{#3cm} m{1cm}  c m{#4cm}}
\setlength{\erhalfHt}{#2cm * \real{0.5}}
\raisebox{-\erhalfHt}{
\input{\erpictureFolder/#1}}& & & \text{\parbox{#4cm}{\raggedright{\small{
\BODY
}}}} \\
\end{tabular}
\end{center} 
}


%\erbulletedModel{#1 pictureFilename}{#2PictureHeight}{#3PictureWidth}{#4TextWidth}

\NewEnviron{erbulletedModel}[4]{%
\begin{center}
$
\begin{array}{m{#3cm} m{1cm} | c m{#4cm}}
\setlength{\erhalfHt}{2cm * \real{0.5}}
\raisebox{-\erhalfHt}{
\input{\erpictureFolder/#1}}& & & \text{\parbox{#4cm}{\raggedright{\footnotesize{
\begin{enumerate}[(i)]
\BODY
\end{enumerate}}}}} \\
\end{array}
$
\end{center} 
}



%\ernotedDimFig{#1 pictureFilename}{#2 figureParam}{#3Caption}{#4PictureHeight}{#5TextWidth}
\NewEnviron{ernotedDimFig}[5]{%
\begin{figure}[#2]
\begin{erexample}
\vspace{-0.5cm}
\begin{center}
$
\begin{array}{c m{0.25cm} | c m{#5cm}}
\setlength{\erhalfHt}{#4cm * \real{0.5}}
\raisebox{-\erhalfHt}{
\input{\erpictureFolder/#1}}& & & \text{\parbox{#5cm}{\raggedright{\footnotesize{
\BODY }}}}\\
\end{array}
$
\end{center}
\caption{#3}
\label{#1}
\end{erexample}
\end{figure} 
}
%\begin{ernotedDimFigPW}{#1 pictureFilename}{#2 figureParam}{#3Caption}{#4PictureHeight}{#5PictureWidth}{#6TextWidth}
\NewEnviron{ernotedDimFigPW}[6]{%
\begin{figure}[#2]
\begin{erexample}
\vspace{-0.5cm}
\begin{center}
$
\begin{array}{>{\centering}m{#5cm} m{0.5cm} | c m{#6cm}}
\setlength{\erhalfHt}{#4cm * \real{0.5}}
\raisebox{-\erhalfHt}{
\centering \input{\erpictureFolder/#1}
}& & & \text{\parbox{#6cm - 0.5cm}{\raggedright{\footnotesize{
\BODY }}}}\\
\end{array}
$ \\
\vspace {0.2cm}
\end{center}
\caption{#3}
\label{#1}
\end{erexample}
\end{figure}
}



\newenvironment{erquote}
{\begin{quote}\itshape}
{\end{quote}}


%
%  erdiagram.tex
%  *************
%  Macros to represent ER diagrams
%  *******************************
% 29/01/2019 Modify so that not reliant on the
%            default fontsize being 10pt by using
%            package anyfontsize and then
%            \fontsize{8}{10}\selectfont to set font to 8pt
% 06/02/2019 Pullback symbol implemented and minor tweaks to positioning 
%            and size of identifier symbol and relationship labels.
%            Accidental forked changes merged on 08/02/2019.
% 15/03/2019 Continuation of 29/01/2019. Need fix fontsize of 
%            ERrelname and ERscope.	 
% ***********************************************************
 \usepackage{anyfontsize}             % 29/01/2019 
  
%\begin{erdiagram}{#1 height}{#2 width} 
% ....
% ....
%\end{erdiagram}
\newenvironment{erdiagram}[2]
{%\pspicture*(-#1,0)(#2,0)
\pspicture*(0,-#1)(#2,0)
%\psgrid
}
{\endpspicture}

\definecolor{lightyellow}{cmyk}{0,0,0.3,0}
\definecolor{verylightgrey}{gray}{0.95}


% \eret{#1 x0} {#2 y0} {#3 x1} {#4 y1} {#5 corner radius} {#6 fill}
\newcommand {\eret}[6]
{ 
\ifthenelse{\equal{#6}{1}}
{\psframe[framearc=#5,fillstyle=solid,fillcolor=lightyellow](#1,#2)(#3,#4)}
{\psframe[framearc=#5,fillstyle=solid,fillcolor=white](#1,#2)(#3,#4)}
}

% et top 
\newcommand {\erettop}[4]
{
%\psframe[linestyle=none,linearc=2pt,cornersize=absolute,fillstyle=solid,fillcolor=lightyellow](#1,#2)(#3,#4)
\psline[linearc=2pt,fillstyle=none,fillcolor=lightyellow](#1,#4)(#1,#2)(#3,#2)(#3,#4)
}

% et bottom 
\newcommand {\eretbtm}[4]
{
%\psframe[linestyle=none,linearc=2pt,cornersize=absolute,fillstyle=solid,fillcolor=lightyellow](#1,#2)(#3,#4)
\psline[linearc=2pt,fillstyle=none,fillcolor=lightyellow](#1,#2)(#1,#4)(#3,#4)(#3,#2)
}

% et bottom left
\newcommand {\eretbl}[4]
{
%\psframe[linestyle=none,linearc=2pt,cornersize=absolute,fillstyle=solid,fillcolor=lightyellow](#1,#2)(#3,#4)
\psline[linearc=2pt,fillstyle=none,fillcolor=lightyellow](#1,#4)(#3,#4)(#3,#2)
}

% et middle left
\newcommand {\eretml}[4]
{
%\psframe[linestyle=none,linearc=2pt,cornersize=absolute,fillstyle=solid,fillcolor=lightyellow](#1,#2)(#3,#4)
\psline[linearc=2pt,fillstyle=none,fillcolor=lightyellow](#1,#2)(#3,#2)(#3,#4)(#1,#4)
}

% et top left
\newcommand {\erettl}[4]
{
%\psframe[linestyle=none,linearc=2pt,cornersize=absolute,fillstyle=solid,fillcolor=lightyellow](#1,#2)(#3,#4)
\psline[linearc=2pt,fillstyle=none,fillcolor=lightyellow](#1,#2)(#3,#2)(#3,#4)
}

% et bottom right
\newcommand {\eretbr}[4]
{
%\psframe[linestyle=none,linearc=2pt,cornersize=absolute,fillstyle=solid,fillcolor=lightyellow](#1,#2)(#3,#4)
\psline[linearc=2pt,fillstyle=none,fillcolor=lightyellow](#1,#2)(#1,#4)(#3,#4)
}

% et middle right
\newcommand {\eretmr}[4]
{
%\psframe[linestyle=none,linearc=2pt,cornersize=absolute,fillstyle=solid,fillcolor=lightyellow](#1,#2)(#3,#4)
\psline[linearc=2pt,fillstyle=none,fillcolor=lightyellow](#3,#4)(#1,#4)(#1,#2)(#3,#2)
}

% et top right
\newcommand {\erettr}[4]
{
\psline[linearc=2pt,fillstyle=none,fillcolor=lightyellow](#1,#4)(#1,#2)(#3,#2)
}

% \ergrp{#1 x0} {#2 y0} {#3 x1} {#4 y1} {#5 corner radius} {#6 fill}
% #5 corner radius is unused!
\newcommand {\ergrp}[6]
{ 
\ifthenelse{\equal{#6}{1}}
{\psframe[fillstyle=solid,fillcolor=verylightgrey](#1,#2)(#3,#4)}
{\psframe[fillstyle=solid,fillcolor=white](#1,#2)(#3,#4)}
}


% \ertext{#1 text}
% 15/03/2019
\newcommand {\erextrasmallitalictext}[1]
{\fontsize{7}{9}\selectfont \textit{#1}}

% 29/01/2019  
\newcommand {\ersmallitalictext}[1]
{\fontsize{8}{10}\selectfont \textit{#1}}

\newcommand {\ermediumitalictext}[1]
{\fontsize{10}{12}\selectfont \textit{#1}}

% \eretname {#1 x left of text} {#2 y top of text} {#3 text}
\newcommand {\olderetname}[3]
{
%shift down 0.1 for height of text the anchor at baseline (B)
\rput[bl]{0}(0,-0.1){\rput[Bl]{0}(#1,#2){\ersmallitalictext{#3}}}
}

% \errelarm {#1 x0} {#2 y0} {#3 x1} {#4 y1} {#5 ismandatory} {#6 isconstructed}
\newcommand {\errelarm}[6]
{
\ifthenelse{\equal{#6}{1}}
{
%%\psline[linewidth=0.5pt,linearc=.05,linestyle=dashed,dash=6pt 6pt]{-}(#1,#2)(#3,#4)}
\ifthenelse{\equal{#5}{1}}
{\psline[linewidth=1.5pt,linearc=.05,linecolor=lightgray]{-}(#1,#2)(#3,#4)}
{\psline[linewidth=1.5pt,linearc=.05,linecolor=lightgray,linestyle=dashed,dash=2pt 2pt]{-}(#1,#2)(#3,#4)}
}
{
\ifthenelse{\equal{#5}{1}}
{\psline[linewidth=0.9pt,linearc=.05]{-}(#1,#2)(#3,#4)}
{\psline[linewidth=0.9pt,linearc=.05,linestyle=dashed,dash=2pt 2pt]{-}(#1,#2)(#3,#4)}
}
}

% \errelangle {#1 x0} {#2 y0} {#3 x1} {#4 y1} {#5 x2} {#6 y2} {#7 ismandatory} {#8 isocnstructed}
\newcommand {\errelangle}[8]
{
\ifthenelse{\equal{#8}{1}}
{
%\psline[linewidth=0.5pt,linearc=.1,linestyle=dashed,dash=6pt 6pt]{-}(#1,#2)(#3,#4)(#5,#6)}
\ifthenelse{\equal{#7}{1}}
{\psline[linewidth=1.5pt,linearc=.05,linecolor=lightgray]{-}(#1,#2)(#3,#4)(#5,#6)}
{\psline[linewidth=1.5pt,linearc=.1,linecolor=lightgray,linestyle=dashed,dash=2pt 2pt]{-}(#1,#2)(#3,#4)(#5,#6)}
}
{
\ifthenelse{\equal{#7}{1}}
{\psline[linewidth=0.9pt,linearc=.1]{-}(#1,#2)(#3,#4)(#5,#6)}
{\psline[linewidth=0.9pt,linearc=.1,linestyle=dashed,dash=2pt 2pt]{-}(#1,#2)(#3,#4)(#5,#6)}
}
}

% \ercrowfoot {#1 x0} {#2 y0} {#3 x11} {#4 y11} {#5 x12} {#6 y12} {#7 x13} {#8 y13} {#9 isconstructed}
\newcommand {\ercrowfoot}[9]
{
\ifthenelse{\equal{#9}{1}}
{
\psline[linewidth=1.5pt,linearc=.05,linecolor=lightgray]{-}(#1,#2)(#3,#4)
\psline[linewidth=1.5pt,linearc=.05,linecolor=lightgray]{-}(#1,#2)(#5,#6)
\psline[linewidth=1.5pt,linearc=.05,linecolor=lightgray]{-}(#1,#2)(#7,#8)
}{
\psline[linewidth=0.9pt,linearc=.05]{-}(#1,#2)(#3,#4)
\psline[linewidth=0.9pt,linearc=.05]{-}(#1,#2)(#5,#6)
\psline[linewidth=0.9pt,linearc=.05]{-}(#1,#2)(#7,#8)
}
}


% \eridcomprel{#1 x1}{#2 x2}{#3 y}
\newcommand {\eridcomprel}[3]
{
\psline[linewidth=0.9pt](#1,#3)(#2,#3)
}

% \eridrefrel{#1 x}{#2 y1}{#3 y2}
\newcommand {\eridrefrel}[3]
{
\psline[linewidth=0.9pt](#1,#2)(#1,#3)
}

% \ertext {#1 x} {#2 y} {#3 text anchor} {#4 text}  PRIVATE
\newcommand {\ertext}[4]
{
\rput[B#3]{0}(#1,#2){\fontsize{8}{10}\selectfont #4}
}

% \eretname {#1 x} {#2 y} {#3 text anchor} {#4 text} 
\newcommand {\eretname}[4]
{
\ertext{#1}{#2}{#3}{#4}
}

% \errelname {#1 x} {#2 y} {#3 text anchor} {#4 text} 
\newcommand {\errelname}[4]
{
\rput[B#3]{0}(#1,#2){\erextrasmallitalictext{#4}}
}


% \erscope {#1 x} {#2 y} {#3 text anchor} {#4 text}  15 March 2019
\newcommand {\erscope}[4]
{
\rput[B#3]{0}(#1,#2){\erextrasmallitalictext{#4}}
}

% \erreletname {#1 x} {#2 y} {#3 text anchor} {#4 text}  15 March 2019
\newcommand {\erreletname}[4]
{
\rput[B#3]{0}(#1,#2){\fontsize{10}{12}\selectfont #4}
}

% \ergroupannotation {#1 x} {#2 y} {#3 text anchor} {#4 text}
\newcommand {\ergroupannotation}[4]
{
\ertext{#1}{#2}{#3}{#4}
}


% \errelseq {#1 x} {#2 y}
\newcommand {\erelseq}[2]
{
}
\newcommand {\erattrmarkermand}
{\fontsize{6}{8}\selectfont $\blacksquare$}
\newcommand {\erattrmarkeropt}
{\fontsize{6}{8}\selectfont \CIRCLE}
\newcommand {\erderattrmarkermand}
{\fontsize{6}{8}\selectfont $\square$}
\newcommand {\erderattrmarkeropt}
{\fontsize{8}{10}\selectfont $\circ$}

% \erattr {#1 x} {#2 y} {#3 ismandatory}{#4 idenitfying} {#5 text}
\newcommand {\erattr}[5]
{
\ifthenelse{\equal{#3}{1}}
{\rput[l]{0}(#1,#2){\erattrmarkermand \ersmallitalictext{\ifthenelse{\equal{#4}{0}}{\underline{#5}}{#5}}}}
{\rput[l]{0}(#1,#2){\erattrmarkeropt \ersmallitalictext{\ifthenelse{\equal{#4}{0}}{\underline{#5}}{#5}}}}
}

\newcommand {\erdattr}[5]
{
\ifthenelse{\equal{#3}{1}}
{\rput[l]{0}(#1,#2){\erderattrmarkermand \ersmallitalictext{\ifthenelse{\equal{#4}{0}}{\underline{#5}}{#5}}}}
{\rput[l]{0}(#1,#2){\erderattrmarkeropt \ersmallitalictext{\ifthenelse{\equal{#4}{0}}{\underline{#5}}{#5}}}}
}


% \erarc {#1 x0} {#2 y0} {#3 x1} {#4 y1} {#5 x2} {#6 y2} {#7 x3} {#8 y3}
\newcommand {\erarc}[8]
{
\psbezier[showpoints=false]{-}(#1,#2) (#3, #4)(#5,#6) (#7, #8)
}

% \erarc {#1 x0} {#2 y0} {#3 x1} {#4 y1} {#5 x2} {#6 y2} {#7 x3} {#8 y3}
\newcommand {\errelseq}[8]
{
\psbezier[showpoints=false]{-}(#1,#2) (#3, #4)(#5,#6) (#7, #8)
}
% \ertrace {#1 trace}   
\newcommand {\ertrace}[1]
{
}

\usepackage{amsthm} % added 7th April 2018
% theorems.macros.tex

\newtheorem{theorem}{Theorem}[section]
\newtheorem{observation}[theorem]{Observation}
\newtheorem{lemma}[theorem]{Lemma}
\newtheorem{proposition}[theorem]{Proposition}
\newtheorem{corollary}[theorem]{Corollary}
\newtheorem{conjecture}[theorem]{Conjecture}
\newtheorem{numbereddefinition}[theorem]{Definition}

\newenvironment{definition}[1][Definition]{\begin{trivlist}
\item[\hskip \labelsep {\bfseries #1}]}{\end{trivlist}}
\newenvironment{examples}[1][Examples]{\begin{trivlist}
\item[\hskip \labelsep {\bfseries #1}]}{\end{trivlist}}
\newenvironment{example}[1][Example]{\begin{trivlist}
\item[\hskip \labelsep {\bfseries #1}]}{\end{trivlist}}
\newenvironment{remark}[1][Remark]{\begin{trivlist}
\item[\hskip \labelsep {\bfseries #1}]}{\end{trivlist}}

\newenvironment{tageqn}[1]
{
\begin{equation}
\stepcounter{equation}
\label{#1}
\tag{\theequation --#1}
}
{
\end{equation}
}

\newenvironment{axiom}[1]
{
\begin{equation}
\label{#1}
\tag{#1}
}
{
\end{equation}
}

% when the tag is required different from the label eg when has math symbols can use:
\newenvironment{axiomtagged}[2]
{
\begin{equation}
\label{#1}
\tag{#2}
}
{
\end{equation}
}

%visible label
\newcommand{\vlabel}[2][]{\label{#2}#1(\textit{#2}):}





\usepackage{mathptmx}  % This changes font to roman
\usepackage{anyfontsize}
\usepackage{mathtools}  % why have we got this?
\usepackage{alltt}    
\usepackage{mnsymbol} %used for rightpitchfork
\usepackage{cmll}
\usepackage{ulem}
\renewcommand{\ttdefault}{txtt}
\usepackage[left=1.5cm, right=4cm, marginparwidth=3cm, top=2cm, bottom=1.5cm]{geometry}
\usepackage{framed}
\usepackage[font=small]{caption}
\setlength{\captionmargin}{2cm}
\newcommand{\commentary}[1]{\marginpar{\footnotesize #1}}

\renewcommand{\erpictureFolder}[0]{../SharedPictures}

\newenvironment{categoricalaside}
{\begin{framed}
\textbf{Categorical Aside}
}
{
\end{framed}
}

%from berkley
\newcommand{\langl}{\begin{picture}(4.5,7)
\put(1.1,2.5){\rotatebox{60}{\line(1,0){5.5}}}
\put(1.1,2.5){\rotatebox{300}{\line(1,0){5.5}}}
\end{picture}}
\newcommand{\rangl}{\begin{picture}(4.5,7)
\put(.9,2.5){\rotatebox{120}{\line(1,0){5.5}}}
\put(.9,2.5){\rotatebox{240}{\line(1,0){5.5}}}
\end{picture}}
\newcommand{\lang}{\begin{picture}(5,7)\put(1.1,2.5){\rotatebox{45}{\line(1,0){6.0}}}\put(1.1,2.5){\rotatebox{315}{\line(1,0){6.0}}}\end{picture}}
\newcommand{\rang}{\begin{picture}(5,7)\put(.1,2.5){\rotatebox{135}{\line(1,0){6.0}}}\put(.1,2.5){\rotatebox{225}{\line(1,0){6.0}}}\end{picture}}
%Try sharper tuple brackets -- except gives errors nested in captions so comment out
%\renewcommand{\tuple}[1]{\lang #1 \rang}

\newcommand{\setsuchthat}[2]{\left\{#1 \ \middle|\ #2\right\}}
\newcommand{\set}[1]{\left\{#1\right\}} 

\newcommand{\genericmodel}{\mathcal{M}}  %PREVIOUSLY
\renewcommand{\genericmodel}{{m}}        %PREVIOUSLY
\renewcommand{\genericmodel}{\gamma}     % TRY THIS FOR A WHILE except texworks isnt happy with greek
\renewcommand{\genericmodel}{M}  %while debugging
\newcommand{\chiZero}{\mathcal{X}_0}
\newcommand{\chiZeroM}{\chiZero(\genericmodel)}
%\newcommand{\chiOne}{\mathcal{X}_1}
%\newcommand{\chiOneM}{\chiOne(\genericmodel)}
\newcommand{\chiM}{\mathcal{X}(\genericmodel)}
\newcommand{\veee}{v}
\newcommand{\Veee}{V}
\newcommand{\et}[1][\genericmodel]{et_{#1}}
\newcommand{\edge}[3][\genericmodel]{Edge_{#1}(#2,#3)}
\newcommand{\iedge}[3][\genericmodel]{IEdge_{#1}(#2,#3)}
\newcommand{\path}[3][\genericmodel]{Path_{#1}(#2,#3)}
\newcommand{\ipath}[3][\genericmodel]{IPath_{#1}(#2,#3)}
\newcommand{\attr}[2] [\genericmodel]{attr_{#1}(#2)}
\newcommand{\iattr}[2] [\genericmodel]{IAttr_{#1}(#2)}
\newcommand{\rel}[3][\genericmodel]{rel_{#1}(#2,#3)}
\newcommand{\irel}[3][\genericmodel]{IRel_{#1}(#2,#3)}
\newcommand{\iedges}[2] [\genericmodel]{i_{#1}(#2)}
\newcommand{\pk}[2] [\genericmodel]{pk_{#1}(#2)}
\newcommand{\fk}[2] [\genericmodel]{fk_{#1}(#2)}
\newcommand{\fkp}[2] [\genericmodel]{fk'_{#1}(#2)}
\newcommand{\fkpp}[2] [\genericmodel]{fk''_{#1}(#2)}

%functional dependencies
\newcommand{\sfd}[2]{\ensuremath{\set{#1} \morph #2}}  %singleton
\newcommand{\fd}[2]{\ensuremath{\sfd{#1}{\set{#2}}}}

\newcommand{\simplepath}[2]{
\ncline[linestyle=none,linewidth=0.1pt]{#1}{#2}   %was linestyle=dotted
\ncput[npos=0.05]{\pnode{dot#21}}
\ncput[npos=0.27]{\dotnode[dotsize=1pt]{dot#22}}
\ncput[npos=0.50]{\dotnode[dotsize=1pt]{dot#23}}
\ncput[npos=0.80]{\dotnode[dotsize=1pt]{dot#24}}
\ncput[npos=0.975]{\pnode{dot#25}}
\ncline[nodesep=2pt]{->}{dot#21}{dot#22}
\ncline[nodesep=2pt]{->}{dot#22}{dot#23}
\ncline[nodesep=2pt]{->}{dot#24}{dot#25}
\ncline[linestyle=dotted,nodesep=8pt]{dot#23}{dot#24} %was 10pt
}

\newcommand{\simplepatha}[3]{
\simplepath{#2}{#3}
\naput[labelsep=1pt]{#1}
}

\newcommand{\simplepathb}[3]{
\simplepath{#2}{#3}
\nbput[labelsep=1pt]{#1}
}
\newcommand{\term}[1]{\textit{{#1}}}
\newcommand{\logtophys}{\mathcal{X}}
\newcommand{\chen}{\mathcal{X}_0}
\newcommand{\chengenericmodel}{\chen(\genericmodel)}
\newcommand{\chigenericmodel}{\logtophys(\genericmodel)}
\newcommand{\phys}[1]{\overline{#1}}
\newcommand{\genericphysical}{\logtophys(\genericmodel)}

\newcommand{\inc}{\subseteq}
\newcommand{\incd}[4]{#1\left[#2\right]\inc#3\left[#4\right]}

\newcommand{\ntuple}[1]{\tuple{#1_1,...#1_n}}
\newcommand{\mtuple}[1]{\tuple{#1_1,...#1_n}}

\newcommand {\bntuple}{\ensuremath{\ntuple{b}}}
\newcommand {\fntuple}{\ensuremath{\ntuple{f}}}
\newcommand {\pntuple}{\ensuremath{\ntuple{p}}}
\newcommand {\qntuple}{\ensuremath{\ntuple{q}}}
\newcommand {\xntuple}{\ensuremath{\ntuple{x}}}
\newcommand{\foreachi}[1][n]{for each $i$, $1 \leq i \leq #1$}
\newcommand{\foreachj}[1][m]{for each $j$, $1 \leq j \leq #1$}
\newcommand{\foreachk}[1][l]{for each $k$, $1 \leq k \leq #1$}



%ccategories.macros.tex 

% Macros for diagrams in contextual categories and related categories

\usepackage{twoopt}
\usepackage{scalerel} 
\usepackage{xargs}

%\usepackage{mathabx}  %Caused font problems
%\usepackage{MnSymbol}  % caused font problems

\newcommand{\conu}
{\mathbf{C}(U)}

\newcommand{\depu}
{\mathbf{D}(U)}


\newcommand{\reqt}{\textbf{R}}
\newcommand{\reqtc}[1][\catc]{\reqt_{#1}}
\newcommand{\reqtcp}[1][\catcp]{\reqt_{#1}}



\newcommand{\cat}[1]{\textbf{#1}}

\newcommand{\catc}{\cat{C}}
\newcommand{\catcw}{\cat{C}\ }
\newcommand{\catcp}[1][C]{\textbf{#1}'}
\newcommand{\catcpp}[1][C]{\textbf{#1}''}
\newcommand{\obj}[1]{\ensuremath{|\cat{#1}|}}
\newcommand{\ccat}[1][C]{\ensuremath{\mathbb{#1}} }
\newcommand{\ccatc}{contextual category \ccat}
\newcommand{\cobj}[2][]{\ensuremath{|\ccat[#2]|_{#1}}}
\newcommand{\cslice}[2]{\ensuremath{\ccat[#1]_{#2}}}
\newcommand{\csliceobj}[3][]{\ensuremath{|\mathbb{#2}_{#3}|_{#1} }}
\newcommand{\varset}[1][]{\ensuremath{V_{#1} }}
\newcommand{\localvarsets}{\ensuremath{\mathcal{V} }}
\newcommand{\Fam}{\ensuremath{\mathbb{F\mathrm{am}} }}
\newcommand{\Fin}{\ensuremath{\textbf{Fin}} }
\newcommand{\Finp}{\ensuremath{\textbf{Finp}} }
\newcommand{\Po}{\ensuremath{\textbf{Po}} }
\newcommand{\Famslice}[1]{\ensuremath{\mathbb{F\mathrm{am}}_{#1} }}
\newcommand{\Famobj}[1][]{\ensuremath{|\mathbb{F\mathrm{am}}|_{#1} }}
\newcommand{\Famsliceobj}[2][]{\ensuremath{|\mathbb{F\mathrm{am}}_{#2}|_{#1} }}
\newcommand{\morph}{\rightarrow}
\newcommand{\epi}{\twoheadrightarrow}
\newcommand{\base}{\triangleleft}
\newcommand{\comp}{\circ}
\newcommand{\cross}{\otimes}
\newcommand{\pc}[2]{d^{#1}_{#2}}
\newcommand{\sub}{^*}
\newcommand{\diag}{\delta}
\newcommand{\pbase}[1]{\tilde{#1}}
\newcommand{\tuple}[1]{\langle#1\rangle}
\newcommand{\ndidly}{\ensuremath{\Join_n}}

\newcommand{\product}[1]{\bigtimes_{#1}}
\newcommand{\productn}{\product{n}}
\newcommand{\crossx}[3]{#1 \underset{#3}{\cross} #2}
\newcommand{\fibrex}[3]{#1 \underset{#3}{\Join} #2}
\newcommand{\powerset}{\mathcal{P}}
\newcommand{\primeds}[1]{
\ensuremath{\mathcal{P}(#1)} }
\newcommand{\compset}{\ \dot{\circ}\, }

% darrow
%\newcommand{\darrow}{\rightarrowtriangle} %use \smorph instead
\newcommand{\smorph}{\rightarrowtriangle}

 
\newcommand\dhead{\scaleobj{0.6}{\triangleright}}
%\newcommand{\dmorph}{\, \mbox{---} \! \cdot \! \raisebox{1.1pt}{\dhead}}    % dot style
\newcommand{\dmorph}{\, \mbox{---}\kern-1pt\raisebox{1.1pt}{\dhead\kern-1.75pt\dhead}}\,     % double triangle style

% projection tree
%\newcommand{\proj}[2]{proj_{#2}(#1)}

\newcommand{\proj}[2]{
\ensuremath{\mathcal{P}_{#2}(#1)} }

%pstrick supplements for arrows

\newlength{\arrnodesepA}
\newlength{\arrnodesepB}
\newlength{\arroffsetA}
\newlength{\arroffsetB}

%Modified to 2pt from 0pt on 23 July 2018
\newcommand{\arreset}{
\setlength{\arrnodesepA}{2pt}
\setlength{\arrnodesepB}{2pt}
\setlength{\arroffsetA}{0pt}
\setlength{\arroffsetB}{0pt}
}
\arreset

\newcommand{\ncarr}[3][0]{\ncarc[arcangle=#1,nodesepA=\arrnodesepA,nodesepB=\arrnodesepB,offsetA=\arroffsetA,offsetB=\arroffsetB,arrowsize=5pt,arrowinset=0.7]{->}{#2}{#3}}
\newcommand{\ncdarr}[3][0]{\ncarc[linestyle=dashed,arcangle=#1,nodesepA=\arrnodesepA,nodesepB=\arrnodesepB,offsetA=\arroffsetA,offsetB=\arroffsetB,arrowsize=5pt,arrowinset=0.7]{->}{#2}{#3}}
\newcommand{\jcbarr}[4][0]{ % ncbarr is defined in some thridy party package so do not use!\emph{}
\ncarr[#1]{#3}{#4}
\nbput[labelsep=2pt]{\footnotesize $#2$}
}

\newcommand{\ncaarr}[4][0]{
\ncarr[#1]{#3}{#4}
\naput[labelsep=2pt]{\footnotesize $#2$}
}

% \alabel{label}[npos][labelsep_pts]
\newcommandx*\alabel[3][2=0.5,3=2,usedefault]{\naput[labelsep=#3pt,npos=#2]{\footnotesize $#1$}}
% \blabel{label}[npos][labelsep_pts]
\newcommandx*\blabel[3][2=0.5,3=2,usedefault]{\nbput[labelsep=#3pt,npos=#2]{\footnotesize $#1$}}


\newif \ifbars
% to supress display of bars use \barsfalse to swith them on use \barstrue
\barstrue 
% \idcomp mark an arrow as one component of an identifier
\newcommand{\idcomp}{\ifbars{\ncput[npos=0, nrot=:U]{\psline(0.2,-0.075)(0.2,0.075)}}\fi}  %add a bar to a node connection arrow
% pstrick supplements for s-arrows (previous name for d-arrow - should convert}

\newlength{\sarnodesepA}
\newlength{\sarnodesepB}
\newlength{\saroffsetA}
\newlength{\saroffsetB}
\newlength{\sarnodesepAsav}
\newlength{\sarnodesepBsav}

\newcommand{\sarreset}{
\setlength{\sarnodesepA}{0pt}
\setlength{\sarnodesepB}{0pt}
\setlength{\saroffsetA}{0pt}
\setlength{\saroffsetB}{0pt}
}

\sarreset

% sar - S-arrow
\newcommand{\ncsar}[3][0]{
\setlength{\sarnodesepAsav}{\sarnodesepA}
\setlength{\sarnodesepBsav}{\sarnodesepB}
\addtolength{\sarnodesepA}{3pt}
\addtolength{\sarnodesepB}{7pt}
\ncarc[nodesepA=\sarnodesepA,nodesepB=\sarnodesepB,offsetA=\saroffsetA,offsetB=\saroffsetB,arcangle=#1]{-}{#2}{#3}
\ncput[nrot=:R,npos=1]{\pstriangle(0,0)(.2,.2)}
\setlength{\sarnodesepA}{\sarnodesepAsav}
\setlength{\sarnodesepB}{\sarnodesepBsav}
}


% bsar - below labelled S-arrow
\newcommand{\ncbsar}[4][0]{
\ncsar[#1]{#3}{#4}
\nbput[labelsep=2pt]{\footnotesize $#2$}
}
% asar - above labelled S-arrow
\newcommand{\ncasar}[4][0]{
\ncsar[#1]{#3}{#4}
\naput[labelsep=2pt]{\footnotesize $#2$}
}

% OLD cdar - composite dependency arrow - dot tyle
\iffalse
\newcommand{\nccdar}[3][0]{
\setlength{\sarnodesepAsav}{\sarnodesepA}
\setlength{\sarnodesepBsav}{\sarnodesepB}
\addtolength{\sarnodesepA}{3pt}
\addtolength{\sarnodesepB}{11pt}
\ncarc[nodesepA=\sarnodesepA,nodesepB=\sarnodesepB,offsetA=\saroffsetA,offsetB=\saroffsetB,arcangle=#1]{-}{#2}{#3}
\ncput[nrot=:R,npos=1]{\pstriangle(0,0.1)(.2,.2)}
\ncput[nrot=:R,npos=1]{\psdot[dotsize=1pt](-0.0075,0.05)}   %!!
\setlength{\sarnodesepA}{\sarnodesepAsav}
\setlength{\sarnodesepB}{\sarnodesepBsav}
}
\fi

% cdar - composite dependency arrow Mark II - double trangle style
\newcommand{\nccdar}[3][0]{
\setlength{\sarnodesepAsav}{\sarnodesepA}
\setlength{\sarnodesepBsav}{\sarnodesepB}
\addtolength{\sarnodesepA}{3pt}
\addtolength{\sarnodesepB}{13pt}
\ncarc[nodesepA=\sarnodesepA,nodesepB=\sarnodesepB,offsetA=\saroffsetA,offsetB=\saroffsetB,arcangle=#1]{-}{#2}{#3}
\ncput[nrot=:R,npos=1]{\pstriangle(0,0)(.2,.2)}
\ncput[nrot=:R,npos=1]{\pstriangle(0,0.2)(.2,.2)}
\setlength{\sarnodesepA}{\sarnodesepAsav}
\setlength{\sarnodesepB}{\sarnodesepBsav}
}


% bcdar - below labelled composite dependency arrow
\newcommand{\ncbcdar}[4][0]{
\nccdar[#1]{#3}{#4}
\nbput[labelsep=2pt]{\footnotesize $#2$}
}
% acdar - above labelled composite dependency arrow
\newcommand{\ncacdar}[4][0]{
\nccdar[#1]{#3}{#4}
\naput[labelsep=2pt]{\footnotesize $#2$}
}


% rsar - recursive S-arrow
\newcommand{\ncrsar}[2]{
\setlength{\sarnodesepAsav}{\sarnodesepA}
\setlength{\sarnodesepBsav}{\sarnodesepB}
\addtolength{\sarnodesepA}{3pt}
\addtolength{\sarnodesepB}{7pt}
\ncloop[nodesepA=\sarnodesepA,nodesepB=\sarnodesepB,
        offsetA=\saroffsetA,offsetB=\saroffsetB,
        armA=0.7cm,armB=0.6cm,angleA=90,angleB=-90,loopsize=-1,linearc=0.4
				]{-}{#1}{#2}
\ncput[nrot=:R,npos=5]{\pstriangle(0,0)(.2,.2)}
\setlength{\sarnodesepA}{\sarnodesepAsav}
\setlength{\sarnodesepB}{\sarnodesepBsav}
}

% pstrick supplements for multi-arrows

\newlength{\marnodesepA}
\newlength{\marnodesepB}
\newlength{\maroffsetB}
\newlength{\marnodesepBsav}

\newcommand{\marreset}{
\setlength{\marnodesepA}{0pt}
\setlength{\marnodesepB}{0pt}
\setlength{\maroffsetB}{0pt}
}

\marreset

%ncmarr[#1 arcangle1][#2 arcangle2]{#3 name}{#4 domain1}{#5 domain2}{#6 junction}{#7 codomain}
\newcommandtwoopt{\ncmarr}[6][8][8]{%
\ncarc[nodesepA=\marnodesepA,nodesepB=0,arcangle=#1]{-}{#3}{#5}
\ncarc[nodesepB=0,arcangle=-#1]{-}{#4}{#5}
\ncarc[arcangle=#2,nodesepB=\marnodesepB,offsetB=\maroffsetB]{->}{#5}{#6}
}%


\newcommandtwoopt{\nchmarr}[6][8][8]{%
\ncarc[nodesepA=\marnodesepA,nodesepB=0,arcangle=#1]{-}{#3}{#5}
\ncarc[nodesepB=0,arcangle=#1]{-}{#4}{#5}
\ncarc[arcangle=#2,nodesepB=\marnodesepB,offsetB=\maroffsetB]{->}{#5}{#6}
}%

\newcommandtwoopt{\ncamarr}[7][8][8]{%
\ncmarr[#1][#2]{#4}{#5}{#6}{#7}
\naput[npos=.05]{$#3$}
}%
\newcommandtwoopt{\ncbmarr}[7][8][8]{%
\ncmarr[#1][#2]{#4}{#5}{#6}{#7}
\nbput[npos=.05]{$#3$}
}%

\newcommandtwoopt{\ncbhmarr}[7][8][8]{%
\nchmarr[#1][#2]{#4}{#5}{#6}{#7}
\nbput[npos=.05]{$#3$}
}%

\newcommandtwoopt{\ncmarrr}[7][8][8]{
\ncarc[nodesepB=0,arcangle=#1]{-}{#3}{#6}
\ncline[nodesepB=0]{-}{#4}{#6}
\ncarc[nodesepB=0,arcangle=-#1]{-}{#5}{#6}
\ncarc[nodesepA=0,arcangle=#2]{->}{#6}{#7}
}

\newcommandtwoopt{\ncamarrr}[8][8][8]{
\ncmarrr[#1][#2]{#4}{#5}{#6}{#7}{#8}
\naput[npos=.05]{$#3$}
}
\newcommandtwoopt{\ncbmarrr}[8][8][8]{
\ncmarrr[#1][#2]{#4}{#5}{#6}{#7}{#8}
\nbput[npos=.05]{$#3$}
}


% 6 June 2020
% Edges representing attributes and relationship graphs
%  Ep   - partial
%  Epm  - partial mono
%  Epe  - partial epi
%  Epme - partial mono epi
%  Et   - total
%  Etm  - total mono
%  Ete  - total epi
%  Etme - total mono epi
%  recursive edges (use nccircle)
%  rEp   - partial
%  rEpm  - partial mono
%  rEpe  - partial epi
%  rEpme - partial mono epi
%  rEt   - total
%  rEtm  - total mono
%  rEte  - total epi
%  rEtme - total mono epi

\newcounter{EangleA}
\newcounter{EangleB}
\newcounter{EmidangleA}
\newcounter{EmidangleB}

% Ep - Edge partial
\newcommandtwoopt{\Ep}[4][0][0]{
\crowsfootedEdge{#1}{#2}{#3}{#4}{dashed}{dashed}
}



% Epm - Edge partial mono
\newcommandtwoopt{\Epm}[4][0][0]{
\monoEdge{#1}{#2}{#3}{#4}{dashed}{dashed}
}


% Epe - Edge partial epi
\newcommandtwoopt{\Epe}[4][0][0]{
\crowsfootedEdge{#1}{#2}{#3}{#4}{dashed}{solid}
}

% Epme - Edge partial mono epi
\newcommandtwoopt{\Epme}[4][0][0]{
\monoEdge{#1}{#2}{#3}{#4}{dashed}{solid}
}

% Et - Edge total
\newcommandtwoopt{\Et}[4][0][0]{
\crowsfootedEdge{#1}{#2}{#3}{#4}{solid}{dashed}
}

% Etm - Edge total mono
\newcommandtwoopt{\Etm}[4][0][0]{
\monoEdge{#1}{#2}{#3}{#4}{solid}{dashed}
}

% Ete - Edge total epi
\newcommandtwoopt{\Ete}[4][0][0]{
\crowsfootedEdge{#1}{#2}{#3}{#4}{solid}{solid}
}

% Etme - Edge total mono epi
\newcommandtwoopt{\Etme}[4][0][0]{
\monoEdge{#1}{#2}{#3}{#4}{solid}{solid}
}

% crowsfootedEdge - \crowsfootedEdge[angleA][midpointangle]{startnode}{endnode}[startstyle][endstyle]
\newcommand{\crowsfootedEdge}[6]{
\setlength{\sarnodesepAsav}{\sarnodesepA}
\setlength{\sarnodesepBsav}{\sarnodesepB}
\addtolength{\sarnodesepA}{3pt}
\addtolength{\sarnodesepB}{3pt}
\setcounter{EangleA}{ #1 + #2}
\setcounter{EangleB}{180  - #1 + #2}
\setcounter{EmidangleA}{#2}
\setcounter{EmidangleB}{#2 + 180}
\nccurve[nodesepA=\sarnodesepA,nodesepB=\sarnodesepB,offsetA=\saroffsetA,offsetB=\saroffsetB,angleA=\theEangleA, angleB=\theEangleB,linestyle=none,linewidth=0]{->}{#3}{#4}
\ncput[nrot=:R,npos=0]{\psline(0,.1)(.075,0)}
\ncput[nrot=:R,npos=0]{\psline(0,.1)(-0.075,0)}
\ncput{\pnode(0,0){xxx}}
\nccurve[nodesepA=0,nodesepB=\sarnodesepB,offsetA=0,offsetB=\saroffsetB,angleA=\theEmidangleA, angleB=\theEangleB, linestyle=#6]{->}{xxx}{#4}
%the following provides context for any following label
\nccurve[nodesepA=\sarnodesepA,nodesepB=0,offsetA=\saroffsetA,offsetB=0,angleA=\theEangleA, angleB=\theEmidangleB,linestyle=#5]{-}{#3}{xxx}
\setlength{\sarnodesepA}{\sarnodesepAsav}
\setlength{\sarnodesepB}{\sarnodesepBsav}
}

% monoEdge - \monoEdge[angleA][midpointangle]{startnode}{endnode}[startstyle][endstyle]
\newcommand{\monoEdge}[6]{ 
\setlength{\sarnodesepAsav}{\sarnodesepA}
\setlength{\sarnodesepBsav}{\sarnodesepB}
\addtolength{\sarnodesepA}{3pt}
\addtolength{\sarnodesepB}{3pt}
\setcounter{EangleA}{ #1 + #2}
\setcounter{EangleB}{180  - #1 + #2}
\setcounter{EmidangleA}{#2}
\setcounter{EmidangleB}{#2 + 180}
\nccurve[nodesepA=\sarnodesepA,nodesepB=\sarnodesepB,offsetA=\saroffsetA,offsetB=\saroffsetB,angleA=\theEangleA, angleB=\theEangleB,linestyle=none,linewidth=0]{->}{#3}{#4}
\ncput{\pnode(0,0){xxx}}
\nccurve[nodesepA=0,nodesepB=\sarnodesepB,offsetA=0,offsetB=\saroffsetB,angleA=\theEmidangleA, angleB=\theEangleB, linestyle=#6]{->}{xxx}{#4}
%the following provides context for any following label
\nccurve[nodesepA=\sarnodesepA,nodesepB=0,offsetA=\saroffsetA,offsetB=0,angleA=\theEangleA, angleB=\theEmidangleB,linestyle=#5]{-}{#3}{xxx}
\setlength{\sarnodesepA}{\sarnodesepAsav}
\setlength{\sarnodesepB}{\sarnodesepBsav}
}


\newcounter{EangleGiven}
\newcounter{EangleComplementary}
\newcounter{EangleStartCorrected}
\newcounter{EangleEndCorrected}


%  rEp   - recursive Edge partial
\newcommand{\rEp}[2][0]{
\setcounter{EangleGiven}{#1}
\setcounter{EangleStartCorrected}{#1-10} %correction required because for nccurve unlike nccircle angle measured at boundary not at centre of node
\setcounter{EangleEndCorrected}{#1+180+10} %correction required because angle measured at boundary not at centre of node
\setcounter{EangleComplementary}{#1 + 180}
\nccircle[angleA=\theEangleComplementary, nodesep=0pt, linestyle=none]{-}{#2}{.4cm} % an invisible circle to hang the midpoint from
\ncput{\pnode(0,0){midpoint}}                                         
\nccurve[nodesepA=1pt,nodesepB=0pt,offsetA=0pt,offsetB=0pt,angleA=\theEangleStartCorrected, angleB=\theEangleGiven, ncurv=1.359, linecolor=black, linestyle=dashed]{-}{#2}{midpoint}
\ncput[nrot=:R,npos=0]{\psline(0,.1)(.075,0)}
\ncput[nrot=:R,npos=0]{\psline(0,.1)(-0.075,0)}
\nccurve[nodesepA=0pt,nodesepB=2pt,offsetA=0pt,offsetB=0pt,angleA=\theEangleComplementary, angleB=\theEangleEndCorrected, ncurv=1.359, linestyle=dashed]{-}{midpoint}{#2}
% 1.359 is e/2 happenchance or algorithmically necessary???
% now draw arrowhead -- dont include in the nccurve because this alters the line position - a strange feature of pstruicks
\ncput[npos=0.9]{\pnode(0,0){yyy}}
\ncline{->}{yyy}{#2}
% repeat from earlier to provide context for label that might follow
\nccurve[nodesepA=1pt,nodesepB=0pt,offsetA=0pt,offsetB=0pt,angleA=\theEangleStartCorrected, angleB=\theEangleGiven, ncurv=1.359, linecolor=black, linestyle=dashed]{-}{#2}{midpoint} 
} 

%  rEpm  - recursive Edge partial mono
\newcommand{\rEpm}[2][0]{
\setcounter{EangleGiven}{#1}
\setcounter{EangleStartCorrected}{#1-10} %correction required because for nccurve unlike nccircle angle measured at boundary not at centre of node
\setcounter{EangleEndCorrected}{#1+180+10} %correction required because angle measured at boundary not at centre of node
\setcounter{EangleComplementary}{#1 + 180}
\nccircle[angleA=\theEangleComplementary, nodesep=0pt, linestyle=none]{-}{#2}{.4cm} % an invisible circle to hang the midpoint from
\ncput{\pnode(0,0){midpoint}}   
\nccurve[nodesepA=0pt,nodesepB=2pt,offsetA=0pt,offsetB=0pt,angleA=\theEangleComplementary, angleB=\theEangleEndCorrected, ncurv=1.359, linestyle=dashed]{-}{midpoint}{#2}
% 1.359 is e/2 happenchance or algorithmically necessary???
% now draw arrowhead -- dont include in the nccurve because this alters the line position - a strange feature of pstruicks
\ncput[npos=0.9]{\pnode(0,0){yyy}}
\ncline{->}{yyy}{#2}
% last to provide context for label that might follow
\nccurve[nodesepA=1pt,nodesepB=0pt,offsetA=0pt,offsetB=0pt,angleA=\theEangleStartCorrected, angleB=\theEangleGiven, ncurv=1.359, linecolor=black, linestyle=dashed]{-}{#2}{midpoint} 
}

%  rEpe  - recursive Edge partial epi
\newcommand{\rEpe}[2][0]{
\setcounter{EangleGiven}{#1}
\setcounter{EangleStartCorrected}{#1-10} %correction required because for nccurve unlike nccircle angle measured at boundary not at centre of node
\setcounter{EangleEndCorrected}{#1+180+10} %correction required because angle measured at boundary not at centre of node
\setcounter{EangleComplementary}{#1 + 180}
\nccircle[angleA=\theEangleComplementary, nodesep=0pt, linestyle=none]{-}{#2}{.4cm} % an invisible circle to hang the midpoint from
\ncput{\pnode(0,0){midpoint}}                                         
\nccurve[nodesepA=1pt,nodesepB=0pt,offsetA=0pt,offsetB=0pt,angleA=\theEangleStartCorrected, angleB=\theEangleGiven, ncurv=1.359, linecolor=black, linestyle=dashed]{-}{#2}{midpoint}
\ncput[nrot=:R,npos=0]{\psline(0,.1)(.075,0)}
\ncput[nrot=:R,npos=0]{\psline(0,.1)(-0.075,0)}
\nccurve[nodesepA=0pt,nodesepB=2pt,offsetA=0pt,offsetB=0pt,angleA=\theEangleComplementary, angleB=\theEangleEndCorrected, ncurv=1.359]{-}{midpoint}{#2}
% 1.359 is e/2 happenchance or algorithmically necessary???
% now draw arrowhead -- dont include in the nccurve because this alters the line position - a strange feature of pstruicks
\ncput[npos=0.9]{\pnode(0,0){yyy}}
\ncline{->}{yyy}{#2}
% repeat from earlier to provide context for label that might follow
\nccurve[nodesepA=1pt,nodesepB=0pt,offsetA=0pt,offsetB=0pt,angleA=\theEangleStartCorrected, angleB=\theEangleGiven, ncurv=1.359, linecolor=black, linestyle=dashed]{-}{#2}{midpoint} 
}

%  rEpme - recursive Edge partial mono epi
\newcommand{\rEpme}[2][0]{
\setcounter{EangleGiven}{#1}
\setcounter{EangleStartCorrected}{#1-10} %correction required because for nccurve unlike nccircle angle measured at boundary not at centre of node
\setcounter{EangleEndCorrected}{#1+180+10} %correction required because angle measured at boundary not at centre of node
\setcounter{EangleComplementary}{#1 + 180}
\nccircle[angleA=\theEangleComplementary, nodesep=0pt, linestyle=none]{-}{#2}{.4cm} % an invisible circle to hang the midpoint from
\ncput{\pnode(0,0){midpoint}}                                         
%\nccurve[nodesepA=0pt,nodesepB=0pt,offsetA=0pt,offsetB=0pt,angleA=\theEangleComplementary, angleB=\theEangleEndCorrected, ncurv=1.359, linestyle=dashed]{->}{xxx}{#2}
\nccurve[nodesepA=0pt,nodesepB=2pt,offsetA=0pt,offsetB=0pt,angleA=\theEangleComplementary, angleB=\theEangleEndCorrected, ncurv=1.359]{-}{midpoint}{#2}
% 1.359 is e/2 happenchance or algorithmically necessary???
% now draw arrowhead -- dont include in the nccurve because this alters the line position - a strange feature of pstruicks
\ncput[npos=0.9]{\pnode(0,0){yyy}}
\ncline{->}{yyy}{#2}
% last so that to provide context for label that might follow
\nccurve[nodesepA=1pt,nodesepB=0pt,offsetA=0pt,offsetB=0pt,angleA=\theEangleStartCorrected, angleB=\theEangleGiven, ncurv=1.359, linecolor=black, linestyle=dashed]{-}{#2}{midpoint} 
}

% rEt - recursive Edge total
\newcommand{\rEt}[2][0]{
\setcounter{EangleGiven}{#1}
\setcounter{EangleStartCorrected}{#1-10} %correction required because for nccurve unlike nccircle angle measured at boundary not at centre of node
\setcounter{EangleEndCorrected}{#1+180+10} %correction required because angle measured at boundary not at centre of node
\setcounter{EangleComplementary}{#1 + 180}
\nccircle[angleA=\theEangleComplementary, nodesep=0pt, linestyle=none]{-}{#2}{.4cm} % an invisible circle to hang the midpoint from
\ncput{\pnode(0,0){midpoint}}                                         
\nccurve[nodesepA=1pt,nodesepB=0pt,offsetA=0pt,offsetB=0pt,angleA=\theEangleStartCorrected, angleB=\theEangleGiven, ncurv=1.359, linecolor=black]{-}{#2}{midpoint}
\ncput[nrot=:R,npos=0]{\psline(0,.1)(.075,0)}
\ncput[nrot=:R,npos=0]{\psline(0,.1)(-0.075,0)}
%\nccurve[nodesepA=0pt,nodesepB=0pt,offsetA=0pt,offsetB=0pt,angleA=\theEangleComplementary, angleB=\theEangleEndCorrected, ncurv=1.359, linestyle=dashed]{->}{xxx}{#2}
\nccurve[nodesepA=0pt,nodesepB=2pt,offsetA=0pt,offsetB=0pt,angleA=\theEangleComplementary, angleB=\theEangleEndCorrected, ncurv=1.359, linestyle=dashed]{-}{midpoint}{#2}
% 1.359 is e/2 happenchance or algorithmically necessary???
% now draw arrowhead -- dont include in the nccurve because this alters the line position - a strange feature of pstruicks
\ncput[npos=0.9]{\pnode(0,0){yyy}}
\ncline{->}{yyy}{#2}
% repeat from earlier to provide context for label that might follow
\nccurve[nodesepA=1pt,nodesepB=0pt,offsetA=0pt,offsetB=0pt,angleA=\theEangleStartCorrected, angleB=\theEangleGiven, ncurv=1.359, linecolor=black]{-}{#2}{midpoint} 
}

%  rEtm  - recursive Edge total mono
\newcommand{\rEtm}[2][0]{
\setcounter{EangleGiven}{#1}
\setcounter{EangleStartCorrected}{#1-10} %correction required because for nccurve unlike nccircle angle measured at boundary not at centre of node
\setcounter{EangleEndCorrected}{#1+180+10} %correction required because angle measured at boundary not at centre of node
\setcounter{EangleComplementary}{#1 + 180}
\nccircle[angleA=\theEangleComplementary, nodesep=0pt, linestyle=none]{-}{#2}{.4cm} % an invisible circle to hang the midpoint from
\ncput{\pnode(0,0){midpoint}}     
\nccurve[nodesepA=0pt,nodesepB=2pt,offsetA=0pt,offsetB=0pt,angleA=\theEangleComplementary, angleB=\theEangleEndCorrected, ncurv=1.359, linestyle=dashed]{-}{midpoint}{#2}
% 1.359 is e/2 happenchance or algorithmically necessary???
% now draw arrowhead -- dont include in the nccurve because this alters the line position - a strange feature of pstruicks
\ncput[npos=0.9]{\pnode(0,0){yyy}}
\ncline{->}{yyy}{#2}
% last to provide context for label that might follow
\nccurve[nodesepA=1pt,nodesepB=0pt,offsetA=0pt,offsetB=0pt,angleA=\theEangleStartCorrected, angleB=\theEangleGiven, ncurv=1.359, linecolor=black]{-}{#2}{midpoint} 
}

%  rEte  - total epi
\newcommand{\rEte}[2][0]{
\setcounter{EangleGiven}{#1}
\setcounter{EangleStartCorrected}{#1-10} %correction required because for nccurve unlike nccircle angle measured at boundary not at centre of node
\setcounter{EangleEndCorrected}{#1+180+10} %correction required because angle measured at boundary not at centre of node
\setcounter{EangleComplementary}{#1 + 180}
\nccircle[angleA=\theEangleComplementary, nodesep=0pt, linestyle=none]{-}{#2}{.4cm} % an invisible circle to hang the midpoint from
\ncput{\pnode(0,0){midpoint}}                                         
\nccurve[nodesepA=1pt,nodesepB=0pt,offsetA=0pt,offsetB=0pt,angleA=\theEangleStartCorrected, angleB=\theEangleGiven, ncurv=1.359, linecolor=black]{-}{#2}{midpoint}
\ncput[nrot=:R,npos=0]{\psline(0,.1)(.075,0)}
\ncput[nrot=:R,npos=0]{\psline(0,.1)(-0.075,0)}
%\nccurve[nodesepA=0pt,nodesepB=0pt,offsetA=0pt,offsetB=0pt,angleA=\theEangleComplementary, angleB=\theEangleEndCorrected, ncurv=1.359, linestyle=dashed]{->}{xxx}{#2}
\nccurve[nodesepA=0pt,nodesepB=2pt,offsetA=0pt,offsetB=0pt,angleA=\theEangleComplementary, angleB=\theEangleEndCorrected, ncurv=1.359]{-}{midpoint}{#2}
% 1.359 is e/2 happenchance or algorithmically necessary???
% now draw arrowhead -- dont include in the nccurve because this alters the line position - a strange feature of pstruicks
\ncput[npos=0.9]{\pnode(0,0){yyy}}
\ncline{->}{yyy}{#2}
% repeat from earlier to provide context for label that might follow
\nccurve[nodesepA=1pt,nodesepB=0pt,offsetA=0pt,offsetB=0pt,angleA=\theEangleStartCorrected, angleB=\theEangleGiven, ncurv=1.359, linecolor=black]{-}{#2}{midpoint} 
}

%  rEtme - recursive Edge total mono epi

\newcommand{\rEtme}[2][0]{
\setcounter{EangleGiven}{#1}
\setcounter{EangleStartCorrected}{#1-10} %correction required because for nccurve unlike nccircle angle measured at boundary not at centre of node
\setcounter{EangleEndCorrected}{#1+180+10} %correction required because angle measured at boundary not at centre of node
\setcounter{EangleComplementary}{#1 + 180}
\nccircle[angleA=\theEangleComplementary, nodesep=0pt, linestyle=none]{-}{#2}{.4cm} % an invisible circle to hang the midpoint from
\ncput{\pnode(0,0){midpoint}}     
\nccurve[nodesepA=0pt,nodesepB=2pt,offsetA=0pt,offsetB=0pt,angleA=\theEangleComplementary, angleB=\theEangleEndCorrected, ncurv=1.359]{-}{midpoint}{#2}
% 1.359 is e/2 happenchance or algorithmically necessary???
% now draw arrowhead -- dont include in the nccurve because this alters the line position - a strange feature of pstruicks
\ncput[npos=0.9]{\pnode(0,0){yyy}}
\ncline{->}{yyy}{#2}
% last to provide context for label that might follow
\nccurve[nodesepA=1pt,nodesepB=0pt,offsetA=0pt,offsetB=0pt,angleA=\theEangleStartCorrected, angleB=\theEangleGiven, ncurv=1.359, linecolor=black]{-}{#2}{midpoint} 
}

%gats.macros.tex

\usepackage{environ}    % also used in ermacros % here used for \NewEnvrion

\newcommand{\gat}[1][U]{
\ensuremath{\mathcal{#1}}}  % used to hav a space in here
\newcommand{\gatw}[1][U]{\gat[#1]\ }  % use this if need trailing space
\newcommand{\ingat}[1][U]{in \gat[#1]}
\newcommand{\isagat}[1][U]{\gat[#1] is a g.a.t.}
\newcommand{\inagat}{in a g.a.t. }

% macro for a generic theory
%\newcommand{\theory}
%{\textit{U}}

\newcommand{\intheory}
{is a derived rule of \gat[U]}

% Macros for GAT rules

\newcommand{\isT}[1]
{#1\mbox{ is a type}}

\newcommand{\ofT}[2]
{#1 \in #2
}

% Macros for GAT rules   <!-- new old -->
\newcommand{\istype}[1]
{#1\mbox{ is a type}}

\newcommand{\oftype}[2]
{#1 \in #2
}

%\context{x}{\Delta}{n}
\newcommand{\context}[3]
{\ofT{#1_1}{#2_1},... \ofT{#1_{#3}}{#2_{#3}(#1_1,...#1_{#3-1})}
}

%\subcontext{x}{\Delta}{i}{k}
\newcommand{\subcontext}[4]
{\ofT{#1_{#3_1}}{#2_{#3_1}},... \ofT{#1_{#3_#4}}{#2_{#3_#4}(#1_1,...#1_{#3_#4-1})}
}

% #schematic context
\newcommand{\schmcon}[3]
{\ofT{#1_1}{#2_1},... \ofT{#1_{#3}}{#2_{#3}}
}
% abbreviated to
\newcommand{\con}[3]
{\schmcon{#1}{#2}{#3}}

% schematic subcontext
%\subcon{x}{\Delta}{i}{k}
\newcommand{\subcon}[4]
{\ofT{#1_{#3_1}}{#2_{#3_1}},... \ofT{#1_{#3_#4}}{#2_{#3_#4}}
}

% permuted context
%\permcon{x}{\Delta}{n}{\sigma}
\newcommand{\permcon}[4]
{\ofT{#1_{#4(1)}}{#2_{#4(1)}},... \ofT{#1_{#4(#3)}}{#2_{#4(#3)}}
}
% permuted term
%\permterm{t}{n}{\sigma}
\newcommand{\permterm}[3]
{
#1_{#3(1)},...#1_{#3(#2)}
}


% Idioms
\newcommand{\xDelta}[1]{\con{x}{\Delta}{#1}}
\newcommand{\xDeltap}[1]{\con{x}{\Delta'}{#1}}
\newcommand{\xOmega}[1]{\con{x}{\Omega}{#1}}
\newcommand{\xOmegap}[1]{\con{x}{\Omega'}{#1}}
\newcommand{\yOmega}[1]{\con{y}{\Omega}{#1}}
\newcommand{\yOmegap}[1]{\con{y}{\Omega'}{#1}}

\newcommand{\xDeltasigma}[1]{\permcon{x}{\Delta}{#1}{\sigma}}
\newcommand{\xDeltapsigma}[1]{\permcon{x}{\Delta'}{#1}{\sigma}}
\newcommand{\xOmegasigma}[1]{\permcon{x}{\Omega}{#1}{\sigma}}
\newcommand{\xOmegapsigma}[1]{\permcon{x}{\Omega'}{#1}{\sigma}}
\newcommand{\yOmegasigma}[1]{\permcon{y}{\Omega}{#1}{\sigma}}
\newcommand{\yOmegapsigma}[1]{\permcon{y}{\Omega'}{#1}{\sigma}}

\newcommand{\xDeltainvsigma}[1]{\permcon{x}{\Delta}{#1}{\sigma^{-1}}}
\newcommand{\xDeltapinvsigma}[1]{\permcon{x}{\Delta'}{#1}{\sigma^{-1}}}
\newcommand{\xOmegainvsigma}[1]{\permcon{x}{\Omega}{#1}{\sigma^{-1}}}
\newcommand{\xOmegapinvsigma}[1]{\permcon{x}{\Omega'}{#1}{\sigma^{-1}}}
\newcommand{\yOmegainvsigma}[1]{\permcon{y}{\Omega}{#1}{\sigma^{-1}}}
\newcommand{\yOmegapinvsigma}[1]{\permcon{y}{\Omega'}{#1}{\sigma^{-1}}}

%Idioms enclosed as tuples
\newcommand{\encxDelta}[1]{\tuple{\con{x}{\Delta}{#1}}}
\newcommand{\encxDeltap}[1]{\tuple{\con{x}{\Delta'}{#1}}}
\newcommand{\encxOmega}[1]{\tuple{\con{x}{\Omega}{#1}}}
\newcommand{\encxOmegap}[1]{\tuple{\con{x}{\Omega'}{#1}}}
\newcommand{\encyOmega}[1]{\tuple{\con{y}{\Omega}{#1}}}
\newcommand{\encyOmegap}[1]{\tuple{\con{y}{\Omega'}{#1}}}

\newcommand{\encxDeltasigma}[1]{\tuple{\permcon{x}{\Delta}{#1}{\sigma}}}
\newcommand{\encxDeltapsigma}[1]{\tuple{\permcon{x}{\Delta'}{#1}{\sigma}}}
\newcommand{\encxOmegasigma}[1]{\tuple{\permcon{x}{\Omega}{#1}{\sigma}}}
\newcommand{\encxOmegapsigma}[1]{\tuple{\permcon{x}{\Omega'}{#1}{\sigma}}}
\newcommand{\encyOmegasigma}[1]{\tuple{\permcon{y}{\Omega}{#1}{\sigma}}}
\newcommand{\encyOmegapsigma}[1]{\tuple{\permcon{y}{\Omega'}{#1}{\sigma}}}

\newcommand{\encxDeltainvsigma}[1]{\tuple{\permcon{x}{\Delta}{#1}{\sigma^{-1}}}}
\newcommand{\encxDeltapinvsigma}[1]{\tuple{\permcon{x}{\Delta'}{#1}{\sigma^{-1}}}}
\newcommand{\encxOmegainvsigma}[1]{\tuple{\permcon{x}{\Omega}{#1}{\sigma^{-1}}}}
\newcommand{\encxOmegapinvsigma}[1]{\tuple{\permcon{x}{\Omega'}{#1}{\sigma^{-1}}}}
\newcommand{\encyOmegainvsigma}[1]{\tuple{\permcon{y}{\Omega}{#1}{\sigma^{-1}}}}
\newcommand{\encyOmegapinvsigma}[1]{\tuple{\permcon{y}{\Omega'}{#1}{\sigma^{-1}}}}

\newcommand{\tstyle}{\vdash}
\newcommand{\gatdisplayrule}[3][]
{
\setlength{\fboxsep}{1pt}       
\setlength{\fboxrule}{0pt}
\fbox{$\displaystyle \frac{#2}{#3\rule[-0.3cm]{0cm}{0cm}}$#1}    %added vertival space using \rule
}
\newcommand{\genericAintroductoryrule} {\gatdisplayrule{\xDelta{n}}{\isT{A(\xn)}}}
\newcommand{\genericfintroductoryrule}  {\gatdisplayrule{\xDelta{n}}{\ofT{f(\xn)}{\Delta}}}

% gat macros developed for cwf paper

% Expressing gats
\newenvironment{gatrules}
{
$$
\begin{array}{l l}
}
{
\end{array}
$$
}
\newcommand{\gatintros}
{
\textbf{Symbol} & \textbf{Introductory\ Rule}                      \\}

\newcommand{\gataxioms}
{\textbf{Axioms}\\}
\newcommand{\gatintro}[3]{\ #1 & #2 \tstyle #3 \\}
\newcommand{\gatlocalintro}[3]{\ #1 & #2 \dashv }
\newcommand{\gataxiom}[2]{\multicolumn{2}{l}{\ \ #1\mbox{,  whenever\ } #2} \\}
\newcommand{\noleft}{\left.\kern-\nulldelimiterspace} % so that no space taken by absent left brace


\newcommand{\gatmultiaxiom}[2]
{\multicolumn{2}{l}{
  \noleft
    \begin{array}{l}
		#1
    \end{array} 
  \right\} \mbox{whenever\ } 	#2 
	}\\}
	
	\newcommand{\axid}[1]{\text{#1}.\ }	

%New context sharing macros
\newcommand{\gatintroducing}[1]{
{\arraycolsep=0pt
  \begin{array}{l}
          #1
  \end{array}} &
}

%*********************************
% \begin{\gatgroup}{context}
%    rules
%  \end{\gatgroup}
%*********************************
\NewEnviron{gatgroup}[1]{%
  \noleft
  {\arraycolsep=0pt
   \begin{array}{l}
\BODY
    \end{array} 
   }
   \ \right\} 
	%\mbox{\ whenever\ } 
	#1
	\vspace{0.1cm} 
}
%*********************************

%*********************************
% \begin{\gatgroupnoshared}
%    rule
%  \end{\gatgroupnoshared}
%*********************************
\NewEnviron{gatgroupnoshared}{%
  {\arraycolsep=0pt
   \begin{array}{l}
\BODY
    \end{array} 
   }
   \ 
	\vspace{0.1cm} 
}
%*********************************

% \gatsingular[width]{context}{conclusion}
\newcommand{\gatsingular}[3][4cm]{
\begin{gatgroupnoshared}
\gatleaf[#1]{#2}{#3} 
\end{gatgroupnoshared}
}

%*********************************
% \gatleaf}[width]{context}{assertion}
%*********************************
\newcommand{\gatleaf}[3][4cm]{%
\makebox[#1]{$#3$ \dotfill} \dotfill \  #2
}
%*********************************
%*********************************
% \gatstandalonesingle}{context}{assertion}
%*********************************
\newcommand{\gatstandalonesingle}[2]{%
#2 \makebox[2.5cm]{\dotfill} \  #1
}
%*********************************

% \gataxiomno{axiomno}
\newcommand{\gataxiomno}[1]{\makebox[0.5cm]{} \axid{#1}}


% metagat.macros.tex

%Meta-theories

%\newcommand{\typ}{\triangleright}
\newcommand{\typ}{\nabla}
\newcommand{\trm}{\tau}
\newcommand{\cross}{\otimes}
\newcommand{\sub}{^*}
\newcommand{\diag}{\delta}

\newcommand{\typeseq}[2]
{\ofT{#1_1}{\typ},... \ofT{#1_{#2}}{\typ(#1_{#2-1})}}

\newcommand{\typeseqcont}[3]
{\ofT{#1_1}{\typ({#2})},... \ofT{#1_{#3}}{\typ(#1_{#3-1})}}

\newcommand{\Ob}{Ob}
\newcommand{\obj}{Ob} % <!-- new old --<
\newcommand{\Hom}{Hom}
\newcommand{\objseq}[2]
{\ofT{#1_1}{\obj},... \ofT{#1_{#2}}{\obj(#1_{#2-1})}}


\def\dottededge{\ncline[linestyle=dotted, nodesep=0.3cm]}
\def\noedge{\ncline[linestyle=none]}
\def\thinedge{\ncline[linewidth=0.4pt]}

\newcommand{\member}[1]
{\ncarc[arcangle=-30,nodesepB=0.03]{->}{\pspred}{\pssucc}
\nbput[labelsep=0.1]{#1}}

\newcommand{\loweraccutemember}[1]
{\ncarc[arcangle=-15,nodesepB=0.03]{->}{\pspred}{\pssucc}
\nbput[labelsep=0.05,npos=0.85]{#1}}

\newcommand{\uppermember}[1]
{\ncarc[arcangle=30,nodesepB=0.03]{->}{\pspred}{\pssucc}\naput{#1}}

\newcommand{\upperaccutemember}[1]
{\ncarc[arcangle=10,nodesepB=0.03]{->}{\pspred}{\pssucc}\naput[npos=0.85]{#1}}

% flexbranch 
% #1 node label
% #2 thislevelsep
% #3 next level sep
% #4 variable (eg x)
% #5 index leter (eg n)
% #6 close parenthesis
% #7 continuation branches
\newcommand{\flexbranch}[7]
{
\pstree[thislevelsep=*#2,nodesep=0.05]
		{\Rnode{#1 1}{\Tr{#4_1 #6}}}
	  {\pstree[thislevelsep=#3]  
				   {\Rnode{#1 2}{\Tr[edge=\dottededge]{#4_{#5} #6}}}
					 {#7}
		}
}

\newcommand{\flexbranchplusleaf}[6]
{
\flexbranch{#1}{#2}{#3}{#4} {#5} {#6}
  {
   %\Rnode{#1 3}{\Tr{#4 #6}}
	 \Tr{\Rnode{#1 3}{#4 #6}}
  }
}

\newcommand{\flexbranchplusarc}[7]
{
\flexbranch{#1}{#2}{#3}{#4} {#5} {#6}
  {
   %\Rnode{#1 3}{\Tr{#4 #6}\member{#7}}
	 \Tr{\Rnode{#1 3}{#4 #6}}\member{#7}
  }
}

\newcommand{\flexbranchinitialarc}[9]
{
\pstree[thislevelsep=*#2,nodesep=0.05]
		{\Rnode{#1 1}{\Tr{#4_#8 #6}}#9}
	  {\pstree[thislevelsep=#3]  
				   {\Rnode{#1 2}{\Tr[edge=\dottededge]{#4_{#5} #6}}}
					 {#7}
		}
}

\newcommand{\equality}[2]
{
\ncline [doubleline=true, nodesep=0.2cm]{#1}{#2}
}
\newcommand{\equalityarc}[2]
{
\ncarc [arcangleA=-30, arcangleB=-20, doubleline=true, nodesep=0.1cm]{#1}{#2}
}

%The following are stylistic so belong in main document not here.
%\usepackage[margin=4.0cm]{geometry} % This shouldn't be here commented out 17 July 2018
%\usepackage{mathptmx}               % This changes font to roman so doesn't belong here
%
\usepackage{amsfonts}
\usepackage{amssymb} % added 08\02\2019 as an experiment. Needed in some instances for \blacksquare
                     % not needed is class is `beamer' but I don't know why not
\usepackage{array}
\usepackage{pstricks}
\usepackage{pst-tree}
\usepackage{pst-plot}
\usepackage{pst-node}
\usepackage{stmaryrd}
\usepackage{amsmath}
\usepackage{verbatim}
\usepackage{graphicx}  
\usepackage{calc}
\usepackage{xifthen}
%\usepackage{xcolor} investigate with beamer
\usepackage{color}
\usepackage{stringstrings}
%\usepackage[small,bf,margin=3pt,format=hang, labelsep=endash,singlelinecheck=false]{caption} %prevuiously justification=justified
%\usepackage{enumerate}
%\usepackage{enumitem}
\usepackage{enumerate}
%\usepackage[shortlabels]{enumitem} %Removed this 28/01/2019 because interfereing with a beamer presentation. 
\usepackage{float}
\usepackage[section]{placeins}
%\setlength{\captionmargin}{5pt}
\usepackage{environ}
\usepackage{multirow}
\usepackage{rotating}
\usepackage{longtable}
\usepackage{afterpage}
\usepackage{needspace}


%DEFINE ENVIRONMENT BLOCK
% Riddle
\newsavebox{\riddlebox}

\newenvironment{erexample}
{\newcommand\colboxcolor{F0F0F0}%was F8F8F8
\begin{lrbox}{\riddlebox}
\begin{minipage}{\dimexpr\columnwidth-2\fboxsep\relax} \textbf{} \\ \itshape}
{\end{minipage}\end{lrbox}%
%\begin{center}
\colorbox[HTML]{\colboxcolor}{\usebox{\riddlebox}}
%\end{center}
}

\newenvironment{erbox}
{\newcommand\colboxcolor{F0F0F0}%was F8F8F8
\begin{lrbox}{\riddlebox}%
\begin{minipage}{\dimexpr\columnwidth-2\fboxsep\relax} }
{\end{minipage}\end{lrbox}%
%\begin{center}
\colorbox[HTML]{\colboxcolor}{\usebox{\riddlebox}}
%\end{center}
}

%\begin{erboxedFigure}{#1 FigureParam}{#2 Label}{#3 Caption}
\NewEnviron{erboxedFigure}[3]{%
\begin{figure}[#1]
\begin{erexample}
\begin{center}
\BODY
\end{center}
\vspace{-0.5cm}
\caption{#3}
\label{#2}
\end{erexample}
\end{figure}
}

\newcommand{\erpictureFolder}[0]{../SharedPictures}

\newcommand{\ercenterPicture}[1]{
\begin{center}
\input{\erpictureFolder/#1}
\end{center}
}


\newlength{\erhalfHt}

%\erinlinePicture{#1 pictureFilename}{#2 pictureHeight}
\newcommand{\erinlinePicture}[2]{
\setlength{\erhalfHt}{#2cm * \real{0.5}}
\raisebox{-\erhalfHt}[\erhalfHt + 0.5cm][\erhalfHt + 0.5cm]{
\input{\erpictureFolder/#1}
} 
}

%\erplainFig{#1 pictureFilename}{#2 figureParam}{#3Caption}
\newcommand{\erplainFig}[3]{
\begin{figure}[#2]
\begin{center}
\input{\erpictureFolder/#1}
\end{center}
\caption{#3}
\label{#1}
\end{figure}
}

%\erboxedFigPicture{#1 pictureFilename}{#2 figureParam}{#3Caption}
\newcommand{\erboxedFigPicture}[3]{
\begin{figure}[#2]
\begin{erexample}
\vspace{-0.5cm}
\begin{center}
\input{\erpictureFolder/#1}
\end{center}
\caption{#3}
\label{#1}
\end{erexample}
\end{figure}
}

%\erLeftSideFig{#1 pictureFilename}{#2 figureParam}{#3Caption}
\newcommand{\erLeftSideFig}[3]{
\begin{figure}[#2]
\begin{erexample}
  \begin{minipage}[c]{0.4\textwidth}
    \caption{#3}
    \label{#1}
  \end{minipage}
  \begin{minipage}[c]{0.5\textwidth}
    \input{\erpictureFolder/#1}
  \end{minipage}
\end{erexample}
\end{figure}
}

%\erbulletedFig{#1 pictureFilename}{#2 figureParam}{#3Caption}
\NewEnviron{erbulletedFig}[3]{%
\begin{figure}[#2]
\begin{erexample}
\vspace{-0.5cm}
\begin{center}
$
\begin{array}{c m{0.25cm} | m{6cm}}
\raisebox{-2.0cm}{
\input{\erpictureFolder/#1}}& & \text{\parbox{6cm}{\raggedright{\footnotesize{
\begin{enumerate}[(i)]
\BODY
\end{enumerate}}}}} \\
\end{array}
$
\end{center}
\caption{#3}
\label{#1}
\end{erexample}
\end{figure} 
}


%\begin{erbulletedDimFig}{#1 pictureFilename}{#2figureParam} {#3Caption} {#4PictureHeight}{#5TextWidth}

\NewEnviron{erbulletedDimFig}[5]{%
\begin{figure}[#2]
\begin{erexample}
\vspace{-0.5cm}
\begin{center}
$
\begin{array}{c m{0.25cm} |  m{#5cm}}
\setlength{\erhalfHt}{#4cm * \real{0.5}}
\raisebox{-\erhalfHt}{
\input{\erpictureFolder/#1}}& & \text{\parbox{#5cm}{\raggedright{\footnotesize{
\begin{enumerate}[(i)]
\BODY
\end{enumerate}}}}} \\
\end{array}
$
\end{center}
\caption{#3}
\label{#1}
\end{erexample}
\end{figure} 
}

%\begin{ernotedModel}{#1 pictureFilename}{#2PictureHeight}{#3PictureWidth}{#4TextWidth}

\NewEnviron{ernotedModel}[4]{%
\begin{center}
$
\begin{array}{m{#3cm} m{1cm} | c m{#4cm}}
\setlength{\erhalfHt}{#2cm * \real{0.5}}
\raisebox{-\erhalfHt}{
\input{\erpictureFolder/#1}}& & & \text{\parbox{#4cm}{\raggedright{\footnotesize{
\BODY
}}}} \\
\end{array}
$
\end{center} 
}

%\begin{ermodelText}{#1 pictureFilename}{#2PictureHeight}{#3PictureWidth}{#4TextWidth}

\NewEnviron{ermodelText}[4]{%
\begin{center}
\begin{tabular}{m{#3cm} m{1cm}  c m{#4cm}}
\setlength{\erhalfHt}{#2cm * \real{0.5}}
\raisebox{-\erhalfHt}{
\input{\erpictureFolder/#1}}& & & \text{\parbox{#4cm}{\raggedright{\small{
\BODY
}}}} \\
\end{tabular}
\end{center} 
}


%\erbulletedModel{#1 pictureFilename}{#2PictureHeight}{#3PictureWidth}{#4TextWidth}

\NewEnviron{erbulletedModel}[4]{%
\begin{center}
$
\begin{array}{m{#3cm} m{1cm} | c m{#4cm}}
\setlength{\erhalfHt}{2cm * \real{0.5}}
\raisebox{-\erhalfHt}{
\input{\erpictureFolder/#1}}& & & \text{\parbox{#4cm}{\raggedright{\footnotesize{
\begin{enumerate}[(i)]
\BODY
\end{enumerate}}}}} \\
\end{array}
$
\end{center} 
}



%\ernotedDimFig{#1 pictureFilename}{#2 figureParam}{#3Caption}{#4PictureHeight}{#5TextWidth}
\NewEnviron{ernotedDimFig}[5]{%
\begin{figure}[#2]
\begin{erexample}
\vspace{-0.5cm}
\begin{center}
$
\begin{array}{c m{0.25cm} | c m{#5cm}}
\setlength{\erhalfHt}{#4cm * \real{0.5}}
\raisebox{-\erhalfHt}{
\input{\erpictureFolder/#1}}& & & \text{\parbox{#5cm}{\raggedright{\footnotesize{
\BODY }}}}\\
\end{array}
$
\end{center}
\caption{#3}
\label{#1}
\end{erexample}
\end{figure} 
}
%\begin{ernotedDimFigPW}{#1 pictureFilename}{#2 figureParam}{#3Caption}{#4PictureHeight}{#5PictureWidth}{#6TextWidth}
\NewEnviron{ernotedDimFigPW}[6]{%
\begin{figure}[#2]
\begin{erexample}
\vspace{-0.5cm}
\begin{center}
$
\begin{array}{>{\centering}m{#5cm} m{0.5cm} | c m{#6cm}}
\setlength{\erhalfHt}{#4cm * \real{0.5}}
\raisebox{-\erhalfHt}{
\centering \input{\erpictureFolder/#1}
}& & & \text{\parbox{#6cm - 0.5cm}{\raggedright{\footnotesize{
\BODY }}}}\\
\end{array}
$ \\
\vspace {0.2cm}
\end{center}
\caption{#3}
\label{#1}
\end{erexample}
\end{figure}
}



\newenvironment{erquote}
{\begin{quote}\itshape}
{\end{quote}}


%
%  erdiagram.tex
%  *************
%  Macros to represent ER diagrams
%  *******************************
% 29/01/2019 Modify so that not reliant on the
%            default fontsize being 10pt by using
%            package anyfontsize and then
%            \fontsize{8}{10}\selectfont to set font to 8pt
% 06/02/2019 Pullback symbol implemented and minor tweaks to positioning 
%            and size of identifier symbol and relationship labels.
%            Accidental forked changes merged on 08/02/2019.
% 15/03/2019 Continuation of 29/01/2019. Need fix fontsize of 
%            ERrelname and ERscope.	 
% ***********************************************************
 \usepackage{anyfontsize}             % 29/01/2019 
  
%\begin{erdiagram}{#1 height}{#2 width} 
% ....
% ....
%\end{erdiagram}
\newenvironment{erdiagram}[2]
{%\pspicture*(-#1,0)(#2,0)
\pspicture*(0,-#1)(#2,0)
%\psgrid
}
{\endpspicture}

\definecolor{lightyellow}{cmyk}{0,0,0.3,0}
\definecolor{verylightgrey}{gray}{0.95}


% \eret{#1 x0} {#2 y0} {#3 x1} {#4 y1} {#5 corner radius} {#6 fill}
\newcommand {\eret}[6]
{ 
\ifthenelse{\equal{#6}{1}}
{\psframe[framearc=#5,fillstyle=solid,fillcolor=lightyellow](#1,#2)(#3,#4)}
{\psframe[framearc=#5,fillstyle=solid,fillcolor=white](#1,#2)(#3,#4)}
}

% et top 
\newcommand {\erettop}[4]
{
%\psframe[linestyle=none,linearc=2pt,cornersize=absolute,fillstyle=solid,fillcolor=lightyellow](#1,#2)(#3,#4)
\psline[linearc=2pt,fillstyle=none,fillcolor=lightyellow](#1,#4)(#1,#2)(#3,#2)(#3,#4)
}

% et bottom 
\newcommand {\eretbtm}[4]
{
%\psframe[linestyle=none,linearc=2pt,cornersize=absolute,fillstyle=solid,fillcolor=lightyellow](#1,#2)(#3,#4)
\psline[linearc=2pt,fillstyle=none,fillcolor=lightyellow](#1,#2)(#1,#4)(#3,#4)(#3,#2)
}

% et bottom left
\newcommand {\eretbl}[4]
{
%\psframe[linestyle=none,linearc=2pt,cornersize=absolute,fillstyle=solid,fillcolor=lightyellow](#1,#2)(#3,#4)
\psline[linearc=2pt,fillstyle=none,fillcolor=lightyellow](#1,#4)(#3,#4)(#3,#2)
}

% et middle left
\newcommand {\eretml}[4]
{
%\psframe[linestyle=none,linearc=2pt,cornersize=absolute,fillstyle=solid,fillcolor=lightyellow](#1,#2)(#3,#4)
\psline[linearc=2pt,fillstyle=none,fillcolor=lightyellow](#1,#2)(#3,#2)(#3,#4)(#1,#4)
}

% et top left
\newcommand {\erettl}[4]
{
%\psframe[linestyle=none,linearc=2pt,cornersize=absolute,fillstyle=solid,fillcolor=lightyellow](#1,#2)(#3,#4)
\psline[linearc=2pt,fillstyle=none,fillcolor=lightyellow](#1,#2)(#3,#2)(#3,#4)
}

% et bottom right
\newcommand {\eretbr}[4]
{
%\psframe[linestyle=none,linearc=2pt,cornersize=absolute,fillstyle=solid,fillcolor=lightyellow](#1,#2)(#3,#4)
\psline[linearc=2pt,fillstyle=none,fillcolor=lightyellow](#1,#2)(#1,#4)(#3,#4)
}

% et middle right
\newcommand {\eretmr}[4]
{
%\psframe[linestyle=none,linearc=2pt,cornersize=absolute,fillstyle=solid,fillcolor=lightyellow](#1,#2)(#3,#4)
\psline[linearc=2pt,fillstyle=none,fillcolor=lightyellow](#3,#4)(#1,#4)(#1,#2)(#3,#2)
}

% et top right
\newcommand {\erettr}[4]
{
\psline[linearc=2pt,fillstyle=none,fillcolor=lightyellow](#1,#4)(#1,#2)(#3,#2)
}

% \ergrp{#1 x0} {#2 y0} {#3 x1} {#4 y1} {#5 corner radius} {#6 fill}
% #5 corner radius is unused!
\newcommand {\ergrp}[6]
{ 
\ifthenelse{\equal{#6}{1}}
{\psframe[fillstyle=solid,fillcolor=verylightgrey](#1,#2)(#3,#4)}
{\psframe[fillstyle=solid,fillcolor=white](#1,#2)(#3,#4)}
}


% \ertext{#1 text}
% 15/03/2019
\newcommand {\erextrasmallitalictext}[1]
{\fontsize{7}{9}\selectfont \textit{#1}}

% 29/01/2019  
\newcommand {\ersmallitalictext}[1]
{\fontsize{8}{10}\selectfont \textit{#1}}

\newcommand {\ermediumitalictext}[1]
{\fontsize{10}{12}\selectfont \textit{#1}}

% \eretname {#1 x left of text} {#2 y top of text} {#3 text}
\newcommand {\olderetname}[3]
{
%shift down 0.1 for height of text the anchor at baseline (B)
\rput[bl]{0}(0,-0.1){\rput[Bl]{0}(#1,#2){\ersmallitalictext{#3}}}
}

% \errelarm {#1 x0} {#2 y0} {#3 x1} {#4 y1} {#5 ismandatory} {#6 isconstructed}
\newcommand {\errelarm}[6]
{
\ifthenelse{\equal{#6}{1}}
{
%%\psline[linewidth=0.5pt,linearc=.05,linestyle=dashed,dash=6pt 6pt]{-}(#1,#2)(#3,#4)}
\ifthenelse{\equal{#5}{1}}
{\psline[linewidth=1.5pt,linearc=.05,linecolor=lightgray]{-}(#1,#2)(#3,#4)}
{\psline[linewidth=1.5pt,linearc=.05,linecolor=lightgray,linestyle=dashed,dash=2pt 2pt]{-}(#1,#2)(#3,#4)}
}
{
\ifthenelse{\equal{#5}{1}}
{\psline[linewidth=0.9pt,linearc=.05]{-}(#1,#2)(#3,#4)}
{\psline[linewidth=0.9pt,linearc=.05,linestyle=dashed,dash=2pt 2pt]{-}(#1,#2)(#3,#4)}
}
}

% \errelangle {#1 x0} {#2 y0} {#3 x1} {#4 y1} {#5 x2} {#6 y2} {#7 ismandatory} {#8 isocnstructed}
\newcommand {\errelangle}[8]
{
\ifthenelse{\equal{#8}{1}}
{
%\psline[linewidth=0.5pt,linearc=.1,linestyle=dashed,dash=6pt 6pt]{-}(#1,#2)(#3,#4)(#5,#6)}
\ifthenelse{\equal{#7}{1}}
{\psline[linewidth=1.5pt,linearc=.05,linecolor=lightgray]{-}(#1,#2)(#3,#4)(#5,#6)}
{\psline[linewidth=1.5pt,linearc=.1,linecolor=lightgray,linestyle=dashed,dash=2pt 2pt]{-}(#1,#2)(#3,#4)(#5,#6)}
}
{
\ifthenelse{\equal{#7}{1}}
{\psline[linewidth=0.9pt,linearc=.1]{-}(#1,#2)(#3,#4)(#5,#6)}
{\psline[linewidth=0.9pt,linearc=.1,linestyle=dashed,dash=2pt 2pt]{-}(#1,#2)(#3,#4)(#5,#6)}
}
}

% \ercrowfoot {#1 x0} {#2 y0} {#3 x11} {#4 y11} {#5 x12} {#6 y12} {#7 x13} {#8 y13} {#9 isconstructed}
\newcommand {\ercrowfoot}[9]
{
\ifthenelse{\equal{#9}{1}}
{
\psline[linewidth=1.5pt,linearc=.05,linecolor=lightgray]{-}(#1,#2)(#3,#4)
\psline[linewidth=1.5pt,linearc=.05,linecolor=lightgray]{-}(#1,#2)(#5,#6)
\psline[linewidth=1.5pt,linearc=.05,linecolor=lightgray]{-}(#1,#2)(#7,#8)
}{
\psline[linewidth=0.9pt,linearc=.05]{-}(#1,#2)(#3,#4)
\psline[linewidth=0.9pt,linearc=.05]{-}(#1,#2)(#5,#6)
\psline[linewidth=0.9pt,linearc=.05]{-}(#1,#2)(#7,#8)
}
}


% \eridcomprel{#1 x1}{#2 x2}{#3 y}
\newcommand {\eridcomprel}[3]
{
\psline[linewidth=0.9pt](#1,#3)(#2,#3)
}

% \eridrefrel{#1 x}{#2 y1}{#3 y2}
\newcommand {\eridrefrel}[3]
{
\psline[linewidth=0.9pt](#1,#2)(#1,#3)
}

% \ertext {#1 x} {#2 y} {#3 text anchor} {#4 text}  PRIVATE
\newcommand {\ertext}[4]
{
\rput[B#3]{0}(#1,#2){\fontsize{8}{10}\selectfont #4}
}

% \eretname {#1 x} {#2 y} {#3 text anchor} {#4 text} 
\newcommand {\eretname}[4]
{
\ertext{#1}{#2}{#3}{#4}
}

% \errelname {#1 x} {#2 y} {#3 text anchor} {#4 text} 
\newcommand {\errelname}[4]
{
\rput[B#3]{0}(#1,#2){\erextrasmallitalictext{#4}}
}


% \erscope {#1 x} {#2 y} {#3 text anchor} {#4 text}  15 March 2019
\newcommand {\erscope}[4]
{
\rput[B#3]{0}(#1,#2){\erextrasmallitalictext{#4}}
}

% \erreletname {#1 x} {#2 y} {#3 text anchor} {#4 text}  15 March 2019
\newcommand {\erreletname}[4]
{
\rput[B#3]{0}(#1,#2){\fontsize{10}{12}\selectfont #4}
}

% \ergroupannotation {#1 x} {#2 y} {#3 text anchor} {#4 text}
\newcommand {\ergroupannotation}[4]
{
\ertext{#1}{#2}{#3}{#4}
}


% \errelseq {#1 x} {#2 y}
\newcommand {\erelseq}[2]
{
}
\newcommand {\erattrmarkermand}
{\fontsize{6}{8}\selectfont $\blacksquare$}
\newcommand {\erattrmarkeropt}
{\fontsize{6}{8}\selectfont \CIRCLE}
\newcommand {\erderattrmarkermand}
{\fontsize{6}{8}\selectfont $\square$}
\newcommand {\erderattrmarkeropt}
{\fontsize{8}{10}\selectfont $\circ$}

% \erattr {#1 x} {#2 y} {#3 ismandatory}{#4 idenitfying} {#5 text}
\newcommand {\erattr}[5]
{
\ifthenelse{\equal{#3}{1}}
{\rput[l]{0}(#1,#2){\erattrmarkermand \ersmallitalictext{\ifthenelse{\equal{#4}{0}}{\underline{#5}}{#5}}}}
{\rput[l]{0}(#1,#2){\erattrmarkeropt \ersmallitalictext{\ifthenelse{\equal{#4}{0}}{\underline{#5}}{#5}}}}
}

\newcommand {\erdattr}[5]
{
\ifthenelse{\equal{#3}{1}}
{\rput[l]{0}(#1,#2){\erderattrmarkermand \ersmallitalictext{\ifthenelse{\equal{#4}{0}}{\underline{#5}}{#5}}}}
{\rput[l]{0}(#1,#2){\erderattrmarkeropt \ersmallitalictext{\ifthenelse{\equal{#4}{0}}{\underline{#5}}{#5}}}}
}


% \erarc {#1 x0} {#2 y0} {#3 x1} {#4 y1} {#5 x2} {#6 y2} {#7 x3} {#8 y3}
\newcommand {\erarc}[8]
{
\psbezier[showpoints=false]{-}(#1,#2) (#3, #4)(#5,#6) (#7, #8)
}

% \erarc {#1 x0} {#2 y0} {#3 x1} {#4 y1} {#5 x2} {#6 y2} {#7 x3} {#8 y3}
\newcommand {\errelseq}[8]
{
\psbezier[showpoints=false]{-}(#1,#2) (#3, #4)(#5,#6) (#7, #8)
}
% \ertrace {#1 trace}   
\newcommand {\ertrace}[1]
{
}

\usepackage{amsthm} % added 7th April 2018
% theorems.macros.tex

\newtheorem{theorem}{Theorem}[section]
\newtheorem{observation}[theorem]{Observation}
\newtheorem{lemma}[theorem]{Lemma}
\newtheorem{proposition}[theorem]{Proposition}
\newtheorem{corollary}[theorem]{Corollary}
\newtheorem{conjecture}[theorem]{Conjecture}
\newtheorem{numbereddefinition}[theorem]{Definition}

\newenvironment{definition}[1][Definition]{\begin{trivlist}
\item[\hskip \labelsep {\bfseries #1}]}{\end{trivlist}}
\newenvironment{examples}[1][Examples]{\begin{trivlist}
\item[\hskip \labelsep {\bfseries #1}]}{\end{trivlist}}
\newenvironment{example}[1][Example]{\begin{trivlist}
\item[\hskip \labelsep {\bfseries #1}]}{\end{trivlist}}
\newenvironment{remark}[1][Remark]{\begin{trivlist}
\item[\hskip \labelsep {\bfseries #1}]}{\end{trivlist}}

\newenvironment{tageqn}[1]
{
\begin{equation}
\stepcounter{equation}
\label{#1}
\tag{\theequation --#1}
}
{
\end{equation}
}

\newenvironment{axiom}[1]
{
\begin{equation}
\label{#1}
\tag{#1}
}
{
\end{equation}
}

% when the tag is required different from the label eg when has math symbols can use:
\newenvironment{axiomtagged}[2]
{
\begin{equation}
\label{#1}
\tag{#2}
}
{
\end{equation}
}

%visible label
\newcommand{\vlabel}[2][]{\label{#2}#1(\textit{#2}):}





\usepackage{imakeidx}
\usepackage{framed}
\makeindex[name=definitions, title=Index of Definitions]
\makeindex[name=lemmas, title=Index of Lemmas]



\newcommand{\seenudgeup}[1]{\rule{0.1cm}{#1}}

\newcommand{\seenudgedown}[1]{\rule[-#1]{0.1cm}{0.1cm}}

\newcommand{\nudgeup}[1]{\rule{0cm}{#1}}

\newcommand{\nudgedown}[1]{\rule[-#1]{0cm}{0.1cm}}

\definecolor{highlight}{cmyk}{0,0,0.7,0}
\newcommand{\commentary}[1]{\marginpar{\footnotesize #1}}
\newcommand{\highlight}[1]{\colorbox{highlight}{#1}}
\newcommand{\whitelight}[1]{\colorbox{white}{#1}}
\newcommand{\term}[1]{\textit{#1}\commentary{\colorbox{lightgray}{\textit{#1}}}\index[definitions]{#1}}
\newcommand{\llabel}[1]{\label{#1}\commentary{\colorbox{pink}{\scriptsize{#1}}}\index[lemmas]{#1}}
\newcommand{\lref}[1]{\ref{#1}\colorbox{pink}{\scriptsize{#1}}\index[lemmas]{#1!use of}}

\newcommand{\daynote}[1]{\commentary{See day notes #1.}}

\newcommand{\newt}[1]{\colorbox{yellow}{#1}}
\newenvironment{newtt}
{  \colorbox{yellow}{$[$ ...} 
}
{  \colorbox{yellow}{... $]$}
}
\newcommand{\oldt}[1]{\colorbox{red}{\sout{#1}}}
\newenvironment{oldtt}
{  \colorbox{red}{$[$ ...} 
}
{  \colorbox{red}{... $]$}
}

\newcommand{\reinstatet}[1]{\colorbox{lime}{#1}}
\newenvironment{reinstatett}
{  \colorbox{lime}{$[$ ...}
}
{  \colorbox{lime}{... $]$}
}

\newcommand{\tbd}{\highlight{TBD}}

%ithprojection function
\newcommand{\proji}[1]{\pi_#1}


\newenvironment{aside}
{\begin{framed}
\textbf{Aside}
}
{
\end{framed}
}

\newenvironment{notebox}[1][Note]
{\begin{framed}
\textbf{#1}
}
{
\end{framed}
}

\newenvironment{categoricalaside}
{\begin{framed}
\textbf{Categorical Aside}
}
{
\end{framed}
}

\newenvironment{noteforfuture}
{\begin{framed}
\textbf{Note For Future}
}
{
\end{framed}
}

\newenvironment{problem}
{\begin{framed}
\textbf{Problem}
}
{
\end{framed}
}

\newenvironment{key}
{
\begin{tabular}{c l p{4cm}}
KEY && \\
\hline
}
{
\end{tabular}
}

\newcommand{\keyentry}[3]{#1 & #2 & #3 \\} 


%quine quote
\newcommand{\qq}[1]{
\left\ulcorner#1\right\urcorner
}

%single quote
\newcommand{\sq}[1]{
\textnormal{\textquotesingle}#1\textnormal{\textquotesingle}
}

%lower quine quote
\newcommand{\lqq}[1]{
\left\llcorner #1\right\lrcorner
}


%from berkley
\newcommand{\langl}{\begin{picture}(4.5,7)
\put(1.1,2.5){\rotatebox{60}{\line(1,0){5.5}}}
\put(1.1,2.5){\rotatebox{300}{\line(1,0){5.5}}}
\end{picture}}
\newcommand{\rangl}{\begin{picture}(4.5,7)
\put(.9,2.5){\rotatebox{120}{\line(1,0){5.5}}}
\put(.9,2.5){\rotatebox{240}{\line(1,0){5.5}}}
\end{picture}}
\newcommand{\lang}{\begin{picture}(5,7)\put(1.1,2.5){\rotatebox{45}{\line(1,0){6.0}}}\put(1.1,2.5){\rotatebox{315}{\line(1,0){6.0}}}\end{picture}}
\newcommand{\rang}{\begin{picture}(5,7)\put(.1,2.5){\rotatebox{135}{\line(1,0){6.0}}}\put(.1,2.5){\rotatebox{225}{\line(1,0){6.0}}}\end{picture}}
%Try sharper tuple brackets -- except gives errors nested in captions so comment out
%\renewcommand{\tuple}[1]{\lang #1 \rang}

\newcommand{\setsuchthat}[2]{\left\{#1 \ \middle|\ #2\right\}}
\newcommand{\set}[1]{\left\{#1\right\}} 

% one to n - wanton
\newcommand{\wanton}[1]{#1_1,...#1_n}
\newcommand{\n}{1...n}
\newcommand{\fn}{\wanton{f}}
\newcommand{\gn}{\wanton{g}}
\newcommand{\pn}{\wanton{p}}
\newcommand{\qn}{\wanton{q}}
\newcommand{\qnprime}{\wanton{q'}}
\newcommand{\tn}{\wanton{t}}
\newcommand{\xn}{\wanton{x}}
\newcommand{\xnp}{\wanton{x'}}
\newcommand{\yn}{\wanton{y}}
\newcommand{\An}{\wanton{A}}
\newcommand{\Bn}{\wanton{B}}
\newcommand{\Cn}{\wanton{C}}
\newcommand{\ntuple}[1]{\tuple{\wanton{#1}}}
\newcommand{\wantom}[2][]{#2_1,...#2_{m#1}}
\newcommand{\m}{1...m}
\newcommand{\mtuple}[1]{\tuple{#1_1,...#1_m}}
\newcommand{\gm}{\wantom{g}}
\newcommand{\qm}{\wantom{q}}
\newcommand{\sm}[1][]{\wantom[#1]{s}}
\newcommand{\smp}{\wantom{s'}}
\newcommand{\ym}{\wantom{y}}
\newcommand{\Bm}{\wantom{B}}
\newcommand {\bntuple}{\ensuremath{\ntuple{b}}}
\newcommand {\fntuple}{\ensuremath{\ntuple{f}}}
\newcommand {\fnptuple}{\ensuremath{\ntuple{f}}}
\newcommand {\pntuple}{\ensuremath{\ntuple{p}}}
\newcommand {\qntuple}{\ensuremath{\ntuple{q}}}
\newcommand {\qnptuple}{\ensuremath{\ntuple{q'}}}
\newcommand {\qmtuple}{\ensuremath{\mtuple{q}}}
\newcommand {\sntuple}{\ensuremath{\ntuple{s}}}
\newcommand {\xntuple}{\ensuremath{\ntuple{x}}}
\newcommand {\xnptuple}{\ensuremath{\ntuple{x'}}}
\newcommand {\ymtuple}{\ensuremath{\mtuple{y}}}
\newcommand{\idef}[1][n]{1 \leq i \leq #1}
\newcommand{\jdef}[1][m]{1 \leq j \leq #1}
\newcommand{\kdef}[1][l]{1 \leq k \leq #1}
\newcommand{\foreachi}[1][n]{for each $i$, $1 \leq i \leq #1$}
\newcommand{\foreachj}[1][m]{for each $j$, $1 \leq j \leq #1$}
\newcommand{\foreachk}[1][l]{for each $k$, $1 \leq k \leq #1$}
\newcommand{\Foreachi}[1][n]{For each $i$, $1 \leq i \leq #1$}
\newcommand{\Foreachj}[1][m]{For each $j$, $1 \leq j \leq #1$}
\newcommand{\Foreachk}[1][l]{For each $k$, $1 \leq k \leq #1$}
\newcommand{\forsomei}[1][n]{for some $i$, $1 \leq i \leq #1$}
\newcommand{\forsomej}[1][m]{for some $j$, $1 \leq j \leq #1$}
\newcommand{\forsomek}[1][l]{for some $k$, $1 \leq k \leq #1$}
\newcommand{\wherei}[1][n]{where $1 \leq i \leq #1$}
\newcommand{\wherej}[1][m]{where $1 \leq j \leq #1$}
\newcommand{\wherek}[1][l]{where $1 \leq k \leq #1$}


\newcommand{\fundep}[3]{#2 \xrightarrow{#1} #3}  %where does this belong? xxxx
% Following used for notes -- indented numbered paras

\newcounter{para}
\newlength{\oldparindent}
\setlength{\oldparindent}{\parindent} % Save \parindent before of change
\newcommand{\ind}{\hspace*{\oldparindent}}
\newcommand\note{
%\setlength{\parskip}{0.5\baselineskip} % Definition of `parskip`
\setlength{\parindent}{0pt}
\par\ind\refstepcounter{para}\thepara.\space
\setlength{\parindent}{\oldparindent}
}



\newcommand{\ncarrNEGZZ}[3][0]{\ncarc[arcangle=#1,nodesepA=2pt,nodesepB=2pt,offsetA=-2pt,offsetB=-2pt,arrowsize=5pt,arrowinset=0.7]{->}{#2}{#3}}
\newcommand{\ncarrZ}[3][0]{\ncarc[arcangle=#1,nodesepA=2pt,nodesepB=2pt,offsetA=0pt,offsetB=0pt,arrowsize=5pt,arrowinset=0.7]{->}{#2}{#3}}
\newcommand{\ncarrZZ}[3][0]{\ncarc[arcangle=#1,nodesepA=2pt,nodesepB=2pt,offsetA=2pt,offsetB=2pt,arrowsize=5pt,arrowinset=0.7]{->}{#2}{#3}}
\newcommand{\ncarrZZZ}[3][0]{\ncarc[arcangle=#1,nodesepA=2pt,nodesepB=2pt,offsetA=4pt,offsetB=4pt,arrowsize=5pt,arrowinset=0.7]{->}{#2}{#3}}
\newcommand{\ncarrZZZZ}[3][0]{\ncarc[arcangle=#1,nodesepA=2pt,nodesepB=2pt,offsetA=6pt,offsetB=6pt,arrowsize=5pt,arrowinset=0.7]{->}{#2}{#3}}


\newcommand{\ccsquareoutline}[6]
{\begin{array}{cp{#1}c}
\Rnode{TL}{#3}  & &  \Rnode{TR}{#4}\\ [#2]
\Rnode{BL}{#5}  & &  \Rnode{BR}{#6}
\end{array}
}
\newcommand{\ccsquareacross}[2]
{\mbox{\ncarr{TL}{TR}
\alabel{#1}
\ncarr{BL}{BR}
\blabel{#2}}
}
\newcommand{\ccsquaredown}[2]
{\mbox{\ncsar{TL}{BL}
\blabel{#1}
\ncsar{TR}{BR}
\alabel{#2}}
}
\newcommand{\ccsquareanddroppers}[6]
{\ccsquareoutline{#1}{#2}{#3}{#4}{#5}{#6}
\ccsquaredown{p_{#3}}{p_{#4}}
}
\usepackage{mathptmx}  % This changes font to roman
\usepackage{anyfontsize}
\usepackage{mathtools}  % why have we got this?
\usepackage{alltt} 
\usepackage{cmll}
\usepackage{ulem}
\renewcommand{\ttdefault}{txtt}
\usepackage[left=1.5cm, right=4cm, marginparwidth=3cm, top=2cm, bottom=2.0cm]{geometry}
\usepackage{framed}
\usepackage[font=small]{caption}
\usepackage{changepage} % used for adjustwidth

\usepackage[hidelinks=true]{hyperref}   % I have become dependent on this get error otherwise... but i don't know why

%\usepackage{enumitem}

\setlength{\captionmargin}{2cm}
\theoremstyle{remark}
\newtheorem*{lemma*}{Lemma}

% have boxed figures
\usepackage{float}
\floatstyle{ruled} 
\restylefloat{figure}
\restylefloat{table}

\usepackage{colortbl}

%\usepackage{mathabox} % wanted this for \lcorners \rcorners but sadly error: mathabox.sty not found

\newtheorem*{lemmastar}{Lemma}

\NewEnviron{tightquote} %italic text indented left and right hand side
{\begin{adjustwidth}{1.5cm}{1.5cm}
\textit{
\BODY
}
\end{adjustwidth}
}

\usepackage{amsmath}
\mathchardef\mhyphen="2D % Define a "math hyphen"

\newcommand{\Ualg}{U \mhyphen alg}
\newcommand{\Upalg}{U' \mhyphen alg}
\newcommand{\Ialg}{I \mhyphen alg}
\newcommand{\catofccs}{\mathbf{Concat}}
\newcommand{\catoflargerccs}{\mathbf{CONCAT}}
%\newcommand{\alg}[1]{{#1}_{alg}}
\newcommand{\alg}[1]{alg(#1)}
\newcommand{\cc}[1]{cc(#1)}
\newcommand{\tccalgebra}{$tcc$-algebra\ }
\newcommand{\tccalgebras}{$tcc$-algebras\ }
\newcommand{\FAM}{\ensuremath{\mathbb{F\mathrm{AM}} }}

\def\therefore{\boldsymbol{\text{ }
\leavevmode
\lower0.4ex\hbox{$\cdot$}
\kern-.5em\raise0.7ex\hbox{$\cdot$}
\kern-0.55em\lower0.4ex\hbox{$\cdot$}
\thinspace\text{ }}}


\NewEnviron{point}
{\begin{adjustwidth}{1.5cm}{1.5cm}
\textbullet\ 
\BODY
\end{adjustwidth}
}

\begin{document}

\title{Instances (Models) of Generalised Algebraic Theories\\
DRAFT }

\author{John Cartmell}

\maketitle

\newcommand{\gatU}{\gat[U]}
\newcommand{\gatUw}{\gatU\ }
\newcommand{\CofU}{\ccat[C](U)}
\newcommand{\KU}{K_{U}}
\newcommand{\KUp}{K_{U'}}
\newcommand{\catCon}{\cat{Con}}
\newcommand{\catGAT}{\cat{GAT}}

\newcommand{\gatrule}[2]{$#1 \tstyle #2$}

\newcommand{\inlinedisplay}[1]
{
\setlength{\fboxsep}{1.5pt}
\setlength{\fboxrule}{0pt}
\fbox{$\displaystyle #1$}
}

\newcommand{\Imappedrule}[2]  {I\left(\gatrawdisplayrule{#1}{#2}\right)}
\newcommand{\sectionsofImappedrule}[2] { Sect \left( \Imappedrule{#1}{#2} \right)}

\newcommand{\Imap}[1]{\setlength{\fboxsep}{1pt}       
\setlength{\fboxrule}{0.2pt}I\left(\mbox{#1}\right)}

\newcommand{\Ipmap}[1]{\setlength{\fboxsep}{1pt}       
\setlength{\fboxrule}{0.2pt}I'\left(\mbox{#1}\right)}

\newcommand{\iI}{\scalebox{1.1}{$\boldsymbol{a}$}}
\newcommand{\Ibar}{\mkern 2.5mu\overline{\mkern-2.5mu \iI\mkern1.5mu}\mkern -1.5mu}

\newcommand{\ibarmappedrule}[2]  {\Ibar\left(\gatrawdisplayrule{#1}{#2}\right)}
\newcommand{\Isort}{\iI_{sort}}
\newcommand{\Iop}{\iI_{op}}

\newcommand{\hatU}{\rule{0pt}{11pt}\widehat {U}}

% workaround when used with Rnode for size of font used under the cross
%XXXXXXXXXXXXXXXXXXXXXXXXXXXXXXXXXXXXXXXXXXXXXXXXXXXXXXXXXXXXXXXXXXXXXXXXXXXX!!!!!!!!!!!!!!!!!!!!!!!!!!!!!!!
\renewcommand{\crossx}[3]{#1 \underset{\tiny #3}{\cross} #2}
\renewcommand{\crossx}[3]{#1 \underset{#3}{\cross} #2}  %Live without for a while to shut up warning messages.
%XXXXXXXXXXXXXXXXXXXXXXXXXXXXXXXXXXXXXXXXXXXXXXXXXXXXXXXXXXXXXXXXXXXXXXXXXXXX!!!!!!!!!!!!!!!!!!!!!!!!!!!!!!!

% macros for closing up * for ease iof readability.
\newcommand{\onestar}   {{_1}\kern-.15em^*}
\newcommand{\twostar}   {{_2}\kern-.15em^*}
\newcommand{\ipstar}    {{_{i-1}}\kern-.2em^*}
\newcommand{\ippstar}    {{_{i-2}}\kern-.2em^*}
\newcommand{\istar}     {{_i}\kern-.2em^*}
\newcommand{\jstar}     {{_j}\kern-.2em^*}
\newcommand{\jpstar}    {{_{j-1}}\kern-.25em^*}
\newcommand{\monestar}{{_{m-1}}\kern-.15em^*}
\newcommand{\mstar}{{_m}\kern-.25em^*}
\newcommand{\nonestar}{{_{n-1}}\kern-.1em^*}
\newcommand{\nstar}{{_n}\kern-.2em^*}
\newcommand{\fonestar}   {f\onestar}             
\newcommand{\ftwostar}   {f\twostar}             
\newcommand{\fippstar}     {f\ippstar}            
\newcommand{\fipstar}     {f\ipstar}
\newcommand{\fistar}     {f\istar}
\newcommand{\fjstar}     {f\jstar}              
\newcommand{\fjpstar}    {f\jpstar}              
\newcommand{\fnstar}     {f\nstar}
\newcommand{\fnonestar}  {f\nonestar} 
\newcommand{\fmonestar}  {f\monestar}        
\newcommand{\fmstar}     {f\mstar}        
\newcommand{\smstar}     {s\mstar}    
\newcommand{\sonestar}   {s\onestar}    
\newcommand{\gonestar}   {g\onestar}
\newcommand{\gprimeonestar}   {g'\onestar}
\newcommand{\gtwostar}   {g\twostar}         
\newcommand{\gjstar}     {g\jstar}  
\newcommand{\gprimejstar}     {g'\jstar}  
\newcommand{\gjpstar}    {g\jpstar}
\newcommand{\gmonestar}  {g\monestar} 
\newcommand{\gmstar}     {g\mstar} 
\newcommand{\gprimemstar}{g'\mstar} 
\newcommand{\gnstar}     {g\nstar}   

\newcommand{\fippvectorstar}{\fippstar...\fonestar}
\newcommand{\fipvectorstar}{\fipstar...\fonestar}
\newcommand{\fivectorstar}{\fistar...\fonestar}
\newcommand{\fnvectorstar}{\fnstar...\fonestar}
\newcommand{\fmvectorstar}{\fmstar...\fonestar}
\newcommand{\gnvectorstar}{\gnstar...\gonestar}
\newcommand{\gmvectorstar}{\gmstar...\gonestar}

\newcommand{\Trule} {T-rule\ }
\newcommand{\trule} {$\in$-rule\ }
\newcommand{\Trules} {T-rules\ }
\newcommand{\trules} {$\in$-rules\ }
\newcommand{\Teqrule} {T=-rule\ }
\newcommand{\teqrule} {$\in=$-rule\ }

\newcommand{\Imapsto}{\scaleto{\mapsto}{8pt}}

\definecolor{lightergrey}{rgb}{0.9,0.9,0.9}
%\gatinterpretationdetail{label}{rulepremise}{ruleconclusion}{mapping}{justification}
\newcommand{\gatinterpretationdetail}[5]{
\refstepcounter{equation}(\theequation)\label{#1}&\gatrule{#2}{#3}&$\mapsto$&&$#4$&#5}

\newcommand{\gatinterpretationdetailcontinuation}[2]{&&&&$#1$&#2}

\newcommand{\gatinterpretationintro}[5]{
\refstepcounter{equation}(\theequation)\label{#1}& \gatrule{#2}{#3}&$\mapsto$&&\cellcolor{lightergrey}$#4$&#5}

%\gatinterpretationmapeqv{equivalentmapping}{justificaton}
\newcommand{\gatinterpretationmapeqv}[2]{&&&=&\ \ $#1$&#2}
\newcommand{\gatinterpretationmapeqvsingle}[1]{&&&=&\multicolumn{2}{l}{\ \ $#1$}}

\newcommand{\gatinterpretationaxcond}[5]{
\refstepcounter{equation}(\theequation)\label{#1}& \gatrule{#2}{#3}&$\scriptstyle iff$&&\cellcolor{lightergrey}$#4$&#5}

\newcommand{\gatinterpretationaxcondrhscontinuation}[2]{ &&&& \cellcolor{lightergrey}\hspace{-0.2
cm} $#1$ &{#2}}

%\gatinterpretationaxeqv{equivalentcondition}{justificatoI(r_{s_j})n}
\newcommand{\gatinterpretationaxeqv}[2]{&&$\scriptstyle iff$&&$#1$&#2}


\newcommand{\bigtuple}[1]{\big \langle #1 \big \rangle}


\newcommand{\highlightpara}[1]{\colorbox{highlight}{%
    \parbox{\dimexpr\linewidth-2\fboxsep}% a box with line-breaks that's just wide enough
        {#1}}
}

\newcommand{\genericcrossxproductdiagram}{
$
\begin{array}{ccccc}
\Rnode{xy}{\crossx{x}{y}{w}} &&               &&               \\[1.3cm]
\Rnode{x}{x}                 &&               && \Rnode{y}{y}  \\[1.3cm]
                             && \Rnode{w}{w}  &&                                                   
\end{array}
$
\makebox[0.2cm]{% technique to avoid throwing white space
\nccdar{xy}{x}
\blabel{p_{\crossx{x}{y}{w},x}}
\nccdar{x}{w}
\blabel{p_{x,w}}
\nccdar{y}{w}
\alabel{p_{y,w}} 
\ncaarr{q(p_{x,w},y)}{xy}{y}
}
}
\iffalse   %     IF FALSE  %%%%%%%%%%%%%%%%%%%%%%%%%%%%%%%%%%%%%%%%%%%%%%%
\begin{abstract}
The notion of 
an \textit{instance of  a generalised algebraic theory $U$ in a contextual category \catc} 
or, equivalently, of an \textit{internal $U$-structure in the contextual category \catc}, is defined. 
From inspection of the given definition it is clear  
that to every generalised algebraic theory $U$ there is generalised algebraic theory $\hatU$ of internal $U$-structures.
The theory $\hatU$ is an extension of the generalised algebraic theory of contextual categories
by a set of rules (introductory rules and axioms) that have  the empty context as premise -- as such it is an extension
by constants and equational identitites between closed terms. Furthermore, every such extension of
the theory of contextual categories arises in this way as a $\hatU$  to some generalised algebraic theory  $U$. 

For $U$ a generalised algebraic theory, a $U$-algebra (previously defined algebraically in \cite{Cartmell78} and published in \cite{Cartmell86} ) can be defined to be any instance of the theory $U$ in the contextual category $\Fam$ of sets,
families of sets, families of families of sets and so on. 

Following the line taken in \cite{BCDEpaper} we study the theory of internal monoids and the theory of internal categories as worked examples. 
The theory of internal monoids is, as to be expected, in agreement with other descriptions such as in the book by Barr and Wells \cite{BarrandWells}. 
\end{abstract}

\section{Introduction}
\subsection{Background}
\note  Metamathematics has well established paradigms for use of the terms
`theory', `signature', `interpretation' and  `model';
in this paper I follow these paradigms
 as well as I am able whilst also substituting the term `instance' for the term `model'\commentary{\highlight{??}}. 

Traditionally, the notion of a `theory', or of a certain class of theories, is  defined syntactically.
For example a metamathematically important class of theories  is the class of `elementary theories' 
which is the class of theories written in first-order predicate logic with equality. 
Such a theory\footnote {As an aside I should mention that some authors use the term `theory' not as we use it here but rather to refer to any set of sentences (of a given signature) that is closed under logical deduction. One such theory in this sense is the total set of true sentences of a model and  
this is called the `diagram of the model'. I have a  memory of a conference in the 1970's and of hearing a German set theorist
 remark that "God doesn't need Logic -- he has the diagram of the universe". He meant, of course, `diagram' in this sense of `set of all true sentences'.
} is defined to consist of a signature (see \cite{HodgesModelTheory}, for example) plus a set of axioms: 
the signature is required to enumerate predicate symbols and functions symbols and assign arities to each, 
the axioms are required  to be a set of closed \term{well formed formulas} (wffs) written
in the language that is defined as the first-order predicate calculus (with equality)
 augmented by predicate symbols and function symbols from the signature subject to use their being used consistently with the arities defined for them. 

\note In this tradition (see \cite{Mendelson}, for example), an `interpretation of an elementary theory' is defined to be a mapping of the symbols defined in the signature 
of the theory to actual predicates and actual functions over some domain that is subject to the requirement that n-ary predicate symbols are mapped to n-ary predicates and n-ary function symbols are mapped to n-ary functions.
As defined by Tarski (Tarski's satisfaction definition), 
such an interpretation induces an interpretation of all
closed well formed formulas of the signature as truth values. 
A `model of a theory' is defined to be an interpretation of the theory such that all axioms of the theory are mapped to true (i.e. are satisfied by the interpretation). 
\commentary{Peter: Note that the term `signature' is only ever used in this sense (apart from in your paper, that is). 
It is never used as a homonym for `theory'.}

\note
Note that the distinction between `signature' and `theory' is significant because it makes possible a definition of `model' 
via a definition of `interpretation': an `interpretation' is an `interpretation of a signature', 
a model is an interpretation in which all axioms are satisfied i.e. have truth value `true'.   
\note
What is meant by an `algebraic theory' prior to Lawvere is a first order theory in a signature having no predicate symbols and in which every axiom is an equational identity between open terms with all variables universally quantified. Thus the notion of `algebraic theory' is  
 a special case of the notion of `elementary theory' and the 
definitions of `interpretation' and `model' for `algebraic theories' are just as given in the broader context of elementary theories but now specialised to the narrower context. Likewise  `Horn theories' and `essentially algebraic theories' are special cases of `elementary theory'. 

\note If pushed, then we need to own up to an  over simplification in this description of the meaning or the class of meanings of the term `theory'.
The term in the precise sense that we have been discussing is often used more loosely. For example for practical purposes there is a single `theory of groups'  but in actual fact at different times the theory might be presented in slightly different ways. Different symbols might be used in one presentation from another. Likewise either of two axiomatisations of the theory might be given -- one in which the inverse to a given element is axiomatised as a left inverse, and one in which the inverse to a given element is axiomatised as a right inverse -- the net effect of either axiomatisation being the same.
So in informal usage `theory' refers to the net effect of a signature plus associated axioms  and if we needed to distinguish the latter, i.e. the signature plus axioms, then we would need refer to as the `presentation of a theory' rather than as the theory itself just as, in group theory, a particular group may be described via a presentation in terms of generators and relations.  Just to be clear, in this paper, by `theory' I will mean in the general sense of signature plus axioms. \commentary{chance to defined generalised signature as a set of symbols along with rules for how they can be used.}

\note
In the traditional view, the domain of interpretation of an algebraic theory may be any set $A$ 
and the interpretation of each n-ary function symbol may be any function $f:A^n \morph A$.
After Lawvere it more usual to consider that an algebraic theory can be interpreted in any category with finite products
 so that $A$ can be any object of the category and  $f$ may be any morphisms $f:A^n \morph A$. 
For $U$ an algebraic theory, the term `model of $U$' therefore has two meanings:
\begin{enumerate}[(i)]
\item model in the most general sense of `model in any category  with finite products',
 such models of $U$ are sometimes said to be internal $U$-objects,
\item model in the traditional sense in which the interpretation is by sets and functions, such models are sometimes said to be algebras or
$U$-algebras. The category of $U$-algebras is denoted $U$-$alg$. 
\end{enumerate} 
Of course, model in sense (ii) is a special case of model in sense (i) --  the case in which the category with finite products is taken to be the category of sets and functions.

\noindent An internal monoid in the category of endofunctors over a category $C$ is precisely a monad. \commentary{\highlight{??}}

\note In addition we might use the term `interpretation' to describe interpretations of the syntax of one theory in terms of the syntax of a second
such as are determined by a mapping of the symbols of one theory into the terms and well formed formulae of another. 
Such an interpretation is valid provided that the axioms of the first are mapped into provable well formed formulae of the second.  Such syntactic interpretations compose and therefore for each class of theory there is a category of theories and interpretations and accordingly such 
interpretations between theories may be said to be theory morphisms.
As an example the two different presentations of the theory of groups mentioned above are different but isomorphic objects in the category of algebrauc theories. 
\subsection{Generalised Algebrauc Theories 	and Contexual Categories}
\note
When we turn to the case of generalised algebraic theories then much of the above carries through except that now 
it is not possible to define an independent notion of `signature of a generalised algebraic theory' and
the approach to defining what a generalised algebraic theory is cannot simply be via a definition of what a signature is followed by a definition 
of a theory  as a signature plus axioms. This is because the rules for introducing symbols need be well-typed and to know that
they are well-typed we already need knowledge of the theory -- in other words the notions of theory and signature are interdependent. 
The  definition of generalised algebraic theory (\cite{Cartmell78},\cite{Cartmell86})  works around this difficult and is by way of 
a definition of pretheory, 
followed by a definition of a theory as a well-typed{\footnote{In this paper I use the term `well-typed' in place of the term `well-formed' defined in \cite{Cartmell78},\cite{Cartmell86}.} pretheory. 

\note The impossibility of predefining the notion of `signature (to be used in a generalised algebraic theory)' has as a consequence
the impossibility of defining a notion of `interpretation of a signature (for a generalised algebraic theory)' prior to 
defining what a `model of a generalised algebraic theory' consists of.  Previously, I avoided this difficulty
in the definition of `model' by stepping over into algebra.
I prove\footnote{
The proof that categories $\catGAT$ and $\catCon$ are equivalent  is entirely trivial but runs to more than 50 pages. I have always interpreted this equivalence as meaning that generalised algebraic theories and contextual categories are more or less the same thing but if this is considered from the point of view of foundations then we have to tread carefully.} the equivalence of generalised algebraic theories and contextual categories: 
\noindent \label{ccgatequivalence}
\begin{point}
there is a category $\catGAT$ of generalised algebraic theories and interpretations,
\end{point}
\begin{point}
there is a category $\catCon$ of contextual categories,
\end{point}
\begin{point}
there is a functor $\ccat[C]: \catGAT \morph \catCon$  (for $U$ a generalised algebraic theory, the category $\ccat[C](U)$ 
has as objects equivalence classes of contexts and realisations, as defined 
in \cite{Cartmell78} and  \cite{Cartmell86}), 
\end{point}
\begin{point}
there is a functor $\gat[U]:\catCon \morph \catGAT$,
\end{point}
\begin{point}
the functor $\ccat[C]$ is an equivalence with inverse $\gat[U]$.
\end{point}
I also describe 
\begin{point}
a contextual category $\Fam$ of sets, families of sets, families of families of sets and so on 
\end{point}
and then am able to go with  algebraic, post-Lawvere
style  definitions of model in either the general or the more restrictive sense : \\
if $U$ is a generalised algebraic theory then
\begin{point}
 a model in the restricted sense of a U-algebra is defined to be a functor $A: \ccat[C](U) \morph Fam$, 
\end{point}
\begin{point}
a model in the more general sense of an internal $U$-structure\footnote{The  term `internal $U$-structure is chosen over `internal $U$-object' because the latter
is no longer appropriate in the broader context in which, generally, theories require complexes of objects for their domain of interpretation.} in a contextual category
$C$ is defined to be a contextual functor $A: \ccat[C](U) \morph C$.
\end{point}
The category of internal $\gat[U]$-structures is defined to be the category whose objects
are pairs $\tuple{\catc,A}$ 
where $\catcw$ is a contextual 
category and $A$ is an internal 
$U$-structure in
 \catcw and whose morphisms between $\tuple{\catc,A}$ and $\tuple{\catc',A'}$ are pairs $\tuple{F, \eta}$ where
$F: \catc \morph \catc'$ is a contextual functor and $\eta: A \circ A \morph  A'$ is a natural transformation.
In otherwords the category of internal $U$-structures
is  the coslice category
$\CofU \downarrow \catCon$. Needless to say this has an initial object which is the identity functor on  $\CofU$.
If\ $\gat[U]$ is a considered a type theory (whatever that is) then this initial object is what I believe Vladimir refers
to as the term model when speaking of the initiality conjecture. 
Specialising to the case of $U$-algebras,  an homomorphism between $U$-algebras $A$ and $A'$ is defined to be a 
natural transformation $\eta: A \morph A'$ and the category of $U$-algebras is a full subcategory of the 
functor category $\Fam^{\CofU}$. 

\note 
Definition of initial $\gat[U]$-algebras. From my thesis:
\begin{tightquote}
Consider for a moment. Every theory $\gat[U]$ has a minimal model denoted $\KU$ built out of the closed terms of \gat[U]. Alternatively this minimal model is described just in terms of the structure $\CofU$. For example
if $1 \base A$ in $\CofU$ then 
$\KU(A)=Hom(1,A)$, otherwise if $1 \base A_1 \base ... \base A_n \base A$ in $\CofU$
then if $a_1 \in \KU(A_1)$, ... if $a_n \in \KU(A_n)(a_1,...a_{n-1})$ then 
$\KU(A)(a_1,...a_n)=\setsuchthat{a\in Hom_{\CofU}(1,A)}{a \circ p_A = a_n}$. \\
\end{tightquote} 

Followed by :
\begin{tightquote}
Now, the free $\gat[U]$-algebras are the algebras $I$-$alg(\KUp)$ for $I: \gat[U] \morph \gat[U']$ an extension of $\gat[U]$ by constants alone. The finitely generated free $\gat[U]$-algebras are those algebras where $\gat[U']$ is an extension by finitely many constants. \\
\end{tightquote}

\subsection{Interpreting Generalised Algebraic Theories}}
\note There is an indirectness in these  algebraic-style definitions of model so that though they are  useful for a 
number of metamathematical purposes they  don't really
help one reason directly about what constitutes a model in the case of any particular generalised algebraic theory in which we might be interested --
although I do think we  have  a good intuition of this.
The definitions given in this paper rectify this indirectness whilst formalising the intuition.
As a bonus, from an examination of these definitions it can be seen, unexpectedly, to me at least, that to every generalised algebraic theory $U$ there is a generalised algebraic theory 
$\hat{U}$ which is the theory of internal $U$-structures\footnote{This is dependent on their being a generalised algebaruc theory of contextual categories and we need to discuss the extent or the manner to which this is true. There are \highlight{nuances} here.}.

\note 
In this paper, in section \ref{sectioninwhichinstanceisdefined}, below, we define the missing definition, so to speak,  of 
an `interpretation of  a generalised algebraic theory $\gat[U]$ in  a contextual category \catc' and say what it is for such an interpretation to be valid. In essence such an interpretation $I$ consists of a \textit{consistent} mapping

\begin{center}
\begin{tabular}{c p{1cm} c}
derived \Trules of $U$           & \raisebox{-0.07cm}{$\Imapsto$} & objects of \catc \\ [0.1cm]
derived \trules of $U$    & \raisebox{-0.07cm}{$\Imapsto$} & sections of \catc \\ [0.1cm]
\end{tabular}
\end{center}
so that derivable equalities in \gatUw map to identical objects, respectively, sections of \catc.
There is a fair amount of detail of what is meant by  `consistent mapping' but what is fundamental is that this detail implies that 
interpetations $I$ of \gatUw in \catcw are completely
determined by their mapping of the introductory rules of \gatU. 
This is the equivalent, in the generalised algebraic case, of 
 the fact that, in regard to algebraic or first-order  theories, interpretations
are determined by a consistent mapping of the symbols within their signatures.

\note
This definition is, then, a traditional style definition  of model in the sense of an internal $\gat[U]$-structure in the contextual category $\catc$:
an internal $U$-structure in $C$ is exactly 
a valid interpretation of $U$ in $C$. This defintion may of course be specialised to a traditional style definition of model in the sense of
$U$-algebra by particularising to the case where $C$ is the contextual category $\Fam$.

\note 
From the details  in
section \ref{sectioninwhichinstanceisdefined} 
of the definition of interpretation
it follows that 
to every gat $\gat[U]$ there is a theory of internal $\gat[U]$-structures. We shall denote this theory as $\hatU$.

Every such theory $\hatU$ is an extension of the generalised algebraic theory of contextual categories
by a set of rules (introductory rules and axioms) that have  the empty context as premise -- as such it is an extension
by constants and equational identities between closed terms -- and, vice-versa, every such extension of
the theory of contextual categories can be interpreted as being a specification of a generalised algebraic theory.  \commentary{Peter: your paper uses this.}

\begin{notebox}[Question]
The above observation would allow someone automating reasoning within  a generalised algebraic theory 
$\gatU$ to reason about $\hatU$ instead. Might this be advantageous? 
\end{notebox}

\note 
The instances of $\hatU$  in $\Fam$ consist of  internal $\gat[U]$-structures  i.e. they consist of contextual categories \catcw along with particular instances $I$ of
the theory $\gat[U]$ in the contextual category \catc. \\
The category of $\hatU$-algebras is (isomorphic to) the category of internal $\gatU$-structures.
\note
An instance of $\hatU$ in an arbitrary contextual category
consists of  an internal internal $\gat[U]$-structure. This sounds a bit crazy but it isn't -- there are after all categories internal to other categories and it isn't much of a stretch to suppose these internal categories have internal $\gat[U]$'s inside of them. 



\note
\label{termmodelEQfreealgebra}For any generalised algebraic theory $\gat[U]$ we have two different 
and therefore isomorphic\commentary{\highlight{isomorphic} in what category??} descriptions of the initial object of the category of internal $\gatU$-structures:\\
\begin{equation}
K_{\hat{U}} \cong \bigtuple{\CofU, I_{triv}}
\end{equation}

where $I_{triv}$ is the trivial instance of $\gatU$ in $\CofU$.

\begin{notebox}[Question]
Is this observation relevant
to the initiality conjecture or to formal (machine checked) theory?  It can be summarised 
by saying that the term model of a theory $\gatU$ is the initial algebra of a theory $\hatU$.
Is that useful? Another way of looking at it (is it helpful?) is that the open terms and types
of $\gatU$ correspond to the closed terms of $\hatU$. 
Is use of the word combinator appropriate here?
\end{notebox}

\note 
Note that if $\gat[U]$ is a single-sorted or many-sorted algebraic theory then 
$\hatU$ is generalised algebraic 
and so solely within these regimes there is no equivalent of the situation described para \ref{termmodelEQfreealgebra}.

\subsection{Outline}
\note As a worked example, we show in lemma \ref{internalmonoidlemma}, below, that 
the generalised algebraic theory of internal monoids can be expressed  as 
the theory of contextual categories plus:

\begin{gatrules}
\gatintros
\gatintroducing{M}
\ofT{M}{Ob} \\
\gatintroducing{unit}
\ofT{unit}{Hom(1,M)} \\
\gatintroducing{mult}
\ofT{mult}{Hom(M \times M,M)} \\
\gataxioms
\gatintroducing{ \gataxiomno{1} }
\tuple{p_M \circ unit,id_M} \circ mult =id_M \\
\gatintroducing{ \gataxiomno{2} }
\tuple{id_M,p_M \circ unit} \circ mult =id_M \\
\gatintroducing{ \gataxiomno{3} }
(mult \times id_M) \circ mult = (id_M \times mult) \circ mult
\end{gatrules}

This is in agreement with Barr and Wells \cite{BarrandWells}, page 232, where they describe
monoids internal to  a category\footnote{Generally we would be thinking of a category with finite products including terminal object here, though, as they say, not all products need be available in the category for there to be an internal monoid.}
as an example of a finite product (FP) sketch.

\note As a second worked example  we derive the generalised algebraic theory of internal categories in lemma \ref{internalcategorylemma}.


%\fi
\section{Background regarding Generalised Algebraic Theories}
\note
We give the broadest possible notion of a model
of a generalised algebraic theory \gat[U] by defining the notion of an interpretation of  \gat[U] in  any given contextual category \catc.

\note
First, as a stepping stone, we need  give an auxillary definition, the definition of  a `preinterpretation of \gat[U] in \catc'.

\newcommand{\Isort}{I_{sort}}
\newcommand{\Iop}{I_{op}}
\note 
If \gat[U] is a generalised algebraic theory  and if \catcw is a contextual category then
a preinterpretation $I$ of  \gat[U] in \catcw consists of a pair :
\begin{itemize}
\item a mapping $\Isort$ that maps each sort symbol of \gat[U] to  an object of \catc,
\item a mapping $\Iop$ that maps each operator symbol of \gat[U] to a section of \catcw (i.e. to a morphism $f: A \morph B$ for some 
$A \base B$ in \catcw such that $f \circ p_B=id_A$).
\end{itemize}

\note 
An interpretation $I$ of \gat[U] in \catcw is a preinterpretation satisfying additional conditions and these 
additional conditions imply that the preinterpretation induces a mapping $\hat{I}$
of derived T- and $\epsilon$- rules of \gat[U] to objects, respectively sections, of \gat[U] such that 
all equational identities of \gat[U] are respected in the sense that :
\begin{itemize}
\item whenever 
$\frac{\context{x}{\Delta}{n}}{t = t' \in \Delta}$
is an axiom/derived rule of \gat[U] then $\hat{I}(R) = \hat{I}(R')$
where $R$ is the rule
$\frac{\context{x}{\Delta}{n}}{\ofT{t}{\Delta}}$
and $R'$ is the rule
$\frac{\context{x}{\Delta}{n}}{\ofT{t'}{\Delta}}$,
\item whenever
$\frac{\context{x}{\Delta}{n}}{\Delta = \Delta'}$
is an axiom/derived rule of \gat[U] then $\hat{I}(R) = \hat{I}(R')$
where $R$ is the rule
$\frac{\context{x}{\Delta}{n}}{\isT{\Delta}}$
and $R'$ is the rule
$\frac{\context{x}{\Delta}{n}}{\isT{\Delta}}$.
\end{itemize}
%\iffalse
%\fi
\section{Contextual Categories }

\label{contextualnotationpartone}



\note 
Terminology: By  the generic term \term{tree} is meant a partially ordered set (poset) $(T, <)$ such that for each $t \in T$, the set $\set{s \in T : s < t}$ is well-ordered by the relation $<$.
In this discussion we restrict ourselves to rooted $\omega$-trees i.e. trees for which the set $\set{s \in T : s < t}$
is finite for all $t \in T$ and for which there is a least element in the partial ordering. 

With respect to a partial ordering $<$, we say that an element $y$ \textit{covers}  an element $x$ in  iff $x<y$ and there does not exist $w$ such that $x < w$ and $w < y$.
If object $y$ covers object $x$ in the partial ordering 
then we write $x \base y$ (we use this in preference to the more usual $x \lessdot y$).


\note We define the rank (sometimes called the grade) of an element $t \in T$ to be the cardinality
of the set $\setsuchthat{s \in T}{s < t}$. If we define the set $T_i$ to be the set of elements of a tree
of rank $i$ then we have that $T= \bigcup_{i \in N}T_i$. 

\note In the  definition of contextual categories (\cite{Cartmell78,Cartmell86}) there is defined to be such a tree-structure on the objects of the category. In a contextual category the root of the tree of objects is also  terminal object $1$
of the category. For $x$ an object of the category we define the set of objects  $Cover(x)$ to be the set of objects covering $x$.

\note
By a \term{tree-structured category}\footnote{Is this what Jacopo refered to as a `stratified category'? If so does this term appear in the lierature somewhere. Why is that adjective used in preference to a choice of either `ranked' or `graded'?} we mean (i) a category with a tree-structure defined on its objects such that the tree of objects has a unique root object and (ii) for every $x \base y$ in the tree of objects  a canonical morphism $p_y:y \rightarrow x$.  I shall shall say morphisms of this form  are \term{direct dependency morphisms} and they will
be distinguished in diagrams by an arrow with  a triangular head so:
\begin{center}
$
\begin{array}{p{2cm}}
\Rnode{y}{y}\\ [1.4cm]
\Rnode{x}{x} \\
\mbox{\ncbsar{p_y}{y}{x}}
\end{array}
$
\end{center}

\note
If $x$ is an object of a tree-structured category \catcw and if $y \in Cover(x)$ in \catcw then we define 
the set  of sections of $y$, denoted $Sect(y)$, to be the set of morphisms $s: x \morph y$ in \catc  such that $s \circ p_y = id_x$. So that if $x$ is a section of $y$ and $y$ covers $x$ then
we have\ \ \ 
\begin{tabular}{cccc}
$
\begin{array}{p{2cm}}
\Rnode{y}{y} \\ [1.4cm]
\Rnode{x}{x} \\
\mbox{
\ncsar{y}{x}
\alabel{p_y}
\ncarrZZ[30]{x}{y} 
\alabel{s}}
\end{array}
$  & so that &
$
\begin{array}{c p{0.5cm}c p{0.5cm}c}
              && \Rnode{y}{y}&&                \\ [1.4cm]
\Rnode{x1}{x} &&             &&   \Rnode{x2}{x}\\
\mbox{
\ncsar{y}{x2}
\alabel{p_y}
\ncarr{x1}{y} 
\alabel{s}
\ncarr{x1}{x2} 
\blabel{id_x}
}
\end{array}
$& commutes
\end{tabular}

\note
The original definition of contextual category given  in [1] and [2], a contextual category is defined to be a tree-structured category 
\cat{C} with the following additional structure:

\noindent 
(i) whenever
$
\begin{array}{cp{.9cm}c}
            & & \Rnode{z}{z} \\ [1.2cm]
\Rnode{x}{x}& & \Rnode{y}{y} \\ [0.5cm]
\end{array}
$
\jcbarr{f}{x}{y}
\ncasar{p_z}{z}{y}

in \cat{C}, an object $f \sub z$ such that $x \base f \sub z$, a morphism $q(f,z): f \sub z \rightarrow z$ such that

\begin{axiom}{q1}
q(f,z) \circ p_z = p_{f \sub z} \circ f
\end{axiom}

i.e. such that the diagram: 
$$
\ccsquareoutline{0.9cm}{1.2cm}{f^*z}{z}{x}{y}
\ccsquareacross{q(f,z)}{f}
\ccsquaredown{p_{f \sub z}}{p_z}
$$
commutes, 

\noindent
and, (ii), so that each such diagram is a pullback diagram, that is: for all objects $w$ of \cat{C}, and for all
morphisms $h_1: w \rightarrow x$ and $h_2: w \rightarrow z$ (see diagram \ref{pullback} below) such that
$h_1 \circ f = h_2 \circ p_z$ 
there exists a unique $h:w \rightarrow f \sub z$ in \cat{C} such that
$h \circ p_{f \sub z} = h_1$ and $h \circ q(f,z) = h_2$, as shown here:

\vspace{3mm}
\begin{center}
\begin{equation}
\label{pullback}
\begin{array}{cp{0.5cm}cp{1.2cm}c}
\Rnode{w}{w} &&                     &&           \\ [0.7cm]
             &&\Rnode{fstarz}{f^*z} && \Rnode{z}{z}\\ [1.2cm]
             &&\Rnode{x}{x}         && \Rnode{y}{y}
\end{array}
\end{equation}
\ncbsar{p_{f \sub z}}{fstarz}{x}
\jcbarr{f}{x}{y}
\ncaarr{q(f,z)}{fstarz}{z}
\ncasar{p_z}{z}{y}
\setlength{\arrnodesepA}{3pt}
\jcbarr[-35]{h_1}{w}{x}
\ncaarr[35]{h_2}{w}{z}
\psset{linestyle=dashed}
\ncaarr{h}{w}{fstarz}
\end{center}

\vspace {0.25cm}
\noindent and so that (iii) whenever $x \base y$ in \cat{C}, 
\begin{axiom}{q2}
id_x^*y=y
\end{axiom}

and

\begin{axiom}{q3}
q(id_x,y) = id_y
\end{axiom}



\noindent and (iv) whenever 
$
\begin{array}{c p{.9cm} c p{.9cm} c}
             &   &             &   & \Rnode{z}{z} \\ [1.2cm]
\Rnode{w}{w} &   &\Rnode{x}{x} &   & \Rnode{y}{y} \\ [0.5cm]
\end{array}
$
\jcbarr{f}{w}{x}
\jcbarr{g}{x}{y}
\ncasar{c}{z}{y}
in \cat{C}, 

then

\begin{axiom}{q4}
(f \circ g)^*z =  f^* (g ^* z)
\end{axiom}

and 
\begin{axiom}{q5}
q(f \circ g,z) = q(f,g^*z) \circ q(g,z)
\end{axiom}



\note
Following Voevodsky we may replace the pullback condition of the original definition by an 
`s' operator along with axioms as follows:

\noindent (ii') for all morphisms $f: x \rightarrow y$, a morphism $s(f) : x \rightarrow f \sub p_y \sub y$ such that both:

\begin{axiom}{s1}
s(f) \circ p_{f\sub p_y \sub y}=id_x
\end{axiom}

\noindent and

\begin{axiom}{s2}
s(f) \circ q( f \circ p_y     ,y)=f
\end{axiom}	

\noindent i.e. such that the following diagrams commute:
\begin{center}
\begin{displaymath}
\begin{array}{cccp{1.cm} cp{.9cm}c}
&\Rnode{fXyyM}{f\sub p_y \sub y}&  & &  \Rnode{fXyy}{f\sub p_y \sub y} & & \Rnode{yXy}{p_y \sub y}\\ [1.2cm]
\Rnode{xL}{x} & &\Rnode{xR}{x} & &\Rnode{x}{x}         & & \Rnode{y}{y}
\end{array}
\end{displaymath}
\ncasar{p_{f\sub p_y \sub y}}{fXyy}{x}
\jcbarr{f}{x}{y}
\ncaarr{q(f,p_y \sub y)}{fXyy}{yXy}
\ncasar{p_{p_y \sub y}}{yXy}{y}
\ncaarr{s(f)}{xL}{fXyyM}
\ncasar{p_{f\sub p_y \sub y}}{fXyyM}{xR}
\jcbarr{id_x}{xL}{xR}
\end{center}

\noindent
and such that whenever

\begin{center}
\begin{displaymath}
\begin{array}{c p{.9cm} c p{.9cm} c}
\Rnode{w}{w}&& \Rnode{g*z}{g \sub z} && \Rnode{z}{z} \\ [1.2cm]
            && \Rnode{x}{x}  && \Rnode{y}{y} \\ [0.2cm]
\end{array}
\end{displaymath}
\jcbarr{f}{w}{g*z}
\jcbarr{g}{x}{y}
\ncaarr{q(g,z)}{g*z}{z}
\ncasar{}{g*z}{x}
\ncasar{}{z}{y}
\end{center}

\noindent in \cat{C} then

\begin{axiom}{s3}
s(f \circ q(g,z))=s(f)
\end{axiom}

\iffalse
\begin{oldtt}
\note
Now consider the pullbacks in the original definition of contextual categories.
According to the definition whenever
$
\begin{array}{cp{.9cm}c}
            & & \Rnode{z}{z} \\ [1.2cm]
\Rnode{x}{x}& & \Rnode{y}{y} \\ [0.5cm]
\mbox{\jcbarr{f}{x}{y}
\ncasar{p_z}{z}{y}}
\end{array}
$
in \catcw then there is a pullback diagram: \ \ 
$
\ccsquareoutline{0.9cm}{1.2cm}{f^*z}{z}{x}{y}
\ccsquareacross{q(f,z)}{f}
\ccsquaredown{p_{f \sub z}}{p_z}
$
in \catcw i.e. such objects and morphisms so that for all objects $w$ of \catc, and for all
morphisms $h_1: w \rightarrow x$ and $h_2: w \rightarrow z$  such that
$h_1 \circ f = h_2 \circ p_z$ 
there exists a unique $h:w \rightarrow f \sub z$ in \catcw such that
$h \circ p_{f \sub z} = h_1$ and $h \circ q(f,z) = h_2$, as shown is diagram (\ref{pullback}).

After reading Peter Dybjer's axioms for categories with families (\cite{dybjer96}) I have been wondering whether I might not use the notation $\tuple{h_1,h_2}_{x,y,f,z}$  for the unique morphism   
$h:w \rightarrow f \sub z$ in \cat{C} such that
$h \circ p_{f \sub z} = h_1$ and $h \circ q(f,z) = h_2$.\\

$\tuple{h_1,h_2}_{x,y,f,z}$ may be be safely elided to $\tuple{h_1,h_2}_{f,z}$ and, rather less safely, to $\tuple{h_1,h_2}$.
In fully elided form we then  have\footnote{Strikingly similar in appearance to Dybjer's axioms but, I think, not like for like i.e. not inter-translatable.} 
\begin{equation}
\mbox{$\tuple{h_1,h_2} \circ p_{f^*z} = h_1$}
\end{equation}
and
\begin{equation}
\mbox{$\tuple{h_1,h_2} \circ q(f,z) = h_2$}
\end{equation}

I will use this notation in the detailed examples that follow.
\end{oldtt}
\fi

\note I use several other notational conveniences when working in contextual categories. 
If $x < y$ in the contextual category \catc, then define the morphism $p_{y,x}:y \morph  x$ in \catc, \\

\begin{tabular}{c c c  c  c  c c}
by defining
& %2 c
$
\begin{array} {c}
\Rnode{midy}{y} \\[2.0cm]
\Rnode{midx}{x}  \\ 
\end{array}
\mbox{\ncarr{midy}{midx}
      \blabel{p_{y,x}}[0.2]
		 }
$
& %3 c
(drawn also  as
& %4 c
$
\begin{array} {c}
\Rnode{lhsy}{y} \\[2.0cm]
\Rnode{lhsx}{x} 
\end{array})
\makebox[0.1cm]{\nccdar{lhsy}{lhsx}
      \blabel{p_{y,x}}[0.275]
		}
$
& %5
 as the composition 
& %6 c
$
\begin{array}{c}
%\Rnode{b}{B}&&\Rnode{xn}{w_n}&&\Rnode{xn1}{w_{n-1}}&&\Rnode{dots}{\ ...\ }&&\Rnode{x1}{w_1}&&\Rnode{a}{A} 
\Rnode{b}{y}\\[0.7cm]
\Rnode{xn}{w_n}\\[0.7cm]
\Rnode{xn1}{w_{n-1}}\\[0.1cm]
\Rnode{dots}{\vdots}\\[0.1cm]
\Rnode{x1}{w_1}\\[0.7cm]
\Rnode{a}{x} 
\end{array}
,
\makebox[0.1cm]{
\ncsar{b}{xn}
\alabel{p_y}
\ncsar{xn}{xn1}
\alabel{p_{w_n}}
\ncsar{xn1}{e1}
\ncline[linestyle=dotted,dotsep=4pt]{e1}{e2}
\ncsar{e2}{x1}
\ncsar{x1}{a}
\alabel{p_{w_1}}}
$ 
& %7 c
,
\end{tabular}

where
$w_1, ... w_n$ is the unique sequence of objects of $C$ such that 
$x \base w_1 \base ... \base w_n \base y$. If $x = y$, then define $p(y, x) = id_x$.
We say that the morphism  $p_{y,x}$ is a dependency morphism. 


\note
The contextual category structure supplies us with pullbacks for any $\smorph$ morphism, 
these given pullbacks can be pieced together to obtain a pullback for
any $\dmorph$ morphism  along any morphism with the same codomain. 

In general, whenever $y \leq z$ in $C$ and whenever $f:x \morph y$ in $C$, then we have
the following canonical pullback for the morphism $p_{z, y}$ along $f$, where
$w_1, ... w_n$ is the unique sequence of objects of $C$ such that 
$y \base w_1 \base ... \base w_n \base z$:

\vspace{3mm}
\begin{center}
\begin{equation}
\label{compositepullbackdefinition}
\begin{array}{cp{2.9cm}c}
\Rnode{TOPL}{q(...q(f, w_1)...w_n)^* z} & & \Rnode{TOPR}{z}\\ [1.2cm]
\Rnode{zOTTOML}{x}         & & \Rnode{zOTTOMR}{y}
\end{array}
\end{equation}
\jcbarr{f}{zOTTOML}{zOTTOMR}
\ncaarr{q(q(...q(f,w_1)...w_n),z)}{TOPL}{TOPR}
\nccdar{TOPL}{zOTTOML}
\blabel{p_{q(...q(f, w_1)...w_n)^* z,x}}
\nccdar{TOPR}{zOTTOMR}
\alabel{p_{z,y}}
\end{center}

Since these constructed pullbacks form an important part of contextual
category structure we would like a simpler notation for them. As no confusion is
likely, we extend the $^*$ and $q$ notation to cover these new pullback diagrams.
From now on if $f:x \morph y$ in $C$ and $y \leq z$ in $C$, then the diagram

\vspace{3mm}
\begin{center}
\begin{equation}
\label{compositepullbackout}
\begin{array}{cp{.9cm}c}
\Rnode{fstarz}{f^*z} & & \Rnode{z}{z}\\ [1.2cm]
\Rnode{x}{x}         & & \Rnode{y}{y}
\end{array}
\end{equation}
\nccdar{fstarz}{x}
\blabel{p_{f \sub z},x}
\jcbarr{f}{x}{y}
\ncaarr{q(f,z)}{fstarz}{z}
\nccdar{z}{y}
\alabel{p_{z,y}}
\end{center}
is the canonical pullback diagram as defined in (\ref{compositepullbackdefinition}) above. 
\note
The following observation
follows from the way the extended pullback diagrams are constructed. 
In the extended notation, if $f: x \morph y$ 
and $y \leq z \leq zz$ in the contextual category C, then
\begin{equation}
f^*zz = q(f, z)^*zz
\end{equation}
 and 
\begin{equation}
q(f, zz) = q(q(f, z), zz)
\end{equation}
and so the outer diagram in
\renewcommand{\pc}[2]{p_{#1,#2}}  % as \pc defined in ccategories macros differently to this
$
\begin{array}{ccp{.9cm}c}
\\[0.25cm]
&\Rnode{TL}{q(f,z)^*zz} & & \Rnode{TR}{zz}\\ [1.2cm]
&\Rnode{ML}{f^*z} & & \Rnode{MR}{z}\\ [1.2cm]
&\Rnode{BL}{x}         & & \Rnode{BR}{y} \\[1.0cm]
\end{array}
$
%composition
\makebox[0.2cm]{   % This make box prevents white space pushing out to the right
                   % cannot see where this white space is comin from. To investigate
									 % change the \makebox[0.2cm] to \fbox and you will see the problem.
\nccdar{TL}{ML}\blabel{P_{q(f,z)^*zz,f^*z}}\nccdar{ML}{BL}\blabel{p_{f \sub z,x}}\nccdar{TR}{MR}\alabel{p_z}
\nccdar{MR}{BR}
}
\alabel{p_z}
%reference
\ncarr{TL}{TR}
\alabel{q(q(f,z),zz)}
\ncarr{ML}{MR}
\alabel{q(f,z)}
\ncarr{BL}{BR}
\blabel{f}
is diagram (\ref{compositepullbackout}). 
On such diagrams as shown here the $p$ labels on $\dmorph$ morphisms (inclusive of their indices) are entirely predictable  and so in diagrams that follows we may omit them.


\note
If we write $\crossx{y}{z}{x}$ in place of ${p_{y,x}}^*z$, for $x < y$, $x < z$  in \ccat then
$\crossx{y}{z}{x}$  represents  in the syntax the `weakening' of a rule of the form
\begin{displaymath}
x, w1,...w_n \tstyle \isT{z}
\end{displaymath}
from a rule with context $x, w_1,...w_n$ to a rule with broader context $y, w_1, ... w_n$: 
\begin{displaymath} 
y, w_1,...w_n \tstyle \isT{z}.
\end{displaymath}

Within the contextual category I think of $\crossx{y}{z}{x}$  as a local cartesian product but of course categorically it is a filtered product i.e. a pullback. If $w < x$ and $w < y$  then 
\genericcrossxproductdiagram % defined in 'paper.tex'
is a pullback diagram in \ccat.

\note
We can extend the $\crossx{}{}{w}$ notation to morphisms. If $f:x \morph x'$ and $g: y \morph y'$ in a contextual
category $\ccat[C]$ and if $w$ is an object such that $w < x$, $w <x'$, $w < y$ and $w < y'$ then 
define $\crossx{f}{g}{w}:\crossx{x}{y}{w} \morph \crossx{x}{y}{w}$ in $\ccat[C]$ by
\begin{equation}
\crossx{f}{g}{w} = \tuple{p_{\crossx{x}{y}{w}, x} \circ f,q(p_{x,w},y) \circ g}
\end{equation}  

\note
In the case special case that $w$ is the terminal object $1$ then the pullback  specialises to give a product diagram:

\begin{displaymath}
\begin{array}{ccccc}
\Rnode{xy}{\crossx{x}{y}{1}} &&               &&               \\[1.3cm]
\Rnode{x}{x}                 &&               && \Rnode{y}{y}  \\                                    
\end{array}
\mbox{\ncsar{xy}{x}
\blabel{p_{\crossx{x}{y}{1},x}}
\ncaarr{q(p_{x,1},y)}{xy}{y}}
\end{displaymath}

In this special case the $\tuple{}$ operation defined earlier is the pairing operation for if
$f: w \morph x$ and $g: \morph y$ then $\tuple{f,g}: w \morph \crossx{x}{y}{1}$ 
and 
\begin{equation}
\tuple{f,g} \circ p_{\crossx{x}{y}{1},x} = f
\end{equation}
and
\begin{equation}
\tuple{f,g} \circ q(p_{x,1},y) = g
\end{equation}

\note 
Note that the product operation $\crossx{}{}{1}$ is far from symmetric 
because if, for example, $1 \base x$ and $1 \base y$ then $x \base \crossx{x}{y}{1}$ and $y \base \crossx{y}{x}{1}$ but we can define 
a swap operation $sw_{x,y} : \crossx{x}{y}{1} \morph \crossx{y}{x}{1}$ by
\begin{equation}
sw_{x,y} = \tuple{q_{p_x,y}, p_{\crossx{x}{y}{1},x}}
\end{equation}

\note
Associativity of $\crossx{}{}w$  follows from the coherence property of the pullbacks in the contextual category. 
For example if $w < x$, $w < y$, $w < z$ in a \ccat then from coherence of pullbacks in \ccat we have:
$\crossx{x}{(\crossx{y}{z}{w})}{w} = \crossx{(\crossx{x}{y}{w})}{z}{w}$ as shown here in this diagram:
 
\begin{displaymath}
\begin{array}{cp{1.0cm}cp{1.0cm}c}
\Rnode{J1}{}\Rnode{D1} {\crossx{(\crossx{x}{y}{w})}{z}{w}}\Rnode{J2}{} \ \ \ \ \   &&  &&  \\ 
= && && \\
\Rnode{D2} {\crossx{x}{(\crossx{y}{z}{w})}{w}}    &&  &&                        \\ [1.3cm]
\Rnode{xy}{\crossx{x}{y}{w}}&& \Rnode{yz}{\crossx{y}{z}{w}} &&                      \\[1.3cm]
\Rnode{x}{x}&& \Rnode{y}{y} && \ \ \ \ \ \ \ \ \ \ \ \ \ \Rnode{z}{z}                                        \\[1.3cm]
             && \Rnode{w}{w} &&                                                     
\end{array}
\end{displaymath}

\ncaarr[50]{q(\pc{\crossx{x}{y}{w}}{w},z)}{J2}{z}
\ncsar{D2}{xy}
\ncsar{xy}{x}
\ncsar{yz}{y}
\ncsar{x}{w}
\ncsar{y}{w} 
\ncsar{z}{w}
\ncaarr{q(\pc{x}{w},y)}{xy}{y}
\ncaarr{q(\pc{y}{w},z)}{yz}{z}
\ncaarr{q(\pc{x}{w},\crossx{y}{z}{w})}{D2}{yz}



\note A number of minor lemmas:

\begin{lemma}
\label{footandstactic}
If $f: A \morph B$ and $f':A \morph B$ in a contextual category \catcw then if 
$f \circ p_B$ = $f' \circ p_B$ and $s(f) = s(f')$ then $f=f'$.
\end{lemma}
\begin{proof}
Follows by axiom (s2) since we have:
$f = s(f) \circ q(f \circ p_B,B)  = s(f') \circ q(f' \circ p_B,B) = f'$.
\end{proof}

From which follows:
\begin{lemma}
\label{stactic}
If $A$ is any object of a contextual category \catcw and if $B$ is an object such that $1 \base B$ in \catcw then
if $f: A \morph B$ and $f':A \morph B$ in a contextual category \catcw then $f=f'$ iff $s(f) = s(f')$.
\end{lemma}

\begin{lemma}
\label{crosssectionlemma}
If 
%\begin{equation*}
$
\begin{array}{ c c c}
\Rnode{B}{B} &              & \Rnode{Bp}{B'} \\[1cm]
             & \Rnode{A}{A} &     
\mbox{\ncsar{B}{A}
\ncsar{Bp}{A}
%\ncarr[-30]{A}{Bp}
\ncrightsimplesection{A}{Bp}
\blabel{g}}
\end{array}
$
%\end{equation*}
in a contextual category \catcw and if $g$ is a section of $B'$ (i.e. if $g \circ p_{B'}= id_A$) so that we have 
\begin{equation*}
\begin{array}{ c c c}
\Rnode{BBp}{\crossx{B}{B'}{A}} \\[1.3cm]
\Rnode{B}{B} &              & \Rnode{Bp}{B'} \\[1.1cm]
             & \Rnode{A}{A} &
\mbox{
\ncsar{BBp}{B}
\ncrightcrosssection{B}{BBp}
\blabel{\crossx{B}{g}{A}}
\ncsar{B}{A}
\blabel{p_B}
\ncsar{Bp}{A}
\ncrightsimplesection{A}{Bp}
\blabel{g}
}														
\end{array}
\end{equation*}
in \catcw,  then
\begin{equation}
\label{crosssectionlemmatarget}
\crossx{B}{g}{A} = s(p_B \circ g).
\end{equation} 
\end{lemma}
\begin{proof}
$\crossx{B}{g}{A}$ is defined to be the unique section of $\crossx{B}{B'}{A}$ such that $(\crossx{B}{g}{A}) \circ q( p_B,B') = p_B \circ g$.

(\ref{crosssectionlemmatarget}) follows because $s(p_B \circ g)$ is also such a section, since it is defined to be the unique section of $(p_B \circ g \circ p_{B'}) ^* B'$
such that $s(p_B \circ g) \circ q( p_B \circ g \circ p_{B'}, B') = p_B \circ g$ and this simplifies,
because $g \circ p_{B'} =id_A$, to
 $s(p_B \circ g)$ being a section of ${p_B} ^* B'$ (i.e of $\crossx{B}{B'}{A}$) satisfying $s(p_B \circ g) \circ q( p_B,B') = p_B \circ g$. 
\end{proof}

% *************************************************************************************
% sfglemma ****************************************************************************
\begin{lemma}
\label{sfglemma}
If $f:A \morph B$ and $g:B\morph C$ in a contextual category \catcw and if $C \in Cover(B)$ then
\begin{equation}
\label{sgflemmagoalone}
ft(f\circ g)^*C = f^*(ft(g)^*C)
\end{equation}
and 
\begin{equation}
\label{sgflemmagoaltwo}
s(f\circ g)=f^*s(g)
\end{equation}
where $ft(g) = g \circ p_B$ so that we have
\begin{displaymath}
\begin{array}{ccp{1.7cm}cp{1.7cm}c}
ft(f\circ g)^*C=\kern-10pt&\Rnode{TL}{f^*(ft(g)^*C)} & & \Rnode{TC}{ft(g)^*C}          \\ [1.7cm]
&\Rnode{BL}{A}         & & \Rnode{BC}{B} && \Rnode{BR}{C}
\end{array}
\mbox{
\ncsar{TL}{BL}
\ncsar{TC}{BC}
\ncarr{BL}{BC}
\blabel{f}
\ncarr{BC}{BR}
\blabel{g}
\ncarr{TL}{TC}
\alabel{q(f,ft(g)^*C)}
\ncarr{TC}{BR}
\alabel{q(ft(g),C)}
\ncleftsimplesection{BL}{TL}
\alabel{s(f\circ g)=f^*s(g)}
\ncleftsimplesection{BC}{TC}
\alabel{s(g)}
}
\end{displaymath}
in \catc.
\end{lemma}
\begin{proof}
That (\ref{sgflemmagoalone}) holds follows by axiom (q4).

To show that (\ref{sgflemmagoaltwo}) holds remember that $s(f \circ g)$is the unique section of $ft(f \circ g)*C$
such that $s(f \circ g) \circ q(f \circ g \circ p_C,C) = f\circ g$. Therefore it suffices to show that
$f^*s(g) \circ q(f \circ g \circ p_C,C) = f\circ g$ and this we can show as follows:
\begin{align*}
(f^*s(g)) \circ q(f \circ g \circ p_C,C) &= (f^*s(g)) \circ q(f ,(g \circ p_C)^*C) \circ q(g \circ p_C ,C) &&\mbox{by axiom (qf)} \\
                             &= f \circ s(g) \circ q(g \circ p_C ,C)                   &&\mbox {from defn. of $f^*s(g)$}\\
														 &= f \circ g                                              &&\mbox {by axiom (s2)}
\end{align*}
\end{proof}

\iffalse
\newcommand{\duplesone}{{\duple{s_1}_{B_1}}}
\newcommand{\duplestwo}{{\duple{s_1,s_2}_{B_2}}}
\newcommand{\duplesn}{\duple{s_1,...s_n}_{B_n}}
\newcommand{\duplesi}{{\duple{s_1,...s_i}_{B_i}}}
\newcommand{\duplesilessone}{\duple{s_1,...s_{i-1}}_{B_{i-1}}}
\newcommand{\duplesj}{{\duple{s_1,...s_j}_{B_j}}}
\newcommand{\duplesjlessone}{\duple{s_1,...s_{j-1}}_{B_{j-1}}}
\newcommand{\duplesisucc}{{\duple{s_1,...s_{i+1}}_{B_{i+1}}}}
\newcommand{\duplesnlessone}{{\duple{s_1,...s_{n-1}}_{B_{n-1}}}}

\newcommand {\sonesub}{{s_1}^*}
\newcommand {\stwosub}{{s_2}^*}
\newcommand {\stwocascade}{\stwosub\sonesub}
\newcommand {\sisub}{{s_i}^*}
\newcommand {\sicascade}{\sisub...\sonesub}
\newcommand {\sisuccsub}{{s_{i+1}}^*}
\newcommand {\sisucccascade}{\sisuccsub...\sonesub}
\newcommand {\snlessonesub}{{s_{n-1}}^*}
\newcommand {\snlessonecascade}{\snlessonesub...\sonesub}
\newcommand {\snsub}{{s_n}^*}
\newcommand {\sncascade}{\snsub...\sonesub}

\note If $A$ is an object of contextual category \catc, if $1 \base B_1 ... \base B_n$ in \catcw and if
\begin{equation*}
\begin{array}{l}
s_1 \in Sect(\crossx{A}{B_1}{1}),                  \\
s_2 \in Sect(\sonesub (\crossx{A}{B_2}{1})),         \\
s_3 \in Sect(\stwocascade (\crossx{A}{B_3}{1})),     \\
\multicolumn{1}{c}{\vdots}                           \\
s_n \in Sect(\snlessonecascade (\crossx{A}{B_n}{1})) \\
\end{array}
\end{equation*}
\mbox{ in \catc},
then  we can define a morphism
$\duplesn:A \morph B_n$ in \catcw such that 
\begin{enumerate}[(i)]
\item $s(\duplesn) = s_n$,
\item for $n> 1$, $\duplesn \circ p_{B_n} = \duplesnlessone$, and 
\item for all objects $B$ of \catcw such that $B_n < B$, 
$\sncascade (\crossx{A}{B}{1}) = \duplesn ^* B$, \\
and for all sections $s$ of $B$,
$\sncascade (\crossx{A}{s}{1}) = \duplesn ^* s$.
\end{enumerate}

The definition of $\duplesn$ proceeds by induction. 
Define $\duplesone= s_1 \circ q(p_{A,1},B_1)$.
By axiom (s1), it follows immediately that $s(\duplesone)=s_1$.

If $B_1 <B$ in \catcw then we have $\sonesub (\crossx{A}{B}{1})=\duplesone ^* B$ because
\begin{align*}
\sonesub (\crossx{A}{B}{1})&= \sonesub q(p_{A,1},B_1)^*B     && \mbox{by definition of $\crossx{1}{}{}$,}\\
                         &= (s_1 \circ q(p_{A,1},B_1))^*B   && \mbox{by pullback coherence axiom (q5),}\\
                         &= \duplesone ^* B                   && \mbox{by definition of $\duplesone ^* B$.}
\end{align*}
Also, if $g$ is a section of $B$ then we can show, by a similar argument, 
that $\sonesub (\crossx{A}{g}{1})=\duplesone ^* g$.

Now assume that $\duplesi$ is defined and satisfies (i) to (iii) above. 
In particular we have  $\sicascade B_{i+1} = \duplesi ^* B_{i+1}$, and therefore that
$s_{i+1} \in Sect(\duplesi ^* B_{i+1})$, and this then allows us to define $\duplesisucc$ by 
\begin{equation*}
\duplesisucc = s_{i+1} \circ q(\duplesi, B_{i+1}).
\end{equation*} 
Immediately by axiom s3
we have that $s(\duplesisucc)=s_{i+1}$.
We have that $\duplesisucc \circ p_{B_{i+1}}= \duplesi$ because
\begin{align*}
\duplesisucc \circ p_{B_{i+1}} &=s_{i+1} \circ q(\duplesi, B_{i+1}) \circ p_{B_{i+1}} && \mbox{by definition of $\duplesisucc$,} \\
                               &=s_{i+1} \circ p_{\duplesi ^* B_{i+1}} \circ \duplesi && \mbox{because the pullback diagram commutes,} \\
															 &= \duplesi                       && \mbox{because $s_{i+1}$ is a section.}
\end{align*}
To establish (iii), suppose $B$ is some object such that $B_{i+1} < B$ in \catcw then we can show that $\sisucccascade (\crossx{A}{B}{1})=\duplesisucc ^* B$ as follows:
\begin{align*}
\sisucccascade (\crossx{A}{B}{1}) 
              &= \sisuccsub \duplesi ^* B && \mbox{by inductive hypothesis,} \\
                         &= \sisuccsub q(\duplesi,B_{i+1})^*B  && \mbox{by definition of extended $^*$,}\\
                         &= (s_{i+1} \circ q(\duplesi,B_{i+1}))^*B   && \mbox{by pullback coherence axiom,}\\
                         &= \duplesisucc ^* B                   && \mbox{by definition of $\duplesisucc$.}
\end{align*}
A similar argument shows that if $g$ is a section of $B$ then $\sisucccascade (\crossx{A}{g}{1})=\duplesisucc ^* g$.


\begin{lemma}
\label{dupledestructionlemma}
If $\duplesn : A \morph B_n$ in a contextual category $\catcw$ then \foreachi, 
\begin{equation}
\duplesn \circ p_{B_n,B_i} = \duplesi
\end{equation} 
\end{lemma}
\begin{proof}
Follows because for each $j$, $i < j \leq n$, $\duplesj \circ p_{B_j} = \duplesjlessone$
and $p_{B_j,B_i} = p_{B_j} \circ p_{B_{j-1},B_i}$.
\end{proof}

\newcommand{\dupletuplerhs}{\bigtuple{\duplesnlessone,g}_{p_{B_{n-1},B_i},C}}
\begin{lemma}
\label{thedupletuplelemma}
If $A$ is an object of contextual category \catc, if $1 \base B_1 ... \base B_n$ in \catcw and if
\begin{equation*}
\begin{array}{l}
s_1 \in Sect(\crossx{A}{B_1}{1}),                  \\
s_2 \in Sect(\sonesub (\crossx{A}{B_2}{1}))=Sect(\duplesone^*B_2),         \\
s_3 \in Sect(\stwocascade (\crossx{A}{B_3}{1}))=Sect(\duplestwo^*B_3),     \\
\multicolumn{1}{c}{\vdots}                           \\
s_n \in Sect(\snlessonecascade (\crossx{A}{B_n}{1})) =Sect(\duplesnlessone^*B_n)\\
\end{array}
\end{equation*}
\mbox{ in \catc}, and if $B_n$ is $\crossx{B_{n-1}}{C}{B_i}$, for some $i$, $0 \leq i < n$, 
(where in the case of $n$ equal to $0$ then 
by $B_0$ we mean the terminal object $1$ of \catc), if $g: A \morph C$ in \catcw and 
$g \circ p_C = \duplesi$, so that
$s(g) \in Sect(\duplesnlessone ^* B_n)$
\footnote {Because by definition of $s(g)$, $s(g) \in Sect((g \circ p_C) ^* C)$ and if 
$g \circ p_C =  \duplesi$ then 
\begin{align*}
(g \circ p_C) ^* C &= \duplesi ^* C  \\
                  &= (\duplesnlessone \circ {p_{B_{n-1},B_i}})^* C\\
									&=\duplesnlessone ^* B_n
\end{align*}
},  then if $s(g)=s_n$ then 


\begin{equation}
\label{dupletuplegoal}
\duplesn = \dupletuplerhs\,.
\end{equation}
\end{lemma}
\begin{proof}


To show that $\dupletuplerhs$ (rhs of (\ref{dupletuplegoal})) is defined we need show that 
\begin{equation}
g \circ p_C = \duplesnlessone \circ p_{B_{n-1},B_i}
\end{equation}
This follows from the assumption that $g \circ p_C = \duplesi$ and from lemma \ref{dupledestructionlemma}.

Now $\dupletuplerhs$ is therefore defined and is the unique morphism such that
\begin{equation}
\dupletuplerhs \circ p_{\crossx{B_{n-1}}{C}{1}} = \duplesnlessone
\end{equation}
and
\begin{equation}
\dupletuplerhs \circ q(p_{B_{n-1},B_i},C) = g
\end{equation}

as shown here

\begin{equation}
\begin{array}{c p{4cm} c p{3cm} c }
\\[1.75cm]
\Rnode{A}{A} && \Rnode{Bn1C}{\crossx{B_{n-1}}{C}{1}} &&                            \\[1.5cm]
						 &&                                      &&\Rnode{C}{C}                \\[0.5cm]
             && \Rnode{Bn1}{B_{n-1}}                 &&                            \\[1.5cm]
						 &&                                      &&\Rnode{Bi}{B_i}             \\[0.5cm]
\end{array}
\mbox{\ncdarr{A}{Bn1C}
\alabel{ \dupletuplerhs}
\ncarr{A}{Bn1}
\blabel{\duplesnlessone}
\ncsar{Bn1C}{Bn1}
\blabel{p_{\crossx{B_{n-1}}{C}{1}}}
\nccdar{Bn1}{Bi}
\blabel{p_{B_{n-1},B_i}}
\ncarr{Bn1C}{C}
\alabel{q(p_{B_{n-1},B_i},C)}[0.3]
\ncsar{C}{Bi}
\alabel{p_C}
\ncarr[60]{A}{C}
\alabel{g}
}
\end{equation}

Therefore to show (\ref{dupletuplegoal}), as we are required,  it suffices  to show that 

\begin{equation}
\label{dupletuplesubgoalone}
\duplesn \circ p_{\crossx{B_{n-1}}{C}{B_i}} = \duplesnlessone
\end{equation}
and
\begin{equation}
\label{dupletuplesubgoaltwo}
\duplesn \circ q(p_{B_{n-1},B_i},C) = g
\end{equation}

(\ref{dupletuplesubgoalone}) holds from the definition of $\duple{}_{B_n}$ because we have initially assumed that $B_n=\crossx{B_{n-1}}{C}{B_i}$.

We show that (\ref{dupletuplesubgoaltwo}) holds by using lemma \ref{footandstactic} and showing that
\begin{equation}
\label{dupletuplesubgoaltwoone}
\duplesn \circ q(p_{B_{n-1},B_i},C) \circ p_C = g \circ p_C
\end{equation}
and
\begin{equation}
\label{dupletuplesubgoaltwotwo}
s(\duplesn \circ q(p_{B_{n-1},B_i},C)) = s(g)
\end{equation}

(\ref{dupletuplesubgoaltwotwo}) follows immediately because by axiom \highlight{(s3)} \commentary{check use of parenthesis in reference to (q1),...(s3)} etc.
$s(\duplesn \circ q(p_{B_{n-1},B_i},C))=s(\duplesn)=s_n$ and from the initial assumption 
$s(g)=s_n$.

We are left with proving (\ref{dupletuplesubgoaltwoone}) which we can do as follows:
\begin{align*}
\duplesn \circ q(p_{B_{n-1},B_i},C) \circ p_C 
              &=  \duplesn \circ p_{\crossx{B_{n-1}}{C}{B_i}} \circ p_{B_{n-1},B_i} 
                                               && \mbox{commutivity of pullback square},               \\
							&=\duplesn \circ p_{\crossx{B_{n-1}}{C}{B_i},B_i} && \mbox{definition of extended $p$,}  \\
							&=\duplesi                                        && \mbox{lemma \ref{dupledestructionlemma},} \\
							&= g \circ p_C                                    && \mbox{from the initial assumption.}
\end{align*}
\end{proof}
\fi


%\fi
%\iffalse
%\section{Meta-GAT algebras}  THIS MATERIAL MOVED INTO PREVIOUS SECTION
%\input{metaGATalgebras} 
%\fi                         % END IF
%\iffalse
\section{Instances of Generalised Algebraic Theories in Contextual Categories}
\label{sectioninwhichinstanceisdefined}


\begin{definition}
An instance $I$ of \gatUw in \catcw is a  mapping 
of derived T- and $\in$- rules of \gatUw to objects, respectively sections, of \gatUw that satisfies the following:
\begin{enumerate}[(i)]
\setlength\itemindent{2cm}
\item \underline{\textbf{T-rules}} 
Suppose that  the rule
\gatdisplayrule{\xDelta{n}}{\isT{\Delta}} is a derived rule of \gatUw which is mapped by $I$ to an object $a$ of \catc. Denote this rule $r$. Recall that because $r$ is a derived rule then it follows  that for each $i$, 
$1 \leq i \leq n$, the rule \gatdisplayrule{\xDelta{i-1}}{\isT{\Delta_i}} is a derived rule of \gatU. Let $r_i$ denote this rule.
Suppose that $I$ maps each rule $r_i$ to an object $a_i$ of \catcw.
It is required that $1 \base a_1 \base ... \base a_n \base a$ in \catc.

Suppose that the  expression $\Delta$ is exactly the expression $\Delta_i$, for some $i$, $1 \leq i \leq n$. In this special case we require that the rule $r$  is mapped by $I$ to the object 
$\crossx{a_n}{a_i}{a_{i-1}}$. 

\item \underline{\textbf{$\boldsymbol {\in}$-rules}} 
In addition to the assumptions made in (i),  suppose that the rule
\gatdisplayrule{\xDelta{n}}{\ofT{t}{\Delta}} is a  derived rule of \gatU. 
Denote this rule $r_t$. It is required that $I$ maps the rule $r_t$ to a section
 $s:a_n \morph a$ in \catcw i.e. to a morphism $s:a_n \morph a$ such that $s \circ p_a = id_{a_n}$. 

Suppose that the  expression $\Delta$ is exactly the expression $\Delta_i$, for some $i$, $1 \leq i \leq n$ and that the expression $t_i$ is simply the variable $x_i$. 
In this special case we require that the rule $r_t$  is mapped by $I$ to the section\footnote{
With these assumptions, $s(p_{a_n,a_i}): a_n \morph \crossx{a_n}{a_i}{a_{i-1}}$ in \catcw because by definition  $s(p_{a_n,a_i}): a_n  \morph (p_{a_n,a_i} \circ p_{a_i})^*a_i$,
and we have 
\begin{align*}
(p_{a_n,a_i} \circ p_{a_i})^*a_i &= {p_{a_n,a_{i-1}}} ^* a_i  && \mbox{ because $p_{a_n,a_i} \circ p_{a_i}=p_{a_n,a_{i-1}}$,} \\
                                 &= \crossx{a_n}{a_i}{a_{i-1}} && \mbox{ by definition of $\crossx{}{}{w}$}.
\end{align*}
} % end footnote
$s(p_{a_n,a_i})$ of the object $\crossx{a_n}{a_i}{a_{i-1}}$. 

\item \underline{\textbf{T=-rules}} 
In addition to the assumptions made in (i), suppose that  
the rule \gatdisplayrule{\xDelta{n}}{\Delta = \Delta'} is a derived rule of \gatU. 
We may deduce that the
\gatdisplayrule{\xDelta{n}}{\isT{\Delta'}} is a derived rule of \gatU. Denote this latter rule $r'$.
It is required that the rule $r'$ is mapped by $I$ to the same object $a$ of \catcw that $r$ is mapped to.

\item \underline{\textbf{$\boldsymbol{\in=}$-rules}} 
In addition to the assumptions made in (ii),  suppose that the rule
\gatdisplayrule{\xDelta{n}}{t = t' \in \Delta}
is a derived rule of \gatU. We may deduce that the rule
\gatdisplayrule{\xDelta{n}}{\ofT{t'}{\Delta}} is a  derived rule of \gatU. 
Denote this latter rule $r'_t$.
It is required that the rule $r'_t$ is mapped by $I$ to the same section $s$ of $a$ that $r_t$ is mapped to.

\item \underline{\textbf{weakening T-rules}} 
Suppose now that $Q$ is any context of \gatUw and that the rule 
\gatdisplayrule{\yOmega{m}}{\isT{\Omega}} is a derived rule of \gatU. Denote this rule $r_\Omega$. 
It follows that \foreachj, the rule   \gatdisplayrule{\yOmega{j-1}}{\isT{\Omega_j}} is a derived rule of \gatU. Denote this rule $r_{\Omega_j}$.
Suppose that each rule $r_{\Omega_j}$ is mapped by $I$ to an object $b_j$ of \catcw and that rule $r_\Omega$ is mapped by $I$ to an object $b$ of \catcw so that
we have that $b_1 \base ... \base b_m \base b$ in \catc.

By the simple weakening lemma it follows that the rules
\gatdisplayrule{Q, \yOmega{j-1}}{\isT{\Omega_j}}, \foreachj, and 
\gatdisplayrule{Q, \yOmega{m}}{\isT{\Omega}} are  derived rule of \gatU. It is required that these rules are mapped by $I$ to the objects
$\crossx{a}{b_1}{1},...\crossx{a}{b_m}{1}$ and to $\crossx{a}{b}{1}$, respectively. 

\item \underline{\textbf{weakening $\boldsymbol {\in}$-rules}} 
Suppose in addition that the rule \gatdisplayrule{\yOmega{m}}{\ofT{s}{\Omega}} is a derived rule of \gatUw 
and that this rule is mapped by $I$ to a section $g$ of object $b$ in \catc.
By the simple weakening lemma it follows that the rule \gatdisplayrule{Q, \yOmega{m}}{\ofT{s}{\Omega}}
is a derived rule of \gatU. It is required that this rule is mapped by $I$ to the section $\crossx{a}{s}{1}$
of object $\crossx{a}{b}{1}$ of \catc.


\item \underline{\textbf{substituting in T-rules}} 
Suppose that, as in (v), above, the rule 
\gatdisplayrule{\yOmega{m}}{\isT{\Omega}} is a derived rule of \gatU.
We may deduce that \foreachj, the rule   \gatdisplayrule{\yOmega{j-1}}{\isT{\Omega_j}} is a derived rule of \gatU. 
Suppose that, as in (v), these rules are mapped by $I$ to objects $b_1,...b_n$ and $b$ so that
we have  $b_1 \base ... \base b_m \base b$ in \catc. Now suppose that for some $j$, $1 \leq j \leq m$, the rule
\gatdisplayrule{\yOmega{j-1}}{\ofT{t}{\Omega_j}} is a derived rule of \gatU. 
Now it follows by the substitution lemma that for each $j'$, $j < j' \leq m$ the rule

\gatdisplayrule{\yOmega{j-1}, y_{j+1}\in \Omega_{j+1}[t|y_j],... y_{j'-1} \in \Omega_{j'-1}[t|y_j] }{\isT{\Omega_j[t|y_j]}} is a derived rule of \gatUw and that likewise the rule

\gatdisplayrule{\yOmega{j-1}, y_{j+1}\in \Omega_{j+1}[t|y_j],... y_m \in \Omega_m[t|y_j] }{\isT{\Omega}[t|y_j]} is a derived rule of \gatU.
It is required that these rules are mapped by $I$ to objects $f^*b_{j+1},...f^*b_m$ and $f^*b$, respectively.
Note that as required we have that $f^*b_{j+1}\base ... \base f^*b_m \base f^*b$ in \catc.

\item \underline{\textbf{substituting in $\boldsymbol {\in}$-rules}} 
Suppose that in addition to the situation in (vii), above, the rule
\gatdisplayrule{\yOmega{m}}{\ofT{s}{\Omega}}
is a derived rule of \gatUw and suppose that this rule is mapped to a section $g$ of object $b$ of \catc.
Now it follows by the substitution lemma that the rule
\gatdisplayrule{\yOmega{j-1}, y_{j+1}\in \Omega_{j+1}[t|y_j],... y_m \in \Omega_m[t|y_j]}{\ofT{s[t|y_j]}{\Omega[t|y_j]}} 
is a derived rule of \gatU.
It is a requirement that this rule is mapped by $I$ to the section $f^*g$ of object $f^*b$ of \catc.
\end{enumerate}
\end{definition}

\begin{lemma}
\label{omegarealisationwrtQ}
Suppose that $I$ is an instance of the generalised algebrauc theory \gatUw in the contextual category \catc,
suppose  that $\encyOmega{m}$ is a context of generalised algebraic theory \gatUw and  that $Q$ is some other context and that for some $m \geq 1$,
 \foreachj, \gatdisplayrule{Q}{\ofT{s_j}{\Omega_j[s_1|y_1,...s_{j-1}|y_{j-1}]}} is a derived rule of \gatU\footnote{Recall that such an m-tuple $\tuple{\sm}$ is said to be a realisation of 
$\encyOmega{m}$ wrt $Q$.}.  Suppose that the interpretation $I$  maps the context $Q$ to an object $a$  of $\catc$ and maps
the context $\encyOmega{j}$ to an object $b_j$ of \catc, \foreachj, so that $1 \base b_1 ... \base b_m$ in \catc. 

Suppose that  the rules 
\gatdisplayrule{\yOmega{m}}{\isT{\Delta}} and  \gatdisplayrule{\yOmega{m}}{\ofT{t}{\Delta}} are derived rules of \gatUw. 
Let us denote these rules $r$ and $r_t$, respectively.  

It follows by the substitution lemma (see \cite{Cartmell86})
that the substituted $r$ and $r_t$ rules: 
\gatdisplayrule{Q}{\isT{\Delta[s_1|y_1...s_m|y_m]}} 
and  \gatdisplayrule{Q}{\ofT{t[s_1|y_1...s_m|y_m]}{\Delta[s_1|y_1...s_m|y_m]}} are derived rules of \gatU. 
The first  will be mapped by $I$ to the object $\smstar...\sonestar\crossx{a}{b}{1}$ and the second will
be mapped by $I$ to the morphism  $\smstar...\sonestar\crossx{a}{g}{1}$ (which is defined since $g$ is a section).
\end{lemma}
\begin{proof}
By induction.
\gatdisplayrule{Q}{\ofT{s_1}{\Omega_1}} will be mapped by $I$ to some section $f_1:a \morph \crossx{a}{b_1}{1}$. The j'th rule,
\gatdisplayrule{Q}{\ofT{s_j}{\Omega_j[s_1|y_1,...s_{j-1}|y_{j-1}]}}, will be mapped to a section $f_j:a \morph \fjpstar ... \fonestar\crossx{a}{b_j}{1}$.

In the case that $m=3$  then in \catcw we will have objects and morphisms as follows:
\begin{displaymath}
\begin{array}{c p{1cm} c p {1cm} c  p{1cm} c}
                                                &&                                           && \Rnode{ab3}{\crossx{a}{b_3}{1}}                       \\[1.2cm]
                                                &&\Rnode{f1axb3}{\fonestar\crossx{a}{b_3}{1}}  && \Rnode{ab2}{\crossx{a}{b_2}{1}}                       \\[1.2cm]
 \Rnode{f3target}{\ftwostar\fonestar\crossx{a}{b_3}{1}} &&\Rnode{f2target}{\fonestar\crossx{a}{b_2}{1}}  && \Rnode{ab1}{\crossx{a}{b_1}{\Rnode{f1target}{1}}}     \\[1.2cm]
                                                &&\Rnode{a}{a}                               &&                                                       \\[-3.0cm]
																								&&                                           &&                         && \Rnode{b3}{b_3}             \\[1.2cm]
																								&&                                           &&                         && \Rnode{b2}{b_2}             \\[1.2cm]
																								&&                                           &&                         && \Rnode{b1}{b_1}             \\[1.1cm]
																								&&                                           && \Rnode{abs}{1} \ \ \ \ \ \ \ \ &&    
\end{array}
\end{displaymath}
\ncarr{ab3}{b3}
\ncarr{ab2}{b2}
\ncarr{ab1}{b1}
\ncarr{f1axb3}{ab3}
\ncarr{f2target}{ab2}
\ncarr{f3target}{f1axb3}
\ncarc[arcangle=10,nodesepA=5pt,offsetA=2pt,nodesepB=2pt,offsetB=2pt]{->}{a}{f1target}
\alabel{f_1}[0.25]
\ncarc[arcangle=15,nodesepA=5pt,offsetA=2pt,nodesepB=2pt,offsetB=2pt]{->}{a}{f2target}
\alabel{f_2}
\ncarc[arcangle=10,nodesepA=5pt,offsetA=2pt,nodesepB=2pt,offsetB=2pt]{->}{a}{f3target}
\alabel{f_3}
\ncsar{f3target}{a}
\ncsar{f2target}{a}
\ncsar{f1target}{a}
\ncsar{ab2}{ab1}
\ncsar{ab3}{ab2}
\ncsar{f1axb3}{f2target}
\ncsar{b3}{b2}
\ncsar{b2}{b1}
\ncsar{b1}{abs}
\nccdar{a}{abs}
\end{proof}

\note
We can show that an instance $I$ of a contextual category \gatUw in a contextual category \catcw is
completely determined by its mapping of the introductory rules of sort symbols and operator symbols to
objects, respectively, sections of \catc. In order to show this first define a preinstance as follows:
\begin{definition}
If \gatUw is a generalised algebraic theory  and if \catcw is a contextual category then
a \term{preinstance} $I$ of  \gatUw in \catcw consists of a pair :
\begin{itemize}
\item a mapping $\Isort$ that maps each sort symbol of \gatUw to  an object of \catc,
\item a mapping $\Iop$ that maps each operator symbol of \gatUw to a section of \catcw (i.e. to a morphism $f: A \morph B$ for some 
$A \base B$ in \catcw such that $f \circ p_B=id_A$).
\end{itemize}
\end{definition}

Note that I will say that an instance $I$ of \gatUw in \catcw extends a preinstance $P$ of \gatUw in \catcw to mean that for each sort symbol $A$ of \gatU,
$\Isort(r_A) = P(A)$, where $r_A$ is the introductory rule for $A$ and that for each operator symbol
$f$ of \gatU,   $\Iop(r_f) = P(f)$, where $r_f$ is the introductory rule for $f$.

\note
 For a preinstance to extend to an instance (we will say that it *is* an instance) 
it will need be type correct and to satisfy the axioms of the theory. The definition of exactly what we mean by this has to proceed by induction because, for example, we need an instance of the rule
\gatdisplayrule{\xDelta{n}}{\isT{\Delta}} (as an object $a$ of \catc, say) before we can say whether a preinstance of an operator with introductory rule \genericfintroductoryrule
is well-typed (i.e. to know that it is object $a$ that it is required to be a section of).
Similarly we need to be able to interpret both sides of an axiom before we can say whether it is respected
by the preinstance. 

\note Suppose $P$ be a preinstance of \gatUw in contextual category \catc.
Let $\gatU_0 \subseteq \gatU_1 \subseteq \gatU_2 \subseteq ...$ be the stratification of \gatU. 
Let $P_i$ be the restriction of $P$ to $U_i$. 
We define $P$ to be well-typed and to respect all axioms 
providing that  each  $P_i$ is well-typed and respects all axioms of $\gatU_i$ and 
what we mean by this we define by induction. 
At the same time we prove that if $P_i$ is well-typed and respects all axioms of $\gatU_i$ then $P_i$
extends uniquely to an instance of $\gatU_i$.  

For the inductive step we require
\begin{lemmastar}
\item $P_{i+1}$ extends uniquely to an instance of $U_{i+1}$ iff  $P_i$ extends uniquely to an instance of $U_{i}$ and additionally:
\begin{enumerate}[(i)]
\item
$P_{i+1}$ is well-typed on sort symbol $A$ of $U_{i+1}$. This we define to mean that
for all sort symbols $A$ in $U_{i+1}$, if $A$ has introductory rule 
\genericAintroductoryrule and if this rule is mapped by $P_{i+1}$
to object $a$ of \catcw then $I_i(r_n) \base a$ in \catc, where $r_n$ is the rule 
\gatdisplayrule{\xDelta{n-1}}{\isT{\Delta_n}} (which, as required and due to the stratification, is a derived rule of $\gatU_i$).
\item  $P_{i+1}$ is well-typed on operator symbols  of $U_{i+1}$. This we define to mean that
for all sort symbols $f$ in $U_{i+1}$, if $f$ has introductory rule 
\genericfintroductoryrule then $P_{i+1}$ maps this rule to a section 
of $I_i(r)$ where $r$ is the rule
\gatdisplayrule{\xDelta{n}}{\isT{\Delta}} (which, as required and due to the stratification, is a derived rule of $\gatU_i$). 
\item
 $I_i$ respects the axioms of $U_{i+1}$. By this we mean that 
\begin{enumerate}[(i)]
\item \underline{\textbf{T=-axioms}} 
for all axioms of $U_{i+1}$ of the form
 \gatdisplayrule{\xDelta{n}}{\Delta = \Delta'},
$I_i(r) = I_i(r')$ where $r$ is the rule
\gatdisplayrule{\xDelta{n}}{\isT{\Delta}} and  
and $r'$ is the rule \gatdisplayrule{\xDelta{n}}{\isT{\Delta'}}
\item \underline{\textbf{$\boldsymbol{\in=}$-axioms}} 
for all axioms of $U_{i+1}$ of the form
\gatdisplayrule{\xDelta{n}}{t = t' \in \Delta}
$I_i(r_t) = I_i(r'_t)$ where $r_t$ is the rule
\gatdisplayrule{\xDelta{n}}{\ofT{r_t}{\Delta}} and  
and $r'_t$ is the rule \gatdisplayrule{\xDelta{n}}{\ofT{t'}{\Delta'}}.
\end{enumerate}
\end{enumerate}
\end{lemmastar}
\begin{proof} 
\tbd
\end{proof}



%\iffalse

\section{Contextual Categories -- The Duple Construction}
\label{contextualnotationparttwo}
\newcommand{\duplesone}{{\duple{s_1}_{B_1}}}
\newcommand{\duplestwo}{{\duple{s_1,s_2}_{B_2}}}
\newcommand{\duplesn}{\duple{s_1,...s_n}_{B_n}}
\newcommand{\duplesi}{{\duple{s_1,...s_i}_{B_i}}}
\newcommand{\duplesilessone}{\duple{s_1,...s_{i-1}}_{B_{i-1}}}
\newcommand{\duplesj}{{\duple{s_1,...s_j}_{B_j}}}
\newcommand{\duplesjlessone}{\duple{s_1,...s_{j-1}}_{B_{j-1}}}
\newcommand{\duplesisucc}{{\duple{s_1,...s_{i+1}}_{B_{i+1}}}}
\newcommand{\duplesnlessone}{{\duple{s_1,...s_{n-1}}_{B_{n-1}}}}

\newcommand {\sonesub}{{s_1}^*}
\newcommand {\stwosub}{{s_2}^*}
\newcommand {\stwocascade}{\stwosub\sonesub}
\newcommand {\sisub}{{s_i}^*}
\newcommand {\sicascade}{\sisub...\sonesub}
\newcommand {\sisuccsub}{{s_{i+1}}^*}
\newcommand {\sisucccascade}{\sisuccsub...\sonesub}
\newcommand {\snlessonesub}{{s_{n-1}}^*}
\newcommand {\snlessonecascade}{\snlessonesub...\sonesub}
\newcommand {\snsub}{{s_n}^*}
\newcommand {\sncascade}{\snsub...\sonesub}

\note If $A$ is an object of contextual category \catc, if $1 \base B_1 ... \base B_n$ in \catcw and if
\begin{equation*}
\begin{array}{l}
s_1 \in Sect(\crossx{A}{B_1}{1}),                  \\
s_2 \in Sect(\sonesub (\crossx{A}{B_2}{1})),         \\
s_3 \in Sect(\stwocascade (\crossx{A}{B_3}{1})),     \\
\multicolumn{1}{c}{\vdots}                           \\
s_n \in Sect(\snlessonecascade (\crossx{A}{B_n}{1})) \\
\end{array}
\end{equation*}
\mbox{ in \catc},
then  we can define a morphism
$\duplesn:A \morph B_n$ in \catcw such that 
\begin{enumerate}[(i)]
\item $s(\duplesn) = s_n$,
\item for $n> 1$, $\duplesn \circ p_{B_n} = \duplesnlessone$, and 
\item for all objects $B$ of \catcw such that $B_n < B$, 
$\sncascade (\crossx{A}{B}{1}) = \duplesn ^* B$, \\
and for all sections $s$ of $B$,
$\sncascade (\crossx{A}{s}{1}) = \duplesn ^* s$.
\end{enumerate}

The definition of $\duplesn$ proceeds by induction. 
Define $\duplesone= s_1 \circ q(p_{A,1},B_1)$.
By axiom (s1), it follows immediately that $s(\duplesone)=s_1$.

If $B_1 <B$ in \catcw then we have $\sonesub (\crossx{A}{B}{1})=\duplesone ^* B$ because
\begin{align*}
\sonesub (\crossx{A}{B}{1})&= \sonesub q(p_{A,1},B_1)^*B     && \mbox{by definition of $\crossx{1}{}{}$,}\\
                         &= (s_1 \circ q(p_{A,1},B_1))^*B   && \mbox{by pullback coherence axiom (q5),}\\
                         &= \duplesone ^* B                   && \mbox{by definition of $\duplesone ^* B$.}
\end{align*}
Also, if $g$ is a section of $B$ then we can show, by a similar argument, 
that $\sonesub (\crossx{A}{g}{1})=\duplesone ^* g$.

Now assume that $\duplesi$ is defined and satisfies (i) to (iii) above. 
In particular we have  $\sicascade B_{i+1} = \duplesi ^* B_{i+1}$, and therefore that
$s_{i+1} \in Sect(\duplesi ^* B_{i+1})$, and this then allows us to define $\duplesisucc$ by 
\begin{equation*}
\duplesisucc = s_{i+1} \circ q(\duplesi, B_{i+1}).
\end{equation*} 
Immediately by axiom s3
we have that $s(\duplesisucc)=s_{i+1}$.
We have that $\duplesisucc \circ p_{B_{i+1}}= \duplesi$ because
\begin{align*}
\duplesisucc \circ p_{B_{i+1}} &=s_{i+1} \circ q(\duplesi, B_{i+1}) \circ p_{B_{i+1}} && \mbox{by definition of $\duplesisucc$,} \\
                               &=s_{i+1} \circ p_{\duplesi ^* B_{i+1}} \circ \duplesi && \mbox{because the pullback diagram commutes,} \\
															 &= \duplesi                       && \mbox{because $s_{i+1}$ is a section.}
\end{align*}
To establish (iii), suppose $B$ is some object such that $B_{i+1} < B$ in \catcw then we can show that $\sisucccascade (\crossx{A}{B}{1})=\duplesisucc ^* B$ as follows:
\begin{align*}
\sisucccascade (\crossx{A}{B}{1}) 
              &= \sisuccsub \duplesi ^* B && \mbox{by inductive hypothesis,} \\
                         &= \sisuccsub q(\duplesi,B_{i+1})^*B  && \mbox{by definition of extended $^*$,}\\
                         &= (s_{i+1} \circ q(\duplesi,B_{i+1}))^*B   && \mbox{by pullback coherence axiom,}\\
                         &= \duplesisucc ^* B                   && \mbox{by definition of $\duplesisucc$.}
\end{align*}
A similar argument shows that if $g$ is a section of $B$ then $\sisucccascade (\crossx{A}{g}{1})=\duplesisucc ^* g$.


\begin{lemma}
\label{dupledestructionlemma}
If $\duplesn : A \morph B_n$ in a contextual category $\catcw$ then \foreachi, 
\begin{equation}
\duplesn \circ p_{B_n,B_i} = \duplesi
\end{equation} 
\end{lemma}
\begin{proof}
Follows because for each $j$, $i < j \leq n$, $\duplesj \circ p_{B_j} = \duplesjlessone$
and $p_{B_j,B_i} = p_{B_j} \circ p_{B_{j-1},B_i}$.
\end{proof}

\newcommand{\dupletuplerhs}{\bigtuple{\duplesnlessone,g}_{p_{B_{n-1},B_i},C}}
\begin{lemma}
\label{thedupletuplelemma}
If $A$ is an object of contextual category \catc, if $1 \base B_1 ... \base B_n$ in \catcw and if
\begin{equation*}
\begin{array}{l}
s_1 \in Sect(\crossx{A}{B_1}{1}),                  \\
s_2 \in Sect(\sonesub (\crossx{A}{B_2}{1}))=Sect(\duplesone^*B_2),         \\
s_3 \in Sect(\stwocascade (\crossx{A}{B_3}{1}))=Sect(\duplestwo^*B_3),     \\
\multicolumn{1}{c}{\vdots}                           \\
s_n \in Sect(\snlessonecascade (\crossx{A}{B_n}{1})) =Sect(\duplesnlessone^*B_n)\\
\end{array}
\end{equation*}
\mbox{ in \catc}, and if $B_n$ is $\crossx{B_{n-1}}{C}{B_i}$, for some $i$, $0 \leq i < n$, 
(where in the case of $n$ equal to $0$ then 
by $B_0$ we mean the terminal object $1$ of \catc), if $g: A \morph C$ in \catcw and 
$g \circ p_C = \duplesi$, so that
$s(g) \in Sect(\duplesnlessone ^* B_n)$
\footnote {Because by definition of $s(g)$, $s(g) \in Sect((g \circ p_C) ^* C)$ and if 
$g \circ p_C =  \duplesi$ then 
\begin{align*}
(g \circ p_C) ^* C &= \duplesi ^* C  \\
                  &= (\duplesnlessone \circ {p_{B_{n-1},B_i}})^* C\\
									&=\duplesnlessone ^* B_n
\end{align*}
},  then if $s(g)=s_n$ then 


\begin{equation}
\label{dupletuplegoal}
\duplesn = \dupletuplerhs\,.
\end{equation}
\end{lemma}
\begin{proof}


To show that $\dupletuplerhs$ (rhs of (\ref{dupletuplegoal})) is defined we need show that 
\begin{equation}
g \circ p_C = \duplesnlessone \circ p_{B_{n-1},B_i}
\end{equation}
This follows from the assumption that $g \circ p_C = \duplesi$ and from lemma \ref{dupledestructionlemma}.

Now $\dupletuplerhs$ is therefore defined and is the unique morphism such that
\begin{equation}
\dupletuplerhs \circ p_{\crossx{B_{n-1}}{C}{1}} = \duplesnlessone
\end{equation}
and
\begin{equation}
\dupletuplerhs \circ q(p_{B_{n-1},B_i},C) = g
\end{equation}

as shown here

\begin{equation}
\begin{array}{c p{4cm} c p{3cm} c }
\\[1.75cm]
\Rnode{A}{A} && \Rnode{Bn1C}{\crossx{B_{n-1}}{C}{1}} &&                            \\[1.5cm]
						 &&                                      &&\Rnode{C}{C}                \\[0.5cm]
             && \Rnode{Bn1}{B_{n-1}}                 &&                            \\[1.5cm]
						 &&                                      &&\Rnode{Bi}{B_i}             \\[0.5cm]
\end{array}
\mbox{\ncdarr{A}{Bn1C}
\alabel{ \dupletuplerhs}
\ncarr{A}{Bn1}
\blabel{\duplesnlessone}
\ncsar{Bn1C}{Bn1}
\blabel{p_{\crossx{B_{n-1}}{C}{1}}}
\nccdar{Bn1}{Bi}
\blabel{p_{B_{n-1},B_i}}
\ncarr{Bn1C}{C}
\alabel{q(p_{B_{n-1},B_i},C)}[0.3]
\ncsar{C}{Bi}
\alabel{p_C}
\ncarr[60]{A}{C}
\alabel{g}
}
\end{equation}

Therefore to show (\ref{dupletuplegoal}), as we are required,  it suffices  to show that 

\begin{equation}
\label{dupletuplesubgoalone}
\duplesn \circ p_{\crossx{B_{n-1}}{C}{B_i}} = \duplesnlessone
\end{equation}
and
\begin{equation}
\label{dupletuplesubgoaltwo}
\duplesn \circ q(p_{B_{n-1},B_i},C) = g
\end{equation}

(\ref{dupletuplesubgoalone}) holds from the definition of $\duple{}_{B_n}$ because we have initially assumed that $B_n=\crossx{B_{n-1}}{C}{B_i}$.

We show that (\ref{dupletuplesubgoaltwo}) holds by using lemma \ref{footandstactic} and showing that
\begin{equation}
\label{dupletuplesubgoaltwoone}
\duplesn \circ q(p_{B_{n-1},B_i},C) \circ p_C = g \circ p_C
\end{equation}
and
\begin{equation}
\label{dupletuplesubgoaltwotwo}
s(\duplesn \circ q(p_{B_{n-1},B_i},C)) = s(g)
\end{equation}

(\ref{dupletuplesubgoaltwotwo}) follows immediately because by axiom \highlight{(s3)} \commentary{check use of parenthesis in reference to (q1),...(s3)} etc.
$s(\duplesn \circ q(p_{B_{n-1},B_i},C))=s(\duplesn)=s_n$ and from the initial assumption 
$s(g)=s_n$.

We are left with proving (\ref{dupletuplesubgoaltwoone}) which we can do as follows:
\begin{align*}
\duplesn \circ q(p_{B_{n-1},B_i},C) \circ p_C 
              &=  \duplesn \circ p_{\crossx{B_{n-1}}{C}{B_i}} \circ p_{B_{n-1},B_i} 
                                               && \mbox{commutivity of pullback square},               \\
							&=\duplesn \circ p_{\crossx{B_{n-1}}{C}{B_i},B_i} && \mbox{definition of extended $p$,}  \\
							&=\duplesi                                        && \mbox{lemma \ref{dupledestructionlemma},} \\
							&= g \circ p_C                                    && \mbox{from the initial assumption.}
\end{align*}
\end{proof}

\section{Supplementary Lemmas}

\begin{lemma}
\llabel{supplementarylemma}
If $I$ is an instance of the generalised algebraic theory $U$ in a contextual category \catcw then
if  $Q$ and $\encyOmega{m}$ are contexts, for some $m \geq 1$, 
and if $\tuple{\sm}$ is a realisation of $\encyOmega{m}$ wrt $Q$ so that
 \foreachj, the rule \IsOmega{j} which we denote $r_{s_j}$ is a derived rule of $U$, then
\begin{enumerate}[(i)]
\item
if the rule \ZOmega which we denote $r_\Omega$ is a derived rule of $U$ then
$$\displaystyle\Imappedrule{Q}{\isT{\Omega[s_1|y_1...s_m|y_m]}}=\duple{I(r_{s_1}),...I(r_{s_m})}^*I(r_\Omega)$$
\item if \ZsOmega which we denote $r_s$ is a derived rule of $U$ then 
$$\displaystyle\Imappedrule{Q}{\ofT{s[s_1|y_1...s_m|y_m]}{\Omega[s_1|y_1...s_m|y_m]}}=\duple{I(r_{s_1}),...I(r_{s_m})}^*I(r_s)$$
\end{enumerate}
\end{lemma}
\begin{proof}
(i) follows because $r_\Omega$ being a derived rule must be consistently interpreted by $I$ and so can use clause (i)(c) of definition \lref{consistentinterpretation} and then simplify using (d3a). (ii) follows likewise using clause (ii)(c) of definition \lref{consistentinterpretation} and then
(d3b).
\end{proof}


\iffalse
\begin{lemma}
\llabel{supplementarytuplelemma}
\highlight{Live without this perhaps because of the ambiguity of the tuple notation and the complexity of making it more precise?}
If $I$ is an instance of the generalised algebraic theory $U$ in a contextual category \catcw then
if  $Q$ and $\encyOmega{m}$ are contexts, for some $m \geq 1$, 
and if $\tuple{\sm}$ is a realisation of $\encyOmega{m}$ wrt $Q$ so that
 \foreachj, the rule \IsOmega{j} which we denote $r_{s_j}$ is a derived rule of $U$, then
 if \foreachj, there is a morphism $f_j$ such that $I(r_{s_j})=s(f_j)$ then \commentary{check that we do not need to specify the domain of each $f_j$.}
\begin{enumerate}[(i)]
\item
if the rule \ZOmega which we denote $r_\Omega$ is a derived rule of $U$ then
$$\displaystyle\Imappedrule{Q}{\isT{\Omega[s_1|y_1...s_m|y_m]}}=\tuple{f_1,...f_m}^*I(r_\Omega)$$
\item if \ZsOmega which we denote $r_s$ is a derived rule of $U$ then \commentary{unfortunate double use of $s$.}
$$\displaystyle\Imappedrule{Q}{\ofT{s[s_1|y_1...s_m|y_m]}{\Omega[s_1|y_1...s_m|y_m]}}=\tuple{f_1,...f_m}^*I(r_s)$$
\end{enumerate}
\end{lemma}
\begin{proof}
\tbd
\end{proof}
\fi


\begin{lemma}
\llabel{supplementaryweakeninglemma}
If $I$ is an instance of the generalised algebraic theory $U$ in a contextual category \catcw then
if  $Q$ and $\encyOmega{m}$ are contexts, for some $m \geq 0$,  then
\begin{enumerate}[(i)]
\item if the rule \ZOmega which we denote $r_\Omega$ is a derived rule of $U$ then
$$\displaystyle\Imappedrule{Q,\, \yOmega{m}}{\isT{\Omega}}=\crossx{I(Q)}{I(r_\Omega)}{1}$$
\item if \ZsOmega which we denote $r_s$ is a derived rule of $U$ then 
$$\displaystyle\Imappedrule{Q,\, \yOmega{m}}{\ofT{s}{\Omega}}=\crossx{I(Q)}{I(r_s)}{1}$$
\end{enumerate}
\end{lemma}
\begin{proof}
The  $m=0$ case was shown in lemma \lref{typeweakeninglemma}.
For $m \geq 1$ we can assume, inductively, that the result holds for all j,  $j \leq m$, so that 
we can assume \foreachj, the rule \gatdisplayrule{Q,\, \yOmega{j-1}}{\isT{\Omega_j}} is mapped by $I$ to
$\crossx{I(Q)}{I(r_{\Omega_j})}{1}$.

\newcommand{\IofyweakenedbyQ}[1]{I(s(p_{\crossx{I(Q)}{I(r_{\Omega_m})}{1}, \crossx{I(Q)}{I(r_{\Omega_#1})}{1}}))}
With this assumption, because $I$ is an instance then from clause (ii)(d) of definition \lref{consistentinterpretation} 
we have that, \foreachj,
\begin{equation}
\label{weakendedyinterpretation}
I(r_{y_j})=\IofyweakenedbyQ{j} 
\end{equation}
where $r_{y_j}$  is the rule \gatdisplayrule{Q,\, \yOmega{m}}{\ofT{y_j}{\Omega_j}}.

Now we  prove (i) as follows:
\begin{align*}
\displaystyle\Imappedrule{Q,\, \yOmega{m}}{\isT{\Omega}}  \kern-2cm   \\
          &= \duple{I(r_{y_1}),... I(r_{y_n})} ^* I(r_\Omega)                   && \mbox{By lemma \lref{supplementarylemma},}     \\ 
          &= \duple{\IofyweakenedbyQ{1},... \IofyweakenedbyQ{m}} ^* I(r_\Omega) && \mbox{using (\ref{weakendedyinterpretation}),}                         \\
          &= q(p_{I(Q),1},I(r_{\Omega_m})) ^* I(r_\Omega)                       && \mbox{by \lref{duplesofplemma},}               \\
          &= {p_{I(Q),1}} ^*  I(r_\Omega)                                         && \mbox{by (Q6),}                                \\
          &= \crossx{I(Q)}{I(r_\Omega)}{1}                                      && \mbox{by definition of $\crossx{}{}{1}$.}  
\end{align*} 
(ii) follows in like manner.

\end{proof}

\begin{lemma}
\llabel{Xnlemma}
If $\iI$ is an interpretation of a generalised algebraic theory $U$ in a contextual category \catcw and if $X$ is an absolute sort symbol of $U$ which is mapped 
by $\iI_{sort}$ to an object $X$ of \catcw (so that $1 \base X$ in \catc) then for any $n \geq 1$ 
\begin{enumerate}[(i)]
\item
The context $\tuple{\ofT{x_1}{X},...\ofT{x_n}{X}}$ is mapped by $\Ibar$ to the object $X^n$ of \catc,
\item the rule 
\gatdisplayrule{\ofT{x_1}{X},... \ofT{x_n}{X}}{\ofT{x_i}{X}} is mapped by $\Ibar$ to the section $s(p_i)$ of $X^{n+1}$, where $p_i$ is the $i$'th projection morphism, $p_i: X^n \morph X$,
\item if $P$ is a context of $U$ that extends the context $\tuple{\ofT{x_1}{X},...\ofT{x_n}{X}}$ and if $P$ is mapped by $\Ibar$ to
the object $Y$ of \catcw (so that $X < Y$ in \catc) then the rule 
\gatdisplayrule{P}{\ofT{x_i}{X}} is mapped by $\Ibar$ to the section $s(p_{Y,X^n}\circ p_i)$ of object $\crossx{Y}{X}{1}$, where $p_i$ is the $i$'th projection morphism, $p_i: X^n \morph X$.
\end{enumerate}
\end{lemma}
\begin{proof}
(i) is proved by induction on $n$. The case of $n=1$ is trivial and is simple because by definition the mapping of the context $\ofT{x_1}{X}$ is simply the mapping of the rule $\tstyle \isT{X}$ which is the mapping under $\iI_{sort}$ of the sort symbol $X$. For the inductive step assume that (i) holds in the case of $n-1$ and therefore that the context $\ofT{x_1}{X},... \ofT{x_{n-1}}{X}$ is mapped by $\Ibar$ to the context $X^{n-1}$. 
By lemma \lref{typeweakeninglemma} (case $m=0$) we determine that 
the mapping under $\Ibar$ of the rule \gatdisplayrule{\ofT{x_1}{X},... \ofT{x_{n-1}}{X}}{\isT{X}} is $\crossx{X^{n-1}}{X}{1}$ i.e. is $X^n$.

(ii) By clause (ii)(d) of definition \lref{consistentinterpretation} 
the rule \gatdisplayrule{P}{\ofT{x_i}{X}} is mapped by $\Ibar$ to $s(p_{Y,X^i})$ but
\begin{align*}
s(p_{Y,X^i}) &= s(p_{Y,X^i} \circ q(p_{X^{i-1},1},X))                     &&\mbox{by (s3),}\\
             &= s(p_{Y,X^n} \circ p_{X^n,X^i} \circ q(p_{X^{i-1},1},X))   &&\mbox{by (P2),}\\
             &= s(p_{Y,X^n} \circ p_i)                                    &&\parbox[t]{4.2cm}{where $p_i$ is the i'th projection morphism $p_i:X^n \morph X$.}
\end{align*}
and so we have shown \gatdisplayrule{P}{\ofT{x_i}{X}} is mapped by $\Ibar$ to $s(p_{Y,X^n}\circ p_i)$, as required.
\end{proof}
\fi


\newcommand{\xyip}{\crossx{x}{y_{i-1}}{1}}
\newcommand{\xyi}{\crossx{x}{y_i}{1}}
\newcommand{\xyn}{\crossx{x}{y_n}{1}}
\newcommand{\ynyi}{\crossx{y_n}{y_i}{y_{i-1}}}
\newcommand{\xynyi}{\crossx{x}{(\ynyi)}{1}}
\newcommand{\xynxyi}{\crossx{(\xyn)}{(\xyi)}{\xyip}}
\newcommand{\xsynyi}{\crossx{x}{s(p_{y_n,y_i})}{1}}
\newcommand{\sxynxyi}{s(p_{\xyn,\xyi})}
\begin{displaymath}
\begin{array}{c}
\fippvectorstar (\xynyi)=  \\
\fippvectorstar \big(\xynxyi \big)= \\
\Rnode{u1}{\crossx{\fippvectorstar(\xyn)}{\kern-0.2cm\fippvectorstar(\xyi)}{\kern-0.15cm\fippvectorstar(\xyip)}} \\[2.6cm]
\Rnode{u2}{\fippvectorstar(\xyn)}     \\[2cm]
\Rnode{s3}{\fippvectorstar(\xyi)}     \\[1.2cm]
\Rnode{s2}{\fippvectorstar(\xyip)}    \\[1.2cm]
\Rnode{s1}{x} \\
\makebox[0cm]{
\ncsar{u1}{u2}
\ncarc[arcangle=-20,nodesepA=5pt,offsetA=-3pt,nodesepB=3pt,offsetB=3pt]{->}{u2}{u1}
\blabel{\fippvectorstar (\xsynyi)=}[0.7]
\blabel{\fippvectorstar (\sxynxyi)=}[0.5]
\blabel{s(p_{\fippvectorstar(\xyn),\fippvectorstar(\xyi)})}[0.3]
\ncdotdotdot{u2}{s3}
\ncsar{s3}{s2}
\ncsar{s2}{s1}
\ncarc[arcangle=-20,nodesepA=5pt,offsetA=-3pt,nodesepB=3pt,offsetB=0pt]{->}{s1}{s2}
\blabel{f_{i-1}} [0.5]
}
\end{array}
\end{displaymath}
\hrulefill
\begin{displaymath}
\begin{array}{c}
\fipvectorstar (\xynyi)=  \\
\fipvectorstar \big( \xynxyi \big)= \\
\Rnode{u1}{\crossx{\fipvectorstar(\xyn)}{\kern-0.2cm\fipvectorstar(\xyi)}{\kern-0.2cm\fipvectorstar(\xyip)}} \\[2.6cm]
\Rnode{u2}{\fipvectorstar(\xyn)}     \\[2cm]
\Rnode{s2}{\fipvectorstar(\xyi)}     \\[1.2cm]
\Rnode{s1}{x} \\
\makebox[0cm]{
\ncsar{u1}{u2}
\ncarc[arcangle=-20,nodesepA=5pt,offsetA=-3pt,nodesepB=3pt,offsetB=3pt]{->}{u2}{u1}
\blabel{\fipvectorstar(\xsynyi)=}[0.7]
\blabel{\fipvectorstar(\sxynxyi)=}[0.5]
\blabel{s(p_{\fipvectorstar(\xyn),\fipvectorstar(\xyi)})}[0.3]
\ncdotdotdot{u2}{s2}
\ncsar{s2}{s1}
\ncarc[arcangle=-20,nodesepA=5pt,offsetA=-3pt,nodesepB=3pt,offsetB=0pt]{->}{s1}{s2}
\blabel{f_i} [0.5]
}
\end{array}
\end{displaymath}
Now  we have that
\begin{align*}
\fistar(\fipvectorstar(\xsynyi)) 
             &= \fistar(s(p_{\fipvectorstar(\xyn),\fipvectorstar(\xyi)})) && \mbox{ as shown above}  \\
             &= \crossx{\big(\fivectorstar(\xyn)\big)}{f_i}{x}            && \mbox{ by lemma \lref{missingsublemma3}}
\end{align*}

\hrulefill
\begin{displaymath}
\begin{array}{c}
\fivectorstar (\xynyi)=  \\
\fivectorstar \big( \xynxyi \big)= \\
\Rnode{u1}{\crossx{\big(\fivectorstar(\xyn)\big)}{\big(\fivectorstar(\xyi)\big)}{x}} \\[2.6cm]
\Rnode{u2}{\fivectorstar(\xyn)}     \\[2cm]
\Rnode{s2}{\fivectorstar(\xyi)}     \\[1.2cm]
\Rnode{s1}{x} \\
\makebox[0cm]{
\ncsar{u1}{u2}
\ncarc[arcangle=-20,nodesepA=5pt,offsetA=-3pt,nodesepB=3pt,offsetB=3pt]{->}{u2}{u1}
\blabel{\fivectorstar(\xsynyi)=} [0.7]
\blabel{\fivectorstar(\sxynxyi)=}[0.5]
\blabel{\crossx{\big(\fivectorstar(\xyn)\big)}{f_i}{x}}[0.3]
\ncdotdotdot{u2}{s2}
\ncsar{s2}{s1}
\ncarc[arcangle=-20,nodesepA=5pt,offsetA=-3pt,nodesepB=3pt,offsetB=0pt]{->}{s1}{s2}
\blabel{f_{i+1}} [0.5]
}
\end{array}
\end{displaymath}
From which by missing sublemma \lref{missingsublemma2}
\begin{align*}
\fiistar (\fivectorstar(\xsynyi)) 
    &= \fiistar (\crossx{\big(\fivectorstar(\xyn)\big)}{f_i}{x}) && \mbox{ as shown above} \\
    &= \crossx{\big(\fiivectorstar(\xyn)\big)}{f_i}{x}           && \mbox{ by lemma \lref{missingsublemma2}}
\end{align*}

\hrulefill \\
\begin{newtt}
ERGO From which by repeated use of missing sublemma \lref{missingsublemma2}

\begin{displaymath}
\begin{array}{c}
\fnpvectorstar (\xynyi)=  \\
\fnpvectorstar \big( \xynxyi \big)= \\
\Rnode{u1}{\crossx{\big(\fnpvectorstar(\xyn)\big)}{\big(\fnpvectorstar(\xyi)\big)}{x}} \\[2.6cm]
\Rnode{u2}{\fnpvectorstar(\xyn)}     \\[1.2cm]
\Rnode{s1}{x} \\
\makebox[0cm]{
\ncsar{u1}{u2}
\ncarc[arcangle=-20,nodesepA=5pt,offsetA=-3pt,nodesepB=3pt,offsetB=3pt]{->}{u2}{u1}
\blabel{\fnpvectorstar(\xsynyi)=} [0.7]
\blabel{\fnpvectorstar(\sxynxyi)=}[0.5]
\blabel{\crossx{\big(\fnpvectorstar(\xyn)\big)}{f_i}{x}}[0.3]
\ncdotdotdot{u2}{s2}
\ncsar{u2}{s1}
\ncarc[arcangle=-20,nodesepA=5pt,offsetA=-3pt,nodesepB=3pt,offsetB=0pt]{->}{s1}{u2}
\blabel{f_{n}} [0.5]
}
\end{array}
\end{displaymath}

From which
\begin{align*}
\fnvectorstar(\xsynyi)&=\fnvectorstar(\sxynxyi)                         &&\mbox{by ??,}                             \\
                      &= \crossx{\big(\fnvectorstar(\xyn)\big)}{f_i}{x} &&\mbox{by ??,}                             \\
                      &= f_n ^* \bigg( \crossx{\big(\fnpvectorstar(\xyn)\big)}{f_i}{x} \bigg)                       \\
                      &= f_i                                            &&\mbox{ by lemma \lref{missingsublemma1}}
\end{align*}
\end{newtt}



\begin{oldtt}
\newcommand{\xnz}{\crossx{x_n}{z}{w}}
\newcommand{\xng}{\crossx{x_n}{g}{w}}
\begin{displaymath}
\begin{array}{ c p{0.75cm} c p{2.25cm} c p{0.1cm} c p{1.0cm} c} 
                    &&                                                    &&                && \Rnode{xnz}{\xnz} &&                       \\ [1.2cm]
                    &&                                                    &&                && \Rnode{xn}{x_n}   &&                       \\ [2.0cm]
                    && \fnpvectorstar(\xnz)                               &&                &&                   &&                       \\ 
                    && \Rnode{fnpz}{=\crossx{(\fnpvectorstar x_n)}{z}{w}} &&                && \Rnode{x2}{x_2}   &&             \\ [2.0cm]
\Rnode{fnxnz}{\fnvectorstar (\xnz)=z} && \Rnode{fnpxnz}{\fnpvectorstar x_n} && \Rnode{f1x2}{f_1 ^* x_2} && \Rnode{x1}{x_1}  && \Rnode{z}{z} \\ [2.4cm]
                                    &&                                    && \circlenode[framesep=1cm, linestyle=none]{w}{w} \ \ \ \ \ \ \ \ \ \ &&   &&   
\makebox[0cm]{
\ncdotdotdot{fnpxnz}{f1x2}
\ncsar{fnpz}{fnpxnz}
\ncarc[arcangle=-10,nodesepA=5pt,offsetA=-4pt,nodesepB=3pt,offsetB=-3pt]{->}{fnpxnz}{fnpz}
\blabel{\fnpvectorstar(\xng)}[0.65]
\blabel{=\crossx{(\fnpvectorstar x_n)}{g}{w}}[0.4]
\ncsar{xnz}{xn}
\ncdotdotdot{xn}{x2}
\ncsar{x2}{x1}
\ncsar{fnxnz}{w}
\ncarc[arcangle=10,nodesepA=5pt,offsetA=-2pt,nodesepB=3pt,offsetB=0pt]{->}{w}{fnxnz}
\alabel{\fnvectorstar(\xng)=g} [0.5]
\ncsar{fnpxnz}{w}
\ncarc[arcangle=10,nodesepA=5pt,offsetA=-2pt,nodesepB=3pt,offsetB=2pt]{->}{w}{fnpxnz}
\alabel{f_n} [0.6]
\ncsar{f1x2}{w}
\ncarc[arcangle=10,nodesepA=5pt,offsetA=-2pt,nodesepB=3pt,offsetB=2pt]{->}{w}{f1x2}
\alabel{f_2} [0.5]
\ncsar{x1}{w}
\ncarc[arcangle=10,nodesepA=5pt,offsetA=-2pt,nodesepB=3pt,offsetB=2pt]{->}{w}{x1}
\alabel{f_1} [0.4]
\ncsar{z}{w}
\ncarc[arcangle=-10,nodesepA=5pt,offsetA=0pt,nodesepB=3pt,offsetB=-2pt]{->}{w}{z}
\blabel{g} [0.4]
}
\end{array}
\end{displaymath}
\end{oldtt}
\hrulefill
\newpage
\section{Examples}
\label{examples}
I give two examples to illustrate how the the main definitions may be applied.
\subsection{Approach -- Conventions}

\subsubsection*{Naming of objects and morphisms}
\label{projectionnaming}
It is convenient to adopt some naming conventions when discussing the structure in a contextual category $\catcw$ into which an interpretation $I$ maps a theory $U$. Firstly it is convenient that the object $I(X)$ which is the interpretation of some sort $X$ of the theory is simply referred to as $X$.
Secondly if $X$ is a sort symbol such that $1 \base X$ in $\catcw$ then it is conventient to adopt some naming conventions for
the projection morphisms: the two canninical projection morphims $X^2 \morph X$ with be referred to as $x_1$ and $x_2$. The three cannonical projection morphisms $X^3 \morph X$ will be referred to as $y_1, y_2$ and $y_3$ and the four projection morphisms $X^4 \morph X$ will be referred to as 
$z_1, z_2, z_3$ and $z_4$. For sake of  uniformity  and considering that  $X^1$ is $X$ with product diagram  $id_X: X \morph X$  in the definitions
below we refer to $id_X$ as $w_1$.
Now in a contextual category, for any object $X$ such that $1 \base X$, $id_X = q(id_1,X) = q(t_1,X)$. Therefore we can define our uses of
$w_1$, $x_1,x_2,y_1,y_2,y_3,z_1,z_2,z_3$ and $z_4$ with respect to powers of an object $X$ by: \\
\begin{tabular} {l p{1cm} l p{1cm} l p{1cm} l}
$w_1 = q(t_1,X)$ &&  $x_1 = p_{X^2} \circ w_1$ && $y_1 = p_{X^3} \circ x_1$  && $z_1 = p_{X^4} \circ y_1$ \\
                 &&  $x_2 = q(t_X,X)$          && $y_2 = p_{X^3} \circ x_2$  && $z_2 = p_{X^4} \circ y_2$ \\
                 &&                            && $y_3 = q(t_{X^2},X)$       && $z_3 = p_{X^4} \circ y_3$ \\
                 &&                            &&                            && $z_4 = q(t_{X^3},X)$
\end{tabular}


Finally, we will see in the examples that it aids  readability to shorten  $p_A \circ f$ to $\dot{f}$, $p_B \circ p_A \circ f$ to $\ddot{f}$ and so on.
Therefore the above definitions can be rewritten as: \\
\begin{tabular} {l p{1cm} l p{1cm} l p{1cm} l}
$w_1 = q(t_1,X)$ &&  $x_1 = \dot{w_1}$ && $y_1 = \dot{x_1}$  && $z_1 = \dot{y_1}$ \\
                 &&  $x_2 = q(t_X,X)$          && $y_2 = \dot{x_2}$  && $z_2 = \dot{y_2}$ \\
                 &&                            && $y_3 = q(t_{X^2},X)$       && $z_3 = \dot{y_3}$ \\
                 &&                            &&                            && $z_4 = q(t_{X^3},X)$
\end{tabular}













 
  

\subsection{Internal Monoids -- instances of the Theory of Monoids within Contextual Categories}
\label{monoidsinstanceexample} 

Now we consider what constitutes a monoid internal to a contextual category \catc. That is to say 
we consider what constitutes an instance $I$ of the theory of monoids   in a contextual category \catc.
For this purpose we shall write the theory of monoids ($tm$) as a generalised algebraic theory like this: 
\begin{gatrules}
\gatintros
\gatintroducing{M}
\isT{M} \\
\gatintroducing{unit}
\ofT{unit}{M} \\
\gatintroducing{mult}
\gatsingular[7.5cm]{\ofT{x_1,x_2}{M}}{\ofT{mult(x_1,x_2)}{M}} \\
\gataxioms

\gatintroducing{ \gataxiomno{1} \\ \gataxiomno{2} }
\begin{gatgroup}{\ofT{w}{M}}
\gatleaf[7.5cm]{}{mult(unit,w)=w} \\
\gatleaf[7.5cm]{}{mult(w,unit)=w}
\end{gatgroup} \\
\gatintroducing{ \gataxiomno{3} }
\gatsingular[7.5cm]{\ofT{y_1,y_2,y_3}{M}}{mult(mult(y_1,y_2),y_3)=mult(y_1,mult(y_2,y_3))} 
\end{gatrules}

For the sakes of readability we write the interpretation $I(M)$ of the sort $M$ simply as $M$. Similarly we write $I(unit)$ as $unit$. We write $I(mult)$ as $m$. I will ask the reader  to distinguish for themselves 
those uses of `$M$' and `$unit$' in reference to sorts and operators of the theory $tm$ from those uses in reference to the interpretation of these sorts by the instance $I$ in the contextual category \catc. 
  In the  lemma that follows we use the naming convention outlined above in section \ref{projectionnaming} and name the three projection morphisms $M^3 \morph M$ i.e. by $y_1, y_2, y_3$.
As discussed above these are defined by:

\begin{align*}
y_1 &= p_{M^3,M}, \\
y_2 &= p_{M^3,M^2} \circ q(p_{M,1},M), \\
y_3 &= q(p_{M^2,1},M). \hspace{1cm}\\
\end{align*}

\newcommand{\wM}{\ofT{w}{M}}
\newcommand{\xM}{\ofT{x_1, x_2}{M}}
\newcommand{\yM}{\ofT{y_1, y_2, y_3}{M}}
\newcommand{\doubleM}{M^2}                       %{\crossx{M}{M}{1}}
\newcommand{\trebleM}{M^3}                       %{\crossx{\big(\doubleM\big)}{M}{1}}
\newcommand{\quadM}{M^4}                         % {\crossx{\big(\trebleM\big)}{M}{1}}
\newcommand{\spi}{s(p_{M^3,M^i})}
\newcommand{\sptrebleone}{s(p_{M^3,M^1})}
\newcommand{\sptrebletwo}{s(p_{M^3,M^2})}
\newcommand{\sptreblethree}{s(p_{M^3,M^3})}
\newcommand{\fmult}{m}  %macro used for name of section that monoidal multiplication maps to

\begin{lemma}
\label{internalmonoidlemma}
An internal monoid in a contextual category \catcw consists of
\begin{enumerate}[(a)]
\item 
\begin{itemize}
\item An object $M$ of \catcw which is the interpretation of the sort $M$ of theory $tm$,
\item A section $\fmult$ of \catc, $\fmult \in Sect(\trebleM)$, that is the interpretation of the operator $mult$,
\item A section $unit$ of \catc, $unit \in Sect(M)$, that is the interpretation of the operator $unit$,
\end{itemize}
such that
\begin{equation}
\label{internalmonoidrepresentation1axiom1}
\tuple{p_M \circ unit,id_M}^*\fmult=s(id_M),
\end{equation}
\begin{equation}
\label{internalmonoidrepresentation1axiom2}
\tuple{id_M,p_M \circ unit}^*\fmult=s(id_M),
\end{equation}
and
\begin{equation}
\label{internalmonoidrepresentation1axiom3}
\bigtuple{(\tuple{y_1,y_2}^*\fmult)\circ q(p_{M^3,1},M),y_3}^*\fmult
=\bigtuple{y_1,(\tuple{y_2,y_3}^*\fmult)\circ q(p_{M^3,1},M)}^*\fmult
\end{equation}.
\end{enumerate}
Or, equivalently, 
\begin{enumerate}[(b)]
\item
\begin{itemize}
\item An object $M$ of \catc,
\item A morphism $mult$ of \catc, $mult: \doubleM \morph M$ in \catc,
\item A morphism $unit$ of \catc, $unit: 1 \morph M$ in \catc
\end{itemize}
such that
\begin{equation}
\label{internalmonoidrepresentation2axiom1}
\tuple{p_M \circ unit,id_M}\circ mult=id_M,
\end{equation}
\begin{equation}
\label{internalmonoidrepresentation2axiom2}
\tuple{{id_M,p_M \circ unit}}\circ mult=id_M
\end{equation}
and
\begin{equation}
\label{internalmonoidrepresentation2axiom3}
\bigtuple{\tuple{y_1,y_2}\circ mult,y_3}\circ mult
=\bigtuple{y_1,\tuple{y_2,y_3} \circ mult}\circ mult
\end{equation}
\end{enumerate}
Note that equation (\ref{internalmonoidrepresentation2axiom3})  can also be written as 
\begin{equation}
\label{internalmonoidrepresentation2axiom3rewritten}
\tuple{\crossx{mult}{id_M}{1}}\circ mult = \tuple{\crossx{id_M}{mult}{1}}\circ mult.
\end{equation}
\end{lemma}
\begin{proof}
We show the derivation of the first of these representations, representation (a),  in table \ref{internalmonoidtable} below. 

It remains to show that the second representation, representation (b) can be derived from the first representation.

To do this, first of all define $mult: \doubleM \morph M$ in \catcw from $\fmult: \doubleM \morph \trebleM$ by defining
$mult = \fmult \circ q(p_{\doubleM,1} , M)$. Now we show that each of the equations (\ref{internalmonoidrepresentation1axiom1}),
(\ref{internalmonoidrepresentation1axiom2}), and (\ref{internalmonoidrepresentation1axiom3}) hold of $\fmult$ iff
the respective equation (\ref{internalmonoidrepresentation2axiom1}),
(\ref{internalmonoidrepresentation2axiom2}) or (\ref{internalmonoidrepresentation2axiom3}) hold of $mult$.

That (\ref{internalmonoidrepresentation1axiom1}) holds iff (\ref{internalmonoidrepresentation2axiom1}) holds follows
because by lemma \lref{stactic}, (\ref{internalmonoidrepresentation2axiom1}) holds iff
\begin{equation}
\label{equivalenceone}
s(\tuple{p_M \circ unit,id_M} \circ mult) = s(id_M)
\end{equation}
which holds iff equation (\ref{internalmonoidrepresentation1axiom1}) holds because 
\begin{align*}
s(\tuple{p_M \circ unit,id_M}\circ mult) 
             &= s(\tuple{p_M \circ unit,id_M}\circ s(mult)) && \mbox{by lemma \lref{sfsglemma},} \\
             &= \tuple{p_M \circ unit,id_M} ^* s(mult )  && \mbox{by lemma \lref{sfglemma},} \\
             &= \tuple{p_M \circ unit,id_M} ^* s(\fmult \circ q(p_{\doubleM,1} , M)) && \mbox{definition of $mult$,} \\
			       &= \tuple{p_M \circ unit,id_M} ^* \fmult                                &&  \mbox{by axiom (s2).}
\end{align*}

\noindent By the same reasoning, it follows that (\ref{internalmonoidrepresentation1axiom2}) holds iff (\ref{internalmonoidrepresentation2axiom2}) holds.

We shall show that (\ref{internalmonoidrepresentation1axiom3}) holds iff (\ref{internalmonoidrepresentation2axiom3}) holds.
First note, though, that
\begin{equation}
\label{thirdaxiomsubgoal}
 \tuple{y_1,y_2}\circ \fmult \circ q(p_{\doubleM,1} , M) = (\tuple{y_1,y_2}^*\fmult)\circ q(p_{M^3,1},M)
\end{equation}
because
\begin{align*}
lhs &= (\tuple{y_1,y_2}^*\fmult) \circ q(\tuple{y_1,y_2},M^3) \circ q(p_{M^2,1},M)  && \mbox{by definition of $^*$, } \\
    &= (\tuple{y_1,y_2}^*\fmult) \circ q(\tuple{y_1,y_2} \circ p_{M^2,1},M)         && \mbox{by (q5),} \\
    &= (\tuple{y_1,y_2}^*\fmult) \circ q(p_{M^3,1},M)                               && \mbox{because $1$ is terminal. }\\
    &= rhs
\end{align*}

Now that (\ref{internalmonoidrepresentation1axiom3}) holds iff (\ref{internalmonoidrepresentation2axiom3}) holds
follows because by lemma \ref{stactic}, (\ref{internalmonoidrepresentation2axiom3}) holds iff

\begin{equation}
\label{sofinternalmonoidrepresentation2axiom3}
s(\bigtuple{\tuple{y_1,y_2}\circ mult,y_3}\circ mult)
=s(\bigtuple{y_1,\tuple{y_2,y_3} \circ mult}\circ mult)
\end{equation}
and the lhs of (\ref{sofinternalmonoidrepresentation2axiom3}) and the lhs of (\ref{internalmonoidrepresentation1axiom3}) 
are identical because
\begin{align*}
s(\bigtuple{\tuple{y_1,y_2}\circ mult,y_3}\circ mult) 
    &= \bigtuple{\tuple{y_1,y_2}\circ mult,y_3} ^* s(mult) && \mbox{by lemma \lref{sfglemma},} \\
		&= \bigtuple{\tuple{y_1,y_2}\circ mult,y_3} ^* s(\fmult \circ q(p_{\doubleM,1} , M)) && \mbox{definition of $mult$,} \\
		&= \bigtuple{\tuple{y_1,y_2}\circ mult,y_3} ^* \fmult                                &&  \mbox{by axiom (s2),} \\
		&= \bigtuple{\tuple{y_1,y_2}\circ \fmult \circ q(p_{\doubleM,1} , M),y_3} ^* \fmult  
		                                                                    &&  \mbox{definition of $mult$,} \\	
	  &= \bigtuple{(\tuple{y_1,y_2}^*\fmult)\circ q(p_{M^3,1},M),y_3} ^* \fmult 
		                                                                    &&  \mbox{by (\ref{thirdaxiomsubgoal}).}
\end{align*}
Similarly, the rhs of (\ref{sofinternalmonoidrepresentation2axiom3}) and the rhs of (\ref{internalmonoidrepresentation1axiom3}) 
can be shown to be identical.

\begin{table}[H]
\caption{Deriving what constitutes an instance of the theory of monoids $tm$ in a contextual category \catc. \\
Part One - Introductory rules for $M$, $unit$ and $\fmult$. 
The result is shown (highlighted) in rows (\ref{tm1}), (\ref{tm11}) and  (\ref{tm13}). Other rows 
show intermediate steps. Each step other than the first is justified by reference to earlier steps.}
\label{internalmonoidtableA}
\setlength{\arrayrulewidth}{1mm}
\setlength{\tabcolsep}{2pt}
\begin{tabular}{l l  c  p{0cm} l  l}
\multicolumn{2}{l}{Derived Rule} &&& Interpretation by $I$ in \catcw & Reason why\\
\hline
\gatinterpretationintro {tm1}{}{\isT{M}}{M \in Cover(1)}{definitions \ref{contextmapping} (ii) and \ref{consistentinterpretation} (i)(a)} \\
\\[-0.1cm]
\gatinterpretationdetail{tm2}{\wM}{\isT{M}}{\doubleM \in Cover(M)}{\highlight{lemma \ref{supplementaryweakeninglemma} (i)} and (\ref{tm1})} \\[0.3cm]
\gatinterpretationdetail{tm4}{\xM}{\isT{M}}{\trebleM \in Cover(\doubleM)}{\highlight{lemma \ref{supplementaryweakeninglemma} (i)}, (\ref{tm2}) and (\ref{tm1}) } \\[0.3cm]
\gatinterpretationintro {tm11}{}{\ofT{unit}{M}}{unit \in Sect(M)}{definition \ref{consistentinterpretation} (ii)(a) and (\ref{tm1})} \\
\\[-0.1cm]
\gatinterpretationintro{tm13}{\xM}{\ofT{\fmult(x_1,x_2)}{M}}{\fmult \in Sect(\trebleM)}{definition \ref{consistentinterpretation} (ii)(a) and (\ref{tm4})} \\
\\[-0.1cm]
\end{tabular}
\end{table}



\begin{table}[H]
\caption{Deriving what constitutes an instance of the theory of monoids $tm$ in a contextual category \catc.\\
Part Two -- Interpretation of the axioms.
The conditions for the axioms to hold are shown (highlighted) in rows (\ref{tmax1}), (\ref{tmax2}) and (\ref{tmax3}).}
\label{internalmonoidtableB}
\setlength{\arrayrulewidth}{1mm}
\setlength{\tabcolsep}{2pt}
\begin{tabular}{l l  c  p{0cm} l  l}
\multicolumn{2}{l}{Derived Rule} &&& Interpretation by $I$ in \catcw & Reason why\\
\hline
%\gatinterpretationintro {tm1}{}{\isT{M}}{M \in Cover(1)}{definitions \ref{contextmapping} (ii) and \ref{consistentinterpretation} (i)(a)} \\
%\\[-0.1cm]
%\gatinterpretationdetail{tm2}{\wM}{\isT{M}}{\doubleM \in Cover(M)}{\highlight{lemma \ref{supplementaryweakeninglemma} (i)} and (\ref{tm1})} \\[0.3cm]
\gatinterpretationdetail{tm3}{\wM}{\ofT{w}{M}}{s(id_M) \in Sect(\doubleM)}{definition \ref{consistentinterpretation} (ii)(d) and (\ref{tm1})} \\[0.3cm]
%\gatinterpretationdetail{tm4}{\xM}{\isT{M}}{\trebleM \in Cover(\doubleM)}{\highlight{lemma \ref{supplementaryweakeninglemma} (i)}, (\ref{tm2}) and (\ref{tm1}) } \\[0.3cm]
% Not USED
%\gatinterpretationdetail{tm5}{\xM}{\ofT{x_1}{M}}{s(p_{M^2}) \in Sect(\trebleM)}{definition \ref{consistentinterpretation} (ii)(d) and (\ref{tm2}) \highlight{USED?}} \\[0.3cm]
%\gatinterpretationdetail{tm6}{\xM}{\ofT{x_2}{M}}{s(id_{M^2}) \in Sect(\trebleM)}{definition \ref{consistentinterpretation} (ii)(d) and (\ref{tm2})\highlight{USED?}} \\[0.3cm]
\gatinterpretationdetail{tm7}{\yM}{\isT{M}}{\quadM \in Cover(\trebleM)}{\highlight{lemma \ref{supplementaryweakeninglemma} (i)}, (\ref{tm1}) and (\ref{tm4})} \\[0.3cm]
\gatinterpretationdetail{tm8}{\yM}{\ofT{y_1}{M}}{\sptrebleone \in Sect(\quadM)}{definition \ref{consistentinterpretation} (ii)(d) and (\ref{tm4})} \\[0.3cm]
\gatinterpretationmapeqv{s(y_1)} 
												{ definition of $y_1$}\\[0.2cm]
\gatinterpretationdetail{tm9}{\yM}{\ofT{y_2}{M}}{\sptrebletwo \in Sect(\quadM)}{definition \ref{consistentinterpretation} (ii)(d) and (\ref{tm4})} \\[0.3cm]
\gatinterpretationmapeqv{s(p_{M^3,M^2} \circ q(p_{M,1},M)) } 
												{by axiom (s3) }\\[0.2cm]
\gatinterpretationmapeqv{ s(y_2)} 
												{definition of $y_2$ }\\[0.2cm]
\gatinterpretationdetail{tm10}{\yM}{\ofT{y_3}{M}}{s(id_{M^3}) \in Sect(\quadM)}{definition \ref{consistentinterpretation} (ii)(d) and (\ref{tm4})} \\[0.3cm]
\gatinterpretationmapeqv{s(q(p_{M^2,1},M)) } 
												{ by axiom (s3)}\\[0.2cm]
\gatinterpretationmapeqv{s(y_3)} 
												{definition of $y_3$ }\\[0.2cm]
%\gatinterpretationintro {tm11}{}{\ofT{unit}{M}}{unit \in Sect(M)}{definition \ref{consistentinterpretation} (ii)(a) and (\ref{tm1})} \\
%\\[-0.1cm]
\gatinterpretationdetail{tm12}{\wM}{\ofT{unit}{M}}{\crossx{M}{unit}{1} \in Sect(\doubleM)}{\highlight{lemma \ref{supplementaryweakeninglemma} (ii)}, (\ref{tm1}) and (\ref{tm11})} \\[0.3cm]\gatinterpretationmapeqv{s(p_M \circ unit)} 
												{by lemma \ref{crosssectionlemma}}\\[0.2cm]
%\gatinterpretationintro{tm13}{\xM}{\ofT{\fmult(x_1,x_2)}{M}}{\fmult \in Sect(\trebleM)}{definition \ref{consistentinterpretation} (ii)(a) and (\ref{tm4})} \\
%\\[-0.1cm]
\gatinterpretationdetail{tm14}{\wM}
                        {\ofT{\fmult(w,unit)}{M}}
                        {\duple{s(id_M),s(p_M \circ unit)}^*\fmult \in Sect(\doubleM)}                   
												{lemma \ref{supplementarylemma} (ii), (\ref{tm3}), (\ref{tm12}) and(\ref{tm13}) }\\[0.2cm]
\gatinterpretationmapeqv{\tuple{id_M,p_M \circ unit}^*\fmult} 
												{lemma \lref{absolutedupletuplelemma}}\\[0.2cm]
\gatinterpretationdetail{tm15}{\wM}
                        {\ofT{\fmult(unit,w)}{M}}
                        {\duple{s(p_M \circ unit),s(id_M)}^*\fmult \in Sect(\doubleM)}
												{lemma \ref{supplementarytuplelemma} (ii), (\ref{tm12}), (\ref{tm3}) and (\ref{tm13}) } \\[0.2cm]
\gatinterpretationmapeqv{\tuple{p_M \circ unit,id_M}^*\fmult}
												{lemma \lref{absolutedupletuplelemma} }\\[0.2cm]
\gatinterpretationdetail{tm16}{\yM}
                        {\ofT{\fmult(y_1,y_2)}{M}}
												{\duple{s(y_1),s(y_2)}^*\fmult}
												{lemma \ref{supplementarylemma} (ii), (\ref{tm8}), (\ref{tm9}) and (\ref{tm13})} \\[0.2cm]
\gatinterpretationmapeqv{\tuple{y_1,y_2}^*\fmult}
												{lemma \lref{absolutedupletuplelemma} (ii)}                                                     \\[0.2cm]
\gatinterpretationdetail{tm17}{\yM}
                        {\ofT{\fmult(y_2,y_3)}{M}}
												{\duple{s(y_2),s(y_3)}^*\fmult}
												{lemma \ref{supplementarylemma} (ii), (\ref{tm9}), (\ref{tm10}) and (\ref{tm13})}  \\[0.2cm]
\gatinterpretationmapeqv{\tuple{y_2,y_3}^*\fmult} 
												{lemma \lref{absolutedupletuplelemma} and axiom (s3)}\\[0.2cm]						
\gatinterpretationdetail{tm18}{\yM}
                        {\fmult(\fmult(y_1,y_2),y_3)}
												{\duple{\tuple{y_1,y_2}^*\fmult,s(y_3)}^*\fmult}
												{lemma \ref{supplementarylemma}, (\ref{tm16}), (\ref{tm10}) and (\ref{tm13})}  \\[0.2cm]
%\gatinterpretationmapeqv{\bigtuple{(\tuple{y_1,y_2}^*\fmult)\circ q(p_{M^3,1},M),y_3}^*\fmult} 
%												{lemma \ref{thedupletuplelemma} and axiom (s3) (twice)} \\[0.2cm]
\gatinterpretationmapeqv{\bigtuple{\duple{\tuple{y_1,y_2}^*\fmult},y_3}^*\fmult} 
												{by definition of $\duple{}$ ??????} \\[0.2cm]
\gatinterpretationmapeqv{\bigtuple{(\tuple{y_1,y_2}^*\fmult)\circ q(p_{M^3,1},M),y_3}^*\fmult}  
												{by definition of $\duple{}$ ????????} \\[0.2cm]
\gatinterpretationdetail{tm19}{\yM}
                        {\fmult(y_1,\fmult(y_2,y_3))}
												{\duple{s(y_1),\tuple{y_2,y_3}^*\fmult}^*\fmult}
												{lemma \ref{supplementarylemma}, (\ref{tm8}), (\ref{tm17}) and (\ref{tm13})} \\[0.2cm]
\gatinterpretationmapeqv{\duple{s(y_1),s(\tuple{y_2,y_3}\circ \fmult)}^*\fmult}
												{by lemma \lref{regardingfstarsection}} \\[0.2cm]
\gatinterpretationmapeqv{\duple{s(y_1),s(\tuple{y_2,y_3}\circ \fmult \circ q(p_{M^2,1},M))}^*\fmult}
												{by (s3)} \\[0.2cm]
\gatinterpretationmapeqv{\bigtuple{y_1,\tuple{y_2,y_3}\circ \fmult \circ q(p_{M^2,1},M)}^*\fmult}
												{by  lemma \lref{absolutedupletuplelemma}} \\[0.2cm]
%\gatinterpretationmapeqv{\bigtuple{y_1, (\tuple{y_2,y_3}^*\fmult) \circ q(p_{M^3,M},M^2) \circ q(p_{M^2,1},M)}^*\fmult}
%												{by definition of $^*$} \\[0.2cm]
\gatinterpretationmapeqv{\bigtuple{y_1,(\tuple{y_2,y_3}^*\fmult)\circ q(p_{M^3,1},M)}^*\fmult} 
												{by definition of $^*$ and (q5)}\\[0.2cm]
\gatinterpretationaxcond{tmax1}{\wM}{\fmult(unit,w)=w}{\tuple{p_M \circ unit,id_M}^*\fmult=s(id_M)}{definition \ref{consistentinterpretation} (iv), (\ref{tm15}) and (\ref{tm3})} \\[0.2cm]
\arrayrulecolor{white}\hline
%\gatinterpretationaxeqv {\tuple{p_M \circ unit,id_M}\comp \duple{\fmult}=id_M}{could push transform back?} \\
%												\rowcolor{lightergrey}
\gatinterpretationaxcond{tmax2}{\wM}{\fmult(w,unit)=w}{\tuple{id_M,p_M \circ unit}^*\fmult=s(id_M)}{definition \ref{consistentinterpretation} (iv), (\ref{tm14}) and (\ref{tm3})} \\[0.2cm]
%\gatinterpretationaxeqv {\tuple{{id_M,p_M \circ unit}}\comp \duple{\fmult}=id_M}{new lemmas}  \\
%												\rowcolor{lightergrey}
\arrayrulecolor{white}\hline
\gatinterpretationaxcond{tmax3}{\yM}{\fmult(\fmult(y_1,y_2),y_3)}
                                     {\bigtuple{(\tuple{y_1,y_2}^*\fmult)\circ q(p_{M^3,1},M),y_3}^*\fmult} \\
																		 &\hspace{2cm}$=\fmult(y_1,\fmult(y_2,y_3))$
																		 &&& \cellcolor{lightergrey}\hspace{0.5cm}
																		    $=\bigtuple{y_1,(\tuple{y_2,y_3}^*\fmult)\circ q(p_{M^3,1},M)}^*\fmult$
																		                           &{definition \ref{consistentinterpretation} (iv), (\ref{tm18}) and (\ref{tm19})} 
\end{tabular}
\end{table}
\end{proof}
\newpage 

\subsection{Internal Categories -- instances of the Theory of Categories within Contextual Categories}
\label{categoriesinstanceexample}
 
\newcommand{\sect}{Sect}
\newcommand{\insect}[2]{#1 \in Sect(#2)}

\newcommand {\OO}{Ob^2}
\newcommand {\OOO}{Ob^3}
\newcommand {\OOOO}{Ob^4}
\newcommand{\HomOb}{\crossx{Hom}{Ob}{1}}
\newcommand{\fid}{\qq{id}}
\newcommand{\fcomp}{\qq{\kern-2pt\circ \kern-2pt}}

\newcommand{\leftidentitylhsterm}{({x_1}^*\qq{id})^*\tuple{x_1,x_1,x_2}^*\fcomp}
\newcommand{\rightidentitylhsterm}{({x_2}^*\qq{id})^*\tuple{x_1,x_2,x_2}^*\fcomp}
\newcommand{\HomHom}{\crossx{Hom}{Hom}{\OO}}

\newcommand {\yOOO}{\ofT{y_1,y_2,y_3}{Ob}}
\newcommand {\yOOOfH}{\yOOO,\,\ofT{f}{Hom(y_1,y_2)}}
\newcommand{\yOOOfHgH}{\yOOOfH,\,\ofT{g}{Hom(y_2,y_3)}}

\newcommand {\yOOOfHmapped}{\tuple{y_1,y_2}^*Hom}
\newcommand {\yOOOfHgHmapped}{\crossx{\yOOOfHmapped}{\tuple{y_2,y_3}^*Hom}{\OOO}}
\newcommand {\yOOOfHgHHmapped}{\crossx{\big(\yOOOfHgHmapped\big)}{{\tuple{y_1,y_3}^*Hom}}{\OOO}}
\newcommand{\gatinterpretationcontext}[1]{&\multicolumn{5}{p{15cm}}{#1}}


%Composition introductory rule
\newcommand{\compdomparent}{\tuple{y_1,y_2}^*Hom}
\newcommand{\compdomain}{\tuple{\dot y_2,\dot y_3}^*Hom}
\newcommand{\compcodomain}{\tuple{\ddot y_1,\ddot y_3}^*Hom}


% Left identity axiom mapping
\newcommand{\leftidentitytuplepp} {\tuple{\dot x_1, \dot x_1, \dot x_2}} 
\newcommand{\leftidentitytuplep} {\tuple{\dot x_1, \dot x_1, \dot x_2, \dot x_1 \circ \fid}} 
\newcommand{\leftidentitytuple} {\tuple{\dot x_1, \dot x_1, \dot x_2, \dot x_1 \circ \fid, id_{Hom}}} 
\newcommand{\leftidentityrhsmapped}{s(id_{Hom})}
\newcommand{\leftidentitylhsremapped}{\leftidentitytuple^*\fcomp}
\newcommand{\leftidentitymapped}{\leftidentitylhsremapped=\leftidentityrhsmapped}

    
\newcommand{\leftidentitylhsmapped}{\duple{s(\dot{x_1}),s(\dot{x_1}),s(\dot x_2),\leftidentityidremapped,s(id_{Hom})}^*\fcomp}


\newcommand{\leftidentityduplepp} {\duple{s(\dot x_1), s(\dot x_1), s(\dot x_2)}} 
\newcommand{\leftidentityduplep} {\duple{s(\dot x_1), s(\dot x_1), s(\dot x_2), s(\dot x_1 \circ \fid)}} 
\newcommand{\leftidentityduple} {\duple{s(\dot x_1), s(\dot x_1), s(\dot x_2), s(\dot x_1 \circ \fid), s(id_{Hom})}} 
% Right identity axiom mapping
\newcommand{\rightidentitymapped}{\tuple{\dot x_1,\dot x_2,\dot x_2,id_{Hom},\dot x_2\circ \fid}^*\fcomp=s(id_{Hom})}

\newcommand{\homdiagram}{
\begin{array}{c}
\Rnode{H}{Hom}                   \\[1.0cm]
\Rnode{OO}{\OO}                  \\[1.0cm]
\Rnode{O}{Ob}                    \\[1.0cm]
\Rnode{abs}{1} 
\end{array} 
\begin{arrows}
\ncsar{H}{OO}
\ncsar{OO}{O}
\ncsar{O}{abs}
\end{arrows}
}

\newcommand{\leftidentitydiagramrhs}{
\begin{array}{c}
\Rnode{TR}{\compcodomain}   \\[1cm]
\Rnode{MR}{\compdomain}     \\[1cm]
\Rnode{LMR}{\compdomparent} \\[1cm]
\Rnode{BR}{\OOO}
\end{array}
\begin{arrows}
%
 \ncsar{TR}{MR}
 \ncsar{MR}{LMR}
 \ncsar{LMR}{BR}
 %
 \ncleftsection{MR}{TR}
 \alabel{\fcomp}
 \end{arrows}
}

\newcommand{\leftidentitydiagramlhs}{
\begin{array}{c }
\leftidentitytuple^*\compcodomain\kern1cm  \\
%=\tuple{x_1,x_2}^*Hom\kern-0.5cm                    \\
%={p_{Hom}}^*Hom\kern0.5cm                  \\                  
\Rnode{HH}{=\crossx{Hom}{Hom}{\OO}} \\[1.3cm]
\homdiagram
\end{array}
\begin{arrows}
\ncsar{HH}{H}
\ncleftcrosssection{H}{HH}
\alabel{\leftidentitymapped}
\end{arrows} 
}

\newcommand{\leftidentitydiagram}{
\begin{array}{c p{1cm} c}
\leftidentitydiagramlhs   && \leftidentitydiagramrhs 
\end{array}
\begin{arrows}
\ncarr{H}{MR}
\alabel{\leftidentitytuple}
\end{arrows}
}





%*****************************
% Associativity axiom mapping
%******************************
\newcommand {\zOOOO}{\ofT{z_1,z_2,z_3,z_4}{Ob}}
\newcommand{\associativitypremisepoppop}
       {\zOOOO,\,\ofT{f}{Hom(z_1,z_2)}}	
\newcommand{\associativitypremisepop}		
			{\associativitypremisepoppop,\,\ofT{g}{Hom(z_2,z_3)}}
\newcommand{\associativitypremise}
       {\associativitypremisepop,\,\ofT{h}{Hom(z_3,z_4)}}	
																											
\newcommand{\associativitypremisepoppopmapped}{\tuple{z_1,z_2}^*Hom}
\newcommand{\associativitypremisepopmapped}{\tuple{\dot z_2,\dot z_3}^*Hom}											
\newcommand{\associativitypremisemapped}{\tuple{\ddot z_3,\ddot z_4}^*Hom}
\newcommand{\Q}{\associativitypremisemapped}
\newcommand{\Qp}{\associativitypremisepopmapped}
\newcommand{\Qpp}{\associativitypremisepoppopmapped}
\newcommand{\assoczimapped}{s(p_{\Q,Ob^i})}
\newcommand{\assoczimappedintermediary}{s(p_{\Q,Ob^i}\circ q(p_{Ob^{i-1},1},Ob))}
\newcommand{\assocziremapped}{{s(\dddot z_i)}}
\newcommand{\assoctripledotzidefiniens}{p_{\Q,\OOOO}\circ z_i}
\newcommand {\assocfmapped}{s(p_{\Q,\Qpp})}
\newcommand {\assocgmapped}{s(p_{\Q,\Qp})}
\newcommand {\assochmapped}{s(id_{\Q})}
\newcommand {\assocfdefiniens}{p_{\Q,\Qpp}\circ q(\tuple{z_1,z_2},Hom)}
\newcommand {\assocgdefiniens}{p_{\Q,\Qp}\circ q(\tuple{\dot z_2, \dot z_3},Hom)}
\newcommand {\assochdefiniens}{q(\tuple{\ddot z_3, \ddot z_4},Hom)}
\newcommand {\assocfmappedintermediary}{s(\assocfdefiniens)}
\newcommand {\assocgmappedintermediary}{s(\assocgdefiniens)}
\newcommand {\assochmappedintermediary}{s(\assochdefiniens)}
\newcommand {\assocfremapped}{s(f)}
\newcommand {\assocgremapped}{s(g)}
\newcommand {\assochremapped}{s(h)}
\newcommand {\associativitylhstype}{\isT{{Hom(z_1,z_4)}}}
\newcommand {\associativitylhstypemapped}{\duple{s(\dddot z_1),s(\dddot z_4)}^*Hom}
\newcommand {\associativitylhstyperemapped}{\tuple{\dddot z_1,\dddot z_4}^*Hom}
\newcommand {\associativitylhstermtyping}{\ofT{(f \circ g) \circ h}{Hom(z_1,z_4)}}
\newcommand {\associativityrhstermtyping}{\ofT{f \circ (g \circ h)}{Hom(z_1,z_4)}}	
\newcommand {\assocfogmapped}{\duple{s(\dddot z_1),s(\dddot z_2),s(\dddot z_3),s(f),s(g)}^*\fcomp }
%\newcommand {\assocfogremapped}{\tuple{\dddot z_1,\dddot z_2,\dddot z_3,f,g}^*\fcomp } %fuller notation
\newcommand {\assocfogremapped}{\tuple{f,g}^*\fcomp}                                    % more abbreviated notation
\newcommand {\assoclhsmapped}{\duple{s(\dddot z_1),s(\dddot z_3),s(\dddot z_4),\assocfogremapped,s(h)}^*\fcomp}
%\newcommand {\assoclhsremapped}{\tuple{\dddot z_1,\dddot z_3,\dddot z_4,(\assocfogremapped) \circ q(\tuple{\dddot z_1,\dddot z_3},Hom),h}^*\fcomp}   % fuller notation

\newcommand {\assoclhsremappingtuple}{\tuple{(\assocfogremapped) \circ q(\tuple{\dddot z_1,\dddot z_3},Hom),h}}
\newcommand {\assoclhsremapped}{\assoclhsremappingtuple^*\fcomp}                                                                                                 % more abbreviated notation
\newcommand {\assocgohmapped}{\duple{s(\dddot z_2),s(\dddot z_3),s(\dddot z_4),s(g),s(h)}^*\fcomp }
%\newcommand {\assocgohremapped}{\tuple{\dddot z_2,\dddot z_3,\dddot z_4,g,h}^*\fcomp } % fuller notation
\newcommand {\assocgohremapped}{\tuple{g,h}^*\fcomp }                                   % more abbreviated notation
\newcommand {\assocrhsmapped}{\duple{s(\dddot z_1),s(\dddot z_2),s(\dddot z_4),s(f),\assocgohremapped}^*\fcomp}
\newcommand {\assocrhsremappingtuple}{\tuple{f,(\assocgohremapped) \circ q(\tuple{\dddot z_2,\dddot z_4},Hom)}}
\newcommand {\assocrhsremapped}{\assocrhsremappingtuple^*\fcomp}

\newcommand{\assocequivalentlhs}{\tuple{\dddot z_1,\dddot z_3,\dddot z_4,\tuple{\dddot z_1,\dddot z_2,\dddot z_3,f,g}\circ \compmorph,h} \circ \compmorph}
\newcommand{\assocequivalentrhs}{\tuple{\dddot z_1,\dddot z_2,\dddot z_4,f,\tuple{\dddot z_2,\dddot z_3,\dddot z_4,g,h}\circ \compmorph} \circ \compmorph}

% remapping
\newcommand{\compmorph}{\text{`$\circ$\kern-2pt'}}%{\odot} %{\llcorner \circ \lrcorner}

% These two should maybe be moved into ccategories shared macros
\newcommand{\ccplaceholder}{\rule[-0.2cm]{0cm}{0.6cm}\kern0.2cm}
\newcommand{\rightend}[1] { \kern-0.2cm\Rnode{#1} {\ccplaceholder} }



\newcommand{\associativitytermdiagramrhs}
{
\begin{array} {cp{1.4cm}c}
 %                       && \Rnode{RTR}{\compcodomain}  \\[0.9cm]
\Rnode{RML}{Hom}        && \Rnode{RMR}{\compdomain} \\[0.9cm]
\Rnode{RBL}{Hom}        && \Rnode{RBR}{\compdomparent} \\[0.9cm]
                        && \Rnode{RVBR}{\OOO}    
\end{array}
\begin{arrows}
% composition
%\ncsar{RTR}{RMR}
\ncsar{RMR}{RBR}
\ncsar{RBR}{RVBR}
\ncarr{RMR}{RML}
\blabel{q(\tuple{\dot y_2,\dot y_3},Hom)}
\ncarr{RBR}{RBL}
\blabel{q(\tuple{y_1,y_2},Hom)}
\end{arrows}
}

% \associativitytermdiagram [angle][width]{first 3 object projection}{Hom projection}{Hom projection}
\newcommandtwoopt{\associativitytermdiagram}[5][30][6cm]
{
\begin{displaymath}
\begin{array}{c p{#2} c}
                                                                                        \\[1.0cm]
\Rnode{L}{\associativitypremisemapped} && \raisebox{-2cm}{$\associativitytermdiagramrhs$} \\
&& \\%[0.25cm]
\end{array}
\begin{arrows}
\ncdarr[#1]{L}{RMR}
\alabel{\tuple{
%#3,
#4,#5}}[0.6]
\ncarr[5]{L}{RML}
\alabel{#5}[0.6]
\ncarr[-5]{L}{RBL}
\blabel{#4}[0.6]
\ncarr[-#1]{L}{RVBR}
\blabel{\tuple{#3}}[0.6]
\end{arrows}
\end{displaymath}
}


In this second example we describe the structure of internal categories by following the main definition and examining
what constitutes an instance of the (generalised algebraic) theory of categories ($tc$) in some \commentary{words accidentally different to monoid example}
 contextual category \catc.

The theory of categories ($tc$) that I work with is presented as follows:
\input{../SharedText/theoryofcategories}

\iffalse
\begin{gatrules}
\gatintros
\gatintroducing{Ob}
\isT{Ob} \\
\gatintroducing{Hom}
  \gatsingular{\ofT{x_1,x_2}{Ob}}{\isT{Hom(x_1,x_2)}} \\	
\gatintroducing{id}
  \gatsingular{\ofT{w}{Ob}}{\ofT{id(w)}{Hom(w,w)}} \\	
\gataxioms
\gatintroducing{  \gataxiomno{1} \\   \gataxiomno{2}}
\begin{gatgroup}{\ofT{f}{Hom(x_1,x_2)},\ \ofT{x_1,x_2}{Ob}}
    \gatleaf{}{id_{x_1} \circ f = f} \\
    \gatleaf{}{f \circ id_{x_2} = f}
\end{gatgroup} \\
\gatintroducing{ \gataxiomno{3} }
\gatsingular{\associativitypremisereversed}{(f \circ g) \circ h = f \circ (g \circ h)} 
\end{gatrules}
\fi

If $I$ is an instance of $tc$ in a contextual category \catcw then the sorts $Ob$ and $Hom$ of $tc$ 
must be mapped by $I$  to objects $I(Ob)$ and  $I(Hom)$ of \catc.
Similarly  the operators symbols
$id$ and $\circ$ must be mapped to sections $I(id)$ and $I(\circ)$ of \catc.

Following the convention described earlier in section \ref{projectionnaming} we simplify  
the description that follows by writing $Ob$ for $I(Ob)$, $Hom$ for $I(Hom)$.
For a further simplification we write $\qq{id}$ for $I(id)$ and   $\qq{\circ}$ for $I(\circ)$.   I will ask the reader  to distinguish for themselves 
those uses of `$Ob$' and `$Hom$' in reference to sorts of the theory $tc$ from those uses in reference to the interpretation of these sorts in the contextual category \catc. 

We will show in lemma \lref{internalcategorylemma} that an instance of the theory $tc$ in a contextual category \catcw, 
i.e. an internal category in \catcw, 
consist of the following:

\begin{itemize}
\item An object $Ob$ of \catc,
\item an object $Hom \in Cover(\OO)$ in \catc,
\item A section $\fid \in Sect(s(id_{Ob})^*Hom)$ in \catc, 								
\end{itemize}
plus a section $\fcomp$, whose codomain we are now going to describe, and such that
 such that the left and right identity axioms and the associativity axiom hold.

\subsection{codomain of $\fcomp$}
In regard to the object $Ob^3$ in \catcw, for $i=1,2,3$, as described in section \ref{projectionnaming}, define,
 $y_i: \OOO \morph Ob$ to be the i'th projection morphism. 
This enables us to consider the following pullback
\begin{equation*}
\begin{array}{r  p{4cm} c}
\compdomparent     \rightend{Qpp} && \Rnode{Hom}{Hom}               \\ [1cm]
\OOO          \rightend{O3}  && \Rnode{O2}{Ob^2}              
\end{array}
\mbox{
\ncsar{Qpp}{O3}
\ncsar{Hom}{O2}
\ncarr{Qpp}{Hom}
\alabel{q(\tuple{y_1,y_2},Hom)}
\ncarr{O3}{O2}
\alabel{\tuple{y_1,y_2}}}
\end{equation*}														

Now define $\dot y_i : \compdomparent \morph Ob$, for $i = 1,2,3$, 
           by $\dot y_i = p_{\compdomparent}\circ y_i$	and consider the pullback:
\begin{equation*}
\begin{array}{r  p{4cm} c}
\compdomain     \rightend{Qpp} && \Rnode{Hom}{Hom}               \\ [1cm]
\compdomparent     \rightend{O3}  && \Rnode{O2}{Ob^2}              
\end{array}
\mbox{
\ncsar{Qpp}{O3}
\ncsar{Hom}{O2}
\ncarr{Qpp}{Hom}
\alabel{q(\tuple{\dot y_2,\dot y_3},Hom)}
\ncarr{O3}{O2}
\alabel{\tuple{\dot y_2,\dot y_3}}}
\end{equation*}	

Finally, define   $\ddot y_i : \compdomain \morph Ob$, for $i = 1,2,3$, 
                                     by $\ddot y_i = p_{\compdomain}\circ \dot y_i$																	
and consider  the pullback
\begin{equation*}
\begin{array}{r  p{4cm} c}
\compcodomain     \rightend{Qpp} && \Rnode{Hom}{Hom}               \\ [1cm]
\compdomain     \rightend{O3}  && \Rnode{O2}{Ob^2}              
\end{array}
\mbox{
\ncsar{Qpp}{O3}
\ncsar{Hom}{O2}
\ncarr{Qpp}{Hom}
\alabel{q(\tuple{\ddot y_1,\ddot y_3},Hom)}
\ncarr{O3}{O2}
\alabel{\tuple{\ddot y_1,\ddot y_3}}}
\end{equation*}	

We will show  that an internal category in a contextual category \catcw consists of
\begin{itemize}
\item An object $Ob$ of \catc,
\item an object $Hom \in Cover(\OO)$ in \catc,
\item A section $\fid \in Sect(s(id_{Ob})^*Hom)$ in \catc, 
\item A section $\fcomp \in Sect(\compcodomain) $ of \catc 											
\end{itemize}
such that the left and right identity axioms and the associativity axiom hold. We turn our attention to the representation of these axioms.

\subsubsection*{Left Identity Axiom}
Suppose that we have such objects $Ob$ and $Hom$ and sections $id$ and $\fcomp$ in a contextual category \catc, as described above. 
In regard to the object $\OO$, for $i=1,2$, define $x_i$ to the the i'th projection function.
In addition we will define $\dot x_1,\dot x_2:Hom \morph Ob$ by
$\dot x_i = p_{Hom} \circ x_i$.

By lemma \lref{absolutedupletuplelemma} we have for any $i$ and $j$, $1 \leq i,j \leq 2$,
\begin{equation}
\label{dupletupledotxtwo}
\tuple{\dot{x_i},\dot{x_j}} = \duple{s(\dot{x_i}),s(\dot{x_j})} 
\end{equation}
By application of the same lemma  we have that for any $i,j$ and $k$, $1 \leq i,j,k \leq 3$,
\begin{equation}
\label{dupletupledotxthree}
\tuple{\dot x_i, \dot x_j, \dot x_k} = \duple{s(\dot x_i), s(\dot x_j), s(\dot x_k)}
\end{equation}


 %known as (zz)
We define $\leftidentitytuplep$
to be the unique morphism $\leftidentitytuplep :  Hom \morph \compdomparent$
such that 
 \begin{equation}
 \leftidentitytuplep \circ p_{\compdomparent} = \leftidentitytuplepp  
\end{equation}
 and 
\begin{equation}
\leftidentitytuplep \circ q(\tuple{y_1,y_2},Hom) = \dot x_1 \circ \fid
\end{equation}
as shown here
\begin{displaymath}
\begin{coneoutline}{1cm}{0.5cm}{Hom}
\ccprimitivepullbacksquare{3cm}{1.2cm}{Hom}{\OOO}{\OO}{\tuple{y_1,y_2}}
\end{coneoutline}
%\conearrowsdefiningtuple{xxx}{yyy}
\conearrowsdefiningtuple{\leftidentitytuplepp}{\dot x_1 \circ \fid}
\end{displaymath}
By  lemma \lref{thegeneraldupletuplelemma} and by use of (\ref{dupletupledotxthree}) it follows that 
\begin{equation}
\label{leftidentitydupletuplepidentity}
\leftidentitytuplep = \leftidentityduplep.
\end{equation}


Next, we define $\leftidentitytuple$
to be the unique morphism $\leftidentitytuple :  Hom \morph \compdomain$
such that 
 \begin{equation}
 \leftidentitytuple \circ p_{\compdomain} = \leftidentitytuplep %changed from pp ending
\end{equation}
 and 
\begin{equation}
\leftidentitytuple \circ q(\tuple{\dot y_2,\dot y_3},Hom) = id_{Hom}
\end{equation}
\begin{displaymath}
\begin{coneoutline}{1cm}{0.5cm}{Hom}
\ccprimitivepullbacksquare{3cm}{1.2cm}{Hom}{\OOO}{\OO}{\tuple{\dot y_2,\dot y_3}}
\end{coneoutline}
\conearrowsdefiningtuple{\leftidentitytuplep}{id_{Hom}}
\end{displaymath}
By  lemma \lref{thegeneraldupletuplelemma} and from (\ref{leftidentitydupletuplepidentity}) it follows that 
\begin{equation}
\label{leftidentitylhsremappingequation}
\leftidentitytuple = \leftidentityduple.
\end{equation}


Since the codomain of $\leftidentitytuple$ is the 
domain of the section $\fcomp$  then $\leftidentitytuple^*\fcomp$
is defined. We will show that the left identity axiom is satisfied iff
\begin{equation}
\leftidentitymapped
\end{equation}
in \catcw as shown in this diagram

\begin{displaymath}
\leftidentitydiagram
\end{displaymath}

\subsection*{Associativity Axiom}				
Next we turn to consideration of $Ob^4$.
Following the earlier convention we define the projection functions 
to be $z_1,z_2,z_3$ and $z_4$ so that for $i = 1, 2,3,4$, $z_i: \OOOO \morph Ob$. \\

Then we proceed to define   $\dot z_i : \associativitypremisepoppopmapped \morph Ob$
                                      by $\dot z_i = p_{\associativitypremisepoppopmapped}\circ z_i$, 
to define  $\ddot z_i : \associativitypremisepopmapped \morph Ob$ 
                                    by $\ddot z_i = p_{\associativitypremisepopmapped, \OOOO}\circ z_i$, 
and, finally, to define $\dddot z_i : \associativitypremisemapped \morph Ob$ 
                                      by $\ddot z_i = p_{\associativitypremisemapped, \OOOO}\circ z_i$ 	
so that for $i = 1, 2,3,4$ we have
\begin{equation*}
\begin{array}{r l p{4cm} c}
\associativitypremisemapped       \rightend{Q}  & \kern-0.2cm\rightend{Qright}                          \\ [1cm]
\associativitypremisepopmapped    \rightend{Qp} &  &&   \\ [1cm]
\associativitypremisepoppopmapped \rightend{Qpp}&  &&   \\ [1cm]
\OOOO                             \rightend{O4} & && \Rnode{Ob}{Ob}              
\end{array}
\mbox{
\ncsar{Q}{Qp}
\ncsar{Qp}{Qpp}
\ncsar{Qpp}{O4}
%\ncarr{Q}{Ob}
\ncarc[nodesepA=5pt,nodesepB=\arrnodesepB,offsetA=\arroffsetA,offsetB=\arroffsetB,arrowsize=5pt,arrowinset=0.7]{->}{Q}{Ob}
\alabel{\dddot z_i}
\ncarr{Qp}{Ob}
\alabel{\ddot z_i}
\ncarr{Qpp}{Ob}
\alabel{\dot z_i}
\ncarr{O4}{Ob}
\alabel{z_i}
}
\end{equation*} in \catcw.

The object $\associativitypremisemapped$ is relevant to us because we will show that it is the interpretation under $I$ of the premise
$\associativitypremise$ of the associativity axiom. This helps explain the next three definitions.
We define $f$, $g$ and $h$ by
\begin{align*}
f &= \assocfdefiniens, \\
g &= \assocgdefiniens, \\
h &= \assochdefiniens.
\end{align*}

With $f$, $g$ ad $h$ so defined, the following diagrams
\vspace{0.3cm}
\begin{equation*}
\ccsquareoutline{1.4cm}{1.2cm}{\associativitypremisemapped}{Hom}{\OOOO}{\OO}
\mbox{
\nccdar{TL}{BL}
\ncsar{TR}{BR}
\ccsquareacross{f}{\tuple{z_1, z_2}}
\kern -1cm %work around bug with lost arrow space bug
}
\ccsquareoutline{1.4cm}{1.2cm}{\associativitypremisemapped}{Hom}{\OOOO}{\OO}
\mbox{
\nccdar{TL}{BL}
\ncsar{TR}{BR}
\ccsquareacross{g}{\tuple{z_2, z_3}}
\kern -1cm %work around bug with lost arrow space bug
}
\ccsquareoutline{1.4cm}{1.2cm}{\associativitypremisemapped}{Hom}{\OOOO}{\OO}
\mbox{
\nccdar{TL}{BL}
\ncsar{TR}{BR}
\ccsquareacross{h}{\tuple{z_3, z_4}}
}
\end{equation*} commute in \catc.	\\

\newcommand{\treblez}[3]{\dddot z_#1, \dddot z_#2,\dddot z_#3}
\newcommand{\trebledottreblez}{\tuple{\treblez{1}{2}{3}}}
\begin{newtt} 
We define $\trebledottreblez$ to be the unique morphism $\associativitypremisemapped \morph \morph \OOO$ such that 
\begin{equation}
\trebledottreblez \circ y_1 = \dddot z_1,
\end{equation}
\begin{equation}
\trebledottreblez \circ y_2 = \dddot z_2
\end{equation}
and
\begin{equation}
\trebledottreblez \circ y_3 = \dddot z_3.
\end{equation}
By application of Lemma \lref{absolutedupletuplelemma} we have that

\begin{equation}
\label{zonezfourdupletuplelemma}
\tuple{\dddot z_1,\ddot z_4} = \duple{s(\dddot z_1),s(\dddot z_4)}
\end{equation}
and similarly that
\begin{equation}
\label{trebledottreblezdupletupleidentity}
\trebledottreblez = \duple{s(\dddot z_1),s(\dddot z_2),s(\dddot z_3)}
\end{equation}

\end{newtt}


\newcommand{\trebledzf}{\tuple{\dddot z_1,\dddot z_2, \dddot z_3, f}}
\begin{newtt} 
We define $\trebledzf$
to be the unique morphism $\trebledzf :  \associativitypremisemapped \morph \compdomparent$
such that 
 \begin{equation}
 \trebledzf \circ p_{\compdomparent} = \trebledottreblez 
\end{equation}
 and 
\begin{equation}
\trebledzf \circ q(\tuple{y_1,y_2},Hom) = f
\end{equation}
as shown here
\begin{displaymath}
\begin{coneoutline}{1cm}{0.5cm}{Hom}
\ccprimitivepullbacksquare{3cm}{1.2cm}{Hom}{\OOO}{\OO}{\tuple{y_1,y_2}}
\end{coneoutline}
\conearrowsdefiningtuple{\dddot z_1,\dddot z_2, \dddot z_3}{f}
\end{displaymath}
By  Lemma \lref{absolutedupletuplelemma} we have 
\begin{equation}
\label{somethingorother}
\trebledzf = \duple{s(\dddot z_1),s(\dddot z_2), s(\dddot z_3), s(f)}.
\end{equation}
\end{newtt}

\begin{newtt} % editing to $\tuple{f,g}$
\newcommand{\tuplefglongform}{\tuple{\dddot z_1,\dddot z_2, \dddot z_3, f,g}}
We define $\tuplefglongform$. which subsequently we abbreviate as $\tuple{f,g}$
to be the unique morphism $\tuplefglongform :  \associativitypremisemapped \morph \compdomain$
such that 
 \begin{equation}
 \tuplefglongform \circ p_{\compdomparent} = \trebledzf 
\end{equation}
 and 
\begin{equation}
\tuplefglongform \circ q(\tuple{y_1,y_2},Hom) = g
\end{equation}
as shown here
\begin{displaymath}
\begin{coneoutline}{1cm}{0.5cm}{Hom}
\ccprimitivepullbacksquare{3cm}{1.2cm}{Hom}{\OOO}{\OO}{\tuple{y_1,y_2}}
\end{coneoutline}
\conearrowsdefiningtuple{\dddot z_1,\dddot z_2, \dddot z_3, f}{g}
\end{displaymath}
We use  $\tuple{f,g}$ as a shortened form of $\tuplefglongform$so that by  Lemma \lref{absolutedupletuplelemma} we have 
\begin{equation}
\label{pairfgdupletuplepidentity}
\tuple{f,g} = \duple{s(\dddot z_1), s(\dddot z_2), s(\dddot z_3),  s(f), s(g)}.
\end{equation}
\end{newtt}

With  $\tuple{f,g}$ so defined the nested diagrams within
% <f,g> definition

\associativitytermdiagram[25][4cm]{\dddot z_1, \dddot z_2,\dddot z_3}{f}{g}

commute.


% <g,h> definition
We  define  $\tuple{\treblez{2}{3}{4}}$ and
$\tuple{g,h}:\associativitypremisemapped \morph \compdomain$ in a similar manner so that nested diagrams in
\associativitytermdiagram[25][4cm]{\dddot z_1, \dddot z_2,\dddot z_3}{g}{h}
commute. With $\tuple{g,h}$ so defined by lemma \lref{absolutedupletuplelemma} we have 
\begin{equation}
\label{pairghdupletuplepidentity}
\tuple{g,h} = \duple{s(\dddot z_2), s(\dddot z_3), s(\dddot z_4),  s(g), s(h)}.
\end{equation}

% associativity lhs definition
We can  define a morphism $\assoclhsremappingtuple:\associativitypremisemapped \morph \compdomain$ so that
nested diagrams in 
\associativitytermdiagram{\dddot z_1, \dddot z_3,\dddot z_4}{(\assocfogremapped) \circ q(\tuple{\dddot z_1,\dddot z_3},Hom)}{h}
commute.
We use this in describing the mapping of the  left hand side of the associativity axiom.

% associativity rhs definition
Repeating the whole exercise we can  define a morphism $\assoclhsremappingtuple$ so that nested diagrams in
\associativitytermdiagram{\dddot z_1, \dddot z_2,\dddot z_4}{f}{(\assocgohremapped) \circ q(\tuple{\dddot z_2,\dddot z_4},Hom)}
commute.


\begin{lemma}
\llabel{associativitycontextmapping}
If $I$ is an interpretation of the theory $tc$ in a contextual category \catcw then
$I$ maps the context $\tuple{\associativitypremise}$ to the object $\associativitypremisemapped$ in \catc.
\end{lemma}
\begin{proof}
% two width forcing commands
\newcommand {\forceSOURCEwidth}{\rule{5cm}{0pt}}  % so as to line up three different arrays
\newcommand {\forceTARGETwidth}{\rule{2.2cm}{0pt}}

From lemma \ref{Xnlemma} we have the following interpretation by $I$, for each $i$, $1 \leq i \leq 4$:
\begin{equation*}
\begin{array}{c c c}
\forceSOURCEwidth & & \forceTARGETwidth \\ [-0.1cm]
\gatdisplayrule{\zOOOO}{\ofT{z_i}{Ob}} & \Imapsto & s(z_i) 
\end{array}
\end{equation*}

From these mappings it follows by lemma \ref{supplementarytuplelemma} that 

\begin{equation*}
\begin{array}{c c c}
\forceSOURCEwidth & & \forceTARGETwidth \\ [-0.1cm]
\gatdisplayrule{\zOOOO}{\isT{Hom(z_1,z_2)}} & \Imapsto & \associativitypremisepoppopmapped 
\end{array}
\end{equation*}

The context $\associativitypremisepoppop$ is therefore mapped to $\associativitypremisepoppopmapped$.

Now it follows by lemma \ref{Xnlemma} that 

\begin{equation*}
\begin{array}{c c c}
\forceSOURCEwidth & & \forceTARGETwidth \\ [-0.1cm]
\gatdisplayrule{\associativitypremisepoppop}{\ofT{z_i}{Ob}} & \Imapsto & s(p_{\associativitypremisepoppopmapped} \circ z_i) \\
                                                            & = & s(\dot{z_i})
\end{array}
\end{equation*}
and therefore by  lemma \ref{supplementarytuplelemma} that 
\begin{equation*}
\begin{array}{c c c}
\forceSOURCEwidth & & \forceTARGETwidth \\ [-0.1cm]
\gatdisplayrule{\associativitypremisepoppop}{\isT{Hom(z_2,z_3)}} & \Imapsto & \associativitypremisepopmapped.
\end{array}
\end{equation*}

The context $\associativitypremisepop$ is therefore mapped to $\associativitypremisepopmapped$.

By application of lemma \ref{Xnlemma} again we establish that 
\begin{equation*}
\begin{array}{c c c}
\forceSOURCEwidth & & \forceTARGETwidth \\ [-0.1cm]
\gatdisplayrule{\associativitypremisepop}{\ofT{z_i}{Ob}} & \Imapsto & s(p_{\associativitypremisepopmapped} \circ z_i) \\
                                                            & = & s(\ddot{z_i})
\end{array}
\end{equation*}
and therefore by  lemma \ref{supplementarytuplelemma} that 
\begin{equation*}
\begin{array}{c c c}
\forceSOURCEwidth & & \forceTARGETwidth \\ [-0.1cm]
\gatdisplayrule{\associativitypremisepop}{\isT{Hom(z_3,z_4)}} & \Imapsto & \associativitypremisemapped.
\end{array}
\end{equation*}
The context $\associativitypremise$ is therefore mapped to $\associativitypremisemapped$.
\end{proof}

\begin{lemma}
\llabel{internalcategorylemma}
An internal category in a contextual category \catcw consists of
\begin{itemize}
\item An object $Ob$ of \catc,
\item an object $Hom \in Cover(\OO)$ in \catc,
\item A section $\fid \in Sect(s(id_{Ob})^*Hom)$ in \catc, 
\item A section $\fcomp \in Sect(\compcodomain) $ of \catc 											
\end{itemize}
such that
\begin{equation}
\label{leftidentityaxiom}
\leftidentitymapped
\end{equation}
\begin{equation}
\label{rightidentityaxiom}
\rightidentitymapped
\end{equation}
and
\begin{multline}
\label{associativityaxiom}
\assoclhsremapped = \assocrhsremapped
\end{multline}


Equivalently an internal category in a contextual category \catcw consists of
\item objects $Ob$ and  $Hom$  and a section $\fid$ in \catc,  as above, along with
\begin{itemize}
\item a morphism $\compmorph$ of \catc, $\compmorph: \compdomain \morph Hom$ in \catc
\end{itemize}
such that
\begin{equation}
\label{leftidentityrepresentation2}
\tuple{\dot x_1,\dot \dot x_1,\dot x_2,\dot x_1\circ \fid,id_{Hom}} \circ \compmorph =id_{Hom}
\end{equation}
\begin{equation}
\label{rightidentityrepresentation2}
\tuple{\dot x_1,\dot x_2,\dot x_2,id_{Hom},\dot x_2 \circ \fid} \circ \compmorph =id_{Hom}
\end{equation}
and
\begin{equation}
\label{associativityrepresentation2}
\assocequivalentlhs = \assocequivalentrhs
\end{equation}.
\end{lemma}
\begin{proof}
Of these two equivalent representations the first results from a literal reading of the definition of instance given earlier
along with the judicious choice of intermediate definitions made with readability in mind.
This is demonstrated in tables \ref{internalcategorytableone}  - \ref{internalcategorytablefour} below. 

To show that the second representation follows from the first then from $\fcomp$ define $\compmorph$ by defining $\compmorph=\fcomp \circ q(p_{\tuple{\ddot y_1,\ddot y_3}},Hom)$ and then it is easy to show that 
(\ref{leftidentityrepresentation2}) follows from (\ref{leftidentityaxiom}), 
(\ref{rightidentityrepresentation2}) follows from (\ref{rightidentityaxiom}) and
(\ref{associativityrepresentation2}) follows from (\ref{associativityaxiom})

Vice-versa, from the second representation follows the first if we define define $\fcomp$ from  $\compmorph$ by defining $\fcomp=s(\compmorph)$.

\begin{table}[H]
\caption{Deriving what constitutes an intepretation of the theory of categories $tc$ in a contextual category \catc.
Part One - Introductory rules for $Ob$, $Hom$ and $id$.
}
\label{internalcategorytableone}
%\setlength{\arrayrulewidth}{1mm}
\setlength{\tabcolsep}{2pt}
\begin{tabular}{l l  c  p{0cm} l  l}
\multicolumn{2}{l}{Derived Rule} &&& Interpretation by $I$ in \catcw & Reason why\\
\hline
\gatinterpretationintro {obintro}{}{\isT{Ob}}{Ob \in Cover(1)}{definition \ref{consistentinterpretation} (i)(a)}                                   \\
\gatinterpretationdetail{homintrohelper}{\ofT{x_1}{Ob}}{\isT{Ob}}{Ob^2 \in Cover(Ob)}
                                                               {\highlight{lemma \ref{supplementaryweakeninglemma} (i)} and (\ref{obintro})}             \\
\gatinterpretationintro {homintro}{\ofT{x_1}{Ob},\ofT{x_2}{Ob}}{\isT{Hom}}{Hom \in Cover(Ob^2)}
                                                               {definition \ref{consistentinterpretation} (i)(a) and (\ref{homintrohelper})}      \\
\gatinterpretationdetail{idintrohelperhelper}{\ofT{w}{Ob}}{\ofT{w}{Ob}}{s(id_{Ob})}
                                                               {definition \ref{consistentinterpretation} (ii)(d) and (\ref{homintrohelper})}  \\
\gatinterpretationdetail{idintrohelper}{\ofT{w}{Ob}}
                                 {\isT{Hom(w,w)}}{s(id_{Ob})^*Hom }
                                 {\highlight{lemma \ref{supplementarylemma} (i) CHECK} (\ref{homintro}) and (\ref{idintrohelperhelper})}           \\
\gatinterpretationintro {idintro}{\ofT{w}{Ob}}{\ofT{id(w)}{Hom(w,w)}} 
                                 {\fid \in Sect(s(id_{Ob})^*Hom) }
                                 {definition \ref{consistentinterpretation} (ii)(a) and (\ref{idintrohelper})}                                      \\
\end{tabular}
\end{table}


\begin{table}[H]
\caption{Deriving what constitutes an intepretation of the theory of categories $tc$ in a contextual category \catc.
Part Two Introductory rule for $\circ$. Indication of the reasoning is not included due to lack of space. 
The reasoning follows the patterns indicated in accompanying tables \ref{internalcategorytableone} and \ref{internalcategorytablethree}.
}
\label{internalcategorytabletwo}
%\setlength{\arrayrulewidth}{1mm}
\setlength{\tabcolsep}{2pt}
\begin{tabular}{l l  c  p{0cm} l  l}
\multicolumn{2}{l}{Derived Rule} &&& Interpretation by $I$ in \catcw \\
\hline
\gatinterpretationdetail{comp1}{\ofT{x_1,x_2}{Ob}}{\isT{Ob}}{ \OOO \in Cover(\OO) }{}              \\
\gatinterpretationdetail{comp2}{\ofT{y_1,y_2,y_3}{Ob}}{\isT{Hom(y_1,y_2)}}{ \compdomparent \in Cover(\OOO) }{} \\
\gatinterpretationdetail{comp3}{\ofT{y_1,y_2,y_3}{Ob}, \ofT{f_1}{Hom(y_1,y_2)}}{\isT{Hom(y_2,y_3)}}
                                                                          {  \compdomain \in Cover(\compdomparent) }{} \\
\gatinterpretationdetail{comp4}{\ofT{y_1,y_2,y_3}{Ob}, \ofT{f}{Hom(y_1,y_2)},\ofT{g}{Hom(y_2,y_3)}} {\isT{Hom(y_1,y_3)}}
                                                                         { \compcodomain \in Cover(\compdomain) }{} \\
\gatinterpretationintro {compintro}	{\ofT{y_1,y_2,y_3}{Ob}, \ofT{f}{Hom(y_1,y_2)},\ofT{g}{Hom(y_2,y_3)}} 
                                    {\ofT{f \circ g}{Hom(y_1,y_3)}}
																    {\fcomp \in Sect(\compcodomain)}
\end{tabular}
\end{table}

\newcommand{\leftidentityidremapped}{s(\dot{x_1}\circ \fid)} 

\begin{table}[H]
\caption{Deriving what constitutes an intepretation of the theory of categories $tc$ in a contextual category \catc.
Part Three. The left identity axiom.
}
\label{internalcategorytablethree}
%\setlength{\arrayrulewidth}{1mm}
\setlength{\tabcolsep}{2pt}
\begin{tabular}{l l  c  p{0cm} l  l}
\gatinterpretationcontext{Let $P$ be the context $\ofT{x_1}{Ob},\,\ofT{x_2}{Ob},\,\ofT{f}{Hom(x_1,x_2)} $
                                 then from (\ref{homintro}) we have $P \mapsto Hom \in Cover(Ob^2)$.} \\
\hline
\multicolumn{2}{l}{Derived Rule} &&& Interpretation by $I$ in \catcw & Reason why\\
\hline
\gatinterpretationdetail{rightidentity1}{P}{\isT{Ob}}{ \HomOb \in Cover(Hom) }{lemma \ref{supplementaryweakeninglemma} (i), (\ref{homintro}) and (\ref{obintro})}              \\
\gatinterpretationdetail{rightidentity2}{P}{\ofT{x_1}{Ob}}{ s(p_{Hom,Ob}) \in Section(\HomOb) }{definition \ref{consistentinterpretation} (ii)(d)}                    \\
\gatinterpretationmapeqv        {s(\dot{x_1})}                                            {defn. of $\dot{x_1}$}             \\
\gatinterpretationdetail{rightidentity3}{P}{\ofT{x_2}{Ob}}{ s(p_{Hom,Ob^2}) \in Section(\HomOb) }{definition \ref{consistentinterpretation} (ii)(d)}                  \\
\gatinterpretationmapeqv        {s(\dot{x_2})}                                            {defn. of $\dot{x_2}$}             \\
\gatinterpretationdetail{rightidentity4}{P}{\isT{Hom(x_1,x_1)}}{\duple{s(\dot{x_1}),s(\dot{x_1})}^*Hom \in Cover(Hom)} 
                                                             {lemma \ref{supplementarylemma} (ii), (\ref{homintro}) and (\ref{rightidentity2})} \\
\gatinterpretationmapeqv       {\tuple{\dot{x_1},\dot{x_1}}^*Hom}                                      {by (\ref{dupletupledotxtwo})}     \\
\gatinterpretationdetail{rightidentityidmapping}{P}{\ofT{id(x_1)}{Hom(x_1,x_1)}}{\duple{s(\dot{x_1})}^*\fid \in Sect(\tuple{\dot{x_1},\dot{x_1}}^*Hom)}  
                                                             {lemma \ref{supplementarylemma} (ii), (\ref{idintro}) and (\ref{rightidentity2})} \\
\gatinterpretationmapeqv       {\dot{x_1}^*\fid}                                      {by (d1)}     \\
\gatinterpretationmapeqv       {\leftidentityidremapped}                              {lemma \ref{regardingfstarsection}}     \\
\gatinterpretationdetail{rightidentityrhsmappping}{P}{\ofT{f}{Hom(x_1,x_2)}}{\leftidentityrhsmapped \in Sect(\HomHom) }{definition \ref{consistentinterpretation} (ii)(d)}                         \\
\gatinterpretationdetail{rightidentitylhsmapping}{P}{id(x_1) \circ f} {\leftidentitylhsmapped     }
                                     {lemma \ref{supplementarylemma} (ii), (\ref{rightidentity2}), (\ref{rightidentity3}) and (\ref{rightidentityidmapping})} \\
				&\hspace{1.2cm}$\ofT{}{Hom(x_1,x_2)}$&&&\hspace{3.5cm}$\in Sect(\HomHom)$&   \\
\gatinterpretationmapeqv   {\leftidentitylhsremapped} { by (\ref{leftidentitylhsremappingequation})}      \\        
\gatinterpretationaxcond{tcaxiomone}{P}{id(x_1) \circ f = f}
                                       {\leftidentitylhsremapped=\leftidentityrhsmapped}
                                       {definition \ref{consistentinterpretation} (iv), (\ref{rightidentitylhsmapping}) and (\ref{rightidentityrhsmappping})}    
\end{tabular}
\end{table}




\begin{table}[H]
\caption{Deriving what constitutes an intepretation of the theory of categories $tc$ in a contextual category \catc.
Part Four. Associativity axiom.
}
\label{internalcategorytablefour}
%\setlength{\arrayrulewidth}{1mm}
\setlength{\tabcolsep}{2pt}
\begin{tabular}{l l  c  p{0cm} l  l}
\gatinterpretationcontext{Let $Q$ be the context $\associativitypremise$} \\
\gatinterpretationcontext{then $Q \mapsto \associativitypremisemapped \in Cover(\associativitypremisepopmapped)$ in \catcw by lemma \ref{associativitycontextmapping}.}\\
\hline

\multicolumn{2}{l}{Derived Rule} &&& Interpretation by $I$ in \catcw & Reason why                   \\
\hline \\[-0.4cm]
\gatinterpretationdetail{assoczimapping}{Q}{\ofT{z_i}{Ob},\mbox{ for } i=1,2,3,4}{\assoczimapped}{definition \ref{consistentinterpretation} (ii)(d)}   \\[0.2cm]
\gatinterpretationmapeqv          {\assoczimappedintermediary}                   {axiom (s3)}                \\[0.2cm]
\gatinterpretationmapeqv          {\assocziremapped}                   {by defn. of $\dddot z_i$}  \\[0.2cm]
\gatinterpretationdetail{assocfmapping}{Q}{\ofT{f}{Hom(z_1,z_2)}}{\assocfmapped}{definition \ref{consistentinterpretation} (ii)(d)}             \\[0.2cm]
\gatinterpretationmapeqv          {\assocfmappedintermediary}                   {axiom (s3)}     \\[0.2cm]
\gatinterpretationmapeqv          {\assocfremapped}                             { by defn. of $f$}      \\[0.2cm]
\gatinterpretationdetail{assocgmapping}{Q}{\ofT{g}{Hom(z_2,z_s)}}{\assocgmapped}{definition \ref{consistentinterpretation} (ii)(d)}              \\[0.2cm]
\gatinterpretationmapeqv                                  {\assocgmappedintermediary} {axiom (s3)}      \\[0.2cm]
\gatinterpretationmapeqv          {\assocgremapped}                             { be defn. of $g$}      \\[0.2cm]
\gatinterpretationdetail{assochmapping}{Q}{\ofT{h}{Hom(z_3,z_4)}}{\assochmapped}{definition \ref{consistentinterpretation} (ii)(d)}               \\[0.2cm]
\gatinterpretationmapeqv                                  {\assochmappedintermediary}  {axiom (s3)}     \\[0.2cm]
\gatinterpretationmapeqv          {\assochremapped}                             { be defn. of $h$}      \\[0.2cm] 
\gatinterpretationdetail{assocfgmapping}{Q}{\ofT{f \circ g}{Hom(z_1,z_3)}}
                                   { \assocfogmapped  }   {lemma \ref{supplementarylemma}, (\ref{assocfmapping}) and (\ref{assocgmapping})}         \\[0.2cm]
\gatinterpretationmapeqv                    {\tuple{f,g}^*\fcomp}{by (\ref{pairfgdupletuplepidentity})}                                                 \\[0.2cm]         
\gatinterpretationdetail{assoctypemapping}{Q}{\associativitylhstype}{\associativitylhstypemapped}
                                                                   {lemma \ref{supplementarylemma} (i), (\ref{homintro}) and (\ref{assoczimapping})}    \\[0.2cm]  
\gatinterpretationmapeqv                     {\associativitylhstyperemapped}{by (\ref{zonezfourdupletuplelemma})}             \\[0.2cm]   
\gatinterpretationdetail{assocLHSmapping}{Q}{\associativitylhstermtyping}{\assoclhsmapped}
                                            {lemma \ref{supplementarylemma}, (\ref{assocfgmapping}) and (\ref{assochmapping})}\\[0.2cm]
\gatinterpretationmapeqv                    {\assoclhsremapped}{lemma \ref{thedupletuplelemma} and  (s3)}\\[0.2cm]
\gatinterpretationdetail{assocghmapping}{Q}{\ofT{g \circ h}{Hom(z_2,z_4)}}
                                   { \assocgohmapped  }{lemma \ref{supplementarylemma} (ii), (\ref{assoczimapping}), (\ref{assocgmapping})}      \\[0.2cm]
\gatinterpretationdetailcontinuation{}{\hspace{2.2cm} and (\ref{assochmapping})}                                                   \\[0.2cm]
\gatinterpretationmapeqv           {\assocgohremapped}{by (\ref{pairghdupletuplepidentity})} \\[0.2cm]
\gatinterpretationdetail{assocRHSmapping}{Q}{\associativityrhstermtyping}
                                            {\assocrhsmapped \iffalse{\in Sect(\associativitylhstyperemapped)}\fi}
			                       {lemma \ref{supplementarylemma}, (\ref{assocfmapping}) and (\ref{assocghmapping})} \\ [0.2cm]
\gatinterpretationmapeqv                    {\assocrhsremapped}{lemma \ref{thedupletuplelemma} and (s3)}\\[0.2cm]
\gatinterpretationaxcond{associativity}{Q}{(f \circ g) \circ h = f \circ (g \circ h)}
                                     { \assoclhsremapped  } {definition \ref{consistentinterpretation} (iv), (\ref{assocLHSmapping})} \\
\gatinterpretationaxcondrhscontinuation{= \assocrhsremapped } { and  (\ref{assocRHSmapping})}\\
\end{tabular}
\end{table}

%\ncarc[arcangle=#1,nodesepA=5pt,nodesepB=5pt,offsetA=#2pt,offsetB=#2pt,arrowsize=5pt,arrowinset=0.7]{->}{#3}{#4}
\iffalse
\begin{equation*}
\begin{array}{c}
\begin{array}{r c p{4cm} c}
\associativitypremisemapped        \rightend{Q}            \\ [1cm]
\associativitypremisepopmapped     \rightend{Qp}           \\ [1cm]
\associativitypremisepoppopmapped  \rightend{Qpp}          \\ [1cm]
\OOOO   \rightend{O4}                                   \\ [1cm]
\OOO    \rightend{O3}      &  & & \Rnode{H}{Hom}            \\ [1cm]
\OO     \rightend{O2}      & & & \Rnode{Hp}{\OO}           \\ [1cm]
Ob      \rightend{O}       & & & \Rnode{Hpp}{Ob}           \\ [1cm]
\Rnode{abs}{1}        \\ 
\end{array} \\
\begin{arrows}
\ncsar{Q}{Qp}
\ncsar{Qp}{Qpp}
\ncsar{Qpp}{O4}
\ncsar{O4}{O3}
\ncsar{O3}{O2}
\ncsar{O2}{O}   
\ncsar{O}{abs}
\ncsar{H}{Hp}
\ncsar{Hp}{Hpp}
\ncsar{Hpp}{abs}
\ncarrNEGZZ[-10]{Q}{H}    \alabel{f}
\ncarrZ{Q}{H}             \alabel{g}
\ncarrZZ[10]{Q}{H}        \alabel{h}
\ncarrNEGZZ[-10]{O4}{Hpp} \alabel{z_1}
\ncarrZ{O4}{Hpp}          \alabel{z_2}
\ncarrZZ[10]{O4}{Hpp}     \alabel{z_3}
\ncarrZZZ[20]{O4}{Hpp}    \alabel{z_4}
\end{arrows}
\end{array}
\end{equation*}
\fi

\end{proof}




%\fi

%unused
%\section{Unused}
%


\note The Inititiality Conjecture is described in \cite{VoevodskyInitialityConjecture}.
\begin{tightquote}
A C-system equipped with additional
operations corresponding to the inference rules of a type theory is called a
model or a C-system model of these rules or of this type theory.
\end{tightquote}
and
\begin{tightquote}
The model whose underlying
C-system is the term C-system is called the term model... for a particular
class of inference rules the term model is an initial object in the category of models.
This is known as the Inititiality Conjecture.
\end{tightquote} 
\ \\
\note Whereas from nLab (\url{https://ncatlab.org/nlab/show/Initiality+Project}) I read
\begin{tightquote}
The Initiality Project is a communal effort to prove an initiality theorem for a dependent type theory: that the categorical structure constructed out of the syntax is the initial object in some category of structured categorical objects.
\end{tightquote}

\note From my 1986 paper (\cite{Cartmell86})based on my thesis (\cite{Cartmell78}):
\begin{tightquote}
An algebraic semantics is witnessed by an equivalence between a category of
theories and a category of structures. In most instances of algebraic semantics
there is a further equivalence in that the usual definition of model of a theory can
be replaced by a definition which uses only the notion of structure. Lawvere has
used the term Functorial Semantics in describing this kind of semantics.
Functorial semantics depends on an equivalence between the category of models
of a theory U and the category of structure preserving morphisms from the
structure $C(U)$ corresponding to $U$ to a special canonical structure (the world
structure?). In the case of algebraic theories (Lawvere \cite{LawvereAlgebraicTheories}) the canonical structure
is taken to be the category of sets Set while in the case of classical proposition
theories the canonical structure is taken to be the Boolean Algebra $\set{0, 1}$.
The present situation is as well-behaved as any if the canonical structure is
taken to be the contextual category $Fam$.
If U is a generalised algebraic theory, then the category of models of U is
equivalent to the category which has contextual functors $C(U)$ to $Fam$ as objects
and natural transformations as morphisms. Thus we can assert
\begin{equation*}
\Ualg \cong ConFunc(C(U), Fam).
\end{equation*}
The inductive construction of $C(U)$ from $U$ has enabled us to replace the usual 
inductive definition of model of $U$ by the definition "a model of $U$ is a contextual
functor $M: C(U): \morph Fam$".
\end{tightquote}

\begin{oldtt}
\note The definition of $\Ihat$ from $\Isort$ and $\Iop$ whenever $I$ is an interpretation proceeds by induction 
on the derivation of rules in  \gatUw 
as described in the principles of derivation in Definition 2(b) of \cite{Cartmell86}. 
The only non-trivial parts of this definition relate to the rules
identified as CF1, CF2(a) and CF2(b)\footnote{So identified, by the way, as a mnemonic for cut-free.}. We consider each of these rules in turn.

\begin{point}
Rule CF1 states that for $n \geq 0$, for $1 \leq i \leq n+1$, from the derived rule 
$\frac{\xDelta{n}}{\isT{\Delta_{n+1}}}$ which we shall denote $R$ 
we may derive the rule
$\frac{\xDelta{n+1}}{\ofT{x_i}{\Delta_i}}$ which, in turn, we shall denote $R_{x_i}$.
Define $\Ihat(R_{x_i}) :  \Ihat(R) \morph \crossx{\Ihat(R)}{\Ihat(R_i)}{\Ihat(R_{i-1})}$
to be $\tuple{p_{\Ihat(R),\Ihat(R_{i-1})},p_{\Ihat(R),\Ihat(R_{i})}}$. 

This presumably is $s(p_{\Ihat(R),\Ihat(R_{i-1})})$ where $s$ is Vladimir's s-operator.
\end{point}
\begin{point}
CF2(a) states that if $A$ is a sort symbol introduced by
$\frac{\xDelta{n}}{\isT{A(\xn)}}$ 
and if $P$ is a context and $\tn$ are expressions then from the following rules, which we shall denote $R_{t_1}$,..$R_{t_n}$,
$\frac{P}{\ofT{t_1}{\Delta_1}}$,
$\frac{P}{\ofT{t_2}{\Delta_2[t_1|x_1]}}$,
... and 
$\frac{P}{\ofT{t_n}{\Delta_n[t_1|x_1,...t_{n-1}|x_{n-1}]}}$
we may derive the rule
$\frac{P}{\isT{A(t_1,...t_n)}}$ which we denote as $R$. 
Define $I(R)$ to be $\Ihat(R_n)^*...\Ihat(R_1)^*\crossx{\Ihat(R_n)}{I(A)}{1}$.\commentary{check this}
\end{point}
\begin{point}
CF2(b) \highlight{fill this in}
\end{point}
\end{oldtt}

\begin{oldtt}
\begin{displaymath}
\begin{array}{c}
\crossx{a_n}{a_i}{\Rnode{cross}{a_{i-1}}} \\[0.9cm]
\Rnode{an}{a_n}\\[0.7cm]
%\Rnode{highervdots}{\vdots}\\
\Rnode{ai}{\begin{array}{c}
\vdots\\
a_i\\
\vdots
\end{array}} \\[1.1cm]
%\Rnode{lowervdots}{\vdots}\\[0.4cm]
\Rnode{a1}{a_1}\\[0.7cm]
\Rnode{abs}{1}
\end{array}
\end{displaymath}
\ncsar{cross}{an}
%\ncsar{an}{highervdots}
%\ncsar{lowervdots}{a1}
\ncsar{an}{ai}
\ncsar{ai}{a1}
\ncsar{a1}{abs}
\ncarc[arcangle=30,nodesepA=5pt,offsetA=2pt,nodesepB=2pt,offsetB=2pt]{->}{an}{cross}
\alabel{s(p_{a_n,a_i})}
\end{oldtt}
\begin{oldtt}
\note 3
\label{omegarealisationwrtQ}
 Suppose also that $\encyOmega{m}$ is a context of generalised algebraic theory \gatUw and suppose that $Q$ is some other context and that for some $m \geq 1$,
 \foreachj, \gatdisplayrule{Q}{\ofT{s_j}{\Omega_j[s_1|y_1,...s_{j-1}|y_{j-1}]}} is a derived rule of \gatU\footnote{Recall that such an m-tuple $\tuple{\sm}$ is said to be a realisation of 
$\encyOmega{m}$ wrt $Q$.}.  Suppose that an interpretation $I$ of \gatUw in a contextual category \catcw maps the context $Q$ to an object $a$  of $\catc$ and maps
the context $\encyOmega{j}$ to an object $b_j$ of \catc, \foreachj, so that $1 \base b_1 ... \base b_m$ in \catc. In this situation the rule 
\gatdisplayrule{Q}{\ofT{s_1}{\Omega_1}} will be mapped by $I$ to some section $f_1:a \morph \crossx{a}{b_1}{1}$. The j'th rule,
\gatdisplayrule{Q}{\ofT{s_j}{\Omega_j[s_1|y_1,...s_{j-1}|y_{j-1}]}}, will be mapped to a section $f_j:a \morph \fjpstar ... \fonestar\crossx{a}{b_j}{1}$.
In the case that $m=3$  then in \catcw we will have objects and morphisms as follows:
\begin{displaymath}
\begin{array}{c p{1cm} c p {1cm} c  p{1cm} c}
                                                &&                                           && \Rnode{ab3}{\crossx{a}{b_3}{1}}                       \\[1.2cm]
                                                &&\Rnode{f1axb3}{\fonestar\crossx{a}{b_3}{1}}  && \Rnode{ab2}{\crossx{a}{b_2}{1}}                       \\[1.2cm]
 \Rnode{f3target}{\ftwostar\fonestar\crossx{a}{b_3}{1}} &&\Rnode{f2target}{\fonestar\crossx{a}{b_2}{1}}  && \Rnode{ab1}{\crossx{a}{b_1}{\Rnode{f1target}{1}}}     \\[1.2cm]
                                                &&\Rnode{a}{a}                               &&                                                       \\[-3.0cm]
																								&&                                           &&                         && \Rnode{b3}{b_3}             \\[1.2cm]
																								&&                                           &&                         && \Rnode{b2}{b_2}             \\[1.2cm]
																								&&                                           &&                         && \Rnode{b1}{b_1}             \\[1.1cm]
																								&&                                           && \Rnode{abs}{1} \ \ \ \ \ \ \ \ &&    
\end{array}
\end{displaymath}
\ncarr{ab3}{b3}
\ncarr{ab2}{b2}
\ncarr{ab1}{b1}
\ncarr{f1axb3}{ab3}
\ncarr{f2target}{ab2}
\ncarr{f3target}{f1axb3}
\ncarc[arcangle=10,nodesepA=5pt,offsetA=2pt,nodesepB=2pt,offsetB=2pt]{->}{a}{f1target}
\alabel{f_1}[0.25]
\ncarc[arcangle=15,nodesepA=5pt,offsetA=2pt,nodesepB=2pt,offsetB=2pt]{->}{a}{f2target}
\alabel{f_2}
\ncarc[arcangle=10,nodesepA=5pt,offsetA=2pt,nodesepB=2pt,offsetB=2pt]{->}{a}{f3target}
\alabel{f_3}
\ncsar{f3target}{a}
\ncsar{f2target}{a}
\ncsar{f1target}{a}
\ncsar{ab2}{ab1}
\ncsar{ab3}{ab2}
\ncsar{f1axb3}{f2target}
\ncsar{b3}{b2}
\ncsar{b2}{b1}
\ncsar{b1}{abs}
\nccdar{a}{abs}

\note 
How suppose that additional to the situation of para. \ref{omegarealisationwrtQ} the rules 
\gatdisplayrule{\yOmega{m}}{\isT{\Delta}} and  \gatdisplayrule{\yOmega{m}}{\ofT{t}{\Delta}} are derived rules of \gatUw. 
Let us denote these rules $r$ and $r_t$, respectively. Suppose that $I$ is an interpretation of $\gatUw$ in \catcw as mapping rules and contexts as described in para. \ref{omegarealisationwrtQ}.
From what we have said in para \ref{omegarealisationwrtQ}, the interpretation  $I$ will map rule $r$ to an object $b$ such that ${b_m \base b}$ in \catcw and it will map the rule $r_t$ to a section $g:b_m \morph b$.  

Now it follows by the substitution lemma (see \cite{Cartmell86})
that the substituted $r$ and $r_t$ rules: 
\gatdisplayrule{Q}{\isT{\Delta[s_1|y_1...s_m|y_m]}} 
and  \gatdisplayrule{Q}{\ofT{t[s_1|y_1...s_m|y_m]}{\Delta[s_1|y_1...s_m|y_m]}} are derived rules of \gatU. 

\highlight{We require that}
the substituted $r$ rule will be mapped by $I$ to the object $\smstar...\sonestar\crossx{a}{b}{1}$ and the substituted $r_t$ rule will
be mapped by $I$ to the morphism  $\smstar...\sonestar\crossx{a}{g}{1}$ (which is defined since $g$ is a section).
\end{oldtt}

\begin{oldtt}

\note There is a large 2-category $\catofccs$ of contextual categories, contextual functors and natural transformations. \\

\note
If \isagat[U] then $\CofU$ is a contextual category. 
$\CofU$ is the structured assembly of contexts and realisations of $\gat[U]$.
There is an interpretation $I_0$ of theory $\gat[U]$ in contextual category
$\CofU$\footnote{
In the context of a type theory I think that this is what Vladimir refers to as the term model though it is safer to think of this as the abstract-syntax model to distinguish it from another model to be described later. It is confusing because both this model and the later model are initial in some category.}.\\

\note If $F : \ccat[C] \morph \ccat[C']$ is a contextual functor then for any $\gat[U]$, 
$F$ induces a mapping of interpretations of $\gat[U]$ in $\ccat[C]$ to interpretations of $\gat[U]$ in $\ccat[C']$. Denote this mapping $\phi_F$. \\

\note
If \isagat[U] then a $\gat[U]$-algebra $A$ is a contextual functor $A: \CofU \morph \Fam$. \\

\note 
Let $tcc$ denote the gat of contextual categories. Then
\begin{point}
to every contextual category $\ccat[C]$ there corresponds a \tccalgebra 
which we shall denote $\alg{C}$  i.e. there is a contextual functor $\alg{C} :\ccat[C](tcc) \morph \Fam$,
\end{point}
\begin{point}
to every a \tccalgebra i.e. to every contextual functor $A :\ccat[C](tcc) \morph \Fam$ there is a contextual category $\cc{A}$
\end{point}
\begin{point}
for all contextual categories $\ccat[C]$,
\begin{equation}
\cc{\alg{\ccat[C]}}= \ccat[C],
\end{equation}
\end{point}
\begin{point}
for all \tccalgebras $A$,
\begin{equation}
\alg{\cc{A}} = A.
\end{equation}
\end{point}

\note
From the previous it follows 
\begin{pointeq}
\label{cualg}
for every gat \gat[U], to the contextual category $\CofU$ corresponds a contextual functor
   $\alg{\CofU} :\ccat[C](tcc) \morph \Fam$. \\
\end{pointeq} 

\note Particularising (\ref{cualg}) to the theory $tcc$ it follows that
\begin{pointeq}
  $\alg{\ccat[C](tcc)}$ is a contextual functor   $\alg{\ccat[C](tcc)} :\ccat[C](tcc) \morph \Fam$.
\end{pointeq}

\note
Let $\Fam$ be the (large) contextual category of sets, indexed families of sets, indexed families of families of sets and so on and
let $\FAM$ be the (larger still) contextual category of large sets, indexed families of large sets, indexed families of families of large sets and so on.
Particularising (\ref{cualg}) to the category $\Fam$ we have
\begin{pointeq}
  \label{inducedalgebra}
  $\alg{\Fam}$ is a contextual functor   $\alg{Fam} :\ccat[C](tcc) \morph \FAM$. 
\end{pointeq}
 
\note
Aside: Assume now that $\catofccs$ is the category of large contextual categories so that $\Fam$ is an object of $\catofccs$. 
Let $\catoflargerccs$ be the category of larger contextual categories. \\



\note Suppose we add a sum type to generalised algebraic theories so that from
types $\Delta$ and $\Delta'$ in context $Delta_n$ we can construct a type $\Delta + \Delta'$
along with inclusion operations and that we can construct $t | t'$. 

\note Suppose we have a gat+ $U$. Then can we construct a term model which is a contextual category?
In the term model is $[\Delta + \Delta']$ actually the coproduct of $[\Delta]$ and $[\
Delta'] $
\end{oldtt}

\begin{oldtt}
\note 
\begin{lemmastar}
\label{finiteinterpretationlemma}
If $F$ is a finite generalised algebraic theory and if $P$ is a preinterpretation of $F$ in a contextual category \catcw then there is at most one interpretation $I$ of $F$ in \catcw that is consistent with $P$.
\end{lemmastar}
\begin{proof}
By induction on the theory $F$. True for the empty theory. Then provably true as sort symbols and operator symbols are added.
\end{proof} 
\begin{lemmastar}
If \gatUw is any finite generalised algebraic theory and if $P$ is a preinterpretation of \gatUw in a contextual category \catcw then there is at most one interpretation $I$ of \gatUw in \catcw that is consistent with $P$.
\end{lemmastar}
\begin{proof}
Use lemma \ref{finiteinterpretationlemma} as follows.
Suppose $I$ and $I'$ are interpretations of \gatUw in \catc.
We aim to show that for all derived T-rules or $\in$-rules $r$ of \gatU, $I(r)=I'(r)$.
Suppose then that $r$ is a derived T-rule or $\in$-rule of \gatU. Since $r$ is a derived rule of \gatUw then by lemma \lref{stratification lemma} it is a derived rule of some finite subtheory $F \subseteq \gatU$. 
Now $I \restriction F$ and $I' \restriction F$ are interpretations
of $F$ that both extend the preinterpretation $P \restriction F$. Therefore $I \restriction F$ = $I' \restriction F$
from which we may derive $I(r) = (I \restriction F) (r) = (I' \restriction F)(r) = I'(r)$, as required. 
\end{proof}
\end{oldtt}


%\bibliographystyle{alpha} 
\bibliographystyle{abbrv}
\bibliography{../SharedBibliography/temp/bibliography}
\end{document}
