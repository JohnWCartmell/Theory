 
\newcommand{\sect}{Sect}
\newcommand{\insect}[2]{#1 \in Sect(#2)}

\newcommand {\OO}{Ob^2}
\newcommand {\OOO}{Ob^3}
\newcommand {\OOOO}{Ob^4}
\newcommand{\HomOb}{\crossx{Hom}{Ob}{1}}
\newcommand{\fid}{\qq{id}}
\newcommand{\fcomp}{\qq{\kern-2pt\circ \kern-2pt}}

\newcommand{\leftidentitylhsterm}{({x_1}^*\qq{id})^*\tuple{x_1,x_1,x_2}^*\fcomp}
\newcommand{\rightidentitylhsterm}{({x_2}^*\qq{id})^*\tuple{x_1,x_2,x_2}^*\fcomp}
\newcommand{\HomHom}{\crossx{Hom}{Hom}{\OO}}

\newcommand {\yOOO}{\ofT{y_1,y_2,y_3}{Ob}}
\newcommand {\yOOOfH}{\yOOO,\,\ofT{f}{Hom(y_1,y_2)}}
\newcommand{\yOOOfHgH}{\yOOOfH,\,\ofT{g}{Hom(y_2,y_3)}}

\newcommand {\yOOOfHmapped}{\tuple{y_1,y_2}^*Hom}
\newcommand {\yOOOfHgHmapped}{\crossx{\yOOOfHmapped}{\tuple{y_2,y_3}^*Hom}{\OOO}}
\newcommand {\yOOOfHgHHmapped}{\crossx{\big(\yOOOfHgHmapped\big)}{{\tuple{y_1,y_3}^*Hom}}{\OOO}}
\newcommand{\gatinterpretationcontext}[1]{&\multicolumn{5}{p{15cm}}{#1}}


%Composition introductory rule
\newcommand{\compdomparent}{\tuple{y_1,y_2}^*Hom}
\newcommand{\compdomain}{\tuple{\dot y_2,\dot y_3}^*Hom}
\newcommand{\compcodomain}{\tuple{\ddot y_1,\ddot y_3}^*Hom}


% Left identity axiom mapping
\newcommand{\leftidentitytuplepp} {\tuple{\dot x_1, \dot x_1, \dot x_2}} 
\newcommand{\leftidentitytuplep} {\tuple{\dot x_1, \dot x_1, \dot x_2, \dot x_1 \circ \fid}} 
\newcommand{\leftidentitytuple} {\tuple{\dot x_1, \dot x_1, \dot x_2, \dot x_1 \circ \fid, id_{Hom}}} 
\newcommand{\leftidentityrhsmapped}{s(id_{Hom})}
\newcommand{\leftidentitylhsremapped}{\leftidentitytuple^*\fcomp}
\newcommand{\leftidentitymapped}{\leftidentitylhsremapped=\leftidentityrhsmapped}

    
\newcommand{\leftidentitylhsmapped}{\duple{s(\dot{x_1}),s(\dot{x_1}),s(\dot x_2),\leftidentityidremapped,s(id_{Hom})}^*\fcomp}


\newcommand{\leftidentityduplepp} {\duple{s(\dot x_1), s(\dot x_1), s(\dot x_2)}} 
\newcommand{\leftidentityduplep} {\duple{s(\dot x_1), s(\dot x_1), s(\dot x_2), s(\dot x_1 \circ \fid)}} 
\newcommand{\leftidentityduple} {\duple{s(\dot x_1), s(\dot x_1), s(\dot x_2), s(\dot x_1 \circ \fid), s(id_{Hom})}} 
% Right identity axiom mapping
\newcommand{\rightidentitymapped}{\tuple{\dot x_1,\dot x_2,\dot x_2,id_{Hom},\dot x_2\circ \fid}^*\fcomp=s(id_{Hom})}

\newcommand{\homdiagram}{
\begin{array}{c}
\Rnode{H}{Hom}                   \\[1.0cm]
\Rnode{OO}{\OO}                  \\[1.0cm]
\Rnode{O}{Ob}                    \\[1.0cm]
\Rnode{abs}{1} 
\end{array} 
\begin{arrows}
\ncsar{H}{OO}
\ncsar{OO}{O}
\ncsar{O}{abs}
\end{arrows}
}

\newcommand{\leftidentitydiagramrhs}{
\begin{array}{c}
\Rnode{TR}{\compcodomain}   \\[1cm]
\Rnode{MR}{\compdomain}     \\[1cm]
\Rnode{LMR}{\compdomparent} \\[1cm]
\Rnode{BR}{\OOO}
\end{array}
\begin{arrows}
%
 \ncsar{TR}{MR}
 \ncsar{MR}{LMR}
 \ncsar{LMR}{BR}
 %
 \ncleftsection{MR}{TR}
 \alabel{\fcomp}
 \end{arrows}
}

\newcommand{\leftidentitydiagramlhs}{
\begin{array}{c }
\leftidentitytuple^*\compcodomain\kern1cm  \\
%=\tuple{x_1,x_2}^*Hom\kern-0.5cm                    \\
%={p_{Hom}}^*Hom\kern0.5cm                  \\                  
\Rnode{HH}{=\crossx{Hom}{Hom}{\OO}} \\[1.3cm]
\homdiagram
\end{array}
\begin{arrows}
\ncsar{HH}{H}
\ncleftcrosssection{H}{HH}
\alabel{\leftidentitymapped}
\end{arrows} 
}

\newcommand{\leftidentitydiagram}{
\begin{array}{c p{1cm} c}
\leftidentitydiagramlhs   && \leftidentitydiagramrhs 
\end{array}
\begin{arrows}
\ncarr{H}{MR}
\alabel{\leftidentitytuple}
\end{arrows}
}





%*****************************
% Associativity axiom mapping
%******************************
\newcommand {\zOOOO}{\ofT{z_1,z_2,z_3,z_4}{Ob}}
\newcommand{\associativitypremisepoppop}
       {\zOOOO,\,\ofT{f}{Hom(z_1,z_2)}}	
\newcommand{\associativitypremisepop}		
			{\associativitypremisepoppop,\,\ofT{g}{Hom(z_2,z_3)}}
\newcommand{\associativitypremise}
       {\associativitypremisepop,\,\ofT{h}{Hom(z_3,z_4)}}	
																											
\newcommand{\associativitypremisepoppopmapped}{\tuple{z_1,z_2}^*Hom}
\newcommand{\associativitypremisepopmapped}{\tuple{\dot z_2,\dot z_3}^*Hom}											
\newcommand{\associativitypremisemapped}{\tuple{\ddot z_3,\ddot z_4}^*Hom}
\newcommand{\Q}{\associativitypremisemapped}
\newcommand{\Qp}{\associativitypremisepopmapped}
\newcommand{\Qpp}{\associativitypremisepoppopmapped}
\newcommand{\assoczimapped}{s(p_{\Q,Ob^i})}
\newcommand{\assoczimappedintermediary}{s(p_{\Q,Ob^i}\circ q(p_{Ob^{i-1},1},Ob))}
\newcommand{\assocziremapped}{{s(\dddot z_i)}}
\newcommand{\assoctripledotzidefiniens}{p_{\Q,\OOOO}\circ z_i}
\newcommand {\assocfmapped}{s(p_{\Q,\Qpp})}
\newcommand {\assocgmapped}{s(p_{\Q,\Qp})}
\newcommand {\assochmapped}{s(id_{\Q})}
\newcommand {\assocfdefiniens}{p_{\Q,\Qpp}\circ q(\tuple{z_1,z_2},Hom)}
\newcommand {\assocgdefiniens}{p_{\Q,\Qp}\circ q(\tuple{\dot z_2, \dot z_3},Hom)}
\newcommand {\assochdefiniens}{q(\tuple{\ddot z_3, \ddot z_4},Hom)}
\newcommand {\assocfmappedintermediary}{s(\assocfdefiniens)}
\newcommand {\assocgmappedintermediary}{s(\assocgdefiniens)}
\newcommand {\assochmappedintermediary}{s(\assochdefiniens)}
\newcommand {\assocfremapped}{s(f)}
\newcommand {\assocgremapped}{s(g)}
\newcommand {\assochremapped}{s(h)}
\newcommand {\associativitylhstype}{\isT{{Hom(z_1,z_4)}}}
\newcommand {\associativitylhstypemapped}{\duple{s(\dddot z_1),s(\dddot z_4)}^*Hom}
\newcommand {\associativitylhstyperemapped}{\tuple{\dddot z_1,\dddot z_4}^*Hom}
\newcommand {\associativitylhstermtyping}{\ofT{(f \circ g) \circ h}{Hom(z_1,z_4)}}
\newcommand {\associativityrhstermtyping}{\ofT{f \circ (g \circ h)}{Hom(z_1,z_4)}}	
\newcommand {\assocfogmapped}{\duple{s(\dddot z_1),s(\dddot z_2),s(\dddot z_3),s(f),s(g)}^*\fcomp }
%\newcommand {\assocfogremapped}{\tuple{\dddot z_1,\dddot z_2,\dddot z_3,f,g}^*\fcomp } %fuller notation
\newcommand {\assocfogremapped}{\tuple{f,g}^*\fcomp}                                    % more abbreviated notation
\newcommand {\assoclhsmapped}{\duple{s(\dddot z_1),s(\dddot z_3),s(\dddot z_4),\assocfogremapped,s(h)}^*\fcomp}
%\newcommand {\assoclhsremapped}{\tuple{\dddot z_1,\dddot z_3,\dddot z_4,(\assocfogremapped) \circ q(\tuple{\dddot z_1,\dddot z_3},Hom),h}^*\fcomp}   % fuller notation

\newcommand {\assoclhsremappingtuple}{\tuple{(\assocfogremapped) \circ q(\tuple{\dddot z_1,\dddot z_3},Hom),h}}
\newcommand {\assoclhsremapped}{\assoclhsremappingtuple^*\fcomp}                                                                                                 % more abbreviated notation
\newcommand {\assocgohmapped}{\duple{s(\dddot z_2),s(\dddot z_3),s(\dddot z_4),s(g),s(h)}^*\fcomp }
%\newcommand {\assocgohremapped}{\tuple{\dddot z_2,\dddot z_3,\dddot z_4,g,h}^*\fcomp } % fuller notation
\newcommand {\assocgohremapped}{\tuple{g,h}^*\fcomp }                                   % more abbreviated notation
\newcommand {\assocrhsmapped}{\duple{s(\dddot z_1),s(\dddot z_2),s(\dddot z_4),s(f),\assocgohremapped}^*\fcomp}
\newcommand {\assocrhsremappingtuple}{\tuple{f,(\assocgohremapped) \circ q(\tuple{\dddot z_2,\dddot z_4},Hom)}}
\newcommand {\assocrhsremapped}{\assocrhsremappingtuple^*\fcomp}

\newcommand{\assocequivalentlhs}{\tuple{\dddot z_1,\dddot z_3,\dddot z_4,\tuple{\dddot z_1,\dddot z_2,\dddot z_3,f,g}\circ \compmorph,h} \circ \compmorph}
\newcommand{\assocequivalentrhs}{\tuple{\dddot z_1,\dddot z_2,\dddot z_4,f,\tuple{\dddot z_2,\dddot z_3,\dddot z_4,g,h}\circ \compmorph} \circ \compmorph}

% remapping
\newcommand{\compmorph}{\text{`$\circ$\kern-2pt'}}%{\odot} %{\llcorner \circ \lrcorner}

% These two should maybe be moved into ccategories shared macros
\newcommand{\ccplaceholder}{\rule[-0.2cm]{0cm}{0.6cm}\kern0.2cm}
\newcommand{\rightend}[1] { \kern-0.2cm\Rnode{#1} {\ccplaceholder} }



\newcommand{\associativitytermdiagramrhs}
{
\begin{array} {cp{1.4cm}c}
 %                       && \Rnode{RTR}{\compcodomain}  \\[0.9cm]
\Rnode{RML}{Hom}        && \Rnode{RMR}{\compdomain} \\[0.9cm]
\Rnode{RBL}{Hom}        && \Rnode{RBR}{\compdomparent} \\[0.9cm]
                        && \Rnode{RVBR}{\OOO}    
\end{array}
\begin{arrows}
% composition
%\ncsar{RTR}{RMR}
\ncsar{RMR}{RBR}
\ncsar{RBR}{RVBR}
\ncarr{RMR}{RML}
\blabel{q(\tuple{\dot y_2,\dot y_3},Hom)}
\ncarr{RBR}{RBL}
\blabel{q(\tuple{y_1,y_2},Hom)}
\end{arrows}
}

% \associativitytermdiagram [angle][width]{first 3 object projection}{Hom projection}{Hom projection}
\newcommandtwoopt{\associativitytermdiagram}[5][30][6cm]
{
\begin{displaymath}
\begin{array}{c p{#2} c}
                                                                                        \\[1.0cm]
\Rnode{L}{\associativitypremisemapped} && \raisebox{-2cm}{$\associativitytermdiagramrhs$} \\
&& \\%[0.25cm]
\end{array}
\begin{arrows}
\ncdarr[#1]{L}{RMR}
\alabel{\tuple{
%#3,
#4,#5}}[0.6]
\ncarr[5]{L}{RML}
\alabel{#5}[0.6]
\ncarr[-5]{L}{RBL}
\blabel{#4}[0.6]
\ncarr[-#1]{L}{RVBR}
\blabel{\tuple{#3}}[0.6]
\end{arrows}
\end{displaymath}
}


In this second example we describe the structure of internal categories by following the main definition and examining
what constitutes an instance of the (generalised algebraic) theory of categories ($tc$) in some
 contextual category \catc.

The theory of categories ($tc$) that I work with is presented as follows:
\newcommand{\associativitypremisereversed}
       {\ofT{f}{Hom(z_1,z_2)},\,\ofT{g}{Hom(z_2,z_3)},\,\ofT{h}{Hom(z_3,z_4)},\,
                      \ofT{z_1,z_2,z_3,z_4}{Ob}
       }
\begin{gatrules}
\gatintros
\gatintroducing{Ob}
\isT{Ob} \\
\gatintroducing{Hom}
  \gatsingular{\ofT{x_1,x_2}{Ob}}{\isT{Hom(x_1,x_2)}} \\	
\gatintroducing{id}
  \gatsingular{\ofT{w}{Ob}}{\ofT{id_w}{Hom(w,w)}} \\	
\gataxioms
\gatintroducing{  \gataxiomno{1} \\   \gataxiomno{2}}
\begin{gatgroup}{\ofT{f}{Hom(x_1,x_2)},\ \ofT{x_1,x_2}{Ob}}
    \gatleaf{}{id_{x_1} \circ f = f} \\
    \gatleaf{}{f \circ id_{x_2} = f}
\end{gatgroup} \\
\gatintroducing{ \gataxiomno{3} }
\gatsingular{\associativitypremisereversed}{(f \circ g) \circ h = f \circ (g \circ h)} 
\end{gatrules}

\iffalse
\begin{gatrules}
\gatintros
\gatintroducing{Ob}
\isT{Ob} \\
\gatintroducing{Hom}
  \gatsingular{\ofT{x_1,x_2}{Ob}}{\isT{Hom(x_1,x_2)}} \\	
\gatintroducing{id}
  \gatsingular{\ofT{w}{Ob}}{\ofT{id(w)}{Hom(w,w)}} \\	
\gataxioms
\gatintroducing{  \gataxiomno{1} \\   \gataxiomno{2}}
\begin{gatgroup}{\ofT{f}{Hom(x_1,x_2)},\ \ofT{x_1,x_2}{Ob}}
    \gatleaf{}{id_{x_1} \circ f = f} \\
    \gatleaf{}{f \circ id_{x_2} = f}
\end{gatgroup} \\
\gatintroducing{ \gataxiomno{3} }
\gatsingular{\associativitypremisereversed}{(f \circ g) \circ h = f \circ (g \circ h)} 
\end{gatrules}
\fi

If $I$ is an instance of $tc$ in a contextual category \catcw then the sorts $Ob$ and $Hom$ of $tc$ 
must be mapped by $I$  to objects $I(Ob)$ and  $I(Hom)$ of \catc.
Similarly  the operators symbols
$id$ and $\circ$ must be mapped to sections $I(id)$ and $I(\circ)$ of \catc.

Following the convention described earlier in section \ref{projectionnaming} we simplify  
the description that follows by writing $Ob$ for $I(Ob)$, $Hom$ for $I(Hom)$.
For a further simplification we write $\qq{id}$ for $I(id)$ and   $\qq{\circ}$ for $I(\circ)$.   I will ask the reader  to distinguish for themselves 
those uses of `$Ob$' and `$Hom$' in reference to sorts of the theory $tc$ from those uses in reference to the interpretation of these sorts in the contextual category \catc. 

We will show in lemma \lref{internalcategorylemma} that an instance of the theory $tc$ in a contextual category \catcw, 
i.e. an internal category in \catcw, 
consist of the following:

\begin{itemize}
\item An object $Ob$ of \catc,
\item an object $Hom \in Cover(\OO)$ in \catc,
\item A section $\fid \in Sect(s(id_{Ob})^*Hom)$ in \catc, 								
\end{itemize}
plus a section $\fcomp$, whose codomain we are now going to describe, and such that
 such that the left and right identity axioms and the associativity axiom hold.

\subsection*{codomain of $\fcomp$}
In regard to the object $Ob^3$ in \catcw, for $i=1,2,3$, as described in section \lref{projectionnaming}, define,
 $y_i: \OOO \morph Ob$ to be the i'th projection morphism. 
This enables us to consider the following pullback
\begin{equation*}
\begin{array}{r  p{4cm} c}
\compdomparent     \rightend{Qpp} && \Rnode{Hom}{Hom}               \\ [1cm]
\OOO          \rightend{O3}  && \Rnode{O2}{Ob^2}              
\end{array}
\mbox{
\ncsar{Qpp}{O3}
\ncsar{Hom}{O2}
\ncarr{Qpp}{Hom}
\alabel{q(\tuple{y_1,y_2},Hom)}
\ncarr{O3}{O2}
\alabel{\tuple{y_1,y_2}}}
\end{equation*}														

Now define $\dot y_i : \compdomparent \morph Ob$, for $i = 1,2,3$, 
           by $\dot y_i = p_{\compdomparent}\circ y_i$	and consider the pullback:
\begin{equation*}
\begin{array}{r  p{4cm} c}
\compdomain     \rightend{Qpp} && \Rnode{Hom}{Hom}               \\ [1cm]
\compdomparent     \rightend{O3}  && \Rnode{O2}{Ob^2}              
\end{array}
\mbox{
\ncsar{Qpp}{O3}
\ncsar{Hom}{O2}
\ncarr{Qpp}{Hom}
\alabel{q(\tuple{\dot y_2,\dot y_3},Hom)}
\ncarr{O3}{O2}
\alabel{\tuple{\dot y_2,\dot y_3}}}
\end{equation*}	

Finally, define   $\ddot y_i : \compdomain \morph Ob$, for $i = 1,2,3$, 
                                     by $\ddot y_i = p_{\compdomain}\circ \dot y_i$																	
and consider  the pullback
\begin{equation*}
\begin{array}{r  p{4cm} c}
\compcodomain     \rightend{Qpp} && \Rnode{Hom}{Hom}               \\ [1cm]
\compdomain     \rightend{O3}  && \Rnode{O2}{Ob^2}              
\end{array}
\mbox{
\ncsar{Qpp}{O3}
\ncsar{Hom}{O2}
\ncarr{Qpp}{Hom}
\alabel{q(\tuple{\ddot y_1,\ddot y_3},Hom)}
\ncarr{O3}{O2}
\alabel{\tuple{\ddot y_1,\ddot y_3}}}
\end{equation*}	

We will show  that an internal category in a contextual category \catcw consists of
\begin{itemize}
\item An object $Ob$ of \catc,
\item an object $Hom \in Cover(\OO)$ in \catc,
\item A section $\fid \in Sect(s(id_{Ob})^*Hom)$ in \catc, 
\item A section $\fcomp \in Sect(\compcodomain) $ of \catc 											
\end{itemize}
such that the left and right identity axioms and the associativity axiom hold. We turn our attention to the representation of these axioms.

\subsubsection*{Left Identity Axiom}
Suppose that we have such objects $Ob$ and $Hom$ and sections $id$ and $\fcomp$ in a contextual category \catc, as described above. 
In regard to the object $\OO$, for $i=1,2$, define $x_i$ to the the i'th projection function.
In addition we will define $\dot x_1,\dot x_2:Hom \morph Ob$ by
$\dot x_i = p_{Hom} \circ x_i$.

By lemma \lref{absolutedupletuplelemma} we have for any $i$ and $j$, $1 \leq i,j \leq 2$,
\begin{equation}
\label{dupletupledotxtwo}
\tuple{\dot{x_i},\dot{x_j}} = \duple{s(\dot{x_i}),s(\dot{x_j})} 
\end{equation}
By application of the same lemma  we have that for any $i,j$ and $k$, $1 \leq i,j,k \leq 3$,
\begin{equation}
\label{dupletupledotxthree}
\tuple{\dot x_i, \dot x_j, \dot x_k} = \duple{s(\dot x_i), s(\dot x_j), s(\dot x_k)}
\end{equation}


 %known as (zz)
We define $\leftidentitytuplep$
to be the unique morphism $\leftidentitytuplep :  Hom \morph \compdomparent$
such that 
 \begin{equation}
 \leftidentitytuplep \circ p_{\compdomparent} = \leftidentitytuplepp  
\end{equation}
 and 
\begin{equation}
\leftidentitytuplep \circ q(\tuple{y_1,y_2},Hom) = \dot x_1 \circ \fid
\end{equation}
as shown here
\begin{displaymath}
\begin{coneoutline}{1cm}{0.5cm}{Hom}
\ccprimitivepullbacksquare{3cm}{1.2cm}{Hom}{\OOO}{\OO}{\tuple{y_1,y_2}}
\end{coneoutline}
%\conearrowsdefiningtuple{xxx}{yyy}
\conearrowsdefiningtuple{\leftidentitytuplepp}{\dot x_1 \circ \fid}
\end{displaymath}
By  lemma \lref{thegeneraldupletuplelemma} and by use of (\lref{dupletupledotxthree}) it follows that 
\begin{equation}
\label{leftidentitydupletuplepidentity}
\leftidentitytuplep = \leftidentityduplep.
\end{equation}


Next, we define $\leftidentitytuple$
to be the unique morphism $\leftidentitytuple :  Hom \morph \compdomain$
such that 
 \begin{equation}
 \leftidentitytuple \circ p_{\compdomain} = \leftidentitytuplep %changed from pp ending
\end{equation}
 and 
\begin{equation}
\leftidentitytuple \circ q(\tuple{\dot y_2,\dot y_3},Hom) = id_{Hom}
\end{equation}
\begin{displaymath}
\begin{coneoutline}{1cm}{0.5cm}{Hom}
\ccprimitivepullbacksquare{3cm}{1.2cm}{Hom}{\OOO}{\OO}{\tuple{\dot y_2,\dot y_3}}
\end{coneoutline}
\conearrowsdefiningtuple{\leftidentitytuplep}{id_{Hom}}
\end{displaymath}
By  lemma \lref{thegeneraldupletuplelemma} and from (\lref{leftidentitydupletuplepidentity}) it follows that 
\begin{equation}
\label{leftidentitylhsremappingequation}
\leftidentitytuple = \leftidentityduple.
\end{equation}


Since the codomain of $\leftidentitytuple$ is the 
domain of the section $\fcomp$  then $\leftidentitytuple^*\fcomp$
is defined. We will show that the left identity axiom is satisfied iff
\begin{equation}
\leftidentitymapped
\end{equation}
in \catcw as shown in this diagram

\begin{displaymath}
\leftidentitydiagram
\end{displaymath}

\subsection*{Associativity Axiom}				
Next we turn to consideration of $Ob^4$.
Following the earlier convention we define the projection functions 
to be $z_1,z_2,z_3$ and $z_4$ so that for $i = 1, 2,3,4$, $z_i: \OOOO \morph Ob$. \\

Then we proceed to define   $\dot z_i : \associativitypremisepoppopmapped \morph Ob$
                                      by $\dot z_i = p_{\associativitypremisepoppopmapped}\circ z_i$, 
to define  $\ddot z_i : \associativitypremisepopmapped \morph Ob$ 
                                    by $\ddot z_i = p_{\associativitypremisepopmapped, \OOOO}\circ z_i$, 
and, finally, to define $\dddot z_i : \associativitypremisemapped \morph Ob$ 
                                      by $\ddot z_i = p_{\associativitypremisemapped, \OOOO}\circ z_i$ 	
so that for $i = 1, 2,3,4$ we have
\begin{equation*}
\begin{array}{r l p{4cm} c}
\associativitypremisemapped       \rightend{Q}  & \kern-0.2cm\rightend{Qright}                          \\ [1cm]
\associativitypremisepopmapped    \rightend{Qp} &  &&   \\ [1cm]
\associativitypremisepoppopmapped \rightend{Qpp}&  &&   \\ [1cm]
\OOOO                             \rightend{O4} & && \Rnode{Ob}{Ob}              
\end{array}
\mbox{
\ncsar{Q}{Qp}
\ncsar{Qp}{Qpp}
\ncsar{Qpp}{O4}
%\ncarr{Q}{Ob}
\ncarc[nodesepA=5pt,nodesepB=\arrnodesepB,offsetA=\arroffsetA,offsetB=\arroffsetB,arrowsize=5pt,arrowinset=0.7]{->}{Q}{Ob}
\alabel{\dddot z_i}
\ncarr{Qp}{Ob}
\alabel{\ddot z_i}
\ncarr{Qpp}{Ob}
\alabel{\dot z_i}
\ncarr{O4}{Ob}
\alabel{z_i}
}
\end{equation*} in \catcw.

The object $\associativitypremisemapped$ is relevant to us because we will show that it is the interpretation under $I$ of the premise
$\associativitypremise$ of the associativity axiom. This helps explain the next three definitions.
We define $f$, $g$ and $h$ by
\begin{align*}
f &= \assocfdefiniens, \\
g &= \assocgdefiniens, \\
h &= \assochdefiniens.
\end{align*}

With $f$, $g$ ad $h$ so defined, the following diagrams
\vspace{0.3cm}
\begin{equation*}
\ccsquareoutline{1.4cm}{1.2cm}{\associativitypremisemapped}{Hom}{\OOOO}{\OO}
\mbox{
\nccdar{TL}{BL}
\ncsar{TR}{BR}
\ccsquareacross{f}{\tuple{z_1, z_2}}
\kern -1cm %work around bug with lost arrow space bug
}
\ccsquareoutline{1.4cm}{1.2cm}{\associativitypremisemapped}{Hom}{\OOOO}{\OO}
\mbox{
\nccdar{TL}{BL}
\ncsar{TR}{BR}
\ccsquareacross{g}{\tuple{z_2, z_3}}
\kern -1cm %work around bug with lost arrow space bug
}
\ccsquareoutline{1.4cm}{1.2cm}{\associativitypremisemapped}{Hom}{\OOOO}{\OO}
\mbox{
\nccdar{TL}{BL}
\ncsar{TR}{BR}
\ccsquareacross{h}{\tuple{z_3, z_4}}
}
\end{equation*} commute in \catc.	\\

\newcommand{\treblez}[3]{\dddot z_#1, \dddot z_#2,\dddot z_#3}
\newcommand{\trebledottreblez}{\tuple{\treblez{1}{2}{3}}}
 
We define $\trebledottreblez$ to be the unique morphism $\associativitypremisemapped \morph \morph \OOO$ such that 
\begin{equation}
\trebledottreblez \circ y_1 = \dddot z_1,
\end{equation}
\begin{equation}
\trebledottreblez \circ y_2 = \dddot z_2
\end{equation}
and
\begin{equation}
\trebledottreblez \circ y_3 = \dddot z_3.
\end{equation}
By application of Lemma \lref{absolutedupletuplelemma} we have that

\begin{equation}
\label{zonezfourdupletuplelemma}
\tuple{\dddot z_1,\ddot z_4} = \duple{s(\dddot z_1),s(\dddot z_4)}
\end{equation}
and similarly that
\begin{equation}
\label{trebledottreblezdupletupleidentity}
\trebledottreblez = \duple{s(\dddot z_1),s(\dddot z_2),s(\dddot z_3)}
\end{equation}



\newcommand{\trebledzf}{\tuple{\dddot z_1,\dddot z_2, \dddot z_3, f}}

We define $\trebledzf$
to be the unique morphism $\trebledzf :  \associativitypremisemapped \morph \compdomparent$
such that 
 \begin{equation}
 \trebledzf \circ p_{\compdomparent} = \trebledottreblez 
\end{equation}
 and 
\begin{equation}
\trebledzf \circ q(\tuple{y_1,y_2},Hom) = f
\end{equation}
as shown here
\begin{displaymath}
\begin{coneoutline}{1cm}{0.5cm}{Hom}
\ccprimitivepullbacksquare{3cm}{1.2cm}{Hom}{\OOO}{\OO}{\tuple{y_1,y_2}}
\end{coneoutline}
\conearrowsdefiningtuple{\dddot z_1,\dddot z_2, \dddot z_3}{f}
\end{displaymath}
By  Lemma \lref{absolutedupletuplelemma} we have 
\begin{equation}
\label{somethingorother}
\trebledzf = \duple{s(\dddot z_1),s(\dddot z_2), s(\dddot z_3), s(f)}.
\end{equation}


\newcommand{\tuplefglongform}{\tuple{\dddot z_1,\dddot z_2, \dddot z_3, f,g}}
We define $\tuplefglongform$. which subsequently we abbreviate as $\tuple{f,g}$
to be the unique morphism $\tuplefglongform :  \associativitypremisemapped \morph \compdomain$
such that 
 \begin{equation}
 \tuplefglongform \circ p_{\compdomparent} = \trebledzf 
\end{equation}
 and 
\begin{equation}
\tuplefglongform \circ q(\tuple{y_1,y_2},Hom) = g
\end{equation}
as shown here
\begin{displaymath}
\begin{coneoutline}{1cm}{0.5cm}{Hom}
\ccprimitivepullbacksquare{3cm}{1.2cm}{Hom}{\OOO}{\OO}{\tuple{y_1,y_2}}
\end{coneoutline}
\conearrowsdefiningtuple{\dddot z_1,\dddot z_2, \dddot z_3, f}{g}
\end{displaymath}
We use  $\tuple{f,g}$ as a shortened form of $\tuplefglongform$so that by  Lemma \lref{absolutedupletuplelemma} we have 
\begin{equation}
\label{pairfgdupletuplepidentity}
\tuple{f,g} = \duple{s(\dddot z_1), s(\dddot z_2), s(\dddot z_3),  s(f), s(g)}.
\end{equation}


With  $\tuple{f,g}$ so defined the nested diagrams within
% <f,g> definition

\associativitytermdiagram[25][4cm]{\dddot z_1, \dddot z_2,\dddot z_3}{f}{g}

commute.


% <g,h> definition
We  define  $\tuple{\treblez{2}{3}{4}}$ and
$\tuple{g,h}:\associativitypremisemapped \morph \compdomain$ in a similar manner so that nested diagrams in
\associativitytermdiagram[25][4cm]{\dddot z_1, \dddot z_2,\dddot z_3}{g}{h}
commute. With $\tuple{g,h}$ so defined by lemma \lref{absolutedupletuplelemma} we have 
\begin{equation}
\label{pairghdupletuplepidentity}
\tuple{g,h} = \duple{s(\dddot z_2), s(\dddot z_3), s(\dddot z_4),  s(g), s(h)}.
\end{equation}

% associativity lhs definition
We can  define a morphism $\assoclhsremappingtuple:\associativitypremisemapped \morph \compdomain$ so that
nested diagrams in 
\associativitytermdiagram{\dddot z_1, \dddot z_3,\dddot z_4}{(\assocfogremapped) \circ q(\tuple{\dddot z_1,\dddot z_3},Hom)}{h}
commute.
We use this in describing the mapping of the  left hand side of the associativity axiom.

% associativity rhs definition
Repeating the whole exercise we can  define a morphism $\assoclhsremappingtuple$ so that nested diagrams in
\associativitytermdiagram{\dddot z_1, \dddot z_2,\dddot z_4}{f}{(\assocgohremapped) \circ q(\tuple{\dddot z_2,\dddot z_4},Hom)}
commute.


\begin{lemma}
\llabel{associativitycontextmapping}
If $I$ is an interpretation of the theory $tc$ in a contextual category \catcw then
$I$ maps the context $\tuple{\associativitypremise}$ to the object $\associativitypremisemapped$ in \catc.
\end{lemma}
\begin{proof}
% two width forcing commands
\newcommand {\forceSOURCEwidth}{\rule{5cm}{0pt}}  % so as to line up three different arrays
\newcommand {\forceTARGETwidth}{\rule{2.2cm}{0pt}}

From lemma \lref{Xnlemma} we have the following interpretation by $I$, for each $i$, $1 \leq i \leq 4$:
\begin{equation*}
\begin{array}{c c c}
\forceSOURCEwidth & & \forceTARGETwidth \\ [-0.1cm]
\gatdisplayrule{\zOOOO}{\ofT{z_i}{Ob}} & \Imapsto & s(z_i) 
\end{array}
\end{equation*}

From these mappings it follows by lemmas \lref{absolutedupletuplelemma} and \lref{supplementarylemma} that 

\begin{equation*}
\begin{array}{c c c}
\forceSOURCEwidth & & \forceTARGETwidth \\ [-0.1cm]
\gatdisplayrule{\zOOOO}{\isT{Hom(z_1,z_2)}} & \Imapsto & \associativitypremisepoppopmapped 
\end{array}
\end{equation*}

The context $\associativitypremisepoppop$ is therefore mapped to $\associativitypremisepoppopmapped$.

Now it follows by lemma \lref{Xnlemma} that 

\begin{equation*}
\begin{array}{c c c}
\forceSOURCEwidth & & \forceTARGETwidth \\ [-0.1cm]
\gatdisplayrule{\associativitypremisepoppop}{\ofT{z_i}{Ob}} & \Imapsto & s(p_{\associativitypremisepoppopmapped} \circ z_i) \\
                                                            & = & s(\dot{z_i})
\end{array}
\end{equation*}
and therefore by  lemmas \lref{absolutedupletuplelemma} and \lref{supplementarylemma} that 
\begin{equation*}
\begin{array}{c c c}
\forceSOURCEwidth & & \forceTARGETwidth \\ [-0.1cm]
\gatdisplayrule{\associativitypremisepoppop}{\isT{Hom(z_2,z_3)}} & \Imapsto & \associativitypremisepopmapped.
\end{array}
\end{equation*}

The context $\associativitypremisepop$ is therefore mapped to $\associativitypremisepopmapped$.

By application of lemma \lref{Xnlemma} again we establish that 
\begin{equation*}
\begin{array}{c c c}
\forceSOURCEwidth & & \forceTARGETwidth \\ [-0.1cm]
\gatdisplayrule{\associativitypremisepop}{\ofT{z_i}{Ob}} & \Imapsto & s(p_{\associativitypremisepopmapped} \circ z_i) \\
                                                            & = & s(\ddot{z_i})
\end{array}
\end{equation*}
and therefore by  lemmas \lref{absolutedupletuplelemma} and \lref{supplementarylemma} that 
\begin{equation*}
\begin{array}{c c c}
\forceSOURCEwidth & & \forceTARGETwidth \\ [-0.1cm]
\gatdisplayrule{\associativitypremisepop}{\isT{Hom(z_3,z_4)}} & \Imapsto & \associativitypremisemapped.
\end{array}
\end{equation*}
The context $\associativitypremise$ is therefore mapped to $\associativitypremisemapped$.
\end{proof}

\begin{lemma}
\llabel{internalcategorylemma}
An internal category in a contextual category \catcw consists of
\begin{itemize}
\item An object $Ob$ of \catc,
\item an object $Hom \in Cover(\OO)$ in \catc,
\item A section $\fid \in Sect(s(id_{Ob})^*Hom)$ in \catc, 
\item A section $\fcomp \in Sect(\compcodomain) $ of \catc 											
\end{itemize}
such that
\begin{equation}
\label{leftidentityaxiom}
\leftidentitymapped
\end{equation}
\begin{equation}
\label{rightidentityaxiom}
\rightidentitymapped
\end{equation}
and
\begin{multline}
\label{associativityaxiom}
\assoclhsremapped = \assocrhsremapped
\end{multline}


Equivalently an internal category in a contextual category \catcw consists of
\item objects $Ob$ and  $Hom$  and a section $\fid$ in \catc,  as above, along with
\begin{itemize}
\item a morphism $\compmorph$ of \catc, $\compmorph: \compdomain \morph Hom$ in \catc
\end{itemize}
such that
\begin{equation}
\label{leftidentityrepresentation2}
\tuple{\dot x_1,\dot \dot x_1,\dot x_2,\dot x_1\circ \fid,id_{Hom}} \circ \compmorph =id_{Hom}
\end{equation}
\begin{equation}
\label{rightidentityrepresentation2}
\tuple{\dot x_1,\dot x_2,\dot x_2,id_{Hom},\dot x_2 \circ \fid} \circ \compmorph =id_{Hom}
\end{equation}
and
\begin{equation}
\label{associativityrepresentation2}
\assocequivalentlhs = \assocequivalentrhs
\end{equation}.
\end{lemma}
\begin{proof}
Of these two equivalent representations the first results from a literal reading of the definition of instance given earlier
along with the judicious choice of intermediate definitions made with readability in mind.
This is demonstrated in tables \ref{internalcategorytableone}  - \ref{internalcategorytablefour} below. 

To show that the second representation follows from the first then from $\fcomp$ define $\compmorph$ by defining $\compmorph=\fcomp \circ q(p_{\tuple{\ddot y_1,\ddot y_3}},Hom)$ and then it is easy to show that 
(\ref{leftidentityrepresentation2}) follows from (\ref{leftidentityaxiom}), 
(\ref{rightidentityrepresentation2}) follows from (\ref{rightidentityaxiom}) and
(\ref{associativityrepresentation2}) follows from (\ref{associativityaxiom})

Vice-versa, from the second representation follows the first if we define define $\fcomp$ from  $\compmorph$ by defining $\fcomp=s(\compmorph)$.

\begin{table}[H]
\caption{Deriving what constitutes an intepretation of the theory of categories $tc$ in a contextual category \catc.
Part One - Introductory rules for $Ob$, $Hom$ and $id$.
}
\label{internalcategorytableone}
%\setlength{\arrayrulewidth}{1mm}
\setlength{\tabcolsep}{2pt}
\begin{tabular}{l l  c  p{0cm} l  l}
\multicolumn{2}{l}{Derived Rule} &&& Interpretation by $I$ in \catcw & Reason why\\
\hline
\gatinterpretationintro {obintro}{}{\isT{Ob}}{Ob \in Cover(1)}{definition \ref{consistentinterpretation} (i)(a)}                                   \\
\gatinterpretationdetail{homintrohelper}{\ofT{x_1}{Ob}}{\isT{Ob}}{Ob^2 \in Cover(Ob)}
                                                               {lemma \ref{supplementaryweakeninglemma} (i) and (\ref{obintro})}             \\
\gatinterpretationintro {homintro}{\ofT{x_1}{Ob},\ofT{x_2}{Ob}}{\isT{Hom}}{Hom \in Cover(Ob^2)}
                                                               {definition \ref{consistentinterpretation} (i)(a) and (\ref{homintrohelper})}      \\
\gatinterpretationdetail{idintrohelperhelper}{\ofT{w}{Ob}}{\ofT{w}{Ob}}{s(id_{Ob})}
                                                               {definition \ref{consistentinterpretation} (ii)(d) and (\ref{homintrohelper})}  \\
\gatinterpretationdetail{idintrohelper}{\ofT{w}{Ob}}
                                 {\isT{Hom(w,w)}}{s(id_{Ob})^*Hom }
                                 {lemma \ref{supplementarylemma} (i),  (\ref{homintro}) and (\ref{idintrohelperhelper})}           \\
\gatinterpretationintro {idintro}{\ofT{w}{Ob}}{\ofT{id(w)}{Hom(w,w)}} 
                                 {\fid \in Sect(s(id_{Ob})^*Hom) }
                                 {definition \ref{consistentinterpretation} (ii)(a) and (\ref{idintrohelper})}                                      \\
\end{tabular}
\end{table}


\begin{table}[H]
\caption{Deriving what constitutes an intepretation of the theory of categories $tc$ in a contextual category \catc.
Part Two Introductory rule for $\circ$. Indication of the reasoning is not included due to lack of space. 
The reasoning follows the patterns indicated in accompanying tables \ref{internalcategorytableone} and \ref{internalcategorytablethree}.
}
\label{internalcategorytabletwo}
%\setlength{\arrayrulewidth}{1mm}
\setlength{\tabcolsep}{2pt}
\begin{tabular}{l l  c  p{0cm} l  l}
\multicolumn{2}{l}{Derived Rule} &&& Interpretation by $I$ in \catcw \\
\hline
\gatinterpretationdetail{comp1}{\ofT{x_1,x_2}{Ob}}{\isT{Ob}}{ \OOO \in Cover(\OO) }{}              \\
\gatinterpretationdetail{comp2}{\ofT{y_1,y_2,y_3}{Ob}}{\isT{Hom(y_1,y_2)}}{ \compdomparent \in Cover(\OOO) }{} \\
\gatinterpretationdetail{comp3}{\ofT{y_1,y_2,y_3}{Ob}, \ofT{f_1}{Hom(y_1,y_2)}}{\isT{Hom(y_2,y_3)}}
                                                                          {  \compdomain \in Cover(\compdomparent) }{} \\
\gatinterpretationdetail{comp4}{\ofT{y_1,y_2,y_3}{Ob}, \ofT{f}{Hom(y_1,y_2)},\ofT{g}{Hom(y_2,y_3)}} {\isT{Hom(y_1,y_3)}}
                                                                         { \compcodomain \in Cover(\compdomain) }{} \\
\gatinterpretationintro {compintro}	{\ofT{y_1,y_2,y_3}{Ob}, \ofT{f}{Hom(y_1,y_2)},\ofT{g}{Hom(y_2,y_3)}} 
                                    {\ofT{f \circ g}{Hom(y_1,y_3)}}
																    {\fcomp \in Sect(\compcodomain)}
\end{tabular}
\end{table}

\newcommand{\leftidentityidremapped}{s(\dot{x_1}\circ \fid)} 

\begin{table}[H]
\caption{Deriving what constitutes an intepretation of the theory of categories $tc$ in a contextual category \catc.
Part Three. The left identity axiom.
}
\label{internalcategorytablethree}
%\setlength{\arrayrulewidth}{1mm}
\setlength{\tabcolsep}{2pt}
\begin{tabular}{l l  c  p{0cm} l  l}
\gatinterpretationcontext{Let $P$ be the context $\ofT{x_1}{Ob},\,\ofT{x_2}{Ob},\,\ofT{f}{Hom(x_1,x_2)} $
                                 then from (\ref{homintro}) we have $P \mapsto Hom \in Cover(Ob^2)$.} \\
\hline
\multicolumn{2}{l}{Derived Rule} &&& Interpretation by $I$ in \catcw & Reason why\\
\hline
\gatinterpretationdetail{rightidentity1}{P}{\isT{Ob}}{ \HomOb \in Cover(Hom) }{lemma \ref{supplementaryweakeninglemma} (i), (\ref{homintro}) and (\ref{obintro})}              \\
\gatinterpretationdetail{rightidentity2}{P}{\ofT{x_1}{Ob}}{ s(p_{Hom,Ob}) \in Section(\HomOb) }{definition \ref{consistentinterpretation} (ii)(d)}                    \\
\gatinterpretationmapeqv        {s(\dot{x_1})}                                            {defn. of $\dot{x_1}$}             \\
\gatinterpretationdetail{rightidentity3}{P}{\ofT{x_2}{Ob}}{ s(p_{Hom,Ob^2}) \in Section(\HomOb) }{definition \ref{consistentinterpretation} (ii)(d)}                  \\
\gatinterpretationmapeqv        {s(\dot{x_2})}                                            {defn. of $\dot{x_2}$}             \\
\gatinterpretationdetail{rightidentity4}{P}{\isT{Hom(x_1,x_1)}}{\duple{s(\dot{x_1}),s(\dot{x_1})}^*Hom \in Cover(Hom)} 
                                                             {lemma \ref{supplementarylemma} (ii), (\ref{homintro}) and (\ref{rightidentity2})} \\
\gatinterpretationmapeqv       {\tuple{\dot{x_1},\dot{x_1}}^*Hom}                                      {by (\ref{dupletupledotxtwo})}     \\
\gatinterpretationdetail{rightidentityidmapping}{P}{\ofT{id(x_1)}{Hom(x_1,x_1)}}{\duple{s(\dot{x_1})}^*\fid \in Sect(\tuple{\dot{x_1},\dot{x_1}}^*Hom)}  
                                                             {lemma \ref{supplementarylemma} (ii), (\ref{idintro}) and (\ref{rightidentity2})} \\
\gatinterpretationmapeqv       {\dot{x_1}^*\fid}                                      {by (d1)}     \\
\gatinterpretationmapeqv       {\leftidentityidremapped}                              {lemma \ref{regardingfstarsection}}     \\
\gatinterpretationdetail{rightidentityrhsmappping}{P}{\ofT{f}{Hom(x_1,x_2)}}{\leftidentityrhsmapped \in Sect(\HomHom) }{definition \ref{consistentinterpretation} (ii)(d)}                         \\
\gatinterpretationdetail{rightidentitylhsmapping}{P}{id(x_1) \circ f} {\leftidentitylhsmapped     }
                                     {lemma \ref{supplementarylemma} (ii), (\ref{rightidentity2}), (\ref{rightidentity3}) and (\ref{rightidentityidmapping})} \\
				&\hspace{1.2cm}$\ofT{}{Hom(x_1,x_2)}$&&&\hspace{3.5cm}$\in Sect(\HomHom)$&   \\
\gatinterpretationmapeqv   {\leftidentitylhsremapped} { by (\ref{leftidentitylhsremappingequation})}      \\        
\gatinterpretationaxcond{tcaxiomone}{P}{id(x_1) \circ f = f}
                                       {\leftidentitylhsremapped=\leftidentityrhsmapped}
                                       {definition \ref{consistentinterpretation} (iv), (\ref{rightidentitylhsmapping}) and (\ref{rightidentityrhsmappping})}    
\end{tabular}
\end{table}

\begin{table}[H]
\caption{Deriving what constitutes an intepretation of the theory of categories $tc$ in a contextual category \catc.
Part Four. Associativity axiom.
}
\label{internalcategorytablefour}
%\setlength{\arrayrulewidth}{1mm}
\setlength{\tabcolsep}{2pt}
\begin{tabular}{l l  c  p{0cm} l  l}
\gatinterpretationcontext{Let $Q$ be the context $\associativitypremise$} \\
\gatinterpretationcontext{then $Q \mapsto \associativitypremisemapped \in Cover(\associativitypremisepopmapped)$ in \catcw by lemma \ref{associativitycontextmapping}.}\\
\hline

\multicolumn{2}{l}{Derived Rule} &&& Interpretation by $I$ in \catcw & Reason why                   \\
\hline \\[-0.4cm]
\gatinterpretationdetail{assoczimapping}{Q}{\ofT{z_i}{Ob},\mbox{ for } i=1,2,3,4}{\assoczimapped}{definition \ref{consistentinterpretation} (ii)(d)}   \\[0.2cm]
\gatinterpretationmapeqv          {\assoczimappedintermediary}                   {axiom (s3)}                \\[0.2cm]
\gatinterpretationmapeqv          {\assocziremapped}                   {by defn. of $\dddot z_i$}  \\[0.2cm]
\gatinterpretationdetail{assocfmapping}{Q}{\ofT{f}{Hom(z_1,z_2)}}{\assocfmapped}{definition \ref{consistentinterpretation} (ii)(d)}             \\[0.2cm]
\gatinterpretationmapeqv          {\assocfmappedintermediary}                   {axiom (s3)}     \\[0.2cm]
\gatinterpretationmapeqv          {\assocfremapped}                             { by defn. of $f$}      \\[0.2cm]
\gatinterpretationdetail{assocgmapping}{Q}{\ofT{g}{Hom(z_2,z_s)}}{\assocgmapped}{definition \ref{consistentinterpretation} (ii)(d)}              \\[0.2cm]
\gatinterpretationmapeqv                                  {\assocgmappedintermediary} {axiom (s3)}      \\[0.2cm]
\gatinterpretationmapeqv          {\assocgremapped}                             { be defn. of $g$}      \\[0.2cm]
\gatinterpretationdetail{assochmapping}{Q}{\ofT{h}{Hom(z_3,z_4)}}{\assochmapped}{definition \ref{consistentinterpretation} (ii)(d)}               \\[0.2cm]
\gatinterpretationmapeqv                                  {\assochmappedintermediary}  {axiom (s3)}     \\[0.2cm]
\gatinterpretationmapeqv          {\assochremapped}                             { be defn. of $h$}      \\[0.2cm] 
\gatinterpretationdetail{assocfgmapping}{Q}{\ofT{f \circ g}{Hom(z_1,z_3)}}
                                   { \assocfogmapped  }   {lemma \ref{supplementarylemma}, (\ref{assocfmapping}) and (\ref{assocgmapping})}         \\[0.2cm]
\gatinterpretationmapeqv                    {\tuple{f,g}^*\fcomp}{by (\ref{pairfgdupletuplepidentity})}                                                 \\[0.2cm]         
\gatinterpretationdetail{assoctypemapping}{Q}{\associativitylhstype}{\associativitylhstypemapped}
                                                                   {lemma \ref{supplementarylemma} (i), (\ref{homintro}) and (\ref{assoczimapping})}    \\[0.2cm]  
\gatinterpretationmapeqv                     {\associativitylhstyperemapped}{by (\ref{zonezfourdupletuplelemma})}             \\[0.2cm]   
\gatinterpretationdetail{assocLHSmapping}{Q}{\associativitylhstermtyping}{\assoclhsmapped}
                                            {lemma \ref{supplementarylemma}, (\ref{assocfgmapping}) and (\ref{assochmapping})}\\[0.2cm]
\gatinterpretationmapeqv                    {\assoclhsremapped}{lemma \ref{thedupletuplelemma} and  (s3)}\\[0.2cm]
\gatinterpretationdetail{assocghmapping}{Q}{\ofT{g \circ h}{Hom(z_2,z_4)}}
                                   { \assocgohmapped  }{lemma \ref{supplementarylemma} (ii), (\ref{assoczimapping}), (\ref{assocgmapping})}      \\[0.2cm]
\gatinterpretationdetailcontinuation{}{\hspace{2.2cm} and (\ref{assochmapping})}                                                   \\[0.2cm]
\gatinterpretationmapeqv           {\assocgohremapped}{by (\ref{pairghdupletuplepidentity})} \\[0.2cm]
\gatinterpretationdetail{assocRHSmapping}{Q}{\associativityrhstermtyping}
                                            {\assocrhsmapped \iffalse{\in Sect(\associativitylhstyperemapped)}\fi}
			                       {lemma \ref{supplementarylemma}, (\ref{assocfmapping}) and (\ref{assocghmapping})} \\ [0.2cm]
\gatinterpretationmapeqv                    {\assocrhsremapped}{lemma \ref{thedupletuplelemma} and (s3)}\\[0.2cm]
\gatinterpretationaxcond{associativity}{Q}{(f \circ g) \circ h = f \circ (g \circ h)}
                                     { \assoclhsremapped  } {definition \ref{consistentinterpretation} (iv), (\ref{assocLHSmapping})} \\
\gatinterpretationaxcondrhscontinuation{= \assocrhsremapped } { and  (\ref{assocRHSmapping})}\\
\end{tabular}
\end{table}

%\ncarc[arcangle=#1,nodesepA=5pt,nodesepB=5pt,offsetA=#2pt,offsetB=#2pt,arrowsize=5pt,arrowinset=0.7]{->}{#3}{#4}
\end{proof}

\iffalse
\begin{equation*}
\begin{array}{c}
\begin{array}{r c p{4cm} c}
\associativitypremisemapped        \rightend{Q}            \\ [1cm]
\associativitypremisepopmapped     \rightend{Qp}           \\ [1cm]
\associativitypremisepoppopmapped  \rightend{Qpp}          \\ [1cm]
\OOOO   \rightend{O4}                                   \\ [1cm]
\OOO    \rightend{O3}      &  & & \Rnode{H}{Hom}            \\ [1cm]
\OO     \rightend{O2}      & & & \Rnode{Hp}{\OO}           \\ [1cm]
Ob      \rightend{O}       & & & \Rnode{Hpp}{Ob}           \\ [1cm]
\Rnode{abs}{1}        \\ 
\end{array} \\
\begin{arrows}
\ncsar{Q}{Qp}
\ncsar{Qp}{Qpp}
\ncsar{Qpp}{O4}
\ncsar{O4}{O3}
\ncsar{O3}{O2}
\ncsar{O2}{O}   
\ncsar{O}{abs}
\ncsar{H}{Hp}
\ncsar{Hp}{Hpp}
\ncsar{Hpp}{abs}
\ncarrNEGZZ[-10]{Q}{H}    \alabel{f}
\ncarrZ{Q}{H}             \alabel{g}
\ncarrZZ[10]{Q}{H}        \alabel{h}
\ncarrNEGZZ[-10]{O4}{Hpp} \alabel{z_1}
\ncarrZ{O4}{Hpp}          \alabel{z_2}
\ncarrZZ[10]{O4}{Hpp}     \alabel{z_3}
\ncarrZZZ[20]{O4}{Hpp}    \alabel{z_4}
\end{arrows}
\end{array}
\end{equation*}
\fi


