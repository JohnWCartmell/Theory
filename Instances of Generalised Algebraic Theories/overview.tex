
\note \label{ccgatequivalence}From my thesis, 
\begin{point}
there is a category $\catGAT$ of generalised algebraic theories and interpretations,
\end{point}
\begin{point}
there is a category $\catCon$ of contextual categories,
\end{point}
\begin{point}
there is a functor $\ccat[C]: \catGAT \morph \catCon$,
\end{point}
\begin{point}
there is a functor $\gat[U]:\catCon \morph \catGAT$,
\end{point}
\begin{point}
the functor $\ccat[C]$ is an equivalence with inverse $\gat[U]$.
\end{point}
\note
The proof that categories $\catGAT$ and $\catCon$ are equivalent  is entirely trivial but runs to more than 50 pages. I have always interpreted this equivalence as meaning that generalised algebraic theories and contextual categories are more or less the same thing but if this is considered from the point of view of foundations then we have to tread carefully.
\note 
In section \ref{sectioninwhichinstanceisdefined}, below, we define the notion of 
an `instance of  generalised algebraic theory $\gat[U]$ in  a contextual category \catc'. 
In essence such an instance $I$ consists of a \textit{consistent} mapping

\begin{center}
\begin{tabular}{c p{1cm} c}
derived \Trules of $U$           & \raisebox{-0.07cm}{$\Imapsto$} & objects of \catc \\ [0.1cm]
derived \trules of $U$    & \raisebox{-0.07cm}{$\Imapsto$} & sections of \catc \\ [0.1cm]
\end{tabular}
\end{center}
so that derivable equalities in \gatUw map to identical objects, respectively, sections of \catc.
There is a fair amount of detail of what is meant by  `consistent mapping'. This detail implies that 
instances $I$ of \gatUw in \catcw are completely
determined by their mapping of the introductory rules of \gatU.

\note
Such an instance we also say is an internal $\gat[U]$-structure in the contextual category $\catc$. 

\note
The category of internal $\gat[U]$-structures is defined to be the category whose objects
are pairs $\tuple{\catc,I}$ where \catcw is a contextual category and $I$ is an instance of the theory $\gat[U]$ in the contextual category \catcw and whose morphisms between $\tuple{\catc,I}$ and $\tuple{\catc',I'}$ are pairs $\tuple{F, \eta}$ where
$F: \catc \morph \catc'$ is a contextual functor and $\eta: I \circ F \morph  I'$ is a natural transformation. 

\note 
As  noted earlier, to every generalised algebraic theory $\gat[U]$ there  is a contextual category $\CofU$ corresponding to $\gat[U]$. This category has as objects equivalence classes of contexts and realisations (as defined 
in \cite{Cartmell78} and  \cite{Cartmell86}). 
From these definitions it also follows that there is a trivial instance
of $\gat[U]$ in  $\CofU$. This is an initial object in
the category of internal $\gat[U]$-structures.

\note 
An alternative definition and one that is offered  in my thesis 
(though the terminology is different\footnote{In my thesis I use the term `model' rather than `instance'  but here I am trying to avoid such use of the term `model'}) is that an internal $\gat[U]$-structure in a contextual category \catcw is precisely 
a contextual functor from the contextualcategory $\CofU$\  to \catc. 
Meta-mathematically the two definitions are equivalent\footnote{The proof that they are equivalent 
definitions is entirely straightforward though I didn't write it up my thesis. I did once write out the proof of a  corresponding lemma in regard to single-sorted algebraic theories; this was in my Msc dissertation.}.
\note
In this way, the category of internal $\gat[U]$-structures  is isomorphic to the coslice category
$\CofU \downarrow \catCon$. Needless to say this has an initial object which is the identity functor on  $\CofU$.
If\ $\gat[U]$ is a considered a type theory (whatever that is) then this initial object is what I believe Vladimir refers
to as the term model when speaking of the initiality conjecture. It is the contextual category
$\CofU$ along with the trivial instance of $\gat[U]$ in $\CofU$.

\note 
Instances of a theory $\gat[U]$ in the contextual category $\Fam$ are said to be $\gat[U]$-algebras. 

\note 
From the details given in
section \ref{sectioninwhichinstanceisdefined} 
in which instances are both defined and characterised 
it follows that 
to every gat $\gat[U]$ there is a theory of internal $\gat[U]$-structures. We shall denote this theory as $\hatU$.

Every such theory $\hatU$ is an extension of the generalised algebraic theory of contextual categories
by a set of rules (introductory rules and axioms) that have  the empty context as premise -- as such it is an extension
by constants and equational identitites between closed terms -- and, vice-versa, every such extension of
the theory of contextual categories can be interpreted as being a specification of a generalised algebraic theory.  

\begin{notebox}[Question]
The above observation would allow someone automating reasoning within  a generalised algebraic theory 
$\gatU$ to reason about $\hatU$ instead. Might this be advantageous? 
\end{notebox}

\note 
The instances of $\hatU$  in $\Fam$ consist of  internal $\gat[U]$-structures  i.e. they consist of contextual categories \catcw along with particular instances $I$ of
the theory $\gat[U]$ in the contextual category \catc. \\
The category of $\hatU$-algebras is (isomorphic to) the category of internal $\gatU$-structures.
\note
An instance of $\hatU$ in an arbitrary contextual category
consists of  an internal internal $\gat[U]$-structure. This sounds a bit crazy but it isn't -- there are after all categories internal to other categories and it isn't much of a stretch to suppose these internal categories have internal $\gat[U]$'s inside of them. 

\note 
Definition of initial $\gat[U]$-algebras. From my thesis:
\begin{tightquote}
Consider for a moment. Every theory $\gat[U]$ has a minimal model denoted $\KU$ built out of the closed terms of \gat[U]. Alternatively this minimal model is described just in terms of the structure $\CofU$. For example
if $1 \base A$ in $\CofU$ then 
$\KU(A)=Hom(1,A)$, otherwise if $1 \base A_1 \base ... \base A_n \base A$ in $\CofU$
then if $a_1 \in \KU(A_1)$, ... if $a_n \in \KU(A_n)(a_1,...a_{n-1})$ then 
$\KU(A)(a_1,...a_n)=\setsuchthat{a\in Hom_{\CofU}(1,A)}{a \circ p_A = a_n}$. \\
\end{tightquote} 

Followed by :
\begin{tightquote}
Now, the free $\gat[U]$-algebras are the algebras $I$-$alg(\KUp)$ for $I: \gat[U] \morph \gat[U']$ an extension of $\gat[U]$ by constants alone. The finitely generated free $\gat[U]$-algebras are those algebras where $\gat[U']$ is an extension by finitely many constants. \\
\end{tightquote}

\note
\label{termmodelEQfreealgebra}For any generalised algebraic theory $\gat[U]$ we have two different 
and therefore isomorphic descriptions of the initial object of the category of internal $\gatU$-structures:\\
\begin{equation}
K_{\hat{U}} \cong \bigtuple{\CofU, I_{triv}}
\end{equation}

where $I_{triv}$ is the trivial instance of $\gatU$ in $\CofU$.

\begin{notebox}[Question]
Is this observation relevant
to the initiality conjecture or to formal (machine checked) theory?  It can be summarised 
by saying that the term model of a theory $\gatU$ is the initial algebra of a theory $\hatU$.
Is that useful? Another way of looking at it (is it helpful?) is that the open terms and types
of $\gatU$ correspond to the closed terms of $\hatU$. 
Is use of the word combinator appropriate here?
\end{notebox}

\note 
Note that if $\gat[U]$ is a single-sorted or many-sorted algebraic theory then 
$\hatU$ is generalised algebraic 
and so solely within these regimes there is no equivalent of the situation described para \ref{termmodelEQfreealgebra}.

\note As a worked example, we show in lemma \ref{internalmonoidlemma}, below, that 
the generalised algebraic theory of internal monoids can be expressed  as 
the theory of contextual categories plus:

\begin{gatrules}
\gatintros
\gatintroducing{M}
\ofT{M}{Ob} \\
\gatintroducing{unit}
\ofT{unit}{Hom(1,M)} \\
\gatintroducing{mult}
\ofT{mult}{Hom(M \times M,M)} \\
\gataxioms
\gatintroducing{ \gataxiomno{1} }
\tuple{p_M \circ unit,id_M} \circ mult =id_M \\
\gatintroducing{ \gataxiomno{2} }
\tuple{id_M,p_M \circ unit} \circ mult =id_M \\
\gatintroducing{ \gataxiomno{3} }
(mult \times id_M) \circ mult = (id_M \times mult) \circ mult
\end{gatrules}

This is in agreement with Barr and Wells \cite{BarrandWells}, page 232, where they describe
monoids internal to  a category\footnote{Generally we would be thinking of a category with finite products including terminal object here, though, as they say, not all products need be available in the category for there to be an internal monoid.}
as an example of a finite product (FP) sketch.

\note As a second worked example  we derive the generalised algebraic theory of internal categories in lemma \ref{internalcategorylemma}.

\note These two examples and the main definition itself serve to illustrate the tie up between generalised algebraic theories and contextual categories.



