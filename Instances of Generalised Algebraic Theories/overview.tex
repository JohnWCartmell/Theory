








\begin{oldtt}
\note 
In section \ref{sectioninwhichinstanceisdefined}, below, we define the notion of 
an `instance of  generalised algebraic theory $\gat[U]$ in  a contextual category \catc'. 
In essence such an instance $I$ consists of a \textit{consistent} mapping

\begin{center}
\begin{tabular}{c p{1cm} c}
derived \Trules of $U$           & \raisebox{-0.07cm}{$\Imapsto$} & objects of \catc \\ [0.1cm]
derived \trules of $U$    & \raisebox{-0.07cm}{$\Imapsto$} & sections of \catc \\ [0.1cm]
\end{tabular}
\end{center}
so that derivable equalities in \gatUw map to identical objects, respectively, sections of \catc.
There is a fair amount of detail of what is meant by  `consistent mapping' but what is fundamental is that this detail implies that 
instances $I$ of \gatUw in \catcw are completely
determined by their mapping of the introductory rules of \gatU. 
This is the equivalent, in the generalised algebraic case, of 
 the fact that, in the case of algebraic or first-order  theories, interpretations
are determined by a consistent mapping of the symbols within the signature.

\note
Such an instance we also say is an internal $\gat[U]$-structure in the contextual category $\catc$. 

\note
The category of internal $\gat[U]$-structures is defined to be the category whose objects \commentary{\highlight{not so!}}
are pairs $\tuple{\catc,I}$ where \catcw is a contextual category and $I$ is an instance of the theory $\gat[U]$ in the contextual category \catcw and whose morphisms between $\tuple{\catc,I}$ and $\tuple{\catc',I'}$ are pairs $\tuple{F, \eta}$ where
$F: \catc \morph \catc'$ is a contextual functor and $\eta: I \circ F \morph  I'$ is a natural transformation. 


\note 
An alternative definition and one that is offered  in my thesis 
(though the terminology is different\footnote{In my thesis I use the term `model' rather than `instance'  but here I am trying to avoid such use of the term `model'}) is that an internal $\gat[U]$-structure in a contextual category \catcw is precisely 
a contextual functor from the contextual category $\CofU$\  to \catc. 
Meta-mathematically the two definitions are equivalent\footnote{The proof that they are equivalent 
definitions is entirely straightforward though I didn't write it up my thesis. I did once write out the proof of a  corresponding lemma in regard to single-sorted algebraic theories; this was presented in my Msc dissertation.}.
\note
In this way, the category of internal $\gat[U]$-structures  is isomorphic to the coslice category
$\CofU \downarrow \catCon$. Needless to say this has an initial object which is the identity functor on  $\CofU$.
If\ $\gat[U]$ is a considered a type theory (whatever that is) then this initial object is what I believe Vladimir refers
to as the term model when speaking of the initiality conjecture. It is the contextual category
$\CofU$ along with the trivial instance of $\gat[U]$ in $\CofU$.

\note 
Instances of a theory $\gat[U]$ in the contextual category $\Fam$ are said to be $\gat[U]$-algebras. 
\end{oldtt}


