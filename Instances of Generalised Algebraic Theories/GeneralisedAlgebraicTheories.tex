\label{generalisedalgrbraictheories}

In this section we recap the definition  of the class of generalised algebraic theories 
as described in detail in \cite{Cartmell78}  and published in summary in \cite{Cartmell86}.
We shall adopt a few changes of terminology.

A generalised algebraic theory consists of a set of symbols each with an introductory rule and a set of axioms. 
Each axiom is an equality rule. We can consider the symbols plus introductory rules within a generalised algebraic theory to be analogous to the signature of an elementary theory in that
the introductory rule for a symbol defines how that symbol may properly be used, as does, within the context of an elementary theory, the arity of a function or predicate symbol within a signature.

From the given rules of a generalised algebraic theory, 
principles of derivation given in \cite{Cartmell86} enable further rules
to be derived and in this way for any generalised algebraic theory 
there is defined the set of all derived rules of the theory. 
This set of derived rules of a generalised algebraic theory is analogous to the set of provable sentences of an elementary theory.
  
Each rule is expressed as the combination of a  \term{Premise} and a \term{Conclusion} 
and these may be arranged on the page in a number of ways as suits the occasion or as befits personal preference. 
In the metamathematical arguments we prefer to write the premise above the conclusion, 
so \gatdisplayrule{Premise}{Conclusion}, but often, in examples, we write  the rule on a single line. 
If we write the premise before the conclusion then we separate by a turnstyle ($\tstyle$). 
It is also possible to foreground the conclusion by writing it before the premise and to group rules having some shared context --- the 
examples in section \ref{examples} are presented in this way.
Of course, in any informal presentation all that is important is that the formal structure shows through.  

In the syntax that we use,  that an expression $t$ represents an instance of a type represented by expression $\Delta$ is written as $\ofT{t}{\Delta}$. Other authors have prefered to use colon at this point in the syntax  and to instead write $t:\Delta$ and such a choice has the benefit of overlap with the syntax
of programming languages, which admittedly does seem appropriate. In our chosen  syntax 
the premise of a rule  takes the form $\xDelta{n}$, for $n \geq 0$, where $\xn$ is a sequence of distinct variables and each $\Delta_i$ is an expression.


Rules can be of four different forms. They either:

\begin{enumerate}[(i)]
\item Assert that an expression $\Delta$ represents a type within the context provided by the premise. For such a rule we write
\gatdisplayrule{\xDelta{n}}{\isT{\Delta}}. Such rules we have referred to as  \Trules in the past but other authors refer to them as \term{type judgements}.

\item Assert that an expression $t$ represents an instance of a type represented by an expression $\Delta$ within the context provided by the premise. Such a rule is written as
\gatdisplayrule{\xDelta{n}}{\ofT{t}{\Delta}}. Such a rule we refer to as an \trule.

\item Assert that two expressions, $\Delta$ and $\Delta'$ represents identical types within the context provided by the premise. Such a rule is written as 
\gatdisplayrule{\xDelta{n}}{\Delta=\Delta'} and we refer to as a \Teqrule.

\item Assert that two expressions, $t$ and $t'$ represent identical instances of a type represented by an expression $\Delta$  within the context provided by the premise. Such a rule is written as 
\gatdisplayrule{\xDelta{n}}{t=\ofT{t'}{\Delta}} and we  refer to it as an \teqrule.
\end{enumerate}


A \term{pretheory} is defined as a collection of symbols each with an introductory rule and a set of axioms. A pretheory is defined to be \term{well-typed}\footnote{In \cite{Cartmell86}, I define what it is for a rule written in the alphabet of a generalised algebraic theory to be \textit{well-formed}. Here I will use the adjective \textit{well-typed} instead.} iff all its introductory rules and axioms are. A \term{generalised algebraic theory} is defined to be a well-typed pretheory. 
With this revised terminology, definition 2(a) from \cite{Cartmell86}, states that in  a generalised algebraic theory $U$:
\begin{enumerate} [(i)]
\item 
a \Trule \gatdisplayrule{\xDelta{n}}{\isT{\Delta}} is well-typed  iff 
\gatdisplayrule{\xDelta{n-1}}{\isT{\Delta_n}} is a derived rule of $U$ and $x_n$ is distinct from all of $x_1,...x_{n-1}$, 
\item 
an \trule \gatdisplayrule{\xDelta{n}}{\ofT{t}{\Delta}} is well-typed iff
the rule \gatdisplayrule{\xDelta{n}}{\isT{\Delta}} is a derived rule of $U$,
\item 
a \Teqrule \gatdisplayrule{\xDelta{n}}{\Delta=\Delta'} is well-typed iff
both \gatdisplayrule{\xDelta{n}}{\isT{\Delta}} and \gatdisplayrule{\xDelta{n}}{\isT{\Delta'}} are derived rules
of $U$,
\item 
an \teqrule \gatdisplayrule{\xDelta{n}}{t=t' \in \Delta} is well-typed iff
both \gatdisplayrule{\xDelta{n}}{\ofT{t}{\Delta}} and \gatdisplayrule{\xDelta{n}}{\ofT{t'}{\Delta}} are derived rules
of $U$.
\end{enumerate}



In a theory $U$, we say that a compound expression of the form $\xDelta{n}$, where $xn$ are variables, is a \term{context} (within theory $U$)  iff
the rule \gatdisplayrule{\xDelta{n-1}}{\isT{\Delta_n}} is a derived rule of $U$. It follows (lemma \ref{contextlemma}, below)  that  $\xDelta{n}$ is a context iff
it is the premise of some derived rule of $U$.

A \term{realisation} of one context from another is defined as follows: if $\xDelta{n}$ and $\yOmega{m}$ are contexts of a generalised algebraic theory $U$  then a \term{realisation} of  $\yOmega{m}$ with respect to $\xDelta{n}$ is an $m$-tuple of expressions $\tuple{s_1,...s_m}$
such that \foreachj, the rule \gatdisplayrule{Q}{\ofT{s_j}{\Omega_j[s_1|y_1,...s_{j-1}|y_{j-1}]}} is a derived rule of $U$.

There are a number of housekeeping lemmas summarised in \cite{Cartmell86}. 
Here it is appropriate to expand on that summary and to name some lemmas that were previously unnamed. The proofs of all of these lemmas 
are given in \cite{Cartmell78}.
In the following, assume that $U$ is a generalised algebraic theory.


\begin{lemma}
\llabel{everyintroruleisderived}
Every introductory rule of $U$ is a derived rule of $U$.
\end{lemma}
\begin{proof}
Can be shown by first showing that for every context $\xDelta{n}$ of $U$, 
\foreachi, the rule 
\gatdisplayrule{\xDelta{n}}{\ofT{x_i}{\Delta_i}} is a derived rule of $U$.
\end{proof}



\begin{lemma}[Context Lemma]
\llabel{contextlemma}
\begin{enumerate}[(i)]
\item the premise of a derived rule of $U$ is a context in $U$.
\item if $\xDelta{n}$ is a context, then \foreachi, the rule \gatdisplayrule{\xDelta{i-1}}{\isT{\Delta_i}} is a derived rule of $U$ i.e.
$\xDelta{i}$ is a context.
\end{enumerate}
\end{lemma}
\begin{proof}
(i) is proved by induction on the derivation of the rule by examining each principle of induction in turn. (ii) follows from repeated use of (i).
\end{proof}

The Substitution Lemma states that every type-correct substitutional instance of a derived rule is a derived rule:
\begin{lemma}[Substitution Lemma]
\llabel{gatsubstitutionlemma}  %label changed to make it unqiue 2 Nov 2021
For $m \geq 1$, if \gatdisplayrule{\yOmega{m}}{Conclusion} is a derived rule of $U$
and  if $\tuple{\sm}$ is a realisation of the context $\yOmega{m}$ wrt some context $Q$ 
then \gatdisplayrule{Q}{Conclusion[s_1|y_1,...s_m|y_m]} is a derived rule of $U$, where
for any expression $e$ we write
$e[s_1|y_1,...s_m|y_m]$ to mean
the result of substituting each instance of variable $y_j$ in $e$ by $s_j$, \foreachj.
\end{lemma}

The following lemma may be found on page 1-33 of \cite{Cartmell78}:
\begin{lemma}[The Well-Typedness Lemma]
\llabel{welltypednesslemma}
Every derived rule of a generalised algebraic theory $U$ is well-typed.
\end{lemma} 
\noindent and this is followed by:
\begin{lemma}[The Derivation Lemma]
\llabel{derivationlemma} 
\begin{enumerate}[(i)]
\item Every derived \Trule of a generalised algebraic theory $U$ is of the form
\gatdisplayrule{\yOmega{m}}{\isT{A(\tn)}} for some sort symbol $A$ of $U$ with introductory rule of the form
\gatdisplayrule{\xDelta{n}}{\isT{A(\xn)}} and for some expressions $\tn$ such that \foreachi, the rule
\gatdisplayrule{\yOmega{m}}{\ofT{t_i}{\Delta_i[t_1|x_1,...t_{i-1}|x_{i-1}]}} is a derived rule of $U$.
\item Every derived \trule of $U$ is 
either of the form \gatdisplayrule{\xDelta{n}}{\ofT{x_i}{\Omega}} for some $n \ge 1$, for some $i$, $1 \leq i \leq n$, 
and for some $\Omega$ such that \gatdisplayrule{\xDelta{n}}{\Delta_i=\Omega} is a derived rule of $U$
or is of the form
\gatdisplayrule{\yOmega{m}}{\ofT{f(\tn)}{\Omega}} for some operator symbol $f$ of $U$ 
with introductory rule of the form
\gatdisplayrule{\xDelta{n}}{\ofT{f(\xn)}{\Delta}} 
and for some expressions $\tn$ such that \foreachi, the rule
\gatdisplayrule{\yOmega{m}}{\ofT{t_i}{\Delta_i[t_1|x_1,...t_{i-1}|x_{i-1}]}} is a derived rule of $U$
and such that
\gatdisplayrule{\yOmega{m}}{\Delta[t_1|x_1,...t_n|x_n]=\Omega} is a derived rule of $U$.
\end{enumerate}
\end{lemma}

Lemma 4 of section 1.7 of \cite{Cartmell78} (page 1.37) may be summarised as saying that if the premise of a derived rule is weakened then the resulting rule is a derived rule. It is expressed as follows:
\begin{lemma}[The Weakening Lemma]
\llabel{weakeninglemma}
If
\gatdisplayrule{\xDelta{n}}{\isT{\Delta}}  and
\gatdisplayrule{\xDelta{n},\,\yOmega{m}}{Conclusion} are both derived rules of $U$ then if $z$ is a variable
distinct from $\xn,\ym$ then
the (weakened) rule \\
\gatdisplayrule{\xDelta{n},\,z \in \Delta,\,\yOmega{m}}{Conclusion} is a derived rule
of $U$.
\end{lemma}

The following lemma suggests an  alternative way of defining the notion of a generalised algebraic theory. 
This is Lemma 3 of section 1.7 of \cite{Cartmell78} (page 1.36):
\begin{lemma}[The Stratification Lemma]
\llabel{stratificationlemma}
 For each generalised algebraic theory $U$  there is a sequence of theories 
$U_0 \subseteq $U$_1 \subseteq $U$_2 \subseteq ...$ such that  \inlinedisplay{$U$ = \bigcup_i $U$_i}
and such that each $U_{i+1}$ is a simple extension of $U_i$ in the sense that each introductory rule and axiom of $U_{i+1}$ is a well-typed  with respect to $U_i$.
\end{lemma}
This tells us that those pretheories that are well-typed (i.e. those that are generalised algebrauc theories) are those that can be expressed as the union of a chain of theories in which each theory is provably well-typed with respect to the previous theory in the chain.  
Expressing generalised algebraic theories by way of such chains would be a good approach to follow
for any environment for managing and machine checking generalised algebraic  rules and derivations.

\begin{example}
If $U$ is a single or multi-sorted algebraic theory considered as an example of a
 generalised algebraic theory then $U$ can be stratified as $U_0 \subseteq U_1 \subseteq U_2=U$
where $U_0$ defines the sort symbol(s), $U_1$ defines the operator symbols and $U_2$ defines the axioms. 
\end{example}

\begin{example}
 The generalised algebraic theory $cc$ of categories stratifies as: $cc_0$ - sort Ob,
$cc_1$ - sort Hom, $cc_2$ - operator symbols $\circ$ and $id$, $cc_3$ - identity and associativity axioms.
\end{example} 

 Finally, note that there is a compactnessl lemma for generalised algebraic theories:
\begin{lemma}[The Compactness Lemma]
\llabel{compactnesslemma}
For each generalised algebraic theory $U$, if r is a derived rule of $U$ then it is a derived rule of some finite subtheory $F \subseteq U$. \
\end{lemma}



