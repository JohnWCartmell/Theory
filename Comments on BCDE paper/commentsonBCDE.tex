
\note  Metamathematical Background. Predating Lawvere, metamathematics has well established paradigms for use of the terms
`theory', `signature', `interpretation' and  `model'.
In this tradition, the notion of a `theory' or of a certain class of theories is  defined syntactically.
The most important class of theories in  metamathematics is the class of `elementary theories' 
which is the class of theories in first-order predicate logic with equality. Such a theory consists of a signature plus a set of axioms. 
The signature describes the predicates and functions symbols and their arities. 
The axioms consist of a set of closed \term{well formed formualas} (wffs) in the language defined by the first-order predicate calculus (with equality)
 using only predicate symbols and function symbols from the signature and used with the arities as defined in the signature. 

\note An `interpretation' is defined to be a mapping of  a signature to actual predicates and actual functions over some domain subject to the requirement that n-ary predicate symbols are mapped to n-ary predicates and n-ary function symbols are mapped to n-ary functions. For a discussion of first order theories and their interpretations see \cite{Mendelson}.
Such an interpretation induces an interpretation of all
closed \term{well formed formualas} (wffs) of the signature as truth values (Tarski's satisfaction definition). 
A `model' is an interpretation in which all the axioms are mapped to true. 
Note that the term `signature' is only ever used in this sense (apart from in your paper, that is). 
It is never used as a homonym for `theory'.  For the definition of `signature' see, for example, Wilfred Hodges. https://plato.stanford.edu/entries/modeltheory-fo/.
\note
As an aside I should mention that there is another usage for the term `theory' in some metamathematical situations. I think `theory' is sometimes used to refer to any set of sentences (of a given signature) that is closed under logical deduction\footnote{One such theory in this sense is the total set of true sentences of a model. 
I was brought up to call this the `diagram of the model'.  
In fact I have a vivid memory of a conference in the 1970's of hearing from a German set theorist (the  name escapes me)
 that "God doesn't need Logic -- he has the diagram of the universe". He meant, of course, 'diagram' in the sense of 'set of all true sentences'. I was quite struck by this.}.
\note
Note that the distinction between `signature' and `theory' is significant because it makes possible a definition of `model' 
via a definition of `interpretation'. An `interpretation' is an `interpretation of a signature'. 
A model is an interpretation in which all axioms are satisfied i.e. have truth value `true'.   
\note
What is always meant by an `algebraic theory' prior to Lawvere is a first order theory in a signature in which there are no predicate symbols and in which every axiom is an equational identity between open terms with all variables universally quantified. Thus the notion of `algebraic theory' is  
 a special case of the notion of `elementary theory' or `first-order theory (of predicate calculus)' and the 
definitions of `interpretation' and `model' for `algebraic theories' are just as given in the broader context of elementary theories specialised to the narrower context. Likewise  `Horn theories' and `essentially algebraic theories' are special cases of `elementary theory'. 
\note
My definition of `generalised algebraic theory' follows this tradition.  In the case of generalised algebraic theories, though, the defintions are not quite so clear cut. I do not define such a thing as the `signature of a generalised algebraic theory' but if I were to do so it would be as
the set of symbols of the theory along with their introductory rules (the equivalent of their arities). 
The approach to defining what a generalised algebraic theory is cannot simply be via a definition of what a signature is followed by a defintion 
of a theory as a signature plus somne axioms in the language of the signature. This is because for a signature to be valid we need knwoledge of derived rules of the theory -- in otherwords the notions of theory and signature are interdependent. My solution is described (and the difficulty itself) 
is described in my APAL paper. 

\note The impossibility of predefinition of the notion of `signature (to be used in a generalised algebarc theory)' as a consequence
makes it impossible to define the notion of `interpretation of a signature (for a generalised algebraic theory)' before defining what a 'model of a generalised algebraic theory' is. 
\note 
We conclude this discussion of background by saying that in the generalised algebaric case we cannot exactly follow the path used in defining 
first order theories and their models and algeabrauic theories and their models. What we have done in the APAL paper is is to follow it as closely as we are able. 
\note

\note
Reading your note brought me to writing  an extended set of comments and explanations regarding gats, contextual categories and categories with families.
These then took on a life on their own and I have already sent you these separately:
\begin{itemize}
\item Note on Generalized Algebraic Theories and Categories with Families - 14 Feb 2021
\item Research Overview - 20 May 2021
\item Instances of Generalised Algebraic Theories in Contextual Categories - 26 May 2021
\end{itemize}

\note
I have been vacillating over the naming of `instance of generalised algebaric theories'. Writing the background notes above I am flipping back to
thinking that maybe I should be using the traditional term `interpretation'. I am however reluctant to use the traditional term `model' for reasons that I describe in that paper. Maybe I can with definitions that give me  `an instance' is a valid `interpretation'. 
Whatever, please note that ``Instances of Generalised Algebraic Theories in Contextual Categories'' needs further work -- especially so
since there are some errors in the details of the proofs. I will send you the next version when available.
\note
On page 2 you write 
\begin{tightquote}
The main novelty of this note is to use CwF
as a framework to define a new notion of a generalized algebraic theory.
\end{tightquote}
which leads me to ask whether you really intend a \textit{new notion} of generalized algebraic theory 
or whether you intend a \textit{new definition} of an old notion? 
If it is a new notion that you intend then why do you use the same name (generalized algebraic theory) for something that is intended to be different?
On the other hand if you intend a new definition of an old notion (if this is your claim) then I don't believe that you have substaniated this claim.

\note
From your abstract it seems clear that you intend a new definition of an old notion:
\begin{tightquote}
We give a new syntax independent definition of 
the notion of a finitely presented generalized algebraic theory...
\end{tightquote}


\note 
Is a theory something syntactic or something structural\footnote{I use the term \textit{structural} in this discussion rather than 
the term \textit{semantic} because Lawvere uses this term and describes
algebraic structure and algebraic semantics as adjoint functors and so for him it seems structural and semamntic are not the same thing. }
(algebraic or algebraic-like)?
Note that the langauge in this area was somewhat stymied at the off (in 1963) by Lawvere when he
named his categorical theories that are equivalent to single-sorted equation theories `algebraic theories'. 
From then on we had two versions of what we meant by algebraic theory.
  
Maybe Lawvere had a revolution in mind and he wanted to take over the  old language and give it new denotation. 
Not sure this is realistic though -- no one in practice describes an actual particular algebraic theory (the theory of groups, for example) by describing one of Lawvere's algebraic theories (an infinite thing)  -- theories in practice are expressed syntacically.  Lawvere in his published work at least relied on us to see the connection. This shouldn't really be acceptable. 

Suppose however that we wish to follow  Lawvere's linguistic lead and give new structural definition of generalised algebaric theory i.e. one that is
non-syntactic. Well we already have numerous such definitions  --  contextual categories -- C-systems -- contextual categories with families -- B-systems -- meta-gat algebras.  

\note
Your defn. of generalized algebraic theory.
I struggle a bit with your main definition and even knowing whether it is a definition as written in section 3. 
In the penultimate paragraph of the introduction you say that you \textit{define
a new finitely presented notion of generalised algebraic theory and ...}. But in section 3 I find  any definition of `finitely presented generalised algebraic theory'.  
What I do read is
``we define inductively how to build a valid $\Sigma$ struucture''.
The langauge in this area was somewhat stymied at the off (in 1963) by Lawvere 
naming his things `algebraic theories'. No doubt the revolution depended on 
taking over the old language and giving it new denotation. Not sure this helps anybody though -- no one in practice describes an actual particular algebraic theory by describing one of Lawvere's algebraic theories  -- theories in practice are expressed syntacically. If you do want a revolutionary definition in line with Lawvere then we have numerous such definitions --  contextual categories -- C-systems -- contextual categories with families -- B-systems -- meta-gat algebras.   

Is your definition closer to being a description of how to construct a contextual category
(with or without families) by generators and relations?

\note
I also had real trouble understanding your example of the generalised algebraic theory of monoids 
and relating this to the description in section 3 of ``how to build a valid $\Sigma$ struucture''.
I had trouble knowing whether this was the theory of monoids or the theory of internal monoids.
The text seems ambiguous. In trying to understand this I have written some notes which, among other things, 
delineate the generalised algebaruc theory of monoids from the generalised algebraic theory
of internal monoids. I won't repeat that discusion here. 
One thing I understand that you might be saying is that it is better to work with theories of internal U-objects rather than U-objects
for any generalised algebraic theory U. I can't see any reason your approach might be cannonical, by the way, though I am open to suggestions as to why.

\note 
Your section EXAMPLES OF GENERALIZED ALGEBRAIC THEORIES
You  shouldn't refer to theory of internal categories, theory of internal monoids etc. you should refer to the
theory of categories, theory of monoids and so on. What you have defined as the theory of monoids for example has
as algebras (i.e. has models in Fam) exactly sets with monoid structure i.e. monoids. When the same theory is interpreted in other cwfs then
the models are monoids internal to these other cwfs. In other words the generalised algebraic theory of xyz's is something that has xyz's as models in Fam and has internal xyz's as models in other categories. 

\note
The signature of a theory consists of s set of sorts and a set of operations. It does not include equations. 

\note
In your paper your construction of a `gat' from a signature seems to include sorts, operators AND equations. Because 
what is included in your signatures differs from what other authors include in signatures it make it harder to understand.

\note
I cannot see a reason for excluding type equations from your presentation.
