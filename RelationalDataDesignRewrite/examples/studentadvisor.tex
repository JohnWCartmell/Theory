\begin{figure} [h]  % student advisor
\begin{center}
\begin{tabular}{c c}
$
\begin{array}{cp{0.05cm}c  p{0.05cm}c p{0.5cm}c}
                & & \Rnode{d}{d} & &              & &             \\ [0.3cm]
								& &              & &              & & \Rnode{v}{v} \\ [0.6cm]     
	 \Rnode{s}{s} & &              & & \Rnode{p}{p} & &             
\end{array}
$
\ncarr{s}{d} 
\alabel{S_1}
\idcomp
\ncarr{d}{v} 
\alabel{Id}
\idcomp
\ncarr{p}{v} 
\blabel{PNo}
\idcomp
\ncarr{p}{d} 
\blabel{R_1}
\idcomp
\ncline[linestyle=dashed,nodesepA=\arrnodesepA,nodesepB=\arrnodesepB]{->}{s}{p} 
\blabel{R_0}
\ncarr[-90]{s}{v} 
\blabel{sNo}
\idcomp
& \footnotesize
\begin{tabular}{c p{1.5cm} p{4cm}}
KEY && \\
\hline
d & department & Having identifying attribute Id the department identifier. \\
s & student & Identified by relationship $S_0$ to department enrolled in and  attribute student number SNo. \\
p & professor & Identified by their relationship $R_1$ to a department and their professor number attribute pNo yyy \\
R0 & advised by & Represents the relationship\footnote{How significant is it that this may be optional?Need draw this out.} between a student and a professor.\\
\end{tabular} 
\end{tabular}
\end{center}
\caption{Following the pattern of the property X Example. This example is based on one given Shlaer-Long. Here we specify that a student optionally has an advisor (suppose that the advisor is selected part way through a course). As before we assume that an advisor must be a professor of the department in which the student is enrolled. }
\label{studentadvisorgraph}
\end{figure}