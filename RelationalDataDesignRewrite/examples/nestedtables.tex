\begin{figure} [h] % nested tables
\begin{center}

\begin{tabular}{c c}
$
\begin{array}{cp{0.5cm}cp{0.5cm}cp{0.3cm}c}
            &&              && \Rnode{v}{v} &&              \\[0.9cm]
            && \Rnode{t}{t} &&              &&              \\[0.5cm] 
\Rnode{r}{r}&&              && \Rnode{c}{c} &&              \\[0.5cm]     
	          && \Rnode{d}{d} &&              &&               
\end{array}
$
\nccircle[angleA=20, nodesep=3pt]{<-}{t}{.4cm}
\blabel{P}[0.7]
\ncarr{r}{t} 
\alabel{S_1}
\idcomp
\ncarr[-35]{t}{v} 
\alabel{tId}
\idcomp
\ncarr[-30]{c}{v} 
\blabel{cNo}
\idcomp
\ncarr{d}{c}
\blabel{R_0}
\idcomp
\ncarr{d}{r}
\alabel{S_0}
\idcomp
\ncarr{c}{t}
\blabel{R_1}
\idcomp
%\ncarr[80]{r}{v}
\nccurve[angleA=120, angleB=170, ncurv=0.9, nodesep=0.1, arrowsize=5pt,arrowinset=0.7]{->}{r}{v}
\alabel{rNo}[0.25]
\idcomp

& \footnotesize
\begin{tabular}{c p{1.5cm} p{3cm}}
KEY && \\
\hline
t & table & Having identifying attribute tNo the name of the table. \\
c & column & Identified by a combination of column number cNo and relationship $R_1$ to the table it is a column of.\\
r & row & Identified by its row number $rN$ and its relationship $S_1$ to the table it is a row of.\\
d & data cell & Identified by relationship $S_0$ to the row it is in and relationship $R_0$ to the row it is in. \\
\end{tabular} 
\end{tabular}
\end{center}
\caption{Nested Tables. Tables are nested within other tables. This is an example of an ER model which is well formulated but which exhibits a transitive functional dependency
$S_0/S_1/tId \morph R_0/R_1/tId$
between attribute-like paths which, because of the following chain of functional dependencies:
$
S_0/S_1/tId \morph S_0/S_1/P \morph S_0/S_1/P/P \morph S_0/S_1/P/P/P  ... \morph R_0/R_1/tId
$
does not factor through an intransitive dependency. This model determines a relational design that is not in 3NF. 
}
\label{datatablegraph}
\end{figure}

This example determines a relational design with the following
relations and inclusion dependencies:
\begin{equation}
table(\underline{tableId},
\underline{parentTableId})
\end{equation}
\begin{equation}
row(
\underline{containingTableId},
\underline{rowNo}
)
\end{equation}
\begin{equation}
row[containingTableId] \subseteq table[tableId]
\end{equation}
\begin{equation}
column(\underline{outermostTableId},
\underline{columnNo})
\end{equation}
\begin{equation}
column[outermostTableId] \subseteq table[tableId]
\end{equation}
\begin{equation}
datum(\underline{containingTableId},
\underline{rowNo},
\underline{outermostTableId},
\underline{columnNo},value)
\end{equation}
\begin{equation}
dataum[containingTableId,rowNo] \subseteq row[containingTableId,rowNo]
\end{equation}
\begin{equation}
dataum[outermostTableId,columNo] \subseteq column[outermostTableId,columNo]
\end{equation}
Additionally the $dataum$ relation has the following functional dependency:
\begin{equation}
\sfd{containingTableId,rowNo}{outermostTableId}
\end{equation}
and so $datum$ is not in normal form. To normalise create replace dataum 
by two relations:
\begin{equation}
datum(\underline{containingTableId},
\underline{rowNo},
\underline{columnNo},value)
\end{equation}
and
\begin{equation}
containingTable(\underline{tableId},
\underline{outermostTableId}
\end{equation}

This amounts to the following revised ER model:

$
\begin{array}{cp{0.5cm}cp{0.5cm}cp{0.3cm}c}
            &&                && \Rnode{v}{v} &&              \\[1.1cm]
            && \Rnode{t}{t}   &&              &&              \\[0.9cm] 
						&& \Rnode{ct}{ct} &&              &&              \\[0.5cm] 
\Rnode{r}{r}&&                && \Rnode{c}{c} &&              \\[0.5cm]
	          && \Rnode{d}{d}   &&              &&               
\end{array}
$
\nccircle[angleA=20, nodesep=3pt]{<-}{t}{.4cm}
\blabel{P}[0.7]
\ncarr{ct}{t} 
\alabel{I}[0.4]
\idcomp
\ncarr[50]{ct}{t}
\alabel{O}[0.4]
\ncarr{r}{ct} 
\alabel{S_1}
\idcomp
\ncarr[-35]{t}{v} 
\alabel{tId}
\idcomp
\ncarr[-30]{c}{v} 
\blabel{cNo}
\idcomp
\ncarr{d}{c}
\blabel{R_0}
\idcomp
\ncarr{d}{r}
\alabel{S_0}
\idcomp
\ncarr{c}{t}
\blabel{R_1}
\idcomp
%\ncarr[80]{r}{v}
\nccurve[angleA=120, angleB=170, ncurv=0.9, nodesep=0.1, arrowsize=5pt,arrowinset=0.7]{->}{r}{v}
\alabel{rNo}[0.25]
\idcomp

Essentially the entity type $ct$ represents the image of $S_0/S_1$
as a sub object of entity type $t$. Now the functional dependency
$\sfd{S_0/S_1}{R_0}/{R_1}$ is represented by the relationship $O$
and we have the equivalence $S_0/S_1/O \simeq R_0/R_1$. The $dataum$
relation determined by entity type $d$ now does not contain an attribute
corresponding to $R_0/R_1/tId$ because this path is subsumed by the path
$S_0/S_1/O/tId$ and this path is not attribute-like
as the relationship $O$ is non-identifying. \commentary{\textit{voil\'a.}}

In this revised ER model there still is a 
a transitive dependency, $S_0/S_1/I/tId \morph R_0/R_1/tId$ between attribute-like paths 
and a chain of intermediate dependencies 
\begin{equation}
S_0/S_1/I/tId \morph S_0/S_1/I/P \morph S_0/S_1/I/P/P \morph S_0/S_1/I/P/P/P  ... \morph R_0/R_1/tId
\end{equation}
and now there is a intransitive dependency $S_0/S_1/O \morph R_0/R_1/tId$ which has a place in this chain like so:
\begin{equation}
S_0/S_1/I/tId \morph S_0/S_1/I/P \morph S_0/S_1/I/P/P \morph S_0/S_1/I/P/P/P  ... \morph S_0/S_1/O \morph R_0/R_1/tId
\end{equation}

and therefore the dependency $S_0/S_1/I/tId \morph R_0/R_1/tId$ can be right transitively factored.
Note that there is still a transitive functional dependency that does not factor right transitively:
\begin{equation}
S_0/S_1/I/tId \morph S_0/S_1/I/P \morph S_0/S_1/I/P/P \morph S_0/S_1/I/P/P/P  ... \morph S_0/S_1/O
\end{equation}
 but this now isn't between attribute-like paths. The revised model meets the condition required for the determined relational model to be in normal form.


