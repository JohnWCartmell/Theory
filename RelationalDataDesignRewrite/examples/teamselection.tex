\begin{figure} [h]
\begin{center}
\begin{tabular}{c c}
\begin{tabular}{c p{1.5cm} c}
   \Rnode{p}{p} & & \Rnode{v}{v}
\end{tabular}
\nccircle[nodesep=3pt]{<-}{p}{.4cm}
\blabel{S}
\ncarr[-40]{p}{v}
\alabel{c}[0.4]
\ncarr{p}{v} 
\alabel{pId}
\idcomp
& \footnotesize
\begin{tabular}{c p{1.5cm} p{4cm}}
KEY && \\
\hline
p & person & Identified by id attribute ($pId$). \\
s & selects & each person selects one other person \\
c & colour & each person is given a coloured vest 
\end{tabular} 
\end{tabular}
\end{center}
\caption{Team Selection Example. This is an example of an ER model 
which is well formulated but which exhibits a transitive functional dependency which does not factor through an intransitive dependency.
}
\label{teamselectionexample}
\end{figure}

In this example the colour of a person's vest is the same as the 
colour of vest of the person they select to be in the same team, which is to say that  the diagram
$
\begin{array}{c p{0.75cm} c}
   \Rnode{p1}{p}  & &                  \\[0.5cm]
	                 & &    \Rnode{v}{v} \\[0.5cm]
   \Rnode{p2}{p}  & &
			
\end{array}
$
\ncarr{p1}{v}
\alabel{c}[0.4]
\ncarr{p2}{v}
\blabel{c}[0.4]
\ncarr{p1}{p2}
\blabel{S}[0.4]
commutes. 


Consider the functional dependency
\begin{equation*}
\sfd{S}{c}
\end{equation*}
sourced at entity type $p$. It is transitive
because though we also have functional dependencies
\begin{equation*}
\sfd{S}{S/S}
\end{equation*}
and
\begin{equation*}
\sfd{S/S}{c}
\end{equation*}
The latter is transitive because we also have
\begin{equation*}
\sfd{S/S}{S/S/S}
\end{equation*}
and
\begin{equation*}
\sfd{S/S/S}{c} 
\end{equation*} 

We have :

\begin{equation}
pId \morph S \morph S/S \morph S/S/S \morph S/S/S/S .. \morph c
\end{equation}
The model has a transitive functional dependency $pId \morph c$ between
attribute-like paths which cannot be factored right-intransitively. Therefore the model doesn't meet
the conditions to establish that the relational design it determines  be
in 3NF.

In the relational design the person entity type, abbreviated $p$ in the diagram, maps to the following relation:
\begin{equation}
\label{personrelation}
person(\underline{pId}, spId, c)
\end{equation}
The  attribute of this relation are related by the following  functional dependencies:
\begin{equation}
\sfd{pId}{spId}
\end{equation}
\begin{equation}
\label{colourfd}
\sfd{pId}{c}
\end{equation}
\begin{equation}
\sfd{spId}{c}
\end{equation}
and the following  inclusion dependency:
\begin{equation}
\label{spIdcolour1}
person[spId,c] \subseteq person[pId,c]
\end{equation}

The $person$ relation (\ref{personrelation}) is not in 3NF because the dependency (\ref{colourfd}) is transitive but its rhs, $c$, is not a key attribute wrt the relation. However this
relation (\ref{personrelation}) can be normalised into relations (\ref{person1relation}) and (\ref{person2relation}):
\begin{equation}
\label{person1relation}
selected(\underline{spId},  c)
\end{equation}
\begin{equation}
\label{person2relation}
person(\underline{pId}, spId)
\end{equation}
satisfying the inclusion dependencies
\begin{equation}
selected[spId] \subseteq person[pId]
\end{equation}
and
\begin{equation}
\label{spIdcolour}
selected[spId,c] \subseteq (selected \bowtie_{spId=pId} 
                                       (person \bowtie selected [pId,c]) )
																			[spId,c];
\end{equation}
The latter inclusion dependency is a database constraint that
is more difficult to police in an implementation
than the prior equivalent (\ref{spIdcolour1}) and
this gives one advantage to the unnormalised design as generated
directly from the entity model. 

A revised model which reflects the normalised relational design is this:
\begin{center}
\begin{tabular}{c p{1.5cm} c}
   \Rnode{sp}{sp} & &           \\[1.4cm]
   \Rnode{p}{p}   & & \Rnode{v}{v}
\end{tabular}
\ncarr[40]{p}{sp}
\alabel{S}
\ncarr{sp}{p}
\alabel{I}
\idcomp
\ncarr[20]{sp}{v}
\alabel{c}[0.4]
\ncarr{p}{v} 
\alabel{pId}
\idcomp
\end{center}

In the revised model we have
\begin{equation}
I/pId \morph I/S \morph I/S/S \morph I/S/S/S ... \morph c
\end{equation}
and $I/pId \morph c$ is an transitive fd between attribute-like paths which cannot
be factored right-intransitively. \commentary{So how come the determined relational design is in normal form?}

