
\begin{figure} [h]  % 
\begin{center}
\begin{tabular}{c c}
$
\begin{array}{cp{0.7cm}c  p{0.7cm}c }
                & & \Rnode{b1}{b_1} & &                \\ [1.2cm]    
	 \Rnode{a}{a} & & \Rnode{c}{c}    & &    \Rnode{v}{v}\\ [1.2cm]  
					      & & \Rnode{b2}{b_2} & &                 
\end{array}
$
\ncarr{a}{b1} 
\alabel{S_1}
\ncarr{b1}{v} 
\alabel{K_{b_1}}
\idcomp
\ncarr{c}{b1} 
\blabel{Q_1}
\idcomp
\ncarr{a}{b2} 
\blabel{S_2}
\ncarr{b2}{v} 
\blabel{K_{b_2}}
\idcomp
\ncarr{c}{b2} 
\alabel{Q_2}
\idcomp
\ncline[linestyle=dashed,nodesepA=\arrnodesepA,nodesepB=\arrnodesepB]{->}{a}{c} 
\blabel{R}
\nccurve[angleA=-90,angleB=-90,nodesep=2pt,ncurv=1.6]{->}{a}{v}
\blabel{K_a}[0.3][-1]
\idcomp
& \footnotesize
\end{tabular}
\end{center}
\caption{Example of failure of property X.  Suppose  $\tuple{R,Q_1} < \tuple{S_1}$ and $\tuple{R,Q_2} < \tuple{S_2}$.}
\label{propertyXfailureexample}
\end{figure}

Consider the schema shown in figure \ref{propertyXfailureexample} and suppose that $\incd{a}{S_1,S_2}{c}{Q_1,Q_2}$. 
There are two possibilities:
\begin{itemize}
\item 
If relationship $R$ represents the inclusion dependency
$\incd{a}{S_1,S_2}{c}{Q_1,Q_2}$ then the model is well-formed and relational even though it does not have property X.
\item
If $R$ does not represent the inclusion dependency then for the model to be well-formed there ought to be a relationship
$\hat{R}:a \morph c$ that represents inclusion dependency $\incd{a}{S_1,S_2}{c}{Q_1,Q_2}$. \\

\begin{tabular}{ p{8.5cm}  c}
We then have that $R \leq \hat{R}$, for in each instance $E$ we have&\parbox{5cm}{ \begin{align*}
E_R&=E_R \circ E_{\tuple{Q_1,Q_2}} \circ E^{-1}_{\tuple{Q_1,Q_2}} \\
   & \leq E_{\tuple{S_1,S_2}} \circ E^{-1}_{\tuple{Q_1,Q2}} \\
	 & = E_{\hat{R}}
\end{align*}}
\end{tabular}
The first-cut relational schema for this model will include a table $a$ as follows:
\begin{equation}
a(\underline{K_a},\qq{S_1/K_{b_1}},\qq{S_2/K_{b_2}}, \qq{R/Q_1/K_{b_1}}, \qq{R/Q_2/K_{b_2}})
\end{equation}
We have the following restrictions
\begin{align}
\qq{R/Q_1/K_{b_1}} = \overline{\qq{R/Q_2/K_{b_2}}} \circ \qq{S_1/K_{b_1}} \\
\qq{R/Q_2/K_{b_2}} = \overline{\qq{R/Q_1/K_{b_1}}} \circ \qq{S_2/K_{b_2}}
\end{align}
and therefore the following functional dependencies:
\begin{align}
\fd{\qq{R/Q_2/K_{b_2}},\qq{S_1/K_{b_1}}}{\qq{R/Q_1/K_{b_1}}} \\
\fd{\qq{R/Q_1/K_{b_1}},\qq{S_2/K_{b_2}}}{\qq{R/Q_2/K_{b_2}}} 
\end{align}
table $a$ therefore is not in S3NF.
If the first cut model is normalised by addressing the first of these functional dependencies then we 
introduce a table $a_0$ and move $\qq{R/Q_1/K_{b_1}}$ to table $a_0$ so that table $a$ is replaced by:
\begin{align}
a(\underline{K_a},\qq{S_1/K_{b_1}},\qq{S_2/K_{b_2}}, \qq{R/Q_2/K_{b_2}}) \\
a_0(\underline{\qq{R/Q_2/K_{b_2}}},\underline{\qq{S_1/K_{b_1}}}, \qq{R/Q_1/K_{b_1}} )
\end{align}
but wierdly now, on table $a_0$,  $\qq{S_1/K_{b_1}} \simeq \qq{R/Q_1/K_{b_1}}$ and so table $a_0$ becomes
\begin{align}
a_0(\underline{\qq{R/Q_2/K_{b_2}}},\underline{\qq{S_1/K_{b_1}}} )
\end{align}
abstracting to a logical model we get:

\begin{figure} [h]  % 
\begin{center}
\begin{tabular}{c c}
$
\begin{array}{cp{0.7cm}c p{0.7cm}c p{0.7cm}c}
                & & \Rnode{b1}{b_1} & &                & &                 \\ [1.2cm]    
	 \Rnode{a}{a} & & \Rnode{a0}{a_0} & &  \Rnode{c}{c}  & &    \Rnode{v}{v} \\ [1.2cm]  
					      & & \Rnode{b2}{b_2} & &                & &             
\end{array}
$
\ncarr{a}{b1} 
\alabel{S_1}
\ncarr[30]{b1}{v} 
\alabel{K_{b_1}}
\idcomp
\ncarr{c}{b1} 
\blabel{Q_1}[0.35]
\idcomp
\ncarr{a}{b2} 
\blabel{S_2}
\ncarr[-30]{b2}{v} 
\blabel{K_{b_2}}
\idcomp
\ncarr{c}{b2} 
\alabel{Q_2}[0.35]
\idcomp
\ncline[linestyle=dashed,nodesepA=\arrnodesepA,nodesepB=\arrnodesepB]{->}{a}{a0} 
\blabel{R_e}
\ncarr{a0}{c}
\blabel{R_t}
\nccurve[angleA=-90,angleB=-90,nodesep=2pt,ncurv=1.1]{->}{a}{v}
\blabel{K_a}[0.3][-1]
\idcomp
\ncarr{a0}{b1}
\idcomp
\ncarr{a0}{b2}
\idcomp
& \footnotesize
\end{tabular}
\end{center}
\caption{Logical version after normalisation. To be well-formed there ought to be added a further relationship $\hat{R} : a \morph c$.}
\label{propertyXfailurenormalisedandabstrcated}
\end{figure}
\vspace{0.5cm}
Question is  has anything been achieved? Certainly property X does not hold of this back-engineered model and the $\chi$ transform as currently defined does not generate the normalised relational model from it either. Much to be preferred as a rework of the original model 
is the following model which now does have property X and which transforms via first-cut transform to a relational schema that is in third normal form:

\begin{figure} [h]  % 
\begin{center}
\begin{tabular}{c c}
$
\begin{array}{cp{0.7cm}c  p{0.7cm}c }
                & & \Rnode{b1}{b_1} & &                \\ [1.2cm]    
	 \Rnode{a}{a} & & \Rnode{c}{c}    & &    \Rnode{v}{v}\\ [1.2cm]  
					      & & \Rnode{b2}{b_2} & &                 
\end{array}
$
\nccircle[angleA=90, nodesep=3pt]{<-}{a}{.4cm}
\blabel{barR}[0.5]
\ncarr{a}{b1} 
\alabel{S_1}
\ncarr{b1}{v} 
\alabel{K_{b_1}}
\idcomp
\ncarr{c}{b1} 
\blabel{Q_1}
\idcomp
\ncarr{a}{b2} 
\blabel{S_2}
\ncarr{b2}{v} 
\blabel{K_{b_2}}
\idcomp
\ncarr{c}{b2} 
\alabel{Q_2}
\idcomp
\ncline[linestyle=dashed,nodesepA=\arrnodesepA,nodesepB=\arrnodesepB]{->}{a}{c} 
\blabel{\hat{R}}
\nccurve[angleA=-90,angleB=-90,nodesep=2pt,ncurv=1.6]{->}{a}{v}
\blabel{K_a}[0.3][-1]
\idcomp
& \footnotesize
\end{tabular}
\end{center}
\caption{Preferred --- An equivalent model which is well-formed and has property X. Relationship $R:a \morph c$ is navigated
as $barR \circ \hat{R}$.}
\label{propertyXfailurecorrection}
\end{figure}

\end{itemize}