
\begin{figure} [h]  % 
\begin{center}
\begin{tabular}{c c}
$
\begin{array}{cp{0.7cm}c  p{0.7cm}c }
                & & \Rnode{b1}{b_1} & &                \\ [1.2cm]    
	 \Rnode{a}{a} & & \Rnode{c}{c}    & &    \Rnode{v}{v}\\ [1.2cm]  
					      & & \Rnode{b2}{b_2} & &                 
\end{array}
$
\ncarr{a}{b1} 
\alabel{S_1}
\ncarr{b1}{v} 
\alabel{K_{b_1}}
\idcomp
\ncarr{c}{b1} 
\blabel{Q_1}
\idcomp
\ncarr{a}{b2} 
\blabel{S_2}
\ncarr{b2}{v} 
\blabel{K_{b_2}}
\idcomp
\ncarr{c}{b2} 
\alabel{Q_2}
\idcomp
\ncline[linestyle=dashed,nodesepA=\arrnodesepA,nodesepB=\arrnodesepB]{->}{a}{c} 
\blabel{R}
\nccurve[angleA=-90,angleB=-90,nodesep=2pt,ncurv=1.6]{->}{a}{v}
\blabel{K_a}[0.3][-1]
\idcomp
& \footnotesize
\end{tabular}
\end{center}
\caption{Example of failure of property X.  Suppose  $\tuple{R,Q_1} < \tuple{S_1}$ and $\tuple{R,Q_2} < \tuple{S_2}$.}

\label{propertyXfailureexample}
\end{figure}

Consider the schema shown in figure \ref{propertyXfailureexample} and suppose that $\incd{a}{S_1,S_2}{c}{Q_1,Q_2}$. 
There are two possibilities:
\begin{itemize}
\item 
If relationship $R$ represents the inclusion dependency
$\incd{a}{S_1,S_2}{c}{Q_1,Q_2}$ then the model is well-formed and relational even though it does not have property X.
\item
If $R$ does not represent the inclusion dependency then for the model to be well-formed there ought to be a relationship
$\hat{R}:a \morph c$ that represents inclusion dependency $\incd{a}{S_1,S_2}{c}{Q_1,Q_2}$. \\

\begin{tabular}{ p{3cm} c c c}
We then have that $R \leq \hat{R}$. &  & For in each instance $E$ we have&\parbox{5cm}{ \begin{align*}
E_R&=E_R \circ E_{\tuple{Q_1,Q_2}} \circ E^{-1}_{\tuple{Q_1,Q_2}} \\
   & \leq E_{\tuple{S_1,S_2}} \circ E^{-1}_{\tuple{Q_1,Q2}} \\
	 & = E_{\hat{R}}
\end{align*}}
\end{tabular}

\end{itemize}