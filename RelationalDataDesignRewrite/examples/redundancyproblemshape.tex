\begin{figure} [h]  % chen fragment with bars
\begin{center}
$
\begin{array}{c p{0.75cm}c p{0.5cm}c p{0.5cm}c}
 \Rnode{a}{a}  && \Rnode{c}{c}  &&              &&              \\ [0.3cm]
	 	           &&               && \Rnode{e}{e} && \Rnode{v}{v} \\ [0.3cm]     
 \Rnode{b}{b}  && \Rnode{d}{d}  &&              &&              \\ 
\end{array}
$
\ncarr[60]{a}{v}
\alabel{K_a}[0.3][0]
\idcomp
\ncarr{a}{c} 
\alabel{S_1}[0.5][0]
\ncarr{c}{e} 
\alabel{S_2}[0.3][0]
\idcomp
\ncarr{e}{v}
\alabel{K_e}[0.4][0]
\idcomp
\ncarr[-30]{a}{b}
\blabel{R_0}[0.35][0]
\ncarr[-30]{b}{a}
\blabel{S_0}[.35][0]
\ncarr{b}{d}
\blabel{R_1}[0.5][1]
\ncarr{d}{e}
\blabel{R_2}[0.3][0]
\idcomp
\ncarr[-60]{b}{v}
\blabel{K_b}[0.3][0]
\idcomp
%\nccurve[angleA=90,angleB=90,nodesep=2pt,ncurv=0.9]{->}{w}{v}
%\alabel{pcnt}[0.3][-1]
\vspace{1.5cm}
\newline
such that \hspace{0.5cm}
$
\begin{array}{c p{0.75cm}c p{0.5cm}c}
 \Rnode{a}{a}  && \Rnode{c}{c}  &&              \\ [0.3cm]
	 	           &&               && \Rnode{e}{e}  \\ [0.3cm]     
 \Rnode{b}{b}  && \Rnode{d}{d}  &&              \\ 
\end{array}
$
\ncarr{a}{c} 
\alabel{S_1}[0.5][0]
\ncarr{c}{e} 
\alabel{S_2}[0.3][0]
\idcomp
\ncarr{b}{a}
\alabel{S_0}[.5][1]
\ncarr{b}{d}
\blabel{R_1}[0.5][1]
\ncarr{d}{e}
\blabel{R_2}[0.3][0]
\idcomp
\hspace {0.25cm} and \hspace{0.5cm}
$
\begin{array}{c p{0.75cm}c p{0.5cm}c}
 \Rnode{a}{a}  && \Rnode{c}{c}  &&              \\ [0.3cm]
	 	           &&               && \Rnode{e}{e}  \\ [0.3cm]     
 \Rnode{b}{b}  && \Rnode{d}{d}  &&               \\ 
\end{array}
$
\ncarr{a}{c} 
\alabel{S_1}[0.5][0]
\ncarr{c}{e} 
\alabel{S_2}[0.3][0]
\idcomp
\ncarr{a}{b}
\blabel{R_0}[0.5][1]
\ncarr{b}{d}
\blabel{R_1}[0.5][1]
\ncarr{d}{e}
\blabel{R_2}[0.3][0]
\idcomp
\hspace{0.2cm} commute.

\end{center}
\caption{Showing that the definition of rendered redundant may lead to circularity in that
$R_1/R_2/K_e$  rendered redundant by a path $S_O/S_1/S_2/K_e$  and, \textit{vice-versa}, 
$S_1/S_2/K_e$ rendered redundant by $R_0/R_1/R_2/K_e$}
\label{redundancyproblemshape}
\end{figure}

The first-cut relational design will include tables $a$ and $b$ as follows
\begin{equation}
\label{relationa}
a(\underline{\qq{K_a}},\qq{R_0/K_b},\qq{S_1/S_2/K_e})
\end{equation}
and
\begin{equation}
\label{relationb}
b(\underline{\qq{K_b}},\qq{S_0/K_a},\qq{R_1/R_2/K_e})
\end{equation}

Because of the equivalence of paths $R_0/R_1/R_2 \simeq S_1/S_2$ (shown as the rightmost of the two
commuting diagram in figure \ref{redundancyproblemshape}) it follows that 
there is a functional dependency
\begin{equation}
\fd{R_0}{S_1/S_2}
\end{equation}
from this and because $K_b$ is a mono-source it follows that
\begin{equation}
\fd{R_0/K_b}{S_1/S_2}
\end{equation}
By an unproven lemma it follows that 
\begin{equation}
\fd{\qq{R_0/K_b}}{\qq{S_1/S_2}}
\end{equation}
Relation $a$ defined in (\ref{relationa}) is, therefore, not in 3NF. Similarly relation $b$ as given in
(\ref{relationb}) is not in 3NF because of this functional dependency:
\begin{equation}
\fd{\qq{S_0/K_a}}{\qq{R_1/R_2}}
\end{equation}

To normalise relations $a$ we need to modify relation $a$ and introduce a relation $X$ keyed by $\qq{R_0/K_b}$ as follows:
\begin{equation}
\label{normalrelationa}
a(\underline{\qq{K_a}},\qq{R_0/K_b})
\end{equation}
\begin{equation}
\label{relationX}
X(\underline{\qq{R_0/K_b}},\qq{S_1/S_2/K_e})
\end{equation}
similarly relations $b$ must be split into
\begin{equation}
\label{normalrelationb}
b(\underline{\qq{K_b}},\qq{S_0/K_a})
\end{equation}
\begin{equation}
\label{relationY}
Y(\underline{\qq{S_0/K_a}},\qq{R_1/R_2/K_e})
\end{equation}
In this design there are referential inclusion dependencies 
\begin{equation}
X[\qq{R_0/K_b}] \subseteq b[K_b]
\end{equation}
and
\begin{equation}
Y[\qq{S_0/K_a}] \subseteq a[K_a]
\end{equation}
Therefore this relational design corresponds to the following ER model:
\vspace{0.75cm}
\begin{center}
$
\begin{array}{c p{0.75cm} c p{0.75cm}c p{0.5cm}c p{0.5cm}c}
 \Rnode{a}{a}  && \Rnode{X}{X}  && \Rnode{c}{c}  &&              &&              \\ [0.35cm]
	 	           &&               &&               && \Rnode{e}{e} && \Rnode{v}{v} \\ [0.35cm]     
 \Rnode{b}{b}  && \Rnode{Y}{Y}  && \Rnode{d}{d}  &&              &&              \\ 
\end{array}
$
\ncarr[60]{a}{v}
\alabel{K_a}[0.3][0]
\idcomp
\ncarr{a}{X}
\alabel{R_0}[0.3][0]
\ncarr{X}{b}
\blabel{I_b}[0.25][0]
\idcomp
\ncarr{X}{c} 
\alabel{S_1}[0.5][0]
\ncarr{c}{e} 
\alabel{S_2}[0.3][0]
\idcomp
\ncarr{e}{v}
\alabel{K_e}[0.4][0]
\idcomp
\ncarr{Y}{d}
\blabel{R_1}[0.5][1]
\ncarr{d}{e}
\blabel{R_2}[0.3][0]
\idcomp
\ncarr{Y}{a}
\alabel{I_a}[0.25][0]
\idcomp
\ncarr{b}{Y}
\blabel{S_0}[0.3][0]
\ncarr[-60]{b}{v}
\blabel{K_b}[0.3][0]
\idcomp
\end{center}
%\nccurve[angleA=90,angleB=90,nodesep=2pt,ncurv=0.9]{->}{w}{v}
%\alabel{pcnt}[0.3][-1]
\vspace{1.5cm}

However we can proceed differently and decide to breal the symetry, omit $R_1/R_2/K_e$  from the relational 
model because it is rendered redundant by a path $S_O/S_1/S_2/K_e$. Then the first-cut relational design will have tables $a$ and $b$ as follows
\begin{equation}
\label{relationa}
a(\underline{\qq{K_a}},\qq{R_0/K_b},\qq{S_1/S_2/K_e})
\end{equation}
and
\begin{equation}
\label{relationb}
b(\underline{\qq{K_b}},\qq{S_0/K_a})
\end{equation}

and normalising we will get:

\begin{equation}
\label{normalrelationa}
a(\underline{\qq{K_a}},\qq{R_0/K_b})
\end{equation}
\begin{equation}
\label{relationX}
X(\underline{\qq{R_0/K_b}},\qq{S_1/S_2/K_e})
\end{equation}
\begin{equation}
\label{normalrelationb}
b(\underline{\qq{K_b}},\qq{S_0/K_a})
\end{equation}
which corresponds to the folllwing ER model:

\vspace{0.75cm}
\begin{center}

$
\begin{array}{c p{0.75cm} c p{0.75cm}c p{0.5cm}c p{0.5cm}c}
 \Rnode{a}{a}  && \Rnode{X}{X}  && \Rnode{c}{c}  &&              &&              \\ [0.35cm]
	 	           &&               &&               && \Rnode{e}{e} && \Rnode{v}{v} \\ [0.35cm]     
 \Rnode{b}{b}  &&               && \Rnode{d}{d}  &&              &&              \\ 
\end{array}
$
\ncarr[60]{a}{v}
\alabel{K_a}[0.3][0]
\idcomp
\ncarr{a}{X}
\alabel{R_0}[0.3][0]
\ncarr{X}{b}
\blabel{I_b}[0.25][0]
\idcomp
\ncarr{X}{c} 
\alabel{S_1}[0.5][0]
\ncarr{c}{e} 
\alabel{S_2}[0.3][0]
\idcomp
\ncarr{e}{v}
\alabel{K_e}[0.4][0]
\idcomp
\ncdarr{b}{d}
\blabel{R_1}[0.5][1]
\ncarr{d}{e}
\blabel{R_2}[0.3][0]
\idcomp
\idcomp
\ncarr{b}{a}
\alabel{S_0}[0.3][0]
\ncarr[-60]{b}{v}
\blabel{K_b}[0.3][0]
\idcomp
\end{center}
\vspace{1.0cm}
relational ER model is then:
\vspace{0.9cm}
\begin{center}
\begin{tabular} {c p{5cm}}
$
\begin{array}{c p{0.75cm} c p{0.75cm}c p{0.5cm}c p{1.5cm}c}
 \Rnode{a}{a}  &&               &&               &&              &&              \\ [0.2cm]
               && \Rnode{X}{X}  &&               &&              &&              \\ [0.2cm]
							 &&               && \Rnode{c}{c}  &&              &&              \\ [0.35cm]
	 	           &&               &&               && \Rnode{e}{e} && \Rnode{v}{v} \\ [0.35cm]     
 \Rnode{b}{b}  &&               && \Rnode{d}{d}  &&              &&              \\ 
\end{array}
$
\nccurve[angleA=90,angleB=90,nodesep=2pt,ncurv=0.7]{->}{a}{v}
\alabel{K_a}[0.3][0]
\idcomp
\ncarr{a}{X}
\alabel{R_0}[0.3][0]
\ncarr{X}{b}
\blabel{I_b}[0.25][0]
\ncarr{X}{c} 
\alabel{S_1}[0.5][0]
\ncarr{c}{e} 
\alabel{S_2}[0.45][0]
\ncarr{e}{v}
\alabel{K_e}[0.4][0]
\idcomp
\ncdarr{b}{d}
\blabel{R_1}[0.5][1]
\ncarr{d}{e}
\blabel{R_2}[0.45][0]
\ncarr{b}{a}
\alabel{S_0}[0.3][0]
\ncarr[-60]{b}{v}
\blabel{K_b}[0.3][0]
\idcomp
\ncarr[60]{X}{v}
\alabel{\qq{I_b/K_b}}[0.3][0]
\idcomp
\ncarr[30]{X}{v}
\alabel{\qq{S1/S2/K_e}}[0.3][0]
\ncarr[55]{a}{v}
\alabel{\qq{R_0/I_b/K_b}}[0.25][0]
\ncarr[-35]{b}{v}
\blabel{\qq{S_0/K_a}}[0.3][0]
\ncarr[25]{c}{v}
\alabel{\qq{S_2/K_e}}[0.2][0]
\idcomp
\ncarr[-25]{d}{v}
\blabel{\qq{R_2/K_e}}[0.2][0]
\idcomp & \raisebox{0cm}{\parbox{4.5cm}{The attribute $\qq{S_1/S_2/K_e}$ 
of entity type $X$ is not identifying and has an equivalent path $I_b/R_1/R_2/K_e$. Therefore the model is not 'strictly dense' because the path $R_1/R_2/K_e$ is not equivalent to a primary key attribute. The model does however meet the condition for being 'dense' since the relationship $I_b$ is a mono-source. }}
\end{tabular}  
\end{center}
\vspace{1.5cm}

