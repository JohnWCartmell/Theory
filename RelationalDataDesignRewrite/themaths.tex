 \section{Definition of ER model}

\begin{definition}
An \term{ER-schema} $\genericmodel$ is a directed graph having the following additional structure:

\begin{enumerate} [(i)]
\item{a distinguished node $\veee$ for which there are no outgoing edges and which represents the type of universals, }

\item{a distinguished acyclic subset $I$ of edges, called the identifying edges, such that
for every node $a$ other than $\veee$ there is a non-empty set $I_a$ of identifying edges with domain $a$. 
}
\end{enumerate}

\end{definition}

If $\genericmodel$ is an ER-schema (or an ER-model which, as we define below, 
includes an ER-schema) then the nodes of $\genericmodel$ other than $\veee$ we say are entity types and we denote by $\et$.

The set $\attr{a}$ of attributes of an entity type $a$ is defined as the set of edges that have $a$ as source and  $\veee$ as destination. 

\begin{definition}
\noindent A \term{database instance}  of an ER schema is
a set of entities $E_x$ for each node $x$ of the graph of the schema and 
a partial function $E_r : E_x \rightarrow E_y$ for each edge of the graph $r:x \rightarrow y$ 
such that $E_\veee=\Veee$
and such that 
for each entity type $x$ if $I_x$ is the subset of all the outgoing and identifying edges
of $x$ then for each $e \in I_x$ the function $E_e$ is total and
the set of functions $\setsuchthat{E_e}{e \in I_x}$, is jointly injective.
\end{definition}

\noindent A  path through an ER schema  is a
sequence of  $n$ edges:
$\overset{r_1}{x_0 \morph   x_1} \overset{r_2}{\morph} x_2 ... \overset{r_n}{\morph} x_n$.  $x_0$ is said to be the source of the path and $x_n$ is said to be the destination of the path. 

\noindent For any database instance $E$ we can  extend  the definition of
$E_f$, for edges $f$,  so that to every path $p$, $p: a \rightarrow b$,  we have defined a function $E_p: E_a \rightarrow E_b$. From the initial definition of $E_f$ that applies to 
edges the definition proceeds recursively as follows: 
\begin{enumerate} [(i)]
\item{  For each entity type $a$, $E_{\langle \rangle}: E_a \rightarrow E_a$ is defined to be the identity function.
}
\item{   if $p$ is a path $p: a \rightarrow b$ and $f$ is an edge $p: b \rightarrow c$ then $E_{\langle p,f \rangle}$ is 
is defined to be the functional composition $E_p \circ E_f$.
}
\end{enumerate}

We can extend this notation further and define $E$ on n-tuples of paths so that if 
$\tuple{p_1,...p_n}$ is an n-tuple of paths from entity type $a$ then in an instance E
the function  $E_{\tuple{p_1,...p_n}}$ is defined to be the function mapping each element $e$ 
of $E_a$ to the n-tuple
$\tuple{E_{p_1}(e),...E_{p_n}(e)}$.


\begin{definition}
If $s$ is a set and if $f_{i, 1\leq i \leq n}$ is a family of partial functions, $f_i: s \rightarrow s_i$   for some sets $s_{i, 1 \leq i \leq n}$,  then
we previously defined that the family of functions $f_{i, 1\leq i \leq n}$, is \term{jointly injective} if the partial function $\langle f_1,...f_n\rangle: s \rightarrow
s_1 \times ... \times s_n $ is injective. 
In this case \commentary{Not sure if this para is reqd and if so where it is required}.
 the partial function $\langle f_1,...f_n\rangle: s \rightarrow
s_1 \times ... \times s_n $ is also invertible i.e. iff there is a partial function $inv_{\langle f_1,...f_n\rangle} : s_1 \times ... \times s_n \rightarrow s$ such that
(i) for all $x \in s$, $inv_{\langle f_1,...f_n\rangle}(\langle f_1(x),...f_2(x) \rangle ) = x$ and (ii) if $y \in s_1 \times ... \times s_n $ and
$y \notin img(\langle f_1,...f_n\rangle)$ then $inv_{\langle f_1,...f_n\rangle}(y)$ is undefined. \\
\end{definition}


\begin{definition}
\noindent An \term{ER model} is an ER schema and a set of database instances of the schema called the
\term{defining instances} of the model. \\
\end{definition}


\subsection{Path equivalence and subsumption}

\noindent If $r$ and $s$ are paths both having source $a$ and destination $b$ then we will say $r \leq s$ iff in all defining instances E, for all 
entities $e \in E_a$, if $E_r(e)$ is defined then $E_s(e)$ is defined and $E_r(e)=E_s(e)$. \\

\noindent If $r$ and $s$ are paths both having source $a$ and destination $b$ then we will say $r \simeq s$ iff $r \leq s$ and $s \leq r$. \\

\noindent With these definitions,  the (meta-relationship) $\leq$ is a partial order on the classes of equivalent paths. \\

\noindent For paths $r$ and $s$ we define $r < s$ to be equivalent to $r \leq s$ and not $r \simeq s$. \\


\begin{definition}
Say that a simple path $\tuple{r_0,r_1,...r_n}$ \textit{is subsumed by} a simple path 
$\tuple{s_0,s_1,...s_m}$ iff $m \geq 1$ and either:
\begin{enumerate} [(i)]

\item $\tuple{r_0,r_1,...r_n} \simeq  \tuple{s_0,s_1,...s_m}$ and 
\commentary{In most general case this will become not $s_1$ uses $s_2$ ... uses $s_m$.}
for some $j$, $1 \leq j \leq m$, $s_j$ is not an identifying edge. \\

 \hspace{1cm} or:
\item $\tuple{r_0,r_1,...r_n} <  \tuple{s_0,s_1,...s_m}$ and $r_0 \neq s_0$.
\end{enumerate}
\end{definition}

\begin{figure} [h]
\begin{center}
\begin{tabular}{c c}
$
\begin{array}{cp{0.75cm}cp{0.75cm}c}
   \Rnode{fk}{fk}     & & \Rnode{t}{t} & & \Rnode{v}{v} \\[1.2cm]     
	 \Rnode{fkc}{fkc}   & & \Rnode{c}{c} & &               
\end{array}
$
\ncarr{fk}{t} 
\alabel{S_1}
\ncarr{t}{v} 
\alabel{tN}
\idcomp
\ncarr{c}{v} 
\blabel{cN}
\idcomp
\ncarr{fkc}{c}
\blabel{R_0}
\ncarr{fkc}{fk}
\alabel{S_0}
\idcomp
\ncarr{c}{t}
\blabel{R_1}
\idcomp
\ncarr[50]{fk}{v}
\alabel{FkN}
\idcomp
\ncarr[-90]{fkc}{v}
\blabel{SeqN}
\idcomp
& \footnotesize
\begin{tabular}{c p{1.5cm} p{4cm}}
KEY && \\
\hline
t & table & Having identifying attribute tN the name of the table. \\
c & column & Identified by a combination of column name cN and relationship $R_1$ to the table it is a column of.\\
fk & foreign key & Identified by its name $FkN$.\\
fkc & foreign key column & Identified by relationship $S_0$ to the foreign key it is a part of and its sequence number $SeqN$ i.e. the position it appears in within the foreign key. \\
\end{tabular} 
\end{tabular}
\end{center}
\caption{Example of path equivalence - the path $\tuple{R_0,R_1}$ is equivalent to path $\tuple{S_0,S_1}$.}
\label{foreignkeygraph}
\end{figure}

\commentary{Relational meta-schema path equivalence example is a little inaccessible therefore consider replacing by
the table row column data cell example from earlier draft as figure 3 where also given as example of path ewuivalence.}

\begin{figure} [h]  % student advisor
\begin{center}
\begin{tabular}{c c}
$
\begin{array}{cp{0.05cm}c  p{0.05cm}c p{0.5cm}c}
                & & \Rnode{d}{d} & &              & &             \\ [0.3cm]
								& &              & &              & & \Rnode{v}{v} \\ [0.6cm]     
	 \Rnode{s}{s} & &              & & \Rnode{p}{p} & &             
\end{array}
$
\ncarr{s}{d} 
\alabel{S_1}
\idcomp
\ncarr{d}{v} 
\alabel{Id}
\idcomp
\ncarr{p}{v} 
\blabel{PNo}
\idcomp
\ncarr{p}{d} 
\blabel{R_1}
\idcomp
\ncline[linestyle=dashed,nodesepA=\arrnodesepA,nodesepB=\arrnodesepB]{->}{s}{p} 
\blabel{R_0}
\ncarr[-90]{s}{v} 
\blabel{sNo}
\idcomp
& \footnotesize
\begin{tabular}{c p{1.5cm} p{4cm}}
KEY && \\
\hline
d & department & Having identifying attribute Id the department identifier. \\
s & student & Identified by relationship $S_0$ to department enrolled in and  attribute student number SNo. \\
p & professor & Identified by their relationship $R_1$ to a department and their professor number attribute pNo yyy \\
R0 & advised by & Represents the relationship\footnote{How significant is it that this may be optional?Need draw this out.} between a student and a professor.\\
\end{tabular} 
\end{tabular}
\end{center}
\caption{Following the pattern of the property X Example. This example is based on one given Shlaer-Long. Here we specify that a student optionally has an advisor (suppose that the advisor is selected part way through a course). As before we assume that an advisor must be a professor of the department in which the student is enrolled. }
\label{studentadvisorgraph}
\end{figure}


\subsection{Jointly Injective Sets of Functions}
\begin{lemma}
\label{jointlyinjectivecomposition}
If $A$ is a set and if $I$  is a jointly injective set of functions with domain $A$, if $F \in I$, $F: A \morph B$ is a function,
and if $J$ is a jointly injective set of functions with domain $B$ then the set of functions
$(I \backslash \set{F}) \cup \setsuchthat{F \circ G}{G \in J}$ is jointly injective.
\end{lemma}
\begin{proof}

\end{proof}


\begin{lemma}
\label{jointlyinjectivefactorisation}
If $A$ is a set and $I$  a set of functions with domain $A$, if $F \in I$, $F: A \morph B$ is a function,
and if $J$ is a jointly injective set of functions with domain $B$ then if the set of functions
$(I \backslash \set{F}) \cup \setsuchthat{F \circ G}{G \in J}$ is jointly injective then the
set $I$ is jointly injective.
\end{lemma}
\begin{proof}
Suppose $x_1,x_2 \in A$ and that for all $f \in I$, $f(x_1)=f(x_2)$. We need show that $x_1=x_2$.
Because $(I \backslash \set{F}) \cup \setsuchthat{F \circ G}{G \in J}$ is jointly injective
it suffices to show that 
\begin{enumerate}[(i)]
\item
for all $f \in I \backslash \set{F}$, $f(x_1)=f(x_2)$, 
\item that for all $f \in \setsuchthat{F \circ G}{G \in J}$, $f(x_1)=f(x_2)$
\end{enumerate}
(i) follows directly from the initial assumption. 
(ii) follows because for such an $f$, 
\begin{align*}
f(x_1)&=G(F(x_1) & & \\
      &=G(F(x_2) & & \mbox{from the initial assumption since }F \in I \\
			&=f(x_2). & &
\end{align*}
\end{proof}


\subsection{Identifying Sets}

The following definition generalises that of super key in the relational model:
\begin{definition}
A set of paths $P$  within a model $\genericmodel$ is said to be a \term{mono-source} at entity type $a$
iff each path $p \in P$ has domain $a$ and such that in all defining instances E, 
(i) the function $E_p$ is total, for each $p \in P$, and (ii) the set of functions
$\setsuchthat{E_p}{p\in P}$ is jointly injective.
\end{definition}

From the definition of ER model it follows that the set of outgoing edges of an entity type 
$a$ in a model \genericmodel\ is a mono-source. 
\subsubsection{Primary Key Paths}
\begin{definition}
If $a$ is an entity type in an entity model \genericmodel\ then a \term{primary key path}
\commentary{Need show equivalence of this and alternate definitions.}
of $a$ is any path $p:a \morph v$ consisting entirely of identifying edges. 
\end{definition}

Denote the set of all primary key paths leaving an entity type $a$ in a model \genericmodel\ by 
$\pk{a}$.

\subsubsection{Equivalent Models}

First we need to define equivalence for sets of paths then we can define what we mean by 
one model being an extension of another.
\begin{definition}
If \genericmodel\ is a model and if $P$ and $P'$ are sets of paths in \genericmodel\ then say that
$P \simeq P'$ in \genericmodel\ iff
\begin{enumerate}
\item for all $p \in P$ there exists $p' \in P'$ such that $p \simeq p'$ in \genericmodel,
\item for all $p' \in P'$ there exists $p \in P$ such that $p \simeq p'$ in \genericmodel,
\end{enumerate} 
\end{definition}

\begin{definition}
If \genericmodel\ is a model then say that a model $\genericmodel'$ is an extension of 
$\genericmodel$ iff 
\begin{enumerate}
\item the underlying graph of $\genericmodel'$ is an extension the graph of \genericmodel\
by edges alone, 
\item for each defining instance $E$ of $\genericmodel$ there is a unique defining
instance $E'$ of $\genericmodel'$ such that 
(i) for each entity types $a$ of \genericmodel, $E'_a=E_a$,
(ii) for each edge  $e$ in \genericmodel,
$E'_e=E_e$ and all defining instances $E'$ of $\genericmodel'$ arise in this way,
\item for all edges $e$ of $\genericmodel'$ that are not in \genericmodel\ there is a path 
$p$ in \genericmodel\ such that for all the defining instances $E'$ of $\genericmodel'$,
$E'_e=E'_p$, or, equivalently, $E'_e=E_p$,
\item for each entity type $a$  then $\pk[\genericmodel]{a} \simeq \pk[\genericmodel']{a}$ in $\genericmodel'$. 
\end{enumerate}
\end{definition}

\begin{definition}
Say that two models are \term{co-extensional} \commentary{was \term{equivalent}} iff there is a third model that they are both extensions of.
\commentary{Co-extsionnality is not a transitive and so we need equivalence needs be the symetric transitive closure.
But do I need this closure? We can show that this closure is the right thing categoricaly.}
\end{definition}

\subsubsection{Composing mono-sources}


A source $i$ at $a$ that includes a path $f:a \morph b$ can be composed with a source $j$ at $b$ by taking
the union of paths in $i$ other than $f$, which we shall denote $i\backslash \set{f}$, and the set 
$\setsuchthat{f \circ g}{g \in j}$. The following diagram gives an idea of the arrangement of sets of 
paths $i$ and $j$ as well as the composed set
$(i \backslash \set{f}) \cup \setsuchthat{f \circ g}{g \in j}$ :

\setlength{\arraycolsep}{.2cm}
\begin{center}
$
\begin{array}{cp{1.5cm}ccp{1.5cm}ccp{1.25cm}c}
             & &         & \dotnode[dotsize=1pt]{b1} & &        &                              && \pnode{bracehigh}  \\ [0.3cm]
						 & &         & \dotnode[dotsize=1pt]{b2} & &        &                                \\ [0.3cm]
\Rnode{a}{a} & & \vdots  &                           & &        &                                \\ [0.02cm]
						 & &         &                           & &        & \dotnode[dotsize=1pt]{x1}      \\ [0.1cm]
             & &         & \Rnode{b}{b}              & & \vdots &                                \\ [0.1cm]
             & &         &                           & &        & \dotnode[dotsize=1pt]{xn}   && \pnode{bracelow}   \\ [0.5cm]
\psbrace[rot=90, nodesepA=-2pt, nodesepB=10pt, braceWidth=1pt, braceWidthInner=3pt](0,0.5)(2.7cm,0.5){i}	
	&  &         & 
\psbrace[rot=90, nodesepA=-2pt, nodesepB=10pt, braceWidth=1pt, braceWidthInner=3pt](0,0.5)(2.7cm,0.5){j} & & \\
\end{array}
$
\psbrace[rot=0, nodesepA=10pt, braceWidth=1pt, braceWidthInner=3pt, ,ref=lC](bracelow)(bracehigh)
{$(i \backslash \set{f}) \cup \setsuchthat{f \circ g}{g \in j}$}
\simplepatha{}{b}{x1}
\simplepatha{}{b}{xn}
\simplepatha{}{a}{b1}
\simplepatha{}{a}{b2}
%\simplepatha{}{a}{bn}
\simplepathb{$f$}{a}{b}
\end{center}
.

In the next lemma we show that if $j$ is a mono-source then the above composition of $i$ with $j$ though $f$ is a mono-source iff $i$ is a mono-source.

\begin{lemma}
\label{identifyingsetdeduction}

In a model \genericmodel, if $i$ is a source with domain $a$,
if $f \in i$, $f: a \morph b$, and if $j$ is a mono-source with domain $b$ then 
$i$ is a mono-source iff $(i \backslash \set{f}) \cup \setsuchthat{f \circ g}{g \in j}$ is a mono-source.
\end{lemma}
\begin{proof}
First note that from the definition of database instance it follows that 
\begin{equation}
\label{eequivalentsets}
\setsuchthat{E_e}{e \in (i \backslash \set{f}) \cup \setsuchthat{f \circ g}{g \in j}}
= \setsuchthat{E_e}{e \in i} \backslash \set{E_f} \cup \setsuchthat{E_f \circ E_g}{g \in j}
\end{equation}

We can show that if $i$ is a mono-source 
then $(i \backslash \set{f}) \cup \setsuchthat{f \circ g}{g \in j}$ is a mono-source
by showing that for any instance $E$ of \genericmodel\ the family of functions 
$\setsuchthat{E_e}{e \in (i \backslash \set{f}) \cup \setsuchthat{f \circ g}{g \in j}}$
is jointly injective. Because of (\ref{eequivalentsets}) this follows by applying lemma \ref{jointlyinjectivecomposition} with $I$ being $\setsuchthat{E_e}{e \in i}$,
$F$ being $E_f$,  and $G$ being $\setsuchthat{E_g}{g \in j}$.

\textit{Vice-versa}, we can show that if $(i \backslash \set{f}) \cup \setsuchthat{f \circ g}{g \in j}$ is a mono-source then $i$ is a mono-source by applying lemma \ref{jointlyinjectivefactorisation} 
with, as above, $I$ being $\setsuchthat{E_e}{e \in i}$,
$F$ being $E_f$,  and $G$ being $\setsuchthat{E_g}{g \in j}$ \commentary{can improve this wording instead of repeating the mapping}.
\end{proof}

From lemma \ref{identifyingsetdeduction} it follows that any set of paths arrived at by repeated composition 
of mono-sources specified for a model \genericmodel\ in $\setsuchthat{I_a}{a \in \et}$ is itelf  
a mono-source in \genericmodel. Denote by $\bar{I}$ the set  of all such mono-sources.

\begin{figure} [h]
\begin{center}
\begin{tabular}{c c}
$
\begin{array}{cp{0.4cm}cp{0.75cm}cp{0.75cm}c}
              &&               &&                &&               \\[0.25cm]
              &&               &&                &&               \\[0.25cm]
\Rnode{p}{p}	&& \Rnode{m}{m}  &&   \Rnode{c}{c} && \Rnode{v}{v}  \\[0.25cm]
	            &&  
\end{array}
$

\ncarr{p}{m} 
\alabel{I}
\idcomp
\ncarr[-35]{p}{v}
\blabel{yr}
\idcomp
\ncarr[30]{m}{v}
\alabel{mNo}
\idcomp
\ncarr{m}{c}
\blabel{M}
\idcomp
\ncarr{c}{v}
\blabel{cId}
\idcomp
& \footnotesize
\begin{tabular}{c p{1.5cm} p{4cm}}
KEY && \\
\hline
p  & president        & Identified by a combination of inclusion relationship ($I$) that identifies a
                       president as being a committee members and a year attribute ($yr$). \\
m  & member           & Identified by a member relationship ($M$) to a club and a by membership 
                         number attribute ($mNo$). \\
c  & club             & Identified by club identifier attribute ($cId$)
\end{tabular} 
\end{tabular}
\end{center}
\caption{Club president example. 
This is an example of an ER model that fails the minimality condition defined in section \ref{minimalitycondition}.
The failure arises at entity type p (president) because 
the set of paths $\set{I\circ mNo, I \circ M \circ cId, yr}$ is
in $\bar{I}$ and is therefore a mono-source but it is not minimal because
 a club has at most one president a year and so the subset $\set{I \circ M \circ cId, yr}$ is  a
mono-source. Further, the path $I \circ mNo$ is not equivalent to either of the two paths in the subset.
This problem can be fixed by introducing a club membership relationship $R: p \morph c$ and extended each defining instance $E$
 by which is defining $E'_R=E_{I \circ M}$  and by re-specifying the identifying features $p$ to be the set $\set{R,yr}$,
so obtaining an extended model in which $R$ it to be equivalent to
path $I \circ M$.
}
\label{clubpresidentbeforenormalisation}
\end{figure}

\commentary{Club president example is isomorphic to the nearest shop example. And is more general than the Zanilo tel-place-location example.}

\subsubsection{Primarily key Paths and Minimality Condition}
\label{minimalitycondition}
 From this definition it follows that $\pk{a} \in \bar{I}$. From this we get the following corollary.


\begin{corollary}
\label{primarykeycorallary}
If \genericmodel\ is an entity model then the set $\pk{a}$of all primary key paths 
leaving an entity type $a$ is a mono-source.
\end{corollary}


\begin{definition}
If $P$ and $Q$ is a set of paths of model \genericmodel\ then say that the set of paths $P$ 
\term{is representative of} $Q$ 
iff $P \simeq Q$ and no proper subset of $P$ is equivalent to $Q$.
\end{definition}

\begin{definition}
A model \genericmodel\ satisfies the minimality condition iff for every entity type $a$,
\begin{enumerate}
\item the set $I_a$ is a minimum mono-source i.e 
there is no proper subset which is a mono-source,
\item whenever a set of paths $P$ is representative
of the set $\pk{a}$  of key paths of \genericmodel\ then no proper subset of $P$ is a mono-source.
\end{enumerate}
\end{definition}

\subsubsection{Primarily Keyed Models}

One of the characteristics of a relational model is that each of the identifying features of a table is a column
and this translates into our unified framework as each identifying edge of each entity type being an attribute i.e. 
an edge with codomain $\veee$. This leads to this definition: 
\begin{definition}
Say that a model is \term{primarily keyed} iff all identifying edges 
have codomain $\veee$.
\end{definition}
To reach the goal of showing that to every ER model there is an equivalent relational model we require the following:
\begin{lemma}
If \genericmodel\ is an ER model then it can be extended to a model $\genericmodel'$
which is primarily keyed. If \genericmodel\ satisfies the minimality condition then the
extended model $\genericmodel'$ satisfies the  condition that each $I_a$ has no proper subsets which
are mono-sources. 
\end{lemma}
\begin{proof}
For each entity type $a$ we extend \genericmodel\ by edges $E_a$, where each edge 
$E_a$ corresponds to a distinct primary key path $p$ from $a$ to $\veee$
and in the extended model \genericmodel'\ by corollary \ref{primarykeycorallary} 
we can define the identifying set $I'_a$ at $a$ to be the set
$E_a$. 
\end{proof}

\noindent
Consider that the various database normal forms (3NF, BCNF, 4NF, 5NF and the like) each 
prescribe that a database schema be complete in some way as a description of the facts of its instances\footnote{Essentially
 because being good as a schema is to be a good theory and a good theory is one that is a good fit to the facts.} and observe 
in particular that BCNF can be paraphrased as saying that those relationships (i.e. functional dependencies) that exist in the data ought to be \term{represented} in the schema\footnote{21/02/2019 Is this not the representation principle?}. These considerations motivate the definitions which now follow and conclude with the definition of a \term{well-formulated} entity model. This definition generalises that of a relational schema being in Boyce-Codd Normal Form (BCNF). 


\subsection{Referential Inclusion Dependencies}

\commentary{aka referential integrity constraints/foreign keys/foreign key constraints}

\begin{definition}
If $\genericmodel$ is an entity model, 
if $a$ and $b$ are entity types of  $\genericmodel$ and  if $\set{q_q,...q_n} \in \bar(I)_b$
\commentary{used to say an identifying tuple with respect to $b$},
if \fntuple is a tuple of outgoing paths from entity type $a$  
so that we have the following diagram of paths in $\genericmodel$
\setlength{\arraycolsep}{.2cm}
\begin{center}
$
\begin{array}{cp{2cm}ccp{2cm}c}
             & &         & \Rnode{b1}{b_1} &&               \\ [0.5cm]
						 & &         & \Rnode{b2}{b_2} &&               \\ [0.2cm]
\Rnode{a}{a} & &         &                 &&  \Rnode{b}{b} \\ [-0.2cm]						
             & &         &    \vdots       &&               \\ [0.2cm]
             & &         & \Rnode{bn}{b_n} &&               \\ 
\end{array}
$
\simplepatha{$f_1$}{a}{b1}
\simplepatha{$f_2$}{a}{b2}
\simplepathb{$f_n$}{a}{bn}
\simplepathb{$q_1$}{b}{b1}
\simplepathb{$q_2$}{b}{b2}
\simplepatha{$q_n$}{b}{bn}
\end{center}
then say that $a$ represented by $\fntuple$ 
has a referential inclusion dependency with $b$ represented by $\qntuple$, and for which we shall summarily write:
$$
\incd{a}{f_1,...f_n}{b}{q_1,...q_n},
$$
iff in all instances $E$ of $\genericmodel$,
$img(E_{\fntuple}) \subseteq img(E_{\qntuple})$. 
\end{definition}

Here or hereabouts maybe need define $E_{\qntuple}^{-1}$ where $\set{q_1,...q_n} \in \bar{I}_a$, for some entity 
type $a$.

\begin{definition}
We say that a path $p:a \morph b$ represents a referential inclusion dependency
$\incd{a}{f_1,...f_n}{b}{q_1,...q_n}$  in an ER model $\genericmodel$ 
  in all instances $E$ of $\genericmodel$, $E_{\fntuple} \circ E_{\qntuple}^{-1}=E_p$.
\end{definition}
We say that an referential inclusion dependency in model \genericmodel\ is represented provided that
there exists a path which represents it. 
\begin{lemma}
\label{refinclusionlemma}
A referential inclusion dependency
$\incd{a}{f_1,...f_n}{b}{q_1,...q_n}$  in an ER model $\genericmodel$ is  represented 
by path $p:a \morph b$ in $\genericmodel$ iff  \foreachi, $p \circ q_i \simeq f_i$.
\end{lemma}
\begin{proof}
\begin{align*}
\text{In the model \genericmodel,}
            &\text{in each instance $E$, }E_p = E_{\fntuple} \circ E_{\qntuple}^{-1}  \\
\text{iff } & \text{in each instance $E$, } E_p \circ E_{\qntuple} = E_{\fntuple}  
                        && \text{because } E_{\qntuple} \text{ injective and total}   \\
\text{iff } & \text{in each instance $E$, } \tuple{E_p \circ E_{q_1},...E_p \circ E_{q_n}} 
                                                      = \tuple{E_{f_1},...E_{f_n}}
									                                            && \text{ by defn. of }E \\
\text{iff } & \text{in each instance $E$, \foreachi, } E_p \circ E_{q_i} = E_{f_i} 
                                                              && \text{because tuples are equal iff they are pointwise equal}  \\
\text{iff } & \text{\foreachi, } p \circ q_i \simeq f_i          && \text{by defn of } \simeq.
\end{align*}
\end{proof}


\begin{categoricalaside}
If an ER schema is represented as a category with finite products (\textit{a la} Johnstone \textit{et al}) then
a referential inclusion dependency is a  diagram
\begin{center}
$
\begin{array}{cp{0.75cm}c}
   \Rnode{a}{a}     & & \Rnode{x}{x}  \\[1.2cm]     
	                  & & \Rnode{b}{b}  
\end{array}
$
\ncarr{a}{x} 
\alabel{f}[0.33]
\ncarr{b}{x}
\blabel{m}[0.3]
\idcomp
\end{center}
\noindent
in \cat{C} such that in all instance functors $F$, 
$F(f)$ factors through $F(m)$.

Such a referential inclusion dependency is explicitly represented iff
 $f$ factors through $m$ i.e. there is an $f_0: a \morph b$ in \cat{C} such that 
$f_0 \circ m =f$. Note that because $m$ is monic then such an $f_0$ is the unique such morphism and if in some instance $F$,
$e: F(a) \morph F(b)$ is a function such that $e \circ F(m) = F(f)$ then $F(f_0)=e$. 
\end{categoricalaside}

\begin{definition}
Say that an inclusion dependency $\incd{a}{x_1,...x_n}{b}{q_1,...q_n}$ is \textit{simple}
iff each $q_i$ is a singleton path, i.e. is simply an edge. 
\end{definition}

\begin{lemma}
\label{simplerepresentationlemma}
If in an ER model \genericmodel\ all simple referential inclusion dependencies
have representations  then all referential inclusion dependencies have representations. 
\end{lemma}
\begin{proof}
Proof by induction TBD. \commentary{Start by drawing a diagram of the inductive step.} 
\end{proof}

\subsection{Functional Dependencies}

\begin{definition} %functional dependency
In an entity model $\genericmodel$  
if for some $n \geq 1$, $a$, $b_{i, 1 \leq i \leq n}$,  and $c$ are entity types and 
if  $x_{i, 1 \leq i \leq n}$, and $y$ are simple paths such
that for each $i$, $x_i : a \rightarrow b_i$, and such that $y: a \rightarrow c$ 
as shown here:
\setlength{\arraycolsep}{.2cm}
\begin{center}
$
\begin{array}{cp{2cm}cc}
             & &         & \Rnode{b1}{b_1} \\ [0.5cm]
						 & &         & \Rnode{b2}{b_2} \\ [0.6cm]
						 & & \vdots  &                 \\ [0.2cm]
\Rnode{a}{a} & &         & \Rnode{bn}{b_n} \\ [1.0cm]
             & &         & \Rnode{c}{c}   \\
\end{array}
$
\simplepatha{$x_1$}{a}{b1}
\simplepatha{$x_2$}{a}{b2}
\simplepatha{$x_n$}{a}{bn}
\simplepathb{$y$}{a}{c}
\end{center}



\noindent 
then path $y$ is said to be \term{functionally dependent} on the set of paths $\{x_1,...x_n\}$, 
for which  we write  \sfd{x_1,...x_n}{y},
iff
 in each instance $E$ there exists a  partial 
function $f_E: E_{b_1} \times E_{b_n} \rightarrow E_c$ 
\noindent such that 
domain of $f_E \subseteq img(E_{\langle x_1,... x_n \rangle})$ 
and  
$E_{\xntuple} \circ f_E = E_y$ 
\begin{center}
$
\begin{array}{cp{2cm}ccp{0.5cm}cc}
						    & &         & \Rnode{Eb1}{E_{b_1}}& &                            &        \\ [0.6cm]
						    & &         & \Rnode{Eb2}{E_{b_2}}& &                            &        \\ [0.6cm]
						    & &\vdots  &                      & &                            &        \\ [0.2cm]												
\Rnode{Ea}{E_a} & &         & \Rnode{Ebn}{E_{b_n}}& & \Rnode{Jnctn}{}&  \\ [1.0cm]
						    & &         & \Rnode{Ec}{{E_c}}   & &                            &        \\
\end{array}
$
\simplepatha{$E_{x_1}$}{Ea}{Eb1}
\simplepatha{$E_{x_2}$}{Ea}{Eb2}
\simplepatha{$E_{x_n}$}{Ea}{Ebn}
\simplepathb{$E_y$}{Ea}{Ec}
\nchmarr[15][45]{Eb1}{Ebn}{Jnctn}{Ec}
\naput[npos=-0.1]{$f_E$}
\ncarc[arcangle=15]{Eb2}{Jnctn}
\end{center}
\end{definition}

\begin{remark}
In the above definition, if within an instance $E$ there exists such a partial function $f$ such that
$E_{\xntuple} \circ f_E = E_y$ then $f$ will be the unique such partial function. For if 
partial function $f'$ is another such then given a $\tuple{E_{x_1}(e),...E_{x_n}(e)} \in img(E_{\tuple{x_1,...x_n}})$
we have that $f(\tuple{E_{x_1}(e),...E_{x_n}(e)})=E_y(e)=f'(\tuple{E_{x_1}(e),...E_{x_n}(e)})$.
\end{remark}

\begin{definition}
A functional dependency \sfd{x_1,...x_n}{y} is said to be \term{trivial} if $y\simeq x_i$, for some $i$, $1 \leq i \leq n$.
\end{definition}

\begin{definition}
In an entity model $\genericmodel$, if
$e$ is an entity type and $\set{x_1,...x_n}$ and $\set{y_1,...y_m}$ are sets of simple paths with source $a$
then we say the set $\set{y_1,...y_m}$ \textit{is functionally dependent on the set} $\set{x_1,...x_n}$ and write
\fd{x_1,...x_n}{y_1,...y_m} 
iff  each $y_j$, $1 \leq j \leq m$, is functionally dependent on $\set{x_1,...x_n}$.
\end{definition}
\begin{definition} %transitive
In an entity model $\genericmodel$, a functional dependency $\set{x_1,...x_n} \morph z$
is said to be \term{transitive} if there exists a set of paths $\{y_1,...y_{m}\}$ such that
$\set{x_1,...x_n} \morph \set{y_1,...y_{m}}$ and $\set{y_1,...y_{m}} \morph y$ but not
$\{y_1,...y_{m}\} \morph \{x_1,...x_n\}$.
\end{definition}
\begin{definition} %intransitive
In an entity model $\genericmodel$, a functional dependency $\set{x_1,...x_n} \morph y$
is said to be \term{intransitive} if it is not transitive.
\end{definition}
\begin{lemma}
\label{transitiveinjectivelemma}
If $\set{x_1,...x_n} \morph \set{y_1,...y_{m}}$ then if in an instances $E$ the function $E_{\tuple{y_1,...y_{m}}}$ is 
injective then the function $E_{\tuple{x_1,...x_n}}$  is injective.
\end{lemma}
\begin{proof}
Follows from the definition of functional dependency that for each $j$, $1 \leq j \leq m$  there is a function ${f_j}$,  such that
$E_{y_j}=E_{\tuple{x_1,...x_n}}\circ f_j$. Therefore there is a function $f=\tuple{f_1,...f_m}$ such that
$E_{\tuple{y_1,...y_m}}= E_{\tuple{x_1,...x_n}}\circ f $. Therefore since  the function $E_{\tuple{y_1,...y_{m}}}$ is 
injective then the function $E_{\tuple{x_1,...x_n}}$  is injective.
\end{proof}

\iffalse
\begin{definition} %elementary
In an entity model $\genericmodel$, a functional dependency $\{x_1,...x_n\} \morph y$
is said to be elementary if there is no  subset $X' \subset$, such that $X' \morph y$.
\end{definition}
\fi

\begin{figure}[h]
\begin{center}
\begin{tabular}{p{3.0cm} c p{1cm} c}
The dependency of &$
\begin{array}{cp{0.5cm}c}
\Rnode{p}{p}   &  &   \Rnode{c}{c}
\end{array}
$
\ncarr{p}{c}
\blabel{R_0}
\idcomp  
                    & on & $
\begin{array}{cp{0.5cm}cp{0.5cm}c}
               &  &                  & &     \Rnode{v1}{v}  \\[-0.35cm]
               &  &   \Rnode{d1}{d}  & &                    \\[0.2cm]
\Rnode{p}{p}   &  &   \Rnode{d2}{d}  & &     \Rnode{v2}{v}  \\[0.35cm]
               &  &   \Rnode{c2}{c}  & &                    \\[-0.35cm]
               &  &                  & &     \Rnode{v3}{v}
\end{array}
$
\ncarr[10]{d1}{v1}
\alabel{oId}
\idcomp 
\ncarr{d2}{v2}
\blabel{dSn}
\idcomp
\ncarr[10]{p}{d1}
\alabel{S_0}
\idcomp 
\ncarr{p}{d2}
\blabel{S_0}
\idcomp
\ncarr[-10]{p}{c2}
\blabel{R_0}
\idcomp 
\ncarr[-5]{c2}{v3} 
\blabel{cSn}
\idcomp \\[0.5cm]
										follows from \\[0.5cm]
the dependency of & $
\begin{array}{cp{0.5cm}c}
\Rnode{p}{p}   &  &   \Rnode{c}{c}
\end{array}
$
\ncarr{p}{c}
\blabel{R_0}
\idcomp  & on    &
$
\begin{array}{cp{0.5cm}cp{0.5cm}c}
               &  &                  & &     \Rnode{w}{w}   \\[-0.35cm]
               &  &   \Rnode{d}{d}  & &                    \\[-0.05cm]
\Rnode{p}{p}   &  &                  & &                    \\[-0.05cm]
               &  &   \Rnode{c2}{c}  & &                    \\[-0.25cm]
               &  &                  & &     \Rnode{v}{v}
\end{array}
$
\ncarr[5]{p}{d}
\alabel{S_0}
\idcomp 
\ncarr[-5]{p}{c2}
\blabel{R_0}
\idcomp 
\ncarr[5]{d}{w}
\alabel{S_1}
\ncarr[-5]{c2}{v} 
\blabel{cSn}
\idcomp        \\[0.5cm]
and from \\[0.5cm]
the dependency of & 
$
\begin{array}{cp{0.5cm}cp{0.5cm}c}
\Rnode{p}{p}   &  &   \Rnode{d}{d}  & &    \Rnode{w}{w}                
\end{array}
$
\ncarr[5]{p}{d}
\alabel{S_0}
\idcomp 
\ncarr[5]{d}{w}
\alabel{S_1}

& on&
$
\begin{array}{cp{0.5cm}cp{0.5cm}c}
               &  &                  & &     \Rnode{v1}{v}   \\[-0.35cm]
               &  &   \Rnode{d1}{d}  & &                    \\[-0.05cm]
\Rnode{p}{p}   &  &                  & &                    \\[-0.05cm]
               &  &   \Rnode{d2}{d}  & &                    \\[-0.25cm]
               &  &                  & &     \Rnode{v2}{v}
\end{array}
$
\ncarr[5]{p}{d1}
\alabel{S_0}
\idcomp 
\ncarr[-5]{p}{d2}
\blabel{S_0}
\idcomp 
\ncarr[5]{d1}{v1}
\alabel{oId}
\ncarr[-5]{d2}{v2} 
\blabel{dSn}
\idcomp 
\end{tabular}
\caption{Example of a Transitive Functional Dependency based on the example in figure \ref{pickexample}}
\end{center}
\end{figure}



\begin{definition}
\noindent In an entity model $\genericmodel$  
if for some $n \geq 1$, $a$, $b_{i, 1 \leq i \leq n}$,  and $c$ are entity types and 
if  $x_{i, 1 \leq i \leq n}$, and $y$ are simple paths such
that there is a \commentary{Removed elementary intransitive}functional dependency $\{x_1,...x_n\} \morph y$
 then the functional dependency $\{x_1,...x_n\} \morph y$ is said to be \term{represented} in the ER model $\genericmodel$ iff  there exists an entity type $d$ and 
an identifying tuple of simple paths with respect to
$d$, $\qntuple$, such that $\incd{a}{x_1,...x_n}{d}{q_1,...q_n}$, and a simple
path $z=\langle z_1,...z_l \rangle$ such that $z:d\rightarrow c$, for some $l \geq 0$ as here:


\setlength{\arraycolsep}{.2cm}
\begin{center}

$
\begin{array}{cp{2cm}c p{2cm} c}
             &  & \Rnode{b1}{b_1} & &               \\ [0.7cm]
						 &  & \Rnode{b2}{b_2} & &               \\ [0.4cm]
						 &  &     \vdots      & &               \\ [0.2cm]
\Rnode{a}{a} &  &                 & & \Rnode{d}{d}  \\ [-0.2cm]
             &  & \Rnode{bn}{b_n} & &               \\ [0.9cm]
             &  & \Rnode{c}{c}    & &               \\
\end{array}
$
\simplepatha{$x_1$}{a}{b1}
\simplepatha{$x_2$}{a}{b2}
\simplepatha{$x_n$}{a}{bn}
\simplepathb{$y$}{a}{c}
\simplepathb{$q_1$}{d}{b1}
\simplepathb{$q_2$}{d}{b2}
\simplepathb{$q_n$}{d}{bn}
\simplepatha{$z$}{d}{c}
\end{center}


and
\commentary {Removed that $z_1$ is not identifying but expect to prove not identifying.}         
such that in all instances $E$, 
$\tuple{ E_{x_1},...E_{x_n}} \circ inv_{E_{\tuple{q_1,...q_n}}}\circ E_{\tuple{z_1,...z_l}} = E_y$
\begin{center}
$
\begin{array}{cp{0.75cm}cp{0.55cm}c}
               &&\Rnode{Eb1}{E_{b_1}}   &&                      \\ [0.6cm]
               &&\Rnode{Eb2}{E_{b_2}}   &&                      \\ [0.2cm]
\Rnode{Jnctn}{}&&  \vdots               &&\Rnode{RightJnctn}{}  \\ [0.1cm]							
               &&\Rnode{Ebn}{E_{b_n}}   &&                      \\ [0cm]
							 &&                       &&\mbox{ }\Rnode{Ed}{E_d}       \\ [0.5cm]
               &&\Rnode{Ec}{{E_c}}      &&                      \\
\end{array}
$
\nchmarr[-20][-45]{Eb1}{Ebn}{Jnctn}{Ec}
\nbput[npos=0.2]{$f_E$}
\ncarc[arcangle=-20]{Eb2}{Jnctn}
\nchmarr[20][15]{Eb1}{Ebn}{RightJnctn}{Ed}
\naput[npos=-0.2]{$ inv_{E_{\qntuple}}   $}
\ncarc[arcangle=20]{Eb2}{RightJnctn}
\ncarc[arcangle=15]{->}{Ed}{Ec}
\naput[npos=0.4]{$E_{\langle z_1,...z_n \rangle}$}
\end{center} 
\end{definition}
 
\begin{definition}
\noindent An ER model $\genericmodel$ is \term{well-formulated} 
 iff 
\begin{enumerate} [(i)]
\item{
it satisfies the minimality condition (see section \ref{minimalitycondition}),
}
\item{ 
all referential inclusion dependencies\footnote{Equivalently, by lemma
\ref{simplerepresentationlemma}, sufficient that all simple referential inclusion dependencies be represented.} are  represented,
}
\item{
for each intransitive functional dependencies $\{x_1,...x_n\} \morph y$ in model \genericmodel\
either in every instance $E$, $E_{\xntuple}$ is injective and total
or else the functional dependency is represented in the model.
}
\end{enumerate}
\end{definition}

This lemma is a converse to lemma \ref{primarypathrepresentativesidentifying}
\begin{lemma} 
\label{identifyingconverse}
In a well-founded model \genericmodel\ if $Q$ is a identifying set of paths from entity type $a$ to $\veee$
then for every primary key path $p$ from $a$ there is a $q$ in $Q$ such that $p \simeq q$.
\end{lemma}
\begin{proof}
We have modified clause (i) of the definition of well-formed to achieve this.
\end{proof}

\begin{lemma}
\label{mainlemma}
In a well-formed model \genericmodel\ if $\{x_1,...x_n\} \morph y$  is a non-trivial
functional dependency then either $E_{\xntuple}$ is injective and total in every instance $E$ 
or else there is a non-empty  simple path $p:a \rightarrow d$
and a non-identifying path $z:d \morph c$ such that $y \simeq p \circ z$.
\end{lemma}
\begin{proof}
Since \genericmodel\ is well formed then either $E_{\xntuple}$ is injective and total in every instance $E$ and we are done or else the
the functional dependency
$\{x_1,...x_n\} \morph y$   is represented 
by  an entity type $d$ and 
an identifying tuple of simple paths with respect to
$d$, $\qntuple$, such that $\incd{a}{x_1,...x_n}{d}{q_1,...q_n}$ and a simple
path $z=\langle z_1,...z_l \rangle$ such that $z:d\rightarrow c$, for some $l \geq 0$ as here:

\setlength{\arraycolsep}{.2cm}
\begin{center}

$
\begin{array}{cp{2cm}c p{2cm} c}
             &  & \Rnode{b1}{b_1} & &               \\ [0.7cm]
						 &  & \Rnode{b2}{b_2} & &               \\ [0.4cm]
						 &  &     \vdots      & &               \\ [0.2cm]
\Rnode{a}{a} &  &                 & & \Rnode{d}{d}  \\ [-0.2cm]
             &  & \Rnode{bn}{b_n} & &               \\ [0.9cm]
             &  & \Rnode{c}{c}    & &               \\
\end{array}
$
\simplepatha{$x_1$}{a}{b1}
\simplepatha{$x_2$}{a}{b2}
\simplepatha{$x_n$}{a}{bn}
\simplepathb{$y$}{a}{c}
\simplepathb{$q_1$}{d}{b1}
\simplepathb{$q_2$}{d}{b2}
\simplepathb{$q_n$}{d}{bn}
\simplepatha{$z$}{d}{c}
\end{center}
such that
\begin{equation}
\label{fundependency}
E_y = E_{\xntuple} \circ E_{\qntuple}^{-1} \circ E_z
\end{equation}

Because $\incd{a}{x_1,...x_n}{d}{q_1,...q_n}$,  and from
clause (ii) of the definition of well-formedness since \genericmodel\ is well formed, it follows that
there  is a simple path $p:a \rightarrow d$ such
that 
%\foreachi, $p \circ q_i \simeq x_i$ and therefore
in all instances $E$,

\begin{equation}
\label{incdependency}
E_{\xntuple} \circ inv_{E_{\qntuple}} = E_p
\end{equation}
Now we have  that in all instances $E$ that 
\begin{align*}
E_y &= E_{\xntuple} \circ E_{\qntuple}^{-1} \circ E_z  && \text{ (\ref{fundependency}), above,}\\
    &= E_p \circ E_z                                   && \text{by (\ref{incdependency}),}\\
		&= E_{p \circ z}                                   && \text{from defn. of instance }E.\\
\end{align*} 
\vspace{-0.3cm}
From which, by definition, $y\simeq p \circ z$, as required. \\
\end{proof}
Now we give three examples of ER models that fail to be well formed. 
Figures \ref{clubpresidentbeforenormalisation}, \ref{rawdatatablegraph} and  \ref{pickexamplebeforenormalisation}
show examples that fail, respectively, conditions (i), (ii) and (iii) respectively of the definition of well-foundedness. 

The failure of the example in figure \ref{rawdatatablegraph} to be well-formed can be rectified by 
replacing the SeqNo attribute of 
the data cell entity (dc) by an identifying relationship $R_0:dc \morph c$. The resuting model was given in figure \ref{datatablegraph}.

\begin{figure} [h]
\begin{center}
\begin{tabular}{c c}
$
\begin{array}{cp{0.4cm}cp{0.5cm}cp{0.75cm}cp{0.75cm}c}
              &&                &&               &&                &&               \\[0.25cm]
              && \Rnode{cm}{cm} &&               &&                &&               \\[0.25cm]
\Rnode{p}{p}	&&                && \Rnode{m}{m}  &&   \Rnode{c}{c} && \Rnode{v}{v}  \\[0.25cm]
	            &&  
\end{array}
$

\ncarr[5]{p}{cm} 
\alabel{I_0}
\idcomp
\ncarr[-30]{p}{v}
\blabel{yr}
\idcomp
\ncarr[-5]{cm}{m}
\blabel{I_1}
\idcomp
\ncarr[50]{cm}{v}
\alabel{yr1}
\idcomp
\ncarr[25]{m}{v}
\alabel{mNo}
\idcomp
\ncarr{m}{c}
\blabel{M}
\idcomp
\ncarr{c}{v}
\blabel{cId}
\idcomp
& \footnotesize
\begin{tabular}{c p{1.5cm} p{4cm}}
KEY && \\
\hline
p  & president        & Identified by a combination of inclusion relationship ($I_0$) that identifies a
                       president as being a committee members and a year attribute ($yr$). \\
cm & committee member & Identified by an inclusion relationship ($I_1$)that identifies a committee member 
                         as a member and a first year of service attribute ($y1$).\\
m  & member           & Identified by a member relationship ($M$) to a club and a by membership 
                         number attribute ($mNo$). \\
c  & club             & Identified by club identifier attribute ($cId$)
\end{tabular} 
\end{tabular}
\end{center}
\caption{Club president example. 
This example fails clause (i) of the definition of well-formedness because
the set of paths $\set{\tuple{I_0,yr1},\tuple{I_0,I_1,mNo}, \tuple{I_0,I_1,M,cId}, \tuple{yr}}$ is 
an identiofying set of paths wrt $p$ but is not minimum because a club has at most one president a year and sothe subset $\set{\tuple{I_0,yr1}, \tuple{I_0,I_1,M,cId}, \tuple{yr}}$ is jointly monomorphic. 
This problem can be fixed by introducing a clum membership relationship $R: p \morph c$ which is equivalent to
path $\tuple{I_0,I_1,M}$ and by specifying $R$ in place of $I_0$ which is to be combined with $yr$ to identify entitites of type $p$.
}
\label{clubpresidentbeforenormalisation}
\end{figure}

\commentary{Club president example is isomorphic to the nearest shop example!}

\begin{figure} [h]
\begin{tabular}{c p {0.5cm} c}
(a) & &
\begin{tabular}{c c}
$
\begin{array}{cp{0.75cm}cp{0.75cm}c}
   \Rnode{r}{r}     & & \Rnode{t}{t} & & \Rnode{v}{v} \\[1.2cm]     
	 \Rnode{d}{d}   & & \Rnode{c}{c} & &               
\end{array}
$
\ncarr{r}{t} 
\alabel{S_1}
\idcomp
\ncarr{t}{v} 
\alabel{tN}
\idcomp
\ncarr{c}{v} 
\blabel{cN}
\idcomp
\ncarr[-90]{d}{v}
\blabel{SeqNo}
\idcomp
\ncarr{d}{r}
\alabel{S_0}
\idcomp
\ncarr{c}{t}
\blabel{R_1}
\idcomp
\ncarr[50]{r}{v}
\alabel{rN}
\idcomp

& \footnotesize
\begin{tabular}{c p{1.5cm} p{4cm}}
KEY && \\
\hline
t & table & Having identifying attribute tN the name of the table. \\
c & column & Identified by a combination of column number cN and relationship $R_1$ to the table it is a column of.\\
r & row & Identified by its row number $rN$ and its relationship $S_1$ to the table it is a row of.\\
d & data cell & Identified by relationship $S_0$ to the row it is in and its $seqNo$ attribute
representing its ordinal position within data cells of the row. \\
\end{tabular} 
\end{tabular} \\
(b) &   & 
$
\begin{array}{cp{0.5cm}c c c }
   \Rnode{d}{d} &                  &                       & \Rnode{c}{c}  &       \\[.5cm]  
		            & \ \ \ \Rnode{r}{r} &                     &               &       \\[.5cm] 
	              &                  & \Rnode{b1}{b_1} \ \ \ &               &  \Rnode{bn}{b_n}  
\end{array}
$
\ncarr{d}{r} 
\blabel{S_0}[0.25]
\ncarr{r}{b1} 
\blabel{S_1}[0.25]
\ncarr{d}{bn}
\alabel{SeqNo}[0.25]
\ncarr{c}{b1}
\blabel{R_1}[0.25][1]
\idcomp
\ncarr{c}{bn}
\alabel{cN}[0.25]
\idcomp
\end{tabular}
\caption{(a) The model fails condition (ii) for being well-formulated
because there is a referential inclusion dependency of $d[ \tuple{S_0,S_1},\tuple{SeqNo}]$
in $c[\tuple{R_0},\tuple{cN}]$, as indicated in (b), which is not explicitly represented in the model.}
\label{rawdatatablegraph}
\end{figure}

\begin{figure} [h]
\begin{center}
\begin{tabular}{c c}
$
\begin{array}{cp{0.75cm}cp{0.75cm}c}
   \Rnode{p}{p}     & & \Rnode{c}{c} & & \Rnode{v}{v}    
\end{array}
$
\ncarr{p}{c} 
\alabel{R}
\idcomp
\ncarr[25]{c}{v} 
\alabel{wId}
\idcomp
\ncarr[-10]{c}{v} 
\blabel{cSn}
\idcomp
\ncarr[-65]{c}{v}
\blabel{iC}
\ncarr[80]{p}{v}
\alabel{dI}
\idcomp
\ncarr[-100]{p}{v}
\blabel{q}
& \footnotesize
\begin{tabular}{c p{1.5cm} p{4cm}}
KEY && \\
\hline
p & pick & Identified by a combination of relationship $R$ to the crate being picked from
                and the delivery id attribute ($dI$). Also  having a quantity picked 
								attribute ($q$).\\
c & crate & Identified by a combination of warehouse name attribute ($wN$)
              and crate serial number attribute ($cSn$). Also has an item type ($iT$) attribute.
 to the table it is 
\end{tabular} 
\end{tabular}
\end{center}
\caption{If all deliveries are made from items picked from a single warehouse 
then this model fails clause (iii) of the definition of well-formulated because, with tespect to entity type $p$, the path $\tuple{R,wN}$ will be dependent on the path $\tuple{dI}$ but neither is the set 
$\{\tuple{dI}\}$ a mono-source nor has the dependency a representation within the model. A well-formulated and fully factored model for this situation was described previously in 
figure \ref{pickexample}.
}
\label{pickexamplebeforenormalisation}
\end{figure}

\commentary{need review mono-source injective terminology}
\section{Definitions Of Logical and Physical Entity Models}

	
\subsection{Definition of Logical ER Model}

\begin{definition}
A well-formulated ER model is \term{purely-logical}  iff it 
also satisfies:
\begin{enumerate}[(i)]
\item
for all edges $r$ if there is a
a simple path $p$  not including $r$ such that
 $r \simeq p$  then $r$ is identifying and there is an edge $e$ in $p$ which is not identifying,
\item the set of identifying sets I of the model is fully-factored.
\end{enumerate} 
\end{definition}

\noindent
We say that an ER model is a \term{logical ER model} iff it is purely logical.

\subsection{Definition of Physical ER Model}
\begin{definition}
\noindent 
A \term{physical ER model} is a well formulated ER model that also satisfies:
\begin{enumerate}[(i)]
\item
all identifying edges are attributes, \commentary{the set of sets of identifying edges $I$ is closed under composition i.e. $I = \bar{I}$.}

\item
for each relationship $r$ there is there is an inclusion dependency that it represents.
\end{enumerate} 
\end{definition}


\section{First Cut Chen Transformation -- $\chiZero$}

\begin{align}
&\et[\chiZeroM]        && = \et        && \\
&\rel[\chiZeroM]{x}{y} && = \rel{x}{y}  &&\\
&\attr[\chiZeroM]{x}   && = \attr{x}  \cup  \setsuchthat{\tuple{r,k}}{\text{for some }
			                  y \in \et, \ r \in \rel{x}{y} \text{ and } k \in \pk{y}} 
\end{align}

Alternatively, we can define $\attr[\chiZeroM]{x}$ as the union of $\attr{x}$ and foreign keys $\fk{x}$
where we define the latter by:
\begin{equation}
\fk{x} = \setsuchthat{\tuple{r,q_1,...q_n,a}}
                               {r \in \rel{x}{y},
                                  \text{ for some } y \in \et,
																	\text{ and } \tuple{q_1,...q_n,a} \in \pk{y},
																	\text{ for some } n \geq 0
			                   }
\end{equation}

\section{Chi Transform - a Revised Chen Transformation}

This then needs revision\footnote{Check this!} as follows:
\begin{multline}
\fkp{x} = \setsuchthat{\tuple{r,q_1,...q_n,a}}
                      {r \in \rel{x}{y},
                                  \text{ for some } y \in \et,
																	\text{ and } \tuple{q_1,...q_n,a} \in \pk{y},
																	\text{ for some } n \geq 0,                   \right. \\        
											\left.		\text{ and } \tuple{r, q_1,...q_n,a}
																	\text{ is not subsumed by any path }
																	      \tuple{s_1,...s_m} \in \path{x}{v}
											}			                   
\end{multline}

Finally we can define:

\begin{equation}
\fkpp{x} = \fkp{x}/\simeq
\end{equation}
and 

\begin{equation}
\attr[\chiM]{x}  = \attr{x}  \cup  \fkpp{x}
\end{equation}


\begin{lemma}
\label{fdcarrythroughlemma}
Functional depedencies in \genericmodel\ carries through to functional depdencies in $\genericphysical$ \commentary{Expand}
\end{lemma}
\begin{proof}

\end{proof}

\section{Boyce-Codd Normal Form}
\noindent One measure of the goodness of a physical model is whether it satisfies the well-formedness condition know as Boyce Codd Normal Form.
Written in the terminology we are using here it can be defined as follows:
\begin{definition} % BCNF
A physical ER model is in Boyce Codd Normal Form (BCNF) \commentary{See Zaniolo definition 2.} iff
for all entity types $a$, for all attributes $x_1,...x_n$ and $y$ of $a$, for $n \geq 1$, 
for which  there is a non-trivial functional dependency \sfd{x_1,...x_n}{y}, 
in all instances $E$  of $\genericmodel$, the function $E_{<x_1,...x_n>}$ is injective and total.  
\end{definition}

\begin{definition} % TNF
A physical ER model is in Third Normal Form (TNF)  iff
for all entity types $a$, for all attributes $x_1,...x_n$ and $y$ of $a$, for $n \geq 1$, 
for which  there is a non-trivial functional dependency \sfd{x_1,...x_n}{y}, 
in all instances $E$  of $\genericmodel$, either the function $E_{<x_1,...x_n>}$ is injective and total
or else $y$ is all identifying.\commentary{new - justify this}.  
\end{definition}

\noindent The next lemma simplifies the requirement for showing BCNF to consideration of non-trivial
intransitive functional dependencies:
\begin{lemma}
\label{BCNFsublemma}
A model $\genericmodel$ is in BCNF iff
for all entity types $a$, for all attributes $x_1,...x_n$ and $y$ of $a$, for $n \geq 1$, 
for which there is an non trivial intransitive functional dependency \sfd{x_1,...x_n}{y},
in all instances $E$, $E_{<x_1,...x_n>}$ is injective and total. 
\end{lemma}
\begin{proof}Use lemma \ref{transitiveinjectivelemma}.
\end{proof}

\noindent Now for the main theorem:\\
\begin{theorem}
\noindent If an ER model $\genericmodel$ is well-formulated and logical then the 
transformed model $\logtophys(\genericmodel)$  is in Third Normal Form. If in $\genericmodel$
there are no identifying edges $r$ such that $r \simeq p$ for some path $p$ of length greater than $1$
then $\logtophys(\genericmodel)$ is in Boyce-Codd Normal form. \commentary{Needs modifying to TNF}.
\end{theorem}
\begin{proof}
By lemma \ref{BCNFsublemma} it suffices to show that if 
$\sfd{\phys{x}_1,...\phys{x}_n}{\phys{y}}$ is a non-trivial intransitive functional dependency of model $\genericphysical$
where $\phys{x}_1,...\phys{x}_n,\phys{y}$ are attributes of the entity type $a$ of model $\genericphysical$ 
then  in all instances
$E$ of $\genericphysical$, $E_{<\phys{x}_1,...\phys{x}_n>}$ is injective and total. \\


\noindent Assume, then, such a functional depedency $\sfd{\phys{x}_1,...\phys{x}_n}{\phys{y}}$ in  $\genericphysical$. 
By lemma \ref{fdcarrythroughlemma} it follows that in the model $\genericmodel$, for each $i$, $1 \leq i \leq n$, 
for some $m_i$, $m_i \geq 1$, we have a 
path of length $m_i$ which we denote $x_i = \langle x_{i,1},...x_{i,m_i} \rangle$  and for some $m$, $m \geq 1$ we have a path 
of length $m$ which we denote 
$y=\langle y_1,..y_m \rangle$ as shown here: 
\setlength{\arraycolsep}{.2cm}
\begin{center}
$
\begin{array}{cp{2cm}cc}
             & &         & \Rnode{b1}{v} \\ [0.5cm]
						 & &         & \Rnode{b2}{v} \\ [0.6cm]
						 & & \vdots  &                 \\ [0.2cm]
\Rnode{a}{a} & &         & \Rnode{bn}{v} \\ [1.0cm]
             & &         & \Rnode{c}{v}   \\
\end{array}
$
\simplepatha{$x_1$}{a}{b1}
\simplepatha{$x_2$}{a}{b2}
\simplepatha{$x_n$}{a}{bn}
\simplepathb{$y$}{a}{c}
\end{center}
and that \sfd{x_1,...x_n}{y} in \genericmodel.

From the assumption that the model $\genericmodel$ is well-formulated and from condition (iii) 
of the definition of well-formulated, either 
$E_{\langle x_1,...x_n \rangle}$ is injective and total in every instance $E$ of $\genericmodel$, in which case 
$\logtophys(E)_{\langle \phys{x}_1,...\phys{x}_n \rangle}$ is injective and total in every instance $E$ of $\genericmodel$ and the proof is completed, or else
the functional dependency  \sfd{x_1,...x_n}{y} is represented in the model $\genericmodel$. 


From the definition of a functional dependency being represented it follows that
there is an entity type $b$ in $\genericmodel$  and an identifying family of simple paths $q_1,...q_n$, 
$q_i: b \rightarrow v$ and a 
path $z: b \rightarrow v$, as shown here: 
such that


\setlength{\arraycolsep}{.2cm}
\begin{center}
$
\begin{array}{cp{2cm}ccp{2cm}c}
             & &         & \Rnode{b1}{v} &&              \\ [0.5cm]
						 & &         & \Rnode{b2}{v} &&              \\ [0.6cm]
						 & &         &    \vdots     &&              \\ [0.2cm]
\Rnode{a}{a} & &         & \Rnode{bn}{v} && \Rnode{b}{b} \\ [1.0cm]
             & &         & \Rnode{c}{v}  &&              \\
\end{array}
$
\simplepatha{$x_1$}{a}{b1}
\simplepatha{$x_2$}{a}{b2}
\simplepatha{$x_n$}{a}{bn}
\simplepathb{$y$}{a}{c}
\simplepathb{$q_1$}{b}{b1}
\simplepathb{$q_2$}{b}{b2}
\simplepathb{$q_n$}{b}{bn}
\simplepatha{$z$}{b}{c}
\end{center}
such that
\begin{equation}
\label{inclusiondependency}
\incd{a}{x_1,...x_n}{d}{q_1,...q_n}
\end{equation}
and such that in every instance $E$ of $\genericmodel$:
%\newcommand{\ineveryinstance}{\forall E \in inst_\genericmodel ,  \hspace{0.25cm} }
\begin{equation}
\label{representationByIdentifyingRelationships}
E_{\langle x_1,...x_n \rangle}\circ inv_{E_{\qntuple}} \circ E_{z} = E_{\langle y_1,..y_m \rangle} 
\end{equation}

Note that $z$ cannot be equivalent to a primary key attribute for if it were then by lemma \ref{identifyingconverse}
it would be equivalent to $q_i$,
for some $i$, $1 \leq i \leq n$ from which it would follow that $E_y=E_{x_i}$ in each instance $E$ and therefore, from the definitions, 
we would have that $y \simeq x_i$ and therefore that $\phys{y}=\phys{x_i}$ contradicting the assumption that
the fucntional dependency $\sfd{\phys{x}_1,...\phys{x}_n}{\phys{y}}$ is non-trivial.


Since there is an inclusion dependency (\ref{inclusiondependency}) in $\genericmodel$ and from the assumption that  model
$\genericmodel$ is well-formulated it follows
 from condition (ii) of the definition of well-formulated  that there exists
a path $\langle p_1,...p_k \rangle:a \rightarrow b$, $k \geq 0$, such that:
\begin{equation}
\label{simpleRepresentationOfFirstPart}
\langle E_{x_1},...E_{x_n} \rangle \circ inv_{E_{\qntuple}} = E_{\langle p_1,...p_k \rangle}
\end{equation}
\noindent Either $k=0$ and $\qntuple = \langle x_1,...x_n \rangle$ in which case $E_{\langle x_1,...x_n \rangle}$ is
injective and total in every instance $E$ of $\genericmodel$ and thus $\logtophys(E)_{\langle \phys{x}_1,...\phys{x}_n \rangle}$ is
injective and total in every instance $\logtophys(E)$ of $\genericphysical$ and the proof is complete 
\noindent or else $k \geq 1$ and it follows from (\ref{representationByIdentifyingRelationships}) and (\ref{simpleRepresentationOfFirstPart}) that:
\begin{equation} 
\label{simpleRepresentation}
 E_{\langle p_1,...p_k \rangle} \circ E_{\langle z_1,...z_l \rangle} = E_{\langle y_1,..y_m \rangle} 
\end{equation}


We will show that this leads to a contradiction and so complete the proof. 

If $m >1$ then
 from  (\ref{simpleRepresentation}),
from clause (i) of the definition of subsumes and from the fact demonstrated above that $z$ is not equivalent to a 
primary key attribute
it follows that $p_1,...p_k,z_1,...z_l$ subsume $\langle y_1,..y_m \rangle$, 
which implies that $\langle y_1,..y_m \rangle$ is excluded from $\fkp{a}$ and thus that 
 $\phys{y}=\left[\langle y_1,..y_m \rangle\right]$ is not an attribute of $\genericphysical$ contrary to our 
initial assumption. 

We must conclude that $m=1$.
In this case we have $y_1$, an attribute of $a$ in $\genericmodel$, and from (\ref{simpleRepresentation})  we have in all instances $E$ of $\genericmodel$:
\begin{equation}
E_{y_1}=E_{\langle p_1,...p_k \rangle} \circ E_{\langle z_1,...z_l \rangle}
\end{equation}
\noindent which is to say that in all instances $E$ of $\genericmodel$:
\begin{equation}
E_{y_1}=E_{\langle p_1,...p_k ,z_1,...z_l \rangle}
\end{equation}
\noindent
We have shown, therefore, that $y_1$ is an outgoing edge of $a$ in $\genericmodel$ 
which is equivalent to a simple path of $\genericmodel$ of length $\geq 2$.
Therefore since $m$ is logical,  $y_1$ is identifying as required to establishing TNF
or if we have assumed no such $y_1$ the assumption is contradicted and we establish BCNF.
This completes the proof.
\end{proof}

\section{Logical Models}
I am using the term \term{logical} in the sense that it is used in certain ER modelling methodolgies such as ???.
\subsection{Fully Factored Models}

\begin{definition}
In an entity model \genericmodel\ if $a$ is an entity type then say that
a \term{factorisation} of the set $I_a$ is a subset $\set{h_1,...h_n} \subset I_a$
and a  path $f:a \morph b$ and a set of paths sourced at $b$, $\set{g_1,...g_n} \in \bar{I}$\commentary{not defined}
such that for each $i$, $1 \leq i \leq n$, $f \circ g_i \simeq h_i$. 
\end{definition}

The factorisation consists  
of a base set of edges: $h=I \backslash \set{h_1,...h_n}$, a mediating path $f:a \morph b$ and a remainder 
set $j$ paths sourced at $b$, $j \in \bar{I}$, 
such that $I_a \simeq h \cup \setsuchthat{f\circ g}{g \in j}$.

\begin{definition}
If \genericmodel\ is a model and if there is a factorisation at entity type $a$ with
base set $h$, mediating path $f: a \morph b$ and remainder $j \in \bar{I}$ then define
a factored model $\genericmodel'$ determined by the factorisation to be the model
whose underlying graph is the graph of \genericmodel\ with edges $h_1,...h_n$ removed 
(defining database instances restricted) and
an edge $k$ added, $k:a \morph b$, each database instance $E$ extends to $E'$, $E'_k = E_f$.
Identifying sets of $\genericmodel'$ are exactly as for \genericmodel\ except for at the entity
type $a$ where the identifying set $I_a$ in $\genericmodel'$ is defined to be $h \cup \set{k}$.
Note that by lemma \ref{identifyingsetdeduction} the modified $E'$ is a database instance because
the family 
$\setsuchthat{E'_e}{e \in \set{h \cup \set{k}}}$ is jointly injective because 
$\setsuchthat{E'_e}{e \in \set{h \cup \set{k}}} =\setsuchthat{E_e}{e \in \set{h \cup \set{f}}}$
and in model \genericmodel, $h \cup \set{f}$ is a mono-source because both
$h \cup \set{f} \backslash f \cup \setsuchthat{f \circ g}{g \in j}
=h  \cup \setsuchthat{f \circ g}{g \in j}=I_a$ and $j$ are  mono-sources.


\end{definition}

\begin{definition}
If $a$ is an entity type in an entity model \genericmodel\ then define the abstraction
level of $a$ to be the sum of the lengths of the primary key paths sourced at $a$.
\end{definition}


\begin{definition}
Define  the abstraction level of a model $\genericmodel$ to be
the sum over all entity types $a$ of the abstraction level of $a$.
\end{definition}

\begin{lemma}
If $\genericmodel'$ is a factorisation of model $\genericmodel$ then abstraction level 
of $\genericmodel'$ is greater than abstraction level of $\genericmodel$ and the number of edges
is not increased. 
\end{lemma}
\begin{proof}
\end{proof}

\begin{definition}
Define an entity model \genericmodel\ to be fully-factored iff there no entity types $a$
for which the set $I_a$ has  factorisations.
\end{definition}

\begin{definition}
If \genericmodel\ is a model then say that a $\genericmodel'$ is a fully-factored 
equivalent to \genericmodel\
if it results from successive factorisations of \genericmodel\ until the resulting model can no longer be
factorised. 
\end{definition}

\begin{lemma}
Every model \genericmodel\ has a fully factorised equivalent. 
\end{lemma}
\begin{proof}
We need to show that the factorisation process will terminate. 
to do this note that we can put a (crude) bound on the abstraction level of an entity model
of number of entity types multiplied by the square of the  number of edges
\commentary{Need revise this most likely}.  Because the number of nodes remains constant during
factorisation and because the number of edges does not increase this bound can be calculated
for the model \genericmodel\ and serves as an upper bound for the abstraction level of 
any successive factorisation of \genericmodel. Since factorisation increases abstraction level
and there is an upper bound to abstraction level then the process cannot continue indefinitely. 
\end{proof}

\subsection{Minimality Condition of Factored Model}

\begin{lemma}
\label{identifyingfactorminimal}
In a model \genericmodel, if $i$ is a source with domain $a$,
if $f \in i$, $f: a \morph b$, 
and if $j$ is a mono-source with domain $b$
then  if $(i \backslash \set{f}) \cup \setsuchthat{f \circ g}{g \in j}$ is a minimum mono-source
then $i$ is a minimum mono-source.
\end{lemma}
\begin{proof}
We assume that $(i \backslash \set{f}) \cup \setsuchthat{f \circ g}{g \in j}$ is a mono-source and that no subset of it is a mono-source.
By lemma \ref{identifyingsetdeduction} it follows that $i$ is a mono-source we need to show that
for no edge $e \in i$ is $i \backslash \set{e}$ a mono source. Assume to the contrary that there is such an $e$ then  either $e = f$ or else
$e \in i \backslash \set{f}$. In the first case $(i \backslash \set{f})$ is a subset of  $(i \backslash \set{f}) \cup \setsuchthat{f \circ g}{g \in j}$ which is a mono-source which contradicts the initial assumption. In the second  case 
by lemma \ref{identifyingsetdeduction} since $i \backslash \set{e}$ is a mono-source it follows that
$((i \backslash \set{e})\backslash \set{f}) \cup \setsuchthat{f \circ g}{g \in j}$ is a mono-source which again is to say that a subset
of $(i \backslash \set{f}) \cup \setsuchthat{f \circ g}{g \in j}$ is a mono-source contradicting the initial assumption.
We conclude that no subset of $i$ is a mono-source.
\end{proof}

\begin{lemma}
If a model \genericmodel\ satisfies the minimality condition then it has a fully factored
equivalent $\genericmodel'$  and $\genericmodel'$ satisfies the minimality condition. 
\end{lemma}
\begin{proof}
By lemma \ref{identifyingfactorminimal} we have the minimality of $I_a$ in $\genericmodel'$.

To show that representative sets of primary key paths in $\genericmodel'$ are minimal mono-sources
first note that to each path $p'$ in $\genericmodel'$ there is a corresponding path 
$p$ in \genericmodel\  
 such that for all instances $E$ of \genericmodel\ with corresponding $E'$ of $\genericmodel'$, 
$E_p=E'_{p'}$ and that this correspondence establishes a 1-1 correspondence between the set of primary key paths $P'$ in $\genericmodel'$
and the set of primary key paths $P'$ in \genericmodel. Because $E_p=E'_{p'}$, for all defining instances
$E$, it follows that a representative set for $P'$ corresponds to a representative set for $P'$ and also
that mono-sources in $\genericmodel'$ correspond to mono-sources in \genericmodel. From this it follows that representative sets of primary key paths in $\genericmodel'$ are minimal mono-sources.
\end{proof}
