
\begin{abstract}
We present a formal notion of ER model as a notion of data specification sufficiently general to encompass  (i) relational schemas of Codd's relational
data model
(ii) hierarchical schemas such as those of the nested relational data model, those such as are commonly
implemented in XML and represented in XML schemas or those represented in variants of Interface Definition Language (IDL) and represented in various open and propriatary formats most recently 
in Google's protocol buffer format,
(iii) entity relationship (ER) models in the binary-relationship style as described by Barker [???] and others [Rosemary Rock-Evans]  and as implemented, for instance, in Oracle's SQL developer tool.
Our notion of ER model  supports definition of goodness criteria generalising 
the normal form criteria of relational data modelling and also transformation of ER models to relational schemas.

 We are also able to give sufficient conditions under which an ER model transforms into a relational schema in an appropriate normal form.  In doing so we provide a theoretical basis for the elimination of the normalisation step from the relational design workflow 
in favour of an emphasis on goodness criteria for entity relationship models in the binary-relationship style.  This style
is strongly influenced by the
 the entity-relationship model  presented by Chen as a unified model of data; this paper provides substance  to
Chen's idea of a unified model of data but exceeeds his ambition by not requiring a separate normalisation step in relational
design. This represents a significant improvement in software maintenance methodology.

This version of the paper covers 
Elementary Key Normal Form (EKNF), and therefore Third Normal Form (3NF), and Boyce-Codd Normal Form (BCNF).  
\end{abstract}
