
\section{Logical Models}
I am using the term \term{logical} in the sense that it is used in certain ER modelling methodolgies such as ???.
\subsection{Fully Factored Models}

\begin{definition}
In an entity model \genericmodel\ if $a$ is an entity type then say that
a \term{factorisation} of the set $I_a$ is a subset $\set{h_1,...h_n} \subset I_a$
and a  path $f:a \morph b$ and a set of paths sourced at $b$, $\set{g_1,...g_n} \in \bar{I}$\commentary{not defined}
such that for each $i$, $1 \leq i \leq n$, $f \circ g_i \simeq h_i$. 
\end{definition}

The factorisation consists  
of a base set of edges: $h=I \backslash \set{h_1,...h_n}$, a mediating path $f:a \morph b$ and a remainder 
set $j$ paths sourced at $b$, $j \in \bar{I}$, 
such that $I_a \simeq h \cup \setsuchthat{f\circ g}{g \in j}$.


\subsection{Definition of Logical ER Model}
\begin{definition}
A well-formulated ER model is \term{purely-logical}  iff it 
also satisfies:
\begin{enumerate}[(i)]
\item
for all edges $r$ if there is a
a simple path $p$  not including $r$ such that
 $r \simeq p$  then $r$ is identifying and there is an edge $e$ in $p$ which is not identifying,
\item the set of identifying sets I of the model is fully-factored.
\end{enumerate} 
\end{definition}

\noindent
We say that an ER model is a \term{logical ER model} iff it is purely logical.

\subsection{Factorisation}
\begin{definition}
If \genericmodel\ is a model and if there is a factorisation at entity type $a$ with
base set $h$, mediating path $f: a \morph b$ and remainder $j \in \bar{I}$ then define
a factored model $\genericmodel'$ determined by the factorisation to be the model
whose underlying graph is the graph of \genericmodel\ with edges $h_1,...h_n$ removed 
(defining database instances restricted) and
an edge $k$ added, $k:a \morph b$, each database instance $E$ extends to $E'$, $E'_k = E_f$.
Identifying sets of $\genericmodel'$ are exactly as for \genericmodel\ except for at the entity
type $a$ where the identifying set $I_a$ in $\genericmodel'$ is defined to be $h \cup \set{k}$.
Note that by lemma \ref{identifyingsetdeduction} the modified $E'$ is a database instance because
the family 
$\setsuchthat{E'_e}{e \in \set{h \cup \set{k}}}$ is jointly injective because 
$\setsuchthat{E'_e}{e \in \set{h \cup \set{k}}} =\setsuchthat{E_e}{e \in \set{h \cup \set{f}}}$
and in model \genericmodel, $h \cup \set{f}$ is a mono-source because both
$h \cup \set{f} \backslash f \cup \setsuchthat{f \circ g}{g \in j}
=h  \cup \setsuchthat{f \circ g}{g \in j}=I_a$ and $j$ are  mono-sources.


\end{definition}

\begin{definition}
If $a$ is an entity type in an entity model \genericmodel\ then define the abstraction
level of $a$ to be the sum of the lengths of the primary key paths sourced at $a$.
\end{definition}


\begin{definition}
Define  the abstraction level of a model $\genericmodel$ to be
the sum over all entity types $a$ of the abstraction level of $a$.
\end{definition}

\begin{lemma}
If $\genericmodel'$ is a factorisation of model $\genericmodel$ then abstraction level 
of $\genericmodel'$ is greater than abstraction level of $\genericmodel$ and the number of edges
is not increased. 
\end{lemma}
\begin{proof}
\end{proof}

\begin{definition}
Define an entity model \genericmodel\ to be fully-factored iff there no entity types $a$
for which the set $I_a$ has  factorisations.
\end{definition}

\begin{definition}
If \genericmodel\ is a model then say that a $\genericmodel'$ is a fully-factored 
equivalent to \genericmodel\
if it results from successive factorisations of \genericmodel\ until the resulting model can no longer be
factorised. 
\end{definition}

\begin{lemma}
Every model \genericmodel\ has a fully factorised equivalent. 
\end{lemma}
\begin{proof}
We need to show that the factorisation process will terminate. 
to do this note that we can put a (crude) bound on the abstraction level of an entity model
of number of entity types multiplied by the square of the  number of edges
\commentary{Need revise this most likely}.  Because the number of nodes remains constant during
factorisation and because the number of edges does not increase this bound can be calculated
for the model \genericmodel\ and serves as an upper bound for the abstraction level of 
any successive factorisation of \genericmodel. Since factorisation increases abstraction level
and there is an upper bound to abstraction level then the process cannot continue indefinitely. 
\end{proof}

\subsection{Minimality Condition of Factored Model}

\begin{lemma}
\label{identifyingfactorminimal}
In a model \genericmodel, if $i$ is a source with domain $a$,
if $f \in i$, $f: a \morph b$, 
and if $j$ is a mono-source with domain $b$
then  if $(i \backslash \set{f}) \cup \setsuchthat{f \circ g}{g \in j}$ is a minimum mono-source
then $i$ is a minimum mono-source.
\end{lemma}
\begin{proof}
We assume that $(i \backslash \set{f}) \cup \setsuchthat{f \circ g}{g \in j}$ is a mono-source and that no subset of it is a mono-source.
By lemma \ref{identifyingsetdeduction} it follows that $i$ is a mono-source we need to show that
for no edge $e \in i$ is $i \backslash \set{e}$ a mono source. Assume to the contrary that there is such an $e$ then  either $e = f$ or else
$e \in i \backslash \set{f}$. In the first case $(i \backslash \set{f})$ is a subset of  $(i \backslash \set{f}) \cup \setsuchthat{f \circ g}{g \in j}$ which is a mono-source which contradicts the initial assumption. In the second  case 
by lemma \ref{identifyingsetdeduction} since $i \backslash \set{e}$ is a mono-source it follows that
$((i \backslash \set{e})\backslash \set{f}) \cup \setsuchthat{f \circ g}{g \in j}$ is a mono-source which again is to say that a subset
of $(i \backslash \set{f}) \cup \setsuchthat{f \circ g}{g \in j}$ is a mono-source contradicting the initial assumption.
We conclude that no subset of $i$ is a mono-source.
\end{proof}

\begin{lemma}
If a model \genericmodel\ satisfies the minimality condition then it has a fully factored
equivalent $\genericmodel'$  and $\genericmodel'$ satisfies the minimality condition. 
\end{lemma}
\begin{proof}
By lemma \ref{identifyingfactorminimal} we have the minimality of $I_a$ in $\genericmodel'$.

To show that representative sets of primary key paths in $\genericmodel'$ are minimal mono-sources
first note that to each path $p'$ in $\genericmodel'$ there is a corresponding path 
$p$ in \genericmodel\  
 such that for all instances $E$ of \genericmodel\ with corresponding $E'$ of $\genericmodel'$, 
$E_p=E'_{p'}$ and that this correspondence establishes a 1-1 correspondence between the set of primary key paths $P'$ in $\genericmodel'$
and the set of primary key paths $P'$ in \genericmodel. Because $E_p=E'_{p'}$, for all defining instances
$E$, it follows that a representative set for $P'$ corresponds to a representative set for $P'$ and also
that mono-sources in $\genericmodel'$ correspond to mono-sources in \genericmodel. From this it follows that representative sets of primary key paths in $\genericmodel'$ are minimal mono-sources.
\end{proof}
