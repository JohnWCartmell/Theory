
 \documentclass[10pt,a4paper]{article}
%


%ccategories.macros.tex 

% Macros for diagrams in contextual categories and related categories

\usepackage{twoopt}
\usepackage{scalerel} 
\usepackage{xargs}

%\usepackage{mathabx}  %Caused font problems
%\usepackage{MnSymbol}  % caused font problems

\newcommand{\conu}
{\mathbf{C}(U)}

\newcommand{\depu}
{\mathbf{D}(U)}


\newcommand{\reqt}{\textbf{R}}
\newcommand{\reqtc}[1][\catc]{\reqt_{#1}}
\newcommand{\reqtcp}[1][\catcp]{\reqt_{#1}}



\newcommand{\cat}[1]{\textbf{#1}}

\newcommand{\catc}{\cat{C}}
\newcommand{\catcw}{\cat{C}\ }
\newcommand{\catcp}[1][C]{\textbf{#1}'}
\newcommand{\catcpp}[1][C]{\textbf{#1}''}
\newcommand{\obj}[1]{\ensuremath{|\cat{#1}|}}
\newcommand{\ccat}[1][C]{\ensuremath{\mathbb{#1}} }
\newcommand{\ccatc}{contextual category \ccat}
\newcommand{\cobj}[2][]{\ensuremath{|\ccat[#2]|_{#1}}}
\newcommand{\cslice}[2]{\ensuremath{\ccat[#1]_{#2}}}
\newcommand{\csliceobj}[3][]{\ensuremath{|\mathbb{#2}_{#3}|_{#1} }}
\newcommand{\varset}[1][]{\ensuremath{V_{#1} }}
\newcommand{\localvarsets}{\ensuremath{\mathcal{V} }}
\newcommand{\Fam}{\ensuremath{\mathbb{F\mathrm{am}} }}
\newcommand{\Fin}{\ensuremath{\textbf{Fin}} }
\newcommand{\Finp}{\ensuremath{\textbf{Finp}} }
\newcommand{\Po}{\ensuremath{\textbf{Po}} }
\newcommand{\Famslice}[1]{\ensuremath{\mathbb{F\mathrm{am}}_{#1} }}
\newcommand{\Famobj}[1][]{\ensuremath{|\mathbb{F\mathrm{am}}|_{#1} }}
\newcommand{\Famsliceobj}[2][]{\ensuremath{|\mathbb{F\mathrm{am}}_{#2}|_{#1} }}
\newcommand{\morph}{\rightarrow}
\newcommand{\epi}{\twoheadrightarrow}
\newcommand{\base}{\triangleleft}
\newcommand{\comp}{\circ}
\newcommand{\cross}{\otimes}
\newcommand{\pc}[2]{d^{#1}_{#2}}
\newcommand{\sub}{^*}
\newcommand{\diag}{\delta}
\newcommand{\pbase}[1]{\tilde{#1}}
\newcommand{\tuple}[1]{\langle#1\rangle}
\newcommand{\ndidly}{\ensuremath{\Join_n}}

\newcommand{\product}[1]{\bigtimes_{#1}}
\newcommand{\productn}{\product{n}}
\newcommand{\crossx}[3]{#1 \underset{#3}{\cross} #2}
\newcommand{\fibrex}[3]{#1 \underset{#3}{\Join} #2}
\newcommand{\powerset}{\mathcal{P}}
\newcommand{\primeds}[1]{
\ensuremath{\mathcal{P}(#1)} }
\newcommand{\compset}{\ \dot{\circ}\, }

% darrow
%\newcommand{\darrow}{\rightarrowtriangle} %use \smorph instead
\newcommand{\smorph}{\rightarrowtriangle}

 
\newcommand\dhead{\scaleobj{0.6}{\triangleright}}
%\newcommand{\dmorph}{\, \mbox{---} \! \cdot \! \raisebox{1.1pt}{\dhead}}    % dot style
\newcommand{\dmorph}{\, \mbox{---}\kern-1pt\raisebox{1.1pt}{\dhead\kern-1.75pt\dhead}}\,     % double triangle style

% projection tree
%\newcommand{\proj}[2]{proj_{#2}(#1)}

\newcommand{\proj}[2]{
\ensuremath{\mathcal{P}_{#2}(#1)} }

%pstrick supplements for arrows

\newlength{\arrnodesepA}
\newlength{\arrnodesepB}
\newlength{\arroffsetA}
\newlength{\arroffsetB}

%Modified to 2pt from 0pt on 23 July 2018
\newcommand{\arreset}{
\setlength{\arrnodesepA}{2pt}
\setlength{\arrnodesepB}{2pt}
\setlength{\arroffsetA}{0pt}
\setlength{\arroffsetB}{0pt}
}
\arreset

\newcommand{\ncarr}[3][0]{\ncarc[arcangle=#1,nodesepA=\arrnodesepA,nodesepB=\arrnodesepB,offsetA=\arroffsetA,offsetB=\arroffsetB,arrowsize=5pt,arrowinset=0.7]{->}{#2}{#3}}
\newcommand{\ncdarr}[3][0]{\ncarc[linestyle=dashed,arcangle=#1,nodesepA=\arrnodesepA,nodesepB=\arrnodesepB,offsetA=\arroffsetA,offsetB=\arroffsetB,arrowsize=5pt,arrowinset=0.7]{->}{#2}{#3}}
\newcommand{\jcbarr}[4][0]{ % ncbarr is defined in some thridy party package so do not use!\emph{}
\ncarr[#1]{#3}{#4}
\nbput[labelsep=2pt]{\footnotesize $#2$}
}

\newcommand{\ncaarr}[4][0]{
\ncarr[#1]{#3}{#4}
\naput[labelsep=2pt]{\footnotesize $#2$}
}

% \alabel{label}[npos][labelsep_pts]
\newcommandx*\alabel[3][2=0.5,3=2,usedefault]{\naput[labelsep=#3pt,npos=#2]{\footnotesize $#1$}}
% \blabel{label}[npos][labelsep_pts]
\newcommandx*\blabel[3][2=0.5,3=2,usedefault]{\nbput[labelsep=#3pt,npos=#2]{\footnotesize $#1$}}


\newif \ifbars
% to supress display of bars use \barsfalse to swith them on use \barstrue
\barstrue 
% \idcomp mark an arrow as one component of an identifier
\newcommand{\idcomp}{\ifbars{\ncput[npos=0, nrot=:U]{\psline(0.2,-0.075)(0.2,0.075)}}\fi}  %add a bar to a node connection arrow
% pstrick supplements for s-arrows (previous name for d-arrow - should convert}

\newlength{\sarnodesepA}
\newlength{\sarnodesepB}
\newlength{\saroffsetA}
\newlength{\saroffsetB}
\newlength{\sarnodesepAsav}
\newlength{\sarnodesepBsav}

\newcommand{\sarreset}{
\setlength{\sarnodesepA}{0pt}
\setlength{\sarnodesepB}{0pt}
\setlength{\saroffsetA}{0pt}
\setlength{\saroffsetB}{0pt}
}

\sarreset

% sar - S-arrow
\newcommand{\ncsar}[3][0]{
\setlength{\sarnodesepAsav}{\sarnodesepA}
\setlength{\sarnodesepBsav}{\sarnodesepB}
\addtolength{\sarnodesepA}{3pt}
\addtolength{\sarnodesepB}{7pt}
\ncarc[nodesepA=\sarnodesepA,nodesepB=\sarnodesepB,offsetA=\saroffsetA,offsetB=\saroffsetB,arcangle=#1]{-}{#2}{#3}
\ncput[nrot=:R,npos=1]{\pstriangle(0,0)(.2,.2)}
\setlength{\sarnodesepA}{\sarnodesepAsav}
\setlength{\sarnodesepB}{\sarnodesepBsav}
}


% bsar - below labelled S-arrow
\newcommand{\ncbsar}[4][0]{
\ncsar[#1]{#3}{#4}
\nbput[labelsep=2pt]{\footnotesize $#2$}
}
% asar - above labelled S-arrow
\newcommand{\ncasar}[4][0]{
\ncsar[#1]{#3}{#4}
\naput[labelsep=2pt]{\footnotesize $#2$}
}

% OLD cdar - composite dependency arrow - dot tyle
\iffalse
\newcommand{\nccdar}[3][0]{
\setlength{\sarnodesepAsav}{\sarnodesepA}
\setlength{\sarnodesepBsav}{\sarnodesepB}
\addtolength{\sarnodesepA}{3pt}
\addtolength{\sarnodesepB}{11pt}
\ncarc[nodesepA=\sarnodesepA,nodesepB=\sarnodesepB,offsetA=\saroffsetA,offsetB=\saroffsetB,arcangle=#1]{-}{#2}{#3}
\ncput[nrot=:R,npos=1]{\pstriangle(0,0.1)(.2,.2)}
\ncput[nrot=:R,npos=1]{\psdot[dotsize=1pt](-0.0075,0.05)}   %!!
\setlength{\sarnodesepA}{\sarnodesepAsav}
\setlength{\sarnodesepB}{\sarnodesepBsav}
}
\fi

% cdar - composite dependency arrow Mark II - double trangle style
\newcommand{\nccdar}[3][0]{
\setlength{\sarnodesepAsav}{\sarnodesepA}
\setlength{\sarnodesepBsav}{\sarnodesepB}
\addtolength{\sarnodesepA}{3pt}
\addtolength{\sarnodesepB}{13pt}
\ncarc[nodesepA=\sarnodesepA,nodesepB=\sarnodesepB,offsetA=\saroffsetA,offsetB=\saroffsetB,arcangle=#1]{-}{#2}{#3}
\ncput[nrot=:R,npos=1]{\pstriangle(0,0)(.2,.2)}
\ncput[nrot=:R,npos=1]{\pstriangle(0,0.2)(.2,.2)}
\setlength{\sarnodesepA}{\sarnodesepAsav}
\setlength{\sarnodesepB}{\sarnodesepBsav}
}


% bcdar - below labelled composite dependency arrow
\newcommand{\ncbcdar}[4][0]{
\nccdar[#1]{#3}{#4}
\nbput[labelsep=2pt]{\footnotesize $#2$}
}
% acdar - above labelled composite dependency arrow
\newcommand{\ncacdar}[4][0]{
\nccdar[#1]{#3}{#4}
\naput[labelsep=2pt]{\footnotesize $#2$}
}


% rsar - recursive S-arrow
\newcommand{\ncrsar}[2]{
\setlength{\sarnodesepAsav}{\sarnodesepA}
\setlength{\sarnodesepBsav}{\sarnodesepB}
\addtolength{\sarnodesepA}{3pt}
\addtolength{\sarnodesepB}{7pt}
\ncloop[nodesepA=\sarnodesepA,nodesepB=\sarnodesepB,
        offsetA=\saroffsetA,offsetB=\saroffsetB,
        armA=0.7cm,armB=0.6cm,angleA=90,angleB=-90,loopsize=-1,linearc=0.4
				]{-}{#1}{#2}
\ncput[nrot=:R,npos=5]{\pstriangle(0,0)(.2,.2)}
\setlength{\sarnodesepA}{\sarnodesepAsav}
\setlength{\sarnodesepB}{\sarnodesepBsav}
}

% pstrick supplements for multi-arrows

\newlength{\marnodesepA}
\newlength{\marnodesepB}
\newlength{\maroffsetB}
\newlength{\marnodesepBsav}

\newcommand{\marreset}{
\setlength{\marnodesepA}{0pt}
\setlength{\marnodesepB}{0pt}
\setlength{\maroffsetB}{0pt}
}

\marreset

%ncmarr[#1 arcangle1][#2 arcangle2]{#3 name}{#4 domain1}{#5 domain2}{#6 junction}{#7 codomain}
\newcommandtwoopt{\ncmarr}[6][8][8]{%
\ncarc[nodesepA=\marnodesepA,nodesepB=0,arcangle=#1]{-}{#3}{#5}
\ncarc[nodesepB=0,arcangle=-#1]{-}{#4}{#5}
\ncarc[arcangle=#2,nodesepB=\marnodesepB,offsetB=\maroffsetB]{->}{#5}{#6}
}%


\newcommandtwoopt{\nchmarr}[6][8][8]{%
\ncarc[nodesepA=\marnodesepA,nodesepB=0,arcangle=#1]{-}{#3}{#5}
\ncarc[nodesepB=0,arcangle=#1]{-}{#4}{#5}
\ncarc[arcangle=#2,nodesepB=\marnodesepB,offsetB=\maroffsetB]{->}{#5}{#6}
}%

\newcommandtwoopt{\ncamarr}[7][8][8]{%
\ncmarr[#1][#2]{#4}{#5}{#6}{#7}
\naput[npos=.05]{$#3$}
}%
\newcommandtwoopt{\ncbmarr}[7][8][8]{%
\ncmarr[#1][#2]{#4}{#5}{#6}{#7}
\nbput[npos=.05]{$#3$}
}%

\newcommandtwoopt{\ncbhmarr}[7][8][8]{%
\nchmarr[#1][#2]{#4}{#5}{#6}{#7}
\nbput[npos=.05]{$#3$}
}%

\newcommandtwoopt{\ncmarrr}[7][8][8]{
\ncarc[nodesepB=0,arcangle=#1]{-}{#3}{#6}
\ncline[nodesepB=0]{-}{#4}{#6}
\ncarc[nodesepB=0,arcangle=-#1]{-}{#5}{#6}
\ncarc[nodesepA=0,arcangle=#2]{->}{#6}{#7}
}

\newcommandtwoopt{\ncamarrr}[8][8][8]{
\ncmarrr[#1][#2]{#4}{#5}{#6}{#7}{#8}
\naput[npos=.05]{$#3$}
}
\newcommandtwoopt{\ncbmarrr}[8][8][8]{
\ncmarrr[#1][#2]{#4}{#5}{#6}{#7}{#8}
\nbput[npos=.05]{$#3$}
}


% 6 June 2020
% Edges representing attributes and relationship graphs
%  Ep   - partial
%  Epm  - partial mono
%  Epe  - partial epi
%  Epme - partial mono epi
%  Et   - total
%  Etm  - total mono
%  Ete  - total epi
%  Etme - total mono epi
%  recursive edges (use nccircle)
%  rEp   - partial
%  rEpm  - partial mono
%  rEpe  - partial epi
%  rEpme - partial mono epi
%  rEt   - total
%  rEtm  - total mono
%  rEte  - total epi
%  rEtme - total mono epi

\newcounter{EangleA}
\newcounter{EangleB}
\newcounter{EmidangleA}
\newcounter{EmidangleB}

% Ep - Edge partial
\newcommandtwoopt{\Ep}[4][0][0]{
\crowsfootedEdge{#1}{#2}{#3}{#4}{dashed}{dashed}
}



% Epm - Edge partial mono
\newcommandtwoopt{\Epm}[4][0][0]{
\monoEdge{#1}{#2}{#3}{#4}{dashed}{dashed}
}


% Epe - Edge partial epi
\newcommandtwoopt{\Epe}[4][0][0]{
\crowsfootedEdge{#1}{#2}{#3}{#4}{dashed}{solid}
}

% Epme - Edge partial mono epi
\newcommandtwoopt{\Epme}[4][0][0]{
\monoEdge{#1}{#2}{#3}{#4}{dashed}{solid}
}

% Et - Edge total
\newcommandtwoopt{\Et}[4][0][0]{
\crowsfootedEdge{#1}{#2}{#3}{#4}{solid}{dashed}
}

% Etm - Edge total mono
\newcommandtwoopt{\Etm}[4][0][0]{
\monoEdge{#1}{#2}{#3}{#4}{solid}{dashed}
}

% Ete - Edge total epi
\newcommandtwoopt{\Ete}[4][0][0]{
\crowsfootedEdge{#1}{#2}{#3}{#4}{solid}{solid}
}

% Etme - Edge total mono epi
\newcommandtwoopt{\Etme}[4][0][0]{
\monoEdge{#1}{#2}{#3}{#4}{solid}{solid}
}

% crowsfootedEdge - \crowsfootedEdge[angleA][midpointangle]{startnode}{endnode}[startstyle][endstyle]
\newcommand{\crowsfootedEdge}[6]{
\setlength{\sarnodesepAsav}{\sarnodesepA}
\setlength{\sarnodesepBsav}{\sarnodesepB}
\addtolength{\sarnodesepA}{3pt}
\addtolength{\sarnodesepB}{3pt}
\setcounter{EangleA}{ #1 + #2}
\setcounter{EangleB}{180  - #1 + #2}
\setcounter{EmidangleA}{#2}
\setcounter{EmidangleB}{#2 + 180}
\nccurve[nodesepA=\sarnodesepA,nodesepB=\sarnodesepB,offsetA=\saroffsetA,offsetB=\saroffsetB,angleA=\theEangleA, angleB=\theEangleB,linestyle=none,linewidth=0]{->}{#3}{#4}
\ncput[nrot=:R,npos=0]{\psline(0,.1)(.075,0)}
\ncput[nrot=:R,npos=0]{\psline(0,.1)(-0.075,0)}
\ncput{\pnode(0,0){xxx}}
\nccurve[nodesepA=0,nodesepB=\sarnodesepB,offsetA=0,offsetB=\saroffsetB,angleA=\theEmidangleA, angleB=\theEangleB, linestyle=#6]{->}{xxx}{#4}
%the following provides context for any following label
\nccurve[nodesepA=\sarnodesepA,nodesepB=0,offsetA=\saroffsetA,offsetB=0,angleA=\theEangleA, angleB=\theEmidangleB,linestyle=#5]{-}{#3}{xxx}
\setlength{\sarnodesepA}{\sarnodesepAsav}
\setlength{\sarnodesepB}{\sarnodesepBsav}
}

% monoEdge - \monoEdge[angleA][midpointangle]{startnode}{endnode}[startstyle][endstyle]
\newcommand{\monoEdge}[6]{ 
\setlength{\sarnodesepAsav}{\sarnodesepA}
\setlength{\sarnodesepBsav}{\sarnodesepB}
\addtolength{\sarnodesepA}{3pt}
\addtolength{\sarnodesepB}{3pt}
\setcounter{EangleA}{ #1 + #2}
\setcounter{EangleB}{180  - #1 + #2}
\setcounter{EmidangleA}{#2}
\setcounter{EmidangleB}{#2 + 180}
\nccurve[nodesepA=\sarnodesepA,nodesepB=\sarnodesepB,offsetA=\saroffsetA,offsetB=\saroffsetB,angleA=\theEangleA, angleB=\theEangleB,linestyle=none,linewidth=0]{->}{#3}{#4}
\ncput{\pnode(0,0){xxx}}
\nccurve[nodesepA=0,nodesepB=\sarnodesepB,offsetA=0,offsetB=\saroffsetB,angleA=\theEmidangleA, angleB=\theEangleB, linestyle=#6]{->}{xxx}{#4}
%the following provides context for any following label
\nccurve[nodesepA=\sarnodesepA,nodesepB=0,offsetA=\saroffsetA,offsetB=0,angleA=\theEangleA, angleB=\theEmidangleB,linestyle=#5]{-}{#3}{xxx}
\setlength{\sarnodesepA}{\sarnodesepAsav}
\setlength{\sarnodesepB}{\sarnodesepBsav}
}


\newcounter{EangleGiven}
\newcounter{EangleComplementary}
\newcounter{EangleStartCorrected}
\newcounter{EangleEndCorrected}


%  rEp   - recursive Edge partial
\newcommand{\rEp}[2][0]{
\setcounter{EangleGiven}{#1}
\setcounter{EangleStartCorrected}{#1-10} %correction required because for nccurve unlike nccircle angle measured at boundary not at centre of node
\setcounter{EangleEndCorrected}{#1+180+10} %correction required because angle measured at boundary not at centre of node
\setcounter{EangleComplementary}{#1 + 180}
\nccircle[angleA=\theEangleComplementary, nodesep=0pt, linestyle=none]{-}{#2}{.4cm} % an invisible circle to hang the midpoint from
\ncput{\pnode(0,0){midpoint}}                                         
\nccurve[nodesepA=1pt,nodesepB=0pt,offsetA=0pt,offsetB=0pt,angleA=\theEangleStartCorrected, angleB=\theEangleGiven, ncurv=1.359, linecolor=black, linestyle=dashed]{-}{#2}{midpoint}
\ncput[nrot=:R,npos=0]{\psline(0,.1)(.075,0)}
\ncput[nrot=:R,npos=0]{\psline(0,.1)(-0.075,0)}
\nccurve[nodesepA=0pt,nodesepB=2pt,offsetA=0pt,offsetB=0pt,angleA=\theEangleComplementary, angleB=\theEangleEndCorrected, ncurv=1.359, linestyle=dashed]{-}{midpoint}{#2}
% 1.359 is e/2 happenchance or algorithmically necessary???
% now draw arrowhead -- dont include in the nccurve because this alters the line position - a strange feature of pstruicks
\ncput[npos=0.9]{\pnode(0,0){yyy}}
\ncline{->}{yyy}{#2}
% repeat from earlier to provide context for label that might follow
\nccurve[nodesepA=1pt,nodesepB=0pt,offsetA=0pt,offsetB=0pt,angleA=\theEangleStartCorrected, angleB=\theEangleGiven, ncurv=1.359, linecolor=black, linestyle=dashed]{-}{#2}{midpoint} 
} 

%  rEpm  - recursive Edge partial mono
\newcommand{\rEpm}[2][0]{
\setcounter{EangleGiven}{#1}
\setcounter{EangleStartCorrected}{#1-10} %correction required because for nccurve unlike nccircle angle measured at boundary not at centre of node
\setcounter{EangleEndCorrected}{#1+180+10} %correction required because angle measured at boundary not at centre of node
\setcounter{EangleComplementary}{#1 + 180}
\nccircle[angleA=\theEangleComplementary, nodesep=0pt, linestyle=none]{-}{#2}{.4cm} % an invisible circle to hang the midpoint from
\ncput{\pnode(0,0){midpoint}}   
\nccurve[nodesepA=0pt,nodesepB=2pt,offsetA=0pt,offsetB=0pt,angleA=\theEangleComplementary, angleB=\theEangleEndCorrected, ncurv=1.359, linestyle=dashed]{-}{midpoint}{#2}
% 1.359 is e/2 happenchance or algorithmically necessary???
% now draw arrowhead -- dont include in the nccurve because this alters the line position - a strange feature of pstruicks
\ncput[npos=0.9]{\pnode(0,0){yyy}}
\ncline{->}{yyy}{#2}
% last to provide context for label that might follow
\nccurve[nodesepA=1pt,nodesepB=0pt,offsetA=0pt,offsetB=0pt,angleA=\theEangleStartCorrected, angleB=\theEangleGiven, ncurv=1.359, linecolor=black, linestyle=dashed]{-}{#2}{midpoint} 
}

%  rEpe  - recursive Edge partial epi
\newcommand{\rEpe}[2][0]{
\setcounter{EangleGiven}{#1}
\setcounter{EangleStartCorrected}{#1-10} %correction required because for nccurve unlike nccircle angle measured at boundary not at centre of node
\setcounter{EangleEndCorrected}{#1+180+10} %correction required because angle measured at boundary not at centre of node
\setcounter{EangleComplementary}{#1 + 180}
\nccircle[angleA=\theEangleComplementary, nodesep=0pt, linestyle=none]{-}{#2}{.4cm} % an invisible circle to hang the midpoint from
\ncput{\pnode(0,0){midpoint}}                                         
\nccurve[nodesepA=1pt,nodesepB=0pt,offsetA=0pt,offsetB=0pt,angleA=\theEangleStartCorrected, angleB=\theEangleGiven, ncurv=1.359, linecolor=black, linestyle=dashed]{-}{#2}{midpoint}
\ncput[nrot=:R,npos=0]{\psline(0,.1)(.075,0)}
\ncput[nrot=:R,npos=0]{\psline(0,.1)(-0.075,0)}
\nccurve[nodesepA=0pt,nodesepB=2pt,offsetA=0pt,offsetB=0pt,angleA=\theEangleComplementary, angleB=\theEangleEndCorrected, ncurv=1.359]{-}{midpoint}{#2}
% 1.359 is e/2 happenchance or algorithmically necessary???
% now draw arrowhead -- dont include in the nccurve because this alters the line position - a strange feature of pstruicks
\ncput[npos=0.9]{\pnode(0,0){yyy}}
\ncline{->}{yyy}{#2}
% repeat from earlier to provide context for label that might follow
\nccurve[nodesepA=1pt,nodesepB=0pt,offsetA=0pt,offsetB=0pt,angleA=\theEangleStartCorrected, angleB=\theEangleGiven, ncurv=1.359, linecolor=black, linestyle=dashed]{-}{#2}{midpoint} 
}

%  rEpme - recursive Edge partial mono epi
\newcommand{\rEpme}[2][0]{
\setcounter{EangleGiven}{#1}
\setcounter{EangleStartCorrected}{#1-10} %correction required because for nccurve unlike nccircle angle measured at boundary not at centre of node
\setcounter{EangleEndCorrected}{#1+180+10} %correction required because angle measured at boundary not at centre of node
\setcounter{EangleComplementary}{#1 + 180}
\nccircle[angleA=\theEangleComplementary, nodesep=0pt, linestyle=none]{-}{#2}{.4cm} % an invisible circle to hang the midpoint from
\ncput{\pnode(0,0){midpoint}}                                         
%\nccurve[nodesepA=0pt,nodesepB=0pt,offsetA=0pt,offsetB=0pt,angleA=\theEangleComplementary, angleB=\theEangleEndCorrected, ncurv=1.359, linestyle=dashed]{->}{xxx}{#2}
\nccurve[nodesepA=0pt,nodesepB=2pt,offsetA=0pt,offsetB=0pt,angleA=\theEangleComplementary, angleB=\theEangleEndCorrected, ncurv=1.359]{-}{midpoint}{#2}
% 1.359 is e/2 happenchance or algorithmically necessary???
% now draw arrowhead -- dont include in the nccurve because this alters the line position - a strange feature of pstruicks
\ncput[npos=0.9]{\pnode(0,0){yyy}}
\ncline{->}{yyy}{#2}
% last so that to provide context for label that might follow
\nccurve[nodesepA=1pt,nodesepB=0pt,offsetA=0pt,offsetB=0pt,angleA=\theEangleStartCorrected, angleB=\theEangleGiven, ncurv=1.359, linecolor=black, linestyle=dashed]{-}{#2}{midpoint} 
}

% rEt - recursive Edge total
\newcommand{\rEt}[2][0]{
\setcounter{EangleGiven}{#1}
\setcounter{EangleStartCorrected}{#1-10} %correction required because for nccurve unlike nccircle angle measured at boundary not at centre of node
\setcounter{EangleEndCorrected}{#1+180+10} %correction required because angle measured at boundary not at centre of node
\setcounter{EangleComplementary}{#1 + 180}
\nccircle[angleA=\theEangleComplementary, nodesep=0pt, linestyle=none]{-}{#2}{.4cm} % an invisible circle to hang the midpoint from
\ncput{\pnode(0,0){midpoint}}                                         
\nccurve[nodesepA=1pt,nodesepB=0pt,offsetA=0pt,offsetB=0pt,angleA=\theEangleStartCorrected, angleB=\theEangleGiven, ncurv=1.359, linecolor=black]{-}{#2}{midpoint}
\ncput[nrot=:R,npos=0]{\psline(0,.1)(.075,0)}
\ncput[nrot=:R,npos=0]{\psline(0,.1)(-0.075,0)}
%\nccurve[nodesepA=0pt,nodesepB=0pt,offsetA=0pt,offsetB=0pt,angleA=\theEangleComplementary, angleB=\theEangleEndCorrected, ncurv=1.359, linestyle=dashed]{->}{xxx}{#2}
\nccurve[nodesepA=0pt,nodesepB=2pt,offsetA=0pt,offsetB=0pt,angleA=\theEangleComplementary, angleB=\theEangleEndCorrected, ncurv=1.359, linestyle=dashed]{-}{midpoint}{#2}
% 1.359 is e/2 happenchance or algorithmically necessary???
% now draw arrowhead -- dont include in the nccurve because this alters the line position - a strange feature of pstruicks
\ncput[npos=0.9]{\pnode(0,0){yyy}}
\ncline{->}{yyy}{#2}
% repeat from earlier to provide context for label that might follow
\nccurve[nodesepA=1pt,nodesepB=0pt,offsetA=0pt,offsetB=0pt,angleA=\theEangleStartCorrected, angleB=\theEangleGiven, ncurv=1.359, linecolor=black]{-}{#2}{midpoint} 
}

%  rEtm  - recursive Edge total mono
\newcommand{\rEtm}[2][0]{
\setcounter{EangleGiven}{#1}
\setcounter{EangleStartCorrected}{#1-10} %correction required because for nccurve unlike nccircle angle measured at boundary not at centre of node
\setcounter{EangleEndCorrected}{#1+180+10} %correction required because angle measured at boundary not at centre of node
\setcounter{EangleComplementary}{#1 + 180}
\nccircle[angleA=\theEangleComplementary, nodesep=0pt, linestyle=none]{-}{#2}{.4cm} % an invisible circle to hang the midpoint from
\ncput{\pnode(0,0){midpoint}}     
\nccurve[nodesepA=0pt,nodesepB=2pt,offsetA=0pt,offsetB=0pt,angleA=\theEangleComplementary, angleB=\theEangleEndCorrected, ncurv=1.359, linestyle=dashed]{-}{midpoint}{#2}
% 1.359 is e/2 happenchance or algorithmically necessary???
% now draw arrowhead -- dont include in the nccurve because this alters the line position - a strange feature of pstruicks
\ncput[npos=0.9]{\pnode(0,0){yyy}}
\ncline{->}{yyy}{#2}
% last to provide context for label that might follow
\nccurve[nodesepA=1pt,nodesepB=0pt,offsetA=0pt,offsetB=0pt,angleA=\theEangleStartCorrected, angleB=\theEangleGiven, ncurv=1.359, linecolor=black]{-}{#2}{midpoint} 
}

%  rEte  - total epi
\newcommand{\rEte}[2][0]{
\setcounter{EangleGiven}{#1}
\setcounter{EangleStartCorrected}{#1-10} %correction required because for nccurve unlike nccircle angle measured at boundary not at centre of node
\setcounter{EangleEndCorrected}{#1+180+10} %correction required because angle measured at boundary not at centre of node
\setcounter{EangleComplementary}{#1 + 180}
\nccircle[angleA=\theEangleComplementary, nodesep=0pt, linestyle=none]{-}{#2}{.4cm} % an invisible circle to hang the midpoint from
\ncput{\pnode(0,0){midpoint}}                                         
\nccurve[nodesepA=1pt,nodesepB=0pt,offsetA=0pt,offsetB=0pt,angleA=\theEangleStartCorrected, angleB=\theEangleGiven, ncurv=1.359, linecolor=black]{-}{#2}{midpoint}
\ncput[nrot=:R,npos=0]{\psline(0,.1)(.075,0)}
\ncput[nrot=:R,npos=0]{\psline(0,.1)(-0.075,0)}
%\nccurve[nodesepA=0pt,nodesepB=0pt,offsetA=0pt,offsetB=0pt,angleA=\theEangleComplementary, angleB=\theEangleEndCorrected, ncurv=1.359, linestyle=dashed]{->}{xxx}{#2}
\nccurve[nodesepA=0pt,nodesepB=2pt,offsetA=0pt,offsetB=0pt,angleA=\theEangleComplementary, angleB=\theEangleEndCorrected, ncurv=1.359]{-}{midpoint}{#2}
% 1.359 is e/2 happenchance or algorithmically necessary???
% now draw arrowhead -- dont include in the nccurve because this alters the line position - a strange feature of pstruicks
\ncput[npos=0.9]{\pnode(0,0){yyy}}
\ncline{->}{yyy}{#2}
% repeat from earlier to provide context for label that might follow
\nccurve[nodesepA=1pt,nodesepB=0pt,offsetA=0pt,offsetB=0pt,angleA=\theEangleStartCorrected, angleB=\theEangleGiven, ncurv=1.359, linecolor=black]{-}{#2}{midpoint} 
}

%  rEtme - recursive Edge total mono epi

\newcommand{\rEtme}[2][0]{
\setcounter{EangleGiven}{#1}
\setcounter{EangleStartCorrected}{#1-10} %correction required because for nccurve unlike nccircle angle measured at boundary not at centre of node
\setcounter{EangleEndCorrected}{#1+180+10} %correction required because angle measured at boundary not at centre of node
\setcounter{EangleComplementary}{#1 + 180}
\nccircle[angleA=\theEangleComplementary, nodesep=0pt, linestyle=none]{-}{#2}{.4cm} % an invisible circle to hang the midpoint from
\ncput{\pnode(0,0){midpoint}}     
\nccurve[nodesepA=0pt,nodesepB=2pt,offsetA=0pt,offsetB=0pt,angleA=\theEangleComplementary, angleB=\theEangleEndCorrected, ncurv=1.359]{-}{midpoint}{#2}
% 1.359 is e/2 happenchance or algorithmically necessary???
% now draw arrowhead -- dont include in the nccurve because this alters the line position - a strange feature of pstruicks
\ncput[npos=0.9]{\pnode(0,0){yyy}}
\ncline{->}{yyy}{#2}
% last to provide context for label that might follow
\nccurve[nodesepA=1pt,nodesepB=0pt,offsetA=0pt,offsetB=0pt,angleA=\theEangleStartCorrected, angleB=\theEangleGiven, ncurv=1.359, linecolor=black]{-}{#2}{midpoint} 
}

%The following are stylistic so belong in main document not here.
%\usepackage[margin=4.0cm]{geometry} % This shouldn't be here commented out 17 July 2018
%\usepackage{mathptmx}               % This changes font to roman so doesn't belong here
%
\usepackage{amsfonts}
\usepackage{amssymb} % added 08\02\2019 as an experiment. Needed in some instances for \blacksquare
                     % not needed is class is `beamer' but I don't know why not
\usepackage{array}
\usepackage{pstricks}
\usepackage{pst-tree}
\usepackage{pst-plot}
\usepackage{pst-node}
\usepackage{stmaryrd}
\usepackage{amsmath}
\usepackage{verbatim}
\usepackage{graphicx}  
\usepackage{calc}
\usepackage{xifthen}
%\usepackage{xcolor} investigate with beamer
\usepackage{color}
\usepackage{stringstrings}
%\usepackage[small,bf,margin=3pt,format=hang, labelsep=endash,singlelinecheck=false]{caption} %prevuiously justification=justified
%\usepackage{enumerate}
%\usepackage{enumitem}
\usepackage{enumerate}
%\usepackage[shortlabels]{enumitem} %Removed this 28/01/2019 because interfereing with a beamer presentation. 
\usepackage{float}
\usepackage[section]{placeins}
%\setlength{\captionmargin}{5pt}
\usepackage{environ}
\usepackage{multirow}
\usepackage{rotating}
\usepackage{longtable}
\usepackage{afterpage}
\usepackage{needspace}


%DEFINE ENVIRONMENT BLOCK
% Riddle
\newsavebox{\riddlebox}

\newenvironment{erexample}
{\newcommand\colboxcolor{F0F0F0}%was F8F8F8
\begin{lrbox}{\riddlebox}
\begin{minipage}{\dimexpr\columnwidth-2\fboxsep\relax} \textbf{} \\ \itshape}
{\end{minipage}\end{lrbox}%
%\begin{center}
\colorbox[HTML]{\colboxcolor}{\usebox{\riddlebox}}
%\end{center}
}

\newenvironment{erbox}
{\newcommand\colboxcolor{F0F0F0}%was F8F8F8
\begin{lrbox}{\riddlebox}%
\begin{minipage}{\dimexpr\columnwidth-2\fboxsep\relax} }
{\end{minipage}\end{lrbox}%
%\begin{center}
\colorbox[HTML]{\colboxcolor}{\usebox{\riddlebox}}
%\end{center}
}

%\begin{erboxedFigure}{#1 FigureParam}{#2 Label}{#3 Caption}
\NewEnviron{erboxedFigure}[3]{%
\begin{figure}[#1]
\begin{erexample}
\begin{center}
\BODY
\end{center}
\vspace{-0.5cm}
\caption{#3}
\label{#2}
\end{erexample}
\end{figure}
}

\newcommand{\erpictureFolder}[0]{../SharedPictures}

\newcommand{\ercenterPicture}[1]{
\begin{center}
\input{\erpictureFolder/#1}
\end{center}
}


\newlength{\erhalfHt}

%\erinlinePicture{#1 pictureFilename}{#2 pictureHeight}
\newcommand{\erinlinePicture}[2]{
\setlength{\erhalfHt}{#2cm * \real{0.5}}
\raisebox{-\erhalfHt}[\erhalfHt + 0.5cm][\erhalfHt + 0.5cm]{
\input{\erpictureFolder/#1}
} 
}

%\erplainFig{#1 pictureFilename}{#2 figureParam}{#3Caption}
\newcommand{\erplainFig}[3]{
\begin{figure}[#2]
\begin{center}
\input{\erpictureFolder/#1}
\end{center}
\caption{#3}
\label{#1}
\end{figure}
}

%\erboxedFigPicture{#1 pictureFilename}{#2 figureParam}{#3Caption}
\newcommand{\erboxedFigPicture}[3]{
\begin{figure}[#2]
\begin{erexample}
\vspace{-0.5cm}
\begin{center}
\input{\erpictureFolder/#1}
\end{center}
\caption{#3}
\label{#1}
\end{erexample}
\end{figure}
}

%\erLeftSideFig{#1 pictureFilename}{#2 figureParam}{#3Caption}
\newcommand{\erLeftSideFig}[3]{
\begin{figure}[#2]
\begin{erexample}
  \begin{minipage}[c]{0.4\textwidth}
    \caption{#3}
    \label{#1}
  \end{minipage}
  \begin{minipage}[c]{0.5\textwidth}
    \input{\erpictureFolder/#1}
  \end{minipage}
\end{erexample}
\end{figure}
}

%\erbulletedFig{#1 pictureFilename}{#2 figureParam}{#3Caption}
\NewEnviron{erbulletedFig}[3]{%
\begin{figure}[#2]
\begin{erexample}
\vspace{-0.5cm}
\begin{center}
$
\begin{array}{c m{0.25cm} | m{6cm}}
\raisebox{-2.0cm}{
\input{\erpictureFolder/#1}}& & \text{\parbox{6cm}{\raggedright{\footnotesize{
\begin{enumerate}[(i)]
\BODY
\end{enumerate}}}}} \\
\end{array}
$
\end{center}
\caption{#3}
\label{#1}
\end{erexample}
\end{figure} 
}


%\begin{erbulletedDimFig}{#1 pictureFilename}{#2figureParam} {#3Caption} {#4PictureHeight}{#5TextWidth}

\NewEnviron{erbulletedDimFig}[5]{%
\begin{figure}[#2]
\begin{erexample}
\vspace{-0.5cm}
\begin{center}
$
\begin{array}{c m{0.25cm} |  m{#5cm}}
\setlength{\erhalfHt}{#4cm * \real{0.5}}
\raisebox{-\erhalfHt}{
\input{\erpictureFolder/#1}}& & \text{\parbox{#5cm}{\raggedright{\footnotesize{
\begin{enumerate}[(i)]
\BODY
\end{enumerate}}}}} \\
\end{array}
$
\end{center}
\caption{#3}
\label{#1}
\end{erexample}
\end{figure} 
}

%\begin{ernotedModel}{#1 pictureFilename}{#2PictureHeight}{#3PictureWidth}{#4TextWidth}

\NewEnviron{ernotedModel}[4]{%
\begin{center}
$
\begin{array}{m{#3cm} m{1cm} | c m{#4cm}}
\setlength{\erhalfHt}{#2cm * \real{0.5}}
\raisebox{-\erhalfHt}{
\input{\erpictureFolder/#1}}& & & \text{\parbox{#4cm}{\raggedright{\footnotesize{
\BODY
}}}} \\
\end{array}
$
\end{center} 
}

%\begin{ermodelText}{#1 pictureFilename}{#2PictureHeight}{#3PictureWidth}{#4TextWidth}

\NewEnviron{ermodelText}[4]{%
\begin{center}
\begin{tabular}{m{#3cm} m{1cm}  c m{#4cm}}
\setlength{\erhalfHt}{#2cm * \real{0.5}}
\raisebox{-\erhalfHt}{
\input{\erpictureFolder/#1}}& & & \text{\parbox{#4cm}{\raggedright{\small{
\BODY
}}}} \\
\end{tabular}
\end{center} 
}


%\erbulletedModel{#1 pictureFilename}{#2PictureHeight}{#3PictureWidth}{#4TextWidth}

\NewEnviron{erbulletedModel}[4]{%
\begin{center}
$
\begin{array}{m{#3cm} m{1cm} | c m{#4cm}}
\setlength{\erhalfHt}{2cm * \real{0.5}}
\raisebox{-\erhalfHt}{
\input{\erpictureFolder/#1}}& & & \text{\parbox{#4cm}{\raggedright{\footnotesize{
\begin{enumerate}[(i)]
\BODY
\end{enumerate}}}}} \\
\end{array}
$
\end{center} 
}



%\ernotedDimFig{#1 pictureFilename}{#2 figureParam}{#3Caption}{#4PictureHeight}{#5TextWidth}
\NewEnviron{ernotedDimFig}[5]{%
\begin{figure}[#2]
\begin{erexample}
\vspace{-0.5cm}
\begin{center}
$
\begin{array}{c m{0.25cm} | c m{#5cm}}
\setlength{\erhalfHt}{#4cm * \real{0.5}}
\raisebox{-\erhalfHt}{
\input{\erpictureFolder/#1}}& & & \text{\parbox{#5cm}{\raggedright{\footnotesize{
\BODY }}}}\\
\end{array}
$
\end{center}
\caption{#3}
\label{#1}
\end{erexample}
\end{figure} 
}
%\begin{ernotedDimFigPW}{#1 pictureFilename}{#2 figureParam}{#3Caption}{#4PictureHeight}{#5PictureWidth}{#6TextWidth}
\NewEnviron{ernotedDimFigPW}[6]{%
\begin{figure}[#2]
\begin{erexample}
\vspace{-0.5cm}
\begin{center}
$
\begin{array}{>{\centering}m{#5cm} m{0.5cm} | c m{#6cm}}
\setlength{\erhalfHt}{#4cm * \real{0.5}}
\raisebox{-\erhalfHt}{
\centering \input{\erpictureFolder/#1}
}& & & \text{\parbox{#6cm - 0.5cm}{\raggedright{\footnotesize{
\BODY }}}}\\
\end{array}
$ \\
\vspace {0.2cm}
\end{center}
\caption{#3}
\label{#1}
\end{erexample}
\end{figure}
}



\newenvironment{erquote}
{\begin{quote}\itshape}
{\end{quote}}


%
%  erdiagram.tex
%  *************
%  Macros to represent ER diagrams
%  *******************************
% 29/01/2019 Modify so that not reliant on the
%            default fontsize being 10pt by using
%            package anyfontsize and then
%            \fontsize{8}{10}\selectfont to set font to 8pt
% 06/02/2019 Pullback symbol implemented and minor tweaks to positioning 
%            and size of identifier symbol and relationship labels.
%            Accidental forked changes merged on 08/02/2019.
% 15/03/2019 Continuation of 29/01/2019. Need fix fontsize of 
%            ERrelname and ERscope.	 
% ***********************************************************
 \usepackage{anyfontsize}             % 29/01/2019 
  
%\begin{erdiagram}{#1 height}{#2 width} 
% ....
% ....
%\end{erdiagram}
\newenvironment{erdiagram}[2]
{%\pspicture*(-#1,0)(#2,0)
\pspicture*(0,-#1)(#2,0)
%\psgrid
}
{\endpspicture}

\definecolor{lightyellow}{cmyk}{0,0,0.3,0}
\definecolor{verylightgrey}{gray}{0.95}


% \eret{#1 x0} {#2 y0} {#3 x1} {#4 y1} {#5 corner radius} {#6 fill}
\newcommand {\eret}[6]
{ 
\ifthenelse{\equal{#6}{1}}
{\psframe[framearc=#5,fillstyle=solid,fillcolor=lightyellow](#1,#2)(#3,#4)}
{\psframe[framearc=#5,fillstyle=solid,fillcolor=white](#1,#2)(#3,#4)}
}

% et top 
\newcommand {\erettop}[4]
{
%\psframe[linestyle=none,linearc=2pt,cornersize=absolute,fillstyle=solid,fillcolor=lightyellow](#1,#2)(#3,#4)
\psline[linearc=2pt,fillstyle=none,fillcolor=lightyellow](#1,#4)(#1,#2)(#3,#2)(#3,#4)
}

% et bottom 
\newcommand {\eretbtm}[4]
{
%\psframe[linestyle=none,linearc=2pt,cornersize=absolute,fillstyle=solid,fillcolor=lightyellow](#1,#2)(#3,#4)
\psline[linearc=2pt,fillstyle=none,fillcolor=lightyellow](#1,#2)(#1,#4)(#3,#4)(#3,#2)
}

% et bottom left
\newcommand {\eretbl}[4]
{
%\psframe[linestyle=none,linearc=2pt,cornersize=absolute,fillstyle=solid,fillcolor=lightyellow](#1,#2)(#3,#4)
\psline[linearc=2pt,fillstyle=none,fillcolor=lightyellow](#1,#4)(#3,#4)(#3,#2)
}

% et middle left
\newcommand {\eretml}[4]
{
%\psframe[linestyle=none,linearc=2pt,cornersize=absolute,fillstyle=solid,fillcolor=lightyellow](#1,#2)(#3,#4)
\psline[linearc=2pt,fillstyle=none,fillcolor=lightyellow](#1,#2)(#3,#2)(#3,#4)(#1,#4)
}

% et top left
\newcommand {\erettl}[4]
{
%\psframe[linestyle=none,linearc=2pt,cornersize=absolute,fillstyle=solid,fillcolor=lightyellow](#1,#2)(#3,#4)
\psline[linearc=2pt,fillstyle=none,fillcolor=lightyellow](#1,#2)(#3,#2)(#3,#4)
}

% et bottom right
\newcommand {\eretbr}[4]
{
%\psframe[linestyle=none,linearc=2pt,cornersize=absolute,fillstyle=solid,fillcolor=lightyellow](#1,#2)(#3,#4)
\psline[linearc=2pt,fillstyle=none,fillcolor=lightyellow](#1,#2)(#1,#4)(#3,#4)
}

% et middle right
\newcommand {\eretmr}[4]
{
%\psframe[linestyle=none,linearc=2pt,cornersize=absolute,fillstyle=solid,fillcolor=lightyellow](#1,#2)(#3,#4)
\psline[linearc=2pt,fillstyle=none,fillcolor=lightyellow](#3,#4)(#1,#4)(#1,#2)(#3,#2)
}

% et top right
\newcommand {\erettr}[4]
{
\psline[linearc=2pt,fillstyle=none,fillcolor=lightyellow](#1,#4)(#1,#2)(#3,#2)
}

% \ergrp{#1 x0} {#2 y0} {#3 x1} {#4 y1} {#5 corner radius} {#6 fill}
% #5 corner radius is unused!
\newcommand {\ergrp}[6]
{ 
\ifthenelse{\equal{#6}{1}}
{\psframe[fillstyle=solid,fillcolor=verylightgrey](#1,#2)(#3,#4)}
{\psframe[fillstyle=solid,fillcolor=white](#1,#2)(#3,#4)}
}


% \ertext{#1 text}
% 15/03/2019
\newcommand {\erextrasmallitalictext}[1]
{\fontsize{7}{9}\selectfont \textit{#1}}

% 29/01/2019  
\newcommand {\ersmallitalictext}[1]
{\fontsize{8}{10}\selectfont \textit{#1}}

\newcommand {\ermediumitalictext}[1]
{\fontsize{10}{12}\selectfont \textit{#1}}

% \eretname {#1 x left of text} {#2 y top of text} {#3 text}
\newcommand {\olderetname}[3]
{
%shift down 0.1 for height of text the anchor at baseline (B)
\rput[bl]{0}(0,-0.1){\rput[Bl]{0}(#1,#2){\ersmallitalictext{#3}}}
}

% \errelarm {#1 x0} {#2 y0} {#3 x1} {#4 y1} {#5 ismandatory} {#6 isconstructed}
\newcommand {\errelarm}[6]
{
\ifthenelse{\equal{#6}{1}}
{
%%\psline[linewidth=0.5pt,linearc=.05,linestyle=dashed,dash=6pt 6pt]{-}(#1,#2)(#3,#4)}
\ifthenelse{\equal{#5}{1}}
{\psline[linewidth=1.5pt,linearc=.05,linecolor=lightgray]{-}(#1,#2)(#3,#4)}
{\psline[linewidth=1.5pt,linearc=.05,linecolor=lightgray,linestyle=dashed,dash=2pt 2pt]{-}(#1,#2)(#3,#4)}
}
{
\ifthenelse{\equal{#5}{1}}
{\psline[linewidth=0.9pt,linearc=.05]{-}(#1,#2)(#3,#4)}
{\psline[linewidth=0.9pt,linearc=.05,linestyle=dashed,dash=2pt 2pt]{-}(#1,#2)(#3,#4)}
}
}

% \errelangle {#1 x0} {#2 y0} {#3 x1} {#4 y1} {#5 x2} {#6 y2} {#7 ismandatory} {#8 isocnstructed}
\newcommand {\errelangle}[8]
{
\ifthenelse{\equal{#8}{1}}
{
%\psline[linewidth=0.5pt,linearc=.1,linestyle=dashed,dash=6pt 6pt]{-}(#1,#2)(#3,#4)(#5,#6)}
\ifthenelse{\equal{#7}{1}}
{\psline[linewidth=1.5pt,linearc=.05,linecolor=lightgray]{-}(#1,#2)(#3,#4)(#5,#6)}
{\psline[linewidth=1.5pt,linearc=.1,linecolor=lightgray,linestyle=dashed,dash=2pt 2pt]{-}(#1,#2)(#3,#4)(#5,#6)}
}
{
\ifthenelse{\equal{#7}{1}}
{\psline[linewidth=0.9pt,linearc=.1]{-}(#1,#2)(#3,#4)(#5,#6)}
{\psline[linewidth=0.9pt,linearc=.1,linestyle=dashed,dash=2pt 2pt]{-}(#1,#2)(#3,#4)(#5,#6)}
}
}

% \ercrowfoot {#1 x0} {#2 y0} {#3 x11} {#4 y11} {#5 x12} {#6 y12} {#7 x13} {#8 y13} {#9 isconstructed}
\newcommand {\ercrowfoot}[9]
{
\ifthenelse{\equal{#9}{1}}
{
\psline[linewidth=1.5pt,linearc=.05,linecolor=lightgray]{-}(#1,#2)(#3,#4)
\psline[linewidth=1.5pt,linearc=.05,linecolor=lightgray]{-}(#1,#2)(#5,#6)
\psline[linewidth=1.5pt,linearc=.05,linecolor=lightgray]{-}(#1,#2)(#7,#8)
}{
\psline[linewidth=0.9pt,linearc=.05]{-}(#1,#2)(#3,#4)
\psline[linewidth=0.9pt,linearc=.05]{-}(#1,#2)(#5,#6)
\psline[linewidth=0.9pt,linearc=.05]{-}(#1,#2)(#7,#8)
}
}


% \eridcomprel{#1 x1}{#2 x2}{#3 y}
\newcommand {\eridcomprel}[3]
{
\psline[linewidth=0.9pt](#1,#3)(#2,#3)
}

% \eridrefrel{#1 x}{#2 y1}{#3 y2}
\newcommand {\eridrefrel}[3]
{
\psline[linewidth=0.9pt](#1,#2)(#1,#3)
}

% \ertext {#1 x} {#2 y} {#3 text anchor} {#4 text}  PRIVATE
\newcommand {\ertext}[4]
{
\rput[B#3]{0}(#1,#2){\fontsize{8}{10}\selectfont #4}
}

% \eretname {#1 x} {#2 y} {#3 text anchor} {#4 text} 
\newcommand {\eretname}[4]
{
\ertext{#1}{#2}{#3}{#4}
}

% \errelname {#1 x} {#2 y} {#3 text anchor} {#4 text} 
\newcommand {\errelname}[4]
{
\rput[B#3]{0}(#1,#2){\erextrasmallitalictext{#4}}
}


% \erscope {#1 x} {#2 y} {#3 text anchor} {#4 text}  15 March 2019
\newcommand {\erscope}[4]
{
\rput[B#3]{0}(#1,#2){\erextrasmallitalictext{#4}}
}

% \erreletname {#1 x} {#2 y} {#3 text anchor} {#4 text}  15 March 2019
\newcommand {\erreletname}[4]
{
\rput[B#3]{0}(#1,#2){\fontsize{10}{12}\selectfont #4}
}

% \ergroupannotation {#1 x} {#2 y} {#3 text anchor} {#4 text}
\newcommand {\ergroupannotation}[4]
{
\ertext{#1}{#2}{#3}{#4}
}


% \errelseq {#1 x} {#2 y}
\newcommand {\erelseq}[2]
{
}
\newcommand {\erattrmarkermand}
{\fontsize{6}{8}\selectfont $\blacksquare$}
\newcommand {\erattrmarkeropt}
{\fontsize{6}{8}\selectfont \CIRCLE}
\newcommand {\erderattrmarkermand}
{\fontsize{6}{8}\selectfont $\square$}
\newcommand {\erderattrmarkeropt}
{\fontsize{8}{10}\selectfont $\circ$}

% \erattr {#1 x} {#2 y} {#3 ismandatory}{#4 idenitfying} {#5 text}
\newcommand {\erattr}[5]
{
\ifthenelse{\equal{#3}{1}}
{\rput[l]{0}(#1,#2){\erattrmarkermand \ersmallitalictext{\ifthenelse{\equal{#4}{0}}{\underline{#5}}{#5}}}}
{\rput[l]{0}(#1,#2){\erattrmarkeropt \ersmallitalictext{\ifthenelse{\equal{#4}{0}}{\underline{#5}}{#5}}}}
}

\newcommand {\erdattr}[5]
{
\ifthenelse{\equal{#3}{1}}
{\rput[l]{0}(#1,#2){\erderattrmarkermand \ersmallitalictext{\ifthenelse{\equal{#4}{0}}{\underline{#5}}{#5}}}}
{\rput[l]{0}(#1,#2){\erderattrmarkeropt \ersmallitalictext{\ifthenelse{\equal{#4}{0}}{\underline{#5}}{#5}}}}
}


% \erarc {#1 x0} {#2 y0} {#3 x1} {#4 y1} {#5 x2} {#6 y2} {#7 x3} {#8 y3}
\newcommand {\erarc}[8]
{
\psbezier[showpoints=false]{-}(#1,#2) (#3, #4)(#5,#6) (#7, #8)
}

% \erarc {#1 x0} {#2 y0} {#3 x1} {#4 y1} {#5 x2} {#6 y2} {#7 x3} {#8 y3}
\newcommand {\errelseq}[8]
{
\psbezier[showpoints=false]{-}(#1,#2) (#3, #4)(#5,#6) (#7, #8)
}
% \ertrace {#1 trace}   
\newcommand {\ertrace}[1]
{
}

\usepackage{amsthm} % added 7th April 2018
% theorems.macros.tex

\newtheorem{theorem}{Theorem}[section]
\newtheorem{observation}[theorem]{Observation}
\newtheorem{lemma}[theorem]{Lemma}
\newtheorem{proposition}[theorem]{Proposition}
\newtheorem{corollary}[theorem]{Corollary}
\newtheorem{conjecture}[theorem]{Conjecture}
\newtheorem{numbereddefinition}[theorem]{Definition}

\newenvironment{definition}[1][Definition]{\begin{trivlist}
\item[\hskip \labelsep {\bfseries #1}]}{\end{trivlist}}
\newenvironment{examples}[1][Examples]{\begin{trivlist}
\item[\hskip \labelsep {\bfseries #1}]}{\end{trivlist}}
\newenvironment{example}[1][Example]{\begin{trivlist}
\item[\hskip \labelsep {\bfseries #1}]}{\end{trivlist}}
\newenvironment{remark}[1][Remark]{\begin{trivlist}
\item[\hskip \labelsep {\bfseries #1}]}{\end{trivlist}}

\newenvironment{tageqn}[1]
{
\begin{equation}
\stepcounter{equation}
\label{#1}
\tag{\theequation --#1}
}
{
\end{equation}
}

\newenvironment{axiom}[1]
{
\begin{equation}
\label{#1}
\tag{#1}
}
{
\end{equation}
}

% when the tag is required different from the label eg when has math symbols can use:
\newenvironment{axiomtagged}[2]
{
\begin{equation}
\label{#1}
\tag{#2}
}
{
\end{equation}
}

%visible label
\newcommand{\vlabel}[2][]{\label{#2}#1(\textit{#2}):}





\usepackage{mathptmx}  % This changes font to roman
\usepackage{anyfontsize}
\usepackage{mathtools}  % why have we got this?
\usepackage{alltt}    
\usepackage{mnsymbol} %used for rightpitchfork
\usepackage{cmll}
\usepackage{ulem}
\renewcommand{\ttdefault}{txtt}
\usepackage[left=1.5cm, right=4cm, marginparwidth=3cm, top=2cm, bottom=1.5cm]{geometry}
\usepackage{framed}
\usepackage[font=small]{caption}
\setlength{\captionmargin}{2cm}
\newcommand{\commentary}[1]{\marginpar{\footnotesize #1}}

\renewcommand{\erpictureFolder}[0]{../SharedPictures}

\newenvironment{categoricalaside}
{\begin{framed}
\textbf{Categorical Aside}
}
{
\end{framed}
}

%from berkley
\newcommand{\langl}{\begin{picture}(4.5,7)
\put(1.1,2.5){\rotatebox{60}{\line(1,0){5.5}}}
\put(1.1,2.5){\rotatebox{300}{\line(1,0){5.5}}}
\end{picture}}
\newcommand{\rangl}{\begin{picture}(4.5,7)
\put(.9,2.5){\rotatebox{120}{\line(1,0){5.5}}}
\put(.9,2.5){\rotatebox{240}{\line(1,0){5.5}}}
\end{picture}}
\newcommand{\lang}{\begin{picture}(5,7)\put(1.1,2.5){\rotatebox{45}{\line(1,0){6.0}}}\put(1.1,2.5){\rotatebox{315}{\line(1,0){6.0}}}\end{picture}}
\newcommand{\rang}{\begin{picture}(5,7)\put(.1,2.5){\rotatebox{135}{\line(1,0){6.0}}}\put(.1,2.5){\rotatebox{225}{\line(1,0){6.0}}}\end{picture}}
%Try sharper tuple brackets -- except gives errors nested in captions so comment out
%\renewcommand{\tuple}[1]{\lang #1 \rang}

\newcommand{\setsuchthat}[2]{\left\{#1 \ \middle|\ #2\right\}}
\newcommand{\set}[1]{\left\{#1\right\}} 

\newcommand{\genericmodel}{\mathcal{M}}  %PREVIOUSLY
\renewcommand{\genericmodel}{{m}}        %PREVIOUSLY
\renewcommand{\genericmodel}{\gamma}     % TRY THIS FOR A WHILE except texworks isnt happy with greek
\renewcommand{\genericmodel}{M}  %while debugging
\newcommand{\chiZero}{\mathcal{X}_0}
\newcommand{\chiZeroM}{\chiZero(\genericmodel)}
%\newcommand{\chiOne}{\mathcal{X}_1}
%\newcommand{\chiOneM}{\chiOne(\genericmodel)}
\newcommand{\chiM}{\mathcal{X}(\genericmodel)}
\newcommand{\veee}{v}
\newcommand{\Veee}{V}
\newcommand{\et}[1][\genericmodel]{et_{#1}}
\newcommand{\edge}[3][\genericmodel]{Edge_{#1}(#2,#3)}
\newcommand{\iedge}[3][\genericmodel]{IEdge_{#1}(#2,#3)}
\newcommand{\path}[3][\genericmodel]{Path_{#1}(#2,#3)}
\newcommand{\ipath}[3][\genericmodel]{IPath_{#1}(#2,#3)}
\newcommand{\attr}[2] [\genericmodel]{attr_{#1}(#2)}
\newcommand{\iattr}[2] [\genericmodel]{IAttr_{#1}(#2)}
\newcommand{\rel}[3][\genericmodel]{rel_{#1}(#2,#3)}
\newcommand{\irel}[3][\genericmodel]{IRel_{#1}(#2,#3)}
\newcommand{\iedges}[2] [\genericmodel]{i_{#1}(#2)}
\newcommand{\pk}[2] [\genericmodel]{pk_{#1}(#2)}
\newcommand{\fk}[2] [\genericmodel]{fk_{#1}(#2)}
\newcommand{\fkp}[2] [\genericmodel]{fk'_{#1}(#2)}
\newcommand{\fkpp}[2] [\genericmodel]{fk''_{#1}(#2)}

%functional dependencies
\newcommand{\sfd}[2]{\ensuremath{\set{#1} \morph #2}}  %singleton
\newcommand{\fd}[2]{\ensuremath{\sfd{#1}{\set{#2}}}}

\newcommand{\simplepath}[2]{
\ncline[linestyle=none,linewidth=0.1pt]{#1}{#2}   %was linestyle=dotted
\ncput[npos=0.05]{\pnode{dot#21}}
\ncput[npos=0.27]{\dotnode[dotsize=1pt]{dot#22}}
\ncput[npos=0.50]{\dotnode[dotsize=1pt]{dot#23}}
\ncput[npos=0.80]{\dotnode[dotsize=1pt]{dot#24}}
\ncput[npos=0.975]{\pnode{dot#25}}
\ncline[nodesep=2pt]{->}{dot#21}{dot#22}
\ncline[nodesep=2pt]{->}{dot#22}{dot#23}
\ncline[nodesep=2pt]{->}{dot#24}{dot#25}
\ncline[linestyle=dotted,nodesep=8pt]{dot#23}{dot#24} %was 10pt
}

\newcommand{\simplepatha}[3]{
\simplepath{#2}{#3}
\naput[labelsep=1pt]{#1}
}

\newcommand{\simplepathb}[3]{
\simplepath{#2}{#3}
\nbput[labelsep=1pt]{#1}
}
\newcommand{\term}[1]{\textit{{#1}}}
\newcommand{\logtophys}{\mathcal{X}}
\newcommand{\chen}{\mathcal{X}_0}
\newcommand{\chengenericmodel}{\chen(\genericmodel)}
\newcommand{\chigenericmodel}{\logtophys(\genericmodel)}
\newcommand{\phys}[1]{\overline{#1}}
\newcommand{\genericphysical}{\logtophys(\genericmodel)}

\newcommand{\inc}{\subseteq}
\newcommand{\incd}[4]{#1\left[#2\right]\inc#3\left[#4\right]}

\newcommand{\ntuple}[1]{\tuple{#1_1,...#1_n}}
\newcommand{\mtuple}[1]{\tuple{#1_1,...#1_n}}

\newcommand {\bntuple}{\ensuremath{\ntuple{b}}}
\newcommand {\fntuple}{\ensuremath{\ntuple{f}}}
\newcommand {\pntuple}{\ensuremath{\ntuple{p}}}
\newcommand {\qntuple}{\ensuremath{\ntuple{q}}}
\newcommand {\xntuple}{\ensuremath{\ntuple{x}}}
\newcommand{\foreachi}[1][n]{for each $i$, $1 \leq i \leq #1$}
\newcommand{\foreachj}[1][m]{for each $j$, $1 \leq j \leq #1$}
\newcommand{\foreachk}[1][l]{for each $k$, $1 \leq k \leq #1$}



%ccategories.macros.tex 

% Macros for diagrams in contextual categories and related categories

\usepackage{twoopt}
\usepackage{scalerel} 
\usepackage{xargs}

%\usepackage{mathabx}  %Caused font problems
%\usepackage{MnSymbol}  % caused font problems

\newcommand{\conu}
{\mathbf{C}(U)}

\newcommand{\depu}
{\mathbf{D}(U)}


\newcommand{\reqt}{\textbf{R}}
\newcommand{\reqtc}[1][\catc]{\reqt_{#1}}
\newcommand{\reqtcp}[1][\catcp]{\reqt_{#1}}



\newcommand{\cat}[1]{\textbf{#1}}

\newcommand{\catc}{\cat{C}}
\newcommand{\catcw}{\cat{C}\ }
\newcommand{\catcp}[1][C]{\textbf{#1}'}
\newcommand{\catcpp}[1][C]{\textbf{#1}''}
\newcommand{\obj}[1]{\ensuremath{|\cat{#1}|}}
\newcommand{\ccat}[1][C]{\ensuremath{\mathbb{#1}} }
\newcommand{\ccatc}{contextual category \ccat}
\newcommand{\cobj}[2][]{\ensuremath{|\ccat[#2]|_{#1}}}
\newcommand{\cslice}[2]{\ensuremath{\ccat[#1]_{#2}}}
\newcommand{\csliceobj}[3][]{\ensuremath{|\mathbb{#2}_{#3}|_{#1} }}
\newcommand{\varset}[1][]{\ensuremath{V_{#1} }}
\newcommand{\localvarsets}{\ensuremath{\mathcal{V} }}
\newcommand{\Fam}{\ensuremath{\mathbb{F\mathrm{am}} }}
\newcommand{\Fin}{\ensuremath{\textbf{Fin}} }
\newcommand{\Finp}{\ensuremath{\textbf{Finp}} }
\newcommand{\Po}{\ensuremath{\textbf{Po}} }
\newcommand{\Famslice}[1]{\ensuremath{\mathbb{F\mathrm{am}}_{#1} }}
\newcommand{\Famobj}[1][]{\ensuremath{|\mathbb{F\mathrm{am}}|_{#1} }}
\newcommand{\Famsliceobj}[2][]{\ensuremath{|\mathbb{F\mathrm{am}}_{#2}|_{#1} }}
\newcommand{\morph}{\rightarrow}
\newcommand{\epi}{\twoheadrightarrow}
\newcommand{\base}{\triangleleft}
\newcommand{\comp}{\circ}
\newcommand{\cross}{\otimes}
\newcommand{\pc}[2]{d^{#1}_{#2}}
\newcommand{\sub}{^*}
\newcommand{\diag}{\delta}
\newcommand{\pbase}[1]{\tilde{#1}}
\newcommand{\tuple}[1]{\langle#1\rangle}
\newcommand{\ndidly}{\ensuremath{\Join_n}}

\newcommand{\product}[1]{\bigtimes_{#1}}
\newcommand{\productn}{\product{n}}
\newcommand{\crossx}[3]{#1 \underset{#3}{\cross} #2}
\newcommand{\fibrex}[3]{#1 \underset{#3}{\Join} #2}
\newcommand{\powerset}{\mathcal{P}}
\newcommand{\primeds}[1]{
\ensuremath{\mathcal{P}(#1)} }
\newcommand{\compset}{\ \dot{\circ}\, }

% darrow
%\newcommand{\darrow}{\rightarrowtriangle} %use \smorph instead
\newcommand{\smorph}{\rightarrowtriangle}

 
\newcommand\dhead{\scaleobj{0.6}{\triangleright}}
%\newcommand{\dmorph}{\, \mbox{---} \! \cdot \! \raisebox{1.1pt}{\dhead}}    % dot style
\newcommand{\dmorph}{\, \mbox{---}\kern-1pt\raisebox{1.1pt}{\dhead\kern-1.75pt\dhead}}\,     % double triangle style

% projection tree
%\newcommand{\proj}[2]{proj_{#2}(#1)}

\newcommand{\proj}[2]{
\ensuremath{\mathcal{P}_{#2}(#1)} }

%pstrick supplements for arrows

\newlength{\arrnodesepA}
\newlength{\arrnodesepB}
\newlength{\arroffsetA}
\newlength{\arroffsetB}

%Modified to 2pt from 0pt on 23 July 2018
\newcommand{\arreset}{
\setlength{\arrnodesepA}{2pt}
\setlength{\arrnodesepB}{2pt}
\setlength{\arroffsetA}{0pt}
\setlength{\arroffsetB}{0pt}
}
\arreset

\newcommand{\ncarr}[3][0]{\ncarc[arcangle=#1,nodesepA=\arrnodesepA,nodesepB=\arrnodesepB,offsetA=\arroffsetA,offsetB=\arroffsetB,arrowsize=5pt,arrowinset=0.7]{->}{#2}{#3}}
\newcommand{\ncdarr}[3][0]{\ncarc[linestyle=dashed,arcangle=#1,nodesepA=\arrnodesepA,nodesepB=\arrnodesepB,offsetA=\arroffsetA,offsetB=\arroffsetB,arrowsize=5pt,arrowinset=0.7]{->}{#2}{#3}}
\newcommand{\jcbarr}[4][0]{ % ncbarr is defined in some thridy party package so do not use!\emph{}
\ncarr[#1]{#3}{#4}
\nbput[labelsep=2pt]{\footnotesize $#2$}
}

\newcommand{\ncaarr}[4][0]{
\ncarr[#1]{#3}{#4}
\naput[labelsep=2pt]{\footnotesize $#2$}
}

% \alabel{label}[npos][labelsep_pts]
\newcommandx*\alabel[3][2=0.5,3=2,usedefault]{\naput[labelsep=#3pt,npos=#2]{\footnotesize $#1$}}
% \blabel{label}[npos][labelsep_pts]
\newcommandx*\blabel[3][2=0.5,3=2,usedefault]{\nbput[labelsep=#3pt,npos=#2]{\footnotesize $#1$}}


\newif \ifbars
% to supress display of bars use \barsfalse to swith them on use \barstrue
\barstrue 
% \idcomp mark an arrow as one component of an identifier
\newcommand{\idcomp}{\ifbars{\ncput[npos=0, nrot=:U]{\psline(0.2,-0.075)(0.2,0.075)}}\fi}  %add a bar to a node connection arrow
% pstrick supplements for s-arrows (previous name for d-arrow - should convert}

\newlength{\sarnodesepA}
\newlength{\sarnodesepB}
\newlength{\saroffsetA}
\newlength{\saroffsetB}
\newlength{\sarnodesepAsav}
\newlength{\sarnodesepBsav}

\newcommand{\sarreset}{
\setlength{\sarnodesepA}{0pt}
\setlength{\sarnodesepB}{0pt}
\setlength{\saroffsetA}{0pt}
\setlength{\saroffsetB}{0pt}
}

\sarreset

% sar - S-arrow
\newcommand{\ncsar}[3][0]{
\setlength{\sarnodesepAsav}{\sarnodesepA}
\setlength{\sarnodesepBsav}{\sarnodesepB}
\addtolength{\sarnodesepA}{3pt}
\addtolength{\sarnodesepB}{7pt}
\ncarc[nodesepA=\sarnodesepA,nodesepB=\sarnodesepB,offsetA=\saroffsetA,offsetB=\saroffsetB,arcangle=#1]{-}{#2}{#3}
\ncput[nrot=:R,npos=1]{\pstriangle(0,0)(.2,.2)}
\setlength{\sarnodesepA}{\sarnodesepAsav}
\setlength{\sarnodesepB}{\sarnodesepBsav}
}


% bsar - below labelled S-arrow
\newcommand{\ncbsar}[4][0]{
\ncsar[#1]{#3}{#4}
\nbput[labelsep=2pt]{\footnotesize $#2$}
}
% asar - above labelled S-arrow
\newcommand{\ncasar}[4][0]{
\ncsar[#1]{#3}{#4}
\naput[labelsep=2pt]{\footnotesize $#2$}
}

% OLD cdar - composite dependency arrow - dot tyle
\iffalse
\newcommand{\nccdar}[3][0]{
\setlength{\sarnodesepAsav}{\sarnodesepA}
\setlength{\sarnodesepBsav}{\sarnodesepB}
\addtolength{\sarnodesepA}{3pt}
\addtolength{\sarnodesepB}{11pt}
\ncarc[nodesepA=\sarnodesepA,nodesepB=\sarnodesepB,offsetA=\saroffsetA,offsetB=\saroffsetB,arcangle=#1]{-}{#2}{#3}
\ncput[nrot=:R,npos=1]{\pstriangle(0,0.1)(.2,.2)}
\ncput[nrot=:R,npos=1]{\psdot[dotsize=1pt](-0.0075,0.05)}   %!!
\setlength{\sarnodesepA}{\sarnodesepAsav}
\setlength{\sarnodesepB}{\sarnodesepBsav}
}
\fi

% cdar - composite dependency arrow Mark II - double trangle style
\newcommand{\nccdar}[3][0]{
\setlength{\sarnodesepAsav}{\sarnodesepA}
\setlength{\sarnodesepBsav}{\sarnodesepB}
\addtolength{\sarnodesepA}{3pt}
\addtolength{\sarnodesepB}{13pt}
\ncarc[nodesepA=\sarnodesepA,nodesepB=\sarnodesepB,offsetA=\saroffsetA,offsetB=\saroffsetB,arcangle=#1]{-}{#2}{#3}
\ncput[nrot=:R,npos=1]{\pstriangle(0,0)(.2,.2)}
\ncput[nrot=:R,npos=1]{\pstriangle(0,0.2)(.2,.2)}
\setlength{\sarnodesepA}{\sarnodesepAsav}
\setlength{\sarnodesepB}{\sarnodesepBsav}
}


% bcdar - below labelled composite dependency arrow
\newcommand{\ncbcdar}[4][0]{
\nccdar[#1]{#3}{#4}
\nbput[labelsep=2pt]{\footnotesize $#2$}
}
% acdar - above labelled composite dependency arrow
\newcommand{\ncacdar}[4][0]{
\nccdar[#1]{#3}{#4}
\naput[labelsep=2pt]{\footnotesize $#2$}
}


% rsar - recursive S-arrow
\newcommand{\ncrsar}[2]{
\setlength{\sarnodesepAsav}{\sarnodesepA}
\setlength{\sarnodesepBsav}{\sarnodesepB}
\addtolength{\sarnodesepA}{3pt}
\addtolength{\sarnodesepB}{7pt}
\ncloop[nodesepA=\sarnodesepA,nodesepB=\sarnodesepB,
        offsetA=\saroffsetA,offsetB=\saroffsetB,
        armA=0.7cm,armB=0.6cm,angleA=90,angleB=-90,loopsize=-1,linearc=0.4
				]{-}{#1}{#2}
\ncput[nrot=:R,npos=5]{\pstriangle(0,0)(.2,.2)}
\setlength{\sarnodesepA}{\sarnodesepAsav}
\setlength{\sarnodesepB}{\sarnodesepBsav}
}

% pstrick supplements for multi-arrows

\newlength{\marnodesepA}
\newlength{\marnodesepB}
\newlength{\maroffsetB}
\newlength{\marnodesepBsav}

\newcommand{\marreset}{
\setlength{\marnodesepA}{0pt}
\setlength{\marnodesepB}{0pt}
\setlength{\maroffsetB}{0pt}
}

\marreset

%ncmarr[#1 arcangle1][#2 arcangle2]{#3 name}{#4 domain1}{#5 domain2}{#6 junction}{#7 codomain}
\newcommandtwoopt{\ncmarr}[6][8][8]{%
\ncarc[nodesepA=\marnodesepA,nodesepB=0,arcangle=#1]{-}{#3}{#5}
\ncarc[nodesepB=0,arcangle=-#1]{-}{#4}{#5}
\ncarc[arcangle=#2,nodesepB=\marnodesepB,offsetB=\maroffsetB]{->}{#5}{#6}
}%


\newcommandtwoopt{\nchmarr}[6][8][8]{%
\ncarc[nodesepA=\marnodesepA,nodesepB=0,arcangle=#1]{-}{#3}{#5}
\ncarc[nodesepB=0,arcangle=#1]{-}{#4}{#5}
\ncarc[arcangle=#2,nodesepB=\marnodesepB,offsetB=\maroffsetB]{->}{#5}{#6}
}%

\newcommandtwoopt{\ncamarr}[7][8][8]{%
\ncmarr[#1][#2]{#4}{#5}{#6}{#7}
\naput[npos=.05]{$#3$}
}%
\newcommandtwoopt{\ncbmarr}[7][8][8]{%
\ncmarr[#1][#2]{#4}{#5}{#6}{#7}
\nbput[npos=.05]{$#3$}
}%

\newcommandtwoopt{\ncbhmarr}[7][8][8]{%
\nchmarr[#1][#2]{#4}{#5}{#6}{#7}
\nbput[npos=.05]{$#3$}
}%

\newcommandtwoopt{\ncmarrr}[7][8][8]{
\ncarc[nodesepB=0,arcangle=#1]{-}{#3}{#6}
\ncline[nodesepB=0]{-}{#4}{#6}
\ncarc[nodesepB=0,arcangle=-#1]{-}{#5}{#6}
\ncarc[nodesepA=0,arcangle=#2]{->}{#6}{#7}
}

\newcommandtwoopt{\ncamarrr}[8][8][8]{
\ncmarrr[#1][#2]{#4}{#5}{#6}{#7}{#8}
\naput[npos=.05]{$#3$}
}
\newcommandtwoopt{\ncbmarrr}[8][8][8]{
\ncmarrr[#1][#2]{#4}{#5}{#6}{#7}{#8}
\nbput[npos=.05]{$#3$}
}


% 6 June 2020
% Edges representing attributes and relationship graphs
%  Ep   - partial
%  Epm  - partial mono
%  Epe  - partial epi
%  Epme - partial mono epi
%  Et   - total
%  Etm  - total mono
%  Ete  - total epi
%  Etme - total mono epi
%  recursive edges (use nccircle)
%  rEp   - partial
%  rEpm  - partial mono
%  rEpe  - partial epi
%  rEpme - partial mono epi
%  rEt   - total
%  rEtm  - total mono
%  rEte  - total epi
%  rEtme - total mono epi

\newcounter{EangleA}
\newcounter{EangleB}
\newcounter{EmidangleA}
\newcounter{EmidangleB}

% Ep - Edge partial
\newcommandtwoopt{\Ep}[4][0][0]{
\crowsfootedEdge{#1}{#2}{#3}{#4}{dashed}{dashed}
}



% Epm - Edge partial mono
\newcommandtwoopt{\Epm}[4][0][0]{
\monoEdge{#1}{#2}{#3}{#4}{dashed}{dashed}
}


% Epe - Edge partial epi
\newcommandtwoopt{\Epe}[4][0][0]{
\crowsfootedEdge{#1}{#2}{#3}{#4}{dashed}{solid}
}

% Epme - Edge partial mono epi
\newcommandtwoopt{\Epme}[4][0][0]{
\monoEdge{#1}{#2}{#3}{#4}{dashed}{solid}
}

% Et - Edge total
\newcommandtwoopt{\Et}[4][0][0]{
\crowsfootedEdge{#1}{#2}{#3}{#4}{solid}{dashed}
}

% Etm - Edge total mono
\newcommandtwoopt{\Etm}[4][0][0]{
\monoEdge{#1}{#2}{#3}{#4}{solid}{dashed}
}

% Ete - Edge total epi
\newcommandtwoopt{\Ete}[4][0][0]{
\crowsfootedEdge{#1}{#2}{#3}{#4}{solid}{solid}
}

% Etme - Edge total mono epi
\newcommandtwoopt{\Etme}[4][0][0]{
\monoEdge{#1}{#2}{#3}{#4}{solid}{solid}
}

% crowsfootedEdge - \crowsfootedEdge[angleA][midpointangle]{startnode}{endnode}[startstyle][endstyle]
\newcommand{\crowsfootedEdge}[6]{
\setlength{\sarnodesepAsav}{\sarnodesepA}
\setlength{\sarnodesepBsav}{\sarnodesepB}
\addtolength{\sarnodesepA}{3pt}
\addtolength{\sarnodesepB}{3pt}
\setcounter{EangleA}{ #1 + #2}
\setcounter{EangleB}{180  - #1 + #2}
\setcounter{EmidangleA}{#2}
\setcounter{EmidangleB}{#2 + 180}
\nccurve[nodesepA=\sarnodesepA,nodesepB=\sarnodesepB,offsetA=\saroffsetA,offsetB=\saroffsetB,angleA=\theEangleA, angleB=\theEangleB,linestyle=none,linewidth=0]{->}{#3}{#4}
\ncput[nrot=:R,npos=0]{\psline(0,.1)(.075,0)}
\ncput[nrot=:R,npos=0]{\psline(0,.1)(-0.075,0)}
\ncput{\pnode(0,0){xxx}}
\nccurve[nodesepA=0,nodesepB=\sarnodesepB,offsetA=0,offsetB=\saroffsetB,angleA=\theEmidangleA, angleB=\theEangleB, linestyle=#6]{->}{xxx}{#4}
%the following provides context for any following label
\nccurve[nodesepA=\sarnodesepA,nodesepB=0,offsetA=\saroffsetA,offsetB=0,angleA=\theEangleA, angleB=\theEmidangleB,linestyle=#5]{-}{#3}{xxx}
\setlength{\sarnodesepA}{\sarnodesepAsav}
\setlength{\sarnodesepB}{\sarnodesepBsav}
}

% monoEdge - \monoEdge[angleA][midpointangle]{startnode}{endnode}[startstyle][endstyle]
\newcommand{\monoEdge}[6]{ 
\setlength{\sarnodesepAsav}{\sarnodesepA}
\setlength{\sarnodesepBsav}{\sarnodesepB}
\addtolength{\sarnodesepA}{3pt}
\addtolength{\sarnodesepB}{3pt}
\setcounter{EangleA}{ #1 + #2}
\setcounter{EangleB}{180  - #1 + #2}
\setcounter{EmidangleA}{#2}
\setcounter{EmidangleB}{#2 + 180}
\nccurve[nodesepA=\sarnodesepA,nodesepB=\sarnodesepB,offsetA=\saroffsetA,offsetB=\saroffsetB,angleA=\theEangleA, angleB=\theEangleB,linestyle=none,linewidth=0]{->}{#3}{#4}
\ncput{\pnode(0,0){xxx}}
\nccurve[nodesepA=0,nodesepB=\sarnodesepB,offsetA=0,offsetB=\saroffsetB,angleA=\theEmidangleA, angleB=\theEangleB, linestyle=#6]{->}{xxx}{#4}
%the following provides context for any following label
\nccurve[nodesepA=\sarnodesepA,nodesepB=0,offsetA=\saroffsetA,offsetB=0,angleA=\theEangleA, angleB=\theEmidangleB,linestyle=#5]{-}{#3}{xxx}
\setlength{\sarnodesepA}{\sarnodesepAsav}
\setlength{\sarnodesepB}{\sarnodesepBsav}
}


\newcounter{EangleGiven}
\newcounter{EangleComplementary}
\newcounter{EangleStartCorrected}
\newcounter{EangleEndCorrected}


%  rEp   - recursive Edge partial
\newcommand{\rEp}[2][0]{
\setcounter{EangleGiven}{#1}
\setcounter{EangleStartCorrected}{#1-10} %correction required because for nccurve unlike nccircle angle measured at boundary not at centre of node
\setcounter{EangleEndCorrected}{#1+180+10} %correction required because angle measured at boundary not at centre of node
\setcounter{EangleComplementary}{#1 + 180}
\nccircle[angleA=\theEangleComplementary, nodesep=0pt, linestyle=none]{-}{#2}{.4cm} % an invisible circle to hang the midpoint from
\ncput{\pnode(0,0){midpoint}}                                         
\nccurve[nodesepA=1pt,nodesepB=0pt,offsetA=0pt,offsetB=0pt,angleA=\theEangleStartCorrected, angleB=\theEangleGiven, ncurv=1.359, linecolor=black, linestyle=dashed]{-}{#2}{midpoint}
\ncput[nrot=:R,npos=0]{\psline(0,.1)(.075,0)}
\ncput[nrot=:R,npos=0]{\psline(0,.1)(-0.075,0)}
\nccurve[nodesepA=0pt,nodesepB=2pt,offsetA=0pt,offsetB=0pt,angleA=\theEangleComplementary, angleB=\theEangleEndCorrected, ncurv=1.359, linestyle=dashed]{-}{midpoint}{#2}
% 1.359 is e/2 happenchance or algorithmically necessary???
% now draw arrowhead -- dont include in the nccurve because this alters the line position - a strange feature of pstruicks
\ncput[npos=0.9]{\pnode(0,0){yyy}}
\ncline{->}{yyy}{#2}
% repeat from earlier to provide context for label that might follow
\nccurve[nodesepA=1pt,nodesepB=0pt,offsetA=0pt,offsetB=0pt,angleA=\theEangleStartCorrected, angleB=\theEangleGiven, ncurv=1.359, linecolor=black, linestyle=dashed]{-}{#2}{midpoint} 
} 

%  rEpm  - recursive Edge partial mono
\newcommand{\rEpm}[2][0]{
\setcounter{EangleGiven}{#1}
\setcounter{EangleStartCorrected}{#1-10} %correction required because for nccurve unlike nccircle angle measured at boundary not at centre of node
\setcounter{EangleEndCorrected}{#1+180+10} %correction required because angle measured at boundary not at centre of node
\setcounter{EangleComplementary}{#1 + 180}
\nccircle[angleA=\theEangleComplementary, nodesep=0pt, linestyle=none]{-}{#2}{.4cm} % an invisible circle to hang the midpoint from
\ncput{\pnode(0,0){midpoint}}   
\nccurve[nodesepA=0pt,nodesepB=2pt,offsetA=0pt,offsetB=0pt,angleA=\theEangleComplementary, angleB=\theEangleEndCorrected, ncurv=1.359, linestyle=dashed]{-}{midpoint}{#2}
% 1.359 is e/2 happenchance or algorithmically necessary???
% now draw arrowhead -- dont include in the nccurve because this alters the line position - a strange feature of pstruicks
\ncput[npos=0.9]{\pnode(0,0){yyy}}
\ncline{->}{yyy}{#2}
% last to provide context for label that might follow
\nccurve[nodesepA=1pt,nodesepB=0pt,offsetA=0pt,offsetB=0pt,angleA=\theEangleStartCorrected, angleB=\theEangleGiven, ncurv=1.359, linecolor=black, linestyle=dashed]{-}{#2}{midpoint} 
}

%  rEpe  - recursive Edge partial epi
\newcommand{\rEpe}[2][0]{
\setcounter{EangleGiven}{#1}
\setcounter{EangleStartCorrected}{#1-10} %correction required because for nccurve unlike nccircle angle measured at boundary not at centre of node
\setcounter{EangleEndCorrected}{#1+180+10} %correction required because angle measured at boundary not at centre of node
\setcounter{EangleComplementary}{#1 + 180}
\nccircle[angleA=\theEangleComplementary, nodesep=0pt, linestyle=none]{-}{#2}{.4cm} % an invisible circle to hang the midpoint from
\ncput{\pnode(0,0){midpoint}}                                         
\nccurve[nodesepA=1pt,nodesepB=0pt,offsetA=0pt,offsetB=0pt,angleA=\theEangleStartCorrected, angleB=\theEangleGiven, ncurv=1.359, linecolor=black, linestyle=dashed]{-}{#2}{midpoint}
\ncput[nrot=:R,npos=0]{\psline(0,.1)(.075,0)}
\ncput[nrot=:R,npos=0]{\psline(0,.1)(-0.075,0)}
\nccurve[nodesepA=0pt,nodesepB=2pt,offsetA=0pt,offsetB=0pt,angleA=\theEangleComplementary, angleB=\theEangleEndCorrected, ncurv=1.359]{-}{midpoint}{#2}
% 1.359 is e/2 happenchance or algorithmically necessary???
% now draw arrowhead -- dont include in the nccurve because this alters the line position - a strange feature of pstruicks
\ncput[npos=0.9]{\pnode(0,0){yyy}}
\ncline{->}{yyy}{#2}
% repeat from earlier to provide context for label that might follow
\nccurve[nodesepA=1pt,nodesepB=0pt,offsetA=0pt,offsetB=0pt,angleA=\theEangleStartCorrected, angleB=\theEangleGiven, ncurv=1.359, linecolor=black, linestyle=dashed]{-}{#2}{midpoint} 
}

%  rEpme - recursive Edge partial mono epi
\newcommand{\rEpme}[2][0]{
\setcounter{EangleGiven}{#1}
\setcounter{EangleStartCorrected}{#1-10} %correction required because for nccurve unlike nccircle angle measured at boundary not at centre of node
\setcounter{EangleEndCorrected}{#1+180+10} %correction required because angle measured at boundary not at centre of node
\setcounter{EangleComplementary}{#1 + 180}
\nccircle[angleA=\theEangleComplementary, nodesep=0pt, linestyle=none]{-}{#2}{.4cm} % an invisible circle to hang the midpoint from
\ncput{\pnode(0,0){midpoint}}                                         
%\nccurve[nodesepA=0pt,nodesepB=0pt,offsetA=0pt,offsetB=0pt,angleA=\theEangleComplementary, angleB=\theEangleEndCorrected, ncurv=1.359, linestyle=dashed]{->}{xxx}{#2}
\nccurve[nodesepA=0pt,nodesepB=2pt,offsetA=0pt,offsetB=0pt,angleA=\theEangleComplementary, angleB=\theEangleEndCorrected, ncurv=1.359]{-}{midpoint}{#2}
% 1.359 is e/2 happenchance or algorithmically necessary???
% now draw arrowhead -- dont include in the nccurve because this alters the line position - a strange feature of pstruicks
\ncput[npos=0.9]{\pnode(0,0){yyy}}
\ncline{->}{yyy}{#2}
% last so that to provide context for label that might follow
\nccurve[nodesepA=1pt,nodesepB=0pt,offsetA=0pt,offsetB=0pt,angleA=\theEangleStartCorrected, angleB=\theEangleGiven, ncurv=1.359, linecolor=black, linestyle=dashed]{-}{#2}{midpoint} 
}

% rEt - recursive Edge total
\newcommand{\rEt}[2][0]{
\setcounter{EangleGiven}{#1}
\setcounter{EangleStartCorrected}{#1-10} %correction required because for nccurve unlike nccircle angle measured at boundary not at centre of node
\setcounter{EangleEndCorrected}{#1+180+10} %correction required because angle measured at boundary not at centre of node
\setcounter{EangleComplementary}{#1 + 180}
\nccircle[angleA=\theEangleComplementary, nodesep=0pt, linestyle=none]{-}{#2}{.4cm} % an invisible circle to hang the midpoint from
\ncput{\pnode(0,0){midpoint}}                                         
\nccurve[nodesepA=1pt,nodesepB=0pt,offsetA=0pt,offsetB=0pt,angleA=\theEangleStartCorrected, angleB=\theEangleGiven, ncurv=1.359, linecolor=black]{-}{#2}{midpoint}
\ncput[nrot=:R,npos=0]{\psline(0,.1)(.075,0)}
\ncput[nrot=:R,npos=0]{\psline(0,.1)(-0.075,0)}
%\nccurve[nodesepA=0pt,nodesepB=0pt,offsetA=0pt,offsetB=0pt,angleA=\theEangleComplementary, angleB=\theEangleEndCorrected, ncurv=1.359, linestyle=dashed]{->}{xxx}{#2}
\nccurve[nodesepA=0pt,nodesepB=2pt,offsetA=0pt,offsetB=0pt,angleA=\theEangleComplementary, angleB=\theEangleEndCorrected, ncurv=1.359, linestyle=dashed]{-}{midpoint}{#2}
% 1.359 is e/2 happenchance or algorithmically necessary???
% now draw arrowhead -- dont include in the nccurve because this alters the line position - a strange feature of pstruicks
\ncput[npos=0.9]{\pnode(0,0){yyy}}
\ncline{->}{yyy}{#2}
% repeat from earlier to provide context for label that might follow
\nccurve[nodesepA=1pt,nodesepB=0pt,offsetA=0pt,offsetB=0pt,angleA=\theEangleStartCorrected, angleB=\theEangleGiven, ncurv=1.359, linecolor=black]{-}{#2}{midpoint} 
}

%  rEtm  - recursive Edge total mono
\newcommand{\rEtm}[2][0]{
\setcounter{EangleGiven}{#1}
\setcounter{EangleStartCorrected}{#1-10} %correction required because for nccurve unlike nccircle angle measured at boundary not at centre of node
\setcounter{EangleEndCorrected}{#1+180+10} %correction required because angle measured at boundary not at centre of node
\setcounter{EangleComplementary}{#1 + 180}
\nccircle[angleA=\theEangleComplementary, nodesep=0pt, linestyle=none]{-}{#2}{.4cm} % an invisible circle to hang the midpoint from
\ncput{\pnode(0,0){midpoint}}     
\nccurve[nodesepA=0pt,nodesepB=2pt,offsetA=0pt,offsetB=0pt,angleA=\theEangleComplementary, angleB=\theEangleEndCorrected, ncurv=1.359, linestyle=dashed]{-}{midpoint}{#2}
% 1.359 is e/2 happenchance or algorithmically necessary???
% now draw arrowhead -- dont include in the nccurve because this alters the line position - a strange feature of pstruicks
\ncput[npos=0.9]{\pnode(0,0){yyy}}
\ncline{->}{yyy}{#2}
% last to provide context for label that might follow
\nccurve[nodesepA=1pt,nodesepB=0pt,offsetA=0pt,offsetB=0pt,angleA=\theEangleStartCorrected, angleB=\theEangleGiven, ncurv=1.359, linecolor=black]{-}{#2}{midpoint} 
}

%  rEte  - total epi
\newcommand{\rEte}[2][0]{
\setcounter{EangleGiven}{#1}
\setcounter{EangleStartCorrected}{#1-10} %correction required because for nccurve unlike nccircle angle measured at boundary not at centre of node
\setcounter{EangleEndCorrected}{#1+180+10} %correction required because angle measured at boundary not at centre of node
\setcounter{EangleComplementary}{#1 + 180}
\nccircle[angleA=\theEangleComplementary, nodesep=0pt, linestyle=none]{-}{#2}{.4cm} % an invisible circle to hang the midpoint from
\ncput{\pnode(0,0){midpoint}}                                         
\nccurve[nodesepA=1pt,nodesepB=0pt,offsetA=0pt,offsetB=0pt,angleA=\theEangleStartCorrected, angleB=\theEangleGiven, ncurv=1.359, linecolor=black]{-}{#2}{midpoint}
\ncput[nrot=:R,npos=0]{\psline(0,.1)(.075,0)}
\ncput[nrot=:R,npos=0]{\psline(0,.1)(-0.075,0)}
%\nccurve[nodesepA=0pt,nodesepB=0pt,offsetA=0pt,offsetB=0pt,angleA=\theEangleComplementary, angleB=\theEangleEndCorrected, ncurv=1.359, linestyle=dashed]{->}{xxx}{#2}
\nccurve[nodesepA=0pt,nodesepB=2pt,offsetA=0pt,offsetB=0pt,angleA=\theEangleComplementary, angleB=\theEangleEndCorrected, ncurv=1.359]{-}{midpoint}{#2}
% 1.359 is e/2 happenchance or algorithmically necessary???
% now draw arrowhead -- dont include in the nccurve because this alters the line position - a strange feature of pstruicks
\ncput[npos=0.9]{\pnode(0,0){yyy}}
\ncline{->}{yyy}{#2}
% repeat from earlier to provide context for label that might follow
\nccurve[nodesepA=1pt,nodesepB=0pt,offsetA=0pt,offsetB=0pt,angleA=\theEangleStartCorrected, angleB=\theEangleGiven, ncurv=1.359, linecolor=black]{-}{#2}{midpoint} 
}

%  rEtme - recursive Edge total mono epi

\newcommand{\rEtme}[2][0]{
\setcounter{EangleGiven}{#1}
\setcounter{EangleStartCorrected}{#1-10} %correction required because for nccurve unlike nccircle angle measured at boundary not at centre of node
\setcounter{EangleEndCorrected}{#1+180+10} %correction required because angle measured at boundary not at centre of node
\setcounter{EangleComplementary}{#1 + 180}
\nccircle[angleA=\theEangleComplementary, nodesep=0pt, linestyle=none]{-}{#2}{.4cm} % an invisible circle to hang the midpoint from
\ncput{\pnode(0,0){midpoint}}     
\nccurve[nodesepA=0pt,nodesepB=2pt,offsetA=0pt,offsetB=0pt,angleA=\theEangleComplementary, angleB=\theEangleEndCorrected, ncurv=1.359]{-}{midpoint}{#2}
% 1.359 is e/2 happenchance or algorithmically necessary???
% now draw arrowhead -- dont include in the nccurve because this alters the line position - a strange feature of pstruicks
\ncput[npos=0.9]{\pnode(0,0){yyy}}
\ncline{->}{yyy}{#2}
% last to provide context for label that might follow
\nccurve[nodesepA=1pt,nodesepB=0pt,offsetA=0pt,offsetB=0pt,angleA=\theEangleStartCorrected, angleB=\theEangleGiven, ncurv=1.359, linecolor=black]{-}{#2}{midpoint} 
}

%gats.macros.tex

\usepackage{environ}    % also used in ermacros % here used for \NewEnvrion

\newcommand{\gat}[1][U]{
\ensuremath{\mathcal{#1}}}  % used to hav a space in here
\newcommand{\gatw}[1][U]{\gat[#1]\ }  % use this if need trailing space
\newcommand{\ingat}[1][U]{in \gat[#1]}
\newcommand{\isagat}[1][U]{\gat[#1] is a g.a.t.}
\newcommand{\inagat}{in a g.a.t. }

% macro for a generic theory
%\newcommand{\theory}
%{\textit{U}}

\newcommand{\intheory}
{is a derived rule of \gat[U]}

% Macros for GAT rules

\newcommand{\isT}[1]
{#1\mbox{ is a type}}

\newcommand{\ofT}[2]
{#1 \in #2
}

% Macros for GAT rules   <!-- new old -->
\newcommand{\istype}[1]
{#1\mbox{ is a type}}

\newcommand{\oftype}[2]
{#1 \in #2
}

%\context{x}{\Delta}{n}
\newcommand{\context}[3]
{\ofT{#1_1}{#2_1},... \ofT{#1_{#3}}{#2_{#3}(#1_1,...#1_{#3-1})}
}

%\subcontext{x}{\Delta}{i}{k}
\newcommand{\subcontext}[4]
{\ofT{#1_{#3_1}}{#2_{#3_1}},... \ofT{#1_{#3_#4}}{#2_{#3_#4}(#1_1,...#1_{#3_#4-1})}
}

% #schematic context
\newcommand{\schmcon}[3]
{\ofT{#1_1}{#2_1},... \ofT{#1_{#3}}{#2_{#3}}
}
% abbreviated to
\newcommand{\con}[3]
{\schmcon{#1}{#2}{#3}}

% schematic subcontext
%\subcon{x}{\Delta}{i}{k}
\newcommand{\subcon}[4]
{\ofT{#1_{#3_1}}{#2_{#3_1}},... \ofT{#1_{#3_#4}}{#2_{#3_#4}}
}

% permuted context
%\permcon{x}{\Delta}{n}{\sigma}
\newcommand{\permcon}[4]
{\ofT{#1_{#4(1)}}{#2_{#4(1)}},... \ofT{#1_{#4(#3)}}{#2_{#4(#3)}}
}
% permuted term
%\permterm{t}{n}{\sigma}
\newcommand{\permterm}[3]
{
#1_{#3(1)},...#1_{#3(#2)}
}


% Idioms
\newcommand{\xDelta}[1]{\con{x}{\Delta}{#1}}
\newcommand{\xDeltap}[1]{\con{x}{\Delta'}{#1}}
\newcommand{\xOmega}[1]{\con{x}{\Omega}{#1}}
\newcommand{\xOmegap}[1]{\con{x}{\Omega'}{#1}}
\newcommand{\yOmega}[1]{\con{y}{\Omega}{#1}}
\newcommand{\yOmegap}[1]{\con{y}{\Omega'}{#1}}

\newcommand{\xDeltasigma}[1]{\permcon{x}{\Delta}{#1}{\sigma}}
\newcommand{\xDeltapsigma}[1]{\permcon{x}{\Delta'}{#1}{\sigma}}
\newcommand{\xOmegasigma}[1]{\permcon{x}{\Omega}{#1}{\sigma}}
\newcommand{\xOmegapsigma}[1]{\permcon{x}{\Omega'}{#1}{\sigma}}
\newcommand{\yOmegasigma}[1]{\permcon{y}{\Omega}{#1}{\sigma}}
\newcommand{\yOmegapsigma}[1]{\permcon{y}{\Omega'}{#1}{\sigma}}

\newcommand{\xDeltainvsigma}[1]{\permcon{x}{\Delta}{#1}{\sigma^{-1}}}
\newcommand{\xDeltapinvsigma}[1]{\permcon{x}{\Delta'}{#1}{\sigma^{-1}}}
\newcommand{\xOmegainvsigma}[1]{\permcon{x}{\Omega}{#1}{\sigma^{-1}}}
\newcommand{\xOmegapinvsigma}[1]{\permcon{x}{\Omega'}{#1}{\sigma^{-1}}}
\newcommand{\yOmegainvsigma}[1]{\permcon{y}{\Omega}{#1}{\sigma^{-1}}}
\newcommand{\yOmegapinvsigma}[1]{\permcon{y}{\Omega'}{#1}{\sigma^{-1}}}

%Idioms enclosed as tuples
\newcommand{\encxDelta}[1]{\tuple{\con{x}{\Delta}{#1}}}
\newcommand{\encxDeltap}[1]{\tuple{\con{x}{\Delta'}{#1}}}
\newcommand{\encxOmega}[1]{\tuple{\con{x}{\Omega}{#1}}}
\newcommand{\encxOmegap}[1]{\tuple{\con{x}{\Omega'}{#1}}}
\newcommand{\encyOmega}[1]{\tuple{\con{y}{\Omega}{#1}}}
\newcommand{\encyOmegap}[1]{\tuple{\con{y}{\Omega'}{#1}}}

\newcommand{\encxDeltasigma}[1]{\tuple{\permcon{x}{\Delta}{#1}{\sigma}}}
\newcommand{\encxDeltapsigma}[1]{\tuple{\permcon{x}{\Delta'}{#1}{\sigma}}}
\newcommand{\encxOmegasigma}[1]{\tuple{\permcon{x}{\Omega}{#1}{\sigma}}}
\newcommand{\encxOmegapsigma}[1]{\tuple{\permcon{x}{\Omega'}{#1}{\sigma}}}
\newcommand{\encyOmegasigma}[1]{\tuple{\permcon{y}{\Omega}{#1}{\sigma}}}
\newcommand{\encyOmegapsigma}[1]{\tuple{\permcon{y}{\Omega'}{#1}{\sigma}}}

\newcommand{\encxDeltainvsigma}[1]{\tuple{\permcon{x}{\Delta}{#1}{\sigma^{-1}}}}
\newcommand{\encxDeltapinvsigma}[1]{\tuple{\permcon{x}{\Delta'}{#1}{\sigma^{-1}}}}
\newcommand{\encxOmegainvsigma}[1]{\tuple{\permcon{x}{\Omega}{#1}{\sigma^{-1}}}}
\newcommand{\encxOmegapinvsigma}[1]{\tuple{\permcon{x}{\Omega'}{#1}{\sigma^{-1}}}}
\newcommand{\encyOmegainvsigma}[1]{\tuple{\permcon{y}{\Omega}{#1}{\sigma^{-1}}}}
\newcommand{\encyOmegapinvsigma}[1]{\tuple{\permcon{y}{\Omega'}{#1}{\sigma^{-1}}}}

\newcommand{\tstyle}{\vdash}
\newcommand{\gatdisplayrule}[3][]
{
\setlength{\fboxsep}{1pt}       
\setlength{\fboxrule}{0pt}
\fbox{$\displaystyle \frac{#2}{#3\rule[-0.3cm]{0cm}{0cm}}$#1}    %added vertival space using \rule
}
\newcommand{\genericAintroductoryrule} {\gatdisplayrule{\xDelta{n}}{\isT{A(\xn)}}}
\newcommand{\genericfintroductoryrule}  {\gatdisplayrule{\xDelta{n}}{\ofT{f(\xn)}{\Delta}}}

% gat macros developed for cwf paper

% Expressing gats
\newenvironment{gatrules}
{
$$
\begin{array}{l l}
}
{
\end{array}
$$
}
\newcommand{\gatintros}
{
\textbf{Symbol} & \textbf{Introductory\ Rule}                      \\}

\newcommand{\gataxioms}
{\textbf{Axioms}\\}
\newcommand{\gatintro}[3]{\ #1 & #2 \tstyle #3 \\}
\newcommand{\gatlocalintro}[3]{\ #1 & #2 \dashv }
\newcommand{\gataxiom}[2]{\multicolumn{2}{l}{\ \ #1\mbox{,  whenever\ } #2} \\}
\newcommand{\noleft}{\left.\kern-\nulldelimiterspace} % so that no space taken by absent left brace


\newcommand{\gatmultiaxiom}[2]
{\multicolumn{2}{l}{
  \noleft
    \begin{array}{l}
		#1
    \end{array} 
  \right\} \mbox{whenever\ } 	#2 
	}\\}
	
	\newcommand{\axid}[1]{\text{#1}.\ }	

%New context sharing macros
\newcommand{\gatintroducing}[1]{
{\arraycolsep=0pt
  \begin{array}{l}
          #1
  \end{array}} &
}

%*********************************
% \begin{\gatgroup}{context}
%    rules
%  \end{\gatgroup}
%*********************************
\NewEnviron{gatgroup}[1]{%
  \noleft
  {\arraycolsep=0pt
   \begin{array}{l}
\BODY
    \end{array} 
   }
   \ \right\} 
	%\mbox{\ whenever\ } 
	#1
	\vspace{0.1cm} 
}
%*********************************

%*********************************
% \begin{\gatgroupnoshared}
%    rule
%  \end{\gatgroupnoshared}
%*********************************
\NewEnviron{gatgroupnoshared}{%
  {\arraycolsep=0pt
   \begin{array}{l}
\BODY
    \end{array} 
   }
   \ 
	\vspace{0.1cm} 
}
%*********************************

% \gatsingular[width]{context}{conclusion}
\newcommand{\gatsingular}[3][4cm]{
\begin{gatgroupnoshared}
\gatleaf[#1]{#2}{#3} 
\end{gatgroupnoshared}
}

%*********************************
% \gatleaf}[width]{context}{assertion}
%*********************************
\newcommand{\gatleaf}[3][4cm]{%
\makebox[#1]{$#3$ \dotfill} \dotfill \  #2
}
%*********************************
%*********************************
% \gatstandalonesingle}{context}{assertion}
%*********************************
\newcommand{\gatstandalonesingle}[2]{%
#2 \makebox[2.5cm]{\dotfill} \  #1
}
%*********************************

% \gataxiomno{axiomno}
\newcommand{\gataxiomno}[1]{\makebox[0.5cm]{} \axid{#1}}


% metagat.macros.tex

%Meta-theories

%\newcommand{\typ}{\triangleright}
\newcommand{\typ}{\nabla}
\newcommand{\trm}{\tau}
\newcommand{\cross}{\otimes}
\newcommand{\sub}{^*}
\newcommand{\diag}{\delta}

\newcommand{\typeseq}[2]
{\ofT{#1_1}{\typ},... \ofT{#1_{#2}}{\typ(#1_{#2-1})}}

\newcommand{\typeseqcont}[3]
{\ofT{#1_1}{\typ({#2})},... \ofT{#1_{#3}}{\typ(#1_{#3-1})}}

\newcommand{\Ob}{Ob}
\newcommand{\obj}{Ob} % <!-- new old --<
\newcommand{\Hom}{Hom}
\newcommand{\objseq}[2]
{\ofT{#1_1}{\obj},... \ofT{#1_{#2}}{\obj(#1_{#2-1})}}


\def\dottededge{\ncline[linestyle=dotted, nodesep=0.3cm]}
\def\noedge{\ncline[linestyle=none]}
\def\thinedge{\ncline[linewidth=0.4pt]}

\newcommand{\member}[1]
{\ncarc[arcangle=-30,nodesepB=0.03]{->}{\pspred}{\pssucc}
\nbput[labelsep=0.1]{#1}}

\newcommand{\loweraccutemember}[1]
{\ncarc[arcangle=-15,nodesepB=0.03]{->}{\pspred}{\pssucc}
\nbput[labelsep=0.05,npos=0.85]{#1}}

\newcommand{\uppermember}[1]
{\ncarc[arcangle=30,nodesepB=0.03]{->}{\pspred}{\pssucc}\naput{#1}}

\newcommand{\upperaccutemember}[1]
{\ncarc[arcangle=10,nodesepB=0.03]{->}{\pspred}{\pssucc}\naput[npos=0.85]{#1}}

% flexbranch 
% #1 node label
% #2 thislevelsep
% #3 next level sep
% #4 variable (eg x)
% #5 index leter (eg n)
% #6 close parenthesis
% #7 continuation branches
\newcommand{\flexbranch}[7]
{
\pstree[thislevelsep=*#2,nodesep=0.05]
		{\Rnode{#1 1}{\Tr{#4_1 #6}}}
	  {\pstree[thislevelsep=#3]  
				   {\Rnode{#1 2}{\Tr[edge=\dottededge]{#4_{#5} #6}}}
					 {#7}
		}
}

\newcommand{\flexbranchplusleaf}[6]
{
\flexbranch{#1}{#2}{#3}{#4} {#5} {#6}
  {
   %\Rnode{#1 3}{\Tr{#4 #6}}
	 \Tr{\Rnode{#1 3}{#4 #6}}
  }
}

\newcommand{\flexbranchplusarc}[7]
{
\flexbranch{#1}{#2}{#3}{#4} {#5} {#6}
  {
   %\Rnode{#1 3}{\Tr{#4 #6}\member{#7}}
	 \Tr{\Rnode{#1 3}{#4 #6}}\member{#7}
  }
}

\newcommand{\flexbranchinitialarc}[9]
{
\pstree[thislevelsep=*#2,nodesep=0.05]
		{\Rnode{#1 1}{\Tr{#4_#8 #6}}#9}
	  {\pstree[thislevelsep=#3]  
				   {\Rnode{#1 2}{\Tr[edge=\dottededge]{#4_{#5} #6}}}
					 {#7}
		}
}

\newcommand{\equality}[2]
{
\ncline [doubleline=true, nodesep=0.2cm]{#1}{#2}
}
\newcommand{\equalityarc}[2]
{
\ncarc [arcangleA=-30, arcangleB=-20, doubleline=true, nodesep=0.1cm]{#1}{#2}
}

%The following are stylistic so belong in main document not here.
%\usepackage[margin=4.0cm]{geometry} % This shouldn't be here commented out 17 July 2018
%\usepackage{mathptmx}               % This changes font to roman so doesn't belong here
%
\usepackage{amsfonts}
\usepackage{amssymb} % added 08\02\2019 as an experiment. Needed in some instances for \blacksquare
                     % not needed is class is `beamer' but I don't know why not
\usepackage{array}
\usepackage{pstricks}
\usepackage{pst-tree}
\usepackage{pst-plot}
\usepackage{pst-node}
\usepackage{stmaryrd}
\usepackage{amsmath}
\usepackage{verbatim}
\usepackage{graphicx}  
\usepackage{calc}
\usepackage{xifthen}
%\usepackage{xcolor} investigate with beamer
\usepackage{color}
\usepackage{stringstrings}
%\usepackage[small,bf,margin=3pt,format=hang, labelsep=endash,singlelinecheck=false]{caption} %prevuiously justification=justified
%\usepackage{enumerate}
%\usepackage{enumitem}
\usepackage{enumerate}
%\usepackage[shortlabels]{enumitem} %Removed this 28/01/2019 because interfereing with a beamer presentation. 
\usepackage{float}
\usepackage[section]{placeins}
%\setlength{\captionmargin}{5pt}
\usepackage{environ}
\usepackage{multirow}
\usepackage{rotating}
\usepackage{longtable}
\usepackage{afterpage}
\usepackage{needspace}


%DEFINE ENVIRONMENT BLOCK
% Riddle
\newsavebox{\riddlebox}

\newenvironment{erexample}
{\newcommand\colboxcolor{F0F0F0}%was F8F8F8
\begin{lrbox}{\riddlebox}
\begin{minipage}{\dimexpr\columnwidth-2\fboxsep\relax} \textbf{} \\ \itshape}
{\end{minipage}\end{lrbox}%
%\begin{center}
\colorbox[HTML]{\colboxcolor}{\usebox{\riddlebox}}
%\end{center}
}

\newenvironment{erbox}
{\newcommand\colboxcolor{F0F0F0}%was F8F8F8
\begin{lrbox}{\riddlebox}%
\begin{minipage}{\dimexpr\columnwidth-2\fboxsep\relax} }
{\end{minipage}\end{lrbox}%
%\begin{center}
\colorbox[HTML]{\colboxcolor}{\usebox{\riddlebox}}
%\end{center}
}

%\begin{erboxedFigure}{#1 FigureParam}{#2 Label}{#3 Caption}
\NewEnviron{erboxedFigure}[3]{%
\begin{figure}[#1]
\begin{erexample}
\begin{center}
\BODY
\end{center}
\vspace{-0.5cm}
\caption{#3}
\label{#2}
\end{erexample}
\end{figure}
}

\newcommand{\erpictureFolder}[0]{../SharedPictures}

\newcommand{\ercenterPicture}[1]{
\begin{center}
\input{\erpictureFolder/#1}
\end{center}
}


\newlength{\erhalfHt}

%\erinlinePicture{#1 pictureFilename}{#2 pictureHeight}
\newcommand{\erinlinePicture}[2]{
\setlength{\erhalfHt}{#2cm * \real{0.5}}
\raisebox{-\erhalfHt}[\erhalfHt + 0.5cm][\erhalfHt + 0.5cm]{
\input{\erpictureFolder/#1}
} 
}

%\erplainFig{#1 pictureFilename}{#2 figureParam}{#3Caption}
\newcommand{\erplainFig}[3]{
\begin{figure}[#2]
\begin{center}
\input{\erpictureFolder/#1}
\end{center}
\caption{#3}
\label{#1}
\end{figure}
}

%\erboxedFigPicture{#1 pictureFilename}{#2 figureParam}{#3Caption}
\newcommand{\erboxedFigPicture}[3]{
\begin{figure}[#2]
\begin{erexample}
\vspace{-0.5cm}
\begin{center}
\input{\erpictureFolder/#1}
\end{center}
\caption{#3}
\label{#1}
\end{erexample}
\end{figure}
}

%\erLeftSideFig{#1 pictureFilename}{#2 figureParam}{#3Caption}
\newcommand{\erLeftSideFig}[3]{
\begin{figure}[#2]
\begin{erexample}
  \begin{minipage}[c]{0.4\textwidth}
    \caption{#3}
    \label{#1}
  \end{minipage}
  \begin{minipage}[c]{0.5\textwidth}
    \input{\erpictureFolder/#1}
  \end{minipage}
\end{erexample}
\end{figure}
}

%\erbulletedFig{#1 pictureFilename}{#2 figureParam}{#3Caption}
\NewEnviron{erbulletedFig}[3]{%
\begin{figure}[#2]
\begin{erexample}
\vspace{-0.5cm}
\begin{center}
$
\begin{array}{c m{0.25cm} | m{6cm}}
\raisebox{-2.0cm}{
\input{\erpictureFolder/#1}}& & \text{\parbox{6cm}{\raggedright{\footnotesize{
\begin{enumerate}[(i)]
\BODY
\end{enumerate}}}}} \\
\end{array}
$
\end{center}
\caption{#3}
\label{#1}
\end{erexample}
\end{figure} 
}


%\begin{erbulletedDimFig}{#1 pictureFilename}{#2figureParam} {#3Caption} {#4PictureHeight}{#5TextWidth}

\NewEnviron{erbulletedDimFig}[5]{%
\begin{figure}[#2]
\begin{erexample}
\vspace{-0.5cm}
\begin{center}
$
\begin{array}{c m{0.25cm} |  m{#5cm}}
\setlength{\erhalfHt}{#4cm * \real{0.5}}
\raisebox{-\erhalfHt}{
\input{\erpictureFolder/#1}}& & \text{\parbox{#5cm}{\raggedright{\footnotesize{
\begin{enumerate}[(i)]
\BODY
\end{enumerate}}}}} \\
\end{array}
$
\end{center}
\caption{#3}
\label{#1}
\end{erexample}
\end{figure} 
}

%\begin{ernotedModel}{#1 pictureFilename}{#2PictureHeight}{#3PictureWidth}{#4TextWidth}

\NewEnviron{ernotedModel}[4]{%
\begin{center}
$
\begin{array}{m{#3cm} m{1cm} | c m{#4cm}}
\setlength{\erhalfHt}{#2cm * \real{0.5}}
\raisebox{-\erhalfHt}{
\input{\erpictureFolder/#1}}& & & \text{\parbox{#4cm}{\raggedright{\footnotesize{
\BODY
}}}} \\
\end{array}
$
\end{center} 
}

%\begin{ermodelText}{#1 pictureFilename}{#2PictureHeight}{#3PictureWidth}{#4TextWidth}

\NewEnviron{ermodelText}[4]{%
\begin{center}
\begin{tabular}{m{#3cm} m{1cm}  c m{#4cm}}
\setlength{\erhalfHt}{#2cm * \real{0.5}}
\raisebox{-\erhalfHt}{
\input{\erpictureFolder/#1}}& & & \text{\parbox{#4cm}{\raggedright{\small{
\BODY
}}}} \\
\end{tabular}
\end{center} 
}


%\erbulletedModel{#1 pictureFilename}{#2PictureHeight}{#3PictureWidth}{#4TextWidth}

\NewEnviron{erbulletedModel}[4]{%
\begin{center}
$
\begin{array}{m{#3cm} m{1cm} | c m{#4cm}}
\setlength{\erhalfHt}{2cm * \real{0.5}}
\raisebox{-\erhalfHt}{
\input{\erpictureFolder/#1}}& & & \text{\parbox{#4cm}{\raggedright{\footnotesize{
\begin{enumerate}[(i)]
\BODY
\end{enumerate}}}}} \\
\end{array}
$
\end{center} 
}



%\ernotedDimFig{#1 pictureFilename}{#2 figureParam}{#3Caption}{#4PictureHeight}{#5TextWidth}
\NewEnviron{ernotedDimFig}[5]{%
\begin{figure}[#2]
\begin{erexample}
\vspace{-0.5cm}
\begin{center}
$
\begin{array}{c m{0.25cm} | c m{#5cm}}
\setlength{\erhalfHt}{#4cm * \real{0.5}}
\raisebox{-\erhalfHt}{
\input{\erpictureFolder/#1}}& & & \text{\parbox{#5cm}{\raggedright{\footnotesize{
\BODY }}}}\\
\end{array}
$
\end{center}
\caption{#3}
\label{#1}
\end{erexample}
\end{figure} 
}
%\begin{ernotedDimFigPW}{#1 pictureFilename}{#2 figureParam}{#3Caption}{#4PictureHeight}{#5PictureWidth}{#6TextWidth}
\NewEnviron{ernotedDimFigPW}[6]{%
\begin{figure}[#2]
\begin{erexample}
\vspace{-0.5cm}
\begin{center}
$
\begin{array}{>{\centering}m{#5cm} m{0.5cm} | c m{#6cm}}
\setlength{\erhalfHt}{#4cm * \real{0.5}}
\raisebox{-\erhalfHt}{
\centering \input{\erpictureFolder/#1}
}& & & \text{\parbox{#6cm - 0.5cm}{\raggedright{\footnotesize{
\BODY }}}}\\
\end{array}
$ \\
\vspace {0.2cm}
\end{center}
\caption{#3}
\label{#1}
\end{erexample}
\end{figure}
}



\newenvironment{erquote}
{\begin{quote}\itshape}
{\end{quote}}


%
%  erdiagram.tex
%  *************
%  Macros to represent ER diagrams
%  *******************************
% 29/01/2019 Modify so that not reliant on the
%            default fontsize being 10pt by using
%            package anyfontsize and then
%            \fontsize{8}{10}\selectfont to set font to 8pt
% 06/02/2019 Pullback symbol implemented and minor tweaks to positioning 
%            and size of identifier symbol and relationship labels.
%            Accidental forked changes merged on 08/02/2019.
% 15/03/2019 Continuation of 29/01/2019. Need fix fontsize of 
%            ERrelname and ERscope.	 
% ***********************************************************
 \usepackage{anyfontsize}             % 29/01/2019 
  
%\begin{erdiagram}{#1 height}{#2 width} 
% ....
% ....
%\end{erdiagram}
\newenvironment{erdiagram}[2]
{%\pspicture*(-#1,0)(#2,0)
\pspicture*(0,-#1)(#2,0)
%\psgrid
}
{\endpspicture}

\definecolor{lightyellow}{cmyk}{0,0,0.3,0}
\definecolor{verylightgrey}{gray}{0.95}


% \eret{#1 x0} {#2 y0} {#3 x1} {#4 y1} {#5 corner radius} {#6 fill}
\newcommand {\eret}[6]
{ 
\ifthenelse{\equal{#6}{1}}
{\psframe[framearc=#5,fillstyle=solid,fillcolor=lightyellow](#1,#2)(#3,#4)}
{\psframe[framearc=#5,fillstyle=solid,fillcolor=white](#1,#2)(#3,#4)}
}

% et top 
\newcommand {\erettop}[4]
{
%\psframe[linestyle=none,linearc=2pt,cornersize=absolute,fillstyle=solid,fillcolor=lightyellow](#1,#2)(#3,#4)
\psline[linearc=2pt,fillstyle=none,fillcolor=lightyellow](#1,#4)(#1,#2)(#3,#2)(#3,#4)
}

% et bottom 
\newcommand {\eretbtm}[4]
{
%\psframe[linestyle=none,linearc=2pt,cornersize=absolute,fillstyle=solid,fillcolor=lightyellow](#1,#2)(#3,#4)
\psline[linearc=2pt,fillstyle=none,fillcolor=lightyellow](#1,#2)(#1,#4)(#3,#4)(#3,#2)
}

% et bottom left
\newcommand {\eretbl}[4]
{
%\psframe[linestyle=none,linearc=2pt,cornersize=absolute,fillstyle=solid,fillcolor=lightyellow](#1,#2)(#3,#4)
\psline[linearc=2pt,fillstyle=none,fillcolor=lightyellow](#1,#4)(#3,#4)(#3,#2)
}

% et middle left
\newcommand {\eretml}[4]
{
%\psframe[linestyle=none,linearc=2pt,cornersize=absolute,fillstyle=solid,fillcolor=lightyellow](#1,#2)(#3,#4)
\psline[linearc=2pt,fillstyle=none,fillcolor=lightyellow](#1,#2)(#3,#2)(#3,#4)(#1,#4)
}

% et top left
\newcommand {\erettl}[4]
{
%\psframe[linestyle=none,linearc=2pt,cornersize=absolute,fillstyle=solid,fillcolor=lightyellow](#1,#2)(#3,#4)
\psline[linearc=2pt,fillstyle=none,fillcolor=lightyellow](#1,#2)(#3,#2)(#3,#4)
}

% et bottom right
\newcommand {\eretbr}[4]
{
%\psframe[linestyle=none,linearc=2pt,cornersize=absolute,fillstyle=solid,fillcolor=lightyellow](#1,#2)(#3,#4)
\psline[linearc=2pt,fillstyle=none,fillcolor=lightyellow](#1,#2)(#1,#4)(#3,#4)
}

% et middle right
\newcommand {\eretmr}[4]
{
%\psframe[linestyle=none,linearc=2pt,cornersize=absolute,fillstyle=solid,fillcolor=lightyellow](#1,#2)(#3,#4)
\psline[linearc=2pt,fillstyle=none,fillcolor=lightyellow](#3,#4)(#1,#4)(#1,#2)(#3,#2)
}

% et top right
\newcommand {\erettr}[4]
{
\psline[linearc=2pt,fillstyle=none,fillcolor=lightyellow](#1,#4)(#1,#2)(#3,#2)
}

% \ergrp{#1 x0} {#2 y0} {#3 x1} {#4 y1} {#5 corner radius} {#6 fill}
% #5 corner radius is unused!
\newcommand {\ergrp}[6]
{ 
\ifthenelse{\equal{#6}{1}}
{\psframe[fillstyle=solid,fillcolor=verylightgrey](#1,#2)(#3,#4)}
{\psframe[fillstyle=solid,fillcolor=white](#1,#2)(#3,#4)}
}


% \ertext{#1 text}
% 15/03/2019
\newcommand {\erextrasmallitalictext}[1]
{\fontsize{7}{9}\selectfont \textit{#1}}

% 29/01/2019  
\newcommand {\ersmallitalictext}[1]
{\fontsize{8}{10}\selectfont \textit{#1}}

\newcommand {\ermediumitalictext}[1]
{\fontsize{10}{12}\selectfont \textit{#1}}

% \eretname {#1 x left of text} {#2 y top of text} {#3 text}
\newcommand {\olderetname}[3]
{
%shift down 0.1 for height of text the anchor at baseline (B)
\rput[bl]{0}(0,-0.1){\rput[Bl]{0}(#1,#2){\ersmallitalictext{#3}}}
}

% \errelarm {#1 x0} {#2 y0} {#3 x1} {#4 y1} {#5 ismandatory} {#6 isconstructed}
\newcommand {\errelarm}[6]
{
\ifthenelse{\equal{#6}{1}}
{
%%\psline[linewidth=0.5pt,linearc=.05,linestyle=dashed,dash=6pt 6pt]{-}(#1,#2)(#3,#4)}
\ifthenelse{\equal{#5}{1}}
{\psline[linewidth=1.5pt,linearc=.05,linecolor=lightgray]{-}(#1,#2)(#3,#4)}
{\psline[linewidth=1.5pt,linearc=.05,linecolor=lightgray,linestyle=dashed,dash=2pt 2pt]{-}(#1,#2)(#3,#4)}
}
{
\ifthenelse{\equal{#5}{1}}
{\psline[linewidth=0.9pt,linearc=.05]{-}(#1,#2)(#3,#4)}
{\psline[linewidth=0.9pt,linearc=.05,linestyle=dashed,dash=2pt 2pt]{-}(#1,#2)(#3,#4)}
}
}

% \errelangle {#1 x0} {#2 y0} {#3 x1} {#4 y1} {#5 x2} {#6 y2} {#7 ismandatory} {#8 isocnstructed}
\newcommand {\errelangle}[8]
{
\ifthenelse{\equal{#8}{1}}
{
%\psline[linewidth=0.5pt,linearc=.1,linestyle=dashed,dash=6pt 6pt]{-}(#1,#2)(#3,#4)(#5,#6)}
\ifthenelse{\equal{#7}{1}}
{\psline[linewidth=1.5pt,linearc=.05,linecolor=lightgray]{-}(#1,#2)(#3,#4)(#5,#6)}
{\psline[linewidth=1.5pt,linearc=.1,linecolor=lightgray,linestyle=dashed,dash=2pt 2pt]{-}(#1,#2)(#3,#4)(#5,#6)}
}
{
\ifthenelse{\equal{#7}{1}}
{\psline[linewidth=0.9pt,linearc=.1]{-}(#1,#2)(#3,#4)(#5,#6)}
{\psline[linewidth=0.9pt,linearc=.1,linestyle=dashed,dash=2pt 2pt]{-}(#1,#2)(#3,#4)(#5,#6)}
}
}

% \ercrowfoot {#1 x0} {#2 y0} {#3 x11} {#4 y11} {#5 x12} {#6 y12} {#7 x13} {#8 y13} {#9 isconstructed}
\newcommand {\ercrowfoot}[9]
{
\ifthenelse{\equal{#9}{1}}
{
\psline[linewidth=1.5pt,linearc=.05,linecolor=lightgray]{-}(#1,#2)(#3,#4)
\psline[linewidth=1.5pt,linearc=.05,linecolor=lightgray]{-}(#1,#2)(#5,#6)
\psline[linewidth=1.5pt,linearc=.05,linecolor=lightgray]{-}(#1,#2)(#7,#8)
}{
\psline[linewidth=0.9pt,linearc=.05]{-}(#1,#2)(#3,#4)
\psline[linewidth=0.9pt,linearc=.05]{-}(#1,#2)(#5,#6)
\psline[linewidth=0.9pt,linearc=.05]{-}(#1,#2)(#7,#8)
}
}


% \eridcomprel{#1 x1}{#2 x2}{#3 y}
\newcommand {\eridcomprel}[3]
{
\psline[linewidth=0.9pt](#1,#3)(#2,#3)
}

% \eridrefrel{#1 x}{#2 y1}{#3 y2}
\newcommand {\eridrefrel}[3]
{
\psline[linewidth=0.9pt](#1,#2)(#1,#3)
}

% \ertext {#1 x} {#2 y} {#3 text anchor} {#4 text}  PRIVATE
\newcommand {\ertext}[4]
{
\rput[B#3]{0}(#1,#2){\fontsize{8}{10}\selectfont #4}
}

% \eretname {#1 x} {#2 y} {#3 text anchor} {#4 text} 
\newcommand {\eretname}[4]
{
\ertext{#1}{#2}{#3}{#4}
}

% \errelname {#1 x} {#2 y} {#3 text anchor} {#4 text} 
\newcommand {\errelname}[4]
{
\rput[B#3]{0}(#1,#2){\erextrasmallitalictext{#4}}
}


% \erscope {#1 x} {#2 y} {#3 text anchor} {#4 text}  15 March 2019
\newcommand {\erscope}[4]
{
\rput[B#3]{0}(#1,#2){\erextrasmallitalictext{#4}}
}

% \erreletname {#1 x} {#2 y} {#3 text anchor} {#4 text}  15 March 2019
\newcommand {\erreletname}[4]
{
\rput[B#3]{0}(#1,#2){\fontsize{10}{12}\selectfont #4}
}

% \ergroupannotation {#1 x} {#2 y} {#3 text anchor} {#4 text}
\newcommand {\ergroupannotation}[4]
{
\ertext{#1}{#2}{#3}{#4}
}


% \errelseq {#1 x} {#2 y}
\newcommand {\erelseq}[2]
{
}
\newcommand {\erattrmarkermand}
{\fontsize{6}{8}\selectfont $\blacksquare$}
\newcommand {\erattrmarkeropt}
{\fontsize{6}{8}\selectfont \CIRCLE}
\newcommand {\erderattrmarkermand}
{\fontsize{6}{8}\selectfont $\square$}
\newcommand {\erderattrmarkeropt}
{\fontsize{8}{10}\selectfont $\circ$}

% \erattr {#1 x} {#2 y} {#3 ismandatory}{#4 idenitfying} {#5 text}
\newcommand {\erattr}[5]
{
\ifthenelse{\equal{#3}{1}}
{\rput[l]{0}(#1,#2){\erattrmarkermand \ersmallitalictext{\ifthenelse{\equal{#4}{0}}{\underline{#5}}{#5}}}}
{\rput[l]{0}(#1,#2){\erattrmarkeropt \ersmallitalictext{\ifthenelse{\equal{#4}{0}}{\underline{#5}}{#5}}}}
}

\newcommand {\erdattr}[5]
{
\ifthenelse{\equal{#3}{1}}
{\rput[l]{0}(#1,#2){\erderattrmarkermand \ersmallitalictext{\ifthenelse{\equal{#4}{0}}{\underline{#5}}{#5}}}}
{\rput[l]{0}(#1,#2){\erderattrmarkeropt \ersmallitalictext{\ifthenelse{\equal{#4}{0}}{\underline{#5}}{#5}}}}
}


% \erarc {#1 x0} {#2 y0} {#3 x1} {#4 y1} {#5 x2} {#6 y2} {#7 x3} {#8 y3}
\newcommand {\erarc}[8]
{
\psbezier[showpoints=false]{-}(#1,#2) (#3, #4)(#5,#6) (#7, #8)
}

% \erarc {#1 x0} {#2 y0} {#3 x1} {#4 y1} {#5 x2} {#6 y2} {#7 x3} {#8 y3}
\newcommand {\errelseq}[8]
{
\psbezier[showpoints=false]{-}(#1,#2) (#3, #4)(#5,#6) (#7, #8)
}
% \ertrace {#1 trace}   
\newcommand {\ertrace}[1]
{
}

\usepackage{amsthm} % added 7th April 2018
% theorems.macros.tex

\newtheorem{theorem}{Theorem}[section]
\newtheorem{observation}[theorem]{Observation}
\newtheorem{lemma}[theorem]{Lemma}
\newtheorem{proposition}[theorem]{Proposition}
\newtheorem{corollary}[theorem]{Corollary}
\newtheorem{conjecture}[theorem]{Conjecture}
\newtheorem{numbereddefinition}[theorem]{Definition}

\newenvironment{definition}[1][Definition]{\begin{trivlist}
\item[\hskip \labelsep {\bfseries #1}]}{\end{trivlist}}
\newenvironment{examples}[1][Examples]{\begin{trivlist}
\item[\hskip \labelsep {\bfseries #1}]}{\end{trivlist}}
\newenvironment{example}[1][Example]{\begin{trivlist}
\item[\hskip \labelsep {\bfseries #1}]}{\end{trivlist}}
\newenvironment{remark}[1][Remark]{\begin{trivlist}
\item[\hskip \labelsep {\bfseries #1}]}{\end{trivlist}}

\newenvironment{tageqn}[1]
{
\begin{equation}
\stepcounter{equation}
\label{#1}
\tag{\theequation --#1}
}
{
\end{equation}
}

\newenvironment{axiom}[1]
{
\begin{equation}
\label{#1}
\tag{#1}
}
{
\end{equation}
}

% when the tag is required different from the label eg when has math symbols can use:
\newenvironment{axiomtagged}[2]
{
\begin{equation}
\label{#1}
\tag{#2}
}
{
\end{equation}
}

%visible label
\newcommand{\vlabel}[2][]{\label{#2}#1(\textit{#2}):}





\usepackage{imakeidx}
\usepackage{framed}
\makeindex[name=definitions, title=Index of Definitions]
\makeindex[name=lemmas, title=Index of Lemmas]



\newcommand{\seenudgeup}[1]{\rule{0.1cm}{#1}}

\newcommand{\seenudgedown}[1]{\rule[-#1]{0.1cm}{0.1cm}}

\newcommand{\nudgeup}[1]{\rule{0cm}{#1}}

\newcommand{\nudgedown}[1]{\rule[-#1]{0cm}{0.1cm}}

\definecolor{highlight}{cmyk}{0,0,0.7,0}
\newcommand{\commentary}[1]{\marginpar{\footnotesize #1}}
\newcommand{\highlight}[1]{\colorbox{highlight}{#1}}
\newcommand{\whitelight}[1]{\colorbox{white}{#1}}
\newcommand{\term}[1]{\textit{#1}\commentary{\colorbox{lightgray}{\textit{#1}}}\index[definitions]{#1}}
\newcommand{\llabel}[1]{\label{#1}\commentary{\colorbox{pink}{\scriptsize{#1}}}\index[lemmas]{#1}}
\newcommand{\lref}[1]{\ref{#1}\colorbox{pink}{\scriptsize{#1}}\index[lemmas]{#1!use of}}

\newcommand{\daynote}[1]{\commentary{See day notes #1.}}

\newcommand{\newt}[1]{\colorbox{yellow}{#1}}
\newenvironment{newtt}
{  \colorbox{yellow}{$[$ ...} 
}
{  \colorbox{yellow}{... $]$}
}
\newcommand{\oldt}[1]{\colorbox{red}{\sout{#1}}}
\newenvironment{oldtt}
{  \colorbox{red}{$[$ ...} 
}
{  \colorbox{red}{... $]$}
}

\newcommand{\reinstatet}[1]{\colorbox{lime}{#1}}
\newenvironment{reinstatett}
{  \colorbox{lime}{$[$ ...}
}
{  \colorbox{lime}{... $]$}
}

\newcommand{\tbd}{\highlight{TBD}}

%ithprojection function
\newcommand{\proji}[1]{\pi_#1}


\newenvironment{aside}
{\begin{framed}
\textbf{Aside}
}
{
\end{framed}
}

\newenvironment{notebox}[1][Note]
{\begin{framed}
\textbf{#1}
}
{
\end{framed}
}

\newenvironment{categoricalaside}
{\begin{framed}
\textbf{Categorical Aside}
}
{
\end{framed}
}

\newenvironment{noteforfuture}
{\begin{framed}
\textbf{Note For Future}
}
{
\end{framed}
}

\newenvironment{problem}
{\begin{framed}
\textbf{Problem}
}
{
\end{framed}
}

\newenvironment{key}
{
\begin{tabular}{c l p{4cm}}
KEY && \\
\hline
}
{
\end{tabular}
}

\newcommand{\keyentry}[3]{#1 & #2 & #3 \\} 


%quine quote
\newcommand{\qq}[1]{
\left\ulcorner#1\right\urcorner
}

%single quote
\newcommand{\sq}[1]{
\textnormal{\textquotesingle}#1\textnormal{\textquotesingle}
}

%lower quine quote
\newcommand{\lqq}[1]{
\left\llcorner #1\right\lrcorner
}


%from berkley
\newcommand{\langl}{\begin{picture}(4.5,7)
\put(1.1,2.5){\rotatebox{60}{\line(1,0){5.5}}}
\put(1.1,2.5){\rotatebox{300}{\line(1,0){5.5}}}
\end{picture}}
\newcommand{\rangl}{\begin{picture}(4.5,7)
\put(.9,2.5){\rotatebox{120}{\line(1,0){5.5}}}
\put(.9,2.5){\rotatebox{240}{\line(1,0){5.5}}}
\end{picture}}
\newcommand{\lang}{\begin{picture}(5,7)\put(1.1,2.5){\rotatebox{45}{\line(1,0){6.0}}}\put(1.1,2.5){\rotatebox{315}{\line(1,0){6.0}}}\end{picture}}
\newcommand{\rang}{\begin{picture}(5,7)\put(.1,2.5){\rotatebox{135}{\line(1,0){6.0}}}\put(.1,2.5){\rotatebox{225}{\line(1,0){6.0}}}\end{picture}}
%Try sharper tuple brackets -- except gives errors nested in captions so comment out
%\renewcommand{\tuple}[1]{\lang #1 \rang}

\newcommand{\setsuchthat}[2]{\left\{#1 \ \middle|\ #2\right\}}
\newcommand{\set}[1]{\left\{#1\right\}} 

% one to n - wanton
\newcommand{\wanton}[1]{#1_1,...#1_n}
\newcommand{\n}{1...n}
\newcommand{\fn}{\wanton{f}}
\newcommand{\gn}{\wanton{g}}
\newcommand{\pn}{\wanton{p}}
\newcommand{\qn}{\wanton{q}}
\newcommand{\qnprime}{\wanton{q'}}
\newcommand{\tn}{\wanton{t}}
\newcommand{\xn}{\wanton{x}}
\newcommand{\xnp}{\wanton{x'}}
\newcommand{\yn}{\wanton{y}}
\newcommand{\An}{\wanton{A}}
\newcommand{\Bn}{\wanton{B}}
\newcommand{\Cn}{\wanton{C}}
\newcommand{\ntuple}[1]{\tuple{\wanton{#1}}}
\newcommand{\wantom}[2][]{#2_1,...#2_{m#1}}
\newcommand{\m}{1...m}
\newcommand{\mtuple}[1]{\tuple{#1_1,...#1_m}}
\newcommand{\gm}{\wantom{g}}
\newcommand{\qm}{\wantom{q}}
\newcommand{\sm}[1][]{\wantom[#1]{s}}
\newcommand{\smp}{\wantom{s'}}
\newcommand{\ym}{\wantom{y}}
\newcommand{\Bm}{\wantom{B}}
\newcommand {\bntuple}{\ensuremath{\ntuple{b}}}
\newcommand {\fntuple}{\ensuremath{\ntuple{f}}}
\newcommand {\fnptuple}{\ensuremath{\ntuple{f}}}
\newcommand {\pntuple}{\ensuremath{\ntuple{p}}}
\newcommand {\qntuple}{\ensuremath{\ntuple{q}}}
\newcommand {\qnptuple}{\ensuremath{\ntuple{q'}}}
\newcommand {\qmtuple}{\ensuremath{\mtuple{q}}}
\newcommand {\sntuple}{\ensuremath{\ntuple{s}}}
\newcommand {\xntuple}{\ensuremath{\ntuple{x}}}
\newcommand {\xnptuple}{\ensuremath{\ntuple{x'}}}
\newcommand {\ymtuple}{\ensuremath{\mtuple{y}}}
\newcommand{\idef}[1][n]{1 \leq i \leq #1}
\newcommand{\jdef}[1][m]{1 \leq j \leq #1}
\newcommand{\kdef}[1][l]{1 \leq k \leq #1}
\newcommand{\foreachi}[1][n]{for each $i$, $1 \leq i \leq #1$}
\newcommand{\foreachj}[1][m]{for each $j$, $1 \leq j \leq #1$}
\newcommand{\foreachk}[1][l]{for each $k$, $1 \leq k \leq #1$}
\newcommand{\Foreachi}[1][n]{For each $i$, $1 \leq i \leq #1$}
\newcommand{\Foreachj}[1][m]{For each $j$, $1 \leq j \leq #1$}
\newcommand{\Foreachk}[1][l]{For each $k$, $1 \leq k \leq #1$}
\newcommand{\forsomei}[1][n]{for some $i$, $1 \leq i \leq #1$}
\newcommand{\forsomej}[1][m]{for some $j$, $1 \leq j \leq #1$}
\newcommand{\forsomek}[1][l]{for some $k$, $1 \leq k \leq #1$}
\newcommand{\wherei}[1][n]{where $1 \leq i \leq #1$}
\newcommand{\wherej}[1][m]{where $1 \leq j \leq #1$}
\newcommand{\wherek}[1][l]{where $1 \leq k \leq #1$}


\newcommand{\fundep}[3]{#2 \xrightarrow{#1} #3}  %where does this belong? xxxx
% Following used for notes -- indented numbered paras

\newcounter{para}
\newlength{\oldparindent}
\setlength{\oldparindent}{\parindent} % Save \parindent before of change
\newcommand{\ind}{\hspace*{\oldparindent}}
\newcommand\note{
%\setlength{\parskip}{0.5\baselineskip} % Definition of `parskip`
\setlength{\parindent}{0pt}
\par\ind\refstepcounter{para}\thepara.\space
\setlength{\parindent}{\oldparindent}
}



\usepackage{mathptmx}  % This changes font to roman
\usepackage{anyfontsize}
\usepackage{mathtools}  % why have we got this?
\usepackage{alltt}    
\usepackage{mnsymbol} %used for rightpitchfork
\usepackage{cmll}
\usepackage{ulem}
\renewcommand{\ttdefault}{txtt}
\usepackage[left=1.5cm, right=4cm, marginparwidth=3cm, top=2cm, bottom=2.0cm]{geometry}
\usepackage{framed}
\usepackage[font=small]{caption}
\usepackage{changepage}

\usepackage{hyperref}
\setlength{\captionmargin}{2cm}
\theoremstyle{remark}
\newtheorem*{lemma*}{Lemma}

% have boxed figures
\usepackage{float}
\floatstyle{ruled} 
\restylefloat{figure}

\newtheorem*{lemmastar}{Lemma}

% from presentation
\newcommand{\attr}[1]{#1}
\renewcommand{\attr}[1]{\psframebox[linecolor=red,framearc=.1]{#1}}
\newcommand{\attrtype}[1]{#1}
\renewcommand{\attrtype}[1]{\psframebox[linecolor=blue,framearc=.1]{#1}}
\newcommand{\etype}[1]{#1}
\renewcommand{\etype}[1]{\psframebox[linecolor=red,framearc=.1]{#1}}

\NewEnviron{tightquote}
{\begin{adjustwidth}{1.5cm}{1.5cm}
\textit{
\BODY
}
\end{adjustwidth}
}

\NewEnviron{thus}
{\begin{adjustwidth}{1.5cm}{1.5cm}
\BODY
\end{adjustwidth}
}

\usepackage{amsmath}
\mathchardef\mhyphen="2D % Define a "math hyphen"

\newcommand{\Ualg}{U \mhyphen alg}
\newcommand{\Upalg}{U' \mhyphen alg}
\newcommand{\Ialg}{I \mhyphen alg}
\newcommand{\catofccs}{\mathbf{Concat}}
\newcommand{\catoflargerccs}{\mathbf{CONCAT}}
%\newcommand{\alg}[1]{{#1}_{alg}}
\newcommand{\alg}[1]{alg(#1)}
\newcommand{\cc}[1]{cc(#1)}
\newcommand{\tccalgebra}{$tcc$-algebra\ }
\newcommand{\tccalgebras}{$tcc$-algebras\ }
\newcommand{\FAM}{\ensuremath{\mathbb{F\mathrm{AM}} }}

\def\therefore{\boldsymbol{\text{ }
\leavevmode
\lower0.4ex\hbox{$\cdot$}
\kern-.5em\raise0.7ex\hbox{$\cdot$}
\kern-0.55em\lower0.4ex\hbox{$\cdot$}
\thinspace\text{ }}}


\NewEnviron{thusly}
{\begin{adjustwidth}{1.5cm}{1.5cm}
$\therefore\ $
\BODY
\end{adjustwidth}
}

\NewEnviron{point}
{\begin{adjustwidth}{1.5cm}{1.5cm}
\textbullet\ 
\BODY
\end{adjustwidth}
}

\NewEnviron{pointeq}
{\begin{equation}
\mbox{\textbullet\ 
\BODY}
\end{equation}
}



\begin{document}

\title{Notes followed by comments regarding inititiality conjecture}

\author{John Cartmell}

\maketitle

\newcommand{\gatU}{\gat[U]}
\newcommand{\gatUw}{\gatU\ }
\newcommand{\CofU}{\ccat[C](\gat[U])}
\newcommand{\KU}{K_{\gat[U]}}
\newcommand{\KUp}{K_{\gat[U']}}
\newcommand{\catCon}{\cat{Con}}
\newcommand{\catGAT}{\cat{GAT}}

\newcommand{\inlinedisplay}[1]
{
\setlength{\fboxsep}{1.5pt}
\setlength{\fboxrule}{0pt}
\fbox{$\displaystyle #1$}
}

\newcommand{\gatdisplayrule}[2]
{
\setlength{\fboxsep}{1pt}
\setlength{\fboxrule}{0pt}
\fbox{$\displaystyle \frac{#1}{#2}$}
}
\newcommand{\Isort}{I_{sort}}
\newcommand{\Iop}{I_{op}}
\newcommand {\Ihat}{\hat{I}}

% workaround when used with Rnode for size of font used under the cross
\renewcommand{\crossx}[3]{#1 \underset{\tiny #3}{\cross} #2}
\newcommand{\fonestar}   {{f_1}\kern-.15em^*}
\newcommand{\ftwostar}   {{f_2}\kern-.15em^*}
\newcommand{\fjstar}     {{f_j}\kern-.2em^*}
\newcommand{\fjpstar}    {{f_{j-1}}\kern-.25em^*}
\newcommand{\smstar}{{s_m}\kern-.25em^*}
\newcommand{\sonestar}{{s_1}\kern-.15em^*}

\newcommand{\Trule} {T-rule\ }
\newcommand{\trule} {$\in$-rule\ }
\newcommand{\Teqrule} {T=-rule\ }
\newcommand{\teqrule} {$\in=$-rule\ }

\newcommand{\genericAintroductoryrule} {\gatdisplayrule{\xDelta{n}}{\isT{A(\xn}}}
\newcommand{\genericfintroductoryrule}  {\gatdisplayrule{\xDelta{n}}{\ofT{f(\xn)}{\Delta}}}


Notes on  generalised algebraic theories, contextual categories, and, the interpretation of the former in the latter. Following this some comments on the draft paper of Peter Dybjer et al. 

\section{Note concerning Generalised Algebraic Theories}
%\note
We give the broadest possible notion of a model
of a generalised algebraic theory \gat[U] by defining the notion of an interpretation of  \gat[U] in  any given contextual category \catc.

\note
First, as a stepping stone, we need  give an auxillary definition, the definition of  a `preinterpretation of \gat[U] in \catc'.

\newcommand{\Isort}{I_{sort}}
\newcommand{\Iop}{I_{op}}
\note 
If \gat[U] is a generalised algebraic theory  and if \catcw is a contextual category then
a preinterpretation $I$ of  \gat[U] in \catcw consists of a pair :
\begin{itemize}
\item a mapping $\Isort$ that maps each sort symbol of \gat[U] to  an object of \catc,
\item a mapping $\Iop$ that maps each operator symbol of \gat[U] to a section of \catcw (i.e. to a morphism $f: A \morph B$ for some 
$A \base B$ in \catcw such that $f \circ p_B=id_A$).
\end{itemize}

\note 
An interpretation $I$ of \gat[U] in \catcw is a preinterpretation satisfying additional conditions and these 
additional conditions imply that the preinterpretation induces a mapping $\hat{I}$
of derived T- and $\epsilon$- rules of \gat[U] to objects, respectively sections, of \gat[U] such that 
all equational identities of \gat[U] are respected in the sense that :
\begin{itemize}
\item whenever 
$\frac{\context{x}{\Delta}{n}}{t = t' \in \Delta}$
is an axiom/derived rule of \gat[U] then $\hat{I}(R) = \hat{I}(R')$
where $R$ is the rule
$\frac{\context{x}{\Delta}{n}}{\ofT{t}{\Delta}}$
and $R'$ is the rule
$\frac{\context{x}{\Delta}{n}}{\ofT{t'}{\Delta}}$,
\item whenever
$\frac{\context{x}{\Delta}{n}}{\Delta = \Delta'}$
is an axiom/derived rule of \gat[U] then $\hat{I}(R) = \hat{I}(R')$
where $R$ is the rule
$\frac{\context{x}{\Delta}{n}}{\isT{\Delta}}$
and $R'$ is the rule
$\frac{\context{x}{\Delta}{n}}{\isT{\Delta}}$.
\end{itemize}

\section{Note concerning Contextual Categories}
%%contextualnotation
\label{contextualnotation}

\note
The original definition of contextual category given  in [1] and [2], a contextual category is defined to be a tree-structured category 
\cat{C} with the following additional structure:

\noindent 
(i) whenever
$
\begin{array}{cp{.9cm}c}
            & & \Rnode{z}{z} \\ [1.2cm]
\Rnode{x}{x}& & \Rnode{y}{y} \\ [0.5cm]
\end{array}
$
\jcbarr{f}{x}{y}
\ncasar{p_z}{z}{y}

in \cat{C}, an object $f \sub z$ such that $x \base f \sub z$, a morphism $q(f,z): f \sub z \rightarrow z$ such that

\begin{axiom}{q1}
q(f,z) \circ p_z = p_{f \sub z} \circ f
\end{axiom}

i.e. such that the diagram: 
$$
\ccsquareoutline{0.9cm}{1.2cm}{f^*z}{z}{x}{y}
\ccsquareacross{q(f,z)}{f}
\ccsquaredown{p_{f \sub z}}{p_z}
$$
commutes, 

\noindent
and, (ii), so that each such diagram is a pullback diagram, that is: for all objects $w$ of \cat{C}, and for all
morphisms $h_1: w \rightarrow x$ and $h_2: w \rightarrow z$ (see diagram \ref{pullback} below) such that
$h_1 \circ f = h_2 \circ p_z$ 
there exists a unique $h:w \rightarrow f \sub z$ in \cat{C} such that
$h \circ p_{f \sub z} = h_1$ and $h \circ q(f,z) = h_2$, as shown here:

\vspace{3mm}
\begin{center}
\begin{equation}
\label{pullback}
\begin{array}{cp{0.5cm}cp{1.2cm}c}
\Rnode{w}{w} &&                     &&           \\ [0.7cm]
             &&\Rnode{fstarz}{f^*z} && \Rnode{z}{z}\\ [1.2cm]
             &&\Rnode{x}{x}         && \Rnode{y}{y}
\end{array}
\end{equation}
\ncbsar{p_{f \sub z}}{fstarz}{x}
\jcbarr{f}{x}{y}
\ncaarr{q(f,z)}{fstarz}{z}
\ncasar{p_z}{z}{y}
\setlength{\arrnodesepA}{3pt}
\jcbarr[-35]{h_1}{w}{x}
\ncaarr[35]{h_2}{w}{z}
\psset{linestyle=dashed}
\ncaarr{h}{w}{fstarz}
\end{center}

\vspace {0.25cm}
\noindent and so that (iii) whenever $x \base y$ in \cat{C}, 
\begin{axiom}{q2}
id_x^*y=y
\end{axiom}

and

\begin{axiom}{q3}
q(id_x,y) = id_y
\end{axiom}



\noindent and (iv) whenever 
$
\begin{array}{c p{.9cm} c p{.9cm} c}
             &   &             &   & \Rnode{z}{z} \\ [1.2cm]
\Rnode{w}{w} &   &\Rnode{x}{x} &   & \Rnode{y}{y} \\ [0.5cm]
\end{array}
$
\jcbarr{f}{w}{x}
\jcbarr{g}{x}{y}
\ncasar{c}{z}{y}
in \cat{C}, 

then

\begin{axiom}{q4}
(f \circ g)^*z =  f^* (g ^* z)
\end{axiom}

and 
\begin{axiom}{q5}
q(f \circ g,z) = q(f,g^*z) \circ q(g,z)
\end{axiom}



\note
Following Voevodsky we may replace the pullback condition of the original definition by an 
`s' operator along with axioms as follows:

\noindent (ii') for all morphisms $f: x \rightarrow y$, a morphism $s(f) : x \rightarrow f \sub p_y \sub y$ such that both:

\begin{axiom}{s1}
s(f) \circ p_{f\sub p_y \sub y}=id_x
\end{axiom}

\noindent and

\begin{axiom}{s2}
s(f) \circ q( f \circ p_y     ,y)=f
\end{axiom}	

\noindent i.e. such that the following diagrams commute:
\begin{center}
\begin{displaymath}
\begin{array}{cccp{1.cm} cp{.9cm}c}
&\Rnode{fXyyM}{f\sub p_y \sub y}&  & &  \Rnode{fXyy}{f\sub p_y \sub y} & & \Rnode{yXy}{p_y \sub y}\\ [1.2cm]
\Rnode{xL}{x} & &\Rnode{xR}{x} & &\Rnode{x}{x}         & & \Rnode{y}{y}
\end{array}
\end{displaymath}
\ncasar{p_{f\sub p_y \sub y}}{fXyy}{x}
\jcbarr{f}{x}{y}
\ncaarr{q(f,p_y \sub y)}{fXyy}{yXy}
\ncasar{p_{p_y \sub y}}{yXy}{y}
\ncaarr{s(f)}{xL}{fXyyM}
\ncasar{p_{f\sub p_y \sub y}}{fXyyM}{xR}
\jcbarr{id_x}{xL}{xR}
\end{center}

\noindent
and such that whenever

\begin{center}
\begin{displaymath}
\begin{array}{c p{.9cm} c p{.9cm} c}
\Rnode{w}{w}&& \Rnode{g*z}{g \sub z} && \Rnode{z}{z} \\ [1.2cm]
            && \Rnode{x}{x}  && \Rnode{y}{y} \\ [0.2cm]
\end{array}
\end{displaymath}
\jcbarr{f}{w}{g*z}
\jcbarr{g}{x}{y}
\ncaarr{q(g,z)}{g*z}{z}
\ncasar{}{g*z}{x}
\ncasar{}{z}{y}
\end{center}

\noindent in \cat{C} then

\begin{axiom}{s3}
s(f \circ q(g,z))=s(f)
\end{axiom}



\note
The contextual category structure supplies us with pullbacks for any $\smorph$ morphism, 
these given pullbacks can be pieced together to obtain a pullback for
any $\dmorph$ morphism  along any morphism with the same codomain. 

In general, whenever $y \leq z$ in $C$ and whenever $f:x \morph y$ in $C$, then we have
the following canonical pullback for the morphism $p_{z, y}$ along $f$, where
$w_1, ... w_n$ is the unique sequence of objects of $C$ such that 
$y \base w_1 \base ... \base w_n \base z$:

\vspace{3mm}
\begin{center}
\begin{equation}
\label{compositepullbackdefinition}
\begin{array}{cp{2.9cm}c}
\Rnode{TOPL}{q(...q(f, w_1)...w_n)^* z} & & \Rnode{TOPR}{z}\\ [1.2cm]
\Rnode{zOTTOML}{x}         & & \Rnode{zOTTOMR}{y}
\end{array}
\end{equation}
\jcbarr{f}{zOTTOML}{zOTTOMR}
\ncaarr{q(q(...q(f,w_1)...w_n),z)}{TOPL}{TOPR}
\nccdar{TOPL}{zOTTOML}
\blabel{p_{q(...q(f, w_1)...w_n)^* z,x}}
\nccdar{TOPR}{zOTTOMR}
\alabel{p_{z,y}}
\end{center}

Since these constructed pullbacks form an important part of contextual
category structure we would like a simpler notation for them. As no confusion is
likely, we extend the $^*$ and $q$ notation to cover these new pullback diagrams.
From now on if $f:x \morph y$ in $C$ and $y \leq z$ in $C$, then the diagram

\vspace{3mm}
\begin{center}
\begin{equation}
\label{compositepullbackout}
\begin{array}{cp{.9cm}c}
\Rnode{fstarz}{f^*z} & & \Rnode{z}{z}\\ [1.2cm]
\Rnode{x}{x}         & & \Rnode{y}{y}
\end{array}
\end{equation}
\nccdar{fstarz}{x}
\blabel{p_{f \sub z},x}
\jcbarr{f}{x}{y}
\ncaarr{q(f,z)}{fstarz}{z}
\nccdar{z}{y}
\alabel{p_{z,y}}
\end{center}
is the canonical pullback diagram as defined in (\ref{compositepullbackdefinition}) above. 
\note
The following observation
follows from the way the extended pullback diagrams are constructed. 
In the extended notation, if $f: x \morph y$ 
and $y \leq z \leq zz$ in the contextual category C, then
\begin{equation}
f^*zz = q(f, z)^*zz
\end{equation}
 and 
\begin{equation}
q(f, zz) = q(q(f, z), zz)
\end{equation}
and so the outer diagram in
\renewcommand{\pc}[2]{p_{#1,#2}}  % as \pc defined in ccategories macros differently to this
$
\begin{array}{ccp{.9cm}c}
\\[0.25cm]
&\Rnode{TL}{q(f,z)^*zz} & & \Rnode{TR}{zz}\\ [1.2cm]
&\Rnode{ML}{f^*z} & & \Rnode{MR}{z}\\ [1.2cm]
&\Rnode{BL}{x}         & & \Rnode{BR}{y} \\[1.0cm]
\end{array}
$
%composition
\makebox[0.2cm]{   % This make box prevents white space pushing out to the right
                   % cannot see where this white space is comin from. To investigate
									 % change the \makebox[0.2cm] to \fbox and you will see the problem.
\nccdar{TL}{ML}\blabel{P_{q(f,z)^*zz,f^*z}}\nccdar{ML}{BL}\blabel{p_{f \sub z,x}}\nccdar{TR}{MR}\alabel{p_z}
\nccdar{MR}{BR}
}
\alabel{p_z}
%reference
\ncarr{TL}{TR}
\alabel{q(q(f,z),zz)}
\ncarr{ML}{MR}
\alabel{q(f,z)}
\ncarr{BL}{BR}
\blabel{f}
is diagram (\ref{compositepullbackout}). 
On such diagrams as shown here the $p$ labels on $\dmorph$ morphisms (inclusive of their indices) are entirely predictable  and so in diagrams that follows we may omit them.


\note
If we write $\crossx{y}{z}{x}$ in place of ${p_{y,x}}^*z$, for $x < y$, $x < z$  in \ccat then
$\crossx{y}{z}{x}$  represents  in the syntax the `weakening' of a rule of the form
\begin{displaymath}
x, w1,...w_n \tstyle \isT{z}
\end{displaymath}
from a rule with context $x, w_1,...w_n$ to a rule with broader context $y, w_1, ... w_n$: 
\begin{displaymath} 
y, w_1,...w_n \tstyle \isT{z}.
\end{displaymath}

Within the contextual category I think of $\crossx{y}{z}{x}$  as a local cartesian product but of course categorically it is a filtered product i.e. a pullback. If $w < x$ and $w < y$  then 
\genericcrossxproductdiagram % defined in 'paper.tex'
is a pullback diagram in \ccat.

\note
We can extend the $\crossx{}{}{w}$ notation to morphisms. If $f:x \morph x'$ and $g: y \morph y'$ in a contextual
category $\ccat[C]$ and if $w$ is an object such that $w < x$, $w <x'$, $w < y$ and $w < y'$ then 
define $\crossx{f}{g}{w}:\crossx{x}{y}{w} \morph \crossx{x}{y}{w}$ in $\ccat[C]$ by
\begin{equation}
\crossx{f}{g}{w} = \tuple{p_{\crossx{x}{y}{w}, x} \circ f,q(p_{x,w},y) \circ g}
\end{equation}  

\note
In the case special case that $w$ is the terminal object $1$ then the pullback  specialises to give a product diagram:

\begin{displaymath}
\begin{array}{ccccc}
\Rnode{xy}{\crossx{x}{y}{1}} &&               &&               \\[1.3cm]
\Rnode{x}{x}                 &&               && \Rnode{y}{y}  \\                                    
\end{array}
\mbox{\ncsar{xy}{x}
\blabel{p_{\crossx{x}{y}{1},x}}
\ncaarr{q(p_{x,1},y)}{xy}{y}}
\end{displaymath}

In this special case the $\tuple{}$ operation defined earlier is the pairing operation for if
$f: w \morph x$ and $g: \morph y$ then $\tuple{f,g}: w \morph \crossx{x}{y}{1}$ 
and 
\begin{equation}
\tuple{f,g} \circ p_{\crossx{x}{y}{1},x} = f
\end{equation}
and
\begin{equation}
\tuple{f,g} \circ q(p_{x,1},y) = g
\end{equation}

\note 
Note that the product operation $\crossx{}{}{1}$ is far from symmetric 
because if, for example, $1 \base x$ and $1 \base y$ then $x \base \crossx{x}{y}{1}$ and $y \base \crossx{y}{x}{1}$ but we can define 
a swap operation $sw_{x,y} : \crossx{x}{y}{1} \morph \crossx{y}{x}{1}$ by
\begin{equation}
sw_{x,y} = \tuple{q_{p_x,y}, p_{\crossx{x}{y}{1},x}}
\end{equation}

\note
Associativity of $\crossx{}{}w$  follows from the coherence property of the pullbacks in the contextual category. 
For example if $w < x$, $w < y$, $w < z$ in a \ccat then from coherence of pullbacks in \ccat we have:
$\crossx{x}{(\crossx{y}{z}{w})}{w} = \crossx{(\crossx{x}{y}{w})}{z}{w}$ as shown here in this diagram:
 
\begin{displaymath}
\begin{array}{cp{1.0cm}cp{1.0cm}c}
\Rnode{J1}{}\Rnode{D1} {\crossx{(\crossx{x}{y}{w})}{z}{w}}\Rnode{J2}{} \ \ \ \ \   &&  &&  \\ 
= && && \\
\Rnode{D2} {\crossx{x}{(\crossx{y}{z}{w})}{w}}    &&  &&                        \\ [1.3cm]
\Rnode{xy}{\crossx{x}{y}{w}}&& \Rnode{yz}{\crossx{y}{z}{w}} &&                      \\[1.3cm]
\Rnode{x}{x}&& \Rnode{y}{y} && \ \ \ \ \ \ \ \ \ \ \ \ \ \Rnode{z}{z}                                        \\[1.3cm]
             && \Rnode{w}{w} &&                                                     
\end{array}
\end{displaymath}

\ncaarr[50]{q(\pc{\crossx{x}{y}{w}}{w},z)}{J2}{z}
\ncsar{D2}{xy}
\ncsar{xy}{x}
\ncsar{yz}{y}
\ncsar{x}{w}
\ncsar{y}{w} 
\ncsar{z}{w}
\ncaarr{q(\pc{x}{w},y)}{xy}{y}
\ncaarr{q(\pc{y}{w},z)}{yz}{z}
\ncaarr{q(\pc{x}{w},\crossx{y}{z}{w})}{D2}{yz}



\note A number of minor lemmas:

\begin{lemma}
\llabel{footandstactic}
If $f: A \morph B$ and $f':A \morph B$ in a contextual category \catcw then if 
$f \circ p_B$ = $f' \circ p_B$ and $s(f) = s(f')$ then $f=f'$.
\end{lemma}
\begin{proof}
Follows by axiom (s2) since we have:
$f = s(f) \circ q(f \circ p_B,B)  = s(f') \circ q(f' \circ p_B,B) = f'$.
\end{proof}

From which follows:
\begin{lemma}
\llabel{stactic}
If $A$ is any object of a contextual category \catcw and if $B$ is an object such that $1 \base B$ in \catcw then
if $f: A \morph B$ and $f':A \morph B$ in a contextual category \catcw then $f=f'$ iff $s(f) = s(f')$.
\end{lemma}

\begin{lemma}
\llabel{crosssectionlemma}
If 
%\begin{equation*}
$
\begin{array}{ c c c}
\Rnode{B}{B} &              & \Rnode{Bp}{B'} \\[1cm]
             & \Rnode{A}{A} &     
\mbox{\ncsar{B}{A}
\ncsar{Bp}{A}
%\ncarr[-30]{A}{Bp}
\ncrightsimplesection{A}{Bp}
\blabel{g}}
\end{array}
$
%\end{equation*}
in a contextual category \catcw and if $g$ is a section of $B'$ (i.e. if $g \circ p_{B'}= id_A$) so that we have 
\begin{equation*}
\begin{array}{ c c c}
\Rnode{BBp}{\crossx{B}{B'}{A}} \\[1.3cm]
\Rnode{B}{B} &              & \Rnode{Bp}{B'} \\[1.1cm]
             & \Rnode{A}{A} &
\mbox{
\ncsar{BBp}{B}
\ncrightcrosssection{B}{BBp}
\blabel{\crossx{B}{g}{A}}
\ncsar{B}{A}
\blabel{p_B}
\ncsar{Bp}{A}
\ncrightsimplesection{A}{Bp}
\blabel{g}
}														
\end{array}
\end{equation*}
in \catcw,  then
\begin{equation}
\label{crosssectionlemmatarget}
\crossx{B}{g}{A} = s(p_B \circ g).
\end{equation} 
\end{lemma}
\begin{proof}
$\crossx{B}{g}{A}$ is defined to be the unique section of $\crossx{B}{B'}{A}$ such that $(\crossx{B}{g}{A}) \circ q( p_B,B') = p_B \circ g$.

(\ref{crosssectionlemmatarget}) follows because $s(p_B \circ g)$ is also such a section, since it is defined to be the unique section of $(p_B \circ g \circ p_{B'}) ^* B'$
such that $s(p_B \circ g) \circ q( p_B \circ g \circ p_{B'}, B') = p_B \circ g$ and this simplifies,
because $g \circ p_{B'} =id_A$, to
 $s(p_B \circ g)$ being a section of ${p_B} ^* B'$ (i.e of $\crossx{B}{B'}{A}$) satisfying $s(p_B \circ g) \circ q( p_B,B') = p_B \circ g$. 
\end{proof}

% *************************************************************************************
% sfglemma ****************************************************************************
\begin{lemma}
\llabel{sfglemma}
If $f:A \morph B$ and $g:B\morph C$ in a contextual category \catcw and if $C \in Cover(B)$ then
\begin{equation}
\label{sgflemmagoalone}
ft(f\circ g)^*C = f^*(ft(g)^*C)
\end{equation}
and 
\begin{equation}
\label{sgflemmagoaltwo}
S(f\circ g)=f^*S(g)
\end{equation}
where $ft(g) = g \circ p_B$ so that we have
\begin{displaymath}
\begin{array}{ccp{1.7cm}cp{1.7cm}c}
ft(f\circ g)^*C=\kern-10pt&\Rnode{TL}{f^*(ft(g)^*C)} & & \Rnode{TC}{ft(g)^*C}          \\ [1.7cm]
&\Rnode{BL}{A}         & & \Rnode{BC}{B} && \Rnode{BR}{C}
\end{array}
\mbox{
\ncsar{TL}{BL}
\ncsar{TC}{BC}
\ncarr{BL}{BC}
\blabel{f}
\ncarr{BC}{BR}
\blabel{g}
\ncarr{TL}{TC}
\alabel{q(f,ft(g)^*C)}
\ncarr{TC}{BR}
\alabel{q(ft(g),C)}
\ncleftsimplesection{BL}{TL}
\alabel{s(f\circ g)=f^*s(g)}
\ncleftsimplesection{BC}{TC}
\alabel{s(g)}
}
\end{displaymath}
in \catc.
\end{lemma}
\begin{proof}
That (\ref{sgflemmagoalone}) holds follows by axiom (q4).

To show that (\ref{sgflemmagoaltwo}) holds remember that $s(f \circ g)$is the unique section of $ft(f \circ g)*C$
such that $s(f \circ g) \circ q(f \circ g \circ p_C,C) = f\circ g$. Therefore it suffices to show that
$f^*s(g) \circ q(f \circ g \circ p_C,C) = f\circ g$ and this we can show as follows:
\begin{align*}
(f^*s(g)) \circ q(f \circ g \circ p_C,C) &= (f^*s(g)) \circ q(f ,(g \circ p_C)^*C) \circ q(g \circ p_C ,C) &&\mbox{by axiom (qf)} \\
                             &= f \circ s(g) \circ q(g \circ p_C ,C)                   &&\mbox {from defn. of $f^*s(g)$}\\
														 &= f \circ g                                              &&\mbox {by axiom (s2)}
\end{align*}
\end{proof}

\newcommand{\duplesone}{{\duple{s_1}_{B_1}}}
\newcommand{\duplestwo}{{\duple{s_1,s_2}_{B_2}}}
\newcommand{\duplesn}{\duple{s_1,...s_n}_{B_n}}
\newcommand{\duplesi}{{\duple{s_1,...s_i}_{B_i}}}
\newcommand{\duplesilessone}{\duple{s_1,...s_{i-1}}_{B_{i-1}}}
\newcommand{\duplesj}{{\duple{s_1,...s_j}_{B_j}}}
\newcommand{\duplesjlessone}{\duple{s_1,...s_{j-1}}_{B_{j-1}}}
\newcommand{\duplesisucc}{{\duple{s_1,...s_{i+1}}_{B_{i+1}}}}
\newcommand{\duplesnlessone}{{\duple{s_1,...s_{n-1}}_{B_{n-1}}}}

\newcommand {\sonesub}{{s_1}^*}
\newcommand {\stwosub}{{s_2}^*}
\newcommand {\stwocascade}{\stwosub\sonesub}
\newcommand {\sisub}{{s_i}^*}
\newcommand {\sicascade}{\sisub...\sonesub}
\newcommand {\sisuccsub}{{s_{i+1}}^*}
\newcommand {\sisucccascade}{\sisuccsub...\sonesub}
\newcommand {\snlessonesub}{{s_{n-1}}^*}
\newcommand {\snlessonecascade}{\snlessonesub...\sonesub}
\newcommand {\snsub}{{s_n}^*}
\newcommand {\sncascade}{\snsub...\sonesub}

\note If $A$ is an object of contextual category \catc, if $1 \base B_1 ... \base B_n$ in \catcw and if
\begin{equation*}
\begin{array}{l}
s_1 \in Sect(\crossx{A}{B_1}{1}),                  \\
s_2 \in Sect(\sonesub (\crossx{A}{B_2}{1})),         \\
s_3 \in Sect(\stwocascade (\crossx{A}{B_3}{1})),     \\
\multicolumn{1}{c}{\vdots}                           \\
s_n \in Sect(\snlessonecascade (\crossx{A}{B_n}{1})) \\
\end{array}
\end{equation*}
\mbox{ in \catc},
then  we can define a morphism
$\duplesn:A \morph B_n$ in \catcw such that 
\begin{enumerate}[(i)]
\item $s(\duplesn) = s_n$,
\item for $n> 1$, $\duplesn \circ p_{B_n} = \duplesnlessone$, and 
\item for all objects $B$ of \catcw such that $B_n < B$, 
$\sncascade (\crossx{A}{B}{1}) = \duplesn ^* B$, \\
and for all sections $s$ of $B$,
$\sncascade (\crossx{A}{s}{1}) = \duplesn ^* s$.
\end{enumerate}

The definition of $\duplesn$ proceeds by induction. 
Define $\duplesone= s_1 \circ q(p_{A,1},B_1)$.
By axiom (s1), it follows immediately that $s(\duplesone)=s_1$.

If $B_1 <B$ in \catcw then we have $\sonesub (\crossx{A}{B}{1})=\duplesone ^* B$ because
\begin{align*}
\sonesub (\crossx{A}{B}{1})&= \sonesub q(p_{A,1},B_1)^*B     && \mbox{by definition of $\crossx{1}{}{}$,}\\
                         &= (s_1 \circ q(p_{A,1},B_1))^*B   && \mbox{by pullback coherence axiom (q5),}\\
                         &= \duplesone ^* B                   && \mbox{by definition of $\duplesone ^* B$.}
\end{align*}
Also, if $g$ is a section of $B$ then we can show, by a similar argument, 
that $\sonesub (\crossx{A}{g}{1})=\duplesone ^* g$.

Now assume that $\duplesi$ is defined and satisfies (i) to (iii) above. 
In particular we have  $\sicascade B_{i+1} = \duplesi ^* B_{i+1}$, and therefore that
$s_{i+1} \in Sect(\duplesi ^* B_{i+1})$, and this then allows us to define $\duplesisucc$ by 
\begin{equation*}
\duplesisucc = s_{i+1} \circ q(\duplesi, B_{i+1}).
\end{equation*} 
Immediately by axiom s3
we have that $s(\duplesisucc)=s_{i+1}$.
We have that $\duplesisucc \circ p_{B_{i+1}}= \duplesi$ because
\begin{align*}
\duplesisucc \circ p_{B_{i+1}} &=s_{i+1} \circ q(\duplesi, B_{i+1}) \circ p_{B_{i+1}} && \mbox{by definition of $\duplesisucc$,} \\
                               &=s_{i+1} \circ p_{\duplesi ^* B_{i+1}} \circ \duplesi && \mbox{because the pullback diagram commutes,} \\
															 &= \duplesi                       && \mbox{because $s_{i+1}$ is a section.}
\end{align*}
To establish (iii), suppose $B$ is some object such that $B_{i+1} < B$ in \catcw then we can show that $\sisucccascade (\crossx{A}{B}{1})=\duplesisucc ^* B$ as follows:
\begin{align*}
\sisucccascade (\crossx{A}{B}{1}) 
              &= \sisuccsub \duplesi ^* B && \mbox{by inductive hypothesis,} \\
                         &= \sisuccsub q(\duplesi,B_{i+1})^*B  && \mbox{by definition of extended $^*$,}\\
                         &= (s_{i+1} \circ q(\duplesi,B_{i+1}))^*B   && \mbox{by pullback coherence axiom,}\\
                         &= \duplesisucc ^* B                   && \mbox{by definition of $\duplesisucc$.}
\end{align*}
A similar argument shows that if $g$ is a section of $B$ then $\sisucccascade (\crossx{A}{g}{1})=\duplesisucc ^* g$.


\begin{lemma}
\label{dupledestructionlemma}
If $\duplesn : A \morph B_n$ in a contextual category $\catcw$ then \foreachi, 
\begin{equation}
\duplesn \circ p_{B_n,B_i} = \duplesi
\end{equation} 
\end{lemma}
\begin{proof}
Follows because for each $j$, $i < j \leq n$, $\duplesj \circ p_{B_j} = \duplesjlessone$
and $p_{B_j,B_i} = p_{B_j} \circ p_{B_{j-1},B_i}$.
\end{proof}

\newcommand{\dupletuplerhs}{\bigtuple{\duplesnlessone,g}_{p_{B_{n-1},B_i},C}}
\begin{lemma}
\label{thedupletuplelemma}
If $A$ is an object of contextual category \catc, if $1 \base B_1 ... \base B_n$ in \catcw and if
\begin{equation*}
\begin{array}{l}
s_1 \in Sect(\crossx{A}{B_1}{1}),                  \\
s_2 \in Sect(\sonesub (\crossx{A}{B_2}{1}))=Sect(\duplesone^*B_2),         \\
s_3 \in Sect(\stwocascade (\crossx{A}{B_3}{1}))=Sect(\duplestwo^*B_3),     \\
\multicolumn{1}{c}{\vdots}                           \\
s_n \in Sect(\snlessonecascade (\crossx{A}{B_n}{1})) =Sect(\duplesnlessone^*B_n)\\
\end{array}
\end{equation*}
\mbox{ in \catc}, and if $B_n$ is $\crossx{B_{n-1}}{C}{B_i}$, for some $i$, $0 \leq i < n$, 
(where in the case of $n$ equal to $0$ then 
by $B_0$ we mean the terminal object $1$ of \catc), if $g: A \morph C$ in \catcw and 
$g \circ p_C = \duplesi$, so that
$s(g) \in Sect(\duplesnlessone ^* B_n)$
\footnote {Because by definition of $s(g)$, $s(g) \in Sect((g \circ p_C) ^* C)$ and if 
$g \circ p_C =  \duplesi$ then 
\begin{align*}
(g \circ p_C) ^* C &= \duplesi ^* C  \\
                  &= (\duplesnlessone \circ {p_{B_{n-1},B_i}})^* C\\
									&=\duplesnlessone ^* B_n
\end{align*}
},  then if $s(g)=s_n$ then 


\begin{equation}
\label{dupletuplegoal}
\duplesn = \dupletuplerhs\,.
\end{equation}
\end{lemma}
\begin{proof}


To show that $\dupletuplerhs$ (rhs of (\ref{dupletuplegoal})) is defined we need show that 
\begin{equation}
g \circ p_C = \duplesnlessone \circ p_{B_{n-1},B_i}
\end{equation}
This follows from the assumption that $g \circ p_C = \duplesi$ and from lemma \ref{dupledestructionlemma}.

Now $\dupletuplerhs$ is therefore defined and is the unique morphism such that
\begin{equation}
\dupletuplerhs \circ p_{\crossx{B_{n-1}}{C}{1}} = \duplesnlessone
\end{equation}
and
\begin{equation}
\dupletuplerhs \circ q(p_{B_{n-1},B_i},C) = g
\end{equation}

as shown here

\begin{equation}
\begin{array}{c p{4cm} c p{3cm} c }
\\[1.75cm]
\Rnode{A}{A} && \Rnode{Bn1C}{\crossx{B_{n-1}}{C}{1}} &&                            \\[1.5cm]
						 &&                                      &&\Rnode{C}{C}                \\[0.5cm]
             && \Rnode{Bn1}{B_{n-1}}                 &&                            \\[1.5cm]
						 &&                                      &&\Rnode{Bi}{B_i}             \\[0.5cm]
\end{array}
\mbox{\ncdarr{A}{Bn1C}
\alabel{ \dupletuplerhs}
\ncarr{A}{Bn1}
\blabel{\duplesnlessone}
\ncsar{Bn1C}{Bn1}
\blabel{p_{\crossx{B_{n-1}}{C}{1}}}
\nccdar{Bn1}{Bi}
\blabel{p_{B_{n-1},B_i}}
\ncarr{Bn1C}{C}
\alabel{q(p_{B_{n-1},B_i},C)}[0.3]
\ncsar{C}{Bi}
\alabel{p_C}
\ncarr[60]{A}{C}
\alabel{g}
}
\end{equation}

Therefore to show (\ref{dupletuplegoal}), as we are required,  it suffices  to show that 

\begin{equation}
\label{dupletuplesubgoalone}
\duplesn \circ p_{\crossx{B_{n-1}}{C}{B_i}} = \duplesnlessone
\end{equation}
and
\begin{equation}
\label{dupletuplesubgoaltwo}
\duplesn \circ q(p_{B_{n-1},B_i},C) = g
\end{equation}

(\ref{dupletuplesubgoalone}) holds from the definition of $\duple{}_{B_n}$ because we have initially assumed that $B_n=\crossx{B_{n-1}}{C}{B_i}$.

We show that (\ref{dupletuplesubgoaltwo}) holds by using lemma \ref{footandstactic} and showing that
\begin{equation}
\label{dupletuplesubgoaltwoone}
\duplesn \circ q(p_{B_{n-1},B_i},C) \circ p_C = g \circ p_C
\end{equation}
and
\begin{equation}
\label{dupletuplesubgoaltwotwo}
s(\duplesn \circ q(p_{B_{n-1},B_i},C)) = s(g)
\end{equation}

(\ref{dupletuplesubgoaltwotwo}) follows immediately because by axiom \highlight{(s3)} \commentary{check use of parenthesis in reference to (q1),...(s3)} etc.
$s(\duplesn \circ q(p_{B_{n-1},B_i},C))=s(\duplesn)=s_n$ and from the initial assumption 
$s(g)=s_n$.

We are left with proving (\ref{dupletuplesubgoaltwoone}) which we can do as follows:
\begin{align*}
\duplesn \circ q(p_{B_{n-1},B_i},C) \circ p_C 
              &=  \duplesn \circ p_{\crossx{B_{n-1}}{C}{B_i}} \circ p_{B_{n-1},B_i} 
                                               && \mbox{commutivity of pullback square},               \\
							&=\duplesn \circ p_{\crossx{B_{n-1}}{C}{B_i},B_i} && \mbox{definition of extended $p$,}  \\
							&=\duplesi                                        && \mbox{lemma \ref{dupledestructionlemma},} \\
							&= g \circ p_C                                    && \mbox{from the initial assumption.}
\end{align*}
\end{proof}




\section{Note concerning the interpretation of Generalised Algebraic Theories within Contextual Categories}
%\note
We give the broadest possible notion of `model
of a generalised algebraic theory \gatUw' by defining the notion of an `interpretation of  a generalised algebraic theory \gatUw in a contextual category \catc'.
\begin{definition}
An interpretation $I$ of \gatUw in \catcw is a  mapping 
of derived T- and $\in$- rules of \gatUw to objects, respectively sections, of \gatUw that satisfies the following:
\begin{enumerate}[(i)]
\setlength\itemindent{2cm}
\item \underline{\textbf{T-rules}} 
Suppose that  the rule
\gatdisplayrule{\xDelta{n}}{\isT{\Delta}} is a derived rule of \gatUw which is mapped by $I$ to an object $a$ of \catc. Denote this rule $r$. Recall that because $r$ is a derived rule then it follows  that for each $i$, 
$1 \leq i \leq n$, the rule \gatdisplayrule{\xDelta{i-1}}{\isT{\Delta_i}} is a derived rule of \gatU. Let $r_i$ denote this rule.
Suppose that $I$ maps each rule $r_i$ to an object $a_i$ of \catcw.
It is required that $1 \base a_1 \base ... \base a_n \base a$ in \catc.

Suppose that the  expression $\Delta$ is exactly the expression $\Delta_i$, for some $i$, $1 \leq i \leq n$. In this special case we require that the rule $r$  is mapped by $I$ to the object 
$\crossx{a_n}{a_i}{a_{i-1}}$. 

\item \underline{\textbf{$\boldsymbol {\in}$-rules}} 
In addition to the assumptions made in (i),  suppose that the rule
\gatdisplayrule{\xDelta{n}}{\ofT{t}{\Delta}} is a  derived rule of \gatU. 
Denote this rule $r_t$. It is required that $I$ maps the rule $r_t$ to a section
 $s:a_n \morph a$ in \catcw i.e. to a morphism $s:a_n \morph a$ such that $s \circ p_a = id_{a_n}$. 

Suppose that the  expression $\Delta$ is exactly the expression $\Delta_i$, for some $i$, $1 \leq i \leq n$ and that the expression $t_i$ is simply the variable $x_i$. 
In this special case we require that the rule $r_t$  is mapped by $I$ to the section\footnote{
With these assumptions, $s(p_{a_n,a_i}): a_n \morph \crossx{a_n}{a_i}{a_{i-1}}$ in \catcw because by definition  $s(p_{a_n,a_i}): a_n  \morph (p_{a_n,a_i} \circ p_{a_i})^*a_i$,
and we have 
\begin{align*}
(p_{a_n,a_i} \circ p_{a_i})^*a_i &= {p_{a_n,a_{i-1}}} ^* a_i  && \mbox{ because $p_{a_n,a_i} \circ p_{a_i}=p_{a_n,a_{i-1}}$,} \\
                                 &= \crossx{a_n}{a_i}{a_{i-1}} && \mbox{ by definition of $\crossx{}{}{w}$}.
\end{align*}
} % end footnote
$s(p_{a_n,a_i})$ of the object $\crossx{a_n}{a_i}{a_{i-1}}$. 

\item \underline{\textbf{T=-rules}} 
In addition to the assumptions made in (i), suppose that  
the rule \gatdisplayrule{\xDelta{n}}{\Delta = \Delta'} is a derived rule of \gatU. 
We may deduce that the
\gatdisplayrule{\xDelta{n}}{\isT{\Delta'}} is a derived rule of \gatU. Denote this latter rule $r'$.
It is required that the rule $r'$ is mapped by $I$ to the same object $a$ of \catcw that $r$ is mapped to.

\item \underline{\textbf{$\boldsymbol{\in=}$-rules}} 
In addition to the assumptions made in (ii),  suppose that the rule
\gatdisplayrule{\xDelta{n}}{t = t' \in \Delta}
is a derived rule of \gatU. We may deduce that the rule
\gatdisplayrule{\xDelta{n}}{\ofT{t'}{\Delta}} is a  derived rule of \gatU. 
Denote this latter rule $r'_t$.
It is required that the rule $r'_t$ is mapped by $I$ to the same section $s$ of $a$ that $r_t$ is mapped to.

\item \underline{\textbf{weakening T-rules}} 
Suppose now that $Q$ is any context of \gatUw and that the rule 
\gatdisplayrule{\yOmega{m}}{\isT{\Omega}} is a derived rule of \gatU. Denote this rule $r_\Omega$. 
It follows that \foreachj, the rule   \gatdisplayrule{\yOmega{j-1}}{\isT{\Omega_j}} is a derived rule of \gatU. Denote this rule $r_{\Omega_j}$.
Suppose that each rule $r_{\Omega_j}$ is mapped by $I$ to an object $b_j$ of \catcw and that rule $r_\Omega$ is mapped by $I$ to an object $b$ of \catcw so that
we have that $b_1 \base ... \base b_m \base b$ in \catc.

By the simple weakening lemma it follows that the rules
\gatdisplayrule{Q, \yOmega{j-1}}{\isT{\Omega_j}}, \foreachj, and 
\gatdisplayrule{Q, \yOmega{m}}{\isT{\Omega}} are  derived rule of \gatU. It is required that these rules are mapped by $I$ to the objects
$\crossx{a}{b_1}{1},...\crossx{a}{b_m}{1}$ and to $\crossx{a}{b}{1}$, respectively. 

\item \underline{\textbf{weakening $\boldsymbol {\in}$-rules}} 
Suppose in addition that the rule \gatdisplayrule{\yOmega{m}}{\ofT{s}{\Omega}} is a derived rule of \gatUw 
and that this rule is mapped by $I$ to a section $g$ of object $b$ in \catc.
By the simple weakening lemma it follows that the rule \gatdisplayrule{Q, \yOmega{m}}{\ofT{s}{\Omega}}
is a derived rule of \gatU. It is required that this rule is mapped by $I$ to the section $\crossx{a}{s}{1}$
of object $\crossx{a}{b}{1}$ of \catc.


\item \underline{\textbf{substituting in T-rules}} 
Suppose that, as in (v), above, the rule 
\gatdisplayrule{\yOmega{m}}{\isT{\Omega}} is a derived rule of \gatU.
We may deduce that \foreachj, the rule   \gatdisplayrule{\yOmega{j-1}}{\isT{\Omega_j}} is a derived rule of \gatU. 
Suppose that, as in (v), these rules are mapped by $I$ to objects $b_1,...b_n$ and $b$ so that
we have  $b_1 \base ... \base b_m \base b$ in \catc. Now suppose that for some $j$, $1 \leq j \leq m$, the rule
\gatdisplayrule{\yOmega{j-1}}{\ofT{t}{\Omega_j}} is a derived rule of \gatU. 
Now it follows by the substitution lemma that for each $j'$, $j < j' \leq m$ the rule

\gatdisplayrule{\yOmega{j-1}, y_{j+1}\in \Omega_{j+1}[t|y_j],... y_{j'-1} \in \Omega_{j'-1}[t|y_j] }{\isT{\Omega_j[t|y_j]}} is a derived rule of \gatUw and that likewise the rule

\gatdisplayrule{\yOmega{j-1}, y_{j+1}\in \Omega_{j+1}[t|y_j],... y_m \in \Omega_m[t|y_j] }{\isT{\Omega}[t|y_j]} is a derived rule of \gatU.
It is required that these rules are mapped by $I$ to objects $f^*b_{j+1},...f^*b_m$ and $f^*b$, respectively.
Note that as required we have that $f^*b_{j+1}\base ... \base f^*b_m \base f^*b$ in \catc.

\item \underline{\textbf{substituting in $\boldsymbol {\in}$-rules}} 
Suppose that in addition to the situation in (vii), above, the rule
\gatdisplayrule{\yOmega{m}}{\ofT{s}{\Omega}}
is a derived rule of \gatUw and suppose that this rule is mapped to a section $g$ of object $b$ of \catc.
Now it follows by the substitution lemma that the rule
\gatdisplayrule{\yOmega{j-1}, y_{j+1}\in \Omega_{j+1}[t|y_j],... y_m \in \Omega_m[t|y_j]}{\ofT{s[t|y_j]}{\Omega[t|y_j]}} 
is a derived rule of \gatU.
It is a requirement that this rule is mapped by $I$ to the section $f^*g$ of object $f^*b$ of \catc.
\end{enumerate}
\end{definition}

\note
We can show that an interpretation $I$ of a contextual category \gatUw in a contextual category \catcw is
completely determined by its mapping of the introductory rules of sort symbols and operator symbols to
objects, respectively, sections of \catc. In order to show this first define a preinterpretation as follows:
\begin{definition}
If \gatUw is a generalised algebraic theory  and if \catcw is a contextual category then
a \term{preinterpretation} $I$ of  \gatUw in \catcw consists of a pair :
\begin{itemize}
\item a mapping $\Isort$ that maps each sort symbol of \gatUw to  an object of \catc,
\item a mapping $\Iop$ that maps each operator symbol of \gatUw to a section of \catcw (i.e. to a morphism $f: A \morph B$ for some 
$A \base B$ in \catcw such that $f \circ p_B=id_A$).
\end{itemize}
\end{definition}

Note that I will say that an interpretation $I$ of \gatUw in \catcw extends a preinterpretation $P$ of \gatUw in \catcw to mean that for each sort symbol $A$ of \gatU,
$\Isort(r_A) = P(A)$, where $r_A$ is the introductory rule for $A$ and that for each operator symbol
$f$ of \gatU,   $\Iop(r_f) = P(f)$, where $r_f$ is the introductory rule for $f$.

\note
 For a preinterpretation to extend to an interpretation (we will say that it *is* an interpretation) 
it will need be type correct and to satisy the axioms of the theory. The definition of exactly what we mean by this has to proceed by induction because, for example, we need an interpretation of the rule
\gatdisplayrule{\xDelta{n}}{\isT{\Delta}} (as an object $a$ of \catc, say) before we can say whether a preinterpretation of an operator with introductory rule \genericfintroductoryrule
is well-typed (i.e. to know that it is object $a$ that it is required to be a section of).
Similarly we need to be able to interpret both sides of an axiom before we can say whether it is respected
by the preinterpretation. 

\note Suppose $P$ be a preinterpretation of \gatUw in contextual category \catc.
Let $\gatU_0 \subseteq \gatU_1 \subseteq \gatU_2 \subseteq ...$ be the stratification of \gatU. 
Let $P_i$ be the restriction of $P$ to $U_i$. 
We define $P$ to be well-typed and to respect all axioms 
providing that  each  $P_i$ is well-typed and respects all axioms of $\gatU_i$ and 
what we mean by this we define by induction. 
At the same time we prove that if $P_i$ is well-typed and respects all axioms of $\gatU_i$ then $P_i$
extends uniquely to an interpretation of $\gatU_i$.  

For the inductive step we require
\begin{lemmastar}
\item $P_{i+1}$ extends uniquely to an interpretation of $U_{i+1}$ iff  $P_i$ extends uniquely to an interpretation of $U_{i}$ and additionally:
\begin{enumerate}[(i)]
\item
$P_{i+1}$ is well-typed on sort symbol $A$ of $U_{i+1}$. This we define to mean that
for all sort symbols $A$ in $U_{i+1}$, if $A$ has introductory rule 
\genericAintroductoryrule and if this rule is mapped by $P_{i+1}$
to object $a$ of \catcw then $I_i(r_n) \base a$ in \catc, where $r_n$ is the rule 
\gatdisplayrule{\xDelta{n-1}}{\isT{\Delta_n}} (which, as required and due to the stratification, is a derived rule of $\gatU_i$).
\item  $P_{i+1}$ is well-typed on operator symbols  of $U_{i+1}$. This we define to mean that
for all sort symbols $f$ in $U_{i+1}$, if $f$ has introductory rule 
\genericfintroductoryrule then $P_{i+1}$ maps this rule to a section 
of $I_i(r)$ where $r$ is the rule
\gatdisplayrule{\xDelta{n}}{\isT{\Delta}} (which, as required and due to the stratification, is a derived rule of $\gatU_i$). 
\item
 $I_i$ respects the axioms of $U_{i+1}$. By this we mean that 
\begin{enumerate}[(i)]
\item \underline{\textbf{T=-axioms}} 
for all axioms of $U_{i+1}$ of the form
 \gatdisplayrule{\xDelta{n}}{\Delta = \Delta'},
$I_i(r) = I_i(r')$ where $r$ is the rule
\gatdisplayrule{\xDelta{n}}{\isT{\Delta}} and  
and $r'$ is the rule \gatdisplayrule{\xDelta{n}}{\isT{\Delta'}}
\item \underline{\textbf{$\boldsymbol{\in=}$-axioms}} 
for all axioms of $U_{i+1}$ of the form
\gatdisplayrule{\xDelta{n}}{t = t' \in \Delta}
$I_i(r_t) = I_i(r'_t)$ where $r_t$ is the rule
\gatdisplayrule{\xDelta{n}}{\ofT{r_t}{\Delta}} and  
and $r'_t$ is the rule \gatdisplayrule{\xDelta{n}}{\ofT{t'}{\Delta'}}.
\end{enumerate}
\end{enumerate}
\end{lemmastar}
\begin{proof} 
\tbd
\end{proof}




  



\newcommand{\Imapsto}{\scaleto{\mapsto}{10pt}}

\newcommand {\OO}{Ob^2}
\newcommand {\OOO}{Ob^3}

\newcommand{\leftidentitylhsterm}{({x_1}^*\qq{id})^*\tuple{x_1,x_1,x_2}^*\qq{\circ}}
\newcommand{\rightidentitylhsterm}{({x_2}^*\qq{id})^*\tuple{x_1,x_2,x_2}^*\qq{\circ}}
\newcommand{\HomHom}{\crossx{Hom}{Hom}{\OO}}

\newcommand {\yOOO}{\ofT{y_1,y_2,y_3}{Ob}}
\newcommand {\yOOOfH}{\yOOO,\,\ofT{f}{Hom(y_1,y_2)}}
\newcommand{\yOOOfHgH}{\yOOOfH,\,\ofT{g}{Hom(y_2,y_3)}}

\newcommand {\yOOOfHmapped}{\tuple{y_1,y_2}^*Hom}
\newcommand {\yOOOfHgHmapped}{\crossx{\yOOOfHmapped}{\tuple{y_2,y_3}^*Hom}{\OOO}}
\newcommand {\yOOOfHgHHmapped}{\crossx{\big(\yOOOfHgHmapped\big)}{{\tuple{y_1,y_3}^*Hom}}{\OOO}}

\newcommand{\associativitypremise}
       {\ofT{z_1,z_2,z_3,z_4}{Ob},\,
                                \ofT{f}{Hom(z_1,z_2)},\,\ofT{g}{Hom(z_2,z_3)},\,\ofT{h}{Hom(z_3,z_4)}}
																
																
\newcommand{\associativitypremisemapped}{\crossx 
                                            {\big(\crossx
																						      {\tuple{z_1,z_2}^*Hom}
																									{\tuple{z_2,z_3}^*Hom} 
                                                  {Ob^4}
																							\big)}
																						{\tuple{z_3,z_4}^*Hom} 
																						{Ob^4}  
																				 }

\newcommand{\associativitylhstype}{\isT{{Hom(z_1,z_4)}}}
\newcommand{\associativitylhstypemapped}{\crossx{\Big(\associativitypremisemapped\Big)}{\tuple{z_1,z_4}^*Hom}{Ob^4}}
\newcommand{\associativitylhstermtyping}{\ofT{(f \circ g) \circ h}{Hom(z_1,z_4)}}
\newcommand{\associativityrhstermtyping}{\ofT{f \circ (g \circ h)}{Hom(z_1,z_4)}}		

\note
Consider as an example
what constitutes an interpretation $I$ of the (generalised algebraic) theory of categories ($tc$) in some 
 contextual category \catc.
The sorts of $tc$ are $Ob$ and $Hom$ 
and an interpretation $I$ must map these to objects $I(Ob)$ and  $I(Hom)$ of \catc.
The operators symbols of
$tc$ are  $id$ and $\circ$ and these must be mapped by $I$ to sections $I(id)$ and $I(\circ)$ of \catc.

A useful simplication for the description that follows is to write $Ob$ for $I(Ob)$, $Hom$ for $I(Hom)$, $\qq{id}$ for $I(id)$ and   $\qq{\circ}$ for $I(\circ)$.   I will ask the reader  to distinguish for themselves 
those uses of `$Ob$' and `$Hom$' in reference to sorts of the theory $tc$ from those uses in reference to the interpretation of these sorts in the contextual category \catc. 

Because we want to explain and to some extent validate the definition of interpretation we will go through the
requirements on the interpretations of $Ob$, $Hom$, $id$ and $\circ$ in some detail.  For this purpose we
need make use of the stratification 
$tc_0 \subseteq tc_1 \subseteq tc_2 \subseteq tc_3 \subseteq tc_4$
of the theory of categories mentioned earlier\footnote{$tc_0$ is the empty theory, $tc_1$ has introductory rule 
$r_{Ob}$ of sort $Ob$, $tc_2$ has introductory rule $r_{Hom}$ of sort symbol $Hom$, $tc_3$ has introductory rules
$r_{id}$ for operator $id$ and  $r_\circ$ for operator $\circ$, $tc_4$ has the three axioms -- left and right identity and associativity.}.


\begin{enumerate}[$tc_1$]
\item By clause (i) of the definition of interpretation (case $n=1$), the sort $Ob$ must be mapped to an object $Ob$ of \catcw such that $1 \base Ob$ in \catc. Given this much, i.e. given the interpretation of the sort $Ob$ in the theory by such an object $Ob$ of the contextual category, then are able to interpret in \catcw any rule of the theory $tc_1$ i.e. any rule that can be derived solely from the introductory rule for $Ob$. 
In particular we can derive the rule \gatdisplayrule{\ofT{x}{Ob}}{\isT{Ob}} 
 and this is interpreted by $I$ (see earlier\commentary{\tbd})
 as the object $\OO$ in \catc . This rule is the one we need  have an interpretation of in \catcw in order to determine whether the interpretation of $Hom$ by $I$ is well-typed.

\item The sort $Hom$ of the theory $tc$ is introduced by the rule 
\gatdisplayrule{\ofT{x}{Ob},\, \ofT{y}{Ob}}{\isT{Hom(x,y)}} 
and therefore, given what we have said of the interpretation of rule 
\gatdisplayrule{\ofT{x}{Ob}}{\isT{Ob}}, 
by clause (i) of the definition of interpretation (case $n=1$), the sort $Hom$ must be mapped to some object of \catc,  which, as we have said, we denote $Hom$, such that $\crossx{Ob}{Ob}{1} \base Hom$ in \catc.

Once we have this interpretation of $Hom$ then we can interpret any derived rule of the theory $tc_2$ i.e.
any rule which can be derived from just the introductory rules
for $Ob$ and $Hom$. In particular we can derive the rules which 
the introductory rules of $\qq{id}$ and $\qq{\circ}$ rely on for their
well-typedness which are \gatdisplayrule{\ofT{x}{Ob}}{\isT{Hom(x,x)}} and
\gatdisplayrule{\ofT{y_1}{Ob},\, \ofT{y_2}{Ob},\,\ofT{y_3}{Ob},\, \ofT{f}{Hom(y_1,y_2)},\, \ofT{g}{Hom(y_2,y_3)} }{\isT{Hom(y_1,y_3)}}.
In fact it follows from something earlier (\tbd) that the first of these rules will be interpreted 
as the object ${\delta_{Ob}}^*Hom$ of \catcw and that the second of these rules will be interpreted by
the object $\yOOOfHgHHmapped$. 

\begin{equation*}
\begin{array}{c c}
\tuple{y''_1,y''_3}^*Hom         \\ [1cm]
\tuple{y'_2,y'_3}^*Hom           \\ [1cm]
\Rnode{e}{\yOOOfHmapped}         \\ [1cm]
\Rnode{d}{\OOO} & \Rnode{h}{Hom} \\ [1cm]
\Rnode{c}{\OO}                   \\ [1cm]
\Rnode{b}{Ob}                    \\ [1cm]
\Rnode{a}{1}           
\end{array}
\ncsar{g}{f}
\ncsar{f}{e}
\ncsar{e}{d}
\ncsar{d}{c}
\ncsar{c}{b}
\ncsar{b}{a}
\ncsar{h}{c}
\end{equation*}


\item 
From clause (ii) of the definition of interpretation and from the observations just  made, it
follows, since the interpretation $I$ must be well typed for operator symbols $id$,
that the operator symbol $id$ must be interpreted by $I$ as a section of the object $\delta_{Ob}^*Hom$ of \catcw
and we shall refer to as $\qq{id}$, so that $\qq{id} : Ob \morph {\delta_{Ob}}^*Hom$ of \catc. 

Now consider the operator symbol $\circ$ whose introcutory rule is
\gatdisplayrule{\yOOOfHgH}{\ofT{f \circ g}{Hom(z_1,z_3)}}.
Let $H_2$ be the premise $\yOOOfHgH$ then $H_2$ is mapped by $I$ to the object $yOOOfHgHmapped$ of \catc.
Let $\qq{H_2}$ be this object. 
 

Likewise for $I$ to be well-typed on the introductory rule for $\circ$, the operator symbol $\circ$ must be interpreted by $I$ as a section of the object 
$\yOOOfHgHHmapped$ of \catcw 
where $y_1,y_2,y_3$ are the projection morphisms $y_1,y_2,vy_3: \OOO \morph Ob$ defined earlier (\tbd\footnote{
$y_1: \OOO \morph Ob$ is defined to be $p_{\OOO} \circ p_{\OO}$ \\
$y_2: \OOO \morph Ob$ is defined to be $p_{\OOO} \circ q(p_{Ob},Ob)$\\
$vy_3: \OOO \morph Ob$ is defined to be $q(p_{\OO},Ob)$
}) and we shall refer to it as $\qq{\circ}$ so that we have $\qq{\circ}: \yOOOfHgHmapped \morph \yOOOfHgHHmapped$ in \catc.
At this stage we are now able to interpret any derived rule of the theory $tc3$ i.e. any rule 
which can be derived just from the introductory rules for $Ob, Hom, id and \circ$. In particular we can
interpret the left and right sides of the three axioms. In fact we will have: 

\newcommand{\sect}{Sect}
\newcommand{\insect}[2]{#1 \in Sect(#2)}

\begin{equation*}
\begin{array}{p{5cm} c l}
\gatdisplayrule{\ofT{y_1}{Ob},\, \ofT{y_2}{Ob},\,\ofT{y_3}{Ob},\, \ofT{f}{Hom(y_1,y_2)},\, \ofT{g}{Hom(y_2,y_3)} }{\isT{Hom(y_1,y_3)}}                                                                     \\[0.15cm]
            & \Imapsto & {\delta_{Ob}}^*Hom                               \\[0.2cm]
\gatdisplayrule{\ofT{y_1}{Ob},\, \ofT{y_2}{Ob},\,\ofT{y_3}{Ob},\, \ofT{f}{Hom(y_1,y_2)},\, \ofT{g}{Hom(y_2,y_3)} }{\ofT{f \circ g}{Hom(y_1,y_3)}}                                                                     \\[0.15cm]
            & \Imapsto & \insect{\qq{\circ}}{{\delta_{Ob}}^*Hom}                               \\[0.2cm]
\multicolumn{3}{l}{\gatdisplayrule{\associativitypremise}{\ofT{f \circ g}{Hom(z_1,z_3)}}}       \\[0.15cm]
        		& \Imapsto & 	\insect{\tuple{z_1,z_2,z_3,f,g}^*\qq{\circ}}{\tuple{z_1,z_2,z_3,f,g}^*{\delta_{Ob}}^*Hom} \\[0.5cm]	
												\\
\gatdisplayrule{\ofT{x_1}{Ob},\,\ofT{x_2}{Ob},\,\ofT{f}{Hom(x_1,x_2)}}
               {\ofT{f}{Hom(x_1,x_2)}} 
            & \Imapsto & \delta_{Ob}: Hom \morph \HomHom                  \\[0.2cm]
\gatdisplayrule{\ofT{x_1}{Ob},\,\ofT{x_2}{Ob},\,\ofT{f}{Hom(x_1,x_2)}}
               {\ofT{id(x_1) \circ f}{Hom(x_1,x_2)}} 
            & \Imapsto &  \leftidentitylhsterm : Hom \morph \HomHom       \\[0.2cm]
\gatdisplayrule{\ofT{x_1}{Ob},\,\ofT{x_2}{Ob},\,\ofT{f}{Hom(x_1,x_2)}}
               {\ofT{f \circ id(x_2)}{Hom(x_1,x_2)}} 
            & \Imapsto & \rightidentitylhsterm : Hom \morph \HomHom       \\[0.2cm]
\multicolumn{3}{l}{\associativitypremise}                                               \\[0.15cm]
        		& \Imapsto & \associativitypremisemapped				              \\[0.5cm]
\multicolumn{3}{l}{\gatdisplayrule{\associativitypremise}{\associativitylhstype}}       \\[0.15cm]
        		& \Imapsto & \associativitylhstypemapped				              \\[0.5cm]					
\multicolumn{3}{l}{\gatdisplayrule{\associativitypremise}{\associativitylhstermtyping}} \\[0.15cm]
            & \Imapsto &                                                  \\[0.5cm] 
\multicolumn{3}{l}{\gatdisplayrule{\associativitypremise}{\associativityrhstermtyping}} \\[0.15cm]
            & \Imapsto &   
\end{array}
\end{equation*}

 
\item % tc_4
\end{enumerate}

\begin{oldtt}
\begin{equation*}
\begin{array}{c c}
\Rnode{g}{\yOOOfHgHHmapped}  \\ [1cm]
\Rnode{f}{\yOOOfHgHmapped}    \\ [1cm]
\Rnode{e}{\yOOOfHmapped}       \\ [1cm]
\Rnode{d}{\OOO} & \Rnode{h}{Hom} \\ [1cm]
\Rnode{c}{\OO}                    \\ [1cm]
\Rnode{b}{Ob}                       \\ [1cm]
\Rnode{a}{1}           
\end{array}
\ncsar{g}{f}
\ncsar{f}{e}
\ncsar{e}{d}
\ncsar{d}{c}
\ncsar{c}{b}
\ncsar{b}{a}
\ncsar{h}{c}
\end{equation*}
\end{oldtt}



\section{Comments on a draft paper regarding the Initiality Conjecture}
%
% initialitynotes
\note From my thesis, 
\begin{point}
there is a category $\catGAT$ of generalised algebraic theories and interpretations,
\end{point}
\begin{point}
there is a category $\catCon$ of contextual categories,
\end{point}
\begin{point}
there is a functor $\ccat[C]: \catCon \morph \catGAT$,
\end{point}
\begin{point}
there is a functor $\gat[U]:\catGAT \morph \catCon$,
\end{point}
\begin{point}
the functor $\ccat[C]$ is an equivalence with inverse $\gat[U]$.
\end{point}

\note
The proof that categories $\catGAT$ and $\catCon$ are equivalent  is entirely trivial but runs to more than 50 pages. For my money what it means is that generalised algebraic theories and contextual categories are more or less the same thing. If a generalised algebraic theory is finite then we have a finitely presented contextual category. 

\note
Presumably we could give a definition for a sketch of contextual category 
similar to  definitions such as those of linear sketch, finite product (FP) sketch, finite discrete (FD) sketch and finite limit (FL) sketch. Such a CC sketch would be syntax-free way a way of defining a generalised algebraic theory and it would parallel and benefit from treatments given to other notions of theory versus category with additional structure.

\note
Since contextual categories with families are the same as contextual categories then also
contextual categories with families are much the same thing as generalised algebraic theories. 

\note For me what it is to be an internal $\gat[U]$-structure in a contextual category $\ccat[C]$ is given by defining a notion of `interpretation of  generalised algebraic theory $\gat[U]$ in  contextual category $\ccat[C]$'. 
An alternative definition is that an internal $\gat[U]$-structure in a contextual category $\ccat[C]$ is precisely a 
contextual functor from the contextual category $\CofU$ to $\ccat[C]$. These two possible definitions are equivalent
\footnote{The proof that they are equivalent definitions is entirely straightforward though I didn't write it up my thesis. I did write out the proof of a  corresponding lemma in regard to single-sorted algebraic theories; this is in my Msc dissertation.}.


\note
For example, an internal monoid in a contextual category $\ccat[C]$
can be defined to be a contextual functor $F: \ccat[C](tm) \morph  \ccat[C]$, where $tm$ is the generalised algebraic theory of categories.
Reminder: The theory of monoids ($tm$)is
\begin{gatrules}
\gatintros
\gatintroducing{M}
\isT{M} \\
\gatintroducing{unit}
\ofT{unit}{M} \\
\gatintroducing{mult}
\gatsingular[6cm]{\ofT{x,y}{M}}{\ofT{mult(x,y)}{M}} \\
\gataxioms

\gatintroducing{ \gataxiomno{1} \\ \gataxiomno{2} }
\begin{gatgroup}{\ofT{x}{M}}
\gatleaf[6cm]{}{mult(unit,x)=x} \\
\gatleaf[6cm]{}{mult(x,unit)=x}
\end{gatgroup} \\
\gatintroducing{ \gataxiomno{3} }
\gatsingular[6cm]{\ofT{x,y,z}{M}}{mult(m(x,y),z)=mult(x,mult(y,z))} 
\end{gatrules}

\note 
Likewise an internal category in a contextual category $\ccat[C]$
can be defined to be a contextual functor $F: \ccat[C](tc) \morph  \ccat[C]$, where $tc$ is the generalised algebraic theory of categories.



\note
As an example of a finite product (FP) sketch Barr and Wells \cite{BarrandWells}, page 232, discuss monoids internal to  category
\footnote{Generally we would be thinking of a category with finite products including terminal object here, though as they say not all products need be available in the category for there to be an internal monoid.}.

\begin{notebox}[In your section EXAMPLES OF GENERALIZED ALGEBRAIC THEORIES]
You  shouldn't refer to theory of internal categories, theory of internal monoids etc. you should refer to the
theory of categories, theory of monoids and so on. What you have defined as the theory of monoids for example has
as algebras (i.e. has models in Fam) exactly sets with monoid structure i.e. monoids. When the same theory is interpreted in other cwfs then
the models are monoids internal to these other cwfs. In other words the generalised algebraic theory of xyz's is something that has xyz's as models in Fam and has internal xyz's as models in other categories. 
\end{notebox}

\note The category of internal $\gat[U]$-structures in contextual categories is the coslice category
$\CofU \downarrow \catCon$. Needless to say this has an initial object which is the identity functor on  $\CofU$.
If\ $\gat[U]$ is a considered a type theory (whatever that is) then this initial object is what I believe Vladimir refers
to as the term model when speaking of the initiality conjecture.



\note From my thesis:
\begin{tightquote}
Consider for a moment. Every theory $\gat[U]$ has a minimal model denoted $\KU$ built out of the closed terms of \gat[U]. Alternatively this minimal model is described just in terms of the structure $\CofU$. For example
if $1 \base A$ in $\CofU$ then 
$\KU(A)=Hom(1,A)$, otherwise if $1 \base A_1 \base ... \base A_n \base A$ in $\CofU$
then if $a_1 \in \KU(A_1)$, ... if $a_n \in \KU(A_n)(a_1,...a_{n-1})$ then 
$\KU(A)(a_1,...a_n)=\setsuchthat{a\in Hom_{\CofU}(1,A)}{a \circ p_A = a_n}$. \\
\end{tightquote} 

Followed by:
\begin{tightquote}
Now, the free U-algebras are the algebras $I-alg(\KUp)$ for $I: \gat[U] \morph \gat[U']$ an extension of $\gat[U]$ by constants alone. The finitely generated free U-algebras are those algebras where $\gat[U']$ is an extension by finitely many constants. \\
\end{tightquote}

\begin{notebox}
Though your example theories are not theories of internal xyz-objects there are such theories.
The models of a theory of internal xyz-objects  in Fam should be  internal xyz's i.e. they should be contextual categories (or cwfs) along with internal xyz's. When the theory of internal xyz's is interpreted in an arbitrary contextual category (or cwf) then what results is an internal internal xyz! This sounds a bit crazy but it isn't -- there are after all categories internal to other categories and it isn't much of a stretch to suppose these internal categories have internal xyz's inside of them. 
\end{notebox}

\note The generalised algebraic theory of internal monoids (where internal means internal to contextual categories) 
is the theory of contexual categories plus:
\begin{gatrules}
\gatintros
\gatintroducing{M}
\ofT{M}{Ob} \\
\gatintroducing{e}
\ofT{unit}{Hom(1,M)} \\
\gatintroducing{m}
\ofT{mult}{Hom(M \cross M,M)} \\
\gataxioms

\gatintroducing{ \gataxiomno{1} }
pair(t_M \circ unit,id_M) \circ m =id_M \\
\gatintroducing{ \gataxiomno{2} }
pair(id_M,t_M \circ unit) \circ m =id_M \\
\gatintroducing{ \gataxiomno{3} }
(m \cross id_M) \circ m = (id_M \cross m) \circ m
\end{gatrules}

\note For any generalised algebraic theory $\gat[U]$ let $\qq{U}$ be the theory of internal $\gat[U]$-objects then
\footnote{This observation seems interesting to me and I have  wondered whether it is a contribution to the initiality contecture.
Sadly I am not sure that it is. }
\begin{equation}
K_{\qq{U}}=\CofU
\end{equation}



\note There is a large 2-category $\catofccs$ of contextual categories, contextual functors and natural transformations. \\

\note
If \isagat[U] then $\CofU$ is a contextual category. 
$\CofU$ is the structured assembly of contexts and realisations of $\gat[U]$.
There is an interpretation $I_0$ of theory $\gat[U]$ in contextual category
$\CofU$\footnote{
In the context of a type theory I think that this is what Vladimir refers to as the term model though it is safer to think of this as the abstract-syntax model to distinguish it from another model to be described later. It is confusing because both this model and the later model are initial in some category.}.\\

\note If $F : \ccat[C] \morph \ccat[C']$ is a contextual functor then for any $\gat[U]$, 
$F$ induces a mapping of interpretations of $\gat[U]$ in $\ccat[C]$ to interpretations of $\gat[U]$ in $\ccat[C']$. Denote this mapping $\phi_F$. \\

\note
If \isagat[U] then a $\gat[U]$-algebra $A$ is a contextual functor $A: \CofU \morph \Fam$. \\

\note 
Let $tcc$ denote the gat of contextual categories. Then
\begin{point}
to every contextual category $\ccat[C]$ there corresponds a \tccalgebra 
which we shall denote $\alg{C}$  i.e. there is a contextual functor $\alg{C} :\ccat[C](tcc) \morph \Fam$,
\end{point}
\begin{point}
to every a \tccalgebra i.e. to every contextual functor $A :\ccat[C](tcc) \morph \Fam$ there is a contextual category $\cc{A}$
\end{point}
\begin{point}
for all contextual categories $\ccat[C]$,
\begin{equation}
\cc{\alg{\ccat[C]}}= \ccat[C],
\end{equation}
\end{point}
\begin{point}
for all \tccalgebras $A$,
\begin{equation}
\alg{\cc{A}} = A.
\end{equation}
\end{point}

\note
From the previous it follows 
\begin{pointeq}
\label{cualg}
for every gat \gat[U], to the contextual category $\CofU$ corresponds a contextual functor
   $\alg{\CofU} :\ccat[C](tcc) \morph \Fam$. \\
\end{pointeq} 

\note Particularising (\ref{cualg}) to the theory $tcc$ it follows that
\begin{pointeq}
  $\alg{\ccat[C](tcc)}$ is a contextual functor   $\alg{\ccat[C](tcc)} :\ccat[C](tcc) \morph \Fam$.
\end{pointeq}

\note
Let $\Fam$ be the (large) contextual category of sets, indexed families of sets, indexed families of families of sets and so on and
let $\FAM$ be the (larger still) contextual category of large sets, indexed families of large sets, indexed families of families of large sets and so on.
Particularising (\ref{cualg}) to the category $\Fam$ we have
\begin{pointeq}
  \label{inducedalgebra}
  $\alg{\Fam}$ is a contextual functor   $\alg{Fam} :\ccat[C](tcc) \morph \FAM$. 
\end{pointeq}

\note
Aside: Assume now that $\catofccs$ is the category of large contextual categories so that $\Fam$ is an object of $\catofccs$. 
Let $\catoflargerccs$ be the category of larger contextual categories. \\

\note The Inititiality Conjecture is described in \cite{VoevodskyInitialityConjecture}.
\begin{tightquote}
A C-system equipped with additional
operations corresponding to the inference rules of a type theory is called a
model or a C-system model of these rules or of this type theory.
\end{tightquote}
and
\begin{tightquote}
The model whose underlying
C-system is the term C-system is called the term model... for a particular
class of inference rules the term model is an initial object in the category of models.
This is known as the Inititiality Conjecture.
\end{tightquote} 
\ \\
\note Whereas from nLab (\url{https://ncatlab.org/nlab/show/Initiality+Project}) I read
\begin{tightquote}
The Initiality Project is a communal effort to prove an initiality theorem for a dependent type theory: that the categorical structure constructed out of the syntax is the initial object in some category of structured categorical objects.
\end{tightquote}

\note Do not use the term `model'.

\note From my 1986 paper (\cite{Cartmell86})based on my thesis (\cite{Cartmell78}):
\begin{tightquote}
An algebraic semantics is witnessed by an equivalence between a category of
theories and a category of structures. In most instances of algebraic semantics
there is a further equivalence in that the usual definition of model of a theory can
be replaced by a definition which uses only the notion of structure. Lawvere has
used the term Functorial Semantics in describing this kind of semantics.
Functorial semantics depends on an equivalence between the category of models
of a theory U and the category of structure preserving morphisms from the
structure $C(U)$ corresponding to $U$ to a special canonical structure (the world
structure?). In the case of algebraic theories (Lawvere \cite{LawvereAlgebraicTheories}) the canonical structure
is taken to be the category of sets Set while in the case of classical proposition
theories the canonical structure is taken to be the Boolean Algebra $\set{0, 1}$.
The present situation is as well-behaved as any if the canonical structure is
taken to be the contextual category $Fam$.
If U is a generalised algebraic theory, then the category of models of U is
equivalent to the category which has contextual functors $C(U)$ to $Fam$ as objects
and natural transformations as morphisms. Thus we can assert
\begin{equation*}
\Ualg \cong ConFunc(C(U), Fam).
\end{equation*}
The inductive construction of $C(U)$ from $U$ has enabled us to replace the usual 
inductive definition of model of $U$ by the definition "a model of $U$ is a contextual
functor $M: C(U): \morph Fam$".

Every interpretation $I:U \morph U'$ induces a contextual functor 
$C(I):C(U) \morph C(U')$. Composition with $C(I)$ is a functor from 
$ConFunc(C(U'),Fam)$ to
$ConFunc(C(U),Fam)$. It is the functor $\Ialg:\Upalg \morph \Ualg$. Those functors
between categories of models which are induced in this way are called generalised
algebraic functors. We can show that all such functors have left adjoints. This is
equivalent to a known generalisation of Lawvere \cite{LawvereAlgebraicTheories}'s theorem that all algebraic
functors have a left adjoint. 
\end{tightquote}

\begin{oldtt}
\note
One possibility for Vladimir's category of models is as follows. Define a model of a generalised algebraic theory to be any pair $\tuple{C,I}$ where $C$ is a contextual category and $I$ is an interpretation of the theory $U$ in the contextual category $C$. Define the morphisms between
model $\tuple{C,I}$ and model $\tuple{C',I'}$ to be pairs $\tuple{F, \eta}$ where
$F: c \morph C'$ is a contextual functor and $\eta: I \circ F \morph  I'$ is a natural transformation.
\end{oldtt}

\note
The signature of a theory consists of s set of sorts and a set of operations. It does not usually include the equations of the theory. 
\begin{notebox}
In your paper your construction of a `gat' from a signature seems to include sorts, operators AND equations. Because 
what is included in your signatures differs from what other authors include in signatures it make it harder to understand.
\end{notebox}
\begin{notebox} 
I cannot see a reason for excluding type equations from your presentation.
\end{notebox}
\begin{oldtt}
\note Suppose we add a sum type to generalised algebraic theories so that from
types $\Delta$ and $\Delta'$ in context $Delta_n$ we can construct a type $\Delta + \Delta'$
along with inclusion operations and that we can construct $t | t'$. 

\note Suppose we have a gat+ $U$. Then can we construct a term model which is a contextual category?
In the term model is $[\Delta + \Delta']$ actually the coproduct of $[\Delta]$ and $[\
Delta'] $
\end{oldtt}

%unused
%\section{Unused}
%input{unused}

%\bibliographystyle{alpha} 
\bibliographystyle{abbrv}
\bibliography{../SharedBibliography/temp/bibliography}
\end{document}
