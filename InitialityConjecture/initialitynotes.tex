
% initialitynotes
\note From my thesis, 
\begin{point}
there is a category $\catGAT$ of generalised algebraic theories and interpretations,
\end{point}
\begin{point}
there is a category $\catCon$ of contextual categories,
\end{point}
\begin{point}
there is a functor $\ccat[C]: \catCon \morph \catGAT$,
\end{point}
\begin{point}
there is a functor $\gat[U]:\catGAT \morph \catCon$,
\end{point}
\begin{point}
the functor $\ccat[C]$ is an equivalence with inverse $\gat[U]$.
\end{point}

\note
The proof that categories $\catGAT$ and $\catCon$ are equivalent  is entirely trivial but runs to more than 50 pages. For my money what it means is that generalised algebraic theories and contextual categories are more or less the same thing. If a generalised algebraic theory is finite then we have a finitely presented contextual category. 

\note
Presumably we could give a definition for a sketch of contextual category 
similar to  definitions such as those of linear sketch, finite product (FP) sketch, finite discrete (FD) sketch and finite limit (FL) sketch. Such a CC sketch would be syntax-free way a way of defining a generalised algebraic theory and it would parallel and benefit from treatments given to other notions of theory versus category with additional structure.

\note
Since contextual categories with families are the same as contextual categories then also
contextual categories with families are much the same thing as generalised algebraic theories. 

\note For me what it is to be an internal $\gat[U]$-structure in a contextual category $\ccat[C]$ is given by defining a notion of `interpretation of  generalised algebraic theory $\gat[U]$ in  contextual category $\ccat[C]$'. 
An alternative definition is that an internal $\gat[U]$-structure in a contextual category $\ccat[C]$ is precisely a 
contextual functor from the contextual category $\CofU$ to $\ccat[C]$. These two possible definitions are equivalent
\footnote{The proof that they are equivalent definitions is entirely straightforward though I didn't write it up my thesis. I did write out the proof of a  corresponding lemma in regard to single-sorted algebraic theories; this is in my Msc dissertation.}.


\note
For example, an internal monoid in a contextual category $\ccat[C]$
can be defined to be a contextual functor $F: \ccat[C](tm) \morph  \ccat[C]$, where $tm$ is the generalised algebraic theory of categories.
Reminder: The theory of monoids ($tm$)is
\begin{gatrules}
\gatintros
\gatintroducing{M}
\isT{M} \\
\gatintroducing{unit}
\ofT{unit}{M} \\
\gatintroducing{mult}
\gatsingular[6cm]{\ofT{x,y}{M}}{\ofT{mult(x,y)}{M}} \\
\gataxioms

\gatintroducing{ \gataxiomno{1} \\ \gataxiomno{2} }
\begin{gatgroup}{\ofT{x}{M}}
\gatleaf[6cm]{}{mult(unit,x)=x} \\
\gatleaf[6cm]{}{mult(x,unit)=x}
\end{gatgroup} \\
\gatintroducing{ \gataxiomno{3} }
\gatsingular[6cm]{\ofT{x,y,z}{M}}{mult(m(x,y),z)=mult(x,mult(y,z))} 
\end{gatrules}

\note 
Likewise an internal category in a contextual category $\ccat[C]$
can be defined to be a contextual functor $F: \ccat[C](tc) \morph  \ccat[C]$, where $tc$ is the generalised algebraic theory of categories.



\note
As an example of a finite product (FP) sketch Barr and Wells \cite{BarrandWells}, page 232, discuss monoids internal to  category
\footnote{Generally we would be thinking of a category with finite products including terminal object here, though as they say not all products need be available in the category for there to be an internal monoid.}.

\begin{notebox}[In your section EXAMPLES OF GENERALIZED ALGEBRAIC THEORIES]
You  shouldn't refer to theory of internal categories, theory of internal monoids etc. you should refer to the
theory of categories, theory of monoids and so on. What you have defined as the theory of monoids for example has
as algebras (i.e. has models in Fam) exactly sets with monoid structure i.e. monoids. When the same theory is interpreted in other cwfs then
the models are monoids internal to these other cwfs. In other words the generalised algebraic theory of xyz's is something that has xyz's as models in Fam and has internal xyz's as models in other categories. 
\end{notebox}

\note The category of internal $\gat[U]$-structures in contextual categories is the coslice category
$\CofU \downarrow \catCon$. Needless to say this has an initial object which is the identity functor on  $\CofU$.
If\ $\gat[U]$ is a considered a type theory (whatever that is) then this initial object is what I believe Vladimir refers
to as the term model when speaking of the initiality conjecture.



\note From my thesis:
\begin{tightquote}
Consider for a moment. Every theory $\gat[U]$ has a minimal model denoted $\KU$ built out of the closed terms of \gat[U]. Alternatively this minimal model is described just in terms of the structure $\CofU$. For example
if $1 \base A$ in $\CofU$ then 
$\KU(A)=Hom(1,A)$, otherwise if $1 \base A_1 \base ... \base A_n \base A$ in $\CofU$
then if $a_1 \in \KU(A_1)$, ... if $a_n \in \KU(A_n)(a_1,...a_{n-1})$ then 
$\KU(A)(a_1,...a_n)=\setsuchthat{a\in Hom_{\CofU}(1,A)}{a \circ p_A = a_n}$. \\
\end{tightquote} 

Followed by:
\begin{tightquote}
Now, the free U-algebras are the algebras $I-alg(\KUp)$ for $I: \gat[U] \morph \gat[U']$ an extension of $\gat[U]$ by constants alone. The finitely generated free U-algebras are those algebras where $\gat[U']$ is an extension by finitely many constants. \\
\end{tightquote}

\begin{notebox}
Though your example theories are not theories of internal xyz-objects there are such theories.
The models of a theory of internal xyz-objects  in Fam should be  internal xyz's i.e. they should be contextual categories (or cwfs) along with internal xyz's. When the theory of internal xyz's is interpreted in an arbitrary contextual category (or cwf) then what results is an internal internal xyz! This sounds a bit crazy but it isn't -- there are after all categories internal to other categories and it isn't much of a stretch to suppose these internal categories have internal xyz's inside of them. 
\end{notebox}

\note The generalised algebraic theory of internal monoids (where internal means internal to contextual categories) 
is the theory of contexual categories plus:
\begin{gatrules}
\gatintros
\gatintroducing{M}
\ofT{M}{Ob} \\
\gatintroducing{e}
\ofT{unit}{Hom(1,M)} \\
\gatintroducing{m}
\ofT{mult}{Hom(M \cross M,M)} \\
\gataxioms

\gatintroducing{ \gataxiomno{1} }
pair(t_M \circ unit,id_M) \circ m =id_M \\
\gatintroducing{ \gataxiomno{2} }
pair(id_M,t_M \circ unit) \circ m =id_M \\
\gatintroducing{ \gataxiomno{3} }
(m \cross id_M) \circ m = (id_M \cross m) \circ m
\end{gatrules}

\note For any generalised algebraic theory $\gat[U]$ let $\qq{U}$ be the theory of internal $\gat[U]$-objects then
\footnote{This observation seems interesting to me and I have  wondered whether it is a contribution to the initiality contecture.
Sadly I am not sure that it is. }
\begin{equation}
K_{\qq{U}}=\CofU
\end{equation}



\note There is a large 2-category $\catofccs$ of contextual categories, contextual functors and natural transformations. \\

\note
If \isagat[U] then $\CofU$ is a contextual category. 
$\CofU$ is the structured assembly of contexts and realisations of $\gat[U]$.
There is an interpretation $I_0$ of theory $\gat[U]$ in contextual category
$\CofU$\footnote{
In the context of a type theory I think that this is what Vladimir refers to as the term model though it is safer to think of this as the abstract-syntax model to distinguish it from another model to be described later. It is confusing because both this model and the later model are initial in some category.}.\\

\note If $F : \ccat[C] \morph \ccat[C']$ is a contextual functor then for any $\gat[U]$, 
$F$ induces a mapping of interpretations of $\gat[U]$ in $\ccat[C]$ to interpretations of $\gat[U]$ in $\ccat[C']$. Denote this mapping $\phi_F$. \\

\note
If \isagat[U] then a $\gat[U]$-algebra $A$ is a contextual functor $A: \CofU \morph \Fam$. \\

\note 
Let $tcc$ denote the gat of contextual categories. Then
\begin{point}
to every contextual category $\ccat[C]$ there corresponds a \tccalgebra 
which we shall denote $\alg{C}$  i.e. there is a contextual functor $\alg{C} :\ccat[C](tcc) \morph \Fam$,
\end{point}
\begin{point}
to every a \tccalgebra i.e. to every contextual functor $A :\ccat[C](tcc) \morph \Fam$ there is a contextual category $\cc{A}$
\end{point}
\begin{point}
for all contextual categories $\ccat[C]$,
\begin{equation}
\cc{\alg{\ccat[C]}}= \ccat[C],
\end{equation}
\end{point}
\begin{point}
for all \tccalgebras $A$,
\begin{equation}
\alg{\cc{A}} = A.
\end{equation}
\end{point}

\note
From the previous it follows 
\begin{pointeq}
\label{cualg}
for every gat \gat[U], to the contextual category $\CofU$ corresponds a contextual functor
   $\alg{\CofU} :\ccat[C](tcc) \morph \Fam$. \\
\end{pointeq} 

\note Particularising (\ref{cualg}) to the theory $tcc$ it follows that
\begin{pointeq}
  $\alg{\ccat[C](tcc)}$ is a contextual functor   $\alg{\ccat[C](tcc)} :\ccat[C](tcc) \morph \Fam$.
\end{pointeq}

\note
Let $\Fam$ be the (large) contextual category of sets, indexed families of sets, indexed families of families of sets and so on and
let $\FAM$ be the (larger still) contextual category of large sets, indexed families of large sets, indexed families of families of large sets and so on.
Particularising (\ref{cualg}) to the category $\Fam$ we have
\begin{pointeq}
  \label{inducedalgebra}
  $\alg{\Fam}$ is a contextual functor   $\alg{Fam} :\ccat[C](tcc) \morph \FAM$. 
\end{pointeq}

\note
Aside: Assume now that $\catofccs$ is the category of large contextual categories so that $\Fam$ is an object of $\catofccs$. 
Let $\catoflargerccs$ be the category of larger contextual categories. \\

\note The Inititiality Conjecture is described in \cite{VoevodskyInitialityConjecture}.
\begin{tightquote}
A C-system equipped with additional
operations corresponding to the inference rules of a type theory is called a
model or a C-system model of these rules or of this type theory.
\end{tightquote}
and
\begin{tightquote}
The model whose underlying
C-system is the term C-system is called the term model... for a particular
class of inference rules the term model is an initial object in the category of models.
This is known as the Inititiality Conjecture.
\end{tightquote} 
\ \\
\note Whereas from nLab (\url{https://ncatlab.org/nlab/show/Initiality+Project}) I read
\begin{tightquote}
The Initiality Project is a communal effort to prove an initiality theorem for a dependent type theory: that the categorical structure constructed out of the syntax is the initial object in some category of structured categorical objects.
\end{tightquote}

\note Do not use the term `model'.

\note From my 1986 paper (\cite{Cartmell86})based on my thesis (\cite{Cartmell78}):
\begin{tightquote}
An algebraic semantics is witnessed by an equivalence between a category of
theories and a category of structures. In most instances of algebraic semantics
there is a further equivalence in that the usual definition of model of a theory can
be replaced by a definition which uses only the notion of structure. Lawvere has
used the term Functorial Semantics in describing this kind of semantics.
Functorial semantics depends on an equivalence between the category of models
of a theory U and the category of structure preserving morphisms from the
structure $C(U)$ corresponding to $U$ to a special canonical structure (the world
structure?). In the case of algebraic theories (Lawvere \cite{LawvereAlgebraicTheories}) the canonical structure
is taken to be the category of sets Set while in the case of classical proposition
theories the canonical structure is taken to be the Boolean Algebra $\set{0, 1}$.
The present situation is as well-behaved as any if the canonical structure is
taken to be the contextual category $Fam$.
If U is a generalised algebraic theory, then the category of models of U is
equivalent to the category which has contextual functors $C(U)$ to $Fam$ as objects
and natural transformations as morphisms. Thus we can assert
\begin{equation*}
\Ualg \cong ConFunc(C(U), Fam).
\end{equation*}
The inductive construction of $C(U)$ from $U$ has enabled us to replace the usual 
inductive definition of model of $U$ by the definition "a model of $U$ is a contextual
functor $M: C(U): \morph Fam$".

Every interpretation $I:U \morph U'$ induces a contextual functor 
$C(I):C(U) \morph C(U')$. Composition with $C(I)$ is a functor from 
$ConFunc(C(U'),Fam)$ to
$ConFunc(C(U),Fam)$. It is the functor $\Ialg:\Upalg \morph \Ualg$. Those functors
between categories of models which are induced in this way are called generalised
algebraic functors. We can show that all such functors have left adjoints. This is
equivalent to a known generalisation of Lawvere \cite{LawvereAlgebraicTheories}'s theorem that all algebraic
functors have a left adjoint. 
\end{tightquote}

\begin{oldtt}
\note
One possibility for Vladimir's category of models is as follows. Define a model of a generalised algebraic theory to be any pair $\tuple{C,I}$ where $C$ is a contextual category and $I$ is an interpretation of the theory $U$ in the contextual category $C$. Define the morphisms between
model $\tuple{C,I}$ and model $\tuple{C',I'}$ to be pairs $\tuple{F, \eta}$ where
$F: c \morph C'$ is a contextual functor and $\eta: I \circ F \morph  I'$ is a natural transformation.
\end{oldtt}

\note
The signature of a theory consists of s set of sorts and a set of operations. It does not usually include the equations of the theory. 
\begin{notebox}
In your paper your construction of a `gat' from a signature seems to include sorts, operators AND equations. Because 
what is included in your signatures differs from what other authors include in signatures it make it harder to understand.
\end{notebox}
\begin{notebox} 
I cannot see a reason for excluding type equations from your presentation.
\end{notebox}
\begin{oldtt}
\note Suppose we add a sum type to generalised algebraic theories so that from
types $\Delta$ and $\Delta'$ in context $Delta_n$ we can construct a type $\Delta + \Delta'$
along with inclusion operations and that we can construct $t | t'$. 

\note Suppose we have a gat+ $U$. Then can we construct a term model which is a contextual category?
In the term model is $[\Delta + \Delta']$ actually the coproduct of $[\Delta]$ and $[\
Delta'] $
\end{oldtt}