\note
We give the broadest possible notion of a model
of a generalised algebraic theory \gatUw by defining the notion of an interpretation of  \gatUw in  any given contextual category \catc.

\note
Suppose $\xDelta{n}$ is a context of a generalised algebraic theory \gatU for some $n \ge 1$
%$r$ is a derived rule of \gatUw  of the form $$\frac{\xDelta{n}}{\isT{\Delta}}$$ 
then recall that for each $i$, 
$1 \leq i \leq n$, the rule $$\frac{\xDelta{i-1}}{\isT{\Delta_i}}$$ will be a derived rule of \gatU. Let $r_i$ denote this rule.
If $I$ is an interpretation of \gatUw in a contextual category \catcw then $I$ we will map the sequence of rules
$r_1, ... r_n$ to a sequence of objects $a_1,...a_n$ of \catcw such that
$1 \base a_1 \base ... \base a_n$ in \catc. \\
% workaround when used with Rnode for size of font used under the cross
\renewcommand{\crossx}[3]{#1 \underset{\tiny #3}{\cross} #2}
\begin{minipage}[t]{8.5cm}
\note Suppose $a_1,..a_n$ are objects in a contexual category \catcw such that $1 \base a_1 \base ... \base a_n$ in \catcw then \foreachi,
$s(p_{a_n,a_i}): a_n \morph \crossx{a_n}{a_i}{a_{i-1}}$ in \catcw because by definition  $s(p_{a_n,a_i}): a_n  \morph (p_{a_n,a_i} \circ p_{a_i})^*a_i$,
and we have 
\begin{align*}
(p_{a_n,a_i} \circ p_{a_i})^*a_i &= {p_{a_n,a_{i-1}}} ^* a_i  && \mbox{ because $p_{a_n,a_i} \circ p_{a_i}=p_{a_n,a_{i-1}}$,} \\
                                 &= \crossx{a_n}{a_i}{a_{i-1}} && \mbox{ by definition of $\crossx{}{}{w}$}.
\end{align*} \\
Suppose now that derived rules $r_1,...r_n$ of \gatUw are mapped to objects $a_1, ...a_n$ of \catcw by an interpretation $I$ of \gatU in \catcw then
$I$ will map the rule 
$\displaystyle \frac{\xDelta{n}}{\isT{\Delta_i}}$
to $\crossx{a_n}{a_i}{a_{i-1}}$
and will map the rule
$\displaystyle \frac{\xDelta{n}}{\ofT{x_i}{\Delta_i}}$
to $s(p_{a_n,a_i})$.
\end{minipage}
\raisebox{0.5cm}{
\begin{minipage}[t]{6cm}
\begin{displaymath}
\begin{array}{c}
\crossx{a_n}{a_i}{\Rnode{cross}{a_{i-1}}} \\[0.9cm]
\Rnode{an}{a_n}\\[0.7cm]
%\Rnode{highervdots}{\vdots}\\
\Rnode{ai}{\begin{array}{c}
\vdots\\
a_i\\
\vdots
\end{array}} \\[1.1cm]
%\Rnode{lowervdots}{\vdots}\\[0.4cm]
\Rnode{a1}{a_1}\\[0.7cm]
\Rnode{abs}{1}
\end{array}
\end{displaymath}
\ncsar{cross}{an}
%\ncsar{an}{highervdots}
%\ncsar{lowervdots}{a1}
\ncsar{an}{ai}
\ncsar{ai}{a1}
\ncsar{a1}{abs}
\ncarc[arcangle=30,nodesepA=5pt,offsetA=2pt,nodesepB=2pt,offsetB=2pt]{->}{an}{cross}
\alabel{s(p_{a_n,a_i})}
\end{minipage}
}
\newpage
\note
First, as a stepping stone, we need  give an auxillary definition, the definition of  a `preinterpretation of \gatUw in \catc'.

\newcommand{\Isort}{I_{sort}}
\newcommand{\Iop}{I_{op}}
\newcommand {\Ihat}{\hat{I}}
\note 
If \gatU is a generalised algebraic theory  and if \catcw is a contextual category then
a preinterpretation $I$ of  \gatU in \catcw consists of a pair :
\begin{itemize}
\item a mapping $\Isort$ that maps each sort symbol of \gatUw to  an object of \catc,
\item a mapping $\Iop$ that maps each operator symbol of \gatUw to a section of \catcw (i.e. to a morphism $f: A \morph B$ for some 
$A \base B$ in \catcw such that $f \circ p_B=id_A$).
\end{itemize}

\note 
An interpretation $I$ of \gatUw in \catcw is a preinterpretation satisfying additional conditions and these 
additional conditions imply that the preinterpretation induces a mapping $\hat{I}$
of derived T- and $\epsilon$- rules of \gatUw to objects, respectively sections, of \gatUw such that:


\begin{enumerate}[(i)]
\setlength\itemindent{2cm}
\item whenever $R$ is a derived rule of the form $\frac{\xDelta{n}}{\isT{\Delta}}$
then $\Ihat(Rn) \base \Ihat(R)$ in \catc, where $Rn$ is the rule
$\frac{\xDelta{n-1}}{\isT{\Delta_n}}$,

\item whenever $R$ and $Rn$ as above and additionally $Rt$ is a derived rule of the form $\frac{\xDelta{n}}{\ofT{t}{\Delta}}$
then $\Ihat(Rt) : \Ihat(Rn) \morph \Ihat(R)$  in \catcw and $\Ihat(Rt)$ is a section, 

\item whenever
$\frac{\xDelta{n}}{\Delta = \Delta'}$
is an derived rule of \gatU then $\Ihat(R) = \Ihat(R')$
where $R$ is the rule
$\frac{\xDelta{n}}{\isT{\Delta}}$
and $R'$ is the rule
$\frac{\xDelta{n}}{\isT{\Delta}}$,

\item whenever
$\frac{\xDelta{n}}{t = t' \in \Delta}$
is an derived rule of \gatUw then $\Ihat(Rt) = \Ihat(Rt')$
where $Rt$ is the rule
$\frac{\xDelta{n}}{\ofT{t}{\Delta}}$
and $Rt'$ is the rule
$\frac{\xDelta{n}}{\ofT{t'}{\Delta}}$.
\end{enumerate}

\note The additional conditions are:
\begin{enumerate}[(i)]
\setlength\itemindent{2cm}
\item
whenever $A$ is a sort symbol of \gatUw that is introduced by the rule
$\frac{\xDelta{n}}{\isT{A(\xn)}}$
then $\Ihat(Rn) \base \Isort(A)$ in \catc, where $Rn$ is the rule
$\frac{\context{x}{\Delta}{n-1}}{\isT{\Delta_n}}$,
\item
whenever $F$ is an operator symbol of \gatUw that is introduced by the rule
$\frac{\xDelta{n}}{\ofT{F(\xn)}{\Delta}}$
then $\Iop(F) : \Ihat(Rn) \morph \Ihat(R)$  in \catcw and $\Iop(F)$ is a section,
\item whenever
$\frac{\xDelta{n}}{\Delta = \Delta'}$
is an axiom of \gatUw then $\Ihat(R) = \Ihat(R')$
where $R$ is the rule
$\frac{\xDelta{n}}{\isT{\Delta}}$
and $R'$ is the rule
$\frac{\xDelta{n}}{\isT{\Delta}}$,
\item whenever
$\frac{\xDelta{n}}{t = t' \in \Delta}$
is an axiom of \gatUw then $\Ihat(Rt) = \Ihat(Rt')$
where $Rt$ is the rule
$\frac{\xDelta{n}}{\ofT{t}{\Delta}}$
and $Rt'$ is the rule
$\frac{\xDelta{n}}{\ofT{t'}{\Delta}}$.
\end{enumerate}

\note The definition of $\Ihat$ from $\Isort$ and $\Iop$ proceeds by induction 
on the derivation of rules in  \gatUw 
as described in the principles of derivation in Definition 2(b) of \cite{Cartmell86}. 
The only non-trivial parts of this definition relate to the rules
identified as CF1, CF2(a) and CF2(b)\footnote{So identified, by the way, as a mnemonic for cut-free.}. We consider each of these rules in turn.

\begin{point}
Rule CF1 states that for $n \geq 0$, for $1 \leq i \leq n+1$, from the derived rule 
$\frac{\xDelta{n}}{\isT{\Delta_{n+1}}}$ which we shall denote $R$ 
we may derive the rule
$\frac{\xDelta{n+1}}{\ofT{x_i}{\Delta_i}}$ which, in turn, we shall denote $R_{x_i}$.
Define $\Ihat(R_{x_i}) :  \Ihat(R) \morph \crossx{\Ihat(R)}{\Ihat(R_i)}{\Ihat(R_{i-1})}$
to be $\tuple{p_{\Ihat(R),\Ihat(R_{i-1})},p_{\Ihat(R),\Ihat(R_{i})}}$. 

This presumably is $s(p_{\Ihat(R),\Ihat(R_{i-1})})$ where $s$ is Vladimir's s-operator.
\end{point}
\begin{point}
CF2(a) states that if $A$ is a sort symbol introduced by
$\frac{\xDelta{n}}{\isT{A(\xn)}}$ 
and if $P$ is a context and $\tn$ are expressions then from the following rules, which we shall denote $R_{t_1}$,..$R_{t_n}$,
$\frac{P}{\ofT{t_1}{\Delta_1}}$,
$\frac{P}{\ofT{t_2}{\Delta_2[t_1|x_1]}}$,
... and 
$\frac{P}{\ofT{t_n}{\Delta_n[t_1|x_1,...t_{n-1}|x_{n-1}]}}$
we may derive the rule
$\frac{P}{\isT{A(t_1,...t_n)}}$ which we denote as $R$. 
Define $I(R)$ to be $\Ihat(R_n)^*...\Ihat(R_1)^*I(A)$.
\end{point}
\begin{point}
CF2(b)
\end{point}

