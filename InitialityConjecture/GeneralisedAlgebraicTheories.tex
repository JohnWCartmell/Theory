\newcommand{\gatdisplayrule}[2]
{
\setlength{\fboxsep}{1pt}
\setlength{\fboxrule}{0pt}
\fbox{$\displaystyle \frac{#1}{#2}$}
}
\newcommand{\Isort}{I_{sort}}
\newcommand{\Iop}{I_{op}}
\newcommand {\Ihat}{\hat{I}}

% workaround when used with Rnode for size of font used under the cross
\renewcommand{\crossx}[3]{#1 \underset{\tiny #3}{\cross} #2}
\newcommand{\fonestar}   {{f_1}\kern-.15em^*}
\newcommand{\ftwostar}   {{f_2}\kern-.15em^*}
\newcommand{\fjstar}     {{f_j}\kern-.2em^*}
\newcommand{\fjpstar}    {{f_{j-1}}\kern-.25em^*}
\newcommand{\smstar}{{s_m}\kern-.25em^*}
\newcommand{\sonestar}{{s_1}\kern-.15em^*}


\note
We give the broadest possible notion of a model
of a generalised algebraic theory \gatUw by defining the notion of an interpretation of  \gatUw in  any given contextual category \catc.

\note 
An interpretation $I$ of \gatUw in \catcw is a  mapping 
of derived T- and $\in$- rules of \gatUw to objects, respectively sections, of \gatUw that satisfies the requirements detailed below.


\begin{enumerate}[(i)]
\setlength\itemindent{2cm}
\item \underline{\textbf{T-rules}} 
Suppose that  the rule
\gatdisplayrule{\xDelta{n}}{\isT{\Delta}} is a derived rule of \gatUw which is mapped by $I$ to an object $a$ of \catc. Denote this rule $r$. Recall that becaue $r$ is a derived rule then it follows  that for each $i$, 
$1 \leq i \leq n$, the rule \gatdisplayrule{\xDelta{i-1}}{\isT{\Delta_i}} is a derived rule of \gatU. Let $r_i$ denote this rule.
Suppose that $I$ maps each rule $r_i$ to an object $a_i$ of \catcw.
It is required that $1 \base a_1 \base ... \base a_n \base a$ in \catc.

Suppose that the  expression $\Delta$ is exactly the expression $\Delta_i$, for some $i$, $1 \leq i \leq n$. In this special case we require that the rule $r$  is mapped by $I$ to the object 
$\crossx{a_n}{a_i}{a_{i-1}}$. 

\item \underline{\textbf{$\boldsymbol {\in}$-rules}} 
Suppose, in addition to the general situiation in (i),  that the rule
\gatdisplayrule{\xDelta{n}}{\ofT{t}{\Delta}} is a  derived rule of \gatU. 
Denote this rule $r_t$. It is required that $I$ maps the rule $r_t$ to a section
 $s:a_n \morph a$ in \catcw i.e. to a morphism $s:a_n \morph a$ such that $s \circ p_a = id_{a_n}$. 

Suppose that the  expression $\Delta$ is exactly the expression $\Delta_i$, for some $i$, $1 \leq i \leq n$ and that the expression $t_i$ is simply the variable $x_i$. 
In this special case we require that the rule $r_t$  is mapped by $I$ to the section\footnote{
With these assumptions, $s(p_{a_n,a_i}): a_n \morph \crossx{a_n}{a_i}{a_{i-1}}$ in \catcw because by definition  $s(p_{a_n,a_i}): a_n  \morph (p_{a_n,a_i} \circ p_{a_i})^*a_i$,
and we have 
\begin{align*}
(p_{a_n,a_i} \circ p_{a_i})^*a_i &= {p_{a_n,a_{i-1}}} ^* a_i  && \mbox{ because $p_{a_n,a_i} \circ p_{a_i}=p_{a_n,a_{i-1}}$,} \\
                                 &= \crossx{a_n}{a_i}{a_{i-1}} && \mbox{ by definition of $\crossx{}{}{w}$}.
\end{align*}
} % end footnote
$s(p_{a_n,a_i})$ of the object $\crossx{a_n}{a_i}{a_{i-1}}$. 

\item \underline{\textbf{T=-rules}} 
Suppose, in addition, that  the rule \gatdisplayrule{\xDelta{n}}{\Delta = \Delta'} is a derived rule of \gatUw and
from which we may deduce that the
\gatdisplayrule{\xDelta{n}}{\isT{\Delta'}} is a derived rule of \gatU. Denote this latter rule $r'$.
It is required that the rule $r'$ is mapped by $I$ to the same object $a$ of \catcw that $r$ is mapped to.

\item \underline{\textbf{$\boldsymbol{\in=}$-rules}} 
Suppose, in addition,  that the rule
\gatdisplayrule{\xDelta{n}}{t = t' \in \Delta}
is a derived rule of \gatUw and from which we may deduce that the rule
\gatdisplayrule{\xDelta{n}}{\ofT{t'}{\Delta}} is a  derived rule of \gatU. 
Denote this latter rule $r'_t$.
It is required that the rule $r'_t$ is mapped by $I$ to the same section $s$ of $a$ that $r_t$ is mapped to.
\end{enumerate}

\newpage

\note 3
\label{omegarealisationwrtQ}
 Suppose also that $\encyOmega{m}$ is a context of generalised algebraic theory \gatUw and suppose that $Q$ is some other context and that for some $m \geq 1$,
 \foreachj, \gatdisplayrule{Q}{\ofT{s_j}{\Omega_j[s_1|y_1,...s_{j-1}|y_{j-1}]}} is a derived rule of \gatU\footnote{Recall that such an m-tuple $\tuple{\sm}$ is said to be a realisation of 
$\encyOmega{m}$ wrt $Q$.}.  Suppose that an interpretation $I$ of \gatUw in a contextual category \catcw maps the context $Q$ to an object $a$  of $\catc$ and maps
the context $\encyOmega{j}$ to an object $b_j$ of \catc, \foreachj, so that $1 \base b_1 ... \base b_m$ in \catc. In this situation the rule 
\gatdisplayrule{Q}{\ofT{s_1}{\Omega_1}} will be mapped by $I$ to some section $f_1:a \morph \crossx{a}{b_1}{1}$. The j'th rule,
\gatdisplayrule{Q}{\ofT{s_j}{\Omega_j[s_1|y_1,...s_{j-1}|y_{j-1}]}}, will be mapped to a section $f_j:a \morph \fjpstar ... \fonestar\crossx{a}{b_j}{1}$.
In the case that $m=3$  then in \catcw we will have objects and morphisms as follows:
\begin{displaymath}
\begin{array}{c p{1cm} c p {1cm} c  p{1cm} c}
                                                &&                                           && \Rnode{ab3}{\crossx{a}{b_3}{1}}                       \\[1.2cm]
                                                &&\Rnode{f1axb3}{\fonestar\crossx{a}{b_3}{1}}  && \Rnode{ab2}{\crossx{a}{b_2}{1}}                       \\[1.2cm]
 \Rnode{f3target}{\ftwostar\fonestar\crossx{a}{b_3}{1}} &&\Rnode{f2target}{\fonestar\crossx{a}{b_2}{1}}  && \Rnode{ab1}{\crossx{a}{b_1}{\Rnode{f1target}{1}}}     \\[1.2cm]
                                                &&\Rnode{a}{a}                               &&                                                       \\[-3.0cm]
																								&&                                           &&                         && \Rnode{b3}{b_3}             \\[1.2cm]
																								&&                                           &&                         && \Rnode{b2}{b_2}             \\[1.2cm]
																								&&                                           &&                         && \Rnode{b1}{b_1}             \\[1.1cm]
																								&&                                           && \Rnode{abs}{1} \ \ \ \ \ \ \ \ &&    
\end{array}
\end{displaymath}
\ncarr{ab3}{b3}
\ncarr{ab2}{b2}
\ncarr{ab1}{b1}
\ncarr{f1axb3}{ab3}
\ncarr{f2target}{ab2}
\ncarr{f3target}{f1axb3}
\ncarc[arcangle=10,nodesepA=5pt,offsetA=2pt,nodesepB=2pt,offsetB=2pt]{->}{a}{f1target}
\alabel{f_1}[0.25]
\ncarc[arcangle=15,nodesepA=5pt,offsetA=2pt,nodesepB=2pt,offsetB=2pt]{->}{a}{f2target}
\alabel{f_2}
\ncarc[arcangle=10,nodesepA=5pt,offsetA=2pt,nodesepB=2pt,offsetB=2pt]{->}{a}{f3target}
\alabel{f_3}
\ncsar{f3target}{a}
\ncsar{f2target}{a}
\ncsar{f1target}{a}
\ncsar{ab2}{ab1}
\ncsar{ab3}{ab2}
\ncsar{f1axb3}{f2target}
\ncsar{b3}{b2}
\ncsar{b2}{b1}
\ncsar{b1}{abs}
\nccdar{a}{abs}

\note 4
How suppose that additional to the situation of para. \ref{omegarealisationwrtQ} the rules 
\gatdisplayrule{\yOmega{m}}{\isT{\Delta}} and  \gatdisplayrule{\yOmega{m}}{\ofT{t}{\Delta}} are derived rules of \gatUw. 
Let us denote these rules $r$ and $r_t$, respectively. Suppose that $I$ is an interpretation of $\gatUw$ in \catcw as mapping rules and contexts as described in para. \ref{omegarealisationwrtQ}.
From what we have said in para \ref{omegarealisationwrtQ}, the interpretation  $I$ will map rule $r$ to an object $b$ such that ${b_m \base b}$ in \catcw and it will map the rule $r_t$ to a section $g:b_m \morph b$.  

Now it follows by the substitution lemma (see \cite{Cartmell86})
that the substituted $r$ and $r_t$ rules: 
\gatdisplayrule{Q}{\isT{\Delta[s_1|y_1...s_m|y_m]}} 
and  \gatdisplayrule{Q}{\ofT{t[s_1|y_1...s_m|y_m]}{\Delta[s_1|y_1...s_m|y_m]}} are derived rules of \gatU. 

\highlight{We require that}
the substituted $r$ rule will be mapped by $I$ to the object $\smstar...\sonestar\crossx{a}{b}{1}$ and the substituted $r_t$ rule will
be mapped by $I$ to the morphism  $\smstar...\sonestar\crossx{a}{g}{1}$ (which is defined since $g$ is a section).

\newpage
\note
\oldt{As a stepping stone, we need  give an auxillary definition, the definition of  a `preinterpretation of \gatUw in \catc'.}
If \gatU is a generalised algebraic theory  and if \catcw is a contextual category then
a preinterpretation $I$ of  \gatU in \catcw consists of a pair :
\begin{itemize}
\item a mapping $\Isort$ that maps each sort symbol of \gatUw to  an object of \catc,
\item a mapping $\Iop$ that maps each operator symbol of \gatUw to a section of \catcw (i.e. to a morphism $f: A \morph B$ for some 
$A \base B$ in \catcw such that $f \circ p_B=id_A$).
\end{itemize}

\note We can state the conditions that a preinterpreation must specify in order that it can be extended to an interpretation.
If a preinterpretation meets these conditions then there is a unique interpretation that it extends to.
\begin{enumerate}[(i)]
\setlength\itemindent{2cm}
\item
whenever $A$ is a sort symbol of \gatUw that is introduced by the rule
$\frac{\xDelta{n}}{\isT{A(\xn)}}$
then $\Ihat(Rn) \base \Isort(A)$ in \catc, where $Rn$ is the rule
$\frac{\context{x}{\Delta}{n-1}}{\isT{\Delta_n}}$,
\item
whenever $F$ is an operator symbol of \gatUw that is introduced by the rule
$\frac{\xDelta{n}}{\ofT{F(\xn)}{\Delta}}$
then $\Iop(F) : \Ihat(Rn) \morph \Ihat(R)$  in \catcw and $\Iop(F)$ is a section,
\item whenever
$\frac{\xDelta{n}}{\Delta = \Delta'}$
is an axiom of \gatUw then $\Ihat(R) = \Ihat(R')$
where $R$ is the rule
$\frac{\xDelta{n}}{\isT{\Delta}}$
and $R'$ is the rule
$\frac{\xDelta{n}}{\isT{\Delta}}$,
\item whenever
$\frac{\xDelta{n}}{t = t' \in \Delta}$
is an axiom of \gatUw then $\Ihat(Rt) = \Ihat(Rt')$
where $Rt$ is the rule
$\frac{\xDelta{n}}{\ofT{t}{\Delta}}$
and $Rt'$ is the rule
$\frac{\xDelta{n}}{\ofT{t'}{\Delta}}$.
\end{enumerate}

\note The definition of $\Ihat$ from $\Isort$ and $\Iop$ whnever $I$ is an interpretation proceeds by induction 
on the derivation of rules in  \gatUw 
as described in the principles of derivation in Definition 2(b) of \cite{Cartmell86}. 
The only non-trivial parts of this definition relate to the rules
identified as CF1, CF2(a) and CF2(b)\footnote{So identified, by the way, as a mnemonic for cut-free.}. We consider each of these rules in turn.

\begin{point}
Rule CF1 states that for $n \geq 0$, for $1 \leq i \leq n+1$, from the derived rule 
$\frac{\xDelta{n}}{\isT{\Delta_{n+1}}}$ which we shall denote $R$ 
we may derive the rule
$\frac{\xDelta{n+1}}{\ofT{x_i}{\Delta_i}}$ which, in turn, we shall denote $R_{x_i}$.
Define $\Ihat(R_{x_i}) :  \Ihat(R) \morph \crossx{\Ihat(R)}{\Ihat(R_i)}{\Ihat(R_{i-1})}$
to be $\tuple{p_{\Ihat(R),\Ihat(R_{i-1})},p_{\Ihat(R),\Ihat(R_{i})}}$. 

This presumably is $s(p_{\Ihat(R),\Ihat(R_{i-1})})$ where $s$ is Vladimir's s-operator.
\end{point}
\begin{point}
CF2(a) states that if $A$ is a sort symbol introduced by
$\frac{\xDelta{n}}{\isT{A(\xn)}}$ 
and if $P$ is a context and $\tn$ are expressions then from the following rules, which we shall denote $R_{t_1}$,..$R_{t_n}$,
$\frac{P}{\ofT{t_1}{\Delta_1}}$,
$\frac{P}{\ofT{t_2}{\Delta_2[t_1|x_1]}}$,
... and 
$\frac{P}{\ofT{t_n}{\Delta_n[t_1|x_1,...t_{n-1}|x_{n-1}]}}$
we may derive the rule
$\frac{P}{\isT{A(t_1,...t_n)}}$ which we denote as $R$. 
Define $I(R)$ to be $\Ihat(R_n)^*...\Ihat(R_1)^*\crossx{\Ihat(R_n)}{I(A)}{1}$.\commentary{check this}
\end{point}
\begin{point}
CF2(b) \highlight{fill this in}
\end{point}

\begin{oldtt}
\begin{displaymath}
\begin{array}{c}
\crossx{a_n}{a_i}{\Rnode{cross}{a_{i-1}}} \\[0.9cm]
\Rnode{an}{a_n}\\[0.7cm]
%\Rnode{highervdots}{\vdots}\\
\Rnode{ai}{\begin{array}{c}
\vdots\\
a_i\\
\vdots
\end{array}} \\[1.1cm]
%\Rnode{lowervdots}{\vdots}\\[0.4cm]
\Rnode{a1}{a_1}\\[0.7cm]
\Rnode{abs}{1}
\end{array}
\end{displaymath}
\ncsar{cross}{an}
%\ncsar{an}{highervdots}
%\ncsar{lowervdots}{a1}
\ncsar{an}{ai}
\ncsar{ai}{a1}
\ncsar{a1}{abs}
\ncarc[arcangle=30,nodesepA=5pt,offsetA=2pt,nodesepB=2pt,offsetB=2pt]{->}{an}{cross}
\alabel{s(p_{a_n,a_i})}
\end{oldtt}
