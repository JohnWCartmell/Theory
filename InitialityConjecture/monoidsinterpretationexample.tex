
\note
Now we consider what constitutes a monoid internal to a contextual category \catc. That is to say 
we consider what constitutes an interpretation $I$ of the theory of monoids   in a contextual category \catc.
For this purpose we shall write the theory of monoids ($tm$) as a generalised algebraic theory like this: 
\begin{gatrules}
\gatintros
\gatintroducing{M}
\isT{M} \\
\gatintroducing{unit}
\ofT{unit}{M} \\
\gatintroducing{mult}
\gatsingular[6cm]{\ofT{x,y}{M}}{\ofT{mult(x,y)}{M}} \\
\gataxioms

\gatintroducing{ \gataxiomno{1} \\ \gataxiomno{2} }
\begin{gatgroup}{\ofT{x}{M}}
\gatleaf[6cm]{}{mult(unit,x)=x} \\
\gatleaf[6cm]{}{mult(x,unit)=x}
\end{gatgroup} \\
\gatintroducing{ \gataxiomno{3} }
\gatsingular[6cm]{\ofT{x,y,z}{M}}{mult(mult(x,y),z)=mult(x,mult(y,z))} 
\end{gatrules}
\note 
For the sakes of readability we write the interpretation $I(M)$ of the sort $M$ simply as $M$. Similarly we write $I(unit)$ as $unit$. We write $I(mult)$ as $m$. I will ask the reader  to distinguish for themselves 
those uses of `$M$' and `$unit$' in reference to sorts and operators of the theory $tm$ from those uses in reference to the interpretation of these sorts by the interpretation $I$ in the contextual category \catc. 

\newcommand{\wM}{\ofT{w}{M}}
\newcommand{\xM}{\ofT{x_1, x_2}{M}}
\newcommand{\yM}{\ofT{y_1, y_2, y_3}{M}}
\newcommand{\doubleM}{M^2}                       %{\crossx{M}{M}{1}}
\newcommand{\trebleM}{M^3}                       %{\crossx{\big(\doubleM\big)}{M}{1}}
\newcommand{\quadM}{M^4}                         % {\crossx{\big(\trebleM\big)}{M}{1}}
\newcommand{\spi}{s(p_{M^3,M^i})}
\newcommand{\sptrebleone}{s(p_{M^3,M^1})}
\newcommand{\sptrebletwo}{s(p_{M^3,M^2})}
\newcommand{\sptreblethree}{s(p_{M^3,M^3})}
\newcommand{\fmult}{m}  %macro used for name of section that monoidal multiplication maps to

\begin{lemma}
An internal monoid in a contextual category \catcw consists of
\begin{itemize}
\item An object $M$ of \catcw which is the interpretation of the sort $M$ of theory $tm$,
\item A section $\fmult$ of \catc, $\fmult \in Sect(\trebleM)$, that is the interpretation of the operator $mult$ of theory $tm$,
\item A section $unit$ of \catc, $unit \in Sect(M)$, that is the interpretation of the operator $unit$ of theory $tm$,
\end{itemize}
such that
\begin{equation}
\label{internalmonoidrepresentation1axiom1}
\tuple{p_M \circ unit,id_M}^*\fmult=s(id_M),
\end{equation}
\begin{equation}
\label{internalmonoidrepresentation1axiom2}
\tuple{id_M,p_M \circ unit}^*\fmult=s(id_M),
\end{equation}
and
\begin{equation}
\label{internalmonoidrepresentation1axiom3}
\bigtuple{(\tuple{y_1,y_2}^*\fmult)\circ q(p_{M^3,1},M),y_3}^*\fmult
=\bigtuple{y_1,(\tuple{y_2,y_3}^*\fmult)\circ q(p_{M^3,1},M)}^*\fmult
\end{equation}.

Equivalently an internal monoid in a contextual category \catc consists of
\begin{itemize}
\item An object $M$ of \catc,
\item A morphism $mult$ of \catc, $mult: \doubleM \morph M$ in \catc,
\item A morphism $unit$ of \catc, $unit: 1 \morph M$ in \catc
\end{itemize}
such that
\begin{equation}
\label{internalmonoidrepresentation2axiom1}
\tuple{p_M \circ unit,id_M}\circ mult=id_M,
\end{equation}
\begin{equation}
\label{internalmonoidrepresentation2axiom2}
\tuple{{id_M,p_M \circ unit}}\circ mult=id_M,
\end{equation}
and
\begin{equation}
\label{internalmonoidrepresentation2axiom3}
\bigtuple{\tuple{y_1,y_2}\circ mult,y_3}\circ mult
=\bigtuple{y_1,\tuple{y_2,y_3} \circ mult}\circ mult
\end{equation}.
\end{lemma}
\begin{proof}
Of these two equivalent representations the first results from a literal reading of the definition of interpretation as we demonstrate in table \ref{internalmonoidtable}. 

From the first representation the second follows as we now show.
First of all define $mult: \doubleM \morph M$ in \catcw from $\fmult: \doubleM \morph \trebleM$ by defining
$mult = \fmult \circ q(p_{\doubleM,1} , M)$. Now we show that each of the equations (\ref{internalmonoidrepresentation1axiom1}),
(\ref{internalmonoidrepresentation1axiom2}), and (\ref{internalmonoidrepresentation1axiom3}) hold of $\fmult$ iff
the respective equation (\ref{internalmonoidrepresentation2axiom1}),
(\ref{internalmonoidrepresentation2axiom2}) or (\ref{internalmonoidrepresentation2axiom3}) hold of $mult$.

That (\ref{internalmonoidrepresentation1axiom1}) holds iff (\ref{internalmonoidrepresentation2axiom1}) holds follows
because by lemma \lref{stactic}, (\ref{internalmonoidrepresentation2axiom1}) holds iff
\begin{equation}
\label{equivalenceone}
s(\tuple{p_M \circ unit,id_M} \circ mult) = s(id_M)
\end{equation}
which holds iff equation (\ref{internalmonoidrepresentation1axiom1}) holds because 

\begin{align*}
s(\tuple{p_M \circ unit,id_M}\circ mult) &= \tuple{p_M \circ unit,id_M} ^* s(mult )  && \mbox{by lemma \lref{sfglemma}} \\
             &= \tuple{p_M \circ unit,id_M} ^* s(\fmult \circ q(p_{\doubleM,1} , M)) && \mbox{definition of $mult$} \\
			       &= \tuple{p_M \circ unit,id_M} ^* \fmult                                &&  \mbox{by axiom (s2)}
\end{align*}

By the same reasoning, it follows that (\ref{internalmonoidrepresentation1axiom2}) holds iff (\ref{internalmonoidrepresentation2axiom2}) holds.

We shall show that (\ref{internalmonoidrepresentation1axiom3}) holds iff (\ref{internalmonoidrepresentation2axiom3}) holds.
First note, though, that
\begin{equation}
\label{thirdaxiomsubgoal}
 \tuple{y_1,y_2}\circ \fmult \circ q(p_{\doubleM,1} , M) = (\tuple{y_2,y_3}^*\fmult)\circ q(p_{M^3,1},M)
\end{equation}
because
\begin{align*}
lhs &= (\tuple{y_2,y_3}^*\fmult) \circ q(\tuple{y_2,y_3},M^3) \circ q(p_{M^2,1},M)  && \mbox{by definition of $^*$} \\
    &= (\tuple{y_2,y_3}^*\fmult) \circ q(\tuple{y_2,y_3} \circ p_{M^2,1},M)         && \mbox{because pullbacks cohere} \\
    &= (\tuple{y_2,y_3}^*\fmult) \circ q(p_{M^3,1},M)                               && \mbox{because $1$ is terminal }\\
    &= rhs
\end{align*}

Now that (\ref{internalmonoidrepresentation1axiom3}) holds iff (\ref{internalmonoidrepresentation2axiom3}) holds
follows because by lemma \lref{stactic}, (\ref{internalmonoidrepresentation2axiom3}) holds iff

\begin{equation}
\label{sofinternalmonoidrepresentation2axiom3}
s(\bigtuple{\tuple{y_1,y_2}\circ mult,y_3}\circ mult)
=s(\bigtuple{y_1,\tuple{y_2,y_3} \circ mult}\circ mult)
\end{equation}
and the lhs of (\ref{sofinternalmonoidrepresentation2axiom3}) and the lhs of (\ref{internalmonoidrepresentation1axiom3}) 
are identical because
\begin{align*}
s(\bigtuple{\tuple{y_1,y_2}\circ mult,y_3}\circ mult) 
    &= \bigtuple{\tuple{y_1,y_2}\circ mult,y_3} ^* s(mult) && \mbox{by lemma \lref{sfglemma}} \\
		&= \bigtuple{\tuple{y_1,y_2}\circ mult,y_3} ^* s(\fmult \circ q(p_{\doubleM,1} , M)) && \mbox{definition of $mult$} \\
		&= \bigtuple{\tuple{y_1,y_2}\circ mult,y_3} ^* \fmult                                &&  \mbox{by axiom (s2)} \\
		&= \bigtuple{\tuple{y_1,y_2}\circ \fmult \circ q(p_{\doubleM,1} , M),y_3} ^* \fmult  
		                                                                    &&  \mbox{definition of $mult$} \\	
	  &= \bigtuple{(\tuple{y_2,y_3}^*\fmult)\circ q(p_{M^3,1},M),y_3} ^* \fmult 
		                                                                    &&  \mbox{by use of (\ref{thirdaxiomsubgoal})}
\end{align*}
Similarly, the rhs of (\ref{sofinternalmonoidrepresentation2axiom3}) and the rhs of (\ref{internalmonoidrepresentation1axiom3}) 
can be shown to be identical.

\begin{table}[H]
\caption{Deriving what constitutes an intepretation of the theory of monoids $tm$ in a contextual category \catc.
The result is shown (highlighted) in rows (\ref{tm1}), (\ref{tm11}), (\ref{tmax1}), (\ref{tmax2}) and (\ref{tmax3}). Other rows 
show how we derive this result from clauses (i) through (viii) of the definition of interpretation.}
\label{internalmonoidtable}

\setlength{\arrayrulewidth}{1mm}
\setlength{\tabcolsep}{2pt}
\begin{tabular}{l l  c  p{0cm} l  l}
\multicolumn{2}{l}{Derived Rule} &&& Interpretation by $I$ in \catcw & Reason why\\
\hline
\gatinterpretationintro {tm1}{}{\isT{M}}{M \in Cover(1)}{(i)} \\
\\[-0.1cm]
\gatinterpretationdetail{tm2}{\wM}{\isT{M}}{\doubleM \in Cover(M)}{(v) and (\ref{tm1})} \\[0.3cm]
\gatinterpretationdetail{tm3}{\wM}{\ofT{w}{M}}{s(id_M) \in Sect(\doubleM)}{(ii)(b) and (\ref{tm1})} \\[0.3cm]
\gatinterpretationdetail{tm4}{\xM}{\isT{M}}{\trebleM \in Cover(\doubleM)}{(v),(\ref{tm1}) and (\ref{tm2})} \\[0.3cm]
\gatinterpretationdetail{tm5}{\xM}{\ofT{x_1}{M}}{s(id_M) \in Sect(\trebleM)}{(ii)(b) and (\ref{tm2})} \\[0.3cm]
\gatinterpretationdetail{tm6}{\xM}{\ofT{x_2}{M}}{s(p_M) \in Sect(\trebleM)}{(ii)(b) and (\ref{tm2})} \\[0.3cm]
\gatinterpretationdetail{tm7}{\yM}{\isT{M}}{\quadM \in Cover(\trebleM)}{(v) and (\ref{tm4})} \\[0.3cm]
\gatinterpretationdetail{tm8}{\yM}{\ofT{y_1}{M}}{\sptrebleone \in Sect(\quadM)}{(ii)(b) and (\ref{tm4})} \\[0.3cm]
\gatinterpretationdetail{tm9}{\yM}{\ofT{y_2}{M}}{\sptrebletwo \in Sect(\quadM)}{(ii)(b) and (\ref{tm4})} \\[0.3cm]
\gatinterpretationdetail{tm10}{\yM}{\ofT{y_3}{M}}{\sptreblethree \in Sect(\quadM)}{(ii)(b) and (\ref{tm4})} \\[0.3cm]
\gatinterpretationintro {tm11}{}{\ofT{unit}{M}}{unit \in Sect(M)}{(ii)(a) wrt (\ref{tm1})} \\
\\[-0.1cm]
\gatinterpretationdetail{tm12}{\wM}{\ofT{unit}{M}}{\crossx{M}{unit}{1} \in Sect(\doubleM)}{(vi),(\ref{tm11}) and (\ref{tm9})} \\[0.3cm]
\gatinterpretationintro{tm13}{\xM}{\ofT{\fmult(x_1,x_2)}{M}}{\fmult \in Sect(\trebleM)}{(ii)(a) wrt (\ref{tm4})} \\
\\[-0.1cm]
\gatinterpretationdetail{tm14}{\wM}
                        {\ofT{\fmult(w,unit)}{M}}
                        {\duple{s(id_M),\crossx{M}{unit}{1}}^*\fmult \in Sect(\doubleM)}
												{(viii'),(\ref{tm3}),(\ref{tm12}) and(\ref{tm13}) }\\[0.2cm]
\gatinterpretationmapeqv {\tuple{id_M,p_M \circ unit}^*\fmult} 
												{lemma \lref{thedupletuplelemma} and lemma \lref{crosssectionlemma}}\\[0.2cm]
\gatinterpretationdetail{tm15}{\wM}
                        {\ofT{\fmult(unit,w)}{M}}
                        {\duple{\crossx{M}{unit}{1},s(id_M)}^*\fmult \in Sect(\doubleM)}
												{(viii),(\ref{tm12}),(\ref{tm3}) and (\ref{tm13}) } \\[0.2cm]
\gatinterpretationmapeqv{\tuple{p_M \circ unit,id_M}^*\fmult}
												{lemma \lref{thedupletuplelemma} and lemma \lref{crosssectionlemma}}\\[0.2cm]
\gatinterpretationdetail{tm16}{\yM}
                        {\ofT{\fmult(y_1,y_2)}{M}}
												{\duple{\sptrebleone,\sptrebletwo}^*\fmult}
												{(viii),(\ref{tm13}),(\ref{tm8}) and (\ref{tm9})}                  \\[0.2cm]
\gatinterpretationmapeqv {\tuple{y_1,y_2}^*\fmult}
												{lemma \lref{thedupletuplelemma} and axiom (s3)}                    \\[0.2cm]										
												&&&&\multicolumn{2}{p{8cm}}{where  $y_i:M^3 \morph M^i$ is defined  
												             if $i = 1$ by  $y_1=p_{M^3,M}$ else, if 
																		 $i >1$, by $y_i=p_{M^3,M^i}\circ q(p_{M^i,1},M)$} \\[0.2cm]
\gatinterpretationdetail{tm17}{\yM}
                        {\ofT{\fmult(y_2,y_3)}{M}}
												{\duple{\sptrebletwo,\sptreblethree}^*\fmult}
												{(viii),(\ref{tm13}),(\ref{tm9}) and (\ref{tm10})}  \\[0.2cm]
\gatinterpretationmapeqv {\tuple{y_2,y_3}^*\fmult} 
												{lemma \lref{thedupletuplelemma} and axiom (s3)}\\[0.2cm]
\gatinterpretationdetail{tm18}{\yM}
                        {\fmult(\fmult(y_1,y_2),y_3)}
												{\duple{\tuple{y_1,y_2}^*\fmult,\sptreblethree}^*\fmult}
												{(viii),(\ref{tm13}),(\ref{tm16}) and (\ref{tm10})}  \\[0.2cm]
\gatinterpretationmapeqv {\bigtuple{(\tuple{y_1,y_2}^*\fmult)\circ q(p_{M^3,1},M),y_3}^*\fmult} 
												{lemma \lref{thedupletuplelemma} and axiom (s3) (twice)} \\[0.2cm]
\gatinterpretationdetail{tm19}{\yM}
                        {\fmult(y_1,\fmult(y_2,y_3))}
												{\duple{\sptrebleone,\tuple{y_2,y_3}^*\fmult}^*\fmult}
												{(viii),(\ref{tm13}),(\ref{tm8}) and (\ref{tm17})} \\[0.2cm]
\gatinterpretationmapeqv {\bigtuple{y_1,(\tuple{y_2,y_3}^*\fmult)\circ q(p_{M^3,1},M)}^*\fmult} 
												{lemma \lref{thedupletuplelemma}   and axiom (s3) (twice)}\\[0.2cm]
%												\rowcolor{lightergrey}
\gatinterpretationaxcond{tmax1}{\wM}{\fmult(unit,w)=w}{\tuple{p_M \circ unit,id_M}^*\fmult=s(id_M)}{(iv), (\ref{tm15}) and (\ref{tm3})} \\[0.2cm]
\arrayrulecolor{white}\hline
%\gatinterpretationaxeqv {\tuple{p_M \circ unit,id_M}\comp \duple{\fmult}=id_M}{could push transform back?} \\
%												\rowcolor{lightergrey}
\gatinterpretationaxcond{tmax2}{\wM}{\fmult(w,unit)=w}{\tuple{id_M,p_M \circ unit}^*\fmult=s(id_M)}{(iv), (\ref{tm14}) and (\ref{tm3})} \\[0.2cm]
%\gatinterpretationaxeqv {\tuple{{id_M,p_M \circ unit}}\comp \duple{\fmult}=id_M}{new lemmas}  \\
%												\rowcolor{lightergrey}
\arrayrulecolor{white}\hline
\gatinterpretationaxcond{tmax3}{\yM}{\fmult(\fmult(y_1,y_2),y_3)}
                                     {\bigtuple{(\tuple{y_1,y_2}^*\fmult)\circ q(p_{M^3,1},M),y_3}^*\fmult} \\
																		 &\hspace{2cm}$=\fmult(y_1,\fmult(y_2,y_3))$
																		 &&& \cellcolor{lightergrey}\hspace{0.5cm}
																		    $=\bigtuple{y_1,(\tuple{y_2,y_3}^*\fmult)\circ q(p_{M^3,1},M)}^*\fmult$
																		                           &{(iv), (\ref{tm18}) and (\ref{tm19})} 
\end{tabular}
\end{table}
\end{proof}
