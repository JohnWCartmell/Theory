
\note
Reminder: The theory of monoids ($tm$) is
\begin{gatrules}
\gatintros
\gatintroducing{M}
\isT{M} \\
\gatintroducing{unit}
\ofT{unit}{M} \\
\gatintroducing{mult}
\gatsingular[6cm]{\ofT{x,y}{M}}{\ofT{mult(x,y)}{M}} \\
\gataxioms

\gatintroducing{ \gataxiomno{1} \\ \gataxiomno{2} }
\begin{gatgroup}{\ofT{x}{M}}
\gatleaf[6cm]{}{mult(unit,x)=x} \\
\gatleaf[6cm]{}{mult(x,unit)=x}
\end{gatgroup} \\
\gatintroducing{ \gataxiomno{3} }
\gatsingular[6cm]{\ofT{x,y,z}{M}}{mult(mult(x,y),z)=mult(x,mult(y,z))} 
\end{gatrules}
\note Let us consider what constitutes an interpretation $I$ of the theory of monoids ($tm$) in a contextual catgeory \catc.
For the sakes of readability we write the interpretation $I(M)$ of the sort $M$ simply as $M$. Similarly we write $I(unit)$ as $unit$ and $I(multt)$ as $mult$. I will ask the reader  to distinguish for themselves 
those uses of `$M$' `$mult$' and `$unit$' in reference to sorts and operators of the theory $tm$ from those uses in reference to the interpretation of these sorts by the interpretation$I$ in the contextual category \catc. 

\newcommand{\wM}{\ofT{w}{M}}
\newcommand{\xM}{\ofT{x_1, x_2}{M}}
\newcommand{\yM}{\ofT{y_1, y_2, y_3}{M}}
\newcommand{\doubleM}{M^2}                       %{\crossx{M}{M}{1}}
\newcommand{\trebleM}{M^3}                       %{\crossx{\big(\doubleM\big)}{M}{1}}
\newcommand{\quadM}{M^4}                         % {\crossx{\big(\trebleM\big)}{M}{1}}
\newcommand{\spi}{s(p_{M^3,M^i})}
\newcommand{\sptrebleone}{s(p_{M^3,M^1})}
\newcommand{\sptrebletwo}{s(p_{M^3,M^2})}
\newcommand{\sptreblethree}{s(p_{M^3,M^3})}
\newcommand{\fmult}{m}  %macro used for name of section that monoidal multiplication maps to

\definecolor{lightergrey}{rgb}{0.9,0.9,0.9}
\setlength{\arrayrulewidth}{1mm}
\setlength{\tabcolsep}{2pt}
\begin{tabular}{l l  c  p{0cm} l  l}
\gatinterpretationintro {tm1}{}{\isT{M}}{M \in Cover(1)}{(i)} \\
\\[-0.1cm]
\gatinterpretationdetail{tm2}{\wM}{\isT{M}}{\doubleM \in Cover(M)}{(v) and (\ref{tm1})} \\[0.3cm]
\gatinterpretationdetail{tm3}{\wM}{\ofT{w}{M}}{s(id_M) \in Sect(\doubleM)}{(ii)(b) and (\ref{tm1})} \\[0.3cm]
\gatinterpretationdetail{tm4}{\xM}{\isT{M}}{\trebleM \in Cover(\doubleM)}{(v),(\ref{tm1}) and (\ref{tm2})} \\[0.3cm]
\gatinterpretationdetail{tm5}{\xM}{\ofT{x_1}{M}}{s(id_M) \in Sect(\trebleM)}{(ii)(b) and (\ref{tm2})} \\[0.3cm]
\gatinterpretationdetail{tm6}{\xM}{\ofT{x_2}{M}}{s(p_M) \in Sect(\trebleM)}{(ii)(b) and (\ref{tm2})} \\[0.3cm]
\gatinterpretationdetail{tm7}{\yM}{\isT{M}}{\quadM \in Cover(\trebleM)}{(v) and (\ref{tm4})} \\[0.3cm]
\gatinterpretationdetail{tm8}{\yM}{\ofT{y_1}{M}}{\sptrebleone \in Sect(\quadM)}{(ii)(b) and (\ref{tm4})} \\[0.3cm]
\gatinterpretationdetail{tm9}{\yM}{\ofT{y_2}{M}}{\sptrebletwo \in Sect(\quadM)}{(ii)(b) and (\ref{tm4})} \\[0.3cm]
\gatinterpretationdetail{tm10}{\yM}{\ofT{y_3}{M}}{\sptreblethree \in Sect(\quadM)}{(ii)(b) and (\ref{tm4})} \\[0.3cm]
\gatinterpretationintro {tm11}{}{\ofT{unit}{M}}{unit \in Sect(M)}{(ii)(a) wrt (\ref{tm1})} \\
\\[-0.1cm]
\gatinterpretationdetail{tm12}{\wM}{\ofT{unit}{M}}{\crossx{M}{unit}{1} \in Sect(\doubleM)}{(vi),(\ref{tm11}) and (\ref{tm9})} \\[0.3cm]
\gatinterpretationintro{tm13}{\xM}{\ofT{\fmult(x_1,x_2)}{M}}{\fmult \in Sect(\trebleM)}{(ii)(a) wrt (\ref{tm4})} \\
\\[-0.1cm]
\gatinterpretationdetail{tm14}{\wM}
                        {\ofT{\fmult(w,unit)}{M}}
                        {\duple{s(id_M),\crossx{M}{unit}{1}}^*\fmult \in Sect(\doubleM)}
												{(viii'),(\ref{tm3}),(\ref{tm12}) and(\ref{tm13}) }\\[0.2cm]
\gatinterpretationmapeqv {\tuple{id_M,p_M \circ unit}^*\fmult} 
												{lemma \ref{thedupletuplelemma} and lemma \ref{crosssectionlemma}}\\[0.2cm]
\gatinterpretationdetail{tm15}{\wM}
                        {\ofT{\fmult(unit,w)}{M}}
                        {\duple{\crossx{M}{unit}{1},s(id_M)}^*\fmult \in Sect(\doubleM)}
												{(viii),(\ref{tm12}),(\ref{tm3}) and (\ref{tm13}) } \\[0.2cm]
\gatinterpretationmapeqv{\tuple{p_M \circ unit,id_M}^*\fmult}
												{lemma \ref{thedupletuplelemma} and lemma \ref{crosssectionlemma}}\\[0.2cm]
\gatinterpretationdetail{tm16}{\yM}
                        {\ofT{\fmult(y_1,y_2)}{M}}
												{\duple{\sptrebleone,\sptrebletwo}^*\fmult}
												{(viii),(\ref{tm13}),(\ref{tm8}) and (\ref{tm9})}                  \\[0.2cm]
\gatinterpretationmapeqv {\tuple{y_1,y_2}^*\fmult}
												{lemma \ref{thedupletuplelemma}}                                      \\[0.2cm]										
												&&&&\multicolumn{2}{l}{where  $y_i:M^3 \morph M^i$ is defined by $y_i=p_{M^3,M^i}\circ q(p_{M^i,1},M)$} \\[0.2cm]
\gatinterpretationdetail{tm17}{\yM}
                        {\ofT{\fmult(y_2,y_3)}{M}}
												{\duple{\sptrebletwo,\sptreblethree}^*\fmult}
												{(viii),(\ref{tm13}),(\ref{tm9}) and (\ref{tm10})}  \\[0.2cm]
\gatinterpretationmapeqv {\tuple{y_2,y_3}^*\fmult} 
												{lemma \ref{thedupletuplelemma}}\\[0.2cm]
\gatinterpretationdetail{tm18}{\yM}
                        {\fmult(\fmult(y_1,y_2),y_3)}
												{\duple{\tuple{y_1,y_2}^*\fmult,\sptreblethree}^*\fmult}
												{(viii),(\ref{tm13}),(\ref{tm16}) and (\ref{tm10})}  \\[0.2cm]
\gatinterpretationmapeqv {\bigtuple{(\tuple{y_1,y_2}^*\fmult)\circ q(p_{m^3,1},M),y_3}^*\fmult} 
												{lemma \ref{thedupletuplelemma}} \\[0.2cm]
\gatinterpretationdetail{tm19}{\yM}
                        {\fmult(y_1,\fmult(y_2,y_3))}
												{\duple{\sptrebleone,\tuple{y_2,y_3}^*\fmult}^*\fmult}
												{(viii),(\ref{tm13}),(\ref{tm8}) and (\ref{tm17})} \\[0.2cm]
\gatinterpretationmapeqv {\bigtuple{y_1,(\tuple{y_2,y_3}^*\fmult)\circ q(p_{m^3,1},M)}^*\fmult} 
												{lemma \ref{thedupletuplelemma}}\\[0.2cm]
%												\rowcolor{lightergrey}
\gatinterpretationaxcond{tmax1}{\wM}{\fmult(unit,w)=w}{\tuple{p_M \circ unit,id_M}^*\fmult=s(id_M)}{(iv), (\ref{tm15}) and (\ref{tm3})} \\[0.2cm]
\arrayrulecolor{white}\hline
%\gatinterpretationaxeqv {\tuple{p_M \circ unit,id_M}\comp \duple{\fmult}=id_M}{could push transform back?} \\
%												\rowcolor{lightergrey}
\gatinterpretationaxcond{tmax1}{\wM}{\fmult(w,unit)=w}{\tuple{id_M,p_M \circ unit}^*\fmult=s(id_M)}{(iv), (\ref{tm14}) and (\ref{tm3})} \\[0.2cm]
%\gatinterpretationaxeqv {\tuple{{id_M,p_M \circ unit}}\comp \duple{\fmult}=id_M}{new lemmas}  \\
%												\rowcolor{lightergrey}
\arrayrulecolor{white}\hline
\gatinterpretationaxcond{tmax1}{\yM}{\fmult(\fmult(y_1,y_2),y_3)}
                                     {\bigtuple{(\tuple{y_1,y_2}^*\fmult)\circ q(p_{m^3,1},M),y_3}^*\fmult} \\
																		 &\hspace{2cm}$=\fmult(y_1,\fmult(y_2,y_3))$
																		 &&& \cellcolor{lightergrey}\hspace{0.5cm}
																		    $=\bigtuple{y_1,(\tuple{y_2,y_3}^*\fmult)\circ q(p_{m^3,1},M)}^*\fmult$
																		                           &{(iv), (\ref{tm18}) and (\ref{tm19})} 
\end{tabular}

