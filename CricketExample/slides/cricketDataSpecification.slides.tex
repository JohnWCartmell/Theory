


\begin{frame}{Cricket Logical Model}
\begin{center}
\scalebox{0.75}{\begin{erdiagram}{10.35}{6.423374999999999}

\eret{1.3}{-2}{5.3}{-1.1}{0.2}{1}\eretname{1.7}{-1.45}{l}{match}
\erattr{1.5}{-1.65}{1}{0}{id}
\eret{0.204}{-4.75}{1.936}{-2.95}{0.2}{1}\eretname{0.377}{-3.3}{l}{innings}
\erattr{0.404}{-3.5}{1}{0}{number}
\eret{4.936}{-4.75}{6.396}{-2.95}{0.2}{1}\eretname{5.082}{-3.3}{l}{side}
\erattr{5.136}{-3.5}{1}{0}{name}
\eret{0.312}{-7.75}{1.827}{-5.95}{0.2}{1}\eretname{0.464}{-6.3}{l}{over}
\erattr{0.513}{-6.5}{1}{0}{number}
\eret{4.908}{-7.75}{6.423}{-5.95}{0.2}{1}\eretname{5.06}{-6.3}{l}{player}
\erattr{5.108}{-6.5}{1}{0}{number}
\erattr{5.108}{-6.8}{1}{1}{name}
\eret{0.234}{-9.85}{1.906}{-8.95}{0.2}{1}\eretname{0.401}{-9.3}{l}{delivery}
\erattr{0.434}{-9.5}{1}{0}{number}
\eret{0}{-0.2}{6.423}{0.3}{0.2}{1}

% relationship 
\errelname{3.45}{-0.5}{l}{}\errelarm{3.3}{-0.2}{3.3}{-0.65}{1}{0}\errelarm{3.3}{-0.65}{3.3}{-1.1}{1}{0}\ercrowfoot{3.3}{-0.95}{3.15}{-1.1}{3.3}{-1.1}{3.45}{-1.1}{0}
% relationship 
\errelname{2.483}{-2.3}{r}{}\errelname{1.22}{-2.8}{l}{..}\errelarm{2.633}{-2}{2.633}{-2.075}{1}{0}\errelarm{1.07}{-2.738}{1.07}{-2.95}{1}{0}\errelangle{2.633}{-2.075}{2.633}{-2.15}{1.852}{-2.338}{1}{0}\errelangle{1.852}{-2.338}{1.07}{-2.525}{1.07}{-2.738}{1}{0}\eridcomprel{0.9699999999999999}{1.17}{-2.7}\ercrowfoot{1.07}{-2.8}{0.92}{-2.95}{1.07}{-2.95}{1.22}{-2.95}{0}
% relationship 
\errelname{4.117}{-2.3}{l}{}\errelname{5.816}{-2.8}{l}{..}\errelarm{3.967}{-2}{3.967}{-2.075}{1}{0}\errelarm{5.666}{-2.738}{5.666}{-2.95}{1}{0}\errelangle{3.967}{-2.075}{3.967}{-2.15}{4.816}{-2.338}{1}{0}\errelangle{4.816}{-2.338}{5.666}{-2.525}{5.666}{-2.738}{1}{0}\eridcomprel{5.565875}{5.765874999999999}{-2.7}\ercrowfoot{5.666}{-2.8}{5.516}{-2.95}{5.666}{-2.95}{5.816}{-2.95}{0}
% relationship 
\errelname{1.22}{-5.05}{l}{}\errelname{1.22}{-5.8}{l}{..}\errelarm{1.07}{-4.75}{1.07}{-5.35}{1}{0}\errelarm{1.07}{-5.35}{1.07}{-5.95}{1}{0}\eridcomprel{0.9699999999999999}{1.17}{-5.699999999999999}\ercrowfoot{1.07}{-5.8}{0.92}{-5.95}{1.07}{-5.95}{1.22}{-5.95}{0}
% relationship battingSide
\errelname{2.086}{-3.4}{l}{battingSide}\errelarm{1.936}{-3.55}{3.436}{-3.55}{1}{0}\errelarm{3.436}{-3.55}{4.936}{-3.55}{0}{0}\ercrowfoot{2.086}{-3.55}{1.936}{-3.4}{1.936}{-3.55}{1.936}{-3.7}{0}
% relationship fieldingSide
\errelname{2.086}{-4}{l}{fieldingSide}\errelarm{1.936}{-4.15}{3.436}{-4.15}{1}{0}\errelarm{3.436}{-4.15}{4.936}{-4.15}{0}{0}\ercrowfoot{2.086}{-4.15}{1.936}{-4}{1.936}{-4.15}{1.936}{-4.3}{0}
% relationship 
\errelname{5.816}{-5.05}{l}{}\errelname{5.816}{-5.8}{l}{..}\errelarm{5.666}{-4.75}{5.666}{-5.35}{1}{0}\errelarm{5.666}{-5.35}{5.666}{-5.95}{1}{0}\eridcomprel{5.565875}{5.765874999999999}{-5.699999999999999}\ercrowfoot{5.666}{-5.8}{5.516}{-5.95}{5.666}{-5.95}{5.816}{-5.95}{0}
% relationship 
\errelname{1.22}{-8.05}{l}{}\errelname{1.22}{-8.8}{l}{..}\errelarm{1.07}{-7.75}{1.07}{-8.35}{1}{0}\errelarm{1.07}{-8.35}{1.07}{-8.95}{1}{0}\eridcomprel{0.9699999999999999}{1.17}{-8.7}\ercrowfoot{1.07}{-8.8}{0.92}{-8.95}{1.07}{-8.95}{1.22}{-8.95}{0}
% relationship bowler
\errelname{1.977}{-6.7}{l}{bowler}\errelarm{1.827}{-6.85}{3.368}{-6.85}{1}{0}\errelarm{3.368}{-6.85}{4.908}{-6.85}{0}{0}\ercrowfoot{1.977}{-6.85}{1.827}{-6.7}{1.827}{-6.85}{1.827}{-7}{0}
% relationship facingBatter
\errelname{2.056}{-9.7}{l}{facingBatter}\errelarm{1.906}{-9.4}{2.506}{-9.4}{1}{0}\errelarm{4.708}{-7.57}{4.908}{-7.57}{0}{0}\errelangle{2.506}{-9.4}{3.106}{-9.4}{3.807}{-8.485}{1}{0}\errelangle{3.807}{-8.485}{4.508}{-7.57}{4.708}{-7.57}{0}{0}\ercrowfoot{2.056}{-9.4}{1.906}{-9.25}{1.906}{-9.4}{1.906}{-9.55}{0}
\end{erdiagram}
}
\end{center}
\end{frame}

%\scopeTriangle{subject}{domain}{codomain}{apex}{diagonal}{riser}
\newcommand{\scopeTriangle}[6]
{
 \begin{array}{c  c  c}
                & \Rnode{apex}{#4} &                \\[1cm]
\Rnode{dom}{#2} &                  & \Rnode{cod}{#3} 
\end{array}
\begin{arrows}
\ncarr{dom}{cod}
\blabel{#1}
\ncsar{dom}{apex}
\alabel{#5}[0.2]
\ncsar{cod}{apex}
\blabel{#6}[0.2]
\end{arrows} 
}


\begin{frame}{the battingSide}
\begin{equation}
\label{battingSideScopeText}
\parbox{9.5cm}{The $battingSide$ within an $innnings$ of a $match$ is a $side$ within that same $match$.}
\end{equation}
\medskip
\begin{equation}
\label{battingSideScopeDiagram}
\scopeTriangle{battingSide}{innings}{side}{match}{..}{..}
\end{equation}
\medskip
\begin{itemize}
\item Statement (\ref{battingSideScopeText}) specifies the \textit{scope} of the $batingSide$ relationship.
\item It means that starting from an $innings$ there are two navigation paths that are equivalent:
\begin{itemize}
  \item $..$ 
  \item $battingSide/..$
\end{itemize}
\item This fact can be represented by saying that diagram (\ref{battingSideScopeDiagram}) commutes.
%\item In ERScript reference relationships may have a scope specified for them. 
\end{itemize}
\end{frame}

\begin{frame}{the fieldingSide}
The $fieldingSide$ likewise. 
\medskip
\begin{equation}
\scopeTriangle{fieldingSide}{innings}{side}{match}{..}{..}
\end{equation}
\end{frame}

\begin{frame}{The bowler}
\begin{equation}
\label{bowlerScopeText}
\parbox{9.5cm}{The $bowler$ within an $over$ of an $innnings$ of a $match$ is a $player$ within the $fieldingSide$ within that $innings$.}
\end{equation}
\medskip
\begin{equation}
\label{bowlerScopeDiagram}
\ccsquareoutline{3cm}{1.0cm}{innings}{side}{over}{player}
\ccsquareacross{fieldingSide}{bowler}
\ccsquareup{..}{..}
\end{equation}
\medskip
\begin{itemize}
\item Statement (\ref{bowlerScopeText}) specifies the \textit{scope} of the $bowler$ relationship.
\item It means that starting from an instance of $over$ there are two navigation paths that are equivalent:
\begin{itemize}
  \item $../fieldingSide$ 
  \item $bowler/..$
\end{itemize}
\item In other words, diagram (\ref{bowlerScopeDiagram}),above, commutes.
%\item In ERScript reference relationships may have a scope specified for them. 
\end{itemize}
\end{frame}

\begin{frame}{The facingBatter}
\begin{equation}
\label{facingBatterScopeText}
\parbox{9.5cm}{The $facingBatter$ for a $delivery$ of an $over$ of an $innnings$ is a $player$ within the $battingingSide$ within that $innings$}.
\end{equation}
\medskip
\begin{equation}
\label{facingBatterScopeDiagram}
\begin{array}{c p{2cm} c}
\Rnode{innings}{innings} && \Rnode{side}{side} \\[0.75cm]
\Rnode{over}{over}    &&      \\[0.75cm]
\Rnode{delivery}{delivery} && \Rnode{player}{player} 
\end{array}
\begin{arrows}
\ncsar{delivery}{over}
\alabel{..}
\ncsar{over}{innings}
\alabel{..}
\ncsar{player}{side}
\blabel{..}
\ncarr{innings}{side}
\alabel{battingSide}
\ncarr{delivery}{player}
\blabel{facingBatter}
\end{arrows}
\end{equation}
\medskip
\begin{itemize}
\item (\ref{facingBatterScopeText}) specifies the \textit{scope} of the $facingBatter$ relationship.
\item It means that starting from an instance of $delivery$ there are two navigation paths that are equivalent:
\begin{itemize}
  \item $../../battingSide$ 
  \item $facingBatter/..$
\end{itemize}
\item In other words, diagram (\ref{facingBatterScopeDiagram}), above, commutes.
%\item In ERScript reference relationships may have a scope specified for them. 
\end{itemize}
\end{frame}


\begin{frame}{Cricket as Data Specification}
\newcommand{\logicalAnnotation}
{\begin{tabular}[b]{p{6.5cm}}
Logical or Conceptual\\
\footnotesize  Start here with along with specification of
scopes for reference relationships (not show)
\end{tabular}
}
\newcommand{\hierarchicalAnnotation}
{\begin{tabular}[b]{p{6.5cm}}
Physical...Hierarchical (auto-generated)\\
\footnotesize Suitable for representation in XML or IDL.
 fk attrs added to represent horizontal reference (R) relationships.
 In this representation the (D) relationships are implemented by structural containment.
\end{tabular}
}
\newcommand{\relationalAnnotation}
{\begin{tabular}[b]{p{6.5cm}}
Physical...Relational (auto-generated)\\
\footnotesize  For representation in a relational schema.
Further fk attrs added to represent the dependency (D) relationships.
\end{tabular}
}
\begin{tabular}{l l}
\raisebox{-1.9cm}{\scalebox{0.75}{\begin{erdiagram}{11.549999999999999}{9.113750000000001}

\eret{2.56}{-2}{6.56}{-1.1}{0.2}{1}\eretname{2.96}{-1.45}{l}{match}
\erattr{2.76}{-1.65}{1}{0}{id}
\eret{0.006}{-4.75}{3.634}{-2.95}{0.2}{1}\eretname{0.369}{-3.3}{l}{innings}
\onslide<3->{\erdattr{0.206}{-3.5}{1}{0}{match\textunderscore id}}
\erattr{0.206}{-3.8}{1}{0}{number}
\onslide<2->{\erdattr{0.206}{-4.1}{1}{1}{battingSide\textunderscore name}}
\onslide<2->{\erdattr{0.206}{-4.4}{1}{1}{fieldingSide\textunderscore name}}
\eret{6.634}{-4.75}{9.114}{-2.95}{0.2}{1}\eretname{6.882}{-3.3}{l}{side}
\onslide<3->{\erdattr{6.834}{-3.5}{1}{0}{match\textunderscore id}}
\erattr{6.834}{-3.8}{1}{0}{name}
\eret{0.298}{-7.75}{3.343}{-5.95}{0.2}{1}\eretname{0.602}{-6.3}{l}{over}
\onslide<3->{\erdattr{0.498}{-6.5}{1}{0}{match\textunderscore id}}
\onslide<3->{\erdattr{0.498}{-6.8}{1}{0}{innings\textunderscore number}}
\erattr{0.498}{-7.1}{1}{0}{number}
\onslide<2->{\erdattr{0.498}{-7.4}{1}{1}{bowler\textunderscore number}}
\eret{6.67}{-7.75}{9.078}{-5.95}{0.2}{1}\eretname{6.911}{-6.3}{l}{player}
\onslide<3->{\erdattr{6.87}{-6.5}{1}{0}{match\textunderscore id}}
\onslide<3->{\erdattr{6.87}{-6.8}{1}{0}{side\textunderscore name}}
\erattr{6.87}{-7.1}{1}{0}{number}
\erattr{6.87}{-7.4}{1}{1}{name}
\eret{-0.021}{-11.05}{3.661}{-8.95}{0.2}{1}\eretname{0.347}{-9.3}{l}{delivery}
\onslide<3->{\erdattr{0.179}{-9.5}{1}{0}{match\textunderscore id}}
\onslide<3->{\erdattr{0.179}{-9.8}{1}{0}{innings\textunderscore number}}
\onslide<3->{\erdattr{0.179}{-10.1}{1}{0}{over\textunderscore number}}
\erattr{0.179}{-10.4}{1}{0}{number}
\onslide<2->{\erdattr{0.179}{-10.7}{1}{1}{facingBatter\textunderscore number}}
\eret{0}{-0.2}{9.114}{0.3}{0.2}{1}

% relationship 
\errelname{4.71}{-0.5}{l}{}\errelarm{4.56}{-0.2}{4.56}{-0.65}{1}{0}\errelarm{4.56}{-0.65}{4.56}{-1.1}{1}{0}\ercrowfoot{4.56}{-0.95}{4.41}{-1.1}{4.56}{-1.1}{4.71}{-1.1}{0}
% relationship 
\errelname{3.743}{-2.3}{r}{}\errelarm{3.893}{-2}{3.893}{-2.075}{1}{0}\errelarm{1.82}{-2.738}{1.82}{-2.95}{1}{0}\errelname{2.707}{-2.188}{r}{D2}\errelangle{3.893}{-2.075}{3.893}{-2.15}{2.857}{-2.338}{1}{0}\errelangle{2.857}{-2.338}{1.82}{-2.525}{1.82}{-2.738}{1}{0}\eridcomprel{1.7200000000000004}{1.9200000000000006}{-2.7}\ercrowfoot{1.82}{-2.8}{1.67}{-2.95}{1.82}{-2.95}{1.97}{-2.95}{0}
% relationship 
\errelname{5.377}{-2.3}{l}{}\errelarm{5.227}{-2}{5.227}{-2.075}{1}{0}\errelarm{7.874}{-2.738}{7.874}{-2.95}{1}{0}\errelname{6.7}{-2.188}{l}{D3}\errelangle{5.227}{-2.075}{5.227}{-2.15}{6.55}{-2.338}{1}{0}\errelangle{6.55}{-2.338}{7.874}{-2.525}{7.874}{-2.738}{1}{0}\eridcomprel{7.7737500000000015}{7.973750000000001}{-2.7}\ercrowfoot{7.874}{-2.8}{7.724}{-2.95}{7.874}{-2.95}{8.024}{-2.95}{0}
% relationship 
\errelname{1.97}{-5.05}{l}{}\errelname{1.97}{-5.2}{l}{D4}\errelarm{1.82}{-4.75}{1.82}{-5.35}{1}{0}\errelarm{1.82}{-5.35}{1.82}{-5.95}{1}{0}\eridcomprel{1.7200000000000004}{1.9200000000000006}{-5.699999999999999}\ercrowfoot{1.82}{-5.8}{1.67}{-5.95}{1.82}{-5.95}{1.97}{-5.95}{0}
% relationship battingSide
\errelname{3.784}{-3.4}{l}{battingSide}\errelname{5.284}{-3.4}{l}{R1}\errelarm{3.634}{-3.55}{5.134}{-3.55}{1}{0}\errelarm{5.134}{-3.55}{6.634}{-3.55}{0}{0}\ercrowfoot{3.784}{-3.55}{3.634}{-3.4}{3.634}{-3.55}{3.634}{-3.7}{0}
% relationship fieldingSide
\errelname{3.784}{-4}{l}{fieldingSide}\errelname{5.284}{-4}{l}{R2}\errelarm{3.634}{-4.15}{5.134}{-4.15}{1}{0}\errelarm{5.134}{-4.15}{6.634}{-4.15}{0}{0}\ercrowfoot{3.784}{-4.15}{3.634}{-4}{3.634}{-4.15}{3.634}{-4.3}{0}
% relationship 
\errelname{8.024}{-5.05}{l}{}\errelname{8.024}{-5.2}{l}{D5}\errelarm{7.874}{-4.75}{7.874}{-5.35}{1}{0}\errelarm{7.874}{-5.35}{7.874}{-5.95}{1}{0}\eridcomprel{7.7737500000000015}{7.973750000000001}{-5.699999999999999}\ercrowfoot{7.874}{-5.8}{7.724}{-5.95}{7.874}{-5.95}{8.024}{-5.95}{0}
% relationship 
\errelname{1.97}{-8.05}{l}{}\errelname{1.97}{-8.2}{l}{D6}\errelarm{1.82}{-7.75}{1.82}{-8.35}{1}{0}\errelarm{1.82}{-8.35}{1.82}{-8.95}{1}{0}\eridcomprel{1.7200000000000004}{1.9200000000000006}{-8.7}\ercrowfoot{1.82}{-8.8}{1.67}{-8.95}{1.82}{-8.95}{1.97}{-8.95}{0}
% relationship bowler
\errelname{3.493}{-6.7}{l}{bowler}\errelname{5.156}{-6.7}{l}{R3}\errelarm{3.343}{-6.85}{5.006}{-6.85}{1}{0}\errelarm{5.006}{-6.85}{6.67}{-6.85}{0}{0}\ercrowfoot{3.493}{-6.85}{3.343}{-6.7}{3.343}{-6.85}{3.343}{-7}{0}
% relationship facingBatter
\errelname{3.811}{-10.3}{l}{facingBatter}\errelarm{3.661}{-10}{4.261}{-10}{1}{0}\errelarm{6.47}{-7.57}{6.67}{-7.57}{0}{0}\errelname{5.416}{-8.635}{r}{R4}\errelangle{4.261}{-10}{4.861}{-10}{5.566}{-8.785}{1}{0}\errelangle{5.566}{-8.785}{6.27}{-7.57}{6.47}{-7.57}{0}{0}\ercrowfoot{3.811}{-10}{3.661}{-9.85}{3.661}{-10}{3.661}{-10.15}{0}
\end{erdiagram}
} }
& \kern-3cm\onslide*<1-1>{\logicalAnnotation} \onslide*<2-2>{\hierarchicalAnnotation}\onslide*<3-3>{\relationalAnnotation}
\end{tabular}
\end{frame}

