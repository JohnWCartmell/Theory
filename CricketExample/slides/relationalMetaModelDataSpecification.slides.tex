


\begin{frame}{Relational Meta Model - Logical Model}
\erDisplayFiveSlideAnimation
\begin{center}
\scalebox{0.75}{\begin{erdiagram}{11.899999999999999}{15.234625000000001}

\eret{4.4}{-2.15}{9.9}{-1.25}{0.2}{1}\eretname{4.95}{-1.6}{l}{table}
\erattr{4.6}{-1.8}{1}{0}{name}{}
\eret{0.957}{-5.25}{3.691}{-3.75}{0.2}{1}\eretname{1.23}{-4.1}{l}{primary key entry}
\erRelationalAttribute{1.157}{-4.3}{1}{0}{table\textunderscore name}{(D2)}
\erHierarchicalAttribute{1.157}{-4.6}{1}{0}{column\textunderscore name}{(R1)}
\erattr{1.157}{-4.9}{1}{1}{seqNo}{}
\eret{5.991}{-4.95}{8.426}{-3.75}{0.2}{1}\eretname{6.234}{-4.1}{l}{column}
\erRelationalAttribute{6.191}{-4.3}{1}{0}{table\textunderscore name}{(D3)}
\erattr{6.191}{-4.6}{1}{0}{name}{}
\eret{10.926}{-5.25}{13.743}{-3.75}{0.2}{1}\eretname{11.208}{-4.1}{l}{foreign key}
\erRelationalAttribute{11.126}{-4.3}{1}{0}{table\textunderscore name}{(D4)}
\erattr{11.126}{-4.6}{1}{0}{name}{}
\erHierarchicalAttribute{11.126}{-4.9}{1}{1}{to\textunderscore table\textunderscore name}{(R3)}
\eret{10.635}{-8.6}{14.035}{-6.8}{0.2}{1}\eretname{11.125}{-7.15}{l}{foreign key entry}
\erRelationalAttribute{10.835}{-7.35}{1}{0}{table\textunderscore name}{(D5)}
\erRelationalAttribute{10.835}{-7.65}{1}{0}{foreign key\textunderscore name}{(D5)}
\erHierarchicalAttribute{10.835}{-7.95}{1}{0}{to\textunderscore column\textunderscore name}{(R5)}
\erHierarchicalAttribute{10.835}{-8.25}{1}{1}{from\textunderscore column\textunderscore name}{(R4)}
\eret{0}{-0.2}{15.235}{0.3}{0.2}{1}

% relationship all tables
\errelname{7.3}{-0.5}{l}{all tables}\errelarm{7.15}{-0.2}{7.15}{-0.725}{1}{0}\errelarm{7.15}{-0.725}{7.15}{-1.25}{1}{0}\ercrowfoot{7.15}{-1.1}{7}{-1.25}{7.15}{-1.25}{7.3}{-1.25}{0}
% relationship 
\errelname{5.65}{-2.45}{l}{}\errelname{2.474}{-3.6}{l}{of}\errelarm{5.5}{-2.15}{5.5}{-2.225}{1}{0}\errelarm{2.324}{-3.463}{2.324}{-3.75}{1}{0}\errelid{3.762}{-2.588}{r}{D2}\errelangle{5.5}{-2.225}{5.5}{-2.3}{3.912}{-2.738}{1}{0}\errelangle{3.912}{-2.738}{2.324}{-3.175}{2.324}{-3.463}{1}{0}\errelseq{2.384}{-3.225}{1.974}{-3.285}{2.674}{-3.345}{2.264}{-3.405}\eridcomprel{2.22375}{2.42375}{-3.5}\ercrowfoot{2.324}{-3.6}{2.174}{-3.75}{2.324}{-3.75}{2.474}{-3.75}{0}
% relationship 
\errelname{7.358}{-2.45}{l}{}\errelname{7.358}{-3.6}{l}{of}\errelid{7.358}{-2.8}{l}{D3}\errelarm{7.208}{-2.15}{7.208}{-2.95}{1}{0}\errelarm{7.208}{-2.95}{7.208}{-3.75}{1}{0}\eridcomprel{7.1083750000000006}{7.308375}{-3.5}\ercrowfoot{7.208}{-3.6}{7.058}{-3.75}{7.208}{-3.75}{7.358}{-3.75}{0}
% relationship 
\errelname{8.95}{-2.45}{l}{}\errelname{12.485}{-3.6}{l}{of}\errelarm{8.8}{-2.15}{8.8}{-2.225}{1}{0}\errelarm{12.335}{-3.538}{12.335}{-3.75}{1}{0}\errelid{10.717}{-2.663}{l}{D4}\errelangle{8.8}{-2.225}{8.8}{-2.3}{10.567}{-2.813}{1}{0}\errelangle{10.567}{-2.813}{12.335}{-3.325}{12.335}{-3.538}{1}{0}\eridcomprel{12.234625000000001}{12.434625}{-3.5}\ercrowfoot{12.335}{-3.6}{12.185}{-3.75}{12.335}{-3.75}{12.485}{-3.75}{0}
% relationship column
\errelname{3.841}{-4.3}{l}{column}\errelid{4.791}{-4.35}{l}{R1}\erscope{4.491}{-4.8}{l}{d:of=s:of}\errelarm{3.691}{-4.5}{4.841}{-4.5}{1}{0}\errelarm{4.841}{-4.5}{5.991}{-4.5}{0}{0}\eridrefrel{3.940875}{-4.4}{-4.6}
% relationship 
\errelname{12.485}{-5.55}{l}{}\errelname{12.585}{-6.65}{l}{partof}\errelid{12.485}{-5.875}{l}{D5}\errelarm{12.335}{-5.25}{12.335}{-6.025}{1}{0}\errelarm{12.335}{-6.025}{12.335}{-6.8}{1}{0}\eridcomprel{12.234625000000001}{12.434625}{-6.55}\ercrowfoot{12.335}{-6.65}{12.185}{-6.8}{12.335}{-6.8}{12.485}{-6.8}{0}
% relationship to_table
\errelname{13.893}{-4.35}{l}{table}\errelname{13.893}{-4.05}{l}{to}\errelarm{13.743}{-4.5}{14.293}{-4.5}{1}{0}\errelarm{2.3}{-1.849}{4.4}{-1.849}{0}{0}\errelangle{14.293}{-4.5}{14.843}{-4.5}{14.843}{-7.1}{1}{0}\errelangle{2.3}{-1.849}{0.2}{-1.849}{0.2}{-5.775}{0}{0}\errelid{7.172}{-9.55}{l}{R3}\erscope{7.172}{-10}{l}{d:\textasciicircum =s:\textasciicircum }\errelangle{14.843}{-7.1}{14.843}{-9.7}{7.522}{-9.7}{1}{0}\errelangle{7.522}{-9.7}{0.2}{-9.7}{0.2}{-5.775}{0}{0}\ercrowfoot{13.893}{-4.5}{13.743}{-4.35}{13.743}{-4.5}{13.743}{-4.65}{0}
% relationship from_column
\errelname{10.485}{-7.7}{r}{from}\errelname{10.485}{-8}{r}{column}\errelarm{10.635}{-7.4}{10.485}{-7.4}{1}{0}\errelarm{8.576}{-4.549}{8.426}{-4.549}{0}{0}\errelid{8.98}{-5.925}{l}{R4}\erscope{7.98}{-6.275}{l}{d:of=s:partof/of}\errelangle{10.485}{-7.4}{10.335}{-7.4}{9.53}{-5.975}{1}{0}\errelangle{9.53}{-5.975}{8.726}{-4.549}{8.576}{-4.549}{0}{0}\ercrowfoot{10.485}{-7.4}{10.635}{-7.25}{10.635}{-7.4}{10.635}{-7.55}{0}
% relationship to
\errelname{14.185}{-7.8}{l}{to}\errelarm{14.035}{-8}{14.285}{-8}{1}{0}\errelarm{0.707}{-4.749}{0.957}{-4.749}{0}{0}\errelangle{14.285}{-8}{14.535}{-8}{14.535}{-8.5}{1}{0}\errelangle{0.707}{-4.749}{0.457}{-4.749}{0.457}{-6.875}{0}{0}\errelid{7.146}{-8.85}{l}{R5}\erscope{6.646}{-9.3}{l}{d:of=s:partof/to\textunderscore table}\errelangle{14.535}{-8.5}{14.535}{-9}{7.496}{-9}{1}{0}\errelangle{7.496}{-9}{0.457}{-9}{0.457}{-6.875}{0}{0}\ercrowfoot{14.185}{-8}{14.035}{-7.85}{14.035}{-8}{14.035}{-8.15}{0}\eridrefrel{14.284625000000002}{-7.9}{-8.1}
\end{erdiagram}
}
\end{center}
\end{frame}


\begin{frame}{primaryKeyEntry.column Scope}
\begin{equation}
\label{primaryKeyEntry.columnScopeText}
\parbox{9.5cm}{The $column$ within a $primary\ key\ entry$ of a $table$ is a $column$ of that same $table$.}
\end{equation}
\medskip
\begin{equation}
\label{primaryKeyEntry.columnScopeDiagram}
\scalebox{0.75}{\only<1>{\begin{erdiagram}{2.8}{6.083774999999999}

\erettop{2.4}{-0.7}{4.3}{-0.1}\eretname{3.35}{-0.45}{}{table}
\eretbl{0.616}{-2.8}{3.25}{-2.2}\eretname{1.933}{-2.55}{}{primary key entry}
\eretbr{4.75}{-2.8}{6.084}{-2.2}\eretname{5.417}{-2.55}{}{column}

% relationship 
\errelname{3.183}{-1}{l}{}\errelname{1.783}{-2.05}{r}{of}\errelarm{3.033}{-0.7}{3.033}{-0.775}{1}{0}\errelarm{1.933}{-1.987}{1.933}{-2.2}{1}{0}\errelangle{3.033}{-0.775}{3.033}{-0.85}{2.483}{-1.312}{1}{0}\errelangle{2.483}{-1.312}{1.933}{-1.775}{1.933}{-1.987}{1}{0}\eridcomprel{1.83335}{2.03335}{-1.9499999999999997}
% relationship 
\errelname{3.817}{-1}{l}{}\errelname{5.567}{-2.05}{l}{of}\errelarm{3.667}{-0.7}{3.667}{-0.775}{1}{0}\errelarm{5.417}{-1.987}{5.417}{-2.2}{1}{0}\errelangle{3.667}{-0.775}{3.667}{-0.85}{4.542}{-1.312}{1}{0}\errelangle{4.542}{-1.312}{5.417}{-1.775}{5.417}{-1.987}{1}{0}\eridcomprel{5.317125}{5.517124999999999}{-1.9499999999999997}\ercrowfoot{5.417}{-2.05}{5.267}{-2.2}{5.417}{-2.2}{5.567}{-2.2}{0}
% relationship column
\errelname{3.4}{-2.35}{l}{column}\errelarm{3.25}{-2.5}{3.5}{-2.5}{1}{0}\errelarm{4.5}{-2.5}{4.75}{-2.5}{0}{0}\errelarm{3.5}{-2.5}{4}{-2.5}{1}{0}\errelarm{4}{-2.5}{4.5}{-2.5}{0}{0}\ercrowfoot{3.4}{-2.5}{3.25}{-2.35}{3.25}{-2.5}{3.25}{-2.65}{0}\eridrefrel{3.5004750000000002}{-2.3999999999999995}{-2.5999999999999996}
\end{erdiagram}
}} \only<2>{\scopeTriangle{column}{primary\ key\ entry}{column}{table}{of}{of}}
\end{equation}
\medskip
\begin{itemize}
\item Statement (\ref{primaryKeyEntry.columnScopeText}) specifies the \textit{scope} of the $column$ relationship.
\item It means that starting from a $primary\ key\ entry$ there are two navigation paths that are equivalent:
\begin{itemize}
  \item $of$ 
  \item $column/of$
\end{itemize}
\item This fact can be represented by saying that the diagram (\ref{primaryKeyEntry.columnScopeDiagram}) commutes.
\only<2>{Alternatively we can represent as a scope annotation \textit{s:of=d:of} on the ER diagram or as a diagram of morphisms.}
\end{itemize}
\end{frame}

\begin{frame}{foreignKeyEntry.from\_column Scope}
\begin{equation}
\label{foreignKeyEntry.toScopeText}
\parbox{9.5cm}{The $from\_column$ within a $foreign\ key\ entry$ of a $foreign\ key$ of a $table$ is a $column$ of that same $table$.}
\end{equation}
\begin{equation}
\label{foreignKeyEntry.toScopeDiagram}
\raisebox{-1cm}{\scalebox{0.65}{\begin{erdiagram}{3.7}{5.8734}

\erettop{2.4}{-0.7}{4.3}{-0.1}\eretname{3.35}{-0.45}{}{table}
\eretml{0.827}{-2.2}{2.619}{-1.6}\eretname{1.723}{-1.95}{}{foreign key}
\eretbl{0.406}{-3.7}{3.04}{-3.1}\eretname{1.723}{-3.45}{}{foreign key entry}
\eretbr{4.54}{-3.7}{5.873}{-3.1}\eretname{5.207}{-3.45}{}{column}

% relationship 
\errelname{3.183}{-1}{l}{}\errelname{1.573}{-1.45}{r}{of}\errelarm{3.033}{-0.7}{3.033}{-0.775}{0}{0}\errelarm{1.723}{-1.388}{1.723}{-1.6}{1}{0}\errelangle{3.033}{-0.775}{3.033}{-0.85}{2.378}{-1.013}{0}{0}\errelangle{2.378}{-1.013}{1.723}{-1.175}{1.723}{-1.388}{1}{0}\eridcomprel{1.6229749999999996}{1.8229749999999998}{-1.3499999999999999}\ercrowfoot{1.723}{-1.45}{1.573}{-1.6}{1.723}{-1.6}{1.873}{-1.6}{0}
% relationship 
\errelname{3.817}{-1}{l}{}\errelname{5.357}{-2.95}{l}{of}\errelarm{3.667}{-0.7}{3.667}{-0.775}{1}{0}\errelarm{5.207}{-2.35}{5.207}{-3.1}{1}{0}\errelangle{3.667}{-0.775}{3.667}{-0.85}{4.437}{-1.225}{1}{0}\errelangle{4.437}{-1.225}{5.207}{-1.6}{5.207}{-2.35}{1}{0}\eridcomprel{5.10675}{5.306749999999999}{-2.8499999999999996}\ercrowfoot{5.207}{-2.95}{5.057}{-3.1}{5.207}{-3.1}{5.357}{-3.1}{0}
% relationship 
\errelname{1.873}{-2.5}{l}{}\errelname{1.573}{-2.95}{r}{of}\errelname{1.573}{-2.65}{r}{part}\errelarm{1.723}{-2.2}{1.723}{-2.65}{1}{0}\errelarm{1.723}{-2.65}{1.723}{-3.1}{1}{0}\eridcomprel{1.6229749999999996}{1.8229749999999998}{-2.8499999999999996}\ercrowfoot{1.723}{-2.95}{1.573}{-3.1}{1.723}{-3.1}{1.873}{-3.1}{0}
% relationship from_column
\errelname{3.19}{-3.25}{l}{column}\errelname{3.19}{-2.95}{l}{from}\errelarm{3.04}{-3.4}{3.29}{-3.4}{1}{0}\errelarm{4.29}{-3.4}{4.54}{-3.4}{0}{0}\errelarm{3.29}{-3.4}{3.79}{-3.4}{1}{0}\errelarm{3.79}{-3.4}{4.29}{-3.4}{0}{0}\ercrowfoot{3.19}{-3.4}{3.04}{-3.25}{3.04}{-3.4}{3.04}{-3.55}{0}
\end{erdiagram}
}}
\end{equation}
\begin{itemize}
\item Statement (\ref{foreignKeyEntry.toScopeText}) specifies the \textit{scope} of the $from\_column$ relationship.
\item So, from $foreign\ key\ entry$ two navigation paths that are equivalent:
\begin{itemize}
  \item $part\_of/of$ 
  \item $from\_column/of$
\end{itemize}
\item In other words, diagram (\ref{foreignKeyEntry.toScopeDiagram}),above, commutes
\item and the scope of the \textit{from column} relationship is annotated as $s:part\_of/of=d:of$
\end{itemize}
\end{frame}


\begin{frame}{foreignKeyEntry.to Scope}
\begin{equation}
\label{foreignKeyEntry.toScopeText}
\parbox{9.5cm}{The $to$ $primary\ key\ entry$ of a $foreign\ key\ entry$ of a $foreign\ key$ is a $primary\ key\ entry$ of the $to\_table$ of that $foreign\ key$.}
\end{equation}
\begin{equation}
\label{foreignKeyEntry.toScopeDiagram}
\raisebox{-1cm}{\scalebox{0.75}{\begin{erdiagram}{2.1999999999999997}{7.817125000000001}

\erettl{1.4}{-0.7}{3.2}{-0.1}\eretname{2.3}{-0.45}{}{foreign key}
\eretbl{0.983}{-2.2}{3.617}{-1.6}\eretname{2.3}{-1.95}{}{foreign key entry}
\erettr{5.6}{-0.7}{7.4}{-0.1}\eretname{6.5}{-0.45}{}{table}
\eretbr{5.183}{-2.2}{7.817}{-1.6}\eretname{6.5}{-1.95}{}{primary key entry}

% relationship 
\errelname{2.45}{-1}{l}{}\errelname{2.15}{-1.45}{r}{of}\errelname{2.15}{-1.15}{r}{part}\errelarm{2.3}{-0.7}{2.3}{-1.15}{1}{0}\errelarm{2.3}{-1.15}{2.3}{-1.6}{1}{0}\eridcomprel{2.1999999999999997}{2.4}{-1.3499999999999999}\ercrowfoot{2.3}{-1.45}{2.15}{-1.6}{2.3}{-1.6}{2.45}{-1.6}{0}
% relationship to_table
\errelname{3.35}{-0.25}{l}{to\textunderscore table}\errelarm{3.2}{-0.4}{4.4}{-0.4}{1}{0}\errelarm{4.4}{-0.4}{5.6}{-0.4}{0}{0}\ercrowfoot{3.35}{-0.4}{3.2}{-0.25}{3.2}{-0.4}{3.2}{-0.55}{0}
% relationship to
\errelname{3.767}{-1.75}{l}{to}\errelarm{3.617}{-1.9}{3.867}{-1.9}{1}{0}\errelarm{4.933}{-1.9}{5.183}{-1.9}{0}{0}\errelarm{3.867}{-1.9}{4.4}{-1.9}{1}{0}\errelarm{4.4}{-1.9}{4.933}{-1.9}{0}{0}\ercrowfoot{3.767}{-1.9}{3.617}{-1.75}{3.617}{-1.9}{3.617}{-2.05}{0}\eridrefrel{3.8671249999999997}{-1.7999999999999998}{-2}
% relationship 
\errelname{6.65}{-1}{l}{}\errelname{6.65}{-1.45}{l}{of}\errelarm{6.5}{-0.7}{6.5}{-1.15}{1}{0}\errelarm{6.5}{-1.15}{6.5}{-1.6}{1}{0}\eridcomprel{6.4}{6.6}{-1.3499999999999999}\ercrowfoot{6.5}{-1.45}{6.35}{-1.6}{6.5}{-1.6}{6.65}{-1.6}{0}
\end{erdiagram}
}}
\end{equation}
\begin{itemize}
\item The scope of the \textit{to} relationship is described by the commutivity of (\ref{foreignKeyEntry.toScopeDiagram}),above,
 and the scope of the \textit{to} relationship may be annotated as $s:part\_of/to\_table=d:of$.
\end{itemize}
\end{frame}

\begin{frame}{foreignKeyEntry.to Scope Being a Pullback}
It is also the case that
\begin{equation}
\label{foreignKeyEntry.to.pullbackText}
\parbox{9.5cm}{
For every \textit{primary key entry} of the \textit{to\_table} of a \textit{foreign key} there is a 
$foreign\ key\ entry$ that is \textit{part\_of} that \textit{foreign key} and which is $to$ the $primary\ key\ entry$.
}
\end{equation}
\begin{equation}
\label{foreignKeyEntry.toScopeDiagram}
\raisebox{-1cm}{\scalebox{0.65}{\begin{erdiagram}{2.1999999999999997}{7.817125000000001}

\erettl{1.4}{-0.7}{3.2}{-0.1}\eretname{2.3}{-0.45}{}{foreign key}
\eretbl{0.983}{-2.2}{3.617}{-1.6}\eretname{2.3}{-1.95}{}{foreign key entry}
\erettr{5.6}{-0.7}{7.4}{-0.1}\eretname{6.5}{-0.45}{}{table}
\eretbr{5.183}{-2.2}{7.817}{-1.6}\eretname{6.5}{-1.95}{}{primary key entry}

% relationship 
\errelname{2.45}{-1}{l}{}\errelname{2.15}{-1.45}{r}{of}\errelname{2.15}{-1.15}{r}{part}\errelarm{2.3}{-0.7}{2.3}{-1.15}{1}{0}\errelarm{2.3}{-1.15}{2.3}{-1.6}{1}{0}\eridcomprel{2.1999999999999997}{2.4}{-1.3499999999999999}\ercrowfoot{2.3}{-1.45}{2.15}{-1.6}{2.3}{-1.6}{2.45}{-1.6}{0}
% relationship to_table
\errelname{3.35}{-0.25}{l}{to\textunderscore table}\errelarm{3.2}{-0.4}{4.4}{-0.4}{1}{0}\errelarm{4.4}{-0.4}{5.6}{-0.4}{0}{0}\ercrowfoot{3.35}{-0.4}{3.2}{-0.25}{3.2}{-0.4}{3.2}{-0.55}{0}
% relationship to
\errelname{3.767}{-1.75}{l}{to}\errelarm{3.617}{-1.9}{3.867}{-1.9}{1}{0}\errelarm{4.933}{-1.9}{5.183}{-1.9}{0}{0}\errelarm{3.867}{-1.9}{4.4}{-1.9}{1}{0}\errelarm{4.4}{-1.9}{4.933}{-1.9}{0}{0}\ercrowfoot{3.767}{-1.9}{3.617}{-1.75}{3.617}{-1.9}{3.617}{-2.05}{0}\eridrefrel{3.8671249999999997}{-1.7999999999999998}{-2}
% relationship 
\errelname{6.65}{-1}{l}{}\errelname{6.65}{-1.45}{l}{of}\errelarm{6.5}{-0.7}{6.5}{-1.15}{1}{0}\errelarm{6.5}{-1.15}{6.5}{-1.6}{1}{0}\eridcomprel{6.4}{6.6}{-1.3499999999999999}\ercrowfoot{6.5}{-1.45}{6.35}{-1.6}{6.5}{-1.6}{6.65}{-1.6}{0}
\end{erdiagram}
}}
\end{equation}
\begin{itemize}
\item Statement (\ref{foreignKeyEntry.to.pullbackText}) expresses the fact that 
diagram  (\ref{foreignKeyEntry.toScopeDiagram}) is  a  pullback diagram.
\end{itemize}
\end{frame}







