


\begin{frame}{Relational Meta Model - Logical Model}
\erDisplayFiveSlideAnimation
\begin{center}
\scalebox{0.75}{\begin{erdiagram}{11.899999999999999}{16.04475}

\eret{4.85}{-2.15}{10.35}{-1.25}{0.2}{1}\ertext{5.4}{-1.6}{l}{table}
\erattr{5.05}{-1.8}{1}{0}{name}
\eret{0.138}{-5.55}{2.912}{-3.75}{0.2}{1}\ertext{0.415}{-4.1}{l}{primary key column}
\erdattr{0.338}{-4.3}{1}{0}{table name(D2)}
\erdattr{0.338}{-4.6}{1}{0}{is name(R1)}
\erattr{0.338}{-4.9}{1}{1}{seq no}
\erattr{0.338}{-5.2}{1}{1}{seqNo}
\eret{5.662}{-5.25}{9.427}{-3.75}{0.2}{1}\ertext{6.039}{-4.1}{l}{column}
\erdattr{5.862}{-4.3}{1}{0}{table name(D3)}
\erattr{5.862}{-4.6}{1}{0}{name}
\erdattr{5.862}{-4.9}{0}{1}{in primary key is name(R2)}
\eret{12.427}{-5.25}{14.662}{-3.75}{0.2}{1}\ertext{12.651}{-4.1}{l}{foreign key}
\erdattr{12.627}{-4.3}{1}{0}{table name(D4)}
\erattr{12.627}{-4.6}{1}{0}{name}
\erdattr{12.627}{-4.9}{1}{1}{to name(R3)}
\eret{12.045}{-8.6}{15.045}{-6.8}{0.2}{1}\ertext{12.495}{-7.15}{l}{foreign key column}
\erdattr{12.245}{-7.35}{1}{0}{table name(D5)}
\erdattr{12.245}{-7.65}{1}{0}{foreign key name(D5)}
\erdattr{12.245}{-7.95}{1}{0}{to is name(R5)}
\erdattr{12.245}{-8.25}{1}{1}{is name(R4)}
\eret{0}{-0.2}{16.045}{0.3}{0.2}{1}

% relationship all tables
\ertext{7.75}{-0.5}{l}{all tables}\errelarm{7.6}{-0.2}{7.6}{-0.725}{1}{0}\errelarm{7.6}{-0.725}{7.6}{-1.25}{1}{0}\ercrowfoot{7.6}{-1.1}{7.45}{-1.25}{7.6}{-1.25}{7.75}{-1.25}{0}
% relationship 
\ertext{6.1}{-2.45}{l}{}\ertext{1.675}{-3.6}{l}{of}\errelarm{5.95}{-2.15}{5.95}{-2.225}{1}{0}\errelarm{1.525}{-3.463}{1.525}{-3.75}{1}{0}\ertext{3.587}{-2.588}{r}{D2}\errelangle{5.95}{-2.225}{5.95}{-2.3}{3.737}{-2.738}{1}{0}\errelangle{3.737}{-2.738}{1.525}{-3.175}{1.525}{-3.463}{1}{0}\errelseq{1.585}{-3.225}{1.175}{-3.285}{1.875}{-3.345}{1.465}{-3.405}\eridcomprel{1.4249999999999996}{1.6249999999999998}{-3.5}\ercrowfoot{1.525}{-3.6}{1.375}{-3.75}{1.525}{-3.75}{1.675}{-3.75}{0}
% relationship 
\ertext{7.695}{-2.45}{l}{}\ertext{7.695}{-3.6}{l}{of}\ertext{7.695}{-2.8}{l}{D3}\errelarm{7.545}{-2.15}{7.545}{-2.95}{1}{0}\errelarm{7.545}{-2.95}{7.545}{-3.75}{1}{0}\eridcomprel{7.444750000000001}{7.64475}{-3.5}\ercrowfoot{7.545}{-3.6}{7.395}{-3.75}{7.545}{-3.75}{7.695}{-3.75}{0}
% relationship 
\ertext{9.4}{-2.45}{l}{}\ertext{13.695}{-3.6}{l}{of}\errelarm{9.25}{-2.15}{9.25}{-2.225}{1}{0}\errelarm{13.545}{-3.538}{13.545}{-3.75}{1}{0}\ertext{11.547}{-2.663}{l}{D4}\errelangle{9.25}{-2.225}{9.25}{-2.3}{11.397}{-2.813}{1}{0}\errelangle{11.397}{-2.813}{13.545}{-3.325}{13.545}{-3.538}{1}{0}\eridcomprel{13.44475}{13.64475}{-3.5}\ercrowfoot{13.545}{-3.6}{13.395}{-3.75}{13.545}{-3.75}{13.695}{-3.75}{0}
% relationship is
\ertext{3.062}{-4.45}{l}{is}\ertext{4.237}{-4.5}{l}{R1}\ertext{3.937}{-4.95}{l}{\textasciitilde /of=of}\errelarm{2.912}{-4.65}{4.287}{-4.65}{1}{0}\errelarm{4.287}{-4.65}{5.662}{-4.65}{0}{0}\eridrefrel{3.16225}{-4.550000000000001}{-4.75}
% relationship 
\ertext{13.695}{-5.55}{l}{}\ertext{13.795}{-6.65}{l}{partof}\ertext{13.695}{-5.875}{l}{D5}\errelarm{13.545}{-5.25}{13.545}{-6.025}{1}{0}\errelarm{13.545}{-6.025}{13.545}{-6.8}{1}{0}\eridcomprel{13.44475}{13.64475}{-6.55}\ercrowfoot{13.545}{-6.65}{13.395}{-6.8}{13.545}{-6.8}{13.695}{-6.8}{0}
% relationship to
\ertext{14.812}{-4.35}{l}{to}\errelarm{14.662}{-4.5}{15.262}{-4.5}{1}{0}\errelarm{2.475}{-1.849}{4.85}{-1.849}{0}{0}\errelangle{15.262}{-4.5}{15.862}{-4.5}{15.862}{-7.2}{1}{0}\errelangle{2.475}{-1.849}{0.1}{-1.849}{0.1}{-5.875}{0}{0}\ertext{7.631}{-9.75}{l}{R3}\ertext{7.631}{-10.2}{l}{\textasciitilde /\textasciicircum =\textasciicircum }\errelangle{15.862}{-7.2}{15.862}{-9.9}{7.981}{-9.9}{1}{0}\errelangle{7.981}{-9.9}{0.1}{-9.9}{0.1}{-5.875}{0}{0}\ercrowfoot{14.812}{-4.5}{14.662}{-4.35}{14.662}{-4.5}{14.662}{-4.65}{0}
% relationship is
\ertext{11.895}{-7.7}{r}{is}\errelarm{12.045}{-7.4}{11.895}{-7.4}{1}{0}\errelarm{9.577}{-4.749}{9.427}{-4.749}{0}{0}\ertext{10.186}{-6.025}{l}{R4}\ertext{9.186}{-6.375}{l}{\textasciitilde /of=partof/of}\errelangle{11.895}{-7.4}{11.745}{-7.4}{10.736}{-6.075}{1}{0}\errelangle{10.736}{-6.075}{9.727}{-4.749}{9.577}{-4.749}{0}{0}\ercrowfoot{11.895}{-7.4}{12.045}{-7.25}{12.045}{-7.4}{12.045}{-7.55}{0}
% relationship to
\ertext{15.195}{-7.8}{l}{to}\errelarm{15.045}{-8}{15.295}{-8}{1}{0}\errelarm{-0.112}{-4.949}{0.138}{-4.949}{0}{0}\errelangle{15.295}{-8}{15.545}{-8}{15.545}{-8.5}{1}{0}\errelangle{-0.112}{-4.949}{-0.362}{-4.949}{-0.362}{-6.974}{0}{0}\ertext{7.241}{-8.85}{l}{R5}\ertext{6.741}{-9.3}{l}{\textasciitilde /of=partof/to}\errelangle{15.545}{-8.5}{15.545}{-9}{7.591}{-9}{1}{0}\errelangle{7.591}{-9}{-0.362}{-9}{-0.362}{-6.974}{0}{0}\ercrowfoot{15.195}{-8}{15.045}{-7.85}{15.045}{-8}{15.045}{-8.15}{0}\eridrefrel{15.29475}{-7.9}{-8.1}
\end{erdiagram}
}
\end{center}
\end{frame}


\begin{frame}{primaryKeyEntry.column Scope}
\begin{equation}
\label{primaryKeyEntry.columnScopeText}
\parbox{9.5cm}{The $column$ within a $primary\ key\ entry$ of a $table$ is a $column$ of that same $table$.}
\end{equation}
\medskip
\begin{equation}
\label{primaryKeyEntry.columnScopeDiagram}
\scalebox{0.75}{\only<1>{\begin{erdiagram}{2.8}{6.083774999999999}

\erettop{2.4}{-0.7}{4.3}{-0.1}\eretname{3.35}{-0.45}{}{table}
\eretbl{0.616}{-2.8}{3.25}{-2.2}\eretname{1.933}{-2.55}{}{primary key entry}
\eretbr{4.75}{-2.8}{6.084}{-2.2}\eretname{5.417}{-2.55}{}{column}

% relationship 
\errelname{3.183}{-1}{l}{}\errelname{1.783}{-2.05}{r}{of}\errelarm{3.033}{-0.7}{3.033}{-0.775}{1}{0}\errelarm{1.933}{-1.987}{1.933}{-2.2}{1}{0}\errelangle{3.033}{-0.775}{3.033}{-0.85}{2.483}{-1.312}{1}{0}\errelangle{2.483}{-1.312}{1.933}{-1.775}{1.933}{-1.987}{1}{0}\eridcomprel{1.83335}{2.03335}{-1.9499999999999997}
% relationship 
\errelname{3.817}{-1}{l}{}\errelname{5.567}{-2.05}{l}{of}\errelarm{3.667}{-0.7}{3.667}{-0.775}{1}{0}\errelarm{5.417}{-1.987}{5.417}{-2.2}{1}{0}\errelangle{3.667}{-0.775}{3.667}{-0.85}{4.542}{-1.312}{1}{0}\errelangle{4.542}{-1.312}{5.417}{-1.775}{5.417}{-1.987}{1}{0}\eridcomprel{5.317125}{5.517124999999999}{-1.9499999999999997}\ercrowfoot{5.417}{-2.05}{5.267}{-2.2}{5.417}{-2.2}{5.567}{-2.2}{0}
% relationship column
\errelname{3.4}{-2.35}{l}{column}\errelarm{3.25}{-2.5}{3.5}{-2.5}{1}{0}\errelarm{4.5}{-2.5}{4.75}{-2.5}{0}{0}\errelarm{3.5}{-2.5}{4}{-2.5}{1}{0}\errelarm{4}{-2.5}{4.5}{-2.5}{0}{0}\ercrowfoot{3.4}{-2.5}{3.25}{-2.35}{3.25}{-2.5}{3.25}{-2.65}{0}\eridrefrel{3.5004750000000002}{-2.3999999999999995}{-2.5999999999999996}
\end{erdiagram}
}} \only<2>{\scopeTriangle{column}{primary\ key\ entry}{column}{table}{of}{of}}
\end{equation}
\medskip
\begin{itemize}
\item Statement (\ref{primaryKeyEntry.columnScopeText}) specifies the \textit{scope} of the $column$ relationship.
\item It means that starting from a $primary\ key\ entry$ there are two navigation paths that are equivalent:
\begin{itemize}
  \item $of$ 
  \item $column/of$
\end{itemize}
\item This fact can be represented by saying that the diagram (\ref{primaryKeyEntry.columnScopeDiagram}) commutes.
\only<2>{Alternatively we can represent as a scope annotation \textit{s:of=d:of} on the ER diagram or as a diagram of morphisms.}
\end{itemize}
\end{frame}

\begin{frame}{foreignKeyEntry.from\_column Scope}
\begin{equation}
\label{foreignKeyEntry.toScopeText}
\parbox{9.5cm}{The $from\_column$ within a $foreign\ key\ entry$ of a $foreign\ key$ of a $table$ is a $column$ of that same $table$.}
\end{equation}
\begin{equation}
\label{foreignKeyEntry.toScopeDiagram}
\raisebox{-1cm}{\scalebox{0.65}{\begin{erdiagram}{3.7}{5.8734}

\erettop{2.4}{-0.7}{4.3}{-0.1}\eretname{3.35}{-0.45}{}{table}
\eretml{0.827}{-2.2}{2.619}{-1.6}\eretname{1.723}{-1.95}{}{foreign key}
\eretbl{0.406}{-3.7}{3.04}{-3.1}\eretname{1.723}{-3.45}{}{foreign key entry}
\eretbr{4.54}{-3.7}{5.873}{-3.1}\eretname{5.207}{-3.45}{}{column}

% relationship 
\errelname{3.183}{-1}{l}{}\errelname{1.573}{-1.45}{r}{of}\errelarm{3.033}{-0.7}{3.033}{-0.775}{0}{0}\errelarm{1.723}{-1.388}{1.723}{-1.6}{1}{0}\errelangle{3.033}{-0.775}{3.033}{-0.85}{2.378}{-1.013}{0}{0}\errelangle{2.378}{-1.013}{1.723}{-1.175}{1.723}{-1.388}{1}{0}\eridcomprel{1.6229749999999996}{1.8229749999999998}{-1.3499999999999999}\ercrowfoot{1.723}{-1.45}{1.573}{-1.6}{1.723}{-1.6}{1.873}{-1.6}{0}
% relationship 
\errelname{3.817}{-1}{l}{}\errelname{5.357}{-2.95}{l}{of}\errelarm{3.667}{-0.7}{3.667}{-0.775}{1}{0}\errelarm{5.207}{-2.35}{5.207}{-3.1}{1}{0}\errelangle{3.667}{-0.775}{3.667}{-0.85}{4.437}{-1.225}{1}{0}\errelangle{4.437}{-1.225}{5.207}{-1.6}{5.207}{-2.35}{1}{0}\eridcomprel{5.10675}{5.306749999999999}{-2.8499999999999996}\ercrowfoot{5.207}{-2.95}{5.057}{-3.1}{5.207}{-3.1}{5.357}{-3.1}{0}
% relationship 
\errelname{1.873}{-2.5}{l}{}\errelname{1.573}{-2.95}{r}{of}\errelname{1.573}{-2.65}{r}{part}\errelarm{1.723}{-2.2}{1.723}{-2.65}{1}{0}\errelarm{1.723}{-2.65}{1.723}{-3.1}{1}{0}\eridcomprel{1.6229749999999996}{1.8229749999999998}{-2.8499999999999996}\ercrowfoot{1.723}{-2.95}{1.573}{-3.1}{1.723}{-3.1}{1.873}{-3.1}{0}
% relationship from_column
\errelname{3.19}{-3.25}{l}{column}\errelname{3.19}{-2.95}{l}{from}\errelarm{3.04}{-3.4}{3.29}{-3.4}{1}{0}\errelarm{4.29}{-3.4}{4.54}{-3.4}{0}{0}\errelarm{3.29}{-3.4}{3.79}{-3.4}{1}{0}\errelarm{3.79}{-3.4}{4.29}{-3.4}{0}{0}\ercrowfoot{3.19}{-3.4}{3.04}{-3.25}{3.04}{-3.4}{3.04}{-3.55}{0}
\end{erdiagram}
}}
\end{equation}
\begin{itemize}
\item Statement (\ref{foreignKeyEntry.toScopeText}) specifies the \textit{scope} of the $from\_column$ relationship.
\item So, from $foreign\ key\ entry$ two navigation paths that are equivalent:
\begin{itemize}
  \item $part\_of/of$ 
  \item $from\_column/of$
\end{itemize}
\item In other words, diagram (\ref{foreignKeyEntry.toScopeDiagram}),above, commutes
\item and the scope of the \textit{from column} relationship is annotated as $s:part\_of/of=d:of$
\end{itemize}
\end{frame}


\begin{frame}{foreignKeyEntry.to Scope}
\begin{equation}
\label{foreignKeyEntry.toScopeText}
\parbox{9.5cm}{The $to$ $primary\ key\ entry$ of a $foreign\ key\ entry$ of a $foreign\ key$ is a $primary\ key\ entry$ of the $to\_table$ of that $foreign\ key$.}
\end{equation}
\begin{equation}
\label{foreignKeyEntry.toScopeDiagram}
\raisebox{-1cm}{\scalebox{0.75}{\begin{erdiagram}{2.1999999999999997}{7.817125000000001}

\erettl{1.4}{-0.7}{3.2}{-0.1}\eretname{2.3}{-0.45}{}{foreign key}
\eretbl{0.983}{-2.2}{3.617}{-1.6}\eretname{2.3}{-1.95}{}{foreign key entry}
\erettr{5.6}{-0.7}{7.4}{-0.1}\eretname{6.5}{-0.45}{}{table}
\eretbr{5.183}{-2.2}{7.817}{-1.6}\eretname{6.5}{-1.95}{}{primary key entry}

% relationship 
\errelname{2.45}{-1}{l}{}\errelname{2.15}{-1.45}{r}{of}\errelname{2.15}{-1.15}{r}{part}\errelarm{2.3}{-0.7}{2.3}{-1.15}{1}{0}\errelarm{2.3}{-1.15}{2.3}{-1.6}{1}{0}\eridcomprel{2.1999999999999997}{2.4}{-1.3499999999999999}\ercrowfoot{2.3}{-1.45}{2.15}{-1.6}{2.3}{-1.6}{2.45}{-1.6}{0}
% relationship to_table
\errelname{3.35}{-0.25}{l}{to\textunderscore table}\errelarm{3.2}{-0.4}{4.4}{-0.4}{1}{0}\errelarm{4.4}{-0.4}{5.6}{-0.4}{0}{0}\ercrowfoot{3.35}{-0.4}{3.2}{-0.25}{3.2}{-0.4}{3.2}{-0.55}{0}
% relationship to
\errelname{3.767}{-1.75}{l}{to}\errelarm{3.617}{-1.9}{3.867}{-1.9}{1}{0}\errelarm{4.933}{-1.9}{5.183}{-1.9}{0}{0}\errelarm{3.867}{-1.9}{4.4}{-1.9}{1}{0}\errelarm{4.4}{-1.9}{4.933}{-1.9}{0}{0}\ercrowfoot{3.767}{-1.9}{3.617}{-1.75}{3.617}{-1.9}{3.617}{-2.05}{0}\eridrefrel{3.8671249999999997}{-1.7999999999999998}{-2}
% relationship 
\errelname{6.65}{-1}{l}{}\errelname{6.65}{-1.45}{l}{of}\errelarm{6.5}{-0.7}{6.5}{-1.15}{1}{0}\errelarm{6.5}{-1.15}{6.5}{-1.6}{1}{0}\eridcomprel{6.4}{6.6}{-1.3499999999999999}\ercrowfoot{6.5}{-1.45}{6.35}{-1.6}{6.5}{-1.6}{6.65}{-1.6}{0}
\end{erdiagram}
}}
\end{equation}
\begin{itemize}
\item The scope of the \textit{to} relationship is described by the commutivity of (\ref{foreignKeyEntry.toScopeDiagram}),above,
 and the scope of the \textit{to} relationship may be annotated as $s:part\_of/to\_table=d:of$.
\end{itemize}
\end{frame}

\begin{frame}{foreignKeyEntry.to Scope Being a Pullback}
It is also the case that
\begin{equation}
\label{foreignKeyEntry.to.pullbackText}
\parbox{9.5cm}{
For every \textit{primary key entry} of the \textit{to\_table} of a \textit{foreign key} there is a 
$foreign\ key\ entry$ that is \textit{part\_of} that \textit{foreign key} and which is $to$ the $primary\ key\ entry$.
}
\end{equation}
\begin{equation}
\label{foreignKeyEntry.toScopeDiagram}
\raisebox{-1cm}{\scalebox{0.65}{\begin{erdiagram}{2.1999999999999997}{7.817125000000001}

\erettl{1.4}{-0.7}{3.2}{-0.1}\eretname{2.3}{-0.45}{}{foreign key}
\eretbl{0.983}{-2.2}{3.617}{-1.6}\eretname{2.3}{-1.95}{}{foreign key entry}
\erettr{5.6}{-0.7}{7.4}{-0.1}\eretname{6.5}{-0.45}{}{table}
\eretbr{5.183}{-2.2}{7.817}{-1.6}\eretname{6.5}{-1.95}{}{primary key entry}

% relationship 
\errelname{2.45}{-1}{l}{}\errelname{2.15}{-1.45}{r}{of}\errelname{2.15}{-1.15}{r}{part}\errelarm{2.3}{-0.7}{2.3}{-1.15}{1}{0}\errelarm{2.3}{-1.15}{2.3}{-1.6}{1}{0}\eridcomprel{2.1999999999999997}{2.4}{-1.3499999999999999}\ercrowfoot{2.3}{-1.45}{2.15}{-1.6}{2.3}{-1.6}{2.45}{-1.6}{0}
% relationship to_table
\errelname{3.35}{-0.25}{l}{to\textunderscore table}\errelarm{3.2}{-0.4}{4.4}{-0.4}{1}{0}\errelarm{4.4}{-0.4}{5.6}{-0.4}{0}{0}\ercrowfoot{3.35}{-0.4}{3.2}{-0.25}{3.2}{-0.4}{3.2}{-0.55}{0}
% relationship to
\errelname{3.767}{-1.75}{l}{to}\errelarm{3.617}{-1.9}{3.867}{-1.9}{1}{0}\errelarm{4.933}{-1.9}{5.183}{-1.9}{0}{0}\errelarm{3.867}{-1.9}{4.4}{-1.9}{1}{0}\errelarm{4.4}{-1.9}{4.933}{-1.9}{0}{0}\ercrowfoot{3.767}{-1.9}{3.617}{-1.75}{3.617}{-1.9}{3.617}{-2.05}{0}\eridrefrel{3.8671249999999997}{-1.7999999999999998}{-2}
% relationship 
\errelname{6.65}{-1}{l}{}\errelname{6.65}{-1.45}{l}{of}\errelarm{6.5}{-0.7}{6.5}{-1.15}{1}{0}\errelarm{6.5}{-1.15}{6.5}{-1.6}{1}{0}\eridcomprel{6.4}{6.6}{-1.3499999999999999}\ercrowfoot{6.5}{-1.45}{6.35}{-1.6}{6.5}{-1.6}{6.65}{-1.6}{0}
\end{erdiagram}
}}
\end{equation}
\begin{itemize}
\item Statement (\ref{foreignKeyEntry.to.pullbackText}) expresses the fact that 
diagram  (\ref{foreignKeyEntry.toScopeDiagram}) is  a  pullback diagram.
\end{itemize}
\end{frame}







