\subsection {Trees of Concepts and the GAT of Trees}

\renewcommand{\highlight}[1]{#1}  % For print copy don't want highlighting.
 
Terminology: By  the generic term \term{tree} is meant a partially ordered set (poset) $(T, <)$ such that for each $t \in T$, the set $\set{s \in T : s < t}$ is well-ordered by the relation $<$.
In this discussion we restrict ourselves to \highlight{rooted $\omega$-trees} i.e. trees for which the set $\set{s \in T : s < t}$
is finite for all $t \in T$ and for which there is a unique element $t$ such that
 the set $\set{s \in T : s < t}$ is empty. We say that such a $t$ is the root of the tree.

With respect to a partial ordering $<$, we say that \highlight{an element $y$ \textit{covers}  an element $x$} in  iff $x<y$ and there does not exist $w$ such that $x < w$ and $w < y$.
If object $y$ covers object $x$ in the partial ordering 
then we write \highlight{$x \base y$} (we use this in preference to the more usual $x \lessdot y$).
For $x$ an element of the tree we define the set of elements  \highlight{$Cover(x)$} to be the set of objects covering $x$.

Such trees as these we can equivalently describe as models of the generalised algebraic theory given below table \ref{GATOFTREES} in which the nodes of height $n+1$ are represented as of a sort $Cover_{n+1}$that is dependent on the sort of nodes of height $n$.

%\newcommand{\Ft}[1]{#1 \kern -0.4em \downarrow}
\newcommand{\Ft}[1]{\downarrow \kern -0.325em #1}
% irrelevant text removed 27 August 2022

\newcommand{\Sz}{Base}
\newcommand{\ofS}[1]{\ofT{#1}{\Sz}}
\newcommand{\Si}[1]{C\kern-1pt over_{#1}}
\newcommand{\ofSi}[3]{\ofT{#1}{\Si{#2}(#3)}}
\vspace{0.03cm} 
\begin{table}[H]
\caption{The Generalised Algebraic Theory of $\omega$-Trees}
\label{GATOFTREES}

%
\begin{tabular}{>{\itshape}l l}
Symbol & \itshape{Introductory Rule} \\
$\Sz  $&$\isT{\Sz}$\\
$\Si{1} $&$\ofS{x_0} \tstyle \isT{\Si{1}(x_0)} $\\
$\Si{2} $&$\ofS{x_0},\ofSi{x_1}{1}{x_0} \tstyle \isT{\Si{2}(x_0,x_1)} $\\
$\vdots$  \\
$\Si{n} $&$\ofS{x_0},\ofSi{x_1}{1}{x_0}, \hdots \ofSi{x_{n-1}}{n-1}{x_0,x_1,\hdots x_{n-2}} \tstyle \isT{\Si{n}(x_0,x_1,\hdots x_{n-1})} $\\
$\vdots$   \\
\end{tabular} \\


\iffalse
% Following would need improving perhaps using pstricks
\begin{gatrules}
\gatintros
\gatintroducing{Base}
\isT{Base} \\
\gatintroducing{Cov_1\\Cov_2\\\vdots \\ \vdots\\Cov_n\\ \vdots \\ \vdots}
\begin{gatgroup}{\ofT{x_0}{Base}}
  \gatleaf[5cm]{}{\isT{Cov_1(x_0)}} \\
  \begin{gatgroup}{\ofT{x_1}{Cov_1(x_0)}}
    \gatleaf[5cm]{}{\isT{Cov_2(x_0,x_1)}} \\
    %\begin{gatgroup}{\ofT{x_2}{Cov_2(x_0,x_1)}}
    \vdots \\
    \vdots \\
    \begin{gatgroup}{\ofT{x_{n-1}} {Cov_{n-1}(x_0,...x_{n-2})}\iddots}
    \gatleaf[5cm]{}{\isT{Cov_n(x_0,...x_{n-1})}} \\
    \vdots \\
    \vdots
    \end{gatgroup}
    %\end{gatgroup}
  \end{gatgroup}
\end{gatgroup}
\end{gatrules}
\fi

\end{table} 

\subsection {Schematic Notation}
%\newcommand{\Ft}[1]{
%#1 \kern-6pt \raisebox{1.45ex}{$\leftrightline$} \kern-3pt \raisebox{.09ex}{$\downarrow$}\kern-3.4pt \raisebox{.25ex} {$|$}}
\newcommand{\ft}[1]{
#1 \kern-6pt \raisebox{1.1ex}{$\leftrightline$} \kern-3pt \raisebox{.1ex}{$\downarrow$}}
%\newcommand{\Bbar}[1]{
%#1 \kern-6pt \raisebox{1.45ex}{$\leftrightline$}
%\overline{#1}}
%\vv{#1}}
%\newcommand{\bbar}[1]{
%#1 \kern-6pt \raisebox{1.0ex}{$\leftrightline$}
%\overline{#1}}
%\vv{#1}}
\newcommand{\bbin}[1]{
\raisebox{-0.5em}{$\stackrel{\displaystyle{\in}} {\scriptstyle{#1}}$}
}
\newcommand{\ofTn}[3]{
\raisebox{0.25pt}{$#1$} \bbin{#2} #3}  % bar removed 27 August 2022

\newcommand{\genericOb}{Ob} % where we have genericOb=Base + Cover

There is a  shorthand that is convenient in the presentation  of the GAT of trees  and then subsequently in the GAT of contextual categories. We use the shorthand
$\ofTn{x}{n}{\genericOb}$ for the context $\ofS{x_0},\ofSi{x_1}{1}{x_0}, \hdots \ofSi{x_n}{n}{x_0,x_1,\hdots x_{n-1}} $. \\

\noindent Using this shorthand, for any $n \geq 0$ the sort $Cover_{n}$  in the theory of trees is introduced as follows: \\

\vspace{0.03cm} 
\begin{tabular}{>{\itshape}l l}
Symbol & \itshape{Introductory Rule} \\
$\Sz  $     & $\isT{\Sz}$\\
$\Si{n+1}, n \geq 0 $ & $\ofTn{x}{n}{\genericOb}    \tstyle \isT{\Si{n+1}(\bar{x})} $\\
\end{tabular} \\
\vspace{.1cm}  \\

