\subsection {An Aside on Recursive Type Definitions}

Using the shorthand, we are quite close to having a recursive definition of a single sort $Cover$.
Such definitions are not possible in generalised algebraic theories but we can imagine a framework in which it is possible to write: \\
\vspace{0.03cm} 
\begin{tabular}{>{\itshape}l l}
Symbol & \itshape{Introductory Rule} \\
$Base $     & $\isT{Base}$\\
$Cover  $     & $\ofT{x_0}{Base}    \tstyle \isT{Cover(x_0)} $\\
$Cover $      & $\ofT{x}{Cover}    \tstyle \isT{Cover(x)} $\\
\end{tabular} \\
\vspace{.1cm}  \\

Such a definition could be represented algebraically in a suitably generalised notion of contextual category (a comulti-contextual category?) these dependencies could be represented
as follows:  

\begin{center}
$
\begin{array}{c c}
\Rnode{abs}{1}  \\ [1.4cm]
\Rnode{S0}{Base} \\ [1.4cm]
\Rnode{SR}{Cover} \\ [1.4cm]
\end{array}
$
\ncsar{S0}{abs}
\ncsar{SR}{S0}
\ncrsar{SR}{SR}
\end{center}

\noindent This is not just an idle thought -- in  data modelling such a tree 
structure is represented in an entity model diagram in which the injections into the coproduct $Ob$ of $Base$ and $Cover$ are represented by containment: \\

\begin{center}
\begin{erdiagram}{3.4499999999999997}{4.4666}

\eret{0.2}{-2.85}{3.867}{-1.4}{0.2}{1}\ertext{0.316}{-1.75}{l}{$Ob$}
\eret{0.45}{-2.6}{1.783}{-2}{0.2}{0}\ertext{1.117}{-2.35}{}{$Base$}
\eret{2.283}{-2.6}{3.617}{-2}{0.2}{0}\ertext{2.95}{-2.35}{}{$Cover$}
\eret{0}{-0.2}{4.467}{0.3}{0.2}{1}

% relationship 
\ertext{1.217}{-0.5}{l}{}\errelarm{1.117}{-0.2}{1.117}{-1.1}{1}{0}\errelarm{1.117}{-1.1}{1.117}{-2}{1}{0}
% relationship 
\ertext{2.133}{-3.15}{l}{}\errelarm{2.033}{-2.85}{2.033}{-3.1}{0}{0}\errelarm{2.95}{-1.5}{2.95}{-2}{1}{0}\errelangle{2.033}{-3.1}{2.033}{-3.35}{3.133}{-3.35}{0}{0}\errelangle{2.95}{-1.5}{2.95}{-1}{3.592}{-1}{1}{0}\errelangle{3.133}{-3.35}{4.233}{-3.35}{4.233}{-2.175}{0}{0}\errelangle{4.233}{-2.175}{4.233}{-1}{3.592}{-1}{1}{0}\ercrowfoot{2.95}{-1.85}{2.8}{-2}{2.95}{-2}{3.1}{-2}{0}
\end{erdiagram}

\end {center}
See \textit{www.entitymodelling.org/tutorialone} for a description of this notation.
For an example of the modelling of recursive relationships in the definition of a phrase structure grammar of English see 
\textit{www.entitymodelling.org/examplesone/englishsentence}.