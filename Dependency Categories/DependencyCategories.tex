\documentclass[10pt,a4paper]{article}
\usepackage[margin=3cm]{geometry}
\usepackage{pstricks}
\usepackage{pst-node}
\usepackage{pst-tree}
\usepackage{stmaryrd}
\usepackage{amsmath}
\usepackage{amssymb}
\usepackage{verbatim}
\usepackage{enumerate}
\usepackage{calc}

\usepackage{amsthm} % added 7th April 2018
% theorems.macros.tex

\newtheorem{theorem}{Theorem}[section]
\newtheorem{observation}[theorem]{Observation}
\newtheorem{lemma}[theorem]{Lemma}

\newtheorem{proposition}[theorem]{Proposition}
\newtheorem{corollary}[theorem]{Corollary}
\newtheorem{conjecture}[theorem]{Conjecture}
\newtheorem{numbereddefinition}[theorem]{Definition}

\newenvironment{definition}[1][Definition]{\begin{trivlist}
\item[\hskip \labelsep {\bfseries #1}]}{\end{trivlist}}
\newenvironment{examples}[1][Examples]{\begin{trivlist}
\item[\hskip \labelsep {\bfseries #1}]}{\end{trivlist}}
\newenvironment{example}[1][Example]{\begin{trivlist}
\item[\hskip \labelsep {\bfseries #1}]}{\end{trivlist}}
\newenvironment{remark}[1][Remark]{\begin{trivlist}
\item[\hskip \labelsep {\bfseries #1}]}{\end{trivlist}}

\newenvironment{tageqn}[1]
{
\begin{equation}
\stepcounter{equation}
\label{#1}
\tag{\theequation --#1}
}
{
\end{equation}
}

\newenvironment{axiom}[1]
{
\begin{equation}
\label{#1}
\tag{#1}
}
{
\end{equation}
}

% when the tag is required different from the label eg when has math symbols can use:
\newenvironment{axiomtagged}[2]
{
\begin{equation}
\label{#1}
\tag{#2}
}
{
\end{equation}
}

%visible label
\newcommand{\vlabel}[2][]{\label{#2}#1(\textit{#2}):}






%ccategories.macros.tex 

% Macros for diagrams in contextual categories and related categories

\usepackage{twoopt}
\usepackage{scalerel} 
\usepackage{xargs}

%\usepackage{mathabx}  %Caused font problems
%\usepackage{MnSymbol}  % caused font problems

\newcommand{\conu}
{\mathbf{C}(U)}

\newcommand{\depu}
{\mathbf{D}(U)}

\newcommand{\cat}[1]{\textbf{#1}}
\newcommand{\obj}[1]{\ensuremath{|\cat{#1}|}}
\newcommand{\ccat}[1][C]{\ensuremath{\mathbb{#1}} }
\newcommand{\ccatc}{contextual category \ccat}
\newcommand{\cobj}[2][]{\ensuremath{|\ccat[#2]|_{#1}}}
\newcommand{\cslice}[2]{\ensuremath{\ccat[#1]_{#2}}}
\newcommand{\csliceobj}[3][]{\ensuremath{|\mathbb{#2}_{#3}|_{#1} }}
\newcommand{\varset}[1][]{\ensuremath{V_{#1} }}
\newcommand{\localvarsets}{\ensuremath{\mathcal{V} }}
\newcommand{\Fam}{\ensuremath{\mathbb{F\mathrm{am}} }}
\newcommand{\Famslice}[1]{\ensuremath{\mathbb{F\mathrm{am}}_{#1} }}
\newcommand{\Famobj}[1][]{\ensuremath{|\mathbb{F\mathrm{am}}|_{#1} }}
\newcommand{\Famsliceobj}[2][]{\ensuremath{|\mathbb{F\mathrm{am}}_{#2}|_{#1} }}
\newcommand{\morph}{\rightarrow}
\newcommand{\epi}{\twoheadrightarrow}
\newcommand{\base}{\triangleleft}
\newcommand{\comp}{\circ}
\newcommand{\cross}{\otimes}
\newcommand{\pc}[2]{d^{#1}_{#2}}
\newcommand{\sub}{^*}
\newcommand{\diag}{\delta}
\newcommand{\pbase}[1]{\tilde{#1}}

\newcommand{\tuple}[1]{\langle#1\rangle}
\newcommand{\ndidly}{\ensuremath{\Join_n}}
\newcommand{\ndidlycospan}{quotiented n-cospan}

\newcommand{\crossx}[3]{#1 \underset{#3}{\cross} #2}
\newcommand{\fibrex}[3]{#1 \underset{#3}{\Join} #2}
\newcommand{\powerset}{\mathcal{P}}
\newcommand{\primeds}[1]{
\ensuremath{\mathcal{P}(#1)} }
\newcommand{\compset}{\ \dot{\circ}\, }

% darrow
%\newcommand{\darrow}{\rightarrowtriangle} %use \smorph instead
\newcommand{\smorph}{\rightarrowtriangle}

 

\newcommand\dhead{\scaleobj{0.6}{\triangleright}}
\newcommand{\dmorph}{\, \mbox{---} \! \cdot \! \raisebox{1.1pt}{\dhead}}

% projection tree
%\newcommand{\proj}[2]{proj_{#2}(#1)}

\newcommand{\proj}[2]{
\ensuremath{\mathcal{P}_{#2}(#1)} }

%pstrick supplements for arrows

\newlength{\arrnodesepA}
\newlength{\arrnodesepB}
\newlength{\arroffsetA}
\newlength{\arroffsetB}

%Modified to 2pt from 0pt on 23 July 2018
\newcommand{\arreset}{
\setlength{\arrnodesepA}{2pt}
\setlength{\arrnodesepB}{2pt}
\setlength{\arroffsetA}{0pt}
\setlength{\arroffsetB}{0pt}
}
\arreset

\newcommand{\ncarr}[3][0]{\ncarc[arcangle=#1,nodesepA=\arrnodesepA,nodesepB=\arrnodesepB,offsetA=\arroffsetA,offsetB=\arroffsetB,arrowsize=5pt,arrowinset=0.7]{->}{#2}{#3}}
\newcommand{\jcbarr}[4][0]{ % ncbarr is defined in some thridy party package so do not use!\emph{}
\ncarr[#1]{#3}{#4}
\nbput[labelsep=2pt]{\footnotesize $#2$}
}

\newcommand{\ncaarr}[4][0]{
\ncarr[#1]{#3}{#4}
\naput[labelsep=2pt]{\footnotesize $#2$}
}

% \alabel{label}[npos][labelsep_pts]
\newcommandx*\alabel[3][2=0.5,3=2,usedefault]{\naput[labelsep=#3pt,npos=#2]{\footnotesize $#1$}}
% \blabel{label}[npos][labelsep_pts]
\newcommandx*\blabel[3][2=0.5,3=2,usedefault]{\nbput[labelsep=#3pt,npos=#2]{\footnotesize $#1$}}

% \idcomp mark an arrow as one component of an identifier
\newcommand{\idcomp}{\ncput[npos=0, nrot=:U]{\psline(0.2,-0.075)(0.2,0.075)}}  %add a bar to a node connection arrow
% pstrick supplements for s-arrows (previous name for d-arrow - should convert}

\newlength{\sarnodesepA}
\newlength{\sarnodesepB}
\newlength{\saroffsetA}
\newlength{\saroffsetB}
\newlength{\sarnodesepAsav}
\newlength{\sarnodesepBsav}

\newcommand{\sarreset}{
\setlength{\sarnodesepA}{0pt}
\setlength{\sarnodesepB}{0pt}
\setlength{\saroffsetA}{0pt}
\setlength{\saroffsetB}{0pt}
}

\sarreset

% sar - S-arrow
\newcommand{\ncsar}[3][0]{
\setlength{\sarnodesepAsav}{\sarnodesepA}
\setlength{\sarnodesepBsav}{\sarnodesepB}
\addtolength{\sarnodesepA}{3pt}
\addtolength{\sarnodesepB}{7pt}
\ncarc[nodesepA=\sarnodesepA,nodesepB=\sarnodesepB,offsetA=\saroffsetA,offsetB=\saroffsetB,arcangle=#1]{-}{#2}{#3}
\ncput[nrot=:R,npos=1]{\pstriangle(0,0)(.2,.2)}
\setlength{\sarnodesepA}{\sarnodesepAsav}
\setlength{\sarnodesepB}{\sarnodesepBsav}
}


% bsar - below labelled S-arrow
\newcommand{\ncbsar}[4][0]{
\ncsar[#1]{#3}{#4}
\nbput[labelsep=2pt]{\footnotesize $#2$}
}
% asar - above labelled S-arrow
\newcommand{\ncasar}[4][0]{
\ncsar[#1]{#3}{#4}
\naput[labelsep=2pt]{\footnotesize $#2$}
}

% cdar - composite dependency arrow
\newcommand{\nccdar}[3][0]{
\setlength{\sarnodesepAsav}{\sarnodesepA}
\setlength{\sarnodesepBsav}{\sarnodesepB}
\addtolength{\sarnodesepA}{3pt}
\addtolength{\sarnodesepB}{11pt}
\ncarc[nodesepA=\sarnodesepA,nodesepB=\sarnodesepB,offsetA=\saroffsetA,offsetB=\saroffsetB,arcangle=#1]{-}{#2}{#3}
\ncput[nrot=:R,npos=1]{\pstriangle(0,0.1)(.2,.2)}
\ncput[nrot=:R,npos=1]{\psdot[dotsize=1pt](-0.0075,0.05)}   %!!
\setlength{\sarnodesepA}{\sarnodesepAsav}
\setlength{\sarnodesepB}{\sarnodesepBsav}
}


% bcdar - below labelled composite dependency arrow
\newcommand{\ncbcdar}[4][0]{
\nccdar[#1]{#3}{#4}
\nbput[labelsep=2pt]{\footnotesize $#2$}
}
% acdar - above labelled composite dependency arrow
\newcommand{\ncacdar}[4][0]{
\nccdar[#1]{#3}{#4}
\naput[labelsep=2pt]{\footnotesize $#2$}
}


% rsar - recursive S-arrow
\newcommand{\ncrsar}[2]{
\setlength{\sarnodesepAsav}{\sarnodesepA}
\setlength{\sarnodesepBsav}{\sarnodesepB}
\addtolength{\sarnodesepA}{3pt}
\addtolength{\sarnodesepB}{7pt}
\ncloop[nodesepA=\sarnodesepA,nodesepB=\sarnodesepB,
        offsetA=\saroffsetA,offsetB=\saroffsetB,
        armA=0.7cm,armB=0.6cm,angleA=90,angleB=-90,loopsize=-1,linearc=0.4
				]{-}{#1}{#2}
\ncput[nrot=:R,npos=5]{\pstriangle(0,0)(.2,.2)}
\setlength{\sarnodesepA}{\sarnodesepAsav}
\setlength{\sarnodesepB}{\sarnodesepBsav}
}

% pstrick supplements for multi-arrows

\newlength{\marnodesepA}
\newlength{\marnodesepB}
\newlength{\maroffsetB}
\newlength{\marnodesepBsav}

\newcommand{\marreset}{
\setlength{\marnodesepA}{0pt}
\setlength{\marnodesepB}{0pt}
\setlength{\maroffsetB}{0pt}
}

\marreset

%ncmarr[#1 arcangle1][#2 arcangle2]{#3 name}{#4 domain1}{#5 domain2}{#6 junction}{#7 codomain}
\newcommandtwoopt{\ncmarr}[6][8][8]{%
\ncarc[nodesepA=\marnodesepA,nodesepB=0,arcangle=#1]{-}{#3}{#5}
\ncarc[nodesepB=0,arcangle=-#1]{-}{#4}{#5}
\ncarc[arcangle=#2,nodesepB=\marnodesepB,offsetB=\maroffsetB]{->}{#5}{#6}
}%


\newcommandtwoopt{\nchmarr}[6][8][8]{%
\ncarc[nodesepA=\marnodesepA,nodesepB=0,arcangle=#1]{-}{#3}{#5}
\ncarc[nodesepB=0,arcangle=#1]{-}{#4}{#5}
\ncarc[arcangle=#2,nodesepB=\marnodesepB,offsetB=\maroffsetB]{->}{#5}{#6}
}%

\newcommandtwoopt{\ncamarr}[7][8][8]{%
\ncmarr[#1][#2]{#4}{#5}{#6}{#7}
\naput[npos=.05]{$#3$}
}%
\newcommandtwoopt{\ncbmarr}[7][8][8]{%
\ncmarr[#1][#2]{#4}{#5}{#6}{#7}
\nbput[npos=.05]{$#3$}
}%

\newcommandtwoopt{\ncbhmarr}[7][8][8]{%
\nchmarr[#1][#2]{#4}{#5}{#6}{#7}
\nbput[npos=.05]{$#3$}
}%

\newcommandtwoopt{\ncmarrr}[7][8][8]{
\ncarc[nodesepB=0,arcangle=#1]{-}{#3}{#6}
\ncline[nodesepB=0]{-}{#4}{#6}
\ncarc[nodesepB=0,arcangle=-#1]{-}{#5}{#6}
\ncarc[nodesepA=0,arcangle=#2]{->}{#6}{#7}
}

\newcommandtwoopt{\ncamarrr}[8][8][8]{
\ncmarrr[#1][#2]{#4}{#5}{#6}{#7}{#8}
\naput[npos=.05]{$#3$}
}
\newcommandtwoopt{\ncbmarrr}[8][8][8]{
\ncmarrr[#1][#2]{#4}{#5}{#6}{#7}{#8}
\nbput[npos=.05]{$#3$}
}

%gats.macros.tex

\usepackage{environ}    % also used in ermacros % here used for \NewEnvrion

\newcommand{\gat}[1][U]{
\ensuremath{\mathcal{#1}}}  % used to hav a space in here
\newcommand{\gatw}[1][U]{\gat[#1]\ }  % use this if need trailing space
\newcommand{\ingat}[1][U]{in \gat[#1]}
\newcommand{\isagat}[1][U]{\gat[#1] is a g.a.t.}
\newcommand{\inagat}{in a g.a.t. }

% macro for a generic theory
%\newcommand{\theory}
%{\textit{U}}

\newcommand{\intheory}
{is a derived rule of \gat[U]}

% Macros for GAT rules

\newcommand{\isT}[1]
{#1\mbox{ is a type}}

\newcommand{\ofT}[2]
{#1 \in #2
}

% Macros for GAT rules   <!-- new old -->
\newcommand{\istype}[1]
{#1\mbox{ is a type}}

\newcommand{\oftype}[2]
{#1 \in #2
}

%\context{x}{\Delta}{n}
\newcommand{\context}[3]
{\ofT{#1_1}{#2_1},... \ofT{#1_{#3}}{#2_{#3}(#1_1,...#1_{#3-1})}
}

%\subcontext{x}{\Delta}{i}{k}
\newcommand{\subcontext}[4]
{\ofT{#1_{#3_1}}{#2_{#3_1}},... \ofT{#1_{#3_#4}}{#2_{#3_#4}(#1_1,...#1_{#3_#4-1})}
}

% #schematic context
\newcommand{\schmcon}[3]
{\ofT{#1_1}{#2_1},... \ofT{#1_{#3}}{#2_{#3}}
}
% abbreviated to
\newcommand{\con}[3]
{\schmcon{#1}{#2}{#3}}

% schematic subcontext
%\subcon{x}{\Delta}{i}{k}
\newcommand{\subcon}[4]
{\ofT{#1_{#3_1}}{#2_{#3_1}},... \ofT{#1_{#3_#4}}{#2_{#3_#4}}
}

% permuted context
%\permcon{x}{\Delta}{n}{\sigma}
\newcommand{\permcon}[4]
{\ofT{#1_{#4(1)}}{#2_{#4(1)}},... \ofT{#1_{#4(#3)}}{#2_{#4(#3)}}
}
% permuted term
%\permterm{t}{n}{\sigma}
\newcommand{\permterm}[3]
{
#1_{#3(1)},...#1_{#3(#2)}
}


% Idioms
\newcommand{\xDelta}[1]{\con{x}{\Delta}{#1}}
\newcommand{\xDeltap}[1]{\con{x}{\Delta'}{#1}}
\newcommand{\xOmega}[1]{\con{x}{\Omega}{#1}}
\newcommand{\xOmegap}[1]{\con{x}{\Omega'}{#1}}
\newcommand{\yOmega}[1]{\con{y}{\Omega}{#1}}
\newcommand{\yOmegap}[1]{\con{y}{\Omega'}{#1}}

\newcommand{\xDeltasigma}[1]{\permcon{x}{\Delta}{#1}{\sigma}}
\newcommand{\xDeltapsigma}[1]{\permcon{x}{\Delta'}{#1}{\sigma}}
\newcommand{\xOmegasigma}[1]{\permcon{x}{\Omega}{#1}{\sigma}}
\newcommand{\xOmegapsigma}[1]{\permcon{x}{\Omega'}{#1}{\sigma}}
\newcommand{\yOmegasigma}[1]{\permcon{y}{\Omega}{#1}{\sigma}}
\newcommand{\yOmegapsigma}[1]{\permcon{y}{\Omega'}{#1}{\sigma}}

\newcommand{\xDeltainvsigma}[1]{\permcon{x}{\Delta}{#1}{\sigma^{-1}}}
\newcommand{\xDeltapinvsigma}[1]{\permcon{x}{\Delta'}{#1}{\sigma^{-1}}}
\newcommand{\xOmegainvsigma}[1]{\permcon{x}{\Omega}{#1}{\sigma^{-1}}}
\newcommand{\xOmegapinvsigma}[1]{\permcon{x}{\Omega'}{#1}{\sigma^{-1}}}
\newcommand{\yOmegainvsigma}[1]{\permcon{y}{\Omega}{#1}{\sigma^{-1}}}
\newcommand{\yOmegapinvsigma}[1]{\permcon{y}{\Omega'}{#1}{\sigma^{-1}}}

%Idioms enclosed as tuples
\newcommand{\encxDelta}[1]{\tuple{\con{x}{\Delta}{#1}}}
\newcommand{\encxDeltap}[1]{\tuple{\con{x}{\Delta'}{#1}}}
\newcommand{\encxOmega}[1]{\tuple{\con{x}{\Omega}{#1}}}
\newcommand{\encxOmegap}[1]{\tuple{\con{x}{\Omega'}{#1}}}
\newcommand{\encyOmega}[1]{\tuple{\con{y}{\Omega}{#1}}}
\newcommand{\encyOmegap}[1]{\tuple{\con{y}{\Omega'}{#1}}}

\newcommand{\encxDeltasigma}[1]{\tuple{\permcon{x}{\Delta}{#1}{\sigma}}}
\newcommand{\encxDeltapsigma}[1]{\tuple{\permcon{x}{\Delta'}{#1}{\sigma}}}
\newcommand{\encxOmegasigma}[1]{\tuple{\permcon{x}{\Omega}{#1}{\sigma}}}
\newcommand{\encxOmegapsigma}[1]{\tuple{\permcon{x}{\Omega'}{#1}{\sigma}}}
\newcommand{\encyOmegasigma}[1]{\tuple{\permcon{y}{\Omega}{#1}{\sigma}}}
\newcommand{\encyOmegapsigma}[1]{\tuple{\permcon{y}{\Omega'}{#1}{\sigma}}}

\newcommand{\encxDeltainvsigma}[1]{\tuple{\permcon{x}{\Delta}{#1}{\sigma^{-1}}}}
\newcommand{\encxDeltapinvsigma}[1]{\tuple{\permcon{x}{\Delta'}{#1}{\sigma^{-1}}}}
\newcommand{\encxOmegainvsigma}[1]{\tuple{\permcon{x}{\Omega}{#1}{\sigma^{-1}}}}
\newcommand{\encxOmegapinvsigma}[1]{\tuple{\permcon{x}{\Omega'}{#1}{\sigma^{-1}}}}
\newcommand{\encyOmegainvsigma}[1]{\tuple{\permcon{y}{\Omega}{#1}{\sigma^{-1}}}}
\newcommand{\encyOmegapinvsigma}[1]{\tuple{\permcon{y}{\Omega'}{#1}{\sigma^{-1}}}}

\newcommand{\tstyle}{\vdash}

% gat macros developed for cwf paper

% Expressing gats
\newenvironment{gatrules}
{
$$
\begin{array}{l l}
}
{
\end{array}
$$
}
\newcommand{\gatintros}
{
\textbf{Symbol} & \textbf{Introductory\ Rule}                      \\}

\newcommand{\gataxioms}
{\textbf{Axioms}\\}
\newcommand{\gatintro}[3]{\ #1 & #2 \tstyle #3 \\}
\newcommand{\gatlocalintro}[3]{\ #1 & #2 \dashv }
\newcommand{\gataxiom}[2]{\multicolumn{2}{l}{\ \ #1\mbox{,  whenever\ } #2} \\}
\newcommand{\noleft}{\left.\kern-\nulldelimiterspace} % so that no space taken by absent left brace


\newcommand{\gatmultiaxiom}[2]
{\multicolumn{2}{l}{
  \noleft
    \begin{array}{l}
		#1
    \end{array} 
  \right\} \mbox{whenever\ } 	#2 
	}\\}
	
	\newcommand{\axid}[1]{\text{#1}.\ }	

%New context sharing macros
\newcommand{\gatintroducing}[1]{
{\arraycolsep=0pt
  \begin{array}{l}
          #1
  \end{array}} &
}

%*********************************
% \begin{\gatgroup}{context}
%    rules
%  \end{\gatgroup}
%*********************************
\NewEnviron{gatgroup}[1]{%
  \noleft
  {\arraycolsep=0pt
   \begin{array}{l}
\BODY
    \end{array} 
   }
   \ \right\} 
	%\mbox{\ whenever\ } 
	#1
	\vspace{0.1cm} 
}
%*********************************

%*********************************
% \begin{\gatgroupnoshared}
%    rule
%  \end{\gatgroupnoshared}
%*********************************
\NewEnviron{gatgroupnoshared}{%
  {\arraycolsep=0pt
   \begin{array}{l}
\BODY
    \end{array} 
   }
   \ 
	\vspace{0.1cm} 
}
%*********************************

% \gatsingular[width]{context}{conclusion}
\newcommand{\gatsingular}[3][4cm]{
\begin{gatgroupnoshared}
\gatleaf[#1]{#2}{#3} 
\end{gatgroupnoshared}
}

%*********************************
% \gatleaf}[width]{context}{assertion}
%*********************************
\newcommand{\gatleaf}[3][4cm]{%
\makebox[#1]{$#3$ \dotfill} \dotfill \  #2
}
%*********************************
%*********************************
% \gatstandalonesingle}{context}{assertion}
%*********************************
\newcommand{\gatstandalonesingle}[2]{%
#2 \makebox[2.5cm]{\dotfill} \  #1
}
%*********************************

% \gataxiomno{axiomno}
\newcommand{\gataxiomno}[1]{\makebox[0.5cm]{} \axid{#1}}


% metagat.macros.tex

%Meta-theories

%\newcommand{\typ}{\triangleright}
\newcommand{\typ}{\nabla}
\newcommand{\trm}{\tau}
\newcommand{\cross}{\otimes}
\newcommand{\sub}{^*}
\newcommand{\diag}{\delta}

\newcommand{\typeseq}[2]
{\ofT{#1_1}{\typ},... \ofT{#1_{#2}}{\typ(#1_{#2-1})}}

\newcommand{\typeseqcont}[3]
{\ofT{#1_1}{\typ({#2})},... \ofT{#1_{#3}}{\typ(#1_{#3-1})}}

\newcommand{\Ob}{Ob}
\newcommand{\obj}{Ob} % <!-- new old --<
\newcommand{\Hom}{Hom}
\newcommand{\objseq}[2]
{\ofT{#1_1}{\obj},... \ofT{#1_{#2}}{\obj(#1_{#2-1})}}


\def\dottededge{\ncline[linestyle=dotted, nodesep=0.3cm]}
\def\noedge{\ncline[linestyle=none]}
\def\thinedge{\ncline[linewidth=0.4pt]}

\newcommand{\member}[1]
{\ncarc[arcangle=-30,nodesepB=0.03]{->}{\pspred}{\pssucc}
\nbput[labelsep=0.1]{#1}}

\newcommand{\loweraccutemember}[1]
{\ncarc[arcangle=-15,nodesepB=0.03]{->}{\pspred}{\pssucc}
\nbput[labelsep=0.05,npos=0.85]{#1}}

\newcommand{\uppermember}[1]
{\ncarc[arcangle=30,nodesepB=0.03]{->}{\pspred}{\pssucc}\naput{#1}}

\newcommand{\upperaccutemember}[1]
{\ncarc[arcangle=10,nodesepB=0.03]{->}{\pspred}{\pssucc}\naput[npos=0.85]{#1}}

% flexbranch 
% #1 node label
% #2 thislevelsep
% #3 next level sep
% #4 variable (eg x)
% #5 index leter (eg n)
% #6 close parenthesis
% #7 continuation branches
\newcommand{\flexbranch}[7]
{
\pstree[thislevelsep=*#2,nodesep=0.05]
		{\Rnode{#1 1}{\Tr{#4_1 #6}}}
	  {\pstree[thislevelsep=#3]  
				   {\Rnode{#1 2}{\Tr[edge=\dottededge]{#4_{#5} #6}}}
					 {#7}
		}
}

\newcommand{\flexbranchplusleaf}[6]
{
\flexbranch{#1}{#2}{#3}{#4} {#5} {#6}
  {
   %\Rnode{#1 3}{\Tr{#4 #6}}
	 \Tr{\Rnode{#1 3}{#4 #6}}
  }
}

\newcommand{\flexbranchplusarc}[7]
{
\flexbranch{#1}{#2}{#3}{#4} {#5} {#6}
  {
   %\Rnode{#1 3}{\Tr{#4 #6}\member{#7}}
	 \Tr{\Rnode{#1 3}{#4 #6}}\member{#7}
  }
}

\newcommand{\flexbranchinitialarc}[9]
{
\pstree[thislevelsep=*#2,nodesep=0.05]
		{\Rnode{#1 1}{\Tr{#4_#8 #6}}#9}
	  {\pstree[thislevelsep=#3]  
				   {\Rnode{#1 2}{\Tr[edge=\dottededge]{#4_{#5} #6}}}
					 {#7}
		}
}

\newcommand{\equality}[2]
{
\ncline [doubleline=true, nodesep=0.2cm]{#1}{#2}
}
\newcommand{\equalityarc}[2]
{
\ncarc [arcangleA=-30, arcangleB=-20, doubleline=true, nodesep=0.1cm]{#1}{#2}
}

\usepackage[margin=4.0cm]{geometry} %was 3cm
\usepackage{mathptmx}
\usepackage{amsfonts}
\usepackage{array}
\usepackage{pstricks}
\usepackage{pst-tree}
\usepackage{pst-plot}
\usepackage{pst-node}
\usepackage{stmaryrd}
\usepackage{amsmath}
\usepackage{verbatim}
\usepackage{graphicx}  
\usepackage{calc}
\usepackage{xifthen}
\usepackage{xcolor}
\usepackage{color}
\usepackage{stringstrings}
%\usepackage[small,bf,margin=3pt,format=hang, labelsep=endash,singlelinecheck=false]{caption} %prevuiously justification=justified
%\usepackage{enumerate}
%\usepackage{enumitem}
\usepackage{enumerate}
\usepackage[shortlabels]{enumitem}
\usepackage{float}
\usepackage[section]{placeins}
%\setlength{\captionmargin}{5pt}
\usepackage{environ}
\usepackage{multirow}
\usepackage{rotating}
\usepackage{longtable}
\usepackage{afterpage}
\usepackage{needspace}


%DEFINE ENVIRONMENT BLOCK
% Riddle
\newsavebox{\riddlebox}

\newenvironment{erexample}
{\newcommand\colboxcolor{F0F0F0}%was F8F8F8
\begin{lrbox}{\riddlebox}
\begin{minipage}{\dimexpr\columnwidth-2\fboxsep\relax} \textbf{} \\ \itshape}
{\end{minipage}\end{lrbox}%
%\begin{center}
\colorbox[HTML]{\colboxcolor}{\usebox{\riddlebox}}
%\end{center}
}

\newenvironment{erbox}
{\newcommand\colboxcolor{F0F0F0}%was F8F8F8
\begin{lrbox}{\riddlebox}%
\begin{minipage}{\dimexpr\columnwidth-2\fboxsep\relax} }
{\end{minipage}\end{lrbox}%
%\begin{center}
\colorbox[HTML]{\colboxcolor}{\usebox{\riddlebox}}
%\end{center}
}

%\begin{erboxedFigure}{#1 FigureParam}{#2 Label}{#3 Caption}
\NewEnviron{erboxedFigure}[3]{%
\begin{figure}[#1]
\begin{erexample}
\begin{center}
\BODY
\end{center}
\vspace{-0.5cm}
\caption{#3}
\label{#2}
\end{erexample}
\end{figure}
}

\newcommand{\erpictureFolder}[0]{../SharedPictures}

\newcommand{\ercenterPicture}[1]{
\begin{center}
\input{\erpictureFolder/#1}
\end{center}
}


\newlength{\erhalfHt}

%\erinlinePicture{#1 pictureFilename}{#2 pictureHeight}
\newcommand{\erinlinePicture}[2]{
\setlength{\erhalfHt}{#2cm * \real{0.5}}
\raisebox{-\erhalfHt}[\erhalfHt + 0.5cm][\erhalfHt + 0.5cm]{
\input{\erpictureFolder/#1}
} 
}

%\erplainFig{#1 pictureFilename}{#2 figureParam}{#3Caption}
\newcommand{\erplainFig}[3]{
\begin{figure}[#2]
\begin{center}
\input{\erpictureFolder/#1}
\end{center}
\caption{#3}
\label{#1}
\end{figure}
}

%\erboxedFigPicture{#1 pictureFilename}{#2 figureParam}{#3Caption}
\newcommand{\erboxedFigPicture}[3]{
\begin{figure}[#2]
\begin{erexample}
\vspace{-0.5cm}
\begin{center}
\input{\erpictureFolder/#1}
\end{center}
\caption{#3}
\label{#1}
\end{erexample}
\end{figure}
}

%\erLeftSideFig{#1 pictureFilename}{#2 figureParam}{#3Caption}
\newcommand{\erLeftSideFig}[3]{
\begin{figure}[#2]
\begin{erexample}
  \begin{minipage}[c]{0.4\textwidth}
    \caption{#3}
    \label{#1}
  \end{minipage}
  \begin{minipage}[c]{0.5\textwidth}
    \input{\erpictureFolder/#1}
  \end{minipage}
\end{erexample}
\end{figure}
}

%\erbulletedFig{#1 pictureFilename}{#2 figureParam}{#3Caption}
\NewEnviron{erbulletedFig}[3]{%
\begin{figure}[#2]
\begin{erexample}
\vspace{-0.5cm}
\begin{center}
$
\begin{array}{c m{0.25cm} | m{6cm}}
\raisebox{-2.0cm}{
\input{\erpictureFolder/#1}}& & \text{\parbox{6cm}{\raggedright{\footnotesize{
\begin{enumerate}[(i)]
\BODY
\end{enumerate}}}}} \\
\end{array}
$
\end{center}
\caption{#3}
\label{#1}
\end{erexample}
\end{figure} 
}


%\begin{erbulletedDimFig}{#1 pictureFilename}{#2figureParam} {#3Caption} {#4PictureHeight}{#5TextWidth}

\NewEnviron{erbulletedDimFig}[5]{%
\begin{figure}[#2]
\begin{erexample}
\vspace{-0.5cm}
\begin{center}
$
\begin{array}{c m{0.25cm} |  m{#5cm}}
\setlength{\erhalfHt}{#4cm * \real{0.5}}
\raisebox{-\erhalfHt}{
\input{\erpictureFolder/#1}}& & \text{\parbox{#5cm}{\raggedright{\footnotesize{
\begin{enumerate}[(i)]
\BODY
\end{enumerate}}}}} \\
\end{array}
$
\end{center}
\caption{#3}
\label{#1}
\end{erexample}
\end{figure} 
}

%\begin{ernotedModel}{#1 pictureFilename}{#2PictureHeight}{#3PictureWidth}{#4TextWidth}

\NewEnviron{ernotedModel}[4]{%
\begin{center}
$
\begin{array}{m{#3cm} m{1cm} | c m{#4cm}}
\setlength{\erhalfHt}{#2cm * \real{0.5}}
\raisebox{-\erhalfHt}{
\input{\erpictureFolder/#1}}& & & \text{\parbox{#4cm}{\raggedright{\footnotesize{
\BODY
}}}} \\
\end{array}
$
\end{center} 
}

%\begin{ermodelText}{#1 pictureFilename}{#2PictureHeight}{#3PictureWidth}{#4TextWidth}

\NewEnviron{ermodelText}[4]{%
\begin{center}
\begin{tabular}{m{#3cm} m{1cm}  c m{#4cm}}
\setlength{\erhalfHt}{#2cm * \real{0.5}}
\raisebox{-\erhalfHt}{
\input{\erpictureFolder/#1}}& & & \text{\parbox{#4cm}{\raggedright{\small{
\BODY
}}}} \\
\end{tabular}
\end{center} 
}


%\erbulletedModel{#1 pictureFilename}{#2PictureHeight}{#3PictureWidth}{#4TextWidth}

\NewEnviron{erbulletedModel}[4]{%
\begin{center}
$
\begin{array}{m{#3cm} m{1cm} | c m{#4cm}}
\setlength{\erhalfHt}{2cm * \real{0.5}}
\raisebox{-\erhalfHt}{
\input{\erpictureFolder/#1}}& & & \text{\parbox{#4cm}{\raggedright{\footnotesize{
\begin{enumerate}[(i)]
\BODY
\end{enumerate}}}}} \\
\end{array}
$
\end{center} 
}



%\ernotedDimFig{#1 pictureFilename}{#2 figureParam}{#3Caption}{#4PictureHeight}{#5TextWidth}
\NewEnviron{ernotedDimFig}[5]{%
\begin{figure}[#2]
\begin{erexample}
\vspace{-0.5cm}
\begin{center}
$
\begin{array}{c m{0.25cm} | c m{#5cm}}
\setlength{\erhalfHt}{#4cm * \real{0.5}}
\raisebox{-\erhalfHt}{
\input{\erpictureFolder/#1}}& & & \text{\parbox{#5cm}{\raggedright{\footnotesize{
\BODY }}}}\\
\end{array}
$
\end{center}
\caption{#3}
\label{#1}
\end{erexample}
\end{figure} 
}
%\begin{ernotedDimFigPW}{#1 pictureFilename}{#2 figureParam}{#3Caption}{#4PictureHeight}{#5PictureWidth}{#6TextWidth}
\NewEnviron{ernotedDimFigPW}[6]{%
\begin{figure}[#2]
\begin{erexample}
\vspace{-0.5cm}
\begin{center}
$
\begin{array}{>{\centering}m{#5cm} m{0.5cm} | c m{#6cm}}
\setlength{\erhalfHt}{#4cm * \real{0.5}}
\raisebox{-\erhalfHt}{
\centering \input{\erpictureFolder/#1}
}& & & \text{\parbox{#6cm - 0.5cm}{\raggedright{\footnotesize{
\BODY }}}}\\
\end{array}
$ \\
\vspace {0.2cm}
\end{center}
\caption{#3}
\label{#1}
\end{erexample}
\end{figure}
}



\newenvironment{erquote}
{\begin{quote}\itshape}
{\end{quote}}


\setcounter{equation}{0}
\bibliographystyle{plain} % was hplain

\title{Dependency Categories}
\author{John Cartmell}
\begin{document}
\maketitle

\section{Context, Information Theory, Categories and Types That Vary}

\cite{Cartmell86B}.

\section{Directed Graphs and Systems of Dependent Type}

\noindent The directed graph:

\begin{center}
\begin{equation}
\begin{array}{p{1.5cm}cccp{2cm}c}
&                & \Rnode{C2}{C}&   \\ [0.8cm]
&                & \Rnode{B}{B} &  \\ [0.8cm]
&\Rnode{A1}{A_1} &              & \Rnode{A2}{A_2}\\ 
\end{array}
\ncsar{C2}{B}
\ncsar{B}{A1}
\ncsar{B}{A2}
\setlength {\saroffsetA}{-2pt}
\setlength {\saroffsetB}{-2pt}
\ncsar[-15]{C3}{B3}
\setlength {\saroffsetA}{2pt}
\setlength {\saroffsetB}{2pt}
\ncsar[15]{C3}{B3}
\sarreset
\ncsar{B3}{A3}
\end{equation}
\end{center}

\noindent can be interpreted as representing the following type system:

%\addtocounter{equation}{-1}
\begin{align}
&A_1\mbox{ is a type} && \tag*{(\theequation a)}\\
&A_2\mbox{ is a type} && \tag*{(\theequation b)}\\
&x_1\in A_1, x_2 \in A_2 : B(x_1,x_2) \mbox{ is a type} && \tag*{(\theequation c)}\\
&x_1\in A_1, x_2 \in A_2, y \in B(x_1,x_2): C(x_1,x_2,y) \mbox{ is a type} && \tag*{(\theequation d)}
\end{align}


\section {Acyclic Categories}
A category C is called \textit{acyclic}, if it has no inverses and no nonidentity
endomorphisms. This definition is given by Kozlov (see \cite{Kozlov2007}) who offers the following intuition:
\begin{erquote}
Another way to visualize acyclic categories is to think of them as those
that can be drawn on a sheet of paper, with dots indicating the objects, and
straight or slightly bent arrows, all pointing down, indicating the nonidentity
morphisms...
\end{erquote}


\noindent
Previous authors had referred to such categories as being \textit{loop-free}. \\

\noindent
The following is an example of an acyclic category given by Kolozov which we have rearranged and relabelled:

\begin{center}
\begin{equation}
\begin{array}{p{1.5cm}cccp{2cm}c}
&                & \Rnode{C}{C}&   \\ [1.4cm]
&                & \Rnode{B}{B} &  \\ [0.8cm]
&                & \Rnode{A}{A} &  
\end{array}
\end{equation}
\ncasar[-30]{y_1}{C}{B}
\ncasar[30]{y_2}{C}{B}
\ncasar{x}{B}{A}
\setlength {\saroffsetA}{-2pt}
\setlength {\saroffsetB}{-2pt}
\ncbsar[-50]{y_1 \circ x = y_2 \circ x}{C}{A}
\sarreset
\end{center}


\noindent This category can be taken as representing the following type system:
\addtocounter{equation}{-1}
\begin{align}
&A\mbox{ is a type} && \tag*{(\theequation a)}\\
&x\in A: B(x) \mbox{ is a type} && \tag*{(\theequation b)}\\
&x\in A_1, y_1 \in B(x), y_2 \in B(x) : C(x,y_1,y_2) \mbox{ is a type} && \tag*{(\theequation c)}
\end{align}

\noindent
Contrast with this acyclic category:
\begin{center}
\begin{equation}
\begin{array}{p{1.5cm}cccp{2cm}c}
&                & \Rnode{C}{C}&   \\ [1.4cm]
&                & \Rnode{B}{B} &  \\ [0.8cm]
&                & \Rnode{A}{A} &  
\end{array}
\end{equation}
\ncasar[-30]{y_1}{C}{B}
\ncasar[30]{y_2}{C}{B}
\ncasar{x}{B}{A}

\setlength {\saroffsetA}{-2pt}
\setlength {\saroffsetB}{-2pt}
\ncbsar[-60]{x_1 = y_1 \circ x}{C}{A}
\setlength {\saroffsetA}{2pt}
\setlength {\saroffsetB}{2pt}
\ncasar[60]{x_2 = y_2 \circ x}{C}{A}
\sarreset
\end{center}
\noindent
which represents :
\addtocounter{equation}{-1}
\begin{align}
&A\mbox{ is a type} && \tag*{(\theequation a)}\\
&x\in A: B(x) \mbox{ is a type} && \tag*{(\theequation b)}\\
&x_1\in A_1, y_1 \in B(x_1), x_2\in A_1, y_2 \in B(x_2) : C(x_1,x_2,y_1,y_2) \mbox{ is a type} && \tag*{(\theequation c)}
\end{align}

\section{Category with Distinguished Morphisms}

\begin{definition}
\noindent A category $\cat{C}$ is \textit{well-founded} provided that for 
any object $A$ of $C$ the set of morphisms with domain $A$ is finite. 
\end{definition}
\begin{definition}
Define a \textit{category with distinguished morphisms} to be a category \cat{C} along with a wide acyclic subcategory \cat{D} that is well-founded. 
Morphisms of the subcategory are referred to as d-morphism. 
We will write $f:B \dmorph A$ in \cat{C}
to mean that 
$f: B \morph A$ in \cat{C} and that $f$ is a d-morphism i.e is in the subcategory \cat{D} of distinguished morphisms\footnote{
Note that this the same notation has been used differently in \cite{Cartmell86} -- a contextual category does give rise to a category with distinguished morphisms but only
if the distinguished morphisms are taken to be not just those
denoted $p_B: B \dmorph A$ in that paper
but also all those in  the subcategory generated by such.
}. 
\end{definition} 

\section{Spans and Cospans}
\noindent
The category $\Lambda$ is defined to be this category:
\begin{center}
\begin{displaymath}
\begin{array}{cp{0.1cm}cp{0.1cm} c}
            & &\Rnode{d}{\bullet}                  \\ [0.4cm]
\Rnode{c1}{\bullet}& &                & & \Rnode{c2}{\bullet} 
\end{array}
\end{displaymath}
\ncarr{d}{c1}
\ncarr{d}{c2}
\end{center}

\noindent
If we need names for the individual objects and morphisms we will use the names show here:
\begin{center}
\begin{displaymath}
\begin{array}{cp{0.1cm}cp{0.1cm} c}
            & &\Rnode{d}{\Delta}                  \\ [0.4cm]
\Rnode{c1}{\Gamma_1}& &                & & \Rnode{c2}{\Gamma_2} 
\end{array}
\end{displaymath}
\jcbarr{\xi}{d}{c1}
\ncaarr{\gamma}{d}{c2}
\end{center}

\noindent
A \textit{span} within a category \cat{C} is exactly a functor 
$S: \Lambda \morph \cat{C}$; it is a pair of morphisms of \cat{C} that have a common domain object i.e it is any diagram of this form:

\begin{center}
\begin{equation}
\label{genericspan}
\begin{array}{cp{0.1cm}cp{0.1cm} c}
            & &\Rnode{d}{D}                  \\ [0.4cm]
\Rnode{c1}{C_1}& &                & & \Rnode{c2}{C_2} 
\end{array}
\end{equation}
\jcbarr{f}{d}{c1}
\ncaarr{g}{d}{c2}
\end{center}

\noindent Similarly a \textit{cospan} in a category \cat{C} is exactly a functor $S: \Lambda^{op} \morph \cat{C}$ and so it is exactly a diagram of this form:
\begin{center}
\begin{equation}
\label{genericcospan}
\begin{array}{cp{0.1cm}cp{0.1cm} c}
\Rnode{c1}{C_1}& &                & & \Rnode{c2}{C_2} \\ [0.4cm]
            & &\Rnode{b}{B}
\end{array}
\end{equation}
\jcbarr{x_1}{c1}{b}
\ncaarr{x_2}{c2}{b}
\end{center}

\noindent
The cospan (\ref{genericcospan}) of $C_1$ and $C_2$ is said to be a 
\textit{coincident cospan} of the span (\ref{genericspan}) iff the 
diagram:
\begin{center}
\begin{equation}
\begin{array}{cp{0.1cm}cp{0.1cm} c}
            & &\Rnode{d}{D}                           \\ [0.4cm]
\Rnode{c1}{C_1}& &                & & \Rnode{c2}{C_2} \\ [0.4cm]
            & &\Rnode{b}{B}
\end{array}
\end{equation}
\jcbarr{f}{d}{c1}
\ncaarr{g}{d}{c2}
\jcbarr{x_1}{c1}{b}
\ncaarr{x_2}{c2}{b}
\end{center}

\noindent commutes. \\

\noindent We will say that the cospan (\ref{genericcospan}) is a 
\textit{minimal coincident cospan} for the span (\ref{genericspan}) iff it is 
coincident  and there does no exist a cospan 
\begin{center}
\begin{equation}
\begin{array}{cp{0.1cm}cp{0.1cm} c}
\Rnode{c1}{C_1}& &                & & \Rnode{c2}{C_2} \\ [0.4cm]
            & &\Rnode{b}{B'}
\end{array}
\end{equation}
\jcbarr{x'_1}{c1}{b}
\ncaarr{x'_2}{c2}{b}
\end{center}

\noindent
that is coincident to (\ref{genericspan}) and such that there is a morphism 
$h: B' \morph B$ such that both \\
\vspace {0.25cm}

$
\begin{array}{cc}
\Rnode{C1}{C_1}                             \\ [0.7cm]
               & \Rnode{Bp}{B'}             \\ [0.8cm]
							 &  \Rnode{B}{B} 
\end{array}
$
\jcbarr[-30]{x_1}{C1}{B}
\ncaarr{h}{Bp}{B}
\ncaarr{x'_1}{C1}{Bp}
and
$
\begin{array}{p{1.5cm}cc}
&                &\Rnode{C2}{C_2}     \\ [0.7cm]
&\Rnode{Bp}{B'}                       \\ [0.8cm]
&\Rnode{B}{B}
\end{array}
$
\ncaarr[30]{x_2}{C2}{B}
\jcbarr{h}{Bp}{B}
\jcbarr{x'_2}{C2}{Bp}
commute. 
\noindent
A cospan $F:\Lambda^{op} \morph \cat{C}$ is said to \textit{factor through}  a diagram 
$J:S \morph \cat{C}$ iff there exists a cospan $F': \Lambda^{op} \morph S$ 
such that $F' \circ J = F$.
  
\section{Higher Cospans and Quotiented Higher Cospans}
The notion of a \textit{higher cospan} was
 introduced in \cite{Grandis2007} for diagrams
in a category of shape $\Lambda^{op^n}$but for us there are significant diagrams whose 
shape category is a quotient of  ${\Lambda^{op}}^n$ and since we are not aware
of further terminology in this area we introduce some here. It is appropriate to make use the join symbol ($\Join$) from relational algebra. \\

\noindent For any $n \geq 2$ define the category \ndidly to be this category:

\begin{center}
\begin{displaymath}
\begin{array}{cp{0.3cm}c         p{0.3cm}  c p{0.2cm} c}
\Rnode{d1}{\Delta_1}& &               & &     & &\Rnode{d2}{\Delta_2}  \\ [1.2cm]
\Rnode{c1}{\Gamma_1}& &\Rnode{c2}{\Gamma_2}& & ... & &\Rnode{cn}{\Gamma_n}
\end{array}
\end{displaymath}
\ncarr{d1}{c1}
\nbput[npos=0.4, labelsep=1pt]{\footnotesize $\xi_1$}
\ncarr{d1}{c2}
\nbput[npos=0.3, labelsep=1pt]{\footnotesize $\xi_2$}
\ncarr{d1}{cn}
\naput[npos=0.3, labelsep=1pt]{\footnotesize $\xi_n$}
\ncarr{d2}{c1}
\nbput[npos=0.3, labelsep=1pt]{\footnotesize $\gamma_1$}
\ncarr{d2}{c2}
\naput[npos=0.25, labelsep=1pt]{\footnotesize $\gamma_2$}
\ncarr{d2}{cn}
\naput[npos=0.4, labelsep=1pt]{\footnotesize $\gamma_n$}
\end{center}

\begin{definition}
A \textit{\ndidlycospan}
 in category \cat{C}  is any diagram in \cat{C} having 
shape $\ndidly$  i.e. it is any function $F: \ndidly \morph \cat{C}$
\end{definition}

\section{ Some Other Preliminary Definitions\protect\footnote{Please regard the terminology introduced here as provisional -- the author would be glad to receive suggested improvements especially where he has failed to use prior established terminology.}}

\begin{definition}
\noindent If \cat{C} is a category  then a \textit{characterising family} for a span $s$ of \cat{C} is defined to be the set of its minimal coincident cospans.
\end{definition}


\noindent 

\begin{definition}
If \cat{C} is a category with distinguished morphisms 
and if $\tuple{f_1,...f_n}$ is a tuple of d-morphisms of \cat{C} 
with common domain then define the \textit{characterising diagram} of 
$\tuple{f_1,...f_n}$
to be the diagram with shape category \ndidly where 
$$n = 
\sum_{\substack{i,\\1 \leq i \leq n}} \  \sum_{\substack{j,\\ 1 \leq j \leq n,\\ j \neq i}} \ | \chi_{i,j} | $$
and where $\chi_{i,j}$ is the characterising family within the subcategory 
of d-morphisms of \cat{C} for the pair $f_i,f_j$ and 
$|\chi_{i,j}|$ is its cardinality,
and with functor $D: S \morph \cat{C}$ defined for $1 \leq i \leq n$, $1 \leq j \leq n$, $i \neq j$, $1 \leq k \leq | \chi_{i,j}$ by
$$ \xi_{i,j,k}   \mapsto F_{i,j,k}(\xi)$$
and
$$ \gamma_{i,j,k}   \mapsto F_{i,j,k}(\gamma) $$
where $F_{i,j,k}$ is the $k$'th cospan within the characteristic family of the pair $f_i,f_j$.
\end{definition}

\begin{definition}
If \cat{C} is a category with distinguished morphisms then a family of 
d-morphisms  $f_i: B \dmorph A_i$ is said to
be a basis for object B iff for every morphism $f: B \dmorph X$ in \cat{C} either 
\begin{enumerate}[(i)]
\item
$f$ factors through one of $f_i$ for some $i$ i.e there exists $g:A_i \morph X$, for some $i$, such that $f_i \circ g = f$ or
\item
there exists $1 \leq k_1 \leq k_2 ... k_m \leq n$, and d-morphisms 
$g_1,...,g_m$ such that for each $j$, $1 \leq j \leq m$, 
$g_j: X \dmorph A_{k_j}$ and

$$
f \circ g_j = f_{k_j}
$$
and such that the diagram

\begin{center}
\psset{arrowsize=5pt,arrowinset=0.7}
\psset{nodesep=7pt}
\setlength{\arraycolsep}{0.3cm}
\begin{displaymath}
\begin{array}{ccp{0.2cm}p{0.075cm}c}
&&\Rnode{top}{$X$}& & \\[1.5cm]
\Rnode{A}{A_{j_1}}&\Rnode{B2}{A_{j_2}}&& ... & \Rnode{Bm}{A_{j_m}}\\
\end{array}
\end{displaymath}
\psset{nodesepA=4pt,nodesepB=7pt}
\ncline{-}{top}{A}
\tlput{\footnotesize $g_1$}
\ncput[nrot=:R,npos=1]{\pstriangle(0,0)(.2,.2)}
\ncline{-}{top}{B2}
\trput{\footnotesize $g_2$}
\ncput[nrot=:R,npos=1]{\pstriangle(0,0)(.2,.2)}
\ncline{-}{top}{Bm}
\trput{\footnotesize $g_m$}
\ncput[nrot=:R,npos=1]{\pstriangle(0,0)(.2,.2)}
\end{center}

is a limit cone\footnote{I want to say cannonical limit here but I can't.
Monitor this situation to see whether a problem or not. I think it is a minor annoyance. The question is whether if X dependent on Y and Y' isomorphic to Y then X is dependent on Y'. This in turn has an effect on the round trip from presentation to category back to presentation. } in category \cat{C} for the characteristic diagram of $g_1,...g_m$.
\end{enumerate}

\end{definition}

\section{Cones and Tight Cones}
Recall the definition of cone $\tuple{ N,\psi}$ to a diagram $F:J \morph C$ of a category $C$ as an object $N$ of $C$ and a family of morphisms $\psi_j$ indexed by objects $j$ of $J$ such that for all $f: j \morph j'$ in $J$, $\psi_j \circ F(f) = \psi_{j'}$. Define a cone $\tuple{ N,\psi}$to be \textit{tight} iff for all pairs $j_1,j_2$ of objects of shape category $J$, every minimal coincident cospan of the span:

\begin{center}
\begin{equation}
\begin{array}{cp{0cm}cp{0cm} c}
            & &\Rnode{d}{N}                           \\ [0.4cm]
\Rnode{c1}{F(j_1)}& &                & & \Rnode{c2}{F(j_2)} 
\end{array}
\end{equation}
\jcbarr{\psi_{j_1}}{d}{c1}
\ncaarr{\psi_{j_2}}{d}{c2}
\end{center}

factors through the diagram $F$.

\section{Definition of Dependency Category}
A \textit{dependency category} is a category \cat{C} with distinguished morphisms and a terminal object $1$ and :

\begin{enumerate} [(i)]
	\item for any object A of \cat{C} the unique morphism $t_A:A \morph 1$ is a d-morphism
	
	
	\item for all diagrams D of d-morphisms \begin{center}
$
\begin{array}{cp{0cm}cp{0cm}c}
\Rnode{B1}{B}& &            &  &\Rnode{B2}{B'} \\ [1.2cm]
               & &\Rnode{A}{A}& &
\end{array}
$
\ncbsar{x}{B1}{A}
\ncasar{x'}{B2}{A}
\end{center}
	
an object  $ \fibrex{B_1}{B_2}{D}$ of \cat{C} and d-morphisms 
$p_1: \fibrex{B_1}{B_2}{D} \dmorph B_1$ and  $p_1: \fibrex{B_1}{B_2}{D} \dmorph B_1$ such that :

\begin{center}
\begin{displaymath}
\begin{array}{cp{0.1cm}cp{0.1cm} c}
            & &\Rnode{d}{\fibrex{B_1}{B_2}{D}}                  \\ [0.4cm]
\Rnode{c1}{B_1}& &                & & \Rnode{c2}{B_2} 
\end{array}
\end{displaymath}
\ncbsar{p_1}{d}{c1}
\ncasar{p_2}{d}{c2}
\end{center}

is a pullback in category \cat{C} and such that for any 
cone $\tuple{N,\psi}$ to diagram $D$
the mediating morphism $h: N \morph B_1 \underset{D}{\Join} B_2$ is a d-moprphism iff 
every morphism in the cone \ $\tuple{N,\psi}$ is a d-morphism and the cone is tight.

	\item Other pullbacks: if
\begin{center}
\begin{displaymath}
\begin{array}{cp{.9cm}c}
            & & \Rnode{X}{X} \\ [1.2cm]
\Rnode{A}{A}& & \Rnode{B}{B} \\
\end{array}
\end{displaymath}
\jcbarr{f}{A}{B}
\ncasar{y}{X}{B}
\end{center}

then there is a pullback 
\vspace{3mm}
\begin{center}
\begin{displaymath}
\begin{array}{cp{.9cm}c}
\Rnode{fstarX}{X[f|y]} & & \Rnode{X}{X}\\ [1.2cm]
\Rnode{A}{A}         & & \Rnode{B}{B}
\end{array}
\end{displaymath}
\ncbsar{f^*y}{fstarX}{A}
\jcbarr{f}{A}{B}
\ncaarr{q(f,y)}{fstarX}{X}
\ncasar{y}{X}{B}
\end{center}
and pullbacks cohere\footnote{exactly as for contextual category}
and if also:
\begin{center}
$
\begin{array}{cp{.9cm}c}
            & & \Rnode{X}{X} \\ [1.2cm]
& & \Rnode{Bprime}{B'} \\
\end{array}
$
\ncasar{$y'$}{X}{Bprime}
\end{center}
and $y \neq y'$ then $q(f,y) \circ y'$ is a d-morphism. 
\noindent Further, if
\begin{center}
$
\begin{array}{p{2cm}p{0.5cm}p{0.5cm}p{0.5cm}}
& & \Rnode{X}{X}& \\ [1.4cm]
\Rnode{A}{A} & \Rnode{B1}{$B_1$} & ... & \Rnode{Bm}{$B_m$}\\
\end{array}
$
\jcbarr{f}{A}{B1}
\ncbsar{y_1}{X}{B1}
\ncasar{y_m}{X}{Bm}
\end{center}
and $y_1,...y_m$ is a basis for $X$ 
\noindent then the set:
\vspace{2mm}
\begin{center}
\psset{arrowsize=5pt,arrowinset=0.7}
\psset{nodesep=7pt}
\setlength{\arraycolsep}{0.3cm}
\begin{displaymath}
\begin{array}{ccp{0.2cm}p{0.075cm}c}
&&\Rnode{top}{$X[f|y_1]$}& & \\[1.5cm]
\Rnode{A}{A}&\Rnode{B2}{B_2}&& ... & \Rnode{Bm}{B_m}\\
\end{array}
\end{displaymath}
\psset{nodesepA=4pt,nodesepB=7pt}
\ncline{-}{top}{A}
\tlput{\footnotesize $f^*y_1$}
\ncput[nrot=:R,npos=1]{\pstriangle(0,0)(.2,.2)}
\ncline{-}{top}{B2}
\trput{\footnotesize $f_*y_2$}
\ncput[nrot=:R,npos=1]{\pstriangle(0,0)(.2,.2)}
\ncline{-}{top}{Bm}
\trput{\footnotesize $f_*y_m$}
\ncput[nrot=:R,npos=1]{\pstriangle(0,0)(.2,.2)}
\end{center}
is a basis for  $f^*X$.


\item If in (iii) the morphism f is a dependency then  $q(f,y)$ is also a dependency and 

\noindent for any diagram D:
\begin{center}
$
\begin{array}{cp{0cm}cp{0cm}c}
\Rnode{B1}{B_1}& &            &  &\Rnode{B2}{B_2} \\ [1.2cm]
               & &\Rnode{A}{A}& &
\end{array}
$
\ncbsar{x_1}{B1}{A}
\ncasar{x_2}{B2}{A}
\end{center}

\noindent then
$B_2[x_1|x_2]=B_1[x_2|x_1] = \fibrex{B_1}{B_2}{D}$ and $q(x_1,x_2)= x_2\sub x_1 $ and $q(x_2,x_1)= x_1 \sub x_2$
\noindent
so that we have:

\begin{center}
$
\begin{array}{cp{0cm}cp{0cm}c}
               && \Rnode{X}{\fibrex{x_1}{x_2}{D}}&&                \\ [1.2cm]
\Rnode{B1}{x_1}&&                                && \Rnode{B2}{x_2} \\ [1.2cm]
               &&\Rnode{A}{A}                    && 
\end{array}
$
\ncbsar{x_1}{B1}{A}
\ncasar{x_2}{B2}{A}
\ncbsar{q(x_2,x_1)=x_1*x_2=p_1}{X}{B1}
\ncasar{q(x_1,x_2)=x_2*x_1=p_2}{X}{B2}
\end{center}






\item Equalisers and properties of equalisers: If
\setlength{\arraycolsep}{1cm}
\psset{arrowsize=5pt,arrowinset=0.7}
\psset{nodesep=7pt}
$
\begin{array}{c}
\Rnode{B}{B} \\ [1.1cm]
\Rnode{A}{A} \\
\end{array}
$
\psset{nodesepA=4pt,nodesepB=7pt}
\ncarc[arcangle=-30]{-}{B}{A}
\nbput{\footnotesize $x_1$}
\ncput[nrot=:R,npos=1]{\pstriangle(0,0)(.2,.2)}
\ncarc[arcangle=30]{-}{B}{A}
\naput{\footnotesize $x_2$}
\ncput[nrot=:R,npos=1]{\pstriangle(0,0)(.2,.2)}
then there is an equaliser:
\setlength{\arraycolsep}{.75cm}
$
\begin{array}{ccc}
\Rnode{Eq}{\underset{x_1=x_2}{\sigma}B} & \Rnode{B}{B}  & \Rnode{A}{A} \\
\end{array}
$
\psset{nodesepA=4pt,nodesepB=7pt}
\ncline[offset=4pt]{-}{B}{A}
\naput{\footnotesize $x_1$}
\ncput[nrot=:R,npos=1]{\pstriangle(0,0)(.2,.2)}
\ncline [offset=-4pt]{-}{B}{A}
\nbput{\footnotesize $x_2$}
\ncput[nrot=:R,npos=1]{\pstriangle(0,0)(.2,.2)}
\ncline{->}{Eq}{B}
\naput{\footnotesize $e$}
\noindent and the morphism $e \circ x_1$ which equals, by definition, $e \circ a_2$  is a d-morphism as shown
in the following diagram:
\vspace{.2cm}
\begin{center}
\setlength{\arraycolsep}{.3cm}
$
\begin{array}{ccc}
\Rnode{Eq}{\underset{x_1=x_2}{\sigma}B} & &\Rnode{B}{B} \\[1.4cm]
 & \Rnode{A}{A} &\\
\end{array}
$
\psset{nodesepA=3pt,nodesepB=9pt}
\ncarc[arcangle=-15]{-}{B}{A}
\nbput{\footnotesize $x_1$}
\ncput[nrot=:R,npos=1]{\pstriangle(0,0)(.2,.2)}
\psset{nodesepA=4pt,nodesepB=7pt}
\ncarc[arcangle=15]{-}{B}{A}
\naput{\footnotesize $x_2$}
\ncput[nrot=:R,npos=1]{\pstriangle(0,0)(.2,.2)}
\ncline{->}{Eq}{B}
\naput{\footnotesize $e$}
\ncarc[arcangle=-5]{-}{Eq}{A}
\nbput{\footnotesize $e \circ x_1 = e \circ x_2$}
\ncput[nrot=:R,npos=1]{\pstriangle(0,0)(.2,.2)}
\end{center}

\noindent For any other d-morphism $a':B\darrow A'$ leaving B, the morphism $e \circ a'$ is a d-morphism; furthermore, if
\begin{center}
\setlength{\arraycolsep}{.1cm}
$
\begin{array}{ccp{.3cm}c}
                  & \multicolumn{2}{c}{\Rnode{B}{B}}& \\ [1.4cm]
\Rnode{A1}{A_1} &\Rnode{A2}{A_2}& ... & \Rnode{Am}{A_m}\\
\end{array}
$
\psset{labelsep=1pt,npos=0.6}
\ncbsar{x_1}{B}{A1}
\ncasar{x_2}{B}{A2}
\ncasar{x_m}{B}{Am}
\end{center}
and $x_1,...x_m$ is a basis for B 
then

\begin{center}
\setlength{\arraycolsep}{.1cm}
$
\begin{array}{cp{.35cm}cp{.4cm}c}
                  & \multicolumn{3}{c}{\Rnode{diamondB}{\underset{x_1=x_2}{\sigma}B}}& \\ [1.4cm]
\Rnode{A1}{A_1} & &\Rnode{A3}{A_3} &... & \Rnode{Am}{A_m}\\
\end{array}
$
\psset{labelsep=1pt,npos=0.6}
\ncbsar{e \circ a_1 = e \circ x_2}{diamondB}{A1}
\psset{labelsep=1pt,npos=0.8}
\ncasar{e \circ x_3}{diamondB}{A3}
\ncasar{e \circ x_m}{diamondB}{Am}
\end{center}
\setlength{\arraycolsep}{0.1cm}
\noindent is a basis for $\underset{x_1=x_2}{\sigma}B$. 

\noindent For any object $C$ and morphism $y: C \morph B$ such that
$y \circ x_1 = y \circ x_2$
the mediating morphism $h: C \morph \underset{x_1=x_2}{\sigma}B$ is a d-morphism iff 
the morphism $y$ is a d-morphism and the cone $\tuple{C,y}$ is tight i.e iff there does not exist an object $A'$ distinct from 
$A$ and morphisms $x'_1: B \dmorph A'$, $x'_2: B \dmorph A'$ and $x:A' \dmorph A$ such that each of the three diagrams contained here:

\begin{center}
$
\begin{array}{cp{0.5cm}cp{0.5cm}c}
               && \Rnode{X}{C}&&                  \\ [0.6cm]
\Rnode{B1}{B}&&                  && \Rnode{B2}{B} \\ [0.6cm]
               &&\Rnode{Ap}{A'}  &&               \\ [1cm]
               &&\Rnode{A}{A}    && 
\end{array}
$
\ncbsar{y}{X}{B1}
\ncasar{y}{X}{B2}
\ncbsar{x'_1}{B1}{Ap}
\ncasar{x'_2}{B2}{Ap}
\ncasar{x}{Ap}{A}
\ncbsar[-30]{x_1}{B1}{A}
\ncasar[30]{x_2}{B2}{A}
\end{center}

\noindent commute. \\


\noindent This completes the definition.

\end{enumerate}

\section{Construction of Limits of Other Dependency Diagrams}

\begin{lemma}
For all diagrams D of d-morphisms with shape \ndidly:
	
\begin{center}
\begin{displaymath}
\begin{array}{cp{0.3cm}c         p{0.3cm}  c p{0.2cm} c}
\Rnode{d1}{B}& &               & &     & &\Rnode{d2}{B'}  \\ [1.4cm]
\Rnode{c1}{A_1}& &\Rnode{c2}{A_2}& & ... & &\Rnode{cn}{A_n}
\end{array}
\end{displaymath}
\ncsar{d1}{c1}
\nbput[npos=0.4, labelsep=1pt]{\footnotesize $x_1$}
\ncsar{d1}{c2}
\nbput[npos=0.3, labelsep=1pt]{\footnotesize $x_2$}
\ncsar{d1}{cn}
\naput[npos=0.3, labelsep=1pt]{\footnotesize $x_n$}
\ncsar{d2}{c1}
\nbput[npos=0.3, labelsep=1pt]{\footnotesize $x'_1$}
\ncsar{d2}{c2}
\naput[npos=0.25, labelsep=1pt]{\footnotesize $x'_2$}
\ncsar{d2}{cn}
\naput[npos=0.4, labelsep=1pt]{\footnotesize $x'_n$}
\end{center}

there is an object  $ \fibrex{B_1}{B_2}{D}$ of \cat{C} and d-morphisms 
$p_1: \fibrex{B_1}{B_2}{D} \dmorph B_1$ and  $p_1: \fibrex{B_1}{B_2}{D} \dmorph B_1$ such that :

\begin{center}
\begin{displaymath}
\begin{array}{cp{0cm}cp{0cm} c}
            & &\Rnode{d}{\fibrex{B_1}{B_2}{D}}                  \\ [0.6cm]
\Rnode{c1}{B_1}& &                & & \Rnode{c2}{B_2} 
\end{array}
\end{displaymath}
\ncbsar{p_1}{d}{c1}
\ncasar{p_2}{d}{c2}
\end{center}

is a limit of diagram $D$ in category \cat{C} and such that for any cone $\tuple{N,\psi}$ to diagram $D$
the mediating morphism $h: N \morph B_1 \underset{D}{\Join} B_2$ is a d-morphism iff 
each morphism in the cone $\tuple{N,\psi}$ is a d-morphism and the 
cone is tight. 

\end{lemma}
\begin{proof}
\noindent This lemma can be proved by induction using the next lemma.
\end{proof}
\begin{lemma}
In any category \cat{C} if $F: S \morph \cat{C}$ is a diagram with limit $\tuple{L,\phi}$,
if $S'$ is a category extending $S$ by an object $\beta_0$ and a pair of morphisms $\xi_1: \beta_1 \morph \beta_0$ and $\xi_2: \beta_2 \morph \beta_0$, where $\beta_1$ and $\beta_2$ are objects of $S$, if $G: S \morph \cat{C}$ is a diagram that extends $F$, then if :
\begin{center}
\begin{equation}
\label{equaliser}
\begin{array}{cp{1cm}cp{1cm} c}
\Rnode{E}{E}& &    \Rnode{L}{L}  & & \Rnode{A}{\beta_0} 
\end{array}
\end{equation}
\ncaarr{e}{E}{L}
\setlength{\arroffsetA}{3pt}
\setlength{\arroffsetB}{3pt}
\ncarr{L}{A}
\naput{$\phi_B \circ G(\xi_1)$}
\setlength{\arroffsetA}{-3pt}
\setlength{\arroffsetB}{-3pt}
\ncarr{L}{A}
\nbput{$\phi_{B'} \circ G(\xi_2)$}
\arreset
\end{center}
is an equaliser in \cat{C} then $\tuple{E,\phi'}$ is a limit of the diagram $G$, where
$\phi'$ is the cone defined by
\begin{equation*}
\phi'_\beta =
\left\{
	\begin{array}{ll}
		e \circ \phi_\beta           & \mbox{if } \beta \mbox{ is an object of } S  \\
		\phi_{\beta_1} \circ G(\xi_1) & \mbox{if } \beta \mbox{ is } \beta_0 
	\end{array}
\right.
\end{equation*}

\end{lemma}
\begin{proof}
If $\tuple{N,\psi'}$is a cone to the diagram $G$ then, the restriction $\psi$ to objects in $F$ is a cone to  $S$. Therefore there exists a unique $h : N \morph L$ such that 
\begin{equation}
\label{hmediates}
h \circ \phi = \psi
\end{equation}

\noindent
Now, we have 
\begin{align*}
h \circ \phi_{\beta_1} \circ G(x)
             & = \psi_{\beta_1} \circ G(\xi_1)         && \mbox{by (\ref{hmediates})}\\
             & = \psi_{\beta_2} \circ G(\xi_2)         && \mbox{because } \psi \mbox{ is a cone to } G \\
             & = h \circ \phi_{\beta_2} \circ G(\xi_2) && \mbox{by (\ref{hmediates}) }\\
\end{align*}

\noindent
and so it follows, since (\ref{equaliser}) is an equaliser diagram,
that there is a unique $g: N \morph E$ such that 
\begin{equation}
\label{gmediates}
g \circ e = h
\end{equation}

\noindent
To show that $\tuple{E,\phi'}$ is a limit to the diagram G we show that 
$g \circ \phi' = \psi'$ and that $g$ is the unique such morphism,
$g: N \morph E$.
\noindent
We need show that for all objects $\beta$ of $S'$,  $(g \circ \phi')_\beta =\psi'_\beta$. 
We need consider two cases. In the first case for objects $\beta$ of $S$ we have:
\begin{align*}
g \circ \phi'_\beta
             & = (g \circ e) \circ \phi_\beta   &&  \mbox{ by definition of } \phi\\
             & = h \circ \phi_\beta             && \mbox{ by (\ref{gmediates})}   \\
             & = \psi_\beta                     && \mbox{ by (\ref{hmediates})}   \\
						 & = \psi'_\beta                    && \mbox{ by definition of } \psi\\
\end{align*}

\noindent
In the second case the object $\beta$ is the additional object $\beta_0$ of $S'$. For this object we have:
\begin{align*}
g \circ \phi'_{\beta_0}
   & = g \circ e \circ \phi_{\beta_1} \circ G(\xi_1)  && \mbox{by definition of }\phi' \\ 
             & = h \circ \phi_{\beta_1} \circ G(\xi_1) && \mbox{ by (\ref{gmediates})} \\
             & = \psi_{\beta_1}  \circ G(\xi_1)       && \mbox{ by (\ref{hmediates})} \\
						 & = \psi'_{\beta_1} \circ G(\xi_1)       && \mbox{ by definition of} \psi \\
						& = \psi'_{\beta_0}                        && \mbox{since }\psi' \mbox{ is a cone} \\ 
\end{align*}
and so, as required, we have shown that 
$g \circ \phi' = \psi'$.
\noindent Finally, if $g':N \morph E$ such that $g' \circ \phi' = \psi'$ then we have 
for any object $\beta$ of S that
$$(g'\circ\phi')_\beta = \psi'_\beta$$
\noindent
and therefore from the defintions of $\phi'$ and $\psi$ we have
that, for all objects $\beta$ of S,
$$
g' \circ e \circ \phi_\beta = \psi_\beta
$$
\noindent
and  from the definition of $h$ as the unique morphism such that
$h \circ \phi = \psi$ we have that 
$g' \circ e = h$.\\

\noindent
Now we have:
 
$$g' \circ e = h = g \circ e$$
from which it follows $g=g'$ because $e$ is an equaliser and therefore is a monomorphism. 
\end{proof}

\bibliography{../SharedBibliography/temp/bibliography}

\end{document}
