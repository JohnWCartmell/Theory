\subsection {Trees of Concepts and the GAT of Trees}



\note 
Terminology: By  the generic term \term{tree} is meant a partially ordered set (poset) $(T, <)$ such that for each $t \in T$, the set $\set{s \in T : s < t}$ is well-ordered by the relation $<$.
In this discussion we restrict ourselves to \highlight{$\omega$-trees} i.e. trees for which the set $\set{s \in T : s < t}$
is finite for all $t \in T$. If the set $\set{s \in T : s < t}$ is empty then we say $t$ is in the base of the tree.

With respect to a partial ordering $<$, we say that \highlight{an element $y$ \textit{covers}  an element $x$} in  iff $x<y$ and there does not exist $w$ such that $x < w$ and $w < y$.
If object $y$ covers object $x$ in the partial ordering 
then we write \highlight{$x \base y$} (we use this in preference to the more usual $x \lessdot y$).
For $x$ an element of the tree we define the set of elements  \highlight{$Cover(x)$} to be the set of objects covering $x$.


Such trees as these we can equivalently describe as models of the generalised algebraic theory given below table \ref{GATOFTREES} in which the nodes of height $n+1$ are represented as of a sort $S_{n+1}$that is dependent on the sort of nodes of height $n$.

%\newcommand{\Ft}[1]{#1 \kern -0.4em \downarrow}
\newcommand{\Ft}[1]{\downarrow \kern -0.325em #1}
If A and B are nodes of  a tree $(S,<)$ then we shall write $A \base B$ to mean that $A < B$ in S and that
there does not exist x such that $A < x < B$. For every node B of tree S other than the root node there exists a unique node A such that $A \base B$.

\newcommand{\Sz}{Base}
\newcommand{\ofS}[1]{\ofT{#1}{\Sz}}
\newcommand{\Si}[1]{C\kern-1pt over_{#1}}
\newcommand{\ofSi}[3]{\ofT{#1}{\Si{#2}(#3)}}
\vspace{0.03cm} 
\begin{table}[H]
\caption{The Generalised Algebraic Theory of $\omega$-Trees}
\label{GATOFTREES}
\begin{tabular}{>{\itshape}l l}
Symbol & \itshape{Introductory Rule} \\
$\Sz  $&$\isT{\Sz}$\\
$\Si{1} $&$\ofS{x_0} \tstyle \isT{\Si{1}(x_0)} $\\
$\Si{2} $&$\ofS{x_0},\ofSi{x_1}{1}{x_0} \tstyle \isT{\Si{2}(x_0,x_1)} $\\
\vdots  \\
$\Si{n} $&$\ofS{x_0},\ofSi{x_1}{1}{x_0}, \hdots \ofSi{x_{n-1}}{n-1}{x_0,x_1,\hdots x_{n-2}} \tstyle \isT{\Si{n}(x_0,x_1,\hdots x_{n-1})} $\\
\vdots   \\
\end{tabular} \\
\end{table} 


\subsection {Schematic Notation}
%\newcommand{\Ft}[1]{
%#1 \kern-6pt \raisebox{1.45ex}{$\leftrightline$} \kern-3pt \raisebox{.09ex}{$\downarrow$}\kern-3.4pt \raisebox{.25ex} {$|$}}
\newcommand{\ft}[1]{
#1 \kern-6pt \raisebox{1.1ex}{$\leftrightline$} \kern-3pt \raisebox{.1ex}{$\downarrow$}}
%\newcommand{\Bbar}[1]{
%#1 \kern-6pt \raisebox{1.45ex}{$\leftrightline$}
%\overline{#1}}
%\vv{#1}}
%\newcommand{\bbar}[1]{
%#1 \kern-6pt \raisebox{1.0ex}{$\leftrightline$}
%\overline{#1}}
%\vv{#1}}
\newcommand{\bbin}[1]{
\raisebox{-0.5em}{$\stackrel{\displaystyle{\in}} {\scriptstyle{#1}}$}
}
\newcommand{\ofTn}[3]{
#1 \bbin{#2} #3}


\newcommand{\genericOb}{Ob} % where we have genericOb=Base + Cover



There is a  shorthand that is convenient in the presentation  of the GAT of trees  and then subsequently in the GAT of contextual categories. We use the shorthand
$\ofTn{x}{n}{\genericOb}$ for the context $\ofS{x_0},\ofSi{x_1}{1}{x_0}, \hdots \ofSi{x_n}{n}{x_0,x_1,\hdots x_{n-1}} $. \\

\noindent Using this shorthand, for any $n \geq 0$ the sort $Cover_{n}$  in the theory of trees is introduced as follows: \\

\vspace{0.03cm} 
\begin{tabular}{>{\itshape}l l}
Symbol & \itshape{Introductory Rule} \\
$\Sz  $     & $\isT{\Sz}$\\
$\Si{n+1}, n \geq 0 $ & $\ofTn{x}{n}{\genericOb}    \tstyle \isT{\Si{n+1}(x)} $\\
\end{tabular} \\
\vspace{.1cm}  \\


\subsection {An Aside on Recursive Type Definitions}

Using the shorthand, we are quite close to having a recursive definition of a single sort $Cover$.
Such definitions are not possible in generalised algebraic theories but we can imagine a framework in which it is possible to write: \\
\vspace{0.03cm} 
\begin{tabular}{>{\itshape}l l}
Symbol & \itshape{Introductory Rule} \\
$Base $     & $\isT{Base}$\\
$Cover  $     & $\ofT{x_0}{Base}    \tstyle \isT{Cover(x_0)} $\\
$Cover $      & $\ofT{x}{Cover}    \tstyle \isT{Cover(x)} $\\
\end{tabular} \\
\vspace{.1cm}  \\

Such a definition could be represented algebraically in a suitably generalised notion of contextual category (a comulti-contextual category?) these dependencies could be represented
as follows:  

\begin{center}
$
\begin{array}{c c}
\Rnode{abs}{1}  \\ [1.4cm]
\Rnode{S0}{Base} \\ [1.4cm]
\Rnode{SR}{Cover} \\ [1.4cm]
\end{array}
$
\ncsar{S0}{abs}
\ncsar{SR}{S0}
\ncrsar{SR}{SR}
\end{center}

\noindent This is not just an idle thought -- in  data modelling such a tree 
structure is represented in an entity model diagram in which the injections into the coproduct $Ob$ of $Base$ and $Cover$ are represented by containment: \\

\begin{center}
\begin{erdiagram}{3.4499999999999997}{4.4666}

\eret{0.2}{-2.85}{3.867}{-1.4}{0.2}{1}\ertext{0.316}{-1.75}{l}{$Ob$}
\eret{0.45}{-2.6}{1.783}{-2}{0.2}{0}\ertext{1.117}{-2.35}{}{$Base$}
\eret{2.283}{-2.6}{3.617}{-2}{0.2}{0}\ertext{2.95}{-2.35}{}{$Cover$}
\eret{0}{-0.2}{4.467}{0.3}{0.2}{1}

% relationship 
\ertext{1.217}{-0.5}{l}{}\errelarm{1.117}{-0.2}{1.117}{-1.1}{1}{0}\errelarm{1.117}{-1.1}{1.117}{-2}{1}{0}
% relationship 
\ertext{2.133}{-3.15}{l}{}\errelarm{2.033}{-2.85}{2.033}{-3.1}{0}{0}\errelarm{2.95}{-1.5}{2.95}{-2}{1}{0}\errelangle{2.033}{-3.1}{2.033}{-3.35}{3.133}{-3.35}{0}{0}\errelangle{2.95}{-1.5}{2.95}{-1}{3.592}{-1}{1}{0}\errelangle{3.133}{-3.35}{4.233}{-3.35}{4.233}{-2.175}{0}{0}\errelangle{4.233}{-2.175}{4.233}{-1}{3.592}{-1}{1}{0}\ercrowfoot{2.95}{-1.85}{2.8}{-2}{2.95}{-2}{3.1}{-2}{0}
\end{erdiagram}

\end {center}
See \textit{www.entitymodelling.org/tutorialone} for a description of this notation.
For an example of the modelling of recursive relationships in the definition of a phrase structure grammar of English see 
\textit{www.entitymodelling.org/examplesone/englishsentence}.