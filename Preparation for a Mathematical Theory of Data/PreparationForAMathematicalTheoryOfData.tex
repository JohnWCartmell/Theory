 \documentclass[10pt,a4paper]{article}
\newcommand{\sharedmacros}{../../SharedMacros}

%ccategories.macros.tex 

% Macros for diagrams in contextual categories and related categories

\usepackage{twoopt}
\usepackage{scalerel} 
\usepackage{xargs}

%\usepackage{mathabx}  %Caused font problems
%\usepackage{MnSymbol}  % caused font problems

\newcommand{\conu}
{\mathbf{C}(U)}

\newcommand{\depu}
{\mathbf{D}(U)}


\newcommand{\reqt}{\textbf{R}}
\newcommand{\reqtc}[1][\catc]{\reqt_{#1}}
\newcommand{\reqtcp}[1][\catcp]{\reqt_{#1}}



\newcommand{\cat}[1]{\textbf{#1}}

\newcommand{\catc}{\cat{C}}
\newcommand{\catcw}{\cat{C}\ }
\newcommand{\catcp}[1][C]{\textbf{#1}'}
\newcommand{\catcpp}[1][C]{\textbf{#1}''}
\newcommand{\obj}[1]{\ensuremath{|\cat{#1}|}}
\newcommand{\ccat}[1][C]{\ensuremath{\mathbb{#1}} }
\newcommand{\ccatc}{contextual category \ccat}
\newcommand{\cobj}[2][]{\ensuremath{|\ccat[#2]|_{#1}}}
\newcommand{\cslice}[2]{\ensuremath{\ccat[#1]_{#2}}}
\newcommand{\csliceobj}[3][]{\ensuremath{|\mathbb{#2}_{#3}|_{#1} }}
\newcommand{\varset}[1][]{\ensuremath{V_{#1} }}
\newcommand{\localvarsets}{\ensuremath{\mathcal{V} }}
\newcommand{\Fam}{\ensuremath{\mathbb{F\mathrm{am}} }}
\newcommand{\Fin}{\ensuremath{\textbf{Fin}} }
\newcommand{\Finp}{\ensuremath{\textbf{Finp}} }
\newcommand{\Po}{\ensuremath{\textbf{Po}} }
\newcommand{\Famslice}[1]{\ensuremath{\mathbb{F\mathrm{am}}_{#1} }}
\newcommand{\Famobj}[1][]{\ensuremath{|\mathbb{F\mathrm{am}}|_{#1} }}
\newcommand{\Famsliceobj}[2][]{\ensuremath{|\mathbb{F\mathrm{am}}_{#2}|_{#1} }}
\newcommand{\morph}{\rightarrow}
\newcommand{\epi}{\twoheadrightarrow}
\newcommand{\base}{\triangleleft}
\newcommand{\comp}{\circ}
\newcommand{\cross}{\otimes}
\newcommand{\pc}[2]{d^{#1}_{#2}}
\newcommand{\sub}{^*}
\newcommand{\diag}{\delta}
\newcommand{\pbase}[1]{\tilde{#1}}
\newcommand{\tuple}[1]{\langle#1\rangle}
\newcommand{\ndidly}{\ensuremath{\Join_n}}

\newcommand{\product}[1]{\bigtimes_{#1}}
\newcommand{\productn}{\product{n}}
\newcommand{\crossx}[3]{#1 \underset{#3}{\cross} #2}
\newcommand{\fibrex}[3]{#1 \underset{#3}{\Join} #2}
\newcommand{\powerset}{\mathcal{P}}
\newcommand{\primeds}[1]{
\ensuremath{\mathcal{P}(#1)} }
\newcommand{\compset}{\ \dot{\circ}\, }

% darrow
%\newcommand{\darrow}{\rightarrowtriangle} %use \smorph instead
\newcommand{\smorph}{\rightarrowtriangle}

 
\newcommand\dhead{\scaleobj{0.6}{\triangleright}}
%\newcommand{\dmorph}{\, \mbox{---} \! \cdot \! \raisebox{1.1pt}{\dhead}}    % dot style
\newcommand{\dmorph}{\, \mbox{---}\kern-1pt\raisebox{1.1pt}{\dhead\kern-1.75pt\dhead}}\,     % double triangle style

% projection tree
%\newcommand{\proj}[2]{proj_{#2}(#1)}

\newcommand{\proj}[2]{
\ensuremath{\mathcal{P}_{#2}(#1)} }

%pstrick supplements for arrows

\newlength{\arrnodesepA}
\newlength{\arrnodesepB}
\newlength{\arroffsetA}
\newlength{\arroffsetB}

%Modified to 2pt from 0pt on 23 July 2018
\newcommand{\arreset}{
\setlength{\arrnodesepA}{2pt}
\setlength{\arrnodesepB}{2pt}
\setlength{\arroffsetA}{0pt}
\setlength{\arroffsetB}{0pt}
}
\arreset

\newcommand{\ncarr}[3][0]{\ncarc[arcangle=#1,nodesepA=\arrnodesepA,nodesepB=\arrnodesepB,offsetA=\arroffsetA,offsetB=\arroffsetB,arrowsize=5pt,arrowinset=0.7]{->}{#2}{#3}}
\newcommand{\ncdarr}[3][0]{\ncarc[linestyle=dashed,arcangle=#1,nodesepA=\arrnodesepA,nodesepB=\arrnodesepB,offsetA=\arroffsetA,offsetB=\arroffsetB,arrowsize=5pt,arrowinset=0.7]{->}{#2}{#3}}
\newcommand{\jcbarr}[4][0]{ % ncbarr is defined in some thridy party package so do not use!\emph{}
\ncarr[#1]{#3}{#4}
\nbput[labelsep=2pt]{\footnotesize $#2$}
}

\newcommand{\ncaarr}[4][0]{
\ncarr[#1]{#3}{#4}
\naput[labelsep=2pt]{\footnotesize $#2$}
}

% \alabel{label}[npos][labelsep_pts]
\newcommandx*\alabel[3][2=0.5,3=2,usedefault]{\naput[labelsep=#3pt,npos=#2]{\footnotesize $#1$}}
% \blabel{label}[npos][labelsep_pts]
\newcommandx*\blabel[3][2=0.5,3=2,usedefault]{\nbput[labelsep=#3pt,npos=#2]{\footnotesize $#1$}}


\newif \ifbars
% to supress display of bars use \barsfalse to swith them on use \barstrue
\barstrue 
% \idcomp mark an arrow as one component of an identifier
\newcommand{\idcomp}{\ifbars{\ncput[npos=0, nrot=:U]{\psline(0.2,-0.075)(0.2,0.075)}}\fi}  %add a bar to a node connection arrow
% pstrick supplements for s-arrows (previous name for d-arrow - should convert}

\newlength{\sarnodesepA}
\newlength{\sarnodesepB}
\newlength{\saroffsetA}
\newlength{\saroffsetB}
\newlength{\sarnodesepAsav}
\newlength{\sarnodesepBsav}

\newcommand{\sarreset}{
\setlength{\sarnodesepA}{0pt}
\setlength{\sarnodesepB}{0pt}
\setlength{\saroffsetA}{0pt}
\setlength{\saroffsetB}{0pt}
}

\sarreset

% sar - S-arrow
\newcommand{\ncsar}[3][0]{
\setlength{\sarnodesepAsav}{\sarnodesepA}
\setlength{\sarnodesepBsav}{\sarnodesepB}
\addtolength{\sarnodesepA}{3pt}
\addtolength{\sarnodesepB}{7pt}
\ncarc[nodesepA=\sarnodesepA,nodesepB=\sarnodesepB,offsetA=\saroffsetA,offsetB=\saroffsetB,arcangle=#1]{-}{#2}{#3}
\ncput[nrot=:R,npos=1]{\pstriangle(0,0)(.2,.2)}
\setlength{\sarnodesepA}{\sarnodesepAsav}
\setlength{\sarnodesepB}{\sarnodesepBsav}
}


% bsar - below labelled S-arrow
\newcommand{\ncbsar}[4][0]{
\ncsar[#1]{#3}{#4}
\nbput[labelsep=2pt]{\footnotesize $#2$}
}
% asar - above labelled S-arrow
\newcommand{\ncasar}[4][0]{
\ncsar[#1]{#3}{#4}
\naput[labelsep=2pt]{\footnotesize $#2$}
}

% OLD cdar - composite dependency arrow - dot tyle
\iffalse
\newcommand{\nccdar}[3][0]{
\setlength{\sarnodesepAsav}{\sarnodesepA}
\setlength{\sarnodesepBsav}{\sarnodesepB}
\addtolength{\sarnodesepA}{3pt}
\addtolength{\sarnodesepB}{11pt}
\ncarc[nodesepA=\sarnodesepA,nodesepB=\sarnodesepB,offsetA=\saroffsetA,offsetB=\saroffsetB,arcangle=#1]{-}{#2}{#3}
\ncput[nrot=:R,npos=1]{\pstriangle(0,0.1)(.2,.2)}
\ncput[nrot=:R,npos=1]{\psdot[dotsize=1pt](-0.0075,0.05)}   %!!
\setlength{\sarnodesepA}{\sarnodesepAsav}
\setlength{\sarnodesepB}{\sarnodesepBsav}
}
\fi

% cdar - composite dependency arrow Mark II - double trangle style
\newcommand{\nccdar}[3][0]{
\setlength{\sarnodesepAsav}{\sarnodesepA}
\setlength{\sarnodesepBsav}{\sarnodesepB}
\addtolength{\sarnodesepA}{3pt}
\addtolength{\sarnodesepB}{13pt}
\ncarc[nodesepA=\sarnodesepA,nodesepB=\sarnodesepB,offsetA=\saroffsetA,offsetB=\saroffsetB,arcangle=#1]{-}{#2}{#3}
\ncput[nrot=:R,npos=1]{\pstriangle(0,0)(.2,.2)}
\ncput[nrot=:R,npos=1]{\pstriangle(0,0.2)(.2,.2)}
\setlength{\sarnodesepA}{\sarnodesepAsav}
\setlength{\sarnodesepB}{\sarnodesepBsav}
}


% bcdar - below labelled composite dependency arrow
\newcommand{\ncbcdar}[4][0]{
\nccdar[#1]{#3}{#4}
\nbput[labelsep=2pt]{\footnotesize $#2$}
}
% acdar - above labelled composite dependency arrow
\newcommand{\ncacdar}[4][0]{
\nccdar[#1]{#3}{#4}
\naput[labelsep=2pt]{\footnotesize $#2$}
}


% rsar - recursive S-arrow
\newcommand{\ncrsar}[2]{
\setlength{\sarnodesepAsav}{\sarnodesepA}
\setlength{\sarnodesepBsav}{\sarnodesepB}
\addtolength{\sarnodesepA}{3pt}
\addtolength{\sarnodesepB}{7pt}
\ncloop[nodesepA=\sarnodesepA,nodesepB=\sarnodesepB,
        offsetA=\saroffsetA,offsetB=\saroffsetB,
        armA=0.7cm,armB=0.6cm,angleA=90,angleB=-90,loopsize=-1,linearc=0.4
				]{-}{#1}{#2}
\ncput[nrot=:R,npos=5]{\pstriangle(0,0)(.2,.2)}
\setlength{\sarnodesepA}{\sarnodesepAsav}
\setlength{\sarnodesepB}{\sarnodesepBsav}
}

% pstrick supplements for multi-arrows

\newlength{\marnodesepA}
\newlength{\marnodesepB}
\newlength{\maroffsetB}
\newlength{\marnodesepBsav}

\newcommand{\marreset}{
\setlength{\marnodesepA}{0pt}
\setlength{\marnodesepB}{0pt}
\setlength{\maroffsetB}{0pt}
}

\marreset

%ncmarr[#1 arcangle1][#2 arcangle2]{#3 name}{#4 domain1}{#5 domain2}{#6 junction}{#7 codomain}
\newcommandtwoopt{\ncmarr}[6][8][8]{%
\ncarc[nodesepA=\marnodesepA,nodesepB=0,arcangle=#1]{-}{#3}{#5}
\ncarc[nodesepB=0,arcangle=-#1]{-}{#4}{#5}
\ncarc[arcangle=#2,nodesepB=\marnodesepB,offsetB=\maroffsetB]{->}{#5}{#6}
}%


\newcommandtwoopt{\nchmarr}[6][8][8]{%
\ncarc[nodesepA=\marnodesepA,nodesepB=0,arcangle=#1]{-}{#3}{#5}
\ncarc[nodesepB=0,arcangle=#1]{-}{#4}{#5}
\ncarc[arcangle=#2,nodesepB=\marnodesepB,offsetB=\maroffsetB]{->}{#5}{#6}
}%

\newcommandtwoopt{\ncamarr}[7][8][8]{%
\ncmarr[#1][#2]{#4}{#5}{#6}{#7}
\naput[npos=.05]{$#3$}
}%
\newcommandtwoopt{\ncbmarr}[7][8][8]{%
\ncmarr[#1][#2]{#4}{#5}{#6}{#7}
\nbput[npos=.05]{$#3$}
}%

\newcommandtwoopt{\ncbhmarr}[7][8][8]{%
\nchmarr[#1][#2]{#4}{#5}{#6}{#7}
\nbput[npos=.05]{$#3$}
}%

\newcommandtwoopt{\ncmarrr}[7][8][8]{
\ncarc[nodesepB=0,arcangle=#1]{-}{#3}{#6}
\ncline[nodesepB=0]{-}{#4}{#6}
\ncarc[nodesepB=0,arcangle=-#1]{-}{#5}{#6}
\ncarc[nodesepA=0,arcangle=#2]{->}{#6}{#7}
}

\newcommandtwoopt{\ncamarrr}[8][8][8]{
\ncmarrr[#1][#2]{#4}{#5}{#6}{#7}{#8}
\naput[npos=.05]{$#3$}
}
\newcommandtwoopt{\ncbmarrr}[8][8][8]{
\ncmarrr[#1][#2]{#4}{#5}{#6}{#7}{#8}
\nbput[npos=.05]{$#3$}
}


% 6 June 2020
% Edges representing attributes and relationship graphs
%  Ep   - partial
%  Epm  - partial mono
%  Epe  - partial epi
%  Epme - partial mono epi
%  Et   - total
%  Etm  - total mono
%  Ete  - total epi
%  Etme - total mono epi
%  recursive edges (use nccircle)
%  rEp   - partial
%  rEpm  - partial mono
%  rEpe  - partial epi
%  rEpme - partial mono epi
%  rEt   - total
%  rEtm  - total mono
%  rEte  - total epi
%  rEtme - total mono epi

\newcounter{EangleA}
\newcounter{EangleB}
\newcounter{EmidangleA}
\newcounter{EmidangleB}

% Ep - Edge partial
\newcommandtwoopt{\Ep}[4][0][0]{
\crowsfootedEdge{#1}{#2}{#3}{#4}{dashed}{dashed}
}



% Epm - Edge partial mono
\newcommandtwoopt{\Epm}[4][0][0]{
\monoEdge{#1}{#2}{#3}{#4}{dashed}{dashed}
}


% Epe - Edge partial epi
\newcommandtwoopt{\Epe}[4][0][0]{
\crowsfootedEdge{#1}{#2}{#3}{#4}{dashed}{solid}
}

% Epme - Edge partial mono epi
\newcommandtwoopt{\Epme}[4][0][0]{
\monoEdge{#1}{#2}{#3}{#4}{dashed}{solid}
}

% Et - Edge total
\newcommandtwoopt{\Et}[4][0][0]{
\crowsfootedEdge{#1}{#2}{#3}{#4}{solid}{dashed}
}

% Etm - Edge total mono
\newcommandtwoopt{\Etm}[4][0][0]{
\monoEdge{#1}{#2}{#3}{#4}{solid}{dashed}
}

% Ete - Edge total epi
\newcommandtwoopt{\Ete}[4][0][0]{
\crowsfootedEdge{#1}{#2}{#3}{#4}{solid}{solid}
}

% Etme - Edge total mono epi
\newcommandtwoopt{\Etme}[4][0][0]{
\monoEdge{#1}{#2}{#3}{#4}{solid}{solid}
}

% crowsfootedEdge - \crowsfootedEdge[angleA][midpointangle]{startnode}{endnode}[startstyle][endstyle]
\newcommand{\crowsfootedEdge}[6]{
\setlength{\sarnodesepAsav}{\sarnodesepA}
\setlength{\sarnodesepBsav}{\sarnodesepB}
\addtolength{\sarnodesepA}{3pt}
\addtolength{\sarnodesepB}{3pt}
\setcounter{EangleA}{ #1 + #2}
\setcounter{EangleB}{180  - #1 + #2}
\setcounter{EmidangleA}{#2}
\setcounter{EmidangleB}{#2 + 180}
\nccurve[nodesepA=\sarnodesepA,nodesepB=\sarnodesepB,offsetA=\saroffsetA,offsetB=\saroffsetB,angleA=\theEangleA, angleB=\theEangleB,linestyle=none,linewidth=0]{->}{#3}{#4}
\ncput[nrot=:R,npos=0]{\psline(0,.1)(.075,0)}
\ncput[nrot=:R,npos=0]{\psline(0,.1)(-0.075,0)}
\ncput{\pnode(0,0){xxx}}
\nccurve[nodesepA=0,nodesepB=\sarnodesepB,offsetA=0,offsetB=\saroffsetB,angleA=\theEmidangleA, angleB=\theEangleB, linestyle=#6]{->}{xxx}{#4}
%the following provides context for any following label
\nccurve[nodesepA=\sarnodesepA,nodesepB=0,offsetA=\saroffsetA,offsetB=0,angleA=\theEangleA, angleB=\theEmidangleB,linestyle=#5]{-}{#3}{xxx}
\setlength{\sarnodesepA}{\sarnodesepAsav}
\setlength{\sarnodesepB}{\sarnodesepBsav}
}

% monoEdge - \monoEdge[angleA][midpointangle]{startnode}{endnode}[startstyle][endstyle]
\newcommand{\monoEdge}[6]{ 
\setlength{\sarnodesepAsav}{\sarnodesepA}
\setlength{\sarnodesepBsav}{\sarnodesepB}
\addtolength{\sarnodesepA}{3pt}
\addtolength{\sarnodesepB}{3pt}
\setcounter{EangleA}{ #1 + #2}
\setcounter{EangleB}{180  - #1 + #2}
\setcounter{EmidangleA}{#2}
\setcounter{EmidangleB}{#2 + 180}
\nccurve[nodesepA=\sarnodesepA,nodesepB=\sarnodesepB,offsetA=\saroffsetA,offsetB=\saroffsetB,angleA=\theEangleA, angleB=\theEangleB,linestyle=none,linewidth=0]{->}{#3}{#4}
\ncput{\pnode(0,0){xxx}}
\nccurve[nodesepA=0,nodesepB=\sarnodesepB,offsetA=0,offsetB=\saroffsetB,angleA=\theEmidangleA, angleB=\theEangleB, linestyle=#6]{->}{xxx}{#4}
%the following provides context for any following label
\nccurve[nodesepA=\sarnodesepA,nodesepB=0,offsetA=\saroffsetA,offsetB=0,angleA=\theEangleA, angleB=\theEmidangleB,linestyle=#5]{-}{#3}{xxx}
\setlength{\sarnodesepA}{\sarnodesepAsav}
\setlength{\sarnodesepB}{\sarnodesepBsav}
}


\newcounter{EangleGiven}
\newcounter{EangleComplementary}
\newcounter{EangleStartCorrected}
\newcounter{EangleEndCorrected}


%  rEp   - recursive Edge partial
\newcommand{\rEp}[2][0]{
\setcounter{EangleGiven}{#1}
\setcounter{EangleStartCorrected}{#1-10} %correction required because for nccurve unlike nccircle angle measured at boundary not at centre of node
\setcounter{EangleEndCorrected}{#1+180+10} %correction required because angle measured at boundary not at centre of node
\setcounter{EangleComplementary}{#1 + 180}
\nccircle[angleA=\theEangleComplementary, nodesep=0pt, linestyle=none]{-}{#2}{.4cm} % an invisible circle to hang the midpoint from
\ncput{\pnode(0,0){midpoint}}                                         
\nccurve[nodesepA=1pt,nodesepB=0pt,offsetA=0pt,offsetB=0pt,angleA=\theEangleStartCorrected, angleB=\theEangleGiven, ncurv=1.359, linecolor=black, linestyle=dashed]{-}{#2}{midpoint}
\ncput[nrot=:R,npos=0]{\psline(0,.1)(.075,0)}
\ncput[nrot=:R,npos=0]{\psline(0,.1)(-0.075,0)}
\nccurve[nodesepA=0pt,nodesepB=2pt,offsetA=0pt,offsetB=0pt,angleA=\theEangleComplementary, angleB=\theEangleEndCorrected, ncurv=1.359, linestyle=dashed]{-}{midpoint}{#2}
% 1.359 is e/2 happenchance or algorithmically necessary???
% now draw arrowhead -- dont include in the nccurve because this alters the line position - a strange feature of pstruicks
\ncput[npos=0.9]{\pnode(0,0){yyy}}
\ncline{->}{yyy}{#2}
% repeat from earlier to provide context for label that might follow
\nccurve[nodesepA=1pt,nodesepB=0pt,offsetA=0pt,offsetB=0pt,angleA=\theEangleStartCorrected, angleB=\theEangleGiven, ncurv=1.359, linecolor=black, linestyle=dashed]{-}{#2}{midpoint} 
} 

%  rEpm  - recursive Edge partial mono
\newcommand{\rEpm}[2][0]{
\setcounter{EangleGiven}{#1}
\setcounter{EangleStartCorrected}{#1-10} %correction required because for nccurve unlike nccircle angle measured at boundary not at centre of node
\setcounter{EangleEndCorrected}{#1+180+10} %correction required because angle measured at boundary not at centre of node
\setcounter{EangleComplementary}{#1 + 180}
\nccircle[angleA=\theEangleComplementary, nodesep=0pt, linestyle=none]{-}{#2}{.4cm} % an invisible circle to hang the midpoint from
\ncput{\pnode(0,0){midpoint}}   
\nccurve[nodesepA=0pt,nodesepB=2pt,offsetA=0pt,offsetB=0pt,angleA=\theEangleComplementary, angleB=\theEangleEndCorrected, ncurv=1.359, linestyle=dashed]{-}{midpoint}{#2}
% 1.359 is e/2 happenchance or algorithmically necessary???
% now draw arrowhead -- dont include in the nccurve because this alters the line position - a strange feature of pstruicks
\ncput[npos=0.9]{\pnode(0,0){yyy}}
\ncline{->}{yyy}{#2}
% last to provide context for label that might follow
\nccurve[nodesepA=1pt,nodesepB=0pt,offsetA=0pt,offsetB=0pt,angleA=\theEangleStartCorrected, angleB=\theEangleGiven, ncurv=1.359, linecolor=black, linestyle=dashed]{-}{#2}{midpoint} 
}

%  rEpe  - recursive Edge partial epi
\newcommand{\rEpe}[2][0]{
\setcounter{EangleGiven}{#1}
\setcounter{EangleStartCorrected}{#1-10} %correction required because for nccurve unlike nccircle angle measured at boundary not at centre of node
\setcounter{EangleEndCorrected}{#1+180+10} %correction required because angle measured at boundary not at centre of node
\setcounter{EangleComplementary}{#1 + 180}
\nccircle[angleA=\theEangleComplementary, nodesep=0pt, linestyle=none]{-}{#2}{.4cm} % an invisible circle to hang the midpoint from
\ncput{\pnode(0,0){midpoint}}                                         
\nccurve[nodesepA=1pt,nodesepB=0pt,offsetA=0pt,offsetB=0pt,angleA=\theEangleStartCorrected, angleB=\theEangleGiven, ncurv=1.359, linecolor=black, linestyle=dashed]{-}{#2}{midpoint}
\ncput[nrot=:R,npos=0]{\psline(0,.1)(.075,0)}
\ncput[nrot=:R,npos=0]{\psline(0,.1)(-0.075,0)}
\nccurve[nodesepA=0pt,nodesepB=2pt,offsetA=0pt,offsetB=0pt,angleA=\theEangleComplementary, angleB=\theEangleEndCorrected, ncurv=1.359]{-}{midpoint}{#2}
% 1.359 is e/2 happenchance or algorithmically necessary???
% now draw arrowhead -- dont include in the nccurve because this alters the line position - a strange feature of pstruicks
\ncput[npos=0.9]{\pnode(0,0){yyy}}
\ncline{->}{yyy}{#2}
% repeat from earlier to provide context for label that might follow
\nccurve[nodesepA=1pt,nodesepB=0pt,offsetA=0pt,offsetB=0pt,angleA=\theEangleStartCorrected, angleB=\theEangleGiven, ncurv=1.359, linecolor=black, linestyle=dashed]{-}{#2}{midpoint} 
}

%  rEpme - recursive Edge partial mono epi
\newcommand{\rEpme}[2][0]{
\setcounter{EangleGiven}{#1}
\setcounter{EangleStartCorrected}{#1-10} %correction required because for nccurve unlike nccircle angle measured at boundary not at centre of node
\setcounter{EangleEndCorrected}{#1+180+10} %correction required because angle measured at boundary not at centre of node
\setcounter{EangleComplementary}{#1 + 180}
\nccircle[angleA=\theEangleComplementary, nodesep=0pt, linestyle=none]{-}{#2}{.4cm} % an invisible circle to hang the midpoint from
\ncput{\pnode(0,0){midpoint}}                                         
%\nccurve[nodesepA=0pt,nodesepB=0pt,offsetA=0pt,offsetB=0pt,angleA=\theEangleComplementary, angleB=\theEangleEndCorrected, ncurv=1.359, linestyle=dashed]{->}{xxx}{#2}
\nccurve[nodesepA=0pt,nodesepB=2pt,offsetA=0pt,offsetB=0pt,angleA=\theEangleComplementary, angleB=\theEangleEndCorrected, ncurv=1.359]{-}{midpoint}{#2}
% 1.359 is e/2 happenchance or algorithmically necessary???
% now draw arrowhead -- dont include in the nccurve because this alters the line position - a strange feature of pstruicks
\ncput[npos=0.9]{\pnode(0,0){yyy}}
\ncline{->}{yyy}{#2}
% last so that to provide context for label that might follow
\nccurve[nodesepA=1pt,nodesepB=0pt,offsetA=0pt,offsetB=0pt,angleA=\theEangleStartCorrected, angleB=\theEangleGiven, ncurv=1.359, linecolor=black, linestyle=dashed]{-}{#2}{midpoint} 
}

% rEt - recursive Edge total
\newcommand{\rEt}[2][0]{
\setcounter{EangleGiven}{#1}
\setcounter{EangleStartCorrected}{#1-10} %correction required because for nccurve unlike nccircle angle measured at boundary not at centre of node
\setcounter{EangleEndCorrected}{#1+180+10} %correction required because angle measured at boundary not at centre of node
\setcounter{EangleComplementary}{#1 + 180}
\nccircle[angleA=\theEangleComplementary, nodesep=0pt, linestyle=none]{-}{#2}{.4cm} % an invisible circle to hang the midpoint from
\ncput{\pnode(0,0){midpoint}}                                         
\nccurve[nodesepA=1pt,nodesepB=0pt,offsetA=0pt,offsetB=0pt,angleA=\theEangleStartCorrected, angleB=\theEangleGiven, ncurv=1.359, linecolor=black]{-}{#2}{midpoint}
\ncput[nrot=:R,npos=0]{\psline(0,.1)(.075,0)}
\ncput[nrot=:R,npos=0]{\psline(0,.1)(-0.075,0)}
%\nccurve[nodesepA=0pt,nodesepB=0pt,offsetA=0pt,offsetB=0pt,angleA=\theEangleComplementary, angleB=\theEangleEndCorrected, ncurv=1.359, linestyle=dashed]{->}{xxx}{#2}
\nccurve[nodesepA=0pt,nodesepB=2pt,offsetA=0pt,offsetB=0pt,angleA=\theEangleComplementary, angleB=\theEangleEndCorrected, ncurv=1.359, linestyle=dashed]{-}{midpoint}{#2}
% 1.359 is e/2 happenchance or algorithmically necessary???
% now draw arrowhead -- dont include in the nccurve because this alters the line position - a strange feature of pstruicks
\ncput[npos=0.9]{\pnode(0,0){yyy}}
\ncline{->}{yyy}{#2}
% repeat from earlier to provide context for label that might follow
\nccurve[nodesepA=1pt,nodesepB=0pt,offsetA=0pt,offsetB=0pt,angleA=\theEangleStartCorrected, angleB=\theEangleGiven, ncurv=1.359, linecolor=black]{-}{#2}{midpoint} 
}

%  rEtm  - recursive Edge total mono
\newcommand{\rEtm}[2][0]{
\setcounter{EangleGiven}{#1}
\setcounter{EangleStartCorrected}{#1-10} %correction required because for nccurve unlike nccircle angle measured at boundary not at centre of node
\setcounter{EangleEndCorrected}{#1+180+10} %correction required because angle measured at boundary not at centre of node
\setcounter{EangleComplementary}{#1 + 180}
\nccircle[angleA=\theEangleComplementary, nodesep=0pt, linestyle=none]{-}{#2}{.4cm} % an invisible circle to hang the midpoint from
\ncput{\pnode(0,0){midpoint}}     
\nccurve[nodesepA=0pt,nodesepB=2pt,offsetA=0pt,offsetB=0pt,angleA=\theEangleComplementary, angleB=\theEangleEndCorrected, ncurv=1.359, linestyle=dashed]{-}{midpoint}{#2}
% 1.359 is e/2 happenchance or algorithmically necessary???
% now draw arrowhead -- dont include in the nccurve because this alters the line position - a strange feature of pstruicks
\ncput[npos=0.9]{\pnode(0,0){yyy}}
\ncline{->}{yyy}{#2}
% last to provide context for label that might follow
\nccurve[nodesepA=1pt,nodesepB=0pt,offsetA=0pt,offsetB=0pt,angleA=\theEangleStartCorrected, angleB=\theEangleGiven, ncurv=1.359, linecolor=black]{-}{#2}{midpoint} 
}

%  rEte  - total epi
\newcommand{\rEte}[2][0]{
\setcounter{EangleGiven}{#1}
\setcounter{EangleStartCorrected}{#1-10} %correction required because for nccurve unlike nccircle angle measured at boundary not at centre of node
\setcounter{EangleEndCorrected}{#1+180+10} %correction required because angle measured at boundary not at centre of node
\setcounter{EangleComplementary}{#1 + 180}
\nccircle[angleA=\theEangleComplementary, nodesep=0pt, linestyle=none]{-}{#2}{.4cm} % an invisible circle to hang the midpoint from
\ncput{\pnode(0,0){midpoint}}                                         
\nccurve[nodesepA=1pt,nodesepB=0pt,offsetA=0pt,offsetB=0pt,angleA=\theEangleStartCorrected, angleB=\theEangleGiven, ncurv=1.359, linecolor=black]{-}{#2}{midpoint}
\ncput[nrot=:R,npos=0]{\psline(0,.1)(.075,0)}
\ncput[nrot=:R,npos=0]{\psline(0,.1)(-0.075,0)}
%\nccurve[nodesepA=0pt,nodesepB=0pt,offsetA=0pt,offsetB=0pt,angleA=\theEangleComplementary, angleB=\theEangleEndCorrected, ncurv=1.359, linestyle=dashed]{->}{xxx}{#2}
\nccurve[nodesepA=0pt,nodesepB=2pt,offsetA=0pt,offsetB=0pt,angleA=\theEangleComplementary, angleB=\theEangleEndCorrected, ncurv=1.359]{-}{midpoint}{#2}
% 1.359 is e/2 happenchance or algorithmically necessary???
% now draw arrowhead -- dont include in the nccurve because this alters the line position - a strange feature of pstruicks
\ncput[npos=0.9]{\pnode(0,0){yyy}}
\ncline{->}{yyy}{#2}
% repeat from earlier to provide context for label that might follow
\nccurve[nodesepA=1pt,nodesepB=0pt,offsetA=0pt,offsetB=0pt,angleA=\theEangleStartCorrected, angleB=\theEangleGiven, ncurv=1.359, linecolor=black]{-}{#2}{midpoint} 
}

%  rEtme - recursive Edge total mono epi

\newcommand{\rEtme}[2][0]{
\setcounter{EangleGiven}{#1}
\setcounter{EangleStartCorrected}{#1-10} %correction required because for nccurve unlike nccircle angle measured at boundary not at centre of node
\setcounter{EangleEndCorrected}{#1+180+10} %correction required because angle measured at boundary not at centre of node
\setcounter{EangleComplementary}{#1 + 180}
\nccircle[angleA=\theEangleComplementary, nodesep=0pt, linestyle=none]{-}{#2}{.4cm} % an invisible circle to hang the midpoint from
\ncput{\pnode(0,0){midpoint}}     
\nccurve[nodesepA=0pt,nodesepB=2pt,offsetA=0pt,offsetB=0pt,angleA=\theEangleComplementary, angleB=\theEangleEndCorrected, ncurv=1.359]{-}{midpoint}{#2}
% 1.359 is e/2 happenchance or algorithmically necessary???
% now draw arrowhead -- dont include in the nccurve because this alters the line position - a strange feature of pstruicks
\ncput[npos=0.9]{\pnode(0,0){yyy}}
\ncline{->}{yyy}{#2}
% last to provide context for label that might follow
\nccurve[nodesepA=1pt,nodesepB=0pt,offsetA=0pt,offsetB=0pt,angleA=\theEangleStartCorrected, angleB=\theEangleGiven, ncurv=1.359, linecolor=black]{-}{#2}{midpoint} 
}

%The following are stylistic so belong in main document not here.
%\usepackage[margin=4.0cm]{geometry} % This shouldn't be here commented out 17 July 2018
%\usepackage{mathptmx}               % This changes font to roman so doesn't belong here
%
\usepackage{amsfonts}
\usepackage{amssymb} % added 08\02\2019 as an experiment. Needed in some instances for \blacksquare
                     % not needed is class is `beamer' but I don't know why not
\usepackage{array}
\usepackage{pstricks}
\usepackage{pst-tree}
\usepackage{pst-plot}
\usepackage{pst-node}
\usepackage{stmaryrd}
\usepackage{amsmath}
\usepackage{verbatim}
\usepackage{graphicx}  
\usepackage{calc}
\usepackage{xifthen}
%\usepackage{xcolor} investigate with beamer
\usepackage{color}
\usepackage{stringstrings}
%\usepackage[small,bf,margin=3pt,format=hang, labelsep=endash,singlelinecheck=false]{caption} %prevuiously justification=justified
%\usepackage{enumerate}
%\usepackage{enumitem}
\usepackage{enumerate}
%\usepackage[shortlabels]{enumitem} %Removed this 28/01/2019 because interfereing with a beamer presentation. 
\usepackage{float}
\usepackage[section]{placeins}
%\setlength{\captionmargin}{5pt}
\usepackage{environ}
\usepackage{multirow}
\usepackage{rotating}
\usepackage{longtable}
\usepackage{afterpage}
\usepackage{needspace}


%DEFINE ENVIRONMENT BLOCK
% Riddle
\newsavebox{\riddlebox}

\newenvironment{erexample}
{\newcommand\colboxcolor{F0F0F0}%was F8F8F8
\begin{lrbox}{\riddlebox}
\begin{minipage}{\dimexpr\columnwidth-2\fboxsep\relax} \textbf{} \\ \itshape}
{\end{minipage}\end{lrbox}%
%\begin{center}
\colorbox[HTML]{\colboxcolor}{\usebox{\riddlebox}}
%\end{center}
}

\newenvironment{erbox}
{\newcommand\colboxcolor{F0F0F0}%was F8F8F8
\begin{lrbox}{\riddlebox}%
\begin{minipage}{\dimexpr\columnwidth-2\fboxsep\relax} }
{\end{minipage}\end{lrbox}%
%\begin{center}
\colorbox[HTML]{\colboxcolor}{\usebox{\riddlebox}}
%\end{center}
}

%\begin{erboxedFigure}{#1 FigureParam}{#2 Label}{#3 Caption}
\NewEnviron{erboxedFigure}[3]{%
\begin{figure}[#1]
\begin{erexample}
\begin{center}
\BODY
\end{center}
\vspace{-0.5cm}
\caption{#3}
\label{#2}
\end{erexample}
\end{figure}
}

\newcommand{\erpictureFolder}[0]{../SharedPictures}

\newcommand{\ercenterPicture}[1]{
\begin{center}
\input{\erpictureFolder/#1}
\end{center}
}


\newlength{\erhalfHt}

%\erinlinePicture{#1 pictureFilename}{#2 pictureHeight}
\newcommand{\erinlinePicture}[2]{
\setlength{\erhalfHt}{#2cm * \real{0.5}}
\raisebox{-\erhalfHt}[\erhalfHt + 0.5cm][\erhalfHt + 0.5cm]{
\input{\erpictureFolder/#1}
} 
}

%\erplainFig{#1 pictureFilename}{#2 figureParam}{#3Caption}
\newcommand{\erplainFig}[3]{
\begin{figure}[#2]
\begin{center}
\input{\erpictureFolder/#1}
\end{center}
\caption{#3}
\label{#1}
\end{figure}
}

%\erboxedFigPicture{#1 pictureFilename}{#2 figureParam}{#3Caption}
\newcommand{\erboxedFigPicture}[3]{
\begin{figure}[#2]
\begin{erexample}
\vspace{-0.5cm}
\begin{center}
\input{\erpictureFolder/#1}
\end{center}
\caption{#3}
\label{#1}
\end{erexample}
\end{figure}
}

%\erLeftSideFig{#1 pictureFilename}{#2 figureParam}{#3Caption}
\newcommand{\erLeftSideFig}[3]{
\begin{figure}[#2]
\begin{erexample}
  \begin{minipage}[c]{0.4\textwidth}
    \caption{#3}
    \label{#1}
  \end{minipage}
  \begin{minipage}[c]{0.5\textwidth}
    \input{\erpictureFolder/#1}
  \end{minipage}
\end{erexample}
\end{figure}
}

%\erbulletedFig{#1 pictureFilename}{#2 figureParam}{#3Caption}
\NewEnviron{erbulletedFig}[3]{%
\begin{figure}[#2]
\begin{erexample}
\vspace{-0.5cm}
\begin{center}
$
\begin{array}{c m{0.25cm} | m{6cm}}
\raisebox{-2.0cm}{
\input{\erpictureFolder/#1}}& & \text{\parbox{6cm}{\raggedright{\footnotesize{
\begin{enumerate}[(i)]
\BODY
\end{enumerate}}}}} \\
\end{array}
$
\end{center}
\caption{#3}
\label{#1}
\end{erexample}
\end{figure} 
}


%\begin{erbulletedDimFig}{#1 pictureFilename}{#2figureParam} {#3Caption} {#4PictureHeight}{#5TextWidth}

\NewEnviron{erbulletedDimFig}[5]{%
\begin{figure}[#2]
\begin{erexample}
\vspace{-0.5cm}
\begin{center}
$
\begin{array}{c m{0.25cm} |  m{#5cm}}
\setlength{\erhalfHt}{#4cm * \real{0.5}}
\raisebox{-\erhalfHt}{
\input{\erpictureFolder/#1}}& & \text{\parbox{#5cm}{\raggedright{\footnotesize{
\begin{enumerate}[(i)]
\BODY
\end{enumerate}}}}} \\
\end{array}
$
\end{center}
\caption{#3}
\label{#1}
\end{erexample}
\end{figure} 
}

%\begin{ernotedModel}{#1 pictureFilename}{#2PictureHeight}{#3PictureWidth}{#4TextWidth}

\NewEnviron{ernotedModel}[4]{%
\begin{center}
$
\begin{array}{m{#3cm} m{1cm} | c m{#4cm}}
\setlength{\erhalfHt}{#2cm * \real{0.5}}
\raisebox{-\erhalfHt}{
\input{\erpictureFolder/#1}}& & & \text{\parbox{#4cm}{\raggedright{\footnotesize{
\BODY
}}}} \\
\end{array}
$
\end{center} 
}

%\begin{ermodelText}{#1 pictureFilename}{#2PictureHeight}{#3PictureWidth}{#4TextWidth}

\NewEnviron{ermodelText}[4]{%
\begin{center}
\begin{tabular}{m{#3cm} m{1cm}  c m{#4cm}}
\setlength{\erhalfHt}{#2cm * \real{0.5}}
\raisebox{-\erhalfHt}{
\input{\erpictureFolder/#1}}& & & \text{\parbox{#4cm}{\raggedright{\small{
\BODY
}}}} \\
\end{tabular}
\end{center} 
}


%\erbulletedModel{#1 pictureFilename}{#2PictureHeight}{#3PictureWidth}{#4TextWidth}

\NewEnviron{erbulletedModel}[4]{%
\begin{center}
$
\begin{array}{m{#3cm} m{1cm} | c m{#4cm}}
\setlength{\erhalfHt}{2cm * \real{0.5}}
\raisebox{-\erhalfHt}{
\input{\erpictureFolder/#1}}& & & \text{\parbox{#4cm}{\raggedright{\footnotesize{
\begin{enumerate}[(i)]
\BODY
\end{enumerate}}}}} \\
\end{array}
$
\end{center} 
}



%\ernotedDimFig{#1 pictureFilename}{#2 figureParam}{#3Caption}{#4PictureHeight}{#5TextWidth}
\NewEnviron{ernotedDimFig}[5]{%
\begin{figure}[#2]
\begin{erexample}
\vspace{-0.5cm}
\begin{center}
$
\begin{array}{c m{0.25cm} | c m{#5cm}}
\setlength{\erhalfHt}{#4cm * \real{0.5}}
\raisebox{-\erhalfHt}{
\input{\erpictureFolder/#1}}& & & \text{\parbox{#5cm}{\raggedright{\footnotesize{
\BODY }}}}\\
\end{array}
$
\end{center}
\caption{#3}
\label{#1}
\end{erexample}
\end{figure} 
}
%\begin{ernotedDimFigPW}{#1 pictureFilename}{#2 figureParam}{#3Caption}{#4PictureHeight}{#5PictureWidth}{#6TextWidth}
\NewEnviron{ernotedDimFigPW}[6]{%
\begin{figure}[#2]
\begin{erexample}
\vspace{-0.5cm}
\begin{center}
$
\begin{array}{>{\centering}m{#5cm} m{0.5cm} | c m{#6cm}}
\setlength{\erhalfHt}{#4cm * \real{0.5}}
\raisebox{-\erhalfHt}{
\centering \input{\erpictureFolder/#1}
}& & & \text{\parbox{#6cm - 0.5cm}{\raggedright{\footnotesize{
\BODY }}}}\\
\end{array}
$ \\
\vspace {0.2cm}
\end{center}
\caption{#3}
\label{#1}
\end{erexample}
\end{figure}
}



\newenvironment{erquote}
{\begin{quote}\itshape}
{\end{quote}}


%
%  erdiagram.tex
%  *************
%  Macros to represent ER diagrams
%  *******************************
% 29/01/2019 Modify so that not reliant on the
%            default fontsize being 10pt by using
%            package anyfontsize and then
%            \fontsize{8}{10}\selectfont to set font to 8pt
% 06/02/2019 Pullback symbol implemented and minor tweaks to positioning 
%            and size of identifier symbol and relationship labels.
%            Accidental forked changes merged on 08/02/2019.
% 15/03/2019 Continuation of 29/01/2019. Need fix fontsize of 
%            ERrelname and ERscope.	 
% ***********************************************************
 \usepackage{anyfontsize}             % 29/01/2019 
  
%\begin{erdiagram}{#1 height}{#2 width} 
% ....
% ....
%\end{erdiagram}
\newenvironment{erdiagram}[2]
{%\pspicture*(-#1,0)(#2,0)
\pspicture*(0,-#1)(#2,0)
%\psgrid
}
{\endpspicture}

\definecolor{lightyellow}{cmyk}{0,0,0.3,0}
\definecolor{verylightgrey}{gray}{0.95}


% \eret{#1 x0} {#2 y0} {#3 x1} {#4 y1} {#5 corner radius} {#6 fill}
\newcommand {\eret}[6]
{ 
\ifthenelse{\equal{#6}{1}}
{\psframe[framearc=#5,fillstyle=solid,fillcolor=lightyellow](#1,#2)(#3,#4)}
{\psframe[framearc=#5,fillstyle=solid,fillcolor=white](#1,#2)(#3,#4)}
}

% et top 
\newcommand {\erettop}[4]
{
%\psframe[linestyle=none,linearc=2pt,cornersize=absolute,fillstyle=solid,fillcolor=lightyellow](#1,#2)(#3,#4)
\psline[linearc=2pt,fillstyle=none,fillcolor=lightyellow](#1,#4)(#1,#2)(#3,#2)(#3,#4)
}

% et bottom 
\newcommand {\eretbtm}[4]
{
%\psframe[linestyle=none,linearc=2pt,cornersize=absolute,fillstyle=solid,fillcolor=lightyellow](#1,#2)(#3,#4)
\psline[linearc=2pt,fillstyle=none,fillcolor=lightyellow](#1,#2)(#1,#4)(#3,#4)(#3,#2)
}

% et bottom left
\newcommand {\eretbl}[4]
{
%\psframe[linestyle=none,linearc=2pt,cornersize=absolute,fillstyle=solid,fillcolor=lightyellow](#1,#2)(#3,#4)
\psline[linearc=2pt,fillstyle=none,fillcolor=lightyellow](#1,#4)(#3,#4)(#3,#2)
}

% et middle left
\newcommand {\eretml}[4]
{
%\psframe[linestyle=none,linearc=2pt,cornersize=absolute,fillstyle=solid,fillcolor=lightyellow](#1,#2)(#3,#4)
\psline[linearc=2pt,fillstyle=none,fillcolor=lightyellow](#1,#2)(#3,#2)(#3,#4)(#1,#4)
}

% et top left
\newcommand {\erettl}[4]
{
%\psframe[linestyle=none,linearc=2pt,cornersize=absolute,fillstyle=solid,fillcolor=lightyellow](#1,#2)(#3,#4)
\psline[linearc=2pt,fillstyle=none,fillcolor=lightyellow](#1,#2)(#3,#2)(#3,#4)
}

% et bottom right
\newcommand {\eretbr}[4]
{
%\psframe[linestyle=none,linearc=2pt,cornersize=absolute,fillstyle=solid,fillcolor=lightyellow](#1,#2)(#3,#4)
\psline[linearc=2pt,fillstyle=none,fillcolor=lightyellow](#1,#2)(#1,#4)(#3,#4)
}

% et middle right
\newcommand {\eretmr}[4]
{
%\psframe[linestyle=none,linearc=2pt,cornersize=absolute,fillstyle=solid,fillcolor=lightyellow](#1,#2)(#3,#4)
\psline[linearc=2pt,fillstyle=none,fillcolor=lightyellow](#3,#4)(#1,#4)(#1,#2)(#3,#2)
}

% et top right
\newcommand {\erettr}[4]
{
\psline[linearc=2pt,fillstyle=none,fillcolor=lightyellow](#1,#4)(#1,#2)(#3,#2)
}

% \ergrp{#1 x0} {#2 y0} {#3 x1} {#4 y1} {#5 corner radius} {#6 fill}
% #5 corner radius is unused!
\newcommand {\ergrp}[6]
{ 
\ifthenelse{\equal{#6}{1}}
{\psframe[fillstyle=solid,fillcolor=verylightgrey](#1,#2)(#3,#4)}
{\psframe[fillstyle=solid,fillcolor=white](#1,#2)(#3,#4)}
}


% \ertext{#1 text}
% 15/03/2019
\newcommand {\erextrasmallitalictext}[1]
{\fontsize{7}{9}\selectfont \textit{#1}}

% 29/01/2019  
\newcommand {\ersmallitalictext}[1]
{\fontsize{8}{10}\selectfont \textit{#1}}

\newcommand {\ermediumitalictext}[1]
{\fontsize{10}{12}\selectfont \textit{#1}}

% \eretname {#1 x left of text} {#2 y top of text} {#3 text}
\newcommand {\olderetname}[3]
{
%shift down 0.1 for height of text the anchor at baseline (B)
\rput[bl]{0}(0,-0.1){\rput[Bl]{0}(#1,#2){\ersmallitalictext{#3}}}
}

% \errelarm {#1 x0} {#2 y0} {#3 x1} {#4 y1} {#5 ismandatory} {#6 isconstructed}
\newcommand {\errelarm}[6]
{
\ifthenelse{\equal{#6}{1}}
{
%%\psline[linewidth=0.5pt,linearc=.05,linestyle=dashed,dash=6pt 6pt]{-}(#1,#2)(#3,#4)}
\ifthenelse{\equal{#5}{1}}
{\psline[linewidth=1.5pt,linearc=.05,linecolor=lightgray]{-}(#1,#2)(#3,#4)}
{\psline[linewidth=1.5pt,linearc=.05,linecolor=lightgray,linestyle=dashed,dash=2pt 2pt]{-}(#1,#2)(#3,#4)}
}
{
\ifthenelse{\equal{#5}{1}}
{\psline[linewidth=0.9pt,linearc=.05]{-}(#1,#2)(#3,#4)}
{\psline[linewidth=0.9pt,linearc=.05,linestyle=dashed,dash=2pt 2pt]{-}(#1,#2)(#3,#4)}
}
}

% \errelangle {#1 x0} {#2 y0} {#3 x1} {#4 y1} {#5 x2} {#6 y2} {#7 ismandatory} {#8 isocnstructed}
\newcommand {\errelangle}[8]
{
\ifthenelse{\equal{#8}{1}}
{
%\psline[linewidth=0.5pt,linearc=.1,linestyle=dashed,dash=6pt 6pt]{-}(#1,#2)(#3,#4)(#5,#6)}
\ifthenelse{\equal{#7}{1}}
{\psline[linewidth=1.5pt,linearc=.05,linecolor=lightgray]{-}(#1,#2)(#3,#4)(#5,#6)}
{\psline[linewidth=1.5pt,linearc=.1,linecolor=lightgray,linestyle=dashed,dash=2pt 2pt]{-}(#1,#2)(#3,#4)(#5,#6)}
}
{
\ifthenelse{\equal{#7}{1}}
{\psline[linewidth=0.9pt,linearc=.1]{-}(#1,#2)(#3,#4)(#5,#6)}
{\psline[linewidth=0.9pt,linearc=.1,linestyle=dashed,dash=2pt 2pt]{-}(#1,#2)(#3,#4)(#5,#6)}
}
}

% \ercrowfoot {#1 x0} {#2 y0} {#3 x11} {#4 y11} {#5 x12} {#6 y12} {#7 x13} {#8 y13} {#9 isconstructed}
\newcommand {\ercrowfoot}[9]
{
\ifthenelse{\equal{#9}{1}}
{
\psline[linewidth=1.5pt,linearc=.05,linecolor=lightgray]{-}(#1,#2)(#3,#4)
\psline[linewidth=1.5pt,linearc=.05,linecolor=lightgray]{-}(#1,#2)(#5,#6)
\psline[linewidth=1.5pt,linearc=.05,linecolor=lightgray]{-}(#1,#2)(#7,#8)
}{
\psline[linewidth=0.9pt,linearc=.05]{-}(#1,#2)(#3,#4)
\psline[linewidth=0.9pt,linearc=.05]{-}(#1,#2)(#5,#6)
\psline[linewidth=0.9pt,linearc=.05]{-}(#1,#2)(#7,#8)
}
}


% \eridcomprel{#1 x1}{#2 x2}{#3 y}
\newcommand {\eridcomprel}[3]
{
\psline[linewidth=0.9pt](#1,#3)(#2,#3)
}

% \eridrefrel{#1 x}{#2 y1}{#3 y2}
\newcommand {\eridrefrel}[3]
{
\psline[linewidth=0.9pt](#1,#2)(#1,#3)
}

% \ertext {#1 x} {#2 y} {#3 text anchor} {#4 text}  PRIVATE
\newcommand {\ertext}[4]
{
\rput[B#3]{0}(#1,#2){\fontsize{8}{10}\selectfont #4}
}

% \eretname {#1 x} {#2 y} {#3 text anchor} {#4 text} 
\newcommand {\eretname}[4]
{
\ertext{#1}{#2}{#3}{#4}
}

% \errelname {#1 x} {#2 y} {#3 text anchor} {#4 text} 
\newcommand {\errelname}[4]
{
\rput[B#3]{0}(#1,#2){\erextrasmallitalictext{#4}}
}


% \erscope {#1 x} {#2 y} {#3 text anchor} {#4 text}  15 March 2019
\newcommand {\erscope}[4]
{
\rput[B#3]{0}(#1,#2){\erextrasmallitalictext{#4}}
}

% \erreletname {#1 x} {#2 y} {#3 text anchor} {#4 text}  15 March 2019
\newcommand {\erreletname}[4]
{
\rput[B#3]{0}(#1,#2){\fontsize{10}{12}\selectfont #4}
}

% \ergroupannotation {#1 x} {#2 y} {#3 text anchor} {#4 text}
\newcommand {\ergroupannotation}[4]
{
\ertext{#1}{#2}{#3}{#4}
}


% \errelseq {#1 x} {#2 y}
\newcommand {\erelseq}[2]
{
}
\newcommand {\erattrmarkermand}
{\fontsize{6}{8}\selectfont $\blacksquare$}
\newcommand {\erattrmarkeropt}
{\fontsize{6}{8}\selectfont \CIRCLE}
\newcommand {\erderattrmarkermand}
{\fontsize{6}{8}\selectfont $\square$}
\newcommand {\erderattrmarkeropt}
{\fontsize{8}{10}\selectfont $\circ$}

% \erattr {#1 x} {#2 y} {#3 ismandatory}{#4 idenitfying} {#5 text}
\newcommand {\erattr}[5]
{
\ifthenelse{\equal{#3}{1}}
{\rput[l]{0}(#1,#2){\erattrmarkermand \ersmallitalictext{\ifthenelse{\equal{#4}{0}}{\underline{#5}}{#5}}}}
{\rput[l]{0}(#1,#2){\erattrmarkeropt \ersmallitalictext{\ifthenelse{\equal{#4}{0}}{\underline{#5}}{#5}}}}
}

\newcommand {\erdattr}[5]
{
\ifthenelse{\equal{#3}{1}}
{\rput[l]{0}(#1,#2){\erderattrmarkermand \ersmallitalictext{\ifthenelse{\equal{#4}{0}}{\underline{#5}}{#5}}}}
{\rput[l]{0}(#1,#2){\erderattrmarkeropt \ersmallitalictext{\ifthenelse{\equal{#4}{0}}{\underline{#5}}{#5}}}}
}


% \erarc {#1 x0} {#2 y0} {#3 x1} {#4 y1} {#5 x2} {#6 y2} {#7 x3} {#8 y3}
\newcommand {\erarc}[8]
{
\psbezier[showpoints=false]{-}(#1,#2) (#3, #4)(#5,#6) (#7, #8)
}

% \erarc {#1 x0} {#2 y0} {#3 x1} {#4 y1} {#5 x2} {#6 y2} {#7 x3} {#8 y3}
\newcommand {\errelseq}[8]
{
\psbezier[showpoints=false]{-}(#1,#2) (#3, #4)(#5,#6) (#7, #8)
}
% \ertrace {#1 trace}   
\newcommand {\ertrace}[1]
{
}

\usepackage{amsthm} % added 7th April 2018
% theorems.macros.tex

\newtheorem{theorem}{Theorem}[section]
\newtheorem{observation}[theorem]{Observation}
\newtheorem{lemma}[theorem]{Lemma}
\newtheorem{proposition}[theorem]{Proposition}
\newtheorem{corollary}[theorem]{Corollary}
\newtheorem{conjecture}[theorem]{Conjecture}
\newtheorem{numbereddefinition}[theorem]{Definition}

\newenvironment{definition}[1][Definition]{\begin{trivlist}
\item[\hskip \labelsep {\bfseries #1}]}{\end{trivlist}}
\newenvironment{examples}[1][Examples]{\begin{trivlist}
\item[\hskip \labelsep {\bfseries #1}]}{\end{trivlist}}
\newenvironment{example}[1][Example]{\begin{trivlist}
\item[\hskip \labelsep {\bfseries #1}]}{\end{trivlist}}
\newenvironment{remark}[1][Remark]{\begin{trivlist}
\item[\hskip \labelsep {\bfseries #1}]}{\end{trivlist}}

\newenvironment{tageqn}[1]
{
\begin{equation}
\stepcounter{equation}
\label{#1}
\tag{\theequation --#1}
}
{
\end{equation}
}

\newenvironment{axiom}[1]
{
\begin{equation}
\label{#1}
\tag{#1}
}
{
\end{equation}
}

% when the tag is required different from the label eg when has math symbols can use:
\newenvironment{axiomtagged}[2]
{
\begin{equation}
\label{#1}
\tag{#2}
}
{
\end{equation}
}

%visible label
\newcommand{\vlabel}[2][]{\label{#2}#1(\textit{#2}):}





\usepackage{imakeidx}
\usepackage{framed}
\makeindex[name=definitions, title=Index of Definitions]
\makeindex[name=lemmas, title=Index of Lemmas]



\newcommand{\seenudgeup}[1]{\rule{0.1cm}{#1}}

\newcommand{\seenudgedown}[1]{\rule[-#1]{0.1cm}{0.1cm}}

\newcommand{\nudgeup}[1]{\rule{0cm}{#1}}

\newcommand{\nudgedown}[1]{\rule[-#1]{0cm}{0.1cm}}

\definecolor{highlight}{cmyk}{0,0,0.7,0}
\newcommand{\commentary}[1]{\marginpar{\footnotesize #1}}
\newcommand{\highlight}[1]{\colorbox{highlight}{#1}}
\newcommand{\whitelight}[1]{\colorbox{white}{#1}}
\newcommand{\term}[1]{\textit{#1}\commentary{\colorbox{lightgray}{\textit{#1}}}\index[definitions]{#1}}
\newcommand{\llabel}[1]{\label{#1}\commentary{\colorbox{pink}{\scriptsize{#1}}}\index[lemmas]{#1}}
\newcommand{\lref}[1]{\ref{#1}\colorbox{pink}{\scriptsize{#1}}\index[lemmas]{#1!use of}}

\newcommand{\daynote}[1]{\commentary{See day notes #1.}}

\newcommand{\newt}[1]{\colorbox{yellow}{#1}}
\newenvironment{newtt}
{  \colorbox{yellow}{$[$ ...} 
}
{  \colorbox{yellow}{... $]$}
}
\newcommand{\oldt}[1]{\colorbox{red}{\sout{#1}}}
\newenvironment{oldtt}
{  \colorbox{red}{$[$ ...} 
}
{  \colorbox{red}{... $]$}
}

\newcommand{\reinstatet}[1]{\colorbox{lime}{#1}}
\newenvironment{reinstatett}
{  \colorbox{lime}{$[$ ...}
}
{  \colorbox{lime}{... $]$}
}

\newcommand{\tbd}{\highlight{TBD}}

%ithprojection function
\newcommand{\proji}[1]{\pi_#1}


\newenvironment{aside}
{\begin{framed}
\textbf{Aside}
}
{
\end{framed}
}

\newenvironment{notebox}[1][Note]
{\begin{framed}
\textbf{#1}
}
{
\end{framed}
}

\newenvironment{categoricalaside}
{\begin{framed}
\textbf{Categorical Aside}
}
{
\end{framed}
}

\newenvironment{noteforfuture}
{\begin{framed}
\textbf{Note For Future}
}
{
\end{framed}
}

\newenvironment{problem}
{\begin{framed}
\textbf{Problem}
}
{
\end{framed}
}

\newenvironment{key}
{
\begin{tabular}{c l p{4cm}}
KEY && \\
\hline
}
{
\end{tabular}
}

\newcommand{\keyentry}[3]{#1 & #2 & #3 \\} 


%quine quote
\newcommand{\qq}[1]{
\left\ulcorner#1\right\urcorner
}

%single quote
\newcommand{\sq}[1]{
\textnormal{\textquotesingle}#1\textnormal{\textquotesingle}
}

%lower quine quote
\newcommand{\lqq}[1]{
\left\llcorner #1\right\lrcorner
}


%from berkley
\newcommand{\langl}{\begin{picture}(4.5,7)
\put(1.1,2.5){\rotatebox{60}{\line(1,0){5.5}}}
\put(1.1,2.5){\rotatebox{300}{\line(1,0){5.5}}}
\end{picture}}
\newcommand{\rangl}{\begin{picture}(4.5,7)
\put(.9,2.5){\rotatebox{120}{\line(1,0){5.5}}}
\put(.9,2.5){\rotatebox{240}{\line(1,0){5.5}}}
\end{picture}}
\newcommand{\lang}{\begin{picture}(5,7)\put(1.1,2.5){\rotatebox{45}{\line(1,0){6.0}}}\put(1.1,2.5){\rotatebox{315}{\line(1,0){6.0}}}\end{picture}}
\newcommand{\rang}{\begin{picture}(5,7)\put(.1,2.5){\rotatebox{135}{\line(1,0){6.0}}}\put(.1,2.5){\rotatebox{225}{\line(1,0){6.0}}}\end{picture}}
%Try sharper tuple brackets -- except gives errors nested in captions so comment out
%\renewcommand{\tuple}[1]{\lang #1 \rang}

\newcommand{\setsuchthat}[2]{\left\{#1 \ \middle|\ #2\right\}}
\newcommand{\set}[1]{\left\{#1\right\}} 

% one to n - wanton
\newcommand{\wanton}[1]{#1_1,...#1_n}
\newcommand{\n}{1...n}
\newcommand{\fn}{\wanton{f}}
\newcommand{\gn}{\wanton{g}}
\newcommand{\pn}{\wanton{p}}
\newcommand{\qn}{\wanton{q}}
\newcommand{\qnprime}{\wanton{q'}}
\newcommand{\tn}{\wanton{t}}
\newcommand{\xn}{\wanton{x}}
\newcommand{\xnp}{\wanton{x'}}
\newcommand{\yn}{\wanton{y}}
\newcommand{\An}{\wanton{A}}
\newcommand{\Bn}{\wanton{B}}
\newcommand{\Cn}{\wanton{C}}
\newcommand{\ntuple}[1]{\tuple{\wanton{#1}}}
\newcommand{\wantom}[2][]{#2_1,...#2_{m#1}}
\newcommand{\m}{1...m}
\newcommand{\mtuple}[1]{\tuple{#1_1,...#1_m}}
\newcommand{\gm}{\wantom{g}}
\newcommand{\qm}{\wantom{q}}
\newcommand{\sm}[1][]{\wantom[#1]{s}}
\newcommand{\smp}{\wantom{s'}}
\newcommand{\ym}{\wantom{y}}
\newcommand{\Bm}{\wantom{B}}
\newcommand {\bntuple}{\ensuremath{\ntuple{b}}}
\newcommand {\fntuple}{\ensuremath{\ntuple{f}}}
\newcommand {\fnptuple}{\ensuremath{\ntuple{f}}}
\newcommand {\pntuple}{\ensuremath{\ntuple{p}}}
\newcommand {\qntuple}{\ensuremath{\ntuple{q}}}
\newcommand {\qnptuple}{\ensuremath{\ntuple{q'}}}
\newcommand {\qmtuple}{\ensuremath{\mtuple{q}}}
\newcommand {\sntuple}{\ensuremath{\ntuple{s}}}
\newcommand {\xntuple}{\ensuremath{\ntuple{x}}}
\newcommand {\xnptuple}{\ensuremath{\ntuple{x'}}}
\newcommand {\ymtuple}{\ensuremath{\mtuple{y}}}
\newcommand{\idef}[1][n]{1 \leq i \leq #1}
\newcommand{\jdef}[1][m]{1 \leq j \leq #1}
\newcommand{\kdef}[1][l]{1 \leq k \leq #1}
\newcommand{\foreachi}[1][n]{for each $i$, $1 \leq i \leq #1$}
\newcommand{\foreachj}[1][m]{for each $j$, $1 \leq j \leq #1$}
\newcommand{\foreachk}[1][l]{for each $k$, $1 \leq k \leq #1$}
\newcommand{\Foreachi}[1][n]{For each $i$, $1 \leq i \leq #1$}
\newcommand{\Foreachj}[1][m]{For each $j$, $1 \leq j \leq #1$}
\newcommand{\Foreachk}[1][l]{For each $k$, $1 \leq k \leq #1$}
\newcommand{\forsomei}[1][n]{for some $i$, $1 \leq i \leq #1$}
\newcommand{\forsomej}[1][m]{for some $j$, $1 \leq j \leq #1$}
\newcommand{\forsomek}[1][l]{for some $k$, $1 \leq k \leq #1$}
\newcommand{\wherei}[1][n]{where $1 \leq i \leq #1$}
\newcommand{\wherej}[1][m]{where $1 \leq j \leq #1$}
\newcommand{\wherek}[1][l]{where $1 \leq k \leq #1$}


\newcommand{\fundep}[3]{#2 \xrightarrow{#1} #3}  %where does this belong? xxxx
% Following used for notes -- indented numbered paras

\newcounter{para}
\newlength{\oldparindent}
\setlength{\oldparindent}{\parindent} % Save \parindent before of change
\newcommand{\ind}{\hspace*{\oldparindent}}
\newcommand\note{
%\setlength{\parskip}{0.5\baselineskip} % Definition of `parskip`
\setlength{\parindent}{0pt}
\par\ind\refstepcounter{para}\thepara.\space
\setlength{\parindent}{\oldparindent}
}



\newcommand{\ncarrNEGZZ}[3][0]{\ncarc[arcangle=#1,nodesepA=2pt,nodesepB=2pt,offsetA=-2pt,offsetB=-2pt,arrowsize=5pt,arrowinset=0.7]{->}{#2}{#3}}
\newcommand{\ncarrZ}[3][0]{\ncarc[arcangle=#1,nodesepA=2pt,nodesepB=2pt,offsetA=0pt,offsetB=0pt,arrowsize=5pt,arrowinset=0.7]{->}{#2}{#3}}
\newcommand{\ncarrZZ}[3][0]{\ncarc[arcangle=#1,nodesepA=2pt,nodesepB=2pt,offsetA=2pt,offsetB=2pt,arrowsize=5pt,arrowinset=0.7]{->}{#2}{#3}}
\newcommand{\ncarrZZZ}[3][0]{\ncarc[arcangle=#1,nodesepA=2pt,nodesepB=2pt,offsetA=4pt,offsetB=4pt,arrowsize=5pt,arrowinset=0.7]{->}{#2}{#3}}
\newcommand{\ncarrZZZZ}[3][0]{\ncarc[arcangle=#1,nodesepA=2pt,nodesepB=2pt,offsetA=6pt,offsetB=6pt,arrowsize=5pt,arrowinset=0.7]{->}{#2}{#3}}


\newcommand{\ccsquareoutline}[6]
{\begin{array}{cp{#1}c}
\Rnode{TL}{#3}  & &  \Rnode{TR}{#4}\\ [#2]
\Rnode{BL}{#5}  & &  \Rnode{BR}{#6}
\end{array}
}
\newcommand{\ccsquareacross}[2]
{\mbox{\ncarr{TL}{TR}
\alabel{#1}
\ncarr{BL}{BR}
\blabel{#2}}
}
\newcommand{\ccsquaredown}[2]
{\mbox{\ncsar{TL}{BL}
\blabel{#1}
\ncsar{TR}{BR}
\alabel{#2}}
}
\newcommand{\ccsquareanddroppers}[6]
{\ccsquareoutline{#1}{#2}{#3}{#4}{#5}{#6}
\ccsquaredown{p_{#3}}{p_{#4}}
}

\usepackage{mathptmx}  % This changes font to roman
\usepackage{anyfontsize}
\usepackage{mathtools}  % why have we got this?
\usepackage{alltt}    
%\usepackage{mnsymbol} %used for rightpitchfork NOT AVAILABLE FROM LINUX
                    % turns out not actually using \rightpitchfork
\usepackage{cmll}
\usepackage{ulem}
\renewcommand{\ttdefault}{txtt}
\usepackage[left=1.5cm, right=4cm, marginparwidth=3cm, top=2cm, bottom=2.0cm]{geometry}
\usepackage{framed}
\usepackage[font=small]{caption}
\setlength{\captionmargin}{2cm}
\theoremstyle{remark}
\newtheorem*{lemma*}{Lemma}

\renewcommand{\term}[1]{\textit{#1}}  %SIMPLE UNINDEXED VERSION
\usepackage{imakeidx}
\makeindex[name=definitions, title=Index of Definitions]
\makeindex[name=lemmas, title=Index of Lemmas]

\newcommand{\fgsourcediag}{$\binarysourcediag{a}{b}{c}{f}{g}$}
\newcommand{\fgparalleldiag}{$\paralleldiag{a}{b}{f}{g}$}

\begin{document}
\title{Preparation for a Mathematical Theory of Data}

\author{John Cartmell}

\date{}

\maketitle

\begin{center}
DRAFTED October 2020 \\
REVISED November 2023
\end{center}

%\newcommand{\seenudgeup}[1]{\rule{0.1cm}{#1}}
%\newcommand{\seenudgedown}[1]{\rule[-#1]{0.1cm}{0.1cm}}
%\newcommand{\nudgeup}[1]{\rule{0cm}{#1}}
%\newcommand{\nudgedown}[1]{\rule[-#1]{0cm}{0.1cm}}

\section{Background}

There is a wide range of situations in which the structure of data is specified
and a wide variation in the syntax used
\begin{itemize}
\item descriptions of data structure are fundamental to most programming languages, 
where they vary depending on the model of computation (object-oriented, functional, symbolic and so on);
\item
they occur as dedicated specification methods, such as the entity relationship method,
\item they are required in database technologies, where they vary by the data model (relational, hierarchical, nested relational,
graph based, etc.),
\item they underlie interfacing technologies and either describe binary formats such as 
the many that implement some variant of IDL (Interface Definition Language) or text formats such as epitomised by XML. 
\end{itemize}

Whatever the method or the context of the description, the thinking is that when viewed abstractly each data specification is a 
theory\footnote{The role of such a theory can be foregrounded by speaking of it
 as a \textit {theory of what is} or as an \textit{ontology}.} and  to each different notion of data specification corresponds a different notion of theory. An exposition of the different notions of theory that can properly be said to be methods of data specification
along with a study of their meta-mathematical properties 
 will constitute a mathematical theory of data which, as described, is therefore in fact a meta-theory. Such a theory has a role to play in improving  the way we think about, discuss, design, develop and transform data specifications. I strongly believe that such a fully elaborated mathematical theory of data will foster significant improvements in  techniques and tools for the management of data. 

The relational model of data underpinning the majority of databases for fifty years or so, is exceptional in that it has a body of theory; this theory includes quality criteria  distinguishing good data specifications from bad. 
One of the goals of a mathematical theory of data is to enable these relational prescriptions of goodness
to be generalised to become generally applicable. The \textit{dry run} below suggests this is possible.

Data is required for a purpose, generally to describe real world things in some or other context. This constitutes an intended usage for a data specification. Of all structurally compliant instances of a data specification some are required for the intended usage and, generally speaking, some are not.
Notionally let there be a requirement $R$ that equates to a subset of the set of all compliant data instances 
of a data specification and serving to characterise its intended use. 

There are two self-complementing principles of good data engineering. 
Firstly, redundancy of data is to be avoided. This first principle is modulated by computational cost for it would be unreasonable not to hold in data all prime factors of a number on account of them being computable and therefore redundant.
Secondly, within each particular methodology a data specification should be as constraining as possible of data instances whilst being general enough for the intended usage; equivalently the corresponding theory should fit as tightly as possible to the facts. 

Meeting the second principle we will describe as achieving \term{maximum constrainedness} for the data specification.
To maximise constrainedness will be to come as  close as we can within any given methodology with given syntax to meeting a formal objective described by Zaniola \cite{zaniolo1982} in the context of relational schema design (data specification, that is, for the relational model of data)  as `the complete \textit{representation} of semantic constraints' (his italics). Zaniola subsequently refers to this as `the representation principle'.

From the two principles we can phrase goodness criteria  for data specifications with respect to requirements $\reqt$
i.e. to intended usages. In the context of relational data design, 3rd, 4th and 5th normal forms are examples of such goodness criteria. 

When viewed abstractly many distinct notions of data specification can be characterised as having
data specifications corresponding to finite presentations of either categories or, if missing data is to be allowed, partial order enriched categories with some additional structure such as certain limits and/or colimits. We use the term \textit{sketch}
in this note to be synonymous with \textit{presentation of category} and as such take it to consist of the combination of a directed graph and a set of path equivalences. In Barr and Wells \cite{BarrandWells} these are more properly called linear sketches.  

Compliant instances of such data specifications correspond to structure preserving functors from the corresponding category to the category of finite sets $\Fin$ or to the category of finite sets and partial functions.

Redundancy of objects or arrows in a presentation corresponds to redundancy of data in instances of a data specification. 
By the first principle it is the goal of data specification to avoid such redundancy. 

Goodness equates to absence of redundancy plus maximal constrainedness to intended usage. Absence of redundancy is a property of a presentation. Maximal constrainedness
is a property of the category $\catc$ generated by the presentation and is relative to a requirement $\reqtc$, where $\reqtc$ is a set of instances where each instance is structure preserving functor $D$, $D: \catc \morph \Fin$, where
$\Fin$ is the category of finite sets and functions (or, subsequently, to other variants of the category of sets and functions as appropriate).

Codd \cite{Codd1970} proposes the relational model of data; he gives the first prescription of goodness for
a relational data specification and describes how it might be achieved through a method which he calls normalisation\cite{Codd1970}\footnote{He also introduces the term foreign key in this first paper and includes a discussion of redundancy of data.}. 
Codd  subsequently defines a third normal form (3NF) \cite{Codd1971} for which purpose he introduces 
the concept of a functional dependency.
The definition of third normal form extends the notion of goodness and the method for achieving it\footnote{By \cite{Codd1971} the stage was set 
for describing conditions of goodness in terms of relational schemas being in normal form -- an  unfortunate terminology  because these schemas that meet the condition
are not canonical in any way as a mathematician might be led to believe from the terminology.}.

Boyce-Codd normal form (BCNF) is a stronger normal form and one that it is not always possible to meet. Zaniola \cite{zaniolo1982}) most clearly elaborates the difference between 3NF and BCNF. 
In Zaniola's description, specifications that are in BCNF meet the representation principle in regard to having all functional dependencies represented in them.

Further standards that a good relational data specification should adhere to were formulated by Fagin \cite{Fagin1977} (fourth normal form) and  \cite{Fagin1979} (projection-join normal form also known as fifth normal form)
using the concept of multi-valued dependencies. 
One paraphrasing would be that it isn't good to store needless copies of data. 
When formulated in category theory this will come down to not needlessly including limit objects in a presentation.

In a different direction many authors describe forms of redundancy in data that are immune to prescriptions
of previous normal forms (up to 5th normal form, say) and to remedy this 
they give definitions of normal forms that take account of inclusion dependencies.
There isn't a single clear concept that arises from this work but the deficiency and the need for a remedy is very clear.  
Inclusion dependencies, like functional dependencies and multi-valued dependencies, are forms of semantic constraint in the sense that this term is used by Zaniola. 

Here we focus on inclusion dependencies that are referential. These in dry run are the equivalents of what elsewhere in the context of relational data specification are referred to as a key-based or a superkey-based inclusion dependencies [\cite{Mannila1986}, \cite{Levene2000}]
or, more pragmatically, as referential constraints\footnote{	Also known colloquially, and rather horribly in my opinion, as foreign key constraints}in the ISO SQL standard\cite{ISOSQL2016} and in relation to XML
(\cite{fan2003}, for instance); whether implicitly, or explicitly as in the relational paradigm, these are lynchpins of  data specifications.

Various authors (\cite{CartmellScopePaper},\cite{Johnson93}) have noted the importance of commutative diagrams in data specifications.
  The fact is that relational designs fairly frequently have commutivity constraints implicitly represented within  and this having been achieved  by designers following prescriptions to normalise data and to eliminate duplicates rather than with awareness of the underlying commutivity. 
Shlaer and Lang illustrate this in \cite{Shlaer96} where they describe alternative paths between two nodes as
relationship loops, when distinct paths are equivalent they say that there are dependencies
between the relationships. Kolp and Zimnyi ((\cite{Kolp1995})) instead use the term
relationship cycle and identify such as a source of superfluous attributes in the
transformation from ER model to relational model. In this paper we speak of commutative digrams as path equivalences.

\section{Investigation -- Data Specification as Sketch of Category}
\mynote
In this investigation -- data specification as presentation of category i.e. as linear sketch \cite{BarrandWells}-- we will formulate 
definitions of maximal constrainedness, path equivalence, functional dependency and referential inclusion dependency
and we will define what it means for such a path equivalence, functional dependency or  referential inclusion dependency to be represented in such data specifications.

\mynote
This investigation with its overtly simplified notion of data specification is worthwhile in that   it establishes some starter definitions which in further work we can subsequently develop to be applicable to more fully elaborated notions.

\mynote 
This investigation establishes a pattern 
that we will follow later in consideration of more fully elaborated definitions of data specifications that we follow up with -- partial order enriched categories, categories with products and others. 
You may also consider that the definitions given in this investigation are embryonic precursors to equivalent relational definitions.


\subsection{Definitions}
\subsection{Directed Graphs}
Regarding directed graphs and reflecting a category theory mindset we will use terminology as follows:
\begin{itemize}
\item
 If $f: a \morph b$ in an edge of a directed graph $G$ then we will say that $a$ is the \term{domain} of $f$ and $b$ is the \term{codomain} of $f$.
\item
If $a$ and $b$ are nodes of a directed graph $G$ then a \term{path} through $G$ with domain $a$ and 
codomain $b$ of length $n$, where $n \geq 0$, we define to be  an n-tuple of  $n$ edges: $p_i: x_i \morph x_{i+1} $ in $G$ where $x_0=a$ and $x_n=b$. We shall write this n-tuple as $p_1 \circ p_2... \circ p_n$. 
We will use the same notation if any of the $p_i$ are edges rather than paths as we will not need distinguish  an edge from a singleton path along that edge. 
\item two paths have the same domain and the same codomain then we shall say that they are \term{commensurate}.
\end{itemize}




\subsection{Sketches}
By a  \term{sketch for a category} we shall mean a directed graph and a set specified path equivalences.
Each path equivalence consists of a pair $f_1 \circ ... \circ f_n$ and $g_1 \circ ... \circ g_m$ of commensurate paths and can be represented as a diagram so
\begin{displaymath}      
\begin{array}{cp{0.5cm}cp{0.5cm}cp{0.25cm}cp{0.25cm}cp{0.5cm}cp{0.5cm}c}
            &&               &&                &&                  &&                &&               && \\[0.1cm] % vertical space
            &&\Rnode{TL}{c_1}&&\Rnode{TIL}{c_2}&&\Rnode{TC}{\hdots}&& \Rnode{TIR}{c_{n-2}} && \Rnode{TR}{c_{n-1}} &&  \\[0.2cm]
\Rnode{a}{a}&&               &&                &&                  &&                &&               && \Rnode{b}{b} \\[0.2cm]
            &&\Rnode{BL}{d_1}&&\Rnode{BIL}{d_2}&&\Rnode{BC}{\hdots}&& \Rnode{BIR}{d_{m-2}} && \Rnode{BR}{d_{m-1}} &&  \\[0.2cm]        
\end{array}
\begin{arrows}
\ncarr{a}{TL}
\alabel{f_1}
\ncarr{TL}{TIL}
\alabel{f_2}
\ncarr{TIL}{TC}
\ncarr{TC}{TIR}
\ncarr{TIR}{TR}
\alabel{f_{n-1}}
\ncarr{TR}{b}
\alabel{f_n}
\ncarr{a}{BL}
\blabel{g_1}
\ncarr{BL}{BIL}
\blabel{g_2}
\ncarr{BIL}{BC}
\ncarr{BC}{BIR}
\ncarr{BIR}{BR}
\blabel{g_{m-1}}
\ncarr{BR}{b}
\blabel{g_n}
\end{arrows}
\end{displaymath}

For what we are here calling a sketch of a category Barr and Wells use the term \term{linear sketch} and define it as a (directed) graph plus a set of diagrams.

We define the equivalence relation $\sim_S$ of path equivalence determined by a sketch $S$ to be the closure of the set of the specified path equivalences under the following

\begin{itemize}
\item for any path $p$, $p \sim_S p$,
\item for any paths $p_1$ and $p_2$ if $p_1 \sim_S p_2$ then $p_2 \sim_S p_1$,
\item for any paths $p_1$,$p_2$ and $p_3$ if $p_1 \sim_S p_2$ and $p_2 \sim_S p_3$ then $p_1 \sim_S p_3$,
\item if \paralleldiag{a}{b}{p_1}{p_2} and $q: b \morph c$ are paths in $G$ and if $p_1 \sim_S p_2$  
then $p_1 \circ q \sim_S p_2 \circ q$,
\item if $p: a \morph b$ and \paralleldiag{b}{c}{q_1}{q_2} and $q: b \morph c$ are paths in $G$ then if $q_1 \sim_S q_2$  
then $q_1 \circ p \sim_S q_2 \circ p$.
\end{itemize}

The category generated by sketch (called the theory of the sketch in Barr and Wells) is the category with
the nodes of the directed graph of $S$ as objects and with equivalence classes of paths as morphisms.
Composition is defined from composition of representative paths and, from the definition of $\sim_S$, is well-defined.

A sketch is said to be \term{redundancy free} 
if there is no smaller sketch which generates the same category (upto isomorphism of categories).

\subsection{Instances of a Data Specification}
Whilst a sketch represents a data specification, an instance of the specification\footnote{In the case that the data specification describing a database then an instance will be a database instance i.e. a snapshot of the data content at a moment in time.} can be considered to be a set of entities for each node within the graph
along with a many-one functional relationship between the entitity sets for each edge of the graph. 
Accordingly we define an instance of a directed graph $G$ to be a mapping of $G$ to the category of finite sets and functions.
If $G$ is a directed graph and $D$ is an instance then for every node $a$ of $G$, $D(a)$ is a finite set
and for every edge $f:a \morph b$ in $G$, $F(f):D(a) \morph D(b)$ in $\Fin$. Given such an instance $D$ of 
directed graph $G$ 
we can interpret every path $p:a \morph b$ through $G$ as a function $D(p): D(a) \morph D(b)$.

we define an instance of a sketch of a category to be an instance $D$ of its directed graph such that
for each path equivalence $p_1 \sim p_2$ specified in the sketch, $D(p_1)=D(p_2)$.

An instance $D$ of a sketch $S$ that generates a category \catcw 
uniquely determines a functor $D:\catc \morph \Fin$ vice-versa. We don't distinguish much in what follows between instances of sketch $S$ and functors $D: \catc \morph Fin$. Both, equally, represent instances of the skecth considered as a data specification.

\subsection{Requirements for Data Specifications}

In what follows we need some representation of the requirement for a data specification. Ultimately it is in respect of this requirement that a data specification is good or bad. It is sufficient to 
represent this requirement as a set of required data instances. 
In what follows a \term{requirement} for a data specification $S$ 
is a set of instances of the sketch $S$ or, equivalently, is a set $R_C$ of functors where for each
$D \in R_C$, $D: \catc \morph \Fin$, where \catcw is the category generated by the sketch $S$.

Two commensurate paths $p_1$ and $p_2$ are  said to be \term{equivalent with respect to a requirement $R$}, 
written $R \models p_1 \sim p_2$, iff 
for all instances $D \in R$, $D(p1)=D(p_2)$.

The following is a trivial consequence of these definitions.
\begin{lemma}
\label{pathequivalenceinference}
If $G$ is a directed graph and if $R$ is a requirement then 
\begin{itemize}
\item if \paralleldiag{a}{b}{p_1}{p_2} and $q: b \morph c$ are paths in $G$ then if $R \models p_1 \sim p_2$  
then $R \models p_1 \circ q \sim p_2 \circ q$,
\item if $p: a \morph b$ and \paralleldiag{b}{c}{q_1}{q_2} and $q: b \morph c$ are paths in $G$ then if $R \models q_1 \sim q_2$  
then $R \models p \circ q_1 \sim p \circ q_2$.
\end{itemize}
\end{lemma}

\subsection{Examples - Sketch of Category as Data Specification}

\newenvironment{graph} % graph with key alongside
{
\begin{tabular}{c p{1cm}c}
}
{
\end{tabular}
}


Is a presentation of a category really a notion of data specification? 
Well, as we might specify data in an XML schema, or in IDL or in a relational schema, then it isn't. But as we might outline
data in a preliminary design of such, say like shown in this fragment 
$
\begin{array}{c p{0.05cm}c p{0.5cm}c}
                        & & \rule[-0.3cm]{0pt}{0.8cm}\Rnode{p}{p}roject& &             \\ [0.3cm]
    project-wo\Rnode{w}{rker} & &                   & &  \\ [0.3cm]     
                         & & \Rnode{e}{e}mployee      & &             \\ [0.6cm]     
                         & & \Rnode{dep}{d}ependent  & &             
\end{array}
\begin{arrows}
\ncarr{w}{p} 
\alabel{R_1}[0.5][-1]
\ncarr{w}{e} 
\blabel{R_2}[0.5][-1]
\ncarr{dep}{e} 
\alabel{S_0}
\end{arrows}
$,
then it might be.

It also might be \textit{after we take away detail} from a complete specification to achieve an abstraction. Thus we can recognise
occurrences of directed graphs within larger and fully detailed data specifications wherever we can find occurrences
of this diagram:
$$
\begin{array}{c p{0.25cm} c  p{0.25cm} c }
\Rnode{e}{e} &&                   && \Rnode{n}{n} \\[0.4cm]
\end{array}
$$
\Etme[20]{e}{n}
\alabel{src}[0.75]
\Etme[-20]{e}{n}
\blabel{tgt}[0.75]

I mention this  because Bachman in his 1969 paper \textit{Data Structure Diagrams} \cite{Bachman1969}
enthuses over multiple occurrences of this shape appearing in  a larger data specification\footnote{Though his notation is different and in particular his arrows are the reverse of ours.}. 
%\subsubsection{Molecular Structure}
\subsubsection{Tabular Data}
\begin{graph}
$
\begin{array}{c p{0.25cm}c p{0.25cm}c}
             && \Rnode{t}{t}   &&              \\ [0.4cm]
\Rnode{r}{r} &&                && \Rnode{c}{c} \\ [0.4cm]     
             && \Rnode{d}{d} &&                    
\end{array}
\begin{arrows}
\ncarr{r}{t} 
\alabel{R_1}[0.5]%[-1]
\ncarr{c}{t} 
\blabel{R_2}[0.5]%[-1]
\ncarr{d}{r} 
\alabel{S_1}
\ncarr{d}{c}
\blabel{S_2} 
\end{arrows}
$
&&
\begin{key}
\keyentry{t} {table}  {a table of data}
\keyentry{r} {row}    {a row within a table}
\keyentry{c} {column} {a column within a table}                
\keyentry{d} {data}   {each data item is located in both a row and a column} 
\end{key}
\end{graph}



\subsubsection{Function Application}
In this example we give a simplified description of data representing a computer program
written in some modular programming language.\\ 
\begin{graph}
$
\begin{array}{c p{0.25cm}c p{0.25cm}c}
               && \Rnode{abs}{p} &&               \\ [0.7cm]  
               && \Rnode{T}{pm} &&               \\ [0.7cm]    
\Rnode{TL}{fc} &&               && \Rnode{TR}{f} \\ [0.8cm] 
\Rnode{BL}{ap} &&               && \Rnode{BR}{fp}   
\end{array}
\begin{arrows}
\ncarr{T}{abs} 
\alabel{R_0}
\ncarr{TL}{T} 
\alabel{R_1}
\ncarr{TR}{T} 
\blabel{R_2}
\ncarr{TL}{TR} 
\alabel{S_1}
\ncarr{BL}{BR} 
\blabel{S_2}
\ncarr{BL}{TL} 
\alabel{Q_1}
\ncarr{BR}{TR}
\blabel{Q_2} 
\end{arrows}
$
&&
\begin{key}
\keyentry{p}  {program}             {a computer program}
\keyentry{pm} {program module}      {part of a computer program ($R_0$)}
\keyentry{fc} {function call}       {point of use of a function in a program}
\keyentry{ap} {actual parameter}    {supply of a value to the formal parameter ($S_2$) of a function in a function call ($Q_1$)}
\keyentry{f } {function}            {the definition of a function within a program}                
\keyentry{fp} {formal parameter}    {defined for a function in its definition} 
\end{key} 
\end{graph} \\
Subject to path equivalences: 
\begin{equation}
\label{programquivalence1}
Q_1\circ S_1 \sim S_2 \circ S_1
\end{equation}
\begin{equation}
\label{programequivalence2}
R_1 \circ R_0 \sim S_1 \circ R_2 \circ R_0
\end{equation}

I ask the reader to interpret these path equivalences and what they say about the programming language. 
In practice we would also specify that functions and possibly formal parameters are named and that 
to each formal parameter there should correspond at most one actual parameter within any given function call.
This means that the pair of functions 
\nudgeup{0.9cm}\nudgedown{0.65cm} \binarysourcediag{D(ap)}{D(fc)}{D(fp)}{D(Q_1)}{D(S_2)}
will be jointly injective 
i.e. comprise a mono-source\footnote{Terminology that is used in the context of data specifications in Piersenns.} in  category $\Fin$.
Note also that in a programming language which does not support default values for formals and as a consequence
actual parameter values must be supplied for all formals (in a programming langauge which does not support default values for parameters) in every function call then in every instance $D$ of a valid program the diagram
\begin{displaymath}
\begin{array}{c p{0.25cm}c p{0.25cm}c} 
                  &&               &&                   \\[0.3cm]
\Rnode{TL}{D(fc)} &&               && \Rnode{TR}{D(f)}  \\[0.8cm] 
\Rnode{BL}{D(ap)} &&               && \Rnode{BR}{D(fp)} \\[0.3cm]
\end{array}
\begin{arrows}
\ncarr{TL}{TR} 
\alabel{D(S_1)}
\ncarr{BL}{BR} 
\blabel{D(S_2)}
\ncarr{BL}{TL} 
\alabel{D(Q_1)}
\ncarr{BR}{TR}
\blabel{D(Q_2)} 
\end{arrows}
\end{displaymath}
will be a pullback diagram in $\Fin$. Representation of pullbacks in data specifications will be a subject of a future note.
Fagin uses the term \term{projection-join dependency} in such situations and his projection-join normal form states when the such
 dependency is justified in a data specification. 

\subsubsection{Relational Meta-Model}
A data specification that describes the relational model of data (in other words, the data specification that is the relational meta-model) when viewed abstractly as a directed graph
includes, amongst others, nodes representing the concepts of table ($t$) , column ($c$), foreign key constraint ($fk$) and foreign key element ($fke$) as follows: \\
\vspace {0.5cm}
\begin{graph}
$
\begin{array}{p {1cm} c p{0.9cm} c p{0.7cm} c}
&                && \Rnode{abs}{rdb}  &&         \\[0.7cm]
&                && \Rnode{T}{t}      &&         \\[0.75cm]
&\Rnode{ML}{pke} && \Rnode{MC}{c}     &&  \Rnode{MR}{fk}  \\[0.7cm]
&                &&                   &&  \Rnode{BR}{fke}  \\[1.3cm]
\end{array}
\begin{arrows}
% composition
\ncarr{T}{abs}
\blabel{S_0}
\ncarr{ML}{T}
\alabel{S_1}[0.4]
\ncarr{MC}{T}
\blabel{S_2}[0.4]
\ncarr{MR}{T}
\blabel{S_3}[0.4]
\ncarr{BR}{MR}
\blabel{S_4}
% reference
\ncarr[-20]{ML}{MC}
\blabel{R_1}
\ncloop[nodesepA=4pt,angleB=180,nodesepB=3pt,loopsize=-2.0cm,armA=0.6cm,armB=3.0cm,linearc=0.2cm]{->}{MR}{T}
\naput[npos=2.5,labelsep=2pt]{\footnotesize $R_2$}
\ncloop[nodesepA=4pt,angleB=180,nodesepB=3pt,loopsize=-0.5cm,armA=0.3cm,armB=0.4cm,linearc=0.15cm]{->}{BR}{ML}
\nbput[npos=2.5,labelsep=2pt]{\footnotesize $R_3$}
\end{arrows}
$
&&
\begin{key}
\keyentry{rdb}{relational database}{}
\keyentry{t}{table}{a table within a relational database($S_0$)}
\keyentry{c}{column}{a column of a table ($S_1$)}
\keyentry{pke}{primary key element}{identifies a column ($R_1$)to be part of the primary key of a table}
\keyentry{fk}{foreign key}{implements a relationship between one table ($S_3$) and another ($R_2$)}
\keyentry{fke}{foreign key element}{associates a referencing column ($R?$) and a referred to column ($R_3$)}
\end{key}
\end{graph}

subject to the following path equivalences:
\begin{align}
\label{rdbR1scope}
&R_1 \circ S_2 = S_1, && \mbox{or, equivalently, }
\begin{array}{cp{0.75cm}c}
   \Rnode{t}{t}       & &              \\[1.2cm] 
   \Rnode{pke}{pke}   & & \Rnode{c}{c} \\[0cm]
                             & &               % horizontal spece needed    
\end{array}
\begin{arrows}
\ncarr{pke}{t} 
\alabel{S_1}
\ncarr{c}{t}
\blabel{S_2}
\ncarr{pke}{c}
\blabel{R_1}
\end{arrows} \mbox{ commutes,} \\
\label{rdbR2scope}
&R_2 \circ  S_0 = S_3 \circ S_0, && \mbox{or, equivalently, \ \ }
\begin{array}{cp{0.75cm}c}
                & \Rnode{rdb}{rdb} &                    \\[1.2cm]
\Rnode{Lt}{t}   &                  &                    \\[1.2cm]  
\Rnode{fk}{fk}  &                  & \Rnode{Rt}{t}       \\[0cm]
                &                  &               % horizontal spece needed    
\end{array}
\begin{arrows}
\ncarr{Lt}{rdb}
\alabel{S_0}
\ncarr{Rt}{rdb}
\blabel{S_0}
\ncarr{fk}{Lt}
\alabel{S_3}
\ncarr{fk}{Rt}
\blabel{R_2}
\end{arrows} \mbox{ commutes,} \\
\label{rdbR3scope}
&R_3 \circ S_1 = S_4 \circ R_2, && \mbox{or, equivalently, \ \ }
\begin{array}{cp{0.75cm}c}
   \Rnode{fk}{fk}     & & \Rnode{t}{t} \\[1.2cm]     
	\Rnode{fke}{fke}   & & \Rnode{pk}{pk}
\end{array}
\begin{arrows}
\ncarr{fk}{t} 
\alabel{R_2}
\ncarr{fke}{pk}
\blabel{R_3}
\ncarr{fke}{fk}
\alabel{S_4}
\ncarr{pk}{t}
\blabel{S_1}
\end{arrows} \mbox{ \ \ \ commutes.} 
\end{align}
It is a striking fact that we find these path equivalence constraints (aka commutivity constraints) right at the heart of the relational model of data. Nonetheless this type of constraint  is absent from relational data theory and this  despite the fact, as mentioned above, 
that they have a direct bearing on the construction of relational schemas in third normal form.
The discussion in Shlaer and Lang \cite{Shlaer96} is an exception. These diagrams, 
(\ref{rdbR1scope}), (\ref{rdbR2scope}) and (\ref{rdbR3scope}), follow a familiar pattern. In \cite{CartmellScopePaper} 
such diagrams are referred to as scope diagrams.



\subsection{Goodness Principles and Goodness Criteria}
\subsubsection{The Principle of No Redundancy}
If $S$ is a sketch for a category \catcw considered as a data specification with set of instances
$R_C$ then the the sketch $S$ then none of the edges or path equivalences in $S$ should be
redundant i.e. there should be no subsketch of $S$ which is a sketch of \catcw. 

Rationale: Redundant edges lead to redundant data. Redundant diagrams give rise to unnecessary program logic.

Example In the example given above of the sketch for the relational meta model the edge $R_?$ and the path equivalence ??? are redundant. Therefore this is not a 'good' data specification.  
\subsubsection{The Requirement}
\subsubsection{The Principle of Maximal Constrainedness}
Maximum constrainedness, as mentioned above, is a property of the category $\catc$ generated by the presentation  rather than of the presentation itself and is defined  relative to a requirement $\reqtc$ by which we mean a set of 
instances where each instance is a functor $D$, $D: \catc \morph \Fin$. In what follows, therefore,  by a requirement $\reqtc$ for category $\catc$ we mean a set  $\reqtc \subseteq | \Fin^{\catc} |$. 

Consider, a theory usually has some slack by which we mean that it has structurally compliant instances that are not part of its requirement.  The definition of maximal constrainedness expresses that a theory is maximally constrained to its requirement if there is no way of extending the theory so as to rule out possible structurally compliant instances that are not part of the requirement (i.e. to rule out slack) whilst remaining consistent with the requirement.

The definition now follows, preceded by an auxiliary definition.
\begin{definition}
If $\catc$ is a category and $\reqtc$ is a requirement for $\catc$,  if $I: \catc \morph \catcp$ is a functor then say that $I$ is \term{consistent with} requirement $\reqtc$ iff for all instances $D \in \reqtc$ there exists a functor $D':\catcp \morph \Fin$ such that $I \circ D'=D$.
\end{definition}
\begin{definition}
If $\catc$ is a category and $\reqtc$ is a requirement for $\catc$ then $\catc$ is \term{maximally constrained} to the requirement $\reqtc$ iff for all categories $\catcp$ and for all functors $I:\catc \morph \catcp$ that are consistent with $\reqtc$, for all functors $F: \catc \morph \Fin$  there exists an $F' : \catcp \morph \Fin$ such that $I \circ F'=F$.
\end{definition}

\textbf{Goodness Principle 2:}
If \catcw is a category and $R_C$ is a set of instances then \catcw should be maximally constrained to $R_C$. 

\subsubsection{Path Equivalence and Completeness}

\newcommand{\fgparalleldiag}
{
 $
\rule[-0.3cm]{0pt}{0.9cm} %to add vertical space of diagram -- based on lowering diagram 0.3cm and heght 0.9cm
                            % change thickness from 0pt to 1 pt to debug
\begin{array}{c p{0.5cm} c  }
 \Rnode{a}{a}            &&   \Rnode{b}{b}
\end{array} 
\begin{arrows}
\ncarc[nodesep=2pt,arcangle=10,offset=2pt]{->}{a}{b}
\alabel{f}
\ncarc[nodesep=2pt,arcangle=-10,offset=-2pt]{->}{a}{b}
\blabel{g}
\end{arrows}
$  
}

\newcommand{\pequiv}[1][R_C]{\underset{#1}{\equiv}}

\begin{definition}
If $\catcw$ is a  category, if $\reqtc$ is a set of instances
 and if \fgparalleldiag in $\catc$, then say that path $f$ is equivalent to path $g$ with respect to the requirement $R_C$ 
 (and write $f \pequiv g$) iff
in all instances $D \in \reqtc$, $D(f)=D(g)$.
\end{definition}

\begin{definition}
If $\catc$ is a  category and $\reqtc$ is a set of instances,
 and if \fgparalleldiag in $\catc$ such that $f \pequiv g$
 then say that the path equivalence $f \pequiv g$ is represented in \catcw iff
 $f=g$.
\end{definition}

\begin{oldtt}
\begin{definition}
If $\catc$ is a  category and $\reqtc$ is a set of instances,
 then say that  $\catc$ is \term{logically complete} with respect 
to the requirement $\reqtc$ iff all path equivalences with respect to $R_C$ are represented in \catcw 
i.e. iff for all diagrams \fgparalleldiag in $\catc$,  
if in all instances $D \in \reqtc$, $D(f)=D(g)$,  then $f=g$ in $\catc$.
\end{definition}
\end{oldtt}

\textbf{Goodness Principle 2A:} If \catcw is a catgeory and $R_C$ is a requirement then \catc should be complete
with respect to $R_C$ that is all path equivalences with respect to $R_C$ as a requirment for \catcw should be represented in \catc.

\subsubsection{The Principle of the Representation of Functional Dependencys}

\begin{definition}
If $\catc$ is a category and $\reqtc$ is a set of instances and if \fgsourcediag
in $\catc$ then there is a  \term{functional dependency} of $g$ on $f$ with respect to $\reqtc$ iff
there is a family of functions $H_D)_{D \in \reqtc}$ such that 
in each instance $D$, $H_D$ is a unique function $H_D: D(b) \morph D(c)$, such that $D(f) \circ H_D = D(g)$. 
If there is such a functional dependency then we say that $\fundep{H}{f}{g}$ in $\catc$ with respect to $\reqtc$.
\end{definition}

Our use of the $\morph$ notation for functional dependencies here is coming from relational database theory where it is usual to represent such a functional dependency as we have here by asserting that 
$$
f \morph g
$$
Note that this use of an $\morph$ notation is independent of our use of $\morph$ as a morphism of a category 
or, for that matter, as an edge in a presentation. Neither are we alluding to a bicategory structure. We have two distinct uses for $\morph$ (three if you distinguish arrows in presentations from arrows in categories). Any particular use will be unambiguous in context.

\begin{definition}
If $\catc$ is a category and $\reqtc$ is a set of instances, if
\fgsourcediag
in $\catc$ 
and if there is a functional dependency $\fundep{H}{f}{g}$ then say that 
the functional dependency $H$ is \term{represented} in $\catc$ 
iff there exists a morphism $h:b \morph c$ in $\catc$ such that for each instance $D \in \reqtc$, $D(h)=H_D$.
\end{definition}

\textbf {Goodness Principle 2B:} If $\catc$ is a category and $\reqtc$ is a set of instances, then all functional dependencies
should be represented.


\subsubsection{The Principle of the Representation of Referential Inclusion Dependencies}

\begin{definition}
If $\catc$ is a category and $\reqtc$ is a set of instances 
and if
\fnsourceqnsource
in $\catc$, then a \term{referential inclusion dependency} $I$, written $a[f_1,...f_n] \overset{I}{\subseteq} c[q_1,..q_n]$, is a family of functions $I_D)_{D \in \reqtc}$
such that each instance $D \in \reqtc$, $I_D$ is a unique function $I_D : D(a) \morph D(c)$ such that
for each $i$, $1 \leq i \le n$, $I_D \circ D(q_i) = D(f_i)$.
\end{definition}

\begin{definition}
If $\catc$ is a category and $\reqtc$ is a set of instances and if
\fnsourceqnsource
in $\catc$ and if $a[f_1,...f_n] \overset{I}{\subseteq} c[q_1,..q_n]$ is a referential inclusion dependency
with respect  to $\reqtc$ then say that the inclusion dependency $I$ is \term{represented} in $\catc$
iff there exists a morphism $i:a \morph c$ in $\catc$ such that in each instance $D \in \reqtc$, $D(i) = I_D$. 
\end{definition}

\textbf {Goodness Principle 2C:} If \catcw is a category and $\reqtc$ is a set of instances, then all referential inclusion dependencies present in $R_C$
should be represented in \catc.
\subsection{Representation Lemmas}
We now show that if we assume local finiteness of the category \catcw generated by a sketch $S$   having a data specification requirement represented by a set of
instances $R_C$ then  if principle 2 (maximal constrainedness) is met then
\newt{goodness criteria} 2A, 2B and 2C are met also. 


Phrasing this differently: for a data specification to be maximally constrained to a
requirement then  all commutivity constraints,  functional dependencies and referential inclusion dependencies arising from the requirement must be represented in the data specification.

\subsubsection{Representation of Path Equivalences}

\begin{lemma}
\llabel{pathequivalencerepresentationlemma}
If $\catc$ is a locally finite category and $\reqtc$ is a set of instances, if $\catc$ is 
\term{maximally constrained} to the requirement $\reqtc$ then all path equivalences with respect
to $R_C$ are represented in \catcw
i.e. for all diagrams
$
\rule[-0.3cm]{0pt}{0.9cm} %to add vertical space of diagram -- based on lowering diagram 0.3cm and heght 0.9cm
                            % change thickness from 0pt to 1 pt to debug
\begin{array}{c p{0.5cm} c  }
 \Rnode{a}{a}            &&   \Rnode{b}{b}
\end{array} 
$
\ncarc[nodesep=2pt,arcangle=10,offset=2pt]{->}{a}{b}
\alabel{f}
\ncarc[nodesep=2pt,arcangle=-10,offset=-2pt]{->}{a}{b}
\blabel{g}
in $\catc$,  if in all instances $D \in \reqtc$, $D(f)=D(g)$, 
then $f=g$ in $\catc$.
\end{lemma}
\begin{proof}
Suppose such a category  $\catcw$  that  is 
\term{maximally constrained} to a requirement $\reqtc$
and suppose 
$
\rule[-0.3cm]{0pt}{0.8cm} %to add vertical space of diagram -- based on lowering diagram 0.3cm and heght 0.9cm
                            % change thickness from 0pt to 1 pt to debug
\begin{array}{c p{0.5cm} c  }
 \Rnode{a}{a}            &&   \Rnode{b}{b}
\end{array} 
$
\ncarc[nodesep=2pt,arcangle=10,offset=2pt]{->}{a}{b}
\alabel{f}
\ncarc[nodesep=2pt,arcangle=-10,offset=-2pt]{->}{a}{b}
\blabel{g}
in $\catc$. From sketch $S$ of $C$ we can construct a sketch $S'$ by formally adding a path equivalence $f=g$.
$S'$ generates a category $C'$ for which we have a functor $I: \catc \morph \catcp$. 
Because $D(f) = D(g)$ in every instance $D \in \reqtc$
it follows that $I$ is consistent with $\reqtc$. Because $\catc$ is maximally constrained to $\reqtc$
and because $I: \catc \morph \catcp$ is consistent with $\reqtc$ it follows that the functor $Hom_{\catc}(a,-): \catc \morph \Fin$ 
can be extended to a functor $F: \catcp \morph \Fin$. Since $I(f)=I(g)$ in $\catcp$ then $F(I(f))=F(I(g))$. But $I \circ F
= Hom_\catc(a,-)$ therefore we have that $Hom(a,f)=Hom(a,g)$ and, applying both sides to $id_a$, that $f=g$ in $C$ as required.
\end{proof}

In this lemma above we have assumed that \catcw is locally finite. Can we not prove this result for all categories? The following example shows not. 
\begin{example}
Suppose \catcw is the category generated by the sketch with directed graph
\begin{displaymath}
\begin{array}{cp{1.4cm}c}
                                    \\[0.1cm]
\Rnode{a}{a}	&& \Rnode{b}{b}     \\[0.25cm]
	            &&  
\end{array}
\begin{arrows}
\ncarr[15]{a}{b}
\alabel{f}[0.35]
\ncarr[-15]{a}{b}
\blabel{g}[0.35]
\ncarr[-70]{a}{b}
\blabel{h'}[0.35]
\ncarr[-70]{b}{a}
\blabel{h}[0.35]
\nccircle[angleA=-90, nodesep=3pt]{->}{b}{.5cm}
\blabel{r}[0.3]
\end{arrows}
\end{displaymath}

subject to the identities
\begin{equation}
\label{fhidentity}
f \circ h = id_a
\end{equation}
\begin{equation}
\label{ghidentity}
g \circ h = id_a
\end{equation}
and 
\begin{equation}
\label{rhhpidentity}
r \circ h \circ h' = id_b
\end{equation}

We can show that for any functor $D:\catc \morph Fin$, $D(f)=D(g)$. For any requirement 
$R_C$, therefore, $f$ is equivalent to $g$ with respect to $R_C$ and as $f \neq g$ this equivalence of $f$ and $g$ 
is not represented in \catc. Lemma \lref{pathequivalencerepresentationlemma} cannot therefore be made applicable to all categories (or to all finitely presented categories) as we might wish.

To show that if $D:\catc \morph Fin$ then $D(f)=D(g)$ 
first note that we have that $D(r) \circ D(h \circ h')=id_{D(b)}$ because this follows from
identity (\ref {rhhpidentity}) and from the functoriality of $D$, Since 
the set $D(b)$ is finite it follows that $D(r): D(b) \morph D(b)$ is surjective and injective 
and that the function $D(h \circ h'):D(b) \morph D(b)$ is surjective and injective and from this later
that the function $D(h)$ is injective.
Because of identities (\ref{fhidentity}) and (\ref{ghidentity}) and from the 
functoriality of $D$ we have that $D(f) \circ D(h) =id_{D(a)}=D(g) \circ D(h)$. Because $D(h)$ is injective it follows 
that $D(f)=D(g)$, as required.
\end{example}

\subsubsection{Representation of Functional Dependencies}
\begin{lemma}
\llabel{functionaldependencyrepresentationlemma}
If $\catc$ is a locally finite category and $\reqtc$ is a set of instances, if $\catc$ is 
\term{maximally constrained} to the requirement $\reqtc$ then
all functional dependencies $\fundep{H}{f}{g}$  with respect to $\reqtc$ are represented in $\catc$.
\end{lemma}
\begin{proof}
Suppose such a category  $\catc$  that  is 
\term{maximally constrained} to a requirement $\reqtc$ and suppose
$
\begin{array}{c p{0.5cm} c  }
             &&   \Rnode{b}{b} \\[0.01cm]
\Rnode{a}{a} &&                \\[0.01cm] 
             &&   \Rnode{c}{c}         
\end{array} 
$
\ncarr{a}{b}
\alabel{f}
\ncarr{a}{c}
\blabel{g}
in $\catc$ 
and that there is a functional dependency $\fundep{H}{f}{g}$ with respect to $\reqtc$.

From sketch $S$ of $C$ we can construct a sketch $S'$ by formally adding a morphism $\qq{h}: b \morph c$
and path equivalence $f \circ \qq{h} = g$. Let $\catcp$ be the category generated by $S'$ and
let $I: C \morph C'$ be the inclusion functor generated by the inclusion of $S$ in $S'$. 

\begin{newtt}
Now we show that $I$ is consistent with $\reqtc$. 
We need to show that any $D \in \reqtc$
extends unqiuely to $D' :\catcp \morph \Fin$. Assume such a $D$. 
$D$  extends to $C'$ iff there is a  function that we can choose as the value of  for $D'(\qq{h})$  such that $D'(f) \circ D'(\qq(h) = D'(g)$ i.e such that
$D(f) \circ D'(\qq(h) = D(g)$ and so we  extend $D$ to $D'$ 
 by defining $D'(\qq{h})=D_H$.
This extension to $D'$ is unique because from the definition of functional depdendency
$D_H$ is the unique function that satisfies $D(f) \circ D_H = D(g)$.
\end{newtt}

\begin{newtt}
In the remainder of this proof 
we define a functor $G: \catc \morph \Fin$ such that
$G(a)$ has a pair of distinct elements, call them $left$ and $right$, say,  that are mapped by $G(f)$ to identical elements
of $G(b)$, i.e. such that $G(f)(left)= G(f)(right)$ 
and such that $G$ is initial with respect to such functors.  
We  argue that because of  maximal constrainedness that $G$ can be extended to a functor 
$G' : \catcp \morph \Fin$. 
Because $f \circ \qq{h} = g$ in \catcpw we can argue that $G'(g)$,
and thus $G(g)$,
maps $left$ and $right$ to identical elements of $G(c)$ 
i.e. is such that $G(g)(left)= G(g)(right)$. 
The initiality of $G$, the fact that  $G(g)(left)= G(g)(right)$ follows
from $G(f)(left)= G(f)(right)$, is then enough 
to ensure that there exists a morphism $h:b \morph c$ in \catc
such that $f \circ g = h$
 with the required properties to represent the functionality dependency $H$.
\end{newtt}

First define the functor $F: \catc \morph \Fin$ be the coproduct $Hom_{\catc}(a,-) + Hom_{\catc}(a,-)$
in the functor category $Fin^{\catc}$ and label the injections $L$ and $R$, respectively so that
for each object $x$ of $\catc$ the diagram
\begin{center}
$
\begin{array}{c p{0.5cm} c p{0.5cm} c  }
\Rnode{h1}{Hom_{\catc}(a,x)}  &&\Rnode{Fx}{F(x)}  &&   \Rnode{h2}{Hom_{\catc}(a,x)}       
\end{array} 
$
\ncarr{h1}{Fx}
\alabel{L_x}
\ncarr{h2}{Fx}
\blabel{R_x}
\end{center}
is a coproduct in $\Fin$. $F$ is a functor such that $F(a)$ has two elements
$L(id_a)$ and $R(id_a)$ and is initial among such functors.

Now for each object $x$ of $\catc$, we define an equivalence relation $\sim_x$ on $F(x)$ by defining,
for $k_1,k_2:a \morph x$ in $\catc$,
\begin{align*}
L_x(k_1) \sim_x R_x(k_2) & \mbox{ iff there exists $k:b \morph x$ in $\catc$ such that $k_1 = f \circ k = k_2$,}\\
L_x(k_1) \sim_x L_x(k_2) & \mbox{ iff $k_1 = k_2$,} \\
R_x(k_1) \sim_x R_x(k_2) & \mbox{ iff $k_1 = k_2$.} \\
\end{align*}
If $j: x_1 \morph x_2$ in $\catc$ and if $y_1,y_2 \in F(x_1)$ such that $y_1 \sim_{x_1} y_2$
then it follows easily by cases and from the definition of $\sim$ that $F(j)(y_1) \sim_{x_2} F(j)(y_2)$.
Therefore we can define a functor 
 $G: \catc \morph \Fin$  so that for any object $x$ of $\catc$
the set $G(x)$ is the quotient $F(x)/{\sim_x}$ and such that 
if $j: x_1 \morph x_2$ in $\catc$ and if $y \in F(x_1)$ then $G(j)([y])=[F(j)(y)]$.
With $G$ so defined then if $k: a \morph x_1$ and $j:x_1 \morph x_2$ in $\catc$
then  $G(j)([L_{x_1}(k)])=[L_{x_2}(k \circ j)]$ and $G(j)([R_{x_1}(k)])=[R_{x_2}(k \circ j)]$. 

Now that we have described the functors  $G: \catc \morph \Fin$ and $I:\catc \morph \catcp$ that is consistent with $\reqtc$
we can use the fact that $\catc$ is maximally constrained to tell us that $G$ extends to a functor 
$G' : \catcp \morph \Fin$. Since $f \circ \qq{h} = g$ in $\catcp$ then we have
 $G'(f) \circ G'(\qq{h}) = G'(g): G(a) \morph G(c)$ in $\Fin$.
Now we have
\begin{align*}
[L_c(g)]&= G'(g)([L_a(id_a)])              & & \mbox{by definition of $G'(g)=G(g)$}           \\
        &= G'(\qq{h}) (G'(f)([L_a(id_a)])) & & \mbox{since $G'(f) \circ G'(\qq{h}) = G'(g)$}  \\
	&= G'(\qq{h}) ([L_b(f)])       & & \mbox{by definition of $G'(f)=G(f)$}           \\
	&= G'(\qq{h}) ([R_b(f)])       & & \mbox{since $L_b(f) \sim_b R_b(f)$, by definition of $\sim_b$} \\
	&= G'(\qq{h}) (G'(f)([R_a(id_a)])) & & \mbox{by definition of $G'(f)=G(f)$}           \\
	&= G'(g)([R_a(id_a)])              & & \mbox{since $G'(f) \circ G'(\qq{h}) = G'(g)$}  \\
				&= [R_c(g)]                        & & \mbox{by definition of $G'(g)=G(g)$}           \\
\end{align*} 
Since $[L_c(g)]=[R_c(g)]$ then it follows from the definition of $\sim_c$ that there exists $k:b \morph c$ in 
$\catc$ such that $f \circ k = g$.  \newt{Need show uniqueness of such $k$.}  We have shown as required that the function dependency
$\fundep{H}{f}{g}$ is represented in $\catc$.

\end{proof}

In this lemma above we have assumed that \catcw is locally finite. Can we not prove this result for all categories? A development of the previous examples shows not.
\begin{example}
Suppose \catcw is the category generated by the directed graph
\begin{displaymath}
\begin{array}{cp{0.6cm}cp{0.6cm}c}
                                                           \\[0.1cm]
                &&  \Rnode{x}{x}       &&                  \\[-0.2cm]
\Rnode{a}{a}	&&                     && \Rnode{b}{b}     \\[0.25cm]
	            &&                     &&
\end{array}
\begin{arrows}
\ncarr[10]{a}{x}
\alabel{f_1}[0.35]
\ncarr[10]{x}{b}
\alabel{f_2}[0.35]
\ncarr[-15]{a}{b}
\blabel{g}[0.35]
\ncarr[-70]{a}{b}
\blabel{h'}[0.35]
\ncarr[-75]{b}{a}
\blabel{h}[0.35]
\nccircle[angleA=-90, nodesep=3pt]{->}{b}{.5cm}
\blabel{r}[0.3]
\end{arrows}
\end{displaymath}

subject to the identities
\begin{equation}
\label{fdcounterfhidentity}
f_1 \circ f_2 \circ h = id_a
\end{equation}
\begin{equation}
\label{fdcounterghidentity}
g \circ h = id_a
\end{equation}
and 
\begin{equation}
\label{fdcounterrhhpidentity}
r \circ h \circ h' = id_b
\end{equation}

As in the previous example, we can show that for any functor $D:\catc \morph Fin$, $D(f1 \circ f_2)=D(g)$. For any requirement 
$R_C$, therefore,  $g$ is functionally dependent on $f_1$ 
with respect to $R_C$. This functional dependency  
is not represented in \catc. \commentary{We also have to prove the claim that the functional dependency is not represented in \catc. }
Therefore Lemma \lref{functionaldependencyrepresentationlemma} cannot therefore be made applicable to all categories (or to all finitely presented categories).

\end{example}

\subsubsection{Representation of Referential Inclusion Dependencies}
\begin{lemma}
\label{catincdsrepresented}
If $\catc$ is a locally finite category and $\reqtc$ is a set of instances, if $\catc$ is 
\term{maximally constrained} to the requirement $\reqtc$ then
every referential inclusion dependency with respect to $\reqtc$ is represented in $\catc$.
\end{lemma}
\begin{proof}
Suppose such  a category $\catc$ and  requirement $\reqtc$ 
 and suppose there is a referential inclusion dependency
$a[f_1,...f_n] \overset{I}{\subseteq} b[q_1,..q_n]$ with respect to $\reqtc$,
where
$
\begin{array}{c p{0.25cm} c  p{0.25cm} c }
             &&   \Rnode{b1}{b_1} &&              \\[0.4cm]
\Rnode{a}{a} &&                   && \Rnode{b}{b} \\[0.4cm]
             &&   \Rnode{bn}{b_n} &&              
\end{array} 
$
\ncarr{a}{b1}
\alabel{f_1}
\ncarr{b}{b1}
\blabel{q_1} 
\ncarr{a}{bn}
\blabel{f_n}
\ncarr{b}{bn}
\alabel{q_n}
in $\catc$, 

From sketch $S$ of $C$ we can construct a sketch $S'$ by formally adding a morphism $\qq{f}: a \morph b$
and path equivalences, for each $i$, $1 \leq i \leq n$, $\qq{f} \circ q_i = f_i$. Let $\catcp$ be the category generated by $S'$ and
let $I: C \morph C'$ be the inclusion functor generated by the inclusion of $S$ in $S'$. $I$ is consistent with $\reqtc$ since
each instance $D \in  \reqtc$ can be \newt{uniquely} extended to a functor $D': \catcp \morph \Fin$ by defining $D'(\qq{f})=I_D$.

Since $\catc$ is locally finite then there is a functor $Hom_{\catc}(a,-): \catc \morph \Fin$ and since $\catc$ is maximally constrained and because $I$ is consistent with $\reqtc$ then there is a functor $F': C' \morph \Fin$
such that $I \circ F' = Hom_{\catc}(a,-)$. In particular we must have 
$F'(a)=Hom_{\catc}(a,a)$,
$F'(b)=Hom_{\catc}(a,b)$ and therefore $F'(\qq{f}):Hom_{\catc}(a,a) \morph Hom_{\catc}(a,b)$ which gives us
$F'(\qq{f})(id_a): a \morph b$ in $\catc$. 

\newt{Do we need show uniqueness of someat?}

We can show that $F'(\qq{f})(id_a)$ represents
the inclusion dependency $a[f_1,...f_n] \overset{I}{\subseteq} b[q_1,..q_n]$ by showing that for 
each $i$, $1 \leq i \leq n$, $F'(\qq{f})(id_a) \circ q_i = f_i$. This follows because we have defined
the sketch $S'$ so that in category $C'$ we have $\qq{f} \circ q_i = f_i$. From this it follows
that $F'(\qq{f}) \circ F'(q_i) = F'(f_i)$ i.e. $F'(\qq{f}) \circ Hom(a,q_i) = Hom(a,f_i)$. 
Applying left and right hand sides to $id_a$ we get $F'(\qq{f})(id_a) \circ q_i = f_i$, as required.
\end{proof}


\subsection{Rephrasing}


\subsubsection{Regarding Jointly-Injectives and Inclusion Dependencies}

\begin{lemma}
If $\catc$ is a category and $\reqtc$ is a set of instances and if
\fnsourceqnsource
in $\catc$
such that $a[f_1,...f_n] \overset{I}{\subseteq} c[q_1,..q_n]$ is an inclusion dependency with respect  to $\reqtc$ then if in each instance $D \in \reqtc$ the family of functions
$q_{i, 1 \leq i \leq n}$ is jointly-injective then $a[f_1,...f_n] \overset{I}{\subseteq} c[q_1,..q_n]$ is a referential inclusion dependency.
\end{lemma}
\begin{proof}
Straightforward.
\end{proof}

\iffalse
\newcommand {\qnsourcediag}{
$
\begin{array}{c p{0.5cm} c  }
             &&   \Rnode{b}{b_1} \\[0.01cm]
\Rnode{a}{a} &&                \\[0.01cm] 
             &&   \Rnode{c}{b_n}         
\end{array} 
\begin{arrows}
\ncarr{a}{b}
\alabel{q_1}
\ncarr{a}{c}
\blabel{q_n}
\end{arrows}
$  
}
\fi




\appendix

\section{Further Analysis}
\subsection{Representation of Binary Joint Functional Dependencies}
\newcommand{\representjointfd}[3]
{
for some $n$, $n \geq 0$, there exists morphisms $w_1,..w_{2n+1}:b \morph #1$ in $\catc$ such that 
 $f_1 \circ w_1 = #2$
 and
$f_2 \circ w_{2n+1} = #3$
and \foreachi, 
                       $f_2 \circ w_{2i-1} = f_2 \circ w_{2i}$
                  and  $f_1 \circ w_{2i} = f_1 \circ w_{2i+1}$}
% End of \representjointfd
\begin{lemma}
\llabel{binaryjointfunctionaldependencyrepresentationlemma}
If $\catc$ is a locally finite category and $\reqtc$ is a set of instances, 
if $\catc$ is \term{maximally constrained} to the requirement $\reqtc$ 
then if $a_1$, $a_2$, $b$ and $c$ are objects of \catcw and   
  $f_1: a_1 \morph b$,  $f_1: a_1 \morph b$, 
$g_1,: a_2 \morph c$ and $g_2: a_2 \morph c$ are morphisms
so that we have 
$
\begin{array}{c p{0.65cm} c  p{0.75cm} c }
\Rnode{a1}{a_1}                  \\[0.5cm]
                 && \Rnode{b}{b} && \Rnode{c}{c}\\[0.5cm]
\Rnode{a2}{a_2}       
\end{array}
\begin{arrows} 
\ncarr{a1}{b}
\blabel{f_1}[0.4][0.1]
\ncarr[15]{a1}{c}
\alabel{g_1}[0.5][0.2]
\ncarr{a2}{b}
\alabel{f_2}[0.4][0.1]
\ncarr[-15]{a2}{c}
\blabel{g_2}
\end{arrows}
$
in \catcw
then if for each $D \in \reqtc$ there is a unique function
$H_{D}: D(b) \morph D(c)$ such that 
$D(f_1) \circ H_D = D(g_1)$ and  $D(f_2) \circ H_D = D(g_2)$
then \representjointfd{c}{g_1}{g_2}.

To illustrate this, in the $n=1$ case we will have the following commuting diagrams in \catc
$$
\begin{array}{c p{0.65cm} c  p{1.0cm} c }
\Rnode{a11}{a_1}                               \\[0.5cm]
               && \Rnode{b1}{b}                \\[0.3cm]
\Rnode{a21}{a_2}                               \\[0.3cm]
               && \Rnode{b2}{b} && \Rnode{c}{c}\\[0.3cm]
\Rnode{a12}{a_1}                               \\[0.3cm]
               && \Rnode{b3}{b}                \\[0.5cm]
\Rnode{a22}{a_2}       
\end{array} 
\begin{arrows} 
\ncarr{a11}{b1}
\blabel{f_1}[0.4][0.1]
\ncarr{a21}{b1}
\alabel{f_2}[0.4][0.1]
\ncarr{a21}{b2}
\blabel{f_2}[0.4][0]
\ncarr{a12}{b2}
\alabel{f_1}[0.4][0.1]
\ncarr{a12}{b3}
\blabel{f_1}[0.4][0.1]
\ncarr{a22}{b3}
\alabel{f_2}[0.4][0] 
\ncarr{b1}{c}
\blabel{w_1}[0.4][0]
\ncarr{b2}{c}
\alabel{w_2}[0.3][0.1]
\ncarr{b3}{c}
\alabel{w_3}[0.4][0] 
\ncarr[30]{a11}{c}
\alabel{g_1}[0.5][0.2]
\ncarr[-30]{a22}{c}
\blabel{g_2}
\end{arrows}
$$
\end{lemma}
\begin{proof}
We can construct a sketch $S'$ extending sketch $S$ 
 by formally adding a morphism $\qq{h}: b \morph c$
and path equivalences $f_1 \circ \qq{h} = g_1$ and $f_2 \circ \qq{h} = g_2$
Let $\catcp$ be the category generated by $S'$ and
let $I: \catc \morph \catcp$ be the inclusion functor generated 
by the inclusion of $S$ in $S'$. 

We can show that $I$ is consistent with $\reqtc$
by showing that any $D \in \reqtc$
extends uniquely to $D' :\catcp \morph \Fin$. Assume such a $D$. 
$D$  extends to \catcpw iff there is a  function that 
we can choose as the value of  for $D'(\qq{h})$  
such that for $i$ being both $1$ and $2$,  $D'(f_i) \circ D'(\qq{h}) = D'(g_i)$ 
i.e  $D(f_i) \circ D'(\qq{h}) = D(g_i)$ and so we  extend $D$ to $D'$ 
 by defining $D'(\qq{h})=H_D$.
This extension to $D'$ is unique because we are given that
$H_D$ is the unique function that satisfies 
for $i$ being both $1$ and $2$, $D(f_i) \circ H_D = D(g_i)$.

Let the functor $F: \catc \morph \Fin$ be the coproduct $Hom_{\catc}(a_i,-) + Hom_{\catc}(a_i,-)$
in the functor category $Fin^{\catc}$ and label the injections $L$ and $R$, respectively so that
for each object $x$ of $\catc$ the diagram
\begin{center}
$
\begin{array}{c p{0.5cm} c p{0.5cm} c  }
\Rnode{h1}{Hom_{\catc}(a_1,x)}  &&\Rnode{Fx}{F(x)}  &&   \Rnode{h2}{Hom_{\catc}(a_2,x)}       
\end{array} 
$
\ncarr{h1}{Fx}
\alabel{L_x}
\ncarr{h2}{Fx}
\blabel{R_x}
\end{center}
is a coproduct in $\Fin$.
Now for each object $x$ of $\catc$, we define an equivalence relation $\sim_x$ on $F(x)$ as follows
\begin{align*}
L_x(k_1) \sim_x R_x(k_2) &&&
 \parbox[t]{8cm}{ iff \representjointfd{x}{k_1}{k_2},}\\
R_x(k_1) \sim_x L_x(k_2) &&& \mbox{ iff $L_x(k_2) \sim_x R_x(k_1)$,} \\
L_x(k_1) \sim_x L_x(k_2) &&&
 \parbox[t]{8cm}{ iff there exists $w,w':b \morph x$ 
                  such that $L_x(k_1) \sim_x R_x(f_2 \circ w)$
                        and $f_2 \circ w = f_2 \circ w'$
                        and $f_1 \circ w' = k_2$,} \\
R_x(k_1) \sim_x R_x(k_2) &&&
 \parbox[t]{8cm}{ iff there exists $w,w':b \morph x$ 
                  such that $R_x(k_1) \sim_x L_x(f_1 \circ w)$
                        and $f_1 \circ w = f_1 \circ w'$
                        and $f_2 \circ w' = k_2$,} \\ 
\end{align*}
With $\sim_x$ so defined then it follows that $\sim_x$ is an eqivalence relation i.e. is symmetric and transitive.  
Also if $j: x_1 \morph x_2$ in \catcw and if $y_1,y_2 \in F(x_1)$
such that $y_1 \sim_{x_1} y_2$
then it follows easily  from the definition of $\sim$
that $F(j)(y_1) \sim_{x_2} F(j)(y_2)$.
Therefore we can define a functor 
$G: \catc \morph \Fin$  so that for any object $x$ of %\catc
the set $G(x)$ is the quotient $F(x)/{\sim_x}$.
Aside: I have defined $\sim_x$ as I have done so as to achieve a minimum relation 
for which $L_a(f_1) \sim_x R_a(f_2)$ and 
which is also symmetric, transitive and preserved by $F(j)$, for each $j$, as described. 

\begin{align*}
[L_c(g_1)]
&= G'(g_1)([L_a(id_a)])     
 	& & \mbox{by definition of $G'(g_1)=G(g_1)$ 
 			            and since $G(g_1)([L_a(id_a)])=[L_c(g_1)]$,}\\
&= G'(\qq{h}) (G'(f_1)([L_a(id_a)])) 
	& & \mbox{since $G'(f_1) \circ G'(\qq{h}) = G'(g_1)$,}  \\
&= G'(\qq{h}) ([L_b(f_1)])       
	& & \mbox{by definition of $G'(f_1)=G(f_1)$
	                            and since $G(f_1)([L_a(id_a)]) = [L_b(f_1)]$,} \\
&= G'(\qq{h}) ([R_b(f_2)])       
	& & \mbox{since $L_b(f_1) \sim_b R_b(f_2)$, by case $n=1$ of the definition of $\sim_b$,} \\
&= G'(\qq{h}) (G'(f_2)([R_b(id_b)])) 
	& & \mbox{by definition of $G'(f_2)=G(f_2)$ 
	                            and since $G(f_2)([R_b(id_b)]) = [R_b(f_2)]$,}\\
&= G'(g_2)([R_b(id_b)])              
	& & \mbox{since $G'(f_2) \circ G'(\qq{h}) = G'(g_2)$,}  \\
&= [R_c(g_2)]                        
	& & \mbox{by definition of $G'(g_2)=G(g_2)$
 			            and since $G(g_2)([R_b(id_b)])=[L_c(g_2)]$.}   
\end{align*} 
and since $[L_c(g_1)]=[R_c(g_2)]$ then it follows from the definition of $\sim_c$ that 
\representjointfd{c}{g_1}{g_2}, as required.
\end{proof}




\bibliographystyle{alpha} 
\bibliography{../../SharedBibliography/temp/bibliography}
\end{document}
