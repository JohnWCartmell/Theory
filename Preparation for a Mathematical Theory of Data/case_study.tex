It is interesting to note that in relational design the requirement to meet the usual goodness criteria and in particular, say, to meet  BCNF, 
does not rule out a column of a table   holding a number which is, in all instances, a square of a natural number. Though
such a data representation would not necessarily fail any of the classic goodness criteria, it is counter to the intuition we have of good data design. It seems that principles of no redundancy and maximum constrainedness are a basis for formalising this intuition as we now illustrate by way of a case study.

Suppose a sketch $S$ has  this graph:
$$
\Rnode{e}{e}
$$
\rEtme[270]{e}
\alabel{u}
and no path equivalences and let $\catc$ be the category presented by $S$. This is the theory of a
set along with a single unary operation (a \term{u-structure} in the terminology of \cite{BarrandWells}). 
Suppose in every instance $D$ of a requirement $\reqtc$ for $\catc$, the function $D(u)$ is a permutation i.e. bijection. 
In this case  
$\catc$ is not maximally constrained to $\reqtc$ (i.e. it fails the goodness criteria) because, since every function $D(u)$ had a unique inverse, there is a functional dependency 
$u \morph id_e$ in  $\catc$ with respect to $\reqtc$ and this functional dependency is not represented in $\catc$. 
In order to meet the goodness criteria, $S$ can be extended to a sketch $S'$ having an additional arrow $v:e \morph e$ to represent this functional
dependency and by extending each $D \in \reqtc$ to a $D': \catc' \morph \Fin$ by defining $D'(v)=D(u)^{-1}$.
With these definitions then commutivity constraints $D'(u) \circ D'(v)=id_{D'(e}$ and $D'(v) \circ D'(u)=id_{D'(e)}$ hold
in all the instances $D' \in \reqtcp$. For this reason path equivalences $u \circ v=id_e$ and $v \circ u =id_e$
must also be added to the sketch $S'$ for its corresponding category $\catcp$  to be maximally constrained to the
requirement $\reqtcp$.

With $S'$ and $\catcp$ so defined then even if all commutivity constraints, all functional dependencies and all referential inclusion dependencies present in the requirement  $\reqtcp$ are represented in $\catcp$  this does not mean that
$\catcp$ necessarily meets the goodness criteria that it be maximally constrained to $\reqtcp$. For example, consider if every permutation 
$D(u)$ is the square of some other permutation $w_D$ on set $e$. In this case in each instance $D \in \reqtcp$ there is a unique
function $w_D: D(e) \morph D(e)$ such that $w_D \circ w_D = D(u)$. 
$\catcp$ is not maximally constrained
to $\reqtcp$ for a morphism $w$ can be added and a constraint $w \circ w = u$ to obtain a sketch $S''$ generating 
a category $C''$ so that the extended theory $\catcpp$ is a
better fit to the requirement than $\catcp$. An arrow must be added to represent the inverse of $w$ and to meet the principle of no redundancy $u$ and $v$ will need be dropped from sketch $S''$ as they will have become redundant.
The net effect of following the goodness criteria will be to represent $w$  and its inverse in data rather than $w^2$ and its inverse.

