Suppose a sketch $S$ has  this graph:
$$
\Rnode{e}{e}
$$
\rEtme[270]{e}
\alabel{u}
and no path equivalences and let $\catc$ be the category presented by $S$. This is the theory of a
set along with a single unary operation (a \term{u-structure} in the terminology of \cite{BarrandWells}). 
Suppose in every instance $D$ of a requirement $\reqtc$ for $\catc$, the function $D(u)$ is a permutation i.e. bijection. 
In this case  
$\catc$ is not maximally constrained to $\reqtc$ (i.e. it fails the goodness criteria) because, since every function $D(u)$ had a unique inverse, there is a functional dependency 
$u \morph id_e$ in  $\catc$ with respect to $\reqtc$ and this functional dependency is not represented in $\catc$. 
In order to meet the goodness criteria, $S$ can be extended to a sketch $S'$ having an additional arrow $v:e \morph e$ to represent this functional
dependency and by extending each $D \in \reqtc$ to a $D': \catc' \morph \Fin$ by defining $D'(v)=D(u)^{-1}$.
With these definitions then commutivity constraints $D'(u) \circ D'(v)=id_{D'(e}$ and $D'(v) \circ D'(u)=id_{D'(e)}$ hold
in all the instances $D' \in \reqtcp$. For this reason path equivalences $u \circ v=id_e$ and $v \circ u =id_e$
must also be added to the sketch $S'$ for its corresponding category $\catcp$  to be maximally constrained to the
requirement $\reqtcp$.