
Is a presentation of a category really a notion of data specification? 
Well, as we might specify data in an XML schema, or in IDL or in a relational schema, then it isn't. But as we might outline
data in a preliminary design of such, say like shown in this fragment 
$
\begin{array}{c p{0.05cm}c p{0.5cm}c}
                        & & \rule[-0.3cm]{0pt}{0.8cm}\Rnode{p}{p}roject& &             \\ [0.3cm]
	 project-wo\Rnode{w}{rker}	& &                   & &  \\ [0.3cm]     
	                      & & \Rnode{e}{e}mployee      & &             \\ [0.6cm]     
	                      & & \Rnode{dep}{d}ependent  & &             
\end{array}
\begin{arrows}
\ncarr{w}{p} 
\alabel{R_1}[0.5][-1]
\ncarr{w}{e} 
\blabel{R_2}[0.5][-1]
\ncarr{dep}{e} 
\alabel{S_0}
\end{arrows}
$
, then it might be.

It also might be \textit{after we take away detail} from a complete specification to achieve an abstraction. Thus we can recognise
occurrences of directed graphs within larger and fully detailed data specifications wherever we can find occurrences
of this diagram:
$$
\begin{array}{c p{0.25cm} c  p{0.25cm} c }
\Rnode{e}{e} &&                   && \Rnode{n}{n} \\[0.4cm]
\end{array}
$$
\Etme[20]{e}{n}
\alabel{src}[0.75]
\Etme[-20]{e}{n}
\blabel{tgt}[0.75]

I mention this  because Bachman in his 1969 paper \textit{Data Structure Diagrams} \cite{Bachman1969}
enthuses over multiple occurrences of this shape appearing in  a larger data specification\footnote{Though his notation is different and in particular his arrows are the reverse of ours.}. 