
Is a presentation of a category really a notion of data specification? 
Well, as we might specify data in an XML schema, or in IDL or in a relational schema, then it isn't. But as we might outline
data in a preliminary design of such, say like shown in this fragment 
$
\begin{array}{c p{0.05cm}c p{0.5cm}c}
                        & & \rule[-0.3cm]{0pt}{0.8cm}\Rnode{p}{p}roject& &             \\ [0.3cm]
    project-wo\Rnode{w}{rker} & &                   & &  \\ [0.3cm]     
                         & & \Rnode{e}{e}mployee      & &             \\ [0.6cm]     
                         & & \Rnode{dep}{d}ependent  & &             
\end{array}
\begin{arrows}
\ncarr{w}{p} 
\alabel{R_1}[0.5][-1]
\ncarr{w}{e} 
\blabel{R_2}[0.5][-1]
\ncarr{dep}{e} 
\alabel{S_0}
\end{arrows}
$,
then it might be.

It also might be \textit{after we take away detail} from a complete specification to achieve an abstraction. Thus we can recognise
occurrences of directed graphs within larger and fully detailed data specifications wherever we can find occurrences
of this diagram:
$$
\begin{array}{c p{0.25cm} c  p{0.25cm} c }
\Rnode{e}{e} &&                   && \Rnode{n}{n} \\[0.4cm]
\end{array}
$$
\Etme[20]{e}{n}
\alabel{src}[0.75]
\Etme[-20]{e}{n}
\blabel{tgt}[0.75]

I mention this  because Bachman in his 1969 paper \textit{Data Structure Diagrams} \cite{Bachman1969}
enthuses over multiple occurrences of this shape appearing in  a larger data specification\footnote{Though his notation is different and in particular his arrows are the reverse of ours.}. 
\subsubsection{Molecular Structure}
\subsubsection{Tabular Data}
\begin{displaymath}
\begin{array}{c p{0.25cm}c p{0.25cm}c}
             && \Rnode{t}{t}   &&              \\ [0.4cm]
\Rnode{r}{r} &&                && \Rnode{c}{c} \\ [0.4cm]     
             && \Rnode{d}{d} &&                    
\end{array}
\begin{arrows}
\ncarr{r}{t} 
\alabel{R_1}[0.5]%[-1]
\ncarr{c}{t} 
\blabel{R_2}[0.5]%[-1]
\ncarr{d}{r} 
\alabel{S_1}
\ncarr{d}{c}
\blabel{S_2} 
\end{arrows}
\end{displaymath}

\subsubsection{Function Application}
\begin{tabular}{c p{1cm}c}
$
\begin{array}{c p{0.25cm}c p{0.25cm}c}
\Rnode{TL}{a} &&                && \Rnode{TR}{f} \\ [0.6cm] 
\Rnode{BL}{ap} &&                && \Rnode{BR}{fp}   
\end{array}
\begin{arrows}
\ncarr{TL}{TR} 
\alabel{R_1}
\ncarr{BL}{BR} 
\blabel{R_2}
\ncarr{BL}{TL} 
\alabel{S_1}
\ncarr{BR}{TR}
\blabel{S_2} 
\end{arrows}
$
&&
\begin{tabular}{c p{0.2cm} c}
a  && function application i.e. use of a function \\
ap && actual parameter supplied to a function \\
f  && a function definition                   \\
fp && a formal parameter of a function 
\end{tabular}
\end{tabular}

\subsubsection{Relational Meta-Model}
A data specification that describes the relational model of data (in other words, the data specification that is the relational meta-model) when viewed abstractly as a directed graph
include nodes representing the concepts of table ($t$) , column ($c$), foreign key constraint ($fk$) and foreign key element ($fke$) as well as others
\iffalse
\raisebox{-0.7cm}
{\footnotesize
\begin{tabular}{cp{0.75cm}cp{13cm}}
                 &$t$&    &  table  \\ [0.1cm]
                 &$c$&    & column  \\ [0.1cm]
                &$fk$&    & foreign key -- consists of one or more foreign key elements \\[0.1cm]
               &$fke$&    & foreign key element --  associates a referencing column and a referred to column. 
\end{tabular}	
}
\vspace{0.25cm}
\fi
and  also includes the following edges:	\\
\begin{tabular}{p{1cm} c}
&
{\footnotesize
$		
\begin{array}{cp{0.75cm}cp{13cm}}	
                                                                  \\			
\Rnode{c}{c}      && \Rnode{t}{t}   & the parent table of a column\\[0.2cm]   
\Rnode{fk}{fk}    && \Rnode{t2}{t}  & the parent table of a foreign key\\[0.2cm] 
\Rnode{fke}{fke}  && \Rnode{fk2}{fk}& the foreign key a foreign key element is part of\\[0.2cm]  
\Rnode{fk3}{fk}   && \Rnode{t3}{t}  & the table that the foreign key defines a reference to\\[0.2cm]     
\Rnode{fke2}{fke} && \Rnode{c2}{c}  & the referencing column identified by the foreign key element\\[0.2cm]     
\Rnode{fke3}{fke} && \Rnode{c3}{c}  & the referred to column identified by the foreign key element         
\end{array}
$
\ncarr{c}{t}
\alabel{p_c}
\ncarr{fk}{t2} 
\alabel{p_f}
\ncarr{fke}{fk2}
\alabel{p_e}
\ncarr{fk3}{t3} 
\alabel{r_0}
\ncarr{fke2}{c2}
\alabel{r_1}
\ncarr{fke3}{c3}
\alabel{r_2}
}
\vspace{0.2cm}
\end{tabular}

It is a striking fact that two non-trivial path equivalences  
will hold between paths constructed from these edges in that both the diagram
\begin{tabular}{ c c c}
\footnotesize{
$
\begin{array}{cp{0.75cm}c}
   \Rnode{t}{t}       & &              \\[1.2cm]   
	 \Rnode{fk}{fk}     & &              \\[1.2cm] 
	 \Rnode{fke}{fke}   & & \Rnode{c}{c} \\[0cm]
							        & &               % horizontal spece needed    
\end{array}
$
\ncarr{fk}{t} 
\alabel{p_f}
\ncarr{fke}{c}
\blabel{r_1}
\ncarr{fke}{fk}
\alabel{p_e}
\ncarr{c}{t}
\blabel{p_c}
}
&and the diagram&
{\footnotesize
$
\begin{array}{cp{0.75cm}c}
   \Rnode{fk}{fk}     & & \Rnode{t}{t} \\[1.2cm]     
	 \Rnode{fke}{fke}   & & \Rnode{c}{c}
\end{array}
$
\ncarr{fk}{t} 
\alabel{r_0}
\ncarr{fke}{c}
\blabel{r_2}
\ncarr{fke}{fk}
\alabel{p_e}
\ncarr{c}{t}
\blabel{p_c}
}  \ \ will commute.
\end{tabular}
Path equivalence constraints (aka commutivity constraints) can therefore be found right at the heart of the relational model of data. Despite this, to my knowledge, this type of constraint  is absent from relational data theory and this  despite the fact, as mentioned above, that they have a direct bearing on the construction of schemas in third normal form.
The discussion in Shlaer and Lang \cite{Shlaer96} is an exception.