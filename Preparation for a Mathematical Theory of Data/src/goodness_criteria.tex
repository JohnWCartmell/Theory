
From the second principle, specific goodness criteria follow. In this section we formulate three specific criteria which are applicable to  sketches considered as data specifications. 

Each criterion is stated as something that `ought to' hold of a data specification with respect to a requirement. 
Collectively, they are intended as an objective measure of whether
a data specification meets the second goodness principle.
One of these criteria (criterion 2B, below) is an archetypal precursor to Codd's third normal form criterion. 


\subsubsection{Criterion 2A --- The Representation of Path Equivalence}
\newcommand{\pequiv}[1][\reqtc]{\underset{#1}{\equiv}}

\begin{definition}
If $\catcw$ is a  category, if $\reqtc$ is a set of instances
 and if \fgparalleldiag in $\catc$, then say that path $f$ is equivalent to path $g$ with respect to the requirement $\reqtc$ 
 (and write $f \pequiv g$) iff
in all instances $D \in \reqtc$, $D(f)=D(g)$.
\end{definition}

\begin{definition}
If $\catc$ is a  category and $\reqtc$ is a set of instances,
 and if \fgparalleldiag in $\catc$ such that $f \pequiv g$
 then say that the path equivalence $f \pequiv g$ is represented in \catcw iff
 $f=g$.
\end{definition}

\begin{definition}
If $\catc$ is a  category and $\reqtc$ is a set of instances,
 then say that  $\catc$ is \term{equationally complete} with respect 
to the requirement $\reqtc$ iff all path equivalences with respect to $\reqtc$ are represented in \catcw 
i.e. iff for all diagrams \fgparalleldiag in $\catc$,  
if in all instances $D \in \reqtc$, $D(f)=D(g)$,  then $f=g$ in $\catc$.
\end{definition}

\textbf{Goodness Criterion 2A} \IfSforCwithRCwords then \catcw should be equationally complete with respect to $\reqtc$ that is all path equivalences with respect to $\reqtc$  ought to be represented in \catc.

\subsubsection{Criterion 2B --- The Representation of Functional Dependencies}

\begin{definition}
If $\catc$ is a category and $\reqtc$ is a set of instances and if \fgsourcediag
in $\catc$ then there is a  \term{functional dependency} of $g$ on $f$ with respect to $\reqtc$ iff
there is a family of functions $H_D)_{D \in \reqtc}$ such that 
in each instance $D$, $H_D$ is a unique function $H_D: D(b) \morph D(c)$, such that $D(f) \circ H_D = D(g)$. 
If there is such a functional dependency then we say that $\fundep{H}{f}{g}$ in $\catc$ with respect to $\reqtc$.
\end{definition}

Our use of the $\morph$ notation for functional dependencies here is coming from relational database theory where it is usual to represent such a functional dependency as we have here by asserting that 
$$
f \morph g
$$
Note that this use of an $\morph$ notation is independent of our use of $\morph$ as a morphism of a category 
or, for that matter, as an edge in a presentation. Neither are we alluding to a bicategory structure. We have two distinct uses for $\morph$ (three if you distinguish arrows in presentations from arrows in categories). Any particular use will be unambiguous in context.

Some consequences of the definition are brought out in the this lemma.
\begin{lemma}
If $\catc$ is a locally finite category and $\reqtc$ is a set of instances, 
if \fgsourcediag in \catcw
then there is a functional dependency $\fundep{H}{f}{g}$  with respect to $\reqtc$ 
iff in every instance $D \in \reqtc$, 
\begin{itemize}
\item
there is a  function $H_D: D(b) \morph D(c)$, 
such that $D(f) \circ H_D = D(g)$ and 
\item either $D(c)$ is a singleton set  or 
           the function $D(f)$ is surjective.
\end{itemize}
\end{lemma}
\begin{proof}
Note that if there is a $D$, such that $D(b)$ has
an element $x$ which is not in the image of $D(f)$, and for which $D(c)$ is not singleton
then $H_D$ cannot be unique for there be multiple values in $D(c)$ that $H_D$ can map $x$ to. 
\end{proof}
\begin{corollary}
For any morphism $f:a \morph a$ is \catcw there is a functional dependency $\fundep{H}{f}{f}$ iff
in each instance  $D \in \reqtc$, $D(f)$ is surjective.
\end{corollary}
\begin{proof}
Follows from the previous lemma because in every instance $D \in \reqtc$ 
we can take the function $H_D: D(A) \morph D(a)$ to be the identity function on the set $D(a)$
so that  $D(f) \circ H_D = D(f)$. 
\end{proof}

\begin{definition}
If $\catc$ is a category and $\reqtc$ is a set of instances, 
if \fgsourcediag in $\catc$ 
and if there is a functional dependency $\fundep{H}{f}{g}$ then say that 
the functional dependency $H$ is \term{represented} in $\catc$  
by a morphism  $h:b \morph c$ in $\catc$ 
iff $f \circ h =g$.
\end{definition}
From which we can quickly see:
\begin{observation}
In the above context, if functional dependency $H$ is represented in \catcw by a morphism $h:b \morph c$ then for each instance $D \in \reqtc$, $D(h)=H_D$.
\end{observation}

\textbf {Goodness Criterion 2B} \IfSforCwithRCwords then all functional dependencies present in $\reqtc$
ought to be represented in \catc.

\subsubsection{Criterion 2C --- The Representation of Referential Inclusion Dependencies}

\begin{definition}
If $\catc$ is a category and $\reqtc$ is a set of instances 
and if
\fnsourceqnsource
in $\catc$, then an \term{inclusion dependency} $I$, written $a[f_1,...f_n] \overset{I}{\subseteq} c[q_1,..q_n]$, is a family of functions $I_D)_{D \in \reqtc}$
such that each instance $D \in \reqtc$, $I_D$ is a function $I_D : D(a) \morph D(c)$ such that
for each $i$, $1 \leq i \le n$, $I_D \circ D(q_i) = D(f_i)$.
\end{definition}

If each function in this family is the unique such function then the inclusion dependency is said to be referential:

\begin{definition}
If $\catc$ is a category and $\reqtc$ is a set of instances 
and if
\fnsourceqnsource
in $\catc$, then a \term{referential inclusion dependency} $R$, written $a[f_1,...f_n] \overset{R}{\subseteq} c[q_1,..q_n]$, is a family of functions $R_D)_{D \in \reqtc}$
such that each instance $D$, $R_D$ is a unique function $R_D : D(a) \morph D(c)$ such that
for each $i$, $1 \leq i \le n$, $R_D \circ D(q_i) = D(f_i)$.
\end{definition}

\begin{lemma}
If $\catc$ is a category and $\reqtc$ is a set of instances and if
\fnsourceqnsource
in $\catc$
such that $a[f_1,...f_n] \overset{I}{\subseteq} c[q_1,..q_n]$ is an inclusion dependency with respect  to $\reqtc$ then if in each instance $D \in \reqtc$ the family of functions
$q_{i, 1 \leq i \leq n}$ is jointly-injective then $a[f_1,...f_n] \overset{I}{\subseteq} c[q_1,..q_n]$ is a referential inclusion dependency.
\end{lemma}
\begin{proof}
Straightforward.
\end{proof}

\begin{definition}
If $\catc$ is a category and $\reqtc$ is a set of instances and if
\fnsourceqnsource
in $\catc$ and if $a[f_1,...f_n] \overset{R}{\subseteq} c[q_1,..q_n]$ is a referential inclusion dependency
with respect  to $\reqtc$ then say that the inclusion dependency $R$ is \term{represented} in $\catc$ by a morphism $r:a \morph c$ by a morphism $r:a \morph c$ in $\catc$
iff in each instance $D \in \reqtc$, $D(r) = R_D$. 
\end{definition}

\textbf {Goodness Criterion 2C} \IfSforCwithRCwords 
then all referential inclusion dependencies present in $\reqtc$
ought to be be represented in \catcp.

We will need the following:
\begin{lemma}
\label{catRefIncSublemma}
If $\catc$ is a category and $\reqtc$ is a set of instances,
if \fnsourceqnsource in $\catc$, 
and if $a[f_1,...f_n] \overset{R}{\subseteq} c[q_1,..q_n]$ 
is a referential inclusion dependency with respect to $\reqtc$ 
then if there is a morphism $r: a \morph c$ in \catcw 
such that for each $i$, $r \circ q_i = f_i$ 
then $r$ is a representation of $R$  in  \catc. 
\end{lemma}
\begin{proof} Straightforward because of the uniqueness condition for each $R_D$ in the definition of referential inclusion dependency and because the functoriality of each $D \in \reqtc$ gives, 
for each $i$, $D(r) \circ D(q_i) = D(f_i)$.
\end{proof}