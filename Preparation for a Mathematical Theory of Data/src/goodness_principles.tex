In this section we formulate two general goodness principles which are applicable to  sketches considered as data specifications. 

Both principles `ought to' hold of a data specification with respect to a requirement. 
Together, they are intended as an objective goal for
a data specification.

  
\subsubsection{The Principle of No Redundancy}
\textbf{Goodness Principle 1:} If a sketch $S$ for a category \catcw 
is considered as a data specification then  there ought to be no subsketch of $S$ which is a sketch of \catcw. 

Rationale: Redundant edges lead to redundant data. Redundant diagrams give rise to unnecessary program logic.

\textbf{Example} In the example given above of the sketch for the relational meta model 
the edge $S_1$ and the path equivalence (\ref{rdbR1scope}) are redundant. This model therefore fails this 
goodness criteria. 
\subsubsection{The Principle of Maximal Constrainedness}
Maximum constrainedness, as mentioned above, is a property of the category $\catc$ generated by the sketch  rather than of the sketch itself and is defined  relative to a requirement $\reqtc$ by which we mean a set of 
instances where each instance is a functor $D$, $D: \catc \morph \Fin$. In what follows, therefore,  by a requirement $\reqtc$ for category $\catc$ we mean a set  $\reqtc \subseteq | \Fin^{\catc} |$. 

Consider, a theory usually has some slack by which we mean that it has structurally compliant instances that are not part of its requirement.  The definition of maximal constrainedness expresses that a theory is maximally constrained to its requirement if there is no way of extending the theory so as to rule out possible structurally compliant instances that are not part of the requirement (i.e. to rule out slack) whilst remaining consistent with the requirement.

The definition now follows, preceded by an auxiliary definition.
\begin{definition}
If $\catc$ is a category and $\reqtc$ is a requirement for $\catc$,  if $I: \catc \morph \catcp$ is a functor then say that $I$ is \term{consistent with} requirement $\reqtc$ iff for all instances $D \in \reqtc$ there exists a unique functor $D':\catcp \morph \Fin$ such that $I \circ D'=D$.
\end{definition}
\newcommand{\IfSforCwithRCwords}{If $S$ is a sketch for category \catcw and if $S$ is considered as a data specification with requirement $\reqtc$\ }
\begin{definition}
 \IfSforCwithRCwords then define \catcw to be \term{maximally constrained} to the requirement $\reqtc$ iff for all categories $\catcp$ and for all functors $I:\catc \morph \catcp$ that are consistent with $\reqtc$, for all functors $F: \catc \morph \Fin$  there exists a  $F' : \catcp \morph \Fin$ such that $I \circ F'=F$.
\end{definition}

\textbf{Goodness Principle 2:}
\IfSforCwithRCwords then \catcw ought to be be maximally constrained to $R_C$. 

