\section{Further Analysis}
\subsection{Representation of Binary Joint Functional Dependencies}
\newcommand{\representjointfd}[3]
{
for some $n$, $n \geq 0$, there exists morphisms $w_1,..w_{2n+1}:b \morph #1$ in $\catc$ such that 
 $f_1 \circ w_1 = #2$
 and
$f_2 \circ w_{2n+1} = #3$
and \foreachi, 
                       $f_2 \circ w_{2i-1} = f_2 \circ w_{2i}$
                  and  $f_1 \circ w_{2i} = f_1 \circ w_{2i+1}$}
% End of \representjointfd
\begin{lemma}
\llabel{binaryjointfunctionaldependencyrepresentationlemma}
If $\catc$ is a locally finite category and $\reqtc$ is a set of instances, 
if $\catc$ is \term{maximally constrained} to the requirement $\reqtc$ 
then if $a_1$, $a_2$, $b$ and $c$ are objects of \catcw and   
  $f_1: a_1 \morph b$,  $f_1: a_1 \morph b$, 
$g_1,: a_2 \morph c$ and $g_2: a_2 \morph c$ are morphisms
so that we have 
$
\begin{array}{c p{0.65cm} c  p{0.75cm} c }
\Rnode{a1}{a_1}                  \\[0.5cm]
                 && \Rnode{b}{b} && \Rnode{c}{c}\\[0.5cm]
\Rnode{a2}{a_2}       
\end{array}
\begin{arrows} 
\ncarr{a1}{b}
\blabel{f_1}[0.4][0.1]
\ncarr[15]{a1}{c}
\alabel{g_1}[0.5][0.2]
\ncarr{a2}{b}
\alabel{f_2}[0.4][0.1]
\ncarr[-15]{a2}{c}
\blabel{g_2}
\end{arrows}
$
in \catcw
then if for each $D \in \reqtc$ there is a unique function
$H_{D}: D(b) \morph D(c)$ such that 
$D(f_1) \circ H_D = D(g_1)$ and  $D(f_2) \circ H_D = D(g_2)$
then \representjointfd{c}{g_1}{g_2}.

To illustrate this, in the $n=1$ case we will have the following commuting diagrams in \catc
$$
\begin{array}{c p{0.65cm} c  p{1.0cm} c }
\Rnode{a11}{a_1}                               \\[0.5cm]
               && \Rnode{b1}{b}                \\[0.3cm]
\Rnode{a21}{a_2}                               \\[0.3cm]
               && \Rnode{b2}{b} && \Rnode{c}{c}\\[0.3cm]
\Rnode{a12}{a_1}                               \\[0.3cm]
               && \Rnode{b3}{b}                \\[0.5cm]
\Rnode{a22}{a_2}       
\end{array} 
\begin{arrows} 
\ncarr{a11}{b1}
\blabel{f_1}[0.4][0.1]
\ncarr{a21}{b1}
\alabel{f_2}[0.4][0.1]
\ncarr{a21}{b2}
\blabel{f_2}[0.4][0]
\ncarr{a12}{b2}
\alabel{f_1}[0.4][0.1]
\ncarr{a12}{b3}
\blabel{f_1}[0.4][0.1]
\ncarr{a22}{b3}
\alabel{f_2}[0.4][0] 
\ncarr{b1}{c}
\blabel{w_1}[0.4][0]
\ncarr{b2}{c}
\alabel{w_2}[0.3][0.1]
\ncarr{b3}{c}
\alabel{w_3}[0.4][0] 
\ncarr[30]{a11}{c}
\alabel{g_1}[0.5][0.2]
\ncarr[-30]{a22}{c}
\blabel{g_2}
\end{arrows}
$$
\end{lemma}
\begin{proof}
We can construct a sketch $S'$ extending sketch $S$ 
 by formally adding a morphism $\qq{h}: b \morph c$
and path equivalences $f_1 \circ \qq{h} = g_1$ and $f_2 \circ \qq{h} = g_2$
Let $\catcp$ be the category generated by $S'$ and
let $I: \catc \morph \catcp$ be the inclusion functor generated 
by the inclusion of $S$ in $S'$. 

We can show that $I$ is consistent with $\reqtc$
by showing that any $D \in \reqtc$
extends uniquely to $D' :\catcp \morph \Fin$. Assume such a $D$. 
$D$  extends to \catcpw iff there is a  function that 
we can choose as the value of  for $D'(\qq{h})$  
such that for $i$ being both $1$ and $2$,  $D'(f_i) \circ D'(\qq{h}) = D'(g_i)$ 
i.e  $D(f_i) \circ D'(\qq{h}) = D(g_i)$ and so we  extend $D$ to $D'$ 
 by defining $D'(\qq{h})=H_D$.
This extension to $D'$ is unique because we are given that
$H_D$ is the unique function that satisfies 
for $i$ being both $1$ and $2$, $D(f_i) \circ H_D = D(g_i)$.

Let the functor $F: \catc \morph \Fin$ be the coproduct $Hom_{\catc}(a_i,-) + Hom_{\catc}(a_i,-)$
in the functor category $Fin^{\catc}$ and label the injections $L$ and $R$, respectively so that
for each object $x$ of $\catc$ the diagram
\begin{center}
$
\begin{array}{c p{0.5cm} c p{0.5cm} c  }
\Rnode{h1}{Hom_{\catc}(a_1,x)}  &&\Rnode{Fx}{F(x)}  &&   \Rnode{h2}{Hom_{\catc}(a_2,x)}       
\end{array} 
$
\ncarr{h1}{Fx}
\alabel{L_x}
\ncarr{h2}{Fx}
\blabel{R_x}
\end{center}
is a coproduct in $\Fin$.
Now for each object $x$ of $\catc$, we define an equivalence relation $\sim_x$ on $F(x)$ as follows
\begin{align*}
L_x(k_1) \sim_x R_x(k_2) &&&
 \parbox[t]{8cm}{ iff \representjointfd{x}{k_1}{k_2},}\\
R_x(k_1) \sim_x L_x(k_2) &&& \mbox{ iff $L_x(k_2) \sim_x R_x(k_1)$,} \\
L_x(k_1) \sim_x L_x(k_2) &&&
 \parbox[t]{8cm}{ iff there exists $w,w':b \morph x$ 
                  such that $L_x(k_1) \sim_x R_x(f_2 \circ w)$
                        and $f_2 \circ w = f_2 \circ w'$
                        and $f_1 \circ w' = k_2$,} \\
R_x(k_1) \sim_x R_x(k_2) &&&
 \parbox[t]{8cm}{ iff there exists $w,w':b \morph x$ 
                  such that $R_x(k_1) \sim_x L_x(f_1 \circ w)$
                        and $f_1 \circ w = f_1 \circ w'$
                        and $f_2 \circ w' = k_2$,} \\ 
\end{align*}
With $\sim_x$ so defined then it follows that $\sim_x$ is an eqivalence relation i.e. is symmetric and transitive.  
Also if $j: x_1 \morph x_2$ in \catcw and if $y_1,y_2 \in F(x_1)$
such that $y_1 \sim_{x_1} y_2$
then it follows easily  from the definition of $\sim$
that $F(j)(y_1) \sim_{x_2} F(j)(y_2)$.
Therefore we can define a functor 
$G: \catc \morph \Fin$  so that for any object $x$ of %\catc
the set $G(x)$ is the quotient $F(x)/{\sim_x}$.
Aside: I have defined $\sim_x$ as I have done so as to achieve a minimum relation 
for which $L_a(f_1) \sim_x R_a(f_2)$ and 
which is also symmetric, transitive and preserved by $F(j)$, for each $j$, as described. 

\begin{align*}
[L_c(g_1)]
&= G'(g_1)([L_a(id_a)])     
 	& & \mbox{by definition of $G'(g_1)=G(g_1)$ 
 			            and since $G(g_1)([L_a(id_a)])=[L_c(g_1)]$,}\\
&= G'(\qq{h}) (G'(f_1)([L_a(id_a)])) 
	& & \mbox{since $G'(f_1) \circ G'(\qq{h}) = G'(g_1)$,}  \\
&= G'(\qq{h}) ([L_b(f_1)])       
	& & \mbox{by definition of $G'(f_1)=G(f_1)$
	                            and since $G(f_1)([L_a(id_a)]) = [L_b(f_1)]$,} \\
&= G'(\qq{h}) ([R_b(f_2)])       
	& & \mbox{since $L_b(f_1) \sim_b R_b(f_2)$, by case $n=1$ of the definition of $\sim_b$,} \\
&= G'(\qq{h}) (G'(f_2)([R_b(id_b)])) 
	& & \mbox{by definition of $G'(f_2)=G(f_2)$ 
	                            and since $G(f_2)([R_b(id_b)]) = [R_b(f_2)]$,}\\
&= G'(g_2)([R_b(id_b)])              
	& & \mbox{since $G'(f_2) \circ G'(\qq{h}) = G'(g_2)$,}  \\
&= [R_c(g_2)]                        
	& & \mbox{by definition of $G'(g_2)=G(g_2)$
 			            and since $G(g_2)([R_b(id_b)])=[L_c(g_2)]$.}   
\end{align*} 
and since $[L_c(g_1)]=[R_c(g_2)]$ then it follows from the definition of $\sim_c$ that 
\representjointfd{c}{g_1}{g_2}, as required.
\end{proof}

