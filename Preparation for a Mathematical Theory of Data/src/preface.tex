% Copied on 21 February 2026 from the MToDrange categories goodness article


Data specifications describe how particulars in some context or discursive domain
are to be represented in data. 
The \textit{mathematical theory of data} is a putative meta-theory 
intended to support technology-independent reasoning about such data specifications, whatever form they may take. 
Its scope spans the relational theory of data that originated in 1970, including the five or so definitions of normal form,
and the later re-targeting of these normal forms to hierarchical representations 
of data as found in message structures generally and as typified in the use of XML.

The putative theory is abstract in so much as it remains neutral about the choice of universals such as \textit{integer}, \textit{float} and \textit{string} and the binary encodings of these universals. 
It explains the viability and the interchangeability of hierarchical and relational forms of a data specification.  

It formulates both general principles and specific criteria for goodness of data specifications and aims to establish these by demonstrating a chain of mathematically provable logical implications
\[
\textit{General Principles}
\;\Longrightarrow\;
\textit{Specific Goodness Criteria}
\;\Longrightarrow\;
\textit{Classic 1970s Normal Forms}.
\]
 
In this way, the general principles provide both an explanation for the classic normal forms --- forms which previosuly have retained an air of mystery --- 
and also a means of generalising them beyond the relational sphere in which they were first formulated.

A prior analysis (notes currently on GitHub) suggests that our everyday descriptions of particular things in common language can be shaped in part by  occurrences of equivalent paths among the network of functional relationships present in any subject domain. 

To achieve good data specifications for a subject domain, the effects of these equivalent paths must be taken into account. 
Since the logic of equivalent paths is the starting point for category theory, we are led to the conclusion that 
category theory should play a role in the mathematical theory of data. Indeed, the contention is that
a data specification must be a sketch of some sort of categorical structure, 
and that it is the use of the language of category theory that enables us to articulate precise 
principles and criteria of goodness.