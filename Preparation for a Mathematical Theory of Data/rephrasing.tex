
The definition of functional dependency  that has been given above 
can be rephrased slightly as we now describe. The rephrasing and an observation regarding
inclusion dependencies and jointly-injective functions points the
way to a future note in which we will describe the use of categories with specified monos, epis and epi-mono splits as data specifications.
\subsubsection{Rephrasing of Principle 2B regarding Representation of Functional Dependencies}
\begin{lemma}
If $\catc$ is a locally finite category and $\reqtc$ is a set of instances, if \fgsourcediag in \catcw
then if there exists a functional dependency $\fundep{H}{f}{g}$  with respect to $R_C$ then
for each instance  $D \in \reqtc$, $D(f)$ is surjective.
\end{lemma}
\begin{proof}
Follows from the uniqueness condition for $H_D$ in the definition of functional dependency which cannot hold in any instance $D \in \reqtc$ in which $D(f)$ is not surjective.
\end{proof}
\begin{corollary}
For any morphism $f:a \morph a$ is \catcw there is a functional dependency $\fundep{H}{f}{f}$ iff
in each instance  $D \in \reqtc$, $D(f)$ is surjective.
\end{corollary}
\begin{proof}
Follows from the previous lemma because in every instance $D \in \reqtc$ 
we can take the function $H_D: D(A) \morph D(a)$ to be the identity function on the set $D(a)$
so that  $D(f) \circ H_D = D(f)$. 
\end{proof}
This lemma and its corallary  mean that we can replace the uniqueness requirement in the 
definition of functional dependency by a requirement that each $D(f)$ be surjective and obtain the following
rephrasing of the definition:
\begin{definition}
If $\catc$ is a locally finite category and $\reqtc$ is a set of instances, if \fgsourcediag in \catcw
then there is a  \term{functional dependency} of $g$ on $f$ with respect to $\reqtc$ iff
\begin{itemize}
\item in every instance $D$ the function $D(f)$ is surjective and
\item
there is a family of functions $H_D)_{D \in \reqtc}$ such that 
in each instance $D$, $H_D$ is a  function $H_D: D(b) \morph D(c)$, 
such that $D(f) \circ H_D = D(g)$. 
\end{itemize}
\end{definition}
\subsubsection{Regarding Jointly-Injectives and Inclusion Dependencies}

\begin{lemma}
If $\catc$ is a category and $\reqtc$ is a set of instances and if
\fnsourceqnsource
in $\catc$
such that $a[f_1,...f_n] \overset{I}{\subseteq} c[q_1,..q_n]$ is an inclusion dependency with respect  to $\reqtc$ then if in each instance $D \in \reqtc$ the family of functions
$q_{i, 1 \leq i \leq n}$ is jointly-injective then $a[f_1,...f_n] \overset{I}{\subseteq} c[q_1,..q_n]$ is a referential inclusion dependency.
\end{lemma}
\begin{proof}
Straightforward.
\end{proof}

\iffalse
\newcommand {\qnsourcediag}{
$
\begin{array}{c p{0.5cm} c  }
             &&   \Rnode{b}{b_1} \\[0.01cm]
\Rnode{a}{a} &&                \\[0.01cm] 
             &&   \Rnode{c}{b_n}         
\end{array} 
\begin{arrows}
\ncarr{a}{b}
\alabel{q_1}
\ncarr{a}{c}
\blabel{q_n}
\end{arrows}
$  
}
\fi

