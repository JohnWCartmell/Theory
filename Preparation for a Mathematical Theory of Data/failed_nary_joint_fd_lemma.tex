 
\subsubsection{Representation of Joint Functional Dependencies}
\begin{lemma}
\llabel{jointfunctionaldependencyrepresentationlemma}
If $\catc$ is a locally finite category and $\reqtc$ is a set of instances, if $\catc$ is 
\term{maximally constrained} to the requirement $\reqtc$ then
all joint functional dependencies $\fundep{H}{f_i)_{1 \le i \leq n}}{g_i)_{1 \le i \leq n}}$  with respect to $\reqtc$ are represented in $\catc$.

Suppose that $b$ and $c$ are objects of \catcw and that for some $n$, $n \geq 1$ that for each $i$,$1\leq i \leq n$  there are morphisms $f_i: a_i \morph b$ and 
$g_i: a_i \morph c$ such that for each $D \in \reqtc$ there is a unique function
$H_{D}: D(b) \morph D(c)$ such that \foreachi $f_i \circ H_D = g_i$
then there exists a morphism $k: b \morph c$ in \catcw such that \foreachi, 
$f_i \circ k = g_i$.


\end{lemma}
\begin{proof}
Suppose such families $f_i)_{1 \le i \leq n}$ and $g_i)_{1 \le i \leq n}$ and suppose 

$
\begin{array}{c p{0.5cm} c  }
             &&   \Rnode{b}{b} \\[0.01cm]
\Rnode{a}{a_i} &&                \\[0.01cm] 
             &&   \Rnode{c}{c}         
\end{array} 
$
\ncarr{a}{b}
\alabel{f_i}
\ncarr{a}{c}
\blabel{g_i}
 we can construct a sketch $S'$ extending sketch $S$ 
 by formally adding a morphism $\qq{h}: b \morph c$
and path equivalences $f_i \circ \qq{h} = g_i$ \foreachi. 
Let $\catcp$ be the category generated by $S'$ and
let $I: C \morph C'$ be the inclusion functor generated by the inclusion of $S$ in $S'$. 

Now we show that $I$ is consistent with $\reqtc$. We need to show that any $D \in \reqtc$
extends uniquely to $D' :\catcp \morph \Fin$. Assume such a $D$. 
$D$  extends to $C'$ iff there is a  function that 
we can choose as the value of  for $D'(\qq{h})$  
such that \foreachi, $D'(f_i) \circ D'(\qq(h) = D'(g_i)$ i.e such that
\foreachi, $D(f_i) \circ D'(\qq(h) = D(g_i)$ and so we  extend $D$ to $D'$ 
 by defining $D'(\qq{h})=D_H$.
This extension to $D'$ is unique because from the definition of functional depdendency
$D_H$ is the unique function that satisfies \foreachi, $D(f_i) \circ D_H = D(g_i)$.

Let the functor $F_i: \catc \morph \Fin$ be the coproduct $Hom_{\catc}(a_i,-) + Hom_{\catc}(a_i,-)$
in the functor category $Fin^{\catc}$ and label the injections ${L_i}_i$ and $R_i$, respectively so that
for each object $x$ of $\catc$ the diagram
\begin{center}
$
\begin{array}{c p{0.5cm} c p{0.5cm} c  }
\Rnode{h1}{Hom_{\catc}(a_i,x)}  &&\Rnode{Fx}{F_i(x)}  &&   \Rnode{h2}{Hom_{\catc}(a_i,x)}       
\end{array} 
$
\ncarr{h1}{Fx}
\alabel{{L_i}_x}
\ncarr{h2}{Fx}
\blabel{{R_i}_x}
\end{center}
is a coproduct in $\Fin$.

Now for each object $x$ of $\catc$, we define an equivalence relation $\sim_x$ on $F_i(x)$ by defining,
for $k_1,k_2:a_i \morph x$ in $\catc$,
\begin{align*}
{L_i}_x(k_1) \sim_x {R_i}_x(k_2) & \mbox{ iff there exists $k:b \morph x$ in $\catc$ such that $k_1 = f_i \circ k = k_2$,}\\
{L_i}_x(k_1) \sim_x {L_i}_x(k_2) & \mbox{ iff $k_1 = k_2$,} \\
{R_i}_x(k_1) \sim_x {R_i}_x(k_2) & \mbox{ iff $k_1 = k_2$.} \\
\end{align*}
If $j: x_1 \morph x_2$ in $\catc$ and if $y_1,y_2 \in F_i(x_1)$ such that $y_1 \sim_{x_1} y_2$
then it follows easily by cases and from the definition of $\sim$ that $F_i(j)(y_1) \sim_{x_2} F_i(j)(y_2)$.
Therefore we can define a functor 
 $G_i: \catc \morph \Fin$  so that for any object $x$ of $\catc$
the set $G_i(x)$ is the quotient $F_i(x)/{\sim_x}$ and such that 
if $j: x_1 \morph x_2$ in $\catc$ and if $y \in F_i(x_1)$ then $G_i(j)([y])=[F_i(j)(y)]$.
With $G_i$ so defined then if $k: a \morph x_1$ and $j:x_1 \morph x_2$ in $\catc$
then  $G_i(j)([{L_i}_{x_1}(k)])=[{L_i}_{x_2}(k \circ j)]$ and $G_i(j)([R_{x_1}(k)])=[R_{x_2}(k \circ j)]$. 

Now that we have described the functors  $G_i: \catc \morph \Fin$ and $I:\catc \morph \catcp$ that is consistent with $\reqtc$
we can use the fact that $\catc$ is maximally constrained to tell us that $G_i$ extends to a functor 
$G'_i : \catcp \morph \Fin$. Since $f_i \circ \qq{h} = g_i$ in $\catcp$ then we have
 $G'_i(f) \circ G'_i(\qq{h}) = G'_i(g): G_i(a) \morph G_i(c)$ in $\Fin$.
Now we have
\begin{align*}
[{L_i}_c(g_i)]&= G'(g_i)([{L_i}_a(id_a)])              & & \mbox{by definition of $G'(g_i)=G(g_i)$}           \\
        &= G'(\qq{h}) (G'(f_i)([{L_i}_a(id_a)])) & & \mbox{since $G'(f_i) \circ G'(\qq{h}) = G'(g_i)$}  \\
				&= G'(\qq{h}) (G'([{L_i}_b(f_i)]))       & & \mbox{by definition of $G'(f_i)=G(f_i)$}           \\
				&= G'(\qq{h}) (G'([{R_i}_b(f_i)]))       & & \mbox{since ${L_i}_b(f_i) \sim_b {R_i}_b(f_i)$, by definition of $\sim_b$} \\
				&= G'(\qq{h}) (G'(f_i)([{R_i}_a(id_a)])) & & \mbox{by definition of $G'(f_i)=G(f_i)$}           \\
		    &= G'(g_i)([{R_i}_a(id_a)])              & & \mbox{since $G'(f_i) \circ G'(\qq{h}) = G'(g_i)$}  \\
				&= [{R_i}_c(g_i)]                        & & \mbox{by definition of $G'(g_i)=G(g_i)$}           \\
\end{align*} 
Since $[{L_i}_c(g_i)]=[{R_i}_c(g_i)]$ then it follows from the definition of $\sim_c$ that there exists $k_i:b \morph c$ in 
$\catc$ such that $f_i \circ k_i = g_i$.

Now argue that all the $k_i$ are equal. 
Use equational completeness and the fact that \foreachi, $D(k_i)=H_D$.


\end{proof}

