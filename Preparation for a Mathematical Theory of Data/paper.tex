 \documentclass[10pt,a4paper]{article}
%

%ccategories.macros.tex 

% Macros for diagrams in contextual categories and related categories

\usepackage{twoopt}
\usepackage{scalerel} 
\usepackage{xargs}

%\usepackage{mathabx}  %Caused font problems
%\usepackage{MnSymbol}  % caused font problems

\newcommand{\conu}
{\mathbf{C}(U)}

\newcommand{\depu}
{\mathbf{D}(U)}

\newcommand{\cat}[1]{\textbf{#1}}
\newcommand{\obj}[1]{\ensuremath{|\cat{#1}|}}
\newcommand{\ccat}[1][C]{\ensuremath{\mathbb{#1}} }
\newcommand{\ccatc}{contextual category \ccat}
\newcommand{\cobj}[2][]{\ensuremath{|\ccat[#2]|_{#1}}}
\newcommand{\cslice}[2]{\ensuremath{\ccat[#1]_{#2}}}
\newcommand{\csliceobj}[3][]{\ensuremath{|\mathbb{#2}_{#3}|_{#1} }}
\newcommand{\varset}[1][]{\ensuremath{V_{#1} }}
\newcommand{\localvarsets}{\ensuremath{\mathcal{V} }}
\newcommand{\Fam}{\ensuremath{\mathbb{F\mathrm{am}} }}
\newcommand{\Famslice}[1]{\ensuremath{\mathbb{F\mathrm{am}}_{#1} }}
\newcommand{\Famobj}[1][]{\ensuremath{|\mathbb{F\mathrm{am}}|_{#1} }}
\newcommand{\Famsliceobj}[2][]{\ensuremath{|\mathbb{F\mathrm{am}}_{#2}|_{#1} }}
\newcommand{\morph}{\rightarrow}
\newcommand{\epi}{\twoheadrightarrow}
\newcommand{\base}{\triangleleft}
\newcommand{\comp}{\circ}
\newcommand{\cross}{\otimes}
\newcommand{\pc}[2]{d^{#1}_{#2}}
\newcommand{\sub}{^*}
\newcommand{\diag}{\delta}
\newcommand{\pbase}[1]{\tilde{#1}}

\newcommand{\tuple}[1]{\langle#1\rangle}
\newcommand{\ndidly}{\ensuremath{\Join_n}}
\newcommand{\ndidlycospan}{quotiented n-cospan}

\newcommand{\crossx}[3]{#1 \underset{#3}{\cross} #2}
\newcommand{\fibrex}[3]{#1 \underset{#3}{\Join} #2}
\newcommand{\powerset}{\mathcal{P}}
\newcommand{\primeds}[1]{
\ensuremath{\mathcal{P}(#1)} }
\newcommand{\compset}{\ \dot{\circ}\, }

% darrow
%\newcommand{\darrow}{\rightarrowtriangle} %use \smorph instead
\newcommand{\smorph}{\rightarrowtriangle}

 

\newcommand\dhead{\scaleobj{0.6}{\triangleright}}
\newcommand{\dmorph}{\, \mbox{---} \! \cdot \! \raisebox{1.1pt}{\dhead}}

% projection tree
%\newcommand{\proj}[2]{proj_{#2}(#1)}

\newcommand{\proj}[2]{
\ensuremath{\mathcal{P}_{#2}(#1)} }

%pstrick supplements for arrows

\newlength{\arrnodesepA}
\newlength{\arrnodesepB}
\newlength{\arroffsetA}
\newlength{\arroffsetB}

%Modified to 2pt from 0pt on 23 July 2018
\newcommand{\arreset}{
\setlength{\arrnodesepA}{2pt}
\setlength{\arrnodesepB}{2pt}
\setlength{\arroffsetA}{0pt}
\setlength{\arroffsetB}{0pt}
}
\arreset

\newcommand{\ncarr}[3][0]{\ncarc[arcangle=#1,nodesepA=\arrnodesepA,nodesepB=\arrnodesepB,offsetA=\arroffsetA,offsetB=\arroffsetB,arrowsize=5pt,arrowinset=0.7]{->}{#2}{#3}}
\newcommand{\jcbarr}[4][0]{ % ncbarr is defined in some thridy party package so do not use!\emph{}
\ncarr[#1]{#3}{#4}
\nbput[labelsep=2pt]{\footnotesize $#2$}
}

\newcommand{\ncaarr}[4][0]{
\ncarr[#1]{#3}{#4}
\naput[labelsep=2pt]{\footnotesize $#2$}
}

% \alabel{label}[npos][labelsep_pts]
\newcommandx*\alabel[3][2=0.5,3=2,usedefault]{\naput[labelsep=#3pt,npos=#2]{\footnotesize $#1$}}
% \blabel{label}[npos][labelsep_pts]
\newcommandx*\blabel[3][2=0.5,3=2,usedefault]{\nbput[labelsep=#3pt,npos=#2]{\footnotesize $#1$}}

% \idcomp mark an arrow as one component of an identifier
\newcommand{\idcomp}{\ncput[npos=0, nrot=:U]{\psline(0.2,-0.075)(0.2,0.075)}}  %add a bar to a node connection arrow
% pstrick supplements for s-arrows (previous name for d-arrow - should convert}

\newlength{\sarnodesepA}
\newlength{\sarnodesepB}
\newlength{\saroffsetA}
\newlength{\saroffsetB}
\newlength{\sarnodesepAsav}
\newlength{\sarnodesepBsav}

\newcommand{\sarreset}{
\setlength{\sarnodesepA}{0pt}
\setlength{\sarnodesepB}{0pt}
\setlength{\saroffsetA}{0pt}
\setlength{\saroffsetB}{0pt}
}

\sarreset

% sar - S-arrow
\newcommand{\ncsar}[3][0]{
\setlength{\sarnodesepAsav}{\sarnodesepA}
\setlength{\sarnodesepBsav}{\sarnodesepB}
\addtolength{\sarnodesepA}{3pt}
\addtolength{\sarnodesepB}{7pt}
\ncarc[nodesepA=\sarnodesepA,nodesepB=\sarnodesepB,offsetA=\saroffsetA,offsetB=\saroffsetB,arcangle=#1]{-}{#2}{#3}
\ncput[nrot=:R,npos=1]{\pstriangle(0,0)(.2,.2)}
\setlength{\sarnodesepA}{\sarnodesepAsav}
\setlength{\sarnodesepB}{\sarnodesepBsav}
}


% bsar - below labelled S-arrow
\newcommand{\ncbsar}[4][0]{
\ncsar[#1]{#3}{#4}
\nbput[labelsep=2pt]{\footnotesize $#2$}
}
% asar - above labelled S-arrow
\newcommand{\ncasar}[4][0]{
\ncsar[#1]{#3}{#4}
\naput[labelsep=2pt]{\footnotesize $#2$}
}

% cdar - composite dependency arrow
\newcommand{\nccdar}[3][0]{
\setlength{\sarnodesepAsav}{\sarnodesepA}
\setlength{\sarnodesepBsav}{\sarnodesepB}
\addtolength{\sarnodesepA}{3pt}
\addtolength{\sarnodesepB}{11pt}
\ncarc[nodesepA=\sarnodesepA,nodesepB=\sarnodesepB,offsetA=\saroffsetA,offsetB=\saroffsetB,arcangle=#1]{-}{#2}{#3}
\ncput[nrot=:R,npos=1]{\pstriangle(0,0.1)(.2,.2)}
\ncput[nrot=:R,npos=1]{\psdot[dotsize=1pt](-0.0075,0.05)}   %!!
\setlength{\sarnodesepA}{\sarnodesepAsav}
\setlength{\sarnodesepB}{\sarnodesepBsav}
}


% bcdar - below labelled composite dependency arrow
\newcommand{\ncbcdar}[4][0]{
\nccdar[#1]{#3}{#4}
\nbput[labelsep=2pt]{\footnotesize $#2$}
}
% acdar - above labelled composite dependency arrow
\newcommand{\ncacdar}[4][0]{
\nccdar[#1]{#3}{#4}
\naput[labelsep=2pt]{\footnotesize $#2$}
}


% rsar - recursive S-arrow
\newcommand{\ncrsar}[2]{
\setlength{\sarnodesepAsav}{\sarnodesepA}
\setlength{\sarnodesepBsav}{\sarnodesepB}
\addtolength{\sarnodesepA}{3pt}
\addtolength{\sarnodesepB}{7pt}
\ncloop[nodesepA=\sarnodesepA,nodesepB=\sarnodesepB,
        offsetA=\saroffsetA,offsetB=\saroffsetB,
        armA=0.7cm,armB=0.6cm,angleA=90,angleB=-90,loopsize=-1,linearc=0.4
				]{-}{#1}{#2}
\ncput[nrot=:R,npos=5]{\pstriangle(0,0)(.2,.2)}
\setlength{\sarnodesepA}{\sarnodesepAsav}
\setlength{\sarnodesepB}{\sarnodesepBsav}
}

% pstrick supplements for multi-arrows

\newlength{\marnodesepA}
\newlength{\marnodesepB}
\newlength{\maroffsetB}
\newlength{\marnodesepBsav}

\newcommand{\marreset}{
\setlength{\marnodesepA}{0pt}
\setlength{\marnodesepB}{0pt}
\setlength{\maroffsetB}{0pt}
}

\marreset

%ncmarr[#1 arcangle1][#2 arcangle2]{#3 name}{#4 domain1}{#5 domain2}{#6 junction}{#7 codomain}
\newcommandtwoopt{\ncmarr}[6][8][8]{%
\ncarc[nodesepA=\marnodesepA,nodesepB=0,arcangle=#1]{-}{#3}{#5}
\ncarc[nodesepB=0,arcangle=-#1]{-}{#4}{#5}
\ncarc[arcangle=#2,nodesepB=\marnodesepB,offsetB=\maroffsetB]{->}{#5}{#6}
}%


\newcommandtwoopt{\nchmarr}[6][8][8]{%
\ncarc[nodesepA=\marnodesepA,nodesepB=0,arcangle=#1]{-}{#3}{#5}
\ncarc[nodesepB=0,arcangle=#1]{-}{#4}{#5}
\ncarc[arcangle=#2,nodesepB=\marnodesepB,offsetB=\maroffsetB]{->}{#5}{#6}
}%

\newcommandtwoopt{\ncamarr}[7][8][8]{%
\ncmarr[#1][#2]{#4}{#5}{#6}{#7}
\naput[npos=.05]{$#3$}
}%
\newcommandtwoopt{\ncbmarr}[7][8][8]{%
\ncmarr[#1][#2]{#4}{#5}{#6}{#7}
\nbput[npos=.05]{$#3$}
}%

\newcommandtwoopt{\ncbhmarr}[7][8][8]{%
\nchmarr[#1][#2]{#4}{#5}{#6}{#7}
\nbput[npos=.05]{$#3$}
}%

\newcommandtwoopt{\ncmarrr}[7][8][8]{
\ncarc[nodesepB=0,arcangle=#1]{-}{#3}{#6}
\ncline[nodesepB=0]{-}{#4}{#6}
\ncarc[nodesepB=0,arcangle=-#1]{-}{#5}{#6}
\ncarc[nodesepA=0,arcangle=#2]{->}{#6}{#7}
}

\newcommandtwoopt{\ncamarrr}[8][8][8]{
\ncmarrr[#1][#2]{#4}{#5}{#6}{#7}{#8}
\naput[npos=.05]{$#3$}
}
\newcommandtwoopt{\ncbmarrr}[8][8][8]{
\ncmarrr[#1][#2]{#4}{#5}{#6}{#7}{#8}
\nbput[npos=.05]{$#3$}
}

\usepackage[margin=4.0cm]{geometry} %was 3cm
\usepackage{mathptmx}
\usepackage{amsfonts}
\usepackage{array}
\usepackage{pstricks}
\usepackage{pst-tree}
\usepackage{pst-plot}
\usepackage{pst-node}
\usepackage{stmaryrd}
\usepackage{amsmath}
\usepackage{verbatim}
\usepackage{graphicx}  
\usepackage{calc}
\usepackage{xifthen}
\usepackage{xcolor}
\usepackage{color}
\usepackage{stringstrings}
%\usepackage[small,bf,margin=3pt,format=hang, labelsep=endash,singlelinecheck=false]{caption} %prevuiously justification=justified
%\usepackage{enumerate}
%\usepackage{enumitem}
\usepackage{enumerate}
\usepackage[shortlabels]{enumitem}
\usepackage{float}
\usepackage[section]{placeins}
%\setlength{\captionmargin}{5pt}
\usepackage{environ}
\usepackage{multirow}
\usepackage{rotating}
\usepackage{longtable}
\usepackage{afterpage}
\usepackage{needspace}


%DEFINE ENVIRONMENT BLOCK
% Riddle
\newsavebox{\riddlebox}

\newenvironment{erexample}
{\newcommand\colboxcolor{F0F0F0}%was F8F8F8
\begin{lrbox}{\riddlebox}
\begin{minipage}{\dimexpr\columnwidth-2\fboxsep\relax} \textbf{} \\ \itshape}
{\end{minipage}\end{lrbox}%
%\begin{center}
\colorbox[HTML]{\colboxcolor}{\usebox{\riddlebox}}
%\end{center}
}

\newenvironment{erbox}
{\newcommand\colboxcolor{F0F0F0}%was F8F8F8
\begin{lrbox}{\riddlebox}%
\begin{minipage}{\dimexpr\columnwidth-2\fboxsep\relax} }
{\end{minipage}\end{lrbox}%
%\begin{center}
\colorbox[HTML]{\colboxcolor}{\usebox{\riddlebox}}
%\end{center}
}

%\begin{erboxedFigure}{#1 FigureParam}{#2 Label}{#3 Caption}
\NewEnviron{erboxedFigure}[3]{%
\begin{figure}[#1]
\begin{erexample}
\begin{center}
\BODY
\end{center}
\vspace{-0.5cm}
\caption{#3}
\label{#2}
\end{erexample}
\end{figure}
}

\newcommand{\erpictureFolder}[0]{../SharedPictures}

\newcommand{\ercenterPicture}[1]{
\begin{center}
\input{\erpictureFolder/#1}
\end{center}
}


\newlength{\erhalfHt}

%\erinlinePicture{#1 pictureFilename}{#2 pictureHeight}
\newcommand{\erinlinePicture}[2]{
\setlength{\erhalfHt}{#2cm * \real{0.5}}
\raisebox{-\erhalfHt}[\erhalfHt + 0.5cm][\erhalfHt + 0.5cm]{
\input{\erpictureFolder/#1}
} 
}

%\erplainFig{#1 pictureFilename}{#2 figureParam}{#3Caption}
\newcommand{\erplainFig}[3]{
\begin{figure}[#2]
\begin{center}
\input{\erpictureFolder/#1}
\end{center}
\caption{#3}
\label{#1}
\end{figure}
}

%\erboxedFigPicture{#1 pictureFilename}{#2 figureParam}{#3Caption}
\newcommand{\erboxedFigPicture}[3]{
\begin{figure}[#2]
\begin{erexample}
\vspace{-0.5cm}
\begin{center}
\input{\erpictureFolder/#1}
\end{center}
\caption{#3}
\label{#1}
\end{erexample}
\end{figure}
}

%\erLeftSideFig{#1 pictureFilename}{#2 figureParam}{#3Caption}
\newcommand{\erLeftSideFig}[3]{
\begin{figure}[#2]
\begin{erexample}
  \begin{minipage}[c]{0.4\textwidth}
    \caption{#3}
    \label{#1}
  \end{minipage}
  \begin{minipage}[c]{0.5\textwidth}
    \input{\erpictureFolder/#1}
  \end{minipage}
\end{erexample}
\end{figure}
}

%\erbulletedFig{#1 pictureFilename}{#2 figureParam}{#3Caption}
\NewEnviron{erbulletedFig}[3]{%
\begin{figure}[#2]
\begin{erexample}
\vspace{-0.5cm}
\begin{center}
$
\begin{array}{c m{0.25cm} | m{6cm}}
\raisebox{-2.0cm}{
\input{\erpictureFolder/#1}}& & \text{\parbox{6cm}{\raggedright{\footnotesize{
\begin{enumerate}[(i)]
\BODY
\end{enumerate}}}}} \\
\end{array}
$
\end{center}
\caption{#3}
\label{#1}
\end{erexample}
\end{figure} 
}


%\begin{erbulletedDimFig}{#1 pictureFilename}{#2figureParam} {#3Caption} {#4PictureHeight}{#5TextWidth}

\NewEnviron{erbulletedDimFig}[5]{%
\begin{figure}[#2]
\begin{erexample}
\vspace{-0.5cm}
\begin{center}
$
\begin{array}{c m{0.25cm} |  m{#5cm}}
\setlength{\erhalfHt}{#4cm * \real{0.5}}
\raisebox{-\erhalfHt}{
\input{\erpictureFolder/#1}}& & \text{\parbox{#5cm}{\raggedright{\footnotesize{
\begin{enumerate}[(i)]
\BODY
\end{enumerate}}}}} \\
\end{array}
$
\end{center}
\caption{#3}
\label{#1}
\end{erexample}
\end{figure} 
}

%\begin{ernotedModel}{#1 pictureFilename}{#2PictureHeight}{#3PictureWidth}{#4TextWidth}

\NewEnviron{ernotedModel}[4]{%
\begin{center}
$
\begin{array}{m{#3cm} m{1cm} | c m{#4cm}}
\setlength{\erhalfHt}{#2cm * \real{0.5}}
\raisebox{-\erhalfHt}{
\input{\erpictureFolder/#1}}& & & \text{\parbox{#4cm}{\raggedright{\footnotesize{
\BODY
}}}} \\
\end{array}
$
\end{center} 
}

%\begin{ermodelText}{#1 pictureFilename}{#2PictureHeight}{#3PictureWidth}{#4TextWidth}

\NewEnviron{ermodelText}[4]{%
\begin{center}
\begin{tabular}{m{#3cm} m{1cm}  c m{#4cm}}
\setlength{\erhalfHt}{#2cm * \real{0.5}}
\raisebox{-\erhalfHt}{
\input{\erpictureFolder/#1}}& & & \text{\parbox{#4cm}{\raggedright{\small{
\BODY
}}}} \\
\end{tabular}
\end{center} 
}


%\erbulletedModel{#1 pictureFilename}{#2PictureHeight}{#3PictureWidth}{#4TextWidth}

\NewEnviron{erbulletedModel}[4]{%
\begin{center}
$
\begin{array}{m{#3cm} m{1cm} | c m{#4cm}}
\setlength{\erhalfHt}{2cm * \real{0.5}}
\raisebox{-\erhalfHt}{
\input{\erpictureFolder/#1}}& & & \text{\parbox{#4cm}{\raggedright{\footnotesize{
\begin{enumerate}[(i)]
\BODY
\end{enumerate}}}}} \\
\end{array}
$
\end{center} 
}



%\ernotedDimFig{#1 pictureFilename}{#2 figureParam}{#3Caption}{#4PictureHeight}{#5TextWidth}
\NewEnviron{ernotedDimFig}[5]{%
\begin{figure}[#2]
\begin{erexample}
\vspace{-0.5cm}
\begin{center}
$
\begin{array}{c m{0.25cm} | c m{#5cm}}
\setlength{\erhalfHt}{#4cm * \real{0.5}}
\raisebox{-\erhalfHt}{
\input{\erpictureFolder/#1}}& & & \text{\parbox{#5cm}{\raggedright{\footnotesize{
\BODY }}}}\\
\end{array}
$
\end{center}
\caption{#3}
\label{#1}
\end{erexample}
\end{figure} 
}
%\begin{ernotedDimFigPW}{#1 pictureFilename}{#2 figureParam}{#3Caption}{#4PictureHeight}{#5PictureWidth}{#6TextWidth}
\NewEnviron{ernotedDimFigPW}[6]{%
\begin{figure}[#2]
\begin{erexample}
\vspace{-0.5cm}
\begin{center}
$
\begin{array}{>{\centering}m{#5cm} m{0.5cm} | c m{#6cm}}
\setlength{\erhalfHt}{#4cm * \real{0.5}}
\raisebox{-\erhalfHt}{
\centering \input{\erpictureFolder/#1}
}& & & \text{\parbox{#6cm - 0.5cm}{\raggedright{\footnotesize{
\BODY }}}}\\
\end{array}
$ \\
\vspace {0.2cm}
\end{center}
\caption{#3}
\label{#1}
\end{erexample}
\end{figure}
}



\newenvironment{erquote}
{\begin{quote}\itshape}
{\end{quote}}


%
%  erdiag
%
  
%\begin{erdiagram}{#1 height}{#2 width} 
% ....
% ....
%\end{erdiagram}
\newenvironment{erdiagram}[2]
{%\pspicture*(-#1,0)(#2,0)
\pspicture*(0,-#1)(#2,0)
%\psgrid
}
{\endpspicture}

\definecolor{lightyellow}{cmyk}{0,0,0.3,0}

% \eret{#1 x0} {#2 y0} {#3 x1} {#4 y1} {#5 corner radius} {#6 fill}
\newcommand {\eret}[6]
{ 
\ifthenelse{\equal{#6}{1}}
{\psframe[framearc=#5,fillstyle=solid,fillcolor=lightyellow](#1,#2)(#3,#4)}
{\psframe[framearc=#5,fillstyle=solid,fillcolor=white](#1,#2)(#3,#4)}
}

% et top 
\newcommand {\erettop}[4]
{
%\psframe[linestyle=none,linearc=2pt,cornersize=absolute,fillstyle=solid,fillcolor=lightyellow](#1,#2)(#3,#4)
\psline[linearc=2pt,fillstyle=none,fillcolor=lightyellow](#1,#4)(#1,#2)(#3,#2)(#3,#4)
}

% et bottom 
\newcommand {\eretbtm}[4]
{
%\psframe[linestyle=none,linearc=2pt,cornersize=absolute,fillstyle=solid,fillcolor=lightyellow](#1,#2)(#3,#4)
\psline[linearc=2pt,fillstyle=none,fillcolor=lightyellow](#1,#2)(#1,#4)(#3,#4)(#3,#2)
}

% et bottom left
\newcommand {\eretbl}[4]
{
%\psframe[linestyle=none,linearc=2pt,cornersize=absolute,fillstyle=solid,fillcolor=lightyellow](#1,#2)(#3,#4)
\psline[linearc=2pt,fillstyle=none,fillcolor=lightyellow](#1,#4)(#3,#4)(#3,#2)
}

% et middle left
\newcommand {\eretml}[4]
{
%\psframe[linestyle=none,linearc=2pt,cornersize=absolute,fillstyle=solid,fillcolor=lightyellow](#1,#2)(#3,#4)
\psline[linearc=2pt,fillstyle=none,fillcolor=lightyellow](#1,#2)(#3,#2)(#3,#4)(#1,#4)
}

% et top left
\newcommand {\erettl}[4]
{
%\psframe[linestyle=none,linearc=2pt,cornersize=absolute,fillstyle=solid,fillcolor=lightyellow](#1,#2)(#3,#4)
\psline[linearc=2pt,fillstyle=none,fillcolor=lightyellow](#1,#2)(#3,#2)(#3,#4)
}

% et bottom right
\newcommand {\eretbr}[4]
{
%\psframe[linestyle=none,linearc=2pt,cornersize=absolute,fillstyle=solid,fillcolor=lightyellow](#1,#2)(#3,#4)
\psline[linearc=2pt,fillstyle=none,fillcolor=lightyellow](#1,#2)(#1,#4)(#3,#4)
}

% et middle right
\newcommand {\eretmr}[4]
{
%\psframe[linestyle=none,linearc=2pt,cornersize=absolute,fillstyle=solid,fillcolor=lightyellow](#1,#2)(#3,#4)
\psline[linearc=2pt,fillstyle=none,fillcolor=lightyellow](#3,#4)(#1,#4)(#1,#2)(#3,#2)
}

% et top right
\newcommand {\erettr}[4]
{
%\psframe[linestyle=none,linearc=2pt,cornersize=absolute,fillstyle=solid,fillcolor=lightyellow](#1,#2)(#3,#4)
\psline[linearc=2pt,fillstyle=none,fillcolor=lightyellow](#1,#4)(#1,#2)(#3,#2)
}

% \ergrp{#1 x0} {#2 y0} {#3 x1} {#4 y1} {#5 corner radius} {#6 fill}
% #5 corner radius is unused!
\newcommand {\ergrp}[6]
{ 
\ifthenelse{\equal{#6}{1}}
{\psframe[fillstyle=solid,fillcolor=lightgray](#1,#2)(#3,#4)}
{\psframe[fillstyle=solid,fillcolor=white](#1,#2)(#3,#4)}
}

% \eretname {#1 x left of text} {#2 y top of text} {#3 text}
\newcommand {\eretname}[3]
{
%shift down 0.1 for height of text the anchor at baseline (B)
\rput[bl]{0}(0,-0.1){\rput[Bl]{0}(#1,#2){\footnotesize \textit{#3}}}
}

% \errelarm {#1 x0} {#2 y0} {#3 x1} {#4 y1} {#5 ismandatory} {#6 isconstructed}
\newcommand {\errelarm}[6]
{
\ifthenelse{\equal{#6}{1}}
{
%%\psline[linewidth=0.5pt,linearc=.05,linestyle=dashed,dash=6pt 6pt]{-}(#1,#2)(#3,#4)}
\ifthenelse{\equal{#5}{1}}
{\psline[linewidth=1.5pt,linearc=.05,linecolor=lightgray]{-}(#1,#2)(#3,#4)}
{\psline[linewidth=1.5pt,linearc=.05,linecolor=lightgray,linestyle=dashed,dash=2pt 2pt]{-}(#1,#2)(#3,#4)}
}
{
\ifthenelse{\equal{#5}{1}}
{\psline[linewidth=0.9pt,linearc=.05]{-}(#1,#2)(#3,#4)}
{\psline[linewidth=0.9pt,linearc=.05,linestyle=dashed,dash=2pt 2pt]{-}(#1,#2)(#3,#4)}
}
}

% \errelangle {#1 x0} {#2 y0} {#3 x1} {#4 y1} {#5 x2} {#6 y2} {#7 ismandatory} {#8 isocnstructed}
\newcommand {\errelangle}[8]
{
\ifthenelse{\equal{#8}{1}}
{
%\psline[linewidth=0.5pt,linearc=.1,linestyle=dashed,dash=6pt 6pt]{-}(#1,#2)(#3,#4)(#5,#6)}
\ifthenelse{\equal{#7}{1}}
{\psline[linewidth=1.5pt,linearc=.05,linecolor=lightgray]{-}(#1,#2)(#3,#4)(#5,#6)}
{\psline[linewidth=1.5pt,linearc=.1,linecolor=lightgray,linestyle=dashed,dash=2pt 2pt]{-}(#1,#2)(#3,#4)(#5,#6)}
}
{
\ifthenelse{\equal{#7}{1}}
{\psline[linewidth=0.9pt,linearc=.1]{-}(#1,#2)(#3,#4)(#5,#6)}
{\psline[linewidth=0.9pt,linearc=.1,linestyle=dashed,dash=2pt 2pt]{-}(#1,#2)(#3,#4)(#5,#6)}
}
}

% \ercrowfoot {#1 x0} {#2 y0} {#3 x11} {#4 y11} {#5 x12} {#6 y12} {#7 x13} {#8 y13} {#9 isconstructed}
\newcommand {\ercrowfoot}[9]
{
\ifthenelse{\equal{#9}{1}}
{
\psline[linewidth=1.5pt,linearc=.05,linecolor=lightgray]{-}(#1,#2)(#3,#4)
\psline[linewidth=1.5pt,linearc=.05,linecolor=lightgray]{-}(#1,#2)(#5,#6)
\psline[linewidth=1.5pt,linearc=.05,linecolor=lightgray]{-}(#1,#2)(#7,#8)
}{
\psline[linewidth=0.9pt,linearc=.05]{-}(#1,#2)(#3,#4)
\psline[linewidth=0.9pt,linearc=.05]{-}(#1,#2)(#5,#6)
\psline[linewidth=0.9pt,linearc=.05]{-}(#1,#2)(#7,#8)
}
}


% \eridcomprel{#1 x1}{#2 x2}{#3 y1}{#4 ymid}{#5 y2}
\newcommand {\eridcomprel}[5]
{
\psline[linewidth=0.9pt](#1,#3)(#1,#5)
\psline[linewidth=0.9pt](#2,#3)(#2,#5)
\psline[linewidth=0.9pt](#1,#4)(#2,#4)
}

% \eridrefrel{#1 x1}{#2 xmid}{#3 x2}{#4 y1}{#5 y2}
\newcommand {\eridrefrel}[5]
{
\psline[linewidth=0.9pt](#1,#4)(#3,#4)
\psline[linewidth=0.9pt](#1,#5)(#3,#5)
\psline[linewidth=0.9pt](#2,#4)(#2,#5)
}


% \errelname {#1 x} {#2 y} {#3 text}
\newcommand {\errelname}[3]
{
\rput[l]{0}(#1,#2){\textit{#3}}
}
% \errelseq {#1 x} {#2 y}
\newcommand {\erelseq}[2]
{
}
% \erattr {#1 x} {#2 y} {#3 ismandatory}{#4 idenitfying} {#5 text}
\newcommand {\erattr}[5]
{
\ifthenelse{\equal{#3}{1}}
{\rput[l]{0}(#1,#2){{\tiny $\square$} {\footnotesize \textit{\ifthenelse{\equal{#4}{0}}{\underline{#5}}{#5}}}}}
{\rput[l]{0}(#1,#2){\footnotesize $\circ$ \textit{\ifthenelse{\equal{#4}{0}}{\underline{#5}}{#5}}}}
}

%\ifthenelse{\equal{#4}{1}}
% \ertext {#1 x} {#2 y} {#3 text anchor} {#4 text}
%{\rput[l]{0}(#1,#2){\footnotesize $\circ$ \underline{\textit{#5}}}}
\newcommand {\ertext}[4]
{
\rput[B#3]{0}(#1,#2){{\footnotesize #4}}
}
% \erarc {#1 x0} {#2 y0} {#3 x1} {#4 y1} {#5 x2} {#6 y2} {#7 x3} {#8 y3}
\newcommand {\erarc}[8]
{
\psbezier[showpoints=false]{-}(#1,#2) (#3, #4)(#5,#6) (#7, #8)
}

% \erarc {#1 x0} {#2 y0} {#3 x1} {#4 y1} {#5 x2} {#6 y2} {#7 x3} {#8 y3}
\newcommand {\errelseq}[8]
{
\psbezier[showpoints=false]{-}(#1,#2) (#3, #4)(#5,#6) (#7, #8)
}
% \ertrace {#1 trace}   
\newcommand {\ertrace}[1]
{
}

\usepackage{amsthm} % added 7th April 2018
% theorems.macros.tex

\newtheorem{theorem}{Theorem}[section]
\newtheorem{observation}[theorem]{Observation}
\newtheorem{lemma}[theorem]{Lemma}

\newtheorem{proposition}[theorem]{Proposition}
\newtheorem{corollary}[theorem]{Corollary}
\newtheorem{conjecture}[theorem]{Conjecture}
\newtheorem{numbereddefinition}[theorem]{Definition}

\newenvironment{definition}[1][Definition]{\begin{trivlist}
\item[\hskip \labelsep {\bfseries #1}]}{\end{trivlist}}
\newenvironment{examples}[1][Examples]{\begin{trivlist}
\item[\hskip \labelsep {\bfseries #1}]}{\end{trivlist}}
\newenvironment{example}[1][Example]{\begin{trivlist}
\item[\hskip \labelsep {\bfseries #1}]}{\end{trivlist}}
\newenvironment{remark}[1][Remark]{\begin{trivlist}
\item[\hskip \labelsep {\bfseries #1}]}{\end{trivlist}}

\newenvironment{tageqn}[1]
{
\begin{equation}
\stepcounter{equation}
\label{#1}
\tag{\theequation --#1}
}
{
\end{equation}
}

\newenvironment{axiom}[1]
{
\begin{equation}
\label{#1}
\tag{#1}
}
{
\end{equation}
}

% when the tag is required different from the label eg when has math symbols can use:
\newenvironment{axiomtagged}[2]
{
\begin{equation}
\label{#1}
\tag{#2}
}
{
\end{equation}
}

%visible label
\newcommand{\vlabel}[2][]{\label{#2}#1(\textit{#2}):}





\usepackage{mathptmx}  % This changes font to roman
\usepackage{anyfontsize}
\usepackage{mathtools}  % why have we got this?
\usepackage{alltt}    
\usepackage{mnsymbol} %used for rightpitchfork
\usepackage{cmll}
\usepackage{ulem}
\renewcommand{\ttdefault}{txtt}
\usepackage[left=1.5cm, right=4cm, marginparwidth=3cm, top=2cm, bottom=1.5cm]{geometry}
\usepackage{framed}
\usepackage[font=small]{caption}
\setlength{\captionmargin}{2cm}
\newcommand{\commentary}[1]{\marginpar{\footnotesize #1}}

\renewcommand{\erpictureFolder}[0]{../SharedPictures}

\newenvironment{categoricalaside}
{\begin{framed}
\textbf{Categorical Aside}
}
{
\end{framed}
}

\newenvironment{noteforfuture}
{\begin{framed}
\textbf{Note For Future}
}
{
\end{framed}
}

\newenvironment{problem}
{\begin{framed}
\textbf{Problem}
}
{
\end{framed}
}


%from berkley
\newcommand{\langl}{\begin{picture}(4.5,7)
\put(1.1,2.5){\rotatebox{60}{\line(1,0){5.5}}}
\put(1.1,2.5){\rotatebox{300}{\line(1,0){5.5}}}
\end{picture}}
\newcommand{\rangl}{\begin{picture}(4.5,7)
\put(.9,2.5){\rotatebox{120}{\line(1,0){5.5}}}
\put(.9,2.5){\rotatebox{240}{\line(1,0){5.5}}}
\end{picture}}
\newcommand{\lang}{\begin{picture}(5,7)\put(1.1,2.5){\rotatebox{45}{\line(1,0){6.0}}}\put(1.1,2.5){\rotatebox{315}{\line(1,0){6.0}}}\end{picture}}
\newcommand{\rang}{\begin{picture}(5,7)\put(.1,2.5){\rotatebox{135}{\line(1,0){6.0}}}\put(.1,2.5){\rotatebox{225}{\line(1,0){6.0}}}\end{picture}}
%Try sharper tuple brackets -- except gives errors nested in captions so comment out
%\renewcommand{\tuple}[1]{\lang #1 \rang}

\newcommand{\setsuchthat}[2]{\left\{#1 \ \middle|\ #2\right\}}
\newcommand{\set}[1]{\left\{#1\right\}} 

\newcommand{\genericmodel}{\mathcal{M}}  %PREVIOUSLY
\renewcommand{\genericmodel}{{m}}        %PREVIOUSLY
\renewcommand{\genericmodel}{\gamma}     % TRY THIS FOR A WHILE except texworks isnt happy with greek
%\renewcommand{\genericmodel}{M}  %while debugging
\newcommand{\chiZero}{\mathcal{X}_0}
\newcommand{\chiZeroM}{\chiZero(\genericmodel)}
\newcommand{\chiOne}{\mathcal{X}_1}
\newcommand{\chiOneM}{\chiOne(\genericmodel)}
\newcommand{\chiM}{\mathcal{X}(\genericmodel)}
\newcommand{\veee}{v}
\newcommand{\Veee}{V}
\newcommand{\et}[1][\genericmodel]{et_{#1}}
\newcommand{\edge}[3][\genericmodel]{Edge_{#1}(#2,#3)}
\newcommand{\iedge}[3][\genericmodel]{IEdge_{#1}(#2,#3)}
\newcommand{\path}[3][\genericmodel]{Path_{#1}(#2,#3)}
\newcommand{\ipath}[3][\genericmodel]{IPath_{#1}(#2,#3)}
\newcommand{\attr}[2] [\genericmodel]{attr_{#1}(#2)}
\newcommand{\iattr}[2] [\genericmodel]{IAttr_{#1}(#2)}
\newcommand{\rel}[3][\genericmodel]{rel_{#1}(#2,#3)}
\newcommand{\irel}[3][\genericmodel]{IRel_{#1}(#2,#3)}
\newcommand{\iedges}[2] [\genericmodel]{i_{#1}(#2)}
\newcommand{\pk}[2] [\genericmodel]{pk_{#1}(#2)}
\newcommand{\fk}[2] [\genericmodel]{fk_{#1}(#2)}
\newcommand{\fkp}[2] [\genericmodel]{fk'_{#1}(#2)}
\newcommand{\fkpp}[2] [\genericmodel]{fk''_{#1}(#2)}

%functional dependencies
\newcommand{\sfd}[2]{\ensuremath{\set{#1} \morph #2}}  %singleton
\newcommand{\fd}[2]{\ensuremath{\sfd{#1}{\set{#2}}}}

\newcommand{\simplepath}[2]{
\ncline[linestyle=none,linewidth=0.1pt]{#1}{#2}   %was linestyle=dotted
\ncput[npos=0.05]{\pnode{dot#21}}
\ncput[npos=0.27]{\dotnode[dotsize=1pt]{dot#22}}
\ncput[npos=0.50]{\dotnode[dotsize=1pt]{dot#23}}
\ncput[npos=0.80]{\dotnode[dotsize=1pt]{dot#24}}
\ncput[npos=0.975]{\pnode{dot#25}}
\ncline[nodesep=2pt]{->}{dot#21}{dot#22}
\ncline[nodesep=2pt]{->}{dot#22}{dot#23}
\ncline[nodesep=2pt]{->}{dot#24}{dot#25}
\ncline[linestyle=dotted,nodesep=8pt]{dot#23}{dot#24} %was 10pt
}

\newcommand{\simplepatha}[3]{
\simplepath{#2}{#3}
\naput[labelsep=1pt]{#1}
}

\newcommand{\simplepathb}[3]{
\simplepath{#2}{#3}
\nbput[labelsep=1pt]{#1}
}
\newcommand{\term}[1]{\textit{{#1}}}
\newcommand{\logtophys}{\mathcal{X}}
\newcommand{\chen}{\mathcal{X}_0}
\newcommand{\chengenericmodel}{\chen(\genericmodel)}
\newcommand{\chigenericmodel}{\logtophys(\genericmodel)}
\newcommand{\phys}[1]{\overline{#1}}
\newcommand{\genericphysical}{\logtophys(\genericmodel)}

\newcommand{\inc}{\subseteq}
\newcommand{\incd}[4]{#1\left[#2\right]\inc#3\left[#4\right]}

\newcommand{\ntuple}[1]{\tuple{#1_1,...#1_n}}
\newcommand{\mtuple}[1]{\tuple{#1_1,...#1_m}}

\newcommand {\bntuple}{\ensuremath{\ntuple{b}}}
\newcommand {\fntuple}{\ensuremath{\ntuple{f}}}
\newcommand {\pntuple}{\ensuremath{\ntuple{p}}}
\newcommand {\qntuple}{\ensuremath{\ntuple{q}}}
\newcommand {\qmtuple}{\ensuremath{\mtuple{q}}}
\newcommand {\xntuple}{\ensuremath{\ntuple{x}}}
\newcommand {\ymtuple}{\ensuremath{\mtuple{y}}}
\newcommand{\foreachi}[1][n]{for each $i$, $1 \leq i \leq #1$}
\newcommand{\foreachj}[1][m]{for each $j$, $1 \leq j \leq #1$}
\newcommand{\foreachk}[1][l]{for each $k$, $1 \leq k \leq #1$}
\newcommand{\fdfactoring}{fd factoring}
\newcommand{\attributelike}{attribute-like}



%ccategories.macros.tex 

% Macros for diagrams in contextual categories and related categories

\usepackage{twoopt}
\usepackage{scalerel} 
\usepackage{xargs}

%\usepackage{mathabx}  %Caused font problems
%\usepackage{MnSymbol}  % caused font problems

\newcommand{\conu}
{\mathbf{C}(U)}

\newcommand{\depu}
{\mathbf{D}(U)}

\newcommand{\cat}[1]{\textbf{#1}}
\newcommand{\obj}[1]{\ensuremath{|\cat{#1}|}}
\newcommand{\ccat}[1][C]{\ensuremath{\mathbb{#1}} }
\newcommand{\ccatc}{contextual category \ccat}
\newcommand{\cobj}[2][]{\ensuremath{|\ccat[#2]|_{#1}}}
\newcommand{\cslice}[2]{\ensuremath{\ccat[#1]_{#2}}}
\newcommand{\csliceobj}[3][]{\ensuremath{|\mathbb{#2}_{#3}|_{#1} }}
\newcommand{\varset}[1][]{\ensuremath{V_{#1} }}
\newcommand{\localvarsets}{\ensuremath{\mathcal{V} }}
\newcommand{\Fam}{\ensuremath{\mathbb{F\mathrm{am}} }}
\newcommand{\Famslice}[1]{\ensuremath{\mathbb{F\mathrm{am}}_{#1} }}
\newcommand{\Famobj}[1][]{\ensuremath{|\mathbb{F\mathrm{am}}|_{#1} }}
\newcommand{\Famsliceobj}[2][]{\ensuremath{|\mathbb{F\mathrm{am}}_{#2}|_{#1} }}
\newcommand{\morph}{\rightarrow}
\newcommand{\epi}{\twoheadrightarrow}
\newcommand{\base}{\triangleleft}
\newcommand{\comp}{\circ}
\newcommand{\cross}{\otimes}
\newcommand{\pc}[2]{d^{#1}_{#2}}
\newcommand{\sub}{^*}
\newcommand{\diag}{\delta}
\newcommand{\pbase}[1]{\tilde{#1}}

\newcommand{\tuple}[1]{\langle#1\rangle}
\newcommand{\ndidly}{\ensuremath{\Join_n}}
\newcommand{\ndidlycospan}{quotiented n-cospan}

\newcommand{\crossx}[3]{#1 \underset{#3}{\cross} #2}
\newcommand{\fibrex}[3]{#1 \underset{#3}{\Join} #2}
\newcommand{\powerset}{\mathcal{P}}
\newcommand{\primeds}[1]{
\ensuremath{\mathcal{P}(#1)} }
\newcommand{\compset}{\ \dot{\circ}\, }

% darrow
%\newcommand{\darrow}{\rightarrowtriangle} %use \smorph instead
\newcommand{\smorph}{\rightarrowtriangle}

 

\newcommand\dhead{\scaleobj{0.6}{\triangleright}}
\newcommand{\dmorph}{\, \mbox{---} \! \cdot \! \raisebox{1.1pt}{\dhead}}

% projection tree
%\newcommand{\proj}[2]{proj_{#2}(#1)}

\newcommand{\proj}[2]{
\ensuremath{\mathcal{P}_{#2}(#1)} }

%pstrick supplements for arrows

\newlength{\arrnodesepA}
\newlength{\arrnodesepB}
\newlength{\arroffsetA}
\newlength{\arroffsetB}

%Modified to 2pt from 0pt on 23 July 2018
\newcommand{\arreset}{
\setlength{\arrnodesepA}{2pt}
\setlength{\arrnodesepB}{2pt}
\setlength{\arroffsetA}{0pt}
\setlength{\arroffsetB}{0pt}
}
\arreset

\newcommand{\ncarr}[3][0]{\ncarc[arcangle=#1,nodesepA=\arrnodesepA,nodesepB=\arrnodesepB,offsetA=\arroffsetA,offsetB=\arroffsetB,arrowsize=5pt,arrowinset=0.7]{->}{#2}{#3}}
\newcommand{\jcbarr}[4][0]{ % ncbarr is defined in some thridy party package so do not use!\emph{}
\ncarr[#1]{#3}{#4}
\nbput[labelsep=2pt]{\footnotesize $#2$}
}

\newcommand{\ncaarr}[4][0]{
\ncarr[#1]{#3}{#4}
\naput[labelsep=2pt]{\footnotesize $#2$}
}

% \alabel{label}[npos][labelsep_pts]
\newcommandx*\alabel[3][2=0.5,3=2,usedefault]{\naput[labelsep=#3pt,npos=#2]{\footnotesize $#1$}}
% \blabel{label}[npos][labelsep_pts]
\newcommandx*\blabel[3][2=0.5,3=2,usedefault]{\nbput[labelsep=#3pt,npos=#2]{\footnotesize $#1$}}

% \idcomp mark an arrow as one component of an identifier
\newcommand{\idcomp}{\ncput[npos=0, nrot=:U]{\psline(0.2,-0.075)(0.2,0.075)}}  %add a bar to a node connection arrow
% pstrick supplements for s-arrows (previous name for d-arrow - should convert}

\newlength{\sarnodesepA}
\newlength{\sarnodesepB}
\newlength{\saroffsetA}
\newlength{\saroffsetB}
\newlength{\sarnodesepAsav}
\newlength{\sarnodesepBsav}

\newcommand{\sarreset}{
\setlength{\sarnodesepA}{0pt}
\setlength{\sarnodesepB}{0pt}
\setlength{\saroffsetA}{0pt}
\setlength{\saroffsetB}{0pt}
}

\sarreset

% sar - S-arrow
\newcommand{\ncsar}[3][0]{
\setlength{\sarnodesepAsav}{\sarnodesepA}
\setlength{\sarnodesepBsav}{\sarnodesepB}
\addtolength{\sarnodesepA}{3pt}
\addtolength{\sarnodesepB}{7pt}
\ncarc[nodesepA=\sarnodesepA,nodesepB=\sarnodesepB,offsetA=\saroffsetA,offsetB=\saroffsetB,arcangle=#1]{-}{#2}{#3}
\ncput[nrot=:R,npos=1]{\pstriangle(0,0)(.2,.2)}
\setlength{\sarnodesepA}{\sarnodesepAsav}
\setlength{\sarnodesepB}{\sarnodesepBsav}
}


% bsar - below labelled S-arrow
\newcommand{\ncbsar}[4][0]{
\ncsar[#1]{#3}{#4}
\nbput[labelsep=2pt]{\footnotesize $#2$}
}
% asar - above labelled S-arrow
\newcommand{\ncasar}[4][0]{
\ncsar[#1]{#3}{#4}
\naput[labelsep=2pt]{\footnotesize $#2$}
}

% cdar - composite dependency arrow
\newcommand{\nccdar}[3][0]{
\setlength{\sarnodesepAsav}{\sarnodesepA}
\setlength{\sarnodesepBsav}{\sarnodesepB}
\addtolength{\sarnodesepA}{3pt}
\addtolength{\sarnodesepB}{11pt}
\ncarc[nodesepA=\sarnodesepA,nodesepB=\sarnodesepB,offsetA=\saroffsetA,offsetB=\saroffsetB,arcangle=#1]{-}{#2}{#3}
\ncput[nrot=:R,npos=1]{\pstriangle(0,0.1)(.2,.2)}
\ncput[nrot=:R,npos=1]{\psdot[dotsize=1pt](-0.0075,0.05)}   %!!
\setlength{\sarnodesepA}{\sarnodesepAsav}
\setlength{\sarnodesepB}{\sarnodesepBsav}
}


% bcdar - below labelled composite dependency arrow
\newcommand{\ncbcdar}[4][0]{
\nccdar[#1]{#3}{#4}
\nbput[labelsep=2pt]{\footnotesize $#2$}
}
% acdar - above labelled composite dependency arrow
\newcommand{\ncacdar}[4][0]{
\nccdar[#1]{#3}{#4}
\naput[labelsep=2pt]{\footnotesize $#2$}
}


% rsar - recursive S-arrow
\newcommand{\ncrsar}[2]{
\setlength{\sarnodesepAsav}{\sarnodesepA}
\setlength{\sarnodesepBsav}{\sarnodesepB}
\addtolength{\sarnodesepA}{3pt}
\addtolength{\sarnodesepB}{7pt}
\ncloop[nodesepA=\sarnodesepA,nodesepB=\sarnodesepB,
        offsetA=\saroffsetA,offsetB=\saroffsetB,
        armA=0.7cm,armB=0.6cm,angleA=90,angleB=-90,loopsize=-1,linearc=0.4
				]{-}{#1}{#2}
\ncput[nrot=:R,npos=5]{\pstriangle(0,0)(.2,.2)}
\setlength{\sarnodesepA}{\sarnodesepAsav}
\setlength{\sarnodesepB}{\sarnodesepBsav}
}

% pstrick supplements for multi-arrows

\newlength{\marnodesepA}
\newlength{\marnodesepB}
\newlength{\maroffsetB}
\newlength{\marnodesepBsav}

\newcommand{\marreset}{
\setlength{\marnodesepA}{0pt}
\setlength{\marnodesepB}{0pt}
\setlength{\maroffsetB}{0pt}
}

\marreset

%ncmarr[#1 arcangle1][#2 arcangle2]{#3 name}{#4 domain1}{#5 domain2}{#6 junction}{#7 codomain}
\newcommandtwoopt{\ncmarr}[6][8][8]{%
\ncarc[nodesepA=\marnodesepA,nodesepB=0,arcangle=#1]{-}{#3}{#5}
\ncarc[nodesepB=0,arcangle=-#1]{-}{#4}{#5}
\ncarc[arcangle=#2,nodesepB=\marnodesepB,offsetB=\maroffsetB]{->}{#5}{#6}
}%


\newcommandtwoopt{\nchmarr}[6][8][8]{%
\ncarc[nodesepA=\marnodesepA,nodesepB=0,arcangle=#1]{-}{#3}{#5}
\ncarc[nodesepB=0,arcangle=#1]{-}{#4}{#5}
\ncarc[arcangle=#2,nodesepB=\marnodesepB,offsetB=\maroffsetB]{->}{#5}{#6}
}%

\newcommandtwoopt{\ncamarr}[7][8][8]{%
\ncmarr[#1][#2]{#4}{#5}{#6}{#7}
\naput[npos=.05]{$#3$}
}%
\newcommandtwoopt{\ncbmarr}[7][8][8]{%
\ncmarr[#1][#2]{#4}{#5}{#6}{#7}
\nbput[npos=.05]{$#3$}
}%

\newcommandtwoopt{\ncbhmarr}[7][8][8]{%
\nchmarr[#1][#2]{#4}{#5}{#6}{#7}
\nbput[npos=.05]{$#3$}
}%

\newcommandtwoopt{\ncmarrr}[7][8][8]{
\ncarc[nodesepB=0,arcangle=#1]{-}{#3}{#6}
\ncline[nodesepB=0]{-}{#4}{#6}
\ncarc[nodesepB=0,arcangle=-#1]{-}{#5}{#6}
\ncarc[nodesepA=0,arcangle=#2]{->}{#6}{#7}
}

\newcommandtwoopt{\ncamarrr}[8][8][8]{
\ncmarrr[#1][#2]{#4}{#5}{#6}{#7}{#8}
\naput[npos=.05]{$#3$}
}
\newcommandtwoopt{\ncbmarrr}[8][8][8]{
\ncmarrr[#1][#2]{#4}{#5}{#6}{#7}{#8}
\nbput[npos=.05]{$#3$}
}

\usepackage[margin=4.0cm]{geometry} %was 3cm
\usepackage{mathptmx}
\usepackage{amsfonts}
\usepackage{array}
\usepackage{pstricks}
\usepackage{pst-tree}
\usepackage{pst-plot}
\usepackage{pst-node}
\usepackage{stmaryrd}
\usepackage{amsmath}
\usepackage{verbatim}
\usepackage{graphicx}  
\usepackage{calc}
\usepackage{xifthen}
\usepackage{xcolor}
\usepackage{color}
\usepackage{stringstrings}
%\usepackage[small,bf,margin=3pt,format=hang, labelsep=endash,singlelinecheck=false]{caption} %prevuiously justification=justified
%\usepackage{enumerate}
%\usepackage{enumitem}
\usepackage{enumerate}
\usepackage[shortlabels]{enumitem}
\usepackage{float}
\usepackage[section]{placeins}
%\setlength{\captionmargin}{5pt}
\usepackage{environ}
\usepackage{multirow}
\usepackage{rotating}
\usepackage{longtable}
\usepackage{afterpage}
\usepackage{needspace}


%DEFINE ENVIRONMENT BLOCK
% Riddle
\newsavebox{\riddlebox}

\newenvironment{erexample}
{\newcommand\colboxcolor{F0F0F0}%was F8F8F8
\begin{lrbox}{\riddlebox}
\begin{minipage}{\dimexpr\columnwidth-2\fboxsep\relax} \textbf{} \\ \itshape}
{\end{minipage}\end{lrbox}%
%\begin{center}
\colorbox[HTML]{\colboxcolor}{\usebox{\riddlebox}}
%\end{center}
}

\newenvironment{erbox}
{\newcommand\colboxcolor{F0F0F0}%was F8F8F8
\begin{lrbox}{\riddlebox}%
\begin{minipage}{\dimexpr\columnwidth-2\fboxsep\relax} }
{\end{minipage}\end{lrbox}%
%\begin{center}
\colorbox[HTML]{\colboxcolor}{\usebox{\riddlebox}}
%\end{center}
}

%\begin{erboxedFigure}{#1 FigureParam}{#2 Label}{#3 Caption}
\NewEnviron{erboxedFigure}[3]{%
\begin{figure}[#1]
\begin{erexample}
\begin{center}
\BODY
\end{center}
\vspace{-0.5cm}
\caption{#3}
\label{#2}
\end{erexample}
\end{figure}
}

\newcommand{\erpictureFolder}[0]{../SharedPictures}

\newcommand{\ercenterPicture}[1]{
\begin{center}
\input{\erpictureFolder/#1}
\end{center}
}


\newlength{\erhalfHt}

%\erinlinePicture{#1 pictureFilename}{#2 pictureHeight}
\newcommand{\erinlinePicture}[2]{
\setlength{\erhalfHt}{#2cm * \real{0.5}}
\raisebox{-\erhalfHt}[\erhalfHt + 0.5cm][\erhalfHt + 0.5cm]{
\input{\erpictureFolder/#1}
} 
}

%\erplainFig{#1 pictureFilename}{#2 figureParam}{#3Caption}
\newcommand{\erplainFig}[3]{
\begin{figure}[#2]
\begin{center}
\input{\erpictureFolder/#1}
\end{center}
\caption{#3}
\label{#1}
\end{figure}
}

%\erboxedFigPicture{#1 pictureFilename}{#2 figureParam}{#3Caption}
\newcommand{\erboxedFigPicture}[3]{
\begin{figure}[#2]
\begin{erexample}
\vspace{-0.5cm}
\begin{center}
\input{\erpictureFolder/#1}
\end{center}
\caption{#3}
\label{#1}
\end{erexample}
\end{figure}
}

%\erLeftSideFig{#1 pictureFilename}{#2 figureParam}{#3Caption}
\newcommand{\erLeftSideFig}[3]{
\begin{figure}[#2]
\begin{erexample}
  \begin{minipage}[c]{0.4\textwidth}
    \caption{#3}
    \label{#1}
  \end{minipage}
  \begin{minipage}[c]{0.5\textwidth}
    \input{\erpictureFolder/#1}
  \end{minipage}
\end{erexample}
\end{figure}
}

%\erbulletedFig{#1 pictureFilename}{#2 figureParam}{#3Caption}
\NewEnviron{erbulletedFig}[3]{%
\begin{figure}[#2]
\begin{erexample}
\vspace{-0.5cm}
\begin{center}
$
\begin{array}{c m{0.25cm} | m{6cm}}
\raisebox{-2.0cm}{
\input{\erpictureFolder/#1}}& & \text{\parbox{6cm}{\raggedright{\footnotesize{
\begin{enumerate}[(i)]
\BODY
\end{enumerate}}}}} \\
\end{array}
$
\end{center}
\caption{#3}
\label{#1}
\end{erexample}
\end{figure} 
}


%\begin{erbulletedDimFig}{#1 pictureFilename}{#2figureParam} {#3Caption} {#4PictureHeight}{#5TextWidth}

\NewEnviron{erbulletedDimFig}[5]{%
\begin{figure}[#2]
\begin{erexample}
\vspace{-0.5cm}
\begin{center}
$
\begin{array}{c m{0.25cm} |  m{#5cm}}
\setlength{\erhalfHt}{#4cm * \real{0.5}}
\raisebox{-\erhalfHt}{
\input{\erpictureFolder/#1}}& & \text{\parbox{#5cm}{\raggedright{\footnotesize{
\begin{enumerate}[(i)]
\BODY
\end{enumerate}}}}} \\
\end{array}
$
\end{center}
\caption{#3}
\label{#1}
\end{erexample}
\end{figure} 
}

%\begin{ernotedModel}{#1 pictureFilename}{#2PictureHeight}{#3PictureWidth}{#4TextWidth}

\NewEnviron{ernotedModel}[4]{%
\begin{center}
$
\begin{array}{m{#3cm} m{1cm} | c m{#4cm}}
\setlength{\erhalfHt}{#2cm * \real{0.5}}
\raisebox{-\erhalfHt}{
\input{\erpictureFolder/#1}}& & & \text{\parbox{#4cm}{\raggedright{\footnotesize{
\BODY
}}}} \\
\end{array}
$
\end{center} 
}

%\begin{ermodelText}{#1 pictureFilename}{#2PictureHeight}{#3PictureWidth}{#4TextWidth}

\NewEnviron{ermodelText}[4]{%
\begin{center}
\begin{tabular}{m{#3cm} m{1cm}  c m{#4cm}}
\setlength{\erhalfHt}{#2cm * \real{0.5}}
\raisebox{-\erhalfHt}{
\input{\erpictureFolder/#1}}& & & \text{\parbox{#4cm}{\raggedright{\small{
\BODY
}}}} \\
\end{tabular}
\end{center} 
}


%\erbulletedModel{#1 pictureFilename}{#2PictureHeight}{#3PictureWidth}{#4TextWidth}

\NewEnviron{erbulletedModel}[4]{%
\begin{center}
$
\begin{array}{m{#3cm} m{1cm} | c m{#4cm}}
\setlength{\erhalfHt}{2cm * \real{0.5}}
\raisebox{-\erhalfHt}{
\input{\erpictureFolder/#1}}& & & \text{\parbox{#4cm}{\raggedright{\footnotesize{
\begin{enumerate}[(i)]
\BODY
\end{enumerate}}}}} \\
\end{array}
$
\end{center} 
}



%\ernotedDimFig{#1 pictureFilename}{#2 figureParam}{#3Caption}{#4PictureHeight}{#5TextWidth}
\NewEnviron{ernotedDimFig}[5]{%
\begin{figure}[#2]
\begin{erexample}
\vspace{-0.5cm}
\begin{center}
$
\begin{array}{c m{0.25cm} | c m{#5cm}}
\setlength{\erhalfHt}{#4cm * \real{0.5}}
\raisebox{-\erhalfHt}{
\input{\erpictureFolder/#1}}& & & \text{\parbox{#5cm}{\raggedright{\footnotesize{
\BODY }}}}\\
\end{array}
$
\end{center}
\caption{#3}
\label{#1}
\end{erexample}
\end{figure} 
}
%\begin{ernotedDimFigPW}{#1 pictureFilename}{#2 figureParam}{#3Caption}{#4PictureHeight}{#5PictureWidth}{#6TextWidth}
\NewEnviron{ernotedDimFigPW}[6]{%
\begin{figure}[#2]
\begin{erexample}
\vspace{-0.5cm}
\begin{center}
$
\begin{array}{>{\centering}m{#5cm} m{0.5cm} | c m{#6cm}}
\setlength{\erhalfHt}{#4cm * \real{0.5}}
\raisebox{-\erhalfHt}{
\centering \input{\erpictureFolder/#1}
}& & & \text{\parbox{#6cm - 0.5cm}{\raggedright{\footnotesize{
\BODY }}}}\\
\end{array}
$ \\
\vspace {0.2cm}
\end{center}
\caption{#3}
\label{#1}
\end{erexample}
\end{figure}
}



\newenvironment{erquote}
{\begin{quote}\itshape}
{\end{quote}}


%
%  erdiag
%
  
%\begin{erdiagram}{#1 height}{#2 width} 
% ....
% ....
%\end{erdiagram}
\newenvironment{erdiagram}[2]
{%\pspicture*(-#1,0)(#2,0)
\pspicture*(0,-#1)(#2,0)
%\psgrid
}
{\endpspicture}

\definecolor{lightyellow}{cmyk}{0,0,0.3,0}

% \eret{#1 x0} {#2 y0} {#3 x1} {#4 y1} {#5 corner radius} {#6 fill}
\newcommand {\eret}[6]
{ 
\ifthenelse{\equal{#6}{1}}
{\psframe[framearc=#5,fillstyle=solid,fillcolor=lightyellow](#1,#2)(#3,#4)}
{\psframe[framearc=#5,fillstyle=solid,fillcolor=white](#1,#2)(#3,#4)}
}

% et top 
\newcommand {\erettop}[4]
{
%\psframe[linestyle=none,linearc=2pt,cornersize=absolute,fillstyle=solid,fillcolor=lightyellow](#1,#2)(#3,#4)
\psline[linearc=2pt,fillstyle=none,fillcolor=lightyellow](#1,#4)(#1,#2)(#3,#2)(#3,#4)
}

% et bottom 
\newcommand {\eretbtm}[4]
{
%\psframe[linestyle=none,linearc=2pt,cornersize=absolute,fillstyle=solid,fillcolor=lightyellow](#1,#2)(#3,#4)
\psline[linearc=2pt,fillstyle=none,fillcolor=lightyellow](#1,#2)(#1,#4)(#3,#4)(#3,#2)
}

% et bottom left
\newcommand {\eretbl}[4]
{
%\psframe[linestyle=none,linearc=2pt,cornersize=absolute,fillstyle=solid,fillcolor=lightyellow](#1,#2)(#3,#4)
\psline[linearc=2pt,fillstyle=none,fillcolor=lightyellow](#1,#4)(#3,#4)(#3,#2)
}

% et middle left
\newcommand {\eretml}[4]
{
%\psframe[linestyle=none,linearc=2pt,cornersize=absolute,fillstyle=solid,fillcolor=lightyellow](#1,#2)(#3,#4)
\psline[linearc=2pt,fillstyle=none,fillcolor=lightyellow](#1,#2)(#3,#2)(#3,#4)(#1,#4)
}

% et top left
\newcommand {\erettl}[4]
{
%\psframe[linestyle=none,linearc=2pt,cornersize=absolute,fillstyle=solid,fillcolor=lightyellow](#1,#2)(#3,#4)
\psline[linearc=2pt,fillstyle=none,fillcolor=lightyellow](#1,#2)(#3,#2)(#3,#4)
}

% et bottom right
\newcommand {\eretbr}[4]
{
%\psframe[linestyle=none,linearc=2pt,cornersize=absolute,fillstyle=solid,fillcolor=lightyellow](#1,#2)(#3,#4)
\psline[linearc=2pt,fillstyle=none,fillcolor=lightyellow](#1,#2)(#1,#4)(#3,#4)
}

% et middle right
\newcommand {\eretmr}[4]
{
%\psframe[linestyle=none,linearc=2pt,cornersize=absolute,fillstyle=solid,fillcolor=lightyellow](#1,#2)(#3,#4)
\psline[linearc=2pt,fillstyle=none,fillcolor=lightyellow](#3,#4)(#1,#4)(#1,#2)(#3,#2)
}

% et top right
\newcommand {\erettr}[4]
{
%\psframe[linestyle=none,linearc=2pt,cornersize=absolute,fillstyle=solid,fillcolor=lightyellow](#1,#2)(#3,#4)
\psline[linearc=2pt,fillstyle=none,fillcolor=lightyellow](#1,#4)(#1,#2)(#3,#2)
}

% \ergrp{#1 x0} {#2 y0} {#3 x1} {#4 y1} {#5 corner radius} {#6 fill}
% #5 corner radius is unused!
\newcommand {\ergrp}[6]
{ 
\ifthenelse{\equal{#6}{1}}
{\psframe[fillstyle=solid,fillcolor=lightgray](#1,#2)(#3,#4)}
{\psframe[fillstyle=solid,fillcolor=white](#1,#2)(#3,#4)}
}

% \eretname {#1 x left of text} {#2 y top of text} {#3 text}
\newcommand {\eretname}[3]
{
%shift down 0.1 for height of text the anchor at baseline (B)
\rput[bl]{0}(0,-0.1){\rput[Bl]{0}(#1,#2){\footnotesize \textit{#3}}}
}

% \errelarm {#1 x0} {#2 y0} {#3 x1} {#4 y1} {#5 ismandatory} {#6 isconstructed}
\newcommand {\errelarm}[6]
{
\ifthenelse{\equal{#6}{1}}
{
%%\psline[linewidth=0.5pt,linearc=.05,linestyle=dashed,dash=6pt 6pt]{-}(#1,#2)(#3,#4)}
\ifthenelse{\equal{#5}{1}}
{\psline[linewidth=1.5pt,linearc=.05,linecolor=lightgray]{-}(#1,#2)(#3,#4)}
{\psline[linewidth=1.5pt,linearc=.05,linecolor=lightgray,linestyle=dashed,dash=2pt 2pt]{-}(#1,#2)(#3,#4)}
}
{
\ifthenelse{\equal{#5}{1}}
{\psline[linewidth=0.9pt,linearc=.05]{-}(#1,#2)(#3,#4)}
{\psline[linewidth=0.9pt,linearc=.05,linestyle=dashed,dash=2pt 2pt]{-}(#1,#2)(#3,#4)}
}
}

% \errelangle {#1 x0} {#2 y0} {#3 x1} {#4 y1} {#5 x2} {#6 y2} {#7 ismandatory} {#8 isocnstructed}
\newcommand {\errelangle}[8]
{
\ifthenelse{\equal{#8}{1}}
{
%\psline[linewidth=0.5pt,linearc=.1,linestyle=dashed,dash=6pt 6pt]{-}(#1,#2)(#3,#4)(#5,#6)}
\ifthenelse{\equal{#7}{1}}
{\psline[linewidth=1.5pt,linearc=.05,linecolor=lightgray]{-}(#1,#2)(#3,#4)(#5,#6)}
{\psline[linewidth=1.5pt,linearc=.1,linecolor=lightgray,linestyle=dashed,dash=2pt 2pt]{-}(#1,#2)(#3,#4)(#5,#6)}
}
{
\ifthenelse{\equal{#7}{1}}
{\psline[linewidth=0.9pt,linearc=.1]{-}(#1,#2)(#3,#4)(#5,#6)}
{\psline[linewidth=0.9pt,linearc=.1,linestyle=dashed,dash=2pt 2pt]{-}(#1,#2)(#3,#4)(#5,#6)}
}
}

% \ercrowfoot {#1 x0} {#2 y0} {#3 x11} {#4 y11} {#5 x12} {#6 y12} {#7 x13} {#8 y13} {#9 isconstructed}
\newcommand {\ercrowfoot}[9]
{
\ifthenelse{\equal{#9}{1}}
{
\psline[linewidth=1.5pt,linearc=.05,linecolor=lightgray]{-}(#1,#2)(#3,#4)
\psline[linewidth=1.5pt,linearc=.05,linecolor=lightgray]{-}(#1,#2)(#5,#6)
\psline[linewidth=1.5pt,linearc=.05,linecolor=lightgray]{-}(#1,#2)(#7,#8)
}{
\psline[linewidth=0.9pt,linearc=.05]{-}(#1,#2)(#3,#4)
\psline[linewidth=0.9pt,linearc=.05]{-}(#1,#2)(#5,#6)
\psline[linewidth=0.9pt,linearc=.05]{-}(#1,#2)(#7,#8)
}
}


% \eridcomprel{#1 x1}{#2 x2}{#3 y1}{#4 ymid}{#5 y2}
\newcommand {\eridcomprel}[5]
{
\psline[linewidth=0.9pt](#1,#3)(#1,#5)
\psline[linewidth=0.9pt](#2,#3)(#2,#5)
\psline[linewidth=0.9pt](#1,#4)(#2,#4)
}

% \eridrefrel{#1 x1}{#2 xmid}{#3 x2}{#4 y1}{#5 y2}
\newcommand {\eridrefrel}[5]
{
\psline[linewidth=0.9pt](#1,#4)(#3,#4)
\psline[linewidth=0.9pt](#1,#5)(#3,#5)
\psline[linewidth=0.9pt](#2,#4)(#2,#5)
}


% \errelname {#1 x} {#2 y} {#3 text}
\newcommand {\errelname}[3]
{
\rput[l]{0}(#1,#2){\textit{#3}}
}
% \errelseq {#1 x} {#2 y}
\newcommand {\erelseq}[2]
{
}
% \erattr {#1 x} {#2 y} {#3 ismandatory}{#4 idenitfying} {#5 text}
\newcommand {\erattr}[5]
{
\ifthenelse{\equal{#3}{1}}
{\rput[l]{0}(#1,#2){{\tiny $\square$} {\footnotesize \textit{\ifthenelse{\equal{#4}{0}}{\underline{#5}}{#5}}}}}
{\rput[l]{0}(#1,#2){\footnotesize $\circ$ \textit{\ifthenelse{\equal{#4}{0}}{\underline{#5}}{#5}}}}
}

%\ifthenelse{\equal{#4}{1}}
% \ertext {#1 x} {#2 y} {#3 text anchor} {#4 text}
%{\rput[l]{0}(#1,#2){\footnotesize $\circ$ \underline{\textit{#5}}}}
\newcommand {\ertext}[4]
{
\rput[B#3]{0}(#1,#2){{\footnotesize #4}}
}
% \erarc {#1 x0} {#2 y0} {#3 x1} {#4 y1} {#5 x2} {#6 y2} {#7 x3} {#8 y3}
\newcommand {\erarc}[8]
{
\psbezier[showpoints=false]{-}(#1,#2) (#3, #4)(#5,#6) (#7, #8)
}

% \erarc {#1 x0} {#2 y0} {#3 x1} {#4 y1} {#5 x2} {#6 y2} {#7 x3} {#8 y3}
\newcommand {\errelseq}[8]
{
\psbezier[showpoints=false]{-}(#1,#2) (#3, #4)(#5,#6) (#7, #8)
}
% \ertrace {#1 trace}   
\newcommand {\ertrace}[1]
{
}

\usepackage{amsthm} % added 7th April 2018
% theorems.macros.tex

\newtheorem{theorem}{Theorem}[section]
\newtheorem{observation}[theorem]{Observation}
\newtheorem{lemma}[theorem]{Lemma}

\newtheorem{proposition}[theorem]{Proposition}
\newtheorem{corollary}[theorem]{Corollary}
\newtheorem{conjecture}[theorem]{Conjecture}
\newtheorem{numbereddefinition}[theorem]{Definition}

\newenvironment{definition}[1][Definition]{\begin{trivlist}
\item[\hskip \labelsep {\bfseries #1}]}{\end{trivlist}}
\newenvironment{examples}[1][Examples]{\begin{trivlist}
\item[\hskip \labelsep {\bfseries #1}]}{\end{trivlist}}
\newenvironment{example}[1][Example]{\begin{trivlist}
\item[\hskip \labelsep {\bfseries #1}]}{\end{trivlist}}
\newenvironment{remark}[1][Remark]{\begin{trivlist}
\item[\hskip \labelsep {\bfseries #1}]}{\end{trivlist}}

\newenvironment{tageqn}[1]
{
\begin{equation}
\stepcounter{equation}
\label{#1}
\tag{\theequation --#1}
}
{
\end{equation}
}

\newenvironment{axiom}[1]
{
\begin{equation}
\label{#1}
\tag{#1}
}
{
\end{equation}
}

% when the tag is required different from the label eg when has math symbols can use:
\newenvironment{axiomtagged}[2]
{
\begin{equation}
\label{#1}
\tag{#2}
}
{
\end{equation}
}

%visible label
\newcommand{\vlabel}[2][]{\label{#2}#1(\textit{#2}):}





\usepackage{imakeidx}
\makeindex[name=definitions, title=Index of Definitions]
\makeindex[name=lemmas, title=Index of Lemmas]



\newcommand{\commentary}[1]{\marginpar{\footnotesize #1}}
\newcommand{\highlight}[1]{\colorbox{orange}{#1}}
\newcommand{\term}[1]{\textit{#1}\commentary{\colorbox{lightgray}{\textit{#1}}}\index[definitions]{#1}}
\newcommand{\llabel}[1]{\label{#1}\commentary{\colorbox{pink}{\scriptsize{#1}}}\index[lemmas]{#1}}
\newcommand{\lref}[1]{\ref{#1}\colorbox{pink}{\scriptsize{#1}}\index[lemmas]{#1!use of}}

\newcommand{\newt}[1]{\colorbox{yellow}{#1}}
\newenvironment{newtt}
{  \colorbox{yellow}{$[$ ...} 
}
{  \colorbox{yellow}{... $]$}
}
\newcommand{\oldt}[1]{\colorbox{yellow}{\sout{#1}}}
\newenvironment{oldtt}
{  \colorbox{red}{$[$ ...} 
}
{  \colorbox{red}{... $]$}
}

\newcommand{\reinstatet}[1]{\colorbox{lime}{#1}}
\newenvironment{reinstatett}
{  \colorbox{lime}{$[$ ...}
}
{  \colorbox{lime}{... $]$}
}

\newcommand{\tbd}{\highlight{TBD}}

%ithprojection function
\newcommand{\proji}[1]{\pi_#1}



\newenvironment{categoricalaside}
{\begin{framed}
\textbf{Categorical Aside}
}
{
\end{framed}
}

\newenvironment{noteforfuture}
{\begin{framed}
\textbf{Note For Future}
}
{
\end{framed}
}

\newenvironment{problem}
{\begin{framed}
\textbf{Problem}
}
{
\end{framed}
}

%quine quote
\newcommand{\qq}[1]{
\left\ulcorner#1\right\urcorner
}

%single quote
\newcommand{\sq}[1]{
\textnormal{\textquotesingle}#1\textnormal{\textquotesingle}
}

%lower quine quote
\newcommand{\lqq}[1]{
\left\llcorner #1\right\lrcorner
}


%from berkley
\newcommand{\langl}{\begin{picture}(4.5,7)
\put(1.1,2.5){\rotatebox{60}{\line(1,0){5.5}}}
\put(1.1,2.5){\rotatebox{300}{\line(1,0){5.5}}}
\end{picture}}
\newcommand{\rangl}{\begin{picture}(4.5,7)
\put(.9,2.5){\rotatebox{120}{\line(1,0){5.5}}}
\put(.9,2.5){\rotatebox{240}{\line(1,0){5.5}}}
\end{picture}}
\newcommand{\lang}{\begin{picture}(5,7)\put(1.1,2.5){\rotatebox{45}{\line(1,0){6.0}}}\put(1.1,2.5){\rotatebox{315}{\line(1,0){6.0}}}\end{picture}}
\newcommand{\rang}{\begin{picture}(5,7)\put(.1,2.5){\rotatebox{135}{\line(1,0){6.0}}}\put(.1,2.5){\rotatebox{225}{\line(1,0){6.0}}}\end{picture}}
%Try sharper tuple brackets -- except gives errors nested in captions so comment out
%\renewcommand{\tuple}[1]{\lang #1 \rang}

\newcommand{\setsuchthat}[2]{\left\{#1 \ \middle|\ #2\right\}}
\newcommand{\set}[1]{\left\{#1\right\}} 

% one to n - wanton
\newcommand{\wanton}[1]{#1_1,...#1_n}
\newcommand{\fn}{\wanton{f}}
\newcommand{\pn}{\wanton{p}}
\newcommand{\qn}{\wanton{q}}
\newcommand{\qnprime}{\wanton{q'}}
\newcommand{\xn}{\wanton{x}}
\newcommand{\xnp}{\wanton{x'}}
\newcommand{\yn}{\wanton{y}}
\newcommand{\ntuple}[1]{\tuple{\wanton{#1}}}
\newcommand{\wantom}[1]{#1_1,...#1_m}
\newcommand{\mtuple}[1]{\tuple{#1_1,...#1_m}}
\newcommand{\qm}{\wantom{q}}
\newcommand{\ym}{\wantom{y}}
\newcommand {\bntuple}{\ensuremath{\ntuple{b}}}
\newcommand {\fntuple}{\ensuremath{\ntuple{f}}}
\newcommand {\fnptuple}{\ensuremath{\ntuple{f}}}
\newcommand {\pntuple}{\ensuremath{\ntuple{p}}}
\newcommand {\qntuple}{\ensuremath{\ntuple{q}}}
\newcommand {\qnptuple}{\ensuremath{\ntuple{q'}}}
\newcommand {\qmtuple}{\ensuremath{\mtuple{q}}}
\newcommand {\sntuple}{\ensuremath{\ntuple{s}}}
\newcommand {\xntuple}{\ensuremath{\ntuple{x}}}
\newcommand {\xnptuple}{\ensuremath{\ntuple{x'}}}
\newcommand {\ymtuple}{\ensuremath{\mtuple{y}}}
\newcommand{\foreachi}[1][n]{for each $i$, $1 \leq i \leq #1$}
\newcommand{\foreachj}[1][m]{for each $j$, $1 \leq j \leq #1$}
\newcommand{\foreachk}[1][l]{for each $k$, $1 \leq k \leq #1$}


\usepackage{mathptmx}  % This changes font to roman
\usepackage{anyfontsize}
\usepackage{mathtools}  % why have we got this?
\usepackage{alltt}    
\usepackage{mnsymbol} %used for rightpitchfork
\usepackage{cmll}
\usepackage{ulem}
\renewcommand{\ttdefault}{txtt}
\usepackage[left=1.5cm, right=4cm, marginparwidth=3cm, top=2cm, bottom=2.0cm]{geometry}
\usepackage{framed}
\usepackage[font=small]{caption}
\setlength{\captionmargin}{2cm}
\theoremstyle{remark}
\newtheorem*{lemma*}{Lemma}


\renewcommand{\term}[1]{\textit{#1}}  %SIMPLE UNINDEXED VERSION

%ENDCOPY

\begin{document}
\title{Preparation for a Mathematical Theory of Data}

% abstract here for ams

\author{John Cartmell}

\date{}

\maketitle

\begin{center}
DRAFTED October 2020 \\
REVISED April 2022
\end{center}

\newcommand{\seenudgeup}[1]{\rule{0.1cm}{#1}}
\newcommand{\seenudgedown}[1]{\rule[-#1]{0.1cm}{0.1cm}}
\newcommand{\nudgeup}[1]{\rule{0cm}{#1}}
\newcommand{\nudgedown}[1]{\rule[-#1]{0cm}{0.1cm}}

\newcommand{\paralleldiag}[4]
{
 $
\rule[-0.3cm]{0pt}{0.9cm} %to add vertical space of diagram -- based on lowering diagram 0.3cm and heght 0.9cm
                            % change thickness from 0pt to 1 pt to debug
\begin{array}{c p{0.5cm} c  }
 \Rnode{a}{#1}     &&   \Rnode{b}{#2}
\end{array} 
\begin{arrows}
\ncarc[nodesep=2pt,arcangle=10,offset=2pt]{->}{a}{b}
\alabel{#3}
\ncarc[nodesep=2pt,arcangle=-10,offset=-2pt]{->}{a}{b}
\blabel{#4}
\end{arrows}
$  
}


\newcommand{\binarysourcediag}[5]{
$
\begin{array}{c p{0.5cm} c  }
             &&   \Rnode{b}{#2} \\[0.01cm]
\Rnode{a}{#1} &&                \\[0.01cm] 
             &&   \Rnode{c}{#3} 
\end{array} 
\begin{arrows}
\ncarr{a}{b}
\alabel{#4}
\ncarr{a}{c}
\blabel{#5}
\end{arrows}
$  
}
\newcommand{\fgsourcediag}{\binarysourcediag{a}{b}{c}{f}{g}}

\newcommand{\fnsourceqnsource}
{
$
\begin{array}{c p{0.25cm} c  p{0.25cm} c }
             &&   \Rnode{b1}{b_1} &&              \\[0.4cm]
\Rnode{a}{a} &&                   && \Rnode{c}{c} \\[0.4cm]
             &&   \Rnode{bn}{b_n} &&              
\end{array} 
\begin{arrows}
\ncarr{a}{b1}
\alabel{f_1}
\ncarr{c}{b1}
\blabel{q_1} 
\ncarr{a}{bn}
\blabel{f_n}
\ncarr{c}{bn}
\alabel{q_n}
\end{arrows}
$   
}




\section{Background}
\note
Methods for describing the structure of data are fundamental to most programming languages, 
where they vary depending on the model of computation (object-oriented, functional, symbolic and so on);
they occur as dedicated specification methods, such as the entity relationship method,
and  in database technologies, where they vary by the data model (relational, hierarchical, nested relational,
graph based, etc.). Finally they underlie interfacing technologies and either describe binary formats such as 
the many that implement some variant of IDL (Interface Definition Language) or text formats such as epitomised by XML. 

\note 
There are similarities between the methods when they are viewed abstractly but there are significant differences too and so a study of these different methods brings us to many essentially distinct notions of data specification.

\note
The thinking is that when viewed abstractly each data specification is a 
theory\footnote{The role of such a theory can be foregrounded by speaking of it
 as a \textit {theory of what is} or as an \textit{ontology}.} and  to each different notion of data specification corresponds a different notion of theory. An exposition of the different notions of theory that can properly be said to be methods of data specification
along with a study of their meta-mathematical properties 
 will constitute a mathematical theory of data which, as described, is therefore in fact a meta-theory. Such a theory has a role to play in improving  the way we think about, discuss, design, develop and transform data specifications. I strongly believe that such a fully elaborated mathematical theory of data will foster significant improvements in  techniques and tools for the management of data. 
\note 
The relational model of data underpinning the majority of databases for fifty years or so, is exceptional in that it has a body of theory; this theory includes quality criteria  distinguishing good data specifications from bad. 
One of the goals of a mathematical theory of data is to enable these relational prescriptions of goodness
to be generalised to become generally applicable. The \textit{dry run} below suggests this is possible.

\note
Data is required for a purpose, generally to describe real world things in some or other context. This constitutes an intended usage for a data specification. Of all structurally compliant instances of a data specification some are required for the intended usage and, generally speaking, some are not.
Notionally let there be a requirement $R$ that equates to a subset of the set of all compliant data instances 
of a data specification and serving to characterise its intended use. 

\note 
There are two self-complementing principles of good data engineering. 
Firstly, redundancy of data is to be avoided. This first principle is modulated by computational cost for it would be unreasonable not to hold in data all prime factors of a number on account of them being computable and therefore redundant.
Secondly, within each particular methodology a data specification should be as constraining as possible of data instances whilst being general enough for the intended usage; equivalently the corresponding theory should fit as tightly as possible to the facts. 

\note 
Meeting the second principle we will describe as achieving \term{maximum constrainedness} for the data specification.
To maximise constrainedness will be to come as  close as we can within any given methodology with given syntax to meeting a formal objective described by Zaniola \cite{zaniolo1982} in the context of relational schema design (data specification, that is, for the relational model of data)  as `the complete \textit{representation} of semantic constraints' (his italics). Zaniola subsequently refers to this as `the representation principle'.

\note
From the two principles we can phrase goodness criteria  for data specifications with respect to requirements $\reqt$
i.e. to intended usages. In the context of relational data design, 3rd, 4th and 5th normal forms are examples of such goodness criteria. 

\note
When viewed abstractly many distinct notions of data specification can be characterised as having
data specifications corresponding to finite presentations of either categories or, if missing data is to be allowed, partial order enriched categories with some additional structure such as certain limits and/or colimits. We use the term \textit{sketch}
in this note to be synonymous with \textit{presentation of category} and as such take it to consist of the combination of a directed graph and a set of path equivalences. In Barr and Wells \cite{BarrandWells} these are more properly called linear sketches.  

\note
Compliant instances of such data specifications correspond to structure preserving functors from the corresponding category to the category of finite sets $\Fin$ or to the category of finite sets and partial functions.

\note
Redundancy of objects or arrows in a presentation corresponds to redundancy of data in instances of a data specification. 
By the first principle it is the goal of data specification to avoid such redundancy. 

\note
Goodness equates to absence of redundancy plus maximal constrainedness to intended usage. Absence of redundancy is a property of a presentation. Maximal constrainedness
is a property of the category $\catc$ generated by the presentation and is relative to a requirement $\reqtc$, where $\reqtc$ is a set of instances where each instance is structure preserving functor $D$, $D: \catc \morph \Fin$, where
$\Fin$ is the category of finite sets and functions (or, subsequently, to other variants of the category of sets and functions as appropriate).

\note 
Codd \cite{Codd1970} proposes the relational model of data; he gives the first prescription of goodness for
a relational data specification and describes how it might be achieved through a method which he calls normalisation\cite{Codd1970}\footnote{He also introduces the term foreign key in this first paper and includes a discussion of redundancy of data.}. 
Codd  subsequently defines a third normal form (3NF) \cite{Codd1971} for which purpose he introduces 
the concept of a functional dependency.
The definition of third normal form extends the notion of goodness and the method for achieving it\footnote{By \cite{Codd1971} the stage was set 
for describing conditions of goodness in terms of relational schemas being in normal form -- an  unfortunate terminology  because these schemas that meet the condition
are not canonical in any way as a mathematician might be led to believe from the terminology.}.

\note Boyce-Codd normal form (BCNF) is a stronger normal form and one that it is not always possible to meet. Zaniola \cite{zaniolo1982}) most clearly elaborates the difference between 3NF and BCNF. 
In Zaniola's description, specifications that are in BCNF meet the representation principle in regard to having all functional dependencies represented in them.

\note Further standards that a good relational data specification should adhere to were formulated by Fagin \cite{Fagin1977} (fourth normal form) and  \cite{Fagin1979} (projection-join normal form also known as fifth normal form)
using the concept of multi-valued dependencies. 
One paraphrasing would be that it isn't good to store needless copies of data. 
When formulated in category theory this will come down to not needlessly including limit objects in a presentation.

\note In a different direction many authors describe forms of redundancy in data that are immune to prescriptions
of previous normal forms (up to 5th normal form, say) and to remedy this 
they give definitions of normal forms that take account of inclusion dependencies.
There isn't a single clear concept that arises from this work but the deficiency and the need for a remedy is very clear.  
Inclusion dependencies, like functional dependencies and multi-valued dependencies, are forms of semantic constraint in the sense that this term is used by Zaniola. 

\note Here we focus on inclusion dependencies that are referential. These in dry run are the equivalents of what elsewhere in the context of relational data specification are referred to as a key-based or a superkey-based inclusion dependencies [\cite{Mannila1986}, \cite{Levene2000}]
or, more pragmatically, as referential constraints\footnote{	Also known colloquially, and rather horribly in my opinion, as foreign key constraints}in the ISO SQL standard\cite{ISOSQL2016} and in relation to XML
(\cite{fan2003}, for instance); whether implicitly, or explicitly as in the relational paradigm, these are lynchpins of  data specifications.

\note Various authors (\cite{CartmellScopePaper},\cite{Johnson93}) have noted the importance of commutative diagrams in data specifications.
  The fact is that relational designs fairly frequently have commutivity constraints implicitly represented within  and this having been achieved  by designers following prescriptions to normalise data and to eliminate duplicates rather than with awareness of the underlying commutivity. 
Shlaer and Lang illustrate this in \cite{Shlaer96} where they describe alternative paths between two nodes as
relationship loops, when distinct paths are equivalent they say that there are dependencies
between the relationships. Kolp and Zimnyi ((\cite{Kolp1995})) instead use the term
relationship cycle and identify such as a source of superfluous attributes in the
transformation from ER model to relational model. In this paper we speak of commutative digrams as path equivalences.

\section{Investigation -- Data Specification as Sketch of Category}
\note
In this investigation -- data specification as presentation of category i.e. as linear sketch \cite{BarrandWells}-- we will formulate 
definitions of maximal constrainedness, path equivalence, functional dependency and referential inclusion dependency
and we will define what it means for such a path equivalence, functional dependency or  referential inclusion dependency to be represented in such data specifications.

\note
This investigation with its overtly simplified notion of data specification is worthwhile in that   it establishes some starter definitions which in further work we can subsequently develop to be applicable to more fully elaborated notions.

\note 
This investigation establishes a pattern that we will follow later in consideration of more fully elaborated 
definitions of 
data specifications that we follow up with -- partial order enriched categories, categories with products and others. 
You may also consider that the definitions given in this investigation are embryonic precursors to equivalent relational definitions.


\subsection{Definitions}
\subsection{Directed Graphs}
Regarding directed graphs and reflecting a category theory mindset we will use terminology as follows:
\begin{itemize}
\item
 If $f: a \morph b$ in an edge of a directed graph $G$ then we will say that $a$ is the \term{domain} of $f$ and $b$ is the \term{codomain} of $f$.
\item
If $a$ and $b$ are nodes of a directed graph $G$ then a \term{path} through $G$ with domain $a$ and 
codomain $b$ of length $n$, where $n \geq 0$, we define to be  an n-tuple of  $n$ edges: $p_i: x_i \morph x_{i+1} $ in $G$ where $x_0=a$ and $x_n=b$. We shall write this n-tuple as $p_1 \circ p_2... \circ p_n$. 
We will use the same notation if any of the $p_i$ are edges rather than paths as we will not need distinguish  an edge from a singleton path along that edge. 
\item two paths have the same domain and the same codomain then we shall say that they are \term{commensurate}.
\end{itemize}




\subsection{Sketches}
By a  \term{sketch for a category} we shall mean a directed graph and a set specified path equivalences.
Each path equivalence consists of a pair $f_1 \circ ... \circ f_n$ and $g_1 \circ ... \circ g_m$ of commensurate paths and can be represented as a diagram so
\begin{displaymath}      
\begin{array}{cp{0.5cm}cp{0.5cm}cp{0.25cm}cp{0.25cm}cp{0.5cm}cp{0.5cm}c}
            &&               &&                &&                  &&                &&               && \\[0.1cm] % vertical space
            &&\Rnode{TL}{c_1}&&\Rnode{TIL}{c_2}&&\Rnode{TC}{\hdots}&& \Rnode{TIR}{c_{n-2}} && \Rnode{TR}{c_{n-1}} &&  \\[0.2cm]
\Rnode{a}{a}&&               &&                &&                  &&                &&               && \Rnode{b}{b} \\[0.2cm]
            &&\Rnode{BL}{d_1}&&\Rnode{BIL}{d_2}&&\Rnode{BC}{\hdots}&& \Rnode{BIR}{d_{m-2}} && \Rnode{BR}{d_{m-1}} &&  \\[0.2cm]        
\end{array}
\begin{arrows}
\ncarr{a}{TL}
\alabel{f_1}
\ncarr{TL}{TIL}
\alabel{f_2}
\ncarr{TIL}{TC}
\ncarr{TC}{TIR}
\ncarr{TIR}{TR}
\alabel{f_{n-1}}
\ncarr{TR}{b}
\alabel{f_n}
\ncarr{a}{BL}
\blabel{g_1}
\ncarr{BL}{BIL}
\blabel{g_2}
\ncarr{BIL}{BC}
\ncarr{BC}{BIR}
\ncarr{BIR}{BR}
\blabel{g_{m-1}}
\ncarr{BR}{b}
\blabel{g_n}
\end{arrows}
\end{displaymath}

For what we are here calling a sketch of a category Barr and Wells use the term \term{linear sketch} and define it as a (directed) graph plus a set of diagrams.

We define the equivalence relation $\sim_S$ of path equivalence determined by a sketch $S$ to be the closure of the set of the specified path equivalences under the following

\begin{itemize}
\item for any path $p$, $p \sim_S p$,
\item for any paths $p_1$ and $p_2$ if $p_1 \sim_S p_2$ then $p_2 \sim_S p_1$,
\item for any paths $p_1$,$p_2$ and $p_3$ if $p_1 \sim_S p_2$ and $p_2 \sim_S p_3$ then $p_1 \sim_S p_3$,
\item if \paralleldiag{a}{b}{p_1}{p_2} and $q: b \morph c$ are paths in $G$ and if $p_1 \sim_S p_2$  
then $p_1 \circ q \sim_S p_2 \circ q$,
\item if $p: a \morph b$ and \paralleldiag{b}{c}{q_1}{q_2} and $q: b \morph c$ are paths in $G$ then if $q_1 \sim_S q_2$  
then $q_1 \circ p \sim_S q_2 \circ p$.
\end{itemize}

The category generated by sketch (called the theory of the sketch in Barr and Wells) is the category with
the nodes of the directed graph of $S$ as objects and with equivalence classes of paths as morphisms.
Composition is defined from composition of representative paths and, from the definition of $\sim_S$, is well-defined.

A sketch is said to be \term{redundancy free} 
if there is no smaller sketch which generates the same category (upto isomorphism of categories).

\subsection{Instances of a Data Specification}
Whilst a sketch represents a data specification, an instance of the specification\footnote{In the case that the data specification describing a database then an instance will be a database instance i.e. a snapshot of the data content at a moment in time.} can be considered to be a set of entities for each node within the graph
along with a many-one functional relationship between the entitity sets for each edge of the graph. 
Accordingly we define an instance of a directed graph $G$ to be a mapping of $G$ to the category of finite sets and functions.
If $G$ is a directed graph and $D$ is an instance then for every node $a$ of $G$, $D(a)$ is a finite set
and for every edge $f:a \morph b$ in $G$, $F(f):D(a) \morph D(b)$ in $\Fin$. Given such an instance $D$ of 
directed graph $G$ 
we can interpret every path $p:a \morph b$ through $G$ as a function $D(p): D(a) \morph D(b)$.

we define an instance of a sketch of a category to be an instance $D$ of its directed graph such that
for each path equivalence $p_1 \sim p_2$ specified in the sketch, $D(p_1)=D(p_2)$.

An instance $D$ of a sketch $S$ that generates a category \catcw 
uniquely determines a functor $D:\catc \morph \Fin$ vice-versa. We don't distinguish much in what follows between instances of sketch $S$ and functors $D: \catc \morph Fin$. Both, equally, represent instances of the skecth considered as a data specification.

\subsection{Requirements for Data Specifications}

In what follows we need some representation of the requirement for a data specification. Ultimately it is in respect of this requirement that a data specification is good or bad. It is sufficient to 
represent this requirement as a set of required data instances. 
In what follows a \term{requirement} for a data specification $S$ 
is a set of instances of the sketch $S$ or, equivalently, is a set $R_C$ of functors where for each
$D \in R_C$, $D: \catc \morph \Fin$, where \catcw is the category generated by the sketch $S$.

Two commensurate paths $p_1$ and $p_2$ are  said to be \term{equivalent with respect to a requirement $R$}, 
written $R \models p_1 \sim p_2$, iff 
for all instances $D \in R$, $D(p1)=D(p_2)$.

The following is a trivial consequence of these definitions.
\begin{lemma}
\label{pathequivalenceinference}
If $G$ is a directed graph and if $R$ is a requirement then 
\begin{itemize}
\item if \paralleldiag{a}{b}{p_1}{p_2} and $q: b \morph c$ are paths in $G$ then if $R \models p_1 \sim p_2$  
then $R \models p_1 \circ q \sim p_2 \circ q$,
\item if $p: a \morph b$ and \paralleldiag{b}{c}{q_1}{q_2} and $q: b \morph c$ are paths in $G$ then if $R \models q_1 \sim q_2$  
then $R \models p \circ q_1 \sim p \circ q_2$.
\end{itemize}
\end{lemma}


\subsection{Examples - Sketch of Category as Data Specification}



\documentclass[10pt,a4paper]{article}



%ccategories.macros.tex 

% Macros for diagrams in contextual categories and related categories

\usepackage{twoopt}
\usepackage{scalerel} 
\usepackage{xargs}

%\usepackage{mathabx}  %Caused font problems
%\usepackage{MnSymbol}  % caused font problems

\newcommand{\conu}
{\mathbf{C}(U)}

\newcommand{\depu}
{\mathbf{D}(U)}

\newcommand{\cat}[1]{\textbf{#1}}
\newcommand{\obj}[1]{\ensuremath{|\cat{#1}|}}
\newcommand{\ccat}[1][C]{\ensuremath{\mathbb{#1}} }
\newcommand{\ccatc}{contextual category \ccat}
\newcommand{\cobj}[2][]{\ensuremath{|\ccat[#2]|_{#1}}}
\newcommand{\cslice}[2]{\ensuremath{\ccat[#1]_{#2}}}
\newcommand{\csliceobj}[3][]{\ensuremath{|\mathbb{#2}_{#3}|_{#1} }}
\newcommand{\varset}[1][]{\ensuremath{V_{#1} }}
\newcommand{\localvarsets}{\ensuremath{\mathcal{V} }}
\newcommand{\Fam}{\ensuremath{\mathbb{F\mathrm{am}} }}
\newcommand{\Famslice}[1]{\ensuremath{\mathbb{F\mathrm{am}}_{#1} }}
\newcommand{\Famobj}[1][]{\ensuremath{|\mathbb{F\mathrm{am}}|_{#1} }}
\newcommand{\Famsliceobj}[2][]{\ensuremath{|\mathbb{F\mathrm{am}}_{#2}|_{#1} }}
\newcommand{\morph}{\rightarrow}
\newcommand{\epi}{\twoheadrightarrow}
\newcommand{\base}{\triangleleft}
\newcommand{\comp}{\circ}
\newcommand{\cross}{\otimes}
\newcommand{\pc}[2]{d^{#1}_{#2}}
\newcommand{\sub}{^*}
\newcommand{\diag}{\delta}
\newcommand{\pbase}[1]{\tilde{#1}}

\newcommand{\tuple}[1]{\langle#1\rangle}
\newcommand{\ndidly}{\ensuremath{\Join_n}}
\newcommand{\ndidlycospan}{quotiented n-cospan}

\newcommand{\crossx}[3]{#1 \underset{#3}{\cross} #2}
\newcommand{\fibrex}[3]{#1 \underset{#3}{\Join} #2}
\newcommand{\powerset}{\mathcal{P}}
\newcommand{\primeds}[1]{
\ensuremath{\mathcal{P}(#1)} }
\newcommand{\compset}{\ \dot{\circ}\, }

% darrow
%\newcommand{\darrow}{\rightarrowtriangle} %use \smorph instead
\newcommand{\smorph}{\rightarrowtriangle}

 

\newcommand\dhead{\scaleobj{0.6}{\triangleright}}
\newcommand{\dmorph}{\, \mbox{---} \! \cdot \! \raisebox{1.1pt}{\dhead}}

% projection tree
%\newcommand{\proj}[2]{proj_{#2}(#1)}

\newcommand{\proj}[2]{
\ensuremath{\mathcal{P}_{#2}(#1)} }

%pstrick supplements for arrows

\newlength{\arrnodesepA}
\newlength{\arrnodesepB}
\newlength{\arroffsetA}
\newlength{\arroffsetB}

%Modified to 2pt from 0pt on 23 July 2018
\newcommand{\arreset}{
\setlength{\arrnodesepA}{2pt}
\setlength{\arrnodesepB}{2pt}
\setlength{\arroffsetA}{0pt}
\setlength{\arroffsetB}{0pt}
}
\arreset

\newcommand{\ncarr}[3][0]{\ncarc[arcangle=#1,nodesepA=\arrnodesepA,nodesepB=\arrnodesepB,offsetA=\arroffsetA,offsetB=\arroffsetB,arrowsize=5pt,arrowinset=0.7]{->}{#2}{#3}}
\newcommand{\jcbarr}[4][0]{ % ncbarr is defined in some thridy party package so do not use!\emph{}
\ncarr[#1]{#3}{#4}
\nbput[labelsep=2pt]{\footnotesize $#2$}
}

\newcommand{\ncaarr}[4][0]{
\ncarr[#1]{#3}{#4}
\naput[labelsep=2pt]{\footnotesize $#2$}
}

% \alabel{label}[npos][labelsep_pts]
\newcommandx*\alabel[3][2=0.5,3=2,usedefault]{\naput[labelsep=#3pt,npos=#2]{\footnotesize $#1$}}
% \blabel{label}[npos][labelsep_pts]
\newcommandx*\blabel[3][2=0.5,3=2,usedefault]{\nbput[labelsep=#3pt,npos=#2]{\footnotesize $#1$}}

% \idcomp mark an arrow as one component of an identifier
\newcommand{\idcomp}{\ncput[npos=0, nrot=:U]{\psline(0.2,-0.075)(0.2,0.075)}}  %add a bar to a node connection arrow
% pstrick supplements for s-arrows (previous name for d-arrow - should convert}

\newlength{\sarnodesepA}
\newlength{\sarnodesepB}
\newlength{\saroffsetA}
\newlength{\saroffsetB}
\newlength{\sarnodesepAsav}
\newlength{\sarnodesepBsav}

\newcommand{\sarreset}{
\setlength{\sarnodesepA}{0pt}
\setlength{\sarnodesepB}{0pt}
\setlength{\saroffsetA}{0pt}
\setlength{\saroffsetB}{0pt}
}

\sarreset

% sar - S-arrow
\newcommand{\ncsar}[3][0]{
\setlength{\sarnodesepAsav}{\sarnodesepA}
\setlength{\sarnodesepBsav}{\sarnodesepB}
\addtolength{\sarnodesepA}{3pt}
\addtolength{\sarnodesepB}{7pt}
\ncarc[nodesepA=\sarnodesepA,nodesepB=\sarnodesepB,offsetA=\saroffsetA,offsetB=\saroffsetB,arcangle=#1]{-}{#2}{#3}
\ncput[nrot=:R,npos=1]{\pstriangle(0,0)(.2,.2)}
\setlength{\sarnodesepA}{\sarnodesepAsav}
\setlength{\sarnodesepB}{\sarnodesepBsav}
}


% bsar - below labelled S-arrow
\newcommand{\ncbsar}[4][0]{
\ncsar[#1]{#3}{#4}
\nbput[labelsep=2pt]{\footnotesize $#2$}
}
% asar - above labelled S-arrow
\newcommand{\ncasar}[4][0]{
\ncsar[#1]{#3}{#4}
\naput[labelsep=2pt]{\footnotesize $#2$}
}

% cdar - composite dependency arrow
\newcommand{\nccdar}[3][0]{
\setlength{\sarnodesepAsav}{\sarnodesepA}
\setlength{\sarnodesepBsav}{\sarnodesepB}
\addtolength{\sarnodesepA}{3pt}
\addtolength{\sarnodesepB}{11pt}
\ncarc[nodesepA=\sarnodesepA,nodesepB=\sarnodesepB,offsetA=\saroffsetA,offsetB=\saroffsetB,arcangle=#1]{-}{#2}{#3}
\ncput[nrot=:R,npos=1]{\pstriangle(0,0.1)(.2,.2)}
\ncput[nrot=:R,npos=1]{\psdot[dotsize=1pt](-0.0075,0.05)}   %!!
\setlength{\sarnodesepA}{\sarnodesepAsav}
\setlength{\sarnodesepB}{\sarnodesepBsav}
}


% bcdar - below labelled composite dependency arrow
\newcommand{\ncbcdar}[4][0]{
\nccdar[#1]{#3}{#4}
\nbput[labelsep=2pt]{\footnotesize $#2$}
}
% acdar - above labelled composite dependency arrow
\newcommand{\ncacdar}[4][0]{
\nccdar[#1]{#3}{#4}
\naput[labelsep=2pt]{\footnotesize $#2$}
}


% rsar - recursive S-arrow
\newcommand{\ncrsar}[2]{
\setlength{\sarnodesepAsav}{\sarnodesepA}
\setlength{\sarnodesepBsav}{\sarnodesepB}
\addtolength{\sarnodesepA}{3pt}
\addtolength{\sarnodesepB}{7pt}
\ncloop[nodesepA=\sarnodesepA,nodesepB=\sarnodesepB,
        offsetA=\saroffsetA,offsetB=\saroffsetB,
        armA=0.7cm,armB=0.6cm,angleA=90,angleB=-90,loopsize=-1,linearc=0.4
				]{-}{#1}{#2}
\ncput[nrot=:R,npos=5]{\pstriangle(0,0)(.2,.2)}
\setlength{\sarnodesepA}{\sarnodesepAsav}
\setlength{\sarnodesepB}{\sarnodesepBsav}
}

% pstrick supplements for multi-arrows

\newlength{\marnodesepA}
\newlength{\marnodesepB}
\newlength{\maroffsetB}
\newlength{\marnodesepBsav}

\newcommand{\marreset}{
\setlength{\marnodesepA}{0pt}
\setlength{\marnodesepB}{0pt}
\setlength{\maroffsetB}{0pt}
}

\marreset

%ncmarr[#1 arcangle1][#2 arcangle2]{#3 name}{#4 domain1}{#5 domain2}{#6 junction}{#7 codomain}
\newcommandtwoopt{\ncmarr}[6][8][8]{%
\ncarc[nodesepA=\marnodesepA,nodesepB=0,arcangle=#1]{-}{#3}{#5}
\ncarc[nodesepB=0,arcangle=-#1]{-}{#4}{#5}
\ncarc[arcangle=#2,nodesepB=\marnodesepB,offsetB=\maroffsetB]{->}{#5}{#6}
}%


\newcommandtwoopt{\nchmarr}[6][8][8]{%
\ncarc[nodesepA=\marnodesepA,nodesepB=0,arcangle=#1]{-}{#3}{#5}
\ncarc[nodesepB=0,arcangle=#1]{-}{#4}{#5}
\ncarc[arcangle=#2,nodesepB=\marnodesepB,offsetB=\maroffsetB]{->}{#5}{#6}
}%

\newcommandtwoopt{\ncamarr}[7][8][8]{%
\ncmarr[#1][#2]{#4}{#5}{#6}{#7}
\naput[npos=.05]{$#3$}
}%
\newcommandtwoopt{\ncbmarr}[7][8][8]{%
\ncmarr[#1][#2]{#4}{#5}{#6}{#7}
\nbput[npos=.05]{$#3$}
}%

\newcommandtwoopt{\ncbhmarr}[7][8][8]{%
\nchmarr[#1][#2]{#4}{#5}{#6}{#7}
\nbput[npos=.05]{$#3$}
}%

\newcommandtwoopt{\ncmarrr}[7][8][8]{
\ncarc[nodesepB=0,arcangle=#1]{-}{#3}{#6}
\ncline[nodesepB=0]{-}{#4}{#6}
\ncarc[nodesepB=0,arcangle=-#1]{-}{#5}{#6}
\ncarc[nodesepA=0,arcangle=#2]{->}{#6}{#7}
}

\newcommandtwoopt{\ncamarrr}[8][8][8]{
\ncmarrr[#1][#2]{#4}{#5}{#6}{#7}{#8}
\naput[npos=.05]{$#3$}
}
\newcommandtwoopt{\ncbmarrr}[8][8][8]{
\ncmarrr[#1][#2]{#4}{#5}{#6}{#7}{#8}
\nbput[npos=.05]{$#3$}
}

\usepackage[margin=4.0cm]{geometry} %was 3cm
\usepackage{mathptmx}
\usepackage{amsfonts}
\usepackage{array}
\usepackage{pstricks}
\usepackage{pst-tree}
\usepackage{pst-plot}
\usepackage{pst-node}
\usepackage{stmaryrd}
\usepackage{amsmath}
\usepackage{verbatim}
\usepackage{graphicx}  
\usepackage{calc}
\usepackage{xifthen}
\usepackage{xcolor}
\usepackage{color}
\usepackage{stringstrings}
%\usepackage[small,bf,margin=3pt,format=hang, labelsep=endash,singlelinecheck=false]{caption} %prevuiously justification=justified
%\usepackage{enumerate}
%\usepackage{enumitem}
\usepackage{enumerate}
\usepackage[shortlabels]{enumitem}
\usepackage{float}
\usepackage[section]{placeins}
%\setlength{\captionmargin}{5pt}
\usepackage{environ}
\usepackage{multirow}
\usepackage{rotating}
\usepackage{longtable}
\usepackage{afterpage}
\usepackage{needspace}


%DEFINE ENVIRONMENT BLOCK
% Riddle
\newsavebox{\riddlebox}

\newenvironment{erexample}
{\newcommand\colboxcolor{F0F0F0}%was F8F8F8
\begin{lrbox}{\riddlebox}
\begin{minipage}{\dimexpr\columnwidth-2\fboxsep\relax} \textbf{} \\ \itshape}
{\end{minipage}\end{lrbox}%
%\begin{center}
\colorbox[HTML]{\colboxcolor}{\usebox{\riddlebox}}
%\end{center}
}

\newenvironment{erbox}
{\newcommand\colboxcolor{F0F0F0}%was F8F8F8
\begin{lrbox}{\riddlebox}%
\begin{minipage}{\dimexpr\columnwidth-2\fboxsep\relax} }
{\end{minipage}\end{lrbox}%
%\begin{center}
\colorbox[HTML]{\colboxcolor}{\usebox{\riddlebox}}
%\end{center}
}

%\begin{erboxedFigure}{#1 FigureParam}{#2 Label}{#3 Caption}
\NewEnviron{erboxedFigure}[3]{%
\begin{figure}[#1]
\begin{erexample}
\begin{center}
\BODY
\end{center}
\vspace{-0.5cm}
\caption{#3}
\label{#2}
\end{erexample}
\end{figure}
}

\newcommand{\erpictureFolder}[0]{../SharedPictures}

\newcommand{\ercenterPicture}[1]{
\begin{center}
\input{\erpictureFolder/#1}
\end{center}
}


\newlength{\erhalfHt}

%\erinlinePicture{#1 pictureFilename}{#2 pictureHeight}
\newcommand{\erinlinePicture}[2]{
\setlength{\erhalfHt}{#2cm * \real{0.5}}
\raisebox{-\erhalfHt}[\erhalfHt + 0.5cm][\erhalfHt + 0.5cm]{
\input{\erpictureFolder/#1}
} 
}

%\erplainFig{#1 pictureFilename}{#2 figureParam}{#3Caption}
\newcommand{\erplainFig}[3]{
\begin{figure}[#2]
\begin{center}
\input{\erpictureFolder/#1}
\end{center}
\caption{#3}
\label{#1}
\end{figure}
}

%\erboxedFigPicture{#1 pictureFilename}{#2 figureParam}{#3Caption}
\newcommand{\erboxedFigPicture}[3]{
\begin{figure}[#2]
\begin{erexample}
\vspace{-0.5cm}
\begin{center}
\input{\erpictureFolder/#1}
\end{center}
\caption{#3}
\label{#1}
\end{erexample}
\end{figure}
}

%\erLeftSideFig{#1 pictureFilename}{#2 figureParam}{#3Caption}
\newcommand{\erLeftSideFig}[3]{
\begin{figure}[#2]
\begin{erexample}
  \begin{minipage}[c]{0.4\textwidth}
    \caption{#3}
    \label{#1}
  \end{minipage}
  \begin{minipage}[c]{0.5\textwidth}
    \input{\erpictureFolder/#1}
  \end{minipage}
\end{erexample}
\end{figure}
}

%\erbulletedFig{#1 pictureFilename}{#2 figureParam}{#3Caption}
\NewEnviron{erbulletedFig}[3]{%
\begin{figure}[#2]
\begin{erexample}
\vspace{-0.5cm}
\begin{center}
$
\begin{array}{c m{0.25cm} | m{6cm}}
\raisebox{-2.0cm}{
\input{\erpictureFolder/#1}}& & \text{\parbox{6cm}{\raggedright{\footnotesize{
\begin{enumerate}[(i)]
\BODY
\end{enumerate}}}}} \\
\end{array}
$
\end{center}
\caption{#3}
\label{#1}
\end{erexample}
\end{figure} 
}


%\begin{erbulletedDimFig}{#1 pictureFilename}{#2figureParam} {#3Caption} {#4PictureHeight}{#5TextWidth}

\NewEnviron{erbulletedDimFig}[5]{%
\begin{figure}[#2]
\begin{erexample}
\vspace{-0.5cm}
\begin{center}
$
\begin{array}{c m{0.25cm} |  m{#5cm}}
\setlength{\erhalfHt}{#4cm * \real{0.5}}
\raisebox{-\erhalfHt}{
\input{\erpictureFolder/#1}}& & \text{\parbox{#5cm}{\raggedright{\footnotesize{
\begin{enumerate}[(i)]
\BODY
\end{enumerate}}}}} \\
\end{array}
$
\end{center}
\caption{#3}
\label{#1}
\end{erexample}
\end{figure} 
}

%\begin{ernotedModel}{#1 pictureFilename}{#2PictureHeight}{#3PictureWidth}{#4TextWidth}

\NewEnviron{ernotedModel}[4]{%
\begin{center}
$
\begin{array}{m{#3cm} m{1cm} | c m{#4cm}}
\setlength{\erhalfHt}{#2cm * \real{0.5}}
\raisebox{-\erhalfHt}{
\input{\erpictureFolder/#1}}& & & \text{\parbox{#4cm}{\raggedright{\footnotesize{
\BODY
}}}} \\
\end{array}
$
\end{center} 
}

%\begin{ermodelText}{#1 pictureFilename}{#2PictureHeight}{#3PictureWidth}{#4TextWidth}

\NewEnviron{ermodelText}[4]{%
\begin{center}
\begin{tabular}{m{#3cm} m{1cm}  c m{#4cm}}
\setlength{\erhalfHt}{#2cm * \real{0.5}}
\raisebox{-\erhalfHt}{
\input{\erpictureFolder/#1}}& & & \text{\parbox{#4cm}{\raggedright{\small{
\BODY
}}}} \\
\end{tabular}
\end{center} 
}


%\erbulletedModel{#1 pictureFilename}{#2PictureHeight}{#3PictureWidth}{#4TextWidth}

\NewEnviron{erbulletedModel}[4]{%
\begin{center}
$
\begin{array}{m{#3cm} m{1cm} | c m{#4cm}}
\setlength{\erhalfHt}{2cm * \real{0.5}}
\raisebox{-\erhalfHt}{
\input{\erpictureFolder/#1}}& & & \text{\parbox{#4cm}{\raggedright{\footnotesize{
\begin{enumerate}[(i)]
\BODY
\end{enumerate}}}}} \\
\end{array}
$
\end{center} 
}



%\ernotedDimFig{#1 pictureFilename}{#2 figureParam}{#3Caption}{#4PictureHeight}{#5TextWidth}
\NewEnviron{ernotedDimFig}[5]{%
\begin{figure}[#2]
\begin{erexample}
\vspace{-0.5cm}
\begin{center}
$
\begin{array}{c m{0.25cm} | c m{#5cm}}
\setlength{\erhalfHt}{#4cm * \real{0.5}}
\raisebox{-\erhalfHt}{
\input{\erpictureFolder/#1}}& & & \text{\parbox{#5cm}{\raggedright{\footnotesize{
\BODY }}}}\\
\end{array}
$
\end{center}
\caption{#3}
\label{#1}
\end{erexample}
\end{figure} 
}
%\begin{ernotedDimFigPW}{#1 pictureFilename}{#2 figureParam}{#3Caption}{#4PictureHeight}{#5PictureWidth}{#6TextWidth}
\NewEnviron{ernotedDimFigPW}[6]{%
\begin{figure}[#2]
\begin{erexample}
\vspace{-0.5cm}
\begin{center}
$
\begin{array}{>{\centering}m{#5cm} m{0.5cm} | c m{#6cm}}
\setlength{\erhalfHt}{#4cm * \real{0.5}}
\raisebox{-\erhalfHt}{
\centering \input{\erpictureFolder/#1}
}& & & \text{\parbox{#6cm - 0.5cm}{\raggedright{\footnotesize{
\BODY }}}}\\
\end{array}
$ \\
\vspace {0.2cm}
\end{center}
\caption{#3}
\label{#1}
\end{erexample}
\end{figure}
}



\newenvironment{erquote}
{\begin{quote}\itshape}
{\end{quote}}


%
%  erdiag
%
  
%\begin{erdiagram}{#1 height}{#2 width} 
% ....
% ....
%\end{erdiagram}
\newenvironment{erdiagram}[2]
{%\pspicture*(-#1,0)(#2,0)
\pspicture*(0,-#1)(#2,0)
%\psgrid
}
{\endpspicture}

\definecolor{lightyellow}{cmyk}{0,0,0.3,0}

% \eret{#1 x0} {#2 y0} {#3 x1} {#4 y1} {#5 corner radius} {#6 fill}
\newcommand {\eret}[6]
{ 
\ifthenelse{\equal{#6}{1}}
{\psframe[framearc=#5,fillstyle=solid,fillcolor=lightyellow](#1,#2)(#3,#4)}
{\psframe[framearc=#5,fillstyle=solid,fillcolor=white](#1,#2)(#3,#4)}
}

% et top 
\newcommand {\erettop}[4]
{
%\psframe[linestyle=none,linearc=2pt,cornersize=absolute,fillstyle=solid,fillcolor=lightyellow](#1,#2)(#3,#4)
\psline[linearc=2pt,fillstyle=none,fillcolor=lightyellow](#1,#4)(#1,#2)(#3,#2)(#3,#4)
}

% et bottom 
\newcommand {\eretbtm}[4]
{
%\psframe[linestyle=none,linearc=2pt,cornersize=absolute,fillstyle=solid,fillcolor=lightyellow](#1,#2)(#3,#4)
\psline[linearc=2pt,fillstyle=none,fillcolor=lightyellow](#1,#2)(#1,#4)(#3,#4)(#3,#2)
}

% et bottom left
\newcommand {\eretbl}[4]
{
%\psframe[linestyle=none,linearc=2pt,cornersize=absolute,fillstyle=solid,fillcolor=lightyellow](#1,#2)(#3,#4)
\psline[linearc=2pt,fillstyle=none,fillcolor=lightyellow](#1,#4)(#3,#4)(#3,#2)
}

% et middle left
\newcommand {\eretml}[4]
{
%\psframe[linestyle=none,linearc=2pt,cornersize=absolute,fillstyle=solid,fillcolor=lightyellow](#1,#2)(#3,#4)
\psline[linearc=2pt,fillstyle=none,fillcolor=lightyellow](#1,#2)(#3,#2)(#3,#4)(#1,#4)
}

% et top left
\newcommand {\erettl}[4]
{
%\psframe[linestyle=none,linearc=2pt,cornersize=absolute,fillstyle=solid,fillcolor=lightyellow](#1,#2)(#3,#4)
\psline[linearc=2pt,fillstyle=none,fillcolor=lightyellow](#1,#2)(#3,#2)(#3,#4)
}

% et bottom right
\newcommand {\eretbr}[4]
{
%\psframe[linestyle=none,linearc=2pt,cornersize=absolute,fillstyle=solid,fillcolor=lightyellow](#1,#2)(#3,#4)
\psline[linearc=2pt,fillstyle=none,fillcolor=lightyellow](#1,#2)(#1,#4)(#3,#4)
}

% et middle right
\newcommand {\eretmr}[4]
{
%\psframe[linestyle=none,linearc=2pt,cornersize=absolute,fillstyle=solid,fillcolor=lightyellow](#1,#2)(#3,#4)
\psline[linearc=2pt,fillstyle=none,fillcolor=lightyellow](#3,#4)(#1,#4)(#1,#2)(#3,#2)
}

% et top right
\newcommand {\erettr}[4]
{
%\psframe[linestyle=none,linearc=2pt,cornersize=absolute,fillstyle=solid,fillcolor=lightyellow](#1,#2)(#3,#4)
\psline[linearc=2pt,fillstyle=none,fillcolor=lightyellow](#1,#4)(#1,#2)(#3,#2)
}

% \ergrp{#1 x0} {#2 y0} {#3 x1} {#4 y1} {#5 corner radius} {#6 fill}
% #5 corner radius is unused!
\newcommand {\ergrp}[6]
{ 
\ifthenelse{\equal{#6}{1}}
{\psframe[fillstyle=solid,fillcolor=lightgray](#1,#2)(#3,#4)}
{\psframe[fillstyle=solid,fillcolor=white](#1,#2)(#3,#4)}
}

% \eretname {#1 x left of text} {#2 y top of text} {#3 text}
\newcommand {\eretname}[3]
{
%shift down 0.1 for height of text the anchor at baseline (B)
\rput[bl]{0}(0,-0.1){\rput[Bl]{0}(#1,#2){\footnotesize \textit{#3}}}
}

% \errelarm {#1 x0} {#2 y0} {#3 x1} {#4 y1} {#5 ismandatory} {#6 isconstructed}
\newcommand {\errelarm}[6]
{
\ifthenelse{\equal{#6}{1}}
{
%%\psline[linewidth=0.5pt,linearc=.05,linestyle=dashed,dash=6pt 6pt]{-}(#1,#2)(#3,#4)}
\ifthenelse{\equal{#5}{1}}
{\psline[linewidth=1.5pt,linearc=.05,linecolor=lightgray]{-}(#1,#2)(#3,#4)}
{\psline[linewidth=1.5pt,linearc=.05,linecolor=lightgray,linestyle=dashed,dash=2pt 2pt]{-}(#1,#2)(#3,#4)}
}
{
\ifthenelse{\equal{#5}{1}}
{\psline[linewidth=0.9pt,linearc=.05]{-}(#1,#2)(#3,#4)}
{\psline[linewidth=0.9pt,linearc=.05,linestyle=dashed,dash=2pt 2pt]{-}(#1,#2)(#3,#4)}
}
}

% \errelangle {#1 x0} {#2 y0} {#3 x1} {#4 y1} {#5 x2} {#6 y2} {#7 ismandatory} {#8 isocnstructed}
\newcommand {\errelangle}[8]
{
\ifthenelse{\equal{#8}{1}}
{
%\psline[linewidth=0.5pt,linearc=.1,linestyle=dashed,dash=6pt 6pt]{-}(#1,#2)(#3,#4)(#5,#6)}
\ifthenelse{\equal{#7}{1}}
{\psline[linewidth=1.5pt,linearc=.05,linecolor=lightgray]{-}(#1,#2)(#3,#4)(#5,#6)}
{\psline[linewidth=1.5pt,linearc=.1,linecolor=lightgray,linestyle=dashed,dash=2pt 2pt]{-}(#1,#2)(#3,#4)(#5,#6)}
}
{
\ifthenelse{\equal{#7}{1}}
{\psline[linewidth=0.9pt,linearc=.1]{-}(#1,#2)(#3,#4)(#5,#6)}
{\psline[linewidth=0.9pt,linearc=.1,linestyle=dashed,dash=2pt 2pt]{-}(#1,#2)(#3,#4)(#5,#6)}
}
}

% \ercrowfoot {#1 x0} {#2 y0} {#3 x11} {#4 y11} {#5 x12} {#6 y12} {#7 x13} {#8 y13} {#9 isconstructed}
\newcommand {\ercrowfoot}[9]
{
\ifthenelse{\equal{#9}{1}}
{
\psline[linewidth=1.5pt,linearc=.05,linecolor=lightgray]{-}(#1,#2)(#3,#4)
\psline[linewidth=1.5pt,linearc=.05,linecolor=lightgray]{-}(#1,#2)(#5,#6)
\psline[linewidth=1.5pt,linearc=.05,linecolor=lightgray]{-}(#1,#2)(#7,#8)
}{
\psline[linewidth=0.9pt,linearc=.05]{-}(#1,#2)(#3,#4)
\psline[linewidth=0.9pt,linearc=.05]{-}(#1,#2)(#5,#6)
\psline[linewidth=0.9pt,linearc=.05]{-}(#1,#2)(#7,#8)
}
}


% \eridcomprel{#1 x1}{#2 x2}{#3 y1}{#4 ymid}{#5 y2}
\newcommand {\eridcomprel}[5]
{
\psline[linewidth=0.9pt](#1,#3)(#1,#5)
\psline[linewidth=0.9pt](#2,#3)(#2,#5)
\psline[linewidth=0.9pt](#1,#4)(#2,#4)
}

% \eridrefrel{#1 x1}{#2 xmid}{#3 x2}{#4 y1}{#5 y2}
\newcommand {\eridrefrel}[5]
{
\psline[linewidth=0.9pt](#1,#4)(#3,#4)
\psline[linewidth=0.9pt](#1,#5)(#3,#5)
\psline[linewidth=0.9pt](#2,#4)(#2,#5)
}


% \errelname {#1 x} {#2 y} {#3 text}
\newcommand {\errelname}[3]
{
\rput[l]{0}(#1,#2){\textit{#3}}
}
% \errelseq {#1 x} {#2 y}
\newcommand {\erelseq}[2]
{
}
% \erattr {#1 x} {#2 y} {#3 ismandatory}{#4 idenitfying} {#5 text}
\newcommand {\erattr}[5]
{
\ifthenelse{\equal{#3}{1}}
{\rput[l]{0}(#1,#2){{\tiny $\square$} {\footnotesize \textit{\ifthenelse{\equal{#4}{0}}{\underline{#5}}{#5}}}}}
{\rput[l]{0}(#1,#2){\footnotesize $\circ$ \textit{\ifthenelse{\equal{#4}{0}}{\underline{#5}}{#5}}}}
}

%\ifthenelse{\equal{#4}{1}}
% \ertext {#1 x} {#2 y} {#3 text anchor} {#4 text}
%{\rput[l]{0}(#1,#2){\footnotesize $\circ$ \underline{\textit{#5}}}}
\newcommand {\ertext}[4]
{
\rput[B#3]{0}(#1,#2){{\footnotesize #4}}
}
% \erarc {#1 x0} {#2 y0} {#3 x1} {#4 y1} {#5 x2} {#6 y2} {#7 x3} {#8 y3}
\newcommand {\erarc}[8]
{
\psbezier[showpoints=false]{-}(#1,#2) (#3, #4)(#5,#6) (#7, #8)
}

% \erarc {#1 x0} {#2 y0} {#3 x1} {#4 y1} {#5 x2} {#6 y2} {#7 x3} {#8 y3}
\newcommand {\errelseq}[8]
{
\psbezier[showpoints=false]{-}(#1,#2) (#3, #4)(#5,#6) (#7, #8)
}
% \ertrace {#1 trace}   
\newcommand {\ertrace}[1]
{
}

\usepackage{amsthm} % added 7th April 2018
% theorems.macros.tex

\newtheorem{theorem}{Theorem}[section]
\newtheorem{observation}[theorem]{Observation}
\newtheorem{lemma}[theorem]{Lemma}

\newtheorem{proposition}[theorem]{Proposition}
\newtheorem{corollary}[theorem]{Corollary}
\newtheorem{conjecture}[theorem]{Conjecture}
\newtheorem{numbereddefinition}[theorem]{Definition}

\newenvironment{definition}[1][Definition]{\begin{trivlist}
\item[\hskip \labelsep {\bfseries #1}]}{\end{trivlist}}
\newenvironment{examples}[1][Examples]{\begin{trivlist}
\item[\hskip \labelsep {\bfseries #1}]}{\end{trivlist}}
\newenvironment{example}[1][Example]{\begin{trivlist}
\item[\hskip \labelsep {\bfseries #1}]}{\end{trivlist}}
\newenvironment{remark}[1][Remark]{\begin{trivlist}
\item[\hskip \labelsep {\bfseries #1}]}{\end{trivlist}}

\newenvironment{tageqn}[1]
{
\begin{equation}
\stepcounter{equation}
\label{#1}
\tag{\theequation --#1}
}
{
\end{equation}
}

\newenvironment{axiom}[1]
{
\begin{equation}
\label{#1}
\tag{#1}
}
{
\end{equation}
}

% when the tag is required different from the label eg when has math symbols can use:
\newenvironment{axiomtagged}[2]
{
\begin{equation}
\label{#1}
\tag{#2}
}
{
\end{equation}
}

%visible label
\newcommand{\vlabel}[2][]{\label{#2}#1(\textit{#2}):}





\usepackage{mathptmx}  % This changes font to roman
\usepackage{anyfontsize}
\usepackage{mathtools}  % why have we got this?
\usepackage{alltt}    
\usepackage{mnsymbol} %used for rightpitchfork
\usepackage{cmll}
\usepackage{ulem}
\renewcommand{\ttdefault}{txtt}
\usepackage[left=1.5cm, right=4cm, marginparwidth=3cm, top=2cm, bottom=1.5cm]{geometry}
\usepackage{framed}
\usepackage[font=small]{caption}
\setlength{\captionmargin}{2cm}
\newcommand{\commentary}[1]{\marginpar{\footnotesize #1}}

\renewcommand{\erpictureFolder}[0]{../SharedPictures}

\newenvironment{categoricalaside}
{\begin{framed}
\textbf{Categorical Aside}
}
{
\end{framed}
}

\newenvironment{noteforfuture}
{\begin{framed}
\textbf{Note For Future}
}
{
\end{framed}
}

\newenvironment{problem}
{\begin{framed}
\textbf{Problem}
}
{
\end{framed}
}


%from berkley
\newcommand{\langl}{\begin{picture}(4.5,7)
\put(1.1,2.5){\rotatebox{60}{\line(1,0){5.5}}}
\put(1.1,2.5){\rotatebox{300}{\line(1,0){5.5}}}
\end{picture}}
\newcommand{\rangl}{\begin{picture}(4.5,7)
\put(.9,2.5){\rotatebox{120}{\line(1,0){5.5}}}
\put(.9,2.5){\rotatebox{240}{\line(1,0){5.5}}}
\end{picture}}
\newcommand{\lang}{\begin{picture}(5,7)\put(1.1,2.5){\rotatebox{45}{\line(1,0){6.0}}}\put(1.1,2.5){\rotatebox{315}{\line(1,0){6.0}}}\end{picture}}
\newcommand{\rang}{\begin{picture}(5,7)\put(.1,2.5){\rotatebox{135}{\line(1,0){6.0}}}\put(.1,2.5){\rotatebox{225}{\line(1,0){6.0}}}\end{picture}}
%Try sharper tuple brackets -- except gives errors nested in captions so comment out
%\renewcommand{\tuple}[1]{\lang #1 \rang}

\newcommand{\setsuchthat}[2]{\left\{#1 \ \middle|\ #2\right\}}
\newcommand{\set}[1]{\left\{#1\right\}} 

\newcommand{\genericmodel}{\mathcal{M}}  %PREVIOUSLY
\renewcommand{\genericmodel}{{m}}        %PREVIOUSLY
\renewcommand{\genericmodel}{\gamma}     % TRY THIS FOR A WHILE except texworks isnt happy with greek
%\renewcommand{\genericmodel}{M}  %while debugging
\newcommand{\chiZero}{\mathcal{X}_0}
\newcommand{\chiZeroM}{\chiZero(\genericmodel)}
\newcommand{\chiOne}{\mathcal{X}_1}
\newcommand{\chiOneM}{\chiOne(\genericmodel)}
\newcommand{\chiM}{\mathcal{X}(\genericmodel)}
\newcommand{\veee}{v}
\newcommand{\Veee}{V}
\newcommand{\et}[1][\genericmodel]{et_{#1}}
\newcommand{\edge}[3][\genericmodel]{Edge_{#1}(#2,#3)}
\newcommand{\iedge}[3][\genericmodel]{IEdge_{#1}(#2,#3)}
\newcommand{\path}[3][\genericmodel]{Path_{#1}(#2,#3)}
\newcommand{\ipath}[3][\genericmodel]{IPath_{#1}(#2,#3)}
\newcommand{\attr}[2] [\genericmodel]{attr_{#1}(#2)}
\newcommand{\iattr}[2] [\genericmodel]{IAttr_{#1}(#2)}
\newcommand{\rel}[3][\genericmodel]{rel_{#1}(#2,#3)}
\newcommand{\irel}[3][\genericmodel]{IRel_{#1}(#2,#3)}
\newcommand{\iedges}[2] [\genericmodel]{i_{#1}(#2)}
\newcommand{\pk}[2] [\genericmodel]{pk_{#1}(#2)}
\newcommand{\fk}[2] [\genericmodel]{fk_{#1}(#2)}
\newcommand{\fkp}[2] [\genericmodel]{fk'_{#1}(#2)}
\newcommand{\fkpp}[2] [\genericmodel]{fk''_{#1}(#2)}

%functional dependencies
\newcommand{\sfd}[2]{\ensuremath{\set{#1} \morph #2}}  %singleton
\newcommand{\fd}[2]{\ensuremath{\sfd{#1}{\set{#2}}}}

\newcommand{\simplepath}[2]{
\ncline[linestyle=none,linewidth=0.1pt]{#1}{#2}   %was linestyle=dotted
\ncput[npos=0.05]{\pnode{dot#21}}
\ncput[npos=0.27]{\dotnode[dotsize=1pt]{dot#22}}
\ncput[npos=0.50]{\dotnode[dotsize=1pt]{dot#23}}
\ncput[npos=0.80]{\dotnode[dotsize=1pt]{dot#24}}
\ncput[npos=0.975]{\pnode{dot#25}}
\ncline[nodesep=2pt]{->}{dot#21}{dot#22}
\ncline[nodesep=2pt]{->}{dot#22}{dot#23}
\ncline[nodesep=2pt]{->}{dot#24}{dot#25}
\ncline[linestyle=dotted,nodesep=8pt]{dot#23}{dot#24} %was 10pt
}

\newcommand{\simplepatha}[3]{
\simplepath{#2}{#3}
\naput[labelsep=1pt]{#1}
}

\newcommand{\simplepathb}[3]{
\simplepath{#2}{#3}
\nbput[labelsep=1pt]{#1}
}
\newcommand{\term}[1]{\textit{{#1}}}
\newcommand{\logtophys}{\mathcal{X}}
\newcommand{\chen}{\mathcal{X}_0}
\newcommand{\chengenericmodel}{\chen(\genericmodel)}
\newcommand{\chigenericmodel}{\logtophys(\genericmodel)}
\newcommand{\phys}[1]{\overline{#1}}
\newcommand{\genericphysical}{\logtophys(\genericmodel)}

\newcommand{\inc}{\subseteq}
\newcommand{\incd}[4]{#1\left[#2\right]\inc#3\left[#4\right]}

\newcommand{\ntuple}[1]{\tuple{#1_1,...#1_n}}
\newcommand{\mtuple}[1]{\tuple{#1_1,...#1_m}}

\newcommand {\bntuple}{\ensuremath{\ntuple{b}}}
\newcommand {\fntuple}{\ensuremath{\ntuple{f}}}
\newcommand {\pntuple}{\ensuremath{\ntuple{p}}}
\newcommand {\qntuple}{\ensuremath{\ntuple{q}}}
\newcommand {\qmtuple}{\ensuremath{\mtuple{q}}}
\newcommand {\xntuple}{\ensuremath{\ntuple{x}}}
\newcommand {\ymtuple}{\ensuremath{\mtuple{y}}}
\newcommand{\foreachi}[1][n]{for each $i$, $1 \leq i \leq #1$}
\newcommand{\foreachj}[1][m]{for each $j$, $1 \leq j \leq #1$}
\newcommand{\foreachk}[1][l]{for each $k$, $1 \leq k \leq #1$}
\newcommand{\fdfactoring}{fd factoring}
\newcommand{\attributelike}{attribute-like}


%\usepackage{needspace}

\begin{document}
\title{Examples}

% abstract here for ams

\author{John Cartmell}

\maketitle
\begin{center}
DRAFT
\end{center}

\appendix
\section{Examples}

\subsection{chen example sans bars}
\begin{figure} [h]  % chen fragment SANS bars
\begin{center}
\begin{tabular}{c c}
\barsfalse % supresses \idcomp macro
$
\begin{array}{c p{0.05cm}c p{0.5cm}c}
                        & & \Rnode{p}{p}& &             \\ [0.3cm]
	 \Rnode{w}{w}	& &                   & & \Rnode{v}{v} \\ [0.3cm]     
	                      & & \Rnode{e}{e}      & &             \\ [0.6cm]     
	                      & & \Rnode{dep}{dep}  & &             
\end{array}
$
\ncarr{w}{p} 
\alabel{R_1}[0.5][-1]
\idcomp
\ncarr{w}{e} 
\blabel{R_2}[0.5][-1]
\idcomp
\ncarr{dep}{e} 
\alabel{S_0}
\idcomp
\ncarr{p}{v}
\alabel{pNo}[0.4][-1]
\idcomp
\ncarr{e}{v}
\alabel{eNo}[0.45][1]
\idcomp
\ncarr[-25]{e}{v}
\blabel{eN}[.3][0]
%\ncarr[-90]{e}{v}
%\blabel{eDob}
\ncarr[-20]{dep}{v}
\blabel{dN}[0.3][-1]
\idcomp
\ncarr[-60]{dep}{v}
\blabel{ddoB}[0.3][-1]
\nccurve[angleA=90,angleB=90,nodesep=2pt,ncurv=0.9]{->}{w}{v}
\alabel{pcnt}[0.3][-1]
& \footnotesize
\begin{tabular}{c p{1.5cm} p{4cm}}
KEY && \\
\hline
p & project & Identified by project number attribute ($pNo$).\\
e & employee & Identified by employee number attribute ($eNo$). \\
w  & project-worker & Identified by combination of relationship $R_0$ to an employee and relationship $R_1$ to a project. Has a percentage of time attribute ($pcnt$) which is percentage of time employee is allocated to work on the project.\\
dep & dependent & Identified by a combination of relationship $S_0$ to an employee and a name attribute ($dN$).\\
\end{tabular} 
\end{tabular}
\end{center}
\caption{A directed graph representing a fragment of Chen's example that illustrates analysis of information for a manufacturing firm. Other attributes such as project name, employee date of birth (dob) and so on not shown.}
\label{chenfragmentSANSbars}
\end{figure}



\subsection{chen example with bars}
\begin{figure} [h]  % chen fragment with bars
\begin{center}
\barstrue
$
\begin{array}{c p{0.05cm}c p{0.5cm}c}
                        & & \Rnode{p}{p}& &             \\ [0.3cm]
	 \Rnode{w}{w}	& &                   & & \Rnode{v}{v} \\ [0.3cm]     
	                      & & \Rnode{e}{e}      & &             \\ [0.6cm]     
	                      & & \Rnode{dep}{dep}  & &             
\end{array}
$
\ncarr{w}{p} 
\alabel{R_1}[0.5][-1]
\idcomp
\ncarr{w}{e} 
\blabel{R_2}[0.5][-1]
\idcomp
\ncarr{dep}{e} 
\alabel{S_0}
\idcomp
\ncarr{p}{v}
\alabel{pNo}[0.4][-1]
\idcomp
\ncarr{e}{v}
\alabel{eNo}[0.45][1]
\idcomp
\ncarr[-25]{e}{v}
\blabel{eN}[.3][0]
%\ncarr[-90]{e}{v}
%\blabel{eDob}
\ncarr[-20]{dep}{v}
\blabel{dN}[0.3][-1]
\idcomp
\ncarr[-60]{dep}{v}
\blabel{ddoB}[0.3][-1]
\nccurve[angleA=90,angleB=90,nodesep=2pt,ncurv=0.9]{->}{w}{v}
\alabel{pcnt}[0.3][-1]
\end{center}
\caption{The same Chen fragment with bars indicating the identifying relationships and attributes. In this example the primary key for each entity type consists of the set of paths through the graph consisting entirely
of identifying (bar'ed) attributes and relationships and leading from itself to the universal type $\veee$ .}
\label{chenfragment}
\end{figure}

\needspace{30\baselineskip}
\subsection{table data cell}
\begin{figure} [h] % data table
\begin{center}
\begin{tabular}{c c}
$
\begin{array}{cp{0.75cm}cp{0.75cm}c}
   \Rnode{r}{r}     & & \Rnode{t}{t} & & \Rnode{v}{v} \\[1.2cm]     
	 \Rnode{d}{d}   & & \Rnode{c}{c} & &               
\end{array}
$
\ncarr{r}{t} 
\alabel{S_1}
\idcomp
\ncarr{t}{v} 
\alabel{tN}
\idcomp
\ncarr{c}{v} 
\blabel{cN}
\idcomp
\ncarr{d}{c}
\blabel{R_0}
\idcomp
\ncarr{d}{r}
\alabel{S_0}
\idcomp
\ncarr{c}{t}
\blabel{R_1}
\idcomp
\ncarr[50]{r}{v}
\alabel{rN}
\idcomp

& \footnotesize
\begin{tabular}{c p{1.5cm} p{4cm}}
KEY && \\
\hline
t & table & Having identifying attribute tN the name of the table. \\
c & column & Identified by a combination of column number cN and relationship $R_1$ to the table it is a column of.\\
r & row & Identified by its row number $rN$ and its relationship $S_1$ to the table it is a row of.\\
d & data cell & Identified by relationship $S_0$ to the row it is in and relationship $R_0$ to the row it is in. \\
\end{tabular} 
\end{tabular}
\end{center}
\caption{Example of path equivalence - the path $\tuple{R_0,R_1}$ is equivalent to path $\tuple{S_0,S_1}$}.
\label{datatablegraph}
\end{figure}

\subsection{relational meta model}
\begin{figure} [h]
\begin{center}
\begin{tabular}{c c}
$
\begin{array}{cp{0.75cm}cp{0.75cm}c}
   \Rnode{fk}{fk}     & & \Rnode{t}{t} & & \Rnode{v}{v} \\[1.2cm]     
	 \Rnode{fkc}{fkc}   & & \Rnode{c}{c} & &               
\end{array}
$
\ncarr{fk}{t} 
\alabel{S_1}
\ncarr{t}{v} 
\alabel{tN}
\idcomp
\ncarr{c}{v} 
\blabel{cN}
\idcomp
\ncarr{fkc}{c}
\blabel{R_0}
\ncarr{fkc}{fk}
\alabel{S_0}
\idcomp
\ncarr{c}{t}
\blabel{R_1}
\idcomp
\ncarr[50]{fk}{v}
\alabel{FkN}
\idcomp
\ncarr[-90]{fkc}{v}
\blabel{SeqN}
\idcomp
& \footnotesize
\begin{tabular}{c p{1.5cm} p{4cm}}
KEY && \\
\hline
t & table & Having identifying attribute tN the name of the table. \\
c & column & Identified by a combination of column name cN and relationship $R_1$ to the table it is a column of.\\
fk & foreign key & Identified by its name $FkN$.\\
fkc & foreign key column & Identified by relationship $S_0$ to the foreign key it is a part of and its sequence number $SeqN$ i.e. the position it appears in within the foreign key. \\
\end{tabular} 
\end{tabular}
\end{center}
\caption{Example of path equivalence - the path $\tuple{R_0,R_1}$ is equivalent to path $\tuple{S_0,S_1}$.}
\label{foreignkeygraph}
\end{figure}



\needspace{30\baselineskip}
\subsection{zaniolo example -- telephone place area}
\begin{figure} [h]
\begin{center}
\begin{tabular}{c c}
$
\begin{array}{cp{0.4cm}cp{0.75cm}cp{0.75cm}c}
              &&               &&                &&               \\[0.25cm]
              &&               &&                &&               \\[0.25cm]
\Rnode{t}{t}	&& \Rnode{p}{p}  &&   \Rnode{a}{a} && \Rnode{v}{v}  \\[0.25cm]
	            &&  
\end{array}
$

\ncarr{t}{p} 
\alabel{ptin}
\idcomp
\ncarr[-35]{t}{v}
\blabel{ltno}
\idcomp
\ncarr[30]{p}{v}
\alabel{pname}
\idcomp
\ncarr{p}{a}
\blabel{apin}
\idcomp
\ncarr{a}{v}
\blabel{acode}
\idcomp
& \footnotesize
\begin{tabular}{c p{1.5cm} p{4cm}}
KEY && \\
\hline
t  & telephone       & Identified by a combination of the place a telephone is in ($ptin$) 
                      and the local telephone number ($ltno$). \\
p  & place           & Identified by a combination of the area a place is in ($apin$)
                         and the place name ($pname$). \\
a  & area            & Identified by the area code ($acode$).
\end{tabular} 
\end{tabular}
\end{center}
\caption{Example based on Zaniolo's telephone,place.area example. 
This is an example of an ER model that fails the minimality condition defined in section \ref{minimalitycondition}.
The failure arises at entity type t (telephone) because 
the set of paths $\set{ptin\circ pname, ptin \circ apin \circ acode, ltno}$ is
in $\bar{I}$ and is therefore a mono-source but it is not minimal because
 an area has at most one telephone per local telephone number and so the 
subset $\set{ptin \circ apin \circ acode, ltno}$ is  a
mono-source. Further, the path $ptin \circ pname$ is not equivalent to either of the two paths in the subset.
This problem can be fixed by introducing an area relationship $atin: t \morph a$ and extended each defining instance $E$
 by which is defining $E'_atin=E_{ptin \circ apin}$  and by re-specifying the identifying features $t$ to be the set $\set{atin,ltno}$,
so obtaining an extended model in which $atin$ it to be equivalent to
path $ptin \circ apin$.
}
\label{clubpresidentbeforenormalisation}
\end{figure}
FOR GOODNESS SAKE SKETCH THE REVISED MODEL!!

\subsection{club president}
\begin{figure} [h]
\begin{center}
\begin{tabular}{c c}
$
\begin{array}{cp{0.4cm}cp{0.75cm}cp{0.75cm}c}
              &&               &&                &&               \\[0.25cm]
              &&               &&                &&               \\[0.25cm]
\Rnode{p}{p}	&& \Rnode{m}{m}  &&   \Rnode{c}{c} && \Rnode{v}{v}  \\[0.25cm]
	            &&  
\end{array}
$

\ncarr{p}{m} 
\alabel{I}
\idcomp
\ncarr[-35]{p}{v}
\blabel{yr}
\idcomp
\ncarr[30]{m}{v}
\alabel{mNo}
\idcomp
\ncarr{m}{c}
\blabel{M}
\idcomp
\ncarr{c}{v}
\blabel{cId}
\idcomp
& \footnotesize
\begin{tabular}{c p{1.5cm} p{4cm}}
KEY && \\
\hline
p  & president        & Identified by a combination of inclusion relationship ($I$) that identifies a
                       president as being a committee members and a year attribute ($yr$). \\
m  & member           & Identified by a member relationship ($M$) to a club and a by membership 
                         number attribute ($mNo$). \\
c  & club             & Identified by club identifier attribute ($cId$)
\end{tabular} 
\end{tabular}
\end{center}
\caption{Club president example. 
This is an example of an ER model that fails the minimality condition defined in section \ref{minimalitycondition}.
The failure arises at entity type p (president) because 
the set of paths $\set{I\circ mNo, I \circ M \circ cId, yr}$ is
in $\bar{I}$ and is therefore a mono-source but it is not minimal because
 a club has at most one president a year and so the subset $\set{I \circ M \circ cId, yr}$ is  a
mono-source. Further, the path $I \circ mNo$ is not equivalent to either of the two paths in the subset.
This problem can be fixed by introducing a club membership relationship $R: p \morph c$ and extended each defining instance $E$
 by which is defining $E'_R=E_{I \circ M}$  and by re-specifying the identifying features $p$ to be the set $\set{R,yr}$,
so obtaining an extended model in which $R$ it to be equivalent to
path $I \circ M$.
}
\label{clubpresidentbeforenormalisation}
\end{figure}

\needspace{30\baselineskip}
\subsection{club president committee member}
\begin{figure} [h]
\begin{center}
\begin{tabular}{c c}
$
\begin{array}{cp{0.4cm}cp{0.5cm}cp{0.75cm}cp{0.75cm}c}
              &&                &&               &&                &&               \\[0.25cm]
              && \Rnode{cm}{cm} &&               &&                &&               \\[0.25cm]
\Rnode{p}{p}	&&                && \Rnode{m}{m}  &&   \Rnode{c}{c} && \Rnode{v}{v}  \\[0.25cm]
	            &&  
\end{array}
$

\ncarr[5]{p}{cm} 
\alabel{I_0}
\idcomp
\ncarr[-30]{p}{v}
\blabel{yr}
\idcomp
\ncarr[-5]{cm}{m}
\blabel{I_1}
\idcomp
\ncarr[50]{cm}{v}
\alabel{yr1}
\idcomp
\ncarr[25]{m}{v}
\alabel{mNo}
\idcomp
\ncarr{m}{c}
\blabel{M}
\idcomp
\ncarr{c}{v}
\blabel{cId}
\idcomp
& \footnotesize
\begin{tabular}{c p{1.5cm} p{4cm}}
KEY && \\
\hline
p  & president        & Identified by a combination of inclusion relationship ($I_0$) that identifies a
                       president as being a committee members and a year attribute ($yr$). \\
cm & committee member & Identified by an inclusion relationship ($I_1$)that identifies a committee member 
                         as a member and a first year of service attribute ($y1$).\\
m  & member           & Identified by a member relationship ($M$) to a club and a by membership 
                         number attribute ($mNo$). \\
c  & club             & Identified by club identifier attribute ($cId$)
\end{tabular} 
\end{tabular}
\end{center}
\caption{Club president example. 
This example fails clause (i) of the definition of well-formedness because
the set of paths $\set{\tuple{I_0,yr1},\tuple{I_0,I_1,mNo}, \tuple{I_0,I_1,M,cId}, \tuple{yr}}$ is 
an identiofying set of paths wrt $p$ but is not minimum because a club has at most one president a year and sothe subset $\set{\tuple{I_0,yr1}, \tuple{I_0,I_1,M,cId}, \tuple{yr}}$ is jointly monomorphic. 
This problem can be fixed by introducing a clum membership relationship $R: p \morph c$ which is equivalent to
path $\tuple{I_0,I_1,M}$ and by specifying $R$ in place of $I_0$ which is to be combined with $yr$ to identify entitites of type $p$.
}
\label{clubpresidentbeforenormalisation}
\end{figure}

\subsection{raw table data cell}
\begin{figure} [h]
\begin{tabular}{c p {0.5cm} c}
(a) & &
\begin{tabular}{c c}
$
\begin{array}{cp{0.75cm}cp{0.75cm}c}
   \Rnode{r}{r}     & & \Rnode{t}{t} & & \Rnode{v}{v} \\[1.2cm]     
	 \Rnode{d}{d}   & & \Rnode{c}{c} & &               
\end{array}
$
\ncarr{r}{t} 
\alabel{S_1}
\idcomp
\ncarr{t}{v} 
\alabel{tN}
\idcomp
\ncarr{c}{v} 
\blabel{cN}
\idcomp
\ncarr[-90]{d}{v}
\blabel{SeqNo}
\idcomp
\ncarr{d}{r}
\alabel{S_0}
\idcomp
\ncarr{c}{t}
\blabel{R_1}
\idcomp
\ncarr[50]{r}{v}
\alabel{rN}
\idcomp

& \footnotesize
\begin{tabular}{c p{1.5cm} p{4cm}}
KEY && \\
\hline
t & table & Having identifying attribute tN the name of the table. \\
c & column & Identified by a combination of column number cN and relationship $R_1$ to the table it is a column of.\\
r & row & Identified by its row number $rN$ and its relationship $S_1$ to the table it is a row of.\\
d & data cell & Identified by relationship $S_0$ to the row it is in and its $seqNo$ attribute
representing its ordinal position within data cells of the row. \\
\end{tabular} 
\end{tabular} \\
(b) &   & 
$
\begin{array}{cp{0.5cm}c c c }
   \Rnode{d}{d} &                  &                       & \Rnode{c}{c}  &       \\[.5cm]  
		            & \ \ \ \Rnode{r}{r} &                     &               &       \\[.5cm] 
	              &                  & \Rnode{b1}{b_1} \ \ \ &               &  \Rnode{bn}{b_n}  
\end{array}
$
\ncarr{d}{r} 
\blabel{S_0}[0.25]
\ncarr{r}{b1} 
\blabel{S_1}[0.25]
\ncarr{d}{bn}
\alabel{SeqNo}[0.25]
\ncarr{c}{b1}
\blabel{R_1}[0.25][1]
\idcomp
\ncarr{c}{bn}
\alabel{cN}[0.25]
\idcomp
\end{tabular}
\caption{(a) The model fails condition (ii) for being well-formulated
because there is a referential inclusion dependency of $d[ \tuple{S_0,S_1},\tuple{SeqNo}]$
in $c[\tuple{R_0},\tuple{cN}]$, as indicated in (b), which is not explicitly represented in the model.}
\label{rawdatatablegraph}
\end{figure}


\needspace{30\baselineskip}
\subsection{pick example}
\begin{figure} [h]
\begin{center}
\begin{tabular}{c c}
\begin{tabular}{cp{0.75cm}cp{0.75cm}c}
   \Rnode{d}{d}    & & \Rnode{w}{w}   & & \Rnode{v}{v}\\[1.2cm]     
	 \Rnode{p}{p}  & & \Rnode{c}{c} & &               
\end{tabular}
\ncarr[90]{d}{v}
\alabel{oId}[0.4]
\idcomp
\ncarr[40]{d}{v}
\alabel{dSn}[0.4]
\idcomp
\ncarr{d}{w} 
\alabel{S_1}
\ncarr{w}{v} 
\alabel{wId}
\idcomp
\ncarr{c}{v} 
\blabel{cSn}
\idcomp
\ncarr{p}{c}
\blabel{R_0}
\idcomp
\ncarr{p}{d}
\alabel{S_0}
\idcomp
\ncarr{c}{w}
\blabel{R_1}
\idcomp
& \footnotesize
\begin{tabular}{c p{1.5cm} p{4cm}}
KEY && \\
\hline
d & delivery & Identified by a combination of an order id attribute ($oId$) and a
delivery serial number  attribute ($dSn$). \\
w & warehouse & Identified by a warehouse id attribute ($wId$). \\
c & crate & Identified by a combination of a crate serial number attribute ($cSn$) and a relationship $R_1$ to the warehouse in which it is held.
             Also has an item type attribute (not shown).\\
p & pick & Identified by relationship $S_0$ to the delivery  it is part of and relationship $R_0$ to the crate being picked from. Has a quantity picked attribute (not shown). \\
\end{tabular} 
\end{tabular}
\end{center}
\caption{Pick Example}
\label{pickexample}
\end{figure}

\subsection{pick example pre normalisation}
\begin{figure} [h]
\begin{center}
\begin{tabular}{c c}
$
\begin{array}{cp{0.75cm}cp{0.75cm}c}
   \Rnode{p}{p}     & & \Rnode{c}{c} & & \Rnode{v}{v}    
\end{array}
$
\ncarr{p}{c} 
\alabel{R}
\idcomp
\ncarr[25]{c}{v} 
\alabel{wId}
\idcomp
\ncarr[-10]{c}{v} 
\blabel{cSn}
\idcomp
\ncarr[-65]{c}{v}
\blabel{iC}
\ncarr[80]{p}{v}
\alabel{dI}
\idcomp
\ncarr[-100]{p}{v}
\blabel{q}
& \footnotesize
\begin{tabular}{c p{1.5cm} p{4cm}}
KEY && \\
\hline
p & pick & Identified by a combination of relationship $R$ to the crate being picked from
                and the delivery id attribute ($dI$). Also  having a quantity picked 
								attribute ($q$).\\
c & crate & Identified by a combination of warehouse name attribute ($wN$)
              and crate serial number attribute ($cSn$). Also has an item type ($iT$) attribute.
 to the table it is 
\end{tabular} 
\end{tabular}
\end{center}
\caption{If all deliveries are made from items picked from a single warehouse 
then this model fails clause (iii) of the definition of well-formulated because, with tespect to entity type $p$, the path $\tuple{R,wN}$ will be dependent on the path $\tuple{dI}$ but neither is the set 
$\{\tuple{dI}\}$ a mono-source nor has the dependency a representation within the model. A well-formulated and fully factored model for this situation was described previously in 
figure \ref{pickexample}.
}
\label{pickexamplebeforenormalisation}
\end{figure}

\needspace{25\baselineskip}
\subsection{pick transitive functional dependency}

\begin{figure}[h]
\begin{center}
\begin{tabular}{p{3.0cm} c p{1cm} c}
The dependency of &$
\begin{array}{cp{0.5cm}c}
\Rnode{p}{p}   &  &   \Rnode{c}{c}
\end{array}
$
\ncarr{p}{c}
\blabel{R_0}
\idcomp  
                    & on & $
\begin{array}{cp{0.5cm}cp{0.5cm}c}
               &  &                  & &     \Rnode{v1}{v}  \\[-0.35cm]
               &  &   \Rnode{d1}{d}  & &                    \\[0.2cm]
\Rnode{p}{p}   &  &   \Rnode{d2}{d}  & &     \Rnode{v2}{v}  \\[0.35cm]
               &  &   \Rnode{c2}{c}  & &                    \\[-0.35cm]
               &  &                  & &     \Rnode{v3}{v}
\end{array}
$
\ncarr[10]{d1}{v1}
\alabel{oId}
\idcomp 
\ncarr{d2}{v2}
\blabel{dSn}
\idcomp
\ncarr[10]{p}{d1}
\alabel{S_0}
\idcomp 
\ncarr{p}{d2}
\blabel{S_0}
\idcomp
\ncarr[-10]{p}{c2}
\blabel{R_0}
\idcomp 
\ncarr[-5]{c2}{v3} 
\blabel{cSn}
\idcomp \\[0.5cm]
										follows from \\[0.5cm]
the dependency of & $
\begin{array}{cp{0.5cm}c}
\Rnode{p}{p}   &  &   \Rnode{c}{c}
\end{array}
$
\ncarr{p}{c}
\blabel{R_0}
\idcomp  & on    &
$
\begin{array}{cp{0.5cm}cp{0.5cm}c}
               &  &                  & &     \Rnode{w}{w}   \\[-0.35cm]
               &  &   \Rnode{d}{d}  & &                    \\[-0.05cm]
\Rnode{p}{p}   &  &                  & &                    \\[-0.05cm]
               &  &   \Rnode{c2}{c}  & &                    \\[-0.25cm]
               &  &                  & &     \Rnode{v}{v}
\end{array}
$
\ncarr[5]{p}{d}
\alabel{S_0}
\idcomp 
\ncarr[-5]{p}{c2}
\blabel{R_0}
\idcomp 
\ncarr[5]{d}{w}
\alabel{S_1}
\ncarr[-5]{c2}{v} 
\blabel{cSn}
\idcomp        \\[0.5cm]
and from \\[0.5cm]
the dependency of & 
$
\begin{array}{cp{0.5cm}cp{0.5cm}c}
\Rnode{p}{p}   &  &   \Rnode{d}{d}  & &    \Rnode{w}{w}                
\end{array}
$
\ncarr[5]{p}{d}
\alabel{S_0}
\idcomp 
\ncarr[5]{d}{w}
\alabel{S_1}

& on&
$
\begin{array}{cp{0.5cm}cp{0.5cm}c}
               &  &                  & &     \Rnode{v1}{v}   \\[-0.35cm]
               &  &   \Rnode{d1}{d}  & &                    \\[-0.05cm]
\Rnode{p}{p}   &  &                  & &                    \\[-0.05cm]
               &  &   \Rnode{d2}{d}  & &                    \\[-0.25cm]
               &  &                  & &     \Rnode{v2}{v}
\end{array}
$
\ncarr[5]{p}{d1}
\alabel{S_0}
\idcomp 
\ncarr[-5]{p}{d2}
\blabel{S_0}
\idcomp 
\ncarr[5]{d1}{v1}
\alabel{oId}
\ncarr[-5]{d2}{v2} 
\blabel{dSn}
\idcomp 
\end{tabular}
\caption{Example of a Transitive Functional Dependency based on the example in figure \ref{pickexample}}
\end{center}
\end{figure}

\needspace{20\baselineskip}
\subsection{Team Selection - Preliminary Model}
\llabel{teamselectionpreliminary}

\commentary {An example of a model that does not have the fd factoring property.}
\commentary{From this model the $\chi$ transform produces a relational model that is not in third normal form.}
We imagine a situation in which a number of people are separated into teams by asking each person to select a person they wish to be teamed up with. Teams of various sizes are picked based on this information and coloured vests are worn by people to be indicative of the teams that they are in.
We choose to represent this team information by the schema shown in figure \ref{teamselectionpreliminaryERschema}. 

\begin{figure} [h]
\begin{center}
\begin{tabular}{p{3.5cm} c}
\begin{tabular}{c p{1.5cm} c}
   \Rnode{p}{p} & & \Rnode{v}{v}
\end{tabular}
%\nccircle[nodesep=3pt]{<-}{p}{.4cm}
\rEt[270]{p}
\alabel{S}[0.6]
\Et[-40]{p}{v}
\blabel{c}[0.6]
\Etm{p}{v} 
\alabel{pId}[0.6]
\idcomp
& \footnotesize
\begin{tabular}{c p{1.5cm} p{4cm}}
KEY && \\
\hline
p & person & Identified by id attribute ($pId$). \\
s & selects & each person selects exactly one other person, each person is selected by zero, one or more persons\\
c & colour & each person is given a coloured vest 
\end{tabular} 
\end{tabular}
\end{center}
\caption{Team Selection Example. The  vest colour  of a person is 
identical to the vest colour of their selected buddy which is to say that the path equivalence $S \circ c \simeq c$ holds. From this
we can show that this model exhibits a transitive functional dependency which does not factor through an intransitive dependency.
}
\label{teamselectionpreliminaryERschema}
\end{figure}

That the  vest colour  of a person is 
identical to the vest colour of their selected buddy can be expressed by the path equivalence $S \circ c \simeq c$. 
\begin{categoricalaside}
Another way of expressing this 
path equivalence, $S \circ c \simeq c$, is by stating that in the category of paths through the schema  the diagram
$
\begin{array}{c p{0.75cm} c}
   \Rnode{p1}{p}  & &                  \\[0.5cm]
	                 & &    \Rnode{v}{v} \\[0.5cm]
   \Rnode{p2}{p}  & &
			
\end{array}
$
\ncarr{p1}{v}
\alabel{c}[0.4]
\ncarr{p2}{v}
\blabel{c}[0.4]
\ncarr{p1}{p2}
\blabel{S}[0.4]
commutes. 
\end{categoricalaside}

\subsubsection{Analysis -- Failure of the fd factoring property}
\llabel{teamselectionpreliminaryanalysis}
We now show that this model does not have the fd factoring property by showing that
the model has a transitive functional dependency $\ssfd{S/pId}{c}$ which is between
attribute-like paths, whose left hand side, $S/pId$, is not a mono-source and  which cannot be factored right-intransitively.

First note that because both the domain and  codomain of $S$ is $p$ then there are paths $S \circ S$, $S \circ S \circ S$ 
and so on in this model and these we shall denote by $S^2$, $S^3$ and so on. 
From the path equivalence $S \circ c \simeq c$ it follows that for any $n$, there is
a path equivalence 
\begin{equation}
\label{tspSncSIMEQc}
S^n \circ c \simeq c
\end{equation}

By lemma \lref{fdfrompathextension} there is a functional dependency
\begin{equation}
\label{tspSc}
\ssfd{S^n}{S^n \circ c}
\end{equation}

(The situation does not constrain $c$ to be injective on the range of $S^n$ and so this functional dependency is not reversible.)

From (\ref{tspSncSIMEQc}) and (\ref{tspSc}) it follows by lemma \ref{fdfrompathequivalence} that for any $n$ there is a functional dependency
\begin{equation}
\label{tspSnc}
\ssfd{S^n}{c}
\end{equation}


It also follows from lemma \lref{fdfrompathextension} that for each $n$ there is a functional dependency
\begin{equation}
\label{tspSnn1}
\ssfd{S^n}{S^{n+1}}
\end{equation}
The situation does not constrain $S$ to be injective on the range of $S^n$ and so this functional dependency is not reversible.

Assuming as we do that selection is unconstrained then $S$ is not a mono-source in this model and there can be arbitrary long chains of selection.
In  all therefore there is an infinity of functional dependencies non of which are reversible

\begin{equation}
S \morph S/S \morph S/S/S \morph S/S/S/S .. \morph c
\end{equation}

Finally by lemma \lref{fdfromfdplusmonosource} from the fact that $\ssfd{S}{S/S}$  is an irreversible functional depedency
we may conclude that $\ssfd{S \circ pId}{S/S}$ is an irreversible functional dependency.

Therefore we have an infinite sequence of functional dependencies
\begin{equation}
S/pId \morph S/S \morph S/S/S \morph S/S/S/S .. \morph c
\end{equation}
and because the first of these is irreversible
the functional dependency
$\ssfd{S/pId}{c} $ 
is an intransitive functional dependency.

We have shown that this model has a transitive functional dependency $\ssfd{S/pId}{c}$ which is between
attribute-like paths but which cannot be factored right-intransitively. The left hand side of this functional dependency, $S/pId$, is not a mono-source, therefore this model doesn't have the fd factoring property 

\subsubsection{The $\chi$-generated Relational Model}
\llabel{teamselectionpreliminaryrelationalschema}
Since this model does not have the fd factoring property then it does not meet the the conditions of theorem \ref{goaltheorem} and  we cannot be sure that the relational model generated by the $\chi$ mapping will  be in 3NF.
In fact, in the relational design produced by the $\chi$ mapping the person entity type, (abbreviated $p$ in the schema diagram above), maps to the following relation:
\begin{equation}
\label{personrelation}
person(\underline{pId}, spId, c)
\end{equation}
The  attributes of this relation are related by the following  functional dependencies:
\begin{equation}
\label{spIdfd}
\ssfd{pId}{spId}
\end{equation}
\begin{equation}
\label{colourfd}
\ssfd{pId}{c}
\end{equation}
\begin{equation}
\label{spIdcfd}
\ssfd{spId}{c}
\end{equation}
and the following  inclusion dependency:
\begin{equation}
\label{spIdcolour1}
person[spId,c] \subseteq person[pId,c]
\end{equation}

The $person$ relation (\ref{personrelation}) is not in BCNF nor in 3NF because the dependency (\ref{colourfd}) is transitive but its rhs, $c$, is not a key attribute wrt the relation\commentary{This uses Codd's way of describing 3NF not Zaniolo's.}. 
In example \lref{teamselectionrevised} we show, significantly,  that a normalised version of this relational schema, one that is in both TNF and BCNF, is 
$\chi$-generated from an ER model that is this preliminary model specifically revised to achieve the fd factoring property. 




\subsection{Team Selection - Revised Model}
\llabel{teamselectionrevised}

The situation modelled in this example is exactly that described in the previous example (see example \lref{teamselectionpreliminary}). The ER model though is different in that the ER schema used to represent the data  is a revised ER schema modified specifically so that in this model the fd factoring property holds with the consequence that the $\chi$-generated relational schema is in TNF and BCNF;  in fact the $\chi$-generated relational schema in this example is a normalised version of the $\chi$-generated relational schema from that 
previous example (as described in section \lref{teamselectionpreliminaryrelationalschema}).

\subsubsection{The ER Schema}
The ER schema in this example is shown in figure \ref{teamselectionrevisedERschema}. It is the ER schema of example \lref{teamselectionpreliminary} (see figure 
\lref{teamselectionpreliminaryERschema}) with an additional entity type $sp$ to represent persons who have been selected by some other person.


\begin{figure} [h]
\begin{center}
\begin{tabular}{p{4cm}  l }
\begin{tabular}{c p{1.5cm} c}
   \Rnode{sp}{\rnode{sps}{s}\rnode{spp}{p}} & &    \\[1.4cm]
   \Rnode{p}{p}   & & \Rnode{v}{v}
\end{tabular}
%\setlength{\sarnodesepA}{-1pt}
%\setlength{\sarnodesepB}{-3pt}
\Ete[40][90]{p}{sps}
\alabel{S}
\Etm[0][-90]{sp}{p}
\alabel{I}
\idcomp
\Et[20][-35]{spp}{v}
\alabel{c}[0.4]
\Etm{p}{v} 
\blabel{pId}
\idcomp &\footnotesize
\begin{tabular}{c p{1.5cm} p{4cm}}
KEY && \\
\hline
p & person & Identified by id attribute ($pId$). \\
sp & selected & a $selected$ person is identified by their inclusion relationship $I$ to the type $person$ \\
S & selects & each $person$ selects exactly one other $selected$ person \\
I & inclusion & each $selected$ person is a $person$ \\
c & colour & each $person$ wears a vest of this colour 
\end{tabular} 
\end{tabular}
\end{center}
\caption{Team Selection Example Revised. 
In the situation modelled all persons have a coloured vest and since this colour is the same as the colour 
worn by the person they select it is this latter colour that is included in the schema and which will therefore to be held as data. 
This schema uses the constraints known of the situation to reduce the amount of data held.
Nonetheless the  vest colour  of a selected person is 
identical to the vest colour of the person they themselves select which is to say that there is a  path equivalence $I \circ S \circ c \simeq c$ in this model.
}
\label{teamselectionrevisedERschema}
\end{figure}

\begin{categoricalaside}

The relationship $S$ of the preliminary model has been epi-mono split into $S$ which is now an epimorphism in the sense that it is onto in all defining instances,
and $I$ which is a monomorphism. The attribute $c$ has been factored through $S$.
\end{categoricalaside}

\subsubsection{Analysis of Functional Dependencies}
 
We show that in this model, compared to the previous model,  there is still
 an intransitive functional dependency between attribute-like paths that cannot be factored right intransitively but that this functional dependency is one in which the left hand side 
is a mono-source; the fd factoring property still holds and,
accordingly, the $\chi$-generated relational schema from this model is in TNF and BCNF.

\commentary{This model meets the conditions of theorem \lref{maintheorem}; is simple without caveats and so the relational model 
determined by the $\chi$ mapping is in BCNF.}

Similar reasoning to that given in section \lref{teamselectionpreliminaryanalysis} can be used to establish that 
in this model we have the following sequence of functional dependencies:
\begin{equation}
I/pId \morph I/S \morph I/S/I/S \morph I/S/I/S/I/S ... \morph c
\end{equation}
and that $I/pId \morph c$ is a transitive functional dependency.  $I/pId \morph c$ is therfore a transitive functional dependency between attribute-like paths which cannot
be factored right-intransitively. However  the left hand side, $I/pId$, is a mono-source and so the condition for the model to have the fd factoring property is not violated.

Note that $\msfd{I/S/I/pId}{c}$ is a functional dependency between  paths that cannot be factorised on the right though an intransitive functional dependency. 
 It is transitive because it factors as $\msfd{I/S/I/pId}{I/S/I/S}$ and $\msfd{I/S/I/S}{c}$ and neither of these are reversible. The path lhs ($I/S/I/pId$) isn't a mono-source 
however neither is it  because $S$ is not identifying and so the condition for the model to have the fd factoring proeprty is not violated.

\subsubsection{The $\chi$-generated Relational Schema}
The $\chi$ mapping applied to this model produces relations (\ref{person1relation}) and (\ref{person2relation}):
\begin{equation}
\label{person1relation}
selected(\underline{spId},  c)
\end{equation}
\begin{equation}
\label{person2relation}
person(\underline{pId}, spId)
\end{equation}
satisfying the inclusion dependencies
\begin{equation}
selected[spId] \subseteq person[pId]
\end{equation}
and
\begin{equation}
\label{spIdselectedcolour}
selected[spId,c] \subseteq (selected \bowtie_{spId=pId} 
                                       (person \bowtie selected [pId,c]) )
																			[spId,c];
\end{equation}
This relational schema is in TNF and BCNF however, to be fair,
it is difficult not to question the efficacy of this normalised schema compared to the unnormalised schema (section \ref{teamselectionpreliminaryrelationalschema}).
In addition,
the inclusion dependency (\ref{spIdselectedcolour}) is a database constraint that
is more difficult to police in an implementation
than the prior equivalent (\ref{spIdcolour1}) in the unnormalised schema and
this too gives advantages to the unnormalised schema. 



\subsection{Team Selection - Variation One}
\llabel{teamselectionvariationone}
This is a variation of the preliminary model in which each person may only be selected by a single other individual.
A person may select themselves in which case the person will be in a team of their own. 

In this example the schema is as shown in the earlier figure \lref{teamselectionpreliminaryschema}. 
In this model the fd factoring property holds because functional dependency $\msfd{S}{c}$ is now intransitive because though
$S \morph S/S \morph c$ we also have $S/S \morph S$.

\begin{figure} [h]
\begin{center}
\begin{tabular}{p{3.5cm} c}
\begin{tabular}{c p{1.5cm} c}
   \Rnode{p}{p} & & \Rnode{v}{v}
\end{tabular}
%\nccircle[nodesep=3pt]{<-}{p}{.4cm}
\rEtm[270]{p}
\alabel{S}[0.6]
\Et[-40]{p}{v}
\blabel{c}[0.6]
\Etm{p}{v} 
\alabel{pId}[0.6]
\idcomp
& \footnotesize
\begin{tabular}{c p{1.5cm} p{4cm}}
KEY && \\
\hline
p & person & Identified by id attribute ($pId$). \\
S & selects & each person selects exactly one other person - each person is selected at most once \\
c & colour & each person is given a coloured vest 
\end{tabular} 
\end{tabular}
\end{center}
\caption{Team Selection Example  - Variant one. This example is different to the example \ref{teamselectionpreliminary} in that the relationship $S$ in this example is a mono-source. 
}
\label{teamselectionvariantoneERschema}
\end{figure}

\subsubsection{The $\chi$-generated Relational Model}
As before in example \lref{teamselectionpreliminary}, from this model the $\chi$ transform generates  a relational schema  consisting of a single
$person$ relation described in (\lref{personrelation}) and for which, as earlier, there are functional dependencies (\ref{spIdfd}),
(\ref{colourfd}) and (\ref{spIdcfd}) but now in addition
there is a functional dependency 
\begin{equation}
S \morph pId
\end{equation}

and now $S/pId$ is a candidate key and thus now, unlike in example \lref{teamselectionpreliminary}, and as predicted by 
theorem \lref{goaltheorem}, this schema is in BCNF and therefore in 3NF. 

\subsection{Team Selection - Variation Two}
\llabel{teamselectionvariationtwo}
\commentary{This variation of earlier models is an example in which the fd factoring property holds but the model is not \textit{reduced}. The $\chi$-generated relational model is neither in BCNF nor in 3NF. }
This is a second variation of the preliminary model - the situation is as for variation one in that  each person may only be selected by a single other individual and a person may select themselves in which case the person will be in a team of their own but persons may elect not to participate in which case
in defining instances $E$, the relationship $s$ and the attribute $c$ are both undefined.

\begin{figure} [h]
\begin{center}
\begin{tabular}{p{3.5cm} c}
\begin{tabular}{c p{1.5cm} c}
   \Rnode{p}{p} & & \Rnode{v}{v}
\end{tabular}
%\nccircle[nodesep=3pt]{<-}{p}{.4cm}
\rEpm[270]{p}
\alabel{S}[0.6]
\Ep[-40]{p}{v}
\blabel{c}[0.6]
\Etm{p}{v} 
\alabel{pId}[0.6]
\idcomp
& \footnotesize
\begin{tabular}{c p{1.5cm} p{4cm}}
KEY && \\
\hline
p & person & Identified by id attribute ($pId$). \\
s & selects & each person selects exactly one other person \\
c & colour & each person is given a coloured vest 
\end{tabular} 
\end{tabular}
\end{center}
\caption{Team Selection Example  - Variant Two. The example is similar to the previous one (example \ref{teamselectionvariantone})
but is different in that the relationship $S$ is not total.
}
\label{teamselectionvarianttwoERschema}
\end{figure}

Assume that\footnote{If we assume otherwise that a person may select a person who has elected out then we have $S \circ c \leq c$ rather than $S \circ c \simeq c$.
and we do not have a situation where $S \morph S/S$ or $S/S \morph S/S/S$ or so on.  The fd factoring property holds and in the relational design we do not
have a functional dependency $\ssfd{spId}{c}$. The relational design is in BCNF.}
the situation is that a person may not select a person who has elected out. In this case there is a  path equivalence $c \simeq S \circ c$.
The fd factoring property holds because as in the previous example though we have $S \morph S/S \morph S/S/S ... \morph c$ 
but unlike the previous example (????) we also have $... S/S/S \morph S/S \morph S$. \commentary{Proof?}

\subsubsection{$\chi$-generated Relational Design}
As in example \ref{teamselectionpreliminaryexample} the $\chi$-generated relational design consists of a single relation:
\begin{equation}
\label{personrelation}
person(\underline{pId}, spId, c)
\end{equation}
and in addition to those implied by the fact that $pId$ is key, the  attributes of this relation are related by the following  functional dependencies:
\begin{equation}
\ssfd{spId}{c}
\end{equation}
and the following  inclusion dependency:
\begin{equation}
\label{spIdcolour1}
person[spId,c] \subseteq person[pId,c]
\end{equation}


The $person$ relation (\ref{personrelation}) is not in BCNF nor in 3NF because the dependency (\ref{colourfd}) is transitive but its rhs, $c$, is not a key attribute wrt the relation\commentary{This uses Codd's way of describing 3NF not Zaniolo's.}. 
In example \lref{teamselectionvarianttworevised} we can show, significantly,  that a normalised version of this relational schema, one that is in both TNF and BCNF, is 
$\chi$-generated from an ER model that is this  model specifically revised to achieve property Z.




\subsection{Team Selection Variation Two - Revised Model}
\llabel{teamselectionrevised}

The situation modelled in this example is exactly that described in the previous example (see example \lref{teamselectionvariationtwo}). The ER model though is different in that the ER schema used to represent the data  is a revised ER schema modified specifically so that in this model property Z holds with the consequence that the $\chi$-generated relational schema is in TNF and BCNF;  in fact the $\chi$-generated relational schema in this example is a normalised version of the $\chi$-generated relational schema from that 
previous example (as described in section \lref{teamselectionvariationtworelationalschema}).

\subsubsection{The ER Schema}
The ER schema in this example is shown in figure \ref{teamselectionvariationtworevisedERschema}. It is the ER schema of example \lref{teamselectionpreliminary} (see figure 
\lref{teamselectionpreliminaryERschema}) with an additional entity type $sp$ to represent persons who have been selected by some other person.


\begin{figure} [h]
\begin{center}
\begin{tabular}{p{4cm}  l }
\begin{tabular}{c p{1.5cm} c}
   \Rnode{sp}{\rnode{sps}{s}\rnode{spp}{p}} & &    \\[1.4cm]
   \Rnode{p}{p}   & & \Rnode{v}{v}
\end{tabular}
%\setlength{\sarnodesepA}{-1pt}
%\setlength{\sarnodesepB}{-3pt}
\Epe[40][90]{p}{sps}
\alabel{S}
\Etm[0][-90]{sp}{p}
\alabel{I}
\idcomp
\Et[20][-35]{spp}{v}
\alabel{c}[0.4]
\Etm{p}{v} 
\blabel{pId}
\idcomp &\footnotesize
\begin{tabular}{c p{1.5cm} p{4cm}}
KEY && \\
\hline
p & person & Identified by id attribute ($pId$). \\
sp & selected & a $selected$ person is identified by their inclusion relationship $I$ to the type $person$ \\
S & selects & each $person$ selects exactly one other $selected$ person \\
I & inclusion & each $selected$ person is a $person$ \\
c & colour & each $person$ wears a vest of this colour 
\end{tabular} 
\end{tabular}
\end{center}
\caption{Team Selection Variation Two Example Revised. 
Persons have a coloured vest only if they elect to take part.
In this case this colour is the same as the colour 
worn by the person they select and it is this latter colour that is included in the schema and which will therefore to be held as data. The  vest colour  of a selected person is 
identical to the vest colour of the person they themselves select which is to say that there is a  path equivalence $I \circ S \circ c \simeq c$ in this model.
}
\label{teamselectionvariationtworevisedERschema}
\end{figure}

\begin{categoricalaside}

The relationship $S$ of the preliminary model has been epi-mono split into $S$ which is now partial and an epimorphism, in the sense that it is onto in all defining instances,
and $I$ which is a monomorphism. The attribute $c$ has been factored through $S$.
\end{categoricalaside}

property Z somewhere

\subsubsection{Analysis of Functional Dependencies}
 


Similar reasoning to that given in section \lref{teamselectionpreliminaryanalysis} can be used to establish that 
in this model we have the following sequence of functional dependencies:
\begin{equation}
I/pId \morph I/S \morph I/S/I/S \morph I/S/I/S/I/S ... \morph c
\end{equation}
but 
\begin{itemize}
\item $\ssfd{I/pId}{c}$ is a transitive functional dependency because $\ssfd{I/pId}{I/S}$ is not reversible.
\item $\ssfd{I/S}{I/S/I/S}$ is reversible \commentary{check}
\item $\ssfd{I/S}{c}$ is intransitive
\item $\ssfd{I/pId}{c}$ can be factored right intransitively
\item the fd factoring property holds of the model
\end{itemize}

\subsubsection{Is this model well-formulated}
Yes. This model is well-formulated is design reduced and meets property Z. 
\commentary{This model meets the conditions of theorem \lref{maintheorem}; and is simple without caveats and so the relational model 
determined by the $\chi$ mapping is in BCNF.}


\subsubsection{The $\chi$-generated Relational Schema}
The $\chi$ mapping applied to this model produces relations (\ref{person1relation}) and (\ref{person2relation}):
\begin{equation}
\label{person1relation}
selected(\underline{spId},  c)
\end{equation}
\begin{equation}
\label{person2relation}
person(\underline{pId}, spId)
\end{equation}
satisfying the inclusion dependencies
\begin{equation}
selected[spId] \subseteq person[pId]
\end{equation}
and
\begin{equation}
\label{spIdselectedcolour}
selected[spId,c] \subseteq (selected \bowtie_{spId=pId} 
                                       (person \bowtie selected [pId,c]) )
																			[spId,c];
\end{equation}
This relational schema is in TNF and BCNF.




\needspace{20\baselineskip}
\subsection{nested rows}
\begin{figure} [h]
\begin{center}
\begin{tabular}{c c}
\begin{tabular}{c p{0.5cm} c p{0.7cm} c}
   \Rnode{r}{r}   & &                  & &              \\[0.8cm]
                      & & \Rnode{c}{c} & & \Rnode{v}{v} \\[0.8cm]
   \Rnode{d}{d} & &                  & & 
\end{tabular}
\nccircle[angleA=80, nodesep=3pt]{->}{r}{.5cm}
\blabel{P}[0.4]
\ncarr[30]{r}{v}
\alabel{rId}[0.4]
\idcomp
\ncarr{c}{v} 
\alabel{cN}
\idcomp
\ncarr[-30]{d}{v} 
\blabel{cId}
\idcomp
\ncarr{d}{r} 
\alabel{S0}
\ncarr{d}{c} 
\blabel{R0}
\ncarr{c}{r} 
\blabel{R1}
& \footnotesize
\begin{tabular}{c p{1.5cm} p{4cm}}
KEY && \\
\hline
r & row & Identified by $rId$ attribute ($pId$) and
           nested within a parent ($P$) row. Root rows in the hierarchy
				  are represented have themselves as their own parent.\\
c & column & Identified by $cId$ attribute and applying to all rows
              nested within a root row given by the relationship $R1$. \\
d & data call & Identified by $cId$ attribute and related to a particular
                row ($S0$) and a particular column ($R0$). \\
\end{tabular} 
\end{tabular}
\end{center}
\caption{Table with Nested Rows. If the rows of a table can be indefinitely nested then the whole table can be equated with,
and therefore modelled as, the root row of the table. This is what is described here and it is an example of an ER model 
which is well formulated but which exhibits a transitive functional dependency which does not factor through an intransitive dependency.
}
\label{nestedrowsexample}
\end{figure}

Since columns are parented by root rows in the hierarchy and these are self-parenting the following diagram commutes:
$
\begin{array}{c p{0.3cm} c p{0.3cm} c}
                  & &              & &                 \\ %need bit of v space
   \Rnode{r1}{r}  & &              & &  \Rnode{r2}{r}  \\[0.8cm]
                  & & \Rnode{c}{c} & &                 
\end{array}
$
\ncarr{r1}{r2}
\alabel{P}[0.4]
\ncarr{c}{r1}
\alabel{R1}[0.4]
\ncarr{c}{r2}
\blabel{R1}[0.4]
commutes. 


There are infinitely many functional dependencies sourced at entity type $d$ as follows
\begin{equation*}
\ssfd{S0}{R0/R1}
\end{equation*}
\begin{equation*}
\ssfd{S0/P}{R0/R1}
\end{equation*}
\begin{equation*}
\ssfd{S0/P/P}{R0/R1} \mbox{ and so on.}
\end{equation*}

Each of these dependencies is an intranstive dependency because each
factors through the one that follows it in the sequence because
\begin{equation}
S0 \morph S0/P \morph s0/P/P \morph s0/P/P/P \mbox{ and so on.}
\end{equation}


The relational design determined by the entity model has relations as follows:
\begin{equation}
\label{rowrelation}
row(\underline{rId}, prId)
\end{equation}
\begin{equation}
\label{columnrelation}
column(\underline{cN}, prId)
\end{equation}
\begin{equation}
\label{rowrelation}
datacell(\underline{dId}, prId, cN)
\end{equation}

These relations are each in BCNF.









\needspace{20\baselineskip}
\subsection{nested tables}
\begin{figure} [h] % nested tables
\begin{center}

\begin{tabular}{c c}
$
\begin{array}{cp{0.5cm}cp{0.5cm}cp{0.3cm}c}
            &&              && \Rnode{v}{v} &&              \\[0.9cm]
            && \Rnode{t}{t} &&              &&              \\[0.5cm] 
\Rnode{r}{r}&&              && \Rnode{c}{c} &&              \\[0.5cm]     
	          && \Rnode{d}{d} &&              &&               
\end{array}
$
\nccircle[angleA=20, nodesep=3pt]{<-}{t}{.4cm}
\blabel{P}[0.7]
\ncarr{r}{t} 
\alabel{S_1}
\idcomp
\ncarr[-35]{t}{v} 
\alabel{tId}
\idcomp
\ncarr[-30]{c}{v} 
\blabel{cNo}
\idcomp
\ncarr{d}{c}
\blabel{R_0}
\idcomp
\ncarr{d}{r}
\alabel{S_0}
\idcomp
\ncarr{c}{t}
\blabel{R_1}
\idcomp
%\ncarr[80]{r}{v}
\nccurve[angleA=120, angleB=170, ncurv=0.9, nodesep=0.1, arrowsize=5pt,arrowinset=0.7]{->}{r}{v}
\alabel{rNo}[0.25]
\idcomp

& \footnotesize
\begin{tabular}{c p{1.5cm} p{3cm}}
KEY && \\
\hline
t & table & Having identifying attribute tNo the name of the table. \\
c & column & Identified by a combination of column number cNo and relationship $R_1$ to the table it is a column of.\\
r & row & Identified by its row number $rN$ and its relationship $S_1$ to the table it is a row of.\\
d & data cell & Identified by relationship $S_0$ to the row it is in and relationship $R_0$ to the row it is in. \\
\end{tabular} 
\end{tabular}
\end{center}
\caption{Nested Tables. Tables are nested within other tables. This is an example of an ER model which is well formulated but which exhibits a transitive functional dependency
$S_0/S_1/tId \morph R_0/R_1/tId$
between attribute-like paths which, because of the following chain of functional dependencies:
$
S_0/S_1/tId \morph S_0/S_1/P \morph S_0/S_1/P/P \morph S_0/S_1/P/P/P  ... \morph R_0/R_1/tId
$
does not factor through an intransitive dependency. This model determines a relational design that is not in 3NF. 
}
\label{datatablegraph}
\end{figure}

This example determines a relational design with the following
relations and inclusion dependencies:
\begin{equation}
table(\underline{tableId},
\underline{parentTableId})
\end{equation}
\begin{equation}
row(
\underline{containingTableId},
\underline{rowNo}
)
\end{equation}
\begin{equation}
row[containingTableId] \subseteq table[tableId]
\end{equation}
\begin{equation}
column(\underline{outermostTableId},
\underline{columnNo})
\end{equation}
\begin{equation}
column[outermostTableId] \subseteq table[tableId]
\end{equation}
\begin{equation}
datum(\underline{containingTableId},
\underline{rowNo},
\underline{outermostTableId},
\underline{columnNo},value)
\end{equation}
\begin{equation}
dataum[containingTableId,rowNo] \subseteq row[containingTableId,rowNo]
\end{equation}
\begin{equation}
dataum[outermostTableId,columNo] \subseteq column[outermostTableId,columNo]
\end{equation}
Additionally the $dataum$ relation has the following functional dependency:
\begin{equation}
\ssfd{containingTableId,rowNo}{outermostTableId}
\end{equation}
and so $datum$ is not in normal form. To normalise create replace dataum 
by two relations:
\begin{equation}
datum(\underline{containingTableId},
\underline{rowNo},
\underline{columnNo},value)
\end{equation}
and
\begin{equation}
containingTable(\underline{tableId},
\underline{outermostTableId}
\end{equation}

This amounts to the following revised ER model:

$
\begin{array}{cp{0.5cm}cp{0.5cm}cp{0.3cm}c}
            &&                && \Rnode{v}{v} &&              \\[1.1cm]
            && \Rnode{t}{t}   &&              &&              \\[0.9cm] 
						&& \Rnode{ct}{ct} &&              &&              \\[0.5cm] 
\Rnode{r}{r}&&                && \Rnode{c}{c} &&              \\[0.5cm]
	          && \Rnode{d}{d}   &&              &&               
\end{array}
$
\nccircle[angleA=20, nodesep=3pt]{<-}{t}{.4cm}
\blabel{P}[0.7]
\ncarr{ct}{t} 
\alabel{I}[0.4]
\idcomp
\ncarr[50]{ct}{t}
\alabel{O}[0.4]
\ncarr{r}{ct} 
\alabel{S_1}
\idcomp
\ncarr[-35]{t}{v} 
\alabel{tId}
\idcomp
\ncarr[-30]{c}{v} 
\blabel{cNo}
\idcomp
\ncarr{d}{c}
\blabel{R_0}
\idcomp
\ncarr{d}{r}
\alabel{S_0}
\idcomp
\ncarr{c}{t}
\blabel{R_1}
\idcomp
%\ncarr[80]{r}{v}
\nccurve[angleA=120, angleB=170, ncurv=0.9, nodesep=0.1, arrowsize=5pt,arrowinset=0.7]{->}{r}{v}
\alabel{rNo}[0.25]
\idcomp

Essentially the entity type $ct$ represents the image of $S_0/S_1$
as a sub object of entity type $t$. Now the functional dependency
$\ssfd{S_0/S_1}{R_0}/{R_1}$ is represented by the relationship $O$
and we have the equivalence $S_0/S_1/O \simeq R_0/R_1$. The $dataum$
relation determined by entity type $d$ now does not contain an attribute
corresponding to $R_0/R_1/tId$ because this path is subsumed by the path
$S_0/S_1/O/tId$ and this path is not attribute-like
as the relationship $O$ is non-identifying. \commentary{\textit{voil\'a.}}

In this revised ER model there still is a 
a transitive dependency, $S_0/S_1/I/tId \morph R_0/R_1/tId$ between attribute-like paths 
and a chain of intermediate dependencies 
\begin{equation}
S_0/S_1/I/tId \morph S_0/S_1/I/P \morph S_0/S_1/I/P/P \morph S_0/S_1/I/P/P/P  ... \morph R_0/R_1/tId
\end{equation}
and now there is a intransitive dependency $S_0/S_1/O \morph R_0/R_1/tId$ which has a place in this chain like so:
\begin{equation}
S_0/S_1/I/tId \morph S_0/S_1/I/P \morph S_0/S_1/I/P/P \morph S_0/S_1/I/P/P/P  ... \morph S_0/S_1/O \morph R_0/R_1/tId
\end{equation}

and therefore the dependency $S_0/S_1/I/tId \morph R_0/R_1/tId$ can be right transitively factored.
Note that there is still a transitive functional dependency that does not factor right transitively:
\begin{equation}
S_0/S_1/I/tId \morph S_0/S_1/I/P \morph S_0/S_1/I/P/P \morph S_0/S_1/I/P/P/P  ... \morph S_0/S_1/O
\end{equation}
 but this now isn't between attribute-like paths. The revised model meets the condition required for the determined relational model to be in normal form.




\needspace{20\baselineskip}
\subsection{simple redundancy problem shape}
\begin{newtt}
I feel that I need to first give the example without $S_0$ and therefore without circularity and to illustrate mono-epi splitting of $R_0$ as the way to normalise to a `reduced' model.
\end{newtt}

\begin{figure} [h]  % chen fragment with bars
\begin{center}
$
\begin{array}{c p{0.75cm}c p{0.5cm}c p{0.5cm}c}
 \Rnode{a}{a}  && \Rnode{c}{c}  &&              &&              \\ [0.3cm]
	 	           &&               && \Rnode{e}{e} && \Rnode{v}{v} \\ [0.3cm]     
 \Rnode{b}{b}  && \Rnode{d}{d}  &&              &&              \\ 
\end{array}
$
\ncarr[60]{a}{v}
\alabel{K_a}[0.3][0]
\idcomp
\ncarr{a}{c} 
\alabel{S_1}[0.5][0]
\ncarr{c}{e} 
\alabel{S_2}[0.3][0]
\idcomp
\ncarr{e}{v}
\alabel{K_e}[0.4][0]
\idcomp
\ncarr[-30]{a}{b}
\blabel{R_0}[0.35][0]
\ncarr[-30]{b}{a}
\blabel{S_0}[.35][0]
\ncarr{b}{d}
\blabel{R_1}[0.5][1]
\ncarr{d}{e}
\blabel{R_2}[0.3][0]
\idcomp
\ncarr[-60]{b}{v}
\blabel{K_b}[0.3][0]
\idcomp
%\nccurve[angleA=90,angleB=90,nodesep=2pt,ncurv=0.9]{->}{w}{v}
%\alabel{pcnt}[0.3][-1]
\vspace{1.5cm}
\newline
such that \hspace{0.5cm}
$
\begin{array}{c p{0.75cm}c p{0.5cm}c}
 \Rnode{a}{a}  && \Rnode{c}{c}  &&              \\ [0.3cm]
	 	           &&               && \Rnode{e}{e}  \\ [0.3cm]     
 \Rnode{b}{b}  && \Rnode{d}{d}  &&              \\ 
\end{array}
$
\ncarr{a}{c} 
\alabel{S_1}[0.5][0]
\ncarr{c}{e} 
\alabel{S_2}[0.3][0]
\idcomp
\ncarr{b}{a}
\alabel{S_0}[.5][1]
\ncarr{b}{d}
\blabel{R_1}[0.5][1]
\ncarr{d}{e}
\blabel{R_2}[0.3][0]
\idcomp
\hspace {0.25cm} and \hspace{0.5cm}
$
\begin{array}{c p{0.75cm}c p{0.5cm}c}
 \Rnode{a}{a}  && \Rnode{c}{c}  &&              \\ [0.3cm]
	 	           &&               && \Rnode{e}{e}  \\ [0.3cm]     
 \Rnode{b}{b}  && \Rnode{d}{d}  &&               \\ 
\end{array}
$
\ncarr{a}{c} 
\alabel{S_1}[0.5][0]
\ncarr{c}{e} 
\alabel{S_2}[0.3][0]
\idcomp
\ncarr{a}{b}
\blabel{R_0}[0.5][1]
\ncarr{b}{d}
\blabel{R_1}[0.5][1]
\ncarr{d}{e}
\blabel{R_2}[0.3][0]
\idcomp
\hspace{0.2cm} commute.

\end{center}
\caption{Showing that the definition of rendered redundant may lead to circularity in that
$R_1/R_2/K_e$  rendered redundant by a path $S_O/S_1/S_2/K_e$  and, \textit{vice-versa}, 
$S_1/S_2/K_e$ rendered redundant by $R_0/R_1/R_2/K_e$}
\label{redundancyproblemshape}
\end{figure}

The first-cut relational design will include tables $a$ and $b$ as follows
\begin{equation}
\label{relationa}
a(\underline{\qq{K_a}},\qq{R_0/K_b},\qq{S_1/S_2/K_e})
\end{equation}
and
\begin{equation}
\label{relationb}
b(\underline{\qq{K_b}},\qq{S_0/K_a},\qq{R_1/R_2/K_e})
\end{equation}

Because of the equivalence of paths $R_0/R_1/R_2 \simeq S_1/S_2$ (shown as the rightmost of the two
commuting diagram in figure \ref{redundancyproblemshape}) it follows that 
there is a functional dependency
\begin{equation}
\fd{R_0}{S_1/S_2}
\end{equation}
from this and because $K_b$ is a mono-source it follows that
\begin{equation}
\fd{R_0/K_b}{S_1/S_2}
\end{equation}
By an unproven lemma it follows that 
\begin{equation}
\fd{\qq{R_0/K_b}}{\qq{S_1/S_2}}
\end{equation}
Relation $a$ defined in (\ref{relationa}) is, therefore, not in 3NF. Similarly relation $b$ as given in
(\ref{relationb}) is not in 3NF because of this functional dependency:
\begin{equation}
\fd{\qq{S_0/K_a}}{\qq{R_1/R_2}}
\end{equation}

To normalise relations $a$ we need to modify relation $a$ and introduce a relation $X$ keyed by $\qq{R_0/K_b}$ as follows:
\begin{equation}
\label{normalrelationa}
a(\underline{\qq{K_a}},\qq{R_0/K_b})
\end{equation}
\begin{equation}
\label{relationX}
X(\underline{\qq{R_0/K_b}},\qq{S_1/S_2/K_e})
\end{equation}
similarly relations $b$ must be split into
\begin{equation}
\label{normalrelationb}
b(\underline{\qq{K_b}},\qq{S_0/K_a})
\end{equation}
\begin{equation}
\label{relationY}
Y(\underline{\qq{S_0/K_a}},\qq{R_1/R_2/K_e})
\end{equation}
In this design there are referential inclusion dependencies 
\begin{equation}
X[\qq{R_0/K_b}] \subseteq b[K_b]
\end{equation}
and
\begin{equation}
Y[\qq{S_0/K_a}] \subseteq a[K_a]
\end{equation}
Therefore this relational design corresponds to the following ER model:
\vspace{0.75cm}
\begin{center}
$
\begin{array}{c p{0.75cm} c p{0.75cm}c p{0.5cm}c p{0.5cm}c}
 \Rnode{a}{a}  && \Rnode{X}{X}  && \Rnode{c}{c}  &&              &&              \\ [0.35cm]
	 	           &&               &&               && \Rnode{e}{e} && \Rnode{v}{v} \\ [0.35cm]     
 \Rnode{b}{b}  && \Rnode{Y}{Y}  && \Rnode{d}{d}  &&              &&              \\ 
\end{array}
$
\ncarr[60]{a}{v}
\alabel{K_a}[0.3][0]
\idcomp
\ncarr{a}{X}
\alabel{R_0}[0.3][0]
\ncarr{X}{b}
\blabel{I_b}[0.25][0]
\idcomp
\ncarr{X}{c} 
\alabel{S_1}[0.5][0]
\ncarr{c}{e} 
\alabel{S_2}[0.3][0]
\idcomp
\ncarr{e}{v}
\alabel{K_e}[0.4][0]
\idcomp
\ncarr{Y}{d}
\blabel{R_1}[0.5][1]
\ncarr{d}{e}
\blabel{R_2}[0.3][0]
\idcomp
\ncarr{Y}{a}
\alabel{I_a}[0.25][0]
\idcomp
\ncarr{b}{Y}
\blabel{S_0}[0.3][0]
\ncarr[-60]{b}{v}
\blabel{K_b}[0.3][0]
\idcomp
\end{center}
%\nccurve[angleA=90,angleB=90,nodesep=2pt,ncurv=0.9]{->}{w}{v}
%\alabel{pcnt}[0.3][-1]
\vspace{1.5cm}

However we can proceed differently and decide to breal the symetry, omit $R_1/R_2/K_e$  from the relational 
model because it is rendered redundant by a path $S_O/S_1/S_2/K_e$. Then the first-cut relational design will have tables $a$ and $b$ as follows
\begin{equation}
\label{relationa}
a(\underline{\qq{K_a}},\qq{R_0/K_b},\qq{S_1/S_2/K_e})
\end{equation}
and
\begin{equation}
\label{relationb}
b(\underline{\qq{K_b}},\qq{S_0/K_a})
\end{equation}

and normalising we will get:

\begin{equation}
\label{normalrelationa}
a(\underline{\qq{K_a}},\qq{R_0/K_b})
\end{equation}
\begin{equation}
\label{relationX}
X(\underline{\qq{R_0/K_b}},\qq{S_1/S_2/K_e})
\end{equation}
\begin{equation}
\label{normalrelationb}
b(\underline{\qq{K_b}},\qq{S_0/K_a})
\end{equation}
which corresponds to the following ER model:

\vspace{0.75cm}
\begin{center}

$
\begin{array}{c p{0.75cm} c p{0.75cm}c p{0.5cm}c p{0.5cm}c}
 \Rnode{a}{a}  && \Rnode{X}{X}  && \Rnode{c}{c}  &&              &&              \\ [0.35cm]
	 	           &&               &&               && \Rnode{e}{e} && \Rnode{v}{v} \\ [0.35cm]     
 \Rnode{b}{b}  &&               && \Rnode{d}{d}  &&              &&              \\ 
\end{array}
$
\ncarr[60]{a}{v}
\alabel{K_a}[0.3][0]
\idcomp
\ncarr{a}{X}
\alabel{R_0}[0.3][0]
\ncarr{X}{b}
\blabel{I_b}[0.25][0]
\idcomp
\ncarr{X}{c} 
\alabel{S_1}[0.5][0]
\ncarr{c}{e} 
\alabel{S_2}[0.3][0]
\idcomp
\ncarr{e}{v}
\alabel{K_e}[0.4][0]
\idcomp
\ncdarr{b}{d}
\blabel{R_1}[0.5][1]
\ncarr{d}{e}
\blabel{R_2}[0.3][0]
\idcomp
\idcomp
\ncarr{b}{a}
\alabel{S_0}[0.3][0]
\ncarr[-60]{b}{v}
\blabel{K_b}[0.3][0]
\idcomp
\end{center}
\vspace{1.0cm}
relational ER model is then:
\vspace{0.9cm}
\begin{center}
\begin{tabular} {c p{5cm}}
$
\begin{array}{c p{0.75cm} c p{0.75cm}c p{0.5cm}c p{1.5cm}c}
 \Rnode{a}{a}  &&               &&               &&              &&              \\ [0.2cm]
               && \Rnode{X}{X}  &&               &&              &&              \\ [0.2cm]
							 &&               && \Rnode{c}{c}  &&              &&              \\ [0.35cm]
	 	           &&               &&               && \Rnode{e}{e} && \Rnode{v}{v} \\ [0.35cm]     
 \Rnode{b}{b}  &&               && \Rnode{d}{d}  &&              &&              \\ 
\end{array}
$
\nccurve[angleA=90,angleB=90,nodesep=2pt,ncurv=0.7]{->}{a}{v}
\alabel{K_a}[0.3][0]
\idcomp
\ncarr{a}{X}
\alabel{R_0}[0.3][0]
\ncarr{X}{b}
\blabel{I_b}[0.25][0]
\ncarr{X}{c} 
\alabel{S_1}[0.5][0]
\ncarr{c}{e} 
\alabel{S_2}[0.45][0]
\ncarr{e}{v}
\alabel{K_e}[0.4][0]
\idcomp
\ncdarr{b}{d}
\blabel{R_1}[0.5][1]
\ncarr{d}{e}
\blabel{R_2}[0.45][0]
\ncarr{b}{a}
\alabel{S_0}[0.3][0]
\ncarr[-60]{b}{v}
\blabel{K_b}[0.3][0]
\idcomp
\ncarr[60]{X}{v}
\alabel{\qq{I_b/K_b}}[0.3][0]
\idcomp
\ncarr[30]{X}{v}
\alabel{\qq{S1/S2/K_e}}[0.3][0]
\ncarr[55]{a}{v}
\alabel{\qq{R_0/I_b/K_b}}[0.25][0]
\ncarr[-35]{b}{v}
\blabel{\qq{S_0/K_a}}[0.3][0]
\ncarr[25]{c}{v}
\alabel{\qq{S_2/K_e}}[0.2][0]
\idcomp
\ncarr[-25]{d}{v}
\blabel{\qq{R_2/K_e}}[0.2][0]
\idcomp & \raisebox{0cm}{\parbox{4.5cm}{The attribute $\qq{S_1/S_2/K_e}$ 
of entity type $X$ is not identifying and has an equivalent path $I_b/R_1/R_2/K_e$. Therefore the model is not 'simple' because the path $R_1/R_2/K_e$ is not equivalent to a primary key attribute. The model does however meet the condition for being 'reduced' since the relationship $I_b$ is a mono-source. }}
\end{tabular}  
\end{center}
\vspace{1.5cm}



\needspace{40\baselineskip}
\subsection{employee intake}
\begin{figure} [h]  % chen fragment with bars
\begin{center}
$
\begin{array}{c p{0.75cm}c p{0.5cm}c p{0.5cm}c}
 \Rnode{project}{project}  && \Rnode{dept}{dept}  &&              &&              \\ [0.3cm]
	 	           &&               && \Rnode{site}{site} && \Rnode{v}{v} \\ [0.3cm]     
 \Rnode{employee}{employee}  && \Rnode{intake}{intake}  &&              &&              \\ 
\end{array}
$
\ncarr[60]{project}{v}
\alabel{pcode}[0.3][0]
\idcomp
\ncarr{project}{dept} 
\alabel{S_1}[0.5][0]
\ncarr[30]{dept}{v}
\alabel{dName}[0.3][0]
\idcomp
\ncarr{dept}{site} 
\alabel{S_2}[0.3][0]
\idcomp
\ncarr{site}{v}
\alabel{sName}[0.4][0]
\idcomp
\ncarr[-30]{project}{employee}
\blabel{R_0}[0.35][0]
\ncarr[-30]{employee}{project}
\blabel{S_0}[.35][0]
\ncarr{employee}{intake}
\blabel{R_1}[0.5][1]
\ncarr{intake}{site}
\blabel{R_2}[0.3][0]
\idcomp
\ncarr[-30]{intake}{v}
\blabel{dName}[0.3][0]
\idcomp
\ncarr[-60]{employee}{v}
\blabel{empNo}[0.3][0]
\idcomp
%\nccurve[angleA=90,angleB=90,nodesep=2pt,ncurv=0.9]{->}{w}{v}
%\alabel{pcnt}[0.3][-1]
\vspace{1.5cm}
\newline
such that \hspace{0.5cm}
$
\begin{array}{c p{0.75cm}c p{0.5cm}c}
 \Rnode{project}{project}  && \Rnode{dept}{dept}  &&              \\ [0.3cm]
	 	           &&               && \Rnode{site}{site}  \\ [0.3cm]     
 \Rnode{employee}{employee}  && \Rnode{intake}{intake}  &&              \\ 
\end{array}
$
\ncarr{project}{dept} 
\alabel{S_1}[0.5][0]
\ncarr{dept}{site} 
\alabel{S_2}[0.3][0]
\idcomp
\ncarr{employee}{project}
\alabel{S_0}[.5][1]
\ncarr{employee}{intake}
\blabel{R_1}[0.5][1]
\ncarr{intake}{site}
\blabel{R_2}[0.3][0]
\idcomp
\hspace {0.25cm} and \hspace{0.5cm}
$
\begin{array}{c p{0.75cm}c p{0.5cm}c}
 \Rnode{project}{project}  && \Rnode{dept}{dept}  &&              \\ [0.3cm]
	 	           &&               && \Rnode{site}{site}  \\ [0.3cm]     
 \Rnode{employee}{employee}  && \Rnode{intake}{intake}  &&               \\ 
\end{array}
$
\ncarr{project}{dept} 
\alabel{S_1}[0.5][0]
\ncarr{dept}{site} 
\alabel{S_2}[0.3][0]
\idcomp
\ncarr{project}{employee}
\blabel{R_0}[0.5][1]
\ncarr{employee}{intake}
\blabel{R_1}[0.5][1]
\ncarr{intake}{site}
\blabel{R_2}[0.3][0]
\idcomp
\hspace{0.2cm} commute.

\end{center}
\caption{An example  of how \textit{rendered redundant} may lead to circularity in that
$R_1/R_2/sName$  rendered redundant by a path $S_O/S_1/S_2/sName$  and, \textit{vice-versa}, 
$S_1/S_2/sName$ rendered redundant by $R_0/R_1/R_2/sName$}
\label{employeeintake}
\end{figure}

\begin{newtt}
Added 14 Jan 2024. 
\begin{itemize}
	\item Perhaps rename intake to briefing withe the meaning of `site briefing'.
	\item Perhaps S0 is that en employee works on a project.
	\item Perhaps R0 is that an employee is the the project sponsor. 
	\item Each employee works on at most one project.
	\item Projects are sponsored by a single employee. One employee may sponsor multiple projects.  
	\item There again might make more sense just to rename site as company and to keep intake.
	\item BTW the attribute dName of intake should be renamed.
	\item the question arises what are the additional types X and Y that are introduced as part of the 'normalisation'???
\end{itemize}
\end{newtt}

\subsection{property X}
\subsection{Property X - a relationship which is not a boolean equivalent}
\begin{figure} [h]  % 
\begin{center}
\begin{tabular}{c c}
$
\begin{array}{cp{0.05cm}c  p{0.05cm}c p{0.5cm}c}
                & & \Rnode{a}{a} & &              & &             \\ [0.3cm]
								& &              & &              & & \Rnode{v}{v} \\ [0.6cm]     
	 \Rnode{b}{b} & &              & & \Rnode{c}{c} & &             
\end{array}
$
\ncarr{b}{a} 
\alabel{S}
\idcomp
\ncarr{a}{v} 
\alabel{K_a}
\idcomp
\ncarr{c}{v} 
\blabel{q}
\idcomp
\ncarr{c}{a} 
\blabel{R_1}
\idcomp
\ncline[linestyle=dashed,nodesepA=\arrnodesepA,nodesepB=\arrnodesepB]{->}{b}{c} 
\blabel{R_0}
\ncarr[-90]{b}{v} 
\blabel{k_b}
\idcomp
& \footnotesize
\end{tabular}
\end{center}
\caption{Even supposing  $\tuple{R_0,R_1} < \tuple{S}$ then the relationship $R_0$ is not a boolean equivalent 
since  the path $R_0/q$ is not dominated. This model is free of boolean equivalents.}
\label{propertyXexample}
\end{figure}

\needspace{20\baselineskip}
\subsection{student advisor}
\begin{figure} [h]  % student advisor
\begin{center}
\begin{tabular}{c c}
$
\begin{array}{cp{0.05cm}c  p{0.05cm}c p{0.5cm}c}
                & & \Rnode{d}{d} & &              & &             \\ [0.3cm]
								& &              & &              & & \Rnode{v}{v} \\ [0.6cm]     
	 \Rnode{s}{s} & &              & & \Rnode{p}{p} & &             
\end{array}
$
\ncarr{s}{d} 
\alabel{S_1}
\idcomp
\ncarr{d}{v} 
\alabel{Id}
\idcomp
\ncarr{p}{v} 
\blabel{PNo}
\idcomp
\ncarr{p}{d} 
\blabel{R_1}
\idcomp
\ncline[linestyle=dashed,nodesepA=\arrnodesepA,nodesepB=\arrnodesepB]{->}{s}{p} 
\blabel{R_0}
\ncarr[-90]{s}{v} 
\blabel{sNo}
\idcomp
& \footnotesize
\begin{tabular}{c p{1.5cm} p{4cm}}
KEY && \\
\hline
d & department & Having identifying attribute Id the department identifier. \\
s & student & Identified by relationship $S_0$ to department enrolled in and  attribute student number SNo. \\
p & professor & Identified by their relationship $R_1$ to a department and their professor number attribute pNo yyy \\
R0 & advised by & Represents the relationship\footnote{How significant is it that this may be optional?Need draw this out.} between a student and a professor.\\
\end{tabular} 
\end{tabular}
\end{center}
\caption{Following the pattern of the property X Example. This example is based on one given Shlaer-Long. Here we specify that a student optionally has an advisor (suppose that the advisor is selected part way through a course). As before we assume that an advisor must be a professor of the department in which the student is enrolled. }
\label{studentadvisorgraph}
\end{figure}

\needspace{20\baselineskip}
\subsection{property X variations}

\begin{figure} [h]  % 
\begin{center}
\begin{tabular}{c c}
$
\begin{array}{cp{0.7cm}c  p{0.7cm}c }
                & & \Rnode{b1}{b_1} & &                \\ [1.2cm]    
	 \Rnode{a}{a} & & \Rnode{c}{c}    & &    \Rnode{v}{v}\\ [1.2cm]  
					      & & \Rnode{b2}{b_2} & &                 
\end{array}
$
\ncarr{a}{b1} 
\alabel{S_1}
\ncarr{b1}{v} 
\alabel{K_{b_1}}
\idcomp
\ncarr{c}{b1} 
\blabel{Q_1}
\idcomp
\ncarr{a}{b2} 
\blabel{S_2}
\ncarr{b2}{v} 
\blabel{K_{b_2}}
\idcomp
\ncarr{c}{b2} 
\alabel{Q_2}
\idcomp
\ncline[linestyle=dashed,nodesepA=\arrnodesepA,nodesepB=\arrnodesepB]{->}{a}{c} 
\blabel{R}
\nccurve[angleA=-90,angleB=-90,nodesep=2pt,ncurv=1.6]{->}{a}{v}
\blabel{K_a}[0.3][-1]
\idcomp
& \footnotesize
\end{tabular}
\end{center}
\caption{A schema for an ER model for which we suppose  $\tuple{R,Q_1} < \tuple{S_1}$ and $\tuple{R,Q_2} < \tuple{S_2}$ and for which we ask the question does property X hold?}
\label{propertyXfailureexample}
\end{figure} 
Consider a model $\gmodel$ with the schema shown in figure \ref{propertyXfailureexample}. It may or may not be the case 
that there is a referential  inclusion dependency $\incd{a}{S_1,S_2}{c}{Q_1,Q_2}$ in $\gmodel$ and if there is such an inclusion dependency it may or may not be the case that relationship $R$ represents this inclusion dependency.  This gives us three possibilities to consider
and we discuss these as the three variations below.
\subsection {Variation One}
Consider a model $\gmodel$ with the schema shown in figure \ref{propertyXfailureexample} and suppose that $\incd{a}{S_1,S_2}{c}{Q_1,Q_2}$ in $\gmodel$ and that this referential inclusion dependency is represented by relationship $R$. 
In this case the model $\gmodel$  has property X because even though each $R \circ Q_i$ is dominated there is a referential inclusion dependency that it represents. The first-cut relational model gnerated from this schema will  not be in third normal form but
the $\chi$ transform does produce a realtional schema in normal form as it omits attributes $\qq{R/Q_1/K_{b_1}}$ and $\qq{R/Q_2/K_{b_2}}$ that are included in the first-cut relational schema  definition of the relation representing entity type $a$. 
They are excluded because each path $R/Q_1/K_{b_1}$ and $R/Q_2/K_{b_2}$  is dominated in $\gmodel$.
\subsection{Variation  Two}
Consider a model $\gmodel$ with the schema shown in figure \ref{propertyXfailureexample} and suppose that $\incd{a}{S_1,S_2}{c}{Q_1,Q_2}$ in $\gmodel$ and that this referential inclusion dependency is not represented by relationship $R$.
In this case the model $\gmodel$ is neither well-formed nor has property X. 
For it to  be well-formed there ought be a relationship, $S$, say,
$S:a \morph c$ that represents referential inclusion dependency $\incd{a}{S_1,S_2}{c}{Q_1,Q_2}$. Consider then a model
which includes in its schema  this additional relationship $S$ \commentary{For completeness should define $E_S$, for each instance $E$}representing the referential inclusion dependency $\incd{a}{S_1,S_2}{c}{Q_1,Q_2}$.\\

\begin{tabular}{ p{8.5cm}  c}
We then have that $R \leq S$, for in each instance $E$ we have&\parbox{5cm}{ \begin{align*}
E_R&=E_R \circ E_{\tuple{Q_1,Q_2}} \circ E^{-1}_{\tuple{Q_1,Q_2}} \\
   & \leq E_{\tuple{S_1,S_2}} \circ E^{-1}_{\tuple{Q_1,Q2}} \\
	 & = E_{S}
\end{align*}}
\end{tabular}

Property X still does not hold of  model as extended with relationship $S$. The  preferred way to achieve property X in  case like this is to replace relationship $R: a \morph b$ by a relationship $barR: a \morph a$ and define in each defining instance $E$,
\begin{equation}
E_{barR} = \overline{E_R}
\end{equation}

The modified model with schema as shown in figure \ref{propertyXfailurecorrection}  now does have property X and it transforms via first-cut transform to a relational schema that is in third normal form.

\begin{figure} [h]  % 
\begin{center}
\begin{tabular}{c c}
$
\begin{array}{cp{0.7cm}c  p{0.7cm}c }
                & & \Rnode{b1}{b_1} & &                \\ [1.2cm]    
	 \Rnode{a}{a} & & \Rnode{c}{c}    & &    \Rnode{v}{v}\\ [1.2cm]  
					      & & \Rnode{b2}{b_2} & &                 
\end{array}
$
\nccircle[linestyle=dashed,angleA=90, nodesep=3pt]{<-}{a}{.4cm}
\blabel{barR}[0.5]
\ncarr{a}{b1} 
\alabel{S_1}
\ncarr{b1}{v} 
\alabel{K_{b_1}}
\idcomp
\ncarr{c}{b1} 
\blabel{Q_1}
\idcomp
\ncarr{a}{b2} 
\blabel{S_2}
\ncarr{b2}{v} 
\blabel{K_{b_2}}
\idcomp
\ncarr{c}{b2} 
\alabel{Q_2}
\idcomp
\ncline[linestyle=dashed,nodesepA=\arrnodesepA,nodesepB=\arrnodesepB]{->}{a}{c} 
\blabel{\hat{R}}
\nccurve[angleA=-90,angleB=-90,nodesep=2pt,ncurv=1.6]{->}{a}{v}
\blabel{K_a}[0.3][-1]
\idcomp
& \footnotesize
\end{tabular}
\end{center}
\captionsetup{singlelinecheck=off}
 \caption[.]{Preferred --- An equivalent model which is well-formed. Relationship $R:a \morph c$ is navigated
as $barR \circ \hat{R}$. This schema has property X  since relationship $\hat{R}$ is implemented by $S_1,S_2$. The  relational schema that represents this model will include a table $a$ as follows:
\begin{equation}
a(\underline{K_a},\qq{S_1/K_{b_1}},\qq{S_2/K_{b_2}}, \qq{barR/K_a})
\end{equation} }
\label{propertyXfailurecorrection}
\end{figure}

Note that the first-cut relational schema for original model $\gmodel$ will include a table $a$ which will not be in S3NF for it
will be as follows:
\begin{equation}
a(\underline{K_a},\qq{S_1/K_{b_1}},\qq{S_2/K_{b_2}}, \qq{R/Q_1/K_{b_1}}, \qq{R/Q_2/K_{b_2}})
\end{equation}
and will have the following restrictions
\begin{align}
\qq{R/Q_1/K_{b_1}} = \overline{\qq{R/Q_2/K_{b_2}}} \circ \qq{S_1/K_{b_1}} \\
\qq{R/Q_2/K_{b_2}} = \overline{\qq{R/Q_1/K_{b_1}}} \circ \qq{S_2/K_{b_2}}
\end{align}
and therefore the following functional dependencies:
\begin{align} 
\fd{\qq{R/Q_1/K_{b_1}},\qq{S_2/K_{b_2}}}{\qq{R/Q_2/K_{b_2}}} \\
\fd{\qq{R/Q_2/K_{b_2}},\qq{S_1/K_{b_1}}}{\qq{R/Q_1/K_{b_1}}}
\end{align}
and has table $a$ therefore that is not in S3NF.
If the first cut model is normalised by addressing the first of these functional dependencies then we 
introduce a table $a'$ and move $\qq{R/Q_2/K_{b_2}}$ to table $a'$ so that table $a$ is replaced by:
\begin{align}
a(\underline{K_a},\qq{S_1/K_{b_1}},\qq{S_2/K_{b_2}}, \qq{R/Q_1/K_{b_1}}) \\
a'(\underline{\qq{R/Q_1/K_{b_1}}},\underline{\qq{S_2/K_{b_2}}}, \qq{R/Q_2/K_{b_2}} )
\end{align}
but wierdly now, on table $a'$,  $\qq{S_2/K_{b_2}} \simeq \qq{R/Q_2/K_{b_2}}$ and so table $a'$ becomes
\begin{align}
a'(\underline{\qq{R/Q_1/K_{b_1}}},\underline{\qq{S_2/K_{b_2}}} )
\end{align}
abstracting to a logical model we get a model with shcema as shown in figure \ref{propertyXfailurenormalisedandabstracted}.

\begin{figure} [H]  % 
\begin{center}
\begin{tabular}{c c}
$
\begin{array}{cp{0.7cm}c p{0.9cm}c p{0.7cm}c}
                & & \Rnode{b1}{b_1} & &                & &                 \\ [1.2cm]    
	 \Rnode{a}{a} & & \Rnode{ap}{a'} & &  \Rnode{c}{c}  & &    \Rnode{v}{v} \\ [1.2cm]  
					      & & \Rnode{b2}{b_2} & &                & &             
\end{array}
$
\ncarr[30]{a}{b1} 
\alabel{\qq{R/Q_1}}
\ncarr{a}{b1} 
\blabel{S_1}
\ncarr[30]{b1}{v} 
\alabel{K_{b_1}}
\idcomp
\ncarr{c}{b1} 
\blabel{Q_1}[0.35]
\idcomp
\ncarr{a}{b2} 
\blabel{S_2}
\ncarr[-30]{b2}{v} 
\blabel{K_{b_2}}
\idcomp
\ncarr{c}{b2} 
\alabel{Q_2}[0.35]
\idcomp
\ncline[linestyle=dashed,nodesepA=\arrnodesepA,nodesepB=\arrnodesepB]{->>}{a}{ap} 
\blabel{R_e}
\ncarr{ap}{c}
\blabel{R_t}
\nccurve[angleA=-90,angleB=-90,nodesep=2pt,ncurv=1.1]{->}{a}{v}
\blabel{K_a}[0.3][-1]
\idcomp
\ncarr{ap}{b1}
\blabel{S'_1}[0.35]
\idcomp
\ncarr{ap}{b2}
\idcomp
\alabel{S'_2}[0.35]
& \footnotesize
\end{tabular}
\end{center}
\caption{Logical version after normalisation. To be well-formed there ought to be added a further relationship $\hat{R} : a \morph c$.
Relationship $R_e$ is represented by the inclusion dependency $\incd{a}{\qq{R/Q_1},S_2}{a'}{S'_1,S'_2}$.
Relationship $R_t$ must be a monomorphim. Equally the original relationship $R$ represents the inclusion dependency 
$\incd{a}{\qq{R/Q_1},S_2}{c}{Q_1,Q_2}$ and the table $a'$ isn't achieving anything. 
}
\label{propertyXfailurenormalisedandabstracted}
\end{figure}
\vspace{0.5cm}


\subsection{Variaton Three}
Now, in regard to the schema shown in figure \ref{propertyXfailureexample} such that $\tuple{R,Q_1} < \tuple{S_1}$ and $\tuple{R,Q_2} < \tuple{S_2}$, suppose that $a\tuple{S_1,S_2} \nsubseteq c\tuple{Q_1,Q_2}$ . Again property X fails to hold and we are led to add a relationship 
$barR$ to this model leading to the schema shown in figure \ref{propertyXfailurecorrection2}.
\begin{figure} [H]  % 
\begin{center}
\begin{tabular}{c c}
$
\begin{array}{cp{0.7cm}c  p{0.7cm}c }
                & & \Rnode{b1}{b_1} & &                \\ [1.2cm]    
	 \Rnode{a}{a} & & \Rnode{c}{c}    & &    \Rnode{v}{v}\\ [1.2cm]  
					      & & \Rnode{b2}{b_2} & &                 
\end{array}
$
\nccircle[linestyle=dashed,angleA=90, nodesep=3pt]{<-}{a}{.4cm}
\blabel{barR}[0.5]
\ncarr{a}{b1} 
\alabel{S_1}
\ncarr{b1}{v} 
\alabel{K_{b_1}}
\idcomp
\ncarr{c}{b1} 
\blabel{Q_1}
\idcomp
\ncarr{a}{b2} 
\blabel{S_2}
\ncarr{b2}{v} 
\blabel{K_{b_2}}
\idcomp
\ncarr{c}{b2} 
\alabel{Q_2}
\idcomp
\ncline[linestyle=dashed,nodesepA=\arrnodesepA,nodesepB=\arrnodesepB]{->}{a}{c} 
\blabel{R}
\nccurve[angleA=-90,angleB=-90,nodesep=2pt,ncurv=1.6]{->}{a}{v}
\blabel{K_a}[0.3][-1]
\idcomp
& \footnotesize
\end{tabular}
\end{center}
\captionsetup{singlelinecheck=off}
 \caption[.]{Preferred -- An equivalent model in the case that $a\tuple{S_1,S_2} \nsubseteq c\tuple{Q_1,Q_2}$. 
Now $R \circ Q_1 \simeq barR \circ S_1$ and $R \circ Q_2 \simeq barR \circ S_2$ and $R$ represents the inclusion dependency 
$\incd{a}{barR \circ S_1,barR \circ S_2}{c}{Q_1,Q_2}$. }
\label{propertyXfailurecorrection2}
\end{figure}
 I recognise, by the way, that $barR$ is nothing but a representation of a boolean attribute and that boolean attributes could be 
represented instead as absolute-valued attributes (ones which are optional i.e may be undefined for some entitites). 

\end{document}

\subsection{Goodness Criteria}
\subsubsection{The Principle of No Redundancy}
\subsubsection{The Requirement}
\subsubsection{The Principle of Maximal Constrainedness}
Maximum constrainedness, as mentioned above, is a property of the category $\catc$ generated by the presentation  rather than of the presentation itself and is defined  relative to a requirement $\reqtc$ by which we mean a set of 
instances where each instance is a functor $D$, $D: \catc \morph \Fin$. In what follows, therefore,  by a requirement $\reqtc$ for category $\catc$ we mean a set  $\reqtc \subseteq | \Fin^{\catc} |$. 

Consider, a theory usually has some slack by which we mean that it has structurally compliant instances that are not part of its requirement.  The definition of maximal constrainedness expresses that a theory is maximally constrained to its requirement if there is no way of extending the theory so as to rule out possible structurally compliant instances that are not part of the requirement (i.e. to rule out slack) whilst remaining consistent with the requirement.

The definition now follows, preceded by an auxiliary definition.
\begin{definition}
If $\catc$ is a category and $\reqtc$ is a requirement for $\catc$,  if $I: \catc \morph \catcp$ is a functor then say that $I$ is \term{consistent with} requirement $\reqtc$ iff for all instances $D \in \reqtc$ there exists a functor $D':\catcp \morph \Fin$ such that $I \circ D'=D$.
\end{definition}
\begin{definition}
If $\catc$ is a category and $\reqtc$ is a requirement for $\catc$ then $\catc$ is \term{maximally constrained} to the requirement $\reqtc$ iff for all categories $\catcp$ and for all functors $I:\catc \morph \catcp$ that are consistent with $\reqtc$, for all functors $F: \catc \morph \Fin$  there exists an $F' : \catcp \morph \Fin$ such that $I \circ F'=F$.
\end{definition}



\subsubsection{The Principle of the Representation of Functional Dependencys}
\newcommand {\fgsourcediag}{
$
\begin{array}{c p{0.5cm} c  }
             &&   \Rnode{b}{b} \\[0.01cm]
\Rnode{a}{a} &&                \\[0.01cm] 
             &&   \Rnode{c}{c}         
\end{array} 
\begin{arrows}
\ncarr{a}{b}
\alabel{f}
\ncarr{a}{c}
\blabel{g}
\end{arrows}
$  
}
\begin{definition}
If $\catc$ is a category and $\reqtc$ is a set of instances and if \fgsourcediag
in $\catc$ then there is a  \term{functional dependency} of $g$ on $f$ with respect to $\reqtc$ iff
there is a family of functions $H_D)_{D \in \reqtc}$ such that 
in each instance $D$, $H_D$ is a unique function $H_D: D(b) \morph D(c)$, such that $D(f) \circ H_D = D(g)$. 
If there is such a functional dependency then we say that $\fundep{H}{f}{g}$ in $\catc$ with respect to $\reqtc$.
\end{definition}

Our use of the $\morph$ notation for functional dependencies here is coming from relational database theory where it is usual to represent such a functional dependency as we have here by asserting that 
$$
f \morph g
$$
Note that this use of an $\morph$ notation is independent of our use of $\morph$ as a morphism of a category 
or, for that matter, as an edge in a presentation. Neither are we alluding to a bicategory structure. We have two distinct uses for $\morph$ (three if you distinguish arrows in presentations from arrows in categories). Any particular use will be unambiguous in context.

\begin{definition}
If $\catc$ is a category and $\reqtc$ is a set of instances, if
\fgsourcediag
in $\catc$ 
and if there is a functional dependency $\fundep{H}{f}{g}$ then say that 
the functional dependency $H$ is \term{represented} in $\catc$ 
iff there exists a morphism $h:b \morph c$ in $\catc$ such that for each instance $D \in \reqtc$, $D(h)=H_D$.
\end{definition}


\subsubsection{The Principle of the Representation of Referential Inclusion Dependencies}
\newcommand{\fnsourceqnsource}
{
$
\begin{array}{c p{0.25cm} c  p{0.25cm} c }
             &&   \Rnode{b1}{b_1} &&              \\[0.4cm]
\Rnode{a}{a} &&                   && \Rnode{c}{c} \\[0.4cm]
             &&   \Rnode{bn}{b_n} &&              
\end{array} 
\begin{arrows}
\ncarr{a}{b1}
\alabel{f_1}
\ncarr{c}{b1}
\blabel{q_1} 
\ncarr{a}{bn}
\blabel{f_n}
\ncarr{c}{bn}
\alabel{q_n}
\end{arrows}
$   
}
\begin{definition}
If $\catc$ is a category and $\reqtc$ is a set of instances 
and if
\fnsourceqnsource
in $\catc$, then a \term{referential inclusion dependency} $I$, written $a[f_1,...f_n] \overset{I}{\subseteq} c[q_1,..q_n]$, is a family of functions $I_D)_{D \in \reqtc}$
such that each instance $D \in \reqtc$, $I_D$ is a unique function $I_D : D(a) \morph D(c)$ such that
for each $i$, $1 \leq i \le n$, $I_D \circ D(q_i) = D(f_i)$.
\end{definition}

\begin{definition}
If $\catc$ is a category and $\reqtc$ is a set of instances and if
\fnsourceqnsource
in $\catc$ and if $a[f_1,...f_n] \overset{I}{\subseteq} c[q_1,..q_n]$ is a referential inclusion dependency
with respect  to $\reqtc$ then say that the inclusion dependency $I$ is \term{represented} in $\catc$
iff there exists a morphism $i:a \morph c$ in $\catc$ such that in each instance $D \in \reqtc$, $D(i) = I_D$. 
\end{definition}

\subsubsection{Path Equivalence}

\newcommand{\fgparalleldiag}
{
 $
\rule[-0.3cm]{0pt}{0.9cm} %to add vertical space of diagram -- based on lowering diagram 0.3cm and heght 0.9cm
                            % change thickness from 0pt to 1 pt to debug
\begin{array}{c p{0.5cm} c  }
 \Rnode{a}{a}            &&   \Rnode{b}{b}
\end{array} 
\begin{arrows}
\ncarc[nodesep=2pt,arcangle=10,offset=2pt]{->}{a}{b}
\alabel{f}
\ncarc[nodesep=2pt,arcangle=-10,offset=-2pt]{->}{a}{b}
\blabel{g}
\end{arrows}
$  
}

\newcommand{\pequiv}[1][R_C]{\underset{#1}{\equiv}}

\begin{definition}
If $\catcw$ is a  category, if $\reqtc$ is a set of instances
 and if \fgparalleldiag in $\catc$, then say that path $f$ is equivalent to path $g$ with respect to the requirement $R_C$ 
 (and write $f \pequiv g$) iff
in all instances $D \in \reqtc$, $D(f)=D(g)$.
\end{definition}

\begin{definition}
If $\catc$ is a  category and $\reqtc$ is a set of instances,
 and if \fgparalleldiag in $\catc$ such that $f \pequiv g$
 then say that the path equivalence $f \pequiv g$ is represented in \catcw iff
 $f=g$.
\end{definition}

\begin{oldtt}
\begin{definition}
If $\catc$ is a  category and $\reqtc$ is a set of instances,
 then say that  $\catc$ is \term{logically complete} with respect 
to the requirement $\reqtc$ iff all path equivalences with respect to $R_C$ are represented in \catcw 
i.e. iff for all diagrams \fgparalleldiag in $\catc$,  
if in all instances $D \in \reqtc$, $D(f)=D(g)$,  then $f=g$ in $\catc$.
\end{definition}
\end{oldtt}
\subsection{Representation Lemmas}
We now show that if we assume local finiteness of the category \catcw generated by a sketch $S$ 
 having a data specification requirement represented by a set of
instances $R_C$ then  principle 2 (maximal constrainedness) holds then
principles 2A, 2B and 2C hold also. 

\subsubsection{Representation of Path Equivalences}
\note
With an assumption of local finiteness (not ideal) we will show that 
the criteria of maximal constrainedness holds then the representation criteria also hold.

if a data specification is maximally constrained to a
requirement then  all commutivity constraints,  functional dependencies and referential inclusion dependencies arising from the requirement are represented in the data specification.

\begin{lemma}
\llabel{pathequivalencerepresentationlemma}
If $\catc$ is a locally finite category and $\reqtc$ is a set of instances, if $\catc$ is 
\term{maximally constrained} to the requirement $\reqtc$ then all path equivalences with respect
to $R_C$ are represented in \catcw
i.e. for all diagrams
$
\rule[-0.3cm]{0pt}{0.9cm} %to add vertical space of diagram -- based on lowering diagram 0.3cm and heght 0.9cm
                            % change thickness from 0pt to 1 pt to debug
\begin{array}{c p{0.5cm} c  }
 \Rnode{a}{a}            &&   \Rnode{b}{b}
\end{array} 
$
\ncarc[nodesep=2pt,arcangle=10,offset=2pt]{->}{a}{b}
\alabel{f}
\ncarc[nodesep=2pt,arcangle=-10,offset=-2pt]{->}{a}{b}
\blabel{g}
in $\catc$,  if in all instances $D \in \reqtc$, $D(f)=D(g)$, 
then $f=g$ in $\catc$.
\end{lemma}
\begin{proof}
Suppose such a category  $\catcw$  that  is 
\term{maximally constrained} to a requirement $\reqtc$
and suppose 
$
\rule[-0.3cm]{0pt}{0.8cm} %to add vertical space of diagram -- based on lowering diagram 0.3cm and heght 0.9cm
                            % change thickness from 0pt to 1 pt to debug
\begin{array}{c p{0.5cm} c  }
 \Rnode{a}{a}            &&   \Rnode{b}{b}
\end{array} 
$
\ncarc[nodesep=2pt,arcangle=10,offset=2pt]{->}{a}{b}
\alabel{f}
\ncarc[nodesep=2pt,arcangle=-10,offset=-2pt]{->}{a}{b}
\blabel{g}
in $\catc$. From sketch $S$ of $C$ we can construct a sketch $S'$ by formally adding a path equivalence $f=g$.
$S'$ generates a category $C'$ for which we have a functor $I: \catc \morph \catcp$. 
Because $D(f) = D(g)$ in every instance $D \in \reqtc$
it follows that $I$ is consistent with $\reqtc$. Because $\catc$ is maximally constrained to $\reqtc$
and because $I: \catc \morph \catcp$ is consistent with $\reqtc$ it follows that the functor $Hom_{\catc}(a,-): \catc \morph \Fin$ 
can be extended to a functor $F: \catcp \morph \Fin$. Since $I(f)=I(g)$ in $\catcp$ then $F(I(f))=F(I(g))$. But $I \circ F
= Hom_\catc(a,-)$ therefore we have that $Hom(a,f)=Hom(a,g)$ and, applying both sides to $id_a$, that $f=g$ in $C$ as required.
\end{proof}

In this lemma above we have assumed that \catcw is locally finite. Can we not prove this result for all categories? The following example shows not. 
\begin{example}
Suppose \catcw is the category generated by the sketch with directed graph
\begin{displaymath}
\begin{array}{cp{1.4cm}c}
                                    \\[0.1cm]
\Rnode{a}{a}	&& \Rnode{b}{b}     \\[0.25cm]
	            &&  
\end{array}
\begin{arrows}
\ncarr[15]{a}{b}
\alabel{f}[0.35]
\ncarr[-15]{a}{b}
\blabel{g}[0.35]
\ncarr[-70]{a}{b}
\blabel{h'}[0.35]
\ncarr[-70]{b}{a}
\blabel{h}[0.35]
\nccircle[angleA=-90, nodesep=3pt]{->}{b}{.5cm}
\blabel{r}[0.3]
\end{arrows}
\end{displaymath}

subject to the identities
\begin{equation}
\label{fhidentity}
f \circ h = id_a
\end{equation}
\begin{equation}
\label{ghidentity}
g \circ h = id_a
\end{equation}
and 
\begin{equation}
\label{rhhpidentity}
r \circ h \circ h' = id_b
\end{equation}

We can show that for any functor $D:\catc \morph Fin$, $D(f)=D(g)$. For any requirement 
$R_C$, therefore, $f$ is equivalent to $g$ with respect to $R_C$ and as $f \neq g$ this equivalence of $f$ and $g$ 
is not represented in \catc. Lemma \lref{pathequivalencerepresentationlemma} cannot therefore be made applicable to all categories (or to all finitely presented categories) as we might wish.

To show that if $D:\catc \morph Fin$ then $D(f)=D(g)$ 
first note that we have that $D(r) \circ D(h \circ h')=id_{D(b)}$ because this follows from
identity (\ref {rhhpidentity}) and from the functoriality of $D$, Since 
the set $D(b)$ is finite it follows that $D(r): D(b) \morph D(b)$ is surjective and injective 
and that the function $D(h \circ h'):D(b) \morph D(b)$ is surjective and injective and from this later
that the function $D(h)$ is injective.
Because of identities (\ref{fhidentity}) and (\ref{ghidentity}) and from the 
functoriality of $D$ we have that $D(f) \circ D(h) =id_{D(a)}=D(g) \circ D(h)$. Because $D(h)$ is injective it follows 
that $D(f)=D(g)$, as required.
\end{example}

\subsubsection{Representation of Functional Dependencies}
\begin{lemma}
\llabel{functionaldependencyrepresentationlemma}
If $\catc$ is a locally finite category and $\reqtc$ is a set of instances, if $\catc$ is 
\term{maximally constrained} to the requirement $\reqtc$ then
all functional dependencies $\fundep{H}{f}{g}$  with respect to $\reqtc$ are represented in $\catc$.
\end{lemma}
\begin{proof}
Suppose such a category  $\catc$  that  is 
\term{maximally constrained} to a requirement $\reqtc$ and suppose
$
\begin{array}{c p{0.5cm} c  }
             &&   \Rnode{b}{b} \\[0.01cm]
\Rnode{a}{a} &&                \\[0.01cm] 
             &&   \Rnode{c}{c}         
\end{array} 
$
\ncarr{a}{b}
\alabel{f}
\ncarr{a}{c}
\blabel{g}
in $\catc$ 
and that there is a functional dependency $\fundep{H}{f}{g}$ with respect to $\reqtc$.

From sketch $S$ of $C$ we can construct a sketch $S'$ by formally adding a morphism $\qq{h}: b \morph c$
and path equivalence $f \circ \qq{h} = g$. Let $\catcp$ be the category generated by $S'$ and
let $I: C \morph C'$ be the inclusion functor generated by the inclusion of $S$ in $S'$. 
$I$ is consistent with $\reqtc$ since for any $D \in \reqtc$ we can extend $D$ to $D' :\catcp \morph \Fin$ by defining $D'(\qq{h})=D_H$.

Let the functor $F: \catc \morph \Fin$ be the coproduct $Hom_{\catc}(a,-) + Hom_{\catc}(a,-)$
in the functor category $Fin^{\catc}$ and label the injections $L$ and $R$, respectively so that
for each object $x$ of $\catc$ the diagram
\begin{center}
$
\begin{array}{c p{0.5cm} c p{0.5cm} c  }
\Rnode{h1}{Hom_{\catc}(a,x)}  &&\Rnode{Fx}{F(x)}  &&   \Rnode{h2}{Hom_{\catc}(a,x)}       
\end{array} 
$
\ncarr{h1}{Fx}
\alabel{L_x}
\ncarr{h2}{Fx}
\blabel{R_x}
\end{center}
is a coproduct in $\Fin$.

Now for each object $x$ of $\catc$, we define an equivalence relation $\sim_x$ on $F(x)$ by defining,
for $k_1,k_2:a \morph x$ in $\catc$,
\begin{align*}
L_x(k_1) \sim_x R_x(k_2) & \mbox{ iff there exists $k:b \morph x$ in $\catc$ such that $k_1 = f \circ k = k_2$,}\\
L_x(k_1) \sim_x L_x(k_2) & \mbox{ iff $k_1 = k_2$,} \\
R_x(k_1) \sim_x R_x(k_2) & \mbox{ iff $k_1 = k_2$.} \\
\end{align*}
If $j: x_1 \morph x_2$ in $\catc$ and if $y_1,y_2 \in F(x_1)$ such that $y_1 \sim_{x_1} y_2$
then it follows easily by cases and from the definition of $\sim$ that $F(j)(y_1) \sim_{x_2} F(j)(y_2)$.
Therefore we can define a functor 
 $G: \catc \morph \Fin$  so that for any object $x$ of $\catc$
the set $G(x)$ is the quotient $F(x)/{\sim_x}$ and such that 
if $j: x_1 \morph x_2$ in $\catc$ and if $y \in F(x_1)$ then $G(j)([y])=[F(j)(y)]$.
With $G$ so defined then if $k: a \morph x_1$ and $j:x_1 \morph x_2$ in $\catc$
then  $G(j)([L_{x_1}(k)])=[L_{x_2}(k \circ j)]$ and $G(j)([R_{x_1}(k)])=[R_{x_2}(k \circ j)]$. 

Now that we have described the functors  $G: \catc \morph \Fin$ and $I:\catc \morph \catcp$ that is consistent with $\reqtc$
we can use the fact that $\catc$ is maximally constrained to tell us that $G$ extends to a functor 
$G' : \catcp \morph \Fin$. Since $f \circ \qq{h} = g$ in $\catcp$ then we have
 $G'(f) \circ G'(\qq{h}) = G'(g): G(a) \morph G(c)$ in $\Fin$.
Now we have
\begin{align*}
[L_c(g)]&= G'(g)([L_a(id_a)])              & & \mbox{by definition of $G'(g)=G(g)$}           \\
        &= G'(\qq{h}) (G'(f)([L_a(id_a)])) & & \mbox{since $G'(f) \circ G'(\qq{h}) = G'(g)$}  \\
				&= G'(\qq{h}) (G'([L_b(f)]))       & & \mbox{by definition of $G'(f)=G(f)$}           \\
				&= G'(\qq{h}) (G'([R_b(f)]))       & & \mbox{since $L_b(f) \sim_b R_b(f)$, by definition of $\sim_b$} \\
				&= G'(\qq{h}) (G'(f)([R_a(id_a)])) & & \mbox{by definition of $G'(f)=G(f)$}           \\
		    &= G'(g)([R_a(id_a)])              & & \mbox{since $G'(f) \circ G'(\qq{h}) = G'(g)$}  \\
				&= [R_c(g)]                        & & \mbox{by definition of $G'(g)=G(g)$}           \\
\end{align*} 
Since $[L_c(g)]=[R_c(g)]$ then it follows from the definition of $\sim_c$ that there exists $k:b \morph c$ in 
$\catc$ such that $f \circ k = g$ and we have shown as required that the function dependency
$\fundep{H}{f}{g}$ is represented in $\catc$.

\end{proof}

In this lemma above we have assumed that \catcw is locally finite. Can we not prove this result for all categories? A development of the previous examples shows not.
\begin{example}
Suppose \catcw is the category generated by the directed graph
\begin{displaymath}
\begin{array}{cp{0.6cm}cp{0.6cm}c}
                                                           \\[0.1cm]
                &&  \Rnode{x}{x}       &&                  \\[-0.2cm]
\Rnode{a}{a}	&&                     && \Rnode{b}{b}     \\[0.25cm]
	            &&                     &&
\end{array}
\begin{arrows}
\ncarr[10]{a}{x}
\alabel{f_1}[0.35]
\ncarr[10]{x}{b}
\alabel{f_2}[0.35]
\ncarr[-15]{a}{b}
\blabel{g}[0.35]
\ncarr[-70]{a}{b}
\blabel{h'}[0.35]
\ncarr[-75]{b}{a}
\blabel{h}[0.35]
\nccircle[angleA=-90, nodesep=3pt]{->}{b}{.5cm}
\blabel{r}[0.3]
\end{arrows}
\end{displaymath}

subject to the identities
\begin{equation}
\label{fdcounterfhidentity}
f_1 \circ f_2 \circ h = id_a
\end{equation}
\begin{equation}
\label{fdcounterghidentity}
g \circ h = id_a
\end{equation}
and 
\begin{equation}
\label{fdcounterrhhpidentity}
r \circ h \circ h' = id_b
\end{equation}

As in the previous example, we can show that for any functor $D:\catc \morph Fin$, $D(f1 \circ f_2)=D(g)$. For any requirement 
$R_C$, therefore,  $g$ is functionally dependent on $f_1$ 
with respect to $R_C$. This functional dependency  
is not represented in \catc. Therefore Lemma \lref{functionaldependencyrepresentationlemma} cannot therefore be made applicable to all categories (or to all finitely presented categories).

\end{example}

\subsubsection{Representation of Referential Inclusion Dependencies}
\begin{lemma}
\label{catincdsrepresented}
If $\catc$ is a locally finite category and $\reqtc$ is a set of instances, if $\catc$ is 
\term{maximally constrained} to the requirement $\reqtc$ then
every referential inclusion dependency with respect to $\reqtc$ is represented in $\catc$.
\end{lemma}
\begin{proof}
Suppose such  a category $\catc$ and  requirement $\reqtc$ 
 and suppose there is a referential inclusion dependency
$a[f_1,...f_n] \overset{I}{\subseteq} b[q_1,..q_n]$ with respect to $\reqtc$,
where
$
\begin{array}{c p{0.25cm} c  p{0.25cm} c }
             &&   \Rnode{b1}{b_1} &&              \\[0.4cm]
\Rnode{a}{a} &&                   && \Rnode{b}{b} \\[0.4cm]
             &&   \Rnode{bn}{b_n} &&              
\end{array} 
$
\ncarr{a}{b1}
\alabel{f_1}
\ncarr{b}{b1}
\blabel{q_1} 
\ncarr{a}{bn}
\blabel{f_n}
\ncarr{b}{bn}
\alabel{q_n}
in $\catc$, 

From sketch $S$ of $C$ we can construct a sketch $S'$ by formally adding a morphism $\qq{f}: a \morph b$
and path equivalences, for each $i$, $1 \leq i \leq n$, $\qq{f} \circ q_i = f_i$. Let $\catcp$ be the category generated by $S'$ and
let $I: C \morph C'$ be the inclusion functor generated by the inclusion of $S$ in $S'$. $I$ is consistent with $\reqtc$ since
each instance $D \in  \reqtc$ can be extended to a functor $D': \catcp \morph \Fin$ by defining $D'(\qq{f})=I_D$.

Since $\catc$ is locally finite then there is a functor $Hom_{\catc}(a,-): \catc \morph \Fin$ and since $\catc$ is maximally constrained and because $I$ is consistent with $\reqtc$ then there is a functor $F': C' \morph \Fin$
such that $I \circ F' = Hom_{\catc}(a,-)$. In particular we must have 
$F'(a)=Hom_{\catc}(a,a)$,
$F'(b)=Hom_{\catc}(a,b)$ and therefore $F'(\qq{f}):Hom_{\catc}(a,a) \morph Hom_{\catc}(a,b)$ which gives us
$F'(\qq{f})(id_a): a \morph b$ in $\catc$. 

We can show that $F'(\qq{f})(id_a)$ represents
the inclusion dependency $a[f_1,...f_n] \overset{I}{\subseteq} b[q_1,..q_n]$ by showing that for 
each $i$, $1 \leq i \leq n$, $F'(\qq{f})(id_a) \circ q_i = f_i$. This follows because we have defined
the sketch $S'$ so that in category $C'$ we have $\qq{f} \circ q_i = f_i$. From this it follows
that $F'(\qq{f}) \circ F'(q_i) = F'(f_i)$ i.e. $F'(\qq{f}) \circ Hom(a,q_i) = Hom(a,f_i)$. 
Applying left and right hand sides to $id_a$ we get $F'(\qq{f})(id_a) \circ q_i = f_i$, as required.
\end{proof}


\subsection{Other Lemmas}

The definition of functional dependency  that has been given above 
can be rephrased slightly as we now describe. The rephrasing and an observation regarding
inclusion dependencies and jointly-injective functions points the
way to a future note in which we will describe the use of categories with specified monos, epis and epi-mono splits as data specifications.
\subsubsection{Rephrasing of Principle 2B regarding Representation of Functional Dependencies}
\begin{lemma}
If $\catc$ is a locally finite category and $\reqtc$ is a set of instances, if \fgsourcediag in \catcw
then if there exists a functional dependency $\fundep{H}{f}{g}$  with respect to $R_C$ then
for each instance  $D \in \reqtc$, $D(f)$ is surjective.
\end{lemma}
\begin{proof}
Follows from the uniqueness condition for $H_D$ in the definition of functional dependency which cannot hold in any instance $D \in \reqtc$ in which $D(f)$ is not surjective.
\end{proof}
\begin{corollary}
For any morphism $f:a \morph a$ is \catcw there is a functional dependency $\fundep{H}{f}{f}$ iff
in each instance  $D \in \reqtc$, $D(f)$ is surjective.
\end{corollary}
\begin{proof}
Follows from the previous lemma because in every instance $D \in \reqtc$ 
we can take the function $H_D: D(A) \morph D(a)$ to be the identity function on the set $D(a)$
so that  $D(f) \circ H_D = D(f)$. 
\end{proof}
This lemma and its corallary  mean that we can replace the uniqueness requirement in the 
definition of functional dependency by a requirement that each $D(f)$ be surjective and obtain the following
rephrasing of the definition:
\begin{definition}
If $\catc$ is a locally finite category and $\reqtc$ is a set of instances, if \fgsourcediag in \catcw
then there is a  \term{functional dependency} of $g$ on $f$ with respect to $\reqtc$ iff
\begin{itemize}
\item in every instance $D$ the function $D(f)$ is surjective and
\item
there is a family of functions $H_D)_{D \in \reqtc}$ such that 
in each instance $D$, $H_D$ is a  function $H_D: D(b) \morph D(c)$, 
such that $D(f) \circ H_D = D(g)$. 
\end{itemize}
\end{definition}
\subsubsection{Regarding Jointly-Injectives and Inclusion Dependencies}

\begin{lemma}
If $\catc$ is a category and $\reqtc$ is a set of instances and if
\fnsourceqnsource
in $\catc$
such that $a[f_1,...f_n] \overset{I}{\subseteq} c[q_1,..q_n]$ is an inclusion dependency with respect  to $\reqtc$ then if in each instance $D \in \reqtc$ the family of functions
$q_{i, 1 \leq i \leq n}$ is jointly-injective then $a[f_1,...f_n] \overset{I}{\subseteq} c[q_1,..q_n]$ is a referential inclusion dependency.
\end{lemma}
\begin{proof}
Straightforward.
\end{proof}

\iffalse
\newcommand {\qnsourcediag}{
$
\begin{array}{c p{0.5cm} c  }
             &&   \Rnode{b}{b_1} \\[0.01cm]
\Rnode{a}{a} &&                \\[0.01cm] 
             &&   \Rnode{c}{b_n}         
\end{array} 
\begin{arrows}
\ncarr{a}{b}
\alabel{q_1}
\ncarr{a}{c}
\blabel{q_n}
\end{arrows}
$  
}
\fi



\bibliographystyle{alpha} 
\bibliography{../SharedBibliography/temp/bibliography}
\end{document}
