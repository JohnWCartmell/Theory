
\note It seems to me that in the syntax of gats, \textit{types and terms} on the one hand and \textit{contexts and realisations} on the other are each mutually dependent. In the world 
of syntax neither can exist without the other.
\begin{center}
$
\begin{array}{ c }
\etype{\Rnode{TandT}{types\ and\ terms}} \\[1.3cm]
\etype{\Rnode{CandR}{contexts\ and\ realisations}}   
\end{array}                     
$
\setlength{\sarnodesepA}{7pt}
\setlength{\sarnodesepB}{7pt}
\setlength{\saroffsetA}{7pt}
\setlength{\saroffsetB}{7pt}
\ncsar[10]{TandT}{CandR}
\ncsar[10]{CandR}{TandT}
\end{center}

\note The theory \textit{meta-gat} (\cite{Cartmell14B}) algebraically describes the substitution structure of equivalence classes of types and terms of a gat. Meta-gat algebras have been described by Vladimir as B-systems and have also been described by Richard Garner (\cite{Garner15}) as algebras of a monad over a functor category.  We can summarise (at a stetch) by saying that ontologising types and terms yields meta-gat algebras also known as B-systems:

\begin{center}
$
\begin{array}{ c p{0.5cm} c}
\etype{\Rnode{TandT}{types\ and\ terms}} & & \attrtype{\Rnode{metagat}{\mbox{\textit{meta-gat = B-system}}}}
\end{array}                     
$
\setlength{\arrnodesepA}{7pt}
\setlength{\arrnodesepB}{7pt}
%\setlength{\arroffsetA}{7pt}
%\setlength{\arroffsetB}{7pt}
\ncarr{TandT}{metagat}
\end{center}

We can formalise this connection by showing that the category of generalised algebrauic theories 
is equivalent to the category of meta-gat algebras.  I wrote out the details of this proof in aproximately 1977 in the draft thesis I mentioned previously. The proof just involves a lot of leg work. The theory meta-gat is generalised algebraic.
 
\note The theory of \textit{contextual catagories} algebraically describes the substitution structure of equivalence classes of contexts and realisations of a gat. Vladimir has called  these C-systems. In this case we can summarise by saying that ontologising contexts and realisations yields contextual categories also known as C-systems:

\begin{center}
$
\begin{array}{ c p{0.5cm} c}
\etype{\Rnode{CandR}{contexts\ and\ realisations}}   & & \attrtype{\Rnode{concat}{\mbox{\textit{contextual category = C-system}}}}
\end{array}                     
$
\setlength{\arrnodesepA}{7pt}
\setlength{\arrnodesepB}{7pt}
\ncarr{CandR}{concat} 
\end{center}

This statement is formalised by showing that the category of generalised algebraic theories 
is equivalent to the category of contextual categories. This proof is written up in detail
in my thesis. The theory of contextual categories is generalised algebraic as shown by Vladimir
(\cite{Cartmell2018A}). 
I missed this in my work. 

\note
From the above it follows that the category of meta-gat algebras and the category of contextual categories are equivalent. I have worked from time to time on equational rewrite systems and, since retirement, I spent a good deal of time trying to write some software to do most of the leg work of such a proof by coming up with a rewrite system for the theory of contextual categories. In the end I gave up on it (which, by the way, doesn't mean to say that such a rewrite system cannot be be developed  -- it is just that the way that I was trying to do it just got too complicated). 

\note 
There is another equivalent structure that I think exists and that I have worked on from time to time without ever bottoming out. In my most recent work toward this I have called these
structures \textit{dependency categories} though realising now that Vladimir wouldn't like me calling them categories and might prefer a name like D-systems. I first described something along these lines in a conference paper in 1986 (\cite{Cartmell86B}). That paper was badly flawed in fact but the sentiment remains. 
\note You defined cwfs to play something like the role of contextual categories and meta-gat algebras (a.k.a C-systems and B-systems). cwfs have ontological representations of both contexts and context realisations and types and terms. 


Unlike contextual categories and meta-gat algebras, there are cwfs that do not correspond to
any generalised algebraic theory. If a cwf corresponds to a gat then each object of the category \catcw with the exception of the terminal object is of the form $\gamma.A$ for some object $\Gamma$ of \catcw and for some $a \in Typ(\Gamma)$. Such conditions are described in your paper called ???. There is such a thing therefore as a contextual cwf. 

\note 
There now follows an aside relevant to the statement of a  condition for a cwf to be contextual. Consider  Lisp-style lists. These  are often described as algebraic data structures but of course such structures are not algebraic in the precise sense that we mean when we speak of many-sorted algebraic theories or generalised algebraic theories rather they require coproducts of types for their definition. In a type theory that is many-sorted algebraic (or generalised algebraic) augmented by coproducts Lisp-style the type of lists over some other type $A$ can be defined as the coproduct of the types of empty lists $Empty$ and non-empty lists $Cons$
 as follows:\\



\begin{minipage}{\textwidth}
\begin{notebox}[Theory of lists of elements of type A]

If the empty list is written as $nil$ and if the LISP list constructor $cons$ is represented
as an infix operator $::$ then the theory of a list of elements of type $A$ can expressed as follows: 
\begin{gatrules}
\gatintros
\gatintro{Empty}{}{\isT{Empty}}   
\gatintro{Cons}{}{\isT{Cons}}     
\gatintro{List}{}{List = Empty + Cons} 
\gatintro{nil}{}{\ofT{nil}{Empty}}    
\gatintro{::}{\ofT{x}{A},\ \ofT{l}{List}}{\ofT{x::l}{List}}
\gatintro{hd}{\ofT{l}{Cons}}{\ofT{hd(l)}{A}} 
\gatintro{tl}{\ofT{l}{Cons}}{\ofT{tl(l)}{List}}   
\gataxioms
\gataxiom{e=nil}{\ofT{e}{Empty}}                            
\gataxiom{hd(x::l)=x}{\ofT{l}{List},\  \ofT{x}{A}}         
\gataxiom{tl(x::l)=l}{\ofT{l}{List},\  \ofT{x}{A}}        
\gataxiom{hd(l)::tl(l)=l}{\ofT{l}{Cons}} 
\end{gatrules}
Expressed as sketch of a category we have a coproduct diagram
expressing $List = Empty + Cons$,
a product diagram expressing the product $A \times List$,
additional morphisms:
$nil: 1 \morph Empty$,
$cons : A \times List \morph Cons$,
$hd: Cons \morph A$ and
$tl: Cons \morph List$.

and identities (commuting diagrams) expressing that
morphisms $nil: 1 \morph Empty$ and $cons : A \times List \morph Cons$ are isomorphims 
with inverses, respectively, of the unique morphism $t_{Empty}: Empty \morph 1$
and the morphism $\tuple{hd,tl}:Cons \morph A \times List$. 

As an entity model:
\begin{center}
\begin{erdiagram}{4.15}{10.5166}

\eret{0.1}{-1.6}{1.433}{-1}{0.2}{1}\eretname{0.767}{-1.35}{}{1}
\eret{0.1}{-3.15}{1.433}{-2.55}{0.2}{1}\eretname{0.767}{-2.9}{}{A}
\eret{2.933}{-3.65}{4.525}{-1.85}{0.2}{1}\eretname{3.729}{-2.2}{}{AxList}
\eret{5.933}{-3.2}{9.717}{-0.5}{0.2}{1}\eretname{6.149}{-0.85}{l}{List}
\eret{7.133}{-1.6}{8.467}{-1}{0.2}{0}\eretname{7.8}{-1.35}{}{Empty}
\eret{7.133}{-2.8}{8.467}{-1.9}{0.2}{0}\eretname{7.8}{-2.25}{}{Cons}

% relationship nil
\errelname{1.583}{-1.15}{l}{nil}\errelname{6.983}{-1.15}{r}{t}\errelarm{1.433}{-1.3}{4.283}{-1.3}{1}{0}\errelarm{4.283}{-1.3}{7.133}{-1.3}{1}{0}
% relationship cons
\errelname{4.675}{-2.15}{l}{cons}\errelname{6.983}{-2.15}{r}{head,tail}\errelarm{4.525}{-2.3}{5.829}{-2.3}{1}{0}\errelarm{5.829}{-2.3}{7.133}{-2.3}{1}{0}
% relationship p2
\errelname{4.675}{-2.69}{l}{p2}\errelarm{4.525}{-2.84}{5.229}{-2.84}{1}{0}\errelarm{5.229}{-2.84}{5.933}{-2.84}{1}{0}\ercrowfoot{4.675}{-2.84}{4.525}{-2.69}{4.525}{-2.84}{4.525}{-2.99}{0}\eridrefrel{4.7748}{-2.7399999999999998}{-2.94}
% relationship p1
\errelname{2.783}{-2.7}{r}{p1}\errelarm{2.933}{-2.85}{2.183}{-2.85}{1}{0}\errelarm{2.183}{-2.85}{1.433}{-2.85}{1}{0}\ercrowfoot{2.783}{-2.85}{2.933}{-2.7}{2.933}{-2.85}{2.933}{-3}{0}\eridrefrel{2.6833}{-2.7499999999999996}{-2.9499999999999997}
\end{erdiagram}
 
\end{center}
\end{notebox}
\end{minipage}

