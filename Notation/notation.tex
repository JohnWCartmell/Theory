\documentclass[10pt,a4paper]{article}
\usepackage{pstricks}
\usepackage{pst-node}
\usepackage{stmaryrd}
\usepackage{amsmath}
\usepackage{amssymb}
\usepackage{verbatim}
\usepackage{enumerate}
\usepackage {textcomp}
\usepackage{ulem}

%ccategories.macros.tex 

% Macros for diagrams in contextual categories and related categories

\usepackage{twoopt}
\usepackage{scalerel} 
\usepackage{xargs}

%\usepackage{mathabx}  %Caused font problems
%\usepackage{MnSymbol}  % caused font problems

\newcommand{\conu}
{\mathbf{C}(U)}

\newcommand{\depu}
{\mathbf{D}(U)}

\newcommand{\cat}[1]{\textbf{#1}}
\newcommand{\obj}[1]{\ensuremath{|\cat{#1}|}}
\newcommand{\ccat}[1][C]{\ensuremath{\mathbb{#1}} }
\newcommand{\ccatc}{contextual category \ccat}
\newcommand{\cobj}[2][]{\ensuremath{|\ccat[#2]|_{#1}}}
\newcommand{\cslice}[2]{\ensuremath{\ccat[#1]_{#2}}}
\newcommand{\csliceobj}[3][]{\ensuremath{|\mathbb{#2}_{#3}|_{#1} }}
\newcommand{\varset}[1][]{\ensuremath{V_{#1} }}
\newcommand{\localvarsets}{\ensuremath{\mathcal{V} }}
\newcommand{\Fam}{\ensuremath{\mathbb{F\mathrm{am}} }}
\newcommand{\Famslice}[1]{\ensuremath{\mathbb{F\mathrm{am}}_{#1} }}
\newcommand{\Famobj}[1][]{\ensuremath{|\mathbb{F\mathrm{am}}|_{#1} }}
\newcommand{\Famsliceobj}[2][]{\ensuremath{|\mathbb{F\mathrm{am}}_{#2}|_{#1} }}
\newcommand{\morph}{\rightarrow}
\newcommand{\epi}{\twoheadrightarrow}
\newcommand{\base}{\triangleleft}
\newcommand{\comp}{\circ}
\newcommand{\cross}{\otimes}
\newcommand{\pc}[2]{d^{#1}_{#2}}
\newcommand{\sub}{^*}
\newcommand{\diag}{\delta}
\newcommand{\pbase}[1]{\tilde{#1}}

\newcommand{\tuple}[1]{\langle#1\rangle}
\newcommand{\ndidly}{\ensuremath{\Join_n}}
\newcommand{\ndidlycospan}{quotiented n-cospan}

\newcommand{\crossx}[3]{#1 \underset{#3}{\cross} #2}
\newcommand{\fibrex}[3]{#1 \underset{#3}{\Join} #2}
\newcommand{\powerset}{\mathcal{P}}
\newcommand{\primeds}[1]{
\ensuremath{\mathcal{P}(#1)} }
\newcommand{\compset}{\ \dot{\circ}\, }

% darrow
%\newcommand{\darrow}{\rightarrowtriangle} %use \smorph instead
\newcommand{\smorph}{\rightarrowtriangle}

 

\newcommand\dhead{\scaleobj{0.6}{\triangleright}}
\newcommand{\dmorph}{\, \mbox{---} \! \cdot \! \raisebox{1.1pt}{\dhead}}

% projection tree
%\newcommand{\proj}[2]{proj_{#2}(#1)}

\newcommand{\proj}[2]{
\ensuremath{\mathcal{P}_{#2}(#1)} }

%pstrick supplements for arrows

\newlength{\arrnodesepA}
\newlength{\arrnodesepB}
\newlength{\arroffsetA}
\newlength{\arroffsetB}

%Modified to 2pt from 0pt on 23 July 2018
\newcommand{\arreset}{
\setlength{\arrnodesepA}{2pt}
\setlength{\arrnodesepB}{2pt}
\setlength{\arroffsetA}{0pt}
\setlength{\arroffsetB}{0pt}
}
\arreset

\newcommand{\ncarr}[3][0]{\ncarc[arcangle=#1,nodesepA=\arrnodesepA,nodesepB=\arrnodesepB,offsetA=\arroffsetA,offsetB=\arroffsetB,arrowsize=5pt,arrowinset=0.7]{->}{#2}{#3}}
\newcommand{\jcbarr}[4][0]{ % ncbarr is defined in some thridy party package so do not use!\emph{}
\ncarr[#1]{#3}{#4}
\nbput[labelsep=2pt]{\footnotesize $#2$}
}

\newcommand{\ncaarr}[4][0]{
\ncarr[#1]{#3}{#4}
\naput[labelsep=2pt]{\footnotesize $#2$}
}

% \alabel{label}[npos][labelsep_pts]
\newcommandx*\alabel[3][2=0.5,3=2,usedefault]{\naput[labelsep=#3pt,npos=#2]{\footnotesize $#1$}}
% \blabel{label}[npos][labelsep_pts]
\newcommandx*\blabel[3][2=0.5,3=2,usedefault]{\nbput[labelsep=#3pt,npos=#2]{\footnotesize $#1$}}

% \idcomp mark an arrow as one component of an identifier
\newcommand{\idcomp}{\ncput[npos=0, nrot=:U]{\psline(0.2,-0.075)(0.2,0.075)}}  %add a bar to a node connection arrow
% pstrick supplements for s-arrows (previous name for d-arrow - should convert}

\newlength{\sarnodesepA}
\newlength{\sarnodesepB}
\newlength{\saroffsetA}
\newlength{\saroffsetB}
\newlength{\sarnodesepAsav}
\newlength{\sarnodesepBsav}

\newcommand{\sarreset}{
\setlength{\sarnodesepA}{0pt}
\setlength{\sarnodesepB}{0pt}
\setlength{\saroffsetA}{0pt}
\setlength{\saroffsetB}{0pt}
}

\sarreset

% sar - S-arrow
\newcommand{\ncsar}[3][0]{
\setlength{\sarnodesepAsav}{\sarnodesepA}
\setlength{\sarnodesepBsav}{\sarnodesepB}
\addtolength{\sarnodesepA}{3pt}
\addtolength{\sarnodesepB}{7pt}
\ncarc[nodesepA=\sarnodesepA,nodesepB=\sarnodesepB,offsetA=\saroffsetA,offsetB=\saroffsetB,arcangle=#1]{-}{#2}{#3}
\ncput[nrot=:R,npos=1]{\pstriangle(0,0)(.2,.2)}
\setlength{\sarnodesepA}{\sarnodesepAsav}
\setlength{\sarnodesepB}{\sarnodesepBsav}
}


% bsar - below labelled S-arrow
\newcommand{\ncbsar}[4][0]{
\ncsar[#1]{#3}{#4}
\nbput[labelsep=2pt]{\footnotesize $#2$}
}
% asar - above labelled S-arrow
\newcommand{\ncasar}[4][0]{
\ncsar[#1]{#3}{#4}
\naput[labelsep=2pt]{\footnotesize $#2$}
}

% cdar - composite dependency arrow
\newcommand{\nccdar}[3][0]{
\setlength{\sarnodesepAsav}{\sarnodesepA}
\setlength{\sarnodesepBsav}{\sarnodesepB}
\addtolength{\sarnodesepA}{3pt}
\addtolength{\sarnodesepB}{11pt}
\ncarc[nodesepA=\sarnodesepA,nodesepB=\sarnodesepB,offsetA=\saroffsetA,offsetB=\saroffsetB,arcangle=#1]{-}{#2}{#3}
\ncput[nrot=:R,npos=1]{\pstriangle(0,0.1)(.2,.2)}
\ncput[nrot=:R,npos=1]{\psdot[dotsize=1pt](-0.0075,0.05)}   %!!
\setlength{\sarnodesepA}{\sarnodesepAsav}
\setlength{\sarnodesepB}{\sarnodesepBsav}
}


% bcdar - below labelled composite dependency arrow
\newcommand{\ncbcdar}[4][0]{
\nccdar[#1]{#3}{#4}
\nbput[labelsep=2pt]{\footnotesize $#2$}
}
% acdar - above labelled composite dependency arrow
\newcommand{\ncacdar}[4][0]{
\nccdar[#1]{#3}{#4}
\naput[labelsep=2pt]{\footnotesize $#2$}
}


% rsar - recursive S-arrow
\newcommand{\ncrsar}[2]{
\setlength{\sarnodesepAsav}{\sarnodesepA}
\setlength{\sarnodesepBsav}{\sarnodesepB}
\addtolength{\sarnodesepA}{3pt}
\addtolength{\sarnodesepB}{7pt}
\ncloop[nodesepA=\sarnodesepA,nodesepB=\sarnodesepB,
        offsetA=\saroffsetA,offsetB=\saroffsetB,
        armA=0.7cm,armB=0.6cm,angleA=90,angleB=-90,loopsize=-1,linearc=0.4
				]{-}{#1}{#2}
\ncput[nrot=:R,npos=5]{\pstriangle(0,0)(.2,.2)}
\setlength{\sarnodesepA}{\sarnodesepAsav}
\setlength{\sarnodesepB}{\sarnodesepBsav}
}

% pstrick supplements for multi-arrows

\newlength{\marnodesepA}
\newlength{\marnodesepB}
\newlength{\maroffsetB}
\newlength{\marnodesepBsav}

\newcommand{\marreset}{
\setlength{\marnodesepA}{0pt}
\setlength{\marnodesepB}{0pt}
\setlength{\maroffsetB}{0pt}
}

\marreset

%ncmarr[#1 arcangle1][#2 arcangle2]{#3 name}{#4 domain1}{#5 domain2}{#6 junction}{#7 codomain}
\newcommandtwoopt{\ncmarr}[6][8][8]{%
\ncarc[nodesepA=\marnodesepA,nodesepB=0,arcangle=#1]{-}{#3}{#5}
\ncarc[nodesepB=0,arcangle=-#1]{-}{#4}{#5}
\ncarc[arcangle=#2,nodesepB=\marnodesepB,offsetB=\maroffsetB]{->}{#5}{#6}
}%


\newcommandtwoopt{\nchmarr}[6][8][8]{%
\ncarc[nodesepA=\marnodesepA,nodesepB=0,arcangle=#1]{-}{#3}{#5}
\ncarc[nodesepB=0,arcangle=#1]{-}{#4}{#5}
\ncarc[arcangle=#2,nodesepB=\marnodesepB,offsetB=\maroffsetB]{->}{#5}{#6}
}%

\newcommandtwoopt{\ncamarr}[7][8][8]{%
\ncmarr[#1][#2]{#4}{#5}{#6}{#7}
\naput[npos=.05]{$#3$}
}%
\newcommandtwoopt{\ncbmarr}[7][8][8]{%
\ncmarr[#1][#2]{#4}{#5}{#6}{#7}
\nbput[npos=.05]{$#3$}
}%

\newcommandtwoopt{\ncbhmarr}[7][8][8]{%
\nchmarr[#1][#2]{#4}{#5}{#6}{#7}
\nbput[npos=.05]{$#3$}
}%

\newcommandtwoopt{\ncmarrr}[7][8][8]{
\ncarc[nodesepB=0,arcangle=#1]{-}{#3}{#6}
\ncline[nodesepB=0]{-}{#4}{#6}
\ncarc[nodesepB=0,arcangle=-#1]{-}{#5}{#6}
\ncarc[nodesepA=0,arcangle=#2]{->}{#6}{#7}
}

\newcommandtwoopt{\ncamarrr}[8][8][8]{
\ncmarrr[#1][#2]{#4}{#5}{#6}{#7}{#8}
\naput[npos=.05]{$#3$}
}
\newcommandtwoopt{\ncbmarrr}[8][8][8]{
\ncmarrr[#1][#2]{#4}{#5}{#6}{#7}{#8}
\nbput[npos=.05]{$#3$}
}

%gats.macros.tex

\usepackage{environ}    % also used in ermacros % here used for \NewEnvrion

\newcommand{\gat}[1][U]{
\ensuremath{\mathcal{#1}}}  % used to hav a space in here
\newcommand{\gatw}[1][U]{\gat[#1]\ }  % use this if need trailing space
\newcommand{\ingat}[1][U]{in \gat[#1]}
\newcommand{\isagat}[1][U]{\gat[#1] is a g.a.t.}
\newcommand{\inagat}{in a g.a.t. }

% macro for a generic theory
%\newcommand{\theory}
%{\textit{U}}

\newcommand{\intheory}
{is a derived rule of \gat[U]}

% Macros for GAT rules

\newcommand{\isT}[1]
{#1\mbox{ is a type}}

\newcommand{\ofT}[2]
{#1 \in #2
}

% Macros for GAT rules   <!-- new old -->
\newcommand{\istype}[1]
{#1\mbox{ is a type}}

\newcommand{\oftype}[2]
{#1 \in #2
}

%\context{x}{\Delta}{n}
\newcommand{\context}[3]
{\ofT{#1_1}{#2_1},... \ofT{#1_{#3}}{#2_{#3}(#1_1,...#1_{#3-1})}
}

%\subcontext{x}{\Delta}{i}{k}
\newcommand{\subcontext}[4]
{\ofT{#1_{#3_1}}{#2_{#3_1}},... \ofT{#1_{#3_#4}}{#2_{#3_#4}(#1_1,...#1_{#3_#4-1})}
}

% #schematic context
\newcommand{\schmcon}[3]
{\ofT{#1_1}{#2_1},... \ofT{#1_{#3}}{#2_{#3}}
}
% abbreviated to
\newcommand{\con}[3]
{\schmcon{#1}{#2}{#3}}

% schematic subcontext
%\subcon{x}{\Delta}{i}{k}
\newcommand{\subcon}[4]
{\ofT{#1_{#3_1}}{#2_{#3_1}},... \ofT{#1_{#3_#4}}{#2_{#3_#4}}
}

% permuted context
%\permcon{x}{\Delta}{n}{\sigma}
\newcommand{\permcon}[4]
{\ofT{#1_{#4(1)}}{#2_{#4(1)}},... \ofT{#1_{#4(#3)}}{#2_{#4(#3)}}
}
% permuted term
%\permterm{t}{n}{\sigma}
\newcommand{\permterm}[3]
{
#1_{#3(1)},...#1_{#3(#2)}
}


% Idioms
\newcommand{\xDelta}[1]{\con{x}{\Delta}{#1}}
\newcommand{\xDeltap}[1]{\con{x}{\Delta'}{#1}}
\newcommand{\xOmega}[1]{\con{x}{\Omega}{#1}}
\newcommand{\xOmegap}[1]{\con{x}{\Omega'}{#1}}
\newcommand{\yOmega}[1]{\con{y}{\Omega}{#1}}
\newcommand{\yOmegap}[1]{\con{y}{\Omega'}{#1}}

\newcommand{\xDeltasigma}[1]{\permcon{x}{\Delta}{#1}{\sigma}}
\newcommand{\xDeltapsigma}[1]{\permcon{x}{\Delta'}{#1}{\sigma}}
\newcommand{\xOmegasigma}[1]{\permcon{x}{\Omega}{#1}{\sigma}}
\newcommand{\xOmegapsigma}[1]{\permcon{x}{\Omega'}{#1}{\sigma}}
\newcommand{\yOmegasigma}[1]{\permcon{y}{\Omega}{#1}{\sigma}}
\newcommand{\yOmegapsigma}[1]{\permcon{y}{\Omega'}{#1}{\sigma}}

\newcommand{\xDeltainvsigma}[1]{\permcon{x}{\Delta}{#1}{\sigma^{-1}}}
\newcommand{\xDeltapinvsigma}[1]{\permcon{x}{\Delta'}{#1}{\sigma^{-1}}}
\newcommand{\xOmegainvsigma}[1]{\permcon{x}{\Omega}{#1}{\sigma^{-1}}}
\newcommand{\xOmegapinvsigma}[1]{\permcon{x}{\Omega'}{#1}{\sigma^{-1}}}
\newcommand{\yOmegainvsigma}[1]{\permcon{y}{\Omega}{#1}{\sigma^{-1}}}
\newcommand{\yOmegapinvsigma}[1]{\permcon{y}{\Omega'}{#1}{\sigma^{-1}}}

%Idioms enclosed as tuples
\newcommand{\encxDelta}[1]{\tuple{\con{x}{\Delta}{#1}}}
\newcommand{\encxDeltap}[1]{\tuple{\con{x}{\Delta'}{#1}}}
\newcommand{\encxOmega}[1]{\tuple{\con{x}{\Omega}{#1}}}
\newcommand{\encxOmegap}[1]{\tuple{\con{x}{\Omega'}{#1}}}
\newcommand{\encyOmega}[1]{\tuple{\con{y}{\Omega}{#1}}}
\newcommand{\encyOmegap}[1]{\tuple{\con{y}{\Omega'}{#1}}}

\newcommand{\encxDeltasigma}[1]{\tuple{\permcon{x}{\Delta}{#1}{\sigma}}}
\newcommand{\encxDeltapsigma}[1]{\tuple{\permcon{x}{\Delta'}{#1}{\sigma}}}
\newcommand{\encxOmegasigma}[1]{\tuple{\permcon{x}{\Omega}{#1}{\sigma}}}
\newcommand{\encxOmegapsigma}[1]{\tuple{\permcon{x}{\Omega'}{#1}{\sigma}}}
\newcommand{\encyOmegasigma}[1]{\tuple{\permcon{y}{\Omega}{#1}{\sigma}}}
\newcommand{\encyOmegapsigma}[1]{\tuple{\permcon{y}{\Omega'}{#1}{\sigma}}}

\newcommand{\encxDeltainvsigma}[1]{\tuple{\permcon{x}{\Delta}{#1}{\sigma^{-1}}}}
\newcommand{\encxDeltapinvsigma}[1]{\tuple{\permcon{x}{\Delta'}{#1}{\sigma^{-1}}}}
\newcommand{\encxOmegainvsigma}[1]{\tuple{\permcon{x}{\Omega}{#1}{\sigma^{-1}}}}
\newcommand{\encxOmegapinvsigma}[1]{\tuple{\permcon{x}{\Omega'}{#1}{\sigma^{-1}}}}
\newcommand{\encyOmegainvsigma}[1]{\tuple{\permcon{y}{\Omega}{#1}{\sigma^{-1}}}}
\newcommand{\encyOmegapinvsigma}[1]{\tuple{\permcon{y}{\Omega'}{#1}{\sigma^{-1}}}}

\newcommand{\tstyle}{\vdash}

% gat macros developed for cwf paper

% Expressing gats
\newenvironment{gatrules}
{
$$
\begin{array}{l l}
}
{
\end{array}
$$
}
\newcommand{\gatintros}
{
\textbf{Symbol} & \textbf{Introductory\ Rule}                      \\}

\newcommand{\gataxioms}
{\textbf{Axioms}\\}
\newcommand{\gatintro}[3]{\ #1 & #2 \tstyle #3 \\}
\newcommand{\gatlocalintro}[3]{\ #1 & #2 \dashv }
\newcommand{\gataxiom}[2]{\multicolumn{2}{l}{\ \ #1\mbox{,  whenever\ } #2} \\}
\newcommand{\noleft}{\left.\kern-\nulldelimiterspace} % so that no space taken by absent left brace


\newcommand{\gatmultiaxiom}[2]
{\multicolumn{2}{l}{
  \noleft
    \begin{array}{l}
		#1
    \end{array} 
  \right\} \mbox{whenever\ } 	#2 
	}\\}
	
	\newcommand{\axid}[1]{\text{#1}.\ }	

%New context sharing macros
\newcommand{\gatintroducing}[1]{
{\arraycolsep=0pt
  \begin{array}{l}
          #1
  \end{array}} &
}

%*********************************
% \begin{\gatgroup}{context}
%    rules
%  \end{\gatgroup}
%*********************************
\NewEnviron{gatgroup}[1]{%
  \noleft
  {\arraycolsep=0pt
   \begin{array}{l}
\BODY
    \end{array} 
   }
   \ \right\} 
	%\mbox{\ whenever\ } 
	#1
	\vspace{0.1cm} 
}
%*********************************

%*********************************
% \begin{\gatgroupnoshared}
%    rule
%  \end{\gatgroupnoshared}
%*********************************
\NewEnviron{gatgroupnoshared}{%
  {\arraycolsep=0pt
   \begin{array}{l}
\BODY
    \end{array} 
   }
   \ 
	\vspace{0.1cm} 
}
%*********************************

% \gatsingular[width]{context}{conclusion}
\newcommand{\gatsingular}[3][4cm]{
\begin{gatgroupnoshared}
\gatleaf[#1]{#2}{#3} 
\end{gatgroupnoshared}
}

%*********************************
% \gatleaf}[width]{context}{assertion}
%*********************************
\newcommand{\gatleaf}[3][4cm]{%
\makebox[#1]{$#3$ \dotfill} \dotfill \  #2
}
%*********************************
%*********************************
% \gatstandalonesingle}{context}{assertion}
%*********************************
\newcommand{\gatstandalonesingle}[2]{%
#2 \makebox[2.5cm]{\dotfill} \  #1
}
%*********************************

% \gataxiomno{axiomno}
\newcommand{\gataxiomno}[1]{\makebox[0.5cm]{} \axid{#1}}


% metagat.macros.tex

%Meta-theories

%\newcommand{\typ}{\triangleright}
\newcommand{\typ}{\nabla}
\newcommand{\trm}{\tau}
\newcommand{\cross}{\otimes}
\newcommand{\sub}{^*}
\newcommand{\diag}{\delta}

\newcommand{\typeseq}[2]
{\ofT{#1_1}{\typ},... \ofT{#1_{#2}}{\typ(#1_{#2-1})}}

\newcommand{\typeseqcont}[3]
{\ofT{#1_1}{\typ({#2})},... \ofT{#1_{#3}}{\typ(#1_{#3-1})}}

\newcommand{\Ob}{Ob}
\newcommand{\obj}{Ob} % <!-- new old --<
\newcommand{\Hom}{Hom}
\newcommand{\objseq}[2]
{\ofT{#1_1}{\obj},... \ofT{#1_{#2}}{\obj(#1_{#2-1})}}


\def\dottededge{\ncline[linestyle=dotted, nodesep=0.3cm]}
\def\noedge{\ncline[linestyle=none]}
\def\thinedge{\ncline[linewidth=0.4pt]}

\newcommand{\member}[1]
{\ncarc[arcangle=-30,nodesepB=0.03]{->}{\pspred}{\pssucc}
\nbput[labelsep=0.1]{#1}}

\newcommand{\loweraccutemember}[1]
{\ncarc[arcangle=-15,nodesepB=0.03]{->}{\pspred}{\pssucc}
\nbput[labelsep=0.05,npos=0.85]{#1}}

\newcommand{\uppermember}[1]
{\ncarc[arcangle=30,nodesepB=0.03]{->}{\pspred}{\pssucc}\naput{#1}}

\newcommand{\upperaccutemember}[1]
{\ncarc[arcangle=10,nodesepB=0.03]{->}{\pspred}{\pssucc}\naput[npos=0.85]{#1}}

% flexbranch 
% #1 node label
% #2 thislevelsep
% #3 next level sep
% #4 variable (eg x)
% #5 index leter (eg n)
% #6 close parenthesis
% #7 continuation branches
\newcommand{\flexbranch}[7]
{
\pstree[thislevelsep=*#2,nodesep=0.05]
		{\Rnode{#1 1}{\Tr{#4_1 #6}}}
	  {\pstree[thislevelsep=#3]  
				   {\Rnode{#1 2}{\Tr[edge=\dottededge]{#4_{#5} #6}}}
					 {#7}
		}
}

\newcommand{\flexbranchplusleaf}[6]
{
\flexbranch{#1}{#2}{#3}{#4} {#5} {#6}
  {
   %\Rnode{#1 3}{\Tr{#4 #6}}
	 \Tr{\Rnode{#1 3}{#4 #6}}
  }
}

\newcommand{\flexbranchplusarc}[7]
{
\flexbranch{#1}{#2}{#3}{#4} {#5} {#6}
  {
   %\Rnode{#1 3}{\Tr{#4 #6}\member{#7}}
	 \Tr{\Rnode{#1 3}{#4 #6}}\member{#7}
  }
}

\newcommand{\flexbranchinitialarc}[9]
{
\pstree[thislevelsep=*#2,nodesep=0.05]
		{\Rnode{#1 1}{\Tr{#4_#8 #6}}#9}
	  {\pstree[thislevelsep=#3]  
				   {\Rnode{#1 2}{\Tr[edge=\dottededge]{#4_{#5} #6}}}
					 {#7}
		}
}

\newcommand{\equality}[2]
{
\ncline [doubleline=true, nodesep=0.2cm]{#1}{#2}
}
\newcommand{\equalityarc}[2]
{
\ncarc [arcangleA=-30, arcangleB=-20, doubleline=true, nodesep=0.1cm]{#1}{#2}
}

\usepackage{imakeidx}
\makeindex[name=definitions, title=Index of Definitions]
\makeindex[name=lemmas, title=Index of Lemmas]



\newcommand{\commentary}[1]{\marginpar{\footnotesize #1}}
\newcommand{\highlight}[1]{\colorbox{orange}{#1}}
\newcommand{\term}[1]{\textit{#1}\commentary{\colorbox{lightgray}{\textit{#1}}}\index[definitions]{#1}}
\newcommand{\llabel}[1]{\label{#1}\commentary{\colorbox{pink}{\scriptsize{#1}}}\index[lemmas]{#1}}
\newcommand{\lref}[1]{\ref{#1}\colorbox{pink}{\scriptsize{#1}}\index[lemmas]{#1!use of}}

\newcommand{\newt}[1]{\colorbox{yellow}{#1}}
\newenvironment{newtt}
{  \colorbox{yellow}{$[$ ...} 
}
{  \colorbox{yellow}{... $]$}
}
\newcommand{\oldt}[1]{\colorbox{yellow}{\sout{#1}}}
\newenvironment{oldtt}
{  \colorbox{red}{$[$ ...} 
}
{  \colorbox{red}{... $]$}
}

\newcommand{\reinstatet}[1]{\colorbox{lime}{#1}}
\newenvironment{reinstatett}
{  \colorbox{lime}{$[$ ...}
}
{  \colorbox{lime}{... $]$}
}

\newcommand{\tbd}{\highlight{TBD}}

%ithprojection function
\newcommand{\proji}[1]{\pi_#1}



\newenvironment{categoricalaside}
{\begin{framed}
\textbf{Categorical Aside}
}
{
\end{framed}
}

\newenvironment{noteforfuture}
{\begin{framed}
\textbf{Note For Future}
}
{
\end{framed}
}

\newenvironment{problem}
{\begin{framed}
\textbf{Problem}
}
{
\end{framed}
}

%quine quote
\newcommand{\qq}[1]{
\left\ulcorner#1\right\urcorner
}

%single quote
\newcommand{\sq}[1]{
\textnormal{\textquotesingle}#1\textnormal{\textquotesingle}
}

%lower quine quote
\newcommand{\lqq}[1]{
\left\llcorner #1\right\lrcorner
}


%from berkley
\newcommand{\langl}{\begin{picture}(4.5,7)
\put(1.1,2.5){\rotatebox{60}{\line(1,0){5.5}}}
\put(1.1,2.5){\rotatebox{300}{\line(1,0){5.5}}}
\end{picture}}
\newcommand{\rangl}{\begin{picture}(4.5,7)
\put(.9,2.5){\rotatebox{120}{\line(1,0){5.5}}}
\put(.9,2.5){\rotatebox{240}{\line(1,0){5.5}}}
\end{picture}}
\newcommand{\lang}{\begin{picture}(5,7)\put(1.1,2.5){\rotatebox{45}{\line(1,0){6.0}}}\put(1.1,2.5){\rotatebox{315}{\line(1,0){6.0}}}\end{picture}}
\newcommand{\rang}{\begin{picture}(5,7)\put(.1,2.5){\rotatebox{135}{\line(1,0){6.0}}}\put(.1,2.5){\rotatebox{225}{\line(1,0){6.0}}}\end{picture}}
%Try sharper tuple brackets -- except gives errors nested in captions so comment out
%\renewcommand{\tuple}[1]{\lang #1 \rang}

\newcommand{\setsuchthat}[2]{\left\{#1 \ \middle|\ #2\right\}}
\newcommand{\set}[1]{\left\{#1\right\}} 

% one to n - wanton
\newcommand{\wanton}[1]{#1_1,...#1_n}
\newcommand{\fn}{\wanton{f}}
\newcommand{\pn}{\wanton{p}}
\newcommand{\qn}{\wanton{q}}
\newcommand{\qnprime}{\wanton{q'}}
\newcommand{\xn}{\wanton{x}}
\newcommand{\xnp}{\wanton{x'}}
\newcommand{\yn}{\wanton{y}}
\newcommand{\ntuple}[1]{\tuple{\wanton{#1}}}
\newcommand{\wantom}[1]{#1_1,...#1_m}
\newcommand{\mtuple}[1]{\tuple{#1_1,...#1_m}}
\newcommand{\qm}{\wantom{q}}
\newcommand{\ym}{\wantom{y}}
\newcommand {\bntuple}{\ensuremath{\ntuple{b}}}
\newcommand {\fntuple}{\ensuremath{\ntuple{f}}}
\newcommand {\fnptuple}{\ensuremath{\ntuple{f}}}
\newcommand {\pntuple}{\ensuremath{\ntuple{p}}}
\newcommand {\qntuple}{\ensuremath{\ntuple{q}}}
\newcommand {\qnptuple}{\ensuremath{\ntuple{q'}}}
\newcommand {\qmtuple}{\ensuremath{\mtuple{q}}}
\newcommand {\sntuple}{\ensuremath{\ntuple{s}}}
\newcommand {\xntuple}{\ensuremath{\ntuple{x}}}
\newcommand {\xnptuple}{\ensuremath{\ntuple{x'}}}
\newcommand {\ymtuple}{\ensuremath{\mtuple{y}}}
\newcommand{\foreachi}[1][n]{for each $i$, $1 \leq i \leq #1$}
\newcommand{\foreachj}[1][m]{for each $j$, $1 \leq j \leq #1$}
\newcommand{\foreachk}[1][l]{for each $k$, $1 \leq k \leq #1$}



\usepackage{mathtools} %for \bigtimes
\title{Notation - supporting macros}
\author{John Cartmell}

\begin{document}
\maketitle


\section{gats.macros.tex}


\subsection{Basics}
\begin{tabular}{| l | l |p{3cm}|}
 \hline 
 \verb'\gat[U]' & \gat[U] & \\
 \hline
 \verb'\isagat[U]' & \isagat[U] \\
 \hline
 \verb!\tstyle!               &   $\tstyle$ & \\
 \hline
 \verb!\isT{X}!            &   $\isT{X}$       &  \\
 \hline
 \verb!\ofT{x}{X}!         &   $\ofT{x}{X}$     & \\
 \hline
 \verb!\context{x}{X}{n}!     &   $\context{x}{X}{n}$ & \\
 \hline
 \hline
 \verb!\subcontext{x}{\Delta}{i}{k}!&$\subcontext{x}{\Delta}{i}{k}$ & \\
 \hline
	\verb!\con{x}{\Delta}{n}!&$\con{x}{\Delta}{n}$ & schematic context\\
	\hline
 \verb!\subon{x}{\Delta}{i}{k}!&$\subcon{x}{\Delta}{i}{k}$ & schematic subcontext\\
 \hline
	\verb!\permcon{x}{\Delta}{n}{\sigma}!&$\permcon{x}{\Delta}{n}{\sigma}$& permuted context\\
 \hline
  \verb!\permterm{t}{n}{\sigma}!&$\permterm{t}{n}{\sigma}$& \\
	\hline
\end{tabular} \\

\vspace{0.5cm}
\noindent For example : \\

\verb!\context{x}{X}{n} \ttsyle \isT{X_n}!   
gives \\
\begin{equation}
\context{x}{X}{n} \tstyle \isT{X}
\end{equation}

\subsection{Macros For Presenting Theories}

The \verb'gatrules' environment frames the entire theory and \verb'\gatintros' and
\verb'\gataxioms' commands enter headers for introductory rules and axioms, respectively.

\begin{verbatim}
\begin{gatrules}
\gatintros                          % generates the 'Symbol Introductory Rule' header
blah blah blah   \\                   
\gataxioms                          % generates the 'Axioms' header
blah blah blah   \\
\end{gatrules}
\end{verbatim}
gives
\begin{gatrules}
\gatintros                          % generates the 'Symbol Introductory Rule' header
blah blah blah   \\                   
\gataxioms                          % generates the 'Axioms' header
blah blah blah   \\
\end{gatrules}

\subsection{Group dotfill style}

An environment \verb'\gatgroup' and a command \verb'\gatleaf'  can be used to hierarchically group together rules which share part or all of their context. The command \verb'\gatintroducing' declares what is being defined in the group. Its argument is a newline separated list of symbols or a newline separated list of axiom numbers.

For example:
\begin{verbatim}
\begin{gatrules}
\gatintros
\gatintroducing{B\\C\\D}
\begin{gatgroup}{\ofT{x}{A}}
\gatleaf{}{\isT{B(x)}} \\
\begin{gatgroup}{\ofT{y}{B(x)}}
\gatleaf{}{\isT{C(x,y}} \\
\gatleaf{\ofT{z}{C(x,y)}}{\isT{D(x,y,z)}} \\
\end{gatgroup}
\end{gatgroup}
\end{gatrules}
\end{verbatim}

gives 

\begin{gatrules}
\gatintros
\gatintroducing{B\\C\\D}
\begin{gatgroup}{\ofT{x}{A}}
\gatleaf{}{\isT{B(x)}} \\
\begin{gatgroup}{\ofT{y}{B(x)}}
\gatleaf{}{\isT{C(x,y}} \\
\gatleaf{\ofT{z}{C(x,y)}}{\isT{D(x,y,z)}} \\
\end{gatgroup}
\end{gatgroup}
\end{gatrules}

For ungrouped rules the \verb'\gatsingular' command is used when there is a context which is to
be places right and dotfilled and no command at all is required when no context is required.

\begin{verbatim}
\begin{gatrules}
\gatintros                           
\gatintroducing{Ob}                  
  \isT{Ob} \\    
\gatintroducing{Hom}
  \gatsingular{\ofT{x,y}{Ob}}{\isT{Hom(x,y)}} \\										
\end{gatrules}
\end{verbatim}

gives

\begin{gatrules}
\gatintros                           
\gatintroducing{Ob}                  
  \isT{Ob} \\    
\gatintroducing{Hom}
  \gatsingular{\ofT{x,y}{Ob}}{\isT{Hom(x,y)}} \\										
\end{gatrules}

Axioms work use the same set of commands. In addition there is an \verb'\axiomno' command that introduces
whitespace before the axiom number.

For example
\begin{verbatim}
\begin{gatrules}
\gataxioms
\gatintroducing{  \gataxiomno{3} \\   \gataxiomno{4}}
\begin{gatgroup}{\ofT{f}{Hom(x,y)},\ \ofT{x,y}{Ob}}
    \gatleaf{}{id_x \circ f = f} \\
    \gatleaf{}{f \circ id_y = f}
\end{gatgroup} \\
\end{gatrules}
\end{verbatim}

gives

\begin{gatrules}
\gataxioms
\gatintroducing{  \gataxiomno{3} \\   \gataxiomno{4}}
\begin{gatgroup}{\ofT{f}{Hom(x,y)},\ \ofT{x,y}{Ob}}
    \gatleaf{}{id_x \circ f = f} \\
    \gatleaf{}{f \circ id_y = f}
\end{gatgroup} \\
\end{gatrules}
 
\subsection{Macros for Rules in MetaGAT}


\begin{tabular}{| l | l |}
 \hline
 \verb!\typ!         &   $\typ$    \\
 \hline
 \verb!\trm!         &   $\trm$     \\
 \hline
 \verb!\typeseq{x}{n}!  & $\typeseq{x}{n}$ \\
 \hline
 \verb!\typeseqcont{y}{x}{m}! & $\typeseqcont{y}{x}{m}$ \\
 \hline
\end{tabular}

\subsection{Idioms}
\begin{tabular}{| l | l |p{3cm}|}
\hline
\verb!\xDelta{i}! & $\xDelta{i}$&\\
 \hline
\verb!\xDeltap{i}! & $\xDeltap{i}$&\\
 \hline
\verb!\xOmega[i]! & $\xOmega{i}$&\\
 \hline
\verb!\xOmegap{i}! & $\xOmegap{i}$&\\
 \hline
\verb!\yOmega[i]! & $\yOmega{i}$&\\
 \hline
\verb!\yOmegap{i}! & $\yOmegap{i}$&\\
 \hline
\hline
\verb!\xDeltasigma{i}! & $\xDeltasigma{i}$&\\
\hline
\verb!\xDeltapsigma{i}! & $\xDeltapsigma{i}$&\\
 \hline
\verb!\xOmegasigma{i}! & $\xOmegasigma{i}$&\\
 \hline
\verb!\xOmegapsigma{i}! & $\xOmegapsigma{i}$&\\
 \hline
\verb!\yOmegasigma{i}! & $\yOmegasigma{i}$&\\
 \hline
\verb!\yOmegapsigma{i}! & $\yOmegapsigma{i}$&\\
 \hline
\hline
\verb!\xDeltainvsigma{i}! & $\xDeltainvsigma{i}$&\\
 \hline
\verb!\xOmegainvsigma{i}! & $\xOmegainvsigma{i}$&\\
\hline
\verb!\xDeltapinvsigma{i}! & $\xDeltapinvsigma{i}$&\\
 \hline
\verb!\xOmegainvsigma{i}! & $\xOmegainvsigma{i}$&\\
 \hline
\verb!\xOmegapinvsigma{i}! & $\xOmegapinvsigma{i}$&\\
 \hline
\verb!\yOmegainvsigma{i}! & $\yOmegainvsigma{i}$&\\
 \hline
\verb!\yOmegapinvsigma{i}! & $\yOmegapinvsigma{i}$&\\
 \hline
\end{tabular} \\

\subsection{Idioms Enclosed as Tuples}
\begin{tabular}{| l | l |p{3cm}|}
\hline
\verb!\encxDelta{i}! & $\encxDelta{i}$&\\
 \hline
\verb!\encxDeltap{i}! & $\encxDeltap{i}$&\\
 \hline
\verb!\encxOmega{i}! & $\encxOmega{i}$&\\
 \hline
\verb!\encxOmegap{i}! & $\encxOmegap{i}$&\\
 \hline
\verb!\encyOmega{i}! & $\encyOmega{i}$&\\
 \hline
\verb!\encyOmegap{i}! & $\encyOmegap{i}$&\\
 \hline
\hline
\verb!\encxDeltasigma{i}! & $\encxDeltasigma{i}$&\\
\hline
\verb!\encxDeltapsigma{i}! & $\encxDeltapsigma{i}$&\\
 \hline
\verb!\encxOmegasigma{i}! & $\encxOmegasigma{i}$&\\
 \hline
\verb!\encxOmegapsigma{i}! & $\encxOmegapsigma{i}$&\\
 \hline
\verb!\encyOmegasigma{i}! & $\encyOmegasigma{i}$&\\
 \hline
\verb!\encyOmegapsigma{i}! & $\encyOmegapsigma{i}$&\\
 \hline
\hline
\verb!\encxDeltainvsigma{i}! & $\encxDeltainvsigma{i}$&\\
 \hline
\verb!\encxOmegainvsigma{i}! & $\encxOmegainvsigma{i}$&\\
\hline
\verb!\encxDeltapinvsigma{i}! & $\encxDeltapinvsigma{i}$&\\
 \hline
\verb!\encxOmegainvsigma{i}! & $\encxOmegainvsigma{i}$&\\
 \hline
\verb!\encxOmegapinvsigma{i}! & $\encxOmegapinvsigma{i}$&\\
 \hline
\verb!\encyOmegainvsigma{i}! & $\encyOmegainvsigma{i}$&\\
 \hline
\verb!\encyOmegapinvsigma{i}! & $\encyOmegapinvsigma{i}$&\\
 \hline
\end{tabular} \\

\section {general.macros}

\subsection{ note macro}

Here is a paragraph before hand. Because it is the first paragrpah it is not indented.

And here a second paragraph which is indented and following this we use the \verb'\note' command twice over to get indented numbered paragraphs:
\note Here is the first note.
\note Here is the second note. This will tend to merge into the paragraph that follows unless the content is sufficeint to get onto a second (unindented) line.

Finally here we revert to a normal blank line introduced paragraph.

\subsection{Idioms}

\begin{tabular}{|l|p{4cm}|}
\hline
\verb!\setsuchthat{x}{condition}! & $\setsuchthat{x}{condition}$\\
\verb!\set{suchandsuch}!&$\set{suchandsuch}$\\
\verb!\qq{x} ! & $\qq{x}$ (quine quote)\\
\verb!\sq{x} ! & $\sq{x}$ (single quote)\\
\verb!\lqq{x }! & $\lqq{x}$ (lower quine quote)\\
\verb!\n  !&$   \n $\\
\verb!\fn !&$   \fn$\\
\verb!\gn !&$   \gn$\\
\verb!\pn !&$   \pn$\\
\verb!\qn !&$   \qn$\\
\verb!\qnp !&$  \qnprime$\\
\verb!\tn !&$   \tn$\\
\verb!\xn !&$   \xn$\\
\verb!\xnp !&$  \xnp$\\
\verb!\yn !&$   \yn$\\
\verb!\An !&$   \An$\\
\verb!\Bn !&$   \Bn$\\
\verb!\Bn !&$   \Cn$\\
\verb!\m  !&$   \m $\\
\verb!\gm !&$   \gm$\\
\verb!\qm !&$   \qm$\\
\verb!\sm !&$   \sm$\\
\verb!\ym !&$   \ym$\\
\verb!\Bm !&$   \Bm$\\
\verb!\wanton{x} !&$ \wanton{x}$\\
\verb!\wantom{x} !&$ \wantom{x}$\\
\verb!\ntuple{x} !&$ \ntuple{x}$\\
\verb!\mtuple{x} !&$ \mtuple{x}$\\
\verb!\bntuple !&$   \bntuple$\\
\verb!\fntuple !&$   \fntuple$\\
\verb!\fnptuple !&$   \fnptuple$\\
\verb!\pntuple !&$   \pntuple$\\
\verb!\qntuple !&$   \qntuple$\\
\verb!\qnptuple !&$   \qnptuple$\\
\verb!\qmtuple !&$   \qmtuple$\\
\verb!\sntuple !&$   \sntuple$\\
\verb!\xntuple !&$   \xntuple$\\
\verb!\xnptuple !&$   \xnptuple$\\
\verb!\ymtuple !&$   \ymtuple$\\
\verb!\foreachi  !&\foreachi\\
\verb!\foreachi[nn]  !&\foreachi[nn]\\
\verb!\foreachj  !&  \foreachj\\
\verb!\foreachj[mm]  !&\foreachj[mm]\\
\verb!\foreachk  !&  \foreachk\\
\verb!\foreachk[ll]  !&\foreachk[ll] \\
\verb!\idef           !&$\idef$\\
\verb!\idef[nn]       !&$\idef[nn]$\\
\verb!\jdef           !&$\jdef$\\
\verb!\jdef[mm]       !&$\jdef[mm]$\\
\verb!\kdef           !&$\kdef$\\
\verb!\kdef[ll]       !&$\kdef[ll]$\\
\hline
\end{tabular} \\

Remember  \verb!\idef! as \textit{i idefnition}. It defaults to $\idef$. 




\subsection{Markup Commands}

\begin{tabular}{|l|p{2cm}|p{6cm}|}
\hline
\verb!\commentary{blah blah} !&       &\textit{places comment in right hand margin -- has to be in outer par mode     }     \\
\verb!\highlight {blah blah} !& \highlight{blah blah} &                                                                     \\
\verb!\term{term}            !&                       &\textit{ italicises and adds to index of definitions}                \\
\verb!\llabel{label}         !&                       & \textit{defines label and makes visible has to be in outer par mode}\\
\verb!\lref{label}           !&  \lref{label}         &                                              \\
\verb!\newt{text}            !& \newt{text}       &\textit{also available as environment newtt}       \\
\verb!\oldt{text}            !& \oldt{text}       &\textit{also available as environment oldtt}       \\
\verb!\reinstatet{text}      !& \reinstatet{text} &\textit{also available as environment reinstatett} \\
\verb!\tbd                   !& \tbd              &                                                   \\             
\hline
\end{tabular}

\section{Framing Environments}

\begin{notebox}
Environment notebox has one optional parameter for its title. The title defaults to \textbf {Note}.
\end{notebox}

\begin{notebox}[Example]
Example of using \verb!\begin{notebox}[Example]...\end{notebox}!
\end{notebox}



\section {cccategories.macros }
\vspace{5mm}
\subsection{General}

\begin{tabular}{|l|p{4cm}|p{6cm}|}
\hline
\verb!\cat{C}!       & \cat{C}            & a category                            \\
\verb!\catc,!        & \catc,                                                    \\
\verb!\catcw and!    & \catcw and                                                \\
\verb!\catcp!        & \catcp                                                     \\
\verb!\catcpp!       & \catcpp                                                    \\
\verb!\obj{C}!       & \obj{C}            & objects of category C                          \\
\verb!\ccat[C]!      & \ccat              & a contextual category                          \\
\verb!\cobj{C}!      & \cobj{C}           & objects of contextual category C  \\
\verb!\cobj[i]{C}!   & \cobj[i]{C}        & objects of contextual category C [at height i] \\
\verb!\ccatc[C]!     & \ccatc[C]          &                                                \\
\verb!\cslice{C}{A}! & \cslice{C}{A}      & contextual slice category                      \\
\verb!\csliceobj[i]{C}{A}!&\csliceobj[i]{C}{A} & objects [at height i] of a contextual slice category                      \\
\verb!\Fam!          & \Fam               & contextual category of sets, families of sets etc. \\
\verb!\Fin!          & \Fin               & category of finite sets                            \\
\verb!\Finp!         & \Finp              & category of finite sets and partial functions      \\
\verb!\Po!           & \Po                & category of partial ordered sets?              \\
\verb!\Famslice{A}!  & \Famslice{A}       & slice of Fam                                   \\
\verb!\Famobj[i]!    & \Famobj[i]         & objects of Fam [at height i]                   \\
\verb!\Famsliceobj[i]{A}!& \Famsliceobj[i]{A} & objects of slice of Fam [at height i]          \\
\verb!A \morph B! & $A \morph B $     &\\
\verb!A \epi B  ! & $A \epi B$        &\\ 
\verb!A \base B! & $A \base B$        &\\
\verb!f \comp g !& $ f \comp g$       &\\
\verb!\pc{B}{A} !& $ \pc{B}{A}$       &\\
\verb!\product{n}  !& $ \product{n} $       &\\
\verb!\productn    !& $ \productn $       &\\
\verb!A \cross B!&$A \cross B$        &\\
\verb!\crossx{A}{B}{X}!& $\crossx{A}{B}{X}$ &\\
\verb!f \sub B!&$f \sub B$                  &\\
\verb!\diag_A!&$\diag_A$                    &\\
\verb!\pbase{A}!&$\pbase{A}$                &\\
\verb!\tuple{x,y,z}!& $\tuple{x,y,z}$       &\\
\verb!\ndidly!& $\ndidly$                   &\\
\verb!\ndidlycospan!& ?????????????       &\\
\verb!\fibrex{B_1}{B_2}{D}!& $\fibrex{B_1}{B_2}{D}$  &\\
\verb!A \smorph B!& $A \smorph B$                    &\\
\verb!A \dmorph B!& $A \dmorph B$                    &\\
\verb!\primeds{A}!& $\primeds{A}$                    &\\
\hline
\end{tabular}

\subsection {pstricks macros for arrows, s-arrows and d-arrows}
The following macros supplement the node connection macros (prefix 'nc') from pstricks and are implemented by combining pstricks 'nc' macros and put macros naput (put above) and nbput (put below). Arrowheads are either open or open triangles (representing S-morphisms). \\
%\begin{table}[h]
%	\centering
		\begin{tabular}{|l |  p{5cm} | c |}
		\hline
		\multicolumn{3}{|c|}{pstricks macros for arrows and s-arrows} \\
		   \hline	   
		   \verb!\ncarr[ang]{A}{B}! &  an arrow between nodes A and B optionally an arc of angle ang (default 0)
			&
			\\
		   \hline
		 	 \verb!\jcbarr[ang]{f}{A}{B}! & an arrow f between nodes A and B labelled below optionally an arc of angle ang (default 0)
			&
			\\
		 	 \hline
		 	 \verb!\ncaarr[ang]{f}{A}{B}! & an arrow f between nodes A and B labelled above optionally an arc of angle ang (default 0)
			&
			\\
			\hline	   
		   \verb!\ncsar[ang]{A}{B}! &  an S-arrow between nodes A and B optionally an arc of angle ang (default 0)
			&
			\raisebox{-0.6cm}{$
\begin{array} {c}
\Rnode{A}{A} \\ [0.6cm]
\Rnode{B}{B}
\end{array}$\ncsar{A}{B}} \\
		 	 \hline			
		 	 \verb!\ncbsar[ang]{f}{A}{B}! & an s-arrow f between nodes A and B labelled below optionally an arc of angle ang (default 0)
			&
			\\
		 	 \hline
		 	 \verb!\ncasar[ang]{f}{A}{B}! & an s-arrow f between nodes A and B labelled above optionally an arc of angle ang (default 0)
			&
			\\
						\hline	   
		   \verb!\nccdar[ang]{A}{B}! &  a composite dependency arrow between nodes A and B optionally an arc of angle ang (default 0)
			&
			\raisebox{-0.75cm}{$
\begin{array} {c}
\Rnode{A}{A} \\ [0.75cm]
\Rnode{B}{B}
\end{array}$\nccdar{A}{B}} \\
		 	 \hline			
		 	 \verb!\ncbcdar[ang]{f}{A}{B}! & a composite dependency arrow f between nodes A and B labelled below optionally an arc of angle ang (default 0)
			&
			\raisebox{-0.75cm}{$
\begin{array} {c}
\Rnode{A}{A} \\ [0.75cm]
\Rnode{B}{B}
\end{array}$\ncbcdar{f}{A}{B}}
			\\
			\hline
		 	 \verb!\ncacdar[ang]{f}{A}{B}! & a composite dependency arrow f between nodes A and B labelled above optionally an arc of angle ang (default 0)
						&
			\raisebox{-0.75cm}{$
\begin{array} {c}
\Rnode{A}{A} \\ [0.75cm]
\Rnode{B}{B}
\end{array}$\ncacdar{f}{A}{B}}
			\\
		 	 \hline	   
		   \verb!\nrsarr{A}{B}! &  a recursive S-arrow between nodes A and B i.e one that loops around
			&
			\\ 
			\hline		
		\end{tabular}
%\end{table}
\vspace{0.5cm}

\noindent For example:
\begin{verbatim}
\begin{center}
$
\begin{array}{p{2cm}p{0.5cm}p{0.5cm}p{0.5cm}}
& & \Rnode{X}{X}& \\ [1.4cm]
\Rnode{A}{A} & \Rnode{B1}{$B_1$} & ... & \Rnode{Bm}{$B_m$}\\
\end{array}
$
\jcbarr{f}{A}{B1}
\ncbsar{b_1}{X}{B1}
\ncasar{b_m}{X}{Bm}
\end{center}
\end{verbatim}

\noindent Results in:
\begin{center}
$
\begin{array}{p{2cm}p{0.5cm}p{0.5cm}p{0.5cm}}
& & \Rnode{X}{X}& \\ [1.4cm]
\Rnode{A}{A} & \Rnode{B1}{$B_{1}$} & ... & \Rnode{Bm}{$B_{m}$}\\
\end{array}
$
\jcbarr{f}{A}{B1}
\ncbsar{b_1}{X}{B1}
\ncasar{b_m}{X}{Bm}
\end{center}
\noindent The optional ang parameter can be used. Replace the s-morphisms in the example by
\begin{verbatim}
\ncbsar[-30]{$b_1$}{X}{B1}
\ncasar[+30]{$b_m$}{X}{Bm}
\end{verbatim}
we get:
\begin{center}
$
\begin{array}{p{2cm}p{0.5cm}p{0.5cm}p{0.5cm}}
& & \Rnode{X}{X}& \\ [1.4cm]
\Rnode{A}{A} & \Rnode{B1}{$B_{1}$} & ... & \Rnode{Bm}{$B_{m}$}\\
\end{array}
$
\jcbarr{f}{A}{B1}
\ncbsar[-30]{b_1}{X}{B1}
\ncasar[30]{b_m}{X}{Bm}
\end{center}
\noindent The following lengths can be set using \\setlength command to control the way that the S-arrows are drawn:
\begin{verbatim}
\sarnodesepA
\sarnodesepB
\saroffsetA
\saroffsetB
\end{verbatim}
For example:
\begin{verbatim}
\ncbsar[-30]{$b_1$}{X}{B1}
\setlength {\saroffsetB}{-4pt}
\ncasar[30]{$b_m$}{X}{Bm}
\end{verbatim}
\indent results in the following adjustment:
\begin{center}
$
\begin{array}{p{2cm}p{0.5cm}p{0.5cm}p{0.5cm}}
& & \Rnode{X}{X}& \\ [1.4cm]
\Rnode{A}{A} & \Rnode{B1}{$B_{1}$} & ... & \Rnode{Bm}{$B_{m}$}\\
\end{array}
$
\jcbarr{f}{A}{B1}
\ncbsar[-30]{b_1}{X}{B1}
\setlength {\saroffsetB}{-4pt}
\ncasar[30]{b_m}{X}{Bm}
\end{center}

\subsubsection {Recursive S-arrow}

\noindent
Use macro \verb'\ncrsar', for example:
\begin{verbatim}
\begin{center}
$
\begin{array}{c c}
\Rnode{abs}{1}  \\ [1.4cm]
\Rnode{S0}{S_0} \\ [1.4cm]
\Rnode{SR}{S_R} \\ [1.4cm]
\end{array}
$
\ncsar{S0}{abs}
\ncsar{SR}{S0}
\ncrsar{SR}{SR}
\end{center}
\end{verbatim}
gives:

\begin{center}
$
\begin{array}{c c}
\Rnode{abs}{1}  \\ [1.4cm]
\Rnode{S0}{S_0} \\ [1.4cm]
\Rnode{SR}{S_R} \\ [1.4cm]
\end{array}
$
\ncsar{S0}{abs}
\ncsar{SR}{S0}
\ncrsar{SR}{SR}
\end{center}


\subsection {multi-arrows}

\noindent Supplemental pstricks style macros for multi-arrows:


		\begin{tabular}{|l |  p{5cm} | }
		   \hline
		   \verb!\ncmarr{A1}{A2}{J}{B}! & multi-arrow with two domains \\
		    \hline
		   \verb!\ncamarr{f}{A1}{A2}{J}{B}! & multi-arrow with two domains and label above \\
		   \hline
		 	 \verb!\ncbmarr{f}{A1}{A2}{J}{B}! & multi-arrow with two domains and label below \\
		 	 \hline
		 	 \verb!\ncmarrr{A1}{A2}{A3}{J}{B}! & multi-arrow with three domains \\
		 	 \hline
		   \verb!\ncamarrr{f}{A1}{A2}{A3}{J}{B}! & multi-arrow with three domains and label above \\
		   \hline
		 	 \verb!\ncbmarrr{f}{A1}{A2}{A3}{J}{B}! & multi-arrow with three domains and label below \\
		 	 \hline
		\end{tabular}

Arcs of angle 8 are used for the branches leading from the domains to the junction point J and for the arrow from the junction J to the codomain.
Two optional parameters allow these to be overridden.
In the following example there is are call \verb!\ncamarr{f}{A1}{A2}{J1}{B1}{f}! and call
\verb!\ncbmarrr{g}{A1}{A2}{A3}{J2}{B2}! :
\setlength{\arraycolsep}{1cm}
\begin{center}
$
\begin{array}{cccc}
\Rnode{A1}{A_1}& & \\
                & \Rnode{J1}{}& \Rnode{B1}{B_1} \\
\Rnode{A2}{A_2}& & \\
                &  \Rnode{J2}{}& \Rnode{B2}{B_2} \\
\Rnode{A3}{A_3}& & \\
\end{array}
$
\ncamarr{f}{A1}{A2}{J1}{B1}
\ncbmarrr{g}{A1}{A2}{A3}{J2}{B2}

\end{center}
\setlength{\arraycolsep}{.2cm}

\subsection{Macros for crowsfeet arrow notation}

In the general case  functions  are many-one functional relationship and can be represented as shown in the following examples.
General form of these commands: \verb'command[angleA][midpointangle]{startnode}{endnode}'.
\begin{center}
\begin{tabular} {c p{5cm}}
\verb'angleA' & similar to like named parameter to \verb'ncarc' command. Angle that arrow leaves the start node relative to a line drwn to the destination. Except\\
\verb'midpointangle' & This parameter needs to be supplied as the angle between start and end node since i cannot calculate it. Defaults to zero. If this parameter is
                        wrongly gicven then results are rubbish.
\end{tabular}
\end{center}


\begin{center}
\begin{tabular}{c p{1.2cm} c p{0.5cm} l p{1.5cm} | p{1.5cm} c p{0.5cm} l}
   \Rnode{a}{a}   & & \Rnode{b}{b}   & \verb'\Et{a}{b}'    \\ [1.2cm]
   \Rnode{a1}{a1} & & \Rnode{b1}{b1} & \verb'\Et[20]{a1}{b1}'  \\ 
	                & &                & \verb'\alabel{R}[0.3]'  \\ [1.2cm]
   \Rnode{a2}{a2} & & \Rnode{b2}{b2} & \verb'\Et[-20]{a2}{b2}' \\ 
	                & &                & \verb'\blabel{R}[0.3]'  \\ [1.2cm]
   \Rnode{a3}{a}  & & \Rnode{b3}{b}  & \verb'\Et[-40]{a3}{b3}' \\ 
	                & &                & \verb'\blabel{R}[0.3]'  \\ [1.2cm]
\end{tabular}
\Et{a}{b}
\alabel{R}[0.3]
\Et[20]{a1}{b1}
\alabel{R}[0.3]
\Et[-20][0]{a2}{b2}
\blabel{R}[0.3]
\Et[-40]{a3}{b3}
\blabel{R}[0.3]
\end{center}


In the command name Et E is for Edge and t is for total.
Other abbreviations p(artial), m(ono), e(pi) and r(recursive) yields the following commands
ann of which have an optional angle parameter:
\vspace{0.5cm}

\begin{center}
\begin{tabular}{c p{1.2cm} c p{0.5cm} l p{1.5cm} | p{1.5cm} c p{0.5cm} l }
   \Rnode{a1}{a} & & \Rnode{b1}{b} & & Ep   & & & \Rnode{x1}{x} & & rEp   \\ [1.2cm]
   \Rnode{a2}{a} & & \Rnode{b2}{b} & & Epm  & & & \Rnode{x2}{x} & & rEpm  \\ [1.2cm]
   \Rnode{a3}{a} & & \Rnode{b3}{b} & & Epe  & & & \Rnode{x3}{x} & & rEpe  \\ [1.2cm]
   \Rnode{a4}{a} & & \Rnode{b4}{b} & & Epme & & & \Rnode{x4}{x} & & rEpme \\ [1.2cm]
   \Rnode{a5}{a} & & \Rnode{b5}{b} & & Et   & & & \Rnode{x5}{x} & & rEt   \\ [1.2cm]
   \Rnode{a6}{a} & & \Rnode{b6}{b} & & Etm  & & & \Rnode{x6}{x} & & rEtm  \\ [1.2cm]
   \Rnode{a7}{a} & & \Rnode{b7}{b} & & Ete  & & & \Rnode{x7}{x} & & rEte  \\ [1.2cm]
   \Rnode{a8}{a} & & \Rnode{b8}{b} & & Etme & & & \Rnode{x8}{x} & & rEtme
\end{tabular}
\Ep[20]{a1}{b1}
\alabel{R}[0.3]
\Epm[20]{a2}{b2}
\alabel{R}[0.3]
\Epe[20]{a3}{b3}
\alabel{R}[0.3]
\Epme[20]{a4}{b4}
\alabel{R}[0.3]
\Et[20]{a5}{b5}
\alabel{R}[0.3]
\Etm[20]{a6}{b6}
\alabel{R}[0.3]
\Ete[20]{a7}{b7}
\alabel{R}[0.3]
\Etme[20]{a8}{b8}
\alabel{R}[0.3]
\rEp[270]{x1} 
\alabel{R}[0.3]
\rEpm[270]{x2}
\alabel{R}[0.3]
\rEpe[270]{x3}
\alabel{R}[0.3]
\rEpme[270]{x4}
\alabel{R}[0.3]
\rEt[270]{x5}
\alabel{R}[0.3]
\rEtm[270]{x6}
\alabel{R}[0.3]
\rEte[270]{x7}
\alabel{R}[0.3]
\rEtme[270]{x8}
\alabel{R}[0.3]
\end{center} 

\vspace{0.5cm}


The recursive commands are used as shown in the following examples:
\begin{center}
\begin{tabular}{l p{.35cm} | p{.35cm} l p{.35cm} | p{.35cm} l p{.35cm} | p{.35cm} l }
   \hspace{.75cm} \Rnode{x1}{$x_1$}    & & & \hspace{.75cm} \Rnode{x2}{$x_2$}    & & &  \hspace{.75cm} \Rnode{x3}{$x_3$}    & & &  \hspace{.75cm} $\Rnode{x4}{x}_4$ \\ [0.5cm]
	 \verb'\rEt[180]{x1}' & & & \verb'\rEt[210]{x2}' & & & \verb'\rEt[240]{x3}'  & & & \verb'\rEt[270]{x4}' \\
	 \verb'\alabel{f}'    & & & \verb'\alabel{f}'    & & & \verb'\alabel{f}'     & & & \verb'\alabel{f}' 
\end{tabular}
\rEt[180]{x1}
\alabel{f}
\rEt[210]{x2}
\alabel{f}
\rEt[240]{x3}
\alabel{f}
\rEt[270]{x4}
\alabel{f}
\end{center}




\end{document}
