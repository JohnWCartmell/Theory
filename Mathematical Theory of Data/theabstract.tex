
\begin{abstract}
We present a formal and abstract definition  of a certain kind of data specification that is sufficiently general to encompass  
(i) relational schemas of Codd's relational data model
(ii) hierarchical schemas such as (a) those of the nested relational data model (b) those such as are commonly implemented in XML and represented in XML schemas and (c) those represented in variants of Interface Definition Language (IDL) and represented in various open and propriatary formats most recently 
in Google's protocol buffer format,
(iii) entity relationship (ER) models in the binary-relationship style as described by Barker [???] and others [Rosemary Rock-Evans]  and as implemented, for instance, in Oracle's SQL developer tool. 
We call such data specifications ER models.

We define  goodness criteria for such ER models that generalise 
the normal form criteria of relational data modelling and we define sufficient conditions under which an ER model can be transformed into a relational schema in an appropriate normal form.  In doing so we provide a theoretical basis for the elimination of the normalisation step from the relational design workflow 
in favour of an emphasis on goodness criteria for entity relationship models in the binary-relationship style.  Historically, this style came about as a practical alternative to the entity-relationship model  presented by Chen as a unified model of data; this paper provides substance  to
Chen's idea of a unified model of data but exceeeds his ambition by not requiring a separate normalisation step in relational design. In this way this paper points the way to a significant improvement in software life-cycle methodology.

This \texttt{<<}\today \texttt{>>} version of the paper covers 
Elementary Key Normal Form (EKNF), and therefore Third Normal Form (3NF), and Boyce-Codd Normal Form (BCNF).  A future version will also cover achieving 4th and 5th normal forms.
\end{abstract}
