
\section{Construction of Restriction Functors}

Denote by \SetP the range category of sets and partial functions
 and let
\FinPar be the range category of finite sets and partial functions. 

\subsection{The partial Hom functor $HomP$}
If \catcw is a category with support then \commentary{Tidy this definition please.}
let $HomP_\catc(a,-): \catc \morph \SetP$ be the  functor
defined as follows. $HomP(a,x)=Hom(a,x)$.
If $j: x \morph y$ then partial function $HomP(a,j): HomP(a,x) \morph HomP(a,y)$
is defined\footnote{
In earlier notes I defined HomP as follows.
\begin{align*}
HomP(a,j)(k) &= k \circ j &&\mbox{ if $k \circ \bar{j} = k$,} \\
             &\bot        && \mbox{otherwise.}
\end{align*}
In a restriction category this is defintion is equivalent to the one given but not in an arbitrary category with support.
} as follows:

\begin{align*}
HomP(a,j)(k) &= k \circ j  &&\mbox{ if $\overline{k \circ j} = \overline{k}$,} \\
             &=\bot        && \mbox{otherwise.}
\end{align*}
\begin{notebox}
CHECK that  I used this fact today, 28 Jan 2026, in order to prove
that a shortening of a tight zigzag is tight. Handwritten notes.
\end{notebox}

It follows from lemma  \ref{RHomFunctorSubLemma} that HomP, so defined, is a functor.
\commentary{\highlight{SPELL THIS OUT}} If \catcw is merely a category with support then, annoyingly, 
HomP does not preserve the support operation. However:
\begin{lemma}
\llabel{originalHomPSupportLemma}
If \catcw is a category with support and $a$ is an object of \catcw 
and if $j: x \morph y$ in \catcw satisfies the condition that for
all $k:a \morph x$ in \catcw 
\begin{equation}
\label{supportPreservationCondition}
k \circ \bar{j} = k \mbox{ iff } \overline{k \circ j} = \overline{k}
\end{equation}
then
\begin{equation}
HomP_\catc(a,\bar{j})=\overline{HomP_\catc(a,j)}.
\end{equation}
\end{lemma}
\begin{proof}
Suppose that  $j: x \morph y$ in \catcw is such a morphism satisfying 
identity (\ref{supportPreservationCondition})
 for all morphisms $k: a \morph x$ in \catcw.

In \SetP we have the following partial functions
\[
\paralleldiag{HomP_C(a,x)}{HomP_C(a,x)}{HomP_C(a,\bar{j})}{\overline{HomP_C(a,j)}}
\]
We need show that for all $k \in HomP_C(a,x)$,
\[
HomP_\catc(a,\bar{j})(k)=\overline{HomP_\catc(a,j)}(k).
\]

To this end,
first consider that  the left hand function, $HomP_\catc(a,\bar{j})(k)$, is defined by
\begin{align*}
HomP(a,\bar{j})(k) &= k \circ \bar{j}  &&\mbox{ if $\overline{k \circ \bar{j}} = \overline{k}$,} \\
             &=\bot        && \mbox{otherwise,}
\end{align*}
which  simplifies using wR.4 to
\begin{align*}
HomP(a,\bar{j})(k) &= k \circ \bar{j} &&\mbox{ if $\overline{k \circ j} = \overline{k}$,} \\
             &=\bot        && \mbox{otherwise.}
\end{align*}
and so using (\ref{supportPreservationCondition}) to
\begin{align*}
\overline{HomP(a,j)}(k) 
            &= k \circ \bar{j} &&\mbox{if $k \circ \bar{j} = k$,} \\
            &=\bot        && \mbox{otherwise,}
\end{align*}
i.e. to
\begin{align*}
\overline{HomP(a,j)}(k) 
            &= k          &&\mbox{if $k \circ \bar{j} = k$,} \\
            &=\bot        && \mbox{otherwise.}
\end{align*}

On the otherhand from the definition of $HomP_\catc(a,j)(k)$ it follows that
\begin{align*}
\overline{HomP(a,j)}(k) 
            &= k         &&\mbox{if $\overline{k \circ j} = \overline{k}$,} \\
            &=\bot        && \mbox{otherwise,}
\end{align*}
which simplfies, using (\ref{supportPreservationCondition}), to defining
\begin{align*}
\overline{HomP(a,j)}(k) 
            &= k  &&\mbox{if $k \circ \bar{j} = k$,} \\
            &=\bot             && \mbox{otherwise.}
\end{align*}
We see the two functions, $HomP_\catc(a,\bar{j})(k)$ and $\overline{HomP_\catc(a,j)}(k)$, are identical, as required.
\end{proof}

From this lemma and from lemma \ref{restrictioncatlemma}(\ref{CHECKLemma}) it follows that:
\begin{corollary}
\llabel{HomPRestrictionCorallary}
If \catcw is a restriction category  and $a$ is an object of \catcw 
then the functor $HomP_\catc(a,-)$ preserves the restriction operator i.e.
for all $j:x \morph y$ in \catc
$$HomP_\catc(a,\bar{j})=\overline{HomP_\catc(a,j)}$$
\end{corollary}


\begin{newtt}
We might end up proceeding via
\begin{itemize}
\item defining a morphism $j$ of a category with support to be 
\textit{friendly} iff for all morphisms k having codomain the domain of j,
\[
\overline{k \circ j} = \bar{k} \mbox{ iff } k \circ \overline{j} = k
\]

which is implied by, but which does not imply,  R.4 holding at $j,k$ i.e. that 
\[
k \circ \bar{j} = \overline{k \circ j} \circ k.
\]
\item showing morphisms embedded from \catcw into the stickleback category  have the HomP suuport preservation propery
i.e have the property that for all morphisms ????
\item
 we will need to show if F and G are zigzags and G is of length 0 then 
the friendliness condition holds,
\item showing that the supports of friendly morphisms are preserved by the HomP functor,
\item if one or more functors preserve supports of a morphism f then the various constructed functors also preserve the support of f.
\item Thereby showing that the constructed functors preserve restrictions of friendl morphisms.
 \end{itemize} 
 \end{newtt}

\subsection{Coproducts of Set valued Restriction Functors}
\tbd
\subsection{Quotients of Set valued Restriction Functors}

\newcommand{\Fquotient}{F/\sim}
Suppose \catcw is a restriction category and that $F: \catc \morph \SetP$ is a restriction functor into the restriction category of sets and partial functions. Suppose that for every object $x$ of \catcw there is an equivalence relationship $\sim_x$ defined on the set $F(x)$. Suppose that this equivalence relationship has the following property(s):
For every morphism $j:x \morph y$ in \catcw, for all elements $k1,k2 \in F(x)$,
\begin{enumerate}
\item $F(j)(k_1)$ is defined iff $F(j)(k_2)$ is defined, 
\item $k_1 \sim_x k_2 \mbox{ and } F(j)(k_1) \mbox{ is defined } \implies F(j)(k_1) \sim_y F(j)(k_2).$
\end{enumerate}
Define functor $\Fquotient: \catcw \morph \SetP$ by defining
\begin{align*}
&x             &&\mapsto F(x)/\sim_x 
                & \mbox{i.e. the set of equivalence classes of $F(x)$ wrt equivalence relation $\sim$}\\
&j: x \morph y &&\mapsto F(j)/\sim
\end{align*} 
where $F(j)/\sim$ is defined by
\begin{align*}
[k] \mapsto [F(j)(k)] && \mbox{provided $F(j)(k)$ is defined, is undefined otherwise.}
\end{align*}
which is well-defined because of the assumption made above.

It is easy to see that $\Fquotient$ is a functor (i.e. respects composition $\circ$ and identity morphisms.)

\begin{lemma}
The functor $\Fquotient$ is a restriction functor.
\end{lemma}
\begin{proof}
We need to show that  $\Fquotient$ respects the restriction operator
i.e. we need to show that for any morphism $j:x \morph y$ in \catcw, $\Fquotient(\bar{j})= \overline{\Fquotient(j)}$.

The LHS, $\Fquotient(\bar{j})$, is defined to  be the partial function
$$ [k] \mapsto [F(\bar{j})(k)] \mbox{ providing $F(\bar{j})(k)$ is defined, is undefined otherwise,}$$
in other words, since  $F$ assumed to be a restriction functor it can be defined as mapping
$$ [k] \mapsto [\overline{F(j)}(k)] \mbox{providing $\overline{F(j)}(k)$ is defined, is undefined otherwise,}$$
and this in turn, because of the definition of restriction in \SetP, means that it maps
$$ [k] \mapsto [k] \mbox{providing $F(j)(k)$ is defined, is undefined otherwise.}$$

Meanwhile the RHS, $\overline{\Fquotient(j)}$  is defined as the partial function that 
maps $[k]$  to $[k]$  provided  $\Fquotient(j)(k)$ is defined, and to be undefined otherwise.

Because,
from the definition of $\Fquotient$, $\Fquotient(j)([k])$ is defined iff $F(j)(k)$ is defined then
we have shown that LHS and RHS are identical functions, as required.
\end{proof}



\newcommand {\catRangeCat}{\mathbf{RC}}
\newcommand {\catRangePSCat}{\mathbf{RPS}}

\section{Free Range Category with Partial Sections}
Let $\catRangeCat$ denote the category of RR.5 range categories and
range-preserving functors, and let $\catRangePSCat$ denote the category
of range categories equipped with partial sections, and structure-
preserving functors.

Let $U_s: \catRangePSCat \morph \catRangeCat$ be the forgetful functor.
Cockett’s axiomatization of range categories, including axiom (RR.5), is quasi-equational. Likewise the extension to categories with partial sections is quasi-equational.
Hence the forgetful functor from range categories with partial sections to range categories is induced by a morphism of quasi-equational theories, and therefore admits a left adjoint by Palmgren–Vickers(\cite{PalmgrenVickers2007}). We have therefore:
\begin{corollary}[Palmgren-Vickers]
The forgetful functor $U_s: \catRangePSCat \morph \catRangeCat$
has a left adjoint $F_s: \catRangeCat \morph \catRangePSCat$.
\end{corollary}

Let $eta: Id_{\catRangeCat} \morph F_s \circ U_s$ be the unit of the adjunction $F_s \dashv U_s$.

\begin{lemma}
If \catcw is a range category  then
the unit $\eta_{\catc}: \catc \morph U(F(\catc))$  is faithful. 
It faithfully embeds an RR.5 category in a freely generated range category with sections.
\end{lemma}
\begin{proof}
Let $\phi$ be the natural isomorphism of hom sets given by the adjunction so that
$$
\phi_{C,A} :
Hom_{\catRangePSCat}(F_s(C),A)
\;\cong\;
Hom_{\catRangeCat}(C,U_s(A))
$$

Let $I_s: \catc \morph  U_s(\mathbf{S})$ be the faithful embedding of \catcw into a catgory with partial sections whose existence is
given by corollary \ref{scheincorollary}. 

The adjunction gives us  $\phi(I_s) : F_s(\catc) \morph U_s(\mathbf(S))$ as the unique morphism such that
\begin{displaymath}
%\composeSevenShaped[nodesize]{A}{B}{C}{f}{g}{h}
\composeSevenShaped[0.5cm]{\catc}{$U_s(F_s(\catc)$)}{$U_s(S)$}{\eta_\catc}{\phi_{C,U_s(S)}(I_s)}{I_s}
\end{displaymath}
commutes. 
Since $I_s$ is faithful it follows from the commutivity of this diagram that $\eta_\catc$ is faithful.
\end{proof}
\newt{We also need the axiom of choice. Where should I  take this step?} 


\section{Construction of Range Functors}

\subsection{$HomP$ as a Range Functor}
Cockett et all show that if every morphism $f$ of a restriction category has a section $f_s$ then if we define
\[
\widehat{f}=f_s \circ f
\]
then \hatItself is a range operator and \catcw is a range category and they show that
$\widehat{f}$ is not dependent on the choice of section.

\begin{notebox}{where I go with this}
I want to define a \hatItself for any morphism which has a section.

I want to define the HomP preservation property a morphims $j$ to as I defined above for \textit{freindly}.

I want to split the \lref{rangePreservationLemma} into three parts. 
\begin{itemize}
\item First, much earlier, that in a restriction category every morphism has the preservation property. This is proved above in the new \ref{restrictioncatlemma} (\ref{CHECKLemma}).
\item Then, Lemma, in any category with support is a morphism $j$ has a partial section and has the preservation property then the pseudo range of j is preserved by $Hom(a,-)$.
\item restate \lref{rangePreservationLemma} as a Corollary.
\end{itemize}
\end{notebox}

\begin{notebox}[\highlight{BUT} very big but]
But isn't that obvious since partial sections are preserved by $HomP$ that therefore
ranges are preserved and we don't need that preservation property. So the above is wrong.
It is easier than that. What we DO need the preservation property for is to show that the support operation is preserved. see lemma \lref{originalHomPRestrictionLemma}. 
\end{notebox}

\begin{lemma}
\llabel{rangePreservationLemma}
If \catcw is a range category in which every morphism has a partial section then
for every object $a$ of \catcw the functor $HomP_\catcw(a,-): \catcw \morph \SetP$ preserves the range operator. 
\end{lemma}
\begin{proof}
\commentary{\highlight{see note above}. This lemma has a trivial proof and can be generalised.}
If $j: x \morph y$ in \catcw then we are required to prove that
$$\reallywidehat{HomP_\catcw(a,j)}=HomP_\catcw(a,\widehat{j})$$
and for this it suffices to show that the range of $HomP_\catcw(a,j)$ 
i.e. this set
\[
\setsuchthat{k \circ j}
            {k: a \morph x \mbox{ and } \overline{k \circ j} = \overline{k} }
\] 
is identical to the range of $HomP_\catcw(a,\hat{j})$ 
i.e. this set
\[
\setsuchthat{k' \circ \hat{j}}
            {k':a \morph y \mbox{ and } \overline{k' \circ \hat{j}} = \overline{k'} }.
\]   

The first set is included in the second because for any morphism of the form $k \circ j$, 
where $\overline{k \circ j}=\bar{k}$, 
is of the form $k' \circ \hat{j}$, for $\overline{k' \circ \hat{j}} = \overline{k'}$, 
because we can take $k'$ to be $k \circ j$ 
and the two conditions are met for we have firstly
\begin{align*}
k' \circ \hat{j}  &= k \circ j \circ \hat{j}  && \mbox{choice of k',}\\
                  &= k \circ j                && \mbox{by RR.2}  
\end{align*}
and secondly, because of this, 
\begin{align*}
\overline{k' \circ \hat{j}} &= \overline{k \circ j } && \mbox{from the first part,}     \\ 
                            &= \overline{k'}         && \mbox{k' as chosen.}
\end{align*} 

Now let $j_s : y \morph x$ be a splitting of $j$ 
i.e. be such that $j_s \circ j = \hat{j}$.\commentary{what else?}

The second set is included in the first because 
if a morphism is of the form $k' \circ \hat{j}$, 
where $\overline{k' \circ \hat{j}} = \overline{k'}$, 
then it of the form $k \circ j$, where $\overline{k \circ j}=\overline{k}$, 
because $k$ can be taken to be $k' \circ j_s$ for firstly we have 
\begin{align*}
k' \circ \hat{j} &= k' \circ j_s \circ j  
                        &&\mbox{because $j_s \circ j = \hat{j}$ \highlight{ref},}\\
                 &= k \circ j
                        &&\mbox{because $k$ chosen as $ k' \circ j_s$}
\end{align*}
and secondly we have 
\begin{align*}
\overline{k \circ j}
           &= \overline{k' \circ j_s \circ j} 
                      && \mbox{choice of $k$,}\\
           &= \overline{k' \circ j_s}       
                      && \mbox{by lemma \ref{supportLemma},} \\
           &          && \mbox{and since 
                                $\overline{k' \circ j_s \circ j}
                                  =\overline{k' \circ \hat{j}} 
                                  = \overline{k'}$,
                               }                             \\
           &= \overline{k} 
                                          && \mbox{because of choice of $k$.}
\end{align*}

\end{proof}



\begin{corollary}
If \catcw is a range category in which every morphism has a partial section then
for every object $a$ of \catcw the functor $HomT_\catcw(a,-): \catcw \morph \SetP$ is a range functor.
\end{corollary}

Satisfyingly, we can show the converse:
\begin{lemma}
If \catcw is a range category and $HomP(a,-)$ is a range functor for all objects a of \catcw
then every morphism $f$ of \catcw has a partial section. 
\end{lemma}
\begin{proof} 
The partial section of a morphism $f:a \morph b$ is the inverse image
under of $\widehat{f}$ under the function $Hom(b,j):Hom(b,a) \morph Hom{b,b}$.
\end{proof}

\subsection{About HomP}
\begin{lemma}
\label{HomOfEffbargbar}
\fgsourcediag
\begin{align*}
[L_c(f)] &= [L_a(Hom(a,f)(\bar{f}\circ\bar{g}))] \\
[L_c(g)] &= [L_a(Hom(a,g)(\bar{f}\circ\bar{g}))]
\end{align*} \commentary{\highlight{what} the heck?}
\end{lemma}
\begin{proof}
\end{proof}
\subsection{Coproducts of Set valued Range Functors}
TBD
\subsection{Quotients of Set valued Range Functors}
\begin{lemma}
\label{rangefunctorquotient}
The functor $\Fquotient$ is a range functor.
\end{lemma}
\begin{proof} \commentary{Proof reviewed on 12 Dec 2023.}
We need to show that  $\Fquotient$ respects the range operator
i.e. we need to show that for any morphism $j:x \morph y$ in \catcw, 
 $\Fquotient(\hat{j})= \widehat{\Fquotient(j)}$.

The LHS, $\Fquotient(\hat{j})$, is defined to  be 
the partial function which maps $[k']$, where $k' \in F(y)$, as follows
$$ [k'] \mapsto [F(\hat{j})(k')] \mbox{ providing $F(\hat{j})(k')$ is defined, and is undefined otherwise.}$$
In other words, since  $F$ assumed to be a range functor it can be defined as mapping
$$ [k'] \mapsto [\widehat{F(j)}(k')] \mbox{ providing $\widehat{F(j)}(k')$ is defined and being undefined otherwise,}$$
and this in turn, because of the definition of ranges in \SetP, means that it maps
$$ [k'] \mapsto [k'] \mbox{providing there exists $k \in F(x)$ such that $k'=F(j)(k)$ and being undefined otherwise.}$$

Meanwhile the RHS, $\widehat{\Fquotient(j)}$  is defined as the partial function that 
maps $[k']$  to $[k']$  provided  there exists $k \in F(x)$ such that
$\Fquotient(j)([k])=[k']$ to be undefined otherwise.

Because,
from the definition of $\Fquotient$ we have that $\Fquotient(j)([k])= [F(j)(k)]$   
we have  that LHS and RHS are identical functions, as required.
\end{proof}
