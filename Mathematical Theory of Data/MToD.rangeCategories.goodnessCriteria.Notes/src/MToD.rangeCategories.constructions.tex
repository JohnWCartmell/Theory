\section{Construction of Restriction Functors}

Denote by \SetP the range category of sets and partial functions
 and let
\FinPar be the range category of finite sets and partial functions. 

\subsection{The partial Hom functor $HomP$}
If \catcw is a category with support then \commentary{Tidy this definition please.}
let $HomP_\catc(a,-): \catc \morph \SetP$ be the  functor
defined as follows. $HomP(a,x)=Hom(a,x)$.
If $j: x \morph y$ then partial function $HomP(a,j): HomP(a,x) \morph HomP(a,y)$
is defined\footnote{
In earlier notes I defined HomP as follows.
\begin{align*}
HomP(a,j)(k) &= k \circ j &&\mbox{ if $k \circ \bar{j} = k$,} \\
             &\bot        && \mbox{otherwise.}
\end{align*}
In a restriction category this is defintion is equivalent to the one given but not in an arbitrary category with support.
} as follows:

\begin{align*}
HomP(a,j)(k) &= k \circ j  &&\mbox{ if $\overline{k \circ j} = \overline{k}$,} \\
             &=\bot        && \mbox{otherwise.}
\end{align*}
\begin{notebox}
CHECK that  I used this fact today, 28 Jan 2026, in order to prove
that a shortening of a tight zigzag is tight. Handwritten notes.
\end{notebox}

It follows from lemma  \lref{supportLemma} (\lref{RHomFunctorSubLemma}) that HomP, so defined, is a functor.
If \catcw is a category with support then
HomP does not necessarily preserve the support operation. However:
\begin{lemma}
\llabel{originalHomPSupportLemma}
If \catcw is a category with support and $a$ is an object of \catcw 
and if $j: x \morph y$ in \catcw satisfies the condition that for
all $k:a \morph x$ in \catcw 
\begin{equation}
\label{supportPreservationCondition}
k \circ \bar{j} = k \mbox{ iff } \overline{k \circ j} = \overline{k}
\end{equation}
then
\begin{equation}
HomP_\catc(a,\bar{j})=\overline{HomP_\catc(a,j)}.
\end{equation}
\end{lemma}
\begin{proof}
Suppose that  $j: x \morph y$ in \catcw is such a morphism satisfying 
identity (\ref{supportPreservationCondition})
 for all morphisms $k: a \morph x$ in \catcw.

In \SetP we have the following partial functions
\[
\paralleldiag{HomP_C(a,x)}{HomP_C(a,x)}{HomP_C(a,\bar{j})}{\overline{HomP_C(a,j)}}
\]
We need show that for all $k \in HomP_C(a,x)$,
\[
HomP_\catc(a,\bar{j})(k)=\overline{HomP_\catc(a,j)}(k).
\]

To this end,
first consider that  the left hand function, $HomP_\catc(a,\bar{j})(k)$, is defined by
\begin{align*}
HomP(a,\bar{j})(k) &= k \circ \bar{j}  &&\mbox{ if $\overline{k \circ \bar{j}} = \overline{k}$,} \\
             &=\bot        && \mbox{otherwise,}
\end{align*}
which  simplifies using wR.4 to
\begin{align*}
HomP(a,\bar{j})(k) &= k \circ \bar{j} &&\mbox{ if $\overline{k \circ j} = \overline{k}$,} \\
             &=\bot        && \mbox{otherwise.}
\end{align*}
and so using (\ref{supportPreservationCondition}) to
\begin{align*}
\overline{HomP(a,j)}(k) 
            &= k \circ \bar{j} &&\mbox{if $k \circ \bar{j} = k$,} \\
            &=\bot        && \mbox{otherwise,}
\end{align*}
i.e. to
\begin{align*}
\overline{HomP(a,j)}(k) 
            &= k          &&\mbox{if $k \circ \bar{j} = k$,} \\
            &=\bot        && \mbox{otherwise.}
\end{align*}

On the otherhand from the definition of $HomP_\catc(a,j)(k)$ it follows that
\begin{align*}
\overline{HomP(a,j)}(k) 
            &= k         &&\mbox{if $\overline{k \circ j} = \overline{k}$,} \\
            &=\bot        && \mbox{otherwise,}
\end{align*}
which simplfies, using (\ref{supportPreservationCondition}), to defining
\begin{align*}
\overline{HomP(a,j)}(k) 
            &= k  &&\mbox{if $k \circ \bar{j} = k$,} \\
            &=\bot             && \mbox{otherwise.}
\end{align*}
We see the two functions, $HomP_\catc(a,\bar{j})(k)$ and $\overline{HomP_\catc(a,j)}(k)$, are identical, as required.
\end{proof}

From this lemma and from lemma \ref{restrictioncatlemma}(\ref{CHECKLemma}) it follows that:
\begin{corollary}
\llabel{HomPRestrictionCorallary}
If \catcw is a restriction category  and $a$ is an object of \catcw 
then the functor $HomP_\catc(a,-)$ preserves the restriction operator i.e.
for all $j:x \morph y$ in \catc
$$HomP_\catc(a,\bar{j})=\overline{HomP_\catc(a,j)}$$
\end{corollary}


\begin{worktt}
We might end up proceeding via
\begin{itemize}

\item showing that morphisms $j$ embedded from \catcw into the stickleback category  have the HomP suuport preservation propery
i.e have the property that for all morphisms $k$ 
\[
\overline{k \circ j} = \bar{k} \mbox{ iff } k \circ \overline{j} = k
\]
\item
 we will need to show if F and G are zigzags and G is of length 0 then 
the  condition holds,
\item we have already shown  that the supports of  morphisms satisfying this condition
are preserved by the HomP functor,
\item if one or more functors preserve supports of a morphism f then the various constructed functors also preserve the support of f.
\item Thereby showing that the constructed functors preserve supports of morphisms
satisfying the condition.
 \end{itemize} 
 \end{worktt}

\subsection{Coproducts of Set valued Restriction Functors}
\tbd
\subsection{Quotients of Set valued Restriction Functors}

\newcommand{\Fquotient}{F/\sim}
Suppose \catcw is a restriction category and that $F: \catc \morph \SetP$ is a restriction functor into the restriction category of sets and partial functions. Suppose that for every object $x$ of \catcw there is an equivalence relationship $\sim_x$ defined on the set $F(x)$. Suppose that this equivalence relationship has the following property(s):
For every morphism $j:x \morph y$ in \catcw, for all elements $k1,k2 \in F(x)$,
\begin{enumerate}
\item $F(j)(k_1)$ is defined iff $F(j)(k_2)$ is defined, 
\item $k_1 \sim_x k_2 \mbox{ and } F(j)(k_1) \mbox{ is defined } \implies F(j)(k_1) \sim_y F(j)(k_2).$
\end{enumerate}
Define functor $\Fquotient: \catcw \morph \SetP$ by defining
\begin{align*}
&x             &&\mapsto F(x)/\sim_x 
                & \mbox{i.e. the set of equivalence classes of $F(x)$ wrt equivalence relation $\sim$}\\
&j: x \morph y &&\mapsto F(j)/\sim
\end{align*} 
where $F(j)/\sim$ is defined by
\begin{align*}
[k] \mapsto [F(j)(k)] && \mbox{provided $F(j)(k)$ is defined, is undefined otherwise.}
\end{align*}
which is well-defined because of the assumption made above.

It is easy to see that $\Fquotient$ is a functor (i.e. respects composition $\circ$ and identity morphisms.)

\begin{lemma}
The functor $\Fquotient$ is a restriction functor.
\end{lemma}
\begin{proof}
We need to show that  $\Fquotient$ respects the restriction operator
i.e. we need to show that for any morphism $j:x \morph y$ in \catcw, $\Fquotient(\bar{j})= \overline{\Fquotient(j)}$.

The LHS, $\Fquotient(\bar{j})$, is defined to  be the partial function
$$ [k] \mapsto [F(\bar{j})(k)] \mbox{ providing $F(\bar{j})(k)$ is defined, is undefined otherwise,}$$
in other words, since  $F$ assumed to be a restriction functor it can be defined as mapping
$$ [k] \mapsto [\overline{F(j)}(k)] \mbox{providing $\overline{F(j)}(k)$ is defined, is undefined otherwise,}$$
and this in turn, because of the definition of restriction in \SetP, means that it maps
$$ [k] \mapsto [k] \mbox{providing $F(j)(k)$ is defined, is undefined otherwise.}$$

Meanwhile the RHS, $\overline{\Fquotient(j)}$  is defined as the partial function that 
maps $[k]$  to $[k]$  provided  $\Fquotient(j)(k)$ is defined, and to be undefined otherwise.

Because,
from the definition of $\Fquotient$, $\Fquotient(j)([k])$ is defined iff $F(j)(k)$ is defined then
we have shown that LHS and RHS are identical functions, as required.
\end{proof}

\section{Construction of Range Functors}

\subsection{$HomP$ as a Range Functor}
Cockett et all show that if every morphism $f$ of a restriction category has a section $f_s$ then if we define
\[
\widehat{f}=f_s \circ f
\]
then \hatItself is a range operator and \catcw is a range category and they show that
$\widehat{f}$ is not dependent on the choice of section.



\begin{observation}
If \catcw is a category with support and j is such and such
and $j$ has a partial section $j_s$ then
$HomP_C(a,j_s)$ is a partial section of $HomP_C(x,j)$.
\commentary{Maybe this is worthy of being a lemma and having a proof.
} 
\begin{newtt}
\workt{{Hang on} this does need proving.} Is the support of $j_s$ preserved.
Need new lemma that if $j$ satisfies the condition and $j_s$ is a partial section of
$j$ then $j_s$ meets the condition also.
\end{newtt}
\end{observation}

\begin{corollary}
If \catcw is a range category in which every morphism has a partial section then
for every object $a$ of \catcw the functor $HomT_\catcw(a,-): \catcw \morph \SetP$ is a range functor.
\end{corollary}

The converse is also true
\begin{lemma}
If \catcw is a range category and $HomP(a,-)$ is a range functor for all objects a of \catcw
then every morphism $f$ of \catcw has a partial section. 
\end{lemma}
\begin{proof} 
First consider that 
$\widehat{f}=id_b \circ \hat{f}=Hom(a,\hat{f})(id_b)$ 
and therefore that $\widehat{f}$ is in the image of $Hom(a,\hat{f})$.
Which is to say that $\widehat{f}$  is in the image of the function
$Hom(b,f)$ because $\reallywidehat{Hom(b,f)} = Hom(b,\hat{f})$ from the assumption that the range operator is preserved by $HomP(a,-)$. Therefore there is
at least one such $g: b \morph a$ such that $Hom(a,f)(g) = \hat{f}$ i.e. such
that
\begin{equation}
\label{gCondition}
g \circ f = \hat{f}.
\end{equation}

For any such $g$ we can show that the morphism $\hat{f} \circ g$ is a partial section of $f$
by showing  both conditions required of a partial section 
\begin{equation}
\label{sectionReqOne}
f \circ \hat{f} \circ g \circ f = f
\end{equation}  
and
\begin{equation}
\label{sectionReqTwo}
\hat{f} \circ g \circ f = \overline{\hat{f} \circ g}
\end{equation}
(\ref{sectionReqOne}) is the case because
\begin{align*}
f \circ \hat{f} \circ g \circ f &=  f \circ \hat{f} \circ \hat{f}     
                                               && \mbox{by (\ref{gCondition}),} \\
                                &=f       
                                               && \mbox{by RR.2.} 
\end{align*}
(\ref{sectionReqTwo}) is the case because
\begin{align*}
\hat{f} \circ g \circ f  &= \hat{f} \circ \hat{f}       
                                               && \mbox{by (\ref{gCondition}),} \\
                         &= \hat{f}     
                                               && \mbox{by lemma \ref{unirangelemma} (ii),} \\
                         &= \bar{\hat{f}}     
                                               && \mbox{by RR.1,} \\
                         &= \overline{f \circ g}    
                                               && \mbox{by (\ref{gCondition}),}\\
                         &= \overline{\hat{f} \circ g}     
                                               && \mbox{by RR.4.} \\
\end{align*}
\end{proof}

\subsection{About HomP}
\begin{lemma}
\label{HomOfEffbargbar}
\fgsourcediag
\begin{align*}
[L_c(f)] &= [L_a(Hom(a,f)(\bar{f}\circ\bar{g}))] \\
[L_c(g)] &= [L_a(Hom(a,g)(\bar{f}\circ\bar{g}))]
\end{align*} \commentary{\highlight{what} the heck?}
\end{lemma}
\begin{proof}
\end{proof}
\subsection{Coproducts of Set valued Range Functors}
TBD
\subsection{Quotients of Set valued Range Functors}
\begin{lemma}
\label{rangefunctorquotient}
The functor $\Fquotient$ is a range functor.
\end{lemma}
\begin{proof} \commentary{Proof reviewed on 12 Dec 2023.}
We need to show that  $\Fquotient$ respects the range operator
i.e. we need to show that for any morphism $j:x \morph y$ in \catcw, 
 $\Fquotient(\hat{j})= \widehat{\Fquotient(j)}$.

The LHS, $\Fquotient(\hat{j})$, is defined to  be 
the partial function which maps $[k']$, where $k' \in F(y)$, as follows
$$ [k'] \mapsto [F(\hat{j})(k')] \mbox{ providing $F(\hat{j})(k')$ is defined, and is undefined otherwise.}$$
In other words, since  $F$ assumed to be a range functor it can be defined as mapping
$$ [k'] \mapsto [\widehat{F(j)}(k')] \mbox{ providing $\widehat{F(j)}(k')$ is defined and being undefined otherwise,}$$
and this in turn, because of the definition of ranges in \SetP, means that it maps
$$ [k'] \mapsto [k'] \mbox{providing there exists $k \in F(x)$ such that $k'=F(j)(k)$ and being undefined otherwise.}$$

Meanwhile the RHS, $\widehat{\Fquotient(j)}$  is defined as the partial function that 
maps $[k']$  to $[k']$  provided  there exists $k \in F(x)$ such that
$\Fquotient(j)([k])=[k']$ to be undefined otherwise.

Because,
from the definition of $\Fquotient$ we have that $\Fquotient(j)([k])= [F(j)(k)]$   
we have  that LHS and RHS are identical functions, as required.
\end{proof}
