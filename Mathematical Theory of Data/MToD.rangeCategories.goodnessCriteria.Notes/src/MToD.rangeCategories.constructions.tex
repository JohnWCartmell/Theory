
\section{Construction of Restriction Functors}

Denote by \SetP the range category of sets and partial functions
 and let
\FinPar be the range category of finite sets and partial functions. 

\subsection{$HomP$ is a Restriction Functor}
If \catcw is a restriction functor then
let $HomP_\catc(a,-): \catc \morph \SetP$ be the  functor
defined as follows. $HomP(a,x)=Hom(a,x)$.
If $j: x \morph y$ then partial function $HomP(a,j): HomP(a,x) \morph HomP(a,y)$
is defined as follows:
\begin{align*}
HomP(a,j)(k) = k \circ j &\mbox{ if $k \circ \bar{j} = k$} \\
                         & \mbox{ is undefined otherwise}
\end{align*}

It follows from lemma  \ref{RHomFunctorSubLemma} that HomP, so defined, is a functor.  From the next lemma we see that it is a restriction functor.
\begin{lemma}
If \catcw is a restriction category and $a$ is an object of \catcw 
then the functor $HomP_\catc(a,-)$ preserves the restriction operator i.e.
for all $j:x \morph y$ in \catc
$$HomP_\catc(a,\bar{j})=\overline{HomP_\catc(a,j)}$$
\end{lemma}
\begin{proof}
Both $HomP_\catc(a,\bar{j})$ and $\overline{HomP_\catc(a,j)}$ are less than id function on the set $Hom(a,x)$. 
The former because by definition of $HomP$ 
if $HomP_\catc(a,\bar{j})(k)$ is defined then $HomP_\catc(a,\bar{j})(k)=k$.
The latter by definition of $\bar{f}$, for any function $f$.

It suffices to show that $HomP_\catc(a,\bar{j})$ is defined iff and only if
$HomP_\catc(a,j)$ is defined which we can do as follows.
 Let $k \in Hom(a,x)$. $HomP_\catc(a,\bar{j})(k)$
is defined iff $k \circ \bar{\bar{j}}$ i.e. $k \circ \bar{j}$ is defined
iff $HomP_\catc(a,j)(k)$ is defined. 
\end{proof}


\subsection{Coproducts of Set valued Range Functors}
\subsection{Quotients of Set valued Restriction Functors}

\newcommand{\Fquotient}{F/\sim}
Suppose \catcw is a restriction category and that $F: \catc \morph \SetP$ is a restriction functor into the restriction category of sets and partial functions. Suppose that for every object $x$ of \catcw there is an equivalence relationship $\sim_x$ defined on the set $F(x)$. Suppose that this equivalence relationship has the following property(s):
For every morphism $j:x \morph y$ in \catcw, for all elements $k1,k2 \in F(x)$,
\begin{enumerate}
\item $F(j)(k_1)$ is defined iff $F(j)(k_2)$ is defined, 
\item $k_1 \sim_x k_2 \mbox{ and } F(j)(k_1) \mbox{ is defined } \implies F(j)(k_1) \sim_y F(j)(k_2).$
\end{enumerate}
Define functor $\Fquotient: \catcw \morph \SetP$ by defining
\begin{align*}
&x             &&\mapsto F(x)/\sim_x 
                & \mbox{i.e. the set of equivalence classes of $F(x)$ wrt equivalence relation $\sim$}\\
&j: x \morph y &&\mapsto F(j)/\sim
\end{align*} 
where $F(j)/\sim$ is defined by
\begin{align*}
[k] \mapsto [F(j)(k)] && \mbox{provided $F(j)(k)$ is defined, is undefined otherwise.}
\end{align*}
which is well-defined because of the assumption made above.

It is easy to see that $\Fquotient$ is a functor (i.e. respects composition $\circ$ and identity morphisms.)

\begin{lemma}
The functor $\Fquotient$ is a restriction functor.
\end{lemma}
\begin{proof}
We need to show that  $\Fquotient$ respects the restriction operator
i.e. we need to show that for any morphism $j:x \morph y$ in \catcw, $\Fquotient(\bar{j})= \overline{\Fquotient(j)}$.

The LHS, $\Fquotient(\bar{j})$, is defined to  be the partial function
$$ [k] \mapsto [F(\bar{j})(k)] \mbox{ providing $F(\bar{j})(k)$ is defined, is undefined otherwise,}$$
in other words, since  $F$ assumed to be a restriction functor it can be defined as mapping
$$ [k] \mapsto [\overline{F(j)}(k)] \mbox{providing $\overline{F(j)}(k)$ is defined, is undefined otherwise,}$$
and this in turn, because of the definition of restriction in \SetP, means that it maps
$$ [k] \mapsto [k] \mbox{providing $F(j)(k)$ is defined, is undefined otherwise.}$$

Meanwhile the RHS, $\overline{\Fquotient(j)}$  is defined as the partial function that 
maps $[k]$  to $[k]$  provided  $\Fquotient(j)(k)$ is defined, and to be undefined otherwise.

Because,
from the definition of $\Fquotient$, $\Fquotient(j)([k])$ is defined iff $F(j)(k)$ is defined then
we have shown that LHS and RHS are identical functions, as required.
\end{proof}

\section{Range Categories with Partial Sections}
\begin{definition}
A \textit{range category with partial sections} is a range category \catcw such that
       for each morphism $f:a \morph b$ in \catcw there is a morphism $f_s: b \morph a $
       in \catcw such that $f_s \circ f = \hat{f}$.
\end{definition}

\begin{lemma}
If \catcw is a range category with partial sections then \catcw is an RR.5 range category.
\end{lemma}
\begin{proof}
Straightforward.
\end{proof}

\section{Generalisation of Schein's Theorem}
\begin{theorem} [\cite{COCKETT2012}]
Every small range category in
which [RR.5] holds admits a faithful embedding into the partial
map category of a regular category, namely sets and partial functions.
\end{theorem}

In particular:
\begin{corollary}
\label{scheincorollary}
Every RR.5 range category can be faithfully mapped into a range category with partial sections.
\end{corollary}

\newcommand {\catRangeCat}{\mathbf{RC}}
\newcommand {\catRangePSCat}{\mathbf{RPS}}

\section{Free Range Category with Partial Sections}
Let $\catRangeCat$ denote the category of RR.5 range categories and
range-preserving functors, and let $\catRangePSCat$ denote the category
of range categories equipped with partial sections, and structure-
preserving functors.

Let $U_s: \catRangePSCat \morph \catRangeCat$ be the forgetful functor.
Cockett’s axiomatization of range categories, including axiom (RR.5), is quasi-equational. Likewise the extension to categories with partial sections is quasi-equational.
Hence the forgetful functor from range categories with partial sections to range categories is induced by a morphism of quasi-equational theories, and therefore admits a left adjoint by Palmgren–Vickers(\cite{PalmgrenVickers2007}). We have therefore:
\begin{corollary}[Palmgren-Vickers]
The forgetful functor $U_s: \catRangePSCat \morph \catRangeCat$
has a left adjoint $F_s: \catRangeCat \morph \catRangePSCat$.
\end{corollary}

Let $eta: Id_{\catRangeCat} \morph F_s \circ U_s$ be the unit of the adjunction $F_s \dashv U_s$.

\begin{lemma}
If \catcw is a range category  then
the unit $\eta_{\catc}: \catc \morph U(F(\catc))$  is faithful. 
It faithfully embeds an RR.5 category in a freely generated range category with sections.
\end{lemma}
\begin{proof}
Let $\phi$ be the natural isomorphism of hom sets given by the adjunction so that
$$
\phi_{C,A} :
Hom_{\catRangePSCat}(F_s(C),A)
\;\cong\;
Hom_{\catRangeCat}(C,U_s(A))
$$

Let $I_s: \catc \morph  U_s(\mathbf{S})$ be the faithful embedding of \catcw into a catgory with partial sections whose existence is
given by corollary \ref{scheincorollary}. 

The adjunction gives us  $\phi(I_s) : F_s(\catc) \morph U_s(\mathbf(S))$ as the unique morphism such that
\begin{displaymath}
%\composeSevenShaped[nodesize]{A}{B}{C}{f}{g}{h}
\composeSevenShaped[0.5cm]{\catc}{$U_s(F_s(\catc)$)}{$U_s(S)$}{\eta_\catc}{\phi_{C,U_s(S)}(I_s)}{I_s}
\end{displaymath}
commutes. 
Since $I_s$ is faithful it follows from the commutivity of this diagram that $\eta_\catc$ is faithful.
\end{proof}
\newt{We also need the axiom of choice. Where should I  take this step?} 



\section{Construction of Range Functors}

\subsection{$HomP$ as a Range Functor}

\begin{lemma}
If \catcw is a range category in which every morphism has a partial section then
for every object $a$ of \catcw the functor $HomP_\catcw(a,-): \catcw \morph \SetP$ preserves the range operator. 
\end{lemma}
\begin{proof}
If $j: x \morph y$ in \catcw then we are required to prove that
$$\widehat{HomP_\catcw(a,j)}=HomP_\catcw(a,\hat{j})$$
and for this it suffices to show that the range of $HomP_\catcw(a,j)$ 
i.e. this set
$$\setsuchthat{ k \circ j}{k: a \morph x \mbox{ and } k \circ \bar{j} = k }$$ 
is identical to the range of $HomP_\catcw(a,\hat{j})$ 
i.e. this set
$$\setsuchthat{k' \circ \hat{j}}{k':a \morph y \mbox{ and } k' \circ \bar{\hat{j}} = k' }.$$   
i.e. this set
$$\setsuchthat{k' \circ \hat{j}}{k':a \morph y \mbox{ and } k' \circ \hat{j} = k' }.$$

The first set is included in the second because for any morphism of the form $k \circ j$, 
where $k \circ \bar{j}=k$, 
is of the form $k' \circ \hat{j}$, for $k' \circ \hat{j} = k'$, 
because we can take $k'$ to be $k \circ j$ 
because then 
$k' \circ \hat{j} = (k \circ j) \circ \hat{j}= k \circ(j \circ \hat{j})= k \circ j = k'$,
as required. 

Let $j_s : y \morph x$ be the splitting of $j$ i.e. be such that $j_s \circ j = \hat{j}$.

The second set is included in the first because if a morphism is of the form $k' \circ \hat{j}$, where $k' \circ \hat{j} = k'$, then it of the form $k \circ j$, where $k \circ \bar{j}=k$, because $k$ can be taken to be $k' \circ j_s$ for then we have 
$k' \circ \hat{j} = k' \circ j_s \circ j = k \circ j$
and we have $k \circ \bar{j}= k' \circ j_s \circ \bar{j} =  k' \circ j_s = k$ using lemma
\ref{jsbarj}, as required. \commentary{no such lemma!}
 \end{proof}


\begin{corollary}
If \catcw is a range category in which every morphism has a partial section then
for every object $a$ of \catcw the functor $HomT_\catcw(a,-): \catcw \morph \SetP$ is a range functor.
\end{corollary}

\subsection{About HomP}
\begin{lemma}
\label{HomOfEffbargbar}
\fgsourcediag
\begin{align*}
[L_c(f)] &= [L_a(Hom(a,f)(\bar{f}\circ\bar{g}))] \\
[L_c(g)] &= [L_a(Hom(a,g)(\bar{f}\circ\bar{g}))]
\end{align*} \commentary{\highlight{what} the heck?}
\end{lemma}
\begin{proof}
\end{proof}
\subsection{Coproducts of Set valued Range Functors}
TBD
\subsection{Quotients of Set valued Range Functors}
\begin{lemma}
\label{rangefunctorquotient}
The functor $\Fquotient$ is a range functor.
\end{lemma}
\begin{proof} \commentary{Proof reviewed on 12 Dec 2023.}
We need to show that  $\Fquotient$ respects the range operator
i.e. we need to show that for any morphism $j:x \morph y$ in \catcw, 
 $\Fquotient(\hat{j})= \widehat{\Fquotient(j)}$.

The LHS, $\Fquotient(\hat{j})$, is defined to  be 
the partial function which maps $[k']$, where $k' \in F(y)$, as follows
$$ [k'] \mapsto [F(\hat{j})(k')] \mbox{ providing $F(\hat{j})(k')$ is defined, and is undefined otherwise.}$$
In other words, since  $F$ assumed to be a range functor it can be defined as mapping
$$ [k'] \mapsto [\widehat{F(j)}(k')] \mbox{ providing $\widehat{F(j)}(k')$ is defined and being undefined otherwise,}$$
and this in turn, because of the definition of ranges in \SetP, means that it maps
$$ [k'] \mapsto [k'] \mbox{providing there exists $k \in F(x)$ such that $k'=F(j)(k)$ and being undefined otherwise.}$$

Meanwhile the RHS, $\widehat{\Fquotient(j)}$  is defined as the partial function that 
maps $[k']$  to $[k']$  provided  there exists $k \in F(x)$ such that
$\Fquotient(j)([k])=[k']$ to be undefined otherwise.

Because,
from the definition of $\Fquotient$ we have that $\Fquotient(j)([k])= [F(j)(k)]$   
we have  that LHS and RHS are identical functions, as required.
\end{proof}
