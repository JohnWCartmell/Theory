
\section{MToD Definitions}

\begin{definition}
If $\catc$ is a range category and $\reqtc$ is a set of instances and if \fgsourcediag
in $\catc$ then there is a  \term{functional dependency} of $g$ on $f$ with respect to $\reqtc$ iff
there is a family of partial functions $H_D)_{D \in \reqtc}$ such that 
in each instance $D$,  $H_D: D(b) \morph D(c)$, such that $D(f) \circ H_D = D(g)$ and $\overline{H_D}=\widehat{D(f)}$.
If there is such a functional dependency then we say that $\fundep{H}{f}{g}$ in $\catc$ with respect to $\reqtc$.
\begin{workt}
Can the equation $\overline{H_D}= \widehat{D(f)}$ be rephrased as the
equation $D(f) \circ\overline{H_D}=D(f)$? If it could be we could manage without range categories and complexity of sticklebacks.
\end{workt}
\begin{notebox}
There is a version of this theory which we should look at some time in which the equation $\overline{H_D}=\widehat{D(f)}$ is replaced by the equation 
$\overline{H_D}\leq \widehat{D(f)}$. Example. Suppose $g$ is $1/f$, 
unless $f$ is zero in which case $g$ is undefined. 
We would want to consider  $g$ tas functionally dependent on $f$. 
\end{notebox}
\end{definition}
\begin{newtt}
Use RR.5 (which holds in \Par) to show that in every $D$, such an $H_D$ is the unique such partial function:
\begin{align*}
&&D(f) \circ H_D &= D(f) \circ H'_D \\
\mbox{therefore }&&\widehat{D(f)} \circ H_D &= \widehat{D(f)} \circ H'_D  & \mbox{by RR.5}\\
\mbox{therefore }&&\overline{H_D} \circ H_D &=  \overline{H'_D} \circ H'_D  &             \\
\mbox{therefore }&&H_D &= H'_D
\end{align*}
\end{newtt}
\begin{lemma}
\label{fdrangesublemma}
If $\fundep{H}{f}{g}$ is a functional dependency then
\begin{enumerate}[(i)]
\item $$\widehat{H_D}=\widehat{D(g)}$$
and
\item $$\overline{D(f)}=\overline{D(g)}.$$
\end{enumerate}
\begin{notebox}
If we need it and assuming equational completeness we therefore also have
$\bar{f} = \bar{g}$. 
\end{notebox}
\end{lemma}
\begin{proof}
\begin{enumerate}[(i)]
    \item
\begin{align*}
\widehat{D(g)} &= \reallywidehat{D(f) \circ H_D} \\
               &= \reallywidehat{hat({D{f})} \circ H_D} &&\mbox{by RR.4,}\\               
               &= \reallywidehat{\overline{H_D} \circ H_D} &&\mbox{by defn of functional dependency,}\\
               &= \widehat{ H_D} &&\mbox{by R.1.}\\
\end{align*}
\item 
\begin{align*}
\overline{D(f)} &= \overline{D(f) \circ \widehat{D(f)}} &&\mbox{by RR.2,}\\
                &= \overline{D(f) \circ \overline{H_D}} &&\mbox{from the definition of functional dependency,}\\
                &= \overline{D(f) \circ H_D} &&\mbox{by wR.4,}\\
                &= \overline{D(g)}           &&\mbox{from the definition of functional dependency.}
\end{align*}
\end{enumerate}
\end{proof}

\begin{definition}
If $\catc$ is a range category and $\reqtc$ is a set of instances, if
\fgsourcediag
in $\catc$ 
and if there is a functional dependency $\fundep{H}{f}{g}$ then say that 
the functional dependency $H$ is \term{represented} in $\catc$ 
iff there exists a morphism $h:b \morph c$ in $\catc$ such that 
$f \circ h = g$. Note that we can deduce that if $h$ is such a representation then
in each instance $D$, $\widehat{D(f)} \circ D(h) = H_D$. \workt{CHECK THIS!!}
\end{definition}

