

\begin{oldtt}
\newcommand {\catRangeCat}{\mathbf{RC}}
\newcommand {\catRangePSCat}{\mathbf{RPS}}
\section{Free Range Category with Partial Sections}
Let $\catRangeCat$ denote the category of RR.5 range categories and
range-preserving functors, and let $\catRangePSCat$ denote the category
of range categories equipped with partial sections, and structure-
preserving functors.

Let $U_s: \catRangePSCat \morph \catRangeCat$ be the forgetful functor.
Cockett’s axiomatization of range categories, including axiom (RR.5), is quasi-equational. Likewise the extension to categories with partial sections is quasi-equational.
Hence the forgetful functor from range categories with partial sections to range categories is induced by a morphism of quasi-equational theories, and therefore admits a left adjoint by Palmgren–Vickers(\cite{PalmgrenVickers2007}). We have therefore:
\begin{corollary}[Palmgren-Vickers]
The forgetful functor $U_s: \catRangePSCat \morph \catRangeCat$
has a left adjoint $F_s: \catRangeCat \morph \catRangePSCat$.
\end{corollary}

Let $eta: Id_{\catRangeCat} \morph F_s \circ U_s$ be the unit of the adjunction $F_s \dashv U_s$.

\begin{lemma}
If \catcw is a range category  then
the unit $\eta_{\catc}: \catc \morph U(F(\catc))$  is faithful. 
It faithfully embeds an RR.5 category in a freely generated range category with sections.
\end{lemma}
\begin{proof}
Let $\phi$ be the natural isomorphism of hom sets given by the adjunction so that
$$
\phi_{C,A} :
Hom_{\catRangePSCat}(F_s(C),A)
\;\cong\;
Hom_{\catRangeCat}(C,U_s(A))
$$

Let $I_s: \catc \morph  U_s(\mathbf{S})$ be the faithful embedding of \catcw into a catgory with partial sections whose existence is
given by corollary \ref{scheincorollary}. 

The adjunction gives us  $\phi(I_s) : F_s(\catc) \morph U_s(\mathbf(S))$ as the unique morphism such that
\begin{displaymath}
%\composeSevenShaped[nodesize]{A}{B}{C}{f}{g}{h}
\composeSevenShaped[0.5cm]{\catc}{$U_s(F_s(\catc)$)}{$U_s(S)$}{\eta_\catc}{\phi_{C,U_s(S)}(I_s)}{I_s}
\end{displaymath}
commutes. 
Since $I_s$ is faithful it follows from the commutivity of this diagram that $\eta_\catc$ is faithful.
\end{proof}
\workt{We also need the axiom of choice. Where should I  take this step?} 
\end{oldtt}
