
Abstract. In this two-part paper, we undertake a systematic study of abstract partial
map categories in which every map has both a restriction (domain of definition) and a
range (image). In this first part, we explore connections with related structures such
as inverse categories and allegories, and establish two representational results. The
first of these explains how every range category can be fully and faithfully embedded
into a category of partial maps equipped with a suitable factorization system. The
second is a generalization of a result from semigroup theory by Boris Schein, and says
that every small range category satisfying the additional condition that every map is
an epimorphism onto its range can be faithfully embedded into the category of sets and
partial functions with the usual notion of range. Finally, we give an explicit construction
of the free range category on a partial map category in terms of certain types of labeled
trees.

\section{By the way}
I cannot help thinking that Cockett et al's proof of Schein's Theorem must break down into three
distinct parts
\begin{itemize}
       \item show that theory of categories with partial sections is a conservative extension of the theory of RR.5 range categories (prove syntactically)
       \item there is a free category with partial sections generated by a range category. 
              More formally:  
              the inclusion of the category of range categories with partial sections  
              into the  category of RR.5 range categories (i.e. the forgetful functor) has a left adjoint.
       \item For every range category with partial sections there is a faithful range functor into the
             category of sets and partial functions. Proof. coproduct over all objects $a$ of Hom functor $(Hom(a,-)$.  
\end{itemize}

Update: 16 December 2025. I guess that there is a category of sticklebacks over a range category C and that this would be a range category. Sadly if it is a category (which I am not sure I have actually checked)
then it certainly isn't a range category. For $f$ a morphism 
$f: A \morph B$
the \catcw the stickle $B \xleftarrow{f} A$ (the partial section of $f$)
doesn't have a range stickeback. I had hoped to show that the stickle 
 $ B \xleftarrow{f} A \xrightarrow{f} B$ was the range but this is clearly nonsense (though I managed to believe it for a day or two).

 I guess that there might be enough structure so that \catcw is embedded in a restriction catgory in which all morphisms in the image have a range and have a partial section(
  the expectation of course is that the stickleback $B \xleftarrow{f} A$ 
 has the stickleback $A \xrightarrow{f} B$ as a section. That is surely the point of the stickleback). Since what I really interested in the is the composition of this functor with the Hom functor then this might be enough for structure for the construction I need. 

I expect that I can define
\begin{itemize}
    \item $\hat{A \xrightarrow{f} B}=_def B \xrightarrow{\hat{f}} B$
\end{itemize}

I need to show $ B \xleftarrow{f} A \xrightarrow{f} B=\xrightarrow{\hat{f}} B$
ooops. very confused.

\subsection{Category of sticklebacks on a category \catcw}
%macros
\newcommand{\Ob}[2]{
\Rnode{#1#2}{#1_#2}
}

%\zigzagZeroOne{x}{y}{f}[ff] 
% This is a zig zag zig zag. 
% zig zero is f0 \circ ff: y0 -> x0             
% zag zero is f'0: y0 -> x1
% zig one is  f1 : y1 -> x1 
% zig one is  f'1 : y1 -> x1
\NewDocumentCommand{\zigzagZeroOne}{m m m o}{
\begin{array} {c c c c c c c c}
 &\Ob{#2}{0}&&\Ob{#2}{1}          &\\[0.8cm]
\Ob{#1}{0}&&\Ob{#1}{1}&&\Ob{#1}{2} 
\end{array}
\begin{arrows}
\ncarr{#20}{#10} \blabel{#3_0 
\IfNoValueTF{#4}
    {}
    {\circ #4}
}[0.45][0.4]
\ncarr{#21}{#11} \blabel{#3_1}[0.45][0.4]
\ncarr{#20}{#11} \alabel{#3'_0}[0.45][0.4]
\ncarr{#21}{#12} \alabel{#3'_1}[0.45][0.4]
\end{arrows}
}

%\zigzagDotsThenFinal{x}{y}{f}{n}
% This is a zig
% fn  : yn -> xn
\NewDocumentCommand{\zigzagDotsThenFinal}{m m m o m}{
\begin{array}{c c c}
\dots &&\Ob{#2}{#5} \\[0.8cm]
\dots &\Ob{#1}{#5}
\end{array}
\begin{arrows}
\ncarr{#2#5}{#1#5} \blabel{#3_#5}[0.45][0.4]
\end{arrows}    
}

%\zigzag{x}{y}{f}[ff]{n}
% This is a zig zag zig zag... zig
% For params 1 to 4, see command \zigzagZeroOne
% For param 5 see command \zigzagDotsThenFinal
%\zigzag{x}{y}{f}{n}
\NewDocumentCommand{\zigzag}{m m m o m}{
\zigzagZeroOne{#1}{#2}{#3}[#4]
\zigzagDotsThenFinal{#1}{#2}{#3}[#4]{#5}
}

A stickleback $f: A \morph B$ is a pair of sequences
$f$, and $f'$ of morphisms of \catcw such that for some $n \ge $, for some objects $x_0,..x_n$ and $y_1,...y_n$ of \catcw we have
$$
\arraycolsep=7pt
\zigzag{x}{y}{f}{n}
$$
in \catcw.

Composition of zigzags is given by
$$
\arraycolsep=7pt
\left( \zigzag{w}{x}{f}{n}\right) \circ \left(\zigzag{y}{z}{g}{m}\right)
$$
=
$$
\arraycolsep=7pt
\zigzagZeroOne{w}{x}{f}
\begin{array}{c}
\dots \\[0.8cm]
\dots
\end{array}
\begin{array}{c c}
\Ob{x}{n} \\[0.8cm]
& \Ob{w}{n}=
\end{array}
\begin{arrows}
\ncarr{xn}{wn} \blabel{f'n}[0.45][0.4]
\end{arrows}
\kern-15pt
\zigzag{y}{z}{g}[f_n]{m}
$$

providing there is no morphism $k: y_1 \morph y_0$ such that the triangle
$$
$$
commutes. If there is such a morphism then the composition is  
$$
\arraycolsep=7pt
\zigzagZeroOne{w}{x}{f}
\begin{array}{c}
\dots \\[0.8cm]
\dots
\end{array}
\begin{array}{c c}
\Ob{x}{n} \\[0.8cm]
& \Ob{w}{n}
\end{array}
\begin{arrows}
\ncarr{xn}{wn} \blabel{f'n}[0.2][0.4]
\end{arrows}
\begin{array}{c c}
\Ob{z}{1}& \\[0.8cm]
& \Ob{y}{2}
\end{array}
\begin{arrows}
\ncarr{z1}{wn}\blabel{g_1 \circ k}[0.45][0.4]
\ncarr{z1}{y2}\alabel{g'_1}[0.45][0.4]
\end{arrows}
\zigzagDotsThenFinal{y}{z}{g}[]{m}
$$



