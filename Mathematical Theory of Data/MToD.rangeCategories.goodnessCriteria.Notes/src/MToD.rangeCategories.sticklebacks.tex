


\section{By the way}
\subsection{One thought}
I cannot help thinking that Cockett et al's proof of Schein's Theorem must break down into three
distinct parts
\begin{itemize}
       \item show that theory of categories with partial sections is a conservative extension of the theory of RR.5 range categories (prove syntactically)
       \item there is a free category with partial sections generated by a range category. 
              More formally:  
              the inclusion of the category of range categories with partial sections  
              into the  category of RR.5 range categories (i.e. the forgetful functor) has a left adjoint.
       \item For every range category with partial sections there is a faithful range functor into the
             category of sets and partial functions. Proof. coproduct over all objects $a$ of Hom functor $(Hom(a,-)$.  
\end{itemize}

\subsection{Another Thought}
Update: 16 December 2025. I guessed that there is a category of short sticklebacks over a range category C and that this might be a range category by virtue of it being a category with choice (check out this terminology and make sure I have it right) and therefore a range category. 

 I guess that there might be enough structure so that \catcw is embedded in a restriction catgory in which all morphisms in the image have a range and have a partial section(
  the expectation of course is that the stickleback that is the image of $f$ in the embedding, i.e. $A \xleftarrow{ \bar{f}} A \xrightarrow{f} B \xleftarrow{\widehat{f}} A $ 
 has the stickleback $B \xleftarrow{f} A$ as a section. That is surely the point of the stickleback). Since what I really interested in  is the composition of this functor with the Hom functor then this might be enough for structure for the construction I need.

 Update: 16 January 2026. I have shown that there is a category of sticklebacks. I have defined an operator
 $\overline{\ \kern0.2cm}$ and proved the restriction axioms R.1, R.2 and R.3. I believe that wR.4 holds 
 (so that $\bar{\ }$ is a support operator (Check terminology!). I belive that I have shown that  R.4 does not hold (check again, I am desparate). Since I require R.4 to hold in order to construct a HompP functor this line (using the category of sticklebacks) seems to have come to a sticky end. It would be worth documenting why R.4 does not hold. It would be worth understanding how Cockett et al successfully use what seems to be the coproduct 
\[
\coprod_{a \in |\catc|}HomP(a,-)
\]
 even though $HomP(a,-)$ isn't functorial.  

\subsection{Plan of action 16th January 2026}
Investigate further 
\begin{itemize}
\item Work in this document for a while and establish that category of sticklebacks is a dead end. 
\item Document that R.4 doesn't hold.
\item Double check that as a consequence  HomP as prevously defined is not therefore functorial. It might be some kind of 2-functor. Guess this would be worth understanding.
\item Understand Cocketts seeming use of coproduct as described above.
\end{itemize}
Unless something gives
\begin{itemize}
\item Tidy up. Move all the category of stickleback stuff in a separate document.
\item Carry on with the free category with sections. 
\item The idea is that I dont have to prove that one theory is a conservative extension of another since Cockett has done this with his generalisation of Shein's lemma (the faithfulness of the embedding).
\end{itemize}

\subsection{Rereading Cockett}
I think I have the category of sticklebacks the wrong way around. I have defined it as the dual of what I need. But it is self-dual so it should make no difference. So that I don't confuse my self further here is what I read in Cockett.  

Functor $S:\catcw \morph \SetP$ is defined as the set of stickelbacks on $X$. That is 
\[ 
\bigcup_{Z \in |\catc|} stickle(X,Z)
\]

My initial interpretation was yields $stickle(X,Z)$ as a hom set in a category of stickles:
\[ Hom_{Stk}(X,Z)\]
so that 
\[
S(X)=\bigcup_{Z \in |\catc|}Hom_{Stk}(X,Z)
\]

For $f:X \morph Y$ in \catcw, $S(f):S(X) \morph S(Y)$ is defined by
\[
\gamma \in S(X) \xmapsto{S(f)} \tuple{f} \circ \gamma
\]
where $\tuple{f}$ is a stickle of length zero and $\circ$ is the composition operation that I have described in my first description of the category of sticklebacks.

This way around I have $f: X \morph Y$ in \catcw, $\tuple{f}: Y \morph X$ in 
category of stickles. My initial interpretation yields $S(f)$ as being
\[
\bigcup_{Z \in |\catc|} Hom_{Stk}(\tuple{f},Z) : \bigcup_{Z \in |\catc|}Hom_{Stk}(X,Z) \morph \bigcup_{Z \in |\catc|} Hom_{Stk}(Y,Z)
\] 

This way around the embedding of \catcw in the stickleback category is contravariant and  $S$ is a composition of the contravarinat embedding with a contravariant functor which is a coproduct of covariant Hom functors. 

Turn the morphisms about and the contravariance disappears. There are other ways around this but overwhelmingly it is better to change the morphisms around which actually is just the same as reversing the directions of all the morphisms in the definition of stickleback. Essentially we are doing this to avoid having morphisms going right to left isntead of left to right.

\subsection{The category of sticklebacks isn't a restriction category}
But is is a category with support. HomP(a,-) as I previously defined it was
functorial on a restriction category but not functorial on a category with support. 
But - HOORAY -- I can define it differently, more similar to Cockett et al, actually.
The HomP with this revised definition on categories with support and therefore on the stickleback category. Whether I absolutely need this I doubt but it is much neater.
Significantly: The revised definition is equivalent on restriction categories.
 
\subsection{Revised plan of action 20 January 2026}
\begin{itemize}
\item I see that Cockett's stickleback's drawn left to right from $x$ to $y$ have to be considered to be morphisms the other way around from $y$ to $x$ (otherwsie I get a contravariant embedding and a contravariant Hom functor).
Having decided this is the direction of stickleback's as morphisms  I now feel the need to do is to draw the sticklebacks the other way around. It makes  no difference mathematically. It just seems odd to have a stickleback running from left to right as a morphism from right to left. As expected when truned about the axiom R.4 still fails to hold. 
\item Since condition R.4 fails and condition wR.4 holds what I haveis (CHECK) a support operation rather than a restriction idempotent. 
I need read Cockett carefully --- is it the case that support operation plus partial sections gives a range operation? Did I read that? \newt{No I didn't.} If so I need check that I have the required conditions to get a range operation and a support operation.
\item If I can make some sense of this I need to check what properties the HomP  functor has. 
\item prove that HomP preserves the restriction operator and the range operator. 
(For the latter, use the sections.) 
\end{itemize}
If that all goes through I am home and hosed in that I have a source of set valued range functors
to use to construct instances, as required.
\begin{itemize}
\item Only when I have checked this out do I have to write up the category of sticklebacks. Somewhere need to introduce short sticklebacks.
\end{itemize}
\subsection{Category of zigzags on a category \catcw}

%macros
\newcommand{\Ob}[2]{
\Rnode{#1#2}{#1_{#2}}
}
% b bare object no subsript
\newcommand{\bOb}[1]{
\Rnode{#1}{#1}
}
% p for predecessor
\newcommand{\Obp}[2]{
\Rnode{#1#2p}{#1}_{#2-1}
}

% s for successor
\newcommand{\Obs}[2]{
\Rnode{#1#2s}{#1_{#2+1}}
}

\newcommand{\Obpp}[2]{
\Rnode{#1#2pp}{#1_{#2-2}}
}

\newcommand{\Obss}[2]{
\Rnode{#1#2ss}{#1_{#2+2}}
}

%\zigzagZeroOne{x}{y}{f}[ff] 
% This is a zig zag zig zag. 
% zig zero is f0 \circ ff: y0 -> x0             
% zag zero is f'0: y0 -> x1
% zig one is  f1 : y1 -> x1 
% zig one is  f'1 : y1 -> x1
\NewDocumentCommand{\zigzagZeroOne}{m m m o}{
\begin{array} {c c c c c c c c}
 &\Ob{#2}{0}&&\Ob{#2}{1}          &\\[0.8cm]
\Ob{#1}{0}&&\Ob{#1}{1}&&\Ob{#1}{2} 
\end{array}
\begin{arrows}
\ncarr{#20}{#10} \blabel{#3_0 
\IfNoValueTF{#4}
    {}
    {\circ #4}
}[0.45][0.4]
\ncarr{#21}{#11} \blabel{#3_1}[0.45][0.4]
\ncarr{#20}{#11} \alabel{#3'_0}[0.45][0.4]
\ncarr{#21}{#12} \alabel{#3'_1}[0.45][0.4]
\end{arrows}
}

\NewDocumentCommand{\zigzagZeroOneShort}{m m m o}{
\begin{array} {c c c c c c c }
 &\Ob{#2}{0}&&\Ob{#2}{1}          \\[0.8cm]
\Ob{#1}{0}&&\Ob{#1}{1}&
\end{array}
\begin{arrows}
\ncarr{#20}{#10} \blabel{#3_0 
\IfNoValueTF{#4}
    {}
    {\circ #4}
}[0.45][0.4]
\ncarr{#21}{#11} \blabel{#3_1}[0.45][0.4]
\ncarr{#20}{#11} \alabel{#3'_0}[0.45][0.4]
%\ncarr{#21}{#12} \alabel{#3'_1}[0.45][0.4]
\end{arrows}
}

%\zigzagStart{x}{y}{f}[n]
% starts from n instead od zero 
\NewDocumentCommand{\zigzagStart}{m m m m}{
\begin{array} {c c c c c c c }
 &\Ob{#2}{0}&&\Ob{#2}{1}    \\[0.8cm]
\Ob{#1}{0}&&\Ob{#1}{1}& 
\end{array}
\begin{arrows}
\ncarr{#20}{#10} \blabel{#3_{#4} }[0.45][0.4]
\ncarr{#20}{#11} \alabel{#3'_{#4} }[0.45][0.4]
\ncarr{#21}{#11} \alabel{#3_{#4+1} }[0.3][0.4]
%\ncarr{#21}{#12} \alabel{#3'_{#4+1} }[0.45][0.4]
\end{arrows}
}



%\zigzagDotsThenFinal{x}{y}{f}{n}
% This is a zig
% fn  : yn -> xn
\NewDocumentCommand{\zigzagDotsThenFinal}{m m m m}{
\begin{array}{c c c}
\dots &&\Ob{#2}{#4} \\[0.8cm]
\dots &\Ob{#1}{#4}
\end{array}
\begin{arrows}
\ncarr{#2#4}{#1#4} \blabel{#3_{#4}}[0.45][0.4]
\end{arrows}    
}

%\zigzagDotsThenFinalFlex{x}{y}{f}{n}{m}
% This is zig
% fm : yn -> xn
\NewDocumentCommand{\zigzagDotsThenFinalFlex}{m m m m m}{
\begin{array}{c c c}
\dots &&\Ob{#2}{#4} \\[0.8cm]
\dots &\Ob{#1}{#4}
\end{array}
\begin{arrows}
\ncarr{#2#4}{#1#4} \blabel{#3_{#5}}[0.45][0.4]
\end{arrows}    
}



%\zigzag{x}{y}{f}[ff]{n}
% This is a zig zag zig zag... zig
% For params 1 to 4, see command \zigzagZeroOne
% For param 5 see command \zigzagDotsThenFinal
%\zigzag{x}{y}{f}{n}
\NewDocumentCommand{\zigzag}{m m m o m}{
\zigzagZeroOne{#1}{#2}{#3}[#4]
\zigzagDotsThenFinal{#1}{#2}{#3}{#5}
}

%\zigzagFlex{x}{y}{f}{n}{n+m}
\NewDocumentCommand{\zigzagFlex}{m m m m m}{
\zigzagStart{#1}{#2}{#3}{#4}
\zigzagDotsThenFinalFlex{#1}{#2}{#3}{#4}{#5}
}

A zigzag  is a pair of sequences
$f$, and $f'$ of morphisms of \catcw such that for some $n \ge $, for some objects $x_0,..x_n$ and $y_1,...y_n$ of \catcw we have
$$
\arraycolsep=7pt
\zigzag{x}{y}{f}{n}
$$
in \catcw. 

Within a zigzag of length $n$ we find a zig for each $i$, $0 \le i \le n-1$, like so
$$
\begin{array} {c c c}
&\Ob{y}{i}&\\[0.8cm]
\Ob{x}{i}&&\Obs{x}{i}
\end{array}
\begin{arrows}
\ncarr{yi}{xi}  \blabel{f_i}[0.45][0.4]
\ncarr {yi}{xis}  \alabel{f'_i}[0.45][0.4]
\end{arrows}
$$

We will say that this i'th zig is \textit{tight} iff
$\overline{f_i}=\overline{f'_i}$.

We also find a  zag \foreachi, like so
%\arreset}
$$
\setlength{\arrnodesepA}{3pt}
\begin{array} {c c c}
\Obp{y}{i}&&\Ob{y}{i} \\[0.8cm]
&\Ob{x}{i}&
\end{array}
\begin{arrows}
\ncarr{yip}{xi}  \blabel{f'_i}[0.45][0.4]
\ncarr {yi}{xi}  \alabel{f_i}[0.45][0.4]
\end{arrows}
$$

and we will say that the i'th zag is tight iff
$\widehat{f'_i}=\widehat{f_i}$.

A zigzag we will say is a \textit{stickleback} iff its every zig and its every zag are tight. 

A stickleback $f: A \morph B$ is a pair of sequences
$f$, and $f'$ of morphisms of \catcw such that for some $n \ge $, for some objects $x_0,..x_n$ and $y_1,...y_n$ of \catcw we have
$$
\arraycolsep=7pt
\zigzag{x}{y}{f}{n}
$$
in \catcw.


\newcommand{\zigzagfgcomposite}
{
\arraycolsep=7pt
\zigzagZeroOne{w}{x}{f}
\begin{array}{c}
\dots \\[0.8cm]
\dots
\end{array}
\begin{array}{c c}
\Ob{x}{n} \\[0.8cm]
& \Ob{w}{n}=
\end{array}
\begin{arrows}
\ncarr{xn}{wn} \blabel{f'_{n-1}}[0.2][0.4] %0.2 was 0.45
\end{arrows}
\kern-15pt
\zigzag{y}{z}{g}[f_n]{m}
}

Composition of zigzags is given by
$$
\arraycolsep=7pt
\left( \zigzag{w}{x}{f}{n}\right) \circ \left(\zigzag{y}{z}{g}{m}\right)
$$
=
$$
\zigzagfgcomposite
$$

providing there is no morphism $k: y_1 \morph y_0$ such that the triangle
$$
$$
commutes. If there is such a morphism then the composition is  
$$
\arraycolsep=7pt
\zigzagZeroOne{w}{x}{f}
\begin{array}{c}
\dots \\[0.8cm]
\dots
\end{array}
\begin{array}{c c}
\Ob{x}{n} \\[0.8cm]
& \Ob{w}{n}
\end{array}
\begin{arrows}
\ncarr{xn}{wn} \blabel{f'_{n-1}}[0.2][0.4]
\end{arrows}
\begin{array}{c c}
\Ob{z}{1}& \\[0.8cm]
& \Ob{y}{2}
\end{array}
\begin{arrows}
\ncarr{z1}{wn}\blabel{g_1 \circ k}[0.45][0.4]
\ncarr{z1}{y2}\alabel{g'_1}[0.45][0.4]
\end{arrows}
\zigzagDotsThenFinal{y}{z}{g}{m}
$$


EXCEPT that the abobe compsitions are not necessarily sticklebacks because
it will not necessarily be the case that $\widehat{g_0)=\widehat{f'n}}$.

Therefore we have to do some modifications to both halves of the composition.
Modifications to first (f,f') half will be along these lines:

$$
\arraycolsep=7pt
\begin{array}{c c c c c}
\Rnode{xnpp}{x_{n-2}} && \kern0.25cm\Rnode{xnp}{x_{n-1}} && \kern0.5cm\Ob{z}{0}\\[1cm]
  & \kern0.25cm\Rnode{wnp}{w_{n-1}}  && \kern0.5cm\Ob{w}{n}
\end{array}
\begin{arrows}
\ncarr{z0}{wn}\alabel{g_1 f_n}[0.25][0.4]
\ncarr{xnp}{wn}\alabel{f'_{n-1}.\reallywidehat{g_1 f_n}}[0.4][0.4]
\ncarr{xnp}{wnp}\alabel{f_n.\overline{f'_{n-1}.\reallywidehat{g_1 f_n}}}[0.5][0.4]
\ncarr{xnpp}{wnp}\blabel{f'_{n-2}.\reallywidehat{f_n.\overline{f'_{n-1}\reallywidehat{g_1 f_n}}}}[0.45][0.4]
\end{arrows}
$$

\textbf{TIGHTENING a loose stickleback.}

start with

$$
\arraycolsep=7pt
\begin{array}{cccccc}
 &\bOb{v}& &\bOb{x}& &\bOb{z} \\[0.8cm]
\bOb{u}& &\bOb{w}& &\bOb{y}&
\end{array}
\begin{arrows}
\ncarr{v}{u}\blabel{e}[0.25][0.4]
\ncarr{v}{w}\alabel{f}[0.25][0.4]
\ncarr{x}{w}\blabel{g}[0.25][0.4]
\ncarr{x}{y}\alabel{h}[0.25][0.4]
\ncarr{z}{y}\blabel{i}[0.25][0.4]
\end{arrows}
$$

This tightens to 

$$
\arraycolsep=7pt
\begin{array}{cccccc}
 &\bOb{v}& &\bOb{x}& &\bOb{z} \\[0.8cm]
\bOb{u}& &\bOb{w}& &\bOb{y}&
\end{array}
\begin{arrows}
\ncarr{v}{u}\blabel{r_e}[0.25][0.4]
\ncarr{v}{w}\alabel{r_f}[0.25][0.4]
\ncarr{x}{w}\blabel{r_g}[0.25][0.4]
\ncarr{x}{y}\alabel{r_h}[0.25][0.4]
\ncarr{z}{y}\blabel{r_i}[0.25][0.4]
\end{arrows}
$$


where
\newcommand{\fConstriction}
{\reallywidehat{\overline{h\circ\widehat{i}}\circ g}}
\newcommand{\eRestriction}
{\overline{f\circ\fConstriction}}
\newcommand{\rebar}{\overline{e} \circ \eRestriction}
\newcommand{\rf}{\overline{e} \circ f \circ \reallywidehat{\overline{h \circ \widehat{i}} \circ g} }
\newcommand{\rfhat}{\widehat{\overline{e} \circ f} \circ \fConstriction}
\newcommand{\hRestriction}
{\overline{g \circ \reallywidehat{\overline{e} \circ f}}}
\newcommand{\rgbar}{\hRestriction\circ\overline{h\circ\widehat{i}}}
\newcommand{\rhhat}{\widehat{i} \circ \reallywidehat{\hRestriction \circ h}}
\newcommand{\ri}
{i \circ \reallywidehat{\hRestriction\circ h}}
\newcommand{\hRestrictionSimplified}
{\overline{g \circ \widehat{f}}}
\begin{align*}
r_e&= \eRestriction \circ e  
& \overline{r_e} &= \overline{e} \circ \eRestriction 
&\widehat{r_e}&=\reallywidehat{\eRestriction \circ e}\\
r_f&=\rf 
& \overline{r_f} &= \rebar
&\widehat{r_f} &= \rfhat\\
r_g&=\overline{h \circ \widehat{i}} \circ g \circ \reallywidehat{\overline{e} \circ f}
& \overline{r_g} &= \rgbar
&\widehat{r_g} &= \rfhat   \\
r_h&=\hRestriction\circ h\circ\widehat{i}
&\overline{r_h}&=\rgbar
&\widehat{r_h}&=\rhhat\\
r_i&=\ri&\overline{r_i}&=\overline{\ri}
&\widehat{r_i}&=\rhhat
\end{align*}
Note that $\overline{r_e}, \overline{r_f}, \overline{r_g}$ and $\overline{r_h}$ have been variously simplified using R.2 and R.3 and that $\overline{r_i}$ does not simplify. Likewise $\widehat{r_f}, \widehat{r_g} , \widehat{r_h}$ and $\widehat{r_i}$ have been variously simplified using lemma \ref{fglemma}, (i) and (iii), and $\widehat{r_e}$ does not simplify.

\textbf{WHAT HAPPENS IF ONLY THE LAST LEG FAILS THE TIGHTNESS CONDITION} 
Suppose $\overline{e}=\overline{f}$, $\widehat{f}=\widehat{g}$, $\overline{g}=\overline{h}$. Then we get the some simplifications as follows.

Stage one $bar{e}=\overline{f}$: 
\begin{align*}
&&& \overline{e}=\overline{f} && \widehat{f} = \widehat{g}  && \overline{g} = \overline{h} \\
r_e&= \eRestriction \circ e  
&& 
&& \\
r_f&=\rf  
&&= f \circ \reallywidehat{\overline{h \circ \widehat{i}} \circ g} 
&& \\
r_g &=\overline{h \circ \widehat{i}} \circ g \circ \reallywidehat{\overline{e} \circ f}
&&= \overline{h \circ \widehat{i}} \circ g \circ \reallywidehat{f}
&&=  \overline{h \circ \widehat{i}} \circ g \\
r_h &=\hRestriction\circ h\circ\widehat{i}
&&= \hRestrictionSimplified\circ h\circ\widehat{i} 
&&= \overline{g} \circ h \circ \widehat{i}
&&= h \circ \widehat{i} \\
r_i &=\ri 
&&=i \circ \reallywidehat{\hRestrictionSimplified \circ h}
&&=i \circ \reallywidehat{\overline{g} \circ h}
&&= i \circ \widehat{h}
\end{align*}

\textbf{Rename}
Rename $e,f,g,h,i$ to be $f_0,f'_0,f_1,f'_1,f_2$.

Rename $r_e,r_f,r_g,r_h,r_i$ to be $t_0,t'_0,t_1,t'_1,t_2$.

Then  \textbf{what happens if only the last leg fails the tightness condition:}

\begin{alignat*}{2}
t_2 &=f_2 \circ \widehat{f'_1} \\
t'_1&=f'_1\circ\widehat{t_2} &&= f'_1 \circ \widehat{f_2}\\
t_1&= \overline{t'_1}\circ f_1&=\overline{f'_1 \circ \widehat{f_2}} \circ f_1\\
t'_0&=f'_0 \circ \widehat{t_1}=f'_0 \circ \reallywidehat{\overline{f'_1 \circ \widehat{f_2}} \circ f_1}\\
t_0&=\overline{t'_0}\circ f_0
\end{alignat*}

\textbf{Same but with just $f_0,f'_0$ and $f_1$}

\begin{alignat*}{2}
t_1&= f_1 \circ \widehat{f'_0}\\
t'_0&=f'_0 \circ \widehat{t_1} &&=f'_0 \circ \widehat{f_1}\\
t_0&=\overline{t'_0} \circ f_0 &&= \overline{f'_0 \circ \widehat{f_1}} \circ f_0 
\end{alignat*}

\textbf{Same but substitute $g_0 \circ f_1$ for $f_1$.}
\begin{alignat*}{2}
t_1&= g_0 \circ f_1 \circ \widehat{f'_0}  &&= g_0 \circ f_1\\
t'_0&=f'_0 \circ \widehat{t_1} &&=f'_0 \circ \widehat{g_0 \circ f_1}\\
t_0&=\overline{t'_0} \circ f_0 &&= \overline{f'_0 \circ \widehat{g_0 \circ f_1}} \circ f_0 
\end{alignat*}



\subsection{Comparison of Zigzags}
%zigzagTuple{f}{n}
\newcommand{\zigzagTuple}[2]
{
\tuple{#1_0,#1'_0,...#1_{#2-1},#1'_{#2-1},#1_{#2}}    
}

Say that zigzags are comparable if they are of the same length
and have identical object sequences.
If $F=\zigzagTuple{f}{n}$ and 
$G=\zigzagTuple{g}{n}$
 are comparable zigzags then define $F \leq G$
by $F \leq G$ iff for each $i$, $0 \leq i \leq n$
$f_i \leq g_i$ and for each $i$, $0 \leq i \leq n-1$ $f'_i \leq g'_i$.

\subsection{Tightening}




If $F=\zigzagTuple{f}{n}$ is a zigzag
then define the one-step tightening $t_F$ to be the comparable zigzag, 
$T_F=\zigzagTuple{t}{n}$
where
\begin{align*}
t_0  &= \overline{f'_0}\circ f_0 \\
t'_i &= \overline{f_i} \circ f'_i \circ \widehat{f_{i+1}} &&\mbox{for each $i$, $0 \leq i \leq n-1$}\\
t_i &= \overline{f'_i}\circ f_i \circ \widehat{f'_{i-1}} &&\mbox{for each $i$, $1 \leq i \leq n-1$}\\
t_n &= f_n \circ \widehat{f_{n-1}}
\end{align*}

\begin{observation}
If $F$ is a zigzag and $T_F$ is its one-step tightening
then $T_F \leq F$.
\end{observation}

Let $T^p_F$ be the result of applying the one-step tightening to $F$ and then applying the tightening to the result and so on $p$ times.
\begin{observation}
For any zigzag $F$, there is a number $p \geq 0$ such that
$T^p_F$ is tight and therefore $T^{p+1}_F=T^p_F$.
\end{observation}
We shall use the notation $T^*_F$ for this and say that it is the absolute tightening of $F$.

\begin{lemma}
\llabel{tighteningStepLemma}
If $F$ is a zigzag and and 
if $G$ is a zigzag that is a comparable  to $F$ and is tight then if $G \leq F$
then $G \leq T_F$. 
\end{lemma}
\begin{proof}
There are four cases to consider.
Holds in the first case as follows.
\begin{align*}
&g'_0 \leq f'_0 \\
\mbox{therefore }&\overline{g'_0} \leq \overline{f'_0} \\
\mbox{therefore }&\overline{g_0} \leq \overline{f'_0} &\mbox{because g tight, $\overline{g'_0} = \overline{g_0}$ }
\end{align*}
Since $g_0 \leq f_0$ and $\overline{g_0} \leq \overline{f'_0}$
then from lemma \ref{leqlemma} it follows that
$g_0 \leq \overline{f'_0}\circ f_0$, as required.
\end{proof}
\begin{corollary}
\llabel{absoluteTighteningCorollary}
If $F$ is a zigzag and $T^*_F$ is tha absolute tightening of $F$, 
if $G$ is a zigzag that is a comparable  to $F$ and is tight then if $G \leq F$
then $G \leq T^*_F$. 
\end{corollary}



\begin{observation}
\llabel{tailSpecialisationLemma}
\commentary{reword as any tightening????}
If $F_a=\tuple{f_0,f'_0,...f_{n-1},f'_{n-1},a}$
and $F_b=\tuple{f_0,f'_0,...f_{n-1},f'_{n-1},b}$
are zigzags that, as you can see, differ only in the final position and
if $\widehat{b} \leq \widehat{a}$
then 
\begin{enumerate}[(i)]
\item if $T_{F_a}=\zigzagTuple{t}{n}$
is the one-step tightening of $F_a$
and  $T_{F_b}=\zigzagTuple{s}{n}$
is the one-step tightening of $F_b$ then
for each $i$, $0 \leq i \leq n-1$, $s_i \leq t_i$ and $s'_i \leq t'_i$,
\item if $T^*_{F_a}=\zigzagTuple{t}{n}$
is the absolute tightening of $F_a$
and  $T^*_{F_b}=\zigzagTuple{s}{n}$
is the absolute tightening of $F_b$ then
for each $i$, $0 \leq i \leq n-1$, $s_i \leq t_i$ and $s'_i \leq t'_i$.
\end{enumerate}
\end{observation}

\begin{observation}
\llabel{tailLemma}
\commentary{reword as any tightening}
If $F_a=\tuple{f_0,f'_0,...f_{n-1},f'_{n-1},a}$
and $F_b=\tuple{f_0,f'_0,...f_{n-1},f'_{n-1},b \circ a}$
are zigzags that, as you can see, differ only in the final position 
then 
\begin{enumerate}[(a)]
\item
if $T_{F_a}=\zigzagTuple{t}{n}$
is the one-step tightening of $F_a$
and  $T_{F_b}=\zigzagTuple{s}{n}$
is the one-step tightening of $F_b$ then
\begin{enumerate}[(i)]
\item for each $i$, $0 \leq i \leq n-1$, 
$s_i \leq t_i$ and $s'_i \leq t'_i$,
\item $s_n = b \circ t_n$,
\end{enumerate}
\item
if $T^*_{F_a}=\zigzagTuple{t}{n}$
is the absolute tightening of $F_a$
and  $T^*_{F_b}=\zigzagTuple{s}{n}$
is the absolute tightening of $F_b$ then
\begin{enumerate}[(i)]
\item for each $i$, $0 \leq i \leq n-1$, 
$s_i \leq t_i$ and $s'_i \leq t'_i$,
\item $s_n = b \circ t_n$.
\end{enumerate}
\end{enumerate}
\end{observation}

\begin{observation}
\llabel{headLemma}
\commentary{reword as any tightening}
If $G_a=\tuple{a,f'_0,...f_{n-1},f'_{n-1},f_n}$
and $G_b=\tuple{a \circ b ,f'_0,...f_{n-1},f'_{n-1},f_n}$
are zigzags that, as you can see, differ only in the first position 
then 
\begin{enumerate}[(a)]
\item
if $T_{G_a}=\zigzagTuple{t}{n}$
is the one-step tightening of $G_a$
and  $T_{G_b}=\zigzagTuple{s}{n}$
is the one-step tightening of $G_b$ then
\begin{enumerate}[(i)]
\item $s_0 = t_n \circ b $,
\item for each $i$, $1 \leq i \leq n$, 
$s_i \leq t_i$,
\item for each $i$, $0 \leq i \leq n-1$, 
$s'_i \leq t'_i$,
\end{enumerate}
\item
if $T^*_{F_a}=\zigzagTuple{t}{n}$
is the absolute tightening of $F_a$
and  $T^*_{F_b}=\zigzagTuple{s}{n}$
is the absolute tightening of $F_b$ then
\begin{enumerate}[(i)]
\item $s_0 = t_n \circ b $,
\item for each $i$, $1 \leq i \leq n$, 
$s_i \leq t_i$,
\item for each $i$, $0 \leq i \leq n-1$, 
$s'_i \leq t'_i$.
\end{enumerate}
\end{enumerate}
\end{observation}



\begin{observation}
\llabel{biggeristighter}
If we have two zigzags $F$ and $G$ and one is an extension of the other
then the tightening of the longer one truncated to be comparable with the smaller is less than or equal the tightening of the smaller one.\commentary{Trivial but needs spelling out.}
\end{observation}


\begin{lemma}
\llabel{tighteningCommutesWithCompositionLemma}

If $F=\zigzagTuple{f}{n}$ is a zigzag shaped like this
$$
\arraycolsep=7pt
\zigzag{w}{x}{f}{n}
$$
and if $G=\zigzagTuple{g}{m}$ is a zigzag shaped like this
$$
\arraycolsep=7pt
\zigzag{y}{z}{g}{m}
$$

and if $w_n = y_0$ so that there is a composite zizzag which we will denote 
$F \circ G$ like so
$$
\arraycolsep=7pt
\zigzagfgcomposite
$$
then if 
\begin{itemize}
\item $T^*_F$ is the absolute tightening of $F$,
\item $T^*_G$ is the absolute tightening of $G$,
\item $T^*_{T^*_F \circ T^*_G}$ is the absolute tightening of $T^*_F \circ T^*_G$
\item $T^*_{F\circ G}$ 
        is the absolute tightening of $F \circ G$
\end{itemize}
then $T^*_{F\circ G}=T^*_{T^*_F \circ T^*_G}$
\end{lemma}

\newcommand{\tightenedComposite}[1]
{
\zigzagZeroOneShort{w}{x}{#1}
\begin{array}{c}
\dots \\[0.8cm]
\dots
\end{array}
\begin{array}{c c}
\Ob{x}{n} \\[0.8cm]
& \Ob{w}{n}=
\end{array}
\begin{arrows}
\ncarr{xn}{wn} \blabel{#1'_{n-1}}[0.2][0.4] %0.2 was 0.45
\end{arrows}
\kern-15pt
\arraycolsep=7pt
\zigzagFlex{y}{z}{#1}{n}{n+m}
}


\begin{proof}
Suppose 
\begin{itemize}
\item $T^*_F=\zigzagTuple{t}{n}$,
\item $T^*_G=\zigzagTuple{s}{m}$,
\item $T^*_{T^*_F \circ T^*_G}=\zigzagTuple{r}{n+m}$so that we have
$$
\arraycolsep=7pt
\tightenedComposite{r},
        $$
\item $T^*_{F\circ G}=\zigzagTuple{q}{n+m}$ 
so that we have
$$
\arraycolsep=7pt
\tightenedComposite{q}.
        $$
\end{itemize}
We will prove the result by showing 
(a) that $T^*_{T^*_F \circ T^*_G} \leq T^*_{F\circ G}$
and (b) that $T^*_{F\circ G} \leq T^*_{T^*_F \circ T^*_G}$.

To establish (a), since $T^*_{T^*_F \circ T^*_G}$ is tight,
it suffices to show that 
$T^*_{T^*_F \circ T^*_G} \leq F\circ G$. 
We know that
$$
T^*_{T^*_F \circ T^*_G}
\leq \tuple{t_0,t'_0,...t'_{n-1},s_0 \circ t_n,s'_0,...s_m}
$$
and so it suffices to show that 
$$
\tuple{t_0,t'_0,...t'_{n-1},s_0 \circ t_n,s'_0,...s_m}
\leq \tuple{f_0,f'_0,...f'_{n-1},g_0 \circ f_n,g'_0,...g_m}
$$
This follows because
$$\zigzagTuple{t}{n} \leq \zigzagTuple{f}{n}$$
as the left zigzag is defined to be a tightening of the right,
likewise we have
$$\zigzagTuple{s}{m} \leq \zigzagTuple{g}{m}$$
 and because
$s_0 \circ t_n \leq g_0 \circ f_n$ follows 
from $s_0 \leq g_0$ and $t_n \leq f_n$ by lemma \ref{leqlemma}.

To establish (b), because $T^*_{F \circ G}$ is tight it suffices to show that
$T^*_{F \circ G} \leq T^*_F \circ T^*_G$. In other words that
$$
\tuple{q_0,q'_0,...q'_{n-1},q_n,q'_n,...q_{n+m}}\leq \tuple{t_0,t'_0,...t'_{n-1},s_0 \circ t_n,s'_0,...s_m}
$$
We prove this piecemeal. 
For the initial segment as far as $q_{n-1}$ 
Because $\tuple{q_1,...q_{n+m}}$ is the tightening of $F \circ G$ it follows that
$$
\tuple{q_1,...q'_{n-1}} \leq \tuple{f_1,...f'_{n-1}} \leq \tuple{t_1,...t'_{n-1}}
$$
For the final segment after $q'_n$ we have
$$
\tuple{q'_n,q_{n+1},...q_{n+m}} \leq \tuple{g'_0,...g_{m}} \leq \tuple{s'_1,...s_m}
$$
That leaves us having to prove that $q_n \leq s_0 \circ t_0$.
Now denote by $F \circ g_0$ the zigzag that is
the initial segment of the zigzag $F \circ G$
that is $\tuple{f_0,...f'_{n-1},g_0 \circ f_n}$
and denote by $f_n \circ G$ the zigzag that is 
the final segment of $F \circ G$
that is $\tuple{g_0 \circ f_n,g'_0,...g_m}$
and let $T_{F \circ g_0}$ and $T_{f_n \circ G}$ be the absolute tightenings of these zigzags.
Because $\tuple{q_1,...q_{n+m}}$ is the tightening of $F \circ G$ it follows
using observation \lref{biggeristighter}  that both
\begin{equation}
\label{qTFg}
\tuple{q_1,...q'_{n-1},q_n} \leq T_{F \circ g_0}
\end{equation}
and
\begin{equation}
\label{qTfG}
\tuple{q_n, q'_n,...q_{n+m}} \leq T_{f_n \circ G}
\end{equation}
By lemma \lref{tailLemma} the final element of $T_{F \circ g_0}$ is $g_0 \circ t_n$.
Therefore from (\ref{qTFg}) we have in particular that
\begin{equation}
\label{qgtn}
q_n \leq g_0 \circ t_n.
\end{equation}
Likewise using lemma \lref{headLemma} we have that the initial element
of $T_{f_n \circ G}$ is $s_0 \circ f_n$ and therefore from 
(\ref{qTfG}) we have the particular case that
\begin{equation}
\label{qsfn}
q_n \leq s_0 \circ f_n.
\end{equation}
Finally we use lemma \lref{pointwiseleqlemma} to establish  from
(\ref{qgtn}) and (\ref{qsfn}) that
$q_n \leq s_0 \circ t_0$ and so complete the proof.
\end{proof}
\subsection{The category of tight zigzags}
If \catcw is a range category then we can define the category
$Z_C$ to be the category whose objects are the objects of \catcw
and such that $Hom_{Z_C}(a,b)$ is the set of tight zigzags of any length $n$
$$
\arraycolsep=7pt
\zigzag{x}{y}{f}{n}
$$
such that $x_0=a$ and $y_n=b$.

\newcommand{\conc}{::}

Composition in $Z_C$ is defined by composing the zigzags and then tightening.  Write this as
$$F \circ G =_{def} T(F :: G)$$
so that what previously we denote $T^*_F$ we now denote T(F).

Composition is associative since 
\begin{align*}
(F \circ G) \circ H 
                &= T(T(F \conc G) \conc H) \\
                &= T(T(F \conc G) \conc T(H))
                        &\mbox{since $H$ is tight,}\\
                &= T((F \conc G) \conc H)  
                        &\mbox{by lemma \ref{tighteningCommutesWithCompositionLemma}}\\
                &= T(F \conc (G \conc H))
                        &\mbox {since $\conc$ is associative} \\
                &= T(T(F) \conc T(G \conc H))       
                        &\mbox{by lemma \ref{tighteningCommutesWithCompositionLemma}}\\
                &= T(F \conc T(G \conc H))))
                        &\mbox{since $F$ is tight} \\
                &= F \circ (G \circ H)                     
\end{align*}

Identity morphisms in $Z_C$ consist of the identity morphisms of \catcw considered 
as zigzags on length 1.

\begin{lemma}
\llabel{leftLowLemma}
If $\tuple{f_0,f'_0,f_1}$ is a zigzag and if
\[
\overline{f_0} \leq \overline{f'_0}
\]
i.e. 
\begin{equation}
\label{leftLowInequality}
\overline{f_0} \circ \overline{f'_0} = \overline{f_0}
\end{equation}
and
\begin{equation}
\label{leftLowEquality}
\widehat{f'_0} = \widehat{f_1}
\end{equation}
then
\[
T(\tuple{f_0,f'_0,f_1})=\tuple{f_0,
                                 \overline{f_0} \circ f'_0,
                                 f_1 \circ \widehat{\overline{f_0} \circ f'_0}
                                }
\]
\end{lemma}
\begin{proof}
\commentary{this is a good example of tightening and need to do a rightLowLemma also}
First see that
\begin{align*}
t(\tuple{f_0,f'_0,f_1}) 
                &= \tuple{f_0 \circ \overline{f'_0}, 
                          \overline{f_0} \circ f'_0 \circ \widehat{f_1},
                           f_1 \circ \widehat{f'_0}
                          }     
                    && \mbox{by following the definition of $t$,}    \\
                &= \tuple{f_0, \overline{f_0} \circ f'_0,f_1} 
                    &&\mbox{after simplifying using (\ref{leftLowInequality}) and (\ref{leftLowEquality}),} 
\end{align*}
from which we calculate that
\begin{align*}
t^2(\tuple{f_0,f'_0,f_1})
           &= \tuple{
                    \overline{\overline{f_0} \circ f'_0} \circ f_0,
                    \overline{f_0} \circ \overline{f_0} \circ f'_0 \circ \reallywidehat{f_1},
                    \reallywidehat{\overline{f_0} \circ f'_0} \circ  f_1 
                    }
           &&  \mbox{by applying the definition of $t$ to $\tuple{f_0, \overline{f_0} \circ f'_0,f_1}$,} \\
           &= \tuple{f_0, 
                     \overline{f_0} \circ f'_0,
                     \reallywidehat{\overline{f_0} \circ f'_0} \circ  f_1 
                     } 
           &&  \mbox{after simplification.}
\end{align*}
Finally, the zig of $\tuple{f_0, \overline{f_0} \circ f'_0,f_1 \circ \widehat{\overline{f_0} \circ f'_0}}$
is tight because we have that  (\ref{leftLowInequality})
and  the zag is tight because
\[ 
  \reallywidehat{\overline{f_0} \circ f'_0} = \reallywidehat{f_1 \circ \widehat{\overline{f_0} \circ f'_0}}
\]
because handwritten sheet A
\begin{align*} 
\reallywidehat{\overline{f_0} \circ f'_0} 
                &=  \widehat{f'_0} \circ \reallywidehat{\overline{f_0} \circ f'_0} 
                &&\mbox{lemma  \ref{fglemma} (ii)}\\
                &= \widehat{f_1} \circ \reallywidehat{\overline{f_0} \circ f'_0}
                &&\mbox{by (\ref{leftLowEquality})} \\ 
                &= \reallywidehat{f_1 \circ \widehat{\overline{f_0} \circ f'_0}}
                &&\mbox{by lemma \ref{fglemma} (iii)}
\end{align*}
Thus 
\begin{align*}
T(\tuple{f_0,f'_0,f_1})&= t^2(\tuple{f_0,f'_0,f_1})\\
                       &=\tuple{f_0,
                                 \overline{f_0} \circ f'_0,
                                 f_1 \circ \widehat{\overline{f_0} \circ f'_0}
                                }
\end{align*}
as required.
\end{proof}

\begin{corollary}
\llabel{rightLowCorollary}
Suppose we have the following morphsisms in a range category \catc:
$$
\arraycolsep=7pt
\begin{array}{ccccc}
 &&\bOb{x}& &\bOb{z} \\[0.8cm]
&\bOb{w}& &\bOb{y}& \\[0.8cm]
\bOb{v}&& &       & 
\end{array}
\begin{arrows}
\ncarr{w}{v}\blabel{g}[0.25][0.4]
\ncarr{x}{w}\blabel{f_0}[0.25][0.4]
\ncarr{x}{y}\alabel{f'_0}[0.25][0.4]
\ncarr{z}{y}\blabel{f_1}[0.25][0.4]
\end{arrows}
$$
so that $\tuple{f_0,f'_0,f1}$ and $\tuple{f_0 \circ g,f'_0,f1}$ are zigzags,
then if the zigzag $\tuple{f_0,f'_0,f1}$ is tight then
\[
T(\tuple{f_0 \circ g,f'_0,f_1})=\tuple{f_0 \circ g,
                                 \overline{f_0 \circ g} \circ f'_0,
                                 f_1 \circ \widehat{\overline{f_0 \circ g} \circ f'_0}
                                }.
\]
\end{corollary}
\begin{proof}
The lemma can be applied to  $\tuple{f_0 \circ g,f'_0,f_1}$ because of the tightness of  $\tuple{f_0,f'_0,f1}$ and
because $\overline{f_0 \circ g} \leq \overline{f_0} = \overline{f'_0}$.
\end{proof}
\subsection{Can we prove that $Z_C$ is a restriction category?}


If $F=\zigzagTuple{f}{n}$ is a morphism in $Z_C$ then define $\overline{F}$ to be $\tuple{\widehat{f_0}}$ the  zigzag of length $1$ determined by the morphism $\widehat{f_0}$ of \catcw.

We need show

R.1 For $F:a \morph b$ in $Z_C$ $$\overline{F} \circ F =F$$.

R.2. If \FGsourcediag in $Z_C$ then
$$\overline{G} \circ \overline{F}=\overline{F} \circ \overline{G}.$$

R.3. If \FGsourcediag in $Z_C$ then
$$\overline{\overline{F} \circ G} = \overline{F} \circ \overline{G}$$.

R.4. If $\sequentialdiag{a}{b}{c}{F}{G}$ in $Z_C$ then
$$F \circ \overline{G} = \overline{F \circ G} \circ F$$.

R.1 follows because in \catcw, $f_0 \circ \widehat{f_0} = f_0$. 

R.2 follows because in \catcw $\widehat{f_0} \circ \widehat{g_0} = \widehat{g_0} \circ \widehat{f_0}$.

R.3 follows because
\commentary{work on this proof}
\begin{align*}
\overline{\overline{F} \circ G} 
                   &= \overline{T(\tuple{g_0 \circ \widehat{f_0},g'_0,...g_m})}\\
                   &= \overline{\tuple{g_0 \circ \widehat{f_0},s'_0,...s_m}}
                                             &&\mbox{\parbox{5cm}{for some $s'_0,...s_m$
                                                       by some new lemma about tighteniong when only head position loose,}}\\
                   &=\tuple{\widehat{g_0 \circ \widehat{f_0}}} 
                                            && \mbox{by definition}\\
                   &=\tuple{\widehat{g_0} \circ \widehat{f_0}}  
                                            && \mbox{by lemma \ref{fglemma} (iii)}\\
                   &=\overline{F} \circ \overline{G}      
\end{align*}



to prove R.4 we need first prove Cockett et als.: wR.4 which is that

\begin{equation}
\label{restrictionlemmiiasR4Antecedent}
\overline{F \circ G}=\overline{F \circ \overline{G}}
\end{equation}


we can then proceed as follows.

\textbf{init(F) :: last(F)}

First consider that if $F$ is the zigzag
$$
\arraycolsep=7pt
\zigzag{x}{y}{f}{n}
$$
then if $n \geq 2$ then $F$ 
then it can be expressed as  the composition of zigzags
\begin{equation}
\label{FheadTail}
F=init(F)::last(F)
\end{equation}
where
\begin{align*}
init(F) &= \tuple{f_0,f'_0,...,f'_{n-2},f_{n-1}} \\
last(F) &= \tuple {\overline{f'_{n-1}},f'_{n-1},f_n}
\end{align*}

We will first prove that R.4 holds for $F$ of length $2$, the case of $n=2$,
and then prove R.4 holds for $F$ of length greater than $2$ by induction on $n$.

\textbf{Case $n=2$}

Suppose $that F$ is the zigzag
$$
\arraycolsep=7pt
\begin{array}{cccc}
 &\bOb{x}& &\bOb{z} \\[0.8cm]
\bOb{w}& &\bOb{y}&
\end{array}
\begin{arrows}
\ncarr{x}{w}\blabel{f_0}[0.25][0.4]
\ncarr{x}{y}\alabel{f'_0}[0.25][0.4]
\ncarr{z}{y}\blabel{f_1}[0.25][0.4]
\end{arrows}
$$
of length $2$.
We need show that

$$
F \circ \overline{G}  = \overline{F \circ G} \circ F 
$$


\begin{align*}
LHS &= \tuple{f_0,f'_0,f_1} \circ \overline{G} \\
    &= T(\tuple{f_0,f'_0,\reallywidehat{g_0} \circ f_1}) &&\mbox{using the definition of the $\bar{}$ operator for zigzags} \\
    &= \tuple{ \overline {f'_0  \circ \reallywidehat{g_0 \circ f_1}} \circ f_0, 
                 f'_0  \circ \reallywidehat{g_0 \circ f_1}, 
                 \reallywidehat{g_0} \circ f_1   
                }  && \mbox{by earlier that needs be lemmarised, and by RR.4 (twice)}
\end{align*}

\begin{align*}
RHS &=  \overline{F \circ G} \circ F \\
    &= \overline{F \circ \overline{G}} \circ F 
                  &&\mbox{ using (\ref{restrictionlemmiiasR4Antecedent}),}\\
    &= \overline{\tuple{ \overline {f'_0  \circ \reallywidehat{g_0 \circ f_1}} \circ f_0, 
                 f'_0  \circ \reallywidehat{g_0 \circ f_1}, 
                 \reallywidehat{g_0} \circ f_1   
                }} \circ F 
                  &&\mbox{using LHS as above,} \\
    &= \reallywidehat{\overline {f'_0  \circ \reallywidehat{g_0 \circ f_1}} \circ f_0} \circ F
                  && \mbox{ from definition of the $\bar{\ }$ operator,} \\
    &= T(\tuple{
            f_0 \circ\reallywidehat{\overline {f'_0  \circ \reallywidehat{g_0 \circ f_1}} \circ f_0},
            f'_0,
            f_1
        })         
                  && \mbox{from definition of $\circ$}\\
    &= \tuple{
            t_0,
            \overline{t_0} \circ f'_0,
            f_1 \circ \reallywidehat{\overline{t_0} \circ f'_0}
        } \mbox{ where } t_0 = f_0 \circ\reallywidehat{\overline {f'_0  \circ \reallywidehat{g_0 \circ f_1}} \circ f_0} 
    && \mbox{ by lemma \ref{leftLowLemma} leftLowLemma}\\
    &= &&\mbox{something that  hopefully matches rhs of LHS, above}   \\
    &= \mbox{\highlight{but it doesn't equate to LHS and we fail!}}
\end{align*}

\textbf{case $n > 2$}
For $F$ of length greater than $2$ we prove by induction. The inductive hypothesis is
\begin{equation}
\label{R4InductiveHypothesis}
init(F) \circ \overline{G} = \overline{init(F) \circ G} \circ init(F)
\end{equation}
Note first that
\begin{equation}
\label{headTailAndG} 
init(F) \circ tail(F) \circ G= F \circ G
\end{equation} 
follows from the definitions and from lemma \ref{tighteningCommutesWithCompositionLemma}.

\begin{align*}
F \circ \overline{G} 
    &= (init(F) \conc last(F)) \circ \overline{G}
                            && \mbox{using (\ref{FheadTail}),}\\
    &= T( init(F) \conc last(F) \conc \overline{G}) 
                            && \mbox{from definition of $\circ$ and assicativity of $\conc$,}\\
    &= T( init(F) \conc T(last(F) \conc \overline{G}) )
                            && \mbox{by lemma \ref{tighteningCommutesWithCompositionLemma},}\\
    &= T( init(F) \conc (last(F) \circ \overline{G}) )
                            && \mbox{using definition of $\circ$,}\\
    &= T(init(F) \conc (\overline{last(F) \circ G} \circ last(F)) )
                            && \mbox{using $n=1$ case,} \\
    &= init(F) \circ \overline{last(F) \circ G} \circ last(F) 
                            && \mbox{using definition of $\circ$,} \\
    &= \overline{init(F) \circ last(F) \circ G} \circ init(F)  \circ last{F} 
                            && \mbox{using the inductive hypothesis (\ref{R4InductiveHypothesis}),} \\
    &= \overline{F \circ G} \circ F 
                                && \mbox{as required, by applying (\ref{headTailAndG}).}
\end{align*}

\subsection{Turn zigzags around}
Looking at Cockett's Shein's lemma construction make me think I have zigzags the wrong way around.
That hios zigzag from x to y should be a morphism from y to x in the category of zigzags.
I can't see this making a difference but I need to check out the preceding failed lemma to see if it succeeds when things are turned around. I can't see how this can be but I better try it anyway. I will edit the previous by reversing the directions of the arrows. It is better this way anyway because the embedding from \catcw into zigzags of \catcw is a more obvious. 
Start with the bit  that I couldn't prove and see if I can now prove it even though I don't see how it can have changed significantly so that it goes through.
\textbf{prove that R.4 holds for $F$ of length 2.}
Suppose $that F$ is the zigzag
$$
\arraycolsep=7pt
\begin{array}{cccc}
 &\bOb{x}& &\bOb{z} \\[0.8cm]
\bOb{w}& &\bOb{y}&
\end{array}
\begin{arrows}
\ncarr{w}{x}\alabel{f_0}[0.5][0.4]
\ncarr{y}{x}\blabel{f'_0}[0.5][0.4]
\ncarr{y}{z}\blabel{f_1}[0.5][0.4]
\end{arrows}
$$
of length $2$.
We need show that

$$
F \circ \overline{G}  = \overline{F \circ G} \circ F 
$$


\highlight{the following looks right}
\begin{align*}
LHS &= \tuple{f_0,f'_0,f_1} \circ \overline{G} \\
    &= T(\tuple{f_0,f'_0, f_1 \circ \overline{g_0}}) 
                   &&\mbox{using the definition of the $\bar{}$ operator for zigzags} \\
    &= \tuple{f_0 \circ \reallywidehat {\overline{f_1 \circ g_0} \circ f'_0}, \ 
                 \overline{f_1 \circ g_0 } \circ f'_0, \
                 f_1 \circ \overline{g_0}   
                }  && \mbox{introduce some earlier lemma to cover this}
\end{align*}



\highlight{begin editing }
\begin{align*}
RHS &=  \overline{F \circ G} \circ F \\
    &= \overline{F \circ \overline{G}} \circ F 
                  &&\mbox{ using a (\ref{restrictionlemmiiasR4Antecedent}) CHECK --- THIS IS wR4,}\\
    &= \overline{\tuple{f_0 \circ \reallywidehat {\overline{f_1 \circ g_0} \circ f'_0}, \ 
                 \overline{f_1 \circ g_0 } \circ f'_0, \
                 f_1 \circ \overline{g_0}   
                }} \circ F 
                  &&\mbox{using LHS as above,} \\
    &= \tuple{\overline{f_0 \circ \reallywidehat {\overline{f_1 \circ g_0} \circ f'_0}}} \circ F
                  && \mbox{ from definition of the $\bar{\ }$ operator,} \\
    &= T(\tuple{
            \overline{f_0 \circ \reallywidehat {\overline{f_1 \circ g_0} \circ f'_0}}
                                \circ f_0,\ 
            f'_0,\ 
            f_1
        })         
                  && \mbox{from definition of $\circ$}\\
    &= \tuple{
            t_0,
            f'_0 \circ \widehat{t_0} ,
            f_1 \circ \overline{\widehat{f'_0 \circ t_0}}
        } \mbox{ where } 
        t_0 = \overline{f_0 \circ \reallywidehat {\overline{f_1 \circ g_0} \circ f'_0}} \circ f_0 
    && \mbox{ by lemma \ref{leftLowLemma} leftLowLemma OR SUCH} \\
    &= &&\mbox{something that  hopefully matches rhs of LHS, above}   \\
    &= \mbox{\highlight{does it equate to LHS or we fail!}}
\end{align*}

We can prove using R.3 and RR.1, for starters, that
\begin{equation}
\overline{f_0 \circ \reallywidehat {\overline{f_1 \circ g_0} \circ f'_0}} \circ f_0 
= f_0 \circ \reallywidehat {\overline{f_1 \circ g_0} \circ f'_0}
\end{equation}

Now the second term of RHS is
\[
f'_0 \circ \reallywidehat{f_0 \circ \reallywidehat {\overline{f_1 \circ g_0} \circ f'_0}}
\]
we need to show this is equal to 
\[
\overline{f_1 \circ g_0 } \circ f'_0
\]

The second term simplifies using fglemma (iii) to
\[
f'_0 \circ \widehat{f_0} \circ \reallywidehat {\overline{f_1 \circ g_0} \circ f'_0}
\]

then simplies to
\[
f'_0 \circ \reallywidehat {\overline{f_1 \circ g_0} \circ f'_0}
\]

which doesn't go thru \highlight{FAILS AS BEFORE} see What isn't true.

\begin{notebox}[Jan 27th 2026]
The key lemma then is to show that when F and G are composable sticklebacks 
where G is of length 1 then 
\[
\overline{F \circ G} = \overline{F} \mbox{ iff } F \circ \overline{G} = F
\]
This, for morphism G, will be the HomP support preservation property.
\end{notebox}

\begin{lemma}
If $F=\zigzagTuple{f}{n}$ and $G=\tuple{g_0}$ are tight zigzags such that
 $F:x \morph y$ and $G: y \morph z$ in the category of sticklebacks 
then  
\[
\overline{F \circ G} = \overline{F} \mbox{ iff } F \circ \overline{G} = F
\]
\end{lemma}
\begin{proof}
The composition of $F$ with $\overline{G}$ is the tightening of 
\[
\tuple{f_0,f'_0,...f_{n-1},f'_{n-1},f_n \circ \overline{g_0}}
\]
This equals $F$ iff $f_n \circ \overline{g_0}=f_n$ iff $\widehat{f_n}=\overline{g_0}$.
NEED LEMMAS FOR THIS.

SIMILar argumant for RHS.
\end{proof}