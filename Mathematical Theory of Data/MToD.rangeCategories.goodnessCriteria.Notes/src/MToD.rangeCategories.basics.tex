


\section{Restriction Categories (\cite{COCKETT2002})}

For the theory of data, 
the notion of \textit{range category} that we are interested in 
is defined in terms of the slightly simpler notion of a 
\textit{restriction category}. These were introduced by 
Cockett \& Lack (\cite{COCKETT2002}) and 
constitute an algebra that nicely represents a system of partially defined terms or maps. Applied to the theory of data, they are relevant because data fields often only optionally have values and can be considered therefore to be partial, rather than total, functional relationships. 

Cockett \& Lack define restriction categories along the following lines:
\begin{definition}
A \textit{restriction category} is a category \catcw with an additional operator on morphisms (the restriction operator $\bar{\ }$), so that from a morphism
$f: a \morph b$ in \catcw there is a 
$$\bar{f}: a \morph a$$,
satisfying

R.1 For $f:a \morph b$ in \catcw $$\bar{f} \circ f =f$$.

R.2. If \fgsourcediag in \catcw then
$$\bar{g} \circ \bar{f}=\bar{f} \circ \bar{g}.$$

R.3. If \fgsourcediag in \catcw then
$$\overline{\bar{f} \circ g} = \bar{f} \circ \bar{g}$$.

R.4. If $\sequentialdiag{a}{b}{c}{f}{g}$ in \catcw then
$$f \circ \bar{g} = \overline{f \circ g} \circ f$$.
\end{definition}


Though we are mainly restriction categories are the structures that we are interested 
one of the constructions we use relies on categories with a such an operator 
Cockett et al \cite{COCKETT2012} describe as categories with support 
such a operator \barItself for which R.4 does not hold, though R.1, R.2 and R.3 so, and for which  the following
weaker condition holds

wR.4  If $\sequentialdiag{a}{b}{c}{f}{g}$ in \catcw then
\[
\overline{f \circ g} = \overline{f \circ \bar{g}}.
\]
Such categories Cockett et al refer to as \textit{categories with support}. 

Some properties of the restriction operator in that follow from the definition of restriction category are as are as follows:




\begin{lemma}
\llabel{restrictioncatlemma}
In any restriction category \catc
\begin{enumerate} [(i)]
\item For $f:a \morph b$ in \catcw,
$$\bar{\bar{f}}=\bar{f}$$
\item w.R4.\footnote{This condition described as wR.4 in Cockett et al.} 
If $\sequentialdiag{a}{b}{c}{f}{g}$ in \catcw 
 then
$$\overline{f \circ g} = \overline{f \circ \bar{g}}.$$

\item \llabel{CHECKLemma}
If \catcw is a restriction category \catcw and if
$\sequentialdiag{a}{b}{c}{f}{g}$ in \catcw then 
$f \circ \bar{g} = f$ iff $\overline{f \circ g} = \overline{f}.$
\item \llabel{restrictioncatlemma:totalk} \commentary{UNUSED}
If $\sequentialdiag{a}{b}{c}{k}{j}$ in \catcw 
and $k$ is total i.e. $\bar{k}=id_a$
and if $k \circ \bar{j}$ is total i.e. $\overline{k \circ \bar{j}}=id_a$
then
$$k \circ \bar{j} = k.$$
\end{enumerate}
\end{lemma}
\begin{proof}
\begin{enumerate} [(i)]
\item Use R3 with $g$ being $id_b$.
\item 
\begin{align*}
\overline{f \circ \bar{g}}
&= \overline{\overline{f \circ g} \circ f}
    && \text{by (R4),} \\
&= \overline{f \circ g} \circ \bar{f}
    && \text{by (R3),} \\
&= \bar{f} \circ \overline{f \circ g}
    && \text{by (R2),} \\
  &= \overline{\bar{f} \circ f \circ g}
    && \text{by (R3),} \\  
&= \overline{f \circ g}
    && \text{by (R1).}
\end{align*}
\item \commentary{CHECKLemma}
LEFT TO RIGHT
Assume $f \circ \bar{g} = f$
then
\begin{align*}
\overline{f \circ g} &= \overline{f \circ \bar{g}} && \mbox{by wR.4,}\\
                     &= \overline{f} && \mbox{from the assumption.}
\end{align*}
RIGHT TO LEFT
Assume $\overline{f \circ g} = \overline{f}$
then
\begin{align*}
f \circ \bar{g} &= \overline{f \circ g} \circ f 
                                  && \mbox{R.4,} \\
                &= \bar{f} \circ f 
                                  && \mbox{from the assumption,} \\
                &= f              && \mbox{by R.1.}
\end{align*}
\item 
\begin{align*}
k \circ \bar{j} &= k \circ \bar{\bar{j}}              && \mbox{from (i),}\\
                &= \overline{k \circ \bar{j}} \circ k && \mbox{by R.4,}   \\
                &= k                                  && \mbox{from assumption $k \circ \bar{j}$ is total.}
\end{align*}

\end{enumerate}
\end{proof}

\begin{lemma}
\llabel{supportLemma}
In any category with support \catc, 
\begin{enumerate}[(i)]
\item
if $\sequentialdiag{a}{b}{c}{f}{g}$ in \catcw 
 then
 \[
 \overline{f} \circ \overline{f \circ g} = \overline{f \circ g}
 \]
 \item \llabel{RHomFunctorSubLemma}
if  $\triplesequentialdiag{a}{x_1}{x_2}{x_3}{k}{j_1}{j_2}$ 
in a \catcw  then 
if $\overline{k \circ j_1 \circ j_2} = \bar{k}$ 
then $\overline{k \circ j_1} = \bar{k}$.
\end{enumerate}
\end{lemma}
\begin{proof}
\begin{enumerate}[(i)]
\item
\begin{align*}
\overline{f} \circ \overline{f \circ g}  
           &= \overline{f} \circ \overline{f \circ \overline{g}}
                               && \mbox{by wR.4} \\
           &= \overline{\overline{f} \circ f \circ \overline{g}}
                               && \mbox{by R.3}  \\
           &= \overline{f \circ \overline{g}}
                               && \mbox{by R.1}\\
           &= \overline{f \circ g}
                               && \mbox{ by wR.4, as required}
\end{align*}
\item

Suppose $\overline{k \circ j_1 \circ j_2} = \bar{k}$  then we have
\begin{align*}
\overline{k \circ j_1} &= \overline{\bar{k} \circ k \circ j_1}
                           &&\mbox{by R.1,} \\
                       &= \overline{\overline{k \circ j_1 \circ j_2} \circ k \circ j_1}
                           &&\mbox{from the initial assumption,} \\
                       &= \overline{k \circ j_1 \circ j_2} \circ \overline{k \circ j_1}
                           &&\mbox{by R.3,} \\
                       &= \overline{k \circ j_1 \circ j_2}
                           &&\mbox{by (i),} \\
                       &= \bar{k}
                           &&\mbox{from the initial assumption, as required.} 
\end{align*}
\end{enumerate}
\end{proof}



\subsection{Partial Ordering of $Hom(A,B)$}

In a category with support, and therefore in a restriction category, we can define a partial ordering on each hom set
Hom(a,b) by defining :
$$f \leq g \mbox{ iff } f = \bar{g} \circ f.$$

We can think of $f \leq g$ as meaning that if $f$ is defined then $g$ is defined and the two are equal.


\begin{lemma}
\llabel{leqlemma}
In a restriction category \catcw,
if 
$
\paralleldiag{a}{b}{f}{f'}
$
and
$
\paralleldiag{b}{c}{g}{g'}
$
in \catcw and if $f' \leq f$ and $g' \leq g$ 
then $f' \circ g' \leq f \circ g$.
\end{lemma}

\begin{lemma}
\llabel{pointwiseleqlemma}

In a restriction category \catcw,
if 
$
\paralleldiag{a}{b}{f}{f'}
$
\nudgedown{0.6cm}
and
$
\paralleldiag{b}{c}{g}{g'}
$
and $h: a \morph c$ in \catcw 
and if $f' \leq f$ and $g' \leq g$ 
and if $h \leq f \circ g'$ and $h \leq f' \circ g$ 
then $h \leq f' \circ g'$.
\end{lemma}
\begin{proof}
Proof is handwritten on tablet.\highlight{CHECK.}
\end{proof}



\section{Range Categories \cite{COCKETT2012}}

Cockett et al. (\cite{COCKETT2012}) define range categories from restriction categories along the following lines.

\begin{definition} 
A \term{range category} is a restriction category with an additional operator on morphisms (the range operator $\hat{\ }$), as follows
if $f: a \morph b$ in  \catcw then
$$\hat{f}: b \morph b$$
satisfying

RR.1 For $f:a \morph b$ in \catcw $$\bar{\hat{f}} = \hat{f}.$$

RR.2 For $f:a \morph b$ in \catcw $$f \circ \hat{f} = f.$$

RR.3. If $\sequentialdiag{a}{b}{c}{f}{g}$ in \catcw then
$$\reallywidehat{f \circ \bar{g}} = \hat{f} \circ \bar{g}.$$

RR.4. If $\sequentialdiag{a}{b}{c}{f}{g}$ in \catcw then
$$\reallywidehat{(\widehat{f} \circ g)} = \widehat{f \circ g}.$$
\end{definition}

In this note we explore range categories that satisfy an  additional condition defined by Cockett et al. as RR.5. We will use the folowing terminology:

\begin{definition}
An \textit{RR.5 range category} is a range category that additionally satisfies:

RR.5 if $\equaliser{a}{f}{\paralleldiag{b}{c}{g}{h}}$ then
 $$f \circ g = f \circ h \Rightarrow  \hat{f} \circ g = \hat{f} \circ h.$$
 \end{definition}
Let the category \FinPar be the range category of finite sets and partial functions.

\begin{lemma}
\llabel{unirangelemma}
If $f:a \morph a$ in a range category \catcw then for any morphism 
$f: a \morph b$
\begin{enumerate}[(i)] 
\item $\hat{\hat{f}}=\hat{f}$,
\item $\hat{f} \comp \hat{f} = \hat{f}$,
\item \llabel{lem:unirange:hatbar} $\hat{\bar{f}}=\bar{f}$.
\end{enumerate}
\end{lemma}
\begin{proof}
Straightforward.
\end{proof}

\begin{lemma}
\llabel{fglemma}
 \begin{enumerate}[(i)]
 \item 
 If \fgsinkdiag in \catcw then
 $\hat{f}\circ \hat{g} = \hat{g} \circ \hat{f},$
 \item
 If $\sequentialdiag{a}{b}{c}{f}{g}$ in \catcw  then
$\hat{g} \circ \reallywidehat{f \circ g}=  \reallywidehat{f \circ g}.$
 \item \commentary{wrong diagram }
If \fgsinkdiag in \catcw  then
$\reallywidehat{f \circ \widehat{g}}=\widehat{f} \circ \widehat{g}.$


\end{enumerate}
\end{lemma}
\begin{proof}
\begin{enumerate}[(i)]
\item Follows from RR.1 and R.2.
\item 
\begin{align*}
\hat{g} \circ \widehat{f \circ g}&=\widehat{f \circ g} \circ \hat{g}&&\mbox{by (i)} \\
&=\widehat{f \circ g} \circ \bar{\hat{g}}
                    &&\mbox{by lemma \ref{unirangelemma} (\ref{lem:unirange:hatbar}),} \\
&= \reallywidehat{f \circ g \circ \bar{\hat{g}}}&&\mbox{by RR.3,} \\
&= \widehat{f \circ g} &&\mbox{by lemma \ref{unirangelemma} (\ref{lem:unirange:hatbar}), then RR.2.}
\end{align*}
\item %(iii)
\begin{align*}
\widehat{f \circ \widehat{g}} 
                                &=\widehat{f \circ \overline {\widehat{g}}} &\mbox{by RR.1}\\
                                &=\widehat{f} \circ \overline {\widehat{g}} &\mbox{by RR.3}\\
                                &=\widehat{f} \circ \widehat{g}             & \mbox{by RR.1}\\
\end{align*}
\end{enumerate}
\end{proof}

\subsection {What isnt true}
\commentary{Document this because I would like it to be true but it isnt.}
If \catcw is a range category and if
 \fgsourcediag in any range category in \catcw then we can probably prove that
 $$\overline{g} \circ f \leq f \circ \reallywidehat{\overline{g} \circ f}.$$

It is not the case however, we cannot  prove that 
$$\overline{g} \circ f = f \circ \reallywidehat{\overline{g} \circ f}$$
for in the range category of partial sets and functions suppose
$a$ is a doubleton set ${x_1,x_2}$ 
and that $f$ is defined at both $x_1$ and $x_2$ and that $f(x_1) = f(x_2)=y$.
Suppose $g$ is defined at $x_1$ but not at $x2$. Then
\begin{align*}
x_1 &\xmapsto{f} y \\
x_2 &\mapsto y \\
             \\
x_1 &\xmapsto{\overline{g}\kern0.05cm} x_1 \\
x_2 &\xmapsto{\kern0.2cm} \bot \\
                \\
x_1 &\xmapsto{\overline{g} \circ f} y \\
x_2 &\xmapsto{\kern0.4cm} \bot
\end{align*}

so that $\reallywidehat{\overline{g} \circ f}=\set{y}$  
and

\begin{align*}
x_1 &\xmapsto{f \circ \reallywidehat{\overline{g} \circ f}} y \\
x_2 &\xmapsto {\kern.75cm}y
\end{align*}

and we see that $$\overline{g} \circ f \neq f \circ \reallywidehat{\overline{g} \circ f}.$$

\subsection{Partial Sections}
Cockett defines partial sections as possible features of restriction categories. 
We are interested in pretty much the same concept applied in the context of categtories with support. If \catcw is a category with support then a morphism s is said to be a partial section of  morphism $f$ iff
$s \circ f = \bar{s}$ and $f \circ s \circ f = f$.

\section{Range Categories with Partial Sections}
\begin{lemma}[Cockett et al]
If \catcw is a restruction category
and every morphism has a partial section then
\catcw is a range category and
\[\widehat{f} = f_s \circ f\]
for any choice $f_s$ of section for $f$.
\end{lemma}


\begin{lemma}[Cockett et al]
If \catcw is a range category with partial sections then \catcw is an RR.5 range category.
\end{lemma}
\begin{proof}
Straightforward.
\end{proof}

\subsection{Generalisation of Schein's Theorem}
\begin{theorem} [\cite{COCKETT2012}]
Every small range category in
which [RR.5] holds admits a faithful embedding into the partial
map category of a regular category, namely sets and partial functions.
\end{theorem}

In particular:
\begin{corollary}
\label{scheincorollary}
Every RR.5 range category can be faithfully mapped into a range category with partial sections.
\end{corollary}

We are interested in picking apart Cockett's proof of this theorem
and reusing some of the pieces. To this end we need to 
work with categories with support and the features of morphisms with partial sections.  

\subsection{Categories with Support, Partial Sections and Pseudo-Range}

The following observation is from  Cockett et al where it is used in the context of restriction categories.
\begin{lemma}
If \catcw is a category with support and there is a morphism $f: a \morph b$
which has two partial sections $s,s':b \morph a$ then
\[
f \circ s = f \circ s' 
\]
\end{lemma}
\begin{proof}
hand written notes
\end{proof}

This leads us to being able to define an operator \hatItself \ for
morphisms $f$ that have partial sections. 
We define
\[\widehat{f} = f_s \circ f\]
where $f_s$ is any partial section of $f$.

\begin{notebox}[Warning]
At this point we have proved none of the required range category properties of \hatItself. We must be very careful not to use any of them. Lets pick over anything we do with it very carefully. Must't accidentally assume any of the properties of the range operator. 
\end{notebox}

To show that HomP preserves the "range" operator I require RR.1.
That $f \circ \widehat{f} = f$. 

\begin{proof}
\begin{align*}
f \circ \widehat{f} &= f \circ f_s \circ f \\
                    &= f &&\mbox{since $f_s$ is a  section of $f$, PHEW.}
\end{align*}
\end{proof}

\begin{noteforfuture}
somewhere I need show that when I compose zigzag 
$\tuple{f}$ with $\tuple{\widehat{f},f,\bar{f}}$
then I get $\tuple{\widehat{f}}$. as well as showing the second os these is a partila section.
\end{noteforfuture}