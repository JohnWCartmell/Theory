

\begin{frame}{Fundamental Principles at this Abstract Level}
\begin{itemize}
    \item Principle 1 -- absence of redundancy in presentation.
    \item Principle 2 -- the theory be the tightest possible fit to the facts.
\end{itemize}
\bigskip
\begin{itemize}
    \item The two principles collectively
    \begin{itemize}
        \item ensure absence of redundancy in data and in data management logic.
    \end{itemize}
\end{itemize}
\bigskip
\begin{itemize}
    \item Note: principle 2 expresses a kind of logical completeness.
\end{itemize}
\end{frame}

\begin{frame}{Data Specifications}
Two kinds of types in play
\begin{itemize}
\item  the \textit{definienda} -- types all of whose instances are \textit{particulars}
\begin{itemize}
\item employee, department, student, account, product, order, shipment, delivery, flight, booking and so on,
\item molecular structure, atom, covalent bond, element, isotope, reaction, metabolite, mass trace, chromatogram, peak.
\end{itemize}
\pause 
\item  the \textit{definiens}  -- types all of whose instances are \textit{universals}
\begin{itemize}
       \item string, integer, float, boolean and so on.
\end{itemize}
\end{itemize}
\pause
\begin{itemize}
%\item In ER modelling 
%\begin{itemize}
%\item the \textit{definienda} are called \textit{entity types}
%\item the \textit{definiens} are called \textit{attribute types} or \textit{domains}.
%\end{itemize}
\item I assume a fixed set $V$ of universals and define data specifications and instances relative to $V$.
\end{itemize}
\end{frame}

\begin{frame}{Data Specifications as Sketches}
A data specification is a sketch of 
\begin{itemize}
\item  an RR.5 range category, 
\item with  designated finite restriction products,
\item designated monomorphisms with partial inverses,
\item an object $v$ representing the set $V$ of universals.
\end{itemize}

Next I go through the background catagory theory that is involved in this definition. 
\end{frame}

