
\newcommand{\rangeplus}{$\mbox{range}^+\ $}

\section{Restriction Categories}

From Cockett \& Lack, a restriction category is a category \catcw with an additiona operator (the restriction operator) such that
if $f: a \morph b$ in \catcw then
$$\bar{f}: a \morph a$$ 
and satisfying

R.1 For $f:a \morph b$ in \catcw $$\bar{f} \circ f =f$$.

R.2. If \fgsourcediag in \catcw then
$$\bar{g} \circ \bar{f}=\bar{f} \circ \bar{g}.$$

R.3. If \fgsourcediag in \catcw then
$$\overline{\bar{f} \circ g} = \bar{f} \circ \bar{g}$$.

R.4. If $\sequentialdiag{a}{b}{c}{f}{g}$ in \catcw then
$$f \circ \bar{g} = \overline{f \circ g} \circ f$$.

\section{Range Categories}

A \term{range category} is a retriction category with an additional operator as follows
if $f: a \morph b$ in  \catcw then
$$\hat{f}: b \morph b$$
satisfying

RR.1 For $f:a \morph b$ in \catcw $$\bar{\hat{f}} = \hat{f}.$$

RR.2 For $f:a \morph b$ in \catcw $$f \circ \hat{f} = f.$$

RR.3. If $\sequentialdiag{a}{b}{c}{f}{g}$ in \catcw then
$$\widehat{f \circ \bar{g}} = \hat{f} \circ \bar{g}.$$

RR.4. If $\sequentialdiag{a}{b}{c}{f}{g}$ in \catcw then
$$\widehat{(hat({f}) \circ g)} = \widehat{f \circ g}.$$

\begin{lemma}
\label{restrictioncatlemma}
In any restriction category \catc
\begin{enumerate} [(i)]
\item For $f:a \morph b$ in \catcw,
$$\bar{\bar{f}}=\bar{f}$$
\item If $\sequentialdiag{a}{b}{c}{k}{j}$ in \catcw 
and $k$ is total i.e. $\bar{k}=id_a$
and if $k \circ \bar{j}$ is total i.e. $\overline{k \circ \bar{j}}=id_a$
then
$$k \circ \bar{j} = k.$$
\item If $\sequentialdiag{a}{b}{c}{k}{j}$ in \catcw 
and $k$ is total i.e. $\bar{k}=id_a$ then
$$\overline{k \circ j} = \overline{k \circ \bar{j}}.$$
\end{enumerate}
\end{lemma}
\begin{proof}
\begin{enumerate} [(i)]
\item Use R3 with $g$ being $id_b$.
\item 
\begin{align*}
k \circ \bar{j} &= k \circ \bar{\bar{j}}              && \mbox{from (i),}\\
                &= \overline{k \circ \bar{j}} \circ k && \mbox{by R.4,}   \\
                &= k                                  && \mbox{from assumption $k \circ \bar{j}$ is total.}
\end{align*}
\item 
\begin{align*}
\overline{k \circ j} &= \overline{k \circ j} \circ \bar{k}    && \mbox{from assumption that $k$ total,}\\
                     &= \overline{\overline{k \circ j} \circ k} && \mbox{by R.3,} \\
                     &= \overline{k \circ \bar{j}}              && \mbox{by R.4.}
\end{align*}
\end{enumerate}
\end{proof}

Let the category \FinPar be the range catgory of finite sets and partial functions.

\subsection{Split assumption}
\begin{definition}
In a range category \catcw then a morphism $f:a \morph b$ is split iff
there exists $f_s:b \morph a$ 
such that
\begin{equation*} 
f_s \circ f = \hat{f} = \bar{f_s}
\end{equation*}
and
\begin{equation*} 
f \circ f_s = \hat{f_s} = \bar{f}
\end{equation*}
\end{definition}

\begin{lemma}
\label{jsbarj}
In a range category \catcw if $f_s$ is a splitting of a morphism $f$ 
then $f_s \circ \bar{f} = f_s.$
\end{lemma}
\begin{proof}
\begin{align*}
f_s \circ \bar{f} 
      &= f_s \circ (f \circ f_s) 
             && \mbox{because $\bar{f}=f \circ f_s$, by definition of splitting, } \\
      &= \bar{f_s} \circ f_s     
             && \mbox{because $f \circ f_s = \bar{f_s}$, from same definition,}    \\
             && 
      &= f_s   &&\mbox{by R.1, as required.}
\end{align*}
\end{proof}

\begin{lemma}
If $f:a \morph a$ in a range category \catcw then for any morphism 
$f: a \morph b$
\begin{enumerate}[(i)] 
\item $\hat{\hat{f}}=\hat{f}$,
\item $\hat{f} \comp \hat{f} = \hat{f}$,
\item $\hat{\bar{f}}=\bar{f}$,
\item $\bar{\hat{f}}=\hat{f}$,
\item $\bar{f}$ is a splitting of itself,
\item $\hat{f}$ is a splitting of itself.
\end{enumerate}
\end{lemma}
\begin{proof}
Straightforward.
\end{proof}

\begin{definition}
A range category with coherent splittings is a range category \catcw along with for every
morphism $f:a \morph b$ a splitting $f_s:b \morph a$ such that if
$\sequentialdiag{a}{b}{c}{f}{g}$ in \catcw then $(f \circ g)_s=g_s \circ f_s$.
\end{definition}

\begin{lemma}
\label{splittingsequencepair}
If \catcw is a range category with coherent splittings then if
$\sequentialdiag{a}{b}{c}{f}{g}$ in \catcw
 then 
 \begin{enumerate}[(i)]
 \item $\overline{f \circ g}=f \circ \bar{g} \circ f_s$,
 \item $\widehat{f \circ g}=g_s\circ \hat{f} \circ g$
\end{enumerate}
 \end{lemma}
\begin{proof}
\end{proof}

\begin{lemma}
\label{splittingalternatepair}
If \catcw is a range category with coherent splittings then if
$\alternatediag{a}{b}{c}{f}{g}$ in \catcw
 then 
 \begin{enumerate}[(i)]
 \item $\overline{f \circ g_s}=f \circ \hat{g} \circ f_s$,
 \item $\widehat{f \circ g_s} = g\circ \hat{f} \circ g_s$
\end{enumerate}
 \end{lemma}
\begin{proof}
Use lemma \ref{splittingsequencepair}.
\end{proof}

\iffalse
\newcommand{\range}[1]{\hat{}(#1)}
\newcommand{\psplit}[1]{#1^{-1}}
\begin{lemma}
\label{splittingalternatesequences}
If \catcw is a range category with coherent splittings then 
\begin{enumerate}[(i)]
\item if $\alternatingchainfg{a}{b}{f}{g}{n}$ in \catcw then
\begin{align*}
\overline{f_1 \circ \psplit{g_1} \circ f_2 \circ \psplit{g_2} 
... \circ f_n \circ \psplit{g_n} } 
= f_1 \circ \psplit{g_1}  ... f_n  \circ \psplit{g_n}   
\circ g_n \circ \psplit{f_n} ... \circ g_1 \circ \psplit{f_1}\\
\range{f_1 \circ \psplit{g_1} \circ f_2 \circ \psplit{g_2} 
... \circ f_n \circ \psplit{g_n} } 
= g_n \circ \psplit{f_n} ... g_1 \circ \psplit{f_1} \circ f_1 \circ \psplit{g_1}
... \circ f_n \circ \psplit{g_n}
\end{align*}
\item if $\alternatingchainff{a}{b}{f}{g}{n}$ in \catcw then 
\begin{align*}
\overline{f_1 \circ \psplit{g_1} \circ f_2 \circ \psplit{g_2} 
... \circ f_n  } 
= f_1 \circ \psplit{g_1}  ... f_n  \circ \psplit{f_n} ... \circ g_1 \circ \psplit{f_1}\\
\range{f_1 \circ \psplit{g_1} \circ f_2 \circ \psplit{g_2} 
... \circ f_n  } 
= \psplit{f_n} ... g_1 \circ \psplit{f_1} \circ f_1 
... \circ f_n \circ \psplit{g_1}
\end{align*}
\item if $\alternatingchaingg{a}{b}{f}{g}{n}$ in \catcw then 
$$etc.$$
\item if $\alternatingchaingf{a}{b}{f}{g}{n}$ in \catcw then
$$etc.$$
\end{enumerate}
\end{lemma}
\begin{proof}
Use lemma \ref{splittingalternatepair}.
\end{proof}
\fi

\section{Construction of a Category with Coherent Splittings from a Range Category}
I tried to do a explicit constructon of a catgeory with coherent splittings from a range category. I was trying to show that a range category could be faithfully embedded in a 
range category with coherent splittings.

Sadly this failed and with good reason.

Consider the range category generated by the  sketch with this graph
$\equaliser{a}{f}{\paralleldiag{b}{c}{g}{h}}$
and subject to  equations $\hat{f} = \bar{g} = \bar{h}$ and $f \circ g = f \circ h$.
In this range category then $f$ and $g$ are distinct morphisms.
If we add a morphsim $f_s$ as a splitting of the morphism $f$  then it follows that $g = h$. This is a counter example, therefore, to the suggestiion that every range 
category can be embedded faithfully in a range category with splittings.

\section{Construction of Restriction Functors}

\subsection{$HomP$ is a Restriction Functor}
If \catcw is a restriction functor then
let $HomP_\catc(a,-): \catc \morph \SetP$ be the  functor
defined as follows. $HomP(a,x)=Hom(a,x)$.
If $j: x \morph y$ then partial function $HomP(a,j): HomP(a,x) \morph HomP(a,y)$
is defined as follows:
\begin{align*}
HomP(a,j)(k) = k \circ j &\mbox{ if $k \circ \bar{j} = k$} \\
                         & \mbox{ is undefined otherwise}
\end{align*}
\begin{lemma}
If \catcw is a restriction catgory and $a$ is an object of \catcw 
then the functor $HomP_\catc(a,-)$ preserves the restriction operator i.e.
for all $j:x \morph y$ in \catc
$$HomP_\catc(a,\bar{j})=\overline{HomP_\catc(a,j)}$$
\end{lemma}
\begin{proof}
Both $HomP_\catc(a,\bar{j})$ and $\overline{HomP_\catc(a,j)}$ are less than id function on the set $Hom(a,x)$. 
The former because by definition of $HomP$ 
if $HomP_\catc(a,\bar{j})(k)$ is defined then $HomP_\catc(a,\bar{j})(k)=k$.
The latter by definition of $\bar{f}$, for any function $f$.

It suffices to show that $HomP_\catc(a,\bar{j})$ is defined iff and only if
$HomP_\catc(a,j)$ is defined which we can do as follows.
 Let $k \in Hom(a,x)$. $HomP_\catc(a,\bar{j})(k)$
is defined iff $k \circ \bar{\bar{j}}$ i.e. $k \circ \bar{j}$ is defined
iff $HomP_\catc(a,j)(k)$ is defined. 
\end{proof}


\subsection{Coproducts of Set valued Range Functors}
\subsection{Quotients of Set valued Restriction Functors}

\newcommand{\Fquotient}{F/\sim}
Suppose \catcw is a restriction category and that $F: \catc \morph \SetP$ is a restriction functor into the restriction category of sets and partial functions. Suppose that for every object $x$ of \catcw there is an equivalence relationship $\sim_x$ defined on the set $F(x)$. Suppose that this equivalence relationship has the following property(s):
For every morphism $j:x \morph y$ in \catcw, for all elements $k1,k2 \in F(x)$,
\begin{enumerate}
\item $F(j)(k_1)$ is defined iff $F(j)(k_2)$ is defined, 
\item $k_1 \sim_x k_2 \mbox{ and } F(j)(k_1) \mbox{ is defined } \implies F(j)(k_1) \sim_y F(j)(k_2).$
\end{enumerate}
Define functor $\Fquotient: \catcw \morph \SetP$ by defining
\begin{align*}
&x             &&\mapsto F(x)/\sim_x 
                & \mbox{i.e. the set of equivalence classes of $F(x)$ wrt equivalence relation $\sim$}\\
&j: x \morph y &&\mapsto F(j)/\sim
\end{align*} 
where $F(j)/\sim$ is defined by
\begin{align*}
[k] \mapsto [F(j)(k)] && \mbox{provided $F(j)(k)$ is defined, is undefined otherwise.}
\end{align*}
which is well-defined because of the assumption made above.

It is easy to see that $\Fquotient$ is a functor (i.e. respects composition $\circ$ and identity morphisms.)

\begin{lemma}
The functor $\Fquotient$ is a restriction functor.
\end{lemma}
\begin{proof}
We need to show that  $\Fquotient$ respects the restriction operator
i.e. we need to show that for any morphism $j:x \morph y$ in \catcw, $\Fquotient(\bar{j})= \overline{\Fquotient(j)}$.

The LHS, $\Fquotient(\bar{j})$, is defined to  be the partial function
$$ [k] \mapsto [F(\bar{j})(k)] \mbox{ providing $F(\bar{j})(k)$ is defined, is undefind otherwise,}$$
in otherwords, since  $F$ assumed to be a restriction functor it can be defined as mapping
$$ [k] \mapsto [\overline{F(j)}(k)] \mbox{providing $\overline{F(j)}(k)$ is defined, is undefind otherwise,}$$
and this in turn, because of the definition of restriction in \SetP, means that it maps
$$ [k] \mapsto [k] \mbox{providing $F(j)(k)$ is defined, is undefined otherwise.}$$

Meanwhile the RHS, $\overline{\Fquotient(j)}$  is defined as the partial function that 
maps $[k]$  to $[k]$  provided  $\Fquotient(j)(k)$ is defined, and to be undefined otherwise.

Because,
from the definition of $\Fquotient$, $\Fquotient(j)([k])$ is defined iff $F(j)(k)$ is defined then
we have shown that LHS and RHS are identical functions, as required.
\end{proof}

\section{Construction of Range Functors}

\subsection{Split assumption and $HomP$ as a Range Functor}

\begin{lemma}
If \catcw is a range category in which every morphism is split then
for every object $a$ of \catcw the functor $HomP_\catcw: \catcw \morph \SetP$ preserves the range operator.
\end{lemma}
\begin{proof}
If $j: x \morph y$ in \catcw then we are required to prove that
$$\widehat{HomP_\catcw(a,j)}=HomP_\catcw(a,\hat{j})$$
and for this it suffices to show that the range of $HomP_\catcw(a,j)$ 
i.e. this set
$$\setsuchthat{ k \circ j}{k: a \morph x \mbox{ and } k \circ \bar{j} = k }$$ 
is identical to the range of $HomP_\catcw(a,\hat{j})$ 
i.e. this set
$$\setsuchthat{k' \circ \hat{j}}{k':a \morph y \mbox{ and } k' \circ \bar{\hat{j}} = k' }.$$   
i.e. this set
$$\setsuchthat{k' \circ \hat{j}}{k':a \morph y \mbox{ and } k' \circ \hat{j} = k' }.$$

The first set is included in the second because for any morphism of the form $k \circ j$, 
where $k \circ \bar{j}=k$, 
is of the form $k' \circ \hat{j}$, for $k' \circ \hat{j} = k'$, 
because we can take $k'$ to be $k \circ j$ 
because then 
$k' \circ \hat{j} = (k \circ j) \circ \hat{j}= k \circ(j \circ \hat{j})= k \circ j = k'$,
as required. 

Let $j_s : y \morph x$ be the splitting of $j$ i.e. be such that $j_s \circ j = \hat{j}$.

The second set is included in the first because if a morphism is of the form $k' \circ \hat{j}$, where $k' \circ \hat{j} = k'$, then it of the form $k \circ j$, where $k \circ \bar{j}=k$, because $k$ can be taken to be $k' \circ j_s$ for then we have 
$k' \circ \hat{j} = k' \circ j_s \circ j = k \circ j$
and we have $k \circ \bar{j}= k' \circ j_s \circ \bar{j} =  k' \circ j_s = k$ using lemma
\ref{jsbarj}, as required. 
 \end{proof}

\begin{corollary}
If \catcw is a range category in which every morphism is split then
for every object $a$ of \catcw the functor $HomT_\catcw(a,-): \catcw \morph \SetP$ is a range functor.
\end{corollary}

\subsection{About HomP}
\begin{lemma}
\label{HomOfEffbargbar}
\fgsourcediag
\begin{align*}
[L_c(f)] &= [L_a(Hom(a,f)(\bar{f}\circ\bar{g}))] \\
[L_c(g)] &= [L_a(Hom(a,g)(\bar{f}\circ\bar{g}))]
\end{align*} 
\end{lemma}
\begin{proof}
\end{proof}
\subsection{Coproducts of Set valued Range Functors}
TBD
\subsection{Quotients of Set valued Range Functors}
\begin{lemma}
\label{rangefunctorquotient}
The functor $\Fquotient$ is a range functor.
\end{lemma}
\begin{proof} \commentary{Proof reviewed on 12 Dec 2023.}
We need to show that  $\Fquotient$ respects the range operator
i.e. we need to show that for any morphism $j:x \morph y$ in \catcw, 
 $\Fquotient(\hat{j})= \widehat{\Fquotient(j)}$.

The LHS, $\Fquotient(\hat{j})$, is defined to  be 
the partial function which maps $[k']$, where $k' \in F(y)$, as follows
$$ [k'] \mapsto [F(\hat{j})(k')] \mbox{ providing $F(\hat{j})(k')$ is defined, and is undefined otherwise.}$$
In otherwords, since  $F$ assumed to be a range functor it can be defined as mapping
$$ [k'] \mapsto [\widehat{F(j)}(k')] \mbox{ providing $\widehat{F(j)}(k')$ is defined and being undefined otherwise,}$$
and this in turn, because of the definition of ranges in \SetP, means that it maps
$$ [k'] \mapsto [k'] \mbox{providing there exists $k \in F(x)$ such that $k'=F(j)(k)$ and being undefined otherwise.}$$

Meanwhile the RHS, $\widehat{\Fquotient(j)}$  is defined as the partial function that 
maps $[k']$  to $[k']$  provided  there exists $k \in F(x)$ such that
$\Fquotient(j)([k])=[k']$ to be undefined otherwise.

Because,
from the definition of $\Fquotient$ we have that $\Fquotient(j)([k])= [F(j)(k)]$   
we have  that LHS and RHS are identical functions, as required.
\end{proof}

\section{\rangeplus categories}
A \rangeplus category is a range category for which the embedding into the freely 
generated range category with partial sections is faithful.