
\newcommand{\studentDeptInclusionDependency}
{student[sDept] \subseteq department[dName]}
\newcommand{\studentSupervisorInclusionDependency}
{student[sDept,sSv] \subseteq professor[pDept,pId]}
\newcommand{\professorDeptInclusionDependency}
{professor[pDept] \subseteq department[dName]}
\newcommand{\headOfDeptInclusionDependency}
{department[dName,dHd]  \subseteq professor[pDept,pId]}

\begin{frame}{Relational Data}
% 3 tables of data
\begin{tabular}{|l|l|l|}
\hline
\rowcolor{myblue}\multicolumn{3}{l}{\colhead{student}} \\
\hline
\rowcolor{myblue}\colhead{\pk{sName}}  & \colhead{\fk{sDept}\kern-7pt} & \colhead{\fk{sSv}\kern-7pt} \\
\hline
gray & phil   & \#1  \\
\hline
bohm &  maths & \#1 \\
\hline
smith & maths & \#2  \\
\hline
doe   & phil  & \#1 \\
\hline
$\hdots$ & $\hdots$  & $\hdots$ \\
\hline
%$\hdots$ & $\hdots$  & $\hdots$ \\
%\hline
\end{tabular}
\begin{tabular}{|l|l|l|}
\hline
\rowcolor{myblue}\multicolumn{3}{l}{\colhead{professor}} \\
\hline
\rowcolor{myblue}\colhead{\fk{\pk{pDept}}\kern-7pt} & \colhead{\pk{pNo}} & \colhead{pName}   \\
\hline
maths 	& \#1 & scott \\
\hline
maths 	& \#2 & smith \\
\hline
maths 	& \#3 & gandy \\
\hline
phil 	& \#1 & smith  \\
\hline
phil 	& \#2 & ayer   \\ 
\hline
$\hdots$ & $\hdots$  & $\hdots$ \\
\hline
\end{tabular}
\begin{tabular}{|l|l|}
\hline
\rowcolor{myblue}\multicolumn{2}{l}{\colhead{department}} \\
\hline
\rowcolor{myblue}\colhead{\pk{dName}} & \colhead{\fk{dHd}\kern-7pt} \\
\hline
maths 	&\#3   \\
\hline
phil  	&\#1 \\
\hline
history &\#5 \\
\hline
physics &\#1 \\
\hline
$\hdots$ & $\hdots$  \\
\hline
%$\hdots$ & $\hdots$  \\
%\hline
\end{tabular}
\begin{overlayarea}{\textwidth}{6cm}
\only<2>{
\begin{itemize}
\item students and departments are identified by name,
\item professors are identified by combination of department and id,
\end{itemize}
}
\only<3->{
\begin{itemize}
\item rows of the \textit{student} table reference the \textit{department} table by virtue of a column that instances values from the identifying column that table,
\onslide<4->{	
\fcolorbox{red}{white}{$\studentDeptInclusionDependency$},
}
\onslide<5->{
\item similarly,
\fcolorbox{red}{white}{$\professorDeptInclusionDependency$},
}
\onslide<6->{
\item rows of the \textit{student} table reference the \textit{professor} table by virtue of two columns  instancing values from the identifying columns of that table,
}
\onslide<7->{
\fcolorbox{red}{white}{$\studentSupervisorInclusionDependency$},
}
\onslide<8->{ 
\item similarly,
\fcolorbox{red}{white}{$\headOfDeptInclusionDependency$}.
}
\end{itemize}
}
\end{overlayarea}
\end{frame}

\iffalse
\newcommand{\studentDeptRangeExpression}
{$\widehat{sDept} \leq \widehat{dName}$}
\newcommand{\studentSupervisorRangeExpression}
{$\widehat{\tuple{sDept,sSv}} \leq \widehat{\tuple{pDept,pId}}$}
\newcommand{\professorDeptRangeExpression}
{$\widehat{pDept} \leq \widehat{dName}$}
\newcommand{\headOfDeptRangeExpression}
{$\widehat{\tuple{dName,dHd}}  \leq \widehat{\tuple{pDept,pId}}$}
\fi

\begin{frame}{Referential Inclusion Dependencies as Range Identities}

\begin{itemize}
\item Now think of each column as a function that maps rows of a table to values.
\item Each inclusion dependency can be expressed as identity on the ranges of these functions.
\item Each $$a[f] \subseteq b[q]$$
can be represented as
$$\widehat{f} \leq \widehat{q} \mbox{ in \Par,}$$
\item Similarly $$a[f_1,...f_n] \subseteq b[q_1,...q_n]$$
can be represented as
$$\widehat{\tuple{f_1,...f_n}} \leq \widehat{\tuple{q_1,...q_n}} \mbox{ in \Par.}$$
\end{itemize}

\end{frame}
