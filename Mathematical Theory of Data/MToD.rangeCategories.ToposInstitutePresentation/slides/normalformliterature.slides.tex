
\begin{frame}{Prior Theory --- 1970 - 1979}
\begin{itemize}
\item In 1970, E.F.Codd introduced the relational model of data and the idea of normal form.
\item A year later he defines the term `functional dependency' and  uses it to define `third normal form' (3NF).
\item In 1977, Fagin defines the concept of a `multivalued dependency' and uses it to define `fourth normal form' (4NF).
\item Two years on, Fagin defines `projection-join normal form' which is also known as `fifth normal form' (5NF).
\end{itemize} 
\end{frame}

\iffalse
\begin{frame}{Inclusion Normal Forms}
Ling and Goh 1992 
\begin{quote}
Since
classical normal forms (including the Improved 3NF)
have failed to consider the effects of INDs on the structure
of a database, they are inadequate in characterizing a
database scheme which is truly devoid of redundancies.
In consideration of the above, we propose a new normal
form, called Inclusion Normal Form (IN-NF)...
\end{quote}
\end{frame}
\fi

\begin{frame}{The success of Codd's Relational Model of Data}
\begin{itemize}
\item Codd's model of data has been very influential. Witness that by 2020 Oracle Corporation had grown from being founded in 1977 to having  a 42\% share of an estimated \$30billion market for relational database technology.  
\item Codd in 1990 says that
\begin{quote}
The relational model is solidly based on two parts of mathematics: 
first-order predicate logic and the theory of relations.
\end{quote}
\item My opinion is that this has been to found data modelling on the wrong mathematics. 
\item Codd's mathematical basis and therefore his  model  do nothing to guide the programmer as navigator, to use Charles W Bachman's phrase, 
\item nor do they encourage thinking about navigation path equivalence, i.e. diagrams that commute ... even though thinking about diagrams that commute is essential to the goodness of data specifications.
\item The right mathematical starting point for the theory of data is category theory.
\end{itemize}
\end{frame}
 


