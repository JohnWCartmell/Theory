\subsection{Team Selection - Variation One}
\llabel{teamselectionvariationone}
This is a variation of the preliminary model in which each person may only be selected by a single other individual.
\commentary{This is a variation of an earlier example. In this example the fd factoring property holds though in the earlier example it did not. In this example, therefore, the $\chi$-generated relational model is in normal form (BCNF, in fact).}
A person may select themselves in which case the person will be in a team of their own. 

In this example the schema is as shown in the earlier figure \lref{teamselectionpreliminaryschema}. 
In this model the fd factoring property holds because functional dependency $\msfd{S}{c}$ is now intransitive because though
$S \morph S/S \morph c$ we also have $S/S \morph S$.

\begin{figure} [h]
\begin{center}
\begin{tabular}{p{3.5cm} c}
\begin{tabular}{c p{1.5cm} c}
   \Rnode{p}{p} & & \Rnode{v}{v}
\end{tabular}
%\nccircle[nodesep=3pt]{<-}{p}{.4cm}
\rEtm[270]{p}
\alabel{S}[0.6]
\Et[-40]{p}{v}
\blabel{c}[0.6]
\Etm{p}{v} 
\alabel{pId}[0.6]
\idcomp
& \footnotesize
\begin{tabular}{c p{1.5cm} p{4cm}}
KEY && \\
\hline
p & person & Identified by id attribute ($pId$). \\
S & selects & each person selects exactly one other person - each person is selected at most once \\
c & colour & each person is given a coloured vest 
\end{tabular} 
\end{tabular}
\end{center}
\caption{Team Selection Example  - Variant one. This example is different to the example \ref{teamselectionpreliminary} in that the relationship $S$ in this example is a mono-source. 
}
\label{teamselectionvariantoneERschema}
\end{figure}

\subsubsection{The $\chi$-generated Relational Model}
As before in example \lref{teamselectionpreliminary}, from this model the $\chi$ transform generates  a relational schema  consisting of a single
$person$ relation described in (\lref{personrelation}) and for which, as earlier, there are functional dependencies (\ref{spIdfd}),
(\ref{colourfd}) and (\ref{spIdcfd}) but now in addition
there is a functional dependency 
\begin{equation}
S \morph pId
\end{equation}

and now $S/pId$ is a candidate key and thus now, unlike in example \lref{teamselectionpreliminary}, and as predicted by 
theorem \lref{goaltheorem}, this schema is in BCNF and therefore in 3NF. 
