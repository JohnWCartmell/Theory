\begin{figure} [h]  % chen fragment with bars
\begin{center}
$
\begin{array}{c p{0.75cm}c p{0.5cm}c p{0.5cm}c}
 \Rnode{project}{project}  && \Rnode{dept}{dept}  &&              &&              \\ [0.3cm]
	 	           &&               && \Rnode{site}{site} && \Rnode{v}{v} \\ [0.3cm]     
 \Rnode{employee}{employee}  && \Rnode{intake}{intake}  &&              &&              \\ 
\end{array}
$
\ncarr[60]{project}{v}
\alabel{pcode}[0.3][0]
\idcomp
\ncarr{project}{dept} 
\alabel{S_1}[0.5][0]
\ncarr[30]{dept}{v}
\alabel{dName}[0.3][0]
\idcomp
\ncarr{dept}{site} 
\alabel{S_2}[0.3][0]
\idcomp
\ncarr{site}{v}
\alabel{sName}[0.4][0]
\idcomp
\ncarr[-30]{project}{employee}
\blabel{R_0}[0.35][0]
\ncarr[-30]{employee}{project}
\blabel{S_0}[.35][0]
\ncarr{employee}{intake}
\blabel{R_1}[0.5][1]
\ncarr{intake}{site}
\blabel{R_2}[0.3][0]
\idcomp
\ncarr[-30]{intake}{v}
\blabel{dName}[0.3][0]
\idcomp
\ncarr[-60]{employee}{v}
\blabel{empNo}[0.3][0]
\idcomp
%\nccurve[angleA=90,angleB=90,nodesep=2pt,ncurv=0.9]{->}{w}{v}
%\alabel{pcnt}[0.3][-1]
\vspace{1.5cm}
\newline
such that \hspace{0.5cm}
$
\begin{array}{c p{0.75cm}c p{0.5cm}c}
 \Rnode{project}{project}  && \Rnode{dept}{dept}  &&              \\ [0.3cm]
	 	           &&               && \Rnode{site}{site}  \\ [0.3cm]     
 \Rnode{employee}{employee}  && \Rnode{intake}{intake}  &&              \\ 
\end{array}
$
\ncarr{project}{dept} 
\alabel{S_1}[0.5][0]
\ncarr{dept}{site} 
\alabel{S_2}[0.3][0]
\idcomp
\ncarr{employee}{project}
\alabel{S_0}[.5][1]
\ncarr{employee}{intake}
\blabel{R_1}[0.5][1]
\ncarr{intake}{site}
\blabel{R_2}[0.3][0]
\idcomp
\hspace {0.25cm} and \hspace{0.5cm}
$
\begin{array}{c p{0.75cm}c p{0.5cm}c}
 \Rnode{project}{project}  && \Rnode{dept}{dept}  &&              \\ [0.3cm]
	 	           &&               && \Rnode{site}{site}  \\ [0.3cm]     
 \Rnode{employee}{employee}  && \Rnode{intake}{intake}  &&               \\ 
\end{array}
$
\ncarr{project}{dept} 
\alabel{S_1}[0.5][0]
\ncarr{dept}{site} 
\alabel{S_2}[0.3][0]
\idcomp
\ncarr{project}{employee}
\blabel{R_0}[0.5][1]
\ncarr{employee}{intake}
\blabel{R_1}[0.5][1]
\ncarr{intake}{site}
\blabel{R_2}[0.3][0]
\idcomp
\hspace{0.2cm} commute.

\end{center}
\caption{An example  of how \textit{rendered redundant} may lead to circularity in that
$R_1/R_2/sName$  rendered redundant by a path $S_O/S_1/S_2/sName$  and, \textit{vice-versa}, 
$S_1/S_2/sName$ rendered redundant by $R_0/R_1/R_2/sName$}
\label{employeeintake}
\end{figure}

\begin{newtt}
Added 14 Jan 2024. 
\begin{itemize}
	\item Perhaps rename intake to briefing withe the meaning of `site briefing'.
	\item Perhaps S0 is that en employee works on a project.
	\item Perhaps R0 is that an employee is the the project sponsor. 
	\item Each employee works on at most one project.
	\item Projects are sponsored by a single employee. One employee may sponsor multiple projects.  
	\item There again might make more sense just to rename site as company and to keep intake.
	\item BTW the attribute dName of intake should be renamed.
	\item the question arises what are the additional types X and Y that are introduced as part of the 'normalisation'???
\end{itemize}
\end{newtt}