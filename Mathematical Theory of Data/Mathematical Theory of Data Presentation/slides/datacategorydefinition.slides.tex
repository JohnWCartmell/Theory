
\begin{frame}{Definition -- \datacat}
\medskip
I shall use the shorthand \textit{\datacat} to mean a triple $\tuple{\catc,M,v}$ where 
\begin{itemize}
\item \catcw is a \rangeplus category with specified finite restriction products,
\item $M$ is a set of designated monomorphisms of \catcw closed under composition
and such that each $m \in M$   has a partial inverse $m^{-1}$,
\item a distinguished object $v$, such that every morphism $f: v \morph x$ in \catcw 
factors through $m^{-1}$, for some monomorphism $m$.
\end{itemize}

\medskip
Note that it follows from this definition that a sketch for a \textit{\datacat} has no need for edges with domain $v$.
\end{frame}

\begin{frame}{Data Specification and Data Specification Instance}
In this presentation, 
\begin{itemize}
\item by \textit{data specification} I shall mean a sketch for a \datacatw such that the designated object $v$ has no outgoing edges -- 
 neither edges  \underline{$v \morph v$} 
nor edges \underline{$v \morph non\localjchyphen v$}.\\
\item If $S$ is a sketch for \datacatw denote by $\catc(S)$ the \datacatw generated from $S$. \\
\pause \item Define an \textit{instance} of a data specification $S$ to be a 
range functor $F: \catc(S) \morph \Par$ 
that preserves the specified restriction products
and maps the object $v$ to the set $V$. \\
\medskip
Note that such an $F$ will preserve designated monomorphisms and their inverses.
\end{itemize} 
I will muddle up data specifications and sketches in these slides. \\ 
I will speak of $\catc(S)$ as the theory category. \\
\medskip
Next I want to give some examples to show how all this works in practice.
The very next next example is of the ... relational model of data.
\end{frame}

\ifNotesnAll
\begin{frame}{Example}
Now I want to give some examples to show how all this works in practice and I want to start with the dear old relational model of data. 
\end{frame}
\fi

\newcommand{\studentDeptInclusionDependency}
{student[dept] \subseteq department[name]}
\newcommand{\studentSupervisorInclusionDependency}
{student[dept,svr] \subseteq professor[dept,no]}
\newcommand{\professorDeptInclusionDependency}
{professor[dept] \subseteq department[name]}
\newcommand{\headOfDeptInclusionDependency}
{department[name,hd]  \subseteq professor[dept,no]}

\begin{frame}{Relational Data Slide}
% 3 tables of data
\begin{tabular}{|l|l|l|}
\hline
\rowcolor{myblue}\multicolumn{3}{l}{\colhead{student}} \\
\hline
\rowcolor{myblue}\colhead{\pk{sName}}  & \colhead{\fk{sDept}\kern-7pt} & \colhead{\fk{sSv}\kern-7pt} \\
\hline
gray & phil   & \#1  \\
\hline
bohm &  maths & \#1 \\
\hline
smith & maths & \#2  \\
\hline
doe   & phil  & \#1 \\
\hline
$\hdots$ & $\hdots$  & $\hdots$ \\
\hline
%$\hdots$ & $\hdots$  & $\hdots$ \\
%\hline
\end{tabular}
\begin{tabular}{|l|l|l|}
\hline
\rowcolor{myblue}\multicolumn{3}{l}{\colhead{professor}} \\
\hline
\rowcolor{myblue}\colhead{\fk{\pk{pDept}}\kern-7pt} & \colhead{\pk{pNo}} & \colhead{pName}   \\
\hline
maths 	& \#1 & scott \\
\hline
maths 	& \#2 & smith \\
\hline
maths 	& \#3 & gandy \\
\hline
phil 	& \#1 & smith  \\
\hline
phil 	& \#2 & ayer   \\ 
\hline
$\hdots$ & $\hdots$  & $\hdots$ \\
\hline
\end{tabular}
\begin{tabular}{|l|l|}
\hline
\rowcolor{myblue}\multicolumn{2}{l}{\colhead{department}} \\
\hline
\rowcolor{myblue}\colhead{\pk{dName}} & \colhead{\fk{dHd}\kern-7pt} \\
\hline
maths 	&\#3   \\
\hline
phil  	&\#1 \\
\hline
history &\#5 \\
\hline
physics &\#1 \\
\hline
$\hdots$ & $\hdots$  \\
\hline
%$\hdots$ & $\hdots$  \\
%\hline
\end{tabular}
\begin{align*}
\studentDeptInclusionDependency  \\
\studentSupervisorInclusionDependency\\
\professorDeptInclusionDependency\\
\headOfDeptInclusionDependency  \\
\end{align*}
{\small{move student table to the left}}
\end{frame}
  % this slide defines inclusion dependencies

%These will be redefined in the next slide
\newcommand{\studentDepartmentAttribute}[1]{#1}
\newcommand{\studentProfessorAttribute}[1]{#1}
\newcommand{\professorDepartmentAttribute}[1]{#1}
\newcommand{\departmentProfessorAttribute}[1]{#1}
\newcommand{\studentDepartmentAttributeStrikeout}[1]{}
\newcommand{\studentProfessorAttributeStrikeout}[1]{}
\newcommand{\professorDepartmentAttributeStrikeout}[1]{}
\newcommand{\departmentProfessorAttributeStrikeout}[1]{}

\newcommand{\professorStudentDepartmentRelationalSchematic}
{
\begin{tabular}{c p{0.25cm} c p{0.5cm} c}
$\begin{array}{c p{0.01cm} c p{0.2cm} c}
\multicolumn{5}{c}{\Rnode{student}{student}} \\[-0.3cm]
\multicolumn{5}{c}{\Rnode{sB1}{}\hspace{0.4cm}\Rnode{sB2}{}\hspace{0.5cm}\Rnode{sB3}{}} \\[1.0cm]
\Rnode{sv1}{v} && \studentDepartmentAttribute{\Rnode{sv2}{v}} && \studentProfessorAttribute{\Rnode{sv3}{v}}
\end{array}
\begin{arrows}
\ncarr{sB1}{sv1}
\blabel{sName}[0.55][0.1]
\addedgebar
\studentDepartmentAttribute{
  \ncarr{sB2}{sv2}
  \alabel{sDept}[0.57][0.1]
  \studentDepartmentAttributeStrikeout{\ncstrikeout{sB2}{sv2}}
}
\studentProfessorAttribute{
  \ncarr{sB3}{sv3}
  \alabel{sSv}[0.55][0.1]
  \studentProfessorAttributeStrikeout{\ncstrikeout{sB3}{sv3}}
}
\end{arrows}$
&&
$\begin{array}{c p{0.1cm} c p{0.01cm} c}
\multicolumn{5}{c}{\Rnode{professor}{professor}} \\[-0.3cm]
\multicolumn{5}{c}{\Rnode{pB1}{}\hspace{0.5cm}\Rnode{pB2}{}\hspace{0.5cm}\Rnode{pB3}{}} \\[1.0cm]
\professorDepartmentAttribute{\Rnode{pv1}{v}} && \Rnode{pv2}{v} && \Rnode{pv3}{v} 
\end{array}
\begin{arrows}
\professorDepartmentAttribute{
  \ncarr{pB1}{pv1}
  \blabel{pDept}[0.5][0.1]
  \addedgebar
  \professorDepartmentAttributeStrikeout{\ncstrikeout{pB1}{pv1}}
}
\ncarr{pB2}{pv2}
\blabel{pId}[0.55][0.1]
\addedgebar
\ncarr{pB3}{pv3}
\alabel{pName}[0.5][0.1]
\end{arrows}$
&&
$\begin{array}{c  c }
\multicolumn{2}{c}{\Rnode{department}{department}} \\[-0.3cm]
\multicolumn{2}{c}{\Rnode{dB1}{}\hspace{0.5cm}\Rnode{dB2}{}} \\[1.0cm]
\hspace{0.2cm}\Rnode{dv1}{v} & \departmentProfessorAttribute{\Rnode{dv2}{v}} 
\end{array}
\begin{arrows}
\ncarr{dB1}{dv1}
\blabel{dName}[0.5][0.1]
\addedgebar
\departmentProfessorAttribute{
  \ncarr{dB2}{dv2}
  \alabel{dHd}[0.5][0.1]
  \departmentProfessorAttributeStrikeout{\ncstrikeout{dB2}{dv2}}
}
\end{arrows}$
\end{tabular}
}


\newcommand{\rightsquiggle}
{
  \textcolor{red}{ \Large$\leftrightsquigarrow$ }
}
\iffalse
\newcommand{\inclusionDependencies}
{
\begin{overprint}
\vspace{1cm}
\begin{itemize}
\item $\studentDeptInclusionDependency$ 
                    \onslide<3->{\rightsquiggle $sDept \circ \widehat{dName} = sDept$}\\
\medskip
\item $\professorDeptInclusionDependency$  
                    \onslide<4->{\rightsquiggle  $pDept \circ \widehat{dName} = pDept$} \\
\medskip
\item $\studentSupervisorInclusionDependency$  \\
\hspace{2.5cm}\onslide<5->{\rightsquiggle $\tuple{sDept,sSv} \circ \widehat{\tuple{pDept,pId}} 
                                                              = \tuple{sDept,sSv}$ }  \\
%\smallskip
\item $\headOfDeptInclusionDependency$   \\
\hspace{2.5cm}\onslide<6->{\rightsquiggle  $\tuple{dName,dHd} \circ \widehat{\tuple{pDept,pId}} 
              = \tuple{dName,dHd}$}
\end{itemize}
\end{overprint}
}
\fi

\newcommand{\studentDeptRangeExpression}
{$\widehat{sDept} \leq \widehat{dName}$}
\newcommand{\studentSupervisorRangeExpression}
{$\widehat{\tuple{sDept,sSv}} \leq \widehat{\tuple{pDept,pId}}$}
\newcommand{\professorDeptRangeExpression}
{$\widehat{pDept} \leq \widehat{dName}$}
\newcommand{\headOfDeptRangeExpression}
{$\widehat{\tuple{dName,dHd}}  \leq \widehat{\tuple{pDept,pId}}$}

\begin{frame}{Sketch  for \datacat (Relational)}
\vspace{0.9cm}
\begin{block}{Directed Graph}
\professorStudentDepartmentRelationalSchematic
\end{block}
\begin{block}{Mono-sources}
Indicated by bars on the graph.
\end{block}
\begin{block}{Identities}
\begin{tabular}{c l}
\studentDeptRangeExpression &
\onslide<2->{\rightsquiggle $sDept \circ \widehat{dName} = sDept$}\\
\professorDeptRangeExpression &
 \onslide<3->{\rightsquiggle  $pDept \circ \widehat{dName} = pDept$}   \\
\studentSupervisorRangeExpression &
\onslide<4->{\rightsquiggle $\tuple{sDept,sSv} \circ \widehat{\tuple{pDept,pId}} 
                                                              = \tuple{sDept,sSv}$ }\\
\headOfDeptRangeExpression &
 \onslide<5->{\rightsquiggle  $\tuple{dName,dHd} \circ \widehat{\tuple{pDept,pId}} 
              = \tuple{dName,dHd}$}      \\
\end{tabular}
\end{block}
\end{frame}



\begin{frame}{Classifying Data Specifications}
\begin{center}
\pstree{\TR{\framebox[3cm][c]{data specification}}}{%
\TR{\framebox[2cm][c]{logical}}
\pstree{\TR{\framebox[2cm][c]{physical}}}{\TR{\framebox[2.5cm][c]{relational}} \TR{\framebox[2.5cm][c]{non-relational}}}
}
\end{center}
\medskip
\begin{itemize}
\pause \item in \textit{relational} sketches all edges are 
of the \underline{$non\localjchyphen v \morph v$} type and each such represents a column of a table/relation,
\pause \item other \textit{physical sketches} (\textit{non-relational}) in addition 
to the \underline{$non\localjchyphen v \morph v$} type edges have edges of the \underline{$non\localjchyphen v \morph non\localjchyphen v$} type and these represent structural containment,
\item non-relational physical data specifications are also said to be \textit{hierarchical}.
\end{itemize}
\end{frame}

\begin{frame}{Characterisation of Relational Data Specifications}
\begin{definition}
A data specification is \textit{relational} iff
\begin{itemize}
	\item all edges are of the \underline{$non\localjchyphen v \morph v$} type,
\item every non-v-node is the domain of at least one v-valued mono-source
i.e. for every non-v-node $a$, for some $n \geq 1$, there exists a source
\begin{displaymath}
\parallelsource{a}{v}{m}{n}
\end{displaymath}
which is designated as a mono-source i.e. for which $\tuple{m_1,...m_n}$ is a designated monomorphism.
\end{itemize}
\end{definition}
\end{frame}

\begin{frame}{Construction -- Transform a Relational Sketch to a Logical Sketch}
\begin{lemma}
For any classic relational data specification 
there is an equivalent data specification
(i.e. one with the same theory category) which is logical.
\end{lemma}

\begin{proof}
In outline: We construct a series of equivalent sketches by eliminating each 
inclusion dependency in turn. When all eliminated the resulting sketch is the required logical sketch. Eliminate the inclusion dependency 
$a[f_1,...f_n] \subseteq b[m_1,...m_n]$
as follows:
\begin{itemize}
\item remove the inclusion dependency,
\item replace by an edge $f: a \morph b$, 
\item remove those $f_i$ that are edges and 
rewrite any occurrence of such $f_i$ in the remaining inclusion dependencies by $f \circ  m_i$, 
\item for those $f_i$ that are not edges add a path equivalence (i.e. a commuting diagram)
$f \circ m_i = f_i$.
\end{itemize}
\end{proof}
An example follows.
\end{frame}


\renewcommand{\studentDepartmentAttribute}[1]             {\onslide<1-5>{#1}}
\newcommand{\popupInitialInclusionDependencies}[1]        {\only<1-4>{#1}}
\renewcommand{\studentDepartmentAttributeStrikeout}[1]    {\onslide<5>{#1}}
\newcommand{\popupMarkedUpInitialInclusionDependencies}[1]{\only<5>{#1}}
\newcommand{\popupAbstractionOfd}[1]                      {\only   <2-5>{#1}} %was 6
\newcommand{\popupAbstractionOfdDetail}[1]                {\onslide<3-5>{#1}} %was 6
\newcommand{\studentDepartmentRelationship}[1]            {\onslide<4->{#1}}
\newcommand{\popupFirstRewrittenInclusionDependencies}    {\only   <6-9>}

\renewcommand{\professorDepartmentAttribute}[1]           {\onslide<1-10>{#1}}
\newcommand{\reprofessorDepartmentAttributeStrikeout}[1]  {\onslide<10>{#1}}
\newcommand{\popupMarkedUpFirstRewrittenInclusionDependencies}[1]
                                                        {\only   <10>{#1}}
\newcommand{\popupAbstractionOfdPrime}[1]               {\only   <7-10>{#1}}
\newcommand{\popupAbstractionOfdPrimeDetail}[1]         {\onslide<8-11>{#1}}
\newcommand{\professorDepartmentRelationship}[1]        {\onslide<9->{#1}}
\newcommand{\professorDepartmentRelationshipBar}[1]     {\onslide<11->{#1}}

\newcommand{\popupSecondRewrittenInclusionDependencies} {\only  <11-16>}
\newcommand{\restudentProfessorAttribute}[1]            {\onslide<1-15>{#1}}
\newcommand{\restudentProfessorAttributeStrikeout}[1]   {\onslide<15>{#1}}
\newcommand{\popupAbstractionOfs}[1]                    {\only   <12-16>{#1}}
\newcommand{\popupAbstractionOfsDetail}[1]              {\onslide<13-16>{#1}}
\newcommand{\studentProfessorRelationship}[1]           {\onslide<14->{#1}}

\newcommand{\popupThirdRewrittenInclusionDependencies}  {\only  <17-20>}
\renewcommand{\departmentProfessorAttribute}[1]         {\onslide<1-20>{#1}}
\renewcommand{\departmentProfessorAttributeStrikeout}[1]{\onslide<20>{#1}}
\newcommand{\popupAbstractionOfhd}[1]                   {\only   <18-20>{#1}}
\newcommand{\popupAbstractionOfhdDetail}[1]             {\onslide<19-20>{#1}}
\newcommand{\departmentProfessorRelationship}[1]        {\onslide<19-20>{#1}}


\newcommand{\abstractionOfd}
{
Remove $sDept$ and replace by an edge $d: student \morph department$.
Rewrite appearances of $sDept$ in the sketch by $d \circ dName$.
}

\newcommand{\abstractionOfdPrime}
{
Remove $pDept$ and replace ny an edge $d': professor \morph department$.
%$\composeSevenShaped[0.85cm]{professor}{department}{v}{d'}{name}{dept}$ commutes. 
Rewrite appearances of $pDept$ in the sketch by $d' \comp dName$.
}

\newcommand{\abstractionOfs}
{
Remove $sSv$ and replace by an edge  $s: student \morph professor$.
Rewrite appearances of $sSv$ in the sketch by $s \comp pNo$.
Add commutative diagram
$\composeSevenShaped[0.85cm]{student}{professor}{v} {s}{d' \circ dName}{d \circ dName}$ 
 simplify to
$\composeSevenShaped[0.85cm]{student}{professor}{department}{s}{d'}{d }$
Keep $\composeSevenShaped[0.85cm]{student}{professor}{v}{s}{no}{svr}$ commute
pending rethink.
}

\newcommand{\abstractionOfhd}
{
Replace $dHd$ by an edge  $h: department \morph professor$,
add commutivity of
$\composeSevenShaped[0.85cm]{department}{professor}{v}{h}{d' \circ dName}{dName}$
 simplfy to
$\composeSevenShaped[0.85cm]{department}{professor}{department}{h}{d'}{id}$
.
Keep $\composeSevenShaped[0.85cm]{department}{professor}{v}{h}{no}{h}$.
}

\newcommand{\circledStudentSupervisorInclusionDependency}
{
{student[\psovalbox[linecolor=red, boxsep=false]{sDept},sSv] \subseteq professor[pDept,pNo]}
}

\newcommand{\rewrittenStudentSupervisorInclusionDependency}
{
{student[d \circ dName,sSv] \subseteq professor[pDept,pNo]}
}

\newcommand{\circledRewrittenStudentSupervisorInclusionDependency}
{
{student[d \circ dName,sSv] 
   \subseteq professor[\psovalbox[linecolor=red, boxsep=false]{pDept},pNo]}
}
\newcommand{\circledHeadOfDeptIncD}
{department[dName,dHd]  
   \subseteq professor[\psovalbox[linecolor=red, boxsep=false]{pDept},pNo]}

\newcommand{\secondRewrittenStudentSupervisorInclusionDependency}
{
{student[d \circ dName,sSv] 
   \subseteq professor[d \circ dName,pNo]}
}
\newcommand{\secondRewrittenHeadOfDeptIncD}
{department[dName,dHd]  
   \subseteq professor[d' \circ dName,pNo]}


\newcommand{\initialInclusionDependencies}
{
\begin{itemize}
\item $\studentDeptInclusionDependency$
\item $\professorDeptInclusionDependency$
\item $\studentSupervisorInclusionDependency$
\item $\headOfDeptInclusionDependency$
\end{itemize}
}

\newcommand{\strikeout}[1]{\colorbox{red}{\sout{#1}}}
\newcommand{\initialMarkedUpInclusionDependencies}
{
\begin{itemize}
\item \st{$\studentDeptInclusionDependency$}
\item $\professorDeptInclusionDependency$
\item $\circledStudentSupervisorInclusionDependency$
\item $\headOfDeptInclusionDependency$
\end{itemize}
}
             
\newcommand{\firstRewrittenInclusionDependencies}
{
\begin{itemize}
\item $\professorDeptInclusionDependency$
\item $\rewrittenStudentSupervisorInclusionDependency$
\item $\headOfDeptInclusionDependency$
\end{itemize}
}

\newcommand{\markedUpFirstRewrittenInclusionDependencies}
{
\begin{itemize}
\item \st{$\professorDeptInclusionDependency$}
\item $\circledRewrittenStudentSupervisorInclusionDependency$
\item $\circledHeadOfDeptIncD$
\end{itemize}
}
\newcommand{\secondRewrittenInclusionDependencies}
{
\begin{itemize}
\item $\secondRewrittenStudentSupervisorInclusionDependency$
\item $\secondRewrittenHeadOfDeptIncD$
\end{itemize}
}

\newcommand{\thirdRewrittenInclusionDependencies}
{
\begin{itemize}
\item $\secondRewrittenHeadOfDeptIncD$ % sic
\end{itemize}
}


\newcommand{\abstractionToSketch}
{
\popupInitialInclusionDependencies{\initialInclusionDependencies}
\popupMarkedUpInitialInclusionDependencies{\initialMarkedUpInclusionDependencies}
\popupFirstRewrittenInclusionDependencies{\firstRewrittenInclusionDependencies}
\popupMarkedUpFirstRewrittenInclusionDependencies
                                    {\markedUpFirstRewrittenInclusionDependencies}
\popupSecondRewrittenInclusionDependencies{\secondRewrittenInclusionDependencies}
\popupThirdRewrittenInclusionDependencies{\thirdRewrittenInclusionDependencies}

\popupAbstractionOfd{                    
\begin{block}{Step 1. Eliminate $\studentDeptInclusionDependency$}
\popupAbstractionOfdDetail{\abstractionOfd}
\end{block}
}

\popupAbstractionOfdPrime{
\begin{block}{Step 2. Eliminate $\professorDeptInclusionDependency$}
\popupAbstractionOfdPrimeDetail{\abstractionOfdPrime}
\end{block}
}
\popupAbstractionOfs{
\begin{block}{Step 3. Eliminate $\secondRewrittenStudentSupervisorInclusionDependency$}
\popupAbstractionOfsDetail{\abstractionOfs}
\end{block}
}
\popupAbstractionOfhd{
\begin{block}{Step 4. Eliminate this final inclusion dependency.}
\popupAbstractionOfhdDetail{\abstractionOfhd}
\end{block}
}
}


\begin{frame}{Transform Relational Sketch to Logical Sketch}
\vspace{0.9cm}
\professorStudentDepartmentRelationalSchematic
$\begin{arrows}
\studentDepartmentRelationship{
  \ncarr[20]{student}{department}
  \alabel{d}[0.4]
}
\studentProfessorRelationship{
  \ncarr[5]{student}{professor}
  \alabel{s}[0.4]
}
\professorDepartmentRelationship{
  \ncarr[5]{professor}{department}
  \alabel{d'}[0.4]
  \professorDepartmentRelationshipBar{\addedgebar}
}
\departmentProfessorRelationship{
  \ncarr[5]{department}{professor}
  \alabel{h}[0.4]
}\end{arrows}
$

\begin{overlayarea}{11cm}{7cm}

\abstractionToSketch
\onslide<20>{END}%need to force slide 20 to appear without hd attribute ??
\end{overlayarea}

\end{frame}




% The following command has a command passed to it as parameter.
% The expected value of the parameter is either the ncarr or the ncsar command.
% This command is used to draw the arrows from student -> department
% and professor -> department.
% In this way this command can be used to draw a sketch of a catagory 
% or a sketch of a contextual category.
\newcommand{\sketchgraph}[1]
{
\begin{displaymath}
\begin{array}{c c c p{1cm} c}
  & \makebox[2cm][c]{\Rnode{department}{department}\Rnode{departmentRight}{}}  & \\[2cm]
\makebox[1cm][c]{\Rnode{student}{student}} 
      && \makebox[1cm][c]{\Rnode{professor}{professor}}
      && \Rnode{v}{v} \\[0.8cm]  %was 1.2cm
\end{array}
\begin{arrows}
\setlength{\arrnodesepA}{4pt}  %2
\setlength{\arroffsetA}{-3pt}  %0
\ncarr[15]{departmentRight}{v}
\alabel{dName}[0.3]
\addedgebar
\arreset
\ncarr[15]{department}{professor}
\alabel{h}[0.3]
#1{student}{department}
\alabel{d}[0.3]
\ncarr[-50]{student}{v}
\blabel{sName}[0.2]
\addedgebar
\ncarr{student}{professor}
\blabel{s}[0.3]
#1[15]{professor}{department}
\alabel{d'}[0.3]
\addedgebar
\ncarr[15]{professor}{v}
\alabel{pId}[0.3]
\addedgebar
\ncarr[-15]{professor}{v}
\blabel{pName}[0.3]
\end{arrows}
\end{displaymath}
}



\begin{frame}{Resulting Logical Data Specification}
\sketchgraph{\ncarr}

\begin{block}{subject to commutivity of }
\vspace{0.25cm}
\studentProfessorDepartmentCommutingDiagrams{\ncarr}
\vspace{0.25cm}
\end{block}
\end{frame}




\begin{frame}{Characterisation of  Logical Data Specification}
\begin{definition}
A data specification $S$ is \textit{logical} iff
\begin{itemize}
	\item there does not exist an edge $e$ of the sketch $S$ for which there is a decomposition in the theory category $\catc(S)$ i.e. such that 
	for some morphisms $f_1$ and $f_2$  distinct from $e$,
	$e = f_1 \circ f_2$. 
\end{itemize}
\end{definition}
\end{frame}

\newcommand{\transone}{\parbox{2cm}{\textit{Chen's 1976\\transform \\(automated)}}}
\newcommand{\transtwo}{\parbox{2cm}{\textit{normalise \\ a la Codd \\(manual)}}}
\begin{frame}{Best Practice -- Structured Systems Analysis and Design}
Current Best Practice
\begin{center}
\begin{tabular}{ p{1.25cm} p{2cm} p{1.25cm} p{2cm}  p{1.25cm}}
\parbox{3cm}{logical \\ data \\ specification}&  $\xRightarrow{\transone}$ & \parbox{3cm}{relational \\ data \\specification} & $ \xRightarrow{\transtwo} $ & \parbox{3.5cm}{relational \\data \\specification}
\end{tabular}
\end{center}
\end{frame}

\begin{frame}{ Chen's Transformation 1976 \onslide<3->{\textcolor{red}{made diagram aware}}}
\begin{construction}
\onslide<3>{\hspace{6.4cm}\textcolor{red}{with the same theory category}\\}
From a logical data specification construct a relational data specification
\onslide<3>{\raisebox{-0.1cm}{\Large\textcolor{red}{$\leftthreetimes$}}}. 
\end{construction}
\textbf{Chen's 1976 Method}
Replace $f:a \morph b$ in the sketch by edges $\fn$
where $\wanton{m}$ is a v-valued mono-source with domain $b$ 
and add inclusion dependency 
$a[f_1,...f_n] \subseteq b[m_1,...m_n].$

\pause
\textbf{Problem with this method}
\begin{itemize}
	\item Doesn't take account of commutative diagrams,
	\item therefore resulting relational specification
	\begin{itemize}
		\item doesn't have equivalent theory category,
		\item often is not be in normal form.  
	\end{itemize}
	\item This weakness negatively impacts how data specifications are written and maintained.
\end{itemize}
\pause
\textbf{Mission}
\begin{itemize}
	\item Theoretically justify an improved algorithm, i.e. one that takes account of commutative diagrams, and thereby change how data specifications are managed and databases are programmed.
\end{itemize}
\end{frame}

%\newcommand{\newtrans}{\parbox{2cm}{\textit{diagram \\ aware \\ transformation \\ (automated)}}

\begin{frame}{Revised Best Practice}
\begin{center}
\begin{tabular}{ p{1.25cm} p{2cm} p{1.25cm}}
\parbox{3cm}{logical \\ data \\ specification}
&  $\xRightarrow{\parbox{2cm}{\textit{diagram \\ aware \\ transformation \\ (automated)}}
                }$ 
& \parbox{3cm}{relational \\ data \\specification}
\end{tabular}
\end{center}
\textbf{Such that}
\begin{itemize}
	\item If appropriate goodness criteria met by the logical specification 
	then the relational specification meets the classic relational goodness criteria.
\end{itemize}
\textbf{Impact}
\begin{itemize}
	\item No manual normalisation process.
	\item No source code required to describe the physical level.
\end{itemize}
\end{frame}


\newcommand{\nestedStudentSupervisorInclusionDependency}
{student[..,svr] \subseteq professor[..,pId]}

\newcommand{\nestedHeadOfDeptInclusionDependency}
{department[identity,hd]  \subseteq professor[..,pId]}

\begin{frame}{Nested Relational Data -- Same information as before.}

\newcommand{\professorAwidth}{35pt}
\newcommand{\professorBwidth}{25pt}
\newcommand{\studentAwidth}{35pt}
\newcommand{\studentBwidth}{25pt}

\begin{tabular} {|l|l|l|l|}
\hline
\rowcolor{myblue}\multicolumn{4}{l}{\colhead{department}} \\
\hline
\rowcolor{myblue}\colhead{\pk{name}} & \colhead{hd}  &
\begin{tabular}{|p{\studentAwidth}|p{\studentBwidth}|}
\multicolumn{2}{c}{\colhead{student}} \\
\hline
	\colhead{\pk{name}} & \colhead{svr} \\
\hline
\end{tabular} &
\begin{tabular}{|p{\professorAwidth}|p{\professorBwidth}|}
\multicolumn{2}{c}{\colhead{professor}} \\
\hline
	\colhead{\pk{no}} & \colhead{name} \\
\hline
\end{tabular} \\
\hline
maths 	&\#3  & 
\begin{tabular}{|p{\studentAwidth}|p{\studentBwidth}|}
\vpad{2}
\hline
bohm & \#1 \\
\hline
smith & \#2 \\
\hline
\vpad{2}
\end{tabular} &
\begin{tabular}{|p{\professorAwidth}|p{\professorBwidth}|}
\vpad{2}
\hline
\#1 & scott \\
\hline
\#2 & smith  \\
\hline
\#3 & gandy  \\
\hline
\vpad{2}
\end{tabular} \\
\hline
phil  	&\#1 & 
\begin{tabular}{|p{\studentAwidth}|p{\studentBwidth}|}
\vpad{2}
\hline
gray & \#1 \\
\hline
doe & \#1 \\
\hline
\vpad{2}
\end{tabular} &
\begin{tabular}{|p{\professorAwidth}|p{\professorBwidth}|}
\vpad{2}
\hline
\#1 & smith \\
\hline
\#2 & ayer  \\
\hline
\vpad{2}
\end{tabular}  \\
\hline
history &\#5  & $\hdots$ & $\hdots$ \\
\hline
physics &\#1  & $\hdots$ & $\hdots$ \\
\hline
\end{tabular}


\begin{overlayarea}{12cm}{5cm}
\only<1>{
\begin{align*}
\nestedStudentSupervisorInclusionDependency\\
\nestedHeadOfDeptInclusionDependency  
\end{align*}
}
\only<2>{
\begin{block}{what we see here  -- a combination of}
\begin{itemize}
    \item structural containment 
	\item relational referencing.
\end{itemize}
\end{block}
}
\only<3>{
\begin{block}{this is all there is}
\begin{itemize}
	\item the sole mechanisms for representing internal relationships in data are 
	\begin{itemize}
	    \item structural containment 
		\item relational referencing.
	\end{itemize}
	\item \raisebox{0.04cm}{$\therefore$} all data can be viewed abstractly as nested relational,
\end{itemize}
\end{block}
}
\end{overlayarea}
\end{frame}


\newcommand{\nestedDataSchematic}
{
\begin{displaymath}
\begin{array}{c c c c}
  & \makebox[1cm][c]{\Rnode{department}{departm\Rnode{departmentRight}{ent}}} \\
  &               &                               &\hspace{0.6cm}\Rnode{v}{v}\\[0.4cm]
  &               &                               &\Rnode{vTWO}{v} \\[0.5cm]
\makebox[1cm][c]{\Rnode{student}{student}} 
      && \makebox[1cm][c]{\Rnode{professor}{professor}}              \\[0.6cm]
\Rnode{sv1}{v}\hspace{0.5cm}\Rnode{sv2}{v} &&   
                       \Rnode{pv1}{v}\hspace{0.5cm}\Rnode{pv2}{v}  
\end{array}
\begin{arrows}
\ncarr{departmentRight}{v}
\alabel{dName}[0.4]
\addedgebar
\ncarr{departmentRight}{vTWO}
\blabel{dHd}[0.4][0]
\arreset
\ncsar{student}{department}
\alabel{d}[0.3]
\ncarr{student}{sv1}
\blabel{sName}[0.4]
\addedgebar
\ncarr{student}{sv2}
\alabel{sSv}[0.4]
\ncsar{professor}{department}
\blabel{d'}[0.3]
\addedgebar
\ncarr{professor}{pv1}
\blabel{pId}[0.4]
\addedgebar
\ncarr{professor}{pv2}
\alabel{pName}[0.4]
\end{arrows}
\end{displaymath}
}

\newcommand{\relabelledNestedStudentSupervisorInclusionDependency}
{student[d,svr] \subseteq professor[d',no]}

\newcommand{\relabelledNestedHeadOfDeptInclusionDependency}
{department[identity,hd]  \subseteq professor[d',no]}

\begin{frame}{Hierarchical Data Specification Sketch }
\nestedDataSchematic
\begin{align*}
\relabelledNestedStudentSupervisorInclusionDependency\\
\relabelledNestedHeadOfDeptInclusionDependency  \\
\end{align*}
\end{frame}




\begin{frame}{Characterisation of Physical Data Specification}
\begin{definition}
A data specification is \textit{physical} iff
\begin{itemize}
\item every non-v-node is the domain of at most one edge of the \underline{$non\localjchyphen v \morph non\localjchyphen v$} type.
\end{itemize}
\end{definition}
In a physical data specification every node and every edge has physical significance
in the database or message structure.
\begin{itemize}
\item Nodes other than v in a physical data specification represent entity types (ER-notation) 
or  tables (relational) or structs (IDL) or similar.  
\item Edges of the \underline{$non\localjchyphen v \morph non\localjchyphen v$} type represent those relationships in the data that are physically represented  by \textit{structural containment}.
\item Remaining edges (i.e. those of the \underline{$non\localjchyphen v \morph v$} type) represent
attributes (ER) or columns of tables (relational) or scalar fields within structs (IDL) or such like.  
\end{itemize}
\end{frame}




\begin{frame}{Subtle annotation of the logical sketch.}
\sketchgraph{\ncsar}
\begin{block}{subject to commutivity of }
\vspace{0.15cm}
\studentProfessorDepartmentCommutingDiagrams{\ncsar} \\
\vspace{0.1cm}
\pause 
\hspace{1cm}\fcolorbox{red}{white}{\parbox{8.5cm}{
	$d \in department, x \in student(d) \tstyle s(x) \in professor(d)$ \\
	and $d \in department  \tstyle h(d) \in professor(d)$
                                     }
                       }

\end{block}

\end{frame}




\begin{frame}{Example -- LCMSMS Data}
\scalebox{0.2}{
\begin{erdiagram}{28.45}{55.81075}

\ergrp{0.2}{-3.7}{12.075}{-0.75}{0.2}{1}\ertext{0.15}{-1.05}{r}{module: task}
\eret{0.4}{-3.45}{4.338}{-1.35}{0.2}{1}\ertext{0.794}{-1.7}{l}{task}
\erattr{0.6}{-1.9}{1}{1}{hostnameOfLabsysAppServer}
\erattr{0.6}{-2.2}{1}{1}{tasknumber}
\erattr{0.6}{-2.5}{1}{1}{SOPnumber}
\erattr{0.6}{-2.8}{1}{1}{procedurename}
\erattr{0.6}{-3.1}{1}{1}{location}
\eret{7.338}{-2.55}{11.075}{-1.35}{0.2}{1}\ertext{7.711}{-1.7}{l}{revised\textunderscore labsys\textunderscore task}
\erattr{7.538}{-1.9}{1}{1}{hostnameOfLabsysAppServer}
\erattr{7.538}{-2.2}{1}{1}{tasknumber}
\ergrp{2.088}{-20.6}{10.188}{-5.7}{0.2}{1}\ertext{2.038}{-6}{r}{Documentation only. Not represented in XML nor in rng or ts.}
\eret{2.288}{-8.3}{9.988}{-6.55}{0.2}{1}\ertext{2.428}{-6.9}{l}{sample\textunderscore group}
\erattr{2.488}{-7.1}{1}{0}{alphacode}
\eret{2.538}{-8.05}{5.838}{-7.45}{0.2}{0}\ertext{4.188}{-7.8}{}{test\textunderscore sample\textunderscore group}
\eret{6.338}{-8.05}{9.638}{-7.45}{0.2}{0}\ertext{7.988}{-7.8}{}{shared\textunderscore sample\textunderscore group}
\eret{7.031}{-10.05}{9.744}{-9.15}{0.2}{1}\ertext{7.302}{-9.5}{l}{MSMS\textunderscore component}
\erattr{7.231}{-9.7}{1}{0}{compoundid}
\eret{4.638}{-14.35}{7.638}{-13.15}{0.2}{1}\ertext{4.938}{-13.5}{l}{sample}
\erattr{4.838}{-13.7}{1}{0}{sampleid}
\erattr{4.838}{-14}{1}{1}{isRTreferencepeak}
\eret{4.388}{-17.65}{7.888}{-16.75}{0.2}{1}\ertext{4.738}{-17.1}{l}{required\textunderscore MSMS\textunderscore data}
\erattr{4.588}{-17.3}{1}{0}{compoundid}
\ergrp{12.575}{-26.3}{24.591}{-2.75}{0.2}{1}\ertext{12.525}{-3.05}{r}{chromatogram tower}
\eret{16.575}{-5.75}{19.775}{-3.35}{0.2}{1}\ertext{16.895}{-3.7}{l}{data\textunderscore collection}
\erattr{16.775}{-3.9}{1}{0}{samplelistname}
\erattr{16.775}{-4.2}{1}{1}{instrumentname}
\erattr{16.775}{-4.5}{1}{1}{instrumenttype}
\erattr{16.775}{-4.8}{1}{1}{chromatography}
\erattr{16.775}{-5.1}{1}{1}{collectiontimestamp}
\erattr{16.775}{-5.4}{0}{1}{labsyssubmissionid}
\eret{16.744}{-7.55}{19.606}{-6.65}{0.2}{1}\ertext{17.03}{-7}{l}{sample\textunderscore group(2)}
\erattr{16.944}{-7.2}{1}{1}{methodfilename}
\eret{13.616}{-15.85}{16.734}{-13.15}{0.2}{1}\ertext{13.928}{-13.5}{l}{injection(2)}
\erattr{13.816}{-13.7}{1}{1}{fullsampleid}
\erattr{13.816}{-14}{1}{1}{samplename}
\erattr{13.816}{-14.3}{1}{1}{sampletype}
\erattr{13.816}{-14.6}{1}{1}{rawdatafilename}
\erattr{13.816}{-14.9}{1}{1}{drawerposition}
\erattr{13.816}{-15.2}{1}{1}{wellposition}
\erattr{13.816}{-15.5}{1}{1}{timestamp}
\eret{20.079}{-10.05}{22.271}{-9.15}{0.2}{1}\ertext{21.175}{-9.5}{}{compound(2)}
\eret{20.289}{-11.85}{22.061}{-10.95}{0.2}{1}\ertext{20.466}{-11.3}{l}{trace(2)}
\erattr{20.489}{-11.5}{1}{0}{trace}
\eret{13.959}{-17.35}{16.392}{-16.75}{0.2}{1}\ertext{15.175}{-17.1}{}{component(2)}
\eret{13.425}{-19.75}{16.925}{-18.25}{0.2}{1}\ertext{13.775}{-18.6}{l}{chromatogram}
\erattr{13.625}{-18.8}{1}{1}{extractiontimestamp}
\erattr{13.625}{-19.1}{1}{1}{extractioneventno}
\erattr{13.625}{-19.4}{1}{1}{timeintensitydata}
\eret{13.059}{-23.3}{22.291}{-20.65}{0.2}{1}\ertext{13.271}{-21}{l}{instrument\textunderscore extraction\textunderscore details}
\eret{13.309}{-22.45}{16.785}{-21.25}{0.2}{0}\ertext{13.657}{-21.6}{l}{xevo\textunderscore extraction\textunderscore details}
\erattr{13.509}{-21.8}{1}{1}{xevofunctionno}
\erattr{13.509}{-22.1}{1}{1}{xevocompoundno}
\eret{17.285}{-23.05}{21.541}{-21.25}{0.2}{0}\ertext{17.71}{-21.6}{l}{ab6600\textunderscore extraction\textunderscore details}
\erattr{17.485}{-21.8}{1}{1}{sampleno}
\erattr{17.485}{-22.1}{1}{1}{periodno}
\erattr{17.485}{-22.4}{1}{1}{experimentno}
\erattr{17.485}{-22.7}{1}{1}{numberofdatapoints}
\eret{17.425}{-25.45}{21.401}{-23.95}{0.2}{1}\ertext{17.823}{-24.3}{l}{step\textunderscore size\textunderscore specification}
\erattr{17.625}{-24.5}{1}{0}{startpointno}
\erattr{17.625}{-24.8}{1}{1}{stepsize}
\erattr{17.625}{-25.1}{1}{1}{changepointno}
\ergrp{27.591}{-28.45}{55.561}{-0.75}{0.2}{1}\ertext{27.541}{-1.05}{r}{interpretation tower}
\eret{31.591}{-2.6}{46.927}{-1.2}{0.2}{1}\ertext{33.125}{-1.55}{l}{interpretation\textunderscore session}
\erattr{31.791}{-1.75}{1}{0}{sessionguid}
\eret{29.957}{-4.9}{48.561}{-3.5}{0.2}{1}\ertext{30.069}{-3.85}{l}{interpretation\textunderscore event}
\erattr{30.157}{-4.05}{1}{1}{dateTimeOpened}
\erattr{30.157}{-4.35}{0}{1}{dateTimeSaved}
\erattr{30.157}{-4.65}{0}{1}{dateTimeSubmitted}
\eret{33.957}{-4.65}{38.755}{-3.75}{0.2}{0}\ertext{34.437}{-4.1}{l}{programmed\textunderscore interpretation\textunderscore event}
\erattr{34.157}{-4.3}{1}{1}{programname}
\eret{42.755}{-4.65}{47.411}{-3.75}{0.2}{0}\ertext{43.22}{-4.1}{l}{user\textunderscore interpretation\textunderscore event}
\erattr{42.955}{-4.3}{1}{1}{username}
\eret{33.179}{-8}{39.533}{-6.65}{0.2}{1}\ertext{36.356}{-7}{}{sample\textunderscore group(3)}
\eret{31.389}{-14.5}{34.322}{-12.9}{0.2}{1}\ertext{32.856}{-13.25}{}{injection(3)}
\eret{38.709}{-10.25}{41.002}{-9.15}{0.2}{1}\ertext{39.856}{-9.5}{}{compound(3)}
\eret{38.795}{-12.05}{40.417}{-10.95}{0.2}{1}\ertext{39.606}{-11.3}{}{trace(3)}
\eret{31.389}{-17.5}{34.322}{-16.6}{0.2}{1}\ertext{32.856}{-16.95}{}{component(3)}
\eret{29.179}{-19.85}{35.533}{-18.25}{0.2}{1}\ertext{32.356}{-18.6}{}{chromatogram(2)}
\eret{41.169}{-26.7}{45.996}{-24.95}{0.2}{1}\ertext{41.309}{-25.3}{l}{annotation}
\erattr{41.369}{-25.5}{1}{1}{text}
\eret{41.419}{-26.45}{43.913}{-25.85}{0.2}{0}\ertext{42.666}{-26.2}{}{reject\textunderscore this\textunderscore data}
\eret{44.413}{-26.45}{45.746}{-25.85}{0.2}{0}\ertext{45.08}{-26.2}{}{comment}
\eret{45.57}{-7.05}{47.595}{-6.15}{0.2}{1}\ertext{45.773}{-6.5}{l}{method}
\erattr{45.77}{-6.7}{1}{0}{methodname}
\eret{45.789}{-9.15}{47.376}{-7.95}{0.2}{1}\ertext{45.948}{-8.3}{l}{param}
\erattr{45.989}{-8.5}{1}{1}{name}
\erattr{45.989}{-8.8}{1}{1}{value}
\eret{27.883}{-22.95}{30.828}{-21.45}{0.2}{1}\ertext{28.178}{-21.8}{l}{timepoint}
\erattr{28.083}{-22}{1}{0}{time}
\erattr{28.083}{-22.3}{1}{1}{rawintensity}
\erattr{28.083}{-22.6}{1}{1}{smoothedintensity}
\eret{31.328}{-28.15}{36.715}{-21.45}{0.2}{1}\ertext{31.759}{-21.8}{l}{peak}
\erattr{31.528}{-22}{1}{1}{RT}
\erattr{31.528}{-22.3}{1}{1}{peakArea}
\erattr{31.528}{-22.6}{1}{1}{peakHeight}
\erattr{31.528}{-22.9}{1}{1}{chromatogramNoise}
\erattr{31.528}{-23.2}{1}{1}{startRT}
\erattr{31.528}{-23.5}{1}{1}{endRT}
\erattr{31.528}{-23.8}{1}{1}{startHght}
\erattr{31.528}{-24.1}{1}{1}{endHght}
\erattr{31.528}{-24.4}{1}{1}{peakWidthHalfHeight}
\erattr{31.528}{-24.7}{1}{1}{peakSkew}
\eret{31.578}{-27.4}{33.651}{-26.55}{0.2}{0}\ertext{32.615}{-26.9}{}{selected\textunderscore peak}
\eret{33.751}{-25.6}{35.965}{-25}{0.2}{0}\ertext{34.858}{-25.35}{}{candidate\textunderscore peak}
\eret{37.215}{-24.55}{41.615}{-21.45}{0.2}{1}\ertext{37.655}{-21.8}{l}{status}
\erattr{37.415}{-22}{1}{1}{timestamp}
\eret{0}{-0.2}{55.811}{0.3}{0.2}{1}

% relationship 
\ertext{2.519}{-0.5}{l}{}\errelarm{2.369}{-0.2}{2.369}{-0.775}{0}{0}\errelarm{2.369}{-0.775}{2.369}{-1.35}{1}{0}
% relationship 
\ertext{6.288}{-0.5}{l}{}\errelarm{6.138}{-0.2}{6.138}{-3.375}{0}{0}\errelarm{6.138}{-3.375}{6.138}{-6.55}{1}{0}
% relationship 
\ertext{18.325}{-0.5}{l}{}\errelarm{18.175}{-0.2}{18.175}{-1.775}{0}{0}\errelarm{18.175}{-1.775}{18.175}{-3.35}{1}{0}
% relationship submitto
\ertext{9.356}{-0.5}{l}{submit}\ertext{9.356}{-0.8}{l}{to}\errelarm{9.206}{-0.2}{9.206}{-0.775}{0}{0}\errelarm{9.206}{-0.775}{9.206}{-1.35}{1}{0}
% relationship 
\ertext{39.409}{-0.5}{l}{}\errelarm{39.259}{-0.2}{39.259}{-0.7}{0}{0}\errelarm{39.259}{-0.7}{39.259}{-1.2}{1}{0}
% relationship revisedin
\ertext{11.225}{-2.25}{l}{revised}\ertext{11.225}{-2.55}{l}{in}\errelarm{11.075}{-1.95}{11.675}{-1.95}{0}{0}\errelarm{31.391}{-1.9}{31.591}{-1.9}{0}{0}\ertext{21.583}{-2.225}{r}{\textasciitilde /\textasciicircum =\textasciicircum }\errelangle{11.675}{-1.95}{12.275}{-1.95}{21.733}{-1.925}{0}{0}\errelangle{21.733}{-1.925}{31.191}{-1.9}{31.391}{-1.9}{0}{0}\ercrowfoot{11.225}{-1.95}{11.075}{-1.8}{11.075}{-1.95}{11.075}{-2.1}{0}
% relationship 
\ertext{5.004}{-8.6}{l}{}\ertext{6.288}{-13}{l}{..}\errelarm{4.854}{-8.3}{4.854}{-8.375}{0}{0}\errelarm{6.138}{-12.862}{6.138}{-13.15}{1}{0}\errelangle{4.854}{-8.375}{4.854}{-8.45}{5.496}{-10.513}{0}{0}\errelangle{5.496}{-10.513}{6.138}{-12.575}{6.138}{-12.862}{1}{0}\errelseq{6.198}{-12.625}{5.788}{-12.685}{6.488}{-12.745}{6.078}{-12.805}\eridcomprel{6.0375000000000005}{6.2375}{-12.899999999999999}
% relationship 
\ertext{7.443}{-8.6}{l}{}\ertext{8.538}{-9}{l}{..}\errelarm{7.293}{-8.3}{7.293}{-8.375}{0}{0}\errelarm{8.388}{-8.937}{8.388}{-9.15}{1}{0}\errelangle{7.293}{-8.375}{7.293}{-8.45}{7.84}{-8.587}{0}{0}\errelangle{7.84}{-8.587}{8.388}{-8.725}{8.388}{-8.937}{1}{0}\eridcomprel{8.287500000000001}{8.4875}{-8.899999999999999}
% relationship 
\ertext{6.288}{-14.65}{l}{}\ertext{6.288}{-16.6}{l}{..}\errelarm{6.138}{-14.35}{6.138}{-15.55}{0}{0}\errelarm{6.138}{-15.55}{6.138}{-16.75}{1}{0}\eridcomprel{6.0375000000000005}{6.2375}{-16.5}
% relationship 
\ertext{18.325}{-6.05}{l}{}\ertext{18.325}{-6.5}{l}{..}\errelarm{18.175}{-5.75}{18.175}{-6.2}{0}{0}\errelarm{18.175}{-6.2}{18.175}{-6.65}{1}{0}\errelseq{18.235}{-6.125}{17.825}{-6.185}{18.525}{-6.245}{18.115}{-6.305}\eridcomprel{18.075}{18.275000000000002}{-6.3999999999999995}
% relationship 
\ertext{17.848}{-7.85}{l}{}\ertext{15.325}{-13}{l}{..}\errelarm{17.698}{-7.55}{17.698}{-7.625}{0}{0}\errelarm{15.175}{-12.4}{15.175}{-13.15}{1}{0}\errelangle{17.698}{-7.625}{17.698}{-7.7}{16.436}{-9.675}{0}{0}\errelangle{16.436}{-9.675}{15.175}{-11.65}{15.175}{-12.4}{1}{0}\errelseq{15.235}{-12.625}{14.825}{-12.685}{15.525}{-12.745}{15.115}{-12.805}\eridcomprel{15.075000000000001}{15.275}{-12.899999999999999}
% relationship 
\ertext{18.802}{-7.85}{l}{}\ertext{21.325}{-9}{l}{..}\errelarm{18.652}{-7.55}{18.652}{-7.625}{0}{0}\errelarm{21.175}{-8.862}{21.175}{-9.15}{1}{0}\errelangle{18.652}{-7.625}{18.652}{-7.7}{19.914}{-8.137}{0}{0}\errelangle{19.914}{-8.137}{21.175}{-8.575}{21.175}{-8.862}{1}{0}\errelseq{21.235}{-8.625}{20.825}{-8.685}{21.525}{-8.745}{21.115}{-8.805}\eridcomprel{21.075000000000003}{21.275000000000006}{-8.899999999999999}
% relationship group
\ertext{16.594}{-7.25}{r}{group}\errelarm{16.744}{-6.95}{16.444}{-6.95}{0}{0}\errelarm{10.188}{-7.425}{9.988}{-7.425}{0}{0}\errelangle{16.444}{-6.95}{16.144}{-6.95}{13.266}{-7.188}{0}{0}\errelangle{13.266}{-7.188}{10.388}{-7.425}{10.188}{-7.425}{0}{0}\ercrowfoot{16.594}{-6.95}{16.744}{-6.8}{16.744}{-6.95}{16.744}{-7.1}{0}\eridrefrel{16.493750000000002}{-6.85}{-7.049999999999999}
% relationship RTreference
\ertext{19.756}{-7.67}{l}{RT}\ertext{19.756}{-7.97}{l}{reference}\errelarm{19.606}{-7.37}{20.206}{-7.37}{0}{0}\errelarm{16.934}{-14.77}{16.734}{-14.77}{0}{0}\errelangle{20.206}{-7.37}{20.806}{-7.37}{22.306}{-7.37}{0}{0}\errelangle{16.934}{-14.77}{17.134}{-14.77}{20.47}{-14.77}{0}{0}\ertext{23.956}{-9.87}{l}{\textasciitilde /..=.}\errelangle{22.306}{-7.37}{23.806}{-7.37}{23.806}{-11.07}{0}{0}\errelangle{23.806}{-11.07}{23.806}{-14.77}{20.47}{-14.77}{0}{0}\ercrowfoot{19.756}{-7.37}{19.606}{-7.22}{19.606}{-7.37}{19.606}{-7.52}{0}
% relationship 
\ertext{15.325}{-16.15}{l}{}\ertext{15.325}{-16.6}{l}{..}\errelarm{15.175}{-15.85}{15.175}{-16.3}{0}{0}\errelarm{15.175}{-16.3}{15.175}{-16.75}{1}{0}\errelseq{15.235}{-16.225}{14.825}{-16.285}{15.525}{-16.345}{15.115}{-16.405}\eridcomprel{15.075000000000001}{15.275}{-16.5}
% relationship subject
\ertext{13.466}{-14.8}{r}{subject}\errelarm{13.616}{-14.5}{13.466}{-14.5}{0}{0}\errelarm{7.838}{-13.75}{7.638}{-13.75}{0}{0}\ertext{10.827}{-14.425}{l}{\textasciitilde /..=../group}\errelangle{13.466}{-14.5}{13.316}{-14.5}{10.677}{-14.125}{0}{0}\errelangle{10.677}{-14.125}{8.038}{-13.75}{7.838}{-13.75}{0}{0}\ercrowfoot{13.466}{-14.5}{13.616}{-14.35}{13.616}{-14.5}{13.616}{-14.65}{0}\eridrefrel{13.36625}{-14.399999999999999}{-14.599999999999998}
% relationship 
\ertext{21.325}{-10.35}{l}{}\ertext{21.325}{-10.8}{l}{..}\errelarm{21.175}{-10.05}{21.175}{-10.5}{0}{0}\errelarm{21.175}{-10.5}{21.175}{-10.95}{1}{0}\eridcomprel{21.075000000000003}{21.275000000000006}{-10.7}
% relationship subject
\ertext{19.929}{-9.75}{r}{subject}\errelarm{20.079}{-9.45}{19.779}{-9.45}{0}{0}\errelarm{9.944}{-9.6}{9.744}{-9.6}{0}{0}\ertext{14.661}{-9.825}{r}{\textasciitilde /..=../group}\errelangle{19.779}{-9.45}{19.479}{-9.45}{14.811}{-9.525}{0}{0}\errelangle{14.811}{-9.525}{10.144}{-9.6}{9.944}{-9.6}{0}{0}\ercrowfoot{19.929}{-9.45}{20.079}{-9.3}{20.079}{-9.45}{20.079}{-9.6}{0}\eridrefrel{19.828625000000002}{-9.35}{-9.549999999999999}
% relationship IS
\ertext{22.421}{-9.9}{l}{IS}\errelangle{22.271}{-9.6}{22.271}{-9.6}{22.571}{-9.6}{0}{0}\errelangle{22.271}{-9.87}{22.271}{-9.87}{22.571}{-9.87}{0}{0}\ertext{23.121}{-10.035}{l}{\textasciitilde /..=..}\errelangle{22.571}{-9.6}{22.871}{-9.6}{22.871}{-9.735}{0}{0}\errelangle{22.871}{-9.735}{22.871}{-9.87}{22.571}{-9.87}{0}{0}\ercrowfoot{22.421}{-9.6}{22.271}{-9.45}{22.271}{-9.6}{22.271}{-9.75}{0}
% relationship 
\ertext{15.325}{-17.65}{l}{}\ertext{15.325}{-18.1}{l}{..}\errelarm{15.175}{-17.35}{15.175}{-17.8}{0}{0}\errelarm{15.175}{-17.8}{15.175}{-18.25}{1}{0}\errelseq{15.235}{-17.725}{14.825}{-17.785}{15.525}{-17.845}{15.115}{-17.905}\eridcomprel{15.075000000000001}{15.275}{-18.000000000000004}
% relationship monitored
\ertext{16.542}{-17.17}{l}{monitored}\errelarm{16.392}{-16.87}{16.992}{-16.87}{0}{0}\errelarm{19.879}{-9.87}{20.079}{-9.87}{0}{0}\ertext{19.235}{-13.67}{r}{\textasciitilde /..=../..}\errelangle{16.992}{-16.87}{17.592}{-16.87}{18.635}{-13.37}{0}{0}\errelangle{18.635}{-13.37}{19.679}{-9.87}{19.879}{-9.87}{0}{0}\ercrowfoot{16.542}{-16.87}{16.392}{-16.72}{16.392}{-16.87}{16.392}{-17.02}{0}\eridrefrel{16.6415}{-16.77}{-16.970000000000002}
% relationship 
\ertext{15.325}{-20.05}{l}{}\errelarm{15.175}{-19.75}{15.175}{-19.825}{0}{0}\errelarm{17.675}{-20.438}{17.675}{-20.65}{1}{0}\errelangle{15.175}{-19.825}{15.175}{-19.9}{16.425}{-20.063}{0}{0}\errelangle{16.425}{-20.063}{17.675}{-20.225}{17.675}{-20.438}{1}{0}\eridcomprel{17.575}{17.775000000000002}{-20.400000000000006}
% relationship extracted
\ertext{17.075}{-18.925}{l}{extracted}\errelarm{16.925}{-18.625}{17.975}{-18.625}{0}{0}\errelarm{20.089}{-11.4}{20.289}{-11.4}{0}{0}\ertext{21.057}{-15.313}{r}{\textasciitilde /..=../monitored}\errelangle{17.975}{-18.625}{19.025}{-18.625}{19.457}{-15.013}{0}{0}\errelangle{19.457}{-15.013}{19.889}{-11.4}{20.089}{-11.4}{0}{0}\ercrowfoot{17.075}{-18.625}{16.925}{-18.475}{16.925}{-18.625}{16.925}{-18.775}{0}\eridrefrel{17.175}{-18.525000000000002}{-18.725000000000005}
% relationship 
\ertext{19.563}{-23.35}{l}{}\errelarm{19.413}{-23.05}{19.413}{-23.5}{0}{0}\errelarm{19.413}{-23.5}{19.413}{-23.95}{1}{0}\eridcomprel{19.312875000000002}{19.512875000000005}{-23.70000000000001}
% relationship 
\ertext{39.409}{-2.9}{l}{}\ertext{39.409}{-3.35}{l}{..}\errelarm{39.259}{-2.6}{39.259}{-3.05}{0}{0}\errelarm{39.259}{-3.05}{39.259}{-3.5}{1}{0}\eridcomprel{39.15875}{39.35875}{-3.2499999999999996}
% relationship 
\ertext{34.758}{-5.2}{l}{}\ertext{36.506}{-6.5}{l}{..}\errelarm{34.608}{-4.9}{34.608}{-4.975}{0}{0}\errelarm{36.356}{-6.437}{36.356}{-6.65}{1}{0}\errelangle{34.608}{-4.975}{34.608}{-5.05}{35.482}{-5.637}{0}{0}\errelangle{35.482}{-5.637}{36.356}{-6.225}{36.356}{-6.437}{1}{0}\eridcomprel{36.205625000000005}{36.505625}{-6.35}\ercrowfoot{36.356}{-6.5}{36.206}{-6.65}{36.356}{-6.65}{36.506}{-6.65}{0}\ercrowfoot{36.356}{-6.5}{36.206}{-6.35}{36.356}{-6.35}{36.506}{-6.35}{0}
% relationship 
\ertext{39.409}{-5.2}{l}{}\ertext{46.733}{-6}{l}{..}\errelarm{39.259}{-4.9}{39.259}{-4.975}{0}{0}\errelarm{46.583}{-6.1}{46.583}{-6.15}{1}{0}\errelangle{39.259}{-4.975}{39.259}{-5.05}{42.921}{-5.55}{0}{0}\errelangle{42.921}{-5.55}{46.583}{-6.05}{46.583}{-6.1}{1}{0}
% relationship 
\ertext{44.06}{-5.2}{l}{}\errelarm{43.91}{-4.9}{43.91}{-4.975}{0}{0}\errelarm{45.272}{-24.9}{45.272}{-24.95}{1}{0}\errelangle{43.91}{-4.975}{43.91}{-5.05}{44.591}{-14.95}{0}{0}\errelangle{44.591}{-14.95}{45.272}{-24.85}{45.272}{-24.9}{1}{0}
% relationship baseevent
\ertext{48.711}{-4.92}{l}{base}\ertext{48.711}{-5.22}{l}{event}\errelangle{48.561}{-4.62}{48.561}{-4.62}{48.861}{-4.62}{0}{0}\errelangle{48.561}{-4.34}{48.561}{-4.34}{48.861}{-4.34}{0}{0}\ertext{50.311}{-5.08}{r}{session}\ertext{50.311}{-4.78}{r}{\textasciitilde /..=base}\errelangle{48.861}{-4.62}{49.161}{-4.62}{49.161}{-4.48}{0}{0}\errelangle{49.161}{-4.48}{49.161}{-4.34}{48.861}{-4.34}{0}{0}\ercrowfoot{48.711}{-4.62}{48.561}{-4.47}{48.561}{-4.62}{48.561}{-4.77}{0}
% relationship basesession
\ertext{48.711}{-4.08}{l}{base}\ertext{48.711}{-4.38}{l}{session}\errelarm{48.561}{-3.78}{49.161}{-3.78}{0}{0}\errelarm{47.127}{-2.32}{46.927}{-2.32}{0}{0}\errelangle{49.161}{-3.78}{49.761}{-3.78}{49.811}{-3.78}{0}{0}\errelangle{47.127}{-2.32}{47.326}{-2.32}{48.594}{-2.32}{0}{0}\ertext{49.561}{-3.35}{r}{\textasciitilde /\textasciicircum =\textasciicircum }\errelangle{49.811}{-3.78}{49.861}{-3.78}{49.861}{-3.05}{0}{0}\errelangle{49.861}{-3.05}{49.861}{-2.32}{48.594}{-2.32}{0}{0}\ercrowfoot{48.711}{-3.78}{48.561}{-3.63}{48.561}{-3.78}{48.561}{-3.93}{0}
% relationship subject
\ertext{29.807}{-4.85}{r}{subject}\errelarm{29.957}{-4.55}{29.357}{-4.55}{0}{0}\errelarm{19.975}{-4.55}{19.775}{-4.55}{0}{0}\ertext{24.616}{-4.85}{l}{\textasciitilde /\textasciicircum =\textasciicircum }\errelarm{29.357}{-4.55}{24.466}{-4.55}{0}{0}\errelarm{24.466}{-4.55}{19.975}{-4.55}{0}{0}\ercrowfoot{29.807}{-4.55}{29.957}{-4.4}{29.957}{-4.55}{29.957}{-4.7}{0}\eridrefrel{29.65675}{-4.399999999999999}{-4.699999999999999}\ercrowfoot{29.807}{-4.55}{29.957}{-4.4}{29.957}{-4.55}{29.957}{-4.7}{0}\ercrowfoot{29.807}{-4.55}{29.657}{-4.4}{29.657}{-4.55}{29.657}{-4.7}{0}
% relationship 
\ertext{34.917}{-8.3}{l}{}\ertext{33.006}{-12.75}{l}{..}\errelarm{34.767}{-8}{34.767}{-8.075}{0}{0}\errelarm{32.856}{-12.15}{32.856}{-12.9}{1}{0}\errelangle{34.767}{-8.075}{34.767}{-8.15}{33.811}{-9.775}{0}{0}\errelangle{33.811}{-9.775}{32.856}{-11.4}{32.856}{-12.15}{1}{0}\eridcomprel{32.705625000000005}{33.005625}{-12.599999999999998}\ercrowfoot{32.856}{-12.75}{32.706}{-12.9}{32.856}{-12.9}{33.006}{-12.9}{0}\ercrowfoot{32.856}{-12.75}{32.706}{-12.6}{32.856}{-12.6}{33.006}{-12.6}{0}
% relationship 
\ertext{37.776}{-8.3}{l}{}\ertext{40.006}{-9}{l}{..}\errelarm{37.626}{-8}{37.626}{-8.075}{0}{0}\errelarm{39.856}{-8.937}{39.856}{-9.15}{1}{0}\errelangle{37.626}{-8.075}{37.626}{-8.15}{38.741}{-8.438}{0}{0}\errelangle{38.741}{-8.438}{39.856}{-8.725}{39.856}{-8.937}{1}{0}\eridcomprel{39.705625000000005}{40.005625}{-8.849999999999998}\ercrowfoot{39.856}{-9}{39.706}{-9.15}{39.856}{-9.15}{40.006}{-9.15}{0}\ercrowfoot{39.856}{-9}{39.706}{-8.85}{39.856}{-8.85}{40.006}{-8.85}{0}
% relationship 
\ertext{39.047}{-8.3}{l}{}\errelarm{38.897}{-8}{38.897}{-8.25}{0}{0}\errelarm{44.427}{-24.9}{44.427}{-24.95}{1}{0}\errelangle{38.897}{-8.25}{38.897}{-8.5}{41.397}{-8.5}{0}{0}\errelangle{44.427}{-24.9}{44.427}{-24.85}{44.427}{-24.85}{1}{0}\errelangle{41.397}{-8.5}{43.897}{-8.5}{43.897}{-9.875}{0}{0}\errelangle{44.427}{-24.85}{44.427}{-24.85}{44.427}{-18.05}{1}{0}\errelangle{43.897}{-9.875}{43.897}{-11.25}{44.162}{-11.25}{0}{0}\errelangle{44.162}{-11.25}{44.427}{-11.25}{44.427}{-18.05}{1}{0}
% relationship subject
\ertext{33.029}{-7.625}{r}{subject}\errelarm{33.179}{-7.325}{32.579}{-7.325}{0}{0}\errelarm{19.806}{-7.1}{19.606}{-7.1}{0}{0}\ertext{25.893}{-7.612}{l}{\textasciitilde /..=../subject}\errelangle{32.579}{-7.325}{31.979}{-7.325}{25.993}{-7.212}{0}{0}\errelangle{25.993}{-7.212}{20.006}{-7.1}{19.806}{-7.1}{0}{0}\ercrowfoot{33.029}{-7.325}{33.179}{-7.175}{33.179}{-7.325}{33.179}{-7.475}{0}\eridrefrel{32.878750000000004}{-7.174999999999999}{-7.475}\ercrowfoot{33.029}{-7.325}{33.179}{-7.175}{33.179}{-7.325}{33.179}{-7.475}{0}\ercrowfoot{33.029}{-7.325}{32.879}{-7.175}{32.879}{-7.325}{32.879}{-7.475}{0}
% relationship RTreference
\ertext{39.683}{-8.098}{l}{RT}\ertext{39.683}{-8.398}{l}{reference}\errelarm{39.533}{-7.797}{40.133}{-7.797}{0}{0}\errelarm{34.522}{-14.18}{34.322}{-14.18}{0}{0}\errelangle{40.133}{-7.797}{40.733}{-7.797}{41.733}{-7.797}{0}{0}\errelangle{34.522}{-14.18}{34.722}{-14.18}{38.727}{-14.18}{0}{0}\ertext{42.883}{-9.789}{l}{\textasciitilde /..=.}\errelangle{41.733}{-7.797}{42.733}{-7.797}{42.733}{-10.989}{0}{0}\errelangle{42.733}{-10.989}{42.733}{-14.18}{38.727}{-14.18}{0}{0}\ercrowfoot{39.683}{-7.797}{39.533}{-7.647}{39.533}{-7.797}{39.533}{-7.948}{0}
% relationship basedon
\ertext{39.683}{-7.49}{l}{based}\ertext{39.683}{-7.79}{l}{on}\errelangle{39.533}{-7.19}{39.533}{-7.19}{39.832}{-7.19}{0}{0}\errelangle{39.533}{-6.852}{39.533}{-6.852}{39.832}{-6.852}{0}{0}\ertext{41.383}{-7.621}{r}{event}\ertext{41.383}{-7.321}{r}{\textasciitilde /..=../base}\errelangle{39.832}{-7.19}{40.133}{-7.19}{40.133}{-7.021}{0}{0}\errelangle{40.133}{-7.021}{40.133}{-6.852}{39.832}{-6.852}{0}{0}\ercrowfoot{39.683}{-7.19}{39.533}{-7.04}{39.533}{-7.19}{39.533}{-7.34}{0}
% relationship 
\ertext{33.006}{-14.8}{l}{}\ertext{33.006}{-16.45}{l}{..}\errelarm{32.856}{-14.5}{32.856}{-15.55}{0}{0}\errelarm{32.856}{-15.55}{32.856}{-16.6}{1}{0}\eridcomprel{32.705625000000005}{33.005625}{-16.3}\ercrowfoot{32.856}{-16.45}{32.706}{-16.6}{32.856}{-16.6}{33.006}{-16.6}{0}\ercrowfoot{32.856}{-16.45}{32.706}{-16.3}{32.856}{-16.3}{33.006}{-16.3}{0}
% relationship subject
\ertext{31.239}{-14}{r}{subject}\errelarm{31.389}{-13.7}{30.789}{-13.7}{0}{0}\errelarm{16.934}{-14.5}{16.734}{-14.5}{0}{0}\ertext{23.511}{-14.4}{r}{\textasciitilde /..=../subject}\errelangle{30.789}{-13.7}{30.189}{-13.7}{23.661}{-14.1}{0}{0}\errelangle{23.661}{-14.1}{17.134}{-14.5}{16.934}{-14.5}{0}{0}\ercrowfoot{31.239}{-13.7}{31.389}{-13.55}{31.389}{-13.7}{31.389}{-13.85}{0}\eridrefrel{31.089125000000003}{-13.549999999999999}{-13.85}\ercrowfoot{31.239}{-13.7}{31.389}{-13.55}{31.389}{-13.7}{31.389}{-13.85}{0}\ercrowfoot{31.239}{-13.7}{31.089}{-13.55}{31.089}{-13.7}{31.089}{-13.85}{0}
% relationship basedon
\ertext{34.472}{-13.84}{l}{based}\ertext{34.472}{-14.14}{l}{on}\errelangle{34.322}{-13.54}{34.322}{-13.54}{34.622}{-13.54}{0}{0}\errelangle{34.322}{-13.14}{34.322}{-13.14}{34.622}{-13.14}{0}{0}\ertext{36.172}{-13.94}{r}{on}\ertext{36.172}{-13.64}{r}{\textasciitilde /..=../based}\errelangle{34.622}{-13.54}{34.922}{-13.54}{34.922}{-13.34}{0}{0}\errelangle{34.922}{-13.34}{34.922}{-13.14}{34.622}{-13.14}{0}{0}\ercrowfoot{34.472}{-13.54}{34.322}{-13.39}{34.322}{-13.54}{34.322}{-13.69}{0}
% relationship 
\ertext{39.624}{-10.55}{l}{}\ertext{39.756}{-10.8}{l}{..}\errelarm{39.474}{-10.25}{39.474}{-10.325}{0}{0}\errelarm{39.606}{-10.737}{39.606}{-10.95}{1}{0}\errelangle{39.474}{-10.325}{39.474}{-10.4}{39.54}{-10.462}{0}{0}\errelangle{39.54}{-10.462}{39.606}{-10.525}{39.606}{-10.737}{1}{0}\eridcomprel{39.455625000000005}{39.755625}{-10.649999999999999}\ercrowfoot{39.606}{-10.8}{39.456}{-10.95}{39.606}{-10.95}{39.756}{-10.95}{0}\ercrowfoot{39.606}{-10.8}{39.456}{-10.65}{39.606}{-10.65}{39.756}{-10.65}{0}
% relationship 
\ertext{40.923}{-10.55}{l}{}\errelarm{40.773}{-10.25}{40.773}{-10.325}{0}{0}\errelarm{43.583}{-24.9}{43.583}{-24.95}{1}{0}\errelangle{40.773}{-10.325}{40.773}{-10.4}{40.773}{-12.4}{0}{0}\errelangle{43.583}{-24.9}{43.583}{-24.85}{43.583}{-19.625}{1}{0}\errelangle{40.773}{-12.4}{40.773}{-14.4}{42.178}{-14.4}{0}{0}\errelangle{42.178}{-14.4}{43.583}{-14.4}{43.583}{-19.625}{1}{0}
% relationship subject
\ertext{38.559}{-9.67}{r}{subject}\errelarm{38.709}{-9.37}{38.109}{-9.37}{0}{0}\errelarm{22.471}{-9.6}{22.271}{-9.6}{0}{0}\ertext{31.44}{-9.785}{r}{\textasciitilde /..=../subject}\errelangle{38.109}{-9.37}{37.509}{-9.37}{30.09}{-9.485}{0}{0}\errelangle{30.09}{-9.485}{22.671}{-9.6}{22.471}{-9.6}{0}{0}\ercrowfoot{38.559}{-9.37}{38.709}{-9.22}{38.709}{-9.37}{38.709}{-9.52}{0}\eridrefrel{38.40925000000001}{-9.219999999999999}{-9.52}\ercrowfoot{38.559}{-9.37}{38.709}{-9.22}{38.709}{-9.37}{38.709}{-9.52}{0}\ercrowfoot{38.559}{-9.37}{38.409}{-9.22}{38.409}{-9.37}{38.409}{-9.52}{0}
% relationship IS
\ertext{41.152}{-9.67}{l}{IS}\errelangle{41.002}{-9.37}{41.002}{-9.37}{41.302}{-9.37}{0}{0}\errelangle{41.002}{-9.7}{41.002}{-9.7}{41.302}{-9.7}{0}{0}\ertext{41.852}{-9.835}{l}{\textasciitilde /..=..}\errelangle{41.302}{-9.37}{41.602}{-9.37}{41.602}{-9.535}{0}{0}\errelangle{41.602}{-9.535}{41.602}{-9.7}{41.302}{-9.7}{0}{0}\ercrowfoot{41.152}{-9.37}{41.002}{-9.22}{41.002}{-9.37}{41.002}{-9.52}{0}
% relationship selected
\ertext{41.152}{-10.33}{l}{selected}\errelarm{41.002}{-10.03}{41.102}{-10.03}{0}{0}\errelarm{40.617}{-11.28}{40.417}{-11.28}{0}{0}\errelangle{41.102}{-10.03}{41.202}{-10.03}{41.252}{-10.03}{0}{0}\errelangle{40.617}{-11.28}{40.817}{-11.28}{41.059}{-11.28}{0}{0}\ertext{41.552}{-11.255}{l}{\textasciitilde /..=.}\errelangle{41.252}{-10.03}{41.302}{-10.03}{41.302}{-10.655}{0}{0}\errelangle{41.302}{-10.655}{41.302}{-11.28}{41.059}{-11.28}{0}{0}\ercrowfoot{41.152}{-10.03}{41.002}{-9.88}{41.002}{-10.03}{41.002}{-10.18}{0}
% relationship subject
\ertext{38.645}{-11.47}{r}{subject}\errelarm{38.795}{-11.17}{38.195}{-11.17}{0}{0}\errelarm{22.261}{-11.4}{22.061}{-11.4}{0}{0}\ertext{31.378}{-11.585}{r}{\textasciitilde /..=../subject}\errelangle{38.195}{-11.17}{37.595}{-11.17}{30.028}{-11.285}{0}{0}\errelangle{30.028}{-11.285}{22.461}{-11.4}{22.261}{-11.4}{0}{0}\ercrowfoot{38.645}{-11.17}{38.795}{-11.02}{38.795}{-11.17}{38.795}{-11.32}{0}\eridrefrel{38.494625000000006}{-11.02}{-11.32}\ercrowfoot{38.645}{-11.17}{38.795}{-11.02}{38.795}{-11.17}{38.795}{-11.32}{0}\ercrowfoot{38.645}{-11.17}{38.495}{-11.02}{38.495}{-11.17}{38.495}{-11.32}{0}
% relationship 
\ertext{32.517}{-17.8}{l}{}\ertext{32.506}{-18.1}{l}{..}\errelarm{32.367}{-17.5}{32.367}{-17.575}{0}{0}\errelarm{32.356}{-18.038}{32.356}{-18.25}{1}{0}\errelangle{32.367}{-17.575}{32.367}{-17.65}{32.361}{-17.738}{0}{0}\errelangle{32.361}{-17.738}{32.356}{-17.825}{32.356}{-18.038}{1}{0}\eridcomprel{32.205625000000005}{32.505625}{-17.950000000000003}\ercrowfoot{32.356}{-18.1}{32.206}{-18.25}{32.356}{-18.25}{32.506}{-18.25}{0}\ercrowfoot{32.356}{-18.1}{32.206}{-17.95}{32.356}{-17.95}{32.506}{-17.95}{0}
% relationship 
\ertext{34.179}{-17.8}{l}{}\errelarm{34.029}{-17.5}{34.029}{-17.575}{0}{0}\errelarm{42.738}{-24.9}{42.738}{-24.95}{1}{0}\errelangle{34.029}{-17.575}{34.029}{-17.65}{34.029}{-17.775}{0}{0}\errelangle{42.738}{-24.9}{42.738}{-24.85}{42.738}{-21.375}{1}{0}\errelangle{34.029}{-17.775}{34.029}{-17.9}{38.383}{-17.9}{0}{0}\errelangle{38.383}{-17.9}{42.738}{-17.9}{42.738}{-21.375}{1}{0}
% relationship subject
\ertext{31.239}{-17.35}{r}{subject}\ertext{24.04}{-17.35}{l}{\textasciitilde /..=../subject}\errelarm{31.389}{-17.05}{23.89}{-17.05}{0}{0}\errelarm{23.89}{-17.05}{16.392}{-17.05}{0}{0}\ercrowfoot{31.239}{-17.05}{31.389}{-16.9}{31.389}{-17.05}{31.389}{-17.2}{0}\eridrefrel{31.089125000000003}{-16.900000000000002}{-17.2}\ercrowfoot{31.239}{-17.05}{31.389}{-16.9}{31.389}{-17.05}{31.389}{-17.2}{0}\ercrowfoot{31.239}{-17.05}{31.089}{-16.9}{31.089}{-17.05}{31.089}{-17.2}{0}
% relationship monitored
\ertext{34.472}{-17.08}{l}{monitored}\errelarm{34.322}{-16.78}{34.922}{-16.78}{0}{0}\errelarm{38.509}{-10.03}{38.709}{-10.03}{0}{0}\ertext{37.516}{-13.705}{r}{\textasciitilde /..=../..}\errelangle{34.922}{-16.78}{35.522}{-16.78}{36.916}{-13.405}{0}{0}\errelangle{36.916}{-13.405}{38.309}{-10.03}{38.509}{-10.03}{0}{0}\ercrowfoot{34.472}{-16.78}{34.322}{-16.63}{34.322}{-16.78}{34.322}{-16.93}{0}
% relationship basedon
\ertext{34.472}{-17.575}{l}{based}\ertext{34.472}{-17.875}{l}{on}\errelangle{34.322}{-17.275}{34.322}{-17.275}{34.622}{-17.275}{0}{0}\errelangle{34.322}{-17.05}{34.322}{-17.05}{34.622}{-17.05}{0}{0}\ertext{36.172}{-17.763}{r}{on}\ertext{36.172}{-17.463}{r}{\textasciitilde /..=../based}\errelangle{34.622}{-17.275}{34.922}{-17.275}{34.922}{-17.163}{0}{0}\errelangle{34.922}{-17.163}{34.922}{-17.05}{34.622}{-17.05}{0}{0}\ercrowfoot{34.472}{-17.275}{34.322}{-17.125}{34.322}{-17.275}{34.322}{-17.425}{0}
% relationship timeseries
\ertext{29.964}{-20.15}{l}{time}\ertext{29.964}{-20.45}{l}{series}\ertext{29.506}{-21.3}{l}{..}\errelarm{29.814}{-19.85}{29.814}{-20.05}{0}{0}\errelarm{29.356}{-21.4}{29.356}{-21.45}{1}{0}\errelangle{29.814}{-20.05}{29.814}{-20.25}{29.585}{-20.8}{0}{0}\errelangle{29.585}{-20.8}{29.356}{-21.35}{29.356}{-21.4}{1}{0}
% relationship 
\ertext{32.506}{-20.15}{l}{}\ertext{32.765}{-26.4}{l}{..}\errelarm{32.356}{-19.85}{32.356}{-19.925}{0}{0}\errelarm{32.615}{-26.5}{32.615}{-26.55}{1}{0}\errelangle{32.356}{-19.925}{32.356}{-20}{32.485}{-23.225}{0}{0}\errelangle{32.485}{-23.225}{32.615}{-26.45}{32.615}{-26.5}{1}{0}
% relationship 
\ertext{33.141}{-20.15}{l}{}\ertext{35.008}{-24.85}{l}{..}\errelarm{32.991}{-19.85}{32.991}{-19.925}{0}{0}\errelarm{34.858}{-24.95}{34.858}{-25}{1}{0}\errelangle{32.991}{-19.925}{32.991}{-20}{33.925}{-22.45}{0}{0}\errelangle{33.925}{-22.45}{34.858}{-24.9}{34.858}{-24.95}{1}{0}
% relationship 
\ertext{34.094}{-20.15}{l}{}\ertext{39.565}{-21.3}{l}{..}\errelarm{33.944}{-19.85}{33.944}{-20.1}{0}{0}\errelarm{39.415}{-21.4}{39.415}{-21.45}{1}{0}\errelangle{33.944}{-20.1}{33.944}{-20.35}{36.679}{-20.85}{0}{0}\errelangle{36.679}{-20.85}{39.415}{-21.35}{39.415}{-21.4}{1}{0}
% relationship 
\ertext{35.047}{-20.15}{l}{}\errelarm{34.897}{-19.85}{34.897}{-20.1}{0}{0}\errelarm{41.893}{-24.9}{41.893}{-24.95}{1}{0}\errelangle{34.897}{-20.1}{34.897}{-20.35}{34.897}{-20.4}{0}{0}\errelangle{41.893}{-24.9}{41.893}{-24.85}{41.893}{-22.65}{1}{0}\errelangle{34.897}{-20.4}{34.897}{-20.45}{38.395}{-20.45}{0}{0}\errelangle{38.395}{-20.45}{41.893}{-20.45}{41.893}{-22.65}{1}{0}
% relationship subject
\ertext{29.029}{-18.95}{r}{subject}\errelarm{29.179}{-18.65}{28.579}{-18.65}{0}{0}\errelarm{17.125}{-19}{16.925}{-19}{0}{0}\ertext{22.502}{-19.125}{r}{\textasciitilde /..=../subject}\errelangle{28.579}{-18.65}{27.979}{-18.65}{22.652}{-18.825}{0}{0}\errelangle{22.652}{-18.825}{17.325}{-19}{17.125}{-19}{0}{0}\ercrowfoot{29.029}{-18.65}{29.179}{-18.5}{29.179}{-18.65}{29.179}{-18.8}{0}\eridrefrel{28.878750000000004}{-18.500000000000004}{-18.8}\ercrowfoot{29.029}{-18.65}{29.179}{-18.5}{29.179}{-18.65}{29.179}{-18.8}{0}\ercrowfoot{29.029}{-18.65}{28.879}{-18.5}{28.879}{-18.65}{28.879}{-18.8}{0}
% relationship extracted
\ertext{35.683}{-18.95}{l}{extracted}\ertext{38.645}{-11.68}{r}{chromatograms}\errelarm{35.533}{-18.65}{36.333}{-18.65}{0}{0}\errelarm{38.595}{-11.83}{38.795}{-11.83}{0}{0}\ertext{39.364}{-15.54}{r}{\textasciitilde /..=../monitored}\errelangle{36.333}{-18.65}{37.133}{-18.65}{37.764}{-15.24}{0}{0}\errelangle{37.764}{-15.24}{38.395}{-11.83}{38.595}{-11.83}{0}{0}
% relationship basedon
\ertext{35.683}{-19.75}{l}{based}\ertext{35.683}{-20.05}{l}{on}\errelangle{35.533}{-19.45}{35.533}{-19.45}{35.833}{-19.45}{0}{0}\errelangle{35.533}{-19.05}{35.533}{-19.05}{35.833}{-19.05}{0}{0}\ertext{37.383}{-19.85}{r}{on}\ertext{37.383}{-19.55}{r}{\textasciitilde /..=../based}\errelangle{35.833}{-19.45}{36.133}{-19.45}{36.133}{-19.25}{0}{0}\errelangle{36.133}{-19.25}{36.133}{-19.05}{35.833}{-19.05}{0}{0}\ercrowfoot{35.683}{-19.45}{35.533}{-19.3}{35.533}{-19.45}{35.533}{-19.6}{0}
% relationship lastmodified
\ertext{46.146}{-26.125}{l}{last}\ertext{46.146}{-26.425}{l}{modified}\errelarm{45.996}{-25.825}{46.146}{-25.825}{0}{0}\errelarm{47.127}{-1.48}{46.927}{-1.48}{0}{0}\errelangle{46.146}{-25.825}{46.296}{-25.825}{50.646}{-25.825}{0}{0}\errelangle{47.127}{-1.48}{47.326}{-1.48}{51.161}{-1.48}{0}{0}\ertext{55.596}{-13.953}{r}{\textasciitilde /\textasciicircum =\textasciicircum }\errelangle{50.646}{-25.825}{54.996}{-25.825}{54.996}{-13.653}{0}{0}\errelangle{54.996}{-13.653}{54.996}{-1.48}{51.161}{-1.48}{0}{0}\ercrowfoot{46.146}{-25.825}{45.996}{-25.675}{45.996}{-25.825}{45.996}{-25.975}{0}\erarc{41.652}{-24.75}{42.617}{-24.55}{44.548}{-24.55}{45.514}{-24.75}
% relationship 
\ertext{46.733}{-7.35}{l}{}\errelarm{46.583}{-7.05}{46.583}{-7.5}{0}{0}\errelarm{46.583}{-7.5}{46.583}{-7.95}{1}{0}\eridcomprel{46.482625}{46.682625}{-7.699999999999998}
% relationship previous
\ertext{33.801}{-27.53}{l}{previous}\errelangle{33.651}{-27.23}{33.651}{-27.23}{33.951}{-27.23}{0}{0}\errelangle{33.651}{-26.975}{33.651}{-26.975}{33.951}{-26.975}{0}{0}\ertext{35.501}{-27.703}{r}{on}\ertext{35.501}{-27.403}{r}{\textasciitilde /..=../based}\errelangle{33.951}{-27.23}{34.251}{-27.23}{34.251}{-27.103}{0}{0}\errelangle{34.251}{-27.103}{34.251}{-26.975}{33.951}{-26.975}{0}{0}\ercrowfoot{33.801}{-27.23}{33.651}{-27.08}{33.651}{-27.23}{33.651}{-27.38}{0}
% relationship method
\ertext{41.765}{-22.215}{l}{method}\errelarm{41.615}{-21.915}{42.215}{-21.915}{0}{0}\errelarm{47.795}{-6.69}{47.595}{-6.69}{0}{0}\errelangle{42.215}{-21.915}{42.815}{-21.915}{47.965}{-21.915}{0}{0}\errelangle{47.795}{-6.69}{47.995}{-6.69}{50.555}{-6.69}{0}{0}\ertext{54.715}{-14.603}{r}{\textasciitilde /..=../../../../..}\errelangle{47.965}{-21.915}{53.115}{-21.915}{53.115}{-14.303}{0}{0}\errelangle{53.115}{-14.303}{53.115}{-6.69}{50.555}{-6.69}{0}{0}\ercrowfoot{41.765}{-21.915}{41.615}{-21.765}{41.615}{-21.915}{41.615}{-22.065}{0}
% relationship lastreviewed
\ertext{41.765}{-22.68}{l}{last}\ertext{41.765}{-22.98}{l}{reviewed}\errelarm{41.615}{-22.38}{42.215}{-22.38}{0}{0}\errelarm{47.127}{-2.04}{46.927}{-2.04}{0}{0}\errelangle{42.215}{-22.38}{42.815}{-22.38}{48.815}{-22.38}{0}{0}\errelangle{47.127}{-2.04}{47.326}{-2.04}{51.071}{-2.04}{0}{0}\ertext{55.415}{-12.51}{r}{\textasciitilde /\textasciicircum =\textasciicircum }\errelangle{48.815}{-22.38}{54.815}{-22.38}{54.815}{-12.21}{0}{0}\errelangle{54.815}{-12.21}{54.815}{-2.04}{51.071}{-2.04}{0}{0}\ercrowfoot{41.765}{-22.38}{41.615}{-22.23}{41.615}{-22.38}{41.615}{-22.53}{0}
% relationship lastmodified
\ertext{41.765}{-23.3}{l}{last}\ertext{41.765}{-23.6}{l}{modified}\errelarm{41.615}{-23}{42.215}{-23}{0}{0}\errelarm{47.127}{-1.76}{46.927}{-1.76}{0}{0}\errelangle{42.215}{-23}{42.815}{-23}{49.24}{-23}{0}{0}\errelangle{47.127}{-1.76}{47.326}{-1.76}{51.496}{-1.76}{0}{0}\ertext{56.265}{-12.68}{r}{\textasciitilde /\textasciicircum =\textasciicircum }\errelangle{49.24}{-23}{55.665}{-23}{55.665}{-12.38}{0}{0}\errelangle{55.665}{-12.38}{55.665}{-1.76}{51.496}{-1.76}{0}{0}\ercrowfoot{41.765}{-23}{41.615}{-22.85}{41.615}{-23}{41.615}{-23.15}{0}
% relationship previous
\ertext{41.765}{-24.54}{l}{previous}\errelangle{41.615}{-24.24}{41.615}{-24.24}{41.915}{-24.24}{0}{0}\errelangle{41.615}{-23.775}{41.615}{-23.775}{41.915}{-23.775}{0}{0}\ertext{43.465}{-24.608}{r}{on}\ertext{43.465}{-24.308}{r}{\textasciitilde /..=../based}\errelangle{41.915}{-24.24}{42.215}{-24.24}{42.215}{-24.008}{0}{0}\errelangle{42.215}{-24.008}{42.215}{-23.775}{41.915}{-23.775}{0}{0}\ercrowfoot{41.765}{-24.24}{41.615}{-24.09}{41.615}{-24.24}{41.615}{-24.39}{0}
\end{erdiagram}

}
\pause
\vspace{-5cm}
\begin{block}{}
This example has 
\begin{itemize}
  \item 33  relationships implemented by structural containment, 
  \item 26  relationships implemented by relational referencing (inclusion depedencies),
  \pause \item 16 non-trivial commutative diagrams,
  \pause \item 6 pullback diagrams.
\end{itemize}
Generated into code in XML, ECMA Javascript and Python. 
\end{block}
\end{frame}

\ifNotesnAll
\begin{frame}Example LCMSMS Notes
\begin{itemize}
  \item This is an example of a data specification written by myself on a project when I worked for a science company. This was a project to build an application for managing and interpreting data from mass spectrometers using liquid chromatography. The application was written in javascript and python and used XML and IDL at various positions in the architecture.
\end{itemize}
\end{frame}
\fi

