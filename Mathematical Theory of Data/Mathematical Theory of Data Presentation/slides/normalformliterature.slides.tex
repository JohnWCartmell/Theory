
\begin{frame}{1970 - 1979}
\begin{itemize}
\item In 1970 E.F.Codd introduces the relational model of data and introduces the idea of normal form.
\item In 1971, he introduces the terms `functional dependency' and  `third normal form' are introduced in an IBM techical report published in a now out of print book.
\item In 1977 Fagin introduces `fourth normal form' (4NF) and `multivalued dependencies'.
\item In 1979, Fagin describes `projection-join normal form' also known as fifth normal form (5NF).
\end{itemize} 
\end{frame}

\begin{frame}{Inclusion Normal Forms}
Ling and Goh 1992 
\begin{quote}
Since
classical normal forms (including the Improved 3NF)
have failed to consider the effects of INDs on the structure
of a database, they are inadequate in characterizing a
database scheme which is truly devoid of redundancies.
In consideration of the above, we propose a new normal
form, called Inclusion Normal Form (IN-NF)...
\end{quote}
\end{frame}

\begin{frame}{The success of Codd's Relational Model of Data}
\begin{itemize}
\item By 2020 Oracle Corporation, founded in 1977, were the world's second largest software company with a 42\% share of an 
estimated \$30billion market for relational database technology. 
\item Codd 1990 says that
\begin{quote}
The relational model is solidly based on two parts of mathematics: first-
order predicate logic and the theory of relations.
\end{quote}
\item My assertion is that this has been to base data modelling on the wrong mathematics. 
\item Codd's mathematical basis and therefore his  model  do nothing to guide the programmer as navigator, to use Charles W Bachman's phrase, 
\item nor do they encourage thinking about navigation path equivalence, i.e. diagrams that commute.
\item I will demonstrate the importance of diagrams that commute to the goodness of data specifications.
\item The right mathematical starting point for the theory of data is category theory.
\end{itemize}
\end{frame}
 


