

\begin{frame}{Fundamental Principles in Outline}
\begin{itemize}
    \item Principle 1 -- absence of redundancy in presentation.
    \item Principle 2 -- the theory be the tightest possible fit to the facts.
\end{itemize}

Regarding which
\begin{itemize}
    \item Principle 2 expresses a kind of logical completeness.
    \item The two principles collectively
    \begin{itemize}
        \item ensure absence of redundancy in data and in data management logic,
        \item imply classic relational normal forms.
    \end{itemize}
\end{itemize}
\end{frame}

\begin{frame}{Data Specifications}
Two kinds of types in play
\begin{itemize}
\item  the \textit{definienda} -- types all of whose instances are \textit{particulars}
\begin{itemize}
\item employee, department, student, account, product, order, shipment, delivery, flight, booking and so on,
\item molecular structure, atom, bond, element, isotope, reaction, metabolite, mass trace, chromatogram, peak,
\item table, column, primary key, foreign key.
\end{itemize}
\pause 
\item  the \textit{definiens}  -- types all of whose instances are \textit{universals}
\begin{itemize}
       \item string, integer, float, boolean and so on.
\end{itemize}
\end{itemize}
\pause
\begin{itemize}
\item In ER modelling 
\begin{itemize}
\item the \textit{definienda} are called \textit{entity types}
\item the \textit{definiens} are called \textit{attribute types} or \textit{domains}.
\end{itemize}
\end{itemize}
\end{frame}

\begin{frame}{Data Specifications}
A data specification is a sketch of a category with some additional structure:
\begin{itemize}
\item that it is a \textit{sketch} is crucial because it is only nodes and edges of the sketch for which data is stored and/or communicated, 
\item that there are commutative diagrams is crucial to construction of representational 
specifications from logical specifications.
\item that the category had additional structure is significant:
\begin{itemize}
\item so that we can distinguish structural from non-structural relationships to describe structure nesting and thereby hierarchical data,
\item so that we can give account of database normal forms 
(BCNF, 3NF, 4NF and 5NF),
\item so that we can allow for missing data as represented by NULL values, 
\item so that types of universals can be distinguished from types of particulars.
\end{itemize}
\end{itemize}
\end{frame}

