
\usepackage{mathptmx}
\usepackage{amsfonts}
\usepackage{wasysym}
\usepackage{url}
\usepackage{hyperref}

\newcommand{\sharedmacros}{../../../SharedMacros}
%\usepackage{imakeidx}
\makeindex[name=definitions, title=Index of Definitions]
\makeindex[name=lemmas, title=Index of Lemmas]



\newcommand{\commentary}[1]{\marginpar{\footnotesize #1}}
\newcommand{\highlight}[1]{\colorbox{orange}{#1}}
\newcommand{\term}[1]{\textit{#1}\commentary{\colorbox{lightgray}{\textit{#1}}}\index[definitions]{#1}}
\newcommand{\llabel}[1]{\label{#1}\commentary{\colorbox{pink}{\scriptsize{#1}}}\index[lemmas]{#1}}
\newcommand{\lref}[1]{\ref{#1}\colorbox{pink}{\scriptsize{#1}}\index[lemmas]{#1!use of}}

\newcommand{\newt}[1]{\colorbox{yellow}{#1}}
\newenvironment{newtt}
{  \colorbox{yellow}{$[$ ...} 
}
{  \colorbox{yellow}{... $]$}
}
\newcommand{\oldt}[1]{\colorbox{yellow}{\sout{#1}}}
\newenvironment{oldtt}
{  \colorbox{red}{$[$ ...} 
}
{  \colorbox{red}{... $]$}
}

\newcommand{\reinstatet}[1]{\colorbox{lime}{#1}}
\newenvironment{reinstatett}
{  \colorbox{lime}{$[$ ...}
}
{  \colorbox{lime}{... $]$}
}

\newcommand{\tbd}{\highlight{TBD}}

%ithprojection function
\newcommand{\proji}[1]{\pi_#1}



\newenvironment{categoricalaside}
{\begin{framed}
\textbf{Categorical Aside}
}
{
\end{framed}
}

\newenvironment{noteforfuture}
{\begin{framed}
\textbf{Note For Future}
}
{
\end{framed}
}

\newenvironment{problem}
{\begin{framed}
\textbf{Problem}
}
{
\end{framed}
}

%quine quote
\newcommand{\qq}[1]{
\left\ulcorner#1\right\urcorner
}

%single quote
\newcommand{\sq}[1]{
\textnormal{\textquotesingle}#1\textnormal{\textquotesingle}
}

%lower quine quote
\newcommand{\lqq}[1]{
\left\llcorner #1\right\lrcorner
}


%from berkley
\newcommand{\langl}{\begin{picture}(4.5,7)
\put(1.1,2.5){\rotatebox{60}{\line(1,0){5.5}}}
\put(1.1,2.5){\rotatebox{300}{\line(1,0){5.5}}}
\end{picture}}
\newcommand{\rangl}{\begin{picture}(4.5,7)
\put(.9,2.5){\rotatebox{120}{\line(1,0){5.5}}}
\put(.9,2.5){\rotatebox{240}{\line(1,0){5.5}}}
\end{picture}}
\newcommand{\lang}{\begin{picture}(5,7)\put(1.1,2.5){\rotatebox{45}{\line(1,0){6.0}}}\put(1.1,2.5){\rotatebox{315}{\line(1,0){6.0}}}\end{picture}}
\newcommand{\rang}{\begin{picture}(5,7)\put(.1,2.5){\rotatebox{135}{\line(1,0){6.0}}}\put(.1,2.5){\rotatebox{225}{\line(1,0){6.0}}}\end{picture}}
%Try sharper tuple brackets -- except gives errors nested in captions so comment out
%\renewcommand{\tuple}[1]{\lang #1 \rang}

\newcommand{\setsuchthat}[2]{\left\{#1 \ \middle|\ #2\right\}}
\newcommand{\set}[1]{\left\{#1\right\}} 

% one to n - wanton
\newcommand{\wanton}[1]{#1_1,...#1_n}
\newcommand{\fn}{\wanton{f}}
\newcommand{\pn}{\wanton{p}}
\newcommand{\qn}{\wanton{q}}
\newcommand{\qnprime}{\wanton{q'}}
\newcommand{\xn}{\wanton{x}}
\newcommand{\xnp}{\wanton{x'}}
\newcommand{\yn}{\wanton{y}}
\newcommand{\ntuple}[1]{\tuple{\wanton{#1}}}
\newcommand{\wantom}[1]{#1_1,...#1_m}
\newcommand{\mtuple}[1]{\tuple{#1_1,...#1_m}}
\newcommand{\qm}{\wantom{q}}
\newcommand{\ym}{\wantom{y}}
\newcommand {\bntuple}{\ensuremath{\ntuple{b}}}
\newcommand {\fntuple}{\ensuremath{\ntuple{f}}}
\newcommand {\fnptuple}{\ensuremath{\ntuple{f}}}
\newcommand {\pntuple}{\ensuremath{\ntuple{p}}}
\newcommand {\qntuple}{\ensuremath{\ntuple{q}}}
\newcommand {\qnptuple}{\ensuremath{\ntuple{q'}}}
\newcommand {\qmtuple}{\ensuremath{\mtuple{q}}}
\newcommand {\sntuple}{\ensuremath{\ntuple{s}}}
\newcommand {\xntuple}{\ensuremath{\ntuple{x}}}
\newcommand {\xnptuple}{\ensuremath{\ntuple{x'}}}
\newcommand {\ymtuple}{\ensuremath{\mtuple{y}}}
\newcommand{\foreachi}[1][n]{for each $i$, $1 \leq i \leq #1$}
\newcommand{\foreachj}[1][m]{for each $j$, $1 \leq j \leq #1$}
\newcommand{\foreachk}[1][l]{for each $k$, $1 \leq k \leq #1$}

    %causes problems when used with bamer

%ccategories.macros.tex 

% Macros for diagrams in contextual categories and related categories

\usepackage{twoopt}
\usepackage{scalerel} 
\usepackage{xargs}

%\usepackage{mathabx}  %Caused font problems
%\usepackage{MnSymbol}  % caused font problems

\newcommand{\conu}
{\mathbf{C}(U)}

\newcommand{\depu}
{\mathbf{D}(U)}

\newcommand{\cat}[1]{\textbf{#1}}
\newcommand{\obj}[1]{\ensuremath{|\cat{#1}|}}
\newcommand{\ccat}[1][C]{\ensuremath{\mathbb{#1}} }
\newcommand{\ccatc}{contextual category \ccat}
\newcommand{\cobj}[2][]{\ensuremath{|\ccat[#2]|_{#1}}}
\newcommand{\cslice}[2]{\ensuremath{\ccat[#1]_{#2}}}
\newcommand{\csliceobj}[3][]{\ensuremath{|\mathbb{#2}_{#3}|_{#1} }}
\newcommand{\varset}[1][]{\ensuremath{V_{#1} }}
\newcommand{\localvarsets}{\ensuremath{\mathcal{V} }}
\newcommand{\Fam}{\ensuremath{\mathbb{F\mathrm{am}} }}
\newcommand{\Famslice}[1]{\ensuremath{\mathbb{F\mathrm{am}}_{#1} }}
\newcommand{\Famobj}[1][]{\ensuremath{|\mathbb{F\mathrm{am}}|_{#1} }}
\newcommand{\Famsliceobj}[2][]{\ensuremath{|\mathbb{F\mathrm{am}}_{#2}|_{#1} }}
\newcommand{\morph}{\rightarrow}
\newcommand{\epi}{\twoheadrightarrow}
\newcommand{\base}{\triangleleft}
\newcommand{\comp}{\circ}
\newcommand{\cross}{\otimes}
\newcommand{\pc}[2]{d^{#1}_{#2}}
\newcommand{\sub}{^*}
\newcommand{\diag}{\delta}
\newcommand{\pbase}[1]{\tilde{#1}}

\newcommand{\tuple}[1]{\langle#1\rangle}
\newcommand{\ndidly}{\ensuremath{\Join_n}}
\newcommand{\ndidlycospan}{quotiented n-cospan}

\newcommand{\crossx}[3]{#1 \underset{#3}{\cross} #2}
\newcommand{\fibrex}[3]{#1 \underset{#3}{\Join} #2}
\newcommand{\powerset}{\mathcal{P}}
\newcommand{\primeds}[1]{
\ensuremath{\mathcal{P}(#1)} }
\newcommand{\compset}{\ \dot{\circ}\, }

% darrow
%\newcommand{\darrow}{\rightarrowtriangle} %use \smorph instead
\newcommand{\smorph}{\rightarrowtriangle}

 

\newcommand\dhead{\scaleobj{0.6}{\triangleright}}
\newcommand{\dmorph}{\, \mbox{---} \! \cdot \! \raisebox{1.1pt}{\dhead}}

% projection tree
%\newcommand{\proj}[2]{proj_{#2}(#1)}

\newcommand{\proj}[2]{
\ensuremath{\mathcal{P}_{#2}(#1)} }

%pstrick supplements for arrows

\newlength{\arrnodesepA}
\newlength{\arrnodesepB}
\newlength{\arroffsetA}
\newlength{\arroffsetB}

%Modified to 2pt from 0pt on 23 July 2018
\newcommand{\arreset}{
\setlength{\arrnodesepA}{2pt}
\setlength{\arrnodesepB}{2pt}
\setlength{\arroffsetA}{0pt}
\setlength{\arroffsetB}{0pt}
}
\arreset

\newcommand{\ncarr}[3][0]{\ncarc[arcangle=#1,nodesepA=\arrnodesepA,nodesepB=\arrnodesepB,offsetA=\arroffsetA,offsetB=\arroffsetB,arrowsize=5pt,arrowinset=0.7]{->}{#2}{#3}}
\newcommand{\jcbarr}[4][0]{ % ncbarr is defined in some thridy party package so do not use!\emph{}
\ncarr[#1]{#3}{#4}
\nbput[labelsep=2pt]{\footnotesize $#2$}
}

\newcommand{\ncaarr}[4][0]{
\ncarr[#1]{#3}{#4}
\naput[labelsep=2pt]{\footnotesize $#2$}
}

% \alabel{label}[npos][labelsep_pts]
\newcommandx*\alabel[3][2=0.5,3=2,usedefault]{\naput[labelsep=#3pt,npos=#2]{\footnotesize $#1$}}
% \blabel{label}[npos][labelsep_pts]
\newcommandx*\blabel[3][2=0.5,3=2,usedefault]{\nbput[labelsep=#3pt,npos=#2]{\footnotesize $#1$}}

% \idcomp mark an arrow as one component of an identifier
\newcommand{\idcomp}{\ncput[npos=0, nrot=:U]{\psline(0.2,-0.075)(0.2,0.075)}}  %add a bar to a node connection arrow
% pstrick supplements for s-arrows (previous name for d-arrow - should convert}

\newlength{\sarnodesepA}
\newlength{\sarnodesepB}
\newlength{\saroffsetA}
\newlength{\saroffsetB}
\newlength{\sarnodesepAsav}
\newlength{\sarnodesepBsav}

\newcommand{\sarreset}{
\setlength{\sarnodesepA}{0pt}
\setlength{\sarnodesepB}{0pt}
\setlength{\saroffsetA}{0pt}
\setlength{\saroffsetB}{0pt}
}

\sarreset

% sar - S-arrow
\newcommand{\ncsar}[3][0]{
\setlength{\sarnodesepAsav}{\sarnodesepA}
\setlength{\sarnodesepBsav}{\sarnodesepB}
\addtolength{\sarnodesepA}{3pt}
\addtolength{\sarnodesepB}{7pt}
\ncarc[nodesepA=\sarnodesepA,nodesepB=\sarnodesepB,offsetA=\saroffsetA,offsetB=\saroffsetB,arcangle=#1]{-}{#2}{#3}
\ncput[nrot=:R,npos=1]{\pstriangle(0,0)(.2,.2)}
\setlength{\sarnodesepA}{\sarnodesepAsav}
\setlength{\sarnodesepB}{\sarnodesepBsav}
}


% bsar - below labelled S-arrow
\newcommand{\ncbsar}[4][0]{
\ncsar[#1]{#3}{#4}
\nbput[labelsep=2pt]{\footnotesize $#2$}
}
% asar - above labelled S-arrow
\newcommand{\ncasar}[4][0]{
\ncsar[#1]{#3}{#4}
\naput[labelsep=2pt]{\footnotesize $#2$}
}

% cdar - composite dependency arrow
\newcommand{\nccdar}[3][0]{
\setlength{\sarnodesepAsav}{\sarnodesepA}
\setlength{\sarnodesepBsav}{\sarnodesepB}
\addtolength{\sarnodesepA}{3pt}
\addtolength{\sarnodesepB}{11pt}
\ncarc[nodesepA=\sarnodesepA,nodesepB=\sarnodesepB,offsetA=\saroffsetA,offsetB=\saroffsetB,arcangle=#1]{-}{#2}{#3}
\ncput[nrot=:R,npos=1]{\pstriangle(0,0.1)(.2,.2)}
\ncput[nrot=:R,npos=1]{\psdot[dotsize=1pt](-0.0075,0.05)}   %!!
\setlength{\sarnodesepA}{\sarnodesepAsav}
\setlength{\sarnodesepB}{\sarnodesepBsav}
}


% bcdar - below labelled composite dependency arrow
\newcommand{\ncbcdar}[4][0]{
\nccdar[#1]{#3}{#4}
\nbput[labelsep=2pt]{\footnotesize $#2$}
}
% acdar - above labelled composite dependency arrow
\newcommand{\ncacdar}[4][0]{
\nccdar[#1]{#3}{#4}
\naput[labelsep=2pt]{\footnotesize $#2$}
}


% rsar - recursive S-arrow
\newcommand{\ncrsar}[2]{
\setlength{\sarnodesepAsav}{\sarnodesepA}
\setlength{\sarnodesepBsav}{\sarnodesepB}
\addtolength{\sarnodesepA}{3pt}
\addtolength{\sarnodesepB}{7pt}
\ncloop[nodesepA=\sarnodesepA,nodesepB=\sarnodesepB,
        offsetA=\saroffsetA,offsetB=\saroffsetB,
        armA=0.7cm,armB=0.6cm,angleA=90,angleB=-90,loopsize=-1,linearc=0.4
				]{-}{#1}{#2}
\ncput[nrot=:R,npos=5]{\pstriangle(0,0)(.2,.2)}
\setlength{\sarnodesepA}{\sarnodesepAsav}
\setlength{\sarnodesepB}{\sarnodesepBsav}
}

% pstrick supplements for multi-arrows

\newlength{\marnodesepA}
\newlength{\marnodesepB}
\newlength{\maroffsetB}
\newlength{\marnodesepBsav}

\newcommand{\marreset}{
\setlength{\marnodesepA}{0pt}
\setlength{\marnodesepB}{0pt}
\setlength{\maroffsetB}{0pt}
}

\marreset

%ncmarr[#1 arcangle1][#2 arcangle2]{#3 name}{#4 domain1}{#5 domain2}{#6 junction}{#7 codomain}
\newcommandtwoopt{\ncmarr}[6][8][8]{%
\ncarc[nodesepA=\marnodesepA,nodesepB=0,arcangle=#1]{-}{#3}{#5}
\ncarc[nodesepB=0,arcangle=-#1]{-}{#4}{#5}
\ncarc[arcangle=#2,nodesepB=\marnodesepB,offsetB=\maroffsetB]{->}{#5}{#6}
}%


\newcommandtwoopt{\nchmarr}[6][8][8]{%
\ncarc[nodesepA=\marnodesepA,nodesepB=0,arcangle=#1]{-}{#3}{#5}
\ncarc[nodesepB=0,arcangle=#1]{-}{#4}{#5}
\ncarc[arcangle=#2,nodesepB=\marnodesepB,offsetB=\maroffsetB]{->}{#5}{#6}
}%

\newcommandtwoopt{\ncamarr}[7][8][8]{%
\ncmarr[#1][#2]{#4}{#5}{#6}{#7}
\naput[npos=.05]{$#3$}
}%
\newcommandtwoopt{\ncbmarr}[7][8][8]{%
\ncmarr[#1][#2]{#4}{#5}{#6}{#7}
\nbput[npos=.05]{$#3$}
}%

\newcommandtwoopt{\ncbhmarr}[7][8][8]{%
\nchmarr[#1][#2]{#4}{#5}{#6}{#7}
\nbput[npos=.05]{$#3$}
}%

\newcommandtwoopt{\ncmarrr}[7][8][8]{
\ncarc[nodesepB=0,arcangle=#1]{-}{#3}{#6}
\ncline[nodesepB=0]{-}{#4}{#6}
\ncarc[nodesepB=0,arcangle=-#1]{-}{#5}{#6}
\ncarc[nodesepA=0,arcangle=#2]{->}{#6}{#7}
}

\newcommandtwoopt{\ncamarrr}[8][8][8]{
\ncmarrr[#1][#2]{#4}{#5}{#6}{#7}{#8}
\naput[npos=.05]{$#3$}
}
\newcommandtwoopt{\ncbmarrr}[8][8][8]{
\ncmarrr[#1][#2]{#4}{#5}{#6}{#7}{#8}
\nbput[npos=.05]{$#3$}
}

%gats.macros.tex

\usepackage{environ}    % also used in ermacros % here used for \NewEnvrion

\newcommand{\gat}[1][U]{
\ensuremath{\mathcal{#1}}}  % used to hav a space in here
\newcommand{\gatw}[1][U]{\gat[#1]\ }  % use this if need trailing space
\newcommand{\ingat}[1][U]{in \gat[#1]}
\newcommand{\isagat}[1][U]{\gat[#1] is a g.a.t.}
\newcommand{\inagat}{in a g.a.t. }

% macro for a generic theory
%\newcommand{\theory}
%{\textit{U}}

\newcommand{\intheory}
{is a derived rule of \gat[U]}

% Macros for GAT rules

\newcommand{\isT}[1]
{#1\mbox{ is a type}}

\newcommand{\ofT}[2]
{#1 \in #2
}

% Macros for GAT rules   <!-- new old -->
\newcommand{\istype}[1]
{#1\mbox{ is a type}}

\newcommand{\oftype}[2]
{#1 \in #2
}

%\context{x}{\Delta}{n}
\newcommand{\context}[3]
{\ofT{#1_1}{#2_1},... \ofT{#1_{#3}}{#2_{#3}(#1_1,...#1_{#3-1})}
}

%\subcontext{x}{\Delta}{i}{k}
\newcommand{\subcontext}[4]
{\ofT{#1_{#3_1}}{#2_{#3_1}},... \ofT{#1_{#3_#4}}{#2_{#3_#4}(#1_1,...#1_{#3_#4-1})}
}

% #schematic context
\newcommand{\schmcon}[3]
{\ofT{#1_1}{#2_1},... \ofT{#1_{#3}}{#2_{#3}}
}
% abbreviated to
\newcommand{\con}[3]
{\schmcon{#1}{#2}{#3}}

% schematic subcontext
%\subcon{x}{\Delta}{i}{k}
\newcommand{\subcon}[4]
{\ofT{#1_{#3_1}}{#2_{#3_1}},... \ofT{#1_{#3_#4}}{#2_{#3_#4}}
}

% permuted context
%\permcon{x}{\Delta}{n}{\sigma}
\newcommand{\permcon}[4]
{\ofT{#1_{#4(1)}}{#2_{#4(1)}},... \ofT{#1_{#4(#3)}}{#2_{#4(#3)}}
}
% permuted term
%\permterm{t}{n}{\sigma}
\newcommand{\permterm}[3]
{
#1_{#3(1)},...#1_{#3(#2)}
}


% Idioms
\newcommand{\xDelta}[1]{\con{x}{\Delta}{#1}}
\newcommand{\xDeltap}[1]{\con{x}{\Delta'}{#1}}
\newcommand{\xOmega}[1]{\con{x}{\Omega}{#1}}
\newcommand{\xOmegap}[1]{\con{x}{\Omega'}{#1}}
\newcommand{\yOmega}[1]{\con{y}{\Omega}{#1}}
\newcommand{\yOmegap}[1]{\con{y}{\Omega'}{#1}}

\newcommand{\xDeltasigma}[1]{\permcon{x}{\Delta}{#1}{\sigma}}
\newcommand{\xDeltapsigma}[1]{\permcon{x}{\Delta'}{#1}{\sigma}}
\newcommand{\xOmegasigma}[1]{\permcon{x}{\Omega}{#1}{\sigma}}
\newcommand{\xOmegapsigma}[1]{\permcon{x}{\Omega'}{#1}{\sigma}}
\newcommand{\yOmegasigma}[1]{\permcon{y}{\Omega}{#1}{\sigma}}
\newcommand{\yOmegapsigma}[1]{\permcon{y}{\Omega'}{#1}{\sigma}}

\newcommand{\xDeltainvsigma}[1]{\permcon{x}{\Delta}{#1}{\sigma^{-1}}}
\newcommand{\xDeltapinvsigma}[1]{\permcon{x}{\Delta'}{#1}{\sigma^{-1}}}
\newcommand{\xOmegainvsigma}[1]{\permcon{x}{\Omega}{#1}{\sigma^{-1}}}
\newcommand{\xOmegapinvsigma}[1]{\permcon{x}{\Omega'}{#1}{\sigma^{-1}}}
\newcommand{\yOmegainvsigma}[1]{\permcon{y}{\Omega}{#1}{\sigma^{-1}}}
\newcommand{\yOmegapinvsigma}[1]{\permcon{y}{\Omega'}{#1}{\sigma^{-1}}}

%Idioms enclosed as tuples
\newcommand{\encxDelta}[1]{\tuple{\con{x}{\Delta}{#1}}}
\newcommand{\encxDeltap}[1]{\tuple{\con{x}{\Delta'}{#1}}}
\newcommand{\encxOmega}[1]{\tuple{\con{x}{\Omega}{#1}}}
\newcommand{\encxOmegap}[1]{\tuple{\con{x}{\Omega'}{#1}}}
\newcommand{\encyOmega}[1]{\tuple{\con{y}{\Omega}{#1}}}
\newcommand{\encyOmegap}[1]{\tuple{\con{y}{\Omega'}{#1}}}

\newcommand{\encxDeltasigma}[1]{\tuple{\permcon{x}{\Delta}{#1}{\sigma}}}
\newcommand{\encxDeltapsigma}[1]{\tuple{\permcon{x}{\Delta'}{#1}{\sigma}}}
\newcommand{\encxOmegasigma}[1]{\tuple{\permcon{x}{\Omega}{#1}{\sigma}}}
\newcommand{\encxOmegapsigma}[1]{\tuple{\permcon{x}{\Omega'}{#1}{\sigma}}}
\newcommand{\encyOmegasigma}[1]{\tuple{\permcon{y}{\Omega}{#1}{\sigma}}}
\newcommand{\encyOmegapsigma}[1]{\tuple{\permcon{y}{\Omega'}{#1}{\sigma}}}

\newcommand{\encxDeltainvsigma}[1]{\tuple{\permcon{x}{\Delta}{#1}{\sigma^{-1}}}}
\newcommand{\encxDeltapinvsigma}[1]{\tuple{\permcon{x}{\Delta'}{#1}{\sigma^{-1}}}}
\newcommand{\encxOmegainvsigma}[1]{\tuple{\permcon{x}{\Omega}{#1}{\sigma^{-1}}}}
\newcommand{\encxOmegapinvsigma}[1]{\tuple{\permcon{x}{\Omega'}{#1}{\sigma^{-1}}}}
\newcommand{\encyOmegainvsigma}[1]{\tuple{\permcon{y}{\Omega}{#1}{\sigma^{-1}}}}
\newcommand{\encyOmegapinvsigma}[1]{\tuple{\permcon{y}{\Omega'}{#1}{\sigma^{-1}}}}

\newcommand{\tstyle}{\vdash}

% gat macros developed for cwf paper

% Expressing gats
\newenvironment{gatrules}
{
$$
\begin{array}{l l}
}
{
\end{array}
$$
}
\newcommand{\gatintros}
{
\textbf{Symbol} & \textbf{Introductory\ Rule}                      \\}

\newcommand{\gataxioms}
{\textbf{Axioms}\\}
\newcommand{\gatintro}[3]{\ #1 & #2 \tstyle #3 \\}
\newcommand{\gatlocalintro}[3]{\ #1 & #2 \dashv }
\newcommand{\gataxiom}[2]{\multicolumn{2}{l}{\ \ #1\mbox{,  whenever\ } #2} \\}
\newcommand{\noleft}{\left.\kern-\nulldelimiterspace} % so that no space taken by absent left brace


\newcommand{\gatmultiaxiom}[2]
{\multicolumn{2}{l}{
  \noleft
    \begin{array}{l}
		#1
    \end{array} 
  \right\} \mbox{whenever\ } 	#2 
	}\\}
	
	\newcommand{\axid}[1]{\text{#1}.\ }	

%New context sharing macros
\newcommand{\gatintroducing}[1]{
{\arraycolsep=0pt
  \begin{array}{l}
          #1
  \end{array}} &
}

%*********************************
% \begin{\gatgroup}{context}
%    rules
%  \end{\gatgroup}
%*********************************
\NewEnviron{gatgroup}[1]{%
  \noleft
  {\arraycolsep=0pt
   \begin{array}{l}
\BODY
    \end{array} 
   }
   \ \right\} 
	%\mbox{\ whenever\ } 
	#1
	\vspace{0.1cm} 
}
%*********************************

%*********************************
% \begin{\gatgroupnoshared}
%    rule
%  \end{\gatgroupnoshared}
%*********************************
\NewEnviron{gatgroupnoshared}{%
  {\arraycolsep=0pt
   \begin{array}{l}
\BODY
    \end{array} 
   }
   \ 
	\vspace{0.1cm} 
}
%*********************************

% \gatsingular[width]{context}{conclusion}
\newcommand{\gatsingular}[3][4cm]{
\begin{gatgroupnoshared}
\gatleaf[#1]{#2}{#3} 
\end{gatgroupnoshared}
}

%*********************************
% \gatleaf}[width]{context}{assertion}
%*********************************
\newcommand{\gatleaf}[3][4cm]{%
\makebox[#1]{$#3$ \dotfill} \dotfill \  #2
}
%*********************************
%*********************************
% \gatstandalonesingle}{context}{assertion}
%*********************************
\newcommand{\gatstandalonesingle}[2]{%
#2 \makebox[2.5cm]{\dotfill} \  #1
}
%*********************************

% \gataxiomno{axiomno}
\newcommand{\gataxiomno}[1]{\makebox[0.5cm]{} \axid{#1}}


% metagat.macros.tex

%Meta-theories

%\newcommand{\typ}{\triangleright}
\newcommand{\typ}{\nabla}
\newcommand{\trm}{\tau}
\newcommand{\cross}{\otimes}
\newcommand{\sub}{^*}
\newcommand{\diag}{\delta}

\newcommand{\typeseq}[2]
{\ofT{#1_1}{\typ},... \ofT{#1_{#2}}{\typ(#1_{#2-1})}}

\newcommand{\typeseqcont}[3]
{\ofT{#1_1}{\typ({#2})},... \ofT{#1_{#3}}{\typ(#1_{#3-1})}}

\newcommand{\Ob}{Ob}
\newcommand{\obj}{Ob} % <!-- new old --<
\newcommand{\Hom}{Hom}
\newcommand{\objseq}[2]
{\ofT{#1_1}{\obj},... \ofT{#1_{#2}}{\obj(#1_{#2-1})}}


\def\dottededge{\ncline[linestyle=dotted, nodesep=0.3cm]}
\def\noedge{\ncline[linestyle=none]}
\def\thinedge{\ncline[linewidth=0.4pt]}

\newcommand{\member}[1]
{\ncarc[arcangle=-30,nodesepB=0.03]{->}{\pspred}{\pssucc}
\nbput[labelsep=0.1]{#1}}

\newcommand{\loweraccutemember}[1]
{\ncarc[arcangle=-15,nodesepB=0.03]{->}{\pspred}{\pssucc}
\nbput[labelsep=0.05,npos=0.85]{#1}}

\newcommand{\uppermember}[1]
{\ncarc[arcangle=30,nodesepB=0.03]{->}{\pspred}{\pssucc}\naput{#1}}

\newcommand{\upperaccutemember}[1]
{\ncarc[arcangle=10,nodesepB=0.03]{->}{\pspred}{\pssucc}\naput[npos=0.85]{#1}}

% flexbranch 
% #1 node label
% #2 thislevelsep
% #3 next level sep
% #4 variable (eg x)
% #5 index leter (eg n)
% #6 close parenthesis
% #7 continuation branches
\newcommand{\flexbranch}[7]
{
\pstree[thislevelsep=*#2,nodesep=0.05]
		{\Rnode{#1 1}{\Tr{#4_1 #6}}}
	  {\pstree[thislevelsep=#3]  
				   {\Rnode{#1 2}{\Tr[edge=\dottededge]{#4_{#5} #6}}}
					 {#7}
		}
}

\newcommand{\flexbranchplusleaf}[6]
{
\flexbranch{#1}{#2}{#3}{#4} {#5} {#6}
  {
   %\Rnode{#1 3}{\Tr{#4 #6}}
	 \Tr{\Rnode{#1 3}{#4 #6}}
  }
}

\newcommand{\flexbranchplusarc}[7]
{
\flexbranch{#1}{#2}{#3}{#4} {#5} {#6}
  {
   %\Rnode{#1 3}{\Tr{#4 #6}\member{#7}}
	 \Tr{\Rnode{#1 3}{#4 #6}}\member{#7}
  }
}

\newcommand{\flexbranchinitialarc}[9]
{
\pstree[thislevelsep=*#2,nodesep=0.05]
		{\Rnode{#1 1}{\Tr{#4_#8 #6}}#9}
	  {\pstree[thislevelsep=#3]  
				   {\Rnode{#1 2}{\Tr[edge=\dottededge]{#4_{#5} #6}}}
					 {#7}
		}
}

\newcommand{\equality}[2]
{
\ncline [doubleline=true, nodesep=0.2cm]{#1}{#2}
}
\newcommand{\equalityarc}[2]
{
\ncarc [arcangleA=-30, arcangleB=-20, doubleline=true, nodesep=0.1cm]{#1}{#2}
}

\usepackage[margin=4.0cm]{geometry} %was 3cm
\usepackage{mathptmx}
\usepackage{amsfonts}
\usepackage{array}
\usepackage{pstricks}
\usepackage{pst-tree}
\usepackage{pst-plot}
\usepackage{pst-node}
\usepackage{stmaryrd}
\usepackage{amsmath}
\usepackage{verbatim}
\usepackage{graphicx}  
\usepackage{calc}
\usepackage{xifthen}
\usepackage{xcolor}
\usepackage{color}
\usepackage{stringstrings}
%\usepackage[small,bf,margin=3pt,format=hang, labelsep=endash,singlelinecheck=false]{caption} %prevuiously justification=justified
%\usepackage{enumerate}
%\usepackage{enumitem}
\usepackage{enumerate}
\usepackage[shortlabels]{enumitem}
\usepackage{float}
\usepackage[section]{placeins}
%\setlength{\captionmargin}{5pt}
\usepackage{environ}
\usepackage{multirow}
\usepackage{rotating}
\usepackage{longtable}
\usepackage{afterpage}
\usepackage{needspace}


%DEFINE ENVIRONMENT BLOCK
% Riddle
\newsavebox{\riddlebox}

\newenvironment{erexample}
{\newcommand\colboxcolor{F0F0F0}%was F8F8F8
\begin{lrbox}{\riddlebox}
\begin{minipage}{\dimexpr\columnwidth-2\fboxsep\relax} \textbf{} \\ \itshape}
{\end{minipage}\end{lrbox}%
%\begin{center}
\colorbox[HTML]{\colboxcolor}{\usebox{\riddlebox}}
%\end{center}
}

\newenvironment{erbox}
{\newcommand\colboxcolor{F0F0F0}%was F8F8F8
\begin{lrbox}{\riddlebox}%
\begin{minipage}{\dimexpr\columnwidth-2\fboxsep\relax} }
{\end{minipage}\end{lrbox}%
%\begin{center}
\colorbox[HTML]{\colboxcolor}{\usebox{\riddlebox}}
%\end{center}
}

%\begin{erboxedFigure}{#1 FigureParam}{#2 Label}{#3 Caption}
\NewEnviron{erboxedFigure}[3]{%
\begin{figure}[#1]
\begin{erexample}
\begin{center}
\BODY
\end{center}
\vspace{-0.5cm}
\caption{#3}
\label{#2}
\end{erexample}
\end{figure}
}

\newcommand{\erpictureFolder}[0]{../SharedPictures}

\newcommand{\ercenterPicture}[1]{
\begin{center}
\input{\erpictureFolder/#1}
\end{center}
}


\newlength{\erhalfHt}

%\erinlinePicture{#1 pictureFilename}{#2 pictureHeight}
\newcommand{\erinlinePicture}[2]{
\setlength{\erhalfHt}{#2cm * \real{0.5}}
\raisebox{-\erhalfHt}[\erhalfHt + 0.5cm][\erhalfHt + 0.5cm]{
\input{\erpictureFolder/#1}
} 
}

%\erplainFig{#1 pictureFilename}{#2 figureParam}{#3Caption}
\newcommand{\erplainFig}[3]{
\begin{figure}[#2]
\begin{center}
\input{\erpictureFolder/#1}
\end{center}
\caption{#3}
\label{#1}
\end{figure}
}

%\erboxedFigPicture{#1 pictureFilename}{#2 figureParam}{#3Caption}
\newcommand{\erboxedFigPicture}[3]{
\begin{figure}[#2]
\begin{erexample}
\vspace{-0.5cm}
\begin{center}
\input{\erpictureFolder/#1}
\end{center}
\caption{#3}
\label{#1}
\end{erexample}
\end{figure}
}

%\erLeftSideFig{#1 pictureFilename}{#2 figureParam}{#3Caption}
\newcommand{\erLeftSideFig}[3]{
\begin{figure}[#2]
\begin{erexample}
  \begin{minipage}[c]{0.4\textwidth}
    \caption{#3}
    \label{#1}
  \end{minipage}
  \begin{minipage}[c]{0.5\textwidth}
    \input{\erpictureFolder/#1}
  \end{minipage}
\end{erexample}
\end{figure}
}

%\erbulletedFig{#1 pictureFilename}{#2 figureParam}{#3Caption}
\NewEnviron{erbulletedFig}[3]{%
\begin{figure}[#2]
\begin{erexample}
\vspace{-0.5cm}
\begin{center}
$
\begin{array}{c m{0.25cm} | m{6cm}}
\raisebox{-2.0cm}{
\input{\erpictureFolder/#1}}& & \text{\parbox{6cm}{\raggedright{\footnotesize{
\begin{enumerate}[(i)]
\BODY
\end{enumerate}}}}} \\
\end{array}
$
\end{center}
\caption{#3}
\label{#1}
\end{erexample}
\end{figure} 
}


%\begin{erbulletedDimFig}{#1 pictureFilename}{#2figureParam} {#3Caption} {#4PictureHeight}{#5TextWidth}

\NewEnviron{erbulletedDimFig}[5]{%
\begin{figure}[#2]
\begin{erexample}
\vspace{-0.5cm}
\begin{center}
$
\begin{array}{c m{0.25cm} |  m{#5cm}}
\setlength{\erhalfHt}{#4cm * \real{0.5}}
\raisebox{-\erhalfHt}{
\input{\erpictureFolder/#1}}& & \text{\parbox{#5cm}{\raggedright{\footnotesize{
\begin{enumerate}[(i)]
\BODY
\end{enumerate}}}}} \\
\end{array}
$
\end{center}
\caption{#3}
\label{#1}
\end{erexample}
\end{figure} 
}

%\begin{ernotedModel}{#1 pictureFilename}{#2PictureHeight}{#3PictureWidth}{#4TextWidth}

\NewEnviron{ernotedModel}[4]{%
\begin{center}
$
\begin{array}{m{#3cm} m{1cm} | c m{#4cm}}
\setlength{\erhalfHt}{#2cm * \real{0.5}}
\raisebox{-\erhalfHt}{
\input{\erpictureFolder/#1}}& & & \text{\parbox{#4cm}{\raggedright{\footnotesize{
\BODY
}}}} \\
\end{array}
$
\end{center} 
}

%\begin{ermodelText}{#1 pictureFilename}{#2PictureHeight}{#3PictureWidth}{#4TextWidth}

\NewEnviron{ermodelText}[4]{%
\begin{center}
\begin{tabular}{m{#3cm} m{1cm}  c m{#4cm}}
\setlength{\erhalfHt}{#2cm * \real{0.5}}
\raisebox{-\erhalfHt}{
\input{\erpictureFolder/#1}}& & & \text{\parbox{#4cm}{\raggedright{\small{
\BODY
}}}} \\
\end{tabular}
\end{center} 
}


%\erbulletedModel{#1 pictureFilename}{#2PictureHeight}{#3PictureWidth}{#4TextWidth}

\NewEnviron{erbulletedModel}[4]{%
\begin{center}
$
\begin{array}{m{#3cm} m{1cm} | c m{#4cm}}
\setlength{\erhalfHt}{2cm * \real{0.5}}
\raisebox{-\erhalfHt}{
\input{\erpictureFolder/#1}}& & & \text{\parbox{#4cm}{\raggedright{\footnotesize{
\begin{enumerate}[(i)]
\BODY
\end{enumerate}}}}} \\
\end{array}
$
\end{center} 
}



%\ernotedDimFig{#1 pictureFilename}{#2 figureParam}{#3Caption}{#4PictureHeight}{#5TextWidth}
\NewEnviron{ernotedDimFig}[5]{%
\begin{figure}[#2]
\begin{erexample}
\vspace{-0.5cm}
\begin{center}
$
\begin{array}{c m{0.25cm} | c m{#5cm}}
\setlength{\erhalfHt}{#4cm * \real{0.5}}
\raisebox{-\erhalfHt}{
\input{\erpictureFolder/#1}}& & & \text{\parbox{#5cm}{\raggedright{\footnotesize{
\BODY }}}}\\
\end{array}
$
\end{center}
\caption{#3}
\label{#1}
\end{erexample}
\end{figure} 
}
%\begin{ernotedDimFigPW}{#1 pictureFilename}{#2 figureParam}{#3Caption}{#4PictureHeight}{#5PictureWidth}{#6TextWidth}
\NewEnviron{ernotedDimFigPW}[6]{%
\begin{figure}[#2]
\begin{erexample}
\vspace{-0.5cm}
\begin{center}
$
\begin{array}{>{\centering}m{#5cm} m{0.5cm} | c m{#6cm}}
\setlength{\erhalfHt}{#4cm * \real{0.5}}
\raisebox{-\erhalfHt}{
\centering \input{\erpictureFolder/#1}
}& & & \text{\parbox{#6cm - 0.5cm}{\raggedright{\footnotesize{
\BODY }}}}\\
\end{array}
$ \\
\vspace {0.2cm}
\end{center}
\caption{#3}
\label{#1}
\end{erexample}
\end{figure}
}



\newenvironment{erquote}
{\begin{quote}\itshape}
{\end{quote}}


%
%  erdiag
%
  
%\begin{erdiagram}{#1 height}{#2 width} 
% ....
% ....
%\end{erdiagram}
\newenvironment{erdiagram}[2]
{%\pspicture*(-#1,0)(#2,0)
\pspicture*(0,-#1)(#2,0)
%\psgrid
}
{\endpspicture}

\definecolor{lightyellow}{cmyk}{0,0,0.3,0}

% \eret{#1 x0} {#2 y0} {#3 x1} {#4 y1} {#5 corner radius} {#6 fill}
\newcommand {\eret}[6]
{ 
\ifthenelse{\equal{#6}{1}}
{\psframe[framearc=#5,fillstyle=solid,fillcolor=lightyellow](#1,#2)(#3,#4)}
{\psframe[framearc=#5,fillstyle=solid,fillcolor=white](#1,#2)(#3,#4)}
}

% et top 
\newcommand {\erettop}[4]
{
%\psframe[linestyle=none,linearc=2pt,cornersize=absolute,fillstyle=solid,fillcolor=lightyellow](#1,#2)(#3,#4)
\psline[linearc=2pt,fillstyle=none,fillcolor=lightyellow](#1,#4)(#1,#2)(#3,#2)(#3,#4)
}

% et bottom 
\newcommand {\eretbtm}[4]
{
%\psframe[linestyle=none,linearc=2pt,cornersize=absolute,fillstyle=solid,fillcolor=lightyellow](#1,#2)(#3,#4)
\psline[linearc=2pt,fillstyle=none,fillcolor=lightyellow](#1,#2)(#1,#4)(#3,#4)(#3,#2)
}

% et bottom left
\newcommand {\eretbl}[4]
{
%\psframe[linestyle=none,linearc=2pt,cornersize=absolute,fillstyle=solid,fillcolor=lightyellow](#1,#2)(#3,#4)
\psline[linearc=2pt,fillstyle=none,fillcolor=lightyellow](#1,#4)(#3,#4)(#3,#2)
}

% et middle left
\newcommand {\eretml}[4]
{
%\psframe[linestyle=none,linearc=2pt,cornersize=absolute,fillstyle=solid,fillcolor=lightyellow](#1,#2)(#3,#4)
\psline[linearc=2pt,fillstyle=none,fillcolor=lightyellow](#1,#2)(#3,#2)(#3,#4)(#1,#4)
}

% et top left
\newcommand {\erettl}[4]
{
%\psframe[linestyle=none,linearc=2pt,cornersize=absolute,fillstyle=solid,fillcolor=lightyellow](#1,#2)(#3,#4)
\psline[linearc=2pt,fillstyle=none,fillcolor=lightyellow](#1,#2)(#3,#2)(#3,#4)
}

% et bottom right
\newcommand {\eretbr}[4]
{
%\psframe[linestyle=none,linearc=2pt,cornersize=absolute,fillstyle=solid,fillcolor=lightyellow](#1,#2)(#3,#4)
\psline[linearc=2pt,fillstyle=none,fillcolor=lightyellow](#1,#2)(#1,#4)(#3,#4)
}

% et middle right
\newcommand {\eretmr}[4]
{
%\psframe[linestyle=none,linearc=2pt,cornersize=absolute,fillstyle=solid,fillcolor=lightyellow](#1,#2)(#3,#4)
\psline[linearc=2pt,fillstyle=none,fillcolor=lightyellow](#3,#4)(#1,#4)(#1,#2)(#3,#2)
}

% et top right
\newcommand {\erettr}[4]
{
%\psframe[linestyle=none,linearc=2pt,cornersize=absolute,fillstyle=solid,fillcolor=lightyellow](#1,#2)(#3,#4)
\psline[linearc=2pt,fillstyle=none,fillcolor=lightyellow](#1,#4)(#1,#2)(#3,#2)
}

% \ergrp{#1 x0} {#2 y0} {#3 x1} {#4 y1} {#5 corner radius} {#6 fill}
% #5 corner radius is unused!
\newcommand {\ergrp}[6]
{ 
\ifthenelse{\equal{#6}{1}}
{\psframe[fillstyle=solid,fillcolor=lightgray](#1,#2)(#3,#4)}
{\psframe[fillstyle=solid,fillcolor=white](#1,#2)(#3,#4)}
}

% \eretname {#1 x left of text} {#2 y top of text} {#3 text}
\newcommand {\eretname}[3]
{
%shift down 0.1 for height of text the anchor at baseline (B)
\rput[bl]{0}(0,-0.1){\rput[Bl]{0}(#1,#2){\footnotesize \textit{#3}}}
}

% \errelarm {#1 x0} {#2 y0} {#3 x1} {#4 y1} {#5 ismandatory} {#6 isconstructed}
\newcommand {\errelarm}[6]
{
\ifthenelse{\equal{#6}{1}}
{
%%\psline[linewidth=0.5pt,linearc=.05,linestyle=dashed,dash=6pt 6pt]{-}(#1,#2)(#3,#4)}
\ifthenelse{\equal{#5}{1}}
{\psline[linewidth=1.5pt,linearc=.05,linecolor=lightgray]{-}(#1,#2)(#3,#4)}
{\psline[linewidth=1.5pt,linearc=.05,linecolor=lightgray,linestyle=dashed,dash=2pt 2pt]{-}(#1,#2)(#3,#4)}
}
{
\ifthenelse{\equal{#5}{1}}
{\psline[linewidth=0.9pt,linearc=.05]{-}(#1,#2)(#3,#4)}
{\psline[linewidth=0.9pt,linearc=.05,linestyle=dashed,dash=2pt 2pt]{-}(#1,#2)(#3,#4)}
}
}

% \errelangle {#1 x0} {#2 y0} {#3 x1} {#4 y1} {#5 x2} {#6 y2} {#7 ismandatory} {#8 isocnstructed}
\newcommand {\errelangle}[8]
{
\ifthenelse{\equal{#8}{1}}
{
%\psline[linewidth=0.5pt,linearc=.1,linestyle=dashed,dash=6pt 6pt]{-}(#1,#2)(#3,#4)(#5,#6)}
\ifthenelse{\equal{#7}{1}}
{\psline[linewidth=1.5pt,linearc=.05,linecolor=lightgray]{-}(#1,#2)(#3,#4)(#5,#6)}
{\psline[linewidth=1.5pt,linearc=.1,linecolor=lightgray,linestyle=dashed,dash=2pt 2pt]{-}(#1,#2)(#3,#4)(#5,#6)}
}
{
\ifthenelse{\equal{#7}{1}}
{\psline[linewidth=0.9pt,linearc=.1]{-}(#1,#2)(#3,#4)(#5,#6)}
{\psline[linewidth=0.9pt,linearc=.1,linestyle=dashed,dash=2pt 2pt]{-}(#1,#2)(#3,#4)(#5,#6)}
}
}

% \ercrowfoot {#1 x0} {#2 y0} {#3 x11} {#4 y11} {#5 x12} {#6 y12} {#7 x13} {#8 y13} {#9 isconstructed}
\newcommand {\ercrowfoot}[9]
{
\ifthenelse{\equal{#9}{1}}
{
\psline[linewidth=1.5pt,linearc=.05,linecolor=lightgray]{-}(#1,#2)(#3,#4)
\psline[linewidth=1.5pt,linearc=.05,linecolor=lightgray]{-}(#1,#2)(#5,#6)
\psline[linewidth=1.5pt,linearc=.05,linecolor=lightgray]{-}(#1,#2)(#7,#8)
}{
\psline[linewidth=0.9pt,linearc=.05]{-}(#1,#2)(#3,#4)
\psline[linewidth=0.9pt,linearc=.05]{-}(#1,#2)(#5,#6)
\psline[linewidth=0.9pt,linearc=.05]{-}(#1,#2)(#7,#8)
}
}


% \eridcomprel{#1 x1}{#2 x2}{#3 y1}{#4 ymid}{#5 y2}
\newcommand {\eridcomprel}[5]
{
\psline[linewidth=0.9pt](#1,#3)(#1,#5)
\psline[linewidth=0.9pt](#2,#3)(#2,#5)
\psline[linewidth=0.9pt](#1,#4)(#2,#4)
}

% \eridrefrel{#1 x1}{#2 xmid}{#3 x2}{#4 y1}{#5 y2}
\newcommand {\eridrefrel}[5]
{
\psline[linewidth=0.9pt](#1,#4)(#3,#4)
\psline[linewidth=0.9pt](#1,#5)(#3,#5)
\psline[linewidth=0.9pt](#2,#4)(#2,#5)
}


% \errelname {#1 x} {#2 y} {#3 text}
\newcommand {\errelname}[3]
{
\rput[l]{0}(#1,#2){\textit{#3}}
}
% \errelseq {#1 x} {#2 y}
\newcommand {\erelseq}[2]
{
}
% \erattr {#1 x} {#2 y} {#3 ismandatory}{#4 idenitfying} {#5 text}
\newcommand {\erattr}[5]
{
\ifthenelse{\equal{#3}{1}}
{\rput[l]{0}(#1,#2){{\tiny $\square$} {\footnotesize \textit{\ifthenelse{\equal{#4}{0}}{\underline{#5}}{#5}}}}}
{\rput[l]{0}(#1,#2){\footnotesize $\circ$ \textit{\ifthenelse{\equal{#4}{0}}{\underline{#5}}{#5}}}}
}

%\ifthenelse{\equal{#4}{1}}
% \ertext {#1 x} {#2 y} {#3 text anchor} {#4 text}
%{\rput[l]{0}(#1,#2){\footnotesize $\circ$ \underline{\textit{#5}}}}
\newcommand {\ertext}[4]
{
\rput[B#3]{0}(#1,#2){{\footnotesize #4}}
}
% \erarc {#1 x0} {#2 y0} {#3 x1} {#4 y1} {#5 x2} {#6 y2} {#7 x3} {#8 y3}
\newcommand {\erarc}[8]
{
\psbezier[showpoints=false]{-}(#1,#2) (#3, #4)(#5,#6) (#7, #8)
}

% \erarc {#1 x0} {#2 y0} {#3 x1} {#4 y1} {#5 x2} {#6 y2} {#7 x3} {#8 y3}
\newcommand {\errelseq}[8]
{
\psbezier[showpoints=false]{-}(#1,#2) (#3, #4)(#5,#6) (#7, #8)
}
% \ertrace {#1 trace}   
\newcommand {\ertrace}[1]
{
}
    %beamer aware version
% 
% general.macros.v2
%
% Rename macros that conflict with beamer 31 Aug 2022
% Don't assume an index                   31 Aug 2022

\usepackage{changepage} % used for adjustwidth

\iffalse % 31 Aug 2022
\usepackage{imakeidx}
\usepackage{framed}
\makeindex[name=definitions, title=Index of Definitions]
\makeindex[name=lemmas, title=Index of Lemmas]
\fi

\definecolor{highlight}{cmyk}{0,0,0.7,0}
\newcommand{\commentary}[1]{\marginpar{\footnotesize #1}}
\newcommand{\highlight}[1]{\colorbox{highlight}{#1}}
\newcommand{\whitelight}[1]{\colorbox{white}{#1}}
\newcommand{\term}[1]{\textit{#1}\commentary{\colorbox{lightgray}{\textit{#1}}}\index[definitions]{#1}}
\newcommand{\llabel}[1]{\label{#1}\commentary{\colorbox{pink}{\scriptsize{#1}}}\index[lemmas]{#1}}
\newcommand{\lref}[1]{\ref{#1}\colorbox{pink}{\scriptsize{#1}}\index[lemmas]{#1!use of}}

\newcommand{\daynote}[1]{\commentary{See day notes #1.}}

\newcommand{\newt}[1]{\colorbox{yellow}{#1}}
\newenvironment{newtt}
{  \colorbox{yellow}{$[$ ...} 
}
{  \colorbox{yellow}{... $]$}
}
\newcommand{\oldt}[1]{\colorbox{yellow}{\sout{#1}}}
\newenvironment{oldtt}
{  \colorbox{red}{$[$ ...} 
}
{  \colorbox{red}{... $]$}
}

\newcommand{\reinstatet}[1]{\colorbox{lime}{#1}}
\newenvironment{reinstatett}
{  \colorbox{lime}{$[$ ...}
}
{  \colorbox{lime}{... $]$}
}

\newcommand{\tbd}{\highlight{TBD}}

%ithprojection function
\newcommand{\proji}[1]{\pi_#1}


\newenvironment{aside}
{\begin{framed}
\textbf{Aside}
}
{
\end{framed}
}

\newenvironment{notebox}[1][Note]
{\begin{framed}
\textbf{#1}
}
{
\end{framed}
}

\newenvironment{categoricalaside}
{\begin{framed}
\textbf{Categorical Aside}
}
{
\end{framed}
}

\newenvironment{noteforfuture}
{\begin{framed}
\textbf{Note For Future}
}
{
\end{framed}
}

\newenvironment{myproblem}       %31 Aug 2022
{\begin{framed}
\textbf{Problem}
}
{
\end{framed}
}

\newenvironment{key}
{
\begin{tabular}{c l p{4cm}}
KEY && \\
\hline
}
{
\end{tabular}
}

%  31 Aug 2022
\NewEnviron{tightquote} %italic text indented left and right hand side
{\begin{adjustwidth}{1.5cm}{1.5cm}
\textit{
\BODY
}
\end{adjustwidth}
}

\newcommand{\keyentry}[3]{#1 & #2 & #3 \\} 


%quine quote
\newcommand{\qq}[1]{
\left\ulcorner#1\right\urcorner
}

%single quote
\newcommand{\sq}[1]{
\textnormal{\textquotesingle}#1\textnormal{\textquotesingle}
}

%lower quine quote
\newcommand{\lqq}[1]{
\left\llcorner #1\right\lrcorner
}


%from berkley
\newcommand{\langl}{\begin{picture}(4.5,7)
\put(1.1,2.5){\rotatebox{60}{\line(1,0){5.5}}}
\put(1.1,2.5){\rotatebox{300}{\line(1,0){5.5}}}
\end{picture}}
\newcommand{\rangl}{\begin{picture}(4.5,7)
\put(.9,2.5){\rotatebox{120}{\line(1,0){5.5}}}
\put(.9,2.5){\rotatebox{240}{\line(1,0){5.5}}}
\end{picture}}
\newcommand{\lang}{\begin{picture}(5,7)\put(1.1,2.5){\rotatebox{45}{\line(1,0){6.0}}}\put(1.1,2.5){\rotatebox{315}{\line(1,0){6.0}}}\end{picture}}
\newcommand{\rang}{\begin{picture}(5,7)\put(.1,2.5){\rotatebox{135}{\line(1,0){6.0}}}\put(.1,2.5){\rotatebox{225}{\line(1,0){6.0}}}\end{picture}}
%Try sharper tuple brackets -- except gives errors nested in captions so comment out
%\renewcommand{\tuple}[1]{\lang #1 \rang}

\newcommand{\setsuchthat}[2]{\left\{#1 \ \middle|\ #2\right\}}
\newcommand{\set}[1]{\left\{#1\right\}} 

% one to n - wanton
\newcommand{\wanton}[1]{#1_1,...#1_n}
\newcommand{\n}{1...n}
\newcommand{\fn}{\wanton{f}}
\newcommand{\gn}{\wanton{g}}
\newcommand{\pn}{\wanton{p}}
\newcommand{\qn}{\wanton{q}}
\newcommand{\qnprime}{\wanton{q'}}
\newcommand{\tn}{\wanton{t}}
\newcommand{\xn}{\wanton{x}}
\newcommand{\xnp}{\wanton{x'}}
\newcommand{\yn}{\wanton{y}}
\newcommand{\An}{\wanton{A}}
\newcommand{\Bn}{\wanton{B}}
\newcommand{\Cn}{\wanton{C}}
\newcommand{\ntuple}[1]{\tuple{\wanton{#1}}}
\newcommand{\wantom}[2][]{#2_1,...#2_{m#1}}
\newcommand{\m}{1...m}
\newcommand{\mtuple}[1]{\tuple{#1_1,...#1_m}}
\newcommand{\gm}{\wantom{g}}
\newcommand{\qm}{\wantom{q}}
\newcommand{\sm}[1][]{\wantom[#1]{s}}
\newcommand{\smp}{\wantom{s'}}
\newcommand{\ym}{\wantom{y}}
\newcommand{\Bm}{\wantom{B}}
\newcommand {\bntuple}{\ensuremath{\ntuple{b}}}
\newcommand {\fntuple}{\ensuremath{\ntuple{f}}}
\newcommand {\fnptuple}{\ensuremath{\ntuple{f}}}
\newcommand {\pntuple}{\ensuremath{\ntuple{p}}}
\newcommand {\qntuple}{\ensuremath{\ntuple{q}}}
\newcommand {\qnptuple}{\ensuremath{\ntuple{q'}}}
\newcommand {\qmtuple}{\ensuremath{\mtuple{q}}}
\newcommand {\sntuple}{\ensuremath{\ntuple{s}}}
\newcommand {\xntuple}{\ensuremath{\ntuple{x}}}
\newcommand {\xnptuple}{\ensuremath{\ntuple{x'}}}
\newcommand {\ymtuple}{\ensuremath{\mtuple{y}}}
\newcommand{\idef}[1][n]{1 \leq i \leq #1}
\newcommand{\jdef}[1][m]{1 \leq j \leq #1}
\newcommand{\kdef}[1][l]{1 \leq k \leq #1}
\newcommand{\foreachi}[1][n]{for each $i$, $1 \leq i \leq #1$}
\newcommand{\foreachj}[1][m]{for each $j$, $1 \leq j \leq #1$}
\newcommand{\foreachk}[1][l]{for each $k$, $1 \leq k \leq #1$}
\newcommand{\Foreachi}[1][n]{For each $i$, $1 \leq i \leq #1$}
\newcommand{\Foreachj}[1][m]{For each $j$, $1 \leq j \leq #1$}
\newcommand{\Foreachk}[1][l]{For each $k$, $1 \leq k \leq #1$}
\newcommand{\forsomei}[1][n]{for some $i$, $1 \leq i \leq #1$}
\newcommand{\forsomej}[1][m]{for some $j$, $1 \leq j \leq #1$}
\newcommand{\forsomek}[1][l]{for some $k$, $1 \leq k \leq #1$}
\newcommand{\wherei}[1][n]{where $1 \leq i \leq #1$}
\newcommand{\wherej}[1][m]{where $1 \leq j \leq #1$}
\newcommand{\wherek}[1][l]{where $1 \leq k \leq #1$}


\newcommand{\fundep}[3]{#2 \xrightarrow{#1} #3}  %where does this belong? xxxx
% Following used for notes -- indented numbered paras

\newcounter{para}
\newlength{\oldparindent}
\setlength{\oldparindent}{\parindent} % Save \parindent before of change
\newcommand{\ind}{\hspace*{\oldparindent}}

\newcommand\mynote{                                                 % renamed 31 Aug 2022
%\setlength{\parskip}{0.5\baselineskip} % Definition of `parskip`
\setlength{\parindent}{0pt}
\par\ind\refstepcounter{para}\thepara.\space
\setlength{\parindent}{\oldparindent}
}


 %try this instead
\newcommand{\ncarrNEGZZ}[3][0]{\ncarc[arcangle=#1,nodesepA=2pt,nodesepB=2pt,offsetA=-2pt,offsetB=-2pt,arrowsize=5pt,arrowinset=0.7]{->}{#2}{#3}}
\newcommand{\ncarrZ}[3][0]{\ncarc[arcangle=#1,nodesepA=2pt,nodesepB=2pt,offsetA=0pt,offsetB=0pt,arrowsize=5pt,arrowinset=0.7]{->}{#2}{#3}}
\newcommand{\ncarrZZ}[3][0]{\ncarc[arcangle=#1,nodesepA=2pt,nodesepB=2pt,offsetA=2pt,offsetB=2pt,arrowsize=5pt,arrowinset=0.7]{->}{#2}{#3}}
\newcommand{\ncarrZZZ}[3][0]{\ncarc[arcangle=#1,nodesepA=2pt,nodesepB=2pt,offsetA=4pt,offsetB=4pt,arrowsize=5pt,arrowinset=0.7]{->}{#2}{#3}}
\newcommand{\ncarrZZZZ}[3][0]{\ncarc[arcangle=#1,nodesepA=2pt,nodesepB=2pt,offsetA=6pt,offsetB=6pt,arrowsize=5pt,arrowinset=0.7]{->}{#2}{#3}}


\newcommand{\ccsquareoutline}[6]
{\begin{array}{c p{#1} c}
\Rnode{TL}{#3}  &\ &  \Rnode{TR}{#4}\\[#2]
\Rnode{BL}{#5}  &\ &  \Rnode{BR}{#6}
\end{array}
}
\newcommand{\ccsquareacross}[2]
{\mbox{\ncarr{TL}{TR}
\alabel{#1}
\ncarr{BL}{BR}
\blabel{#2}}
}
\newcommand{\ccsquaredown}[2]
{\mbox{\ncsar{TL}{BL}
\blabel{#1}
\ncsar{TR}{BR}
\alabel{#2}}
}

\newcommand{\ccsquareup}[2]
{\mbox{\ncsar{BL}{TL}
\blabel{#1}
\ncsar{BR}{TR}
\alabel{#2}}
}

\newcommand{\ccsquareanddroppers}[6]
{\ccsquareoutline{#1}{#2}{#3}{#4}{#5}{#6}
\ccsquaredown{p_{#3}}{p_{#4}}
}
\newcommand{\ccsquareprimitivedroppers}
{
\begin{arrows}
\ncsar{TL}{BL}
\ncsar{TR}{BR}
\end{arrows}
}

% ccprimitivepullbacksquare{width}{height}{TR}{BL}{BR}{f}
\newcommand{\ccprimitivepullbacksquare}[6]
{\ccsquareoutline{#1}{#2}{#6^*#3}{#3}{#4}{#5}
\ccsquareprimitivedroppers
\ccsquareacross{q(#6,#3)}{#6}
}

 
\NewEnviron{coneoutline}[3]{% width depth source
\begin{array} {c p{#1} c}
                      \\[0.5cm]
\Rnode{OTL}{#3} &\ &  \\[#2]
&&
\BODY
\end{array}
}

\newcommand{\conearrows}[3]
{\begin{arrows}
\ncarr[-15]{OTL}{BL}
\blabel{#1}
\ncarr[20]{OTL}{TR}
\alabel{#2}
\ncline{OTL}{TL}  % want \ncdarr for dashed I think
\trput[tpos=0.2,labelsep=6pt]{\footnotesize{$#3$}}
%\salabel{sa label #3}[0.5]
\end{arrows}
}

\newcommand{\conearrowsdefiningtuple}[2]
{\conearrows{#1}{#2}{\tuple{#1,#2}}
}

%\scopeTriangle{subject}{domain}{codomain}{apex}{diagonal}{riser}
\newcommand{\scopeTriangle}[6]
{
 \begin{array}{c  c  c}
                & \Rnode{apex}{#4} &                \\[1cm]
\Rnode{dom}{#2} &                  & \Rnode{cod}{#3} 
\end{array}
\begin{arrows}
\ncarr{dom}{cod}
\blabel{#1}
\ncsar{dom}{apex}
\alabel{#5}[0.2]
\ncsar{cod}{apex}
\blabel{#6}[0.2]
\end{arrows} 
}


% \composeSevenShaped[nodesize]{A}{B}{C}{f}{g}{h}
\newcommand{\composeSevenShaped}[7][1cm]
{\begin{array}{c c c}
\makebox[#1][c]{\nudgeup{0.5cm}\Rnode{A}{#2}} && \makebox[#1][c]{\Rnode{B}{#3}} \\[0.5cm]
           &    \makebox[#1][c]{\Rnode{C}{#4}}    &
\end{array}
\begin{arrows}
\ncarr{A}{B}
\alabel{#5}
\ncarr{B}{C}
\alabel{#6}[0.2][0.05]
\ncarr{A}{C}
\blabel{#7}[0.2][0.05]
\end{arrows}
}



%
% beamermacros
%

%re-impmentation of highlight for beamer
%\newcommand<>\highlightbox[2]{%
%  \alt#3{\makebox[\dimexpr\width-2\fboxsep]{\colorbox{#1}{#2}}}{#2}%
%}
%By copying above from internet
\definecolor{highlightcolor}{cmyk}{0,0,0.7,0}
\newcommand<>{\highlight}[1]{%
  \alt#2{\makebox[\dimexpr\width-2\fboxsep]{\colorbox{highlightcolor}{#1}}}{#1}%
}



%%All these macros are copied from SharedMacros/general.tex which doesnt seem to work with beamer
% Some macros in SharedMacros/general.tex thought to have name clashes with beamer.


\newcommand{\fundep}[3]{#2 \xrightarrow{#1} #3}                                                 
\newcommand{\term}[1]{\textit{#1}} 
\newcommand{\setsuchthat}[2]{\left\{#1 \ \middle|\ #2\right\}}
\newcommand{\set}[1]{\left\{#1\right\}}



\newcommand{\wanton}[1]{#1_1,...#1_n}
\newcommand{\ntuple}[1]{\tuple{\wanton{#1}}}

\newcommand{\xntuple}{\ensuremath{\ntuple{x}}}

% maybe not in general.macros 
\newcommand{\xnset}{\ensuremath{\set{\wanton{x}}}} %get rid of this
%
% othermacros
%

% copied and edited from \idcomp to make stronger linestyle
\newcommand{\addedgebar}{
\ncput[npos=0, nrot=:U]{\psline[linewidth=1.25pt](0.2,-0.1)(0.2,0.1)}
}
\newcommand{\addedgedoublebar}{
\ncput[npos=0, nrot=:U]{\psline[linewidth=1.25pt](0.2,-0.1)(0.2,0.1)}
\ncput[npos=0, nrot=:U]{\psline[linewidth=1.25pt](0.3,-0.1)(0.3,0.1)}
}
\newcommand{\addedgetriplebar}{
\ncput[npos=0, nrot=:U]{\psline[linewidth=1.25pt](0.2,-0.1)(0.2,0.1)}
\ncput[npos=0, nrot=:U]{\psline[linewidth=1.25pt](0.3,-0.1)(0.3,0.1)}
\ncput[npos=0, nrot=:U]{\psline[linewidth=1.25pt](0.4,-0.1)(0.4,0.1)}
}

%\newcommand{\addedgebar}{\ifbars{\ncput[npos=0, nrot=:U]{\psline(0.2,-0.075)(0.2,0.075)}}\fi}

%copied from database literature review
\newcommand{\displaybibentry}[1]
{\begin{framed}
\bibentry{#1}
\end{framed}
}

% used in data tables
\newcommand{\colhead}[1]{\textbf{\textcolor{white}{#1}}}
\definecolor{myblue}{RGB}{71,71,186}
\newcommand{\largeAsterisk}{\mathop{\scalebox{1.5}{\raisebox{-0.2ex}{$\ast$}}}}
\newcommand{\fk}[1]{#1$^{\largeAsterisk}$}
\newcommand{\pk}[1]{\underline{#1}}
\newcommand{\seck}[1]{\dashuline{#1}}  % secondary key
\newcommand{\pkfk}[1]{\underline{#1}$^{\largeAsterisk}$} % primary key that is a foreign key
% \vpad gives vertical padding in a tabular
\newcommand{\vpad}[1]{\multicolumn{#1}{c}{}\\[-0.25cm]}
% used in slides
\newcommand{\outerbullet}{{$\color{blue}{\blacktriangleright}$}\ }% please dont remove final space
\newcommand{\innerbullet}{{\footnotesize $\color{blue}{\blacktriangleright}$}\ }% please dont remove final space
\newcommand{\braceLabel}[3]{\psbrace[ref=lC,braceWidth=1pt,braceWidthInner=3pt,braceWidthOuter=3pt](#2)(#1){#3} }

% words words words
\newcommand{\catMEterm}{category with designated monomorphisms and epimorphisms\ }
\newcommand{\IfSforCwithRCwords}{
If $S$ is a sketch for category \catcw considered as a data specification with requirement $\reqtc$\ }
\newcommand{\IfSforCwithRCwordsvariant}{
If $S$ is a sketch for structured category \catcw and if $S$ is considered as a data specification with requirement $\reqtc$\ }
\newcommand{\IfSforepimonoCwithRCwords}{
If $S$ is a sketch for a category \catcw with designated monomorphisms and epimorphisms considered as a data specification with requirement $\reqtc$\ }
\iffalse
\newcommand{\scmonosketchwording}{
If $S$ is a sketch for such a category
%of a category with finite products and designated monomorphisms and epimorphisms
considered as a data specification
with requirement $\reqtc$\ }
\fi
\newcommand{\spacechar}{\ }
\newcommand{\thirdstructure}{designated monomorphisms and epimorphisms and with finite products}
\newcommand{\IfSforproductepimonoCwithRCwords}{
If $S$ is a sketch for  a category with \thirdstructure \spacechar
%category \catcw with finite products and designated monomorphisms and epimorphisms 
considered as a data specification with requirement $\reqtc$\ }

\newcommand{\goodnesscriteria}[1]{\textbf{Goodness Criteria #1:}}

\newcommand{\goodnessoneA}{
\goodnesscriteria{1A} There ought not to be an edge $e$ in $G$ for which there is an equivalent path $p$ which  does not containing $e$
}
\newcommand{\goodnessoneB}{
\goodnesscriteria{1B} 
There ought not exist $d \in PE$ such that $d \in \overline{PE \setminus d}$
}

\newcommand{\goodnessoneC}{
\goodnesscriteria{1C} \\
There ought not exist $m \in M$ such that $m \in \overline{M \setminus m}$
}
\newcommand{\goodnessoneD}{
\goodnesscriteria{1D} \\
There ought not exist $e \in E$ such that $e \in \overline{E \setminus e}$
}




% From the Mathematical Theory of data paper
\newcommand{\ssfd}[2]{\ensuremath{#1 \morph #2}}  % singleton-singleton
\newcommand{\smfd}[2]{\ensuremath{\ssfd{#1}{\set{#2}}}}  % singleton-many
\newcommand{\msfd}[2]{\ensuremath{\ssfd{\set{#1}}{#2}}}  % many-singleton
\newcommand{\mmfd}[2]{\ensuremath{\msfd{#1}{\set{#2}}}}  % many-many



% All these should find a home in SharedMacros eventually 

% Commands for making a bit of vertical space. used when arrows and particularly labels of arrows
% use spec that is otherwise accounted for.
\newcommand{\seeroomup}[1]{\rule{0.1cm}{#1}}
\newcommand{\seeroomdown}[1]{\rule[-#1]{0.1cm}{0.1cm}}
\newcommand{\roomup}[1]{\rule{0cm}{#1}}
\newcommand{\roomdown}[1]{\rule[-#1]{0cm}{0.1cm}}


% BOX DIAGRAMS
\newcommand{\attr}[1]{#1}
\renewcommand{\attr}[1]{\psframebox[linecolor=red,framearc=.1]{#1}}
\newcommand{\attrtype}[1]{#1}
\renewcommand{\attrtype}[1]{\psframebox[linecolor=blue,framearc=.1]{#1}}
\newcommand{\etype}[1]{#1}
\renewcommand{\etype}[1]{\psframebox[linecolor=red,framearc=.1]{#1}}


\newcommand{\regularizetextheight}{\roomup{0.3cm}\roomdown{0.1cm}}

\newcommand{\unarystructurediagramnodes}[3][]{
\rput[tc](2.4,3){\Rnode{#1A}{\psframebox[framesep=10pt]{\regularizetextheight#2}}} 
\rput[tc](2.4,1){\rnode{#1B}{\psframebox[framesep=10pt]{\regularizetextheight#3}}}           
}

\newcommand{\binarystructurediagramnodes}[4][]{
\rput[tr](4.0,3){\Rnode{#1A}{\psframebox[framesep=10pt]{\regularizetextheight#2}}} 
\rput[tr](2.4,1){\rnode{#1B}{\psframebox[framesep=10pt]{\regularizetextheight#3}}}     
\rput[tr](5.6,1){\rnode{#1C}{\psframebox[framesep=10pt]{\regularizetextheight#4}}}        
}

\newcommand{\triplestructurediagramnodes}[5][]{ 
\rput[tc](2.0,3){\rnode{#1A}{\psframebox[framesep=10pt]{\regularizetextheight#2}}}     
\rput[tc](-1.05,1){\rnode{#1B}{\psframebox[framesep=10pt]{\regularizetextheight#3}}}  
\rput[tc](2.0,1){\rnode{#1C}{\psframebox[framesep=10pt]{\regularizetextheight#4}}}   
\rput[tc](5.0,1){\rnode{#1D}{\psframebox[framesep=10pt]{\regularizetextheight#5}}}   
}

\newcommand{\jacksonbinarydiagram}[3]
{
\pspicture(-0.4,0)(5.7,3)  % lower left is 0,0 upper right is 8,3
%\psgrid
\binarystructurediagramnodes{#1}{#2}{#3}
\rput[tr](2.3,0.9){*}
\rput[tr](5.4,0.9){*}
\ncangle[offsetA=-0.5cm, angleA=-90,angleB=90,armB=0.5cm]{A}{B}
\ncangle[offsetA=0.5cm, angleA=-90,angleB=90,armB=0.5cm]{A}{C}
\endpspicture      
}

\newcommand{\bachmanbinarydiagram}[4][]
{
\pspicture(-0.4,0)(5.7,3)  % lower left is 0,0 upper right is 8,3
%\psgrid
\binarystructurediagramnodes[#1]{#2}{#3}{#4}
\ncline[linewidth=3pt]{->}{#1A}{#1B}
\ncline[linewidth=3pt]{->}{#1A}{#1C}
\endpspicture      
}

\newcommand{\unarystructurediagram}[3][]
{
\pspicture(0.9,0)(3.9,3.5)  
%\psgrid
\unarystructurediagramnodes[#1]{#2}{#3}
\endpspicture      
}

\newcommand{\binarystructurediagram}[4][]
{
\pspicture(-0.4,0)(5.7,3)  
%\psgrid
\binarystructurediagramnodes[#1]{#2}{#3}{#4}
\endpspicture      
}

\newcommand{\triplestructurediagram}[5][]
{
\pspicture(-2.5,0)(6.4,3.5)  
%\psgrid
\triplestructurediagramnodes[#1]{#2}{#3}{#4}{#5}
\endpspicture      
}


\newcommand{\binarynetworkdiagramnodes}[3]{ 
\rput[tr](2.4,3){\rnode{A}{\psframebox[framesep=10pt]{\regularizetextheight#1}}}     
\rput[tr](5.6,3){\rnode{B}{\psframebox[framesep=10pt]{\regularizetextheight#2}}} 
\rput[tr](4.0,1){\Rnode{C}{\psframebox[framesep=10pt]{\regularizetextheight#3}}}       
}

\newcommand{\bachmannetworkdiagram}[3]
{
\pspicture(-0.4,0)(5.7,3)  % lower left is 0,0 upper right is 8,3
%\psgrid
\binarynetworkdiagramnodes{#1}{#2}{#3}
\ncline[linewidth=3pt]{->}{A}{C}
\ncline[linewidth=3pt]{->}{B}{C}
\endpspicture      
}

%craft bachman nwtrok share diagram
\newcommand{\doublebinarynetworkdiagramnodes}[6]{ 
\rput[tc](-2.5,3){\rnode{A}{\psframebox[framesep=10pt]{\regularizetextheight#1}}}  
\rput[tc](2.0,3){\rnode{B}{\psframebox[framesep=10pt]{\regularizetextheight#2}}}    
\rput[tc](-4.1,1){\rnode{C}{\psframebox[framesep=10pt]{\regularizetextheight#3}}} 
\rput[tc](-1.05,1){\rnode{D}{\psframebox[framesep=10pt]{\regularizetextheight#4}}}  
\rput[tc](2.0,1){\rnode{E}{\psframebox[framesep=10pt]{\regularizetextheight#5}}}   
\rput[tc](5.0,1){\rnode{F}{\psframebox[framesep=10pt]{\regularizetextheight#6}}}   
}

\newcommand{\doublebachmannetworkdiagram}[6]
{
\pspicture(-5.6,0)(6.5,3)  % lower left is 0,0 upper right is 8,3
%\psgrid
\doublebinarynetworkdiagramnodes{#1}{#2}{#3}{#4}{#5}{#6}
\ncline[linewidth=3pt]{->}{A}{C}
\ncline[linewidth=3pt]{->}{A}{D}
\ncline[linewidth=3pt]{->}{B}{D}
\ncline[linewidth=3pt]{->}{B}{E}
\ncline[linewidth=3pt]{->}{B}{F}
\endpspicture      
}

\newcommand{\doublecategorynetworkdiagram}[6]
{
\pspicture(-5.6,0)(6.5,3)  % lower left is 0,0 upper right is 8,3
%\psgrid
\doublebinarynetworkdiagramnodes{#1}{#2}{#3}{#4}{#5}{#6}
\ncarr{C}{A}
\ncarr{D}{A}
\ncarr{D}{B}
\ncarr{E}{B}
\ncarr{F}{B}
\endpspicture      
}

\newcommand{\mixedcategorynetworkdiagram}[6]
{
\pspicture(-5.6,0)(6.5,3)  % lower left is 0,0 upper right is 8,3
%\psgrid
\doublebinarynetworkdiagramnodes{#1}{#2}{#3}{#4}{#5}{#6}
\ncline[linewidth=2.5pt]{->}{C}{A}
\ncline[linewidth=2.5pt]{->}{D}{A}
\ncarr{D}{B}
\ncline[linewidth=2.5pt]{->}{E}{B}
\ncline[linewidth=2.5pt]{->}{F}{B}
\endpspicture      
}

\newcommand{\contextualcategoryblockstyleexamplekernel}[6]{
\pspicture(-5.6,0)(6.5,3)  % lower left is 0,0 upper right is 8,3
%\psgrid
\doublebinarynetworkdiagramnodes{#1}{#2}{#3}{#4}{#5}{#6}
\ncsar{C}{A}
\ncsar{E}{B}
\ncsar{F}{B}
\endpspicture
}

\newcommand{\contextualcategorynetworkdiagram}[6]
{
\contextualcategoryblockstyleexamplekernel{#1}{#2}{#3}{#4}{#5}{#6}
\ncsar{D}{A}
\ncarr{D}{B}
}

\newcommand{\contextualcategorynetworkdiagramreorganised}[6]
{
\contextualcategoryblockstyleexamplekernel{#1}{#2}{#3}{#4}{#5}{#6}
\ncarr{D}{A}
\ncsar{D}{B}
}


\newcommand{\contextualcategorynetworkdiagramtopologised}[6]
{
\begin{tabular}{c c c}
\scalebox{0.9}{\binarystructurediagram[left]{compound\kern0.1cm}{alias \kern1.2cm}{occurence}}
&&
\scalebox{0.9}{\binarystructurediagram[right]{element\kern0.4cm}{valency \kern0.8cm}{allotrope\kern0.3cm}}
\end{tabular}
\ncangle[offsetA=0.15cm, angleA=0,offsetB=-0.25cm, angleB=180, armB=2.5cm]{->}{leftC}{rightA}
\ncsar{leftB}{leftA}
\ncsar{leftC}{leftA}
\ncsar{rightB}{rightA}
\ncsar{rightC}{rightA}
}

\newcommand{\contextualcategorynetworkdiagramreorganisedtopologised}[6]
{
\begin{tabular}{c c c}
\scalebox{0.9}{\unarystructurediagram[left]{compound\kern0.1cm}{alias \kern1.2cm}}
&&
\scalebox{0.9}{\triplestructurediagram[right]{element\kern0.4cm}{occurence}{valency \kern0.8cm}{allotrope\kern0.3cm}}
\end{tabular}
\ncangle[offsetA=0.15cm, angleA=180,offsetB=-0.25cm, angleB=0, armB=0.9cm]{->}{rightB}{leftA}
\ncsar{leftB}{leftA}
\ncsar{leftC}{leftA}
\ncsar{rightB}{rightA}
\ncsar{rightC}{rightA}
\ncsar{rightD}{rightA}
}

\iffalse
\newcommand{\contextualcategorynetworkdiagramreorganised}[6]
{
\pspicture(-5.6,0)(6.5,3)  % lower left is 0,0 upper right is 8,3
%\psgrid
\doublebinarynetworkdiagramnodes{#1}{#2}{#3}{#4}{#5}{#6}
\ncsar{C}{A}
\ncarr{D}{A}
\ncsar{D}{B} 
\ncsar{E}{B}
\ncsar{F}{B}
\endpspicture      
}
\fi

% Category DIAGRAMS START HERE


\newcommand{\factorisationfdiagram}{
    $
    \begin{array}{c p{1cm} c p{1.0cm} c}
    \Rnode{a}{a}&&\Rnode{Imf}{Im(f)}&&\Rnode{b}{b}
    \end{array}
    \begin{arrows}
    \ncline{->>}{a}{Imf}\alabel{f_e}
    \ncarr{Imf}{b}\alabel{f_m}\idcomp
    \end{arrows}
    $
}
\newcommand{\nakedbinarysourcediagram}[5]{
\begin{array}{c p{0.5cm} c}
             &&   \Rnode{b}{#2}\\[0.01cm]
\Rnode{a}{#1} &&               \\[0.01cm] 
             &&   \Rnode{c}{#3}
\end{array} 
\begin{arrows}
\ncarr{a}{b}
\alabel{#4}
\ncarr{a}{c}
\blabel{#5}
\end{arrows}
}

\newcommand{\binarysourcediagram}[5]{$\nakedbinarysourcediagram{#1}{#2}{#3}{#4}{#5}$}
\newcommand{\fgsourcediagram}{\binarysourcediagram{a}{b}{c}{f}{g}}

%  binary source diagram with arrows pointing SE and SW
% nakedSWSEsourcediagram{prefix}{a}{b}{c}{f}{g}
\newcommand{\nakedSWSEsourcediagram}[6]{
\begin{array}{c c c}
              & \Rnode{#1a}{#2} &               \\[1.0cm] 
\Rnode{#1b}{#3} &               &\Rnode{#1c}{#4}
\end{array} 
\begin{arrows}
\ncarr{#1a}{#1b}
\alabel{#5}
\ncarr{#1a}{#1c}
\blabel{#6}
\end{arrows}
}


%  binary sink diagram with arrows pointing SE and SW
\newcommand{\nakedNWNEsinkdiagram}[5]{
\begin{array}{c c c}
              & \Rnode{a}{#1} &               \\[0.5cm] 
\Rnode{b}{#2} &               &\Rnode{c}{#3}
\end{array} 
\begin{arrows}
\ncarr{b}{a}
\alabel{#4}
\ncarr{c}{a}
\blabel{#5}
\end{arrows}
}

\newcommand{\simpleunaryfdrepresentationdiagram}[6]{
$
\nakedbinarysourcediagram{#1}{#2}{#3}{#4}{#5}
\begin{arrows}
\ncarr{b}{c}
\alabel{#6}
\end{arrows}
$
}

\newcommand{\unaryfdrepresentationdiagram}[8]{
$
\begin{array}{c p{0.2cm} c}
\nakedbinarysourcediagram{#1}{#2}{#3}{#4}{#5}&& \Rnode{d}{#6}
\end{array}
\begin{arrows}
\ncarr{d}{b}
\idcomp
\blabel{#7}
\ncarr{d}{c}
\alabel{#8}
\end{arrows}
$
}

\newcommand{\unaryfdrepresentationmappeddiagram}[8]{
$
\begin{array}{c p{0.2cm} c}
\nakedbinarysourcediagram{D(#1)}{D(#2)}{D(#3)}{D(#4)}{D(#5)}&& \Rnode{d}{D(#6)}
\end{array}
\begin{arrows}
\ncarr{b}{d}
\alabel{D(#7)^-1}
\ncarr{d}{c}
\alabel{D(#8)}
\end{arrows}
$
}

\newcommand{\commutativetrianglediagram}[6]{
$
\begin{array}{c p{0.4cm} c p{0.4cm} c}
              && \Rnode{b}{#2}  &&                 \\[0.6cm]
\Rnode{a}{#1} &&                && \Rnode{c}{#3}  
\end{array}
\begin{arrows}
\ncarr{a}{b}
\alabel{#4}
\ncarr{b}{c}
\alabel{#5}
\ncarr{a}{c}
\blabel{#6}
\end{arrows}
$
}

\newcommand{\commutativetrianglediagrammutant}[6]{
$
\begin{array}{c  c  c}
              & \Rnode{b}{#2}  &                 \\[0.85cm]
\Rnode{a}{#1} &                & \Rnode{c}{#3}  
\end{array}
\begin{arrows}
\ncarr{a}{b}
\alabel{#4}[0.15]
\ncarr{b}{c}
\alabel{#5}[0.6]
\ncarr{a}{c}
\blabel{#6}
\end{arrows}
$
}

\newcommand{\epimonosplitdiagram}[3]{
\commutativetrianglediagram{#1}{img(#3)}{#2}{#3_e}{#3_m}{#3}   
}


\iffalse %saved
\begin{array}{c p{2.0cm} c }                
               &&  \Rnode{b1}{#3_1}    \\ [0.75cm]
               &&  \Rnode{b2}{#3_2}    \\ [0.5cm]
\Rnode{a}{#2}  &&                      \\ [-0.5cm]
               &&       \vdots         \\ [0.85cm]
               &&  \Rnode{bn}{#3_{#1}}  
\end{array}
\fi

%nakedmultisourceobjects{n}{a}{b}
\newcommand{\nakedmultisourceobjects}[3]{
\begin{array}{c p{2.0cm} c }
\Rnode{a}{#2}   &&
\begin{array}{c }                
\Rnode{b1}{#3_1}   \\ [0.75cm]
\Rnode{b2}{#3_2}   \\ [0.25cm]
\vdots             \\ [0.35cm]
\Rnode{bn}{#3_{#1}}  
\end{array}
\end{array}
}

% \nakedmultisourcediagram{n}{a}{b}{f}
\newcommand{\nakedmultisourcediagram}[4]{
\nakedmultisourceobjects{#1}{#2}{#3}
\begin{arrows}
\ncarr{a}{b1}
\alabel{#4_1}[0.5]
\ncarr{a}{b2}
\alabel{#4_2}[0.5][-1]
\ncarr{a}{bn}
\blabel{#4_{#1}}[0.5][-1]
\end{arrows}
}

% \nakedmultisourcepathdiagram{n}{a}{b}{f}
\newcommand{\nakedmultisourcepathdiagram}[4]{
\nakedmultisourceobjects{#1}{#2}{#3}{#4}
\begin{arrows}
\simplepath{a}{b1}
\alabel{#4_1}[0.5]
\simplepath{a}{b2}
\alabel{#4_2}[0.5][-1]
\simplepath{a}{bn}
\blabel{#4_{#1}}[0.5][-1]
\end{arrows}
}


\newcommand{\multisourcediagram}[4]{$\nakedmultisourcediagram{#1}{#2}{#3}{#4}$}
\newcommand{\multisourcepathdiagram}[4]{$\nakedmultisourcepathdiagram{#1}{#2}{#3}{#4}$}


% \monosourcedefinitiondiagram{x}{g}{h}{n}{a}{b}{f}
\newcommand{\monosourcedefinitiondiagram}[7]{
$
\begin{array}{c p{1.5cm} c}
\Rnode{x}{#1} && \nakedmultisourcediagram{#4}{#5}{#6}{#7}
\end{array}
\begin{arrows}
\parallelarrows{x}{a}{#2}{#3}
\end{arrows}
$
}

%\multisourcenplusonediagram{n}{a}{b}{f}{c}{g}
\newcommand{\multisourcenplusonediagram}[6]{
$
\begin{array}{c p{2.0cm} c }
\Rnode{a}{#2}   &&
\begin{array}{c }                
\Rnode{b1}{#3_1}   \\ [0.55cm]
\Rnode{b2}{#3_2}   \\ 
\vdots             \\ 
\Rnode{bn}{#3_{#1}} \\ [0.65cm] 
\Rnode{c}{#5} 
\end{array}
\end{array}
\begin{arrows}
\ncarr{a}{b1}
\alabel{#4_1}[0.6][1]
\ncarr{a}{b2}
\alabel{#4_2}[0.6][0]
\ncarr{a}{bn}
\blabel{#4_{#1}}[0.6][0]
\ncarr{a}{c}\blabel{#6}[0.6][0]
\end{arrows}
$
}

\newcommand{\fghfactordiagram}[6]
{
\binarysourcediagram{#1}{#2\roomup{0.5cm}}{#3}{#4}{#5}
\begin{arrows}
\ncarr{b}{c}
\alabel{#6}
\end{arrows}
}

\newcommand{\fghpartialfactordiagram}[6]{
\binarysourcediagram{#1}{#2\roomup{0.5cm}}{#3}{#4}{#5}
\begin{arrows}
\ncdarr{b}{c} %dashed arrow
\alabel{#6}
\end{arrows}
}

\newcommand{\fnsourceqnsource}{
$
\begin{array}{c p{0.25cm} c  p{0.25cm} c }
             &&   \Rnode{b1}{b_1} &&              \\[0.4cm]
\Rnode{a}{a} &&                   && \Rnode{c}{c} \\[0.4cm]
             &&   \Rnode{bn}{b_n} &&              
\end{array} 
\begin{arrows}
\ncarr{a}{b1}
\alabel{f_1}
\ncarr{c}{b1}
\blabel{q_1} 
\ncarr{a}{bn}
\blabel{f_n}
\ncarr{c}{bn}
\alabel{q_n}
\end{arrows}
$   
}

\newcommand{\parallelarrows}[4]{
\ncarc[nodesep=2pt,arcangle=10,offset=2pt]{->}{#1}{#2}
\alabel{#3}
\ncarc[nodesep=2pt,arcangle=-10,offset=-2pt]{->}{#1}{#2}
\blabel{#4}
}

\newcommand{\paralleldiagram}[4]{
$
\rule[-0.3cm]{0pt}{0.9cm} %to add vertical space of diagram -- based on lowering diagram 0.3cm and heght 0.9cm
                            % change thickness from 0pt to 1 pt to debug
\begin{array}{c p{0.5cm} c}
\Rnode{a}{#1}       &&   \Rnode{b}{#2}
\end{array} 
\begin{arrows}
\parallelarrows{a}{b}{#3}{#4}
\end{arrows}
$
}

\newcommand{\fgparalleldiagram}{
 $
\rule[-0.3cm]{0pt}{0.9cm} %to add vertical space of diagram -- based on lowering diagram 0.3cm and heght 0.9cm
                            % change thickness from 0pt to 1 pt to debug
\begin{array}{c p{0.5cm} c  }
 \Rnode{a}{a}            &&   \Rnode{b}{b}
\end{array} 
\begin{arrows}
\parallelarrows{a}{b}{f}{g}
\end{arrows}
$  
}

\newcommand{\fgcomposablediagram}[5]{
\mbox{
\roomup{0.45cm}
$
\begin{array}{c p{0.5cm}cp{0.5cm}c}
\Rnode{x}{#1}&&\Rnode{y}{#2}&&\Rnode{z}{#3}
\end{array}
\begin{arrows}
\ncarr{x}{y}
\alabel{#4}
\ncarr{y}{z}
\alabel{#5}
\end{arrows}
$    
}
}



% copied from MToD paper (preamble.tex)
\newcommand{\simplepath}[2]{
\ncline[linestyle=none,linewidth=0.1pt]{#1}{#2}   %was linestyle=dotted
\ncput[npos=0.05]{\pnode{dot#21}}
\ncput[npos=0.27]{\dotnode[dotsize=1pt]{dot#22}}
\ncput[npos=0.50]{\dotnode[dotsize=1pt]{dot#23}}
\ncput[npos=0.80]{\dotnode[dotsize=1pt]{dot#24}}
\ncput[npos=0.975]{\pnode{dot#25}}
\ncline[nodesep=2pt]{->}{dot#21}{dot#22}
\ncline[nodesep=2pt]{->}{dot#22}{dot#23}
\ncline[nodesep=2pt]{->}{dot#24}{dot#25}
\ncline[linestyle=dotted,nodesep=8pt]{dot#23}{dot#24} %was 10pt
}



\newcommand{\stringtype}{text}
\newcommand{\numbertype}{number}

\newcommand{\dgsrcedge}
{
\setlength{\arroffsetA}{3pt}
\setlength{\arroffsetB}{3pt}
\ncarr[5]{edge}{node} 
\arreset  
}
\newcommand{\structuraldgsrcedge}
{
\setlength{\saroffsetA}{3pt}
\setlength{\saroffsetB}{3pt}
\ncsar[5]{edge}{node} 
\sarreset 
}
\newcommand{\dgtargetedge}
{
\setlength{\arroffsetA}{-3pt}
\setlength{\arroffsetB}{-3pt}
\ncarr[-5]{edge}{node} 
\arreset   
}
\newcommand{\dgbasic}
{
\begin{array}{c}
\Rnode{node}{node}  \\[2cm]
\Rnode{edge}{edge}       
\end{array}
\begin{arrows}
\dgsrcedge
\alabel{src}
\dgtargetedge
\blabel{trg}
\end{arrows}    
}

% dgbasiclabelled{labeltype}
\newcommand{\dgbasiclabelled}[1]
{
\begin{array}{cp{0.75cm}c}
%rule [raise-height]{width}{height}
\dgbasic   &&  \Rnode{labeltypelhs}{\rule[0cm]{0cm}{0.3cm}}\Rnode{labeltype}{#1} 
\end{array}
\begin{arrows}
\ncarr{node}{labeltypelhs}
\alabel{label}
\ncarr{edge}{labeltypelhs}
\blabel{label}
\end{arrows}       
}

\newcommand{\setoflabelleddgs}
{
\begin{array}{cp{1.0cm} : p{0.5cm}c}
\setofdg   &&&
\begin{array}{l}
\Rnode{text}{}\stringtype \\[1cm]
\Rnode{number}{}\numbertype 
\end{array}
\end{array}
\begin{arrows}
\ncarr{dg}{text}
\alabel{name}[0.3]
\ncarr{node}{number}
\alabel{label}[0.3]
\ncarr{edge}{number}
\blabel{label}[0.3]   
\end{arrows}
}


\newcommand{\structuraldgbasic}
{
\begin{array}{c}
\Rnode{node}{node}  \\[1.5cm]
\Rnode{edge}{edge}       
\end{array}
\begin{arrows}
\structuraldgsrcedge
\alabel{src}
\dgtargetedge
\blabel{trg}
\end{arrows}    
}

\newcommand{\nodepartof}
{
\ncarr{node}{dg}
\alabel{p}     
}
\newcommand{\structuralnodepartof}
{
\ncsar{node}{dg}     
}

\newcommand{\setofdg}
{
\begin{array}{c}
\rnode{dg}{dg} \\[1.5cm]
\dgbasic
\begin{arrows}
\nodepartof
\end{arrows}
\end{array}
}

\newcommand{\setofdgcommutativediagram}
{
\begin{array}{c c c}
&\rnode{dg}{dg} &\\[0.75cm]
\rnode{Lnode}{node}&&\rnode{Rnode}{node}\\[0.75cm]
&\rnode{edge}{edge}&
\end{array}
\begin{arrows}
\ncarr{Lnode}{dg}
\alabel{p}
\ncarr{Rnode}{dg}
\blabel{p}
\ncarr{edge}{Lnode}
\alabel{src}
\ncarr{edge}{Rnode}
\blabel{trg}
\end{arrows}
}


\newcommand{\structuralsetofdg}
{
\begin{array}{c}
\rnode{dg}{dg} \\[2cm]
\structuraldgbasic
\begin{arrows}
\structuralnodepartof
\end{arrows}
\end{array}
}


\newcommand{\dgnumericallylabelleddetail}
{
\begin{array} {c}
\begin{array} {p{1.5cm} c}
     & \Rnode{abs}{\ \ 1\ \ }  
\end{array} \\[1.0cm]
\dgbasiclabelled{number}  
\end{array}
\begin{arrows}
%\setlength{\arroffsetA}{3pt}
\setlength{\arroffsetA}{-6pt}
\setlength{\arroffsetB}{-12pt}
\ncarr{abs}{labeltype}
\alabel{0}
\setlength{\arroffsetA}{3pt}
\setlength{\arroffsetB}{0pt}
\ncarr{abs}{labeltype}
\alabel{1}
\setlength{\arroffsetA}{9pt}
\setlength{\arroffsetB}{12pt}
\ncarr{abs}{labeltype}
\alabel{2 \hdots}
\end{arrows}
}



\newcommand{\dgabsuniquelylabelled}
{
\begin{array}{cp{0.75cm}c}
\dgbasic   &&  \Rnode{l}{l} 
\end{array}
\begin{arrows}
\ncarr{node}{l}
\alabel{label}
\idcomp
\ncarr{edge}{l}
\blabel{label}
\idcomp
\end{arrows}
}


\newcommand{\dglocallyuniquelylabelleddirectedgraph}
{
\begin{array}{cp{0.75cm}c}
\dgbasic   &&  \Rnode{l}{l} 
\end{array}
\begin{arrows}
\ncarr{node}{l}
\alabel{label}
\idcomp
\ncarr{edge}{l}
\blabel{label}
\idcomp
\dgsrcedge  % redrawn so that I can bar it with \idcomp
\idcomp
\end{arrows}
}


\newcommand{\setdgexitsuniquelylabelled}
{
\setoflabelleddgs
%
\begin{arrows}
%redraw arrows and bar using \idcomp
\ncarr{dg}{text}
\idcomp
\ncarr{node}{number}
\idcomp
\ncarr{edge}{number}
\idcomp
\dgsrcedge  % redrawn so that I can bar it with \idcomp
\idcomp
\nodepartof  % repeat so thatI can bar with \idcomp
\idcomp
\end{arrows}
}




\newcommand{\structuralsetofsgincludingabs}
{
\begin{array}{cp{0.75cm}c}
\Rnode{abs}{abs}                       \\[1cm]
\structuralsetofdg   &&  \Rnode{u}{u} 
\end{array}
\begin{arrows}
\ncsar{dg}{abs}
\ncarr{dg}{u}
\alabel{name}
\idcomp
\ncarr{node}{u}
\alabel{label}
\idcomp
\ncarr{edge}{u}
\blabel{label}
\idcomp
\structuraldgsrcedge  % redrawn so that I can bar it with \idcomp
\idcomp
\structuralnodepartof  % repeat so that I can bar with \idcomp
\idcomp
\end{arrows}
}









\renewcommand{\erpictureFolder}[0]{../../../SharedPictures}
\setcounter{equation}{0}

\makeatletter
\newcommand{\xRightarrow}[2][]{\ext@arrow 0359\Rightarrowfill@{#1}{#2}}
\makeatother


\title[John Cartmell]{Exploring the Mathematical Theory of Data}
%% Which is to say types as they are used in practice in software development and as represented in theory in categories and in syntactic type theories.
%% There is also a subplot concerning representation of context which certain types depend on -- again represented in practice and in theory. 

\author{John Cartmell}
\institute{\\}
\date{Jan 18, 2024}
\bibliographystyle{plainnat}
\usepackage{framed}
\usepackage{bibentry}
\usepackage{colortbl}
\usepackage{ulem}   % for \dashuline{dashing} for seconday key
\usepackage{soul,xcolor} % also used for striking out text 

\usepackage{listings}
\lstset{%
  escapeinside={(*}{*)},%
}
\usepackage{arydshln} % vertical dashed lines between columns of an array
\usepackage{pst-arrow} %for \bigarrow


%Redefine the \Fin macro to be category of sets
%\renewcommand{\Fin}{\Set}



\newcommand{\fgsourcediag}{$\binarysourcediag{a}{b}{c}{f}{g}$}
\newcommand{\rangeplus}{RR.5 range } % has trailing space
\newcommand{\datacat}{$\gamma$-structured category}
\newcommand{\datacatw}{\datacat\ }
\mathchardef\localjchyphen="2D
\newtheorem{construction}[theorem]{Construction}

\newcommand{\IfSforGammaCwithRCwords}{
If $S$ is a sketch for \datacatw \catcw considered as a data specification with requirement $\reqtc$\ }
\newcommand{\IfSforGammaCwithRCwordsvariant}{
If $S$ is a sketch for \datacatw  \catcw and if $S$ is considered as a data specification with requirement $\reqtc$\ }

\begin{document}

\setstcolor{red}
\begin{frame}
\titlepage
\nobibliography{../../SharedBibliography/temp/bibliography}
\end{frame} 


% see top level tex for NotesnAll
\ifNotesnAll
\begin{frame}{Title page Notes}
\begin{itemize}
  \item HELLO EVERYBODY. 
  \item Thank you Shaowei for inviting me to present.
  \item Thank you everybody for coming to listen.
  \item As you see, I will be speaking of the mathematical theory of data.
\end{itemize}
\end{frame}
\fi


\begin{frame}{Preamble}
\underline{\url{www.researchgate.net/profile/John-Cartmell}}
\medskip
\begin{itemize}
\item Preparation for a Mathematical Theory Of Data
\item Concept Instance Algebras (\textit{circa} 1973)
\item Instances of Generalised Algebraic Theories 
\end{itemize}
\end{frame}
\ifNotesnAll
\begin{frame}{Preamble Notes (1)}
\begin{itemize}
\item I would like to start just by mentioning three papers that I have posted on my Research Gate page.
\item The first is contains useful background to the talk I give here.
  \item The second describes some algebras that I first defined about 50 years ago to be the algebraic equivalents of the syntactic notion of a Generalised Algebraic Theory. 
\begin{itemize}
  \item I was being supervised by Professor Dana Scott. 
  \item I took Dana a draft of my thesis in which I defined GATs, these algebras and a lengthy proof of equivalence.
  \item Dana didn't look too impressed and simply said ``couldn't you make these algebras into categories like Lawvere has done''. I went away crestfallen.
  \item By the next morning I had the definition of contextual categories and had the onerous task of proving contextual categories and GATs were equivalent ... which of course I did do ... and I went on to write up the proof in my thesis.
  \item So, thank you Dana for taking me into your research group at Oxford and thank you for posing that ``couldn't you just..'' question.
\end{itemize}
\end{itemize}
\end{frame}
\begin{frame}{Preamble Notes(2)}
\begin{itemize}
  \item The third paper here also relates to my thesis. It expores the definition of models of generalised algebriac theories in sets and families of sets WITHOUT going via contextual categories. BUT, and it is a big but, in this paper ...and generally, I speak of INSTANCES rather than models. 
  This is because I want to use  the word model  AS OTHER SCIENTISTS USE IT and ... NOT as it used in mathematical logic. Therefore for me theories have instances, not models.
\end{itemize}
\end{frame}
\fi

\begin{frame}{Terminology}
\begin{itemize}
\item \textit{modelling} for me is \textit{theorizing}
\item I speak of \textit{instances of theories}  rather then \textit{models of theories}
\item I speak of \textit{data specifications} except when I forget and I call them \textit{data models}
\item the act of constructing data specifications is \textit{data modelling}
\item a \textit{model of data} is a meta-theory (a meta-model) describing what constitutes a data specification. Most significantly there are
\begin{itemize}
\item \textit{relational} and 
\item \textit{nested relational} 
models of data
\end{itemize}
\item the \textit{mathematical theory of data} is a meta-theory of data that supports technology  independent reasoning about data specifications in all their forms.
\end{itemize}
\end{frame}

\begin{frame}{Why?}
\begin{itemize}[<+->]
  \item There are gross inefficiencies in the methodologies and working practices used in  a key activity in s/w systems development and maintenance namely in  the creation and maintenance of specifications of the data stored in databases and 
  represented in messages variously intra-communicated between components of systems and inter-communicated between systems. 
  \item These inefficiences have been established and endorsed by a theory which is grossly inadequate.
  \item A new theory is required to expose and remedy the shortcomings.
  \item The challenge is to positively impact best practice.
\end{itemize}
\end{frame}

\begin{frame}{Mathematical Theory of Data}
\begin{itemize}
\item is a meta-theory,
\item it covers \textit{principles} and \textit{criteria} for goodness of data specifications,
\item it reveals the significance of commutative diagrams and therefore category theory.
\item The slogan on the tin is \textit{Good Data Modelling is Good Theorising}. 
\end{itemize}
\end{frame}



\begin{frame}{1970 - 1979}
\begin{itemize}
\item In 1970, E.F.Codd introduced the relational model of data and the idea of normal form.
\item In 1971, he introduces the terms `functional dependency' and  `third normal form' are introduced in an IBM techical report published in a now out of print book.
\item In 1977, Fagin introduces `fourth normal form' (4NF) and `multivalued dependencies'.
\item In 1979, Fagin describes `projection-join normal form' also known as `'fifth normal form' (5NF).
\end{itemize} 
\end{frame}

\begin{frame}{Inclusion Normal Forms}
Ling and Goh 1992 
\begin{quote}
Since
classical normal forms (including the Improved 3NF)
have failed to consider the effects of INDs on the structure
of a database, they are inadequate in characterizing a
database scheme which is truly devoid of redundancies.
In consideration of the above, we propose a new normal
form, called Inclusion Normal Form (IN-NF)...
\end{quote}
\end{frame}

\begin{frame}{The success of Codd's Relational Model of Data}
\begin{itemize}
\item By 2020 Oracle Corporation, founded in 1977, were the world's second largest software company with a 42\% share of an 
estimated \$30billion market for relational database technology. 
\item Codd 1990 says that
\begin{quote}
The relational model is solidly based on two parts of mathematics: first-
order predicate logic and the theory of relations.
\end{quote}
\item My opinion is that this has been to found data modelling on the wrong mathematics. 
\item Codd's mathematical basis and therefore his  model  do nothing to guide the programmer as navigator, to use Charles W Bachman's phrase, 
\item nor do they encourage thinking about navigation path equivalence, i.e. diagrams that commute.
\item I will demonstrate the importance of diagrams that commute to the goodness of data specifications.
\item The right mathematical starting point for the theory of data is category theory.
\end{itemize}
\end{frame}
 




\begin{frame}{Relational Normal Form Criteria}
Classic relational database normal form criteria 
\begin{itemize}
    \item from a mathematical perspective are not really normal forms!
    \item they are goodness criteria (GC) that articulate good engineering principles
    \item they include:
        \begin{center}
        \begin{tabular}{p{6.0cm}  l }
         \innerbullet first normal form (1NF)            &\Rnode{A1}{}                       \\
         \innerbullet 2nd normal form (2NF)              &                                   \\
         \innerbullet 3rd normal form (3NF)              &
                      \Rnode{A2}{}\braceLabel{A1}{A2}{Codd 1970,1971}                        \\
         \innerbullet Boyce-Codd normal form (BCNF)      &                                   \\
         \innerbullet 4th normal form (4NF)              & -- Fagin 1977                     \\
         \innerbullet projection-join normal form (5NF)  & -- Fagin 1979                     \\
         \innerbullet inclusion normal form (IN-NF)      & -- Ling and Goh  1992
        \end{tabular}
        \end{center}
\end{itemize}
\end{frame}

\begin{frame}{Goodness Criteria}
I wish to 
\begin{itemize}
\item show that we can genericise relational database normal form criteria into abstract logical terms,
\item achieve goodness criteria that are generic i.e. can be applied to any data specifications not just to relational schema,
 \item prove that the classic relational database normal form criteria (2NF, 3NF, BCNF, INC-NF, 4NF, 5NF)  are  consequences of these generic goodness criteria.
\end{itemize}
\medskip
\begin{center}
\begin{tabular}{c p{0.75cm} c p{0.75cm} c}
\Rnode{A}{\parbox{2cm}{\textit{Two \\Fundamental Principles}}} 
             &$\Longrightarrow$& \Rnode{B}{\parbox{1.5cm}{\textit{Generic Goodness Criteria}}} 
             &$\Longrightarrow$& \Rnode{C}{\parbox{1.5cm}{\textit{Classic Normal Forms}}}
\begin{arrows}
%\ncarr{A}{B}
%\ncarr{B}{C}
\end{arrows}
\end{tabular}
\end{center}
\end{frame}

\begin{frame}{View}
A data specification  
\begin{itemize}
\item is a  presentation of a theory (of what is),
\item choice of primitives in a given presentation is choice of which data to be stored or communicated.
\end{itemize}
\medskip
A data specification method 
\begin{itemize}
\item is a method for expressing such a  theory,
\item unequivocally it enables definitions of types and certain relationships between these types,
\item types are equally types of data and types of real world entity.
\end{itemize}

\end{frame}





\begin{frame}{Goodness Principles in Outline}
\begin{itemize}
    \item Principle 1 -- absence of redundancy in presentation.
    \item Principle 2 -- the theory be the tightest possible fit to the facts.
\end{itemize}

Regarding which
\begin{itemize}
    \item Principle 2 expresses a kind of logical completeness.
    \item The two principles collectively
    \begin{itemize}
        \item ensure absence of redundancy in data and in data management logic,
        \item Imply classic relational normal forms.
    \end{itemize}
\end{itemize}
\end{frame}

\begin{frame}{Data Specifications}
Two kinds of types in play
\begin{itemize}
\item  the \textit{definienda} -- types all of whose instances are \textit{particulars}
\begin{itemize}
\item employee, department, student, account, product, order, shipment, delivery, flight, booking and so on
\item molecular structure, atom, bond, element, isotope, reaction, metabolite, mass trace, chromatogram, peak
\item table, column, primary key, foreign key
\item node and edge. 
\end{itemize}
\pause 
\item  the \textit{definiens}  -- types all of whose instances are \textit{universals}
\begin{itemize}
       \item string, integer, float, boolean and so on
\end{itemize}
\end{itemize}
\pause
\begin{itemize}
\item in ER modelling 
\begin{itemize}
\item the \textit{definienda} are called \textit{entity types}
\item the \textit{definiens} are called \textit{attribute types} or \textit{domains}.
\end{itemize}
\end{itemize}
\end{frame}

\begin{frame}{Data Specifications}
A data specification is a sketch of a category with some additional structure:
\begin{itemize}
\item that it is a \textit{sketch} is crucial because it is only nodes and edges of the sketch for which data is stored and/or communicated, 
\item that there are commutative diagrams is crucial to construction of representational 
specifications from logical specifications.
\item that the category had additional structure is significant:
\begin{itemize}
\item so that we can distinguish structural from non-structural relationships to describe structure nesting and thereby hierarchical data,
\item so that we can give account of database normal forms 
(BCNF, 3NF, 4NF and 5NF),
\item so that we can allow for missing data as represented by NULL values, 
\item so that types of universals can be distinguished from types of particulars.
\end{itemize}
\end{itemize}
\end{frame}



\begin{frame}{Definitions} 
In a category \catc, a  \textit{source} is a family of morphisms with common domain: \\
\scalebox{0.65}{
\multisourcediagram{n}{a}{b}{f}
} 
\medskip
Such a source is said to be a \textit{mono source}  iff for all $g,h:x \morph a$ in \catcw 
so that \scalebox{0.65}{
\monosourcedefinitiondiagram{x}{g}{h}{n}{a}{b}{f}
} 
in \catcw then if $g \circ f_i = h \circ f_i$, for each $i$,  then $g=h$.
\medskip
OR, in presence of cartesian products, $<f_1,...f_n>$ is a monomorphim.
\end{frame}


\begin{frame}{Restriction Categories I(2002, Cockett and Lack)}
There is a category $\Par$ of sets and partial functions.
\medskip
For a partial function $f : A \morph B$ define its restriction idempotent to
be the  function
$\bar{f} : A \morph A$ is defined by
  \begin{equation*}
    \bar{f}(a)  =
    \begin{cases}
      a           & \mbox{if $f$ defined at $a$,}\\
      undefined   & \mbox{otherwise.}
    \end{cases}
  \end{equation*}

  This $bar$ operator satisfies four algebraic identities R.1, R.2, R3, and R.4.
\end{frame}

\begin{frame}{Restriction Categories I(2002, Cockett and Lack)}
A \textit{restriction category} is a  category along with an operator
that maps every morphism $f$ to an idempotent $\bar{f}$ on its domain satisfying

R.1 For $f:a \morph b$ in \catcw $$\bar{f} \circ f =f$$.

R.2. If \fgsourcediag in \catcw then
$$\bar{g} \circ \bar{f}=\bar{f} \circ \bar{g}.$$

R.3. If \fgsourcediag in \catcw then
$$\overline{\bar{f} \circ g} = \bar{f} \circ \bar{g}$$.

R.4. If $\sequentialdiag{a}{b}{c}{f}{g}$ in \catcw then
$$f \circ \bar{g} = \overline{f \circ g} \circ f$$. 

\end{frame}

\begin{frame}{Restriction Categories III (2006, Cockett and Lack)}

In this paper Cockett and Lack define
\begin{itemize}
\item the \textit{restriction product} of a pair of objects in a restriction category,

\item give the cartesian product of sets in $\Par$ as an example.
\end{itemize}
\end{frame}

\begin{frame}
\begin{itemize}
\item In a restriction category there is a partial ordering on each hom set
Hom(a,b) defined by:
$$f \leq g \mbox{ iff } f = \bar{g} \circ f$$,
\item meaning if left hand side defined the right hand side defined and equal,
\item there are lots of data specification near commutative 
diagrams i.e. instances of relationships $f$, $g$ and $h$
satisfying
$$ f \circ g \leq h$$.
\end{itemize}
\end{frame}

\begin{frame}{Range Categories (2012, Cockett, Guo and Hofstra)}

A \textit{range category} is a restriction category 
with an additional operator as follows
if $f: a \morph b$ in  \catcw then
$$\hat{f}: b \morph b$$
satisfying

RR.1 For $f:a \morph b$ in \catcw $$\bar{\hat{f}} = \hat{f}.$$

RR.2 For $f:a \morph b$ in \catcw $$f \circ \hat{f} = f.$$

RR.3. If $\sequentialdiag{a}{b}{c}{f}{g}$ in \catcw then
$$\widehat{f \circ \bar{g}} = \hat{f} \circ \bar{g}.$$

RR.4. If $\sequentialdiag{a}{b}{c}{f}{g}$ in \catcw then
$$\widehat{(hat({f}) \circ g)} = \widehat{f \circ g}.$$

\end{frame}





\begin{frame}{Schein’s Theorem for Range Categories}
\begin{itemize}
\item A range category may specify
RR.5 whenever $\equaliser{a}{f}{\paralleldiag{b}{c}{g}{h}}$ in \catcw then
 $$f \circ g = f \circ h \Rightarrow  \hat{f} \circ g = \hat{f} \circ h$$.

\item  The category of sets and partial functions satisfies RR.5.

\item Cockett \textit{et al} prove that if a range category \catcw satisfies 
RR.5
then there is a faithful functor $S: \catc \morph \Par$.
\end{itemize}
\end{frame}

\begin{frame}{Partial Inverse of a Monomorphism}
If  $m:a \morph b$ is a monomorphim in range category \catcw then
a map $m^{-1}: b \morph a$ 
$$
\begin{array}{c p{1cm} c}
\Rnode{a}{a} && \Rnode{b}{b}
\end{array}
\begin{arrows}
\ncarr{a}{b}
\alabel{m}
\ncarr[30]{b}{a}
\alabel{m^{-1}}
\end{arrows}
$$
I will say $m^{-1}$ is the \textit{(partial) inverse} of $m$ iff
\begin{center}
$m \circ m^{-1}= id_a$
\hspace{0.75cm} and \hspace{0.75cm}
$m^{-1} \circ m= \widehat{m}.$
\end{center} 
\end{frame}

\newcommand{\datacat}{$\gamma$-category }

\begin{frame}{Definition}

I assume a fixed set $V$ of universals and define data specifications and instances relative to $V$. 

\begin{itemize}
\item A \textit{\datacat} is a triple $\tuple{\catc,M,v}$ where 
\begin{itemize}
\item \catcw is a \rangeplus category with specified finite products,
\item $M$ is a set of designated monomorphisms of \catcw closed under composition and including all identity morphisms 
and such that each $m \in M$   has a partial inverse $m^{-1}$,
\item a distinguished object $v$, such that every morphism $f: v \morph x$ in \catcw 
factors through $m^{-1}$, for some monomorphism $m$
\end{itemize}
\item Define an instance $F$ of a \datacat to be a 
range functor $F: \catc \morph \Par$ 
that preserves the specified finite products
and maps the object $v$ to the set $V$.
\end{itemize}
\end{frame}

\begin{frame}{Data Specifications}
A data specification is a restricted kind of sketch for a \datacat category
with a designated object $v$ and designated finite products and designated monomorphisms.

The most significant restrictions are 
\begin{itemize}
\item the designated object $v$ has no outgoing edges.
\item every non-v-node is the domain of at least one v-valued mono-source
i.e. for every non-v-node $a$, for some $n \geq 1$, there exists a source
\begin{displaymath}
\parallelsource{a}{v}{m}{n}
\end{displaymath}
which is designated as a mono-source i.e. for which $\tuple{m_1,...m_n}$ is a designated monomorphism.
\end{itemize}

There are other restrictions which I won't describe here.

\end{frame}

\begin{frame}{Relational Data Specifications}
What is much more interesting is that we can give additional conditions for such a sketch to be a relational data specification or a logical data specification (akin to an entity relational model).
\medskip
A relational data specification is a data specification 
such that the sketch has no edges between non-v-nodes.

How can that be? 
\end{frame}

\iffalse
\begin{frame}
Well in a relational specification if I wish to represent a morphism 
$f: a \morph b$ in the category, without representing $f$ in the sketch
then I select a mono-source
$\parallelsource{b}{v}{m}{n}$ is a designated and add edges

$\parallelsource{a}{v}{f}{n}$ to the sketch along with 
the commuting diagram
$\tuple{f_1,...f_n} \circ \widehat{\tuple{m_1,...m_n}}=\tuple{f_1,...f_n}$

Then $f$ in the category is constructed from the sketch as $\tuple{f_1,...f_n} \circ \tuple{m_1,...m_n}^{-1}$
\end{frame}
\fi

\begin{frame}{Relational Data Specification}
\begin{definition}
A \textit{relational data specification} is one whose only edges have a non-v-node as a domain and the $v$ node as codomain. 
\end{definition}

In relational-speak the non-v-nodes of the sketch are 
said to be tables and the edges are said to be columns. 

\begin{definition}
A \textit{classic relational data specification} is a
relational data specification
in which every specified path equivalence is of the form 
$$\ntuple{f} \circ \widehat{\ntuple{m}}=\ntuple{f}$$,
where the $f_i$ and $m_i$ are  edges, 
i.e. is representative of a referential inclusion dependency
$$a[f_1,...f_n] \subseteq b[m_1,...m_n].$$
Additional condition regarding the $\bar{f_i}$.
\end{definition}
\end{frame}

\begin{frame}{Chen's Method 1976}
\begin{lemma}
For any data specification there is an equivalent data specification
(one with the same theory category) which is relational.
\end{lemma}
\begin{proof}
The first-cut construction is
given a data specification in which contrary to the definition of relational there is an edge $f:a \morph b$ replace $f$ in the sketch by edges $\fn$
where $\wanton{m}$ is a v-valued mono-source with domain $a$ 
and inclusion dependency 
$$a[f_1,...f_n] \subseteq b[m_1,...m_n]$$
to the sketch.
In path equivalences replace uses of $f$ by uses of
$$\ntuple{f}\circ \ntuple{m}^{-1}$$.
Need to be more sophisicated to take account of commutative diagrams.
\end{proof}
\end{frame}

\begin{frame}{Logical Data Specifications (Entity Relationship Models)}
The extreme opposite of a relational model. In principle at least:
\begin{definition}
A data specification is said to be \textit{pure logical} iff 
it does not contain any path equivalence of the form
$$\ntuple{p} \comp \widehat{\ntuple{m}}=\ntuple{p}$$
where  each $p_i: a \morph v$  is a path and 
where $\ntuple{m}$ is a  v-valued mono-source at $b$,
 i.e. does not have a path equivalence such 
 as would represent a referential inclusion dependency
$$a[p_1,...p_n] \subseteq b[m_1,...m_n]$$.
\end{definition}
\end{frame}

\begin{frame}{Abstraction of Logical Model}
\begin{lemma}
For any classic relational data specification 
there is an equivalent data specification
(i.e. one with the same theory category) which is logical.
\end{lemma}

\begin{proof}
In outline: We construct a series of equivalent models by eliminating each 
inclusion dependency in turn. When all eliminated the resulting model is the required logical model. Eliminate the inclusion dependency 
$a[f_1,...f_n] \subseteq b[m_1,...m_n]$
as follows:
\begin{itemize}
\item remove the inclusion dependency,
\item replace by an edge $f: a \morph b$, 
\item remove those $f_i$ that are edges and 
rewrite any occurrence of such $f_i$ in the remaining inclusion depdencies by $f \circ  m_i$, 
\item for those $f_i$ that are not edges add a path equivalence (i.e. a commuting diagram)
$f \circ m_i = f_i$.
\end{itemize}
\end{proof}
An example follows.
\end{frame}



\begin{frame}{Data Specification and Instance}
A \textit{data specification} is a sketch $S$ for a \datacat \catc.
\\
\medskip
An \textit{instance} of a data specification $S$ is a functor $D: \catc(S) \morph \Par$,
where $\catc(S)$ is the \datacat generated by $S$.
\end{frame}

\begin{frame}{The Requirement}
A \textit{requirement} for a data specification $S$ 
is a set of instances of the sketch $S$ or, equivalently, is a set $R_C$ of functors where for each
$D \in R_C$, $D: \catc(S) \morph \Par$, where $\catc(S)$ is the \datacat generated by the sketch $S$.
\end{frame}

\begin{frame}{Fundamental Principles of Data Specification}
\IfSforCwithRCwordsvariant 
\begin{itemize}
\item 
\textbf {Principle 1 :} No redundancy. The sketch $S$ ought to be a minimum sketch for structured category \catcw i.e. there should be no subsketch of $S$ which generates  \catc.
\item
\textbf {Principle 2:} \catcw ought to be \textit{maximally constrained} to $\reqtc$.
\end{itemize}
\end{frame}



\begin{frame}{Key definition: \catcw \textit{maximally constrained} to $\reqtc$}
\begin{itemize}


\item  question -- is there a $\catcp$ that extends \catcw and that will do a better job. 

\item Is there a $\catcp$ and an $I: \catc \morph \catcp$  such that 
all instances in the requirement $\reqtc$ uniquely factor though $I$
$$
\begin{array} {c p{2cm} c}
\Rnode{Cp}{C'} && \\ [0.25cm]
             && \Rnode{finset}{\Fin} \\ [0.15cm]
\Rnode{C}{C}  
\end{array}
\begin{arrows}
\ncarr {C}{finset}
\alabel{D}
\ncarr{C}{Cp}
\alabel{I}
\ncarr{Cp}{finset}
\alabel{D'} 
\end{arrows}
$$
\pause  and at least one other instance $F$ of \catcw does not factor through $I$.
\ncarr[-20]{C}{finset}
\blabel{F} 
\end{itemize}
\pause If there is no such $I: \catc \morph \catcp$ then we shall say that 
\catcw is \textit{maximally constrained} with respect to $\reqtc$.\\
\medskip
 ...meaning  that structured category \catcw  is the tightest possible fit to facts i.e. to the requirement $R_{\catc}$.
\end{frame}

\begin{frame}{Representational Completeness}
Alternative way of approaching tightest fit:
\begin{itemize}
\item That which is in the data and can be represented in the theory should be represented in the theory.
\item To make precise we can give definitions
 of \textit{representational completeness} wrt $\reqtc$ 
\begin{center}
\begin{tabular}{>{\bfseries}l l} 
2A. & equationally complete   \\
2B. & functionally complete   \\
2C. & referentially complete  \\
2D. & mono complete           \\
2E. & epi complete            \\
2F. & product complete        \\
\end{tabular}
\end{center}
\pause \item In these defintions that \catcw is $x$ complete wrt $\reqtc$ will mean exactly that the set of instances $\reqtc$ are jointly reflective of $x$.
\end{itemize}
\end{frame}


\begin{frame}{BCNF in the abstract (based on Zaniolo 1982 Definition 2)}
If $S$ is a sketch for a \rangeplus category $C$ with designated mono-sources with inverses   and designated finite restriction products
and $S$ is considered as a data specification with requirement $\reqtc$, then it ought to be the case
that if 
\scalebox{0.9}{\multisourcenplusonediagram{n}{a}{b}{x}{c}{y}}
are edges of $S$
and if \msfd{x_1,...x_n}{y}  is a non-trivial functional dependency  between these edges
then \onslide<1>{...}
\onslide<2>{\scalebox{0.8}{\multisourcediagram{n}{a}{b}{x}} is a designated mono-source.}
\end{frame}



\begin{frame}{Equational Completeness and Goodness Criteria 2A}
\pause \begin{definition}
If $\catc$ is a  structured category and $\reqtc$ is a set of instances,
 then say that  $\catc$ is \textit{equationally complete} with respect 
to the requirement $\reqtc$ iff all path equivalences with respect to $R_C$ are represented in \catcw 
i.e. iff for all diagrams \fgparalleldiagram in $\catc$,  
if in all instances $D \in \reqtc$, $D(f)=D(g)$,  then $f=g$.
\end{definition}
\medskip
\pause In other words:
the set of functors $\reqtc$ is jointly faithful. \\
\medskip
\pause \goodnesscriteria{2A} \IfSforCwithRCwords then \catcw ought to be equationally complete
with respect to $R_C$.
\end{frame}

\begin{frame}{Equational completeness cont. }
Does equational completeness follow from maximal constrainedness?
Scein's theorem for range categories implied that it does.
\begin{lemma}[Path Equivalence Representation Lemma]
If $\catc$ is a locally finite category and $\reqtc$ is a set of instances, if $\catc$ 
 is
\textit{maximally constrained} to the requirement $\reqtc$ then it is equationally
complete with respect to $\reqtc$.
\end{lemma}
\begin{proof}
Suppose \fgparalleldiagram  in $\catc$ and that in all instances $D \in \reqtc$, $D(f)=D(g)$. 
Define $\catcp$ to be \catc plus $f=g$. Consider the functor given by Schein's theorem.
This is an instance of $\catc$ which therefore extends to an instance $H$ of $\catcp$.  Then
$Hom_{\catc}(a,f)=H(I(f))=H(I(g))=Hom(a,g)$ and applying to $id_a$ we have that $f=g$.
????????????????????????
\end{proof}
\end{frame}


\begin{frame}{Functional Dependencies}
To describe Goodness Criteria 2B I first need to
\begin{itemize}
\item Define what we mean by \textit{functional dependency}
-- abstracted and simplified from definition given by Codd 1971.
\item Define what we mean by a functional dependency being \textit{represented}
-- inspired by language found in Zaniolo 1982.
\item State as the criteria that all functional dependencies ought to be represented -- 
the spirit of Zaniolo's paper. 
\end{itemize}
\end{frame}

\begin{frame}{Definition of Functional Dependency}
\begin{definition}
If $\catc$ is a range category and $\reqtc$ is a set of instances and if \fgsourcediag
in $\catc$ then there is a  \textit{functional dependency} of $g$ on $f$ with respect to $\reqtc$ iff
there is a family of functions $H_D)_{D \in \reqtc}$ such that 
in each instance $D$, $H_D$ is a unique function $H_D: D(b) \morph D(c)$, such that $D(f) \circ H_D = D(g)$ and $\overline{H_D}=\widehat{D(f)}$.
\end{definition}
\pause If $H$ is such a functional dependency then we say that $\fundep{H}{f}{g}$ in $\catc$ with respect to $\reqtc$.
\end{frame}

\begin{frame}{Definition of Functional Dependency (2)}
\begin{definition}
If $\catc$ is a range category with finite products and if $\reqtc$ is a set of instances 
of \catcw then if 
\scalebox{0.9}{\multisourcenplusonediagram{n}{a}{b}{f}{c}{g}}
in \catcw then say that $g$ is functionally dependent of $f_1,...f_n$
iff $g$ is functionally dependent on $\fntuple$.
\end{definition}
\pause If $H$ is such a functional dependency then we say that $\fundep{H}{f}{g}$ in $\catc$ with respect to $\reqtc$. 

This notation is adapted  from relational database theory. It doesn't imply a 2-category structure.
\end{frame}


\begin{frame}{Definition of Representation of Functional Dependency}
\begin{definition}
If $\catc$ is a range category and $\reqtc$ is a set of instances, if
\fgsourcediag
in $\catc$ 
and if there is a functional dependency $\fundep{H}{f}{g}$ then say that 
the functional dependency $H$ is \textit{represented} in $\catc$ 
iff there exists a morphism $h:b \morph c$ in $\catc$ such that 
\oldt{for each instance $D \in \reqtc$, $D(h)=H_D$}
$f \circ h = g$. Note that we can deduce that if $h$ is such a representation then
in each instance $D$, $\widehat{D(f)} \circ D(h) = H_D$.
\end{definition}
\medskip
\pause If \catcw is a requirement and $\reqtc$ a set of instances then \catcw is said to be 
\textit{functionally complete} with respect to $\reqtc$ iff every functional dependency
present in $\reqtc$ is represented in \catc.\\
\medskip
\pause \goodnesscriteria{2B}\IfSforCwithRCwords then \catcw ought to be functionally complete with respect to $\reqtc$.
\end{frame}


\begin{frame}{Representation Lemma for Functional Dependencies}
\begin{lemma}
If $\catc$ is a \datacat category and $\reqtc$ is a set of instances, if $\catc$ is 
\textit{maximally constrained} to the requirement $\reqtc$ then 
\catcw is functionally complete with respect to $\reqtc$.
\end{lemma}
\begin{proof}
Suppose$\fundep{H}{f}{g}$  is a functional dependency with respect to $\reqtc$.
Hence in all instances $D \in \reqtc$, $D(f)$ is surjective.

Extend \catcw to $\catcp$ by formally adding a morphism $\qq{h}$ such that $f \circ \qq{h}=g$. Extend each $D$ to a $D'$ by defining $D(h)=H_D$. 

Define a functor from \catcw to $\Fin$ as  a certain quotient of the coproduct of functor $HomP_{\catc}(a,-)$ with itself. 
Extend to $\catcp$ and demonstrate that there exists $k:b \morph c$ in \catcw such that $f \circ k=g$. 
which shows that $\fundep{H}{f}{g}$ is represented in $\catc$.
\end{proof}
\end{frame}



\newcommand{\incdsetup}{$
\begin{array}{c p{0.5cm} c p{0.5cm} c}
             &&\Rnode{bi}{b_i} &&              \\[0.5cm]
\Rnode{a}{a} &&                && \Rnode{c}{c}
\end{array}
\begin{arrows}
\ncarr{a}{bi}\alabel{f_i}
\ncarr{c}{bi}\blabel{q_i}
\end{arrows}
$}

\newcommand{\incdresolution}{$
\begin{array}{c p{0.5cm} c p{0.5cm} c}
             &&\Rnode{bi}{D(b_i)} &&              \\[0.5cm]
\Rnode{a}{D(a)} &&                && \Rnode{c}{D(c)}
\end{array}
\begin{arrows}
\ncarr{a}{bi}\alabel{D(f_i)}
\ncarr{c}{bi}\blabel{D(q_i)}
\ncarr{a}{c}\blabel{J_D}
\end{arrows}
$}



\begin{frame}{Definition of Inclusion Dependencies}
If $\catc$ is a category and $\reqtc$ is a set of instances 
and if
\incdsetup
in $\catc$, for $i$, $1 \leq i \leq n$, then an \textit{inclusion dependency} $J$, written $a[f_1,...f_n] \overset{J}{\subseteq} c[q_1,..q_n]$, is a family of functions $J_D)_{D \in \reqtc}$
such that each instance $D \in \reqtc$, $J_D$ is a function 
 such that
 \incdresolution commutes, for $i$, $1 \leq i \leq n$. \\

\medskip
\begin{itemize}
\item If each $J_D$ is the unique such function then the inclusion dependency is said to be referential. 
\end{itemize}
\end{frame}

\begin{frame}{Definition of Referentially Complete}
\begin{definition}
If $\catc$ is a category and $\reqtc$ is a set of instances and if
\fnsourceqnsource
in $\catc$ and if $a[f_1,...f_n] \overset{J}{\subseteq} c[q_1,..q_n]$ is a referential inclusion dependency
with respect  to $\reqtc$ then say that the inclusion dependency $J$ is \textit{represented} in $\catc$
iff there exists a morphism $j:a \morph c$ in $\catc$ such that in each instance $D \in \reqtc$, $D(j) = J_D$. 
\end{definition}
If \catcw is a category and $\reqtc$ a set of instances then 
\catcw is \textit{referentially complete} with respect to $\reqtc$ 
iff all referential inclusion dependencies present in $\reqtc$ are represented in \catc.
\end{frame}

\begin{frame}{Goodness Criteria 2C}
\goodnesscriteria{2C} \IfSforCwithRCwords 
then \catcw ought to be referentially complete with respect to $\reqtc$.
\end{frame}

\begin{frame}{Representation Lemma for Referential Inclusion Dependencies}
If \catcw is a locally finite category and $\reqtc$ is a set of instances, if \catcw is 
\textit{maximally constrained} to the requirement $\reqtc$ then
every referential inclusion dependency with respect to $\reqtc$ is represented in $\catc$.
\end{frame}

\begin{frame}{Proof}
Suppose $a[f_1,...f_n] \overset{J}{\subseteq} c[q_1,..q_n]$ is a referential inclusion dependency
where for each $i$, $1 \leq i \leq n$,
\scalebox{0.9}{\incdsetup} in \catc.

Extend category \catcw to a category \catcpw such that in \catcpw there is
a morphism $\qq{r}: a \morph c$ such that for each $i$, $1 \leq i \leq n$, 
$\qq{r} \circ q_i = f_i$.
Every $D \in \reqtc$ extends uniquely to functor $D: \catcp \morph \Fin$. Therefore, since \catcw is maximally constrained then every functor $F: \catc \morph \Fin$ can be extended to $F':\catcp \morph \Fin$. \\
\pause Let $H: \catcp \morph \Fin$ be the extension of the hom functor $Hom(a,-): \catc \morph \Fin$ to 
\catcp. \\
\pause For each $i$, we have $H(\qq{r}) \circ H(q_i) = H(f_i)$, \\
\pause \hspace {3cm} i.e. $H(\qq{r}) \circ Hom(a,q_i) = Hom(a,f_i)$. \\
\pause Therefore in particular $H(\qq{r})(id_a) \circ q_i =  f_i$. \\
\pause Hence $H(\qq{r})(id_a):a \morph c$ in \catcw represents the inclusion dependency $J$.

\highlight{Why? Do we know that $H(c)=Hom(a,c)$?} How do we know this?
I guess in the category case because $I \circ H = Hom$? \highlight{Does this still go through?}
\end{frame}

\begin{frame}{Three Goodness Criteria}{for Categories as Data Specifications}
Definition of Goodness Criteria 2A, 2B and 2C:
\begin{itemize}
\item If \catcw is a category and $\reqtc$ is a set of instances of \catcw then
\medskip
\begin{tabular}{>{\bfseries}l l} 
2A: & \catcw ought to be equationally complete wrt $\reqtc$  \\
2B: & \catcw ought to be  functionally complete wrt $\reqtc$  \\
2C: & \catcw ought to be referentially complete wrt $\reqtc$ \\
\end{tabular}
\pause \item We have shown that if \catcw is locally finite and meets Criteria 2 that it is maximally constrained then it also meets Criteria 2A, 2B and 2C.

\pause \item We will move on to consider categories with designated monomorphisms and epimorphisms
\item First though we will describe a possible improvement to this situation described above.
\end{itemize}
\end{frame}


\begin{frame}{Revised Best Practice}
\begin{center}
\begin{tabular}{ p{1.25cm} p{2cm} p{1.25cm}}
\parbox{3cm}{logical \\ data \\ specification}
&  $\xRightarrow{\parbox{2cm}{\textit{diagram \\ aware \\ transformation \\ (automated)}}
                }$ 
& \parbox{3cm}{relational \\ data \\specification}
\end{tabular}
\end{center}
\textbf{Such that}
\begin{itemize}
	\item If appropriate goodness criteria met by the logical specification 
	then the relational specification meets the classic relational goodness criteria.
\end{itemize}
\textbf{Impact}
\begin{itemize}
	\item No manual normalisation process.
	\item No source code required to describe the physical level.
\end{itemize}
\end{frame}


\end{document}