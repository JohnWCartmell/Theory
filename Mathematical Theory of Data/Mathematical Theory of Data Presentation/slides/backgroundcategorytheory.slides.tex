
\begin{frame}{Mono Sources} 
In a category \catc, a  \textit{source} is a family of morphisms with common domain: \\
\scalebox{0.65}{
\multisourcediagram{n}{a}{b}{f}
} 
\medskip
Such a source is said to be a \textit{mono source}  iff for all $g,h:x \morph a$ in \catcw 
so that \scalebox{0.65}{
\monosourcedefinitiondiagram{x}{g}{h}{n}{a}{b}{f}
} 
in \catcw then if $g \circ f_i = h \circ f_i$, for each $i$,  then $g=h$.
\medskip
OR, in presence of cartesian products, $<f_1,...f_n>$ is a monomorphim.
\end{frame}

\begin{frame}{Category of Sets and Partial Functions}
\begin{itemize}
\item There is a category $\Par$ of sets and partial functions.
\item
For a partial function $f : A \morph B$ define its restriction idempotent to
be the  function
$\bar{f} : A \morph A$ is defined by
  \begin{equation*}
    \bar{f}(a)  =
    \begin{cases}
      a           & \mbox{if $f$ defined at $a$,}\\
      undefined   & \mbox{otherwise.}
    \end{cases}
  \end{equation*}
\item  This $bar$ operator satisfies four algebraic identities R.1, R.2, R3, and R.4.
  \item Also for a partial $f : A \morph B$ define its range idempotent to
be the  function
$\hat{f} : B \morph B$ is defined by
  \begin{equation*}
    \hat{f}(b)  =
    \begin{cases}
      b           & \mbox{if there exists $a \in A$ such that $f(a)=b$,}\\
      undefined   & \mbox{otherwise.}
    \end{cases}
  \end{equation*}
  \item This \textit{hat} operator satisfies identities
  RR.1, ...RR.5.
\end{itemize}
\end{frame}

\begin{frame}{Restriction Categories I(2002, Cockett and Lack)}
A \textit{restriction category} is a  category along with an operator
that maps every morphism $f$ to an idempotent $\bar{f}$ on its domain satisfying

R.1 For $f:a \morph b$ in \catcw $$\bar{f} \circ f =f$$.

R.2. If \fgsourcediag in \catcw then
$$\bar{g} \circ \bar{f}=\bar{f} \circ \bar{g}.$$

R.3. If \fgsourcediag in \catcw then
$$\overline{\bar{f} \circ g} = \bar{f} \circ \bar{g}$$.

R.4. If $\sequentialdiag{a}{b}{c}{f}{g}$ in \catcw then
$$f \circ \bar{g} = \overline{f \circ g} \circ f$$. 

\end{frame}

\begin{frame}{Range Categories (2012, Cockett, Guo and Hofstra)}
\begin{itemize}
\pause \item
A \textit{range category} is a restriction category 
with an additional operator as follows
if $f: a \morph b$ in  \catcw then
$\hat{f}: b \morph b$ satisfying \\
\medskip
\begin{quote}
RR.1 For $f:a \morph b$ in \catc, $\bar{\hat{f}} = \hat{f}.$ \\
\medskip
RR.2 For $f:a \morph b$ in \catc, $f \circ \hat{f} = f.$ \\
\medskip
RR.3. If $\sequentialdiag{a}{b}{c}{f}{g}$ in \catcw then
$\widehat{f \circ \bar{g}} = \hat{f} \circ \bar{g}.$ \\
\medskip
RR.4. If $\sequentialdiag{a}{b}{c}{f}{g}$ in \catcw then
$\widehat{(hat({f}) \circ g)} = \widehat{f \circ g}.$
\end{quote}
\pause \item A range category may additionally satisfy
RR.5 if $\equaliser{a}{f}{\paralleldiag{b}{c}{g}{h}}$ then
 $f \circ g = f \circ h \Rightarrow  \hat{f} \circ g = \hat{f} \circ h.$
\end{itemize}
\end{frame}


\begin{frame}{Restriction Products}
\begin{itemize}
\item The ususal cartesian product of sets in the category of sets and partial functions
  $\Par$ does not satisfy the ususal categorical cartesian product conditions.
\item In 2006 ``Restriction Categories III'' Cockett and Lack define 
the appropriate notion of product.
\item They define \textit{restriction product} of a pair of objects in a restriction category.
\end{itemize}
\end{frame}

\begin{frame}{Partial Ordering of $Hom(A,B)$}
\begin{itemize}
\item In a restriction category we can define a partial ordering on each hom set
Hom(a,b) by defining :
$$f \leq g \mbox{ iff } f = \bar{g} \circ f$$,
\item we can think of $f \leq g$ as meaning that if $f$ is defined then $g$ is defined and the two are equal,
%in my experience
\pause \item there are lots of data specifications having near commutative 
diagrams i.e. instances of relationships $f$, $g$ and $h$
satisfying
$$ f \circ g \leq h$$.
\end{itemize}
\end{frame}


\begin{frame}{Partial Inverse of a Monomorphism}
If  $m:a \morph b$ is a monomorphim in range category \catcw then
a map $m^{-1}: b \morph a$ 
$$
\begin{array}{c p{1cm} c}
\Rnode{a}{a} && \Rnode{b}{b}
\end{array}
\begin{arrows}
\ncarr{a}{b}
\alabel{m}
\ncarr[30]{b}{a}
\alabel{m^{-1}}
\end{arrows}
$$
I will say $m^{-1}$ is the \textit{(partial) inverse} of $m$ iff
\begin{center}
$m \circ m^{-1}= id_a$
\hspace{0.75cm} and \hspace{0.75cm}
$m^{-1} \circ m= \widehat{m}.$
\end{center} 
\end{frame}
