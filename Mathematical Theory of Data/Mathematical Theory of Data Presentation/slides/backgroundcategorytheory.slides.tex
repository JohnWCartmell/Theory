
\begin{frame}{Definitions} 
In a category \catc, a  \textit{source} is a family of morphisms with common domain: \\
\scalebox{0.65}{
\multisourcediagram{n}{a}{b}{f}
} 
\medskip
Such a source is said to be a \textit{mono source}  iff for all $g,h:x \morph a$ in \catcw 
so that \scalebox{0.65}{
\monosourcedefinitiondiagram{x}{g}{h}{n}{a}{b}{f}
} 
in \catcw then if $g \circ f_i = h \circ f_i$, for each $i$,  then $g=h$.
\medskip
OR, in presence of cartesian products, $<f_1,...f_n>$ is a monomorphim.
\end{frame}

\iffalse DONT NEED THIS
\begin{frame}{Alternative Definition}{Mono Source Limit Cone}  
In a category \cat{C},
\scalebox{0.65}{


$
\begin{array}{c p{2.0cm} c p{2.0cm} c}				
                   &&	 \Rnode{B1}{B_1}  \\ [0.75cm]
									 &&  \Rnode{B2}{B_2}  \\ [0.5cm]
		\Rnode{A}{A}  &&                    \\ [-0.5cm]
				           &&       \vdots      \\ [0.85cm]
                   &&	 \Rnode{Bn}{B_n}  
\end{array}
$
%\setlength{\arrnodesepA}{7pt}
%\setlength{\arrnodesepB}{8pt}
%\setlength{\arroffsetA}{2pt}
%\setlength{\arroffsetB}{0pt}
\begin{arrows}
\ncarr{A}{B1}
\alabel{f_1}[0.5]
\ncarr{A}{B2}
\alabel{f_2}[0.5][-1]
%\blabel{\vdots}[0.4][-2]  % move up 5pts -- dont know why I need this to get position for vdots
\ncarr{A}{Bn}
\blabel{f_n}[0.5][-1]
\end{arrows}


} is a mono source iff \\
\begin{center}
\scalebox{0.65}{
$
\begin{array}{c p{2.0cm} c p{2.0cm} c}				
                           &&	\Rnode{At}{A}  &&          \Rnode{B1}{B_1}  \\ [0.65cm]
													 &&                &&          \Rnode{B2}{B_2}  \\ [0.5cm]
		\Rnode{Al}{A}          &&                &&                           \\ [0cm]
				                   &&                &&           \vdots      \\ [0.85cm]
                           &&	\Rnode{Ab}{A}  &&          \Rnode{Bn}{B_n}  
\end{array}
$
%\setlength{\arrnodesepA}{7pt}
%\setlength{\arrnodesepB}{8pt}
%\setlength{\arroffsetA}{2pt}
%\setlength{\arroffsetB}{0pt}
\ncarr{Al}{At}
\alabel{id_A}
\ncarr{Al}{Ab}
\blabel{id_A}
\ncarr{At}{B1}
\alabel{f_1}[0.5]
\ncarr{At}{B2}
\alabel{f_2}[0.4][-1]
%\blabel{\vdots}[0.4][-2]  % move up 5pts -- dont know why I need this to get position for vdots
\ncarr{At}{Bn}
\blabel{f_n}[0.3][-2]
\ncarr{Ab}{B1}
\alabel{f_1}[0.3][-1]
\ncarr{Ab}{B2}
\blabel{f_2}[0.3][-1]
\ncarr{Ab}{Bn}
\blabel{f_n}[0.4]
%\alabel{\vdots}[0.4]

} 
is a limit cone.
\end{center}
\end{frame}
\fi

\begin{frame}{Restriction Catgeories (2002, Cockett and Lack)}
A \textit{restriction category} is a  category along with an operator
that maps every morphism $f$ to an idempotent $\bar{f}$ on its domain satisfying

R.1 For $f:a \morph b$ in \catcw $$\bar{f} \circ f =f$$.

R.2. If \fgsourcediag in \catcw then
$$\bar{g} \circ \bar{f}=\bar{f} \circ \bar{g}.$$

R.3. If \fgsourcediag in \catcw then
$$\overline{\bar{f} \circ g} = \bar{f} \circ \bar{g}$$.

R.4. If $\sequentialdiag{a}{b}{c}{f}{g}$ in \catcw then
$$f \circ \bar{g} = \overline{f \circ g} \circ f$$.
\end{frame}
\begin{frame}{Range Categories (2012, Cockett, Guo and Hofstra)}

A \textit{\rangeplus category} is a restriction category 
with an additional operator as follows
if $f: a \morph b$ in  \catcw then
$$\hat{f}: b \morph b$$
satisfying

RR.1 For $f:a \morph b$ in \catcw $$\bar{\hat{f}} = \hat{f}.$$

RR.2 For $f:a \morph b$ in \catcw $$f \circ \hat{f} = f.$$

RR.3. If $\sequentialdiag{a}{b}{c}{f}{g}$ in \catcw then
$$\widehat{f \circ \bar{g}} = \hat{f} \circ \bar{g}.$$

RR.4. If $\sequentialdiag{a}{b}{c}{f}{g}$ in \catcw then
$$\widehat{(hat({f}) \circ g)} = \widehat{f \circ g}.$$

RR.5 whenever $\equaliser{a}{f}{\paralleldiag{b}{c}{g}{h}}$ in \catcw then
 $$f \circ g = f \circ h \Rightarrow  \hat{f} \circ g = \hat{f} \circ h$$.
\end{frame}

\begin{frame}
In a restriction category there is a partial ordering on each hom set
Hom(a,b) defined by:
$$f \leq g iff f = \bar{g} \circ f$$
\end{frame}



\begin{frame}{Schein’s Theorem for Range Categories}
Cockett \textit{et al} prove that if a range category \catcw satisfies 
RR.5
then there is a faithful functor $S: \catc \morph \Par$. \\
\end{frame}

\begin{frame}{Partial Inverse of a Monomorphism}
If  $m:a \morph b$ is a monomorphim in range category \catcw then
a map $m^{-1}: b \morph a$ 
$$
\begin{array}{c p{1cm} c}
\Rnode{a}{a} && \Rnode{b}{b}
\end{array}
\begin{arrows}
\ncarr{a}{b}
\alabel{m}
\ncarr[30]{b}{a}
\alabel{m^{-1}}
\end{arrows}
$$
I will say $m^{-1}$ is the \textit{(partial) inverse} of $m$ iff
\begin{center}
$m \circ m^{-1}= id_a$
\hspace{0.75cm} and \hspace{0.75cm}
$m^{-1} \circ m= \widehat{m}.$
\end{center} 
\end{frame}
