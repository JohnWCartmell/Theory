
\begin{frame}{Universals}
\begin{quote}
In language, there are proper names, and there are adjectives. 
The proper names apply to `things' or `persons', each of which is the only thing or person
to  which the name in question applies. The sun, the moon, France, Napolean, are unique; there are not a number of instances of things to which these names apply. On the other hand, words like `cat', `dog', `man' apply to many different things. 
The problem of universals is concerned with the meaning of such words, and also of adjectives, such as `white', `hard' `round' and so on.
He says `By the term ``universal'' I mean that which is of such a nature as to be predicated of many subjects, by ``individual'' that which is not thus predicated.'  
\end{quote} Betrand Russell, describing Aritotle's Metaphysics.
\end{frame}

\begin{frame}{A.J.Cotnoir}
\begin{itemize}
\item Contrast storing or communicating a reference to a particular such as a 
\textit{customer} or a \textit{supplier}
with storing or communicating a reference to a universal like the number \textit{909}. 
\item In the latter case the number is wholly located within the reference
--- the reference to a number is the number itself.
\item This is the same for all universals --- a reference to a universal is the entirety of that universal.
\item This is not the case for particulars and this distinguishes universals from particulars.
\item I found a discussion of universals in the context of mereology
written by A.J.Cotnoir. There you find that 
\begin{quote}
Universals are typically said to be ‘wholly located wherever they
are instantiated’.
\end{quote}
\end{itemize}
\end{frame}

\begin{frame}{Attributes}
\begin{quote}
The term attribute is
adopted as a specific term meaning a relationship between a particular on the
one hand and a universal on the other. Actually it would be more accurate
to say that the term attribute is used for a functional relationship between a
type of particulars and a type of universals, i.e. between a type all of whose
instances are particulars a type all of whose instances are universals. When this
functional relationship is evaluated at a particular entity then the result is said
to be the value of the attribute. For this reason the types which I describe as
types all of whose instances are universals are called value sets by Chen in his
1976 paper
\end{quote}
\end{frame}

\begin{frame}{Data is the universal describing the particular.}
\begin{itemize}
\item I will tend talk about how they are commuincated in data but every-
thing that is said applies to how they are communicated generally including nay
especially in natural language. 
\item
What is universally true need not be communicated. Only
that which is particular and the relation of the particular with the universal
need be communicated.
\item An entity model systematically describes all that can be known
of an entity in terms of core functional relationships with universals and core
functional relationships with particulars i.e. in terms of core attributes and core
outgoing directional relationships.
\item But of all this which may be known of an entity all we can commu-
nicate of an entity is its functional relationships with universals. Other rela-
tionships must be communicated indirectly via derivative
relationships with universals --- foreign keys, in other words.
\end{itemize}
\end{frame}




\begin{frame}{The world of universals ...}
From `The Problems of Philosopy', Betrand Russell
\begin{quote}
“The world of universals, therefore, may also be described as the
world of being. The world of being is unchangeable, rigid, exact, de-
lightful to the mathematician, the logician, the builder of metaphysi-
cal systems, and all who love perfection more than life. The world of
existence is fleeting, vague, without sharp boundaries, without any
clear plan or arrangement, but it contains all thoughts and feelings,
all the data of sense, and all physical objects, everything that can
do either good or harm, everything that makes any difference to the
value of life and the world. According to our temperaments, we shall
prefer the contemplation of the one or of the other. The one we do
not prefer will probably seem to us a pale shadow of the one we pre-
fer, and hardly worthy to be regarded as in any sense real. But the
truth is that both have the same claim on our impartial attention,
both are real, and both are important to the metaphysician.”
\end{quote}
\end{frame}



