
\begin{frame}{Goodness Criteria}
\begin{itemize}
    \item From a mathematical perspective are not really normal forms!
    \item They are goodness criteria (GC) that articulate good engineering principles.
\item I wish to show that we can 
\begin{itemize}
\item genericise relational database normal form criteria into abstract logical terms,
\item define goodness criteria that are generic i.e. can be applied to any data specifications not just to relational schema,
 \item prove that the classic relational database normal form criteria (2NF, 3NF, BCNF, INC-NF, 4NF, 5NF)  are  consequences of these generic goodness criteria,
 \item articulate principles from which the generic goodness criteria follow.
\end{itemize}
\end{itemize}
\medskip
\pause \begin{center}
\begin{tabular}{c p{0.75cm} c p{0.75cm} c}
\Rnode{A}{\parbox{2cm}{\textit{Fundamental Principles}}} %Two \\
             &$\Longrightarrow$& \Rnode{B}{\parbox{1.5cm}{\textit{Generic Goodness Criteria}}} 
             &$\Longrightarrow$& \Rnode{C}{\parbox{1.5cm}{\textit{Classic Normal Forms}}}
\begin{arrows}
%\ncarr{A}{B}
%\ncarr{B}{C}
\end{arrows}
\end{tabular}
\end{center}
\end{frame}

\begin{frame}{An abstract view}
\begin{itemize}
\item A data specification is a  presentation of a theory (of what is).
\item There can be many different presentations of a single theory and these have different roles depending on their properties
\begin{itemize}
\item some presentations are said to be \textit{physical} --- the
choice of primitives in such a presentation is a choice of the individual elements to be represented in the data,
\item other presentations are said to be \textit{logical} --- these seek to describe the data  by directly describing its internal relationships. 
\end{itemize}
\item Both the theory and its logical presentations express the overall information content of the data independently of the details of its representation. 
\end{itemize}

\end{frame}


