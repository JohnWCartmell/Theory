
\begin{frame}{Relational Normal Form Criteria}
Classic relational database normal form criteria 
\begin{itemize}
    \item from a mathematical perspective are not really normal forms!
    \item they are goodness criteria (GC) that articulate good engineering principles
    \item they include:
        \begin{center}
        \begin{tabular}{p{6.0cm}  l }
         \innerbullet first normal form (1NF)            &\Rnode{A1}{}                       \\
         \innerbullet 2nd normal form (2NF)              &                                   \\
         \innerbullet 3rd normal form (3NF)              &
                      \Rnode{A2}{}\braceLabel{A1}{A2}{Codd 1970,1971}                        \\
         \innerbullet Boyce-Codd normal form (BCNF)      &                                   \\
         \innerbullet 4th normal form (4NF)              & -- Fagin 1977                     \\
         \innerbullet projection-join normal form (5NF)  & -- Fagin 1979                     \\
         \innerbullet inclusion normal form (IN-NF)      & -- Ling and Goh  1992
        \end{tabular}
        \end{center}
\end{itemize}
\end{frame}

\begin{frame}{Goodness Criteria}
I wish to 
\begin{itemize}
\item show that we can genericise relational database normal form criteria into abstract logical terms,
\item achieve goodness criteria that are generic i.e. can be applied to any data specifications not just to relational schema,
 \item prove that the classic relational database normal form criteria (2NF, 3NF, BCNF, INC-NF, 4NF, 5NF)  are  consequences of these generic goodness criteria.
\end{itemize}
\medskip
\begin{center}
\begin{tabular}{c p{0.75cm} c p{0.75cm} c}
\Rnode{A}{\parbox{2cm}{\textit{Two \\Fundamental Principles}}} 
             &$\Longrightarrow$& \Rnode{B}{\parbox{1.5cm}{\textit{Generic Goodness Criteria}}} 
             &$\Longrightarrow$& \Rnode{C}{\parbox{1.5cm}{\textit{Classic Normal Forms}}}
\begin{arrows}
%\ncarr{A}{B}
%\ncarr{B}{C}
\end{arrows}
\end{tabular}
\end{center}
\end{frame}

\begin{frame}{View}
A data specification  
\begin{itemize}
\item is a  theory (of what is)
\end{itemize}
\medskip
A data specification method 
\begin{itemize}
\item is a method for expressing such a  theory
\item unequivocally it enables definitions of types and certain relationships between these types
\item types are equally types of data and types of real world entity
\end{itemize}
More precisely, 
\begin{itemize}
\item data specification are \underline{presentations} of theories of what is,
\item choice of primitives in a given presentation is choice of which data to be stored or communicated.
\end{itemize}



\end{frame}


