

\begin{frame}{Data Specification Instances and Requirements}
\begin{itemize}
\item A \textit{data specification} is a sketch $S$ for a \datacatw $\catc(S)$.
\item An \textit{instance} of a data specification $S$ is a structure preserving functor $D: \catc(S) \morph \Par$.
\item A \textit{requirement} for a data specification $S$ 
is a set of such instances i.e.  is a set $R_C$ of structure preserving functors where for each
$D \in R_C$, $D: \catc(S) \morph \Par$.
\end{itemize}
\end{frame}

\begin{frame}{Fundamental Principles of Data Specification}
\IfSforGammaCwithRCwordsvariant 
\begin{itemize}
\item 
\textbf {Principle 1 :} No redundancy. The sketch $S$ ought to be a minimum sketch for structured category \catcw i.e. there should be no subsketch of $S$ which generates  \catc.
\item
\textbf {Principle 2:} \catcw ought to be \textit{maximally constrained} to $\reqtc$.
When defined, this will be the most fundamental way of saying that \catcw is a tightest fit to the facts $\reqtc$.
\end{itemize}
\end{frame}


\begin{frame}{Representational Completeness --- Goodness Critera}
Another way of approaching tightest fit:
\begin{itemize}
\item That which is in the requirement and can be represented in the theory should be represented in the theory.
\pause \item To make precise we can give definitions
 of \textit{representational completeness} wrt $\reqtc$ 
\begin{center}
\begin{tabular}{l l} 
Goodness Criteria 2A. & equational completeness,   \\
Goodness Criteria 2B. & functional completeness,   \\
Goodness Criteria 2C. & referential completeness,  \\
\multicolumn{2}{c}{\textit{and others beside.}}      \\
\end{tabular}
\end{center}
\pause \item In these definitions that \catcw is $x$ complete wrt $\reqtc$ will mean exactly that the set of instances $\reqtc$ are jointly reflective of $x$.
\end{itemize}
\end{frame}

\begin{frame}{Equational Completeness --- Goodness Criteria 2A}
\pause \begin{definition}
If $\catc$ is a  \datacatw and $\reqtc$ is a set of instances,
 then say that  $\catc$ is \textit{equationally complete} with respect 
to the requirement $\reqtc$ iff all path equivalences with respect to $R_C$ are represented in \catcw 
i.e. iff for all diagrams \fgparalleldiagram in $\catc$,  
if in all instances $D \in \reqtc$, $D(f)=D(g)$,  then $f=g$.
\end{definition}
In other words,
\begin{itemize}
\item loosely speaking ... if $f=g$ in all data instances then $f=g$,
\item or...the set of functors $\reqtc$ is jointly faithful. 
\end{itemize}
\medskip
\pause \goodnesscriteria{2A} \IfSforGammaCwithRCwords then \catcw ought to be equationally complete
with respect to $R_C$.
\end{frame}

\begin{frame}{Functional Dependencies --- Goodness Criteria 2B}
To describe Goodness Criteria 2B I first need to
\begin{itemize}
\item Define what we mean by \textit{functional dependency}
-- abstracted and simplified from definition given by Codd 1971.
\item Define what we mean by a functional dependency being \textit{represented}
-- inspired by language found in Zaniolo 1982.
\item State as the criteria that all functional dependencies ought to be represented -- 
the spirit of Zaniolo's paper. 
\end{itemize}
\end{frame}

\begin{frame}{Definition of Functional Dependency}
\begin{definition}
If $\catc$ is a \datacatw and $\reqtc$ is a set of instances and if \fgsourcediag
in $\catc$ then there is a  \textit{functional dependency} of $g$ on $f$ with respect to $\reqtc$ iff
there is a family of functions $H_D)_{D \in \reqtc}$ such that 
in each instance $D$, $H_D$ is a partial function $H_D: D(b) \morph D(c)$, such that both
$$\overline{H_D}=\widehat{D(f)}$$
and 
$$D(f) \circ H_D = D(g).$$
\end{definition}
\begin{itemize}
\item  $H_D$ will be the unique such partial function (this follows from RR.5), 
\pause \item If $H$ is such a functional dependency then we say that $\fundep{H}{f}{g}$ in $\catc$ with respect to $\reqtc$.
\end{itemize}
\end{frame}

\iffalse
\begin{frame}{Definition of Functional Dependency (2)}
\begin{definition}
If $\catc$ is a \datacatw and if $\reqtc$ is a set of instances 
of \catcw then if 
\scalebox{0.9}{\multisourcenplusonediagram{n}{a}{b}{f}{c}{g}}
in \catcw then say that $g$ is functionally dependent of $f_1,...f_n$
iff $g$ is functionally dependent on $\fntuple$.
\end{definition}
\pause If $H$ is such a functional dependency then 
we say that $\fundep{H}{\set{f_1,f_2,...f_n}}{g}$ in $\catc$ with respect to $\reqtc$. 

This notation is adapted  from relational database theory. It doesn't imply a 2-category structure.
\end{frame}
\fi

\begin{frame}{Functional Dependencies --- Goodness Criteria 2B}
\begin{definition}
If $\catc$ is a \datacatw and $\reqtc$ is a set of instances, if
\fgsourcediag
in $\catc$ 
and if there is a functional dependency $\fundep{H}{f}{g}$ then say that 
the functional dependency $H$ is \textit{represented} in $\catc$ 
iff there exists a morphism $h:b \morph c$ in $\catc$ such that 
$D(h) = H_D$.
\end{definition}
\medskip
\pause If \catcw is a \datacatw and $\reqtc$ a set of instances then \catcw is said to be 
\textit{functionally complete} with respect to $\reqtc$ iff every functional dependency
present in $\reqtc$ is represented in \catc. Loosely speaking ... whenever $g$ factors through $f$ in every data instance then $g$ should factor through $f$.\\
\medskip
\pause \goodnesscriteria{2B}\IfSforGammaCwithRCwords then \catcw ought to be functionally complete with respect to $\reqtc$.
\end{frame}

\newcommand{\incdsetup}{$
\begin{array}{c p{0.5cm} c p{0.5cm} c}
             &&\Rnode{bi}{b_i} &&              \\[0.5cm]
\Rnode{a}{a} &&                && \Rnode{c}{c}
\end{array}
\begin{arrows}
\ncarr{a}{bi}\alabel{f_i}
\ncarr{c}{bi}\blabel{q_i}
\end{arrows}
$}

\newcommand{\incdresolution}{$
\begin{array}{c p{0.5cm} c p{0.5cm} c}
             &&\Rnode{bi}{D(b_i)} &&              \\[0.5cm]
\Rnode{a}{D(a)} &&                && \Rnode{c}{D(c)}
\end{array}
\begin{arrows}
\ncarr{a}{bi}\alabel{D(f_i)}
\ncarr{c}{bi}\blabel{D(q_i)}
\ncarr{a}{c}\blabel{J_D}
\end{arrows}
$}

\begin{frame}{Definition of Inclusion Dependencies}
If $\catc$ is a \datacatw and $\reqtc$ is a set of instances 
and if
\incdsetup
in $\catc$, for $i$, $1 \leq i \leq n$, then an \textit{inclusion dependency} $J$, written $a[f_1,...f_n] \overset{J}{\subseteq} c[q_1,..q_n]$, is a family of functions $J_D)_{D \in \reqtc}$
such that each instance $D \in \reqtc$, $J_D: D(a) \morph D(c)$ is a partial function
(so that we have this diagram in \Par \incdresolution
) 
and $J_D$ satisfies
\begin{equation}
\overline{J_D} = \overline{\tuple{D(f_1),...D(f_n)}} \tag{1}
\end{equation}
 and, for each $i$, $1 \leq i \leq n$,
\begin{equation}
  J_D \comp D(q_i) = \overline(J_D) \comp D(f_i) \tag{2}
\end{equation}
  or, equivalent to (2) in the presence of (1): 
\begin{equation}
J_D \comp \tuple{D(q_1),...D(q_n)} = \tuple{D(f_1),...D(f_n)} \tag{3}
\end{equation}

\medskip
\begin{itemize}
\item If each $J_D$ is the unique such function then the inclusion dependency is said to be referential. 
\end{itemize}
\end{frame}

\begin{frame}{Referential Completeness and Goodness Criteria 2C}
\begin{definition}
If $\catc$ is a category and $\reqtc$ is a set of instances and if
\fnsourceqnsource
in $\catc$ and if $a[f_1,...f_n] \overset{J}{\subseteq} c[q_1,..q_n]$ is a referential inclusion dependency
with respect  to $\reqtc$ then say that the inclusion dependency $J$ is \textit{represented} in $\catc$
iff there exists a morphism $j:a \morph c$ in $\catc$ such that in each instance $D \in \reqtc$, $D(j) = J_D$. 
\end{definition}
If \catcw is a category and $\reqtc$ a set of instances then 
\catcw is \textit{referentially complete} with respect to $\reqtc$ 
iff all referential inclusion dependencies present in $\reqtc$ are represented in \catc.

\goodnesscriteria{2C} \IfSforGammaCwithRCwords 
then \catcw ought to be referentially complete with respect to $\reqtc$.
\end{frame}


\begin{frame}{BCNF in the abstract (based on Zaniolo 1982 Definition 2)}
If $S$ is a simple relational sketch for a \datacatw \catcw 
and $S$ is considered as a data specification with requirement $\reqtc$, then it ought to be the case
that if 
\scalebox{0.9}{\multisourcenplusonediagram{n}{a}{b}{x}{c}{y}}
are edges of $S$
and if \msfd{x_1,...x_n}{y}  is a non-trivial functional dependency  between these edges
%then \onslide<1>{...}
\onslide<2>{then \scalebox{0.8}{\multisourcediagram{n}{a}{b}{x}} is a designated mono-source.}
\end{frame}

\begin{frame}{Deriving the classic normal form criteria}
\begin{lemma}
\begin{enumerate}[(i)]
\item For a simple relational data specification $S$ with requirement $\reqtc$, if $S$ meets the minimality condition (principle 1) and $\catc(S)$ meets the goodness condition 2B then $S$ meets the condtions of Codd's third normal form.
\item In addition to (i), if for each designated mono-source $<m_1, ...m_n>$ of the associated logical sketch,
each $m_i$ is an edge then the data specification $S$ meets the conditions of Boyce-Codd normal form (BCNF).
\item In addition to (i), if we follow principle 1 and \textbf{do not introduce
limits into a sketch needlessly} then the data specification $S$ meets the 
fourth and fifth normal form criteria of Fagin.
\end{enumerate}
\end{lemma}
\pause \begin{block}{Significance}
We have defined criteria which are generic in the sense that they apply to any kind of data specification. They genericise the classic relational normal form criteria.
\end{block}
\end{frame}

\begin{frame}{Definition: \catcw \textit{maximally constrained} to $\reqtc$}
\begin{itemize}
\item  Question -- is there a $\catcp$ that extends \catcw and that will do a better job. 
\item Is there a $\catcp$ and an $I: \catc \morph \catcp$  such that 
all instances in the requirement $\reqtc$ uniquely factor though $I$
$$
\begin{array} {c p{2cm} c}
\Rnode{Cp}{C'} && \\ [0.25cm]
             && \Rnode{finset}{\Par} \\ [0.15cm]
\Rnode{C}{C}  
\end{array}
\begin{arrows}
\ncarr {C}{finset}
\alabel{D}
\ncarr{C}{Cp}
\alabel{I}
\ncarr{Cp}{finset}
\alabel{D'} 
\end{arrows}
$$
and at least one other instance $F$ of \catcw does not factor through $I$.
\ncarr[-20]{C}{finset}
\blabel{F} 
\end{itemize}
\pause If there is no such $I: \catc \morph \catcp$ then we shall say that 
\catcw is \textit{maximally constrained} with respect to $\reqtc$.\\
\medskip
 ...meaning  that structured category \catcw  is the tightest possible fit to facts i.e. to the requirement $R_{\catc}$.
\end{frame}

\begin{frame}{Principles imply specific criteria}
\begin{itemize}
\pause \item We would like to show (the grand plan) that sketch $S$ meets 
Principle 1 (minimality of the sketch) and if $\catc(S)$  meets 
Principle 2 (that it should be maximally constrained) then it also meets specific representational completeness criteria 2A, 2B, 2C and so on. 
\pause \item If we can get to this then  we have fundamental principles which are both generic across all kinds of data specifications and which imply the 
specific representation completeness criteria which in turn imply the classic relational normal forms.
\end{itemize}
\end{frame}

