
% The following command has a command passed to it as parameter.
% The expected value of the parameter is either the ncarr or the ncsar command.
% This command is used to draw the arrows from student -> department
% and professor -> department.
% In this way this command can be used to draw a sketch of a catagory 
% or a sketch of a contextual category.
\newcommand{\sketchgraph}[1]
{
\begin{displaymath}
\begin{array}{c c c p{1cm} c}
  & \makebox[2cm][c]{\Rnode{department}{department}\Rnode{departmentRight}{}}  & \\[2cm]
\makebox[1cm][c]{\Rnode{student}{student}} 
      && \makebox[1cm][c]{\Rnode{professor}{professor}}
      && \Rnode{v}{v} \\[0.8cm]  %was 1.2cm
\end{array}
\begin{arrows}
\setlength{\arrnodesepA}{4pt}  %2
\setlength{\arroffsetA}{-3pt}  %0
\ncarr[15]{departmentRight}{v}
\alabel{dName}[0.3]
\addedgebar
\arreset
\ncarr[15]{department}{professor}
\alabel{h}[0.3]
#1{student}{department}
\alabel{d}[0.3]
\ncarr[-50]{student}{v}
\blabel{sName}[0.2]
\addedgebar
\ncarr{student}{professor}
\blabel{s}[0.3]
#1[15]{professor}{department}
\alabel{d'}[0.3]
\addedgebar
\ncarr[15]{professor}{v}
\alabel{pId}[0.3]
\addedgebar
\ncarr[-15]{professor}{v}
\blabel{pName}[0.3]
\end{arrows}
\end{displaymath}
}



\begin{frame}{Resulting Logical Data Specification}
\sketchgraph{\ncarr}

\begin{block}{subject to commutivity of }
\vspace{0.25cm}
\studentProfessorDepartmentCommutingDiagrams{\ncarr}
\vspace{0.25cm}
\end{block}
\end{frame}


