
\newcommand{\studentDepartmentAttribute}[1]             {\onslide<1-5>{#1}}
\newcommand{\studentDepartmentAttributeStrikeout}[1]    {\onslide<5>{#1}}
\newcommand{\popupAbstractionOfd}[1]                    {\only   <2-6>{#1}}
\newcommand{\popupAbstractionOfdDetail}[1]              {\onslide<3-6>{#1}}
\newcommand{\studentDepartmentRelationship}[1]          {\onslide<4->{#1}}

\newcommand{\professorDepartmentAttribute}[1]           {\onslide<1-10>{#1}}
\newcommand{\professorDepartmentAttributeStrikeout}[1]  {\onslide<10>{#1}}
\newcommand{\popupAbstractionOfdPrime}[1]               {\only   <7-11>{#1}}
\newcommand{\popupAbstractionOfdPrimeDetail}[1]         {\onslide<8-11>{#1}}
\newcommand{\professorDepartmentRelationship}[1]        {\onslide<9->{#1}}
\newcommand{\professorDepartmentRelationshipBar}[1]     {\onslide<11>{#1}}

\newcommand{\studentProfessorAttribute}[1]              {\onslide<1-15>{#1}}
\newcommand{\studentProfessorAttributeStrikeout}[1]     {\onslide<15>{#1}}
\newcommand{\popupAbstractionOfs}[1]                    {\only   <12-16>{#1}}
\newcommand{\popupAbstractionOfsDetail}[1]              {\onslide<13-16>{#1}}
\newcommand{\studentProfessorRelationship}[1]           {\onslide<14->{#1}}

\newcommand{\departmentProfessorAttribute}[1]           {\onslide<1-20>{#1}}
\newcommand{\departmentProfessorAttributeStrikeout}[1]  {\onslide<20>{#1}}
\newcommand{\popupAbstractionOfhd}[1]                   {\only   <17-21>{#1}}
\newcommand{\popupAbstractionOfhdDetail}[1]             {\onslide<18-21>{#1}}
\newcommand{\departmentProfessorRelationship}[1]        {\onslide<19->{#1}}


\newcommand{\professorStudentDepartmentRelationalSchematic}
{
\begin{tabular}{c p{0.25cm} c p{0.25cm} c}
$\begin{array}{c p{0.3cm} c p{0.1cm} c}
\multicolumn{5}{c}{\Rnode{student}{student}} \\[-0.3cm]
\multicolumn{5}{c}{\Rnode{sB1}{}\hspace{0.5cm}\Rnode{sB2}{}\hspace{0.5cm}\Rnode{sB3}{}} \\[1.0cm]
\Rnode{sv1}{v} && \studentDepartmentAttribute{\Rnode{sv2}{v}} && \studentProfessorAttribute{\Rnode{sv3}{v}}
\end{array}
\begin{arrows}
\ncarr{sB1}{sv1}
\alabel{name}[0.55][0.1]
\addedgebar
\studentDepartmentAttribute{
  \ncarr{sB2}{sv2}
  \alabel{dept}[0.57][0.1]
  \studentDepartmentAttributeStrikeout{\ncstrikeout{sB2}{sv2}}
}
\studentProfessorAttribute{
  \ncarr{sB3}{sv3}
  \alabel{svr}[0.6][0.1]
  \studentProfessorAttributeStrikeout{\ncstrikeout{sB3}{sv3}}
}
\end{arrows}$
&&
$\begin{array}{c p{0.3cm} c p{0.1cm} c}
\multicolumn{5}{c}{\Rnode{professor}{professor}} \\[-0.3cm]
\multicolumn{5}{c}{\Rnode{pB1}{}\hspace{0.5cm}\Rnode{pB2}{}\hspace{0.5cm}\Rnode{pB3}{}} \\[1.0cm]
\professorDepartmentAttribute{\Rnode{pv1}{v}} && \Rnode{pv2}{v} && \Rnode{pv3}{v} 
\end{array}
\begin{arrows}
\professorDepartmentAttribute{
  \ncarr{pB1}{pv1}
  \alabel{dept}[0.5][0.1]
  \addedgebar
  \professorDepartmentAttributeStrikeout{\ncstrikeout{pB1}{pv1}}
}
\ncarr{pB2}{pv2}
\alabel{no}[0.575][0.1]
\addedgebar
\ncarr{pB3}{pv3}
\alabel{name}[0.6][0.1]
\end{arrows}$
&&
$\begin{array}{c p{0.2cm} c }
\multicolumn{3}{c}{\Rnode{department}{department}} \\[-0.3cm]
\multicolumn{3}{c}{\Rnode{dB1}{}\hspace{0.5cm}\Rnode{dB2}{}} \\[1.0cm]
\hspace{0.2cm}\Rnode{dv1}{v} && \departmentProfessorAttribute{\Rnode{dv2}{v}} 
\end{array}
\begin{arrows}
\ncarr{dB1}{dv1}
\alabel{name}[0.55][0.1]
\addedgebar
\departmentProfessorAttribute{
  \ncarr{dB2}{dv2}
  \alabel{hd}[0.575][0.1]
  \departmentProfessorAttributeStrikeout{\ncstrikeout{dB2}{dv2}}
}
\end{arrows}$
\end{tabular}
}

\newcommand{\abstractionOfd}
{
 establishes a functional relationship $d: student \morph department$ such that
$\composeSevenShaped[0.85cm]{student}{department}{v}{d}{name}{dept}$
commutes.
}

\newcommand{\abstractionOfdPrime}
{
 establishes a functional relationship $d': professor \morph department$ such that
$\composeSevenShaped[0.85cm]{professor}{department}{v}{d'}{name}{dept}$ commutes. 
}

\newcommand{\abstractionOfs}
{
establishes a functional relationship $s: student \morph professor$ such that 
$\composeSevenShaped[0.85cm]{student}{professor}{v} {s}{dept}{dept}$ 
and
$\composeSevenShaped[0.85cm]{student}{professor}{v}{s}{no}{svr}$ commute.

The commutivity of left hand triangle simplifies to commutivity of
$\composeSevenShaped[0.85cm]{student}{professor}{department}{s}{d'}{d}$
.
}

\newcommand{\abstractionOfhd}
{
establishes a functional relationship
$department \morph professor$ such that 
$\composeSevenShaped[0.85cm]{department}{professor}{v}{h}{dept}{name}$ commute and
$\composeSevenShaped[0.85cm]{department}{professor}{v}{h}{no}{hd}$.
The commutivity of left hand triangle simplifies to commutivity of
$\composeSevenShaped[0.85cm]{department}{professor}{department}{h}{d'}{id}$
.
}


\newcommand{\abstractionToSketch}
{
\popupAbstractionOfd{
\begin{block}{$\studentDeptInclusionDependency$}
\popupAbstractionOfdDetail{\abstractionOfd}
\end{block}
}
\popupAbstractionOfdPrime{
\begin{block}{$\professorDeptInclusionDependency$}
\popupAbstractionOfdPrimeDetail{\abstractionOfdPrime}
\end{block}
}
\popupAbstractionOfs{
\begin{block}{$\studentSupervisorInclusionDependency$}
\popupAbstractionOfsDetail{\abstractionOfs}
\end{block}
}
\popupAbstractionOfhd{
\begin{block}{$\headOfDeptInclusionDependency$}
\popupAbstractionOfhdDetail{\abstractionOfhd}
\end{block}
}
}


\begin{frame}{Relational Data Schematic}
\vspace{0.9cm}
\professorStudentDepartmentRelationalSchematic
$\begin{arrows}
\studentDepartmentRelationship{
  \ncarr[20]{student}{department}
  \alabel{d}[0.4]
}
\studentProfessorRelationship{
  \ncarr[5]{student}{professor}
  \alabel{s}[0.4]
}
\professorDepartmentRelationship{
  \ncarr[5]{professor}{department}
  \alabel{d'}[0.4]
  \professorDepartmentRelationshipBar{\addedgebar}
}
\departmentProfessorRelationship{
  \ncarr[5]{department}{professor}
  \alabel{hd}[0.4]
}\end{arrows}
$

\begin{overlayarea}{11cm}{7cm}
\abstractionToSketch
\onslide<20>{END}%need to force slide 20 to appear without hd attribute ??
\end{overlayarea}

\end{frame}

