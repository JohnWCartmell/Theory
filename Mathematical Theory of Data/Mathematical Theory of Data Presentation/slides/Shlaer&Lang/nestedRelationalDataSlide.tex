

\newcommand{\nestedStudentSupervisorInclusionDependency}
{student[..,svr] \subseteq professor[..,pId]}

\newcommand{\nestedHeadOfDeptInclusionDependency}
{department[identity,hd]  \subseteq professor[..,pId]}

\begin{frame}{Nested Relational Data -- Same information as before.}

\newcommand{\professorAwidth}{35pt}
\newcommand{\professorBwidth}{25pt}
\newcommand{\studentAwidth}{35pt}
\newcommand{\studentBwidth}{25pt}

\begin{tabular} {|l|l|l|l|}
\hline
\rowcolor{myblue}\multicolumn{4}{l}{\colhead{department}} \\
\hline
\rowcolor{myblue}\colhead{\pk{name}} & \colhead{hd}  &
\begin{tabular}{|p{\studentAwidth}|p{\studentBwidth}|}
\multicolumn{2}{c}{\colhead{student}} \\
\hline
	\colhead{\pk{name}} & \colhead{svr} \\
\hline
\end{tabular} &
\begin{tabular}{|p{\professorAwidth}|p{\professorBwidth}|}
\multicolumn{2}{c}{\colhead{professor}} \\
\hline
	\colhead{\pk{no}} & \colhead{name} \\
\hline
\end{tabular} \\
\hline
maths 	&\#3  & 
\begin{tabular}{|p{\studentAwidth}|p{\studentBwidth}|}
\vpad{2}
\hline
bohm & \#1 \\
\hline
smith & \#2 \\
\hline
\vpad{2}
\end{tabular} &
\begin{tabular}{|p{\professorAwidth}|p{\professorBwidth}|}
\vpad{2}
\hline
\#1 & scott \\
\hline
\#2 & smith  \\
\hline
\#3 & gandy  \\
\hline
\vpad{2}
\end{tabular} \\
\hline
phil  	&\#1 & 
\begin{tabular}{|p{\studentAwidth}|p{\studentBwidth}|}
\vpad{2}
\hline
gray & \#1 \\
\hline
doe & \#1 \\
\hline
\vpad{2}
\end{tabular} &
\begin{tabular}{|p{\professorAwidth}|p{\professorBwidth}|}
\vpad{2}
\hline
\#1 & smith \\
\hline
\#2 & ayer  \\
\hline
\vpad{2}
\end{tabular}  \\
\hline
history &\#5  & $\hdots$ & $\hdots$ \\
\hline
physics &\#1  & $\hdots$ & $\hdots$ \\
\hline
\end{tabular}


\begin{overlayarea}{12cm}{5cm}
\only<1>{
\begin{align*}
\nestedStudentSupervisorInclusionDependency\\
\nestedHeadOfDeptInclusionDependency  
\end{align*}
}
\only<2>{
\begin{block}{what we see here  -- a combination of}
\begin{itemize}
    \item structural containment 
	\item relational referencing.
\end{itemize}
\end{block}
}
\only<3>{
\begin{block}{this is all there is}
\begin{itemize}
	\item the sole mechanisms for representing internal relationships in data are 
	\begin{itemize}
	    \item structural containment 
		\item relational referencing.
	\end{itemize}
	\item \raisebox{0.04cm}{$\therefore$} all data can be viewed abstractly as nested relational,
\end{itemize}
\end{block}
}
\end{overlayarea}
\end{frame}
