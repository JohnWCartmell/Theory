
This is a second variation of the preliminary model - the situation is as for variation one in that  each person may only be selected by a single other individual and a person may select themselves in which case the person will be in a team of their own but persons may elect not to participate in which case
in defining instances $E$, the relationship $s$ and the attribute $c$ are both undefined.

\begin{figure} [h]
\begin{center}
\begin{tabular}{p{3.5cm} c}
\begin{tabular}{c p{1.5cm} c}
   \Rnode{p}{p} & & \Rnode{v}{v}
\end{tabular}
%\nccircle[nodesep=3pt]{<-}{p}{.4cm}
\rEpm[270]{p}
\alabel{S}[0.6]
\Ep[-40]{p}{v}
\blabel{c}[0.6]
\Etm{p}{v} 
\alabel{pId}[0.6]
\idcomp
& \footnotesize
\begin{tabular}{c p{1.5cm} p{4cm}}
KEY && \\
\hline
p & person & Identified by id attribute ($pId$). \\
s & selects & each person selects exactly one other person \\
c & colour & each person is given a coloured vest 
\end{tabular} 
\end{tabular}
\end{center}
\caption{Team Selection Example  - Variant Two. The  vest colour  of a person is 
identical to the vest colour of their selected buddy which is to say that the path equivalence $S \circ c \simeq c$ holds. From this
we can show that this model exhibits a transitive functional dependency which does not factor through an intransitive dependency.
}
\label{teamselectionvarianttwoERschema}
\end{figure}

\begin{enumerate}
\item
A person may not select a person who has eleccted out 
Now the fd factoring property does hold because though we have $S \morph S/S \morph S/S/S ... \morph c$ 
we do have $... S/S/S \morph S/S \morph S$.
\item
a person may select a person who has elected out
now we do not have $S/S \morph S$ but then neither do we have $S \morph S/S$ or $S/S \morph S/S/S$ or so on and so we do have the fd factoring property.
\end{enumerate}

