\subsection{Team Selection - Variation Two}
\llabel{teamselectionvariationtwo}
\commentary{This variation of earlier models is an example in which the fd factoring property holds but the model is not \textit{reduced}. The $\chi$-generated relational model is neither in BCNF nor in 3NF. }
This is a second variation of the preliminary model - the situation is as for variation one in that  each person may only be selected by a single other individual and a person may select themselves in which case the person will be in a team of their own but persons may elect not to participate in which case
in defining instances $E$, the relationship $s$ and the attribute $c$ are both undefined.

\begin{figure} [h]
\begin{center}
\begin{tabular}{p{3.5cm} c}
\begin{tabular}{c p{1.5cm} c}
   \Rnode{p}{p} & & \Rnode{v}{v}
\end{tabular}
%\nccircle[nodesep=3pt]{<-}{p}{.4cm}
\rEpm[270]{p}
\alabel{S}[0.6]
\Ep[-40]{p}{v}
\blabel{c}[0.6]
\Etm{p}{v} 
\alabel{pId}[0.6]
\idcomp
& \footnotesize
\begin{tabular}{c p{1.5cm} p{4cm}}
KEY && \\
\hline
p & person & Identified by id attribute ($pId$). \\
s & selects & each person selects exactly one other person \\
c & colour & each person is given a coloured vest 
\end{tabular} 
\end{tabular}
\end{center}
\caption{Team Selection Example  - Variant Two. The example is similar to the previous one (example \ref{teamselectionvariantone})
but is different in that the relationship $S$ is not total.
}
\label{teamselectionvarianttwoERschema}
\end{figure}

Assume that\footnote{If we assume otherwise that a person may select a person who has elected out then we have $S \circ c \leq c$ rather than $S \circ c \simeq c$.
and we do not have a situation where $S \morph S/S$ or $S/S \morph S/S/S$ or so on.  The fd factoring property holds and in the relational design we do not
have a functional dependency $\ssfd{spId}{c}$. The relational design is in BCNF.}
the situation is that a person may not select a person who has elected out. In this case there is a  path equivalence $c \simeq S \circ c$.
The fd factoring property holds because as in the previous example though we have $S \morph S/S \morph S/S/S ... \morph c$ 
but unlike the previous example (????) we also have $... S/S/S \morph S/S \morph S$. \commentary{Proof?}

\subsubsection{$\chi$-generated Relational Design}
As in example \ref{teamselectionpreliminaryexample} the $\chi$-generated relational design consists of a single relation:
\begin{equation}
\label{personrelation}
person(\underline{pId}, spId, c)
\end{equation}
and in addition to those implied by the fact that $pId$ is key, the  attributes of this relation are related by the following  functional dependencies:
\begin{equation}
\ssfd{spId}{c}
\end{equation}
and the following  inclusion dependency:
\begin{equation}
\label{spIdcolour1}
person[spId,c] \subseteq person[pId,c]
\end{equation}


The $person$ relation (\ref{personrelation}) is not in BCNF nor in 3NF because the dependency (\ref{colourfd}) is transitive but its rhs, $c$, is not a key attribute wrt the relation\commentary{This uses Codd's way of describing 3NF not Zaniolo's.}. 
In example \lref{teamselectionvarianttworevised} we can show, significantly,  that a normalised version of this relational schema, one that is in both TNF and BCNF, is 
$\chi$-generated from an ER model that is this  model specifically revised to achieve \newt{the condition for being reduced}.



