\begin{figure} [h] % data table
\begin{center}
\begin{tabular}{c c}
$
\begin{array}{cp{0.75cm}cp{0.75cm}c}
   \Rnode{r}{r}     & & \Rnode{t}{t} & & \Rnode{v}{v} \\[1.2cm]     
	 \Rnode{d}{d}   & & \Rnode{c}{c} & &               
\end{array}
$
\ncarr{r}{t} 
\alabel{S_1}
\idcomp
\ncarr{t}{v} 
\alabel{tN}
\idcomp
\ncarr{c}{v} 
\blabel{cN}
\idcomp
\ncarr{d}{c}
\blabel{R_0}
\idcomp
\ncarr{d}{r}
\alabel{S_0}
\idcomp
\ncarr{c}{t}
\blabel{R_1}
\idcomp
\ncarr[50]{r}{v}
\alabel{rN}
\idcomp

& \footnotesize
\begin{tabular}{c p{1.5cm} p{4cm}}
KEY && \\
\hline
t & table & Having identifying attribute tN the name of the table. \\
c & column & Identified by a combination of column number cN and relationship $R_1$ to the table it is a column of.\\
r & row & Identified by its row number $rN$ and its relationship $S_1$ to the table it is a row of.\\
d & data cell & Identified by relationship $S_0$ to the row it is in and relationship $R_0$ to the row it is in. \\
\end{tabular} 
\end{tabular}
\end{center}
\caption{Example of path equivalence - the path $\tuple{R_0,R_1}$ is equivalent to path $\tuple{S_0,S_1}$}.
\label{datatablegraph}
\end{figure}