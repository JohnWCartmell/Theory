\subsection{Team Selection - Revised Model}
\llabel{teamselectionrevised}

The situation modelled in this example is exactly that described in the previous example (see example \lref{teamselectionpreliminary}). The ER model though is different in that the ER schema used to represent the data  is a revised ER schema modified specifically so that in this model the fd factoring property holds with the consequence that the $\chi$-generated relational schema is in TNF and BCNF;  in fact the $\chi$-generated relational schema in this example is a normalised version of the $\chi$-generated relational schema from that 
previous example (as described in section \lref{teamselectionpreliminaryrelationalschema}).

\subsubsection{The ER Schema}
The ER schema in this example is shown in figure \ref{teamselectionrevisedERschema}. It is the ER schema of example \lref{teamselectionpreliminary} (see figure 
\lref{teamselectionpreliminaryERschema}) with an additional entity type $sp$ to represent persons who have been selected by some other person.


\begin{figure} [h]
\begin{center}
\begin{tabular}{p{4cm}  l }
\begin{tabular}{c p{1.5cm} c}
   \Rnode{sp}{\rnode{sps}{s}\rnode{spp}{p}} & &    \\[1.4cm]
   \Rnode{p}{p}   & & \Rnode{v}{v}
\end{tabular}
%\setlength{\sarnodesepA}{-1pt}
%\setlength{\sarnodesepB}{-3pt}
\Ete[40][90]{p}{sps}
\alabel{S}
\Etm[0][-90]{sp}{p}
\alabel{I}
\idcomp
\Et[20][-35]{spp}{v}
\alabel{c}[0.4]
\Etm{p}{v} 
\blabel{pId}
\idcomp &\footnotesize
\begin{tabular}{c p{1.5cm} p{4cm}}
KEY && \\
\hline
p & person & Identified by id attribute ($pId$). \\
sp & selected & a $selected$ person is identified by their inclusion relationship $I$ to the type $person$ \\
S & selects & each $person$ selects exactly one other $selected$ person \\
I & inclusion & each $selected$ person is a $person$ \\
c & colour & each $person$ wears a vest of this colour 
\end{tabular} 
\end{tabular}
\end{center}
\caption{Team Selection Example Revised. 
In the situation modelled all persons have a coloured vest and since this colour is the same as the colour 
worn by the person they select it is this latter colour that is included in the schema and which will therefore to be held as data. 
This schema uses the constraints known of the situation to reduce the amount of data held.
Nonetheless the  vest colour  of a selected person is 
identical to the vest colour of the person they themselves select which is to say that there is a  path equivalence $I \circ S \circ c \simeq c$ in this model.
}
\label{teamselectionrevisedERschema}
\end{figure}

\begin{categoricalaside}

The relationship $S$ of the preliminary model has been epi-mono split into $S$ which is now an epimorphism in the sense that it is onto in all defining instances,
and $I$ which is a monomorphism. The attribute $c$ has been factored through $S$.
\end{categoricalaside}

\subsubsection{Analysis of Functional Dependencies}
 
We show that in this model, compared to the previous model,  there is still
 an intransitive functional dependency between attribute-like paths that cannot be factored right intransitively but that this functional dependency is one in which the left hand side 
is a mono-source; the fd factoring property still holds and,
accordingly, the $\chi$-generated relational schema from this model is in TNF and BCNF.

\commentary{This model meets the conditions of theorem \lref{maintheorem}; is simple without caveats and so the relational model 
determined by the $\chi$ mapping is in BCNF.}

Similar reasoning to that given in section \lref{teamselectionpreliminaryanalysis} can be used to establish that 
in this model we have the following sequence of functional dependencies:
\begin{equation}
I/pId \morph I/S \morph I/S/I/S \morph I/S/I/S/I/S ... \morph c
\end{equation}
and that $I/pId \morph c$ is a transitive functional dependency.  $I/pId \morph c$ is therfore a transitive functional dependency between attribute-like paths which cannot
be factored right-intransitively. However  the left hand side, $I/pId$, is a mono-source and so the condition for the model to have the fd factoring property is not violated.

Note that $\msfd{I/S/I/pId}{c}$ is a functional dependency between  paths that cannot be factorised on the right though an intransitive functional dependency. 
 It is transitive because it factors as $\msfd{I/S/I/pId}{I/S/I/S}$ and $\msfd{I/S/I/S}{c}$ and neither of these are reversible. The path lhs ($I/S/I/pId$) isn't a mono-source 
however neither is it  because $S$ is not identifying and so the condition for the model to have the fd factoring proeprty is not violated.

\subsubsection{The $\chi$-generated Relational Schema}
The $\chi$ mapping applied to this model produces relations (\ref{person1relation}) and (\ref{person2relation}):
\begin{equation}
\label{person1relation}
selected(\underline{spId},  c)
\end{equation}
\begin{equation}
\label{person2relation}
person(\underline{pId}, spId)
\end{equation}
satisfying the inclusion dependencies
\begin{equation}
selected[spId] \subseteq person[pId]
\end{equation}
and
\begin{equation}
\label{spIdselectedcolour}
selected[spId,c] \subseteq (selected \bowtie_{spId=pId} 
                                       (person \bowtie selected [pId,c]) )
																			[spId,c];
\end{equation}
This relational schema is in TNF and BCNF however, to be fair,
it is difficult not to question the efficacy of this normalised schema compared to the unnormalised schema (section \ref{teamselectionpreliminaryrelationalschema}).
In addition,
the inclusion dependency (\ref{spIdselectedcolour}) is a database constraint that
is more difficult to police in an implementation
than the prior equivalent (\ref{spIdcolour1}) in the unnormalised schema and
this too gives advantages to the unnormalised schema. 


