\begin{figure} [h]
\begin{center}
\begin{tabular}{c c}
$
\begin{array}{cp{0.75cm}cp{0.75cm}c}
   \Rnode{fk}{fk}     & & \Rnode{t}{t} & & \Rnode{v}{v} \\[1.2cm]     
	 \Rnode{fkc}{fkc}   & & \Rnode{c}{c} & &               
\end{array}
$
\ncarr{fk}{t} 
\alabel{S_1}
\ncarr{t}{v} 
\alabel{tN}
\idcomp
\ncarr{c}{v} 
\blabel{cN}
\idcomp
\ncarr{fkc}{c}
\blabel{R_0}
\ncarr{fkc}{fk}
\alabel{S_0}
\idcomp
\ncarr{c}{t}
\blabel{R_1}
\idcomp
\ncarr[50]{fk}{v}
\alabel{FkN}
\idcomp
\ncarr[-90]{fkc}{v}
\blabel{SeqN}
\idcomp
& \footnotesize
\begin{tabular}{c p{1.5cm} p{4cm}}
KEY && \\
\hline
t & table & Having identifying attribute tN the name of the table. \\
c & column & Identified by a combination of column name cN and relationship $R_1$ to the table it is a column of.\\
fk & foreign key & Identified by its name $FkN$.\\
fkc & foreign key column & Identified by relationship $S_0$ to the foreign key it is a part of and its sequence number $SeqN$ i.e. the position it appears in within the foreign key. \\
\end{tabular} 
\end{tabular}
\end{center}
\caption{Example of path equivalence - the path $\tuple{R_0,R_1}$ is equivalent to path $\tuple{S_0,S_1}$.}
\label{foreignkeygraph}
\end{figure}