
\begin{figure} [h]  % 
\begin{center}
\begin{tabular}{c c}
$
\begin{array}{cp{0.7cm}c  p{0.7cm}c }
                & & \Rnode{b1}{b_1} & &                \\ [1.2cm]    
	 \Rnode{a}{a} & & \Rnode{c}{c}    & &    \Rnode{v}{v}\\ [1.2cm]  
					      & & \Rnode{b2}{b_2} & &                 
\end{array}
$
\ncarr{a}{b1} 
\alabel{S_1}
\ncarr{b1}{v} 
\alabel{K_{b_1}}
\idcomp
\ncarr{c}{b1} 
\blabel{Q_1}
\idcomp
\ncarr{a}{b2} 
\blabel{S_2}
\ncarr{b2}{v} 
\blabel{K_{b_2}}
\idcomp
\ncarr{c}{b2} 
\alabel{Q_2}
\idcomp
\ncline[linestyle=dashed,nodesepA=\arrnodesepA,nodesepB=\arrnodesepB]{->}{a}{c} 
\blabel{R}
\nccurve[angleA=-90,angleB=-90,nodesep=2pt,ncurv=1.6]{->}{a}{v}
\blabel{K_a}[0.3][-1]
\idcomp
& \footnotesize
\end{tabular}
\end{center}
\caption{A schema for an ER model for which we suppose  $\tuple{R,Q_1} < \tuple{S_1}$ and $\tuple{R,Q_2} < \tuple{S_2}$ and for which we ask the question does property X hold?}
\label{propertyXfailureexample}
\end{figure} 
Consider a model $\gmodel$ with the schema shown in figure \ref{propertyXfailureexample}. It may or may not be the case 
that there is a referential  inclusion dependency $\incd{a}{S_1,S_2}{c}{Q_1,Q_2}$ in $\gmodel$ and if there is such an inclusion dependency it may or may not be the case that relationship $R$ represents this inclusion dependency.  This gives us three possibilities to consider
and we discuss these as the three variations below.
\subsection {Variation One}
Consider a model $\gmodel$ with the schema shown in figure \ref{propertyXfailureexample} and suppose that $\incd{a}{S_1,S_2}{c}{Q_1,Q_2}$ in $\gmodel$ and that this referential inclusion dependency is represented by relationship $R$. 
In this case the model $\gmodel$  has property X because even though each $R \circ Q_i$ is dominated there is a referential inclusion dependency that it represents. The first-cut relational model gnerated from this schema will  not be in third normal form but
the $\chi$ transform does produce a realtional schema in normal form as it omits attributes $\qq{R/Q_1/K_{b_1}}$ and $\qq{R/Q_2/K_{b_2}}$ that are included in the first-cut relational schema  definition of the relation representing entity type $a$. 
They are excluded because each path $R/Q_1/K_{b_1}$ and $R/Q_2/K_{b_2}$  is dominated in $\gmodel$.
\subsection{Variation  Two}
Consider a model $\gmodel$ with the schema shown in figure \ref{propertyXfailureexample} and suppose that $\incd{a}{S_1,S_2}{c}{Q_1,Q_2}$ in $\gmodel$ and that this referential inclusion dependency is not represented by relationship $R$.
In this case the model $\gmodel$ is neither well-formed nor has property X. 
For it to  be well-formed there ought be a relationship, $S$, say,
$S:a \morph c$ that represents referential inclusion dependency $\incd{a}{S_1,S_2}{c}{Q_1,Q_2}$. Consider then a model
which includes in its schema  this additional relationship $S$ \commentary{For completeness should define $E_S$, for each instance $E$}representing the referential inclusion dependency $\incd{a}{S_1,S_2}{c}{Q_1,Q_2}$.\\

\begin{tabular}{ p{8.5cm}  c}
We then have that $R \leq S$, for in each instance $E$ we have&\parbox{5cm}{ \begin{align*}
E_R&=E_R \circ E_{\tuple{Q_1,Q_2}} \circ E^{-1}_{\tuple{Q_1,Q_2}} \\
   & \leq E_{\tuple{S_1,S_2}} \circ E^{-1}_{\tuple{Q_1,Q2}} \\
	 & = E_{S}
\end{align*}}
\end{tabular}

Property X still does not hold of  model as extended with relationship $S$. The  preferred way to achieve property X in  case like this is to replace relationship $R: a \morph b$ by a relationship $barR: a \morph a$ and define in each defining instance $E$,
\begin{equation}
E_{barR} = \overline{E_R}
\end{equation}

The modified model with schema as shown in figure \ref{propertyXfailurecorrection}  now does have property X and it transforms via first-cut transform to a relational schema that is in third normal form.

\begin{figure} [h]  % 
\begin{center}
\begin{tabular}{c c}
$
\begin{array}{cp{0.7cm}c  p{0.7cm}c }
                & & \Rnode{b1}{b_1} & &                \\ [1.2cm]    
	 \Rnode{a}{a} & & \Rnode{c}{c}    & &    \Rnode{v}{v}\\ [1.2cm]  
					      & & \Rnode{b2}{b_2} & &                 
\end{array}
$
\nccircle[linestyle=dashed,angleA=90, nodesep=3pt]{<-}{a}{.4cm}
\blabel{barR}[0.5]
\ncarr{a}{b1} 
\alabel{S_1}
\ncarr{b1}{v} 
\alabel{K_{b_1}}
\idcomp
\ncarr{c}{b1} 
\blabel{Q_1}
\idcomp
\ncarr{a}{b2} 
\blabel{S_2}
\ncarr{b2}{v} 
\blabel{K_{b_2}}
\idcomp
\ncarr{c}{b2} 
\alabel{Q_2}
\idcomp
\ncline[linestyle=dashed,nodesepA=\arrnodesepA,nodesepB=\arrnodesepB]{->}{a}{c} 
\blabel{\hat{R}}
\nccurve[angleA=-90,angleB=-90,nodesep=2pt,ncurv=1.6]{->}{a}{v}
\blabel{K_a}[0.3][-1]
\idcomp
& \footnotesize
\end{tabular}
\end{center}
\captionsetup{singlelinecheck=off}
 \caption[.]{Preferred --- An equivalent model which is well-formed. Relationship $R:a \morph c$ is navigated
as $barR \circ \hat{R}$. This schema has property X  since relationship $\hat{R}$ is implemented by $S_1,S_2$. The  relational schema that represents this model will include a table $a$ as follows:
\begin{equation}
a(\underline{K_a},\qq{S_1/K_{b_1}},\qq{S_2/K_{b_2}}, \qq{barR/K_a})
\end{equation} }
\label{propertyXfailurecorrection}
\end{figure}

Note that the first-cut relational schema for original model $\gmodel$ will include a table $a$ which will not be in S3NF for it
will be as follows:
\begin{equation}
a(\underline{K_a},\qq{S_1/K_{b_1}},\qq{S_2/K_{b_2}}, \qq{R/Q_1/K_{b_1}}, \qq{R/Q_2/K_{b_2}})
\end{equation}
and will have the following restrictions
\begin{align}
\qq{R/Q_1/K_{b_1}} = \overline{\qq{R/Q_2/K_{b_2}}} \circ \qq{S_1/K_{b_1}} \\
\qq{R/Q_2/K_{b_2}} = \overline{\qq{R/Q_1/K_{b_1}}} \circ \qq{S_2/K_{b_2}}
\end{align}
and therefore the following functional dependencies:
\begin{align} 
\msfd{\qq{R/Q_1/K_{b_1}},\qq{S_2/K_{b_2}}}{\qq{R/Q_2/K_{b_2}}} \\
\msfd{\qq{R/Q_2/K_{b_2}},\qq{S_1/K_{b_1}}}{\qq{R/Q_1/K_{b_1}}}
\end{align}
and has table $a$ therefore that is not in S3NF.
If the first cut model is normalised by addressing the first of these functional dependencies then we 
introduce a table $a'$ and move $\qq{R/Q_2/K_{b_2}}$ to table $a'$ so that table $a$ is replaced by:
\begin{align}
a(\underline{K_a},\qq{S_1/K_{b_1}},\qq{S_2/K_{b_2}}, \qq{R/Q_1/K_{b_1}}) \\
a'(\underline{\qq{R/Q_1/K_{b_1}}},\underline{\qq{S_2/K_{b_2}}}, \qq{R/Q_2/K_{b_2}} )
\end{align}
but wierdly now, on table $a'$,  $\qq{S_2/K_{b_2}} \simeq \qq{R/Q_2/K_{b_2}}$ and so table $a'$ becomes
\begin{align}
a'(\underline{\qq{R/Q_1/K_{b_1}}},\underline{\qq{S_2/K_{b_2}}} )
\end{align}
abstracting to a logical model we get a model with shcema as shown in figure \ref{propertyXfailurenormalisedandabstracted}.

\begin{figure} [H]  % 
\begin{center}
\begin{tabular}{c c}
$
\begin{array}{cp{0.7cm}c p{0.9cm}c p{0.7cm}c}
                & & \Rnode{b1}{b_1} & &                & &                 \\ [1.2cm]    
	 \Rnode{a}{a} & & \Rnode{ap}{a'} & &  \Rnode{c}{c}  & &    \Rnode{v}{v} \\ [1.2cm]  
					      & & \Rnode{b2}{b_2} & &                & &             
\end{array}
$
\ncarr[30]{a}{b1} 
\alabel{\qq{R/Q_1}}
\ncarr{a}{b1} 
\blabel{S_1}
\ncarr[30]{b1}{v} 
\alabel{K_{b_1}}
\idcomp
\ncarr{c}{b1} 
\blabel{Q_1}[0.35]
\idcomp
\ncarr{a}{b2} 
\blabel{S_2}
\ncarr[-30]{b2}{v} 
\blabel{K_{b_2}}
\idcomp
\ncarr{c}{b2} 
\alabel{Q_2}[0.35]
\idcomp
\ncline[linestyle=dashed,nodesepA=\arrnodesepA,nodesepB=\arrnodesepB]{->>}{a}{ap} 
\blabel{R_e}
\ncarr{ap}{c}
\blabel{R_t}
\nccurve[angleA=-90,angleB=-90,nodesep=2pt,ncurv=1.1]{->}{a}{v}
\blabel{K_a}[0.3][-1]
\idcomp
\ncarr{ap}{b1}
\blabel{S'_1}[0.35]
\idcomp
\ncarr{ap}{b2}
\idcomp
\alabel{S'_2}[0.35]
& \footnotesize
\end{tabular}
\end{center}
\caption{Logical version after normalisation. To be well-formed there ought to be added a further relationship $\hat{R} : a \morph c$.
Relationship $R_e$ is represented by the inclusion dependency $\incd{a}{\qq{R/Q_1},S_2}{a'}{S'_1,S'_2}$.
Relationship $R_t$ must be a monomorphim. Equally the original relationship $R$ represents the inclusion dependency 
$\incd{a}{\qq{R/Q_1},S_2}{c}{Q_1,Q_2}$ and the table $a'$ isn't achieving anything. 
}
\label{propertyXfailurenormalisedandabstracted}
\end{figure}
\vspace{0.5cm}


\subsection{Variaton Three}
Now, in regard to the schema shown in figure \ref{propertyXfailureexample} such that $\tuple{R,Q_1} < \tuple{S_1}$ and $\tuple{R,Q_2} < \tuple{S_2}$, suppose that $a\tuple{S_1,S_2} \nsubseteq c\tuple{Q_1,Q_2}$ . Again property X fails to hold and we are led to add a relationship 
$barR$ to this model leading to the schema shown in figure \ref{propertyXfailurecorrection2}.
\begin{figure} [H]  % 
\begin{center}
\begin{tabular}{c c}
$
\begin{array}{cp{0.7cm}c  p{0.7cm}c }
                & & \Rnode{b1}{b_1} & &                \\ [1.2cm]    
	 \Rnode{a}{a} & & \Rnode{c}{c}    & &    \Rnode{v}{v}\\ [1.2cm]  
					      & & \Rnode{b2}{b_2} & &                 
\end{array}
$
\nccircle[linestyle=dashed,angleA=90, nodesep=3pt]{<-}{a}{.4cm}
\blabel{barR}[0.5]
\ncarr{a}{b1} 
\alabel{S_1}
\ncarr{b1}{v} 
\alabel{K_{b_1}}
\idcomp
\ncarr{c}{b1} 
\blabel{Q_1}
\idcomp
\ncarr{a}{b2} 
\blabel{S_2}
\ncarr{b2}{v} 
\blabel{K_{b_2}}
\idcomp
\ncarr{c}{b2} 
\alabel{Q_2}
\idcomp
\ncline[linestyle=dashed,nodesepA=\arrnodesepA,nodesepB=\arrnodesepB]{->}{a}{c} 
\blabel{R}
\nccurve[angleA=-90,angleB=-90,nodesep=2pt,ncurv=1.6]{->}{a}{v}
\blabel{K_a}[0.3][-1]
\idcomp
& \footnotesize
\end{tabular}
\end{center}
\captionsetup{singlelinecheck=off}
 \caption[.]{Preferred -- An equivalent model in the case that $a\tuple{S_1,S_2} \nsubseteq c\tuple{Q_1,Q_2}$. 
Now $R \circ Q_1 \simeq barR \circ S_1$ and $R \circ Q_2 \simeq barR \circ S_2$ and $R$ represents the inclusion dependency 
$\incd{a}{barR \circ S_1,barR \circ S_2}{c}{Q_1,Q_2}$. }
\label{propertyXfailurecorrection2}
\end{figure}
 I recognise, by the way, that $barR$ is nothing but a representation of a boolean attribute and that boolean attributes could be 
represented instead as absolute-valued attributes (ones which are optional i.e may be undefined for some entitites). 