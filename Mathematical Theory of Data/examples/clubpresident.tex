\begin{figure} [h]
\begin{center}
\begin{tabular}{c c}
$
\begin{array}{cp{0.4cm}cp{0.75cm}cp{0.75cm}c}
              &&               &&                &&               \\[0.25cm]
              &&               &&                &&               \\[0.25cm]
\Rnode{p}{p}	&& \Rnode{m}{m}  &&   \Rnode{c}{c} && \Rnode{v}{v}  \\[0.25cm]
	            &&  
\end{array}
$

\ncarr{p}{m} 
\alabel{I}
\idcomp
\ncarr[-35]{p}{v}
\blabel{yr}
\idcomp
\ncarr[30]{m}{v}
\alabel{mNo}
\idcomp
\ncarr{m}{c}
\blabel{M}
\idcomp
\ncarr{c}{v}
\blabel{cId}
\idcomp
& \footnotesize
\begin{tabular}{c p{1.5cm} p{4cm}}
KEY && \\
\hline
p  & president        & Identified by a combination of inclusion relationship ($I$) that identifies a
                       president as being a committee members and a year attribute ($yr$). \\
m  & member           & Identified by a member relationship ($M$) to a club and a by membership 
                         number attribute ($mNo$). \\
c  & club             & Identified by club identifier attribute ($cId$)
\end{tabular} 
\end{tabular}
\end{center}
\caption{Club president example. 
This is an example of an ER model that fails the minimality condition defined in section \ref{minimalitycondition}.
The failure arises at entity type p (president) because 
the set of paths $\set{I\circ mNo, I \circ M \circ cId, yr}$ is
in $\bar{I}$ and is therefore a mono-source but it is not minimal because
 a club has at most one president a year and so the subset $\set{I \circ M \circ cId, yr}$ is  a
mono-source. Further, the path $I \circ mNo$ is not equivalent to either of the two paths in the subset.
This problem can be fixed by introducing a club membership relationship $R: p \morph c$ and extended each defining instance $E$
 by which is defining $E'_R=E_{I \circ M}$  and by re-specifying the identifying features $p$ to be the set $\set{R,yr}$,
so obtaining an extended model in which $R$ it to be equivalent to
path $I \circ M$.
}
\label{clubpresidentbeforenormalisation}
\end{figure}