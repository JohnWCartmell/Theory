\begin{figure} [h]
\begin{center}
\begin{tabular}{c c}
\begin{tabular}{c p{0.5cm} c p{0.7cm} c}
   \Rnode{r}{r}   & &                  & &              \\[0.8cm]
                      & & \Rnode{c}{c} & & \Rnode{v}{v} \\[0.8cm]
   \Rnode{d}{d} & &                  & & 
\end{tabular}
\nccircle[angleA=80, nodesep=3pt]{->}{r}{.5cm}
\blabel{P}[0.4]
\ncarr[30]{r}{v}
\alabel{rId}[0.4]
\idcomp
\ncarr{c}{v} 
\alabel{cN}
\idcomp
\ncarr[-30]{d}{v} 
\blabel{cId}
\idcomp
\ncarr{d}{r} 
\alabel{S0}
\ncarr{d}{c} 
\blabel{R0}
\ncarr{c}{r} 
\blabel{R1}
& \footnotesize
\begin{tabular}{c p{1.5cm} p{4cm}}
KEY && \\
\hline
r & row & Identified by $rId$ attribute ($pId$) and
           nested within a parent ($P$) row. Root rows in the hierarchy
				  are represented have themselves as their own parent.\\
c & column & Identified by $cId$ attribute and applying to all rows
              nested within a root row given by the relationship $R1$. \\
d & data call & Identified by $cId$ attribute and related to a particular
                row ($S0$) and a particular column ($R0$). \\
\end{tabular} 
\end{tabular}
\end{center}
\caption{Table with Nested Rows. If the rows of a table can be indefinitely nested then the whole table can be equated with,
and therefore modelled as, the root row of the table. This is what is described here and it is an example of an ER model 
which is well formulated but which exhibits a transitive functional dependency which does not factor through an intransitive dependency.
}
\label{nestedrowsexample}
\end{figure}

Since columns are parented by root rows in the hierarchy and these are self-parenting the following diagram commutes:
$
\begin{array}{c p{0.3cm} c p{0.3cm} c}
                  & &              & &                 \\ %need bit of v space
   \Rnode{r1}{r}  & &              & &  \Rnode{r2}{r}  \\[0.8cm]
                  & & \Rnode{c}{c} & &                 
\end{array}
$
\ncarr{r1}{r2}
\alabel{P}[0.4]
\ncarr{c}{r1}
\alabel{R1}[0.4]
\ncarr{c}{r2}
\blabel{R1}[0.4]
commutes. 


There are infinitely many functional dependencies sourced at entity type $d$ as follows
\begin{equation*}
\sfd{S0}{R0/R1}
\end{equation*}
\begin{equation*}
\sfd{S0/P}{R0/R1}
\end{equation*}
\begin{equation*}
\sfd{S0/P/P}{R0/R1} \mbox{ and so on.}
\end{equation*}

Each of these dependencies is an intranstive dependency because each
factors through the one that follows it in the sequence because
\begin{equation}
S0 \morph S0/P \morph s0/P/P \morph s0/P/P/P \mbox{ and so on.}
\end{equation}


The relational design determined by the entity model has relations as follows:
\begin{equation}
\label{rowrelation}
row(\underline{rId}, prId)
\end{equation}
\begin{equation}
\label{columnrelation}
column(\underline{cN}, prId)
\end{equation}
\begin{equation}
\label{rowrelation}
datacell(\underline{dId}, prId, cN)
\end{equation}

These relations are each in BCNF.






