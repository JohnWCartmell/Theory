\begin{figure} [h]
\begin{center}
\begin{tabular}{c c}
$
\begin{array}{cp{0.9cm}c p{0.05cm} c p{0.05cm}    cp{0.9cm}c}
             && \Rnode{b1}{b_1} &&                && \Rnode{v1}{v} &&               \\[0.45cm]
             && \Rnode{b2}{b_2} &&                && \Rnode{v2}{v} &&               \\[-0.1cm]
\Rnode{a}{a} &&                 &&                &&               &&  \Rnode{c}{c} \\[0.2cm]
	           &&                 && \Rnode{ap}{a'} &&               &&
\end{array}
$

\ncarr[15]{a}{b1} 
\alabel{s_1}[0.4][0.1]
\ncarr[10]{a}{b2} 
\alabel{s_2}[0.4][0.1]
\ncarr{b1}{v1}
\alabel{k_1}[0.4]
\idcomp
\ncarr{b2}{v2}
\alabel{k_2}[0.4]
\idcomp
\ncarr[-15]{c}{v1}
\blabel{q_1}[0.4][0.1]
\idcomp
\ncarr[-10]{c}{v2}
\blabel{q_2}[0.4][0.1]
\idcomp
\ncarr[-10]{a}{ap} 
\blabel{r}[0.4]
\ncarr[-10]{ap}{c} 
\blabel{r'}[0.4]
\end{tabular}
\end{center}
\caption{Scenario for failure of clause (iii)
}
\label{clauseiiifail}
\end{figure}

Suppose that in an ER model there are entity types and relationships as shown and suppose that 
\begin{equation}
\label{clauseiiifailurescenario}
\incd{a}{s_1 \circ k_1,s_2 \circ k_2}{c}{q_1,q_2}
\end{equation}
 is a leaf primary referential inclusion dependency that is represented by path $r \circ r'$. Suppose that
the relationship $r'$ is not identifying-like so that  the path $r \circ r'$ is not relationship-like. Assume also that 
there is no relationship-like path that is equivalent to $r \circ r'$. In such a situation
clause (iii) of the definition of well-formulated does not hold.

The first question I need to answer is does this example suggest clause (iii) is really needed and if it does how can 
it be achieved with violating the requirement for the model to be reduced?
\subsection {Does failure of clause (iii) cause bad?}
Suppose $a$ is identified by a single attribute $k_a : a \morph \veee$ and likewise $a'$ by  $k_a' : a' \morph \veee$.
Suppose $b_1$ is identified by $k_1$ and $b_2$ by $k_2$. There are quite a number of possibilities here.
\subsubsection{Two inequalities}
One possible problematic case for the relational model is if attributes $\qq{s_1 \circ k_1}$ and $\qq{s_2 \circ k_2}$ both make it into the relational model. For this to happen must \underline{not}  be the case that for each
$i$, $i = 1,2$, $s_i \circ k_i \simeq r \circ r' \circ q_i$. 

Note that by lemma \lref{inclusiondependencyrestrictionlemma} we have
\begin{equation}
\label{clauseiiifailureinequality}
r \circ r' \circ q_i \leq s_i \circ k_i
\end{equation}

and therefore 
\begin{equation}
\label{clauseiiifailureinequality}
r \circ r' \circ q_i < s_i \circ k_i
\end{equation}

In the relational model we have
\begin{equation}
\label{errantfd}
\msfd{\qq{s_1 \circ k_1},\qq{s_2 \circ k_2}}{r \circ r'}
\end{equation}
Now two cases: 
\paragraph{ $\msfd{\qq{s_1 \circ k_1},\qq{s_2 \circ k_2}}{r \circ r'}$ is Intransitive}

One way that this can happen is in the following schema
\begin{figure} [h]
\begin{center}
\begin{tabular}{c c}
$
\begin{array}{cp{0.9cm}c p{0.05cm} c p{0.05cm}    cp{0.9cm}c}
             &&                 &&                && \Rnode{v0}{v} &&               \\[0.45cm]
             && \Rnode{b1}{b_1} &&                && \Rnode{v1}{v} &&               \\[0.45cm]
             && \Rnode{b2}{b_2} &&                && \Rnode{v2}{v} &&               \\[-0.1cm]
\Rnode{a}{a} &&                 &&                &&               &&  \Rnode{c}{c} \\[0.3cm]
	           &&                 && \Rnode{ap}{a'} &&               &&               \\[0.2cm]
						 &&                 &&                && \Rnode{v3}{v} &&               
\end{array}
$
\ncarr[50]{a}{v0} 
\alabel{K_a}[0.4][0.1]
\idcomp
\ncarr[15]{a}{b1} 
\alabel{s_1}[0.4][0.1]
\ncarr[10]{a}{b2} 
\alabel{s_2}[0.4][0.1]
\ncarr{b1}{v1}
\alabel{k_1}[0.4]
\idcomp
\ncarr{b2}{v2}
\alabel{k_2}[0.4]
\idcomp
\ncarr[-15]{c}{v1}
\blabel{q_1}[0.4][0.1]
\idcomp
\ncarr[-10]{c}{v2}
\blabel{q_2}[0.4][0.1]
\idcomp
\ncarr[-10]{a}{ap} 
\blabel{r}[0.4]
\ncarr[-10]{ap}{c} 
\blabel{r'}[0.4]
\ncarr[-10]{c}{ap} 
\blabel{r''}[0.5]
\ncarr[-10]{ap}{v3} 
\blabel{K_{a'}}[0.4]
\idcomp
\end{tabular}
\end{center}
\caption{Schema for failure of clause (iii) with intransitive functional dependency
}
\label{clauseiiifailschema}
\end{figure}

in which $r \circ r' \circ r'' \simeq r$.

The relational model $\chi(\gamma)$ has table $a$ as follows:
\begin{equation}
a(\underline{K_a},\qq{s_1 \circ k_1},\qq{s_2 \circ k_2},\qq{r \circ K_{a'}})
\end{equation}
and an intransitive functional dependency
\begin{equation}
\label{clauseiiifailtransitivecounterTNFfd}
\msfd{\qq{s_1 \circ k_1},\qq{s_2 \circ k_2}}{\qq{r \circ K_{a'}}}
\end{equation}

on account of which this table is not in BCNF nor in TNF.

If we normalise this schema then table $a$ is split into $a$ and $a_0$ as follows:
\begin{equation}
a(\underline{K_a},\qq{s_1 \circ k_1},\qq{s_2 \circ k_2})
\end{equation}
\begin{equation}
a_0(\underline{\qq{s_1 \circ k_1}},\underline{\qq{s_2 \circ k_2}},\qq{r \circ K_{a'}})
\end{equation}
which can be abstracted to the following logical ER model:

\begin{figure} [h]
\begin{center}
\begin{tabular}{c c}
$
\begin{array}{cp{0.9cm} cp{0.9cm} c p{0.05cm} c p{0.05cm}    cp{0.9cm}c}
             &&                &&                 &&                && \Rnode{v0}{v} &&               \\[0.45cm]
             &&                && \Rnode{b1}{b_1} &&                && \Rnode{v1}{v} &&               \\[0.45cm]
             &&                && \Rnode{b2}{b_2} &&                && \Rnode{v2}{v} &&               \\[-0.1cm]
\Rnode{a}{a} &&\Rnode{a0}{a_0} &&                 &&                &&               &&  \Rnode{c}{c} \\[0.3cm]
             &&	               &&                 && \Rnode{ap}{a'} &&               &&               \\[0.2cm]
             &&						     &&                 &&                && \Rnode{v3}{v} &&               
\end{array}
$
\ncarr[40]{a}{v0} 
\alabel{K_a}[0.4][0.1]
\idcomp
\ncarr[15]{a}{b1} 
\alabel{s_1}[0.4][0.1]
\ncarr[10]{a}{b2} 
\alabel{s_2}[0.4][0.1]
\ncarr{a}{a0} 
%\alabel{e}[0.4][0.1]
\blabel{e (epi)}[0.3][0.1]
\ncarr[15]{a0}{b1} 
\alabel{s_1}[0.4][0.1]
\idcomp
\ncarr[10]{a0}{b2} 
\alabel{s_2}[0.4][0.1]
\idcomp
\ncarr{b1}{v1}
\alabel{k_1}[0.4]
\idcomp
\ncarr{b2}{v2}
\alabel{k_2}[0.4]
\idcomp
\ncarr[-15]{c}{v1}
\blabel{q_1}[0.4][0.1]
\idcomp
\ncarr[-10]{c}{v2}
\blabel{q_2}[0.4][0.1]
\idcomp
\ncarr[-10]{a0}{ap} 
\blabel{r}[0.4]
\blabel{(total)}[0.4][0.25cm]
\ncarr[-10]{ap}{c} 
\blabel{r'}[0.4]
\ncarr[-10]{c}{ap} 
\blabel{r''}[0.5]
\ncarr[-10]{ap}{v3} 
\blabel{K_{a'}}[0.4]
\idcomp
\end{tabular}
\end{center}
\caption{Normalised schema for failure of clause (iii) with intransitive functional dependency
}
\label{clauseiiifailnormalisedschema}
\end{figure}

{\large THE BIG QUESTION -- what property does this ER model exhibit which distinguishes it goodnesswise from the starting model?}

For one thing $r \circ r' \circ q_1 \simeq s_1 \circ k_1$ and $r \circ r' \circ q_2 \simeq s_2 \circ k_2$
and so you could say that the inclusion dependency $\incd{a_0}{s_1 \circ k_1,s_2 \circ k_2}{c}{q_1,q_2}$ 
is \underline{explicitly} represented by $r \circ r'$. 

What this would mean for clause iii would be a further possibility so that a leaf inclusion dependency is 
represented by a relationship-like path \underline{or else has an explicit representation}. 
\footnote{If this is required then I feel it is worthwhile since I regard ER modelling as Hilbert did that which Cantor created. }.

\paragraph{Transitive}

\vspace {0.5cm}
\subsubsection{Another possibility}
Another is if $r' \circ q_1$ is equivalent to an identifying path $q'_1$, say, so that the relational schema
has an attribute $\qq{r \circ q'}$ and a functional dependency $\ssfd{\qq{s_1 \circ k_1}}{\qq{r \circ q'_1}}$. \\

\vspace {0.5cm}

\begin{enumerate}[(A)]
\item Suppose $r \circ r' \circ q_1 < s_1 \circ k_1$ and $r \circ r' \circ q_2 < s_2 \circ k_2$. 

\end{enumerate}

