

\begin{frame}{Finitary Property}
\begin{definition}
Define a category \catcw to have the \term{finitary property} iff for all objects $x$ and for all endomorphisms 
$f: x \morph x$ the following are equivalent
\begin{itemize}
\item $f$ is a monomorphism,
\item $f$ is an epimorphism.
\end{itemize}
\end{definition}
\end{frame}

\begin{frame}{Deducing Monos and Epis}
Consider, in any category \catcw, whenever \fgcomposablediagram{x}{y}{z}{f}{g} then
\begin{itemize}
\item $f \circ g$ is a monomorphism implies $f$ is a monomorphism.
\item $f \circ g$ is an epimorphism implies $g$ is an epimorphism.
\end{itemize}
\medskip
Further to this if \catcw has the \term{finitary property} then whenever \fgcomposablediagram{x}{x}{x}{f}{g} 
such that $f \circ g = id_x$ then
\begin{itemize}
\item $f$ is a monomorphism and an epimorphism,
\item $g$ is a monomorphism and an epimorphism.
\end{itemize}
\end{frame}

\begin{frame}
Now return to this example.
\begin{displaymath}
\begin{array}{cp{1.4cm}c}
                                    \\[0.1cm]
\Rnode{a}{a}	&& \Rnode{b}{b}     \\[0.25cm]
	            &&  
\end{array}
\begin{arrows}
\ncarr[15]{a}{b}
\alabel{f}[0.35]
\ncarr[-15]{a}{b}
\blabel{g}[0.35]
\ncarr[-70]{a}{b}
\blabel{h'}[0.35]
\ncarr[-70]{b}{a}
\blabel{h}[0.35]
\nccircle[angleA=-90, nodesep=3pt]{->}{b}{.5cm}
\blabel{r}[0.3]
\end{arrows}
\end{displaymath}

subject to the identities
\begin{equation}
\label{fhidentity}
f \circ h = id_a,
\end{equation}
\begin{equation}
\label{ghidentity}
g \circ h = id_a,
\end{equation}
\begin{equation}
\label{rhhpidentity}
r \circ h \circ h' = id_b.
\end{equation}

If we see this as a sketch for a category with the finitary property then we can deduce that $h$ is a monomorphism
and from this that $f=g$.  It is no longer a counter example (to criteria 2 implies criteria 2A).
\medskip
I am left wondering ...  
\end{frame}

