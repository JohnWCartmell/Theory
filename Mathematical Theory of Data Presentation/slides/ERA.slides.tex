
\begin{frame}{Chemical Element DS as Directed Graph}
\scalebox{0.65}{


$
\begin{array}{c c c p{4.5cm} l}
                        &                        &                            &&\attrtype{\Rnode{symboltype}{\CEsymboltype}  }       \\ [0.7cm]
                        &\etype{\Rnode{element}{chemical\ element}\Rnode{elementR}{}}&  &&                                        \\ [0.3cm]
%												&                        &           &&\attrtype{\Rnode{atomicnumbertypeL}{n}\Rnode{atomicnumbertype}{umber(1,1000)}}\\ [0.55cm]
												&                        &           &&\attrtype{\Rnode{atomicnumbertypeL}{\CEatomicnumbertype}}\\ [0.55cm]
												&                        &\etype{\Rnode{isotope}{isotope}\Rnode{isotopeR}{}}&&                          \\ [0.3cm]
                        &                        &                            &&\attrtype{\Rnode{floattype}{\CEfloattype}}                \\ [0.45cm]
                        &\etype{\Rnode{allotrope}{allotrope}\Rnode{allotropeR}{}}&    &&                                \\ [0.45cm]
												&                        &                            &&\attrtype{\Rnode{nametype}{\CEnametype}}           \\ [1.0cm]
\etype{\Rnode{valency}{valency}\Rnode{valencyR}{}}&      &                            &&\attrtype{\Rnode{valencynumbertype}{\CEvalencynumbertype}}\\
\end{array}
$
\setlength{\arrnodesepA}{7pt}
\setlength{\arrnodesepB}{8pt}
\setlength{\arroffsetB}{7pt}
\ncarr[10]{valency}{element}
\setlength{\arrnodesepB}{7pt}
\setlength{\arroffsetB}{0pt}
\ncarr[-5]{allotrope}{element}
\setlength{\arrnodesepB}{9pt}
\setlength{\arroffsetB}{-5pt}
\ncarr[-5]{isotope}{element}
\setlength{\arroffsetB}{0pt}
\setlength{\arrnodesepB}{3pt}
\setlength{\arroffsetB}{-2pt}
\setlength{\arroffsetA}{5pt}
\ncarr[15]{elementR}{symboltype}
\alabel{symbol}[0.4][0]
\setlength{\arroffsetA}{0pt}
\setlength{\arrnodesepB}{3pt}
\setlength{\arroffsetB}{0pt}
\ncarr[5]{elementR}{atomicnumbertypeL}
\alabel{atomic\,number}[0.4]
\setlength{\arroffsetA}{2pt}
\setlength{\arroffsetB}{-2pt}
\ncarr[5]{isotopeR}{atomicnumbertypeL}
\alabel{neutron\,count}[0.35][0]
\setlength{\arroffsetA}{-2pt}
\ncarr[5]{isotopeR}{floattype}
\alabel{mass}[0.35]
\setlength{\arroffsetA}{2pt}
\ncarr[5]{allotropeR}{floattype}
\alabel{melting\,point}[0.35][-1]
\setlength{\arroffsetA}{-2pt}
\setlength{\arroffsetB}{-4pt}
\ncarr[5]{allotropeR}{nametype}
\blabel{name}[0.3]
\setlength{\arroffsetA}{2pt}
\setlength{\arroffsetB}{-2pt}
\ncarr[5]{valencyR}{valencynumbertype}
\alabel{number}[0.3]
} \\
\vspace {0.25cm}
\textit{
\begin{tabular} {c p{0cm} p{2.5cm} l}
the types of entities and       & & the attributes   & the  types  \\
their inter-relationships       & & of the entities     & of the attributes \\
  & &            &
\end{tabular}
}
\end{frame}

\begin{frame}{Chemical Element DS viewed as ER diagram}
\begin{center}
\scalebox{0.85}{
\begin{erdiagram}{4.8}{4.85}

\eret{1}{-1.9}{4}{-0.1}{0.2}{1}\eretname{1.3}{-0.45}{l}{element}
\erattr{1.2}{-0.65}{1}{0}{symbol}
\erattr{1.2}{-0.95}{1}{1}{name}
\erattr{1.2}{-1.25}{1}{1}{atomic number}
\erattr{1.2}{-1.55}{1}{1}{relative atomic mass}
\eret{0.15}{-4.8}{2.25}{-3}{0.2}{1}\eretname{0.36}{-3.35}{l}{allotrope}
\erattr{0.35}{-3.55}{1}{0}{name}
\erattr{0.35}{-3.85}{0}{1}{melting point}
\erattr{0.35}{-4.15}{0}{1}{boiling point}
\erattr{0.35}{-4.45}{0}{1}{density}
\eret{2.75}{-3.9}{4.85}{-3}{0.2}{1}\eretname{2.96}{-3.35}{l}{valency}
\erattr{2.95}{-3.55}{1}{0}{number}

% relationship 
\errelname{2.15}{-2.2}{l}{}\errelarm{2}{-1.9}{2}{-1.975}{1}{0}\errelarm{1.2}{-2.875}{1.2}{-3}{1}{0}\errelangle{2}{-1.975}{2}{-2.05}{1.6}{-2.4}{1}{0}\errelangle{1.6}{-2.4}{1.2}{-2.75}{1.2}{-2.875}{1}{0}\ercrowfoot{1.2}{-2.85}{1.05}{-3}{1.2}{-3}{1.35}{-3}{0}
% relationship 
\errelname{3.15}{-2.2}{l}{}\errelarm{3}{-1.9}{3}{-1.975}{1}{0}\errelarm{3.8}{-2.788}{3.8}{-3}{1}{0}\errelangle{3}{-1.975}{3}{-2.05}{3.4}{-2.313}{1}{0}\errelangle{3.4}{-2.313}{3.8}{-2.575}{3.8}{-2.788}{1}{0}\eridcomprel{3.6999999999999997}{3.9}{-2.75}\ercrowfoot{3.8}{-2.85}{3.65}{-3}{3.8}{-3}{3.95}{-3}{0}
\end{erdiagram}

}
\end{center}

\end{frame}

\begin{frame}{Basic Chemistry viewed as ER diagram}
\begin{center}
\scalebox{0.5}{
\begin{erdiagram}{7.699999999999999}{15.7075}

\eret{1}{-2.4}{4}{-1.5}{0.2}{1}\eretname{1.3}{-1.85}{l}{molecularStructure}
\erattr{1.2}{-2.05}{1}{0}{name}
\eret{1.793}{-4.8}{3.207}{-3.3}{0.2}{1}\eretname{1.934}{-3.65}{l}{atom}
\erattr{1.993}{-3.85}{1}{0}{atomId}
\erattr{1.993}{-4.15}{1}{1}{x}
\erattr{1.993}{-4.45}{1}{1}{y}
\eret{1}{-6.9}{4}{-5.7}{0.2}{1}\eretname{1.3}{-6.05}{l}{bond formed}
\erdattr{1.2}{-6.25}{1}{0}{withAtomId(R2)}
\erattr{1.2}{-6.55}{1}{1}{bondType}
\eret{9.208}{-4.8}{12.208}{-3.3}{0.2}{1}\eretname{9.508}{-3.65}{l}{element}
\erattr{9.407}{-3.85}{1}{0}{symbol}
\erattr{9.407}{-4.15}{1}{1}{name}
\erattr{9.407}{-4.45}{1}{1}{atomic number}
\eret{5.708}{-7.4}{8.708}{-5.9}{0.2}{1}\eretname{6.007}{-6.25}{l}{isotope}
\erattr{5.908}{-6.45}{1}{0}{numberOfNeutrons}
\erattr{5.908}{-6.75}{1}{1}{mass}
\erattr{5.908}{-7.05}{1}{1}{abundancy}
\eret{9.208}{-7.7}{12.208}{-5.9}{0.2}{1}\eretname{9.508}{-6.25}{l}{allotrope}
\erattr{9.407}{-6.45}{1}{0}{name}
\erattr{9.407}{-6.75}{0}{1}{melting point}
\erattr{9.407}{-7.05}{0}{1}{boiling point}
\erattr{9.407}{-7.35}{0}{1}{density}
\eret{12.708}{-6.8}{15.708}{-5.9}{0.2}{1}\eretname{13.008}{-6.25}{l}{valency}
\erattr{12.907}{-6.45}{1}{0}{number}
\eret{0}{-0.2}{15.708}{0.3}{0.2}{1}

% relationship 
\errelname{2.65}{-0.5}{l}{}\errelname{2.65}{-1.35}{l}{..}\errelarm{2.5}{-0.2}{2.5}{-0.85}{1}{0}\errelarm{2.5}{-0.85}{2.5}{-1.5}{1}{0}\ercrowfoot{2.5}{-1.35}{2.35}{-1.5}{2.5}{-1.5}{2.65}{-1.5}{0}
% relationship 
\errelname{10.858}{-0.5}{l}{}\errelname{10.858}{-3.15}{l}{..}\errelarm{10.708}{-0.2}{10.708}{-1.75}{1}{0}\errelarm{10.708}{-1.75}{10.708}{-3.3}{1}{0}\ercrowfoot{10.708}{-3.15}{10.558}{-3.3}{10.708}{-3.3}{10.858}{-3.3}{0}
% relationship 
\errelname{2.65}{-2.7}{l}{}\errelname{2.65}{-3.15}{l}{..}\errelarm{2.5}{-2.4}{2.5}{-2.85}{1}{0}\errelarm{2.5}{-2.85}{2.5}{-3.3}{1}{0}\eridcomprel{2.4}{2.6}{-3.05}\ercrowfoot{2.5}{-3.15}{2.35}{-3.3}{2.5}{-3.3}{2.65}{-3.3}{0}
% relationship 
\errelname{2.65}{-5.1}{l}{}\errelname{2.65}{-5.55}{l}{of}\errelarm{2.5}{-4.8}{2.5}{-5.25}{0}{0}\errelarm{2.5}{-5.25}{2.5}{-5.7}{1}{0}\eridcomprel{2.4}{2.6}{-5.449999999999999}\ercrowfoot{2.5}{-5.55}{2.35}{-5.7}{2.5}{-5.7}{2.65}{-5.7}{0}
% relationship element
\errelname{3.357}{-4.35}{l}{element}\errelarm{3.207}{-4.05}{6.208}{-4.05}{1}{0}\errelarm{6.208}{-4.05}{9.208}{-4.05}{0}{0}\ercrowfoot{3.357}{-4.05}{3.207}{-3.9}{3.207}{-4.05}{3.207}{-4.2}{0}
% relationship with
\errelname{4.15}{-6.6}{l}{with}\errelname{4.15}{-6.15}{l}{with}\errelarm{4}{-6.3}{4.5}{-6.3}{1}{0}\errelarm{4.5}{-6.3}{4}{-6.3}{1}{0}\errelname{5.15}{-6.15}{l}{R2}\eridrefrel{4.25}{-6.199999999999999}{-6.399999999999999}
% relationship 
\errelname{10.108}{-5.1}{l}{}\errelname{7.358}{-5.75}{l}{..}\errelarm{9.958}{-4.8}{9.958}{-4.875}{1}{0}\errelarm{7.208}{-5.687}{7.208}{-5.9}{1}{0}\errelangle{9.958}{-4.875}{9.958}{-4.95}{8.583}{-5.213}{1}{0}\errelangle{8.583}{-5.213}{7.208}{-5.475}{7.208}{-5.687}{1}{0}\eridcomprel{7.1075}{7.307499999999999}{-5.6499999999999995}\ercrowfoot{7.208}{-5.75}{7.057}{-5.9}{7.208}{-5.9}{7.358}{-5.9}{0}
% relationship 
\errelname{10.858}{-5.1}{l}{}\errelname{10.858}{-5.75}{l}{..}\errelarm{10.708}{-4.8}{10.708}{-5.35}{1}{0}\errelarm{10.708}{-5.35}{10.708}{-5.9}{1}{0}\ercrowfoot{10.708}{-5.75}{10.558}{-5.9}{10.708}{-5.9}{10.858}{-5.9}{0}
% relationship 
\errelname{11.608}{-5.1}{l}{}\errelname{14.358}{-5.75}{l}{..}\errelarm{11.458}{-4.8}{11.458}{-4.875}{1}{0}\errelarm{14.208}{-5.687}{14.208}{-5.9}{1}{0}\errelangle{11.458}{-4.875}{11.458}{-4.95}{12.833}{-5.213}{1}{0}\errelangle{12.833}{-5.213}{14.208}{-5.475}{14.208}{-5.687}{1}{0}\eridcomprel{14.1075}{14.3075}{-5.6499999999999995}\ercrowfoot{14.208}{-5.75}{14.058}{-5.9}{14.208}{-5.9}{14.358}{-5.9}{0}
\end{erdiagram}

}
\end{center}
\end{frame}


\begin{frame}{MolecularGeometry ERA diagram  -- Logical }
\begin{center}
\scalebox{0.5}{
\begin{erdiagram}{7}{15.1075}

\eret{1}{-2.6}{3.6}{-1.5}{0.2}{1}\eretname{1.26}{-1.85}{l}{conformation}
\erattr{1.2}{-2.05}{1}{0}{id}
\eret{1.514}{-5.2}{3.086}{-3.7}{0.2}{1}\eretname{1.671}{-4.05}{l}{position}
\erattr{1.714}{-4.25}{1}{1}{x}
\erattr{1.714}{-4.55}{1}{1}{y}
\erattr{1.714}{-4.85}{1}{1}{z}
\eret{7.1}{-2.6}{9.7}{-1.5}{0.2}{1}\eretname{7.36}{-1.85}{l}{molStruct}
\erattr{7.3}{-2.05}{1}{0}{name}
\eret{7.693}{-5.2}{9.108}{-3.7}{0.2}{1}\eretname{7.834}{-4.05}{l}{atom}
\erattr{7.893}{-4.25}{1}{0}{number}
\eret{7.354}{-7}{9.446}{-6.1}{0.2}{1}\eretname{7.563}{-6.45}{l}{bond formed}
\erattr{7.554}{-6.65}{1}{1}{bondType}
\eret{12.108}{-5.2}{15.108}{-3.7}{0.2}{1}\eretname{12.408}{-4.05}{l}{element}
\erattr{12.308}{-4.25}{1}{0}{symbol}
\erattr{12.308}{-4.55}{1}{1}{name}
\erattr{12.308}{-4.85}{1}{1}{atomic number}
\eret{0}{-0.2}{15.108}{0.3}{0.2}{1}

% relationship 
\errelname{2.45}{-0.5}{l}{}\errelname{2.45}{-1.35}{l}{..}\errelarm{2.3}{-0.2}{2.3}{-0.85}{1}{0}\errelarm{2.3}{-0.85}{2.3}{-1.5}{1}{0}\ercrowfoot{2.3}{-1.35}{2.15}{-1.5}{2.3}{-1.5}{2.45}{-1.5}{0}
% relationship 
\errelname{8.55}{-0.5}{l}{}\errelname{8.55}{-1.35}{l}{..}\errelarm{8.4}{-0.2}{8.4}{-0.85}{1}{0}\errelarm{8.4}{-0.85}{8.4}{-1.5}{1}{0}\ercrowfoot{8.4}{-1.35}{8.25}{-1.5}{8.4}{-1.5}{8.55}{-1.5}{0}
% relationship 
\errelname{13.758}{-0.5}{l}{}\errelname{13.758}{-3.55}{l}{..}\errelarm{13.608}{-0.2}{13.608}{-1.95}{1}{0}\errelarm{13.608}{-1.95}{13.608}{-3.7}{1}{0}\ercrowfoot{13.608}{-3.55}{13.458}{-3.7}{13.608}{-3.7}{13.758}{-3.7}{0}
% relationship 
\errelname{2.45}{-2.9}{l}{}\errelname{2.45}{-3.55}{l}{..}\errelarm{2.3}{-2.6}{2.3}{-3.15}{1}{0}\errelarm{2.3}{-3.15}{2.3}{-3.7}{1}{0}\eridcomprel{2.15}{2.4499999999999997}{-3.4000000000000004}\ercrowfoot{2.3}{-3.55}{2.15}{-3.7}{2.3}{-3.7}{2.45}{-3.7}{0}\ercrowfoot{2.3}{-3.55}{2.15}{-3.4}{2.3}{-3.4}{2.45}{-3.4}{0}\ercrowfoot{2.3}{-3.55}{2.15}{-3.7}{2.3}{-3.7}{2.45}{-3.7}{0}
% relationship of
\errelname{3.75}{-2.35}{l}{of}\errelarm{3.6}{-2.05}{5.35}{-2.05}{1}{0}\errelarm{5.35}{-2.05}{7.1}{-2.05}{0}{0}\ercrowfoot{3.75}{-2.05}{3.6}{-1.9}{3.6}{-2.05}{3.6}{-2.2}{0}
% relationship atom
\errelname{3.236}{-4.75}{l}{atom}\erscope{5.139}{-4.75}{l}{d:..=s:..}\errelarm{3.086}{-4.45}{5.389}{-4.45}{1}{0}\errelarm{5.389}{-4.45}{7.693}{-4.45}{0}{0}\ercrowfoot{3.236}{-4.45}{3.086}{-4.3}{3.086}{-4.45}{3.086}{-4.6}{0}\eridrefrel{3.3859999999999997}{-4.3}{-4.6000000000000005}\ercrowfoot{3.236}{-4.45}{3.086}{-4.3}{3.086}{-4.45}{3.086}{-4.6}{0}\ercrowfoot{3.236}{-4.45}{3.386}{-4.3}{3.386}{-4.45}{3.386}{-4.6}{0}
% relationship 
\errelname{8.55}{-2.9}{l}{}\errelname{8.55}{-3.55}{l}{..}\errelarm{8.4}{-2.6}{8.4}{-3.15}{1}{0}\errelarm{8.4}{-3.15}{8.4}{-3.7}{1}{0}\eridcomprel{8.3}{8.5}{-3.45}\ercrowfoot{8.4}{-3.55}{8.25}{-3.7}{8.4}{-3.7}{8.55}{-3.7}{0}
% relationship 
\errelname{8.55}{-5.5}{l}{}\errelname{8.55}{-5.95}{l}{of}\errelarm{8.4}{-5.2}{8.4}{-5.65}{0}{0}\errelarm{8.4}{-5.65}{8.4}{-6.1}{1}{0}\eridcomprel{8.3}{8.5}{-5.85}\ercrowfoot{8.4}{-5.95}{8.25}{-6.1}{8.4}{-6.1}{8.55}{-6.1}{0}
% relationship element
\errelname{9.258}{-4.75}{l}{element}\errelarm{9.108}{-4.45}{10.608}{-4.45}{1}{0}\errelarm{10.608}{-4.45}{12.108}{-4.45}{0}{0}\ercrowfoot{9.258}{-4.45}{9.108}{-4.3}{9.108}{-4.45}{9.108}{-4.6}{0}
% relationship with
\errelname{9.596}{-6.85}{l}{with}\errelname{9.596}{-6.4}{l}{with}\errelarm{9.446}{-6.55}{9.946}{-6.55}{1}{0}\errelarm{9.946}{-6.55}{9.446}{-6.55}{1}{0}\erscope{10.496}{-6.75}{l}{d:of/..=s:of/..}\eridrefrel{9.696375}{-6.45}{-6.6499999999999995}
\end{erdiagram}

}
\end{center}
\end{frame}

\begin{frame}{MolecularGeometry ERA diagram -- Hierarchical}
\begin{center}
\scalebox{0.5}{
\begin{erdiagram}{7.6}{15.8725}

\eret{1}{-2.6}{3.6}{-1.5}{0.2}{1}\eretname{1.26}{-1.85}{l}{conformation}
\erattr{1.2}{-2.05}{1}{0}{id}
\erdattr{1.2}{-2.35}{1}{1}{of\_name(R1)}
\eret{1.019}{-5.5}{3.581}{-3.7}{0.2}{1}\eretname{1.275}{-4.05}{l}{position}
\erdattr{1.219}{-4.25}{1}{0}{atom\_number(R2)}
\erattr{1.219}{-4.55}{1}{1}{x}
\erattr{1.219}{-4.85}{1}{1}{y}
\erattr{1.219}{-5.15}{1}{1}{z}
\eret{7.1}{-2.6}{9.7}{-1.5}{0.2}{1}\eretname{7.36}{-1.85}{l}{molStruct}
\erattr{7.3}{-2.05}{1}{0}{name}
\eret{6.928}{-5.5}{9.873}{-3.7}{0.2}{1}\eretname{7.222}{-4.05}{l}{atom}
\erattr{7.128}{-4.25}{1}{0}{number}
\erdattr{7.128}{-4.55}{1}{1}{element\_symbol(R3)}
\eret{6.75}{-7.6}{10.05}{-6.4}{0.2}{1}\eretname{7.08}{-6.75}{l}{bond formed}
\erdattr{6.95}{-6.95}{1}{0}{with\_atom\_number(R4)}
\erattr{6.95}{-7.25}{1}{1}{bondType}
\eret{12.873}{-5.2}{15.873}{-3.7}{0.2}{1}\eretname{13.173}{-4.05}{l}{element}
\erattr{13.073}{-4.25}{1}{0}{symbol}
\erattr{13.073}{-4.55}{1}{1}{name}
\erattr{13.073}{-4.85}{1}{1}{atomic number}
\eret{0}{-0.2}{15.873}{0.3}{0.2}{1}

% relationship 
\errelname{2.45}{-0.5}{l}{}\errelname{2.45}{-1.35}{l}{..}\errelarm{2.3}{-0.2}{2.3}{-0.85}{1}{0}\errelarm{2.3}{-0.85}{2.3}{-1.5}{1}{0}\ercrowfoot{2.3}{-1.35}{2.15}{-1.5}{2.3}{-1.5}{2.45}{-1.5}{0}
% relationship 
\errelname{8.55}{-0.5}{l}{}\errelname{8.55}{-1.35}{l}{..}\errelarm{8.4}{-0.2}{8.4}{-0.85}{1}{0}\errelarm{8.4}{-0.85}{8.4}{-1.5}{1}{0}\ercrowfoot{8.4}{-1.35}{8.25}{-1.5}{8.4}{-1.5}{8.55}{-1.5}{0}
% relationship 
\errelname{14.523}{-0.5}{l}{}\errelname{14.523}{-3.55}{l}{..}\errelarm{14.373}{-0.2}{14.373}{-1.95}{1}{0}\errelarm{14.373}{-1.95}{14.373}{-3.7}{1}{0}\ercrowfoot{14.373}{-3.55}{14.223}{-3.7}{14.373}{-3.7}{14.523}{-3.7}{0}
% relationship 
\errelname{2.45}{-2.9}{l}{}\errelname{2.45}{-3.55}{l}{..}\errelarm{2.3}{-2.6}{2.3}{-3.15}{1}{0}\errelarm{2.3}{-3.15}{2.3}{-3.7}{1}{0}\eridcomprel{2.15}{2.4499999999999997}{-3.4000000000000004}\ercrowfoot{2.3}{-3.55}{2.15}{-3.7}{2.3}{-3.7}{2.45}{-3.7}{0}\ercrowfoot{2.3}{-3.55}{2.15}{-3.4}{2.3}{-3.4}{2.45}{-3.4}{0}\ercrowfoot{2.3}{-3.55}{2.15}{-3.7}{2.3}{-3.7}{2.45}{-3.7}{0}
% relationship of
\errelname{3.75}{-2.35}{l}{of}\errelname{5.5}{-1.9}{l}{R1}\errelarm{3.6}{-2.05}{5.35}{-2.05}{1}{0}\errelarm{5.35}{-2.05}{7.1}{-2.05}{0}{0}\ercrowfoot{3.75}{-2.05}{3.6}{-1.9}{3.6}{-2.05}{3.6}{-2.2}{0}
% relationship atom
\errelname{3.731}{-4.9}{l}{atom}\errelname{5.404}{-4.45}{l}{R2}\erscope{5.004}{-4.9}{l}{\textasciitilde /..=..}\errelarm{3.581}{-4.6}{5.254}{-4.6}{1}{0}\errelarm{5.254}{-4.6}{6.928}{-4.6}{0}{0}\ercrowfoot{3.731}{-4.6}{3.581}{-4.45}{3.581}{-4.6}{3.581}{-4.75}{0}\eridrefrel{3.8812499999999996}{-4.449999999999999}{-4.75}\ercrowfoot{3.731}{-4.6}{3.581}{-4.45}{3.581}{-4.6}{3.581}{-4.75}{0}\ercrowfoot{3.731}{-4.6}{3.881}{-4.45}{3.881}{-4.6}{3.881}{-4.75}{0}
% relationship 
\errelname{8.55}{-2.9}{l}{}\errelname{8.55}{-3.55}{l}{..}\errelarm{8.4}{-2.6}{8.4}{-3.15}{1}{0}\errelarm{8.4}{-3.15}{8.4}{-3.7}{1}{0}\eridcomprel{8.3}{8.5}{-3.45}\ercrowfoot{8.4}{-3.55}{8.25}{-3.7}{8.4}{-3.7}{8.55}{-3.7}{0}
% relationship 
\errelname{8.55}{-5.8}{l}{}\errelname{8.55}{-6.25}{l}{of}\errelarm{8.4}{-5.5}{8.4}{-5.95}{0}{0}\errelarm{8.4}{-5.95}{8.4}{-6.4}{1}{0}\eridcomprel{8.3}{8.5}{-6.1499999999999995}\ercrowfoot{8.4}{-6.25}{8.25}{-6.4}{8.4}{-6.4}{8.55}{-6.4}{0}
% relationship element
\errelname{10.023}{-4.9}{l}{element}\errelname{11.523}{-4.45}{l}{R3}\errelarm{9.873}{-4.6}{11.373}{-4.6}{1}{0}\errelarm{11.373}{-4.6}{12.873}{-4.6}{0}{0}\ercrowfoot{10.023}{-4.6}{9.873}{-4.45}{9.873}{-4.6}{9.873}{-4.75}{0}
% relationship with
\errelname{10.2}{-7.3}{l}{with}\errelname{10.2}{-6.85}{l}{with}\errelarm{10.05}{-7}{10.55}{-7}{1}{0}\errelarm{10.55}{-7}{10.05}{-7}{1}{0}\errelname{11.2}{-6.85}{l}{R4}\erscope{11.1}{-7.2}{l}{\textasciitilde /of/..=of/..}\eridrefrel{10.3}{-6.8999999999999995}{-7.099999999999999}
\end{erdiagram}

}
\end{center}
\end{frame}

\begin{frame}{MolecularGeometry  ERA diagram -- Relational}
\begin{center}
\scalebox{0.5}{
\begin{erdiagram}{8.5}{15.8725}

\eret{1}{-2.6}{3.6}{-1.5}{0.2}{1}\eretname{1.26}{-1.85}{l}{conformation}
\erattr{1.2}{-2.05}{1}{0}{id}
\erdattr{1.2}{-2.35}{1}{1}{of\_name(R1)}
\eret{0.764}{-5.8}{3.836}{-3.7}{0.2}{1}\eretname{1.071}{-4.05}{l}{position}
\erdattr{0.964}{-4.25}{1}{0}{conformation\_id(D4)}
\erdattr{0.964}{-4.55}{1}{0}{atom\_number(R2)}
\erattr{0.964}{-4.85}{1}{1}{x}
\erattr{0.964}{-5.15}{1}{1}{y}
\erattr{0.964}{-5.45}{1}{1}{z}
\eret{7.1}{-2.6}{9.7}{-1.5}{0.2}{1}\eretname{7.36}{-1.85}{l}{molStruct}
\erattr{7.3}{-2.05}{1}{0}{name}
\eret{6.928}{-5.8}{9.873}{-3.7}{0.2}{1}\eretname{7.222}{-4.05}{l}{atom}
\erdattr{7.128}{-4.25}{1}{0}{molStruct\_name(D5)}
\erattr{7.128}{-4.55}{1}{0}{number}
\erdattr{7.128}{-4.85}{1}{1}{element\_symbol(R3)}
\eret{6.75}{-8.5}{10.05}{-6.7}{0.2}{1}\eretname{7.08}{-7.05}{l}{bond formed}
\erdattr{6.95}{-7.25}{1}{0}{molStruct\_name(D6)}
\erdattr{6.95}{-7.55}{1}{0}{atom\_number(D6)}
\erdattr{6.95}{-7.85}{1}{0}{with\_atom\_number(R4)}
\erattr{6.95}{-8.15}{1}{1}{bondType}
\eret{12.873}{-5.2}{15.873}{-3.7}{0.2}{1}\eretname{13.173}{-4.05}{l}{element}
\erattr{13.073}{-4.25}{1}{0}{symbol}
\erattr{13.073}{-4.55}{1}{1}{name}
\erattr{13.073}{-4.85}{1}{1}{atomic number}
\eret{0}{-0.2}{15.873}{0.3}{0.2}{1}

% relationship 
\errelname{2.45}{-0.5}{l}{}\errelname{2.45}{-1.35}{l}{..}\errelarm{2.3}{-0.2}{2.3}{-0.85}{1}{0}\errelarm{2.3}{-0.85}{2.3}{-1.5}{1}{0}\ercrowfoot{2.3}{-1.35}{2.15}{-1.5}{2.3}{-1.5}{2.45}{-1.5}{0}
% relationship 
\errelname{8.55}{-0.5}{l}{}\errelname{8.55}{-1.35}{l}{..}\errelarm{8.4}{-0.2}{8.4}{-0.85}{1}{0}\errelarm{8.4}{-0.85}{8.4}{-1.5}{1}{0}\ercrowfoot{8.4}{-1.35}{8.25}{-1.5}{8.4}{-1.5}{8.55}{-1.5}{0}
% relationship 
\errelname{14.523}{-0.5}{l}{}\errelname{14.523}{-3.55}{l}{..}\errelarm{14.373}{-0.2}{14.373}{-1.95}{1}{0}\errelarm{14.373}{-1.95}{14.373}{-3.7}{1}{0}\ercrowfoot{14.373}{-3.55}{14.223}{-3.7}{14.373}{-3.7}{14.523}{-3.7}{0}
% relationship 
\errelname{2.45}{-2.9}{l}{}\errelname{2.45}{-3.55}{l}{..}\errelname{2.45}{-3}{l}{D4}\errelarm{2.3}{-2.6}{2.3}{-3.15}{1}{0}\errelarm{2.3}{-3.15}{2.3}{-3.7}{1}{0}\eridcomprel{2.15}{2.4499999999999997}{-3.4000000000000004}\ercrowfoot{2.3}{-3.55}{2.15}{-3.7}{2.3}{-3.7}{2.45}{-3.7}{0}\ercrowfoot{2.3}{-3.55}{2.15}{-3.4}{2.3}{-3.4}{2.45}{-3.4}{0}\ercrowfoot{2.3}{-3.55}{2.15}{-3.7}{2.3}{-3.7}{2.45}{-3.7}{0}
% relationship of
\errelname{3.75}{-2.35}{l}{of}\errelname{5.5}{-1.9}{l}{R1}\errelarm{3.6}{-2.05}{5.35}{-2.05}{1}{0}\errelarm{5.35}{-2.05}{7.1}{-2.05}{0}{0}\ercrowfoot{3.75}{-2.05}{3.6}{-1.9}{3.6}{-2.05}{3.6}{-2.2}{0}
% relationship atom
\errelname{3.986}{-5.05}{l}{atom}\errelname{5.532}{-4.6}{l}{R2}\erscope{5.132}{-5.05}{l}{\textasciitilde /..=..}\errelarm{3.836}{-4.75}{5.382}{-4.75}{1}{0}\errelarm{5.382}{-4.75}{6.928}{-4.75}{0}{0}\ercrowfoot{3.986}{-4.75}{3.836}{-4.6}{3.836}{-4.75}{3.836}{-4.9}{0}\eridrefrel{4.1362499999999995}{-4.6}{-4.9}\ercrowfoot{3.986}{-4.75}{3.836}{-4.6}{3.836}{-4.75}{3.836}{-4.9}{0}\ercrowfoot{3.986}{-4.75}{4.136}{-4.6}{4.136}{-4.75}{4.136}{-4.9}{0}
% relationship 
\errelname{8.55}{-2.9}{l}{}\errelname{8.55}{-3.55}{l}{..}\errelname{8.55}{-3}{l}{D5}\errelarm{8.4}{-2.6}{8.4}{-3.15}{1}{0}\errelarm{8.4}{-3.15}{8.4}{-3.7}{1}{0}\eridcomprel{8.3}{8.5}{-3.45}\ercrowfoot{8.4}{-3.55}{8.25}{-3.7}{8.4}{-3.7}{8.55}{-3.7}{0}
% relationship 
\errelname{8.55}{-6.1}{l}{}\errelname{8.55}{-6.55}{l}{of}\errelname{8.55}{-6.1}{l}{D6}\errelarm{8.4}{-5.8}{8.4}{-6.25}{0}{0}\errelarm{8.4}{-6.25}{8.4}{-6.7}{1}{0}\eridcomprel{8.3}{8.5}{-6.45}\ercrowfoot{8.4}{-6.55}{8.25}{-6.7}{8.4}{-6.7}{8.55}{-6.7}{0}
% relationship element
\errelname{10.023}{-5.05}{l}{element}\errelname{11.523}{-4.6}{l}{R3}\errelarm{9.873}{-4.75}{11.373}{-4.75}{1}{0}\errelarm{11.373}{-4.75}{12.873}{-4.75}{0}{0}\ercrowfoot{10.023}{-4.75}{9.873}{-4.6}{9.873}{-4.75}{9.873}{-4.9}{0}
% relationship with
\errelname{10.2}{-7.9}{l}{with}\errelname{10.2}{-7.45}{l}{with}\errelarm{10.05}{-7.6}{10.55}{-7.6}{1}{0}\errelarm{10.55}{-7.6}{10.05}{-7.6}{1}{0}\errelname{11.2}{-7.45}{l}{R4}\erscope{11.1}{-7.8}{l}{\textasciitilde /of/..=of/..}\eridrefrel{10.3}{-7.5}{-7.699999999999999}
\end{erdiagram}

}
\end{center}
\end{frame}


\begin{frame}{ER modelling Meta-Model}
\scalebox{0.23}{
\begin{erdiagram}{24.550000000000004}{49.63012500000001}

\eret{1.5}{-2.2}{2.833}{-1}{0.2}{1}\ertext{1.633}{-1.35}{l}{diagram}
\erattr{1.7}{-1.55}{0}{1}{deltaw}
\erattr{1.7}{-1.85}{0}{1}{deltah}
\eret{3.333}{-2.8}{6.461}{-1}{0.2}{1}\ertext{3.646}{-1.35}{l}{defaults}
\erattr{3.533}{-1.55}{0}{1}{etwidth}
\erattr{3.533}{-1.85}{0}{1}{etheight}
\erattr{3.533}{-2.15}{0}{1}{etyseparation}
\erattr{3.533}{-2.45}{0}{1}{etydeltaseparation}
\eret{47.04}{-1.9}{49.248}{-1}{0.2}{1}\ertext{47.261}{-1.35}{l}{xml}
\erattr{47.24}{-1.55}{0}{1}{namespaceuri}
\eret{10.461}{-3.75}{26.408}{-1}{0.2}{1}\ertext{10.681}{-1.35}{l}{ENTITY\textunderscore TYPE}
\erattr{10.661}{-1.55}{1}{0}{name}
\erattr{10.661}{-1.85}{0}{1}{description}
\erattr{10.661}{-2.15}{0}{1}{xpath}
\erattr{10.661}{-2.45}{0}{1}{modulename}
\eret{18.786}{-2.15}{20.158}{-1.55}{0.2}{0}\ertext{19.472}{-1.9}{}{absolute}
\eret{12.961}{-3.4}{18.286}{-1.55}{0.2}{0}\ertext{13.109}{-1.9}{l}{entity\textunderscore type\textunderscore like}
\eret{13.211}{-3.15}{15.811}{-2.15}{0.2}{1}\ertext{14.511}{-2.5}{}{entity\textunderscore type}
\eret{16.311}{-3.05}{18.036}{-2.15}{0.2}{1}\ertext{16.483}{-2.5}{l}{group}
\erattr{16.511}{-2.7}{0}{1}{annotation}
\eret{33.213}{-7.95}{46.066}{-6.75}{0.2}{1}\ertext{34.499}{-7.1}{l}{attribute}
\erattr{33.413}{-7.3}{1}{0}{name}
\erattr{33.413}{-7.6}{0}{1}{description}
\eret{8.808}{-6.65}{10.461}{-5.75}{0.2}{1}\ertext{9.635}{-6.1}{}{dependency}\ertext{9.635}{-6.4}{}{group}
\eret{11.461}{-12.05}{27.213}{-5.75}{0.2}{1}\ertext{11.965}{-6.1}{l}{Relationship}
\erattr{11.661}{-6.3}{0}{0}{name}
\erattr{11.661}{-6.6}{0}{1}{description}
\erattr{11.661}{-6.9}{0}{1}{id}
\erattr{11.661}{-7.2}{0}{1}{scope}
\erattr{11.661}{-7.5}{0}{1}{physicalprefix}
\eret{15.461}{-9}{26.463}{-6.1}{0.2}{0}\ertext{15.693}{-6.45}{l}{reference\textunderscore or\textunderscore dependency}
\eret{16.461}{-7.55}{18.113}{-6.95}{0.2}{1}\ertext{17.287}{-7.3}{}{dependency}
\eret{18.613}{-8.45}{24.613}{-6.45}{0.2}{1}\ertext{19.213}{-6.8}{l}{reference}
\erattr{18.813}{-7}{0}{1}{js}
\erattr{18.813}{-7.3}{0}{1}{xpathevaluate}
\eret{18.613}{-10.45}{20.613}{-9.55}{0.2}{0}\ertext{19.613}{-9.9}{}{constructed}\ertext{19.613}{-10.2}{}{relationship}
\eret{15.887}{-10.65}{17.687}{-10.05}{0.2}{0}\ertext{16.787}{-10.4}{}{composition}
\eret{1.385}{-11.95}{7.885}{-9.25}{0.2}{1}\ertext{2.035}{-9.6}{l}{presentation}
\erattr{1.585}{-9.8}{0}{1}{x}
\erattr{1.585}{-10.1}{0}{1}{y}
\erattr{1.585}{-10.4}{0}{1}{h}
\erattr{1.585}{-10.7}{0}{1}{w}
\erattr{1.585}{-11}{0}{1}{deltah}
\erattr{1.585}{-11.3}{0}{1}{deltaw}
\erattr{1.585}{-11.6}{0}{1}{sign}
\eret{9.761}{-15.35}{13.061}{-12.95}{0.2}{1}\ertext{10.091}{-13.3}{l}{path}
\erattr{9.961}{-13.5}{0}{1}{srcsign}
\erattr{9.961}{-13.8}{0}{1}{srcarmlen}
\erattr{9.961}{-14.1}{0}{1}{srcattach}
\erattr{9.961}{-14.4}{0}{1}{destsign}
\erattr{9.961}{-14.7}{0}{1}{destarmlen}
\erattr{9.961}{-15}{0}{1}{destattach}
\eret{13.561}{-13.55}{14.933}{-12.95}{0.2}{1}\ertext{14.247}{-13.3}{}{sequence}
\eret{23.205}{-13.55}{24.757}{-12.95}{0.2}{1}\ertext{23.981}{-13.3}{}{projection}
\eret{24.007}{-14.55}{26.121}{-13.95}{0.2}{1}\ertext{25.064}{-14.3}{}{identifying(1)}
\eret{24.308}{-15.75}{25.82}{-15.15}{0.2}{1}\ertext{25.064}{-15.5}{}{inherited}
\eret{25.121}{-21.1}{27.564}{-16.95}{0.2}{1}\ertext{25.316}{-17.3}{l}{cardinality}
\eret{25.521}{-18.15}{27.033}{-17.55}{0.2}{0}\ertext{26.277}{-17.9}{}{ZeroOrOne}
\eret{25.451}{-19.05}{27.103}{-18.45}{0.2}{0}\ertext{26.277}{-18.8}{}{ExactlyOne}
\eret{25.24}{-19.95}{27.314}{-19.35}{0.2}{0}\ertext{26.277}{-19.7}{}{ZeroOneOrMore}
\eret{25.521}{-20.85}{27.033}{-20.25}{0.2}{0}\ertext{26.277}{-20.6}{}{OneOrMore}
\eret{15.095}{-13.55}{16.607}{-12.95}{0.2}{1}\ertext{15.851}{-13.3}{}{transient}
\eret{16.607}{-16.3}{18.479}{-13.95}{0.2}{1}\ertext{16.757}{-14.3}{l}{initialiser}
\eret{16.857}{-15.15}{18.229}{-14.55}{0.2}{0}\ertext{17.543}{-14.9}{}{pullback}
\eret{16.877}{-16.05}{18.21}{-15.45}{0.2}{0}\ertext{17.543}{-15.8}{}{copy}
\eret{18.479}{-23.85}{24.212}{-17.95}{0.2}{1}\ertext{18.938}{-18.3}{l}{navigation}
\erattr{18.679}{-18.5}{0}{1}{xpathevaluate}
\eret{19.979}{-21.2}{22.279}{-18.85}{0.2}{0}\ertext{20.163}{-19.2}{l}{complex}
\eret{20.229}{-20.05}{22.029}{-19.45}{0.2}{1}\ertext{21.129}{-19.8}{}{join}
\eret{20.229}{-20.95}{22.029}{-20.35}{0.2}{1}\ertext{21.129}{-20.7}{}{aggregate}
\eret{20.379}{-22.7}{21.879}{-22.1}{0.2}{0}\ertext{21.129}{-22.45}{}{component}
\eret{19.297}{-23.6}{20.669}{-23}{0.2}{0}\ertext{19.983}{-23.35}{}{identity}
\eret{21.169}{-23.6}{22.962}{-23}{0.2}{0}\ertext{22.065}{-23.35}{}{theabsolute}
\eret{1.303}{-17.6}{4.653}{-12.85}{0.2}{1}\ertext{1.571}{-13.2}{l}{position}
\erattr{1.503}{-13.4}{0}{1}{d}
\eret{1.703}{-14.65}{4.403}{-13.75}{0.2}{0}\ertext{1.973}{-14.1}{l}{relative}
\erattr{1.903}{-14.3}{0}{1}{ratio}
\eret{2.386}{-15.55}{3.72}{-14.95}{0.2}{0}\ertext{3.053}{-15.3}{}{abs}
\eret{2.386}{-16.45}{3.72}{-15.85}{0.2}{0}\ertext{3.053}{-16.2}{}{local}
\eret{2.386}{-17.35}{3.72}{-16.75}{0.2}{0}\ertext{3.053}{-17.1}{}{default}
\eret{5.153}{-20.6}{7.366}{-12.85}{0.2}{1}\ertext{5.33}{-13.2}{l}{CustomShape}
\eret{5.553}{-14.05}{6.886}{-13.45}{0.2}{0}\ertext{6.22}{-13.8}{}{Top}
\eret{5.553}{-14.95}{6.886}{-14.35}{0.2}{0}\ertext{6.22}{-14.7}{}{TopLeft}
\eret{5.534}{-15.85}{6.906}{-15.25}{0.2}{0}\ertext{6.22}{-15.6}{}{TopRight}
\eret{5.323}{-16.75}{7.116}{-16.15}{0.2}{0}\ertext{6.22}{-16.5}{}{MiddleRight}
\eret{5.393}{-17.65}{7.046}{-17.05}{0.2}{0}\ertext{6.22}{-17.4}{}{MiddleLeft}
\eret{5.393}{-18.55}{7.046}{-17.95}{0.2}{0}\ertext{6.22}{-18.3}{}{BottomLeft}
\eret{5.323}{-19.45}{7.116}{-18.85}{0.2}{0}\ertext{6.22}{-19.2}{}{BottomRight}
\eret{5.553}{-20.35}{6.886}{-19.75}{0.2}{0}\ertext{6.22}{-20.1}{}{Bottom}
\eret{33.122}{-13.85}{35.616}{-8.85}{0.2}{1}\ertext{34.369}{-9.2}{}{implementationOf}
\eret{36.116}{-17.5}{39.42}{-8.85}{0.2}{1}\ertext{36.38}{-9.2}{l}{value\textunderscore type}
\eret{37.116}{-10.05}{38.449}{-9.45}{0.2}{0}\ertext{37.782}{-9.8}{}{boolean}
\eret{37.116}{-10.95}{38.449}{-10.35}{0.2}{0}\ertext{37.782}{-10.7}{}{date}
\eret{37.096}{-11.85}{38.468}{-11.25}{0.2}{0}\ertext{37.782}{-11.6}{}{dateTime}
\eret{37.116}{-12.75}{38.449}{-12.15}{0.2}{0}\ertext{37.782}{-12.5}{}{integer}
\eret{37.116}{-13.65}{38.449}{-13.05}{0.2}{0}\ertext{37.782}{-13.4}{}{float}
\eret{36.395}{-14.55}{39.17}{-13.95}{0.2}{0}\ertext{37.782}{-14.3}{}{nonNegativeInteger}
\eret{36.605}{-15.45}{38.959}{-14.85}{0.2}{0}\ertext{37.782}{-15.2}{}{positiveInteger}
\eret{37.116}{-16.35}{38.449}{-15.75}{0.2}{0}\ertext{37.782}{-16.1}{}{string}
\eret{37.116}{-17.25}{38.449}{-16.65}{0.2}{0}\ertext{37.782}{-17}{}{time}
\eret{39.92}{-9.45}{42.133}{-8.85}{0.2}{1}\ertext{41.026}{-9.2}{}{identifying(2)}
\eret{42.633}{-9.75}{44.005}{-8.85}{0.2}{1}\ertext{42.77}{-9.2}{l}{optional}
\erattr{42.833}{-9.4}{0}{1}{value}
\eret{44.505}{-9.45}{46.158}{-8.85}{0.2}{1}\ertext{45.331}{-9.2}{}{deprecated}
\eret{8.578}{-18.15}{11.578}{-16.95}{0.2}{1}\ertext{8.878}{-17.3}{l}{label}
\erattr{8.778}{-17.5}{0}{1}{xAdjustment}
\erattr{8.778}{-17.8}{0}{1}{yAdjustment}
\eret{12.078}{-19.3}{13.911}{-16.95}{0.2}{1}\ertext{12.224}{-17.3}{l}{src\textunderscore or\textunderscore dest}
\eret{12.328}{-18.15}{13.661}{-17.55}{0.2}{0}\ertext{12.994}{-17.9}{}{ToSrc}
\eret{12.328}{-19.05}{13.661}{-18.45}{0.2}{0}\ertext{12.994}{-18.8}{}{ToDest}
\eret{14.411}{-19.3}{16.244}{-16.95}{0.2}{1}\ertext{14.557}{-17.3}{l}{step}
\eret{14.661}{-18.15}{15.994}{-17.55}{0.2}{0}\ertext{15.327}{-17.9}{}{vstep}
\eret{14.661}{-19.05}{15.994}{-18.45}{0.2}{0}\ertext{15.327}{-18.8}{}{hstep}
\eret{13.494}{-22.25}{17.161}{-20.2}{0.2}{1}\ertext{13.658}{-20.55}{l}{dimension}
\erattr{13.694}{-20.75}{0}{1}{src}
\erattr{13.694}{-21.05}{0}{1}{dest}
\eret{13.744}{-22}{15.077}{-21.4}{0.2}{0}\ertext{14.411}{-21.75}{}{reldim}
\eret{15.577}{-22}{16.911}{-21.4}{0.2}{0}\ertext{16.244}{-21.75}{}{absdim}
\eret{7.594}{-23.65}{9.427}{-20.4}{0.2}{1}\ertext{7.74}{-20.75}{l}{render}
\eret{7.844}{-21.6}{9.177}{-21}{0.2}{0}\ertext{8.51}{-21.35}{}{None}
\eret{7.844}{-22.5}{9.177}{-21.9}{0.2}{0}\ertext{8.51}{-22.25}{}{Split}
\eret{7.844}{-23.4}{9.177}{-22.8}{0.2}{0}\ertext{8.51}{-23.15}{}{NoSplit}
\eret{9.927}{-24.55}{12.561}{-20.4}{0.2}{1}\ertext{10.138}{-20.75}{l}{relative\textunderscore position}
\eret{10.177}{-21.6}{11.51}{-21}{0.2}{0}\ertext{10.844}{-21.35}{}{Right}
\eret{10.177}{-22.5}{11.51}{-21.9}{0.2}{0}\ertext{10.844}{-22.25}{}{Left}
\eret{10.177}{-23.4}{11.51}{-22.8}{0.2}{0}\ertext{10.844}{-23.15}{}{Upside}
\eret{10.158}{-24.3}{11.53}{-23.7}{0.2}{0}\ertext{10.844}{-24.05}{}{Downside}
\eret{46.658}{-12.1}{49.63}{-8.85}{0.2}{1}\ertext{46.895}{-9.2}{l}{xmlStyle(2)}
\eret{47.658}{-10.05}{49.17}{-9.45}{0.2}{0}\ertext{48.414}{-9.8}{}{Attribute}
\eret{47.587}{-10.95}{49.24}{-10.35}{0.2}{0}\ertext{48.414}{-10.7}{}{Element(2)}
\eret{47.447}{-11.85}{49.38}{-11.25}{0.2}{0}\ertext{48.414}{-11.6}{}{Anonymous(2)}
\eret{9.013}{-9.9}{11.256}{-7.55}{0.2}{1}\ertext{9.193}{-7.9}{l}{xmlStyle(1)}
\eret{9.213}{-8.75}{10.866}{-8.15}{0.2}{0}\ertext{10.039}{-8.5}{}{Element(1)}
\eret{9.073}{-9.65}{11.006}{-9.05}{0.2}{0}\ertext{10.039}{-9.4}{}{Anonymous(1)}
\eret{0}{-0.2}{49.63}{0.3}{0.2}{1}

% relationship 
\ertext{16.625}{-0.5}{l}{}\errelarm{16.475}{-0.2}{16.475}{-0.875}{0}{0}\errelarm{16.475}{-0.875}{16.475}{-1.55}{1}{0}\ercrowfoot{16.475}{-1.4}{16.325}{-1.55}{16.475}{-1.55}{16.625}{-1.55}{0}
% relationship 
\ertext{19.622}{-0.5}{l}{}\errelarm{19.472}{-0.2}{19.472}{-0.875}{1}{0}\errelarm{19.472}{-0.875}{19.472}{-1.55}{1}{0}
% relationship presentation
\ertext{2.317}{-0.5}{l}{presentation}\errelarm{2.167}{-0.2}{2.167}{-0.6}{0}{0}\errelarm{2.167}{-0.6}{2.167}{-1}{1}{0}
% relationship 
\ertext{5.047}{-0.5}{l}{}\errelarm{4.897}{-0.2}{4.897}{-0.6}{0}{0}\errelarm{4.897}{-0.6}{4.897}{-1}{1}{0}
% relationship 
\ertext{48.294}{-0.5}{l}{}\errelarm{48.144}{-0.2}{48.144}{-0.6}{0}{0}\errelarm{48.144}{-0.6}{48.144}{-1}{1}{0}
% relationship attributeDefault
\ertext{48.294}{-2.2}{l}{attributeDefault}\errelarm{48.144}{-1.9}{48.144}{-5.375}{0}{0}\errelarm{48.144}{-5.375}{48.144}{-8.85}{1}{0}
% relationship 
\ertext{11.009}{-4.05}{l}{}\errelarm{10.859}{-3.75}{10.859}{-3.825}{0}{0}\errelarm{4.635}{-8.875}{4.635}{-9.25}{1}{0}\errelangle{10.859}{-3.825}{10.859}{-3.9}{7.747}{-6.2}{0}{0}\errelangle{7.747}{-6.2}{4.635}{-8.5}{4.635}{-8.875}{1}{0}
% relationship 
\ertext{11.408}{-4.05}{l}{}\errelarm{11.258}{-3.75}{11.258}{-3.825}{0}{0}\errelarm{9.635}{-5.625}{9.635}{-5.75}{1}{0}\errelangle{11.258}{-3.825}{11.258}{-3.9}{10.446}{-4.7}{0}{0}\errelangle{10.446}{-4.7}{9.635}{-5.5}{9.635}{-5.625}{1}{0}\ercrowfoot{9.635}{-5.6}{9.485}{-5.75}{9.635}{-5.75}{9.785}{-5.75}{0}
% relationship 
\ertext{20.179}{-4.05}{l}{}\ertext{20.179}{-5.6}{l}{..}\errelarm{20.029}{-3.75}{20.029}{-4.75}{0}{0}\errelarm{20.029}{-4.75}{20.029}{-5.75}{1}{0}\eridcomprel{19.928999999999995}{20.128999999999998}{-5.5}\ercrowfoot{20.029}{-5.6}{19.879}{-5.75}{20.029}{-5.75}{20.179}{-5.75}{0}
% relationship 
\ertext{23.368}{-4.05}{l}{}\ertext{39.79}{-6.6}{l}{..}\errelarm{23.218}{-3.75}{23.218}{-3.825}{0}{0}\errelarm{39.64}{-6.538}{39.64}{-6.75}{1}{0}\errelangle{23.218}{-3.825}{23.218}{-3.9}{31.429}{-5.113}{0}{0}\errelangle{31.429}{-5.113}{39.64}{-6.325}{39.64}{-6.538}{1}{0}\eridcomprel{39.53955}{39.73955}{-6.5}\ercrowfoot{39.64}{-6.6}{39.49}{-6.75}{39.64}{-6.75}{39.79}{-6.75}{0}
% relationship 
\ertext{14.868}{-3.7}{l}{}\errelarm{14.718}{-3.4}{14.718}{-3.5}{0}{0}\errelarm{14.718}{-1.4}{14.718}{-1.55}{1}{0}\errelangle{14.718}{-3.5}{14.718}{-3.6}{13.718}{-3.6}{0}{0}\errelangle{14.718}{-1.4}{14.718}{-1.25}{13.718}{-1.25}{1}{0}\errelangle{13.718}{-3.6}{12.718}{-3.6}{12.718}{-2.425}{0}{0}\errelangle{12.718}{-2.425}{12.718}{-1.25}{13.718}{-1.25}{1}{0}\ercrowfoot{14.718}{-1.4}{14.568}{-1.55}{14.718}{-1.55}{14.868}{-1.55}{0}\erarc{14.123}{-1.45}{14.873}{-1.25}{16.373}{-1.25}{17.123}{-1.45}
% relationship xmlRepresentation
\ertext{15.311}{-3.6}{l}{xmlRepresentation}\errelarm{15.161}{-3.15}{15.161}{-3.95}{0}{0}\errelarm{10.807}{-7.5}{10.807}{-7.55}{1}{0}\errelangle{15.161}{-3.95}{15.161}{-4.75}{12.984}{-4.75}{0}{0}\errelangle{10.807}{-7.5}{10.807}{-7.45}{10.807}{-7.45}{1}{0}\errelangle{12.984}{-4.75}{10.807}{-4.75}{10.807}{-6.1}{0}{0}\errelangle{10.807}{-6.1}{10.807}{-7.45}{10.807}{-7.45}{1}{0}
% relationship 
\ertext{34.519}{-8.25}{l}{}\ertext{34.519}{-8.7}{l}{..}\errelarm{34.369}{-7.95}{34.369}{-8.4}{0}{0}\errelarm{34.369}{-8.4}{34.369}{-8.85}{1}{0}
% relationship type
\ertext{37.918}{-8.25}{l}{type}\errelarm{37.768}{-7.95}{37.768}{-8.4}{1}{0}\errelarm{37.768}{-8.4}{37.768}{-8.85}{1}{0}
% relationship 
\ertext{41.176}{-8.25}{l}{}\errelarm{41.026}{-7.95}{41.026}{-8.4}{0}{0}\errelarm{41.026}{-8.4}{41.026}{-8.85}{1}{0}
% relationship 
\ertext{43.469}{-8.25}{l}{}\errelarm{43.319}{-7.95}{43.319}{-8.4}{0}{0}\errelarm{43.319}{-8.4}{43.319}{-8.85}{1}{0}
% relationship 
\ertext{45.481}{-8.25}{l}{}\errelarm{45.331}{-7.95}{45.331}{-8.4}{0}{0}\errelarm{45.331}{-8.4}{45.331}{-8.85}{1}{0}
% relationship xmlStyle
\ertext{45.894}{-8.25}{l}{xmlStyle}\errelarm{45.744}{-7.95}{45.744}{-8.15}{0}{0}\errelarm{47.549}{-8.7}{47.549}{-8.85}{1}{0}\errelangle{45.744}{-8.15}{45.744}{-8.35}{46.647}{-8.45}{0}{0}\errelangle{46.647}{-8.45}{47.549}{-8.55}{47.549}{-8.7}{1}{0}
% relationship 
\ertext{9.785}{-6.95}{l}{}\errelarm{9.635}{-6.65}{9.635}{-6.725}{0}{0}\errelarm{6.26}{-8.925}{6.26}{-9.25}{1}{0}\errelangle{9.635}{-6.725}{9.635}{-6.8}{7.947}{-7.7}{0}{0}\errelangle{7.947}{-7.7}{6.26}{-8.6}{6.26}{-8.925}{1}{0}
% relationship diagram
\ertext{11.705}{-12.35}{r}{diagram}\errelarm{11.855}{-12.05}{11.855}{-12.5}{0}{0}\errelarm{11.855}{-12.5}{11.855}{-12.95}{1}{0}
% relationship 
\ertext{14.397}{-12.35}{l}{}\errelarm{14.247}{-12.05}{14.247}{-12.5}{0}{0}\errelarm{14.247}{-12.5}{14.247}{-12.95}{1}{0}
% relationship 
\ertext{25.214}{-12.35}{l}{}\errelarm{25.064}{-12.05}{25.064}{-13}{0}{0}\errelarm{25.064}{-13}{25.064}{-13.95}{1}{0}
% relationship cardinality
\ertext{26.492}{-12.35}{l}{cardinality}\errelarm{26.342}{-12.05}{26.342}{-14.5}{1}{0}\errelarm{26.342}{-14.5}{26.342}{-16.95}{1}{0}
% relationship type
\ertext{27.363}{-7.94}{l}{type}\errelarm{27.213}{-7.64}{27.313}{-7.64}{1}{0}\errelarm{26.608}{-3.2}{26.408}{-3.2}{0}{0}\errelangle{27.313}{-7.64}{27.413}{-7.64}{27.563}{-7.64}{1}{0}\errelangle{26.608}{-3.2}{26.808}{-3.2}{27.261}{-3.2}{0}{0}\errelangle{27.563}{-7.64}{27.713}{-7.64}{27.713}{-5.42}{1}{0}\errelangle{27.713}{-5.42}{27.713}{-3.2}{27.261}{-3.2}{0}{0}\ercrowfoot{27.363}{-7.64}{27.213}{-7.49}{27.213}{-7.64}{27.213}{-7.79}{0}
% relationship diagonal
\ertext{21.463}{-8.75}{r}{diagonal}\errelarm{21.613}{-8.45}{21.613}{-13.2}{0}{0}\errelarm{21.613}{-13.2}{21.613}{-17.95}{1}{0}
% relationship riser
\ertext{22.063}{-8.75}{l}{riser}\errelarm{21.913}{-8.45}{21.913}{-13.2}{0}{0}\errelarm{21.913}{-13.2}{21.913}{-17.95}{1}{0}
% relationship key
\ertext{23.263}{-8.75}{l}{key}\errelarm{23.113}{-8.45}{23.113}{-13.2}{0}{0}\errelarm{23.113}{-13.2}{23.113}{-17.95}{1}{0}
% relationship 
\ertext{24.131}{-8.75}{l}{}\errelarm{23.981}{-8.45}{23.981}{-10.7}{0}{0}\errelarm{23.981}{-10.7}{23.981}{-12.95}{1}{0}
% relationship inverse
\ertext{24.763}{-7.15}{l}{inverse}\ertext{24.763}{-6.7}{l}{inverse}\errelarm{24.613}{-6.85}{25.113}{-6.85}{0}{0}\ertext{25.263}{-6.85}{l}{\textasciitilde /..=type}\errelarm{25.113}{-6.85}{25.113}{-6.85}{0}{0}\errelarm{25.113}{-6.85}{24.613}{-6.85}{0}{0}
% relationship 
\ertext{20.349}{-10.75}{l}{}\errelarm{20.199}{-10.45}{20.199}{-14.2}{1}{0}\errelarm{20.199}{-14.2}{20.199}{-17.95}{1}{0}
% relationship inverse
\ertext{20.763}{-10.3}{l}{inverse}\ertext{20.763}{-9.85}{l}{inverse}\errelarm{20.613}{-10}{21.113}{-10}{0}{0}\errelarm{21.113}{-10}{21.113}{-10}{0}{0}\errelarm{21.113}{-10}{20.613}{-10}{0}{0}
% relationship 
\ertext{16.637}{-10.95}{l}{}\errelarm{16.487}{-10.65}{16.487}{-10.725}{0}{0}\errelarm{15.851}{-12.9}{15.851}{-12.95}{1}{0}\errelangle{16.487}{-10.725}{16.487}{-10.8}{16.169}{-11.825}{0}{0}\errelangle{16.169}{-11.825}{15.851}{-12.85}{15.851}{-12.9}{1}{0}
% relationship 
\ertext{17.237}{-10.95}{l}{}\ertext{17.693}{-13.8}{l}{..}\errelarm{17.087}{-10.65}{17.087}{-10.725}{0}{0}\errelarm{17.543}{-13.4}{17.543}{-13.95}{1}{0}\errelangle{17.087}{-10.725}{17.087}{-10.8}{17.315}{-11.825}{0}{0}\errelangle{17.315}{-11.825}{17.543}{-12.85}{17.543}{-13.4}{1}{0}
% relationship inverse
\ertext{15.737}{-10.65}{r}{inverse}\ertext{16.311}{-7.1}{r}{of}\ertext{16.311}{-6.8}{r}{inverse}\errelarm{15.887}{-10.35}{15.287}{-10.35}{0}{0}\errelarm{16.261}{-7.25}{16.461}{-7.25}{0}{0}\errelangle{15.287}{-10.35}{14.687}{-10.35}{14.587}{-10.35}{0}{0}\errelangle{16.261}{-7.25}{16.061}{-7.25}{15.274}{-7.25}{0}{0}\errelangle{14.587}{-10.35}{14.487}{-10.35}{14.487}{-8.8}{0}{0}\errelangle{14.487}{-8.8}{14.487}{-7.25}{15.274}{-7.25}{0}{0}
% relationship xl
\ertext{2.51}{-12.25}{l}{xl}\errelarm{2.36}{-11.95}{2.36}{-12.4}{0}{0}\errelarm{2.36}{-12.4}{2.36}{-12.85}{1}{0}
% relationship xc
\ertext{2.9}{-12.25}{l}{xc}\errelarm{2.75}{-11.95}{2.75}{-12.85}{0}{0}\errelarm{2.75}{-12.85}{2.75}{-13.75}{1}{0}
% relationship xr
\ertext{3.29}{-12.25}{l}{xr}\errelarm{3.14}{-11.95}{3.14}{-12.85}{0}{0}\errelarm{3.14}{-12.85}{3.14}{-13.75}{1}{0}
% relationship yt
\ertext{3.68}{-12.25}{l}{yt}\errelarm{3.53}{-11.95}{3.53}{-12.4}{0}{0}\errelarm{3.53}{-12.4}{3.53}{-12.85}{1}{0}
% relationship ym
\ertext{4.07}{-12.25}{l}{ym}\errelarm{3.92}{-11.95}{3.92}{-12.85}{0}{0}\errelarm{3.92}{-12.85}{3.92}{-13.75}{1}{0}
% relationship yb
\ertext{4.525}{-12.25}{l}{yb}\errelarm{4.375}{-11.95}{4.375}{-12.15}{0}{0}\errelarm{4.349}{-13.7}{4.349}{-13.75}{1}{0}\errelangle{4.375}{-12.15}{4.375}{-12.35}{4.362}{-13}{0}{0}\errelangle{4.362}{-13}{4.349}{-13.65}{4.349}{-13.7}{1}{0}
% relationship shape
\ertext{6.41}{-12.25}{l}{shape}\errelarm{6.26}{-11.95}{6.26}{-12.4}{0}{0}\errelarm{6.26}{-12.4}{6.26}{-12.85}{1}{0}
% relationship name
\ertext{7.872}{-12.25}{l}{name}\errelarm{7.722}{-11.95}{7.722}{-12.15}{0}{0}\errelarm{8.235}{-20.35}{8.235}{-20.4}{1}{0}\errelangle{7.722}{-12.15}{7.722}{-12.35}{7.979}{-16.325}{0}{0}\errelangle{7.979}{-16.325}{8.235}{-20.3}{8.235}{-20.35}{1}{0}
% relationship rightOf
\ertext{1.235}{-10.45}{r}{rightOf}\errelarm{1.385}{-10.15}{1.135}{-10.15}{0}{0}\errelarm{10.261}{-3.475}{10.461}{-3.475}{0}{0}\errelangle{1.135}{-10.15}{0.885}{-10.15}{0.785}{-10.15}{0}{0}\errelangle{10.261}{-3.475}{10.061}{-3.475}{5.373}{-3.475}{0}{0}\ertext{0.535}{-7.113}{r}{\textasciitilde /\textasciicircum =\textasciicircum }\errelangle{0.785}{-10.15}{0.685}{-10.15}{0.685}{-6.813}{0}{0}\errelangle{0.685}{-6.813}{0.685}{-3.475}{5.373}{-3.475}{0}{0}\ercrowfoot{1.235}{-10.15}{1.385}{-10}{1.385}{-10.15}{1.385}{-10.3}{0}
% relationship below
\ertext{1.235}{-11.35}{r}{below}\errelarm{1.385}{-11.05}{0.935}{-11.05}{0}{0}\errelarm{10.261}{-3.2}{10.461}{-3.2}{0}{0}\errelangle{0.935}{-11.05}{0.485}{-11.05}{0.435}{-11.05}{0}{0}\errelangle{10.261}{-3.2}{10.061}{-3.2}{5.223}{-3.2}{0}{0}\ertext{0.235}{-7.425}{r}{\textasciitilde /\textasciicircum =\textasciicircum }\errelangle{0.435}{-11.05}{0.385}{-11.05}{0.385}{-7.125}{0}{0}\errelangle{0.385}{-7.125}{0.385}{-3.2}{5.223}{-3.2}{0}{0}\ercrowfoot{1.235}{-11.05}{1.385}{-10.9}{1.385}{-11.05}{1.385}{-11.2}{0}\erarc{4.235}{-9.05}{4.86}{-8.85}{6.11}{-8.85}{6.735}{-9.05}
% relationship 
\ertext{10.241}{-15.65}{l}{}\errelarm{10.091}{-15.35}{10.091}{-16.15}{0}{0}\errelarm{10.091}{-16.15}{10.091}{-16.95}{1}{0}
% relationship inverse
\ertext{11.101}{-15.65}{l}{inverse}\errelarm{10.751}{-15.35}{10.751}{-16.15}{0}{0}\errelarm{10.751}{-16.15}{10.751}{-16.95}{1}{0}
% relationship id
\ertext{11.1}{-15.95}{l}{id}\errelarm{10.85}{-15.35}{10.85}{-16.15}{0}{0}\errelarm{10.85}{-16.15}{10.85}{-16.95}{1}{0}
% relationship scope
\ertext{11.099}{-16.25}{l}{scope}\errelarm{10.949}{-15.35}{10.949}{-16.15}{0}{0}\errelarm{10.949}{-16.15}{10.949}{-16.95}{1}{0}
% relationship align
\ertext{12.056}{-15.65}{l}{align}\errelarm{11.906}{-15.35}{11.906}{-15.55}{0}{0}\errelarm{12.994}{-16.9}{12.994}{-16.95}{1}{0}\errelangle{11.906}{-15.55}{11.906}{-15.75}{12.45}{-16.3}{0}{0}\errelangle{12.45}{-16.3}{12.994}{-16.85}{12.994}{-16.9}{1}{0}
% relationship 
\ertext{12.881}{-15.65}{l}{}\errelarm{12.731}{-15.35}{12.731}{-15.425}{0}{0}\errelarm{15.327}{-16.75}{15.327}{-16.95}{1}{0}\errelangle{12.731}{-15.425}{12.731}{-15.5}{14.029}{-16.025}{0}{0}\errelangle{14.029}{-16.025}{15.327}{-16.55}{15.327}{-16.75}{1}{0}
% relationship 
\ertext{25.214}{-14.85}{l}{}\errelarm{25.064}{-14.55}{25.064}{-14.85}{0}{0}\errelarm{25.064}{-14.85}{25.064}{-15.15}{1}{0}
% relationship along
\ertext{17.019}{-16.6}{r}{along}\errelarm{17.169}{-16.3}{17.169}{-16.6}{1}{0}\errelarm{19.052}{-17.9}{19.052}{-17.95}{1}{0}\errelangle{17.169}{-16.6}{17.169}{-16.9}{18.111}{-17.375}{1}{0}\errelangle{18.111}{-17.375}{19.052}{-17.85}{19.052}{-17.9}{1}{0}
% relationship type
\ertext{18.629}{-15.798}{l}{type}\errelarm{18.479}{-15.948}{19.079}{-15.948}{1}{0}\errelarm{26.608}{-3.063}{26.408}{-3.063}{0}{0}\errelangle{19.079}{-15.948}{19.679}{-15.948}{23.779}{-15.948}{1}{0}\errelangle{26.608}{-3.063}{26.808}{-3.063}{27.343}{-3.063}{0}{0}\errelangle{23.779}{-15.948}{27.879}{-15.948}{27.879}{-9.505}{1}{0}\errelangle{27.879}{-9.505}{27.879}{-3.063}{27.343}{-3.063}{0}{0}\ercrowfoot{18.629}{-15.948}{18.479}{-15.798}{18.479}{-15.948}{18.479}{-16.098}{0}
% relationship projectionrel
\ertext{18.679}{-14.97}{l}{projection}\ertext{18.679}{-15.27}{l}{rel}\errelarm{18.229}{-14.67}{18.529}{-14.67}{0}{0}\errelarm{18.413}{-7.95}{18.613}{-7.95}{0}{0}\ertext{18.671}{-11.61}{l}{\textasciitilde /..=../type}\errelangle{18.529}{-14.67}{18.829}{-14.67}{18.521}{-11.31}{0}{0}\errelangle{18.521}{-11.31}{18.213}{-7.95}{18.413}{-7.95}{0}{0}\ercrowfoot{18.379}{-14.67}{18.229}{-14.52}{18.229}{-14.67}{18.229}{-14.82}{0}
% relationship riser2
\ertext{17.96}{-16.65}{l}{riser2}\errelarm{17.81}{-16.05}{17.81}{-16.3}{1}{0}\errelarm{19.626}{-17.9}{19.626}{-17.95}{1}{0}\errelangle{17.81}{-16.3}{17.81}{-16.55}{18.718}{-17.2}{1}{0}\errelangle{18.718}{-17.2}{19.626}{-17.85}{19.626}{-17.9}{1}{0}
% relationship src
\ertext{24.362}{-22.97}{l}{src}\errelarm{24.212}{-22.67}{24.662}{-22.67}{0}{0}\errelarm{26.608}{-2.238}{26.408}{-2.238}{0}{0}\errelangle{24.662}{-22.67}{25.112}{-22.67}{27.512}{-22.67}{0}{0}\errelangle{26.608}{-2.238}{26.808}{-2.238}{28.36}{-2.238}{0}{0}\errelangle{27.512}{-22.67}{29.912}{-22.67}{29.912}{-12.454}{0}{0}\errelangle{29.912}{-12.454}{29.912}{-2.238}{28.36}{-2.238}{0}{0}\ercrowfoot{24.362}{-22.67}{24.212}{-22.52}{24.212}{-22.67}{24.212}{-22.82}{0}
% relationship dest
\ertext{24.362}{-23.56}{l}{dest}\errelarm{24.212}{-23.26}{24.762}{-23.26}{0}{0}\errelarm{26.608}{-2.1}{26.408}{-2.1}{0}{0}\errelangle{24.762}{-23.26}{25.312}{-23.26}{27.712}{-23.26}{0}{0}\errelangle{26.608}{-2.1}{26.808}{-2.1}{28.46}{-2.1}{0}{0}\errelangle{27.712}{-23.26}{30.112}{-23.26}{30.112}{-12.68}{0}{0}\errelangle{30.112}{-12.68}{30.112}{-2.1}{28.46}{-2.1}{0}{0}\ercrowfoot{24.362}{-23.26}{24.212}{-23.11}{24.212}{-23.26}{24.212}{-23.41}{0}\erarc{18.479}{-17.75}{19.729}{-17.55}{22.229}{-17.55}{23.479}{-17.75}
% relationship 
\ertext{21.279}{-21.5}{l}{}\errelarm{21.129}{-21.2}{21.129}{-21.65}{0}{0}\errelarm{21.129}{-21.65}{21.129}{-22.1}{1}{0}\errelseq{21.189}{-21.65}{20.779}{-21.71}{21.479}{-21.77}{21.069}{-21.83}\ercrowfoot{21.129}{-21.95}{20.979}{-22.1}{21.129}{-22.1}{21.279}{-22.1}{0}
% relationship rel
\ertext{22.029}{-22.7}{l}{rel}\errelarm{21.879}{-22.4}{22.029}{-22.4}{1}{0}\errelarm{27.413}{-10.475}{27.213}{-10.475}{0}{0}\errelangle{22.029}{-22.4}{22.179}{-22.4}{25.429}{-22.4}{1}{0}\errelangle{27.413}{-10.475}{27.613}{-10.475}{28.146}{-10.475}{0}{0}\ertext{28.529}{-16.738}{r}{\textasciitilde /..=src}\errelangle{25.429}{-22.4}{28.679}{-22.4}{28.679}{-16.438}{1}{0}\errelangle{28.679}{-16.438}{28.679}{-10.475}{28.146}{-10.475}{0}{0}\ercrowfoot{22.029}{-22.4}{21.879}{-22.25}{21.879}{-22.4}{21.879}{-22.55}{0}\erarc{1.853}{-12.65}{2.578}{-12.45}{4.028}{-12.45}{4.753}{-12.65}
% relationship to
\ertext{1.553}{-14.5}{r}{to}\errelarm{1.703}{-14.2}{1.103}{-14.2}{1}{0}\errelarm{10.261}{-2.925}{10.461}{-2.925}{0}{0}\errelangle{1.103}{-14.2}{0.503}{-14.2}{0.303}{-14.2}{1}{0}\errelangle{10.261}{-2.925}{10.061}{-2.925}{5.082}{-2.925}{0}{0}\ertext{-0.047}{-8.863}{r}{\textasciitilde /\textasciicircum =\textasciicircum }\errelangle{0.303}{-14.2}{0.103}{-14.2}{0.103}{-8.563}{1}{0}\errelangle{0.103}{-8.563}{0.103}{-2.925}{5.082}{-2.925}{0}{0}\ercrowfoot{1.553}{-14.2}{1.703}{-14.05}{1.703}{-14.2}{1.703}{-14.35}{0}
% relationship destattr
\ertext{32.972}{-9.7}{r}{destattr}\errelarm{33.122}{-9.85}{32.672}{-9.85}{1}{0}\errelarm{33.013}{-7.59}{33.213}{-7.59}{0}{0}\errelangle{32.672}{-9.85}{32.222}{-9.85}{32.197}{-9.85}{1}{0}\errelangle{33.013}{-7.59}{32.813}{-7.59}{32.492}{-7.59}{0}{0}\ertext{33.322}{-9.02}{r}{\textasciitilde /..=rel/type}\errelangle{32.197}{-9.85}{32.172}{-9.85}{32.172}{-8.72}{1}{0}\errelangle{32.172}{-8.72}{32.172}{-7.59}{32.492}{-7.59}{0}{0}\ercrowfoot{32.972}{-9.85}{33.122}{-9.7}{33.122}{-9.85}{33.122}{-10}{0}
% relationship attrOfOrigin
\ertext{32.972}{-10.65}{r}{attrOfOrigin}\errelarm{33.122}{-10.35}{32.622}{-10.35}{1}{0}\errelarm{33.013}{-7.35}{33.213}{-7.35}{0}{0}\errelangle{32.622}{-10.35}{32.122}{-10.35}{32.097}{-10.35}{1}{0}\errelangle{33.013}{-7.35}{32.813}{-7.35}{32.442}{-7.35}{0}{0}\ertext{33.222}{-9.15}{r}{\textasciitilde /..=typeOfOrigin}\errelangle{32.097}{-10.35}{32.072}{-10.35}{32.072}{-8.85}{1}{0}\errelangle{32.072}{-8.85}{32.072}{-7.35}{32.442}{-7.35}{0}{0}\ercrowfoot{32.972}{-10.35}{33.122}{-10.2}{33.122}{-10.35}{33.122}{-10.5}{0}
% relationship typeOfOrigin
\ertext{32.972}{-12.15}{r}{typeOfOrigin}\errelarm{33.122}{-11.85}{32.622}{-11.85}{1}{0}\errelarm{26.608}{-1.55}{26.408}{-1.55}{0}{0}\errelangle{32.622}{-11.85}{32.122}{-11.85}{31.622}{-11.85}{1}{0}\errelangle{26.608}{-1.55}{26.808}{-1.55}{28.965}{-1.55}{0}{0}\ertext{32.272}{-7}{r}{\textasciitilde /\textasciicircum =\textasciicircum }\errelangle{31.622}{-11.85}{31.122}{-11.85}{31.122}{-6.7}{1}{0}\errelangle{31.122}{-6.7}{31.122}{-1.55}{28.965}{-1.55}{0}{0}\ercrowfoot{32.972}{-11.85}{33.122}{-11.7}{33.122}{-11.85}{33.122}{-12}{0}
% relationship rel
\ertext{32.972}{-13.15}{r}{rel}\errelarm{33.122}{-12.85}{32.522}{-12.85}{1}{0}\errelarm{26.663}{-8.42}{26.463}{-8.42}{0}{0}\ertext{29.942}{-11.385}{l}{\textasciitilde /..=../..}\errelangle{32.522}{-12.85}{31.922}{-12.85}{29.392}{-10.635}{1}{0}\errelangle{29.392}{-10.635}{26.863}{-8.42}{26.663}{-8.42}{0}{0}\ercrowfoot{32.972}{-12.85}{33.122}{-12.7}{33.122}{-12.85}{33.122}{-13}{0}
% relationship name
\ertext{9.728}{-18.45}{l}{name}\errelarm{9.578}{-18.15}{9.578}{-18.35}{0}{0}\errelarm{8.785}{-20.35}{8.785}{-20.4}{1}{0}\errelangle{9.578}{-18.35}{9.578}{-18.55}{9.181}{-19.425}{0}{0}\errelangle{9.181}{-19.425}{8.785}{-20.3}{8.785}{-20.35}{1}{0}
% relationship position
\ertext{10.728}{-18.45}{l}{position}\errelarm{10.578}{-18.15}{10.578}{-18.35}{0}{0}\errelarm{11.244}{-20.35}{11.244}{-20.4}{1}{0}\errelangle{10.578}{-18.35}{10.578}{-18.55}{10.911}{-19.425}{0}{0}\errelangle{10.911}{-19.425}{11.244}{-20.3}{11.244}{-20.35}{1}{0}\erarc{9.828}{-16.85}{10.203}{-16.65}{10.953}{-16.65}{11.328}{-16.85}
% relationship 
\ertext{15.172}{-19.6}{l}{}\errelarm{15.022}{-19.3}{15.022}{-19.375}{0}{0}\errelarm{15.052}{-16.75}{15.052}{-16.95}{1}{0}\errelangle{15.022}{-19.375}{15.022}{-19.45}{14.622}{-19.45}{0}{0}\errelangle{15.052}{-16.75}{15.052}{-16.55}{14.637}{-16.55}{1}{0}\errelangle{14.622}{-19.45}{14.222}{-19.45}{14.222}{-18}{0}{0}\errelangle{14.222}{-18}{14.222}{-16.55}{14.637}{-16.55}{1}{0}
% relationship 
\ertext{15.783}{-19.6}{l}{}\errelarm{15.633}{-19.3}{15.633}{-19.75}{0}{0}\errelarm{15.633}{-19.75}{15.633}{-20.2}{1}{0}\erarc{14.786}{-16.85}{14.973}{-16.65}{15.348}{-16.65}{15.536}{-16.85}\erarc{7.777}{-20.2}{8.144}{-20}{8.877}{-20}{9.244}{-20.2}\erarc{47.008}{-8.75}{47.383}{-8.55}{48.133}{-8.55}{48.508}{-8.75}
\end{erdiagram}

}
\end{frame}



\begin{frame}{Data Specification viewed as ER diagram}
\begin{center}
\scalebox{0.5}{
\begin{erdiagram}{8.1}{17.04625}

\eret{1}{-2.4}{4}{-1.5}{0.2}{1}\ertext{1.3}{-1.85}{l}{molecularStructure}
\erattr{1.2}{-2.05}{1}{0}{name}
\eret{0.454}{-5.4}{4.546}{-3.3}{0.2}{1}\ertext{0.863}{-3.65}{l}{atom}
\erdattr{0.654}{-3.85}{1}{0}{molecularStructurename(D3)}
\erattr{0.654}{-4.15}{1}{0}{atomId}
\erdattr{0.654}{-4.45}{1}{1}{elementsymbol(R1)}
\erattr{0.654}{-4.75}{1}{1}{x}
\erattr{0.654}{-5.05}{1}{1}{y}
\eret{0.554}{-8.1}{4.446}{-6.3}{0.2}{1}\ertext{0.943}{-6.65}{l}{bond formed}
\erdattr{0.754}{-6.85}{1}{0}{molecularStructurename(D4)}
\erdattr{0.754}{-7.15}{1}{0}{atomatomId(D4)}
\erdattr{0.754}{-7.45}{1}{0}{withAtomId(R2)}
\erattr{0.754}{-7.75}{1}{1}{bondType}
\eret{10.546}{-4.8}{13.546}{-3.3}{0.2}{1}\ertext{10.846}{-3.65}{l}{element}
\erattr{10.746}{-3.85}{1}{0}{symbol}
\erattr{10.746}{-4.15}{1}{1}{name}
\erattr{10.746}{-4.45}{1}{1}{atomic number}
\eret{7.046}{-7.7}{10.046}{-5.9}{0.2}{1}\ertext{7.346}{-6.25}{l}{isotope}
\erdattr{7.246}{-6.45}{1}{0}{elementsymbol(D5)}
\erattr{7.246}{-6.75}{1}{0}{numberOfNeutrons}
\erattr{7.246}{-7.05}{1}{1}{mass}
\erattr{7.246}{-7.35}{1}{1}{abundancy}
\eret{10.546}{-8}{13.546}{-5.9}{0.2}{1}\ertext{10.846}{-6.25}{l}{allotrope}
\erattr{10.746}{-6.45}{1}{0}{name}
\erdattr{10.746}{-6.75}{1}{1}{elementsymbol(D6)}
\erattr{10.746}{-7.05}{0}{1}{melting point}
\erattr{10.746}{-7.35}{0}{1}{boiling point}
\erattr{10.746}{-7.65}{0}{1}{density}
\eret{14.046}{-7.1}{17.046}{-5.9}{0.2}{1}\ertext{14.346}{-6.25}{l}{valency}
\erdattr{14.246}{-6.45}{1}{0}{elementsymbol(D7)}
\erattr{14.246}{-6.75}{1}{0}{number}
\eret{0}{-0.2}{17.046}{0.3}{0.2}{1}

% relationship 
\ertext{2.65}{-0.5}{l}{}\ertext{2.65}{-1.35}{l}{..}\errelarm{2.5}{-0.2}{2.5}{-0.85}{1}{0}\errelarm{2.5}{-0.85}{2.5}{-1.5}{1}{0}\ercrowfoot{2.5}{-1.35}{2.35}{-1.5}{2.5}{-1.5}{2.65}{-1.5}{0}
% relationship 
\ertext{12.196}{-0.5}{l}{}\ertext{12.196}{-3.15}{l}{..}\errelarm{12.046}{-0.2}{12.046}{-1.75}{1}{0}\errelarm{12.046}{-1.75}{12.046}{-3.3}{1}{0}\ercrowfoot{12.046}{-3.15}{11.896}{-3.3}{12.046}{-3.3}{12.196}{-3.3}{0}
% relationship 
\ertext{2.65}{-2.7}{l}{}\ertext{2.65}{-3.15}{l}{..}\ertext{2.65}{-2.7}{l}{D3}\errelarm{2.5}{-2.4}{2.5}{-2.85}{1}{0}\errelarm{2.5}{-2.85}{2.5}{-3.3}{1}{0}\eridcomprel{2.4}{2.6}{-3.05}\ercrowfoot{2.5}{-3.15}{2.35}{-3.3}{2.5}{-3.3}{2.65}{-3.3}{0}
% relationship 
\ertext{2.65}{-5.7}{l}{}\ertext{2.65}{-6.15}{l}{of}\ertext{2.65}{-5.7}{l}{D4}\errelarm{2.5}{-5.4}{2.5}{-5.85}{0}{0}\errelarm{2.5}{-5.85}{2.5}{-6.3}{1}{0}\eridcomprel{2.4}{2.6}{-6.05}\ercrowfoot{2.5}{-6.15}{2.35}{-6.3}{2.5}{-6.3}{2.65}{-6.3}{0}
% relationship element
\ertext{4.696}{-4.65}{l}{element}\ertext{7.696}{-4.2}{l}{R1}\errelarm{4.546}{-4.35}{7.546}{-4.35}{1}{0}\errelarm{7.546}{-4.35}{10.546}{-4.35}{0}{0}\ercrowfoot{4.696}{-4.35}{4.546}{-4.2}{4.546}{-4.35}{4.546}{-4.5}{0}
% relationship with
\ertext{4.596}{-7.5}{l}{with}\ertext{4.596}{-7.05}{l}{with}\errelarm{4.446}{-7.2}{4.946}{-7.2}{1}{0}\errelarm{4.946}{-7.2}{4.446}{-7.2}{1}{0}\ertext{5.596}{-7.05}{l}{R2}\eridrefrel{4.69625}{-7.1}{-7.299999999999999}
% relationship 
\ertext{11.446}{-5.1}{l}{}\ertext{8.696}{-5.75}{l}{..}\errelarm{11.296}{-4.8}{11.296}{-4.875}{1}{0}\errelarm{8.546}{-5.687}{8.546}{-5.9}{1}{0}\ertext{9.771}{-5.063}{r}{D5}\errelangle{11.296}{-4.875}{11.296}{-4.95}{9.921}{-5.213}{1}{0}\errelangle{9.921}{-5.213}{8.546}{-5.475}{8.546}{-5.687}{1}{0}\eridcomprel{8.446250000000001}{8.64625}{-5.6499999999999995}\ercrowfoot{8.546}{-5.75}{8.396}{-5.9}{8.546}{-5.9}{8.696}{-5.9}{0}
% relationship 
\ertext{12.196}{-5.1}{l}{}\ertext{12.196}{-5.75}{l}{..}\ertext{12.196}{-5.2}{l}{D6}\errelarm{12.046}{-4.8}{12.046}{-5.35}{1}{0}\errelarm{12.046}{-5.35}{12.046}{-5.9}{1}{0}\ercrowfoot{12.046}{-5.75}{11.896}{-5.9}{12.046}{-5.9}{12.196}{-5.9}{0}
% relationship 
\ertext{12.946}{-5.1}{l}{}\ertext{15.696}{-5.75}{l}{..}\errelarm{12.796}{-4.8}{12.796}{-4.875}{1}{0}\errelarm{15.546}{-5.687}{15.546}{-5.9}{1}{0}\ertext{14.321}{-5.063}{l}{D7}\errelangle{12.796}{-4.875}{12.796}{-4.95}{14.171}{-5.213}{1}{0}\errelangle{14.171}{-5.213}{15.546}{-5.475}{15.546}{-5.687}{1}{0}\eridcomprel{15.446250000000001}{15.64625}{-5.6499999999999995}\ercrowfoot{15.546}{-5.75}{15.396}{-5.9}{15.546}{-5.9}{15.696}{-5.9}{0}
\end{erdiagram}

}
\end{center}
\end{frame}
\begin{frame}{ER modelling Meta-Model}
\scalebox{0.23}{
\begin{erdiagram}{24.550000000000004}{49.63012500000001}

\eret{1.5}{-2.2}{2.833}{-1}{0.2}{1}\ertext{1.633}{-1.35}{l}{diagram}
\erattr{1.7}{-1.55}{0}{1}{deltaw}
\erattr{1.7}{-1.85}{0}{1}{deltah}
\eret{3.333}{-2.8}{6.461}{-1}{0.2}{1}\ertext{3.646}{-1.35}{l}{defaults}
\erattr{3.533}{-1.55}{0}{1}{etwidth}
\erattr{3.533}{-1.85}{0}{1}{etheight}
\erattr{3.533}{-2.15}{0}{1}{etyseparation}
\erattr{3.533}{-2.45}{0}{1}{etydeltaseparation}
\eret{47.04}{-1.9}{49.248}{-1}{0.2}{1}\ertext{47.261}{-1.35}{l}{xml}
\erattr{47.24}{-1.55}{0}{1}{namespaceuri}
\eret{10.461}{-3.75}{26.408}{-1}{0.2}{1}\ertext{10.681}{-1.35}{l}{ENTITY\textunderscore TYPE}
\erattr{10.661}{-1.55}{1}{0}{name}
\erattr{10.661}{-1.85}{0}{1}{description}
\erattr{10.661}{-2.15}{0}{1}{xpath}
\erattr{10.661}{-2.45}{0}{1}{modulename}
\eret{18.786}{-2.15}{20.158}{-1.55}{0.2}{0}\ertext{19.472}{-1.9}{}{absolute}
\eret{12.961}{-3.4}{18.286}{-1.55}{0.2}{0}\ertext{13.109}{-1.9}{l}{entity\textunderscore type\textunderscore like}
\eret{13.211}{-3.15}{15.811}{-2.15}{0.2}{1}\ertext{14.511}{-2.5}{}{entity\textunderscore type}
\eret{16.311}{-3.05}{18.036}{-2.15}{0.2}{1}\ertext{16.483}{-2.5}{l}{group}
\erattr{16.511}{-2.7}{0}{1}{annotation}
\eret{33.213}{-7.95}{46.066}{-6.75}{0.2}{1}\ertext{34.499}{-7.1}{l}{attribute}
\erattr{33.413}{-7.3}{1}{0}{name}
\erattr{33.413}{-7.6}{0}{1}{description}
\eret{8.808}{-6.65}{10.461}{-5.75}{0.2}{1}\ertext{9.635}{-6.1}{}{dependency}\ertext{9.635}{-6.4}{}{group}
\eret{11.461}{-12.05}{27.213}{-5.75}{0.2}{1}\ertext{11.965}{-6.1}{l}{Relationship}
\erattr{11.661}{-6.3}{0}{0}{name}
\erattr{11.661}{-6.6}{0}{1}{description}
\erattr{11.661}{-6.9}{0}{1}{id}
\erattr{11.661}{-7.2}{0}{1}{scope}
\erattr{11.661}{-7.5}{0}{1}{physicalprefix}
\eret{15.461}{-9}{26.463}{-6.1}{0.2}{0}\ertext{15.693}{-6.45}{l}{reference\textunderscore or\textunderscore dependency}
\eret{16.461}{-7.55}{18.113}{-6.95}{0.2}{1}\ertext{17.287}{-7.3}{}{dependency}
\eret{18.613}{-8.45}{24.613}{-6.45}{0.2}{1}\ertext{19.213}{-6.8}{l}{reference}
\erattr{18.813}{-7}{0}{1}{js}
\erattr{18.813}{-7.3}{0}{1}{xpathevaluate}
\eret{18.613}{-10.45}{20.613}{-9.55}{0.2}{0}\ertext{19.613}{-9.9}{}{constructed}\ertext{19.613}{-10.2}{}{relationship}
\eret{15.887}{-10.65}{17.687}{-10.05}{0.2}{0}\ertext{16.787}{-10.4}{}{composition}
\eret{1.385}{-11.95}{7.885}{-9.25}{0.2}{1}\ertext{2.035}{-9.6}{l}{presentation}
\erattr{1.585}{-9.8}{0}{1}{x}
\erattr{1.585}{-10.1}{0}{1}{y}
\erattr{1.585}{-10.4}{0}{1}{h}
\erattr{1.585}{-10.7}{0}{1}{w}
\erattr{1.585}{-11}{0}{1}{deltah}
\erattr{1.585}{-11.3}{0}{1}{deltaw}
\erattr{1.585}{-11.6}{0}{1}{sign}
\eret{9.761}{-15.35}{13.061}{-12.95}{0.2}{1}\ertext{10.091}{-13.3}{l}{path}
\erattr{9.961}{-13.5}{0}{1}{srcsign}
\erattr{9.961}{-13.8}{0}{1}{srcarmlen}
\erattr{9.961}{-14.1}{0}{1}{srcattach}
\erattr{9.961}{-14.4}{0}{1}{destsign}
\erattr{9.961}{-14.7}{0}{1}{destarmlen}
\erattr{9.961}{-15}{0}{1}{destattach}
\eret{13.561}{-13.55}{14.933}{-12.95}{0.2}{1}\ertext{14.247}{-13.3}{}{sequence}
\eret{23.205}{-13.55}{24.757}{-12.95}{0.2}{1}\ertext{23.981}{-13.3}{}{projection}
\eret{24.007}{-14.55}{26.121}{-13.95}{0.2}{1}\ertext{25.064}{-14.3}{}{identifying(1)}
\eret{24.308}{-15.75}{25.82}{-15.15}{0.2}{1}\ertext{25.064}{-15.5}{}{inherited}
\eret{25.121}{-21.1}{27.564}{-16.95}{0.2}{1}\ertext{25.316}{-17.3}{l}{cardinality}
\eret{25.521}{-18.15}{27.033}{-17.55}{0.2}{0}\ertext{26.277}{-17.9}{}{ZeroOrOne}
\eret{25.451}{-19.05}{27.103}{-18.45}{0.2}{0}\ertext{26.277}{-18.8}{}{ExactlyOne}
\eret{25.24}{-19.95}{27.314}{-19.35}{0.2}{0}\ertext{26.277}{-19.7}{}{ZeroOneOrMore}
\eret{25.521}{-20.85}{27.033}{-20.25}{0.2}{0}\ertext{26.277}{-20.6}{}{OneOrMore}
\eret{15.095}{-13.55}{16.607}{-12.95}{0.2}{1}\ertext{15.851}{-13.3}{}{transient}
\eret{16.607}{-16.3}{18.479}{-13.95}{0.2}{1}\ertext{16.757}{-14.3}{l}{initialiser}
\eret{16.857}{-15.15}{18.229}{-14.55}{0.2}{0}\ertext{17.543}{-14.9}{}{pullback}
\eret{16.877}{-16.05}{18.21}{-15.45}{0.2}{0}\ertext{17.543}{-15.8}{}{copy}
\eret{18.479}{-23.85}{24.212}{-17.95}{0.2}{1}\ertext{18.938}{-18.3}{l}{navigation}
\erattr{18.679}{-18.5}{0}{1}{xpathevaluate}
\eret{19.979}{-21.2}{22.279}{-18.85}{0.2}{0}\ertext{20.163}{-19.2}{l}{complex}
\eret{20.229}{-20.05}{22.029}{-19.45}{0.2}{1}\ertext{21.129}{-19.8}{}{join}
\eret{20.229}{-20.95}{22.029}{-20.35}{0.2}{1}\ertext{21.129}{-20.7}{}{aggregate}
\eret{20.379}{-22.7}{21.879}{-22.1}{0.2}{0}\ertext{21.129}{-22.45}{}{component}
\eret{19.297}{-23.6}{20.669}{-23}{0.2}{0}\ertext{19.983}{-23.35}{}{identity}
\eret{21.169}{-23.6}{22.962}{-23}{0.2}{0}\ertext{22.065}{-23.35}{}{theabsolute}
\eret{1.303}{-17.6}{4.653}{-12.85}{0.2}{1}\ertext{1.571}{-13.2}{l}{position}
\erattr{1.503}{-13.4}{0}{1}{d}
\eret{1.703}{-14.65}{4.403}{-13.75}{0.2}{0}\ertext{1.973}{-14.1}{l}{relative}
\erattr{1.903}{-14.3}{0}{1}{ratio}
\eret{2.386}{-15.55}{3.72}{-14.95}{0.2}{0}\ertext{3.053}{-15.3}{}{abs}
\eret{2.386}{-16.45}{3.72}{-15.85}{0.2}{0}\ertext{3.053}{-16.2}{}{local}
\eret{2.386}{-17.35}{3.72}{-16.75}{0.2}{0}\ertext{3.053}{-17.1}{}{default}
\eret{5.153}{-20.6}{7.366}{-12.85}{0.2}{1}\ertext{5.33}{-13.2}{l}{CustomShape}
\eret{5.553}{-14.05}{6.886}{-13.45}{0.2}{0}\ertext{6.22}{-13.8}{}{Top}
\eret{5.553}{-14.95}{6.886}{-14.35}{0.2}{0}\ertext{6.22}{-14.7}{}{TopLeft}
\eret{5.534}{-15.85}{6.906}{-15.25}{0.2}{0}\ertext{6.22}{-15.6}{}{TopRight}
\eret{5.323}{-16.75}{7.116}{-16.15}{0.2}{0}\ertext{6.22}{-16.5}{}{MiddleRight}
\eret{5.393}{-17.65}{7.046}{-17.05}{0.2}{0}\ertext{6.22}{-17.4}{}{MiddleLeft}
\eret{5.393}{-18.55}{7.046}{-17.95}{0.2}{0}\ertext{6.22}{-18.3}{}{BottomLeft}
\eret{5.323}{-19.45}{7.116}{-18.85}{0.2}{0}\ertext{6.22}{-19.2}{}{BottomRight}
\eret{5.553}{-20.35}{6.886}{-19.75}{0.2}{0}\ertext{6.22}{-20.1}{}{Bottom}
\eret{33.122}{-13.85}{35.616}{-8.85}{0.2}{1}\ertext{34.369}{-9.2}{}{implementationOf}
\eret{36.116}{-17.5}{39.42}{-8.85}{0.2}{1}\ertext{36.38}{-9.2}{l}{value\textunderscore type}
\eret{37.116}{-10.05}{38.449}{-9.45}{0.2}{0}\ertext{37.782}{-9.8}{}{boolean}
\eret{37.116}{-10.95}{38.449}{-10.35}{0.2}{0}\ertext{37.782}{-10.7}{}{date}
\eret{37.096}{-11.85}{38.468}{-11.25}{0.2}{0}\ertext{37.782}{-11.6}{}{dateTime}
\eret{37.116}{-12.75}{38.449}{-12.15}{0.2}{0}\ertext{37.782}{-12.5}{}{integer}
\eret{37.116}{-13.65}{38.449}{-13.05}{0.2}{0}\ertext{37.782}{-13.4}{}{float}
\eret{36.395}{-14.55}{39.17}{-13.95}{0.2}{0}\ertext{37.782}{-14.3}{}{nonNegativeInteger}
\eret{36.605}{-15.45}{38.959}{-14.85}{0.2}{0}\ertext{37.782}{-15.2}{}{positiveInteger}
\eret{37.116}{-16.35}{38.449}{-15.75}{0.2}{0}\ertext{37.782}{-16.1}{}{string}
\eret{37.116}{-17.25}{38.449}{-16.65}{0.2}{0}\ertext{37.782}{-17}{}{time}
\eret{39.92}{-9.45}{42.133}{-8.85}{0.2}{1}\ertext{41.026}{-9.2}{}{identifying(2)}
\eret{42.633}{-9.75}{44.005}{-8.85}{0.2}{1}\ertext{42.77}{-9.2}{l}{optional}
\erattr{42.833}{-9.4}{0}{1}{value}
\eret{44.505}{-9.45}{46.158}{-8.85}{0.2}{1}\ertext{45.331}{-9.2}{}{deprecated}
\eret{8.578}{-18.15}{11.578}{-16.95}{0.2}{1}\ertext{8.878}{-17.3}{l}{label}
\erattr{8.778}{-17.5}{0}{1}{xAdjustment}
\erattr{8.778}{-17.8}{0}{1}{yAdjustment}
\eret{12.078}{-19.3}{13.911}{-16.95}{0.2}{1}\ertext{12.224}{-17.3}{l}{src\textunderscore or\textunderscore dest}
\eret{12.328}{-18.15}{13.661}{-17.55}{0.2}{0}\ertext{12.994}{-17.9}{}{ToSrc}
\eret{12.328}{-19.05}{13.661}{-18.45}{0.2}{0}\ertext{12.994}{-18.8}{}{ToDest}
\eret{14.411}{-19.3}{16.244}{-16.95}{0.2}{1}\ertext{14.557}{-17.3}{l}{step}
\eret{14.661}{-18.15}{15.994}{-17.55}{0.2}{0}\ertext{15.327}{-17.9}{}{vstep}
\eret{14.661}{-19.05}{15.994}{-18.45}{0.2}{0}\ertext{15.327}{-18.8}{}{hstep}
\eret{13.494}{-22.25}{17.161}{-20.2}{0.2}{1}\ertext{13.658}{-20.55}{l}{dimension}
\erattr{13.694}{-20.75}{0}{1}{src}
\erattr{13.694}{-21.05}{0}{1}{dest}
\eret{13.744}{-22}{15.077}{-21.4}{0.2}{0}\ertext{14.411}{-21.75}{}{reldim}
\eret{15.577}{-22}{16.911}{-21.4}{0.2}{0}\ertext{16.244}{-21.75}{}{absdim}
\eret{7.594}{-23.65}{9.427}{-20.4}{0.2}{1}\ertext{7.74}{-20.75}{l}{render}
\eret{7.844}{-21.6}{9.177}{-21}{0.2}{0}\ertext{8.51}{-21.35}{}{None}
\eret{7.844}{-22.5}{9.177}{-21.9}{0.2}{0}\ertext{8.51}{-22.25}{}{Split}
\eret{7.844}{-23.4}{9.177}{-22.8}{0.2}{0}\ertext{8.51}{-23.15}{}{NoSplit}
\eret{9.927}{-24.55}{12.561}{-20.4}{0.2}{1}\ertext{10.138}{-20.75}{l}{relative\textunderscore position}
\eret{10.177}{-21.6}{11.51}{-21}{0.2}{0}\ertext{10.844}{-21.35}{}{Right}
\eret{10.177}{-22.5}{11.51}{-21.9}{0.2}{0}\ertext{10.844}{-22.25}{}{Left}
\eret{10.177}{-23.4}{11.51}{-22.8}{0.2}{0}\ertext{10.844}{-23.15}{}{Upside}
\eret{10.158}{-24.3}{11.53}{-23.7}{0.2}{0}\ertext{10.844}{-24.05}{}{Downside}
\eret{46.658}{-12.1}{49.63}{-8.85}{0.2}{1}\ertext{46.895}{-9.2}{l}{xmlStyle(2)}
\eret{47.658}{-10.05}{49.17}{-9.45}{0.2}{0}\ertext{48.414}{-9.8}{}{Attribute}
\eret{47.587}{-10.95}{49.24}{-10.35}{0.2}{0}\ertext{48.414}{-10.7}{}{Element(2)}
\eret{47.447}{-11.85}{49.38}{-11.25}{0.2}{0}\ertext{48.414}{-11.6}{}{Anonymous(2)}
\eret{9.013}{-9.9}{11.256}{-7.55}{0.2}{1}\ertext{9.193}{-7.9}{l}{xmlStyle(1)}
\eret{9.213}{-8.75}{10.866}{-8.15}{0.2}{0}\ertext{10.039}{-8.5}{}{Element(1)}
\eret{9.073}{-9.65}{11.006}{-9.05}{0.2}{0}\ertext{10.039}{-9.4}{}{Anonymous(1)}
\eret{0}{-0.2}{49.63}{0.3}{0.2}{1}

% relationship 
\ertext{16.625}{-0.5}{l}{}\errelarm{16.475}{-0.2}{16.475}{-0.875}{0}{0}\errelarm{16.475}{-0.875}{16.475}{-1.55}{1}{0}\ercrowfoot{16.475}{-1.4}{16.325}{-1.55}{16.475}{-1.55}{16.625}{-1.55}{0}
% relationship 
\ertext{19.622}{-0.5}{l}{}\errelarm{19.472}{-0.2}{19.472}{-0.875}{1}{0}\errelarm{19.472}{-0.875}{19.472}{-1.55}{1}{0}
% relationship presentation
\ertext{2.317}{-0.5}{l}{presentation}\errelarm{2.167}{-0.2}{2.167}{-0.6}{0}{0}\errelarm{2.167}{-0.6}{2.167}{-1}{1}{0}
% relationship 
\ertext{5.047}{-0.5}{l}{}\errelarm{4.897}{-0.2}{4.897}{-0.6}{0}{0}\errelarm{4.897}{-0.6}{4.897}{-1}{1}{0}
% relationship 
\ertext{48.294}{-0.5}{l}{}\errelarm{48.144}{-0.2}{48.144}{-0.6}{0}{0}\errelarm{48.144}{-0.6}{48.144}{-1}{1}{0}
% relationship attributeDefault
\ertext{48.294}{-2.2}{l}{attributeDefault}\errelarm{48.144}{-1.9}{48.144}{-5.375}{0}{0}\errelarm{48.144}{-5.375}{48.144}{-8.85}{1}{0}
% relationship 
\ertext{11.009}{-4.05}{l}{}\errelarm{10.859}{-3.75}{10.859}{-3.825}{0}{0}\errelarm{4.635}{-8.875}{4.635}{-9.25}{1}{0}\errelangle{10.859}{-3.825}{10.859}{-3.9}{7.747}{-6.2}{0}{0}\errelangle{7.747}{-6.2}{4.635}{-8.5}{4.635}{-8.875}{1}{0}
% relationship 
\ertext{11.408}{-4.05}{l}{}\errelarm{11.258}{-3.75}{11.258}{-3.825}{0}{0}\errelarm{9.635}{-5.625}{9.635}{-5.75}{1}{0}\errelangle{11.258}{-3.825}{11.258}{-3.9}{10.446}{-4.7}{0}{0}\errelangle{10.446}{-4.7}{9.635}{-5.5}{9.635}{-5.625}{1}{0}\ercrowfoot{9.635}{-5.6}{9.485}{-5.75}{9.635}{-5.75}{9.785}{-5.75}{0}
% relationship 
\ertext{20.179}{-4.05}{l}{}\ertext{20.179}{-5.6}{l}{..}\errelarm{20.029}{-3.75}{20.029}{-4.75}{0}{0}\errelarm{20.029}{-4.75}{20.029}{-5.75}{1}{0}\eridcomprel{19.928999999999995}{20.128999999999998}{-5.5}\ercrowfoot{20.029}{-5.6}{19.879}{-5.75}{20.029}{-5.75}{20.179}{-5.75}{0}
% relationship 
\ertext{23.368}{-4.05}{l}{}\ertext{39.79}{-6.6}{l}{..}\errelarm{23.218}{-3.75}{23.218}{-3.825}{0}{0}\errelarm{39.64}{-6.538}{39.64}{-6.75}{1}{0}\errelangle{23.218}{-3.825}{23.218}{-3.9}{31.429}{-5.113}{0}{0}\errelangle{31.429}{-5.113}{39.64}{-6.325}{39.64}{-6.538}{1}{0}\eridcomprel{39.53955}{39.73955}{-6.5}\ercrowfoot{39.64}{-6.6}{39.49}{-6.75}{39.64}{-6.75}{39.79}{-6.75}{0}
% relationship 
\ertext{14.868}{-3.7}{l}{}\errelarm{14.718}{-3.4}{14.718}{-3.5}{0}{0}\errelarm{14.718}{-1.4}{14.718}{-1.55}{1}{0}\errelangle{14.718}{-3.5}{14.718}{-3.6}{13.718}{-3.6}{0}{0}\errelangle{14.718}{-1.4}{14.718}{-1.25}{13.718}{-1.25}{1}{0}\errelangle{13.718}{-3.6}{12.718}{-3.6}{12.718}{-2.425}{0}{0}\errelangle{12.718}{-2.425}{12.718}{-1.25}{13.718}{-1.25}{1}{0}\ercrowfoot{14.718}{-1.4}{14.568}{-1.55}{14.718}{-1.55}{14.868}{-1.55}{0}\erarc{14.123}{-1.45}{14.873}{-1.25}{16.373}{-1.25}{17.123}{-1.45}
% relationship xmlRepresentation
\ertext{15.311}{-3.6}{l}{xmlRepresentation}\errelarm{15.161}{-3.15}{15.161}{-3.95}{0}{0}\errelarm{10.807}{-7.5}{10.807}{-7.55}{1}{0}\errelangle{15.161}{-3.95}{15.161}{-4.75}{12.984}{-4.75}{0}{0}\errelangle{10.807}{-7.5}{10.807}{-7.45}{10.807}{-7.45}{1}{0}\errelangle{12.984}{-4.75}{10.807}{-4.75}{10.807}{-6.1}{0}{0}\errelangle{10.807}{-6.1}{10.807}{-7.45}{10.807}{-7.45}{1}{0}
% relationship 
\ertext{34.519}{-8.25}{l}{}\ertext{34.519}{-8.7}{l}{..}\errelarm{34.369}{-7.95}{34.369}{-8.4}{0}{0}\errelarm{34.369}{-8.4}{34.369}{-8.85}{1}{0}
% relationship type
\ertext{37.918}{-8.25}{l}{type}\errelarm{37.768}{-7.95}{37.768}{-8.4}{1}{0}\errelarm{37.768}{-8.4}{37.768}{-8.85}{1}{0}
% relationship 
\ertext{41.176}{-8.25}{l}{}\errelarm{41.026}{-7.95}{41.026}{-8.4}{0}{0}\errelarm{41.026}{-8.4}{41.026}{-8.85}{1}{0}
% relationship 
\ertext{43.469}{-8.25}{l}{}\errelarm{43.319}{-7.95}{43.319}{-8.4}{0}{0}\errelarm{43.319}{-8.4}{43.319}{-8.85}{1}{0}
% relationship 
\ertext{45.481}{-8.25}{l}{}\errelarm{45.331}{-7.95}{45.331}{-8.4}{0}{0}\errelarm{45.331}{-8.4}{45.331}{-8.85}{1}{0}
% relationship xmlStyle
\ertext{45.894}{-8.25}{l}{xmlStyle}\errelarm{45.744}{-7.95}{45.744}{-8.15}{0}{0}\errelarm{47.549}{-8.7}{47.549}{-8.85}{1}{0}\errelangle{45.744}{-8.15}{45.744}{-8.35}{46.647}{-8.45}{0}{0}\errelangle{46.647}{-8.45}{47.549}{-8.55}{47.549}{-8.7}{1}{0}
% relationship 
\ertext{9.785}{-6.95}{l}{}\errelarm{9.635}{-6.65}{9.635}{-6.725}{0}{0}\errelarm{6.26}{-8.925}{6.26}{-9.25}{1}{0}\errelangle{9.635}{-6.725}{9.635}{-6.8}{7.947}{-7.7}{0}{0}\errelangle{7.947}{-7.7}{6.26}{-8.6}{6.26}{-8.925}{1}{0}
% relationship diagram
\ertext{11.705}{-12.35}{r}{diagram}\errelarm{11.855}{-12.05}{11.855}{-12.5}{0}{0}\errelarm{11.855}{-12.5}{11.855}{-12.95}{1}{0}
% relationship 
\ertext{14.397}{-12.35}{l}{}\errelarm{14.247}{-12.05}{14.247}{-12.5}{0}{0}\errelarm{14.247}{-12.5}{14.247}{-12.95}{1}{0}
% relationship 
\ertext{25.214}{-12.35}{l}{}\errelarm{25.064}{-12.05}{25.064}{-13}{0}{0}\errelarm{25.064}{-13}{25.064}{-13.95}{1}{0}
% relationship cardinality
\ertext{26.492}{-12.35}{l}{cardinality}\errelarm{26.342}{-12.05}{26.342}{-14.5}{1}{0}\errelarm{26.342}{-14.5}{26.342}{-16.95}{1}{0}
% relationship type
\ertext{27.363}{-7.94}{l}{type}\errelarm{27.213}{-7.64}{27.313}{-7.64}{1}{0}\errelarm{26.608}{-3.2}{26.408}{-3.2}{0}{0}\errelangle{27.313}{-7.64}{27.413}{-7.64}{27.563}{-7.64}{1}{0}\errelangle{26.608}{-3.2}{26.808}{-3.2}{27.261}{-3.2}{0}{0}\errelangle{27.563}{-7.64}{27.713}{-7.64}{27.713}{-5.42}{1}{0}\errelangle{27.713}{-5.42}{27.713}{-3.2}{27.261}{-3.2}{0}{0}\ercrowfoot{27.363}{-7.64}{27.213}{-7.49}{27.213}{-7.64}{27.213}{-7.79}{0}
% relationship diagonal
\ertext{21.463}{-8.75}{r}{diagonal}\errelarm{21.613}{-8.45}{21.613}{-13.2}{0}{0}\errelarm{21.613}{-13.2}{21.613}{-17.95}{1}{0}
% relationship riser
\ertext{22.063}{-8.75}{l}{riser}\errelarm{21.913}{-8.45}{21.913}{-13.2}{0}{0}\errelarm{21.913}{-13.2}{21.913}{-17.95}{1}{0}
% relationship key
\ertext{23.263}{-8.75}{l}{key}\errelarm{23.113}{-8.45}{23.113}{-13.2}{0}{0}\errelarm{23.113}{-13.2}{23.113}{-17.95}{1}{0}
% relationship 
\ertext{24.131}{-8.75}{l}{}\errelarm{23.981}{-8.45}{23.981}{-10.7}{0}{0}\errelarm{23.981}{-10.7}{23.981}{-12.95}{1}{0}
% relationship inverse
\ertext{24.763}{-7.15}{l}{inverse}\ertext{24.763}{-6.7}{l}{inverse}\errelarm{24.613}{-6.85}{25.113}{-6.85}{0}{0}\ertext{25.263}{-6.85}{l}{\textasciitilde /..=type}\errelarm{25.113}{-6.85}{25.113}{-6.85}{0}{0}\errelarm{25.113}{-6.85}{24.613}{-6.85}{0}{0}
% relationship 
\ertext{20.349}{-10.75}{l}{}\errelarm{20.199}{-10.45}{20.199}{-14.2}{1}{0}\errelarm{20.199}{-14.2}{20.199}{-17.95}{1}{0}
% relationship inverse
\ertext{20.763}{-10.3}{l}{inverse}\ertext{20.763}{-9.85}{l}{inverse}\errelarm{20.613}{-10}{21.113}{-10}{0}{0}\errelarm{21.113}{-10}{21.113}{-10}{0}{0}\errelarm{21.113}{-10}{20.613}{-10}{0}{0}
% relationship 
\ertext{16.637}{-10.95}{l}{}\errelarm{16.487}{-10.65}{16.487}{-10.725}{0}{0}\errelarm{15.851}{-12.9}{15.851}{-12.95}{1}{0}\errelangle{16.487}{-10.725}{16.487}{-10.8}{16.169}{-11.825}{0}{0}\errelangle{16.169}{-11.825}{15.851}{-12.85}{15.851}{-12.9}{1}{0}
% relationship 
\ertext{17.237}{-10.95}{l}{}\ertext{17.693}{-13.8}{l}{..}\errelarm{17.087}{-10.65}{17.087}{-10.725}{0}{0}\errelarm{17.543}{-13.4}{17.543}{-13.95}{1}{0}\errelangle{17.087}{-10.725}{17.087}{-10.8}{17.315}{-11.825}{0}{0}\errelangle{17.315}{-11.825}{17.543}{-12.85}{17.543}{-13.4}{1}{0}
% relationship inverse
\ertext{15.737}{-10.65}{r}{inverse}\ertext{16.311}{-7.1}{r}{of}\ertext{16.311}{-6.8}{r}{inverse}\errelarm{15.887}{-10.35}{15.287}{-10.35}{0}{0}\errelarm{16.261}{-7.25}{16.461}{-7.25}{0}{0}\errelangle{15.287}{-10.35}{14.687}{-10.35}{14.587}{-10.35}{0}{0}\errelangle{16.261}{-7.25}{16.061}{-7.25}{15.274}{-7.25}{0}{0}\errelangle{14.587}{-10.35}{14.487}{-10.35}{14.487}{-8.8}{0}{0}\errelangle{14.487}{-8.8}{14.487}{-7.25}{15.274}{-7.25}{0}{0}
% relationship xl
\ertext{2.51}{-12.25}{l}{xl}\errelarm{2.36}{-11.95}{2.36}{-12.4}{0}{0}\errelarm{2.36}{-12.4}{2.36}{-12.85}{1}{0}
% relationship xc
\ertext{2.9}{-12.25}{l}{xc}\errelarm{2.75}{-11.95}{2.75}{-12.85}{0}{0}\errelarm{2.75}{-12.85}{2.75}{-13.75}{1}{0}
% relationship xr
\ertext{3.29}{-12.25}{l}{xr}\errelarm{3.14}{-11.95}{3.14}{-12.85}{0}{0}\errelarm{3.14}{-12.85}{3.14}{-13.75}{1}{0}
% relationship yt
\ertext{3.68}{-12.25}{l}{yt}\errelarm{3.53}{-11.95}{3.53}{-12.4}{0}{0}\errelarm{3.53}{-12.4}{3.53}{-12.85}{1}{0}
% relationship ym
\ertext{4.07}{-12.25}{l}{ym}\errelarm{3.92}{-11.95}{3.92}{-12.85}{0}{0}\errelarm{3.92}{-12.85}{3.92}{-13.75}{1}{0}
% relationship yb
\ertext{4.525}{-12.25}{l}{yb}\errelarm{4.375}{-11.95}{4.375}{-12.15}{0}{0}\errelarm{4.349}{-13.7}{4.349}{-13.75}{1}{0}\errelangle{4.375}{-12.15}{4.375}{-12.35}{4.362}{-13}{0}{0}\errelangle{4.362}{-13}{4.349}{-13.65}{4.349}{-13.7}{1}{0}
% relationship shape
\ertext{6.41}{-12.25}{l}{shape}\errelarm{6.26}{-11.95}{6.26}{-12.4}{0}{0}\errelarm{6.26}{-12.4}{6.26}{-12.85}{1}{0}
% relationship name
\ertext{7.872}{-12.25}{l}{name}\errelarm{7.722}{-11.95}{7.722}{-12.15}{0}{0}\errelarm{8.235}{-20.35}{8.235}{-20.4}{1}{0}\errelangle{7.722}{-12.15}{7.722}{-12.35}{7.979}{-16.325}{0}{0}\errelangle{7.979}{-16.325}{8.235}{-20.3}{8.235}{-20.35}{1}{0}
% relationship rightOf
\ertext{1.235}{-10.45}{r}{rightOf}\errelarm{1.385}{-10.15}{1.135}{-10.15}{0}{0}\errelarm{10.261}{-3.475}{10.461}{-3.475}{0}{0}\errelangle{1.135}{-10.15}{0.885}{-10.15}{0.785}{-10.15}{0}{0}\errelangle{10.261}{-3.475}{10.061}{-3.475}{5.373}{-3.475}{0}{0}\ertext{0.535}{-7.113}{r}{\textasciitilde /\textasciicircum =\textasciicircum }\errelangle{0.785}{-10.15}{0.685}{-10.15}{0.685}{-6.813}{0}{0}\errelangle{0.685}{-6.813}{0.685}{-3.475}{5.373}{-3.475}{0}{0}\ercrowfoot{1.235}{-10.15}{1.385}{-10}{1.385}{-10.15}{1.385}{-10.3}{0}
% relationship below
\ertext{1.235}{-11.35}{r}{below}\errelarm{1.385}{-11.05}{0.935}{-11.05}{0}{0}\errelarm{10.261}{-3.2}{10.461}{-3.2}{0}{0}\errelangle{0.935}{-11.05}{0.485}{-11.05}{0.435}{-11.05}{0}{0}\errelangle{10.261}{-3.2}{10.061}{-3.2}{5.223}{-3.2}{0}{0}\ertext{0.235}{-7.425}{r}{\textasciitilde /\textasciicircum =\textasciicircum }\errelangle{0.435}{-11.05}{0.385}{-11.05}{0.385}{-7.125}{0}{0}\errelangle{0.385}{-7.125}{0.385}{-3.2}{5.223}{-3.2}{0}{0}\ercrowfoot{1.235}{-11.05}{1.385}{-10.9}{1.385}{-11.05}{1.385}{-11.2}{0}\erarc{4.235}{-9.05}{4.86}{-8.85}{6.11}{-8.85}{6.735}{-9.05}
% relationship 
\ertext{10.241}{-15.65}{l}{}\errelarm{10.091}{-15.35}{10.091}{-16.15}{0}{0}\errelarm{10.091}{-16.15}{10.091}{-16.95}{1}{0}
% relationship inverse
\ertext{11.101}{-15.65}{l}{inverse}\errelarm{10.751}{-15.35}{10.751}{-16.15}{0}{0}\errelarm{10.751}{-16.15}{10.751}{-16.95}{1}{0}
% relationship id
\ertext{11.1}{-15.95}{l}{id}\errelarm{10.85}{-15.35}{10.85}{-16.15}{0}{0}\errelarm{10.85}{-16.15}{10.85}{-16.95}{1}{0}
% relationship scope
\ertext{11.099}{-16.25}{l}{scope}\errelarm{10.949}{-15.35}{10.949}{-16.15}{0}{0}\errelarm{10.949}{-16.15}{10.949}{-16.95}{1}{0}
% relationship align
\ertext{12.056}{-15.65}{l}{align}\errelarm{11.906}{-15.35}{11.906}{-15.55}{0}{0}\errelarm{12.994}{-16.9}{12.994}{-16.95}{1}{0}\errelangle{11.906}{-15.55}{11.906}{-15.75}{12.45}{-16.3}{0}{0}\errelangle{12.45}{-16.3}{12.994}{-16.85}{12.994}{-16.9}{1}{0}
% relationship 
\ertext{12.881}{-15.65}{l}{}\errelarm{12.731}{-15.35}{12.731}{-15.425}{0}{0}\errelarm{15.327}{-16.75}{15.327}{-16.95}{1}{0}\errelangle{12.731}{-15.425}{12.731}{-15.5}{14.029}{-16.025}{0}{0}\errelangle{14.029}{-16.025}{15.327}{-16.55}{15.327}{-16.75}{1}{0}
% relationship 
\ertext{25.214}{-14.85}{l}{}\errelarm{25.064}{-14.55}{25.064}{-14.85}{0}{0}\errelarm{25.064}{-14.85}{25.064}{-15.15}{1}{0}
% relationship along
\ertext{17.019}{-16.6}{r}{along}\errelarm{17.169}{-16.3}{17.169}{-16.6}{1}{0}\errelarm{19.052}{-17.9}{19.052}{-17.95}{1}{0}\errelangle{17.169}{-16.6}{17.169}{-16.9}{18.111}{-17.375}{1}{0}\errelangle{18.111}{-17.375}{19.052}{-17.85}{19.052}{-17.9}{1}{0}
% relationship type
\ertext{18.629}{-15.798}{l}{type}\errelarm{18.479}{-15.948}{19.079}{-15.948}{1}{0}\errelarm{26.608}{-3.063}{26.408}{-3.063}{0}{0}\errelangle{19.079}{-15.948}{19.679}{-15.948}{23.779}{-15.948}{1}{0}\errelangle{26.608}{-3.063}{26.808}{-3.063}{27.343}{-3.063}{0}{0}\errelangle{23.779}{-15.948}{27.879}{-15.948}{27.879}{-9.505}{1}{0}\errelangle{27.879}{-9.505}{27.879}{-3.063}{27.343}{-3.063}{0}{0}\ercrowfoot{18.629}{-15.948}{18.479}{-15.798}{18.479}{-15.948}{18.479}{-16.098}{0}
% relationship projectionrel
\ertext{18.679}{-14.97}{l}{projection}\ertext{18.679}{-15.27}{l}{rel}\errelarm{18.229}{-14.67}{18.529}{-14.67}{0}{0}\errelarm{18.413}{-7.95}{18.613}{-7.95}{0}{0}\ertext{18.671}{-11.61}{l}{\textasciitilde /..=../type}\errelangle{18.529}{-14.67}{18.829}{-14.67}{18.521}{-11.31}{0}{0}\errelangle{18.521}{-11.31}{18.213}{-7.95}{18.413}{-7.95}{0}{0}\ercrowfoot{18.379}{-14.67}{18.229}{-14.52}{18.229}{-14.67}{18.229}{-14.82}{0}
% relationship riser2
\ertext{17.96}{-16.65}{l}{riser2}\errelarm{17.81}{-16.05}{17.81}{-16.3}{1}{0}\errelarm{19.626}{-17.9}{19.626}{-17.95}{1}{0}\errelangle{17.81}{-16.3}{17.81}{-16.55}{18.718}{-17.2}{1}{0}\errelangle{18.718}{-17.2}{19.626}{-17.85}{19.626}{-17.9}{1}{0}
% relationship src
\ertext{24.362}{-22.97}{l}{src}\errelarm{24.212}{-22.67}{24.662}{-22.67}{0}{0}\errelarm{26.608}{-2.238}{26.408}{-2.238}{0}{0}\errelangle{24.662}{-22.67}{25.112}{-22.67}{27.512}{-22.67}{0}{0}\errelangle{26.608}{-2.238}{26.808}{-2.238}{28.36}{-2.238}{0}{0}\errelangle{27.512}{-22.67}{29.912}{-22.67}{29.912}{-12.454}{0}{0}\errelangle{29.912}{-12.454}{29.912}{-2.238}{28.36}{-2.238}{0}{0}\ercrowfoot{24.362}{-22.67}{24.212}{-22.52}{24.212}{-22.67}{24.212}{-22.82}{0}
% relationship dest
\ertext{24.362}{-23.56}{l}{dest}\errelarm{24.212}{-23.26}{24.762}{-23.26}{0}{0}\errelarm{26.608}{-2.1}{26.408}{-2.1}{0}{0}\errelangle{24.762}{-23.26}{25.312}{-23.26}{27.712}{-23.26}{0}{0}\errelangle{26.608}{-2.1}{26.808}{-2.1}{28.46}{-2.1}{0}{0}\errelangle{27.712}{-23.26}{30.112}{-23.26}{30.112}{-12.68}{0}{0}\errelangle{30.112}{-12.68}{30.112}{-2.1}{28.46}{-2.1}{0}{0}\ercrowfoot{24.362}{-23.26}{24.212}{-23.11}{24.212}{-23.26}{24.212}{-23.41}{0}\erarc{18.479}{-17.75}{19.729}{-17.55}{22.229}{-17.55}{23.479}{-17.75}
% relationship 
\ertext{21.279}{-21.5}{l}{}\errelarm{21.129}{-21.2}{21.129}{-21.65}{0}{0}\errelarm{21.129}{-21.65}{21.129}{-22.1}{1}{0}\errelseq{21.189}{-21.65}{20.779}{-21.71}{21.479}{-21.77}{21.069}{-21.83}\ercrowfoot{21.129}{-21.95}{20.979}{-22.1}{21.129}{-22.1}{21.279}{-22.1}{0}
% relationship rel
\ertext{22.029}{-22.7}{l}{rel}\errelarm{21.879}{-22.4}{22.029}{-22.4}{1}{0}\errelarm{27.413}{-10.475}{27.213}{-10.475}{0}{0}\errelangle{22.029}{-22.4}{22.179}{-22.4}{25.429}{-22.4}{1}{0}\errelangle{27.413}{-10.475}{27.613}{-10.475}{28.146}{-10.475}{0}{0}\ertext{28.529}{-16.738}{r}{\textasciitilde /..=src}\errelangle{25.429}{-22.4}{28.679}{-22.4}{28.679}{-16.438}{1}{0}\errelangle{28.679}{-16.438}{28.679}{-10.475}{28.146}{-10.475}{0}{0}\ercrowfoot{22.029}{-22.4}{21.879}{-22.25}{21.879}{-22.4}{21.879}{-22.55}{0}\erarc{1.853}{-12.65}{2.578}{-12.45}{4.028}{-12.45}{4.753}{-12.65}
% relationship to
\ertext{1.553}{-14.5}{r}{to}\errelarm{1.703}{-14.2}{1.103}{-14.2}{1}{0}\errelarm{10.261}{-2.925}{10.461}{-2.925}{0}{0}\errelangle{1.103}{-14.2}{0.503}{-14.2}{0.303}{-14.2}{1}{0}\errelangle{10.261}{-2.925}{10.061}{-2.925}{5.082}{-2.925}{0}{0}\ertext{-0.047}{-8.863}{r}{\textasciitilde /\textasciicircum =\textasciicircum }\errelangle{0.303}{-14.2}{0.103}{-14.2}{0.103}{-8.563}{1}{0}\errelangle{0.103}{-8.563}{0.103}{-2.925}{5.082}{-2.925}{0}{0}\ercrowfoot{1.553}{-14.2}{1.703}{-14.05}{1.703}{-14.2}{1.703}{-14.35}{0}
% relationship destattr
\ertext{32.972}{-9.7}{r}{destattr}\errelarm{33.122}{-9.85}{32.672}{-9.85}{1}{0}\errelarm{33.013}{-7.59}{33.213}{-7.59}{0}{0}\errelangle{32.672}{-9.85}{32.222}{-9.85}{32.197}{-9.85}{1}{0}\errelangle{33.013}{-7.59}{32.813}{-7.59}{32.492}{-7.59}{0}{0}\ertext{33.322}{-9.02}{r}{\textasciitilde /..=rel/type}\errelangle{32.197}{-9.85}{32.172}{-9.85}{32.172}{-8.72}{1}{0}\errelangle{32.172}{-8.72}{32.172}{-7.59}{32.492}{-7.59}{0}{0}\ercrowfoot{32.972}{-9.85}{33.122}{-9.7}{33.122}{-9.85}{33.122}{-10}{0}
% relationship attrOfOrigin
\ertext{32.972}{-10.65}{r}{attrOfOrigin}\errelarm{33.122}{-10.35}{32.622}{-10.35}{1}{0}\errelarm{33.013}{-7.35}{33.213}{-7.35}{0}{0}\errelangle{32.622}{-10.35}{32.122}{-10.35}{32.097}{-10.35}{1}{0}\errelangle{33.013}{-7.35}{32.813}{-7.35}{32.442}{-7.35}{0}{0}\ertext{33.222}{-9.15}{r}{\textasciitilde /..=typeOfOrigin}\errelangle{32.097}{-10.35}{32.072}{-10.35}{32.072}{-8.85}{1}{0}\errelangle{32.072}{-8.85}{32.072}{-7.35}{32.442}{-7.35}{0}{0}\ercrowfoot{32.972}{-10.35}{33.122}{-10.2}{33.122}{-10.35}{33.122}{-10.5}{0}
% relationship typeOfOrigin
\ertext{32.972}{-12.15}{r}{typeOfOrigin}\errelarm{33.122}{-11.85}{32.622}{-11.85}{1}{0}\errelarm{26.608}{-1.55}{26.408}{-1.55}{0}{0}\errelangle{32.622}{-11.85}{32.122}{-11.85}{31.622}{-11.85}{1}{0}\errelangle{26.608}{-1.55}{26.808}{-1.55}{28.965}{-1.55}{0}{0}\ertext{32.272}{-7}{r}{\textasciitilde /\textasciicircum =\textasciicircum }\errelangle{31.622}{-11.85}{31.122}{-11.85}{31.122}{-6.7}{1}{0}\errelangle{31.122}{-6.7}{31.122}{-1.55}{28.965}{-1.55}{0}{0}\ercrowfoot{32.972}{-11.85}{33.122}{-11.7}{33.122}{-11.85}{33.122}{-12}{0}
% relationship rel
\ertext{32.972}{-13.15}{r}{rel}\errelarm{33.122}{-12.85}{32.522}{-12.85}{1}{0}\errelarm{26.663}{-8.42}{26.463}{-8.42}{0}{0}\ertext{29.942}{-11.385}{l}{\textasciitilde /..=../..}\errelangle{32.522}{-12.85}{31.922}{-12.85}{29.392}{-10.635}{1}{0}\errelangle{29.392}{-10.635}{26.863}{-8.42}{26.663}{-8.42}{0}{0}\ercrowfoot{32.972}{-12.85}{33.122}{-12.7}{33.122}{-12.85}{33.122}{-13}{0}
% relationship name
\ertext{9.728}{-18.45}{l}{name}\errelarm{9.578}{-18.15}{9.578}{-18.35}{0}{0}\errelarm{8.785}{-20.35}{8.785}{-20.4}{1}{0}\errelangle{9.578}{-18.35}{9.578}{-18.55}{9.181}{-19.425}{0}{0}\errelangle{9.181}{-19.425}{8.785}{-20.3}{8.785}{-20.35}{1}{0}
% relationship position
\ertext{10.728}{-18.45}{l}{position}\errelarm{10.578}{-18.15}{10.578}{-18.35}{0}{0}\errelarm{11.244}{-20.35}{11.244}{-20.4}{1}{0}\errelangle{10.578}{-18.35}{10.578}{-18.55}{10.911}{-19.425}{0}{0}\errelangle{10.911}{-19.425}{11.244}{-20.3}{11.244}{-20.35}{1}{0}\erarc{9.828}{-16.85}{10.203}{-16.65}{10.953}{-16.65}{11.328}{-16.85}
% relationship 
\ertext{15.172}{-19.6}{l}{}\errelarm{15.022}{-19.3}{15.022}{-19.375}{0}{0}\errelarm{15.052}{-16.75}{15.052}{-16.95}{1}{0}\errelangle{15.022}{-19.375}{15.022}{-19.45}{14.622}{-19.45}{0}{0}\errelangle{15.052}{-16.75}{15.052}{-16.55}{14.637}{-16.55}{1}{0}\errelangle{14.622}{-19.45}{14.222}{-19.45}{14.222}{-18}{0}{0}\errelangle{14.222}{-18}{14.222}{-16.55}{14.637}{-16.55}{1}{0}
% relationship 
\ertext{15.783}{-19.6}{l}{}\errelarm{15.633}{-19.3}{15.633}{-19.75}{0}{0}\errelarm{15.633}{-19.75}{15.633}{-20.2}{1}{0}\erarc{14.786}{-16.85}{14.973}{-16.65}{15.348}{-16.65}{15.536}{-16.85}\erarc{7.777}{-20.2}{8.144}{-20}{8.877}{-20}{9.244}{-20.2}\erarc{47.008}{-8.75}{47.383}{-8.55}{48.133}{-8.55}{48.508}{-8.75}
\end{erdiagram}

}
\end{frame}
\begin{frame}{Example -- LCMSMS Data -- Seven Pullback Diagrams}
\scalebox{0.2}{
\begin{erdiagram}{28.45}{55.81075}

\ergrp{0.2}{-3.7}{12.075}{-0.75}{0.2}{1}\ertext{0.15}{-1.05}{r}{module: task}
\eret{0.4}{-3.45}{4.338}{-1.35}{0.2}{1}\ertext{0.794}{-1.7}{l}{task}
\erattr{0.6}{-1.9}{1}{1}{hostnameOfLabsysAppServer}
\erattr{0.6}{-2.2}{1}{1}{tasknumber}
\erattr{0.6}{-2.5}{1}{1}{SOPnumber}
\erattr{0.6}{-2.8}{1}{1}{procedurename}
\erattr{0.6}{-3.1}{1}{1}{location}
\eret{7.338}{-2.55}{11.075}{-1.35}{0.2}{1}\ertext{7.711}{-1.7}{l}{revised\textunderscore labsys\textunderscore task}
\erattr{7.538}{-1.9}{1}{1}{hostnameOfLabsysAppServer}
\erattr{7.538}{-2.2}{1}{1}{tasknumber}
\ergrp{2.088}{-20.6}{10.188}{-5.7}{0.2}{1}\ertext{2.038}{-6}{r}{Documentation only. Not represented in XML nor in rng or ts.}
\eret{2.288}{-8.3}{9.988}{-6.55}{0.2}{1}\ertext{2.428}{-6.9}{l}{sample\textunderscore group}
\erattr{2.488}{-7.1}{1}{0}{alphacode}
\eret{2.538}{-8.05}{5.838}{-7.45}{0.2}{0}\ertext{4.188}{-7.8}{}{test\textunderscore sample\textunderscore group}
\eret{6.338}{-8.05}{9.638}{-7.45}{0.2}{0}\ertext{7.988}{-7.8}{}{shared\textunderscore sample\textunderscore group}
\eret{7.031}{-10.05}{9.744}{-9.15}{0.2}{1}\ertext{7.302}{-9.5}{l}{MSMS\textunderscore component}
\erattr{7.231}{-9.7}{1}{0}{compoundid}
\eret{4.638}{-14.35}{7.638}{-13.15}{0.2}{1}\ertext{4.938}{-13.5}{l}{sample}
\erattr{4.838}{-13.7}{1}{0}{sampleid}
\erattr{4.838}{-14}{1}{1}{isRTreferencepeak}
\eret{4.388}{-17.65}{7.888}{-16.75}{0.2}{1}\ertext{4.738}{-17.1}{l}{required\textunderscore MSMS\textunderscore data}
\erattr{4.588}{-17.3}{1}{0}{compoundid}
\ergrp{12.575}{-26.3}{24.591}{-2.75}{0.2}{1}\ertext{12.525}{-3.05}{r}{chromatogram tower}
\eret{16.575}{-5.75}{19.775}{-3.35}{0.2}{1}\ertext{16.895}{-3.7}{l}{data\textunderscore collection}
\erattr{16.775}{-3.9}{1}{0}{samplelistname}
\erattr{16.775}{-4.2}{1}{1}{instrumentname}
\erattr{16.775}{-4.5}{1}{1}{instrumenttype}
\erattr{16.775}{-4.8}{1}{1}{chromatography}
\erattr{16.775}{-5.1}{1}{1}{collectiontimestamp}
\erattr{16.775}{-5.4}{0}{1}{labsyssubmissionid}
\eret{16.744}{-7.55}{19.606}{-6.65}{0.2}{1}\ertext{17.03}{-7}{l}{sample\textunderscore group(2)}
\erattr{16.944}{-7.2}{1}{1}{methodfilename}
\eret{13.616}{-15.85}{16.734}{-13.15}{0.2}{1}\ertext{13.928}{-13.5}{l}{injection(2)}
\erattr{13.816}{-13.7}{1}{1}{fullsampleid}
\erattr{13.816}{-14}{1}{1}{samplename}
\erattr{13.816}{-14.3}{1}{1}{sampletype}
\erattr{13.816}{-14.6}{1}{1}{rawdatafilename}
\erattr{13.816}{-14.9}{1}{1}{drawerposition}
\erattr{13.816}{-15.2}{1}{1}{wellposition}
\erattr{13.816}{-15.5}{1}{1}{timestamp}
\eret{20.079}{-10.05}{22.271}{-9.15}{0.2}{1}\ertext{21.175}{-9.5}{}{compound(2)}
\eret{20.289}{-11.85}{22.061}{-10.95}{0.2}{1}\ertext{20.466}{-11.3}{l}{trace(2)}
\erattr{20.489}{-11.5}{1}{0}{trace}
\eret{13.959}{-17.35}{16.392}{-16.75}{0.2}{1}\ertext{15.175}{-17.1}{}{component(2)}
\eret{13.425}{-19.75}{16.925}{-18.25}{0.2}{1}\ertext{13.775}{-18.6}{l}{chromatogram}
\erattr{13.625}{-18.8}{1}{1}{extractiontimestamp}
\erattr{13.625}{-19.1}{1}{1}{extractioneventno}
\erattr{13.625}{-19.4}{1}{1}{timeintensitydata}
\eret{13.059}{-23.3}{22.291}{-20.65}{0.2}{1}\ertext{13.271}{-21}{l}{instrument\textunderscore extraction\textunderscore details}
\eret{13.309}{-22.45}{16.785}{-21.25}{0.2}{0}\ertext{13.657}{-21.6}{l}{xevo\textunderscore extraction\textunderscore details}
\erattr{13.509}{-21.8}{1}{1}{xevofunctionno}
\erattr{13.509}{-22.1}{1}{1}{xevocompoundno}
\eret{17.285}{-23.05}{21.541}{-21.25}{0.2}{0}\ertext{17.71}{-21.6}{l}{ab6600\textunderscore extraction\textunderscore details}
\erattr{17.485}{-21.8}{1}{1}{sampleno}
\erattr{17.485}{-22.1}{1}{1}{periodno}
\erattr{17.485}{-22.4}{1}{1}{experimentno}
\erattr{17.485}{-22.7}{1}{1}{numberofdatapoints}
\eret{17.425}{-25.45}{21.401}{-23.95}{0.2}{1}\ertext{17.823}{-24.3}{l}{step\textunderscore size\textunderscore specification}
\erattr{17.625}{-24.5}{1}{0}{startpointno}
\erattr{17.625}{-24.8}{1}{1}{stepsize}
\erattr{17.625}{-25.1}{1}{1}{changepointno}
\ergrp{27.591}{-28.45}{55.561}{-0.75}{0.2}{1}\ertext{27.541}{-1.05}{r}{interpretation tower}
\eret{31.591}{-2.6}{46.927}{-1.2}{0.2}{1}\ertext{33.125}{-1.55}{l}{interpretation\textunderscore session}
\erattr{31.791}{-1.75}{1}{0}{sessionguid}
\eret{29.957}{-4.9}{48.561}{-3.5}{0.2}{1}\ertext{30.069}{-3.85}{l}{interpretation\textunderscore event}
\erattr{30.157}{-4.05}{1}{1}{dateTimeOpened}
\erattr{30.157}{-4.35}{0}{1}{dateTimeSaved}
\erattr{30.157}{-4.65}{0}{1}{dateTimeSubmitted}
\eret{33.957}{-4.65}{38.755}{-3.75}{0.2}{0}\ertext{34.437}{-4.1}{l}{programmed\textunderscore interpretation\textunderscore event}
\erattr{34.157}{-4.3}{1}{1}{programname}
\eret{42.755}{-4.65}{47.411}{-3.75}{0.2}{0}\ertext{43.22}{-4.1}{l}{user\textunderscore interpretation\textunderscore event}
\erattr{42.955}{-4.3}{1}{1}{username}
\eret{33.179}{-8}{39.533}{-6.65}{0.2}{1}\ertext{36.356}{-7}{}{sample\textunderscore group(3)}
\eret{31.389}{-14.5}{34.322}{-12.9}{0.2}{1}\ertext{32.856}{-13.25}{}{injection(3)}
\eret{38.709}{-10.25}{41.002}{-9.15}{0.2}{1}\ertext{39.856}{-9.5}{}{compound(3)}
\eret{38.795}{-12.05}{40.417}{-10.95}{0.2}{1}\ertext{39.606}{-11.3}{}{trace(3)}
\eret{31.389}{-17.5}{34.322}{-16.6}{0.2}{1}\ertext{32.856}{-16.95}{}{component(3)}
\eret{29.179}{-19.85}{35.533}{-18.25}{0.2}{1}\ertext{32.356}{-18.6}{}{chromatogram(2)}
\eret{41.169}{-26.7}{45.996}{-24.95}{0.2}{1}\ertext{41.309}{-25.3}{l}{annotation}
\erattr{41.369}{-25.5}{1}{1}{text}
\eret{41.419}{-26.45}{43.913}{-25.85}{0.2}{0}\ertext{42.666}{-26.2}{}{reject\textunderscore this\textunderscore data}
\eret{44.413}{-26.45}{45.746}{-25.85}{0.2}{0}\ertext{45.08}{-26.2}{}{comment}
\eret{45.57}{-7.05}{47.595}{-6.15}{0.2}{1}\ertext{45.773}{-6.5}{l}{method}
\erattr{45.77}{-6.7}{1}{0}{methodname}
\eret{45.789}{-9.15}{47.376}{-7.95}{0.2}{1}\ertext{45.948}{-8.3}{l}{param}
\erattr{45.989}{-8.5}{1}{1}{name}
\erattr{45.989}{-8.8}{1}{1}{value}
\eret{27.883}{-22.95}{30.828}{-21.45}{0.2}{1}\ertext{28.178}{-21.8}{l}{timepoint}
\erattr{28.083}{-22}{1}{0}{time}
\erattr{28.083}{-22.3}{1}{1}{rawintensity}
\erattr{28.083}{-22.6}{1}{1}{smoothedintensity}
\eret{31.328}{-28.15}{36.715}{-21.45}{0.2}{1}\ertext{31.759}{-21.8}{l}{peak}
\erattr{31.528}{-22}{1}{1}{RT}
\erattr{31.528}{-22.3}{1}{1}{peakArea}
\erattr{31.528}{-22.6}{1}{1}{peakHeight}
\erattr{31.528}{-22.9}{1}{1}{chromatogramNoise}
\erattr{31.528}{-23.2}{1}{1}{startRT}
\erattr{31.528}{-23.5}{1}{1}{endRT}
\erattr{31.528}{-23.8}{1}{1}{startHght}
\erattr{31.528}{-24.1}{1}{1}{endHght}
\erattr{31.528}{-24.4}{1}{1}{peakWidthHalfHeight}
\erattr{31.528}{-24.7}{1}{1}{peakSkew}
\eret{31.578}{-27.4}{33.651}{-26.55}{0.2}{0}\ertext{32.615}{-26.9}{}{selected\textunderscore peak}
\eret{33.751}{-25.6}{35.965}{-25}{0.2}{0}\ertext{34.858}{-25.35}{}{candidate\textunderscore peak}
\eret{37.215}{-24.55}{41.615}{-21.45}{0.2}{1}\ertext{37.655}{-21.8}{l}{status}
\erattr{37.415}{-22}{1}{1}{timestamp}
\eret{0}{-0.2}{55.811}{0.3}{0.2}{1}

% relationship 
\ertext{2.519}{-0.5}{l}{}\errelarm{2.369}{-0.2}{2.369}{-0.775}{0}{0}\errelarm{2.369}{-0.775}{2.369}{-1.35}{1}{0}
% relationship 
\ertext{6.288}{-0.5}{l}{}\errelarm{6.138}{-0.2}{6.138}{-3.375}{0}{0}\errelarm{6.138}{-3.375}{6.138}{-6.55}{1}{0}
% relationship 
\ertext{18.325}{-0.5}{l}{}\errelarm{18.175}{-0.2}{18.175}{-1.775}{0}{0}\errelarm{18.175}{-1.775}{18.175}{-3.35}{1}{0}
% relationship submitto
\ertext{9.356}{-0.5}{l}{submit}\ertext{9.356}{-0.8}{l}{to}\errelarm{9.206}{-0.2}{9.206}{-0.775}{0}{0}\errelarm{9.206}{-0.775}{9.206}{-1.35}{1}{0}
% relationship 
\ertext{39.409}{-0.5}{l}{}\errelarm{39.259}{-0.2}{39.259}{-0.7}{0}{0}\errelarm{39.259}{-0.7}{39.259}{-1.2}{1}{0}
% relationship revisedin
\ertext{11.225}{-2.25}{l}{revised}\ertext{11.225}{-2.55}{l}{in}\errelarm{11.075}{-1.95}{11.675}{-1.95}{0}{0}\errelarm{31.391}{-1.9}{31.591}{-1.9}{0}{0}\ertext{21.583}{-2.225}{r}{\textasciitilde /\textasciicircum =\textasciicircum }\errelangle{11.675}{-1.95}{12.275}{-1.95}{21.733}{-1.925}{0}{0}\errelangle{21.733}{-1.925}{31.191}{-1.9}{31.391}{-1.9}{0}{0}\ercrowfoot{11.225}{-1.95}{11.075}{-1.8}{11.075}{-1.95}{11.075}{-2.1}{0}
% relationship 
\ertext{5.004}{-8.6}{l}{}\ertext{6.288}{-13}{l}{..}\errelarm{4.854}{-8.3}{4.854}{-8.375}{0}{0}\errelarm{6.138}{-12.862}{6.138}{-13.15}{1}{0}\errelangle{4.854}{-8.375}{4.854}{-8.45}{5.496}{-10.513}{0}{0}\errelangle{5.496}{-10.513}{6.138}{-12.575}{6.138}{-12.862}{1}{0}\errelseq{6.198}{-12.625}{5.788}{-12.685}{6.488}{-12.745}{6.078}{-12.805}\eridcomprel{6.0375000000000005}{6.2375}{-12.899999999999999}
% relationship 
\ertext{7.443}{-8.6}{l}{}\ertext{8.538}{-9}{l}{..}\errelarm{7.293}{-8.3}{7.293}{-8.375}{0}{0}\errelarm{8.388}{-8.937}{8.388}{-9.15}{1}{0}\errelangle{7.293}{-8.375}{7.293}{-8.45}{7.84}{-8.587}{0}{0}\errelangle{7.84}{-8.587}{8.388}{-8.725}{8.388}{-8.937}{1}{0}\eridcomprel{8.287500000000001}{8.4875}{-8.899999999999999}
% relationship 
\ertext{6.288}{-14.65}{l}{}\ertext{6.288}{-16.6}{l}{..}\errelarm{6.138}{-14.35}{6.138}{-15.55}{0}{0}\errelarm{6.138}{-15.55}{6.138}{-16.75}{1}{0}\eridcomprel{6.0375000000000005}{6.2375}{-16.5}
% relationship 
\ertext{18.325}{-6.05}{l}{}\ertext{18.325}{-6.5}{l}{..}\errelarm{18.175}{-5.75}{18.175}{-6.2}{0}{0}\errelarm{18.175}{-6.2}{18.175}{-6.65}{1}{0}\errelseq{18.235}{-6.125}{17.825}{-6.185}{18.525}{-6.245}{18.115}{-6.305}\eridcomprel{18.075}{18.275000000000002}{-6.3999999999999995}
% relationship 
\ertext{17.848}{-7.85}{l}{}\ertext{15.325}{-13}{l}{..}\errelarm{17.698}{-7.55}{17.698}{-7.625}{0}{0}\errelarm{15.175}{-12.4}{15.175}{-13.15}{1}{0}\errelangle{17.698}{-7.625}{17.698}{-7.7}{16.436}{-9.675}{0}{0}\errelangle{16.436}{-9.675}{15.175}{-11.65}{15.175}{-12.4}{1}{0}\errelseq{15.235}{-12.625}{14.825}{-12.685}{15.525}{-12.745}{15.115}{-12.805}\eridcomprel{15.075000000000001}{15.275}{-12.899999999999999}
% relationship 
\ertext{18.802}{-7.85}{l}{}\ertext{21.325}{-9}{l}{..}\errelarm{18.652}{-7.55}{18.652}{-7.625}{0}{0}\errelarm{21.175}{-8.862}{21.175}{-9.15}{1}{0}\errelangle{18.652}{-7.625}{18.652}{-7.7}{19.914}{-8.137}{0}{0}\errelangle{19.914}{-8.137}{21.175}{-8.575}{21.175}{-8.862}{1}{0}\errelseq{21.235}{-8.625}{20.825}{-8.685}{21.525}{-8.745}{21.115}{-8.805}\eridcomprel{21.075000000000003}{21.275000000000006}{-8.899999999999999}
% relationship group
\ertext{16.594}{-7.25}{r}{group}\errelarm{16.744}{-6.95}{16.444}{-6.95}{0}{0}\errelarm{10.188}{-7.425}{9.988}{-7.425}{0}{0}\errelangle{16.444}{-6.95}{16.144}{-6.95}{13.266}{-7.188}{0}{0}\errelangle{13.266}{-7.188}{10.388}{-7.425}{10.188}{-7.425}{0}{0}\ercrowfoot{16.594}{-6.95}{16.744}{-6.8}{16.744}{-6.95}{16.744}{-7.1}{0}\eridrefrel{16.493750000000002}{-6.85}{-7.049999999999999}
% relationship RTreference
\ertext{19.756}{-7.67}{l}{RT}\ertext{19.756}{-7.97}{l}{reference}\errelarm{19.606}{-7.37}{20.206}{-7.37}{0}{0}\errelarm{16.934}{-14.77}{16.734}{-14.77}{0}{0}\errelangle{20.206}{-7.37}{20.806}{-7.37}{22.306}{-7.37}{0}{0}\errelangle{16.934}{-14.77}{17.134}{-14.77}{20.47}{-14.77}{0}{0}\ertext{23.956}{-9.87}{l}{\textasciitilde /..=.}\errelangle{22.306}{-7.37}{23.806}{-7.37}{23.806}{-11.07}{0}{0}\errelangle{23.806}{-11.07}{23.806}{-14.77}{20.47}{-14.77}{0}{0}\ercrowfoot{19.756}{-7.37}{19.606}{-7.22}{19.606}{-7.37}{19.606}{-7.52}{0}
% relationship 
\ertext{15.325}{-16.15}{l}{}\ertext{15.325}{-16.6}{l}{..}\errelarm{15.175}{-15.85}{15.175}{-16.3}{0}{0}\errelarm{15.175}{-16.3}{15.175}{-16.75}{1}{0}\errelseq{15.235}{-16.225}{14.825}{-16.285}{15.525}{-16.345}{15.115}{-16.405}\eridcomprel{15.075000000000001}{15.275}{-16.5}
% relationship subject
\ertext{13.466}{-14.8}{r}{subject}\errelarm{13.616}{-14.5}{13.466}{-14.5}{0}{0}\errelarm{7.838}{-13.75}{7.638}{-13.75}{0}{0}\ertext{10.827}{-14.425}{l}{\textasciitilde /..=../group}\errelangle{13.466}{-14.5}{13.316}{-14.5}{10.677}{-14.125}{0}{0}\errelangle{10.677}{-14.125}{8.038}{-13.75}{7.838}{-13.75}{0}{0}\ercrowfoot{13.466}{-14.5}{13.616}{-14.35}{13.616}{-14.5}{13.616}{-14.65}{0}\eridrefrel{13.36625}{-14.399999999999999}{-14.599999999999998}
% relationship 
\ertext{21.325}{-10.35}{l}{}\ertext{21.325}{-10.8}{l}{..}\errelarm{21.175}{-10.05}{21.175}{-10.5}{0}{0}\errelarm{21.175}{-10.5}{21.175}{-10.95}{1}{0}\eridcomprel{21.075000000000003}{21.275000000000006}{-10.7}
% relationship subject
\ertext{19.929}{-9.75}{r}{subject}\errelarm{20.079}{-9.45}{19.779}{-9.45}{0}{0}\errelarm{9.944}{-9.6}{9.744}{-9.6}{0}{0}\ertext{14.661}{-9.825}{r}{\textasciitilde /..=../group}\errelangle{19.779}{-9.45}{19.479}{-9.45}{14.811}{-9.525}{0}{0}\errelangle{14.811}{-9.525}{10.144}{-9.6}{9.944}{-9.6}{0}{0}\ercrowfoot{19.929}{-9.45}{20.079}{-9.3}{20.079}{-9.45}{20.079}{-9.6}{0}\eridrefrel{19.828625000000002}{-9.35}{-9.549999999999999}
% relationship IS
\ertext{22.421}{-9.9}{l}{IS}\errelangle{22.271}{-9.6}{22.271}{-9.6}{22.571}{-9.6}{0}{0}\errelangle{22.271}{-9.87}{22.271}{-9.87}{22.571}{-9.87}{0}{0}\ertext{23.121}{-10.035}{l}{\textasciitilde /..=..}\errelangle{22.571}{-9.6}{22.871}{-9.6}{22.871}{-9.735}{0}{0}\errelangle{22.871}{-9.735}{22.871}{-9.87}{22.571}{-9.87}{0}{0}\ercrowfoot{22.421}{-9.6}{22.271}{-9.45}{22.271}{-9.6}{22.271}{-9.75}{0}
% relationship 
\ertext{15.325}{-17.65}{l}{}\ertext{15.325}{-18.1}{l}{..}\errelarm{15.175}{-17.35}{15.175}{-17.8}{0}{0}\errelarm{15.175}{-17.8}{15.175}{-18.25}{1}{0}\errelseq{15.235}{-17.725}{14.825}{-17.785}{15.525}{-17.845}{15.115}{-17.905}\eridcomprel{15.075000000000001}{15.275}{-18.000000000000004}
% relationship monitored
\ertext{16.542}{-17.17}{l}{monitored}\errelarm{16.392}{-16.87}{16.992}{-16.87}{0}{0}\errelarm{19.879}{-9.87}{20.079}{-9.87}{0}{0}\ertext{19.235}{-13.67}{r}{\textasciitilde /..=../..}\errelangle{16.992}{-16.87}{17.592}{-16.87}{18.635}{-13.37}{0}{0}\errelangle{18.635}{-13.37}{19.679}{-9.87}{19.879}{-9.87}{0}{0}\ercrowfoot{16.542}{-16.87}{16.392}{-16.72}{16.392}{-16.87}{16.392}{-17.02}{0}\eridrefrel{16.6415}{-16.77}{-16.970000000000002}
% relationship 
\ertext{15.325}{-20.05}{l}{}\errelarm{15.175}{-19.75}{15.175}{-19.825}{0}{0}\errelarm{17.675}{-20.438}{17.675}{-20.65}{1}{0}\errelangle{15.175}{-19.825}{15.175}{-19.9}{16.425}{-20.063}{0}{0}\errelangle{16.425}{-20.063}{17.675}{-20.225}{17.675}{-20.438}{1}{0}\eridcomprel{17.575}{17.775000000000002}{-20.400000000000006}
% relationship extracted
\ertext{17.075}{-18.925}{l}{extracted}\errelarm{16.925}{-18.625}{17.975}{-18.625}{0}{0}\errelarm{20.089}{-11.4}{20.289}{-11.4}{0}{0}\ertext{21.057}{-15.313}{r}{\textasciitilde /..=../monitored}\errelangle{17.975}{-18.625}{19.025}{-18.625}{19.457}{-15.013}{0}{0}\errelangle{19.457}{-15.013}{19.889}{-11.4}{20.089}{-11.4}{0}{0}\ercrowfoot{17.075}{-18.625}{16.925}{-18.475}{16.925}{-18.625}{16.925}{-18.775}{0}\eridrefrel{17.175}{-18.525000000000002}{-18.725000000000005}
% relationship 
\ertext{19.563}{-23.35}{l}{}\errelarm{19.413}{-23.05}{19.413}{-23.5}{0}{0}\errelarm{19.413}{-23.5}{19.413}{-23.95}{1}{0}\eridcomprel{19.312875000000002}{19.512875000000005}{-23.70000000000001}
% relationship 
\ertext{39.409}{-2.9}{l}{}\ertext{39.409}{-3.35}{l}{..}\errelarm{39.259}{-2.6}{39.259}{-3.05}{0}{0}\errelarm{39.259}{-3.05}{39.259}{-3.5}{1}{0}\eridcomprel{39.15875}{39.35875}{-3.2499999999999996}
% relationship 
\ertext{34.758}{-5.2}{l}{}\ertext{36.506}{-6.5}{l}{..}\errelarm{34.608}{-4.9}{34.608}{-4.975}{0}{0}\errelarm{36.356}{-6.437}{36.356}{-6.65}{1}{0}\errelangle{34.608}{-4.975}{34.608}{-5.05}{35.482}{-5.637}{0}{0}\errelangle{35.482}{-5.637}{36.356}{-6.225}{36.356}{-6.437}{1}{0}\eridcomprel{36.205625000000005}{36.505625}{-6.35}\ercrowfoot{36.356}{-6.5}{36.206}{-6.65}{36.356}{-6.65}{36.506}{-6.65}{0}\ercrowfoot{36.356}{-6.5}{36.206}{-6.35}{36.356}{-6.35}{36.506}{-6.35}{0}
% relationship 
\ertext{39.409}{-5.2}{l}{}\ertext{46.733}{-6}{l}{..}\errelarm{39.259}{-4.9}{39.259}{-4.975}{0}{0}\errelarm{46.583}{-6.1}{46.583}{-6.15}{1}{0}\errelangle{39.259}{-4.975}{39.259}{-5.05}{42.921}{-5.55}{0}{0}\errelangle{42.921}{-5.55}{46.583}{-6.05}{46.583}{-6.1}{1}{0}
% relationship 
\ertext{44.06}{-5.2}{l}{}\errelarm{43.91}{-4.9}{43.91}{-4.975}{0}{0}\errelarm{45.272}{-24.9}{45.272}{-24.95}{1}{0}\errelangle{43.91}{-4.975}{43.91}{-5.05}{44.591}{-14.95}{0}{0}\errelangle{44.591}{-14.95}{45.272}{-24.85}{45.272}{-24.9}{1}{0}
% relationship baseevent
\ertext{48.711}{-4.92}{l}{base}\ertext{48.711}{-5.22}{l}{event}\errelangle{48.561}{-4.62}{48.561}{-4.62}{48.861}{-4.62}{0}{0}\errelangle{48.561}{-4.34}{48.561}{-4.34}{48.861}{-4.34}{0}{0}\ertext{50.311}{-5.08}{r}{session}\ertext{50.311}{-4.78}{r}{\textasciitilde /..=base}\errelangle{48.861}{-4.62}{49.161}{-4.62}{49.161}{-4.48}{0}{0}\errelangle{49.161}{-4.48}{49.161}{-4.34}{48.861}{-4.34}{0}{0}\ercrowfoot{48.711}{-4.62}{48.561}{-4.47}{48.561}{-4.62}{48.561}{-4.77}{0}
% relationship basesession
\ertext{48.711}{-4.08}{l}{base}\ertext{48.711}{-4.38}{l}{session}\errelarm{48.561}{-3.78}{49.161}{-3.78}{0}{0}\errelarm{47.127}{-2.32}{46.927}{-2.32}{0}{0}\errelangle{49.161}{-3.78}{49.761}{-3.78}{49.811}{-3.78}{0}{0}\errelangle{47.127}{-2.32}{47.326}{-2.32}{48.594}{-2.32}{0}{0}\ertext{49.561}{-3.35}{r}{\textasciitilde /\textasciicircum =\textasciicircum }\errelangle{49.811}{-3.78}{49.861}{-3.78}{49.861}{-3.05}{0}{0}\errelangle{49.861}{-3.05}{49.861}{-2.32}{48.594}{-2.32}{0}{0}\ercrowfoot{48.711}{-3.78}{48.561}{-3.63}{48.561}{-3.78}{48.561}{-3.93}{0}
% relationship subject
\ertext{29.807}{-4.85}{r}{subject}\errelarm{29.957}{-4.55}{29.357}{-4.55}{0}{0}\errelarm{19.975}{-4.55}{19.775}{-4.55}{0}{0}\ertext{24.616}{-4.85}{l}{\textasciitilde /\textasciicircum =\textasciicircum }\errelarm{29.357}{-4.55}{24.466}{-4.55}{0}{0}\errelarm{24.466}{-4.55}{19.975}{-4.55}{0}{0}\ercrowfoot{29.807}{-4.55}{29.957}{-4.4}{29.957}{-4.55}{29.957}{-4.7}{0}\eridrefrel{29.65675}{-4.399999999999999}{-4.699999999999999}\ercrowfoot{29.807}{-4.55}{29.957}{-4.4}{29.957}{-4.55}{29.957}{-4.7}{0}\ercrowfoot{29.807}{-4.55}{29.657}{-4.4}{29.657}{-4.55}{29.657}{-4.7}{0}
% relationship 
\ertext{34.917}{-8.3}{l}{}\ertext{33.006}{-12.75}{l}{..}\errelarm{34.767}{-8}{34.767}{-8.075}{0}{0}\errelarm{32.856}{-12.15}{32.856}{-12.9}{1}{0}\errelangle{34.767}{-8.075}{34.767}{-8.15}{33.811}{-9.775}{0}{0}\errelangle{33.811}{-9.775}{32.856}{-11.4}{32.856}{-12.15}{1}{0}\eridcomprel{32.705625000000005}{33.005625}{-12.599999999999998}\ercrowfoot{32.856}{-12.75}{32.706}{-12.9}{32.856}{-12.9}{33.006}{-12.9}{0}\ercrowfoot{32.856}{-12.75}{32.706}{-12.6}{32.856}{-12.6}{33.006}{-12.6}{0}
% relationship 
\ertext{37.776}{-8.3}{l}{}\ertext{40.006}{-9}{l}{..}\errelarm{37.626}{-8}{37.626}{-8.075}{0}{0}\errelarm{39.856}{-8.937}{39.856}{-9.15}{1}{0}\errelangle{37.626}{-8.075}{37.626}{-8.15}{38.741}{-8.438}{0}{0}\errelangle{38.741}{-8.438}{39.856}{-8.725}{39.856}{-8.937}{1}{0}\eridcomprel{39.705625000000005}{40.005625}{-8.849999999999998}\ercrowfoot{39.856}{-9}{39.706}{-9.15}{39.856}{-9.15}{40.006}{-9.15}{0}\ercrowfoot{39.856}{-9}{39.706}{-8.85}{39.856}{-8.85}{40.006}{-8.85}{0}
% relationship 
\ertext{39.047}{-8.3}{l}{}\errelarm{38.897}{-8}{38.897}{-8.25}{0}{0}\errelarm{44.427}{-24.9}{44.427}{-24.95}{1}{0}\errelangle{38.897}{-8.25}{38.897}{-8.5}{41.397}{-8.5}{0}{0}\errelangle{44.427}{-24.9}{44.427}{-24.85}{44.427}{-24.85}{1}{0}\errelangle{41.397}{-8.5}{43.897}{-8.5}{43.897}{-9.875}{0}{0}\errelangle{44.427}{-24.85}{44.427}{-24.85}{44.427}{-18.05}{1}{0}\errelangle{43.897}{-9.875}{43.897}{-11.25}{44.162}{-11.25}{0}{0}\errelangle{44.162}{-11.25}{44.427}{-11.25}{44.427}{-18.05}{1}{0}
% relationship subject
\ertext{33.029}{-7.625}{r}{subject}\errelarm{33.179}{-7.325}{32.579}{-7.325}{0}{0}\errelarm{19.806}{-7.1}{19.606}{-7.1}{0}{0}\ertext{25.893}{-7.612}{l}{\textasciitilde /..=../subject}\errelangle{32.579}{-7.325}{31.979}{-7.325}{25.993}{-7.212}{0}{0}\errelangle{25.993}{-7.212}{20.006}{-7.1}{19.806}{-7.1}{0}{0}\ercrowfoot{33.029}{-7.325}{33.179}{-7.175}{33.179}{-7.325}{33.179}{-7.475}{0}\eridrefrel{32.878750000000004}{-7.174999999999999}{-7.475}\ercrowfoot{33.029}{-7.325}{33.179}{-7.175}{33.179}{-7.325}{33.179}{-7.475}{0}\ercrowfoot{33.029}{-7.325}{32.879}{-7.175}{32.879}{-7.325}{32.879}{-7.475}{0}
% relationship RTreference
\ertext{39.683}{-8.098}{l}{RT}\ertext{39.683}{-8.398}{l}{reference}\errelarm{39.533}{-7.797}{40.133}{-7.797}{0}{0}\errelarm{34.522}{-14.18}{34.322}{-14.18}{0}{0}\errelangle{40.133}{-7.797}{40.733}{-7.797}{41.733}{-7.797}{0}{0}\errelangle{34.522}{-14.18}{34.722}{-14.18}{38.727}{-14.18}{0}{0}\ertext{42.883}{-9.789}{l}{\textasciitilde /..=.}\errelangle{41.733}{-7.797}{42.733}{-7.797}{42.733}{-10.989}{0}{0}\errelangle{42.733}{-10.989}{42.733}{-14.18}{38.727}{-14.18}{0}{0}\ercrowfoot{39.683}{-7.797}{39.533}{-7.647}{39.533}{-7.797}{39.533}{-7.948}{0}
% relationship basedon
\ertext{39.683}{-7.49}{l}{based}\ertext{39.683}{-7.79}{l}{on}\errelangle{39.533}{-7.19}{39.533}{-7.19}{39.832}{-7.19}{0}{0}\errelangle{39.533}{-6.852}{39.533}{-6.852}{39.832}{-6.852}{0}{0}\ertext{41.383}{-7.621}{r}{event}\ertext{41.383}{-7.321}{r}{\textasciitilde /..=../base}\errelangle{39.832}{-7.19}{40.133}{-7.19}{40.133}{-7.021}{0}{0}\errelangle{40.133}{-7.021}{40.133}{-6.852}{39.832}{-6.852}{0}{0}\ercrowfoot{39.683}{-7.19}{39.533}{-7.04}{39.533}{-7.19}{39.533}{-7.34}{0}
% relationship 
\ertext{33.006}{-14.8}{l}{}\ertext{33.006}{-16.45}{l}{..}\errelarm{32.856}{-14.5}{32.856}{-15.55}{0}{0}\errelarm{32.856}{-15.55}{32.856}{-16.6}{1}{0}\eridcomprel{32.705625000000005}{33.005625}{-16.3}\ercrowfoot{32.856}{-16.45}{32.706}{-16.6}{32.856}{-16.6}{33.006}{-16.6}{0}\ercrowfoot{32.856}{-16.45}{32.706}{-16.3}{32.856}{-16.3}{33.006}{-16.3}{0}
% relationship subject
\ertext{31.239}{-14}{r}{subject}\errelarm{31.389}{-13.7}{30.789}{-13.7}{0}{0}\errelarm{16.934}{-14.5}{16.734}{-14.5}{0}{0}\ertext{23.511}{-14.4}{r}{\textasciitilde /..=../subject}\errelangle{30.789}{-13.7}{30.189}{-13.7}{23.661}{-14.1}{0}{0}\errelangle{23.661}{-14.1}{17.134}{-14.5}{16.934}{-14.5}{0}{0}\ercrowfoot{31.239}{-13.7}{31.389}{-13.55}{31.389}{-13.7}{31.389}{-13.85}{0}\eridrefrel{31.089125000000003}{-13.549999999999999}{-13.85}\ercrowfoot{31.239}{-13.7}{31.389}{-13.55}{31.389}{-13.7}{31.389}{-13.85}{0}\ercrowfoot{31.239}{-13.7}{31.089}{-13.55}{31.089}{-13.7}{31.089}{-13.85}{0}
% relationship basedon
\ertext{34.472}{-13.84}{l}{based}\ertext{34.472}{-14.14}{l}{on}\errelangle{34.322}{-13.54}{34.322}{-13.54}{34.622}{-13.54}{0}{0}\errelangle{34.322}{-13.14}{34.322}{-13.14}{34.622}{-13.14}{0}{0}\ertext{36.172}{-13.94}{r}{on}\ertext{36.172}{-13.64}{r}{\textasciitilde /..=../based}\errelangle{34.622}{-13.54}{34.922}{-13.54}{34.922}{-13.34}{0}{0}\errelangle{34.922}{-13.34}{34.922}{-13.14}{34.622}{-13.14}{0}{0}\ercrowfoot{34.472}{-13.54}{34.322}{-13.39}{34.322}{-13.54}{34.322}{-13.69}{0}
% relationship 
\ertext{39.624}{-10.55}{l}{}\ertext{39.756}{-10.8}{l}{..}\errelarm{39.474}{-10.25}{39.474}{-10.325}{0}{0}\errelarm{39.606}{-10.737}{39.606}{-10.95}{1}{0}\errelangle{39.474}{-10.325}{39.474}{-10.4}{39.54}{-10.462}{0}{0}\errelangle{39.54}{-10.462}{39.606}{-10.525}{39.606}{-10.737}{1}{0}\eridcomprel{39.455625000000005}{39.755625}{-10.649999999999999}\ercrowfoot{39.606}{-10.8}{39.456}{-10.95}{39.606}{-10.95}{39.756}{-10.95}{0}\ercrowfoot{39.606}{-10.8}{39.456}{-10.65}{39.606}{-10.65}{39.756}{-10.65}{0}
% relationship 
\ertext{40.923}{-10.55}{l}{}\errelarm{40.773}{-10.25}{40.773}{-10.325}{0}{0}\errelarm{43.583}{-24.9}{43.583}{-24.95}{1}{0}\errelangle{40.773}{-10.325}{40.773}{-10.4}{40.773}{-12.4}{0}{0}\errelangle{43.583}{-24.9}{43.583}{-24.85}{43.583}{-19.625}{1}{0}\errelangle{40.773}{-12.4}{40.773}{-14.4}{42.178}{-14.4}{0}{0}\errelangle{42.178}{-14.4}{43.583}{-14.4}{43.583}{-19.625}{1}{0}
% relationship subject
\ertext{38.559}{-9.67}{r}{subject}\errelarm{38.709}{-9.37}{38.109}{-9.37}{0}{0}\errelarm{22.471}{-9.6}{22.271}{-9.6}{0}{0}\ertext{31.44}{-9.785}{r}{\textasciitilde /..=../subject}\errelangle{38.109}{-9.37}{37.509}{-9.37}{30.09}{-9.485}{0}{0}\errelangle{30.09}{-9.485}{22.671}{-9.6}{22.471}{-9.6}{0}{0}\ercrowfoot{38.559}{-9.37}{38.709}{-9.22}{38.709}{-9.37}{38.709}{-9.52}{0}\eridrefrel{38.40925000000001}{-9.219999999999999}{-9.52}\ercrowfoot{38.559}{-9.37}{38.709}{-9.22}{38.709}{-9.37}{38.709}{-9.52}{0}\ercrowfoot{38.559}{-9.37}{38.409}{-9.22}{38.409}{-9.37}{38.409}{-9.52}{0}
% relationship IS
\ertext{41.152}{-9.67}{l}{IS}\errelangle{41.002}{-9.37}{41.002}{-9.37}{41.302}{-9.37}{0}{0}\errelangle{41.002}{-9.7}{41.002}{-9.7}{41.302}{-9.7}{0}{0}\ertext{41.852}{-9.835}{l}{\textasciitilde /..=..}\errelangle{41.302}{-9.37}{41.602}{-9.37}{41.602}{-9.535}{0}{0}\errelangle{41.602}{-9.535}{41.602}{-9.7}{41.302}{-9.7}{0}{0}\ercrowfoot{41.152}{-9.37}{41.002}{-9.22}{41.002}{-9.37}{41.002}{-9.52}{0}
% relationship selected
\ertext{41.152}{-10.33}{l}{selected}\errelarm{41.002}{-10.03}{41.102}{-10.03}{0}{0}\errelarm{40.617}{-11.28}{40.417}{-11.28}{0}{0}\errelangle{41.102}{-10.03}{41.202}{-10.03}{41.252}{-10.03}{0}{0}\errelangle{40.617}{-11.28}{40.817}{-11.28}{41.059}{-11.28}{0}{0}\ertext{41.552}{-11.255}{l}{\textasciitilde /..=.}\errelangle{41.252}{-10.03}{41.302}{-10.03}{41.302}{-10.655}{0}{0}\errelangle{41.302}{-10.655}{41.302}{-11.28}{41.059}{-11.28}{0}{0}\ercrowfoot{41.152}{-10.03}{41.002}{-9.88}{41.002}{-10.03}{41.002}{-10.18}{0}
% relationship subject
\ertext{38.645}{-11.47}{r}{subject}\errelarm{38.795}{-11.17}{38.195}{-11.17}{0}{0}\errelarm{22.261}{-11.4}{22.061}{-11.4}{0}{0}\ertext{31.378}{-11.585}{r}{\textasciitilde /..=../subject}\errelangle{38.195}{-11.17}{37.595}{-11.17}{30.028}{-11.285}{0}{0}\errelangle{30.028}{-11.285}{22.461}{-11.4}{22.261}{-11.4}{0}{0}\ercrowfoot{38.645}{-11.17}{38.795}{-11.02}{38.795}{-11.17}{38.795}{-11.32}{0}\eridrefrel{38.494625000000006}{-11.02}{-11.32}\ercrowfoot{38.645}{-11.17}{38.795}{-11.02}{38.795}{-11.17}{38.795}{-11.32}{0}\ercrowfoot{38.645}{-11.17}{38.495}{-11.02}{38.495}{-11.17}{38.495}{-11.32}{0}
% relationship 
\ertext{32.517}{-17.8}{l}{}\ertext{32.506}{-18.1}{l}{..}\errelarm{32.367}{-17.5}{32.367}{-17.575}{0}{0}\errelarm{32.356}{-18.038}{32.356}{-18.25}{1}{0}\errelangle{32.367}{-17.575}{32.367}{-17.65}{32.361}{-17.738}{0}{0}\errelangle{32.361}{-17.738}{32.356}{-17.825}{32.356}{-18.038}{1}{0}\eridcomprel{32.205625000000005}{32.505625}{-17.950000000000003}\ercrowfoot{32.356}{-18.1}{32.206}{-18.25}{32.356}{-18.25}{32.506}{-18.25}{0}\ercrowfoot{32.356}{-18.1}{32.206}{-17.95}{32.356}{-17.95}{32.506}{-17.95}{0}
% relationship 
\ertext{34.179}{-17.8}{l}{}\errelarm{34.029}{-17.5}{34.029}{-17.575}{0}{0}\errelarm{42.738}{-24.9}{42.738}{-24.95}{1}{0}\errelangle{34.029}{-17.575}{34.029}{-17.65}{34.029}{-17.775}{0}{0}\errelangle{42.738}{-24.9}{42.738}{-24.85}{42.738}{-21.375}{1}{0}\errelangle{34.029}{-17.775}{34.029}{-17.9}{38.383}{-17.9}{0}{0}\errelangle{38.383}{-17.9}{42.738}{-17.9}{42.738}{-21.375}{1}{0}
% relationship subject
\ertext{31.239}{-17.35}{r}{subject}\ertext{24.04}{-17.35}{l}{\textasciitilde /..=../subject}\errelarm{31.389}{-17.05}{23.89}{-17.05}{0}{0}\errelarm{23.89}{-17.05}{16.392}{-17.05}{0}{0}\ercrowfoot{31.239}{-17.05}{31.389}{-16.9}{31.389}{-17.05}{31.389}{-17.2}{0}\eridrefrel{31.089125000000003}{-16.900000000000002}{-17.2}\ercrowfoot{31.239}{-17.05}{31.389}{-16.9}{31.389}{-17.05}{31.389}{-17.2}{0}\ercrowfoot{31.239}{-17.05}{31.089}{-16.9}{31.089}{-17.05}{31.089}{-17.2}{0}
% relationship monitored
\ertext{34.472}{-17.08}{l}{monitored}\errelarm{34.322}{-16.78}{34.922}{-16.78}{0}{0}\errelarm{38.509}{-10.03}{38.709}{-10.03}{0}{0}\ertext{37.516}{-13.705}{r}{\textasciitilde /..=../..}\errelangle{34.922}{-16.78}{35.522}{-16.78}{36.916}{-13.405}{0}{0}\errelangle{36.916}{-13.405}{38.309}{-10.03}{38.509}{-10.03}{0}{0}\ercrowfoot{34.472}{-16.78}{34.322}{-16.63}{34.322}{-16.78}{34.322}{-16.93}{0}
% relationship basedon
\ertext{34.472}{-17.575}{l}{based}\ertext{34.472}{-17.875}{l}{on}\errelangle{34.322}{-17.275}{34.322}{-17.275}{34.622}{-17.275}{0}{0}\errelangle{34.322}{-17.05}{34.322}{-17.05}{34.622}{-17.05}{0}{0}\ertext{36.172}{-17.763}{r}{on}\ertext{36.172}{-17.463}{r}{\textasciitilde /..=../based}\errelangle{34.622}{-17.275}{34.922}{-17.275}{34.922}{-17.163}{0}{0}\errelangle{34.922}{-17.163}{34.922}{-17.05}{34.622}{-17.05}{0}{0}\ercrowfoot{34.472}{-17.275}{34.322}{-17.125}{34.322}{-17.275}{34.322}{-17.425}{0}
% relationship timeseries
\ertext{29.964}{-20.15}{l}{time}\ertext{29.964}{-20.45}{l}{series}\ertext{29.506}{-21.3}{l}{..}\errelarm{29.814}{-19.85}{29.814}{-20.05}{0}{0}\errelarm{29.356}{-21.4}{29.356}{-21.45}{1}{0}\errelangle{29.814}{-20.05}{29.814}{-20.25}{29.585}{-20.8}{0}{0}\errelangle{29.585}{-20.8}{29.356}{-21.35}{29.356}{-21.4}{1}{0}
% relationship 
\ertext{32.506}{-20.15}{l}{}\ertext{32.765}{-26.4}{l}{..}\errelarm{32.356}{-19.85}{32.356}{-19.925}{0}{0}\errelarm{32.615}{-26.5}{32.615}{-26.55}{1}{0}\errelangle{32.356}{-19.925}{32.356}{-20}{32.485}{-23.225}{0}{0}\errelangle{32.485}{-23.225}{32.615}{-26.45}{32.615}{-26.5}{1}{0}
% relationship 
\ertext{33.141}{-20.15}{l}{}\ertext{35.008}{-24.85}{l}{..}\errelarm{32.991}{-19.85}{32.991}{-19.925}{0}{0}\errelarm{34.858}{-24.95}{34.858}{-25}{1}{0}\errelangle{32.991}{-19.925}{32.991}{-20}{33.925}{-22.45}{0}{0}\errelangle{33.925}{-22.45}{34.858}{-24.9}{34.858}{-24.95}{1}{0}
% relationship 
\ertext{34.094}{-20.15}{l}{}\ertext{39.565}{-21.3}{l}{..}\errelarm{33.944}{-19.85}{33.944}{-20.1}{0}{0}\errelarm{39.415}{-21.4}{39.415}{-21.45}{1}{0}\errelangle{33.944}{-20.1}{33.944}{-20.35}{36.679}{-20.85}{0}{0}\errelangle{36.679}{-20.85}{39.415}{-21.35}{39.415}{-21.4}{1}{0}
% relationship 
\ertext{35.047}{-20.15}{l}{}\errelarm{34.897}{-19.85}{34.897}{-20.1}{0}{0}\errelarm{41.893}{-24.9}{41.893}{-24.95}{1}{0}\errelangle{34.897}{-20.1}{34.897}{-20.35}{34.897}{-20.4}{0}{0}\errelangle{41.893}{-24.9}{41.893}{-24.85}{41.893}{-22.65}{1}{0}\errelangle{34.897}{-20.4}{34.897}{-20.45}{38.395}{-20.45}{0}{0}\errelangle{38.395}{-20.45}{41.893}{-20.45}{41.893}{-22.65}{1}{0}
% relationship subject
\ertext{29.029}{-18.95}{r}{subject}\errelarm{29.179}{-18.65}{28.579}{-18.65}{0}{0}\errelarm{17.125}{-19}{16.925}{-19}{0}{0}\ertext{22.502}{-19.125}{r}{\textasciitilde /..=../subject}\errelangle{28.579}{-18.65}{27.979}{-18.65}{22.652}{-18.825}{0}{0}\errelangle{22.652}{-18.825}{17.325}{-19}{17.125}{-19}{0}{0}\ercrowfoot{29.029}{-18.65}{29.179}{-18.5}{29.179}{-18.65}{29.179}{-18.8}{0}\eridrefrel{28.878750000000004}{-18.500000000000004}{-18.8}\ercrowfoot{29.029}{-18.65}{29.179}{-18.5}{29.179}{-18.65}{29.179}{-18.8}{0}\ercrowfoot{29.029}{-18.65}{28.879}{-18.5}{28.879}{-18.65}{28.879}{-18.8}{0}
% relationship extracted
\ertext{35.683}{-18.95}{l}{extracted}\ertext{38.645}{-11.68}{r}{chromatograms}\errelarm{35.533}{-18.65}{36.333}{-18.65}{0}{0}\errelarm{38.595}{-11.83}{38.795}{-11.83}{0}{0}\ertext{39.364}{-15.54}{r}{\textasciitilde /..=../monitored}\errelangle{36.333}{-18.65}{37.133}{-18.65}{37.764}{-15.24}{0}{0}\errelangle{37.764}{-15.24}{38.395}{-11.83}{38.595}{-11.83}{0}{0}
% relationship basedon
\ertext{35.683}{-19.75}{l}{based}\ertext{35.683}{-20.05}{l}{on}\errelangle{35.533}{-19.45}{35.533}{-19.45}{35.833}{-19.45}{0}{0}\errelangle{35.533}{-19.05}{35.533}{-19.05}{35.833}{-19.05}{0}{0}\ertext{37.383}{-19.85}{r}{on}\ertext{37.383}{-19.55}{r}{\textasciitilde /..=../based}\errelangle{35.833}{-19.45}{36.133}{-19.45}{36.133}{-19.25}{0}{0}\errelangle{36.133}{-19.25}{36.133}{-19.05}{35.833}{-19.05}{0}{0}\ercrowfoot{35.683}{-19.45}{35.533}{-19.3}{35.533}{-19.45}{35.533}{-19.6}{0}
% relationship lastmodified
\ertext{46.146}{-26.125}{l}{last}\ertext{46.146}{-26.425}{l}{modified}\errelarm{45.996}{-25.825}{46.146}{-25.825}{0}{0}\errelarm{47.127}{-1.48}{46.927}{-1.48}{0}{0}\errelangle{46.146}{-25.825}{46.296}{-25.825}{50.646}{-25.825}{0}{0}\errelangle{47.127}{-1.48}{47.326}{-1.48}{51.161}{-1.48}{0}{0}\ertext{55.596}{-13.953}{r}{\textasciitilde /\textasciicircum =\textasciicircum }\errelangle{50.646}{-25.825}{54.996}{-25.825}{54.996}{-13.653}{0}{0}\errelangle{54.996}{-13.653}{54.996}{-1.48}{51.161}{-1.48}{0}{0}\ercrowfoot{46.146}{-25.825}{45.996}{-25.675}{45.996}{-25.825}{45.996}{-25.975}{0}\erarc{41.652}{-24.75}{42.617}{-24.55}{44.548}{-24.55}{45.514}{-24.75}
% relationship 
\ertext{46.733}{-7.35}{l}{}\errelarm{46.583}{-7.05}{46.583}{-7.5}{0}{0}\errelarm{46.583}{-7.5}{46.583}{-7.95}{1}{0}\eridcomprel{46.482625}{46.682625}{-7.699999999999998}
% relationship previous
\ertext{33.801}{-27.53}{l}{previous}\errelangle{33.651}{-27.23}{33.651}{-27.23}{33.951}{-27.23}{0}{0}\errelangle{33.651}{-26.975}{33.651}{-26.975}{33.951}{-26.975}{0}{0}\ertext{35.501}{-27.703}{r}{on}\ertext{35.501}{-27.403}{r}{\textasciitilde /..=../based}\errelangle{33.951}{-27.23}{34.251}{-27.23}{34.251}{-27.103}{0}{0}\errelangle{34.251}{-27.103}{34.251}{-26.975}{33.951}{-26.975}{0}{0}\ercrowfoot{33.801}{-27.23}{33.651}{-27.08}{33.651}{-27.23}{33.651}{-27.38}{0}
% relationship method
\ertext{41.765}{-22.215}{l}{method}\errelarm{41.615}{-21.915}{42.215}{-21.915}{0}{0}\errelarm{47.795}{-6.69}{47.595}{-6.69}{0}{0}\errelangle{42.215}{-21.915}{42.815}{-21.915}{47.965}{-21.915}{0}{0}\errelangle{47.795}{-6.69}{47.995}{-6.69}{50.555}{-6.69}{0}{0}\ertext{54.715}{-14.603}{r}{\textasciitilde /..=../../../../..}\errelangle{47.965}{-21.915}{53.115}{-21.915}{53.115}{-14.303}{0}{0}\errelangle{53.115}{-14.303}{53.115}{-6.69}{50.555}{-6.69}{0}{0}\ercrowfoot{41.765}{-21.915}{41.615}{-21.765}{41.615}{-21.915}{41.615}{-22.065}{0}
% relationship lastreviewed
\ertext{41.765}{-22.68}{l}{last}\ertext{41.765}{-22.98}{l}{reviewed}\errelarm{41.615}{-22.38}{42.215}{-22.38}{0}{0}\errelarm{47.127}{-2.04}{46.927}{-2.04}{0}{0}\errelangle{42.215}{-22.38}{42.815}{-22.38}{48.815}{-22.38}{0}{0}\errelangle{47.127}{-2.04}{47.326}{-2.04}{51.071}{-2.04}{0}{0}\ertext{55.415}{-12.51}{r}{\textasciitilde /\textasciicircum =\textasciicircum }\errelangle{48.815}{-22.38}{54.815}{-22.38}{54.815}{-12.21}{0}{0}\errelangle{54.815}{-12.21}{54.815}{-2.04}{51.071}{-2.04}{0}{0}\ercrowfoot{41.765}{-22.38}{41.615}{-22.23}{41.615}{-22.38}{41.615}{-22.53}{0}
% relationship lastmodified
\ertext{41.765}{-23.3}{l}{last}\ertext{41.765}{-23.6}{l}{modified}\errelarm{41.615}{-23}{42.215}{-23}{0}{0}\errelarm{47.127}{-1.76}{46.927}{-1.76}{0}{0}\errelangle{42.215}{-23}{42.815}{-23}{49.24}{-23}{0}{0}\errelangle{47.127}{-1.76}{47.326}{-1.76}{51.496}{-1.76}{0}{0}\ertext{56.265}{-12.68}{r}{\textasciitilde /\textasciicircum =\textasciicircum }\errelangle{49.24}{-23}{55.665}{-23}{55.665}{-12.38}{0}{0}\errelangle{55.665}{-12.38}{55.665}{-1.76}{51.496}{-1.76}{0}{0}\ercrowfoot{41.765}{-23}{41.615}{-22.85}{41.615}{-23}{41.615}{-23.15}{0}
% relationship previous
\ertext{41.765}{-24.54}{l}{previous}\errelangle{41.615}{-24.24}{41.615}{-24.24}{41.915}{-24.24}{0}{0}\errelangle{41.615}{-23.775}{41.615}{-23.775}{41.915}{-23.775}{0}{0}\ertext{43.465}{-24.608}{r}{on}\ertext{43.465}{-24.308}{r}{\textasciitilde /..=../based}\errelangle{41.915}{-24.24}{42.215}{-24.24}{42.215}{-24.008}{0}{0}\errelangle{42.215}{-24.008}{42.215}{-23.775}{41.915}{-23.775}{0}{0}\ercrowfoot{41.765}{-24.24}{41.615}{-24.09}{41.615}{-24.24}{41.615}{-24.39}{0}
\end{erdiagram}

}
\end{frame}

\begin{frame}{Croswfoot notation for binary relationships}
\begin{center}
\scalebox{0.9}{
\begin{erdiagram}{1.4}{4.6666}

\eret{0}{-1}{1.333}{-0.4}{0.2}{1}\ertext{0.667}{-0.75}{}{egg}
\eret{3.333}{-1}{4.667}{-0.4}{0.2}{1}\ertext{4}{-0.75}{}{chicken}

% relationship lays
\ertext{3.183}{-1}{r}{lays}\ertext{1.483}{-0.55}{l}{by}\ertext{1.483}{-0.25}{l}{is laid}\errelarm{3.333}{-0.7}{2.333}{-0.7}{0}{0}\errelarm{2.333}{-0.7}{1.333}{-0.7}{1}{0}\ercrowfoot{1.483}{-0.7}{1.333}{-0.55}{1.333}{-0.7}{1.333}{-0.85}{0}
\end{erdiagram}

}
\end{center}
is read as
\begin{center}
\begin{enumerate}
\item Each egg \textit{is laid by} \textbf{exactly one} chicken.
\item Each chicken \textit{lays} \textbf{zero,one or more} eggs.
\end{enumerate}
\end{center}
 \textit{is laid by} therefore represents a partial function with inverse \textit{lays}.
\end{frame}

\begin{frame}{Chicken and Egg}
A further relationship 
\begin{center}
\scalebox{0.9}{
\begin{erdiagram}{1.7999999999999998}{5.066599999999999}

\eret{0.1}{-1.4}{1.433}{-0.4}{0.2}{1}\ertext{0.767}{-0.75}{}{egg}
\eret{3.733}{-1.4}{5.067}{-0.4}{0.2}{1}\ertext{4.4}{-0.75}{}{chicken}

% relationship hatched_from
\ertext{3.583}{-0.55}{r}{from}\ertext{3.583}{-0.25}{r}{hatched}\ertext{1.583}{-0.55}{l}{into}\ertext{1.583}{-0.25}{l}{hatches}\errelarm{3.733}{-0.7}{2.583}{-0.7}{1}{0}\errelarm{2.583}{-0.7}{1.433}{-0.7}{0}{0}
% relationship lays
\ertext{3.583}{-1.3}{r}{lays}\ertext{1.583}{-1.3}{l}{is laid}\ertext{1.583}{-1.6}{l}{by}\errelarm{3.733}{-1}{2.583}{-1}{0}{0}\errelarm{2.583}{-1}{1.433}{-1}{1}{0}\ercrowfoot{1.583}{-1}{1.433}{-0.85}{1.433}{-1}{1.433}{-1.15}{0}
\end{erdiagram}

}
\end{center}
reads
\begin{center}
\begin{enumerate}
\item Each egg \textit{hatches into} \textbf{zero or one} chickens.
\item Each chicken \textit{hatched from} \textbf{exactly one} egg.
\end{enumerate}
\end{center}
 \textit{hatched from} therefore represents an injective function with inverse \textit{hatches into}.
\end{frame}


\begin{frame}{Coproducts and Inheritance in ER}
\begin{itemize}
\pause \item In either formal grammar or in IDL from Carnegie-Melon we may write A ::= A1 \textbar\  A2
\pause \item In an ER model  A is said to generalise A1 and A2, (A1 and A2 are said to inherit from A) and this is represented
(in Barker's book for example) so:
\begin{center}
\scalebox{0.85}{
\begin{erdiagram}{1.45}{4}

\eret{0}{-1.45}{4}{-0}{0.2}{1}\ertext{0.116}{-0.35}{l}{A}
\eret{0.25}{-1.2}{1.75}{-0.6}{0.2}{0}\ertext{1}{-0.95}{}{A1}
\eret{2.25}{-1.2}{3.75}{-0.6}{0.2}{0}\ertext{3}{-0.95}{}{A2}

\end{erdiagram}

}
\end{center}
\pause \item In category theory this situation is represented by a coproductL A = A1 + A2  
\end{itemize}
\end{frame}




\begin{frame}{ER's Binary Relationships as Morphisms}
\footnotesize
\begin{tabular} {l l l l}
cardinality        & inverse cardinality &     &\\
exactly one        & zero, one or many   &     & $f: A \morph B$\\
exactly one        & zero or one         &     & f is injective $f: A \hookrightarrow B$\\
one or many        &                     &     &f is partial $f:A \rightharpoonup B$\\
zero, one or many  &                     &     &\\
zero or one        &                     &     &\\
\end{tabular}
\end{frame}

\iffalse
\begin{frame}{\textit{a priori}s}
\begin{itemize}
\item In language theory, formal grammars have terminals (and non-terminals)
\pause \item In data specifications, we have use of basic types string, integer, float, boolean and so on
              in addition maybe define enumerations 
\pause \item in categorical data types we have \textit{a priori}s i.e coproducts of terminal object
\pause \item In relational data models we have domains
\pause \item in ER models many-one relationships to the basic types are called attributes and are graphically distinct from other relationships
\end{itemize}
\end{frame}
\fi

\begin{frame}{Enumerated Types}
\pause for example
\begin{itemize} 
\pause \item pascal: type Boolean=(True,False)
\pause \item C-M IDL: boolean ::= true | false
\pause \item In an ER+ model:
\scalebox{0.85}{
\begin{erdiagram}{2.45}{3.8166}

\eret{0.1}{-2.45}{3.817}{-1}{0.2}{1}\ertext{0.216}{-1.35}{l}{boolean}
\eret{0.4}{-2.2}{1.733}{-1.6}{0.2}{0}\ertext{1.067}{-1.95}{}{true}
\eret{2.233}{-2.2}{3.567}{-1.6}{0.2}{0}\ertext{2.9}{-1.95}{}{false}
\eret{0}{-0.2}{3.817}{0.3}{0.2}{1}

% relationship 
\ertext{1.217}{-0.5}{l}{}\errelarm{1.067}{-0.2}{1.067}{-0.9}{1}{0}\errelarm{1.067}{-0.9}{1.067}{-1.6}{1}{0}
% relationship 
\ertext{3.05}{-0.5}{l}{}\errelarm{2.9}{-0.2}{2.9}{-0.9}{1}{0}\errelarm{2.9}{-0.9}{2.9}{-1.6}{1}{0}
\end{erdiagram}

}
\end{itemize}
\end{frame}

\begin{frame}{Types and relationships}
The following ER diagram:
\begin{center}
\scalebox{0.9}{
\begin{erdiagram}{1.4}{4.6666}

\eret{0}{-1}{1.333}{-0.4}{0.2}{1}\ertext{0.667}{-0.75}{}{egg}
\eret{3.333}{-1}{4.667}{-0.4}{0.2}{1}\ertext{4}{-0.75}{}{chicken}

% relationship lays
\ertext{3.183}{-1}{r}{lays}\ertext{1.483}{-0.55}{l}{by}\ertext{1.483}{-0.25}{l}{is laid}\errelarm{3.333}{-0.7}{2.333}{-0.7}{0}{0}\errelarm{2.333}{-0.7}{1.333}{-0.7}{1}{0}\ercrowfoot{1.483}{-0.7}{1.333}{-0.55}{1.333}{-0.7}{1.333}{-0.85}{0}
\end{erdiagram}

}
\end{center}
\begin{center}
\begin{enumerate}
\item Can be considered a data specification.
\item Is \textbf{not} a database specification. 
\end{enumerate}
\end{center}
Note: This is the arrow category -- morphisms interpreted by partial functions. 
\end{frame}

\begin{frame}{Identifying Features in Database Specifications}
\begin{itemize}
\item database specifications are data specifications in which types of entity have 
identifying features
\item combination of relationships which can identify an entity unquely
i.e a mono source
\item achieves principle of identity of indiscernibles
\end {itemize}
\end{frame}

\begin{frame}{bar notation}
\begin{itemize}
\item bar across a relationship indicates that it is one of the identifying features
\item if departments identified by department code
\item and employee identified by employee number within department 
\end{itemize}
\end{frame}

\iffalse
\begin{frame}
\begin{center}{ER modelling notation}
(departmentEmployeeComposite here)
\scalebox{0.50}{
\input{departmentEmployeeComposite}
}
\end{center}
\end{frame}
\fi


\begin{frame}{ER modelling notation}
\begin{center}
\scalebox{0.85}{
\begin{erdiagram}{12.2}{8.7}

\eret{1.7}{-3}{3.7}{-1}{0.2}{1}\ertext{1.9}{-1.35}{l}{department}
\erattr{1.9}{-1.55}{1}{0}{deptCode}
\erattr{1.9}{-1.85}{1}{1}{name}
\eret{5.7}{-2}{7.7}{-1}{0.2}{1}\ertext{5.9}{-1.35}{l}{location}
\erattr{5.9}{-1.55}{1}{0}{name}
\eret{1.45}{-6.9}{5.95}{-3.9}{0.2}{1}\ertext{1.9}{-4.25}{l}{employee}
\erattr{1.65}{-4.45}{1}{0}{empCode}
\erattr{1.65}{-4.75}{1}{1}{name}
\erattr{1.65}{-5.05}{1}{1}{startDate}
\eret{0.868}{-12.2}{2.702}{-8.65}{0.2}{1}\ertext{1.015}{-9}{l}{telephone}
\erattr{1.068}{-9.2}{1}{0}{number}
\eret{1.118}{-10.15}{2.452}{-9.55}{0.2}{0}\ertext{1.785}{-9.9}{}{home}
\eret{1.118}{-11.05}{2.452}{-10.45}{0.2}{0}\ertext{1.785}{-10.8}{}{work}
\eret{1.118}{-11.95}{2.452}{-11.35}{0.2}{0}\ertext{1.785}{-11.7}{}{mobile}
\eret{3.552}{-12.15}{5.532}{-10.15}{0.2}{1}\ertext{3.75}{-10.5}{l}{package}
\erattr{3.752}{-10.7}{1}{0}{from}
\erattr{3.752}{-11}{0}{1}{to}
\erattr{3.752}{-11.3}{1}{1}{annualSalary}
\eret{6.132}{-9.6}{7.382}{-8.3}{0.2}{1}\ertext{6.257}{-8.65}{l}{review}
\erattr{6.332}{-8.85}{1}{0}{on}
\eret{0}{-0.2}{8.7}{0.3}{0.2}{1}

% relationship 
\ertext{2.85}{-0.5}{l}{}\errelarm{2.7}{-0.2}{2.7}{-0.6}{1}{0}\errelarm{2.7}{-0.6}{2.7}{-1}{1}{0}\ercrowfoot{2.7}{-0.85}{2.55}{-1}{2.7}{-1}{2.85}{-1}{0}
% relationship 
\ertext{6.85}{-0.5}{l}{}\errelarm{6.7}{-0.2}{6.7}{-0.6}{1}{0}\errelarm{6.7}{-0.6}{6.7}{-1}{1}{0}\ercrowfoot{6.7}{-0.85}{6.55}{-1}{6.7}{-1}{6.85}{-1}{0}
% relationship 
\ertext{2.85}{-3.3}{l}{}\errelarm{2.7}{-3}{2.7}{-3.075}{1}{0}\errelarm{3.7}{-3.688}{3.7}{-3.9}{1}{0}\errelangle{2.7}{-3.075}{2.7}{-3.15}{3.2}{-3.313}{1}{0}\errelangle{3.2}{-3.313}{3.7}{-3.475}{3.7}{-3.688}{1}{0}\eridcomprel{3.6}{3.8000000000000003}{-3.65}\ercrowfoot{3.7}{-3.75}{3.55}{-3.9}{3.7}{-3.9}{3.85}{-3.9}{0}
% relationship location
\ertext{3.85}{-1.35}{l}{location}\errelarm{3.7}{-1.5}{4.7}{-1.5}{1}{0}\errelarm{4.7}{-1.5}{5.7}{-1.5}{0}{0}\ercrowfoot{3.85}{-1.5}{3.7}{-1.35}{3.7}{-1.5}{3.7}{-1.65}{0}
% relationship headOfDept
\ertext{1.55}{-2.183}{r}{headOfDept}\ertext{1.3}{-4.5}{r}{isHeadOf}\errelarm{1.7}{-2.333}{1.1}{-2.333}{1}{0}\errelarm{1.25}{-4.65}{1.45}{-4.65}{0}{0}\errelangle{1.1}{-2.333}{0.5}{-2.333}{0.35}{-2.333}{1}{0}\errelangle{1.25}{-4.65}{1.05}{-4.65}{0.625}{-4.65}{0}{0}\errelangle{0.35}{-2.333}{0.2}{-2.333}{0.2}{-3.492}{1}{0}\errelangle{0.2}{-3.492}{0.2}{-4.65}{0.625}{-4.65}{0}{0}
% relationship 
\ertext{1.935}{-7.2}{l}{}\errelarm{1.785}{-6.9}{1.785}{-7.775}{0}{0}\errelarm{1.785}{-7.775}{1.785}{-8.65}{1}{0}\eridcomprel{1.685}{1.8850000000000002}{-8.4}\ercrowfoot{1.785}{-8.5}{1.635}{-8.65}{1.785}{-8.65}{1.935}{-8.65}{0}
% relationship 
\ertext{4.692}{-7.2}{l}{}\errelarm{4.542}{-6.9}{4.542}{-8.525}{1}{0}\errelarm{4.542}{-8.525}{4.542}{-10.15}{1}{0}\eridcomprel{4.44165}{4.641649999999999}{-9.9}\ercrowfoot{4.542}{-10}{4.392}{-10.15}{4.542}{-10.15}{4.692}{-10.15}{0}
% relationship 
\ertext{5.65}{-7.2}{l}{}\errelarm{5.5}{-6.9}{5.5}{-6.975}{0}{0}\errelarm{6.757}{-8.088}{6.757}{-8.3}{1}{0}\errelangle{5.5}{-6.975}{5.5}{-7.05}{6.128}{-7.463}{0}{0}\errelangle{6.128}{-7.463}{6.757}{-7.875}{6.757}{-8.088}{1}{0}\eridcomprel{6.65665}{6.856649999999999}{-8.05}\ercrowfoot{6.757}{-8.15}{6.607}{-8.3}{6.757}{-8.3}{6.907}{-8.3}{0}
% relationship manager
\ertext{1.3}{-5.25}{r}{manager}\ertext{1.3}{-6.45}{r}{managerOf}\errelangle{1.45}{-5.4}{1.45}{-5.4}{0.85}{-5.4}{1}{0}\errelangle{1.45}{-6.15}{1.45}{-6.15}{0.85}{-6.15}{0}{0}\errelangle{0.85}{-5.4}{0.25}{-5.4}{0.25}{-5.775}{1}{0}\errelangle{0.25}{-5.775}{0.25}{-6.15}{0.85}{-6.15}{0}{0}\ercrowfoot{1.3}{-5.4}{1.45}{-5.25}{1.45}{-5.4}{1.45}{-5.55}{0}
% relationship reviewBy
\ertext{7.532}{-8.579}{l}{reviewBy}\errelarm{7.382}{-8.729}{7.457}{-8.729}{1}{0}\errelarm{6.15}{-6.3}{5.95}{-6.3}{0}{0}\errelangle{7.457}{-8.729}{7.532}{-8.729}{8.032}{-8.729}{1}{0}\errelangle{6.15}{-6.3}{6.35}{-6.3}{7.441}{-6.3}{0}{0}\errelangle{8.032}{-8.729}{8.532}{-8.729}{8.532}{-7.515}{1}{0}\errelangle{8.532}{-7.515}{8.532}{-6.3}{7.441}{-6.3}{0}{0}\ercrowfoot{7.532}{-8.729}{7.382}{-8.579}{7.382}{-8.729}{7.382}{-8.879}{0}
\end{erdiagram}

}
\end{center}
\end{frame}


\begin{frame}{Grammar or ER}
\begin{itemize}
\pause \item Based on syntax given by Brinton (Structure of English Sentence)
\end{itemize}
\begin{center}
\scalebox{0.85}{
\begin{erdiagram}{4.5}{8.6}

\eret{0}{-3.3}{8.6}{-0}{0.2}{1}\ertext{0.264}{-0.35}{l}{verb phrase}
\eret{0.25}{-1.2}{2.65}{-0.6}{0.2}{0}\ertext{1.45}{-0.95}{}{intransitive}
\eret{2.85}{-2.55}{8.35}{-0.6}{0.2}{0}\ertext{3.006}{-0.95}{l}{transitive}
\eret{3.1}{-1.8}{5.5}{-1.2}{0.2}{1}\ertext{4.3}{-1.55}{}{mono transitive}
\eret{5.7}{-1.8}{8.1}{-1.2}{0.2}{1}\ertext{6.9}{-1.55}{}{ditransitive}
\eret{0.25}{-4.5}{2.65}{-3.9}{0.2}{1}\ertext{1.45}{-4.25}{}{verb}
\eret{4.1}{-4.5}{7.1}{-3.9}{0.2}{1}\ertext{5.6}{-4.25}{}{noun phrase}

% relationship head
\ertext{1.3}{-3.6}{r}{head}\errelarm{1.45}{-3.3}{1.45}{-3.6}{1}{0}\errelarm{1.45}{-3.6}{1.45}{-3.9}{1}{0}
% relationship direct_object
\ertext{4.7}{-2.85}{r}{direct}\ertext{4.7}{-3.15}{r}{object}\errelarm{4.85}{-2.55}{4.85}{-3.225}{1}{0}\errelarm{4.85}{-3.225}{4.85}{-3.9}{1}{0}
% relationship indirect_object
\ertext{6.57}{-2.1}{l}{indirect}\ertext{6.57}{-2.4}{l}{object}\errelarm{6.42}{-1.8}{6.42}{-2.85}{1}{0}\errelarm{6.42}{-2.85}{6.42}{-3.9}{1}{0}\erarc{4.35}{-3.7}{4.975}{-3.5}{6.225}{-3.5}{6.85}{-3.7}
\end{erdiagram}

}
\end{center}
\begin{itemize}
\pause \item This is a fragment of either or both of a data specification (ER model) and/or a grammar.
\end{itemize}
\end{frame}


\begin{frame}{Chicken and Egg}
The next ER diagram:
\begin{center}
\scalebox{0.9}{
\begin{erdiagram}{1.7999999999999998}{5.066599999999999}

\eret{0.1}{-1.4}{1.433}{-0.4}{0.2}{1}\ertext{0.767}{-0.75}{}{egg}
\eret{3.733}{-1.4}{5.067}{-0.4}{0.2}{1}\ertext{4.4}{-0.75}{}{chicken}

% relationship hatched_from
\ertext{3.583}{-0.55}{r}{from}\ertext{3.583}{-0.25}{r}{hatched}\ertext{1.583}{-0.55}{l}{into}\ertext{1.583}{-0.25}{l}{hatches}\errelarm{3.733}{-0.7}{2.583}{-0.7}{1}{0}\errelarm{2.583}{-0.7}{1.433}{-0.7}{0}{0}
% relationship lays
\ertext{3.583}{-1.3}{r}{lays}\ertext{1.583}{-1.3}{l}{is laid}\ertext{1.583}{-1.6}{l}{by}\errelarm{3.733}{-1}{2.583}{-1}{0}{0}\errelarm{2.583}{-1}{1.433}{-1}{1}{0}\ercrowfoot{1.583}{-1}{1.433}{-0.85}{1.433}{-1}{1.433}{-1.15}{0}
\end{erdiagram}

}
\end{center}
\begin{center}
\begin{enumerate}
\item Has faults as a data specification.
\item Is still not a database specification. 
\end{enumerate}
\end{center}
\end{frame}




\begin{frame}{Relational Model of Data}
\scalebox{0.6}{
\begin{erdiagram}{10.1}{14.9495}

\eret{4.85}{-2.15}{10.35}{-1.25}{0.2}{1}\ertext{5.4}{-1.6}{l}{table}
\erattr{5.05}{-1.8}{1}{0}{name}
\eret{1.338}{-4.65}{4.112}{-3.75}{0.2}{1}\ertext{1.615}{-4.1}{l}{primary key column}
\erattr{1.538}{-4.3}{1}{1}{seq no}
\eret{6.862}{-4.65}{8.662}{-3.75}{0.2}{1}\ertext{7.042}{-4.1}{l}{column}
\erattr{7.062}{-4.3}{1}{0}{name}
\eret{11.662}{-4.65}{13.462}{-3.75}{0.2}{1}\ertext{11.842}{-4.1}{l}{foreign key}
\erattr{11.862}{-4.3}{1}{0}{name}
\eret{11.175}{-6.8}{13.95}{-6.2}{0.2}{1}\ertext{12.562}{-6.55}{}{foreign key column}
\eret{0}{-0.2}{14.95}{0.3}{0.2}{1}

% relationship all tables
\ertext{7.75}{-0.5}{l}{all tables}\errelarm{7.6}{-0.2}{7.6}{-0.725}{1}{0}\errelarm{7.6}{-0.725}{7.6}{-1.25}{1}{0}\ercrowfoot{7.6}{-1.1}{7.45}{-1.25}{7.6}{-1.25}{7.75}{-1.25}{0}
% relationship 
\ertext{6.1}{-2.45}{l}{}\ertext{2.875}{-3.6}{l}{of}\errelarm{5.95}{-2.15}{5.95}{-2.225}{1}{0}\errelarm{2.725}{-3.463}{2.725}{-3.75}{1}{0}\errelangle{5.95}{-2.225}{5.95}{-2.3}{4.337}{-2.738}{1}{0}\errelangle{4.337}{-2.738}{2.725}{-3.175}{2.725}{-3.463}{1}{0}\errelseq{2.785}{-3.225}{2.375}{-3.285}{3.075}{-3.345}{2.665}{-3.405}\eridcomprel{2.6249999999999987}{2.824999999999999}{-3.5}\ercrowfoot{2.725}{-3.6}{2.575}{-3.75}{2.725}{-3.75}{2.875}{-3.75}{0}
% relationship 
\ertext{7.912}{-2.45}{l}{}\ertext{7.912}{-3.6}{l}{of}\errelarm{7.762}{-2.15}{7.762}{-2.95}{1}{0}\errelarm{7.762}{-2.95}{7.762}{-3.75}{1}{0}\eridcomprel{7.66225}{7.8622499999999995}{-3.5}\ercrowfoot{7.762}{-3.6}{7.612}{-3.75}{7.762}{-3.75}{7.912}{-3.75}{0}
% relationship 
\ertext{9.4}{-2.45}{l}{}\ertext{12.712}{-3.6}{l}{of}\errelarm{9.25}{-2.15}{9.25}{-2.225}{1}{0}\errelarm{12.562}{-3.538}{12.562}{-3.75}{1}{0}\errelangle{9.25}{-2.225}{9.25}{-2.3}{10.906}{-2.813}{1}{0}\errelangle{10.906}{-2.813}{12.562}{-3.325}{12.562}{-3.538}{1}{0}\eridcomprel{12.462250000000001}{12.66225}{-3.5}\ercrowfoot{12.562}{-3.6}{12.412}{-3.75}{12.562}{-3.75}{12.712}{-3.75}{0}
% relationship is
\ertext{4.262}{-4}{l}{is}\ertext{5.137}{-4.5}{l}{\textasciitilde /of=of}\errelarm{4.112}{-4.2}{5.487}{-4.2}{1}{0}\errelarm{5.487}{-4.2}{6.862}{-4.2}{0}{0}\eridrefrel{4.3622499999999995}{-4.1000000000000005}{-4.3}
% relationship 
\ertext{12.712}{-4.95}{l}{}\ertext{12.812}{-6.05}{l}{partof}\errelarm{12.562}{-4.65}{12.562}{-5.425}{1}{0}\errelarm{12.562}{-5.425}{12.562}{-6.2}{1}{0}\eridcomprel{12.462250000000001}{12.66225}{-5.95}\ercrowfoot{12.562}{-6.05}{12.412}{-6.2}{12.562}{-6.2}{12.712}{-6.2}{0}
% relationship to
\ertext{13.612}{-4.05}{l}{to}\errelarm{13.462}{-4.2}{14.062}{-4.2}{1}{0}\errelarm{2.475}{-1.849}{4.85}{-1.849}{0}{0}\errelangle{14.062}{-4.2}{14.662}{-4.2}{14.662}{-6.9}{1}{0}\errelangle{2.475}{-1.849}{0.1}{-1.849}{0.1}{-5.725}{0}{0}\ertext{7.031}{-9.9}{l}{\textasciitilde /\textasciicircum =\textasciicircum }\errelangle{14.662}{-6.9}{14.662}{-9.6}{7.381}{-9.6}{1}{0}\errelangle{7.381}{-9.6}{0.1}{-9.6}{0.1}{-5.725}{0}{0}\ercrowfoot{13.612}{-4.2}{13.462}{-4.05}{13.462}{-4.2}{13.462}{-4.35}{0}
% relationship is
\ertext{11.025}{-6.7}{r}{is}\errelarm{11.175}{-6.4}{11.025}{-6.4}{1}{0}\errelarm{8.812}{-4.349}{8.662}{-4.349}{0}{0}\ertext{8.369}{-5.675}{l}{\textasciitilde /of=partof/of}\errelangle{11.025}{-6.4}{10.875}{-6.4}{9.919}{-5.375}{1}{0}\errelangle{9.919}{-5.375}{8.962}{-4.349}{8.812}{-4.349}{0}{0}\ercrowfoot{11.025}{-6.4}{11.175}{-6.25}{11.175}{-6.4}{11.175}{-6.55}{0}
% relationship to
\ertext{14.1}{-6.4}{l}{to}\errelarm{13.95}{-6.6}{14.2}{-6.6}{1}{0}\errelarm{1.088}{-4.349}{1.338}{-4.349}{0}{0}\errelangle{14.2}{-6.6}{14.45}{-6.6}{14.45}{-7.1}{1}{0}\errelangle{1.088}{-4.349}{0.838}{-4.349}{0.838}{-5.975}{0}{0}\ertext{6.794}{-7.9}{l}{\textasciitilde /of=partof/to}\errelangle{14.45}{-7.1}{14.45}{-7.6}{7.644}{-7.6}{1}{0}\errelangle{7.644}{-7.6}{0.838}{-7.6}{0.838}{-5.975}{0}{0}\ercrowfoot{14.1}{-6.6}{13.95}{-6.45}{13.95}{-6.6}{13.95}{-6.75}{0}\eridrefrel{14.1995}{-6.500000000000001}{-6.7}
\end{erdiagram}

}
\end{frame}

\begin{frame}{Relational Model of Data}
\scalebox{0.6}{
\begin{erdiagram}{11.899999999999999}{16.04475}

\eret{4.85}{-2.15}{10.35}{-1.25}{0.2}{1}\ertext{5.4}{-1.6}{l}{table}
\erattr{5.05}{-1.8}{1}{0}{name}
\eret{0.138}{-5.55}{2.912}{-3.75}{0.2}{1}\ertext{0.415}{-4.1}{l}{primary key column}
\erdattr{0.338}{-4.3}{1}{0}{table name(D2)}
\erdattr{0.338}{-4.6}{1}{0}{is name(R1)}
\erattr{0.338}{-4.9}{1}{1}{seq no}
\erattr{0.338}{-5.2}{1}{1}{seqNo}
\eret{5.662}{-5.25}{9.427}{-3.75}{0.2}{1}\ertext{6.039}{-4.1}{l}{column}
\erdattr{5.862}{-4.3}{1}{0}{table name(D3)}
\erattr{5.862}{-4.6}{1}{0}{name}
\erdattr{5.862}{-4.9}{0}{1}{in primary key is name(R2)}
\eret{12.427}{-5.25}{14.662}{-3.75}{0.2}{1}\ertext{12.651}{-4.1}{l}{foreign key}
\erdattr{12.627}{-4.3}{1}{0}{table name(D4)}
\erattr{12.627}{-4.6}{1}{0}{name}
\erdattr{12.627}{-4.9}{1}{1}{to name(R3)}
\eret{12.045}{-8.6}{15.045}{-6.8}{0.2}{1}\ertext{12.495}{-7.15}{l}{foreign key column}
\erdattr{12.245}{-7.35}{1}{0}{table name(D5)}
\erdattr{12.245}{-7.65}{1}{0}{foreign key name(D5)}
\erdattr{12.245}{-7.95}{1}{0}{to is name(R5)}
\erdattr{12.245}{-8.25}{1}{1}{is name(R4)}
\eret{0}{-0.2}{16.045}{0.3}{0.2}{1}

% relationship all tables
\ertext{7.75}{-0.5}{l}{all tables}\errelarm{7.6}{-0.2}{7.6}{-0.725}{1}{0}\errelarm{7.6}{-0.725}{7.6}{-1.25}{1}{0}\ercrowfoot{7.6}{-1.1}{7.45}{-1.25}{7.6}{-1.25}{7.75}{-1.25}{0}
% relationship 
\ertext{6.1}{-2.45}{l}{}\ertext{1.675}{-3.6}{l}{of}\errelarm{5.95}{-2.15}{5.95}{-2.225}{1}{0}\errelarm{1.525}{-3.463}{1.525}{-3.75}{1}{0}\ertext{3.587}{-2.588}{r}{D2}\errelangle{5.95}{-2.225}{5.95}{-2.3}{3.737}{-2.738}{1}{0}\errelangle{3.737}{-2.738}{1.525}{-3.175}{1.525}{-3.463}{1}{0}\errelseq{1.585}{-3.225}{1.175}{-3.285}{1.875}{-3.345}{1.465}{-3.405}\eridcomprel{1.4249999999999996}{1.6249999999999998}{-3.5}\ercrowfoot{1.525}{-3.6}{1.375}{-3.75}{1.525}{-3.75}{1.675}{-3.75}{0}
% relationship 
\ertext{7.695}{-2.45}{l}{}\ertext{7.695}{-3.6}{l}{of}\ertext{7.695}{-2.8}{l}{D3}\errelarm{7.545}{-2.15}{7.545}{-2.95}{1}{0}\errelarm{7.545}{-2.95}{7.545}{-3.75}{1}{0}\eridcomprel{7.444750000000001}{7.64475}{-3.5}\ercrowfoot{7.545}{-3.6}{7.395}{-3.75}{7.545}{-3.75}{7.695}{-3.75}{0}
% relationship 
\ertext{9.4}{-2.45}{l}{}\ertext{13.695}{-3.6}{l}{of}\errelarm{9.25}{-2.15}{9.25}{-2.225}{1}{0}\errelarm{13.545}{-3.538}{13.545}{-3.75}{1}{0}\ertext{11.547}{-2.663}{l}{D4}\errelangle{9.25}{-2.225}{9.25}{-2.3}{11.397}{-2.813}{1}{0}\errelangle{11.397}{-2.813}{13.545}{-3.325}{13.545}{-3.538}{1}{0}\eridcomprel{13.44475}{13.64475}{-3.5}\ercrowfoot{13.545}{-3.6}{13.395}{-3.75}{13.545}{-3.75}{13.695}{-3.75}{0}
% relationship is
\ertext{3.062}{-4.45}{l}{is}\ertext{4.237}{-4.5}{l}{R1}\ertext{3.937}{-4.95}{l}{\textasciitilde /of=of}\errelarm{2.912}{-4.65}{4.287}{-4.65}{1}{0}\errelarm{4.287}{-4.65}{5.662}{-4.65}{0}{0}\eridrefrel{3.16225}{-4.550000000000001}{-4.75}
% relationship 
\ertext{13.695}{-5.55}{l}{}\ertext{13.795}{-6.65}{l}{partof}\ertext{13.695}{-5.875}{l}{D5}\errelarm{13.545}{-5.25}{13.545}{-6.025}{1}{0}\errelarm{13.545}{-6.025}{13.545}{-6.8}{1}{0}\eridcomprel{13.44475}{13.64475}{-6.55}\ercrowfoot{13.545}{-6.65}{13.395}{-6.8}{13.545}{-6.8}{13.695}{-6.8}{0}
% relationship to
\ertext{14.812}{-4.35}{l}{to}\errelarm{14.662}{-4.5}{15.262}{-4.5}{1}{0}\errelarm{2.475}{-1.849}{4.85}{-1.849}{0}{0}\errelangle{15.262}{-4.5}{15.862}{-4.5}{15.862}{-7.2}{1}{0}\errelangle{2.475}{-1.849}{0.1}{-1.849}{0.1}{-5.875}{0}{0}\ertext{7.631}{-9.75}{l}{R3}\ertext{7.631}{-10.2}{l}{\textasciitilde /\textasciicircum =\textasciicircum }\errelangle{15.862}{-7.2}{15.862}{-9.9}{7.981}{-9.9}{1}{0}\errelangle{7.981}{-9.9}{0.1}{-9.9}{0.1}{-5.875}{0}{0}\ercrowfoot{14.812}{-4.5}{14.662}{-4.35}{14.662}{-4.5}{14.662}{-4.65}{0}
% relationship is
\ertext{11.895}{-7.7}{r}{is}\errelarm{12.045}{-7.4}{11.895}{-7.4}{1}{0}\errelarm{9.577}{-4.749}{9.427}{-4.749}{0}{0}\ertext{10.186}{-6.025}{l}{R4}\ertext{9.186}{-6.375}{l}{\textasciitilde /of=partof/of}\errelangle{11.895}{-7.4}{11.745}{-7.4}{10.736}{-6.075}{1}{0}\errelangle{10.736}{-6.075}{9.727}{-4.749}{9.577}{-4.749}{0}{0}\ercrowfoot{11.895}{-7.4}{12.045}{-7.25}{12.045}{-7.4}{12.045}{-7.55}{0}
% relationship to
\ertext{15.195}{-7.8}{l}{to}\errelarm{15.045}{-8}{15.295}{-8}{1}{0}\errelarm{-0.112}{-4.949}{0.138}{-4.949}{0}{0}\errelangle{15.295}{-8}{15.545}{-8}{15.545}{-8.5}{1}{0}\errelangle{-0.112}{-4.949}{-0.362}{-4.949}{-0.362}{-6.974}{0}{0}\ertext{7.241}{-8.85}{l}{R5}\ertext{6.741}{-9.3}{l}{\textasciitilde /of=partof/to}\errelangle{15.545}{-8.5}{15.545}{-9}{7.591}{-9}{1}{0}\errelangle{7.591}{-9}{-0.362}{-9}{-0.362}{-6.974}{0}{0}\ercrowfoot{15.195}{-8}{15.045}{-7.85}{15.045}{-8}{15.045}{-8.15}{0}\eridrefrel{15.29475}{-7.9}{-8.1}
\end{erdiagram}

}
\end{frame}


