\documentclass[12pt,a4paper]{article}
\begin{document}

Over years spent programming and designing software and software systems I became aware
that wherever there was data to be modelled -- which was every project that I worked on -- then there was a category that expressed the shape of the data and the paths through the data.  
I also became aware that data was very often represented hierarchically 
-- each piece of data being held in the context of another -- and that the hierarchy that is present in  such is the very same as is present in dependent type theory and in contextual categories.
Having left contextual catgeories behind I found them staring me in the face.
I wrote a paper on this as long ago as 1986 though I now belive that it has major flaws. 

Since then and upto retiring from software development in 2017, a category theoretic mindset has increasingly guided my work modelling data and implementing  systems software. I have written  tools 
encapsulating categorically inspired concepts  to assist in the programming. More recently I have been able to think about  making some of this more mathematically precise and  to give it  theoretical respectability. 

Over the early years I have been programming I sawe the relational model of data emerge 
and then take over as the dominant model of data 
and just recently I saw that relational concepts (table, column, primary key, foreign key) 
have entered the school curriculum at GCSE level.  
As it happens the relational model is not my favourite model of data but it is ubiquitous and so cannot be ignored. Famously the relational model of data is supported by a number of goodness criteria called normal forms. The first three of these were proposed by E.F. Codd the creator of the model. 
Despite Codd's protestations that the model and the normal forms are mathematically inspired I have 
always found them difficult to come to terms with. 
Many papers have been written about the goodness criteria (i.e. the normal forms) 
and about concepts such as that of functional dependency
which are used in their expression. Somehow these concepts have seemed disconnected from my mathematical experience -- which given my backgound in mathematical logic 
and the fact that databases describe the world just as logical theries do -- has seemed odd. 
Recently I have found that a categorical framework brings me to a better understanding of these 
goodness criteria and the first principles that are behind them. 

I don't belive that there is a mathematical theory of data, at least not one that has rigor and generality. 
I believe that there is a need to create one and, wild claim,  I believe that the concepts encapsulated in such a theory should back up into mainstream database theory and practice and, who knows, surplant relational concepts in the school curriculum.

Some of which I have to say has been said before or is a development on what has been said before
most notably by Michael Johnson and coworkers Robert Rosebrugh, C.N.G Dampney and R.J. Wood.

\end{document}



