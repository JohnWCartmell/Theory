\documentclass{beamer}
%\documentclass{scrartcl}
%\usepackage{pstricks}
%\usepackage{pst-node}
%\usepackage{pst-tree}
%\usepackage{stmaryrd}
%\usepackage{amsmath}
%\usepackage{amssymb}
%\usepackage{verbatim}
%\usepackage{enumerate}
%\usepackage{calc}
\usepackage{mathptmx}
\usepackage{amsfonts}
\usepackage{wasysym}
\usepackage{url}
\usepackage{hyperref}
%\usepackage{environ}

%\usepackage{amsthm} % added 7th April 2018
% theorems.macros.tex

\newtheorem{theorem}{Theorem}[section]
\newtheorem{observation}[theorem]{Observation}
\newtheorem{lemma}[theorem]{Lemma}

\newtheorem{proposition}[theorem]{Proposition}
\newtheorem{corollary}[theorem]{Corollary}
\newtheorem{conjecture}[theorem]{Conjecture}
\newtheorem{numbereddefinition}[theorem]{Definition}

\newenvironment{definition}[1][Definition]{\begin{trivlist}
\item[\hskip \labelsep {\bfseries #1}]}{\end{trivlist}}
\newenvironment{examples}[1][Examples]{\begin{trivlist}
\item[\hskip \labelsep {\bfseries #1}]}{\end{trivlist}}
\newenvironment{example}[1][Example]{\begin{trivlist}
\item[\hskip \labelsep {\bfseries #1}]}{\end{trivlist}}
\newenvironment{remark}[1][Remark]{\begin{trivlist}
\item[\hskip \labelsep {\bfseries #1}]}{\end{trivlist}}

\newenvironment{tageqn}[1]
{
\begin{equation}
\stepcounter{equation}
\label{#1}
\tag{\theequation --#1}
}
{
\end{equation}
}

\newenvironment{axiom}[1]
{
\begin{equation}
\label{#1}
\tag{#1}
}
{
\end{equation}
}

% when the tag is required different from the label eg when has math symbols can use:
\newenvironment{axiomtagged}[2]
{
\begin{equation}
\label{#1}
\tag{#2}
}
{
\end{equation}
}

%visible label
\newcommand{\vlabel}[2][]{\label{#2}#1(\textit{#2}):}






%ccategories.macros.tex 

% Macros for diagrams in contextual categories and related categories

\usepackage{twoopt}
\usepackage{scalerel} 
\usepackage{xargs}

%\usepackage{mathabx}  %Caused font problems
%\usepackage{MnSymbol}  % caused font problems

\newcommand{\conu}
{\mathbf{C}(U)}

\newcommand{\depu}
{\mathbf{D}(U)}

\newcommand{\cat}[1]{\textbf{#1}}
\newcommand{\obj}[1]{\ensuremath{|\cat{#1}|}}
\newcommand{\ccat}[1][C]{\ensuremath{\mathbb{#1}} }
\newcommand{\ccatc}{contextual category \ccat}
\newcommand{\cobj}[2][]{\ensuremath{|\ccat[#2]|_{#1}}}
\newcommand{\cslice}[2]{\ensuremath{\ccat[#1]_{#2}}}
\newcommand{\csliceobj}[3][]{\ensuremath{|\mathbb{#2}_{#3}|_{#1} }}
\newcommand{\varset}[1][]{\ensuremath{V_{#1} }}
\newcommand{\localvarsets}{\ensuremath{\mathcal{V} }}
\newcommand{\Fam}{\ensuremath{\mathbb{F\mathrm{am}} }}
\newcommand{\Famslice}[1]{\ensuremath{\mathbb{F\mathrm{am}}_{#1} }}
\newcommand{\Famobj}[1][]{\ensuremath{|\mathbb{F\mathrm{am}}|_{#1} }}
\newcommand{\Famsliceobj}[2][]{\ensuremath{|\mathbb{F\mathrm{am}}_{#2}|_{#1} }}
\newcommand{\morph}{\rightarrow}
\newcommand{\epi}{\twoheadrightarrow}
\newcommand{\base}{\triangleleft}
\newcommand{\comp}{\circ}
\newcommand{\cross}{\otimes}
\newcommand{\pc}[2]{d^{#1}_{#2}}
\newcommand{\sub}{^*}
\newcommand{\diag}{\delta}
\newcommand{\pbase}[1]{\tilde{#1}}

\newcommand{\tuple}[1]{\langle#1\rangle}
\newcommand{\ndidly}{\ensuremath{\Join_n}}
\newcommand{\ndidlycospan}{quotiented n-cospan}

\newcommand{\crossx}[3]{#1 \underset{#3}{\cross} #2}
\newcommand{\fibrex}[3]{#1 \underset{#3}{\Join} #2}
\newcommand{\powerset}{\mathcal{P}}
\newcommand{\primeds}[1]{
\ensuremath{\mathcal{P}(#1)} }
\newcommand{\compset}{\ \dot{\circ}\, }

% darrow
%\newcommand{\darrow}{\rightarrowtriangle} %use \smorph instead
\newcommand{\smorph}{\rightarrowtriangle}

 

\newcommand\dhead{\scaleobj{0.6}{\triangleright}}
\newcommand{\dmorph}{\, \mbox{---} \! \cdot \! \raisebox{1.1pt}{\dhead}}

% projection tree
%\newcommand{\proj}[2]{proj_{#2}(#1)}

\newcommand{\proj}[2]{
\ensuremath{\mathcal{P}_{#2}(#1)} }

%pstrick supplements for arrows

\newlength{\arrnodesepA}
\newlength{\arrnodesepB}
\newlength{\arroffsetA}
\newlength{\arroffsetB}

%Modified to 2pt from 0pt on 23 July 2018
\newcommand{\arreset}{
\setlength{\arrnodesepA}{2pt}
\setlength{\arrnodesepB}{2pt}
\setlength{\arroffsetA}{0pt}
\setlength{\arroffsetB}{0pt}
}
\arreset

\newcommand{\ncarr}[3][0]{\ncarc[arcangle=#1,nodesepA=\arrnodesepA,nodesepB=\arrnodesepB,offsetA=\arroffsetA,offsetB=\arroffsetB,arrowsize=5pt,arrowinset=0.7]{->}{#2}{#3}}
\newcommand{\jcbarr}[4][0]{ % ncbarr is defined in some thridy party package so do not use!\emph{}
\ncarr[#1]{#3}{#4}
\nbput[labelsep=2pt]{\footnotesize $#2$}
}

\newcommand{\ncaarr}[4][0]{
\ncarr[#1]{#3}{#4}
\naput[labelsep=2pt]{\footnotesize $#2$}
}

% \alabel{label}[npos][labelsep_pts]
\newcommandx*\alabel[3][2=0.5,3=2,usedefault]{\naput[labelsep=#3pt,npos=#2]{\footnotesize $#1$}}
% \blabel{label}[npos][labelsep_pts]
\newcommandx*\blabel[3][2=0.5,3=2,usedefault]{\nbput[labelsep=#3pt,npos=#2]{\footnotesize $#1$}}

% \idcomp mark an arrow as one component of an identifier
\newcommand{\idcomp}{\ncput[npos=0, nrot=:U]{\psline(0.2,-0.075)(0.2,0.075)}}  %add a bar to a node connection arrow
% pstrick supplements for s-arrows (previous name for d-arrow - should convert}

\newlength{\sarnodesepA}
\newlength{\sarnodesepB}
\newlength{\saroffsetA}
\newlength{\saroffsetB}
\newlength{\sarnodesepAsav}
\newlength{\sarnodesepBsav}

\newcommand{\sarreset}{
\setlength{\sarnodesepA}{0pt}
\setlength{\sarnodesepB}{0pt}
\setlength{\saroffsetA}{0pt}
\setlength{\saroffsetB}{0pt}
}

\sarreset

% sar - S-arrow
\newcommand{\ncsar}[3][0]{
\setlength{\sarnodesepAsav}{\sarnodesepA}
\setlength{\sarnodesepBsav}{\sarnodesepB}
\addtolength{\sarnodesepA}{3pt}
\addtolength{\sarnodesepB}{7pt}
\ncarc[nodesepA=\sarnodesepA,nodesepB=\sarnodesepB,offsetA=\saroffsetA,offsetB=\saroffsetB,arcangle=#1]{-}{#2}{#3}
\ncput[nrot=:R,npos=1]{\pstriangle(0,0)(.2,.2)}
\setlength{\sarnodesepA}{\sarnodesepAsav}
\setlength{\sarnodesepB}{\sarnodesepBsav}
}


% bsar - below labelled S-arrow
\newcommand{\ncbsar}[4][0]{
\ncsar[#1]{#3}{#4}
\nbput[labelsep=2pt]{\footnotesize $#2$}
}
% asar - above labelled S-arrow
\newcommand{\ncasar}[4][0]{
\ncsar[#1]{#3}{#4}
\naput[labelsep=2pt]{\footnotesize $#2$}
}

% cdar - composite dependency arrow
\newcommand{\nccdar}[3][0]{
\setlength{\sarnodesepAsav}{\sarnodesepA}
\setlength{\sarnodesepBsav}{\sarnodesepB}
\addtolength{\sarnodesepA}{3pt}
\addtolength{\sarnodesepB}{11pt}
\ncarc[nodesepA=\sarnodesepA,nodesepB=\sarnodesepB,offsetA=\saroffsetA,offsetB=\saroffsetB,arcangle=#1]{-}{#2}{#3}
\ncput[nrot=:R,npos=1]{\pstriangle(0,0.1)(.2,.2)}
\ncput[nrot=:R,npos=1]{\psdot[dotsize=1pt](-0.0075,0.05)}   %!!
\setlength{\sarnodesepA}{\sarnodesepAsav}
\setlength{\sarnodesepB}{\sarnodesepBsav}
}


% bcdar - below labelled composite dependency arrow
\newcommand{\ncbcdar}[4][0]{
\nccdar[#1]{#3}{#4}
\nbput[labelsep=2pt]{\footnotesize $#2$}
}
% acdar - above labelled composite dependency arrow
\newcommand{\ncacdar}[4][0]{
\nccdar[#1]{#3}{#4}
\naput[labelsep=2pt]{\footnotesize $#2$}
}


% rsar - recursive S-arrow
\newcommand{\ncrsar}[2]{
\setlength{\sarnodesepAsav}{\sarnodesepA}
\setlength{\sarnodesepBsav}{\sarnodesepB}
\addtolength{\sarnodesepA}{3pt}
\addtolength{\sarnodesepB}{7pt}
\ncloop[nodesepA=\sarnodesepA,nodesepB=\sarnodesepB,
        offsetA=\saroffsetA,offsetB=\saroffsetB,
        armA=0.7cm,armB=0.6cm,angleA=90,angleB=-90,loopsize=-1,linearc=0.4
				]{-}{#1}{#2}
\ncput[nrot=:R,npos=5]{\pstriangle(0,0)(.2,.2)}
\setlength{\sarnodesepA}{\sarnodesepAsav}
\setlength{\sarnodesepB}{\sarnodesepBsav}
}

% pstrick supplements for multi-arrows

\newlength{\marnodesepA}
\newlength{\marnodesepB}
\newlength{\maroffsetB}
\newlength{\marnodesepBsav}

\newcommand{\marreset}{
\setlength{\marnodesepA}{0pt}
\setlength{\marnodesepB}{0pt}
\setlength{\maroffsetB}{0pt}
}

\marreset

%ncmarr[#1 arcangle1][#2 arcangle2]{#3 name}{#4 domain1}{#5 domain2}{#6 junction}{#7 codomain}
\newcommandtwoopt{\ncmarr}[6][8][8]{%
\ncarc[nodesepA=\marnodesepA,nodesepB=0,arcangle=#1]{-}{#3}{#5}
\ncarc[nodesepB=0,arcangle=-#1]{-}{#4}{#5}
\ncarc[arcangle=#2,nodesepB=\marnodesepB,offsetB=\maroffsetB]{->}{#5}{#6}
}%


\newcommandtwoopt{\nchmarr}[6][8][8]{%
\ncarc[nodesepA=\marnodesepA,nodesepB=0,arcangle=#1]{-}{#3}{#5}
\ncarc[nodesepB=0,arcangle=#1]{-}{#4}{#5}
\ncarc[arcangle=#2,nodesepB=\marnodesepB,offsetB=\maroffsetB]{->}{#5}{#6}
}%

\newcommandtwoopt{\ncamarr}[7][8][8]{%
\ncmarr[#1][#2]{#4}{#5}{#6}{#7}
\naput[npos=.05]{$#3$}
}%
\newcommandtwoopt{\ncbmarr}[7][8][8]{%
\ncmarr[#1][#2]{#4}{#5}{#6}{#7}
\nbput[npos=.05]{$#3$}
}%

\newcommandtwoopt{\ncbhmarr}[7][8][8]{%
\nchmarr[#1][#2]{#4}{#5}{#6}{#7}
\nbput[npos=.05]{$#3$}
}%

\newcommandtwoopt{\ncmarrr}[7][8][8]{
\ncarc[nodesepB=0,arcangle=#1]{-}{#3}{#6}
\ncline[nodesepB=0]{-}{#4}{#6}
\ncarc[nodesepB=0,arcangle=-#1]{-}{#5}{#6}
\ncarc[nodesepA=0,arcangle=#2]{->}{#6}{#7}
}

\newcommandtwoopt{\ncamarrr}[8][8][8]{
\ncmarrr[#1][#2]{#4}{#5}{#6}{#7}{#8}
\naput[npos=.05]{$#3$}
}
\newcommandtwoopt{\ncbmarrr}[8][8][8]{
\ncmarrr[#1][#2]{#4}{#5}{#6}{#7}{#8}
\nbput[npos=.05]{$#3$}
}

%gats.macros.tex

\usepackage{environ}    % also used in ermacros % here used for \NewEnvrion

\newcommand{\gat}[1][U]{
\ensuremath{\mathcal{#1}}}  % used to hav a space in here
\newcommand{\gatw}[1][U]{\gat[#1]\ }  % use this if need trailing space
\newcommand{\ingat}[1][U]{in \gat[#1]}
\newcommand{\isagat}[1][U]{\gat[#1] is a g.a.t.}
\newcommand{\inagat}{in a g.a.t. }

% macro for a generic theory
%\newcommand{\theory}
%{\textit{U}}

\newcommand{\intheory}
{is a derived rule of \gat[U]}

% Macros for GAT rules

\newcommand{\isT}[1]
{#1\mbox{ is a type}}

\newcommand{\ofT}[2]
{#1 \in #2
}

% Macros for GAT rules   <!-- new old -->
\newcommand{\istype}[1]
{#1\mbox{ is a type}}

\newcommand{\oftype}[2]
{#1 \in #2
}

%\context{x}{\Delta}{n}
\newcommand{\context}[3]
{\ofT{#1_1}{#2_1},... \ofT{#1_{#3}}{#2_{#3}(#1_1,...#1_{#3-1})}
}

%\subcontext{x}{\Delta}{i}{k}
\newcommand{\subcontext}[4]
{\ofT{#1_{#3_1}}{#2_{#3_1}},... \ofT{#1_{#3_#4}}{#2_{#3_#4}(#1_1,...#1_{#3_#4-1})}
}

% #schematic context
\newcommand{\schmcon}[3]
{\ofT{#1_1}{#2_1},... \ofT{#1_{#3}}{#2_{#3}}
}
% abbreviated to
\newcommand{\con}[3]
{\schmcon{#1}{#2}{#3}}

% schematic subcontext
%\subcon{x}{\Delta}{i}{k}
\newcommand{\subcon}[4]
{\ofT{#1_{#3_1}}{#2_{#3_1}},... \ofT{#1_{#3_#4}}{#2_{#3_#4}}
}

% permuted context
%\permcon{x}{\Delta}{n}{\sigma}
\newcommand{\permcon}[4]
{\ofT{#1_{#4(1)}}{#2_{#4(1)}},... \ofT{#1_{#4(#3)}}{#2_{#4(#3)}}
}
% permuted term
%\permterm{t}{n}{\sigma}
\newcommand{\permterm}[3]
{
#1_{#3(1)},...#1_{#3(#2)}
}


% Idioms
\newcommand{\xDelta}[1]{\con{x}{\Delta}{#1}}
\newcommand{\xDeltap}[1]{\con{x}{\Delta'}{#1}}
\newcommand{\xOmega}[1]{\con{x}{\Omega}{#1}}
\newcommand{\xOmegap}[1]{\con{x}{\Omega'}{#1}}
\newcommand{\yOmega}[1]{\con{y}{\Omega}{#1}}
\newcommand{\yOmegap}[1]{\con{y}{\Omega'}{#1}}

\newcommand{\xDeltasigma}[1]{\permcon{x}{\Delta}{#1}{\sigma}}
\newcommand{\xDeltapsigma}[1]{\permcon{x}{\Delta'}{#1}{\sigma}}
\newcommand{\xOmegasigma}[1]{\permcon{x}{\Omega}{#1}{\sigma}}
\newcommand{\xOmegapsigma}[1]{\permcon{x}{\Omega'}{#1}{\sigma}}
\newcommand{\yOmegasigma}[1]{\permcon{y}{\Omega}{#1}{\sigma}}
\newcommand{\yOmegapsigma}[1]{\permcon{y}{\Omega'}{#1}{\sigma}}

\newcommand{\xDeltainvsigma}[1]{\permcon{x}{\Delta}{#1}{\sigma^{-1}}}
\newcommand{\xDeltapinvsigma}[1]{\permcon{x}{\Delta'}{#1}{\sigma^{-1}}}
\newcommand{\xOmegainvsigma}[1]{\permcon{x}{\Omega}{#1}{\sigma^{-1}}}
\newcommand{\xOmegapinvsigma}[1]{\permcon{x}{\Omega'}{#1}{\sigma^{-1}}}
\newcommand{\yOmegainvsigma}[1]{\permcon{y}{\Omega}{#1}{\sigma^{-1}}}
\newcommand{\yOmegapinvsigma}[1]{\permcon{y}{\Omega'}{#1}{\sigma^{-1}}}

%Idioms enclosed as tuples
\newcommand{\encxDelta}[1]{\tuple{\con{x}{\Delta}{#1}}}
\newcommand{\encxDeltap}[1]{\tuple{\con{x}{\Delta'}{#1}}}
\newcommand{\encxOmega}[1]{\tuple{\con{x}{\Omega}{#1}}}
\newcommand{\encxOmegap}[1]{\tuple{\con{x}{\Omega'}{#1}}}
\newcommand{\encyOmega}[1]{\tuple{\con{y}{\Omega}{#1}}}
\newcommand{\encyOmegap}[1]{\tuple{\con{y}{\Omega'}{#1}}}

\newcommand{\encxDeltasigma}[1]{\tuple{\permcon{x}{\Delta}{#1}{\sigma}}}
\newcommand{\encxDeltapsigma}[1]{\tuple{\permcon{x}{\Delta'}{#1}{\sigma}}}
\newcommand{\encxOmegasigma}[1]{\tuple{\permcon{x}{\Omega}{#1}{\sigma}}}
\newcommand{\encxOmegapsigma}[1]{\tuple{\permcon{x}{\Omega'}{#1}{\sigma}}}
\newcommand{\encyOmegasigma}[1]{\tuple{\permcon{y}{\Omega}{#1}{\sigma}}}
\newcommand{\encyOmegapsigma}[1]{\tuple{\permcon{y}{\Omega'}{#1}{\sigma}}}

\newcommand{\encxDeltainvsigma}[1]{\tuple{\permcon{x}{\Delta}{#1}{\sigma^{-1}}}}
\newcommand{\encxDeltapinvsigma}[1]{\tuple{\permcon{x}{\Delta'}{#1}{\sigma^{-1}}}}
\newcommand{\encxOmegainvsigma}[1]{\tuple{\permcon{x}{\Omega}{#1}{\sigma^{-1}}}}
\newcommand{\encxOmegapinvsigma}[1]{\tuple{\permcon{x}{\Omega'}{#1}{\sigma^{-1}}}}
\newcommand{\encyOmegainvsigma}[1]{\tuple{\permcon{y}{\Omega}{#1}{\sigma^{-1}}}}
\newcommand{\encyOmegapinvsigma}[1]{\tuple{\permcon{y}{\Omega'}{#1}{\sigma^{-1}}}}

\newcommand{\tstyle}{\vdash}

% gat macros developed for cwf paper

% Expressing gats
\newenvironment{gatrules}
{
$$
\begin{array}{l l}
}
{
\end{array}
$$
}
\newcommand{\gatintros}
{
\textbf{Symbol} & \textbf{Introductory\ Rule}                      \\}

\newcommand{\gataxioms}
{\textbf{Axioms}\\}
\newcommand{\gatintro}[3]{\ #1 & #2 \tstyle #3 \\}
\newcommand{\gatlocalintro}[3]{\ #1 & #2 \dashv }
\newcommand{\gataxiom}[2]{\multicolumn{2}{l}{\ \ #1\mbox{,  whenever\ } #2} \\}
\newcommand{\noleft}{\left.\kern-\nulldelimiterspace} % so that no space taken by absent left brace


\newcommand{\gatmultiaxiom}[2]
{\multicolumn{2}{l}{
  \noleft
    \begin{array}{l}
		#1
    \end{array} 
  \right\} \mbox{whenever\ } 	#2 
	}\\}
	
	\newcommand{\axid}[1]{\text{#1}.\ }	

%New context sharing macros
\newcommand{\gatintroducing}[1]{
{\arraycolsep=0pt
  \begin{array}{l}
          #1
  \end{array}} &
}

%*********************************
% \begin{\gatgroup}{context}
%    rules
%  \end{\gatgroup}
%*********************************
\NewEnviron{gatgroup}[1]{%
  \noleft
  {\arraycolsep=0pt
   \begin{array}{l}
\BODY
    \end{array} 
   }
   \ \right\} 
	%\mbox{\ whenever\ } 
	#1
	\vspace{0.1cm} 
}
%*********************************

%*********************************
% \begin{\gatgroupnoshared}
%    rule
%  \end{\gatgroupnoshared}
%*********************************
\NewEnviron{gatgroupnoshared}{%
  {\arraycolsep=0pt
   \begin{array}{l}
\BODY
    \end{array} 
   }
   \ 
	\vspace{0.1cm} 
}
%*********************************

% \gatsingular[width]{context}{conclusion}
\newcommand{\gatsingular}[3][4cm]{
\begin{gatgroupnoshared}
\gatleaf[#1]{#2}{#3} 
\end{gatgroupnoshared}
}

%*********************************
% \gatleaf}[width]{context}{assertion}
%*********************************
\newcommand{\gatleaf}[3][4cm]{%
\makebox[#1]{$#3$ \dotfill} \dotfill \  #2
}
%*********************************
%*********************************
% \gatstandalonesingle}{context}{assertion}
%*********************************
\newcommand{\gatstandalonesingle}[2]{%
#2 \makebox[2.5cm]{\dotfill} \  #1
}
%*********************************

% \gataxiomno{axiomno}
\newcommand{\gataxiomno}[1]{\makebox[0.5cm]{} \axid{#1}}


% metagat.macros.tex

%Meta-theories

%\newcommand{\typ}{\triangleright}
\newcommand{\typ}{\nabla}
\newcommand{\trm}{\tau}
\newcommand{\cross}{\otimes}
\newcommand{\sub}{^*}
\newcommand{\diag}{\delta}

\newcommand{\typeseq}[2]
{\ofT{#1_1}{\typ},... \ofT{#1_{#2}}{\typ(#1_{#2-1})}}

\newcommand{\typeseqcont}[3]
{\ofT{#1_1}{\typ({#2})},... \ofT{#1_{#3}}{\typ(#1_{#3-1})}}

\newcommand{\Ob}{Ob}
\newcommand{\obj}{Ob} % <!-- new old --<
\newcommand{\Hom}{Hom}
\newcommand{\objseq}[2]
{\ofT{#1_1}{\obj},... \ofT{#1_{#2}}{\obj(#1_{#2-1})}}


\def\dottededge{\ncline[linestyle=dotted, nodesep=0.3cm]}
\def\noedge{\ncline[linestyle=none]}
\def\thinedge{\ncline[linewidth=0.4pt]}

\newcommand{\member}[1]
{\ncarc[arcangle=-30,nodesepB=0.03]{->}{\pspred}{\pssucc}
\nbput[labelsep=0.1]{#1}}

\newcommand{\loweraccutemember}[1]
{\ncarc[arcangle=-15,nodesepB=0.03]{->}{\pspred}{\pssucc}
\nbput[labelsep=0.05,npos=0.85]{#1}}

\newcommand{\uppermember}[1]
{\ncarc[arcangle=30,nodesepB=0.03]{->}{\pspred}{\pssucc}\naput{#1}}

\newcommand{\upperaccutemember}[1]
{\ncarc[arcangle=10,nodesepB=0.03]{->}{\pspred}{\pssucc}\naput[npos=0.85]{#1}}

% flexbranch 
% #1 node label
% #2 thislevelsep
% #3 next level sep
% #4 variable (eg x)
% #5 index leter (eg n)
% #6 close parenthesis
% #7 continuation branches
\newcommand{\flexbranch}[7]
{
\pstree[thislevelsep=*#2,nodesep=0.05]
		{\Rnode{#1 1}{\Tr{#4_1 #6}}}
	  {\pstree[thislevelsep=#3]  
				   {\Rnode{#1 2}{\Tr[edge=\dottededge]{#4_{#5} #6}}}
					 {#7}
		}
}

\newcommand{\flexbranchplusleaf}[6]
{
\flexbranch{#1}{#2}{#3}{#4} {#5} {#6}
  {
   %\Rnode{#1 3}{\Tr{#4 #6}}
	 \Tr{\Rnode{#1 3}{#4 #6}}
  }
}

\newcommand{\flexbranchplusarc}[7]
{
\flexbranch{#1}{#2}{#3}{#4} {#5} {#6}
  {
   %\Rnode{#1 3}{\Tr{#4 #6}\member{#7}}
	 \Tr{\Rnode{#1 3}{#4 #6}}\member{#7}
  }
}

\newcommand{\flexbranchinitialarc}[9]
{
\pstree[thislevelsep=*#2,nodesep=0.05]
		{\Rnode{#1 1}{\Tr{#4_#8 #6}}#9}
	  {\pstree[thislevelsep=#3]  
				   {\Rnode{#1 2}{\Tr[edge=\dottededge]{#4_{#5} #6}}}
					 {#7}
		}
}

\newcommand{\equality}[2]
{
\ncline [doubleline=true, nodesep=0.2cm]{#1}{#2}
}
\newcommand{\equalityarc}[2]
{
\ncarc [arcangleA=-30, arcangleB=-20, doubleline=true, nodesep=0.1cm]{#1}{#2}
}

\usepackage[margin=4.0cm]{geometry} %was 3cm
\usepackage{mathptmx}
\usepackage{amsfonts}
\usepackage{array}
\usepackage{pstricks}
\usepackage{pst-tree}
\usepackage{pst-plot}
\usepackage{pst-node}
\usepackage{stmaryrd}
\usepackage{amsmath}
\usepackage{verbatim}
\usepackage{graphicx}  
\usepackage{calc}
\usepackage{xifthen}
\usepackage{xcolor}
\usepackage{color}
\usepackage{stringstrings}
%\usepackage[small,bf,margin=3pt,format=hang, labelsep=endash,singlelinecheck=false]{caption} %prevuiously justification=justified
%\usepackage{enumerate}
%\usepackage{enumitem}
\usepackage{enumerate}
\usepackage[shortlabels]{enumitem}
\usepackage{float}
\usepackage[section]{placeins}
%\setlength{\captionmargin}{5pt}
\usepackage{environ}
\usepackage{multirow}
\usepackage{rotating}
\usepackage{longtable}
\usepackage{afterpage}
\usepackage{needspace}


%DEFINE ENVIRONMENT BLOCK
% Riddle
\newsavebox{\riddlebox}

\newenvironment{erexample}
{\newcommand\colboxcolor{F0F0F0}%was F8F8F8
\begin{lrbox}{\riddlebox}
\begin{minipage}{\dimexpr\columnwidth-2\fboxsep\relax} \textbf{} \\ \itshape}
{\end{minipage}\end{lrbox}%
%\begin{center}
\colorbox[HTML]{\colboxcolor}{\usebox{\riddlebox}}
%\end{center}
}

\newenvironment{erbox}
{\newcommand\colboxcolor{F0F0F0}%was F8F8F8
\begin{lrbox}{\riddlebox}%
\begin{minipage}{\dimexpr\columnwidth-2\fboxsep\relax} }
{\end{minipage}\end{lrbox}%
%\begin{center}
\colorbox[HTML]{\colboxcolor}{\usebox{\riddlebox}}
%\end{center}
}

%\begin{erboxedFigure}{#1 FigureParam}{#2 Label}{#3 Caption}
\NewEnviron{erboxedFigure}[3]{%
\begin{figure}[#1]
\begin{erexample}
\begin{center}
\BODY
\end{center}
\vspace{-0.5cm}
\caption{#3}
\label{#2}
\end{erexample}
\end{figure}
}

\newcommand{\erpictureFolder}[0]{../SharedPictures}

\newcommand{\ercenterPicture}[1]{
\begin{center}
\input{\erpictureFolder/#1}
\end{center}
}


\newlength{\erhalfHt}

%\erinlinePicture{#1 pictureFilename}{#2 pictureHeight}
\newcommand{\erinlinePicture}[2]{
\setlength{\erhalfHt}{#2cm * \real{0.5}}
\raisebox{-\erhalfHt}[\erhalfHt + 0.5cm][\erhalfHt + 0.5cm]{
\input{\erpictureFolder/#1}
} 
}

%\erplainFig{#1 pictureFilename}{#2 figureParam}{#3Caption}
\newcommand{\erplainFig}[3]{
\begin{figure}[#2]
\begin{center}
\input{\erpictureFolder/#1}
\end{center}
\caption{#3}
\label{#1}
\end{figure}
}

%\erboxedFigPicture{#1 pictureFilename}{#2 figureParam}{#3Caption}
\newcommand{\erboxedFigPicture}[3]{
\begin{figure}[#2]
\begin{erexample}
\vspace{-0.5cm}
\begin{center}
\input{\erpictureFolder/#1}
\end{center}
\caption{#3}
\label{#1}
\end{erexample}
\end{figure}
}

%\erLeftSideFig{#1 pictureFilename}{#2 figureParam}{#3Caption}
\newcommand{\erLeftSideFig}[3]{
\begin{figure}[#2]
\begin{erexample}
  \begin{minipage}[c]{0.4\textwidth}
    \caption{#3}
    \label{#1}
  \end{minipage}
  \begin{minipage}[c]{0.5\textwidth}
    \input{\erpictureFolder/#1}
  \end{minipage}
\end{erexample}
\end{figure}
}

%\erbulletedFig{#1 pictureFilename}{#2 figureParam}{#3Caption}
\NewEnviron{erbulletedFig}[3]{%
\begin{figure}[#2]
\begin{erexample}
\vspace{-0.5cm}
\begin{center}
$
\begin{array}{c m{0.25cm} | m{6cm}}
\raisebox{-2.0cm}{
\input{\erpictureFolder/#1}}& & \text{\parbox{6cm}{\raggedright{\footnotesize{
\begin{enumerate}[(i)]
\BODY
\end{enumerate}}}}} \\
\end{array}
$
\end{center}
\caption{#3}
\label{#1}
\end{erexample}
\end{figure} 
}


%\begin{erbulletedDimFig}{#1 pictureFilename}{#2figureParam} {#3Caption} {#4PictureHeight}{#5TextWidth}

\NewEnviron{erbulletedDimFig}[5]{%
\begin{figure}[#2]
\begin{erexample}
\vspace{-0.5cm}
\begin{center}
$
\begin{array}{c m{0.25cm} |  m{#5cm}}
\setlength{\erhalfHt}{#4cm * \real{0.5}}
\raisebox{-\erhalfHt}{
\input{\erpictureFolder/#1}}& & \text{\parbox{#5cm}{\raggedright{\footnotesize{
\begin{enumerate}[(i)]
\BODY
\end{enumerate}}}}} \\
\end{array}
$
\end{center}
\caption{#3}
\label{#1}
\end{erexample}
\end{figure} 
}

%\begin{ernotedModel}{#1 pictureFilename}{#2PictureHeight}{#3PictureWidth}{#4TextWidth}

\NewEnviron{ernotedModel}[4]{%
\begin{center}
$
\begin{array}{m{#3cm} m{1cm} | c m{#4cm}}
\setlength{\erhalfHt}{#2cm * \real{0.5}}
\raisebox{-\erhalfHt}{
\input{\erpictureFolder/#1}}& & & \text{\parbox{#4cm}{\raggedright{\footnotesize{
\BODY
}}}} \\
\end{array}
$
\end{center} 
}

%\begin{ermodelText}{#1 pictureFilename}{#2PictureHeight}{#3PictureWidth}{#4TextWidth}

\NewEnviron{ermodelText}[4]{%
\begin{center}
\begin{tabular}{m{#3cm} m{1cm}  c m{#4cm}}
\setlength{\erhalfHt}{#2cm * \real{0.5}}
\raisebox{-\erhalfHt}{
\input{\erpictureFolder/#1}}& & & \text{\parbox{#4cm}{\raggedright{\small{
\BODY
}}}} \\
\end{tabular}
\end{center} 
}


%\erbulletedModel{#1 pictureFilename}{#2PictureHeight}{#3PictureWidth}{#4TextWidth}

\NewEnviron{erbulletedModel}[4]{%
\begin{center}
$
\begin{array}{m{#3cm} m{1cm} | c m{#4cm}}
\setlength{\erhalfHt}{2cm * \real{0.5}}
\raisebox{-\erhalfHt}{
\input{\erpictureFolder/#1}}& & & \text{\parbox{#4cm}{\raggedright{\footnotesize{
\begin{enumerate}[(i)]
\BODY
\end{enumerate}}}}} \\
\end{array}
$
\end{center} 
}



%\ernotedDimFig{#1 pictureFilename}{#2 figureParam}{#3Caption}{#4PictureHeight}{#5TextWidth}
\NewEnviron{ernotedDimFig}[5]{%
\begin{figure}[#2]
\begin{erexample}
\vspace{-0.5cm}
\begin{center}
$
\begin{array}{c m{0.25cm} | c m{#5cm}}
\setlength{\erhalfHt}{#4cm * \real{0.5}}
\raisebox{-\erhalfHt}{
\input{\erpictureFolder/#1}}& & & \text{\parbox{#5cm}{\raggedright{\footnotesize{
\BODY }}}}\\
\end{array}
$
\end{center}
\caption{#3}
\label{#1}
\end{erexample}
\end{figure} 
}
%\begin{ernotedDimFigPW}{#1 pictureFilename}{#2 figureParam}{#3Caption}{#4PictureHeight}{#5PictureWidth}{#6TextWidth}
\NewEnviron{ernotedDimFigPW}[6]{%
\begin{figure}[#2]
\begin{erexample}
\vspace{-0.5cm}
\begin{center}
$
\begin{array}{>{\centering}m{#5cm} m{0.5cm} | c m{#6cm}}
\setlength{\erhalfHt}{#4cm * \real{0.5}}
\raisebox{-\erhalfHt}{
\centering \input{\erpictureFolder/#1}
}& & & \text{\parbox{#6cm - 0.5cm}{\raggedright{\footnotesize{
\BODY }}}}\\
\end{array}
$ \\
\vspace {0.2cm}
\end{center}
\caption{#3}
\label{#1}
\end{erexample}
\end{figure}
}



\newenvironment{erquote}
{\begin{quote}\itshape}
{\end{quote}}


%
%  erdiag
%
  
%\begin{erdiagram}{#1 height}{#2 width} 
% ....
% ....
%\end{erdiagram}
\newenvironment{erdiagram}[2]
{%\pspicture*(-#1,0)(#2,0)
\pspicture*(0,-#1)(#2,0)
%\psgrid
}
{\endpspicture}

\definecolor{lightyellow}{cmyk}{0,0,0.3,0}

% \eret{#1 x0} {#2 y0} {#3 x1} {#4 y1} {#5 corner radius} {#6 fill}
\newcommand {\eret}[6]
{ 
\ifthenelse{\equal{#6}{1}}
{\psframe[framearc=#5,fillstyle=solid,fillcolor=lightyellow](#1,#2)(#3,#4)}
{\psframe[framearc=#5,fillstyle=solid,fillcolor=white](#1,#2)(#3,#4)}
}

% et top 
\newcommand {\erettop}[4]
{
%\psframe[linestyle=none,linearc=2pt,cornersize=absolute,fillstyle=solid,fillcolor=lightyellow](#1,#2)(#3,#4)
\psline[linearc=2pt,fillstyle=none,fillcolor=lightyellow](#1,#4)(#1,#2)(#3,#2)(#3,#4)
}

% et bottom 
\newcommand {\eretbtm}[4]
{
%\psframe[linestyle=none,linearc=2pt,cornersize=absolute,fillstyle=solid,fillcolor=lightyellow](#1,#2)(#3,#4)
\psline[linearc=2pt,fillstyle=none,fillcolor=lightyellow](#1,#2)(#1,#4)(#3,#4)(#3,#2)
}

% et bottom left
\newcommand {\eretbl}[4]
{
%\psframe[linestyle=none,linearc=2pt,cornersize=absolute,fillstyle=solid,fillcolor=lightyellow](#1,#2)(#3,#4)
\psline[linearc=2pt,fillstyle=none,fillcolor=lightyellow](#1,#4)(#3,#4)(#3,#2)
}

% et middle left
\newcommand {\eretml}[4]
{
%\psframe[linestyle=none,linearc=2pt,cornersize=absolute,fillstyle=solid,fillcolor=lightyellow](#1,#2)(#3,#4)
\psline[linearc=2pt,fillstyle=none,fillcolor=lightyellow](#1,#2)(#3,#2)(#3,#4)(#1,#4)
}

% et top left
\newcommand {\erettl}[4]
{
%\psframe[linestyle=none,linearc=2pt,cornersize=absolute,fillstyle=solid,fillcolor=lightyellow](#1,#2)(#3,#4)
\psline[linearc=2pt,fillstyle=none,fillcolor=lightyellow](#1,#2)(#3,#2)(#3,#4)
}

% et bottom right
\newcommand {\eretbr}[4]
{
%\psframe[linestyle=none,linearc=2pt,cornersize=absolute,fillstyle=solid,fillcolor=lightyellow](#1,#2)(#3,#4)
\psline[linearc=2pt,fillstyle=none,fillcolor=lightyellow](#1,#2)(#1,#4)(#3,#4)
}

% et middle right
\newcommand {\eretmr}[4]
{
%\psframe[linestyle=none,linearc=2pt,cornersize=absolute,fillstyle=solid,fillcolor=lightyellow](#1,#2)(#3,#4)
\psline[linearc=2pt,fillstyle=none,fillcolor=lightyellow](#3,#4)(#1,#4)(#1,#2)(#3,#2)
}

% et top right
\newcommand {\erettr}[4]
{
%\psframe[linestyle=none,linearc=2pt,cornersize=absolute,fillstyle=solid,fillcolor=lightyellow](#1,#2)(#3,#4)
\psline[linearc=2pt,fillstyle=none,fillcolor=lightyellow](#1,#4)(#1,#2)(#3,#2)
}

% \ergrp{#1 x0} {#2 y0} {#3 x1} {#4 y1} {#5 corner radius} {#6 fill}
% #5 corner radius is unused!
\newcommand {\ergrp}[6]
{ 
\ifthenelse{\equal{#6}{1}}
{\psframe[fillstyle=solid,fillcolor=lightgray](#1,#2)(#3,#4)}
{\psframe[fillstyle=solid,fillcolor=white](#1,#2)(#3,#4)}
}

% \eretname {#1 x left of text} {#2 y top of text} {#3 text}
\newcommand {\eretname}[3]
{
%shift down 0.1 for height of text the anchor at baseline (B)
\rput[bl]{0}(0,-0.1){\rput[Bl]{0}(#1,#2){\footnotesize \textit{#3}}}
}

% \errelarm {#1 x0} {#2 y0} {#3 x1} {#4 y1} {#5 ismandatory} {#6 isconstructed}
\newcommand {\errelarm}[6]
{
\ifthenelse{\equal{#6}{1}}
{
%%\psline[linewidth=0.5pt,linearc=.05,linestyle=dashed,dash=6pt 6pt]{-}(#1,#2)(#3,#4)}
\ifthenelse{\equal{#5}{1}}
{\psline[linewidth=1.5pt,linearc=.05,linecolor=lightgray]{-}(#1,#2)(#3,#4)}
{\psline[linewidth=1.5pt,linearc=.05,linecolor=lightgray,linestyle=dashed,dash=2pt 2pt]{-}(#1,#2)(#3,#4)}
}
{
\ifthenelse{\equal{#5}{1}}
{\psline[linewidth=0.9pt,linearc=.05]{-}(#1,#2)(#3,#4)}
{\psline[linewidth=0.9pt,linearc=.05,linestyle=dashed,dash=2pt 2pt]{-}(#1,#2)(#3,#4)}
}
}

% \errelangle {#1 x0} {#2 y0} {#3 x1} {#4 y1} {#5 x2} {#6 y2} {#7 ismandatory} {#8 isocnstructed}
\newcommand {\errelangle}[8]
{
\ifthenelse{\equal{#8}{1}}
{
%\psline[linewidth=0.5pt,linearc=.1,linestyle=dashed,dash=6pt 6pt]{-}(#1,#2)(#3,#4)(#5,#6)}
\ifthenelse{\equal{#7}{1}}
{\psline[linewidth=1.5pt,linearc=.05,linecolor=lightgray]{-}(#1,#2)(#3,#4)(#5,#6)}
{\psline[linewidth=1.5pt,linearc=.1,linecolor=lightgray,linestyle=dashed,dash=2pt 2pt]{-}(#1,#2)(#3,#4)(#5,#6)}
}
{
\ifthenelse{\equal{#7}{1}}
{\psline[linewidth=0.9pt,linearc=.1]{-}(#1,#2)(#3,#4)(#5,#6)}
{\psline[linewidth=0.9pt,linearc=.1,linestyle=dashed,dash=2pt 2pt]{-}(#1,#2)(#3,#4)(#5,#6)}
}
}

% \ercrowfoot {#1 x0} {#2 y0} {#3 x11} {#4 y11} {#5 x12} {#6 y12} {#7 x13} {#8 y13} {#9 isconstructed}
\newcommand {\ercrowfoot}[9]
{
\ifthenelse{\equal{#9}{1}}
{
\psline[linewidth=1.5pt,linearc=.05,linecolor=lightgray]{-}(#1,#2)(#3,#4)
\psline[linewidth=1.5pt,linearc=.05,linecolor=lightgray]{-}(#1,#2)(#5,#6)
\psline[linewidth=1.5pt,linearc=.05,linecolor=lightgray]{-}(#1,#2)(#7,#8)
}{
\psline[linewidth=0.9pt,linearc=.05]{-}(#1,#2)(#3,#4)
\psline[linewidth=0.9pt,linearc=.05]{-}(#1,#2)(#5,#6)
\psline[linewidth=0.9pt,linearc=.05]{-}(#1,#2)(#7,#8)
}
}


% \eridcomprel{#1 x1}{#2 x2}{#3 y1}{#4 ymid}{#5 y2}
\newcommand {\eridcomprel}[5]
{
\psline[linewidth=0.9pt](#1,#3)(#1,#5)
\psline[linewidth=0.9pt](#2,#3)(#2,#5)
\psline[linewidth=0.9pt](#1,#4)(#2,#4)
}

% \eridrefrel{#1 x1}{#2 xmid}{#3 x2}{#4 y1}{#5 y2}
\newcommand {\eridrefrel}[5]
{
\psline[linewidth=0.9pt](#1,#4)(#3,#4)
\psline[linewidth=0.9pt](#1,#5)(#3,#5)
\psline[linewidth=0.9pt](#2,#4)(#2,#5)
}


% \errelname {#1 x} {#2 y} {#3 text}
\newcommand {\errelname}[3]
{
\rput[l]{0}(#1,#2){\textit{#3}}
}
% \errelseq {#1 x} {#2 y}
\newcommand {\erelseq}[2]
{
}
% \erattr {#1 x} {#2 y} {#3 ismandatory}{#4 idenitfying} {#5 text}
\newcommand {\erattr}[5]
{
\ifthenelse{\equal{#3}{1}}
{\rput[l]{0}(#1,#2){{\tiny $\square$} {\footnotesize \textit{\ifthenelse{\equal{#4}{0}}{\underline{#5}}{#5}}}}}
{\rput[l]{0}(#1,#2){\footnotesize $\circ$ \textit{\ifthenelse{\equal{#4}{0}}{\underline{#5}}{#5}}}}
}

%\ifthenelse{\equal{#4}{1}}
% \ertext {#1 x} {#2 y} {#3 text anchor} {#4 text}
%{\rput[l]{0}(#1,#2){\footnotesize $\circ$ \underline{\textit{#5}}}}
\newcommand {\ertext}[4]
{
\rput[B#3]{0}(#1,#2){{\footnotesize #4}}
}
% \erarc {#1 x0} {#2 y0} {#3 x1} {#4 y1} {#5 x2} {#6 y2} {#7 x3} {#8 y3}
\newcommand {\erarc}[8]
{
\psbezier[showpoints=false]{-}(#1,#2) (#3, #4)(#5,#6) (#7, #8)
}

% \erarc {#1 x0} {#2 y0} {#3 x1} {#4 y1} {#5 x2} {#6 y2} {#7 x3} {#8 y3}
\newcommand {\errelseq}[8]
{
\psbezier[showpoints=false]{-}(#1,#2) (#3, #4)(#5,#6) (#7, #8)
}
% \ertrace {#1 trace}   
\newcommand {\ertrace}[1]
{
}


%indexedsets.macros.tex

% Macros for sets and families of sets
\newlength{\xl}
\newlength{\yb}
\newlength{\xr}
\newlength{\yt}
\newlength{\ytm}
\newlength{\ybm}
\newlength{\dotxl}
\newlength{\dotxr}
\newlength{\dotym}
\newlength{\basex} 
\newlength{\basey} 
\newlength{\childx} 
\newlength{\childy}
\newcommand{\putthreeset}[5][0]{
  \setlength{\xl}{-1.6cm * \real{#2}}
  \setlength{\xr}{1.8cm * \real{#2}}
  \setlength{\yt}{0.55cm * \real{#2}}
  \setlength{\ytm}{0.75cm * \real{#2}}
  \setlength{\yb}{-0.55cm * \real{#2}}
  \setlength{\ybm}{-0.80cm * \real{#2}}
  \setlength{\dotxl}{-1cm * \real{#2}}
  \setlength{\dotxr}{0.9cm * \real{#2}}
  \setlength{\dotym}{0.15cm * \real{#2}}
  %
  \rput{#1}(#3,#4){        
           {\psccurve%[showpoints=true]
                     (\xl ,\yt)(\xl,\yb)(0,\ybm )(\xr,\yb)(\xr,\yt) (0,\ytm)  }
            \dotnode[dotscale=0.4](\dotxl,0){#5l}
            \dotnode[dotscale=0.4](0,\dotym){#5m}
            \dotnode[dotscale=0.4](\dotxr,0){#5r}
            \pnode(0,\ybm){#5c}
           }
}
\newcommand{\puttwoset}[5][0]{
  \setlength{\xl}{-1.0cm * \real{#2}}
  \setlength{\xr}{1cm * \real{#2}}
  \setlength{\yt}{0.55cm * \real{#2}}
  \setlength{\ytm}{0.75cm * \real{#2}}
  \setlength{\yb}{-0.55cm * \real{#2}}
  \setlength{\ybm}{-0.80cm * \real{#2}}
  \setlength{\dotxl}{-0.75cm * \real{#2}}
  \setlength{\dotxr}{0.25cm * \real{#2}}
  %
  \rput{#1}(#3,#4){        
           {\psccurve%[showpoints=true]
                     (\xl ,\yt)(\xl,\yb)(0,\ybm )(\xr,\yb)(\xr,\yt) (0,\ytm)  }
            
            \dotnode[dotscale=0.4](\dotxl,0){#5l} 
            \dotnode[dotscale=0.4](\dotxr,0){#5r}
            \pnode(0,\ybm){#5c}
           }
}

%\putfamilyOfSets[#1 rotation]{#2 basescale}{#3 childscale}{#4 x}{#5 y}{#6 childoffset}{#7nodeprefix}
\newcommand{\putfamilyOfSets}[7][0]{
  \setlength{\basex}{#4}
  \setlength{\basey}{#5}
  \putthreeset[#1]{#2}{\basex}{\basey}{#7BASE} 
  %child 1
  \setlength{\childx} {#4 - (4cm * \real{#3})}
  \setlength{\childy} {#5 + #6}
  \putthreeset[#1]{#3}{\childx}{\childy}{L}
  %child 2
  \setlength{\childy}{\childy + 0.5cm}
  \putthreeset[#1]{#3}{#4}{\childy}{M}
  %child 3
  \setlength{\childx} {#4 + (4cm * \real{#3})}
  \setlength{\childy}{\childy - 0.5cm}
  \putthreeset[#1]{#3}{\childx}{\childy}{R}
  \ncline[nodesep=3pt]{|->}{#7BASEl}{Lc}
  \ncline[nodesep=3pt]{|->}{#7BASEm}{Mc}
  \ncline[nodesep=3pt]{|->}{#7BASEr}{Rc}
}

%putFunction[#1 rotation]{#2 basescale}{#3 childscale}{#4 x}{#5 y}{#6 childoffset}{#7nodeprefix}
\newcommand{\putFunction}[7][0]{
  \setlength{\basex}{#4}
  \setlength{\basey}{#5}
  \putthreeset[#1]{#2}{\basex}{\basey}{#7BASE} 
  %child 1
  %\setlength{\childx} {#4 - (4cm * \real{#3})}
	\setlength{\childx} {#4 }
  \setlength{\childy} {#5 + #6}
  \putthreeset[#1]{#3}{\childx}{\childy}{DEST}
  \ncline[nodesep=3pt]{|->}{#7BASEl}{DESTl}
  \ncline[nodesep=3pt]{|->}{#7BASEm}{DESTm}
  \ncline[nodesep=3pt]{|->}{#7BASEr}{DESTm}
}

\renewcommand{\erpictureFolder}[0]{../SharedPictures}

\usetheme{Szeged}
\usecolortheme{dolphin}

%\setbeamertemplate{navigation symbols}{}

\setcounter{equation}{0}


\title[Types in practice and in theory]{Data Specifications, Categories, Theories\\ -- Types in practice and in theory}
%% Which is to say types as they are used in practice in software development and as represented in theory in categories and in syntactic type theories.
%% There is also a subplot concerning representation of context which certain types depend on -- again represented in practice and in theory. 
\author{John Cartmell}
\institute{Otium}
\date{Jan 25, 2019}
\bibliographystyle{plainnat}
\begin{document}

\begin{frame}
\titlepage
\end{frame}

\AtBeginSubsection[]
{
  \begin{frame}<beamer>
    \frametitle{Layout}
    \tableofcontents[currentsection,currentsubsection]
  \end{frame}
} 
\begin{frame}{Table of Contents}
\tableofcontents
\end{frame}

\section{Introduction -- Notions of Type}
\begin{frame}{Introduction}
\begin{itemize}
\item
What types of things are there and how are they related? 
\pause \item Data specifications provide the answer to this question in the context of a software development. 
\pause \item Types theories provide the answer in the context of mathematics. 
\pause \item Category theory abstracts across both these domains.
\end{itemize}

\end{frame}

\begin{frame}{Notions of ...}
\begin{columns}[t]
\column{6.0cm}
\pause \begin{itemize}
\item {... data specification
   \begin{itemize}
	    \item relational data model\Rnode{RDB}{\ }
			\item interface defn. language{\scriptsize(IDL)}$^*$\Rnode{IDL}{}
			\item {\scriptsize XML} schema$^*$\Rnode{XML}{\ }
			\item entity/{\scriptsize ER} model$^*$
			\item extd. {\scriptsize ER} model$^*$\Rnode{ER}{}
	 \end{itemize}
	 }
\end{itemize}
\column{5cm}
\pause \begin{itemize}
	\item {... theory
   \begin{itemize}
	    \item many-sorted algebraic
			\item essentially algebraic
			\item generalised algebraic
			\item intuitionistic logic
			\item Martin-L\"of type theory
	 \end{itemize}
	}
	\end{itemize}
\end{columns}
\begin{columns}[t]
\column{7.0cm}
\pause \begin{itemize}
\item {... structure
   \begin{itemize}
	    \item category
			\item contextual category (c-system)
			\item category with attributes$^*$
			\item lextensive category
			\item topos
	 \end{itemize}
	}
\end{itemize}
\column{4cm}
\onslide<2-4>\raisebox{-2.5cm}{$^*$ \footnotesize \textit{various definitions of these}}
\end{columns}
\onslide<5-7> \ncdiag[linecolor=red,linewidth=2pt,angleA=0, angleB=0, armA=0.5, armB=0.8, linearc=.2]{->}{ER}{XML}
\onslide<6-7> \ncdiag[linecolor=red,linewidth=2pt,angleA=0, angleB=0, armA=2.5, armB=0.5, linearc=.2]{->}{ER}{IDL}
\onslide<7> \ncdiag[linecolor=red,linewidth=2pt,angleA=0, angleB=0, armA=2.8, armB=1.95, linearc=.2]{->}{ER}{RDB}
\end{frame}
\begin{frame}{Introduction}
\begin{itemize}
\pause \item In either formal grammar or in IDL from Carnegie-Melon we may write A ::= A1 \textbar\  A2
\pause \item In category theory this situation is represented by a coproductL A = A1 + A2  
\pause \item In an ER model  A is said to generalise A1 and A2, (A1 and A2 are said to inherit from A) and this is represented
(in Barker's book for example) so:
\end{itemize}
\begin{center}
\scalebox{0.85}{
\begin{erdiagram}{1.45}{4}

\eret{0}{-1.45}{4}{-0}{0.2}{1}\ertext{0.116}{-0.35}{l}{A}
\eret{0.25}{-1.2}{1.75}{-0.6}{0.2}{0}\ertext{1}{-0.95}{}{A1}
\eret{2.25}{-1.2}{3.75}{-0.6}{0.2}{0}\ertext{3}{-0.95}{}{A2}

\end{erdiagram}

}
\end{center}
\end{frame}
\begin{frame}{Grammar or ER}
\begin{itemize}
\pause \item Based on syntax given by Brinton (Structure of English Sentence)
\end{itemize}
\begin{center}
\scalebox{0.85}{
\begin{erdiagram}{4.5}{8.6}

\eret{0}{-3.3}{8.6}{-0}{0.2}{1}\ertext{0.264}{-0.35}{l}{verb phrase}
\eret{0.25}{-1.2}{2.65}{-0.6}{0.2}{0}\ertext{1.45}{-0.95}{}{intransitive}
\eret{2.85}{-2.55}{8.35}{-0.6}{0.2}{0}\ertext{3.006}{-0.95}{l}{transitive}
\eret{3.1}{-1.8}{5.5}{-1.2}{0.2}{1}\ertext{4.3}{-1.55}{}{mono transitive}
\eret{5.7}{-1.8}{8.1}{-1.2}{0.2}{1}\ertext{6.9}{-1.55}{}{ditransitive}
\eret{0.25}{-4.5}{2.65}{-3.9}{0.2}{1}\ertext{1.45}{-4.25}{}{verb}
\eret{4.1}{-4.5}{7.1}{-3.9}{0.2}{1}\ertext{5.6}{-4.25}{}{noun phrase}

% relationship head
\ertext{1.3}{-3.6}{r}{head}\errelarm{1.45}{-3.3}{1.45}{-3.6}{1}{0}\errelarm{1.45}{-3.6}{1.45}{-3.9}{1}{0}
% relationship direct_object
\ertext{4.7}{-2.85}{r}{direct}\ertext{4.7}{-3.15}{r}{object}\errelarm{4.85}{-2.55}{4.85}{-3.225}{1}{0}\errelarm{4.85}{-3.225}{4.85}{-3.9}{1}{0}
% relationship indirect_object
\ertext{6.57}{-2.1}{l}{indirect}\ertext{6.57}{-2.4}{l}{object}\errelarm{6.42}{-1.8}{6.42}{-2.85}{1}{0}\errelarm{6.42}{-2.85}{6.42}{-3.9}{1}{0}\erarc{4.35}{-3.7}{4.975}{-3.5}{6.225}{-3.5}{6.85}{-3.7}
\end{erdiagram}

}
\end{center}
\begin{itemize}
\pause \item This is a fragment of either or both of a data specification (ER model) and/or a grammar.
\end{itemize}
\end{frame}

\begin{frame}{\textit{a priori}s}
\begin{itemize}
\item In language theory, formal grammars have terminals (and non-terminals)
\pause \item In data specifications, we have \textit{a priori}s
\pause \item In relational data models we have domains
\pause in ER models we have attributes 

\end{itemize}

\end{frame}

\section{Part One - Data Specification}


\subsection{Chicken and Egg}
\begin{frame}{Types and relationships}
The following ER diagram:
\begin{center}
\scalebox{0.9}{
\begin{erdiagram}{1.4}{4.6666}

\eret{0}{-1}{1.333}{-0.4}{0.2}{1}\ertext{0.667}{-0.75}{}{egg}
\eret{3.333}{-1}{4.667}{-0.4}{0.2}{1}\ertext{4}{-0.75}{}{chicken}

% relationship lays
\ertext{3.183}{-1}{r}{lays}\ertext{1.483}{-0.55}{l}{by}\ertext{1.483}{-0.25}{l}{is laid}\errelarm{3.333}{-0.7}{2.333}{-0.7}{0}{0}\errelarm{2.333}{-0.7}{1.333}{-0.7}{1}{0}\ercrowfoot{1.483}{-0.7}{1.333}{-0.55}{1.333}{-0.7}{1.333}{-0.85}{0}
\end{erdiagram}

}
\end{center}
\begin{center}
\begin{enumerate}
\item Can be considered a data specification.
\item Is \textbf{not} a database specification. 
\end{enumerate}
\end{center}
Note: This is the arrow category -- morphisms interpreted by partial functions. 
\end{frame}
\begin{frame}{Chicken and Egg}
The next ER diagram:
\begin{center}
\scalebox{0.9}{
\begin{erdiagram}{1.7999999999999998}{5.066599999999999}

\eret{0.1}{-1.4}{1.433}{-0.4}{0.2}{1}\ertext{0.767}{-0.75}{}{egg}
\eret{3.733}{-1.4}{5.067}{-0.4}{0.2}{1}\ertext{4.4}{-0.75}{}{chicken}

% relationship hatched_from
\ertext{3.583}{-0.55}{r}{from}\ertext{3.583}{-0.25}{r}{hatched}\ertext{1.583}{-0.55}{l}{into}\ertext{1.583}{-0.25}{l}{hatches}\errelarm{3.733}{-0.7}{2.583}{-0.7}{1}{0}\errelarm{2.583}{-0.7}{1.433}{-0.7}{0}{0}
% relationship lays
\ertext{3.583}{-1.3}{r}{lays}\ertext{1.583}{-1.3}{l}{is laid}\ertext{1.583}{-1.6}{l}{by}\errelarm{3.733}{-1}{2.583}{-1}{0}{0}\errelarm{2.583}{-1}{1.433}{-1}{1}{0}\ercrowfoot{1.583}{-1}{1.433}{-0.85}{1.433}{-1}{1.433}{-1.15}{0}
\end{erdiagram}

}
\end{center}
\begin{center}
\begin{enumerate}
\item Has faults as a data specification.
\item Is still not a database specification. 
\end{enumerate}
\end{center}
\end{frame}


\subsection{Relational Model of Data}
\begin{frame}{Relational Model of Data}
\scalebox{0.6}{
\begin{erdiagram}{10.1}{14.9495}

\eret{4.85}{-2.15}{10.35}{-1.25}{0.2}{1}\ertext{5.4}{-1.6}{l}{table}
\erattr{5.05}{-1.8}{1}{0}{name}
\eret{1.338}{-4.65}{4.112}{-3.75}{0.2}{1}\ertext{1.615}{-4.1}{l}{primary key column}
\erattr{1.538}{-4.3}{1}{1}{seq no}
\eret{6.862}{-4.65}{8.662}{-3.75}{0.2}{1}\ertext{7.042}{-4.1}{l}{column}
\erattr{7.062}{-4.3}{1}{0}{name}
\eret{11.662}{-4.65}{13.462}{-3.75}{0.2}{1}\ertext{11.842}{-4.1}{l}{foreign key}
\erattr{11.862}{-4.3}{1}{0}{name}
\eret{11.175}{-6.8}{13.95}{-6.2}{0.2}{1}\ertext{12.562}{-6.55}{}{foreign key column}
\eret{0}{-0.2}{14.95}{0.3}{0.2}{1}

% relationship all tables
\ertext{7.75}{-0.5}{l}{all tables}\errelarm{7.6}{-0.2}{7.6}{-0.725}{1}{0}\errelarm{7.6}{-0.725}{7.6}{-1.25}{1}{0}\ercrowfoot{7.6}{-1.1}{7.45}{-1.25}{7.6}{-1.25}{7.75}{-1.25}{0}
% relationship 
\ertext{6.1}{-2.45}{l}{}\ertext{2.875}{-3.6}{l}{of}\errelarm{5.95}{-2.15}{5.95}{-2.225}{1}{0}\errelarm{2.725}{-3.463}{2.725}{-3.75}{1}{0}\errelangle{5.95}{-2.225}{5.95}{-2.3}{4.337}{-2.738}{1}{0}\errelangle{4.337}{-2.738}{2.725}{-3.175}{2.725}{-3.463}{1}{0}\errelseq{2.785}{-3.225}{2.375}{-3.285}{3.075}{-3.345}{2.665}{-3.405}\eridcomprel{2.6249999999999987}{2.824999999999999}{-3.5}\ercrowfoot{2.725}{-3.6}{2.575}{-3.75}{2.725}{-3.75}{2.875}{-3.75}{0}
% relationship 
\ertext{7.912}{-2.45}{l}{}\ertext{7.912}{-3.6}{l}{of}\errelarm{7.762}{-2.15}{7.762}{-2.95}{1}{0}\errelarm{7.762}{-2.95}{7.762}{-3.75}{1}{0}\eridcomprel{7.66225}{7.8622499999999995}{-3.5}\ercrowfoot{7.762}{-3.6}{7.612}{-3.75}{7.762}{-3.75}{7.912}{-3.75}{0}
% relationship 
\ertext{9.4}{-2.45}{l}{}\ertext{12.712}{-3.6}{l}{of}\errelarm{9.25}{-2.15}{9.25}{-2.225}{1}{0}\errelarm{12.562}{-3.538}{12.562}{-3.75}{1}{0}\errelangle{9.25}{-2.225}{9.25}{-2.3}{10.906}{-2.813}{1}{0}\errelangle{10.906}{-2.813}{12.562}{-3.325}{12.562}{-3.538}{1}{0}\eridcomprel{12.462250000000001}{12.66225}{-3.5}\ercrowfoot{12.562}{-3.6}{12.412}{-3.75}{12.562}{-3.75}{12.712}{-3.75}{0}
% relationship is
\ertext{4.262}{-4}{l}{is}\ertext{5.137}{-4.5}{l}{\textasciitilde /of=of}\errelarm{4.112}{-4.2}{5.487}{-4.2}{1}{0}\errelarm{5.487}{-4.2}{6.862}{-4.2}{0}{0}\eridrefrel{4.3622499999999995}{-4.1000000000000005}{-4.3}
% relationship 
\ertext{12.712}{-4.95}{l}{}\ertext{12.812}{-6.05}{l}{partof}\errelarm{12.562}{-4.65}{12.562}{-5.425}{1}{0}\errelarm{12.562}{-5.425}{12.562}{-6.2}{1}{0}\eridcomprel{12.462250000000001}{12.66225}{-5.95}\ercrowfoot{12.562}{-6.05}{12.412}{-6.2}{12.562}{-6.2}{12.712}{-6.2}{0}
% relationship to
\ertext{13.612}{-4.05}{l}{to}\errelarm{13.462}{-4.2}{14.062}{-4.2}{1}{0}\errelarm{2.475}{-1.849}{4.85}{-1.849}{0}{0}\errelangle{14.062}{-4.2}{14.662}{-4.2}{14.662}{-6.9}{1}{0}\errelangle{2.475}{-1.849}{0.1}{-1.849}{0.1}{-5.725}{0}{0}\ertext{7.031}{-9.9}{l}{\textasciitilde /\textasciicircum =\textasciicircum }\errelangle{14.662}{-6.9}{14.662}{-9.6}{7.381}{-9.6}{1}{0}\errelangle{7.381}{-9.6}{0.1}{-9.6}{0.1}{-5.725}{0}{0}\ercrowfoot{13.612}{-4.2}{13.462}{-4.05}{13.462}{-4.2}{13.462}{-4.35}{0}
% relationship is
\ertext{11.025}{-6.7}{r}{is}\errelarm{11.175}{-6.4}{11.025}{-6.4}{1}{0}\errelarm{8.812}{-4.349}{8.662}{-4.349}{0}{0}\ertext{8.369}{-5.675}{l}{\textasciitilde /of=partof/of}\errelangle{11.025}{-6.4}{10.875}{-6.4}{9.919}{-5.375}{1}{0}\errelangle{9.919}{-5.375}{8.962}{-4.349}{8.812}{-4.349}{0}{0}\ercrowfoot{11.025}{-6.4}{11.175}{-6.25}{11.175}{-6.4}{11.175}{-6.55}{0}
% relationship to
\ertext{14.1}{-6.4}{l}{to}\errelarm{13.95}{-6.6}{14.2}{-6.6}{1}{0}\errelarm{1.088}{-4.349}{1.338}{-4.349}{0}{0}\errelangle{14.2}{-6.6}{14.45}{-6.6}{14.45}{-7.1}{1}{0}\errelangle{1.088}{-4.349}{0.838}{-4.349}{0.838}{-5.975}{0}{0}\ertext{6.794}{-7.9}{l}{\textasciitilde /of=partof/to}\errelangle{14.45}{-7.1}{14.45}{-7.6}{7.644}{-7.6}{1}{0}\errelangle{7.644}{-7.6}{0.838}{-7.6}{0.838}{-5.975}{0}{0}\ercrowfoot{14.1}{-6.6}{13.95}{-6.45}{13.95}{-6.6}{13.95}{-6.75}{0}\eridrefrel{14.1995}{-6.500000000000001}{-6.7}
\end{erdiagram}

}
\end{frame}
\begin{frame}{Relational Model of Data}
\scalebox{0.6}{
\begin{erdiagram}{11.899999999999999}{16.04475}

\eret{4.85}{-2.15}{10.35}{-1.25}{0.2}{1}\ertext{5.4}{-1.6}{l}{table}
\erattr{5.05}{-1.8}{1}{0}{name}
\eret{0.138}{-5.55}{2.912}{-3.75}{0.2}{1}\ertext{0.415}{-4.1}{l}{primary key column}
\erdattr{0.338}{-4.3}{1}{0}{table name(D2)}
\erdattr{0.338}{-4.6}{1}{0}{is name(R1)}
\erattr{0.338}{-4.9}{1}{1}{seq no}
\erattr{0.338}{-5.2}{1}{1}{seqNo}
\eret{5.662}{-5.25}{9.427}{-3.75}{0.2}{1}\ertext{6.039}{-4.1}{l}{column}
\erdattr{5.862}{-4.3}{1}{0}{table name(D3)}
\erattr{5.862}{-4.6}{1}{0}{name}
\erdattr{5.862}{-4.9}{0}{1}{in primary key is name(R2)}
\eret{12.427}{-5.25}{14.662}{-3.75}{0.2}{1}\ertext{12.651}{-4.1}{l}{foreign key}
\erdattr{12.627}{-4.3}{1}{0}{table name(D4)}
\erattr{12.627}{-4.6}{1}{0}{name}
\erdattr{12.627}{-4.9}{1}{1}{to name(R3)}
\eret{12.045}{-8.6}{15.045}{-6.8}{0.2}{1}\ertext{12.495}{-7.15}{l}{foreign key column}
\erdattr{12.245}{-7.35}{1}{0}{table name(D5)}
\erdattr{12.245}{-7.65}{1}{0}{foreign key name(D5)}
\erdattr{12.245}{-7.95}{1}{0}{to is name(R5)}
\erdattr{12.245}{-8.25}{1}{1}{is name(R4)}
\eret{0}{-0.2}{16.045}{0.3}{0.2}{1}

% relationship all tables
\ertext{7.75}{-0.5}{l}{all tables}\errelarm{7.6}{-0.2}{7.6}{-0.725}{1}{0}\errelarm{7.6}{-0.725}{7.6}{-1.25}{1}{0}\ercrowfoot{7.6}{-1.1}{7.45}{-1.25}{7.6}{-1.25}{7.75}{-1.25}{0}
% relationship 
\ertext{6.1}{-2.45}{l}{}\ertext{1.675}{-3.6}{l}{of}\errelarm{5.95}{-2.15}{5.95}{-2.225}{1}{0}\errelarm{1.525}{-3.463}{1.525}{-3.75}{1}{0}\ertext{3.587}{-2.588}{r}{D2}\errelangle{5.95}{-2.225}{5.95}{-2.3}{3.737}{-2.738}{1}{0}\errelangle{3.737}{-2.738}{1.525}{-3.175}{1.525}{-3.463}{1}{0}\errelseq{1.585}{-3.225}{1.175}{-3.285}{1.875}{-3.345}{1.465}{-3.405}\eridcomprel{1.4249999999999996}{1.6249999999999998}{-3.5}\ercrowfoot{1.525}{-3.6}{1.375}{-3.75}{1.525}{-3.75}{1.675}{-3.75}{0}
% relationship 
\ertext{7.695}{-2.45}{l}{}\ertext{7.695}{-3.6}{l}{of}\ertext{7.695}{-2.8}{l}{D3}\errelarm{7.545}{-2.15}{7.545}{-2.95}{1}{0}\errelarm{7.545}{-2.95}{7.545}{-3.75}{1}{0}\eridcomprel{7.444750000000001}{7.64475}{-3.5}\ercrowfoot{7.545}{-3.6}{7.395}{-3.75}{7.545}{-3.75}{7.695}{-3.75}{0}
% relationship 
\ertext{9.4}{-2.45}{l}{}\ertext{13.695}{-3.6}{l}{of}\errelarm{9.25}{-2.15}{9.25}{-2.225}{1}{0}\errelarm{13.545}{-3.538}{13.545}{-3.75}{1}{0}\ertext{11.547}{-2.663}{l}{D4}\errelangle{9.25}{-2.225}{9.25}{-2.3}{11.397}{-2.813}{1}{0}\errelangle{11.397}{-2.813}{13.545}{-3.325}{13.545}{-3.538}{1}{0}\eridcomprel{13.44475}{13.64475}{-3.5}\ercrowfoot{13.545}{-3.6}{13.395}{-3.75}{13.545}{-3.75}{13.695}{-3.75}{0}
% relationship is
\ertext{3.062}{-4.45}{l}{is}\ertext{4.237}{-4.5}{l}{R1}\ertext{3.937}{-4.95}{l}{\textasciitilde /of=of}\errelarm{2.912}{-4.65}{4.287}{-4.65}{1}{0}\errelarm{4.287}{-4.65}{5.662}{-4.65}{0}{0}\eridrefrel{3.16225}{-4.550000000000001}{-4.75}
% relationship 
\ertext{13.695}{-5.55}{l}{}\ertext{13.795}{-6.65}{l}{partof}\ertext{13.695}{-5.875}{l}{D5}\errelarm{13.545}{-5.25}{13.545}{-6.025}{1}{0}\errelarm{13.545}{-6.025}{13.545}{-6.8}{1}{0}\eridcomprel{13.44475}{13.64475}{-6.55}\ercrowfoot{13.545}{-6.65}{13.395}{-6.8}{13.545}{-6.8}{13.695}{-6.8}{0}
% relationship to
\ertext{14.812}{-4.35}{l}{to}\errelarm{14.662}{-4.5}{15.262}{-4.5}{1}{0}\errelarm{2.475}{-1.849}{4.85}{-1.849}{0}{0}\errelangle{15.262}{-4.5}{15.862}{-4.5}{15.862}{-7.2}{1}{0}\errelangle{2.475}{-1.849}{0.1}{-1.849}{0.1}{-5.875}{0}{0}\ertext{7.631}{-9.75}{l}{R3}\ertext{7.631}{-10.2}{l}{\textasciitilde /\textasciicircum =\textasciicircum }\errelangle{15.862}{-7.2}{15.862}{-9.9}{7.981}{-9.9}{1}{0}\errelangle{7.981}{-9.9}{0.1}{-9.9}{0.1}{-5.875}{0}{0}\ercrowfoot{14.812}{-4.5}{14.662}{-4.35}{14.662}{-4.5}{14.662}{-4.65}{0}
% relationship is
\ertext{11.895}{-7.7}{r}{is}\errelarm{12.045}{-7.4}{11.895}{-7.4}{1}{0}\errelarm{9.577}{-4.749}{9.427}{-4.749}{0}{0}\ertext{10.186}{-6.025}{l}{R4}\ertext{9.186}{-6.375}{l}{\textasciitilde /of=partof/of}\errelangle{11.895}{-7.4}{11.745}{-7.4}{10.736}{-6.075}{1}{0}\errelangle{10.736}{-6.075}{9.727}{-4.749}{9.577}{-4.749}{0}{0}\ercrowfoot{11.895}{-7.4}{12.045}{-7.25}{12.045}{-7.4}{12.045}{-7.55}{0}
% relationship to
\ertext{15.195}{-7.8}{l}{to}\errelarm{15.045}{-8}{15.295}{-8}{1}{0}\errelarm{-0.112}{-4.949}{0.138}{-4.949}{0}{0}\errelangle{15.295}{-8}{15.545}{-8}{15.545}{-8.5}{1}{0}\errelangle{-0.112}{-4.949}{-0.362}{-4.949}{-0.362}{-6.974}{0}{0}\ertext{7.241}{-8.85}{l}{R5}\ertext{6.741}{-9.3}{l}{\textasciitilde /of=partof/to}\errelangle{15.545}{-8.5}{15.545}{-9}{7.591}{-9}{1}{0}\errelangle{7.591}{-9}{-0.362}{-9}{-0.362}{-6.974}{0}{0}\ercrowfoot{15.195}{-8}{15.045}{-7.85}{15.045}{-8}{15.045}{-8.15}{0}\eridrefrel{15.29475}{-7.9}{-8.1}
\end{erdiagram}

}
\end{frame}
\subsection{Database Normal Forms}
\begin{frame}{Database Normal Forms}
\begin{itemize}
\item third normal form (3NF)
\item Boyce-Codd normal form (BCNF)
\item fourth normal form (4NF)
\item fifth normal form (5NF)
\end{itemize}
\end{frame}

\subsection{XML Schemas}
\begin{frame}{XML schemas}
\end{frame}

\subsection{Entity/ER Models}
\begin{frame}{Entity Modelling ER modelling}
Entity Modelling/ER Modelling
\end{frame}
\subsection{Extended Entity Modelling}

\begin{frame}{Extended Entity Modelling}
Single model mapping to both relational schema and to XML schema
\end{frame}
\begin{frame}{Composition v. Reference}
\end{frame}
\begin{frame}{Network v Hierarchy}
\end{frame}
\begin{frame}{Integrity Constraints}
\end{frame}
\subsection{ER $\Longrightarrow$ Relational (Chen's Transformation)}
\begin{frame}
\end{frame}
\begin{frame}{Database Normalisation}
\end{frame}

\subsection{The solution to a big problem}
\begin{frame}{A big problem}
\end{frame}
\begin{frame}{Commutative diagrams to the rescue}
\end{frame}
\begin{frame}{Is it really a solution?}
\end{frame}

\section{Part Two}
\subsection{Gats and Contextual Categories}
\begin{frame}{Generalised Algebriac Theories}
\end{frame}
\begin{frame}{Contextual Categories}
\end{frame}
\subsection{Alternatives To Contextual Catgories}
\begin{frame}{Real or Imagined}
\end{frame}

\begin{frame}{Citations}
\cite{Cartmell78}
\nocite{erhard88}
\end{frame}

\begin{frame}{erinlinePicture}
\begin{erdiagram}{4.6}{5.3}

\eret{2.7}{-1.6}{4.2}{-1}{0.2}{1}\ertext{3.45}{-1.35}{}{language}
\eret{1.7}{-3.1}{3.2}{-2.5}{0.2}{1}\ertext{2.45}{-2.85}{}{sentence}
\eret{3.7}{-3.1}{5.2}{-2.5}{0.2}{1}\ertext{4.45}{-2.85}{}{word}
\eret{0.094}{-4.6}{1.594}{-4}{0.2}{1}\ertext{0.844}{-4.35}{}{noun}
\eret{1.694}{-4.6}{3.194}{-4}{0.2}{1}\ertext{2.444}{-4.35}{}{verb}
\eret{3.294}{-4.6}{4.806}{-4}{0.2}{1}\ertext{4.05}{-4.35}{}{adjective}
\eret{0}{-0.2}{5.3}{0.3}{0.2}{1}

% relationship 
\ertext{3.55}{-0.5}{l}{}\errelarm{3.45}{-0.2}{3.45}{-0.6}{1}{0}\errelarm{3.45}{-0.6}{3.45}{-1}{1}{0}\ercrowfoot{3.45}{-0.85}{3.3}{-1}{3.45}{-1}{3.6}{-1}{0}
% relationship 
\ertext{3.3}{-1.9}{l}{}\errelarm{3.2}{-1.6}{3.2}{-1.675}{1}{0}\errelarm{2.45}{-2.375}{2.45}{-2.5}{1}{0}\errelangle{3.2}{-1.675}{3.2}{-1.75}{2.825}{-2}{1}{0}\errelangle{2.825}{-2}{2.45}{-2.25}{2.45}{-2.375}{1}{0}\ercrowfoot{2.45}{-2.35}{2.3}{-2.5}{2.45}{-2.5}{2.6}{-2.5}{0}
% relationship 
\ertext{3.8}{-1.9}{l}{}\errelarm{3.7}{-1.6}{3.7}{-1.675}{1}{0}\errelarm{4.45}{-2.375}{4.45}{-2.5}{1}{0}\errelangle{3.7}{-1.675}{3.7}{-1.75}{4.075}{-2}{1}{0}\errelangle{4.075}{-2}{4.45}{-2.25}{4.45}{-2.375}{1}{0}\ercrowfoot{4.45}{-2.35}{4.3}{-2.5}{4.45}{-2.5}{4.6}{-2.5}{0}
% relationship 
\ertext{2.175}{-3.4}{l}{}\errelarm{2.075}{-3.1}{2.075}{-3.175}{1}{0}\errelarm{0.844}{-3.875}{0.844}{-4}{1}{0}\errelangle{2.075}{-3.175}{2.075}{-3.25}{1.459}{-3.5}{1}{0}\errelangle{1.459}{-3.5}{0.844}{-3.75}{0.844}{-3.875}{1}{0}\ercrowfoot{0.844}{-3.85}{0.694}{-4}{0.844}{-4}{0.994}{-4}{0}
% relationship 
\ertext{2.55}{-3.4}{l}{}\errelarm{2.45}{-3.1}{2.45}{-3.175}{1}{0}\errelarm{2.444}{-3.875}{2.444}{-4}{1}{0}\errelangle{2.45}{-3.175}{2.45}{-3.25}{2.447}{-3.5}{1}{0}\errelangle{2.447}{-3.5}{2.444}{-3.75}{2.444}{-3.875}{1}{0}\ercrowfoot{2.444}{-3.85}{2.294}{-4}{2.444}{-4}{2.594}{-4}{0}
% relationship 
\ertext{2.925}{-3.4}{l}{}\errelarm{2.825}{-3.1}{2.825}{-3.175}{1}{0}\errelarm{4.05}{-3.875}{4.05}{-4}{1}{0}\errelangle{2.825}{-3.175}{2.825}{-3.25}{3.438}{-3.5}{1}{0}\errelangle{3.438}{-3.5}{4.05}{-3.75}{4.05}{-3.875}{1}{0}\ercrowfoot{4.05}{-3.85}{3.9}{-4}{4.05}{-4}{4.2}{-4}{0}
\end{erdiagram}

%\erinlinePicture{partsOfSpeech}{4}
\end{frame}

\begin{frame}{Bibliography}
\bibliography{../SharedBibliography/temp/bibliography}
\end{frame}
\end{document}