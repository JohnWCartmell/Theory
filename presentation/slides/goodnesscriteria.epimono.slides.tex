


\begin{frame}{Definition}
\begin{itemize}
\item A \term{\catMEterm} is a triple $\tuple{\catc,M,E}$ where 
\begin{itemize}
\item \catcw is a category,
\item $M$ is a set of designated monomorphisms of \catcw closed under composition and including all identity morphisms,
\item $E$ is a set of designated epimorphisms of \catcw closed under composition and including all identity morphisms.
\end{itemize}
\item Define an instance $F$ of a \catMEterm to be a functor $F: \catc \morph \Fin$ 
that preserves the designated monomorphisms and epimorphisms.

%Note that \catcw may well have other monomorphisms and epimorphisms but these need not be preserved by an instance $F$.

\item A sketch for a \catMEterm is a 4-tuple $\tuple{G,PE_0,M_0,E_0}$ ...
\begin{itemize}
 \item   $M_0$ and $E_0$ are sets of $G$-paths  
\end{itemize}
\end{itemize}
\end{frame}


\begin{frame}{Goodness Type 1 Criteria}
\medskip
\goodnessoneA.
\medskip
\goodnessoneB.

\newcommand{\goodnessoneC}{
\goodnesscriteria{1C} \\
There ought not exist $m \in M$ such that $m \in \overline{M \setminus m}$
}
\newcommand{\goodnessoneD}{
\goodnesscriteria{1D} \\
There ought not exist $e \in E$ such that $e \in \overline{E \setminus e}$
}

\goodnessoneC. \\
\medskip
\goodnessoneD. \\
\medskip

Definition of closure of set of monomorphics includes
\begin{itemize}
\item $f \in \overline{M}$ and $g \in \overline{M}$ then $f \circ g \in \overline{M}$
\item $f \circ g \in \overline{M}$ then $f \in \overline{M}$
\end{itemize}
and to which we might add the rule that expresses the finitary property.
\end{frame}

\begin{frame}{Definitions}
A functional depdency is said to be \term{transitive} ...

Otherwise a functional dependency is said to be \term{intransitive}.
\end{frame}


\begin{frame}{Revised Definition of Functional Dependency}
\begin{definition}
If \scalebox{0.9}{\fgsourcediagram} in a \catMEterm \catcw  and if $\reqtc$ is a set of instances of \catcw
then we say that there is a  \term{functional dependency} of $g$ on $f$ with respect to $\reqtc$ iff
for every $D \in \reqtc$, $D(g)$ factors through $D(f)$. \\
\medskip
Equivalently there is a family of unique functions $H_D:img(D(f)) \morph D(c))_{D \in \reqtc}$
such that in each instance $D \in \reqtc$,
\scalebox{0.9}{\fghfactordiagram{\roomup{0.4cm}\roomdown{0.3cm}D(a)}{img(D(f))}{D(c)}{D(f)}{D(g)}{H_D}} commutes.
\end{definition}

If $H$ is such a functional dependency then we say that $\fundep{H}{f}{g}$ in $\catc$ with respect to $\reqtc$.
\end{frame}

\begin{frame}{Representation of FDs}
\begin{definition}
%If there is a functional dependency $\fundep{H}{f}{g}$ $\catc$ is such a category and $\reqtc$ is a set of instances, 
If $\fundep{H}{f}{g}$ is an \highlight{intransitive}  functional dependency wrt such a category \catcw and set of instances $\reqtc$, 
where \scalebox{0.9}{\roomdown{0.8cm}\fgsourcediagram} in $\catc$, 
then $H$  is \term{represented} in $\catcw$ 
iff there exists an object $d$ of \catc, a monomorphism $q$ and a morphism $h$
like so:
\scalebox{0.9}{\roomup{1.2cm}\unaryfdrepresentationdiagram{a}{b}{c}{f}{g}{d}{q}{h}}
such that in every instance $D \in \reqtc$,
\scalebox{0.9}{\roomup{1.2cm}\unaryfdrepresentationmappeddiagram{a}{b}{c}{f}{g}{d}{q}{h}} commutes.
\end{definition}
\end{frame}


\begin{frame}{Lemma}
\begin{lemma}
If \catcw is a \catMEterm with a finite sketch and if $f$ is any  morphism 
not having an epi-mono factorisation then \catcw can be extended to a category $\catcp$ with finite sketch so that
\begin{itemize} 
\item $f$ has an epi-mono factorisation in $\catcp$,
\item every morphism $h:im(f) \morph x$ in $\catcp$ factors through $f_m$,
\item every instance of \catcw uniquely extends, upto isomorphism, to an instance of $\catcp$,
\item \catcw is maximally constrained to $\reqtc$ iff $\catcp$ is maximally constrained to $\reqtc$ (extended to $\catcp$).
\end{itemize} 
\end{lemma} 
\end{frame}
\begin{frame}{Representation Lemma for FDs}
{\small Note: I can prove this only in the case that \catcw is locally finite.

No longer restricted to the case that each $D(f)$ is surjective.}

\begin{lemma}
If $\catc$ is a locally finite \catMEterm that is 
\textit{maximally constrained} to a requirement $\reqtc$ then
if $\fundep{H}{f}{g}$  is an \highlight{intransitive}  functional dependency with respect to $\reqtc$
then $\fundep{H}{f}{g}$ is represented in $\catc$.
\end{lemma}
\begin{proof} If $f$ has a mono-epi factorisation then
\scalebox{0.9}{\unaryfdrepresentationdiagram{\roomup{0.6cm}\roomdown{0.4cm}a}{b}{c}{f}{g}{im(f)}{f_m}{h}$\begin{arrows}\ncarr{a}{d}\alabel{f_e}\end{arrows}$} commutes in \catcw  where $h$ represents 
the functional dependency of $g$ on $f_e$.
%whose existence is given by previous version of this lemma.
Otherwise extend \catcw so that $f$ has an epi-mono factorisation.
Establish a $h: b \morph c$ in \catcw that represents $H$.
\end{proof}
\end{frame}

\begin{frame}{Epi Mono Type 2 Goodness Criteria}
From local finiteness and maximal constrainedness it follows that all the following goodness criteria hold: 
\begin{itemize}
\item \textbf{2A}:  \catcw ought to be equationally complete wrt $\reqtc$,         
\item \textbf{2B}:  all $\reqtc$ intransitive functional depedencies ought to be represented in \catc,
\item \textbf{2C}:  all $\reqtc$ referential inclusion dependencies ought to be represented in \catc, 
\pause \item \textbf{2D}:  if $f$ is injective in every $D \in \reqtc$ then $f$ ought to be a designated monomorphism in \catc,
\pause \item \textbf{2E}:  if $f$ is surjective in every $D \in \reqtc$ then $f$ ought to be a designated epimorphism in \catc.
\end{itemize}
We can also show that \textbf{2B} can be weakened. If all intransitive dependencies \textbf{upon edges} from sketch $S$ are represented then 
all intransitive depedencies are represented.
\end{frame}

\begin{frame}{Simplifying Assumption}
Say that the sketch $S$ is simple if it is minimal (no equivalent subsketches) and if there are no path equivalences in $S$ in which one side path or the other is of length 1.  
\end{frame}


\begin{frame}{Tellingly...}
With the additional assumption now that $S$ is simple we can show that if
ff goodness criteria 2A and 2B then 
\begin{itemize}
\item If $\fundep{H}{f}{g}$ is a non-trivial \highlight{intransitive}  functional dependency wrt  category \catcw and set of instances $\reqtc$
 then $f$ is a monomorphism.

\item This moves us into sight of the goal of "explaining the BCNF (normal form) criteria from first principle". \highlight{Much earlier need to state BCNF.}

\item To get closer we need add finite products to the mix.
\end{itemize}
\end{frame}
