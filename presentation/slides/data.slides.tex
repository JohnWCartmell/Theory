
\begin{frame}{Chemical Element Data}
\scalebox{0.75}{

\newcommand{\allotropeAwidth}{44pt}
\newcommand{\allotropeBwidth}{25pt}
\newcommand{\allotropeCwidth}{25pt}
\newcommand{\allotropeDwidth}{25pt}

\newcommand{\valencywidth}{15pt}


\hspace{-1cm}
\begin{tabular}{|l|l|l|l|c|l|}
\hline
\rowcolor{myblue}\colhead{name}    &
\colhead{symbol}  &
\multicolumn{1}{|p{1.0cm}|}{\colhead{atomic no}}&
\colhead{r.a.m.}   &
\begin{tabular}{|c|c|}
\multicolumn{1}{c}{\colhead{valency}} \\
\hline 
\colhead{number} \\
\hline 
\end{tabular} &
\begin{tabular}{|p{\allotropeAwidth}|p{\allotropeBwidth}|p{\allotropeCwidth}|p{\allotropeDwidth}|}
\hline
 \multicolumn{4}{c}{\colhead{allotrope}}     \\
 \hline
 \colhead{name}   & \colhead{m.p.}  & \colhead{b.p.}  & \colhead{density} \\
 \hline 
 \end{tabular} \\
 \hline
oxygen & O &  8 &  15.99 &   
\begin{tabular}{|p{\valencywidth}|}
\hline
2\\
\hline
\end{tabular} & 
\begin{tabular}{|p{\allotropeAwidth}|p{\allotropeBwidth}|p{\allotropeCwidth}|p{\allotropeDwidth}|}
\vpad{4}
\hline   
dioxygen  &  -218  &  -183  &  .0014 \\
\hline
Ozone     & -192 & -119 & .0032 \\
\hline 
\vpad{4} 
\end{tabular} \\
\hline
chlorine  &  Cl & 17 & 35.45 & 
\begin{tabular}{|p{\valencywidth}|}
\vpad{1}
\hline
1- \\
\hline
3+\\
\hline
5+\\
\hline
7+\\
\hline
\vpad{1} 
\end{tabular}
&
\begin{tabular}{|p{\allotropeAwidth}|p{\allotropeBwidth}|p{\allotropeCwidth}|p{\allotropeDwidth}|}
\hline
Dichlorine & -101 &   -34.6 &  >?? \\
\hline
\end{tabular}\\
\hline
sulphur & S &  16 &  32.06  &
\begin{tabular}{|p{\valencywidth}|}
\vpad{1} 
\hline
2\\
\hline
4\\
\hline
6\\
\hline
\multicolumn{1}{c}{}\\[-0.25cm] 
\end{tabular} &
\begin{tabular}{|p{\allotropeAwidth}|p{\allotropeBwidth}|p{\allotropeCwidth}|p{\allotropeDwidth}|}
\vpad{4} 
\hline 
Rhombic    & 112.8 &  ???  & 2.07 \\
\hline
Monoclinic & 119   & 444.6 &  1.96 \\
\hline 
\vpad{4} 
\end{tabular} \\
\hline
\end{tabular}
}
\begin{itemize}
	\item Underline indicates some kind of key attribute i.e. some kind of uniqueness constraint.
\end{itemize}
An example of data organised hierarchically. 
I will use the A-word. Columns I will say are attributes. Attributes are string valued or integer valued or float  valued.
Certain columns hold unique values. In database speak these are key attributes. In the language we use
symbol is the primary key. Name is also unque and could have been chosen as the key instead. It is called a candidate key attribute.


This data is displayed as described by the nested relational model of data. This is a hierarchical model.
Other hierarchical models are that of XML. And that of Google ProtocolBuffer messaging system as described in protocol buffer IDL language.

\end{frame}


\begin{frame}[fragile]{Message Structure}
A corresponding message structure might be defined 
in an IDL style language by:

\begin{lstlisting}[basicstyle=\footnotesize,language=IDL]
struct compound => 
    symbol: string,
    name : string,
    atomic_no: integer,
    relative_atomoc_mass : float,
    valencies: valency repeated,
    allotropes : allotrope repeated

struct valency =>  number : integer 

struct allotrope =>
   name : string,
   melting_point : float,
   boiling_point : float,
   density : float
\end{lstlisting}
\end{frame}


\begin{frame}{Chemical Compound Data}
\scalebox{0.75}{
\newcommand{\aliasnamewidth}{90pt}
\newcommand{\formulaAwidth}{35pt}
\newcommand{\formulaBwidth}{25pt}
\begin{tabular} {|l|l|l|l|}
\hline
\rowcolor{myblue}\multicolumn{4}{l}{\colhead{compound}} \\
\hline
\rowcolor{myblue}\colhead{\pk{name}} & \colhead{molar mass} &
\begin{tabular}{|p{\aliasnamewidth}|}
\multicolumn{1}{c}{\colhead{aliases}} \\
\multicolumn{1}{c}{\colhead{aliased name}} \\
\hline
\end{tabular} &
\begin{tabular}{|p{\formulaAwidth}|p{\formulaBwidth}|}
\multicolumn{2}{c}{\colhead{formula}} \\
\hline
	\colhead{\pkfk{symbol}} & \colhead{count} \\
\hline
\end{tabular} \\
\hline
aspirin	& 180.16 &
\begin{tabular}{|p{\aliasnamewidth}|}
\vpad{1}
\hline
acetylsalicylic acid \\
\hline
\vpad{1}
\end{tabular} &
\begin{tabular}{|p{\formulaAwidth}|p{\formulaBwidth}|}
\vpad{2}
\hline 
C&9\\
\hline
H&8\\
\hline
O&4\\
\hline
\vpad{2}
\end{tabular} \\
\hline
sodium chloride & 58.44	&	
\begin{tabular}{|p{\aliasnamewidth}|}
\vpad{1}
\hline 
common salt \\
\hline
halite \\
\hline
saline \\	
\hline
\vpad{1}
\end{tabular} &
\begin{tabular}{|p{\formulaAwidth}|p{\formulaBwidth}|}
\vpad{2}
\hline 	 
Na	& \\
\hline
Cl	& \\
\hline
\vpad{2}
\end{tabular} \\
\hline
water &	18.01 &
\begin{tabular}{|p{\aliasnamewidth}|}
\vpad{1}
\hline 
oxidane \\
\hline
\vpad{1}
\end{tabular} &
\begin{tabular}{|p{\formulaAwidth}|p{\formulaBwidth}|}
\vpad{2}
\hline 	 
H	& 2\\
\hline
O	& \\
\hline
\vpad{2}
\end{tabular} \\
\hline
\end{tabular}
		 
	
}
\begin{itemize}
\item $\largeAsterisk$ Asterisk here denotes a foreign key attribute. 
\pause This witnesses a relationship between this part of a formula and an element. \pause It implements a pointer.
\end{itemize}
\end{frame}

\begin{frame}{Jackson Structure Diagrams}
\begin{tabular}{c c}
\scalebox{0.9}{\jacksonbinarydiagram{compound\kern0.1cm}{alias \kern1.2cm}{occurrence\kern0cm}}
&
\scalebox{0.9}{\jacksonbinarydiagram{element\kern0.4cm}{valency \kern0.8cm}{allotrope\kern0.3cm}}
\end{tabular}
\end{frame}

\begin{frame}{Bachman Structure Diagrams}
\begin{tabular}{c c}
\scalebox{0.9}{\bachmanbinarydiagram[left]{compound\kern0.1cm}{alias \kern1.2cm}{occurrence}}
&
\scalebox{0.9}{\bachmanbinarydiagram[right]{element\kern0.4cm}{valency \kern0.8cm}{allotrope\kern0.3cm}}
\end{tabular}
\begin{itemize}
	\item This diagram describes the hierarchcal structure of the nested tables of data I started with.
	\item It doesn't show all the relationships in the data
	\pause \item there is a set of zero, one or more occurrences that point to any given element
	\pause \ncline[linewidth=1.5pt, linestyle=dashed]{<-}{leftC}{rightA}
\end{itemize}
\end{frame}

\begin{frame} 
In my tables the relationship between occurence and element is implemented by the symbol attribute of the compound data.
Looking up the symbol of an occurence in the symbol of the element data.
The symbol attribue of compound is called a foreign key
and the symbol attribute of the compound table is called a primary key. 
\end{frame}

\begin{frame}{Bachman Network Structure Representation}
\begin{itemize}
	\item Bachman is credited with the concept of network database.
    \item A network structure can have parts `contained in' many non-commensurate wholes.
    \item Using such we don't need commit to one hierarchy or the other. 
 \end{itemize}
\bachmannetworkdiagram{compound\kern0.3cm}{element\kern0.6cm}{occurrence\kern0.2cm}
\end{frame}

\begin{frame}{Restructured Chemical Element Data}
\scalebox{0.75}{

\newcommand{\occursinAwidth}{88pt}
\newcommand{\occursinBwidth}{25pt}

\newcommand{\valencywidth}{15pt}


\hspace{-1cm}
\begin{tabular}{|c|l|l|l|l|}
\hline
\rowcolor{myblue}\multicolumn{5}{l}{\colhead{element}} \\
\hline
\rowcolor{myblue}\colhead{\pk{symbol}}  &
\colhead{name}    &
\multicolumn{1}{|p{1.0cm}|}{\colhead{atomic no}}&
\colhead{...}   &
\begin{tabular}{|p{\occursinAwidth}|p{\occursinBwidth}|}
\hline
 \multicolumn{2}{c}{\colhead{occurs in}}     \\
 \hline
 \colhead{\pkfk{compound}}   & \colhead{count}   \\
 \hline 
 \end{tabular} \\
 \hline
 O & oxygen & 8 &  ... &  
\begin{tabular}{|p{\occursinAwidth}|p{\occursinBwidth}|}
\vpad{2}
\hline   
water  &  1  \\
\hline
aspirin & 2  \\
\hline
thionyle chloride & 1 \\
\hline 
... & ... \\
\hline 
\vpad{2} 
\end{tabular} \\
\hline
Cl & chlorine  & 17 & ... & 
\begin{tabular}{|p{\occursinAwidth}|p{\occursinBwidth}|}
\hline
sodium chloride & 1   \\
\hline
thionyle chloride & 2   \\
\hline
... & ... \\
\hline
\end{tabular}\\
\hline
S & sulphur &  16 &  ...  &
\begin{tabular}{|p{\occursinAwidth}|p{\occursinBwidth}|}
\vpad{2} 
\hline 
sulphur dioxide    & 1  \\
\hline 
... & ... \\
\hline 
\vpad{2} 
\end{tabular} \\
\hline
\end{tabular}
}
\end{frame}
