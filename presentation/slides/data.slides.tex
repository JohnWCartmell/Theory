
\newcommand{\CDsymboltype}{varchar(2)}
\newcommand{\CDatomicnumbertype}{number(1,1000)}
\newcommand{\CDfloattype}{float}
\newcommand{\CDnametype}{string}
\newcommand{\CDvalencynumbertype}{number(-7,7)}
\begin{frame}{Chemical Element Data}
\begin{center}
\scalebox{0.75}{

\begin{tabular}{|l|c|l|}
\hline
\rowcolor{myblue}\multicolumn{3}{l}{\colhead{allotrope}} \\
\hline
\rowcolor{myblue}\colhead{\pk{name}}  &
\colhead{\pk{symbol}}  &
\multicolumn{1}{|p{1.0cm}|}{\colhead{atomic no}}\\
 \hline
 Dioxygen & O &  8     \\
\hline
 Ozone    & O &  8     \\
\hline
Dichlorine & Cl & 17   \\
\hline
Rhombic & S &  16      \\
\hline
Monoclinic & S &   16  \\ 
\hline
Rhombic & Sn & 50      \\
\hline
\end{tabular}}
\hspace{0.5cm}
\onslide<2->{\scalebox{0.75}{
$
\begin{array}{c p{2.0cm} l}
                                                         && \attrtype{\Rnode{nametype}{\CDnametype}}               \\ [0.9cm]
\etype{\Rnode{allotrope}{allotrope}\Rnode{allotropeR}{}} && \attrtype{\Rnode{symboltype}{\CDsymboltype}  }         \\ [0.9cm]
                                                         && \attrtype{\Rnode{atomicnumbertypeL}{} \CDatomicnumbertype} 
\end{array}
$
\begin{arrows}
\newcommand{\CDelementnameattribute}{
\setlength{\arrnodesepA}{8pt}
\setlength{\arrnodesepB}{4pt}
\setlength{\arroffsetA}{5pt}
\setlength{\arroffsetB}{4pt}
\ncarr[7]{allotropeR}{nametype}
}
\CDelementnameattribute
\alabel{name}[0.3]

\newcommand{\CDsymbolattribute}{
\setlength{\arrnodesepA}{6pt}
\setlength{\arrnodesepB}{4pt}
\setlength{\arroffsetA}{0pt}
\setlength{\arroffsetB}{-2pt}
\ncarr[8]{allotropeR}{symboltype}
}
\CDsymbolattribute
\alabel{symbol}[0.4][0]

\newcommand{\CDatomicnumberattribute}{
\setlength{\arrnodesepA}{8pt}
\setlength{\arroffsetA}{-6pt}
\setlength{\arrnodesepB}{6pt}
\setlength{\arroffsetB}{0pt}
\ncarr[3]{allotropeR}{atomicnumbertypeL}
}
\CDatomicnumberattribute
\blabel{atomic\,no.}[0.4][0]


%barring of mono source i.e. uniqueness constraints start here
\onslide<3->{
% type chemical element
\CDsymbolattribute
\addedgebar
\CDelementnameattribute
\addedgebar}
\end{arrows}}} 
\end{center}
\begin{itemize}
	\item An example of data organised in a relational table.
	\onslide<3->{\item The key to this table is a combination of the name and the symbol columns.} 
	\onslide<4->{\item Now to meet normal form criteria each column needs ``\textit{depend on the key, the whole key and nothing but the key}''.} 
	\onslide<5->{ \item This table does not meet the 2NF criteria because the  `atomic no' column depends only on symbol.}
\end{itemize}
\end{frame}

\begin{frame}{Graph of the Allotrope Table Structure}
\scalebox{0.9}{

$
\begin{array}{c p{2.0cm} l}
                                                         && \attrtype{\Rnode{nametype}{\CDnametype}}               \\ [0.9cm]
\etype{\Rnode{allotrope}{allotrope}\Rnode{allotropeR}{}} && \attrtype{\Rnode{symboltype}{\CDsymboltype}  }         \\ [0.9cm]
                                                         && \attrtype{\Rnode{atomicnumbertypeL}{} \CDatomicnumbertype} 
\end{array}
$
\begin{arrows}
\newcommand{\CDelementnameattribute}{
\setlength{\arrnodesepA}{8pt}
\setlength{\arrnodesepB}{4pt}
\setlength{\arroffsetA}{5pt}
\setlength{\arroffsetB}{4pt}
\ncarr[7]{allotropeR}{nametype}
}
\CDelementnameattribute
\alabel{name}[0.3]

\newcommand{\CDsymbolattribute}{
\setlength{\arrnodesepA}{6pt}
\setlength{\arrnodesepB}{4pt}
\setlength{\arroffsetA}{0pt}
\setlength{\arroffsetB}{-2pt}
\ncarr[8]{allotropeR}{symboltype}
}
\CDsymbolattribute
\alabel{symbol}[0.4][0]

\newcommand{\CDatomicnumberattribute}{
\setlength{\arrnodesepA}{8pt}
\setlength{\arroffsetA}{-6pt}
\setlength{\arrnodesepB}{6pt}
\setlength{\arroffsetB}{0pt}
\ncarr[3]{allotropeR}{atomicnumbertypeL}
}
\CDatomicnumberattribute
\blabel{atomic\,no.}[0.4][0]


%barring of mono source i.e. uniqueness constraints start here
\onslide<3->{
% type chemical element
\CDsymbolattribute
\addedgebar
\CDelementnameattribute
\addedgebar}
\end{arrows}
} 
\end{frame}


\begin{frame}{Chemical Element Data}
\scalebox{0.75}{

\newcommand{\allotropeAwidth}{44pt}
\newcommand{\allotropeBwidth}{25pt}
\newcommand{\allotropeCwidth}{25pt}
\newcommand{\allotropeDwidth}{25pt}
\newcommand{\valencywidth}{15pt}

\hspace{-1cm}
\begin{tabular}{|c|l|l|l|c|l|}
\hline
\rowcolor{myblue}\multicolumn{6}{l}{\colhead{element}} \\
\hline
\rowcolor{myblue}\colhead{\pk{symbol}}  &
\colhead{\seck{name}}    &
\multicolumn{1}{|p{1.0cm}|}{\colhead{atomic no}}&
\colhead{r.a.m.}   &
\begin{tabular}{|c|c|}
\multicolumn{1}{c}{\colhead{valency}} \\
\hline 
\colhead{\pk{number}} \\
\hline 
\end{tabular} &
\begin{tabular}{|p{\allotropeAwidth}|p{\allotropeBwidth}|p{\allotropeCwidth}|p{\allotropeDwidth}|}
\hline
 \multicolumn{4}{c}{\colhead{allotrope}}     \\
 \hline
 \colhead{\pk{name}}   & \colhead{m.p.}  & \colhead{b.p.}  & \colhead{density} \\
 \hline 
 \end{tabular} \\
 \hline
 O & oxygen & 8 &  15.99 &   
\begin{tabular}{|p{\valencywidth}|}
\hline
2\\
\hline
\end{tabular} & 
\begin{tabular}{|p{\allotropeAwidth}|p{\allotropeBwidth}|p{\allotropeCwidth}|p{\allotropeDwidth}|}
\vpad{4}
\hline   
dioxygen  &  -218  &  -183  &  .0014 \\
\hline
Ozone     & -192 & -119 & .0032 \\
\hline 
\vpad{4} 
\end{tabular} \\
\hline
Cl & chlorine  & 17 & 35.45 & 
\begin{tabular}{|p{\valencywidth}|}
\vpad{1}
\hline
1- \\
\hline
3+\\
\hline
5+\\
\hline
7+\\
\hline
\vpad{1} 
\end{tabular}
&
\begin{tabular}{|p{\allotropeAwidth}|p{\allotropeBwidth}|p{\allotropeCwidth}|p{\allotropeDwidth}|}
\hline
Dichlorine & -101 &   -34.6 &  >?? \\
\hline
\end{tabular}\\
\hline
S & sulphur &  16 &  32.06  &
\begin{tabular}{|p{\valencywidth}|}
\vpad{1} 
\hline
2\\
\hline
4\\
\hline
6\\
\hline
\multicolumn{1}{c}{}\\[-0.25cm] 
\end{tabular} &
\begin{tabular}{|p{\allotropeAwidth}|p{\allotropeBwidth}|p{\allotropeCwidth}|p{\allotropeDwidth}|}
\vpad{4} 
\hline 
Rhombic    & 112.8 &  ???  & 2.07 \\
\hline
Monoclinic & 119   & 444.6 &  1.96 \\
\hline 
\vpad{4} 
\end{tabular} \\
\hline
\end{tabular}
}

\begin{itemize}
	\item An example of data organised in nested relations.
	\item Underlined column indicates a key attribute and thereby a uniqueness constraint.
	\item Some data relationships are represented by physical containment relationships in the message structure i.e. by nesting.
\end{itemize}
\end{frame}

\lstdefinelanguage{MyIDL}[]{IDL}
{morekeywords={Set,Of, struct},sensitive=true}


\begin{frame}[fragile]{Description of Chemical Element Table}
In an IDL data specification it would look something like this:
\only<1>{\lstinputlisting
[basicstyle=\footnotesize,language=myIDL]
{../content/chemistryData/element.idl}
}
\onslide<2->{\lstinputlisting
[basicstyle=\footnotesize,keywordstyle={\ttfamily\color{green}\bfseries},language=myIDL]
{../content/chemistryData/element.idl}
}

\end{frame}

\begin{frame}{Chemical Compound Data}
\scalebox{0.75}{
\newcommand{\aliasnamewidth}{90pt}
\newcommand{\formulaAwidth}{35pt}
\newcommand{\formulaBwidth}{25pt}
\begin{tabular} {|l|l|l|l|}
\hline
\rowcolor{myblue}\multicolumn{4}{l}{\colhead{compound}} \\
\hline
\rowcolor{myblue}\colhead{\pk{name}} & \colhead{molar mass} &
\begin{tabular}{|p{\aliasnamewidth}|}
\multicolumn{1}{c}{\colhead{aliases}} \\
\multicolumn{1}{c}{\colhead{aliased name}} \\
\hline
\end{tabular} &

\begin{tabular}{|p{\formulaAwidth}|p{\formulaBwidth}|}
\multicolumn{2}{c}{\colhead{formula}} \\
\hline
	\colhead{\pkfk{symbol}} & \colhead{count} \\
\hline
\end{tabular} \\
\hline
aspirin	& 180.16 &
\begin{tabular}{|p{\aliasnamewidth}|}
\vpad{1}
\hline
acetylsalicylic acid \\
\hline
\vpad{1}
\end{tabular} &
\begin{tabular}{|p{\formulaAwidth}|p{\formulaBwidth}|}
\vpad{2}
\hline 
C&9\\
\hline
H&8\\
\hline
O&4\\
\hline
\vpad{2}
\end{tabular} \\
\hline
sodium chloride & 58.44	&	
\begin{tabular}{|p{\aliasnamewidth}|}
\vpad{1}
\hline 
common salt \\
\hline
halite \\
\hline
saline \\	
\hline
\vpad{1}
\end{tabular} &
\begin{tabular}{|p{\formulaAwidth}|p{\formulaBwidth}|}
\vpad{2}
\hline 	 
Na	& \\
\hline
Cl	& \\
\hline
\vpad{2}
\end{tabular} \\
\hline
water &	18.01 &
\begin{tabular}{|p{\aliasnamewidth}|}
\vpad{1}
\hline 
oxidane \\
\hline
\vpad{1}
\end{tabular} &
\begin{tabular}{|p{\formulaAwidth}|p{\formulaBwidth}|}
\vpad{2}
\hline 	 
H	& 2\\
\hline
O	& \\
\hline
\vpad{2}
\end{tabular} \\
\hline
\end{tabular}
		 
	
}
\begin{itemize}
\item $\largeAsterisk$ Asterisk here denotes a foreign key attribute. 
\item The 'symbol' attribute is a foreign key into the `element' table i.e.
	\begin{equation}
	              formula[symbol] \subseteq element[symbol]
	\end{equation}
\end{itemize}	
\end{frame}

\begin{frame}[fragile]{Message Structure}
A corresponding message structure might be defined 
in an IDL style language by:

\only<1>{\lstinputlisting
[basicstyle=\footnotesize,language=myIDL]
{../content/chemistryData/compound.idl}
}
\onslide<2->{\lstinputlisting
[basicstyle=\footnotesize,keywordstyle={\ttfamily\color{green}\bfseries},language=myIDL]
{../content/chemistryData/compound.idl}
}
\end{frame}


\newcommand{\occurence}{\parbox{1.5cm}{occurring element}}

\begin{frame}{Jackson Structure Diagrams}
\begin{tabular}{c c}
\scalebox{0.9}{\jacksonbinarydiagram{compound\kern0.1cm}{alias \kern1.2cm}{\occurence\kern0cm}}
&
\scalebox{0.9}{\jacksonbinarydiagram{element\kern0.4cm}{valency \kern0.8cm}{allotrope\kern0.3cm}}
\end{tabular}
\end{frame}

\begin{frame}{Jackson Structure Diagrams}
Diagrammatic niceties call an `'occuring element' an 'occurence'. 

\begin{center}
\begin{tabular}{c c}
\scalebox{0.9}{\jacksonbinarydiagram{compound\kern0.1cm}{alias \kern1.2cm}{occurence\kern0cm}}
&
\scalebox{0.9}{\jacksonbinarydiagram{element\kern0.4cm}{valency \kern0.8cm}{allotrope\kern0.3cm}}
\end{tabular}
\end{center}
\end{frame}

\begin{frame}{Bachman Structure Diagrams}
\begin{tabular}{c c}
\scalebox{0.9}{\bachmanbinarydiagram[left]{compound\kern0.1cm}{alias \kern1.2cm}{occurence}}
&
\scalebox{0.9}{\bachmanbinarydiagram[right]{element\kern0.4cm}{valency \kern0.8cm}{allotrope\kern0.3cm}}
\end{tabular}
\begin{itemize}
	\item This diagram describes the hierarchcal structure of the nested tables of data I started with.
	\item It doesn't show all the relationships in the data
	\pause \item there is a set of zero, one or more occurrences that point to any given element
	\pause \ncline[linewidth=1.5pt, linestyle=dashed]{->}{leftC}{rightA}
\end{itemize}
\end{frame}

\begin{frame}{Bachman Network Diagram!!}
\doublebachmannetworkdiagram{compound\kern0.1cm}{element\kern0.4cm}{alias \kern1.2cm}{occurence\kern0cm}{valency \kern0.8cm}{allotrope\kern0.3cm}
\end{frame}

\begin{frame}{Network Diagram Morphisms Version}
\doublecategorynetworkdiagram{compound\kern0.1cm}{element\kern0.4cm}{alias \kern1.2cm}{occurence\kern0cm}{valency \kern0.8cm}{allotrope\kern0.3cm}

\begin{itemize}
	\item arrows now represent functional many-one relationships - these are morphisms
\end{itemize}
\end{frame}


\begin{frame}{Network Diagram Distinguished Morphisms Version}
\contextualcategorynetworkdiagram{compound\kern0.1cm}{element\kern0.4cm}{alias \kern1.2cm}{occurence\kern0cm}{valency \kern0.8cm}{allotrope\kern0.3cm}
\begin{itemize}
	\item distinguished arrows now represent containment relationships
	\item others represent foreign key implemented relationships
\end{itemize}
\end{frame}

\begin{frame}{Network Diagram Distinguished Morphisms Version Reorganised}
\contextualcategorynetworkdiagramreorganised{compound\kern0.1cm}{element\kern0.4cm}{alias \kern1.2cm}{occurence\kern0cm}{valency \kern0.8cm}{allotrope\kern0.3cm}
\begin{itemize}
	\item distinguished arrows now represent containment relationships
	\item others represent foreign key implemented relationships
\end{itemize}
\end{frame}

\begin{frame}{Made clearer}
\contextualcategorynetworkdiagramtopologised{compound\kern0.1cm}{element\kern0.4cm}{alias \kern0.6cm}{occurence\kern0cm}{valency \kern0.8cm}{allotrope\kern0.3cm}
\end{frame}

\begin{frame}{What if}
\contextualcategorynetworkdiagramreorganisedtopologised{compound\kern0.1cm}{element\kern0.4cm}{alias \kern0.6cm}{occurence\kern0cm}{valency \kern0.8cm}{allotrope\kern0.3cm}
\end{frame}

\begin{frame}{Restructured Chemical Element Data}
\scalebox{0.75}{

\newcommand{\occursinAwidth}{88pt}
\newcommand{\occursinBwidth}{25pt}

\newcommand{\valencywidth}{15pt}


\hspace{-1cm}
\begin{tabular}{|c|l|l|l|l|}
\hline
\rowcolor{myblue}\multicolumn{5}{l}{\colhead{element}} \\
\hline
\rowcolor{myblue}\colhead{\pk{symbol}}  &
\colhead{name}    &
\multicolumn{1}{|p{1.0cm}|}{\colhead{atomic no}}&
\colhead{...}   &
\begin{tabular}{|p{\occursinAwidth}|p{\occursinBwidth}|}
\hline
 \multicolumn{2}{c}{\colhead{occurs in}}     \\
 \hline
 \colhead{\pkfk{compound}}   & \colhead{count}   \\
 \hline 
 \end{tabular} \\
 \hline
 O & oxygen & 8 &  ... &  
\begin{tabular}{|p{\occursinAwidth}|p{\occursinBwidth}|}
\vpad{2}
\hline   
water  &  1  \\
\hline
aspirin & 4  \\
\hline
thionyle chloride & 1 \\
\hline 
... & ... \\
\hline 
\vpad{2} 
\end{tabular} \\
\hline
Cl & chlorine  & 17 & ... & 
\begin{tabular}{|p{\occursinAwidth}|p{\occursinBwidth}|}
\hline
sodium chloride & 1   \\
\hline
thionyle chloride & 2   \\
\hline
... & ... \\
\hline
\end{tabular}\\
\hline
S & sulphur &  16 &  ...  &
\begin{tabular}{|p{\occursinAwidth}|p{\occursinBwidth}|}
\vpad{2} 
\hline 
sulphur dioxide    & 1  \\
\hline 
... & ... \\
\hline 
\vpad{2} 
\end{tabular} \\
\hline
\end{tabular}
}
\end{frame}

\begin{frame}{Purely Relational Data}
\begin{itemize}
	\item There is no nesting whatsoever
	\item All relationships (arrows) implemented by foreign keys
\end{itemize}
\begin{tabular}[t] {c c}
\multicolumn{2}{c}{\scalebox{0.75}{

\begin{tabular} {|l|l|}
\hline
\rowcolor{myblue}\multicolumn{2}{l}{\colhead{compound}} \\
\hline
\rowcolor{myblue}\colhead{\pk{name}} & \colhead{molar mass} \\
\hline
aspirin	& 180.16  \\
\hline
sodium chloride & 58.44	\\
\hline
water &	18.01  \\
\hline
\end{tabular}
		 
	
}} \\[1cm]
\scalebox{0.75}{

%
% formula(compound,element,count)
%
{ % enclosing scope for macro definitions


\newcommand{\tableheader}{\rowcolor{myblue}\multicolumn{2}{l}{\colhead{aliases}} \\}
\newcommand{\columnheaders}{\rowcolor{myblue}\colhead{\pk{alias}} & \colhead{\fk{aliased}} \\}

\begin{tabular}[t] {|l|l|} %[t] added so as to top align 
\hline
\tableheader
\hline
\columnheaders
\hline 
acetylsalicylic acid & aspirin	\\
\hline
 common salt  & sodium chloride \\
\hline
halite  & sodium chloride \\
\hline
saline & sodium chloride  \\	
\hline
oxidane & water 	 \\
\hline
\end{tabular}
}
		 
	
} &
\scalebox{0.75}{

%
% formula(compound,element,count)
%
{ % enclosing scope for macro definitions


\newcommand{\tableheader}{\rowcolor{myblue}\multicolumn{3}{l}{\colhead{formula}} \\}
\newcommand{\columnheaders}{\rowcolor{myblue}\colhead{\pkfk{compound}} & \colhead{\fk{\pkfk{element}}} & \colhead{count}\\}


\begin{tabular} {|l|l|l|}
\hline
\tableheader
\hline
\columnheaders
\hline 
aspirin	&   C&9 \\
\hline
aspirin	&   H&8 \\
\hline
aspirin	&   O&4 \\
\hline 	 
sodium chloride & Na & \\
\hline
sodium chloride & Cl & \\
\hline
water &	H	& 2\\
\hline
water &	O	&  \\
\hline
\end{tabular}
}
		 
	
}
\end{tabular}

\end{frame}




\begin{frame}{Logical Data Specification of the Chemical Elements as a Directed Graph}
\scalebox{0.9}{


$
\begin{array}{c c c p{3.5cm} l}
        &              &          &&\attrtype{\Rnode{symboltype}{\CDsymboltype}  }         \\ [0.25cm]
        &\etype{\Rnode{element}{chemical\ element}\Rnode{elementR}{}}\kern-1.5cm&  &&                 \\ [0.1cm]
		&              &      &&\attrtype{\Rnode{atomicnumbertypeL}{\CDatomicnumbertype}}   \\ [0.75cm]
        &              &                        &&\attrtype{\Rnode{nametype}{\CDnametype}}   \\ [0.45cm]
        &              & \etype{\Rnode{allotrope}{allotrope}\Rnode{allotropeR}{}}    &&      \\ [0.45cm]
        &              &                        &&\attrtype{\Rnode{floattype}{\CDfloattype}} \\ [0.75cm]
\etype{\Rnode{valency}{valency}\Rnode{valencyR}{}}\kern-1.5cm& &&&\attrtype{\Rnode{valencynumbertype}{\CDvalencynumbertype}}\\
\end{array}
$
\newcommand{\CDvalencyuparrow}{
\setlength{\arrnodesepA}{7pt}
\setlength{\arrnodesepB}{8pt}
\setlength{\arroffsetA}{0pt}
\setlength{\arroffsetB}{7pt}
\ncarr[10]{valency}{element}
}
\CDvalencyuparrow
\newcommand{\CDallotropeuparrow}{
\setlength{\arrnodesepB}{7pt}
\setlength{\arroffsetA}{0pt}
\setlength{\arroffsetB}{0pt}
\ncarr[-5]{allotrope}{element}
}
\CDallotropeuparrow
\newcommand{\CDsymbolattribute}{
\setlength{\arroffsetB}{0pt}
\setlength{\arrnodesepB}{3pt}
\setlength{\arroffsetB}{-2pt}
\setlength{\arroffsetA}{5pt}
\ncarr[15]{elementR}{symboltype}
}
\CDsymbolattribute
\alabel{symbol}[0.4][0]



\newcommand{\CDatomicnumberattribute}{
\setlength{\arroffsetA}{0pt}
\setlength{\arrnodesepB}{3pt}
\setlength{\arroffsetB}{0pt}
\ncarr[5]{elementR}{atomicnumbertypeL}
}
\CDatomicnumberattribute
\alabel{atomic\,number}[0.4]

\newcommand{\CDelementnameattribute}{
\setlength{\arroffsetA}{-5pt}
\setlength{\arroffsetB}{4pt}
\ncarr[5]{elementR}{nametype}
}
\CDelementnameattribute
\alabel{name}[0.3]

\newcommand{\CDallotropenameattribute}{
\setlength{\arroffsetA}{2pt}
\setlength{\arroffsetB}{-4pt}
\ncarr[5]{allotropeR}{nametype}
}
\CDallotropenameattribute
\alabel{name}[0.3]

\newcommand{\CDmeltingpointattribute}{
\setlength{\arroffsetA}{-2pt}
\setlength{\arroffsetB}{2pt}
\ncarr[5]{allotropeR}{floattype}
}
\CDmeltingpointattribute
\alabel{melting\,point}[0.35][-1]

\newcommand{\CDboilingpointattribute}
{\setlength{\arroffsetA}{-6pt}
\setlength{\arroffsetB}{-2pt}
\ncarr[-5]{allotropeR}{floattype}
}
\CDboilingpointattribute
\blabel{boiling\,point}[0.35][-1]

\newcommand{\CDnumberattribute}
{\setlength{\arroffsetA}{2pt}
\setlength{\arroffsetB}{-2pt}
\ncarr[5]{valencyR}{valencynumbertype}
}
\CDnumberattribute
\alabel{number}[0.3]

%barring of mono sources i.e. uniqueness constraints start here
\pause
% type chemical element
\CDsymbolattribute
\addedgebar

% type valency
\CDvalencyuparrow
\addedgebar
\CDnumberattribute
\addedgebar

% type allotrope
\CDallotropeuparrow
\addedgebar
\CDallotropenameattribute
\addedgebar

% Second uniqueness constraint for type chemical element- use double bar
\CDatomicnumberattribute
\addedgedoublebar
% Third uniqueness constraint for chemical element - use triplebar
\CDelementnameattribute
\addedgetriplebar
} 
\end{frame}








