
\documentclass[10pt,a4paper]{scrartcl}
\usepackage[margin=3cm]{geometry}
\usepackage{pstricks}
\usepackage{pst-node}
\usepackage{pst-tree}
\usepackage{stmaryrd}
\usepackage{amsmath}
\usepackage{amssymb}
\usepackage{verbatim}
\usepackage{enumerate}
\usepackage{calc}
\usepackage{url}
\usepackage{hyperref}
\usepackage{environ}
\NewEnviron{shrunkdisplay}{%
\begin{equation*}
\scalebox{0.9}{$\BODY$}
\end{equation*}
}

\usepackage{amsthm} % added 7th April 2018
% theorems.macros.tex

\newtheorem{theorem}{Theorem}[section]
\newtheorem{observation}[theorem]{Observation}
\newtheorem{lemma}[theorem]{Lemma}
\newtheorem{proposition}[theorem]{Proposition}
\newtheorem{corollary}[theorem]{Corollary}
\newtheorem{conjecture}[theorem]{Conjecture}
\newtheorem{numbereddefinition}[theorem]{Definition}

\newenvironment{definition}[1][Definition]{\begin{trivlist}
\item[\hskip \labelsep {\bfseries #1}]}{\end{trivlist}}
\newenvironment{examples}[1][Examples]{\begin{trivlist}
\item[\hskip \labelsep {\bfseries #1}]}{\end{trivlist}}
\newenvironment{example}[1][Example]{\begin{trivlist}
\item[\hskip \labelsep {\bfseries #1}]}{\end{trivlist}}
\newenvironment{remark}[1][Remark]{\begin{trivlist}
\item[\hskip \labelsep {\bfseries #1}]}{\end{trivlist}}

\newenvironment{tageqn}[1]
{
\begin{equation}
\stepcounter{equation}
\label{#1}
\tag{\theequation --#1}
}
{
\end{equation}
}

\newenvironment{axiom}[1]
{
\begin{equation}
\label{#1}
\tag{#1}
}
{
\end{equation}
}

% when the tag is required different from the label eg when has math symbols can use:
\newenvironment{axiomtagged}[2]
{
\begin{equation}
\label{#1}
\tag{#2}
}
{
\end{equation}
}

%visible label
\newcommand{\vlabel}[2][]{\label{#2}#1(\textit{#2}):}






%ccategories.macros.tex 

% Macros for diagrams in contextual categories and related categories

\usepackage{twoopt}
\usepackage{scalerel} 
\usepackage{xargs}

%\usepackage{mathabx}  %Caused font problems
%\usepackage{MnSymbol}  % caused font problems

\newcommand{\conu}
{\mathbf{C}(U)}

\newcommand{\depu}
{\mathbf{D}(U)}


\newcommand{\reqt}{\textbf{R}}
\newcommand{\reqtc}[1][\catc]{\reqt_{#1}}
\newcommand{\reqtcp}[1][\catcp]{\reqt_{#1}}



\newcommand{\cat}[1]{\textbf{#1}}

\newcommand{\catc}{\cat{C}}
\newcommand{\catcw}{\cat{C}\ }
\newcommand{\catcp}[1][C]{\textbf{#1}'}
\newcommand{\catcpp}[1][C]{\textbf{#1}''}
\newcommand{\obj}[1]{\ensuremath{|\cat{#1}|}}
\newcommand{\ccat}[1][C]{\ensuremath{\mathbb{#1}} }
\newcommand{\ccatc}{contextual category \ccat}
\newcommand{\cobj}[2][]{\ensuremath{|\ccat[#2]|_{#1}}}
\newcommand{\cslice}[2]{\ensuremath{\ccat[#1]_{#2}}}
\newcommand{\csliceobj}[3][]{\ensuremath{|\mathbb{#2}_{#3}|_{#1} }}
\newcommand{\varset}[1][]{\ensuremath{V_{#1} }}
\newcommand{\localvarsets}{\ensuremath{\mathcal{V} }}
\newcommand{\Fam}{\ensuremath{\mathbb{F\mathrm{am}} }}
\newcommand{\Fin}{\ensuremath{\textbf{Fin}} }
\newcommand{\Finp}{\ensuremath{\textbf{Finp}} }
\newcommand{\Po}{\ensuremath{\textbf{Po}} }
\newcommand{\Famslice}[1]{\ensuremath{\mathbb{F\mathrm{am}}_{#1} }}
\newcommand{\Famobj}[1][]{\ensuremath{|\mathbb{F\mathrm{am}}|_{#1} }}
\newcommand{\Famsliceobj}[2][]{\ensuremath{|\mathbb{F\mathrm{am}}_{#2}|_{#1} }}
\newcommand{\morph}{\rightarrow}
\newcommand{\epi}{\twoheadrightarrow}
\newcommand{\base}{\triangleleft}
\newcommand{\comp}{\circ}
\newcommand{\cross}{\otimes}
\newcommand{\pc}[2]{d^{#1}_{#2}}
\newcommand{\sub}{^*}
\newcommand{\diag}{\delta}
\newcommand{\pbase}[1]{\tilde{#1}}
\newcommand{\tuple}[1]{\langle#1\rangle}
\newcommand{\ndidly}{\ensuremath{\Join_n}}

\newcommand{\product}[1]{\bigtimes_{#1}}
\newcommand{\productn}{\product{n}}
\newcommand{\crossx}[3]{#1 \underset{#3}{\cross} #2}
\newcommand{\fibrex}[3]{#1 \underset{#3}{\Join} #2}
\newcommand{\powerset}{\mathcal{P}}
\newcommand{\primeds}[1]{
\ensuremath{\mathcal{P}(#1)} }
\newcommand{\compset}{\ \dot{\circ}\, }

% darrow
%\newcommand{\darrow}{\rightarrowtriangle} %use \smorph instead
\newcommand{\smorph}{\rightarrowtriangle}

 
\newcommand\dhead{\scaleobj{0.6}{\triangleright}}
%\newcommand{\dmorph}{\, \mbox{---} \! \cdot \! \raisebox{1.1pt}{\dhead}}    % dot style
\newcommand{\dmorph}{\, \mbox{---}\kern-1pt\raisebox{1.1pt}{\dhead\kern-1.75pt\dhead}}\,     % double triangle style

% projection tree
%\newcommand{\proj}[2]{proj_{#2}(#1)}

\newcommand{\proj}[2]{
\ensuremath{\mathcal{P}_{#2}(#1)} }

%pstrick supplements for arrows

\newlength{\arrnodesepA}
\newlength{\arrnodesepB}
\newlength{\arroffsetA}
\newlength{\arroffsetB}

%Modified to 2pt from 0pt on 23 July 2018
\newcommand{\arreset}{
\setlength{\arrnodesepA}{2pt}
\setlength{\arrnodesepB}{2pt}
\setlength{\arroffsetA}{0pt}
\setlength{\arroffsetB}{0pt}
}
\arreset

\newcommand{\ncarr}[3][0]{\ncarc[arcangle=#1,nodesepA=\arrnodesepA,nodesepB=\arrnodesepB,offsetA=\arroffsetA,offsetB=\arroffsetB,arrowsize=5pt,arrowinset=0.7]{->}{#2}{#3}}
\newcommand{\ncdarr}[3][0]{\ncarc[linestyle=dashed,arcangle=#1,nodesepA=\arrnodesepA,nodesepB=\arrnodesepB,offsetA=\arroffsetA,offsetB=\arroffsetB,arrowsize=5pt,arrowinset=0.7]{->}{#2}{#3}}
\newcommand{\jcbarr}[4][0]{ % ncbarr is defined in some thridy party package so do not use!\emph{}
\ncarr[#1]{#3}{#4}
\nbput[labelsep=2pt]{\footnotesize $#2$}
}

\newcommand{\ncaarr}[4][0]{
\ncarr[#1]{#3}{#4}
\naput[labelsep=2pt]{\footnotesize $#2$}
}

% \alabel{label}[npos][labelsep_pts]
\newcommandx*\alabel[3][2=0.5,3=2,usedefault]{\naput[labelsep=#3pt,npos=#2]{\footnotesize $#1$}}
% \blabel{label}[npos][labelsep_pts]
\newcommandx*\blabel[3][2=0.5,3=2,usedefault]{\nbput[labelsep=#3pt,npos=#2]{\footnotesize $#1$}}


\newif \ifbars
% to supress display of bars use \barsfalse to swith them on use \barstrue
\barstrue 
% \idcomp mark an arrow as one component of an identifier
\newcommand{\idcomp}{\ifbars{\ncput[npos=0, nrot=:U]{\psline(0.2,-0.075)(0.2,0.075)}}\fi}  %add a bar to a node connection arrow
% pstrick supplements for s-arrows (previous name for d-arrow - should convert}

\newlength{\sarnodesepA}
\newlength{\sarnodesepB}
\newlength{\saroffsetA}
\newlength{\saroffsetB}
\newlength{\sarnodesepAsav}
\newlength{\sarnodesepBsav}

\newcommand{\sarreset}{
\setlength{\sarnodesepA}{0pt}
\setlength{\sarnodesepB}{0pt}
\setlength{\saroffsetA}{0pt}
\setlength{\saroffsetB}{0pt}
}

\sarreset

% sar - S-arrow
\newcommand{\ncsar}[3][0]{
\setlength{\sarnodesepAsav}{\sarnodesepA}
\setlength{\sarnodesepBsav}{\sarnodesepB}
\addtolength{\sarnodesepA}{3pt}
\addtolength{\sarnodesepB}{7pt}
\ncarc[nodesepA=\sarnodesepA,nodesepB=\sarnodesepB,offsetA=\saroffsetA,offsetB=\saroffsetB,arcangle=#1]{-}{#2}{#3}
\ncput[nrot=:R,npos=1]{\pstriangle(0,0)(.2,.2)}
\setlength{\sarnodesepA}{\sarnodesepAsav}
\setlength{\sarnodesepB}{\sarnodesepBsav}
}


% bsar - below labelled S-arrow
\newcommand{\ncbsar}[4][0]{
\ncsar[#1]{#3}{#4}
\nbput[labelsep=2pt]{\footnotesize $#2$}
}
% asar - above labelled S-arrow
\newcommand{\ncasar}[4][0]{
\ncsar[#1]{#3}{#4}
\naput[labelsep=2pt]{\footnotesize $#2$}
}

% OLD cdar - composite dependency arrow - dot tyle
\iffalse
\newcommand{\nccdar}[3][0]{
\setlength{\sarnodesepAsav}{\sarnodesepA}
\setlength{\sarnodesepBsav}{\sarnodesepB}
\addtolength{\sarnodesepA}{3pt}
\addtolength{\sarnodesepB}{11pt}
\ncarc[nodesepA=\sarnodesepA,nodesepB=\sarnodesepB,offsetA=\saroffsetA,offsetB=\saroffsetB,arcangle=#1]{-}{#2}{#3}
\ncput[nrot=:R,npos=1]{\pstriangle(0,0.1)(.2,.2)}
\ncput[nrot=:R,npos=1]{\psdot[dotsize=1pt](-0.0075,0.05)}   %!!
\setlength{\sarnodesepA}{\sarnodesepAsav}
\setlength{\sarnodesepB}{\sarnodesepBsav}
}
\fi

% cdar - composite dependency arrow Mark II - double trangle style
\newcommand{\nccdar}[3][0]{
\setlength{\sarnodesepAsav}{\sarnodesepA}
\setlength{\sarnodesepBsav}{\sarnodesepB}
\addtolength{\sarnodesepA}{3pt}
\addtolength{\sarnodesepB}{13pt}
\ncarc[nodesepA=\sarnodesepA,nodesepB=\sarnodesepB,offsetA=\saroffsetA,offsetB=\saroffsetB,arcangle=#1]{-}{#2}{#3}
\ncput[nrot=:R,npos=1]{\pstriangle(0,0)(.2,.2)}
\ncput[nrot=:R,npos=1]{\pstriangle(0,0.2)(.2,.2)}
\setlength{\sarnodesepA}{\sarnodesepAsav}
\setlength{\sarnodesepB}{\sarnodesepBsav}
}


% bcdar - below labelled composite dependency arrow
\newcommand{\ncbcdar}[4][0]{
\nccdar[#1]{#3}{#4}
\nbput[labelsep=2pt]{\footnotesize $#2$}
}
% acdar - above labelled composite dependency arrow
\newcommand{\ncacdar}[4][0]{
\nccdar[#1]{#3}{#4}
\naput[labelsep=2pt]{\footnotesize $#2$}
}


% rsar - recursive S-arrow
\newcommand{\ncrsar}[2]{
\setlength{\sarnodesepAsav}{\sarnodesepA}
\setlength{\sarnodesepBsav}{\sarnodesepB}
\addtolength{\sarnodesepA}{3pt}
\addtolength{\sarnodesepB}{7pt}
\ncloop[nodesepA=\sarnodesepA,nodesepB=\sarnodesepB,
        offsetA=\saroffsetA,offsetB=\saroffsetB,
        armA=0.7cm,armB=0.6cm,angleA=90,angleB=-90,loopsize=-1,linearc=0.4
				]{-}{#1}{#2}
\ncput[nrot=:R,npos=5]{\pstriangle(0,0)(.2,.2)}
\setlength{\sarnodesepA}{\sarnodesepAsav}
\setlength{\sarnodesepB}{\sarnodesepBsav}
}

% pstrick supplements for multi-arrows

\newlength{\marnodesepA}
\newlength{\marnodesepB}
\newlength{\maroffsetB}
\newlength{\marnodesepBsav}

\newcommand{\marreset}{
\setlength{\marnodesepA}{0pt}
\setlength{\marnodesepB}{0pt}
\setlength{\maroffsetB}{0pt}
}

\marreset

%ncmarr[#1 arcangle1][#2 arcangle2]{#3 name}{#4 domain1}{#5 domain2}{#6 junction}{#7 codomain}
\newcommandtwoopt{\ncmarr}[6][8][8]{%
\ncarc[nodesepA=\marnodesepA,nodesepB=0,arcangle=#1]{-}{#3}{#5}
\ncarc[nodesepB=0,arcangle=-#1]{-}{#4}{#5}
\ncarc[arcangle=#2,nodesepB=\marnodesepB,offsetB=\maroffsetB]{->}{#5}{#6}
}%


\newcommandtwoopt{\nchmarr}[6][8][8]{%
\ncarc[nodesepA=\marnodesepA,nodesepB=0,arcangle=#1]{-}{#3}{#5}
\ncarc[nodesepB=0,arcangle=#1]{-}{#4}{#5}
\ncarc[arcangle=#2,nodesepB=\marnodesepB,offsetB=\maroffsetB]{->}{#5}{#6}
}%

\newcommandtwoopt{\ncamarr}[7][8][8]{%
\ncmarr[#1][#2]{#4}{#5}{#6}{#7}
\naput[npos=.05]{$#3$}
}%
\newcommandtwoopt{\ncbmarr}[7][8][8]{%
\ncmarr[#1][#2]{#4}{#5}{#6}{#7}
\nbput[npos=.05]{$#3$}
}%

\newcommandtwoopt{\ncbhmarr}[7][8][8]{%
\nchmarr[#1][#2]{#4}{#5}{#6}{#7}
\nbput[npos=.05]{$#3$}
}%

\newcommandtwoopt{\ncmarrr}[7][8][8]{
\ncarc[nodesepB=0,arcangle=#1]{-}{#3}{#6}
\ncline[nodesepB=0]{-}{#4}{#6}
\ncarc[nodesepB=0,arcangle=-#1]{-}{#5}{#6}
\ncarc[nodesepA=0,arcangle=#2]{->}{#6}{#7}
}

\newcommandtwoopt{\ncamarrr}[8][8][8]{
\ncmarrr[#1][#2]{#4}{#5}{#6}{#7}{#8}
\naput[npos=.05]{$#3$}
}
\newcommandtwoopt{\ncbmarrr}[8][8][8]{
\ncmarrr[#1][#2]{#4}{#5}{#6}{#7}{#8}
\nbput[npos=.05]{$#3$}
}


% 6 June 2020
% Edges representing attributes and relationship graphs
%  Ep   - partial
%  Epm  - partial mono
%  Epe  - partial epi
%  Epme - partial mono epi
%  Et   - total
%  Etm  - total mono
%  Ete  - total epi
%  Etme - total mono epi
%  recursive edges (use nccircle)
%  rEp   - partial
%  rEpm  - partial mono
%  rEpe  - partial epi
%  rEpme - partial mono epi
%  rEt   - total
%  rEtm  - total mono
%  rEte  - total epi
%  rEtme - total mono epi

\newcounter{EangleA}
\newcounter{EangleB}
\newcounter{EmidangleA}
\newcounter{EmidangleB}

% Ep - Edge partial
\newcommandtwoopt{\Ep}[4][0][0]{
\crowsfootedEdge{#1}{#2}{#3}{#4}{dashed}{dashed}
}



% Epm - Edge partial mono
\newcommandtwoopt{\Epm}[4][0][0]{
\monoEdge{#1}{#2}{#3}{#4}{dashed}{dashed}
}


% Epe - Edge partial epi
\newcommandtwoopt{\Epe}[4][0][0]{
\crowsfootedEdge{#1}{#2}{#3}{#4}{dashed}{solid}
}

% Epme - Edge partial mono epi
\newcommandtwoopt{\Epme}[4][0][0]{
\monoEdge{#1}{#2}{#3}{#4}{dashed}{solid}
}

% Et - Edge total
\newcommandtwoopt{\Et}[4][0][0]{
\crowsfootedEdge{#1}{#2}{#3}{#4}{solid}{dashed}
}

% Etm - Edge total mono
\newcommandtwoopt{\Etm}[4][0][0]{
\monoEdge{#1}{#2}{#3}{#4}{solid}{dashed}
}

% Ete - Edge total epi
\newcommandtwoopt{\Ete}[4][0][0]{
\crowsfootedEdge{#1}{#2}{#3}{#4}{solid}{solid}
}

% Etme - Edge total mono epi
\newcommandtwoopt{\Etme}[4][0][0]{
\monoEdge{#1}{#2}{#3}{#4}{solid}{solid}
}

% crowsfootedEdge - \crowsfootedEdge[angleA][midpointangle]{startnode}{endnode}[startstyle][endstyle]
\newcommand{\crowsfootedEdge}[6]{
\setlength{\sarnodesepAsav}{\sarnodesepA}
\setlength{\sarnodesepBsav}{\sarnodesepB}
\addtolength{\sarnodesepA}{3pt}
\addtolength{\sarnodesepB}{3pt}
\setcounter{EangleA}{ #1 + #2}
\setcounter{EangleB}{180  - #1 + #2}
\setcounter{EmidangleA}{#2}
\setcounter{EmidangleB}{#2 + 180}
\nccurve[nodesepA=\sarnodesepA,nodesepB=\sarnodesepB,offsetA=\saroffsetA,offsetB=\saroffsetB,angleA=\theEangleA, angleB=\theEangleB,linestyle=none,linewidth=0]{->}{#3}{#4}
\ncput[nrot=:R,npos=0]{\psline(0,.1)(.075,0)}
\ncput[nrot=:R,npos=0]{\psline(0,.1)(-0.075,0)}
\ncput{\pnode(0,0){xxx}}
\nccurve[nodesepA=0,nodesepB=\sarnodesepB,offsetA=0,offsetB=\saroffsetB,angleA=\theEmidangleA, angleB=\theEangleB, linestyle=#6]{->}{xxx}{#4}
%the following provides context for any following label
\nccurve[nodesepA=\sarnodesepA,nodesepB=0,offsetA=\saroffsetA,offsetB=0,angleA=\theEangleA, angleB=\theEmidangleB,linestyle=#5]{-}{#3}{xxx}
\setlength{\sarnodesepA}{\sarnodesepAsav}
\setlength{\sarnodesepB}{\sarnodesepBsav}
}

% monoEdge - \monoEdge[angleA][midpointangle]{startnode}{endnode}[startstyle][endstyle]
\newcommand{\monoEdge}[6]{ 
\setlength{\sarnodesepAsav}{\sarnodesepA}
\setlength{\sarnodesepBsav}{\sarnodesepB}
\addtolength{\sarnodesepA}{3pt}
\addtolength{\sarnodesepB}{3pt}
\setcounter{EangleA}{ #1 + #2}
\setcounter{EangleB}{180  - #1 + #2}
\setcounter{EmidangleA}{#2}
\setcounter{EmidangleB}{#2 + 180}
\nccurve[nodesepA=\sarnodesepA,nodesepB=\sarnodesepB,offsetA=\saroffsetA,offsetB=\saroffsetB,angleA=\theEangleA, angleB=\theEangleB,linestyle=none,linewidth=0]{->}{#3}{#4}
\ncput{\pnode(0,0){xxx}}
\nccurve[nodesepA=0,nodesepB=\sarnodesepB,offsetA=0,offsetB=\saroffsetB,angleA=\theEmidangleA, angleB=\theEangleB, linestyle=#6]{->}{xxx}{#4}
%the following provides context for any following label
\nccurve[nodesepA=\sarnodesepA,nodesepB=0,offsetA=\saroffsetA,offsetB=0,angleA=\theEangleA, angleB=\theEmidangleB,linestyle=#5]{-}{#3}{xxx}
\setlength{\sarnodesepA}{\sarnodesepAsav}
\setlength{\sarnodesepB}{\sarnodesepBsav}
}


\newcounter{EangleGiven}
\newcounter{EangleComplementary}
\newcounter{EangleStartCorrected}
\newcounter{EangleEndCorrected}


%  rEp   - recursive Edge partial
\newcommand{\rEp}[2][0]{
\setcounter{EangleGiven}{#1}
\setcounter{EangleStartCorrected}{#1-10} %correction required because for nccurve unlike nccircle angle measured at boundary not at centre of node
\setcounter{EangleEndCorrected}{#1+180+10} %correction required because angle measured at boundary not at centre of node
\setcounter{EangleComplementary}{#1 + 180}
\nccircle[angleA=\theEangleComplementary, nodesep=0pt, linestyle=none]{-}{#2}{.4cm} % an invisible circle to hang the midpoint from
\ncput{\pnode(0,0){midpoint}}                                         
\nccurve[nodesepA=1pt,nodesepB=0pt,offsetA=0pt,offsetB=0pt,angleA=\theEangleStartCorrected, angleB=\theEangleGiven, ncurv=1.359, linecolor=black, linestyle=dashed]{-}{#2}{midpoint}
\ncput[nrot=:R,npos=0]{\psline(0,.1)(.075,0)}
\ncput[nrot=:R,npos=0]{\psline(0,.1)(-0.075,0)}
\nccurve[nodesepA=0pt,nodesepB=2pt,offsetA=0pt,offsetB=0pt,angleA=\theEangleComplementary, angleB=\theEangleEndCorrected, ncurv=1.359, linestyle=dashed]{-}{midpoint}{#2}
% 1.359 is e/2 happenchance or algorithmically necessary???
% now draw arrowhead -- dont include in the nccurve because this alters the line position - a strange feature of pstruicks
\ncput[npos=0.9]{\pnode(0,0){yyy}}
\ncline{->}{yyy}{#2}
% repeat from earlier to provide context for label that might follow
\nccurve[nodesepA=1pt,nodesepB=0pt,offsetA=0pt,offsetB=0pt,angleA=\theEangleStartCorrected, angleB=\theEangleGiven, ncurv=1.359, linecolor=black, linestyle=dashed]{-}{#2}{midpoint} 
} 

%  rEpm  - recursive Edge partial mono
\newcommand{\rEpm}[2][0]{
\setcounter{EangleGiven}{#1}
\setcounter{EangleStartCorrected}{#1-10} %correction required because for nccurve unlike nccircle angle measured at boundary not at centre of node
\setcounter{EangleEndCorrected}{#1+180+10} %correction required because angle measured at boundary not at centre of node
\setcounter{EangleComplementary}{#1 + 180}
\nccircle[angleA=\theEangleComplementary, nodesep=0pt, linestyle=none]{-}{#2}{.4cm} % an invisible circle to hang the midpoint from
\ncput{\pnode(0,0){midpoint}}   
\nccurve[nodesepA=0pt,nodesepB=2pt,offsetA=0pt,offsetB=0pt,angleA=\theEangleComplementary, angleB=\theEangleEndCorrected, ncurv=1.359, linestyle=dashed]{-}{midpoint}{#2}
% 1.359 is e/2 happenchance or algorithmically necessary???
% now draw arrowhead -- dont include in the nccurve because this alters the line position - a strange feature of pstruicks
\ncput[npos=0.9]{\pnode(0,0){yyy}}
\ncline{->}{yyy}{#2}
% last to provide context for label that might follow
\nccurve[nodesepA=1pt,nodesepB=0pt,offsetA=0pt,offsetB=0pt,angleA=\theEangleStartCorrected, angleB=\theEangleGiven, ncurv=1.359, linecolor=black, linestyle=dashed]{-}{#2}{midpoint} 
}

%  rEpe  - recursive Edge partial epi
\newcommand{\rEpe}[2][0]{
\setcounter{EangleGiven}{#1}
\setcounter{EangleStartCorrected}{#1-10} %correction required because for nccurve unlike nccircle angle measured at boundary not at centre of node
\setcounter{EangleEndCorrected}{#1+180+10} %correction required because angle measured at boundary not at centre of node
\setcounter{EangleComplementary}{#1 + 180}
\nccircle[angleA=\theEangleComplementary, nodesep=0pt, linestyle=none]{-}{#2}{.4cm} % an invisible circle to hang the midpoint from
\ncput{\pnode(0,0){midpoint}}                                         
\nccurve[nodesepA=1pt,nodesepB=0pt,offsetA=0pt,offsetB=0pt,angleA=\theEangleStartCorrected, angleB=\theEangleGiven, ncurv=1.359, linecolor=black, linestyle=dashed]{-}{#2}{midpoint}
\ncput[nrot=:R,npos=0]{\psline(0,.1)(.075,0)}
\ncput[nrot=:R,npos=0]{\psline(0,.1)(-0.075,0)}
\nccurve[nodesepA=0pt,nodesepB=2pt,offsetA=0pt,offsetB=0pt,angleA=\theEangleComplementary, angleB=\theEangleEndCorrected, ncurv=1.359]{-}{midpoint}{#2}
% 1.359 is e/2 happenchance or algorithmically necessary???
% now draw arrowhead -- dont include in the nccurve because this alters the line position - a strange feature of pstruicks
\ncput[npos=0.9]{\pnode(0,0){yyy}}
\ncline{->}{yyy}{#2}
% repeat from earlier to provide context for label that might follow
\nccurve[nodesepA=1pt,nodesepB=0pt,offsetA=0pt,offsetB=0pt,angleA=\theEangleStartCorrected, angleB=\theEangleGiven, ncurv=1.359, linecolor=black, linestyle=dashed]{-}{#2}{midpoint} 
}

%  rEpme - recursive Edge partial mono epi
\newcommand{\rEpme}[2][0]{
\setcounter{EangleGiven}{#1}
\setcounter{EangleStartCorrected}{#1-10} %correction required because for nccurve unlike nccircle angle measured at boundary not at centre of node
\setcounter{EangleEndCorrected}{#1+180+10} %correction required because angle measured at boundary not at centre of node
\setcounter{EangleComplementary}{#1 + 180}
\nccircle[angleA=\theEangleComplementary, nodesep=0pt, linestyle=none]{-}{#2}{.4cm} % an invisible circle to hang the midpoint from
\ncput{\pnode(0,0){midpoint}}                                         
%\nccurve[nodesepA=0pt,nodesepB=0pt,offsetA=0pt,offsetB=0pt,angleA=\theEangleComplementary, angleB=\theEangleEndCorrected, ncurv=1.359, linestyle=dashed]{->}{xxx}{#2}
\nccurve[nodesepA=0pt,nodesepB=2pt,offsetA=0pt,offsetB=0pt,angleA=\theEangleComplementary, angleB=\theEangleEndCorrected, ncurv=1.359]{-}{midpoint}{#2}
% 1.359 is e/2 happenchance or algorithmically necessary???
% now draw arrowhead -- dont include in the nccurve because this alters the line position - a strange feature of pstruicks
\ncput[npos=0.9]{\pnode(0,0){yyy}}
\ncline{->}{yyy}{#2}
% last so that to provide context for label that might follow
\nccurve[nodesepA=1pt,nodesepB=0pt,offsetA=0pt,offsetB=0pt,angleA=\theEangleStartCorrected, angleB=\theEangleGiven, ncurv=1.359, linecolor=black, linestyle=dashed]{-}{#2}{midpoint} 
}

% rEt - recursive Edge total
\newcommand{\rEt}[2][0]{
\setcounter{EangleGiven}{#1}
\setcounter{EangleStartCorrected}{#1-10} %correction required because for nccurve unlike nccircle angle measured at boundary not at centre of node
\setcounter{EangleEndCorrected}{#1+180+10} %correction required because angle measured at boundary not at centre of node
\setcounter{EangleComplementary}{#1 + 180}
\nccircle[angleA=\theEangleComplementary, nodesep=0pt, linestyle=none]{-}{#2}{.4cm} % an invisible circle to hang the midpoint from
\ncput{\pnode(0,0){midpoint}}                                         
\nccurve[nodesepA=1pt,nodesepB=0pt,offsetA=0pt,offsetB=0pt,angleA=\theEangleStartCorrected, angleB=\theEangleGiven, ncurv=1.359, linecolor=black]{-}{#2}{midpoint}
\ncput[nrot=:R,npos=0]{\psline(0,.1)(.075,0)}
\ncput[nrot=:R,npos=0]{\psline(0,.1)(-0.075,0)}
%\nccurve[nodesepA=0pt,nodesepB=0pt,offsetA=0pt,offsetB=0pt,angleA=\theEangleComplementary, angleB=\theEangleEndCorrected, ncurv=1.359, linestyle=dashed]{->}{xxx}{#2}
\nccurve[nodesepA=0pt,nodesepB=2pt,offsetA=0pt,offsetB=0pt,angleA=\theEangleComplementary, angleB=\theEangleEndCorrected, ncurv=1.359, linestyle=dashed]{-}{midpoint}{#2}
% 1.359 is e/2 happenchance or algorithmically necessary???
% now draw arrowhead -- dont include in the nccurve because this alters the line position - a strange feature of pstruicks
\ncput[npos=0.9]{\pnode(0,0){yyy}}
\ncline{->}{yyy}{#2}
% repeat from earlier to provide context for label that might follow
\nccurve[nodesepA=1pt,nodesepB=0pt,offsetA=0pt,offsetB=0pt,angleA=\theEangleStartCorrected, angleB=\theEangleGiven, ncurv=1.359, linecolor=black]{-}{#2}{midpoint} 
}

%  rEtm  - recursive Edge total mono
\newcommand{\rEtm}[2][0]{
\setcounter{EangleGiven}{#1}
\setcounter{EangleStartCorrected}{#1-10} %correction required because for nccurve unlike nccircle angle measured at boundary not at centre of node
\setcounter{EangleEndCorrected}{#1+180+10} %correction required because angle measured at boundary not at centre of node
\setcounter{EangleComplementary}{#1 + 180}
\nccircle[angleA=\theEangleComplementary, nodesep=0pt, linestyle=none]{-}{#2}{.4cm} % an invisible circle to hang the midpoint from
\ncput{\pnode(0,0){midpoint}}     
\nccurve[nodesepA=0pt,nodesepB=2pt,offsetA=0pt,offsetB=0pt,angleA=\theEangleComplementary, angleB=\theEangleEndCorrected, ncurv=1.359, linestyle=dashed]{-}{midpoint}{#2}
% 1.359 is e/2 happenchance or algorithmically necessary???
% now draw arrowhead -- dont include in the nccurve because this alters the line position - a strange feature of pstruicks
\ncput[npos=0.9]{\pnode(0,0){yyy}}
\ncline{->}{yyy}{#2}
% last to provide context for label that might follow
\nccurve[nodesepA=1pt,nodesepB=0pt,offsetA=0pt,offsetB=0pt,angleA=\theEangleStartCorrected, angleB=\theEangleGiven, ncurv=1.359, linecolor=black]{-}{#2}{midpoint} 
}

%  rEte  - total epi
\newcommand{\rEte}[2][0]{
\setcounter{EangleGiven}{#1}
\setcounter{EangleStartCorrected}{#1-10} %correction required because for nccurve unlike nccircle angle measured at boundary not at centre of node
\setcounter{EangleEndCorrected}{#1+180+10} %correction required because angle measured at boundary not at centre of node
\setcounter{EangleComplementary}{#1 + 180}
\nccircle[angleA=\theEangleComplementary, nodesep=0pt, linestyle=none]{-}{#2}{.4cm} % an invisible circle to hang the midpoint from
\ncput{\pnode(0,0){midpoint}}                                         
\nccurve[nodesepA=1pt,nodesepB=0pt,offsetA=0pt,offsetB=0pt,angleA=\theEangleStartCorrected, angleB=\theEangleGiven, ncurv=1.359, linecolor=black]{-}{#2}{midpoint}
\ncput[nrot=:R,npos=0]{\psline(0,.1)(.075,0)}
\ncput[nrot=:R,npos=0]{\psline(0,.1)(-0.075,0)}
%\nccurve[nodesepA=0pt,nodesepB=0pt,offsetA=0pt,offsetB=0pt,angleA=\theEangleComplementary, angleB=\theEangleEndCorrected, ncurv=1.359, linestyle=dashed]{->}{xxx}{#2}
\nccurve[nodesepA=0pt,nodesepB=2pt,offsetA=0pt,offsetB=0pt,angleA=\theEangleComplementary, angleB=\theEangleEndCorrected, ncurv=1.359]{-}{midpoint}{#2}
% 1.359 is e/2 happenchance or algorithmically necessary???
% now draw arrowhead -- dont include in the nccurve because this alters the line position - a strange feature of pstruicks
\ncput[npos=0.9]{\pnode(0,0){yyy}}
\ncline{->}{yyy}{#2}
% repeat from earlier to provide context for label that might follow
\nccurve[nodesepA=1pt,nodesepB=0pt,offsetA=0pt,offsetB=0pt,angleA=\theEangleStartCorrected, angleB=\theEangleGiven, ncurv=1.359, linecolor=black]{-}{#2}{midpoint} 
}

%  rEtme - recursive Edge total mono epi

\newcommand{\rEtme}[2][0]{
\setcounter{EangleGiven}{#1}
\setcounter{EangleStartCorrected}{#1-10} %correction required because for nccurve unlike nccircle angle measured at boundary not at centre of node
\setcounter{EangleEndCorrected}{#1+180+10} %correction required because angle measured at boundary not at centre of node
\setcounter{EangleComplementary}{#1 + 180}
\nccircle[angleA=\theEangleComplementary, nodesep=0pt, linestyle=none]{-}{#2}{.4cm} % an invisible circle to hang the midpoint from
\ncput{\pnode(0,0){midpoint}}     
\nccurve[nodesepA=0pt,nodesepB=2pt,offsetA=0pt,offsetB=0pt,angleA=\theEangleComplementary, angleB=\theEangleEndCorrected, ncurv=1.359]{-}{midpoint}{#2}
% 1.359 is e/2 happenchance or algorithmically necessary???
% now draw arrowhead -- dont include in the nccurve because this alters the line position - a strange feature of pstruicks
\ncput[npos=0.9]{\pnode(0,0){yyy}}
\ncline{->}{yyy}{#2}
% last to provide context for label that might follow
\nccurve[nodesepA=1pt,nodesepB=0pt,offsetA=0pt,offsetB=0pt,angleA=\theEangleStartCorrected, angleB=\theEangleGiven, ncurv=1.359, linecolor=black]{-}{#2}{midpoint} 
}

%gats.macros.tex

\usepackage{environ}    % also used in ermacros % here used for \NewEnvrion

\newcommand{\gat}[1][U]{
\ensuremath{\mathcal{#1}}}  % used to hav a space in here
\newcommand{\gatw}[1][U]{\gat[#1]\ }  % use this if need trailing space
\newcommand{\ingat}[1][U]{in \gat[#1]}
\newcommand{\isagat}[1][U]{\gat[#1] is a g.a.t.}
\newcommand{\inagat}{in a g.a.t. }

% macro for a generic theory
%\newcommand{\theory}
%{\textit{U}}

\newcommand{\intheory}
{is a derived rule of \gat[U]}

% Macros for GAT rules

\newcommand{\isT}[1]
{#1\mbox{ is a type}}

\newcommand{\ofT}[2]
{#1 \in #2
}

% Macros for GAT rules   <!-- new old -->
\newcommand{\istype}[1]
{#1\mbox{ is a type}}

\newcommand{\oftype}[2]
{#1 \in #2
}

%\context{x}{\Delta}{n}
\newcommand{\context}[3]
{\ofT{#1_1}{#2_1},... \ofT{#1_{#3}}{#2_{#3}(#1_1,...#1_{#3-1})}
}

%\subcontext{x}{\Delta}{i}{k}
\newcommand{\subcontext}[4]
{\ofT{#1_{#3_1}}{#2_{#3_1}},... \ofT{#1_{#3_#4}}{#2_{#3_#4}(#1_1,...#1_{#3_#4-1})}
}

% #schematic context
\newcommand{\schmcon}[3]
{\ofT{#1_1}{#2_1},... \ofT{#1_{#3}}{#2_{#3}}
}
% abbreviated to
\newcommand{\con}[3]
{\schmcon{#1}{#2}{#3}}

% schematic subcontext
%\subcon{x}{\Delta}{i}{k}
\newcommand{\subcon}[4]
{\ofT{#1_{#3_1}}{#2_{#3_1}},... \ofT{#1_{#3_#4}}{#2_{#3_#4}}
}

% permuted context
%\permcon{x}{\Delta}{n}{\sigma}
\newcommand{\permcon}[4]
{\ofT{#1_{#4(1)}}{#2_{#4(1)}},... \ofT{#1_{#4(#3)}}{#2_{#4(#3)}}
}
% permuted term
%\permterm{t}{n}{\sigma}
\newcommand{\permterm}[3]
{
#1_{#3(1)},...#1_{#3(#2)}
}


% Idioms
\newcommand{\xDelta}[1]{\con{x}{\Delta}{#1}}
\newcommand{\xDeltap}[1]{\con{x}{\Delta'}{#1}}
\newcommand{\xOmega}[1]{\con{x}{\Omega}{#1}}
\newcommand{\xOmegap}[1]{\con{x}{\Omega'}{#1}}
\newcommand{\yOmega}[1]{\con{y}{\Omega}{#1}}
\newcommand{\yOmegap}[1]{\con{y}{\Omega'}{#1}}

\newcommand{\xDeltasigma}[1]{\permcon{x}{\Delta}{#1}{\sigma}}
\newcommand{\xDeltapsigma}[1]{\permcon{x}{\Delta'}{#1}{\sigma}}
\newcommand{\xOmegasigma}[1]{\permcon{x}{\Omega}{#1}{\sigma}}
\newcommand{\xOmegapsigma}[1]{\permcon{x}{\Omega'}{#1}{\sigma}}
\newcommand{\yOmegasigma}[1]{\permcon{y}{\Omega}{#1}{\sigma}}
\newcommand{\yOmegapsigma}[1]{\permcon{y}{\Omega'}{#1}{\sigma}}

\newcommand{\xDeltainvsigma}[1]{\permcon{x}{\Delta}{#1}{\sigma^{-1}}}
\newcommand{\xDeltapinvsigma}[1]{\permcon{x}{\Delta'}{#1}{\sigma^{-1}}}
\newcommand{\xOmegainvsigma}[1]{\permcon{x}{\Omega}{#1}{\sigma^{-1}}}
\newcommand{\xOmegapinvsigma}[1]{\permcon{x}{\Omega'}{#1}{\sigma^{-1}}}
\newcommand{\yOmegainvsigma}[1]{\permcon{y}{\Omega}{#1}{\sigma^{-1}}}
\newcommand{\yOmegapinvsigma}[1]{\permcon{y}{\Omega'}{#1}{\sigma^{-1}}}

%Idioms enclosed as tuples
\newcommand{\encxDelta}[1]{\tuple{\con{x}{\Delta}{#1}}}
\newcommand{\encxDeltap}[1]{\tuple{\con{x}{\Delta'}{#1}}}
\newcommand{\encxOmega}[1]{\tuple{\con{x}{\Omega}{#1}}}
\newcommand{\encxOmegap}[1]{\tuple{\con{x}{\Omega'}{#1}}}
\newcommand{\encyOmega}[1]{\tuple{\con{y}{\Omega}{#1}}}
\newcommand{\encyOmegap}[1]{\tuple{\con{y}{\Omega'}{#1}}}

\newcommand{\encxDeltasigma}[1]{\tuple{\permcon{x}{\Delta}{#1}{\sigma}}}
\newcommand{\encxDeltapsigma}[1]{\tuple{\permcon{x}{\Delta'}{#1}{\sigma}}}
\newcommand{\encxOmegasigma}[1]{\tuple{\permcon{x}{\Omega}{#1}{\sigma}}}
\newcommand{\encxOmegapsigma}[1]{\tuple{\permcon{x}{\Omega'}{#1}{\sigma}}}
\newcommand{\encyOmegasigma}[1]{\tuple{\permcon{y}{\Omega}{#1}{\sigma}}}
\newcommand{\encyOmegapsigma}[1]{\tuple{\permcon{y}{\Omega'}{#1}{\sigma}}}

\newcommand{\encxDeltainvsigma}[1]{\tuple{\permcon{x}{\Delta}{#1}{\sigma^{-1}}}}
\newcommand{\encxDeltapinvsigma}[1]{\tuple{\permcon{x}{\Delta'}{#1}{\sigma^{-1}}}}
\newcommand{\encxOmegainvsigma}[1]{\tuple{\permcon{x}{\Omega}{#1}{\sigma^{-1}}}}
\newcommand{\encxOmegapinvsigma}[1]{\tuple{\permcon{x}{\Omega'}{#1}{\sigma^{-1}}}}
\newcommand{\encyOmegainvsigma}[1]{\tuple{\permcon{y}{\Omega}{#1}{\sigma^{-1}}}}
\newcommand{\encyOmegapinvsigma}[1]{\tuple{\permcon{y}{\Omega'}{#1}{\sigma^{-1}}}}

\newcommand{\tstyle}{\vdash}
\newcommand{\gatdisplayrule}[3][]
{
\setlength{\fboxsep}{1pt}       
\setlength{\fboxrule}{0pt}
\fbox{$\displaystyle \frac{#2}{#3\rule[-0.3cm]{0cm}{0cm}}$#1}    %added vertival space using \rule
}
\newcommand{\genericAintroductoryrule} {\gatdisplayrule{\xDelta{n}}{\isT{A(\xn)}}}
\newcommand{\genericfintroductoryrule}  {\gatdisplayrule{\xDelta{n}}{\ofT{f(\xn)}{\Delta}}}

% gat macros developed for cwf paper

% Expressing gats
\newenvironment{gatrules}
{
$$
\begin{array}{l l}
}
{
\end{array}
$$
}
\newcommand{\gatintros}
{
\textbf{Symbol} & \textbf{Introductory\ Rule}                      \\}

\newcommand{\gataxioms}
{\textbf{Axioms}\\}
\newcommand{\gatintro}[3]{\ #1 & #2 \tstyle #3 \\}
\newcommand{\gatlocalintro}[3]{\ #1 & #2 \dashv }
\newcommand{\gataxiom}[2]{\multicolumn{2}{l}{\ \ #1\mbox{,  whenever\ } #2} \\}
\newcommand{\noleft}{\left.\kern-\nulldelimiterspace} % so that no space taken by absent left brace


\newcommand{\gatmultiaxiom}[2]
{\multicolumn{2}{l}{
  \noleft
    \begin{array}{l}
		#1
    \end{array} 
  \right\} \mbox{whenever\ } 	#2 
	}\\}
	
	\newcommand{\axid}[1]{\text{#1}.\ }	

%New context sharing macros
\newcommand{\gatintroducing}[1]{
{\arraycolsep=0pt
  \begin{array}{l}
          #1
  \end{array}} &
}

%*********************************
% \begin{\gatgroup}{context}
%    rules
%  \end{\gatgroup}
%*********************************
\NewEnviron{gatgroup}[1]{%
  \noleft
  {\arraycolsep=0pt
   \begin{array}{l}
\BODY
    \end{array} 
   }
   \ \right\} 
	%\mbox{\ whenever\ } 
	#1
	\vspace{0.1cm} 
}
%*********************************

%*********************************
% \begin{\gatgroupnoshared}
%    rule
%  \end{\gatgroupnoshared}
%*********************************
\NewEnviron{gatgroupnoshared}{%
  {\arraycolsep=0pt
   \begin{array}{l}
\BODY
    \end{array} 
   }
   \ 
	\vspace{0.1cm} 
}
%*********************************

% \gatsingular[width]{context}{conclusion}
\newcommand{\gatsingular}[3][4cm]{
\begin{gatgroupnoshared}
\gatleaf[#1]{#2}{#3} 
\end{gatgroupnoshared}
}

%*********************************
% \gatleaf}[width]{context}{assertion}
%*********************************
\newcommand{\gatleaf}[3][4cm]{%
\makebox[#1]{$#3$ \dotfill} \dotfill \  #2
}
%*********************************
%*********************************
% \gatstandalonesingle}{context}{assertion}
%*********************************
\newcommand{\gatstandalonesingle}[2]{%
#2 \makebox[2.5cm]{\dotfill} \  #1
}
%*********************************

% \gataxiomno{axiomno}
\newcommand{\gataxiomno}[1]{\makebox[0.5cm]{} \axid{#1}}


% metagat.macros.tex

%Meta-theories

%\newcommand{\typ}{\triangleright}
\newcommand{\typ}{\nabla}
\newcommand{\trm}{\tau}
\newcommand{\cross}{\otimes}
\newcommand{\sub}{^*}
\newcommand{\diag}{\delta}

\newcommand{\typeseq}[2]
{\ofT{#1_1}{\typ},... \ofT{#1_{#2}}{\typ(#1_{#2-1})}}

\newcommand{\typeseqcont}[3]
{\ofT{#1_1}{\typ({#2})},... \ofT{#1_{#3}}{\typ(#1_{#3-1})}}

\newcommand{\Ob}{Ob}
\newcommand{\obj}{Ob} % <!-- new old --<
\newcommand{\Hom}{Hom}
\newcommand{\objseq}[2]
{\ofT{#1_1}{\obj},... \ofT{#1_{#2}}{\obj(#1_{#2-1})}}


\def\dottededge{\ncline[linestyle=dotted, nodesep=0.3cm]}
\def\noedge{\ncline[linestyle=none]}
\def\thinedge{\ncline[linewidth=0.4pt]}

\newcommand{\member}[1]
{\ncarc[arcangle=-30,nodesepB=0.03]{->}{\pspred}{\pssucc}
\nbput[labelsep=0.1]{#1}}

\newcommand{\loweraccutemember}[1]
{\ncarc[arcangle=-15,nodesepB=0.03]{->}{\pspred}{\pssucc}
\nbput[labelsep=0.05,npos=0.85]{#1}}

\newcommand{\uppermember}[1]
{\ncarc[arcangle=30,nodesepB=0.03]{->}{\pspred}{\pssucc}\naput{#1}}

\newcommand{\upperaccutemember}[1]
{\ncarc[arcangle=10,nodesepB=0.03]{->}{\pspred}{\pssucc}\naput[npos=0.85]{#1}}

% flexbranch 
% #1 node label
% #2 thislevelsep
% #3 next level sep
% #4 variable (eg x)
% #5 index leter (eg n)
% #6 close parenthesis
% #7 continuation branches
\newcommand{\flexbranch}[7]
{
\pstree[thislevelsep=*#2,nodesep=0.05]
		{\Rnode{#1 1}{\Tr{#4_1 #6}}}
	  {\pstree[thislevelsep=#3]  
				   {\Rnode{#1 2}{\Tr[edge=\dottededge]{#4_{#5} #6}}}
					 {#7}
		}
}

\newcommand{\flexbranchplusleaf}[6]
{
\flexbranch{#1}{#2}{#3}{#4} {#5} {#6}
  {
   %\Rnode{#1 3}{\Tr{#4 #6}}
	 \Tr{\Rnode{#1 3}{#4 #6}}
  }
}

\newcommand{\flexbranchplusarc}[7]
{
\flexbranch{#1}{#2}{#3}{#4} {#5} {#6}
  {
   %\Rnode{#1 3}{\Tr{#4 #6}\member{#7}}
	 \Tr{\Rnode{#1 3}{#4 #6}}\member{#7}
  }
}

\newcommand{\flexbranchinitialarc}[9]
{
\pstree[thislevelsep=*#2,nodesep=0.05]
		{\Rnode{#1 1}{\Tr{#4_#8 #6}}#9}
	  {\pstree[thislevelsep=#3]  
				   {\Rnode{#1 2}{\Tr[edge=\dottededge]{#4_{#5} #6}}}
					 {#7}
		}
}

\newcommand{\equality}[2]
{
\ncline [doubleline=true, nodesep=0.2cm]{#1}{#2}
}
\newcommand{\equalityarc}[2]
{
\ncarc [arcangleA=-30, arcangleB=-20, doubleline=true, nodesep=0.1cm]{#1}{#2}
}

%The following are stylistic so belong in main document not here.
%\usepackage[margin=4.0cm]{geometry} % This shouldn't be here commented out 17 July 2018
%\usepackage{mathptmx}               % This changes font to roman so doesn't belong here
%
\usepackage{amsfonts}
\usepackage{amssymb} % added 08\02\2019 as an experiment. Needed in some instances for \blacksquare
                     % not needed is class is `beamer' but I don't know why not
\usepackage{array}
\usepackage{pstricks}
\usepackage{pst-tree}
\usepackage{pst-plot}
\usepackage{pst-node}
\usepackage{stmaryrd}
\usepackage{amsmath}
\usepackage{verbatim}
\usepackage{graphicx}  
\usepackage{calc}
\usepackage{xifthen}
%\usepackage{xcolor} investigate with beamer
\usepackage{color}
\usepackage{stringstrings}
%\usepackage[small,bf,margin=3pt,format=hang, labelsep=endash,singlelinecheck=false]{caption} %prevuiously justification=justified
%\usepackage{enumerate}
%\usepackage{enumitem}
\usepackage{enumerate}
%\usepackage[shortlabels]{enumitem} %Removed this 28/01/2019 because interfereing with a beamer presentation. 
\usepackage{float}
\usepackage[section]{placeins}
%\setlength{\captionmargin}{5pt}
\usepackage{environ}
\usepackage{multirow}
\usepackage{rotating}
\usepackage{longtable}
\usepackage{afterpage}
\usepackage{needspace}


%DEFINE ENVIRONMENT BLOCK
% Riddle
\newsavebox{\riddlebox}

\newenvironment{erexample}
{\newcommand\colboxcolor{F0F0F0}%was F8F8F8
\begin{lrbox}{\riddlebox}
\begin{minipage}{\dimexpr\columnwidth-2\fboxsep\relax} \textbf{} \\ \itshape}
{\end{minipage}\end{lrbox}%
%\begin{center}
\colorbox[HTML]{\colboxcolor}{\usebox{\riddlebox}}
%\end{center}
}

\newenvironment{erbox}
{\newcommand\colboxcolor{F0F0F0}%was F8F8F8
\begin{lrbox}{\riddlebox}%
\begin{minipage}{\dimexpr\columnwidth-2\fboxsep\relax} }
{\end{minipage}\end{lrbox}%
%\begin{center}
\colorbox[HTML]{\colboxcolor}{\usebox{\riddlebox}}
%\end{center}
}

%\begin{erboxedFigure}{#1 FigureParam}{#2 Label}{#3 Caption}
\NewEnviron{erboxedFigure}[3]{%
\begin{figure}[#1]
\begin{erexample}
\begin{center}
\BODY
\end{center}
\vspace{-0.5cm}
\caption{#3}
\label{#2}
\end{erexample}
\end{figure}
}

\newcommand{\erpictureFolder}[0]{../SharedPictures}

\newcommand{\ercenterPicture}[1]{
\begin{center}
\input{\erpictureFolder/#1}
\end{center}
}


\newlength{\erhalfHt}

%\erinlinePicture{#1 pictureFilename}{#2 pictureHeight}
\newcommand{\erinlinePicture}[2]{
\setlength{\erhalfHt}{#2cm * \real{0.5}}
\raisebox{-\erhalfHt}[\erhalfHt + 0.5cm][\erhalfHt + 0.5cm]{
\input{\erpictureFolder/#1}
} 
}

%\erplainFig{#1 pictureFilename}{#2 figureParam}{#3Caption}
\newcommand{\erplainFig}[3]{
\begin{figure}[#2]
\begin{center}
\input{\erpictureFolder/#1}
\end{center}
\caption{#3}
\label{#1}
\end{figure}
}

%\erboxedFigPicture{#1 pictureFilename}{#2 figureParam}{#3Caption}
\newcommand{\erboxedFigPicture}[3]{
\begin{figure}[#2]
\begin{erexample}
\vspace{-0.5cm}
\begin{center}
\input{\erpictureFolder/#1}
\end{center}
\caption{#3}
\label{#1}
\end{erexample}
\end{figure}
}

%\erLeftSideFig{#1 pictureFilename}{#2 figureParam}{#3Caption}
\newcommand{\erLeftSideFig}[3]{
\begin{figure}[#2]
\begin{erexample}
  \begin{minipage}[c]{0.4\textwidth}
    \caption{#3}
    \label{#1}
  \end{minipage}
  \begin{minipage}[c]{0.5\textwidth}
    \input{\erpictureFolder/#1}
  \end{minipage}
\end{erexample}
\end{figure}
}

%\erbulletedFig{#1 pictureFilename}{#2 figureParam}{#3Caption}
\NewEnviron{erbulletedFig}[3]{%
\begin{figure}[#2]
\begin{erexample}
\vspace{-0.5cm}
\begin{center}
$
\begin{array}{c m{0.25cm} | m{6cm}}
\raisebox{-2.0cm}{
\input{\erpictureFolder/#1}}& & \text{\parbox{6cm}{\raggedright{\footnotesize{
\begin{enumerate}[(i)]
\BODY
\end{enumerate}}}}} \\
\end{array}
$
\end{center}
\caption{#3}
\label{#1}
\end{erexample}
\end{figure} 
}


%\begin{erbulletedDimFig}{#1 pictureFilename}{#2figureParam} {#3Caption} {#4PictureHeight}{#5TextWidth}

\NewEnviron{erbulletedDimFig}[5]{%
\begin{figure}[#2]
\begin{erexample}
\vspace{-0.5cm}
\begin{center}
$
\begin{array}{c m{0.25cm} |  m{#5cm}}
\setlength{\erhalfHt}{#4cm * \real{0.5}}
\raisebox{-\erhalfHt}{
\input{\erpictureFolder/#1}}& & \text{\parbox{#5cm}{\raggedright{\footnotesize{
\begin{enumerate}[(i)]
\BODY
\end{enumerate}}}}} \\
\end{array}
$
\end{center}
\caption{#3}
\label{#1}
\end{erexample}
\end{figure} 
}

%\begin{ernotedModel}{#1 pictureFilename}{#2PictureHeight}{#3PictureWidth}{#4TextWidth}

\NewEnviron{ernotedModel}[4]{%
\begin{center}
$
\begin{array}{m{#3cm} m{1cm} | c m{#4cm}}
\setlength{\erhalfHt}{#2cm * \real{0.5}}
\raisebox{-\erhalfHt}{
\input{\erpictureFolder/#1}}& & & \text{\parbox{#4cm}{\raggedright{\footnotesize{
\BODY
}}}} \\
\end{array}
$
\end{center} 
}

%\begin{ermodelText}{#1 pictureFilename}{#2PictureHeight}{#3PictureWidth}{#4TextWidth}

\NewEnviron{ermodelText}[4]{%
\begin{center}
\begin{tabular}{m{#3cm} m{1cm}  c m{#4cm}}
\setlength{\erhalfHt}{#2cm * \real{0.5}}
\raisebox{-\erhalfHt}{
\input{\erpictureFolder/#1}}& & & \text{\parbox{#4cm}{\raggedright{\small{
\BODY
}}}} \\
\end{tabular}
\end{center} 
}


%\erbulletedModel{#1 pictureFilename}{#2PictureHeight}{#3PictureWidth}{#4TextWidth}

\NewEnviron{erbulletedModel}[4]{%
\begin{center}
$
\begin{array}{m{#3cm} m{1cm} | c m{#4cm}}
\setlength{\erhalfHt}{2cm * \real{0.5}}
\raisebox{-\erhalfHt}{
\input{\erpictureFolder/#1}}& & & \text{\parbox{#4cm}{\raggedright{\footnotesize{
\begin{enumerate}[(i)]
\BODY
\end{enumerate}}}}} \\
\end{array}
$
\end{center} 
}



%\ernotedDimFig{#1 pictureFilename}{#2 figureParam}{#3Caption}{#4PictureHeight}{#5TextWidth}
\NewEnviron{ernotedDimFig}[5]{%
\begin{figure}[#2]
\begin{erexample}
\vspace{-0.5cm}
\begin{center}
$
\begin{array}{c m{0.25cm} | c m{#5cm}}
\setlength{\erhalfHt}{#4cm * \real{0.5}}
\raisebox{-\erhalfHt}{
\input{\erpictureFolder/#1}}& & & \text{\parbox{#5cm}{\raggedright{\footnotesize{
\BODY }}}}\\
\end{array}
$
\end{center}
\caption{#3}
\label{#1}
\end{erexample}
\end{figure} 
}
%\begin{ernotedDimFigPW}{#1 pictureFilename}{#2 figureParam}{#3Caption}{#4PictureHeight}{#5PictureWidth}{#6TextWidth}
\NewEnviron{ernotedDimFigPW}[6]{%
\begin{figure}[#2]
\begin{erexample}
\vspace{-0.5cm}
\begin{center}
$
\begin{array}{>{\centering}m{#5cm} m{0.5cm} | c m{#6cm}}
\setlength{\erhalfHt}{#4cm * \real{0.5}}
\raisebox{-\erhalfHt}{
\centering \input{\erpictureFolder/#1}
}& & & \text{\parbox{#6cm - 0.5cm}{\raggedright{\footnotesize{
\BODY }}}}\\
\end{array}
$ \\
\vspace {0.2cm}
\end{center}
\caption{#3}
\label{#1}
\end{erexample}
\end{figure}
}



\newenvironment{erquote}
{\begin{quote}\itshape}
{\end{quote}}


%
%  erdiagram.tex
%  *************
%  Macros to represent ER diagrams
%  *******************************
% 29/01/2019 Modify so that not reliant on the
%            default fontsize being 10pt by using
%            package anyfontsize and then
%            \fontsize{8}{10}\selectfont to set font to 8pt
% 06/02/2019 Pullback symbol implemented and minor tweaks to positioning 
%            and size of identifier symbol and relationship labels.
%            Accidental forked changes merged on 08/02/2019.
% 15/03/2019 Continuation of 29/01/2019. Need fix fontsize of 
%            ERrelname and ERscope.	 
% ***********************************************************
 \usepackage{anyfontsize}             % 29/01/2019 
  
%\begin{erdiagram}{#1 height}{#2 width} 
% ....
% ....
%\end{erdiagram}
\newenvironment{erdiagram}[2]
{%\pspicture*(-#1,0)(#2,0)
\pspicture*(0,-#1)(#2,0)
%\psgrid
}
{\endpspicture}

\definecolor{lightyellow}{cmyk}{0,0,0.3,0}
\definecolor{verylightgrey}{gray}{0.95}


% \eret{#1 x0} {#2 y0} {#3 x1} {#4 y1} {#5 corner radius} {#6 fill}
\newcommand {\eret}[6]
{ 
\ifthenelse{\equal{#6}{1}}
{\psframe[framearc=#5,fillstyle=solid,fillcolor=lightyellow](#1,#2)(#3,#4)}
{\psframe[framearc=#5,fillstyle=solid,fillcolor=white](#1,#2)(#3,#4)}
}

% et top 
\newcommand {\erettop}[4]
{
%\psframe[linestyle=none,linearc=2pt,cornersize=absolute,fillstyle=solid,fillcolor=lightyellow](#1,#2)(#3,#4)
\psline[linearc=2pt,fillstyle=none,fillcolor=lightyellow](#1,#4)(#1,#2)(#3,#2)(#3,#4)
}

% et bottom 
\newcommand {\eretbtm}[4]
{
%\psframe[linestyle=none,linearc=2pt,cornersize=absolute,fillstyle=solid,fillcolor=lightyellow](#1,#2)(#3,#4)
\psline[linearc=2pt,fillstyle=none,fillcolor=lightyellow](#1,#2)(#1,#4)(#3,#4)(#3,#2)
}

% et bottom left
\newcommand {\eretbl}[4]
{
%\psframe[linestyle=none,linearc=2pt,cornersize=absolute,fillstyle=solid,fillcolor=lightyellow](#1,#2)(#3,#4)
\psline[linearc=2pt,fillstyle=none,fillcolor=lightyellow](#1,#4)(#3,#4)(#3,#2)
}

% et middle left
\newcommand {\eretml}[4]
{
%\psframe[linestyle=none,linearc=2pt,cornersize=absolute,fillstyle=solid,fillcolor=lightyellow](#1,#2)(#3,#4)
\psline[linearc=2pt,fillstyle=none,fillcolor=lightyellow](#1,#2)(#3,#2)(#3,#4)(#1,#4)
}

% et top left
\newcommand {\erettl}[4]
{
%\psframe[linestyle=none,linearc=2pt,cornersize=absolute,fillstyle=solid,fillcolor=lightyellow](#1,#2)(#3,#4)
\psline[linearc=2pt,fillstyle=none,fillcolor=lightyellow](#1,#2)(#3,#2)(#3,#4)
}

% et bottom right
\newcommand {\eretbr}[4]
{
%\psframe[linestyle=none,linearc=2pt,cornersize=absolute,fillstyle=solid,fillcolor=lightyellow](#1,#2)(#3,#4)
\psline[linearc=2pt,fillstyle=none,fillcolor=lightyellow](#1,#2)(#1,#4)(#3,#4)
}

% et middle right
\newcommand {\eretmr}[4]
{
%\psframe[linestyle=none,linearc=2pt,cornersize=absolute,fillstyle=solid,fillcolor=lightyellow](#1,#2)(#3,#4)
\psline[linearc=2pt,fillstyle=none,fillcolor=lightyellow](#3,#4)(#1,#4)(#1,#2)(#3,#2)
}

% et top right
\newcommand {\erettr}[4]
{
\psline[linearc=2pt,fillstyle=none,fillcolor=lightyellow](#1,#4)(#1,#2)(#3,#2)
}

% \ergrp{#1 x0} {#2 y0} {#3 x1} {#4 y1} {#5 corner radius} {#6 fill}
% #5 corner radius is unused!
\newcommand {\ergrp}[6]
{ 
\ifthenelse{\equal{#6}{1}}
{\psframe[fillstyle=solid,fillcolor=verylightgrey](#1,#2)(#3,#4)}
{\psframe[fillstyle=solid,fillcolor=white](#1,#2)(#3,#4)}
}


% \ertext{#1 text}
% 15/03/2019
\newcommand {\erextrasmallitalictext}[1]
{\fontsize{7}{9}\selectfont \textit{#1}}

% 29/01/2019  
\newcommand {\ersmallitalictext}[1]
{\fontsize{8}{10}\selectfont \textit{#1}}

\newcommand {\ermediumitalictext}[1]
{\fontsize{10}{12}\selectfont \textit{#1}}

% \eretname {#1 x left of text} {#2 y top of text} {#3 text}
\newcommand {\olderetname}[3]
{
%shift down 0.1 for height of text the anchor at baseline (B)
\rput[bl]{0}(0,-0.1){\rput[Bl]{0}(#1,#2){\ersmallitalictext{#3}}}
}

% \errelarm {#1 x0} {#2 y0} {#3 x1} {#4 y1} {#5 ismandatory} {#6 isconstructed}
\newcommand {\errelarm}[6]
{
\ifthenelse{\equal{#6}{1}}
{
%%\psline[linewidth=0.5pt,linearc=.05,linestyle=dashed,dash=6pt 6pt]{-}(#1,#2)(#3,#4)}
\ifthenelse{\equal{#5}{1}}
{\psline[linewidth=1.5pt,linearc=.05,linecolor=lightgray]{-}(#1,#2)(#3,#4)}
{\psline[linewidth=1.5pt,linearc=.05,linecolor=lightgray,linestyle=dashed,dash=2pt 2pt]{-}(#1,#2)(#3,#4)}
}
{
\ifthenelse{\equal{#5}{1}}
{\psline[linewidth=0.9pt,linearc=.05]{-}(#1,#2)(#3,#4)}
{\psline[linewidth=0.9pt,linearc=.05,linestyle=dashed,dash=2pt 2pt]{-}(#1,#2)(#3,#4)}
}
}

% \errelangle {#1 x0} {#2 y0} {#3 x1} {#4 y1} {#5 x2} {#6 y2} {#7 ismandatory} {#8 isocnstructed}
\newcommand {\errelangle}[8]
{
\ifthenelse{\equal{#8}{1}}
{
%\psline[linewidth=0.5pt,linearc=.1,linestyle=dashed,dash=6pt 6pt]{-}(#1,#2)(#3,#4)(#5,#6)}
\ifthenelse{\equal{#7}{1}}
{\psline[linewidth=1.5pt,linearc=.05,linecolor=lightgray]{-}(#1,#2)(#3,#4)(#5,#6)}
{\psline[linewidth=1.5pt,linearc=.1,linecolor=lightgray,linestyle=dashed,dash=2pt 2pt]{-}(#1,#2)(#3,#4)(#5,#6)}
}
{
\ifthenelse{\equal{#7}{1}}
{\psline[linewidth=0.9pt,linearc=.1]{-}(#1,#2)(#3,#4)(#5,#6)}
{\psline[linewidth=0.9pt,linearc=.1,linestyle=dashed,dash=2pt 2pt]{-}(#1,#2)(#3,#4)(#5,#6)}
}
}

% \ercrowfoot {#1 x0} {#2 y0} {#3 x11} {#4 y11} {#5 x12} {#6 y12} {#7 x13} {#8 y13} {#9 isconstructed}
\newcommand {\ercrowfoot}[9]
{
\ifthenelse{\equal{#9}{1}}
{
\psline[linewidth=1.5pt,linearc=.05,linecolor=lightgray]{-}(#1,#2)(#3,#4)
\psline[linewidth=1.5pt,linearc=.05,linecolor=lightgray]{-}(#1,#2)(#5,#6)
\psline[linewidth=1.5pt,linearc=.05,linecolor=lightgray]{-}(#1,#2)(#7,#8)
}{
\psline[linewidth=0.9pt,linearc=.05]{-}(#1,#2)(#3,#4)
\psline[linewidth=0.9pt,linearc=.05]{-}(#1,#2)(#5,#6)
\psline[linewidth=0.9pt,linearc=.05]{-}(#1,#2)(#7,#8)
}
}


% \eridcomprel{#1 x1}{#2 x2}{#3 y}
\newcommand {\eridcomprel}[3]
{
\psline[linewidth=0.9pt](#1,#3)(#2,#3)
}

% \eridrefrel{#1 x}{#2 y1}{#3 y2}
\newcommand {\eridrefrel}[3]
{
\psline[linewidth=0.9pt](#1,#2)(#1,#3)
}

% \ertext {#1 x} {#2 y} {#3 text anchor} {#4 text}  PRIVATE
\newcommand {\ertext}[4]
{
\rput[B#3]{0}(#1,#2){\fontsize{8}{10}\selectfont #4}
}

% \eretname {#1 x} {#2 y} {#3 text anchor} {#4 text} 
\newcommand {\eretname}[4]
{
\ertext{#1}{#2}{#3}{#4}
}

% \errelname {#1 x} {#2 y} {#3 text anchor} {#4 text} 
\newcommand {\errelname}[4]
{
\rput[B#3]{0}(#1,#2){\erextrasmallitalictext{#4}}
}


% \erscope {#1 x} {#2 y} {#3 text anchor} {#4 text}  15 March 2019
\newcommand {\erscope}[4]
{
\rput[B#3]{0}(#1,#2){\erextrasmallitalictext{#4}}
}

% \erreletname {#1 x} {#2 y} {#3 text anchor} {#4 text}  15 March 2019
\newcommand {\erreletname}[4]
{
\rput[B#3]{0}(#1,#2){\fontsize{10}{12}\selectfont #4}
}

% \ergroupannotation {#1 x} {#2 y} {#3 text anchor} {#4 text}
\newcommand {\ergroupannotation}[4]
{
\ertext{#1}{#2}{#3}{#4}
}


% \errelseq {#1 x} {#2 y}
\newcommand {\erelseq}[2]
{
}
\newcommand {\erattrmarkermand}
{\fontsize{6}{8}\selectfont $\blacksquare$}
\newcommand {\erattrmarkeropt}
{\fontsize{6}{8}\selectfont \CIRCLE}
\newcommand {\erderattrmarkermand}
{\fontsize{6}{8}\selectfont $\square$}
\newcommand {\erderattrmarkeropt}
{\fontsize{8}{10}\selectfont $\circ$}

% \erattr {#1 x} {#2 y} {#3 ismandatory}{#4 idenitfying} {#5 text}
\newcommand {\erattr}[5]
{
\ifthenelse{\equal{#3}{1}}
{\rput[l]{0}(#1,#2){\erattrmarkermand \ersmallitalictext{\ifthenelse{\equal{#4}{0}}{\underline{#5}}{#5}}}}
{\rput[l]{0}(#1,#2){\erattrmarkeropt \ersmallitalictext{\ifthenelse{\equal{#4}{0}}{\underline{#5}}{#5}}}}
}

\newcommand {\erdattr}[5]
{
\ifthenelse{\equal{#3}{1}}
{\rput[l]{0}(#1,#2){\erderattrmarkermand \ersmallitalictext{\ifthenelse{\equal{#4}{0}}{\underline{#5}}{#5}}}}
{\rput[l]{0}(#1,#2){\erderattrmarkeropt \ersmallitalictext{\ifthenelse{\equal{#4}{0}}{\underline{#5}}{#5}}}}
}


% \erarc {#1 x0} {#2 y0} {#3 x1} {#4 y1} {#5 x2} {#6 y2} {#7 x3} {#8 y3}
\newcommand {\erarc}[8]
{
\psbezier[showpoints=false]{-}(#1,#2) (#3, #4)(#5,#6) (#7, #8)
}

% \erarc {#1 x0} {#2 y0} {#3 x1} {#4 y1} {#5 x2} {#6 y2} {#7 x3} {#8 y3}
\newcommand {\errelseq}[8]
{
\psbezier[showpoints=false]{-}(#1,#2) (#3, #4)(#5,#6) (#7, #8)
}
% \ertrace {#1 trace}   
\newcommand {\ertrace}[1]
{
}


%indexedsets.macros.tex

% Macros for sets and families of sets
\newlength{\xl}
\newlength{\yb}
\newlength{\xr}
\newlength{\yt}
\newlength{\ytm}
\newlength{\ybm}
\newlength{\dotxl}
\newlength{\dotxr}
\newlength{\dotym}
\newlength{\basex} 
\newlength{\basey} 
\newlength{\childx} 
\newlength{\childy}
\newcommand{\putthreeset}[5][0]{
  \setlength{\xl}{-1.6cm * \real{#2}}
  \setlength{\xr}{1.8cm * \real{#2}}
  \setlength{\yt}{0.55cm * \real{#2}}
  \setlength{\ytm}{0.75cm * \real{#2}}
  \setlength{\yb}{-0.55cm * \real{#2}}
  \setlength{\ybm}{-0.80cm * \real{#2}}
  \setlength{\dotxl}{-1cm * \real{#2}}
  \setlength{\dotxr}{0.9cm * \real{#2}}
  \setlength{\dotym}{0.15cm * \real{#2}}
  %
  \rput{#1}(#3,#4){        
           {\psccurve%[showpoints=true]
                     (\xl ,\yt)(\xl,\yb)(0,\ybm )(\xr,\yb)(\xr,\yt) (0,\ytm)  }
            \dotnode[dotscale=0.4](\dotxl,0){#5l}
            \dotnode[dotscale=0.4](0,\dotym){#5m}
            \dotnode[dotscale=0.4](\dotxr,0){#5r}
            \pnode(0,\ybm){#5c}
           }
}
\newcommand{\puttwoset}[5][0]{
  \setlength{\xl}{-1.0cm * \real{#2}}
  \setlength{\xr}{1cm * \real{#2}}
  \setlength{\yt}{0.55cm * \real{#2}}
  \setlength{\ytm}{0.75cm * \real{#2}}
  \setlength{\yb}{-0.55cm * \real{#2}}
  \setlength{\ybm}{-0.80cm * \real{#2}}
  \setlength{\dotxl}{-0.75cm * \real{#2}}
  \setlength{\dotxr}{0.25cm * \real{#2}}
  %
  \rput{#1}(#3,#4){        
           {\psccurve%[showpoints=true]
                     (\xl ,\yt)(\xl,\yb)(0,\ybm )(\xr,\yb)(\xr,\yt) (0,\ytm)  }
            
            \dotnode[dotscale=0.4](\dotxl,0){#5l} 
            \dotnode[dotscale=0.4](\dotxr,0){#5r}
            \pnode(0,\ybm){#5c}
           }
}

%\putfamilyOfSets[#1 rotation]{#2 basescale}{#3 childscale}{#4 x}{#5 y}{#6 childoffset}{#7nodeprefix}
\newcommand{\putfamilyOfSets}[7][0]{
  \setlength{\basex}{#4}
  \setlength{\basey}{#5}
  \putthreeset[#1]{#2}{\basex}{\basey}{#7BASE} 
  %child 1
  \setlength{\childx} {#4 - (4cm * \real{#3})}
  \setlength{\childy} {#5 + #6}
  \putthreeset[#1]{#3}{\childx}{\childy}{L}
  %child 2
  \setlength{\childy}{\childy + 0.5cm}
  \putthreeset[#1]{#3}{#4}{\childy}{M}
  %child 3
  \setlength{\childx} {#4 + (4cm * \real{#3})}
  \setlength{\childy}{\childy - 0.5cm}
  \putthreeset[#1]{#3}{\childx}{\childy}{R}
  \ncline[nodesep=3pt]{|->}{#7BASEl}{Lc}
  \ncline[nodesep=3pt]{|->}{#7BASEm}{Mc}
  \ncline[nodesep=3pt]{|->}{#7BASEr}{Rc}
}

%putFunction[#1 rotation]{#2 basescale}{#3 childscale}{#4 x}{#5 y}{#6 childoffset}{#7nodeprefix}
\newcommand{\putFunction}[7][0]{
  \setlength{\basex}{#4}
  \setlength{\basey}{#5}
  \putthreeset[#1]{#2}{\basex}{\basey}{#7BASE} 
  %child 1
  %\setlength{\childx} {#4 - (4cm * \real{#3})}
	\setlength{\childx} {#4 }
  \setlength{\childy} {#5 + #6}
  \putthreeset[#1]{#3}{\childx}{\childy}{DEST}
  \ncline[nodesep=3pt]{|->}{#7BASEl}{DESTl}
  \ncline[nodesep=3pt]{|->}{#7BASEm}{DESTm}
  \ncline[nodesep=3pt]{|->}{#7BASEr}{DESTm}
}


\renewcommand{\erpictureFolder}[0]{../SharedPictures}



\setcounter{equation}{0}
\bibliographystyle{plain} % was hplain

\title{Notes on Contextual Categories - Definition of Fam}
\author{John Cartmell\footnote{john.w.cartmell gmail.com}\\ \normalsize{\textit{Ad Otium}}}
\begin{document}
\maketitle

\section{Definitions}
\subsection{Terminology}
In my thesis I use the terms  `type' and also `types that vary' and develop the syntax of generalised algebraic theories based on concepts and syntax found in Martin-Lof type theory.
Since then, in this area of type theory,  it has become the practice to speak of `dependent types'. In this document I stay faithful to the term `types that vary' because, though it might be a little awkward, I believe it gets closer to the heart of a matter which is both important and elusive.
By way of approach I suggest that the first use of functions in mathematics comes about as representations of quantities which vary; understanding quantities that vary, distance, velocity, temperature, for instance and understanding how quantities vary is core to much of science;
In physics I find reference to scale-dependent quantities, wave-number dependent quantities, gauge dependent quantities and so on but though it is commonplace to speak of dependent and independent variables I don't think it commonplace to speak of dependent quantities in science
Am I right or wrong here? 
Further, I would also say that a quantity that varies,  electric potential for instance, is not a function it \textit{is} a quantity that varies -- and for which we then have a mathematical best representation as a function; scientific enquiries concern quantities which vary, not functions and in 
this manner `quantities that vary' precede `functions'. 
\noindent
In fact my interest in type theory would be better served if we could use the
word `concept' in place of the word `type'. For instance I have a concept `citizen' and I might use it in several different contexts - in one context it might mean a citizen of Europe and in another a citizen of the US - likewise in one context it might mean citizen past, present and future whilst in another it might mean citizen at this very moment\footnote{the rub is that what we mean by `at this very moment' or `Now' varies also ref Hegel passage on this Here and this Now}  -- the conclusion is that `citizen' is a concept that varies and at very least it varies as `state' or `country' vary. If we were concept modelling or data modelling\footnote{??} we would model the dependency relationship between `citizen' 
and `state'. On the otherhand we can formalise mathematically by asserting that there is a set of countries
and also, for each country, a set of citizens of that country and we could say that there is a country indexed family of sets of citizens. Finally, in type theory we can formalise 'citizen' as 
a type that varies, in fact as a country-dependent type.

\section {Acyclic Categories and Wide Subcategories}
We need some standard definitions.

A category C is called \textit{acyclic}, if it has no inverses and no nonidentity
endomorphisms. This definition is given by Kozlov (see \cite{Kozlov2007}) who offers the following intuition:
\begin{erquote}
Another way to visualize acyclic categories is to think of them as those
that can be drawn on a sheet of paper, with dots indicating the objects, and
straight or slightly bent arrows, all pointing down, indicating the nonidentity
morphisms...
\end{erquote}
\noindent
A \textit{wide} subcategory of a category \cat{C} is a subcategory which contains all the objects of \cat{C}.

\section{Category with Distinguished Morphisms}

\begin{definition}
Define a \textit{category with distinguished morphisms} to be a category \cat{C} along with a wide acyclic subcategory \cat{D} such that for any object $A$ of $C$ the set of morphisms with domain $A$ is finite.
\end{definition} 

\noindent
Notation:
If \cat{C} is a category with distinguished morphisms then :
\begin{enumerate}[(i)]
\item Morphisms of the subcategory we will refer to as d-morphisms and we will write  
$f: A \dmorph B$ in \cat{C}
to mean that 
$f: B \morph A$ in \cat{C} and that $f$ is a d-morphism i.e. is in the subcategory \cat{D} of distinguished morphisms. 
\item If a morphism $f: B \morph A$ in \cat{C} is a d-morphisms and in addition
there does not exist an object $X$ distinct from $A$ and $B$ and d-morphisms $d_1$ and $d_2$
such that $f=d_1 \comp d_2$ then we will say f is a simple distinguished morphism (or s-morphism). We will write  $f:B \smorph A$ in \cat{C}
to mean that 
$f: B \morph A$ in \cat{C} and that $f$ is a s-morphism.
\end{enumerate}

\noindent
Previous authors had referred to such categories as being \textit{loop-free}. \\
\subsubsection{Examples}
These examples are not directly translatable to contextual categories but 
in future will illustrate an algebraic representation that is currently under development.
\noindent
The following is an example of an acyclic category given by Kolozov which we have rearranged and relabelled:
\noindent
The following is an example of an acyclic category given by Kolozov which we have rearranged and relabelled:

\begin{center}
\begin{equation}
\begin{array}{p{1.5cm}cccp{2cm}c}
&                & \Rnode{C}{C}&   \\ [1.4cm]
&                & \Rnode{B}{B} &  \\ [0.8cm]
&                & \Rnode{A}{A} &  
\end{array}
\end{equation}
\ncasar[-30]{y_1}{C}{B}
\ncasar[30]{y_2}{C}{B}
\ncasar{x}{B}{A}
\setlength {\saroffsetA}{-2pt}
\setlength {\saroffsetB}{-2pt}
\ncbsar[-50]{y_1 \comp x = y_2 \comp x}{C}{A}
\sarreset
\end{center}

\noindent This category can be taken as representing the following type system:
\addtocounter{equation}{-1}
\begin{align}
&A\mbox{ is a type} && \tag*{(\theequation a)}\\
&x\in A: B(x) \mbox{ is a type} && \tag*{(\theequation b)}\\
&x\in A_1, y_1 \in B(x), y_2 \in B(x) : C(x,y_1,y_2) \mbox{ is a type} && \tag*{(\theequation c)}
\end{align}

\noindent
Contrast with this acyclic category:
\begin{center}
\begin{equation}
\begin{array}{p{1.5cm}cccp{2cm}c}
&                & \Rnode{C}{C}&   \\ [1.4cm]
&                & \Rnode{B}{B} &  \\ [0.8cm]
&                & \Rnode{A}{A} &  
\end{array}
\end{equation}
\ncasar[-30]{y_1}{C}{B}
\ncasar[30]{y_2}{C}{B}
\ncasar{x}{B}{A}
\setlength {\saroffsetA}{-2pt}
\setlength {\saroffsetB}{-2pt}
\ncbsar[-60]{x_1 = y_1 \comp x}{C}{A}
\setlength {\saroffsetA}{2pt}
\setlength {\saroffsetB}{2pt}
\ncasar[60]{x_2 = y_2 \comp x}{C}{A}
\sarreset
\end{center}
\noindent
which represents :
\addtocounter{equation}{-1}
\begin{align}
&A\mbox{ is a type} && \tag*{(\theequation a)}\\
&x\in A: B(x) \mbox{ is a type} && \tag*{(\theequation b)}\\
&x_1\in A_1, y_1 \in B(x_1), x_2\in A_1, y_2 \in B(x_2) : C(x_1,x_2,y_1,y_2) \mbox{ is a type} && \tag*{(\theequation c)}
\end{align}

\subsection{Hierarchical Category with Distinguished Morphisms}
A category with distinguished morphisms is \textit{hierarchical} 


\subsection{Variable sets}
\noindent In the past I have written of `sets, 
set indexed families of sets, indexed families of families of sets and so on' (literally so) which is  a both a mouthful and slightly unhelpful when it comes to
defining what a model of a generalised algebraic theory $U$ (aka $U$-algebra) is.
\noindent
Here I shall use the term `variable set', which I will write as `varset', to include all of the above and use the term `element'  of `variable set' in place of the term `operator'.
\noindent
Varsets have a height which is some positive natural number i.e. some $n$, $n \geq 1$. A varsets of height $n+1$ are defined to vary wrt some other varset of height $n$. The large set of all variable sets of height $i$ will be written $\varset[i]$.
\varset[1] is defined to be the large set $U$ of all sets. Thus, a varset of height 1 is exactly a set. 

\subsubsection{Definitions in Type Theory}
I find it helpful to first consider how families and families of families of sets and so on might be defined using quantifiers in the style of Martin-L\"of but playing fast and loose with the notation and blurring the distinction between sets and types. Afterwards we recode the definitions in the language of set theory. 

\noindent
If $U$ is the universe of small sets  and $A \in U$ then the type 
of $A$-indexed families of sets will be given by:
\begin{equation*}
\prod_{x \in A}U
\end{equation*}
and so the type of all indexed families of sets, $\varset[2]$ will be given by
\begin{equation*}
\varset[2]=\sum_{A \in U}\prod_{x \in A}U
\end{equation*}
With \varset[2] so defined then if $\tuple{A,B} \in \varset[2]$ then
\begin{equation*}
x \in A \tstyle B(x) \in U 
\end{equation*}
as we require.
\noindent
Likewise if  $A \in U$ and $B \in \prod_{x \in A}U$ the the type of all
family $B$ indexed families of sets is
\begin{equation*}
\prod_{x \in A}\prod_{y \in B(a)}U
\end{equation*}
and so the type of all families of families of sets is:
\begin{equation*}
\varset[3]=\sum_{A \in U}\sum_{B \in \prod_{x \in A}U} \prod_{x \in A}\prod_{y \in B(a)}U
\end{equation*}

\noindent
With \varset[3] so defined then we have that if $\tuple{A,B,C} \in \varset[3]$ then
\begin{equation*}
x \in A, y \in B(x) \tstyle C(x,y) \in U 
\end{equation*}
which, again, is as required.

\subsubsection{In set theory}
Now we recode these definitions, so to speak, in (good old ZF) set theory.
Recall that a function in set theory is defined as a set $f$ all of whose elements are ordered pairs\footnote{In turn an ordered pair is defined to be a set of the form \{a,\{a,b\}\} for some sets $a$ and $b$ and is which is then written as $\tuple{a,b}$} 
and satisfying the condition that for all 
 $\tuple{a,b} \in f$ and for all $\tuple{a',b'} \in f$, $a=a'$ implies $b=b'$. Recall also that from this definition it follows that for such a set $f$ we can define a set which is the domain of the function $f$ and furthermore a set which is the range of $f$. In this way
a large set of all functions and also (large) domain and range functions can be defined. \\ 

\noindent Note also that if $s$ is a set of $n+1$-tuples, for some $n \geq 1$, then there 
is a set of
all $n$-tuples $\tuple{x_1,...x_n}$ satisfying the condition that there exists $x_{n+1}$ 
such that $\tuple{x_1,...x_n,x_{n+1}} \in s$. 
We will denote by $dom_n$ the (large) function that delivers this 
	set of $n$ tuples for any set of $n+1$-tuples. \\
	
\noindent
We now proceed to  define \varset[2], \varset[3], etc. set theoretically, and to define functions 
$dom_i: \varset[i+1] \morph \varset[i]$. If $v_{i+1} \in \varset[i+1]$ and $v_i \in \varset[i]$
then we will say the $v_{i+1}$ depends on $v_i$ iff $dom_i(v_{i+1})=v{i}$.

\noindent
We define \varset[1] to be the large set of all (small) sets $U$. \\

\noindent
We will say a set is \varset[2]-like if it satisfies the definition above of being a function. We will interpret the codomain as being $U$ and so if $v$ is \varset[2]-like then it is a set of ordered pairs such that for all $x \in dom_1(v)$, there exists a unique $y \in U$ such that
$\tuple{x,y} \in U$ and we will write $v(x)$ for the unique such $y$. We define the large set \varset[2] to be the large set of all \varset[2]-like  sets.\\

\noindent
We will say a set $v$ is \varset[3]-like if it is a set of ordered triples and 
$dom_2(v)$ is \varset[2]-like and for all $x \in dom_1(dom_2(v))$, for all $y \in dom_2(v)(x)$ 
there exists a unique $z$ such that $\tuple{x,y,z} \in v$. We will write $v(x,y)$ for the
unique such $z$. We will define the large set \varset[3] to be the large set of  all \varset[3]-like sets.

\noindent
Checkout these definitions: If $a \in \varset[1]$ then $a$ is a set. If $b \in \varset[2]$ and
$b$ depend on $a$ then from definition of \varset[2]-like it follows that:
\begin{equation*}
\mbox{for all }x \in a \mbox{, } b(x) \in U
\end{equation*}  
and, in other words, $b$ is an A-index family of sets. Futhermore, if $c \in \varset[3]$
and $c$ depends on $b$ then it folows from the definition of \varset[3]-like that
\begin{equation*}
\mbox{for all }x \in a \mbox{, for all } y \in b(x) \mbox{, }c(x,y) \in U
\end{equation*}
so the definitions check out OK and deliver up families  of sets and families of families of sets as
required.

\subsection{Old material}
\begin{equation*}
\varset[n+1]=\bigcup_{A \in U}{\varset[n]^A}
\end{equation*}
So defined, a varset of height $n+1$ consists of a pair $\tuple{A,X}$ where
$A$ is any set and $X:A \morph U$ is a function and this completes the definition of varset.

\noindent
Now we define for any varset of height $n+1$ the varset of height $n$ that it is defined to vary with respect to (wrt) by defining a functions $wrt_n$, for each $n >1$ such that $wrt_n: \varset[n+1] \morph \varset[n]$. 
We define $wrt_2$ to by
\begin{align*}
  &wrt_{2}                                               \\
\left\langle A,X\right\rangle &\longmapsto A
\end{align*}
\noindent
and define   $wrt_{n+1}$ for each $n$, $n > 2$, by:
\begin{align*}
  &wrt_{n+1}                 \\
\left\langle A,F\right\rangle &\longmapsto F \circ wrt_n
\end{align*}


Finally, denote the functional inverse to 
function $wrt_n: \varset[n+1] \morph \varset[n]$ by $\localvarsets_n :
\varset[n] \morph \powerset(\varset[n+1])$. 

We have the notation that we need now, for example if $A \in U$ and $B \in \localvarsets_2(A)$ then $B$ represents an $A$-indexed family of sets, for there is a set $p2(B)(a)$ for each element $a \in A$.

If in addition $C \in \localvarsets_3(B)$ then $C$ is  an $A$-indexed family of families of sets because for each $a \in A$,
for each $b \in p2(B)(a)$,
$p2(p2(C)(a))(b)$ is a set.

(see figures  \ref{BindexedbyA} and \ref{familyoffamilies}).
 
\subsection{Notation}
If \ccat[C] is a contextual category then the set of objects of \ccat[C]
will be written as \cobj{C} 
and the subset of objects of \ccat[C] at height $i$, for some $i$, $i \geq 0$  will be written as \cobj[i]{C}.

\noindent
If \ccat[C] is a contextual category and if $A$ is an object of \ccat[C] then \cslice{C}{A} is the contextual category equivalent of the slice category at A.
 %Recall that the categorical slice category has morphisms with codomian A as objects. 
The objects of the contextual slice category \cslice{C}{A} are exactly those distinguished morphisms of the contextual category \ccat[C] having codomain $A$.  The morphisms between any two such objects are as defined for the categorical slice category.

\noindent
\noindent
We have
\begin{equation}
\label{objectsall}
\cobj{C}=\bigcup_{n \geq 0}\cobj[n]{C}
\end{equation}
and also, since $\cobj[1]{C}=\cobj{C}$, we have
\begin{equation}
\label{objectsone}
\csliceobj[1]{C}{1}=\cobj[1]{C}
\end{equation}
From the definitions, we also have that
\begin{equation}
\label{objectssuccessor}
\cobj[n+1]{C}=\bigcup_{A \in \cobj[n]{C}}\csliceobj[1]{C}{A}
\end{equation}



\subsection{Sets, indexed families of sets, indexed famillies of indexed families and so on.}
\subsubsection{Set Indexed Families of Sets}
An indexed family of sets $B_{a)a\in A}$ (see figure \ref{BindexedbyA}) 
% Example family - BindexedbyA
\begin{figure}[h]
\begin{pspicture}(0,0.5)(9,5.5)
%\psgrid
%\putfamilyOfSets[rot]{basescale}{childscale}{x}{y}{childoffset}{nodeprefix}
\rput{*270}(1,5){
  \putfamilyOfSets[*270]{.75}{0.4}{2cm}{2.5cm}{2.3cm}{FAM}
  }
\rput[l](1.5,4.2){ \psframebox*{set A} }
\rput[l](6.5,2){ \psframebox*{set $B_a$, for every $a \in A$} }
\end{pspicture}
\caption{A set indexed families of sets}
\label{BindexedbyA}
\end{figure}
can be defined as a function $B:A \rightarrow U$. 
\subsubsection{An Indexed Family of Families of Sets}
\noindent An indexed family of families of sets(see figure \ref{familyoffamilies})
% Example family of families
\begin{figure}[h]
\begin{center}
\begin{pspicture}(0,0)(10,10)
%\psgrid
%\putfamilyOfSets[rot]{basescale}{childscale}{x}{y}{childoffset}{nodeprefix}
\rput{*270}(1,10){
  \putthreeset[*270]{.75}{5cm}{1cm}{BASE}
  \putfamilyOfSets[*270]{.5}{0.25}{2cm}{2.5cm}{1.3cm}{FAML}
  \putfamilyOfSets[*270]{.5}{0.25}{5cm}{3cm}{1.3cm}{FAMM}
  \putfamilyOfSets[*270]{.5}{0.25}{8cm}{2.5cm}{1.3cm}{FAMR}
}
\ncline[nodesep=3pt]{|->}{BASEl}{FAMLBASEc}
\ncline[nodesep=3pt]{|->}{BASEm}{FAMMBASEc}
\ncline[nodesep=3pt]{|->}{BASEr}{FAMRBASEc}
\rput[l](.6,6.6){\psframebox*{set A} }
\rput[l](6,2){\psframebox*{a set indexed family $B_a$, for every $a \in A$} }
\end{pspicture}
\caption {A set indexed family of set indexed families of sets}
\label{familyoffamilies}
\end{center}
\end{figure}
\noindent can be defined to be a function from a set A to the (large) set \Fam[2] of all set indexed families. 

\subsection{Motivation for definition of Fam}
In the definition of a model of a generalised theory just as we interpret types by sets of instances so we interpret types that vary by sets which vary and by this we mean
set indexed families of sets, set indexed families of set indexed famillies of sets and so on.

In speaking this way, by keep having to say `and so on' is an indication that there isn't any good terminology in this area.  We want to be able to define what we mean by a 'set which varies' or maybe a better term `a variable set'. 

\subsection{Initial Definition of \Fam}

\noindent We define a large set $\Famobj[n]$, for all $n, n > 0$,  and we define functions $dep_n:\,\Famobj[n] \rightarrow\,\Famobj[n-1]$, for all $n, n \ge 1$.
$\Fam[1]$ is defined to the U, the set of all sets.
\Famobj[2] is defined to the set of all set indexed families. 
We define $dep_1$ to be the unique function $dep_1: \Famobj[1] \morph \Famobj[0]$
$dep_2$ is defined to the function that maps the A-indexed family B to the set A.
\Famobj[n+1], for $n \geq 2$ is defined to the set of all set-indexed families of elements of \Famobj[n]. That is:
\begin{equation}
\Famobj[n+1] = \bigcup_{A\in U}(\Famobj[n])^A
\end{equation}


\noindent To define $dep_{n+1}$, for $n \geq 2$ we need to define $dep_{n+1}(X_{n+1})$, for any $X_{n+1} \in\,\Famobj[n+1]$. Such an $X_{n+1}$ consists of a pair $\left\langle A,F\right\rangle$,
where A is a set and $F:A \rightarrow\, \Famobj[n]$. Define:
\begin{align}
  &dep_{n+1}              \notag     \\
\left\langle A,F\right\rangle &\longmapsto F \circ dep_n
\end{align}
\noindent In summary, we have defined sets and functions:

\vspace{0.25cm}
\begin{center}
\begin{math}
\setlength{\arraycolsep}{.5cm}
\begin{array}{cccccccc}
\Rnode{U0}{\Famobj[0]}&\Rnode{U1}{\Famobj[1]}&\Rnode{U2}{\Famobj[2]}&\cdots&\Rnode{Unp}{\Famobj[n-1]}&\Rnode{Un}{\Famobj[n]}&\Rnode{Uns}{\Famobj[n+1]}&\cdots
\end{array}
\ncline{<-}{U0}{U1}
\naput{dep_1}
\ncline{<-}{U1}{U2}
\naput{dep_2}
\ncline{<-}{Unp}{Un}
\naput{dep_n}
\ncline{<-}{Un}{Uns}
\naput{dep_{n+1}}
\end{math}
\end{center}

\noindent 
\subsection{Initial Uniform Definition}
The initial definition had $\Famobj[1]=U$. If instead we represent $\Famobj[1]$ as 
$1^U$ where 1 is a chosen singleton set - the set containing the empty set, say, which we can do because $1^U \simeq U$, then the definition is more uniform. 
We define the terminal object of \Fam\  to be the set containing the empty set i.e. the Von Neuman ordinal 1. 
So, define
\begin{equation*}
\Famobj[0] = \left\{\{\{\}\}\right\} 
\end{equation*}
Then we can define \Famobj[n+1],
for each $n$, $n \geq 0$, by
\begin{align*}
\Famobj[n+1] &= \bigcup_{A \in U} (\Famobj[n])^A  
\end{align*}
where $U$ is the large set of all sets.
We define $dep_1$ to be the unique function $dep_1: \Famobj[1] \morph \Famobj[0]$.
and define $depn+1$ for each 
\noindent Now we can define $dep_{n+1}$ for each $n$, $n geq 1$, by:
\begin{align}
  &dep_{n+1}              \notag     \\
\left\langle A,F\right\rangle &\longmapsto F \circ dep_n
\end{align}

\subsection {Sections}
\subsubsection{Sections of an indexed family of sets}
If $B{a)a\in A}$ in an A-indexed family of sets \textit{a section} of B is a function $b:A \rightarrow \bigcup_{a \in A}{B(a)}$ such that
for each $a \in A, b(a) \in B(a)$. Here is a picture:  
\begin{figure}[h]

\begin{pspicture}(0,0.5)(9,5.5)
%\psgrid
%\putfamilyOfSets[rot]{basescale}{childscale}{x}{y}{childoffset}{nodeprefix}
\rput{*270}(1,5){
  \putfamilyOfSets[*270]{.75}{0.4}{2cm}{2.5cm}{2.3cm}{FAM}
  }
\rput[l](1.5,4.2){ \psframebox*{set A} }
\rput[l](6.5,2){ \psframebox*{set $B_a$, for every $a \in A$} }
\rput[l](6.5,1.5){ \psframebox*{elemnt b(a) of the set $B_a$, for every $a \in A$} }

\ncarc[nodesep=2pt,arcangle=20]{->}{FAMBASEl}{Ll}
\ncarc[nodesep=2pt,arcangle=20]{->}{FAMBASEm}{Mm}
\ncarc[nodesep=2pt,arcangle=-20]{->}{FAMBASEr}{Rr}
\end{pspicture}

\caption{A section b of A-indexed family of sets B}
\end{figure}

\subsubsection{Sections of a indexed family of family of sets}
Diagram here.
Mathematical definition.
\subsubsection{Sections in General}
Diagram here.
Mathematical definition.

\subsection{Section Structures}
Each element of $^\omega U$ has a set of sections and sections can be combined to give new ones by functional composition and the combination operation satisfies simple associativity/distributive laws (can't remember them right now - need to work them out again or find 30-odd year old notes).

We define a section structure to be any dependency graph endowed with a set of sections for each element and a composition operator which satisfies the aforesaid associativity/distributive laws.

If B is an A-indexed family of sets and then we say: $A \triangleleft B$ in $^\omega U$ and we represent sections of B as arrows.
$\langle$ Diagram Here $\rangle$
//

\noindent A cartesian section structure si defined to be a section structure with product operation obeying the following rules:
$\langle$ Can't remember them right now - need to work them out again? $\rangle$

\noindent What is troubling me here is whether I can manage to make sense of product types before I introduce sections or non-d-morphisms. I *can* but what meaningful things can be said: (i) each dependency graph can be represented by a hierarchical graph with product structure - use dependency coverings to define it (ii) I can define a network version of 
omega U (iii) I can show that models(realisations) of a dependency graph in network set correspond to realisations of its hierarchical equivalent in omega U.
  
  
\bibliography{../SharedBibliography/temp/bibliography}

\end{document}

\begin{frame}{Chemical Element Data}
\begin{center}
\scalebox{0.75}{

\begin{tabular}{|l|c|l|}
\hline
\rowcolor{myblue}\multicolumn{3}{l}{\colhead{allotrope}} \\
\hline
\rowcolor{myblue}\colhead{\pk{name}}  &
\colhead{\pk{symbol}}  &
\multicolumn{1}{|p{1.0cm}|}{\colhead{atomic no}}\\
 \hline
 Dioxygen & O &  8     \\
\hline
 Ozone    & O &  8     \\
\hline
Dichlorine & Cl & 17   \\
\hline
Rhombic & S &  16      \\
\hline
Monoclinic & S &   16  \\ 
\hline
Rhombic & Sn & 50      \\
\hline
\end{tabular}}
\hspace{0.5cm}
\onslide<2->{\scalebox{0.75}{
$
\begin{array}{c p{2.0cm} l}
                                                         && \attrtype{\Rnode{nametype}{\CDnametype}}               \\ [0.9cm]
\etype{\Rnode{allotrope}{allotrope}\Rnode{allotropeR}{}} && \attrtype{\Rnode{symboltype}{\CDsymboltype}  }         \\ [0.9cm]
                                                         && \attrtype{\Rnode{atomicnumbertypeL}{} \CDatomicnumbertype} 
\end{array}
$
\begin{arrows}
\newcommand{\CDelementnameattribute}{
\setlength{\arrnodesepA}{8pt}
\setlength{\arrnodesepB}{4pt}
\setlength{\arroffsetA}{5pt}
\setlength{\arroffsetB}{4pt}
\ncarr[7]{allotropeR}{nametype}
}
\CDelementnameattribute
\alabel{name}[0.3]

\newcommand{\CDsymbolattribute}{
\setlength{\arrnodesepA}{6pt}
\setlength{\arrnodesepB}{4pt}
\setlength{\arroffsetA}{0pt}
\setlength{\arroffsetB}{-2pt}
\ncarr[8]{allotropeR}{symboltype}
}
\CDsymbolattribute
\alabel{symbol}[0.4][0]

\newcommand{\CDatomicnumberattribute}{
\setlength{\arrnodesepA}{8pt}
\setlength{\arroffsetA}{-6pt}
\setlength{\arrnodesepB}{6pt}
\setlength{\arroffsetB}{0pt}
\ncarr[3]{allotropeR}{atomicnumbertypeL}
}
\CDatomicnumberattribute
\blabel{atomic\,no.}[0.4][0]


%barring of mono source i.e. uniqueness constraints start here
\onslide<3->{
% type chemical element
\CDsymbolattribute
\addedgebar
\CDelementnameattribute
\addedgebar}
\end{arrows}}} 
\end{center}
\begin{itemize}
	\item An example of data organised in a relational table.
	\onslide<3->{\item The key to this table is a combination of the name and the symbol columns.} 
	\onslide<4->{\item Now to meet normal form criteria each column needs ``\textit{depend on the key, the whole key and nothing but the key}''.} 
	\onslide<5->{ \item This table does not meet the 2NF criteria because the  `atomic no' column depends only on symbol.}
\end{itemize}
\end{frame}



\begin{frame}{Chemical Element Data}
\scalebox{0.75}{

\newcommand{\allotropeAwidth}{44pt}
\newcommand{\allotropeBwidth}{25pt}
\newcommand{\allotropeCwidth}{25pt}
\newcommand{\allotropeDwidth}{25pt}

\newcommand{\valencywidth}{15pt}


\hspace{-1cm}
\begin{tabular}{|l|l|l|l|c|l|}
\hline
\rowcolor{myblue}\colhead{name}    &
\colhead{symbol}  &
\multicolumn{1}{|p{1.0cm}|}{\colhead{atomic no}}&
\colhead{r.a.m.}   &
\begin{tabular}{|c|c|}
\multicolumn{1}{c}{\colhead{valency}} \\
\hline 
\colhead{number} \\
\hline 
\end{tabular} &
\begin{tabular}{|p{\allotropeAwidth}|p{\allotropeBwidth}|p{\allotropeCwidth}|p{\allotropeDwidth}|}
\hline
 \multicolumn{4}{c}{\colhead{allotrope}}     \\
 \hline
 \colhead{name}   & \colhead{m.p.}  & \colhead{b.p.}  & \colhead{density} \\
 \hline 
 \end{tabular} \\
 \hline
oxygen & O &  8 &  15.99 &   
\begin{tabular}{|p{\valencywidth}|}
\hline
2\\
\hline
\end{tabular} & 
\begin{tabular}{|p{\allotropeAwidth}|p{\allotropeBwidth}|p{\allotropeCwidth}|p{\allotropeDwidth}|}
\vpad{4}
\hline   
dioxygen  &  -218  &  -183  &  .0014 \\
\hline
Ozone     & -192 & -119 & .0032 \\
\hline 
\vpad{4} 
\end{tabular} \\
\hline
chlorine  &  Cl & 17 & 35.45 & 
\begin{tabular}{|p{\valencywidth}|}
\vpad{1}
\hline
1- \\
\hline
3+\\
\hline
5+\\
\hline
7+\\
\hline
\vpad{1} 
\end{tabular}
&
\begin{tabular}{|p{\allotropeAwidth}|p{\allotropeBwidth}|p{\allotropeCwidth}|p{\allotropeDwidth}|}
\hline
Dichlorine & -101 &   -34.6 &  >?? \\
\hline
\end{tabular}\\
\hline
sulphur & S &  16 &  32.06  &
\begin{tabular}{|p{\valencywidth}|}
\vpad{1} 
\hline
2\\
\hline
4\\
\hline
6\\
\hline
\multicolumn{1}{c}{}\\[-0.25cm] 
\end{tabular} &
\begin{tabular}{|p{\allotropeAwidth}|p{\allotropeBwidth}|p{\allotropeCwidth}|p{\allotropeDwidth}|}
\vpad{4} 
\hline 
Rhombic    & 112.8 &  ???  & 2.07 \\
\hline
Monoclinic & 119   & 444.6 &  1.96 \\
\hline 
\vpad{4} 
\end{tabular} \\
\hline
\end{tabular}
}

\begin{itemize}
	\item An example of data organised in nested relations.
	\item Underlined column indicates a key attribute and thereby a uniqueness constraint.
	\item Some data relationships are represented by physical containment relationships in the message structure i.e. by nesting.
\end{itemize}
\end{frame}

\lstdefinelanguage{MyIDL}[]{IDL}
{morekeywords={Set,Of, struct},sensitive=true}


\begin{frame}[fragile]{Description of Chemical Element Table}
In an IDL data specification it would look something like this:
\only<1>{\lstinputlisting
[basicstyle=\footnotesize,language=myIDL]
{../content/chemistryData/element.idl}
}
\onslide<2->{\lstinputlisting
[basicstyle=\footnotesize,keywordstyle={\ttfamily\color{green}\bfseries},language=myIDL]
{../content/chemistryData/element.idl}
}
\end{frame}

\begin{frame}{Chemical Compound Data}
\scalebox{0.75}{
\newcommand{\aliasnamewidth}{90pt}
\newcommand{\formulaAwidth}{35pt}
\newcommand{\formulaBwidth}{25pt}
\begin{tabular} {|l|l|l|l|}
\hline
\rowcolor{myblue}\multicolumn{4}{l}{\colhead{compound}} \\
\hline
\rowcolor{myblue}\colhead{\pk{name}} & \colhead{molar mass} &
\begin{tabular}{|p{\aliasnamewidth}|}
\multicolumn{1}{c}{\colhead{aliases}} \\
\multicolumn{1}{c}{\colhead{aliased name}} \\
\hline
\end{tabular} &
\begin{tabular}{|p{\formulaAwidth}|p{\formulaBwidth}|}
\multicolumn{2}{c}{\colhead{formula}} \\
\hline
	\colhead{\pkfk{symbol}} & \colhead{count} \\
\hline
\end{tabular} \\
\hline
aspirin	& 180.16 &
\begin{tabular}{|p{\aliasnamewidth}|}
\vpad{1}
\hline
acetylsalicylic acid \\
\hline
\vpad{1}
\end{tabular} &
\begin{tabular}{|p{\formulaAwidth}|p{\formulaBwidth}|}
\vpad{2}
\hline 
C&9\\
\hline
H&8\\
\hline
O&4\\
\hline
\vpad{2}
\end{tabular} \\
\hline
sodium chloride & 58.44	&	
\begin{tabular}{|p{\aliasnamewidth}|}
\vpad{1}
\hline 
common salt \\
\hline
halite \\
\hline
saline \\	
\hline
\vpad{1}
\end{tabular} &
\begin{tabular}{|p{\formulaAwidth}|p{\formulaBwidth}|}
\vpad{2}
\hline 	 
Na	& \\
\hline
Cl	& \\
\hline
\vpad{2}
\end{tabular} \\
\hline
water &	18.01 &
\begin{tabular}{|p{\aliasnamewidth}|}
\vpad{1}
\hline 
oxidane \\
\hline
\vpad{1}
\end{tabular} &
\begin{tabular}{|p{\formulaAwidth}|p{\formulaBwidth}|}
\vpad{2}
\hline 	 
H	& 2\\
\hline
O	& \\
\hline
\vpad{2}
\end{tabular} \\
\hline
\end{tabular}
		 
	
}
\begin{itemize}
\item $\largeAsterisk$ Asterisk here denotes a foreign key attribute. 
\item The 'symbol' attribute is a foreign key into the `element' table i.e.
	\begin{equation}
	              formula[symbol] \subseteq element[symbol]
	\end{equation}
\end{itemize}	
\end{frame}

\begin{frame}[fragile]{Message Structure}
A corresponding message structure might be defined 
in an IDL style language by:

\only<1>{\lstinputlisting
[basicstyle=\footnotesize,language=myIDL]
{../content/chemistryData/compound.idl}
}
\onslide<2->{\lstinputlisting
[basicstyle=\footnotesize,keywordstyle={\ttfamily\color{green}\bfseries},language=myIDL]
{../content/chemistryData/compound.idl}
}
\end{frame}


\newcommand{\occurence}{\parbox{1.5cm}{occurring element}}

\iffalse %remove this to reduce number of slides
\begin{frame}{Jackson Structure Diagrams}
\begin{tabular}{c c}
\scalebox{0.9}{\jacksonbinarydiagram{compound\kern0.1cm}{alias \kern1.2cm}{\occurence\kern0cm}}
&
\scalebox{0.9}{\jacksonbinarydiagram{element\kern0.4cm}{valency \kern0.8cm}{allotrope\kern0.3cm}}
\end{tabular}
\end{frame}


\begin{frame}{Jackson Structure Diagrams}
Diagrammatic niceties call an `'occuring element' an 'occurence'. 
\begin{center}
\begin{tabular}{c c}
\scalebox{0.9}{\jacksonbinarydiagram{compound\kern0.1cm}{alias \kern1.2cm}{occurence\kern0cm}}
&
\scalebox{0.9}{\jacksonbinarydiagram{element\kern0.4cm}{valency \kern0.8cm}{allotrope\kern0.3cm}}
\end{tabular}
\end{center}
\end{frame}


\begin{frame}{Bachman Structure Diagrams}
\begin{tabular}{c c}
\scalebox{0.9}{\bachmanbinarydiagram[left]{compound\kern0.1cm}{alias \kern1.2cm}{occurence}}
&
\scalebox{0.9}{\bachmanbinarydiagram[right]{element\kern0.4cm}{valency \kern0.8cm}{allotrope\kern0.3cm}}
\end{tabular}
\begin{itemize}
	\item This diagram describes the hierarchcal structure of the nested tables of data I started with.
	\item It doesn't show all the relationships in the data
	\pause \item there is a set of zero, one or more occurrences that point to any given element
	\pause \ncline[linewidth=1.5pt, linestyle=dashed]{->}{leftC}{rightA}
\end{itemize}
\end{frame}

\begin{frame}{Bachman Network Diagram!!}
\doublebachmannetworkdiagram{compound\kern0.1cm}{element\kern0.4cm}{alias \kern1.2cm}{occurence\kern0cm}{valency \kern0.8cm}{allotrope\kern0.3cm}
\end{frame}
\fi

\begin{frame}{Network Diagram Morphisms Version}
\doublecategorynetworkdiagram{compound\kern0.1cm}{element\kern0.4cm}{alias \kern1.2cm}{occurence\kern0cm}{valency \kern0.8cm}{allotrope\kern0.3cm}

\begin{itemize}
	\item arrows now represent functional many-one relationships - these are morphisms
\end{itemize}
\end{frame}


\begin{frame}{Network Diagram Distinguished Morphisms Version}
\contextualcategorynetworkdiagram{compound\kern0.1cm}{element\kern0.4cm}{alias \kern1.2cm}{occurence\kern0cm}{valency \kern0.8cm}{allotrope\kern0.3cm}
\begin{itemize}
	\item distinguished arrows now represent containment relationships
	\item others represent foreign key implemented relationships
\end{itemize}
\end{frame}

\begin{frame}{Network Diagram Distinguished Morphisms Version Reorganised}
\contextualcategorynetworkdiagramreorganised{compound\kern0.1cm}{element\kern0.4cm}{alias \kern1.2cm}{occurence\kern0cm}{valency \kern0.8cm}{allotrope\kern0.3cm}
\begin{itemize}
	\item distinguished arrows now represent containment relationships
	\item others represent foreign key implemented relationships
\end{itemize}
\end{frame}

\begin{frame}{Made clearer}
\contextualcategorynetworkdiagramtopologised{compound\kern0.1cm}{element\kern0.4cm}{alias \kern0.6cm}{occurence\kern0cm}{valency \kern0.8cm}{allotrope\kern0.3cm}
\end{frame}

\begin{frame}{What if}
\contextualcategorynetworkdiagramreorganisedtopologised{compound\kern0.1cm}{element\kern0.4cm}{alias \kern0.6cm}{occurence\kern0cm}{valency \kern0.8cm}{allotrope\kern0.3cm}
\end{frame}

\begin{frame}{Restructured Chemical Element Data}
\scalebox{0.75}{

\newcommand{\occursinAwidth}{88pt}
\newcommand{\occursinBwidth}{25pt}

\newcommand{\valencywidth}{15pt}


\hspace{-1cm}
\begin{tabular}{|c|l|l|l|l|}
\hline
\rowcolor{myblue}\multicolumn{5}{l}{\colhead{element}} \\
\hline
\rowcolor{myblue}\colhead{\pk{symbol}}  &
\colhead{name}    &
\multicolumn{1}{|p{1.0cm}|}{\colhead{atomic no}}&
\colhead{...}   &
\begin{tabular}{|p{\occursinAwidth}|p{\occursinBwidth}|}
\hline
 \multicolumn{2}{c}{\colhead{occurs in}}     \\
 \hline
 \colhead{\pkfk{compound}}   & \colhead{count}   \\
 \hline 
 \end{tabular} \\
 \hline
 O & oxygen & 8 &  ... &  
\begin{tabular}{|p{\occursinAwidth}|p{\occursinBwidth}|}
\vpad{2}
\hline   
water  &  1  \\
\hline
aspirin & 2  \\
\hline
thionyle chloride & 1 \\
\hline 
... & ... \\
\hline 
\vpad{2} 
\end{tabular} \\
\hline
Cl & chlorine  & 17 & ... & 
\begin{tabular}{|p{\occursinAwidth}|p{\occursinBwidth}|}
\hline
sodium chloride & 1   \\
\hline
thionyle chloride & 2   \\
\hline
... & ... \\
\hline
\end{tabular}\\
\hline
S & sulphur &  16 &  ...  &
\begin{tabular}{|p{\occursinAwidth}|p{\occursinBwidth}|}
\vpad{2} 
\hline 
sulphur dioxide    & 1  \\
\hline 
... & ... \\
\hline 
\vpad{2} 
\end{tabular} \\
\hline
\end{tabular}
}
\end{frame}

\begin{frame}{Purely Relational Data}
\begin{itemize}
	\item There is no nesting whatsoever
	\item All relationships (arrows) implemented by foreign keys
\end{itemize}
\begin{tabular}[t] {c c}
\multicolumn{2}{c}{\scalebox{0.75}{

\begin{tabular} {|l|l|}
\hline
\rowcolor{myblue}\multicolumn{2}{l}{\colhead{compound}} \\
\hline
\rowcolor{myblue}\colhead{\pk{name}} & \colhead{molar mass} \\
\hline
aspirin	& 180.16  \\
\hline
sodium chloride & 58.44	\\
\hline
water &	18.01  \\
\hline
\end{tabular}
		 
	
}} \\[1cm]
\scalebox{0.75}{

%
% formula(compound,element,count)
%
{ % enclosing scope for macro definitions


\newcommand{\tableheader}{\rowcolor{myblue}\multicolumn{2}{l}{\colhead{aliases}} \\}
\newcommand{\columnheaders}{\rowcolor{myblue}\colhead{\pk{alias}} & \colhead{\fk{aliased}} \\}

\begin{tabular} {|l|l|}
\hline
\tableheader
\hline
\columnheaders
\hline 
aspirin	& acetylsalicylic acid \\
\hline
sodium chloride & common salt \\
\hline
sodium chloride & halite \\
\hline
sodium chloride & saline \\	
\hline
water &	oxidane \\
\hline
\end{tabular}
}
		 
	
} &
\scalebox{0.75}{

%
% formula(compound,element,count)
%
{ % enclosing scope for macro definitions


\newcommand{\tableheader}{\rowcolor{myblue}\multicolumn{3}{l}{\colhead{formula}} \\}
\newcommand{\columnheaders}{\rowcolor{myblue}\colhead{\pkfk{compound}} & \colhead{\fk{\pkfk{element}}} & \colhead{count}\\}


\begin{tabular} {|l|l|l|}
\hline
\tableheader
\hline
\columnheaders
\hline 
aspirin	&   C&9 \\
\hline
aspirin	&   H&8 \\
\hline
aspirin	&   O&4 \\
\hline 	 
sodium chloride & Na & \\
\hline
sodium chloride & Cl & \\
\hline
water &	H	& 2\\
\hline
water &	O	&  \\
\hline
\end{tabular}
}
		 
	
}
\end{tabular}

\end{frame}




\begin{frame}{Logical Data Specification of the Chemical Elements as a Directed Graph}
\scalebox{0.9}{


$
\begin{array}{c c c p{3.5cm} l}
        &              &          &&\attrtype{\Rnode{symboltype}{\CDsymboltype}  }         \\ [0.25cm]
        &\etype{\Rnode{element}{chemical\ element}\Rnode{elementR}{}}\kern-1.5cm&  &&                 \\ [0.1cm]
		&              &      &&\attrtype{\Rnode{atomicnumbertypeL}{\CDatomicnumbertype}}   \\ [0.75cm]
        &              &                        &&\attrtype{\Rnode{nametype}{\CDnametype}}   \\ [0.45cm]
        &              & \etype{\Rnode{allotrope}{allotrope}\Rnode{allotropeR}{}}    &&      \\ [0.45cm]
        &              &                        &&\attrtype{\Rnode{floattype}{\CDfloattype}} \\ [0.75cm]
\etype{\Rnode{valency}{valency}\Rnode{valencyR}{}}\kern-1.5cm& &&&\attrtype{\Rnode{valencynumbertype}{\CDvalencynumbertype}}\\
\end{array}
$
\newcommand{\CDvalencyuparrow}{
\setlength{\arrnodesepA}{7pt}
\setlength{\arrnodesepB}{8pt}
\setlength{\arroffsetA}{0pt}
\setlength{\arroffsetB}{7pt}
\ncarr[10]{valency}{element}
}
\CDvalencyuparrow
\newcommand{\CDallotropeuparrow}{
\setlength{\arrnodesepB}{7pt}
\setlength{\arroffsetA}{0pt}
\setlength{\arroffsetB}{0pt}
\ncarr[-5]{allotrope}{element}
}
\CDallotropeuparrow
\newcommand{\CDsymbolattribute}{
\setlength{\arroffsetB}{0pt}
\setlength{\arrnodesepB}{3pt}
\setlength{\arroffsetB}{-2pt}
\setlength{\arroffsetA}{5pt}
\ncarr[15]{elementR}{symboltype}
}
\CDsymbolattribute
\alabel{symbol}[0.4][0]



\newcommand{\CDatomicnumberattribute}{
\setlength{\arroffsetA}{0pt}
\setlength{\arrnodesepB}{3pt}
\setlength{\arroffsetB}{0pt}
\ncarr[5]{elementR}{atomicnumbertypeL}
}
\CDatomicnumberattribute
\alabel{atomic\,number}[0.4]

\newcommand{\CDelementnameattribute}{
\setlength{\arroffsetA}{-5pt}
\setlength{\arroffsetB}{4pt}
\ncarr[5]{elementR}{nametype}
}
\CDelementnameattribute
\alabel{name}[0.3]

\newcommand{\CDallotropenameattribute}{
\setlength{\arroffsetA}{2pt}
\setlength{\arroffsetB}{-4pt}
\ncarr[5]{allotropeR}{nametype}
}
\CDallotropenameattribute
\alabel{name}[0.3]

\newcommand{\CDmeltingpointattribute}{
\setlength{\arroffsetA}{-2pt}
\setlength{\arroffsetB}{2pt}
\ncarr[5]{allotropeR}{floattype}
}
\CDmeltingpointattribute
\alabel{melting\,point}[0.35][-1]

\newcommand{\CDboilingpointattribute}
{\setlength{\arroffsetA}{-6pt}
\setlength{\arroffsetB}{-2pt}
\ncarr[-5]{allotropeR}{floattype}
}
\CDboilingpointattribute
\blabel{boiling\,point}[0.35][-1]

\newcommand{\CDnumberattribute}
{\setlength{\arroffsetA}{2pt}
\setlength{\arroffsetB}{-2pt}
\ncarr[5]{valencyR}{valencynumbertype}
}
\CDnumberattribute
\alabel{number}[0.3]

%barring of mono sources i.e. uniqueness constraints start here
\pause
% type chemical element
\CDsymbolattribute
\addedgebar

% type valency
\CDvalencyuparrow
\addedgebar
\CDnumberattribute
\addedgebar

% type allotrope
\CDallotropeuparrow
\addedgebar
\CDallotropenameattribute
\addedgebar

% Second uniqueness constraint for type chemical element- use double bar
\CDatomicnumberattribute
\addedgedoublebar
% Third uniqueness constraint for chemical element - use triplebar
\CDelementnameattribute
\addedgetriplebar
} 
\end{frame}








