

\usepackage{mathptmx}
\usepackage{amsfonts}
\usepackage{wasysym}
\usepackage{url}
\usepackage{hyperref}

\newcommand{\sharedmacros}{../../SharedMacros}
%\usepackage{imakeidx}
\usepackage{framed}
\makeindex[name=definitions, title=Index of Definitions]
\makeindex[name=lemmas, title=Index of Lemmas]



\newcommand{\seenudgeup}[1]{\rule{0.1cm}{#1}}

\newcommand{\seenudgedown}[1]{\rule[-#1]{0.1cm}{0.1cm}}

\newcommand{\nudgeup}[1]{\rule{0cm}{#1}}

\newcommand{\nudgedown}[1]{\rule[-#1]{0cm}{0.1cm}}

\definecolor{highlight}{cmyk}{0,0,0.7,0}
\newcommand{\commentary}[1]{\marginpar{\footnotesize #1}}
\newcommand{\highlight}[1]{\colorbox{highlight}{#1}}
\newcommand{\whitelight}[1]{\colorbox{white}{#1}}
\newcommand{\term}[1]{\textit{#1}\commentary{\colorbox{lightgray}{\textit{#1}}}\index[definitions]{#1}}
\newcommand{\llabel}[1]{\label{#1}\commentary{\colorbox{pink}{\scriptsize{#1}}}\index[lemmas]{#1}}
\newcommand{\lref}[1]{\ref{#1}\colorbox{pink}{\scriptsize{#1}}\index[lemmas]{#1!use of}}

\newcommand{\daynote}[1]{\commentary{See day notes #1.}}

\newcommand{\newt}[1]{\colorbox{yellow}{#1}}
\newenvironment{newtt}
{  \colorbox{yellow}{$[$ ...} 
}
{  \colorbox{yellow}{... $]$}
}
\newcommand{\oldt}[1]{\colorbox{red}{\sout{#1}}}
\newenvironment{oldtt}
{  \colorbox{red}{$[$ ...} 
}
{  \colorbox{red}{... $]$}
}

\newcommand{\reinstatet}[1]{\colorbox{lime}{#1}}
\newenvironment{reinstatett}
{  \colorbox{lime}{$[$ ...}
}
{  \colorbox{lime}{... $]$}
}

\newcommand{\tbd}{\highlight{TBD}}

%ithprojection function
\newcommand{\proji}[1]{\pi_#1}


\newenvironment{aside}
{\begin{framed}
\textbf{Aside}
}
{
\end{framed}
}

\newenvironment{notebox}[1][Note]
{\begin{framed}
\textbf{#1}
}
{
\end{framed}
}

\newenvironment{categoricalaside}
{\begin{framed}
\textbf{Categorical Aside}
}
{
\end{framed}
}

\newenvironment{noteforfuture}
{\begin{framed}
\textbf{Note For Future}
}
{
\end{framed}
}

\newenvironment{problem}
{\begin{framed}
\textbf{Problem}
}
{
\end{framed}
}

\newenvironment{key}
{
\begin{tabular}{c l p{4cm}}
KEY && \\
\hline
}
{
\end{tabular}
}

\newcommand{\keyentry}[3]{#1 & #2 & #3 \\} 


%quine quote
\newcommand{\qq}[1]{
\left\ulcorner#1\right\urcorner
}

%single quote
\newcommand{\sq}[1]{
\textnormal{\textquotesingle}#1\textnormal{\textquotesingle}
}

%lower quine quote
\newcommand{\lqq}[1]{
\left\llcorner #1\right\lrcorner
}


%from berkley
\newcommand{\langl}{\begin{picture}(4.5,7)
\put(1.1,2.5){\rotatebox{60}{\line(1,0){5.5}}}
\put(1.1,2.5){\rotatebox{300}{\line(1,0){5.5}}}
\end{picture}}
\newcommand{\rangl}{\begin{picture}(4.5,7)
\put(.9,2.5){\rotatebox{120}{\line(1,0){5.5}}}
\put(.9,2.5){\rotatebox{240}{\line(1,0){5.5}}}
\end{picture}}
\newcommand{\lang}{\begin{picture}(5,7)\put(1.1,2.5){\rotatebox{45}{\line(1,0){6.0}}}\put(1.1,2.5){\rotatebox{315}{\line(1,0){6.0}}}\end{picture}}
\newcommand{\rang}{\begin{picture}(5,7)\put(.1,2.5){\rotatebox{135}{\line(1,0){6.0}}}\put(.1,2.5){\rotatebox{225}{\line(1,0){6.0}}}\end{picture}}
%Try sharper tuple brackets -- except gives errors nested in captions so comment out
%\renewcommand{\tuple}[1]{\lang #1 \rang}

\newcommand{\setsuchthat}[2]{\left\{#1 \ \middle|\ #2\right\}}
\newcommand{\set}[1]{\left\{#1\right\}} 

% one to n - wanton
\newcommand{\wanton}[1]{#1_1,...#1_n}
\newcommand{\n}{1...n}
\newcommand{\fn}{\wanton{f}}
\newcommand{\gn}{\wanton{g}}
\newcommand{\pn}{\wanton{p}}
\newcommand{\qn}{\wanton{q}}
\newcommand{\qnprime}{\wanton{q'}}
\newcommand{\tn}{\wanton{t}}
\newcommand{\xn}{\wanton{x}}
\newcommand{\xnp}{\wanton{x'}}
\newcommand{\yn}{\wanton{y}}
\newcommand{\An}{\wanton{A}}
\newcommand{\Bn}{\wanton{B}}
\newcommand{\Cn}{\wanton{C}}
\newcommand{\ntuple}[1]{\tuple{\wanton{#1}}}
\newcommand{\wantom}[2][]{#2_1,...#2_{m#1}}
\newcommand{\m}{1...m}
\newcommand{\mtuple}[1]{\tuple{#1_1,...#1_m}}
\newcommand{\gm}{\wantom{g}}
\newcommand{\qm}{\wantom{q}}
\newcommand{\sm}[1][]{\wantom[#1]{s}}
\newcommand{\smp}{\wantom{s'}}
\newcommand{\ym}{\wantom{y}}
\newcommand{\Bm}{\wantom{B}}
\newcommand {\bntuple}{\ensuremath{\ntuple{b}}}
\newcommand {\fntuple}{\ensuremath{\ntuple{f}}}
\newcommand {\fnptuple}{\ensuremath{\ntuple{f}}}
\newcommand {\pntuple}{\ensuremath{\ntuple{p}}}
\newcommand {\qntuple}{\ensuremath{\ntuple{q}}}
\newcommand {\qnptuple}{\ensuremath{\ntuple{q'}}}
\newcommand {\qmtuple}{\ensuremath{\mtuple{q}}}
\newcommand {\sntuple}{\ensuremath{\ntuple{s}}}
\newcommand {\xntuple}{\ensuremath{\ntuple{x}}}
\newcommand {\xnptuple}{\ensuremath{\ntuple{x'}}}
\newcommand {\ymtuple}{\ensuremath{\mtuple{y}}}
\newcommand{\idef}[1][n]{1 \leq i \leq #1}
\newcommand{\jdef}[1][m]{1 \leq j \leq #1}
\newcommand{\kdef}[1][l]{1 \leq k \leq #1}
\newcommand{\foreachi}[1][n]{for each $i$, $1 \leq i \leq #1$}
\newcommand{\foreachj}[1][m]{for each $j$, $1 \leq j \leq #1$}
\newcommand{\foreachk}[1][l]{for each $k$, $1 \leq k \leq #1$}
\newcommand{\Foreachi}[1][n]{For each $i$, $1 \leq i \leq #1$}
\newcommand{\Foreachj}[1][m]{For each $j$, $1 \leq j \leq #1$}
\newcommand{\Foreachk}[1][l]{For each $k$, $1 \leq k \leq #1$}
\newcommand{\forsomei}[1][n]{for some $i$, $1 \leq i \leq #1$}
\newcommand{\forsomej}[1][m]{for some $j$, $1 \leq j \leq #1$}
\newcommand{\forsomek}[1][l]{for some $k$, $1 \leq k \leq #1$}
\newcommand{\wherei}[1][n]{where $1 \leq i \leq #1$}
\newcommand{\wherej}[1][m]{where $1 \leq j \leq #1$}
\newcommand{\wherek}[1][l]{where $1 \leq k \leq #1$}


\newcommand{\fundep}[3]{#2 \xrightarrow{#1} #3}  %where does this belong? xxxx
% Following used for notes -- indented numbered paras

\newcounter{para}
\newlength{\oldparindent}
\setlength{\oldparindent}{\parindent} % Save \parindent before of change
\newcommand{\ind}{\hspace*{\oldparindent}}
\newcommand\note{
%\setlength{\parskip}{0.5\baselineskip} % Definition of `parskip`
\setlength{\parindent}{0pt}
\par\ind\refstepcounter{para}\thepara.\space
\setlength{\parindent}{\oldparindent}
}


    %causes problems when used with bamer

%ccategories.macros.tex 

% Macros for diagrams in contextual categories and related categories

\usepackage{twoopt}
\usepackage{scalerel} 
\usepackage{xargs}

%\usepackage{mathabx}  %Caused font problems
%\usepackage{MnSymbol}  % caused font problems

\newcommand{\conu}
{\mathbf{C}(U)}

\newcommand{\depu}
{\mathbf{D}(U)}


\newcommand{\reqt}{\textbf{R}}
\newcommand{\reqtc}[1][\catc]{\reqt_{#1}}
\newcommand{\reqtcp}[1][\catcp]{\reqt_{#1}}



\newcommand{\cat}[1]{\textbf{#1}}

\newcommand{\catc}{\cat{C}}
\newcommand{\catcw}{\cat{C}\ }
\newcommand{\catcp}[1][C]{\textbf{#1}'}
\newcommand{\catcpp}[1][C]{\textbf{#1}''}
\newcommand{\obj}[1]{\ensuremath{|\cat{#1}|}}
\newcommand{\ccat}[1][C]{\ensuremath{\mathbb{#1}} }
\newcommand{\ccatc}{contextual category \ccat}
\newcommand{\cobj}[2][]{\ensuremath{|\ccat[#2]|_{#1}}}
\newcommand{\cslice}[2]{\ensuremath{\ccat[#1]_{#2}}}
\newcommand{\csliceobj}[3][]{\ensuremath{|\mathbb{#2}_{#3}|_{#1} }}
\newcommand{\varset}[1][]{\ensuremath{V_{#1} }}
\newcommand{\localvarsets}{\ensuremath{\mathcal{V} }}
\newcommand{\Fam}{\ensuremath{\mathbb{F\mathrm{am}} }}
\newcommand{\Fin}{\ensuremath{\textbf{Fin}} }
\newcommand{\Finp}{\ensuremath{\textbf{Finp}} }
\newcommand{\Po}{\ensuremath{\textbf{Po}} }
\newcommand{\Famslice}[1]{\ensuremath{\mathbb{F\mathrm{am}}_{#1} }}
\newcommand{\Famobj}[1][]{\ensuremath{|\mathbb{F\mathrm{am}}|_{#1} }}
\newcommand{\Famsliceobj}[2][]{\ensuremath{|\mathbb{F\mathrm{am}}_{#2}|_{#1} }}
\newcommand{\morph}{\rightarrow}
\newcommand{\epi}{\twoheadrightarrow}
\newcommand{\base}{\triangleleft}
\newcommand{\comp}{\circ}
\newcommand{\cross}{\otimes}
\newcommand{\pc}[2]{d^{#1}_{#2}}
\newcommand{\sub}{^*}
\newcommand{\diag}{\delta}
\newcommand{\pbase}[1]{\tilde{#1}}
\newcommand{\tuple}[1]{\langle#1\rangle}
\newcommand{\ndidly}{\ensuremath{\Join_n}}

\newcommand{\product}[1]{\bigtimes_{#1}}
\newcommand{\productn}{\product{n}}
\newcommand{\crossx}[3]{#1 \underset{#3}{\cross} #2}
\newcommand{\fibrex}[3]{#1 \underset{#3}{\Join} #2}
\newcommand{\powerset}{\mathcal{P}}
\newcommand{\primeds}[1]{
\ensuremath{\mathcal{P}(#1)} }
\newcommand{\compset}{\ \dot{\circ}\, }

% darrow
%\newcommand{\darrow}{\rightarrowtriangle} %use \smorph instead
\newcommand{\smorph}{\rightarrowtriangle}

 
\newcommand\dhead{\scaleobj{0.6}{\triangleright}}
%\newcommand{\dmorph}{\, \mbox{---} \! \cdot \! \raisebox{1.1pt}{\dhead}}    % dot style
\newcommand{\dmorph}{\, \mbox{---}\kern-1pt\raisebox{1.1pt}{\dhead\kern-1.75pt\dhead}}\,     % double triangle style

% projection tree
%\newcommand{\proj}[2]{proj_{#2}(#1)}

\newcommand{\proj}[2]{
\ensuremath{\mathcal{P}_{#2}(#1)} }

%pstrick supplements for arrows

\newlength{\arrnodesepA}
\newlength{\arrnodesepB}
\newlength{\arroffsetA}
\newlength{\arroffsetB}

%Modified to 2pt from 0pt on 23 July 2018
\newcommand{\arreset}{
\setlength{\arrnodesepA}{2pt}
\setlength{\arrnodesepB}{2pt}
\setlength{\arroffsetA}{0pt}
\setlength{\arroffsetB}{0pt}
}
\arreset

\newcommand{\ncarr}[3][0]{\ncarc[arcangle=#1,nodesepA=\arrnodesepA,nodesepB=\arrnodesepB,offsetA=\arroffsetA,offsetB=\arroffsetB,arrowsize=5pt,arrowinset=0.7]{->}{#2}{#3}}
\newcommand{\ncdarr}[3][0]{\ncarc[linestyle=dashed,arcangle=#1,nodesepA=\arrnodesepA,nodesepB=\arrnodesepB,offsetA=\arroffsetA,offsetB=\arroffsetB,arrowsize=5pt,arrowinset=0.7]{->}{#2}{#3}}
\newcommand{\jcbarr}[4][0]{ % ncbarr is defined in some thridy party package so do not use!\emph{}
\ncarr[#1]{#3}{#4}
\nbput[labelsep=2pt]{\footnotesize $#2$}
}

\newcommand{\ncaarr}[4][0]{
\ncarr[#1]{#3}{#4}
\naput[labelsep=2pt]{\footnotesize $#2$}
}

% \alabel{label}[npos][labelsep_pts]
\newcommandx*\alabel[3][2=0.5,3=2,usedefault]{\naput[labelsep=#3pt,npos=#2]{\footnotesize $#1$}}
% \blabel{label}[npos][labelsep_pts]
\newcommandx*\blabel[3][2=0.5,3=2,usedefault]{\nbput[labelsep=#3pt,npos=#2]{\footnotesize $#1$}}


\newif \ifbars
% to supress display of bars use \barsfalse to swith them on use \barstrue
\barstrue 
% \idcomp mark an arrow as one component of an identifier
\newcommand{\idcomp}{\ifbars{\ncput[npos=0, nrot=:U]{\psline(0.2,-0.075)(0.2,0.075)}}\fi}  %add a bar to a node connection arrow
% pstrick supplements for s-arrows (previous name for d-arrow - should convert}

\newlength{\sarnodesepA}
\newlength{\sarnodesepB}
\newlength{\saroffsetA}
\newlength{\saroffsetB}
\newlength{\sarnodesepAsav}
\newlength{\sarnodesepBsav}

\newcommand{\sarreset}{
\setlength{\sarnodesepA}{0pt}
\setlength{\sarnodesepB}{0pt}
\setlength{\saroffsetA}{0pt}
\setlength{\saroffsetB}{0pt}
}

\sarreset

% sar - S-arrow
\newcommand{\ncsar}[3][0]{
\setlength{\sarnodesepAsav}{\sarnodesepA}
\setlength{\sarnodesepBsav}{\sarnodesepB}
\addtolength{\sarnodesepA}{3pt}
\addtolength{\sarnodesepB}{7pt}
\ncarc[nodesepA=\sarnodesepA,nodesepB=\sarnodesepB,offsetA=\saroffsetA,offsetB=\saroffsetB,arcangle=#1]{-}{#2}{#3}
\ncput[nrot=:R,npos=1]{\pstriangle(0,0)(.2,.2)}
\setlength{\sarnodesepA}{\sarnodesepAsav}
\setlength{\sarnodesepB}{\sarnodesepBsav}
}


% bsar - below labelled S-arrow
\newcommand{\ncbsar}[4][0]{
\ncsar[#1]{#3}{#4}
\nbput[labelsep=2pt]{\footnotesize $#2$}
}
% asar - above labelled S-arrow
\newcommand{\ncasar}[4][0]{
\ncsar[#1]{#3}{#4}
\naput[labelsep=2pt]{\footnotesize $#2$}
}

% OLD cdar - composite dependency arrow - dot tyle
\iffalse
\newcommand{\nccdar}[3][0]{
\setlength{\sarnodesepAsav}{\sarnodesepA}
\setlength{\sarnodesepBsav}{\sarnodesepB}
\addtolength{\sarnodesepA}{3pt}
\addtolength{\sarnodesepB}{11pt}
\ncarc[nodesepA=\sarnodesepA,nodesepB=\sarnodesepB,offsetA=\saroffsetA,offsetB=\saroffsetB,arcangle=#1]{-}{#2}{#3}
\ncput[nrot=:R,npos=1]{\pstriangle(0,0.1)(.2,.2)}
\ncput[nrot=:R,npos=1]{\psdot[dotsize=1pt](-0.0075,0.05)}   %!!
\setlength{\sarnodesepA}{\sarnodesepAsav}
\setlength{\sarnodesepB}{\sarnodesepBsav}
}
\fi

% cdar - composite dependency arrow Mark II - double trangle style
\newcommand{\nccdar}[3][0]{
\setlength{\sarnodesepAsav}{\sarnodesepA}
\setlength{\sarnodesepBsav}{\sarnodesepB}
\addtolength{\sarnodesepA}{3pt}
\addtolength{\sarnodesepB}{13pt}
\ncarc[nodesepA=\sarnodesepA,nodesepB=\sarnodesepB,offsetA=\saroffsetA,offsetB=\saroffsetB,arcangle=#1]{-}{#2}{#3}
\ncput[nrot=:R,npos=1]{\pstriangle(0,0)(.2,.2)}
\ncput[nrot=:R,npos=1]{\pstriangle(0,0.2)(.2,.2)}
\setlength{\sarnodesepA}{\sarnodesepAsav}
\setlength{\sarnodesepB}{\sarnodesepBsav}
}


% bcdar - below labelled composite dependency arrow
\newcommand{\ncbcdar}[4][0]{
\nccdar[#1]{#3}{#4}
\nbput[labelsep=2pt]{\footnotesize $#2$}
}
% acdar - above labelled composite dependency arrow
\newcommand{\ncacdar}[4][0]{
\nccdar[#1]{#3}{#4}
\naput[labelsep=2pt]{\footnotesize $#2$}
}


% rsar - recursive S-arrow
\newcommand{\ncrsar}[2]{
\setlength{\sarnodesepAsav}{\sarnodesepA}
\setlength{\sarnodesepBsav}{\sarnodesepB}
\addtolength{\sarnodesepA}{3pt}
\addtolength{\sarnodesepB}{7pt}
\ncloop[nodesepA=\sarnodesepA,nodesepB=\sarnodesepB,
        offsetA=\saroffsetA,offsetB=\saroffsetB,
        armA=0.7cm,armB=0.6cm,angleA=90,angleB=-90,loopsize=-1,linearc=0.4
				]{-}{#1}{#2}
\ncput[nrot=:R,npos=5]{\pstriangle(0,0)(.2,.2)}
\setlength{\sarnodesepA}{\sarnodesepAsav}
\setlength{\sarnodesepB}{\sarnodesepBsav}
}

% pstrick supplements for multi-arrows

\newlength{\marnodesepA}
\newlength{\marnodesepB}
\newlength{\maroffsetB}
\newlength{\marnodesepBsav}

\newcommand{\marreset}{
\setlength{\marnodesepA}{0pt}
\setlength{\marnodesepB}{0pt}
\setlength{\maroffsetB}{0pt}
}

\marreset

%ncmarr[#1 arcangle1][#2 arcangle2]{#3 name}{#4 domain1}{#5 domain2}{#6 junction}{#7 codomain}
\newcommandtwoopt{\ncmarr}[6][8][8]{%
\ncarc[nodesepA=\marnodesepA,nodesepB=0,arcangle=#1]{-}{#3}{#5}
\ncarc[nodesepB=0,arcangle=-#1]{-}{#4}{#5}
\ncarc[arcangle=#2,nodesepB=\marnodesepB,offsetB=\maroffsetB]{->}{#5}{#6}
}%


\newcommandtwoopt{\nchmarr}[6][8][8]{%
\ncarc[nodesepA=\marnodesepA,nodesepB=0,arcangle=#1]{-}{#3}{#5}
\ncarc[nodesepB=0,arcangle=#1]{-}{#4}{#5}
\ncarc[arcangle=#2,nodesepB=\marnodesepB,offsetB=\maroffsetB]{->}{#5}{#6}
}%

\newcommandtwoopt{\ncamarr}[7][8][8]{%
\ncmarr[#1][#2]{#4}{#5}{#6}{#7}
\naput[npos=.05]{$#3$}
}%
\newcommandtwoopt{\ncbmarr}[7][8][8]{%
\ncmarr[#1][#2]{#4}{#5}{#6}{#7}
\nbput[npos=.05]{$#3$}
}%

\newcommandtwoopt{\ncbhmarr}[7][8][8]{%
\nchmarr[#1][#2]{#4}{#5}{#6}{#7}
\nbput[npos=.05]{$#3$}
}%

\newcommandtwoopt{\ncmarrr}[7][8][8]{
\ncarc[nodesepB=0,arcangle=#1]{-}{#3}{#6}
\ncline[nodesepB=0]{-}{#4}{#6}
\ncarc[nodesepB=0,arcangle=-#1]{-}{#5}{#6}
\ncarc[nodesepA=0,arcangle=#2]{->}{#6}{#7}
}

\newcommandtwoopt{\ncamarrr}[8][8][8]{
\ncmarrr[#1][#2]{#4}{#5}{#6}{#7}{#8}
\naput[npos=.05]{$#3$}
}
\newcommandtwoopt{\ncbmarrr}[8][8][8]{
\ncmarrr[#1][#2]{#4}{#5}{#6}{#7}{#8}
\nbput[npos=.05]{$#3$}
}


% 6 June 2020
% Edges representing attributes and relationship graphs
%  Ep   - partial
%  Epm  - partial mono
%  Epe  - partial epi
%  Epme - partial mono epi
%  Et   - total
%  Etm  - total mono
%  Ete  - total epi
%  Etme - total mono epi
%  recursive edges (use nccircle)
%  rEp   - partial
%  rEpm  - partial mono
%  rEpe  - partial epi
%  rEpme - partial mono epi
%  rEt   - total
%  rEtm  - total mono
%  rEte  - total epi
%  rEtme - total mono epi

\newcounter{EangleA}
\newcounter{EangleB}
\newcounter{EmidangleA}
\newcounter{EmidangleB}

% Ep - Edge partial
\newcommandtwoopt{\Ep}[4][0][0]{
\crowsfootedEdge{#1}{#2}{#3}{#4}{dashed}{dashed}
}



% Epm - Edge partial mono
\newcommandtwoopt{\Epm}[4][0][0]{
\monoEdge{#1}{#2}{#3}{#4}{dashed}{dashed}
}


% Epe - Edge partial epi
\newcommandtwoopt{\Epe}[4][0][0]{
\crowsfootedEdge{#1}{#2}{#3}{#4}{dashed}{solid}
}

% Epme - Edge partial mono epi
\newcommandtwoopt{\Epme}[4][0][0]{
\monoEdge{#1}{#2}{#3}{#4}{dashed}{solid}
}

% Et - Edge total
\newcommandtwoopt{\Et}[4][0][0]{
\crowsfootedEdge{#1}{#2}{#3}{#4}{solid}{dashed}
}

% Etm - Edge total mono
\newcommandtwoopt{\Etm}[4][0][0]{
\monoEdge{#1}{#2}{#3}{#4}{solid}{dashed}
}

% Ete - Edge total epi
\newcommandtwoopt{\Ete}[4][0][0]{
\crowsfootedEdge{#1}{#2}{#3}{#4}{solid}{solid}
}

% Etme - Edge total mono epi
\newcommandtwoopt{\Etme}[4][0][0]{
\monoEdge{#1}{#2}{#3}{#4}{solid}{solid}
}

% crowsfootedEdge - \crowsfootedEdge[angleA][midpointangle]{startnode}{endnode}[startstyle][endstyle]
\newcommand{\crowsfootedEdge}[6]{
\setlength{\sarnodesepAsav}{\sarnodesepA}
\setlength{\sarnodesepBsav}{\sarnodesepB}
\addtolength{\sarnodesepA}{3pt}
\addtolength{\sarnodesepB}{3pt}
\setcounter{EangleA}{ #1 + #2}
\setcounter{EangleB}{180  - #1 + #2}
\setcounter{EmidangleA}{#2}
\setcounter{EmidangleB}{#2 + 180}
\nccurve[nodesepA=\sarnodesepA,nodesepB=\sarnodesepB,offsetA=\saroffsetA,offsetB=\saroffsetB,angleA=\theEangleA, angleB=\theEangleB,linestyle=none,linewidth=0]{->}{#3}{#4}
\ncput[nrot=:R,npos=0]{\psline(0,.1)(.075,0)}
\ncput[nrot=:R,npos=0]{\psline(0,.1)(-0.075,0)}
\ncput{\pnode(0,0){xxx}}
\nccurve[nodesepA=0,nodesepB=\sarnodesepB,offsetA=0,offsetB=\saroffsetB,angleA=\theEmidangleA, angleB=\theEangleB, linestyle=#6]{->}{xxx}{#4}
%the following provides context for any following label
\nccurve[nodesepA=\sarnodesepA,nodesepB=0,offsetA=\saroffsetA,offsetB=0,angleA=\theEangleA, angleB=\theEmidangleB,linestyle=#5]{-}{#3}{xxx}
\setlength{\sarnodesepA}{\sarnodesepAsav}
\setlength{\sarnodesepB}{\sarnodesepBsav}
}

% monoEdge - \monoEdge[angleA][midpointangle]{startnode}{endnode}[startstyle][endstyle]
\newcommand{\monoEdge}[6]{ 
\setlength{\sarnodesepAsav}{\sarnodesepA}
\setlength{\sarnodesepBsav}{\sarnodesepB}
\addtolength{\sarnodesepA}{3pt}
\addtolength{\sarnodesepB}{3pt}
\setcounter{EangleA}{ #1 + #2}
\setcounter{EangleB}{180  - #1 + #2}
\setcounter{EmidangleA}{#2}
\setcounter{EmidangleB}{#2 + 180}
\nccurve[nodesepA=\sarnodesepA,nodesepB=\sarnodesepB,offsetA=\saroffsetA,offsetB=\saroffsetB,angleA=\theEangleA, angleB=\theEangleB,linestyle=none,linewidth=0]{->}{#3}{#4}
\ncput{\pnode(0,0){xxx}}
\nccurve[nodesepA=0,nodesepB=\sarnodesepB,offsetA=0,offsetB=\saroffsetB,angleA=\theEmidangleA, angleB=\theEangleB, linestyle=#6]{->}{xxx}{#4}
%the following provides context for any following label
\nccurve[nodesepA=\sarnodesepA,nodesepB=0,offsetA=\saroffsetA,offsetB=0,angleA=\theEangleA, angleB=\theEmidangleB,linestyle=#5]{-}{#3}{xxx}
\setlength{\sarnodesepA}{\sarnodesepAsav}
\setlength{\sarnodesepB}{\sarnodesepBsav}
}


\newcounter{EangleGiven}
\newcounter{EangleComplementary}
\newcounter{EangleStartCorrected}
\newcounter{EangleEndCorrected}


%  rEp   - recursive Edge partial
\newcommand{\rEp}[2][0]{
\setcounter{EangleGiven}{#1}
\setcounter{EangleStartCorrected}{#1-10} %correction required because for nccurve unlike nccircle angle measured at boundary not at centre of node
\setcounter{EangleEndCorrected}{#1+180+10} %correction required because angle measured at boundary not at centre of node
\setcounter{EangleComplementary}{#1 + 180}
\nccircle[angleA=\theEangleComplementary, nodesep=0pt, linestyle=none]{-}{#2}{.4cm} % an invisible circle to hang the midpoint from
\ncput{\pnode(0,0){midpoint}}                                         
\nccurve[nodesepA=1pt,nodesepB=0pt,offsetA=0pt,offsetB=0pt,angleA=\theEangleStartCorrected, angleB=\theEangleGiven, ncurv=1.359, linecolor=black, linestyle=dashed]{-}{#2}{midpoint}
\ncput[nrot=:R,npos=0]{\psline(0,.1)(.075,0)}
\ncput[nrot=:R,npos=0]{\psline(0,.1)(-0.075,0)}
\nccurve[nodesepA=0pt,nodesepB=2pt,offsetA=0pt,offsetB=0pt,angleA=\theEangleComplementary, angleB=\theEangleEndCorrected, ncurv=1.359, linestyle=dashed]{-}{midpoint}{#2}
% 1.359 is e/2 happenchance or algorithmically necessary???
% now draw arrowhead -- dont include in the nccurve because this alters the line position - a strange feature of pstruicks
\ncput[npos=0.9]{\pnode(0,0){yyy}}
\ncline{->}{yyy}{#2}
% repeat from earlier to provide context for label that might follow
\nccurve[nodesepA=1pt,nodesepB=0pt,offsetA=0pt,offsetB=0pt,angleA=\theEangleStartCorrected, angleB=\theEangleGiven, ncurv=1.359, linecolor=black, linestyle=dashed]{-}{#2}{midpoint} 
} 

%  rEpm  - recursive Edge partial mono
\newcommand{\rEpm}[2][0]{
\setcounter{EangleGiven}{#1}
\setcounter{EangleStartCorrected}{#1-10} %correction required because for nccurve unlike nccircle angle measured at boundary not at centre of node
\setcounter{EangleEndCorrected}{#1+180+10} %correction required because angle measured at boundary not at centre of node
\setcounter{EangleComplementary}{#1 + 180}
\nccircle[angleA=\theEangleComplementary, nodesep=0pt, linestyle=none]{-}{#2}{.4cm} % an invisible circle to hang the midpoint from
\ncput{\pnode(0,0){midpoint}}   
\nccurve[nodesepA=0pt,nodesepB=2pt,offsetA=0pt,offsetB=0pt,angleA=\theEangleComplementary, angleB=\theEangleEndCorrected, ncurv=1.359, linestyle=dashed]{-}{midpoint}{#2}
% 1.359 is e/2 happenchance or algorithmically necessary???
% now draw arrowhead -- dont include in the nccurve because this alters the line position - a strange feature of pstruicks
\ncput[npos=0.9]{\pnode(0,0){yyy}}
\ncline{->}{yyy}{#2}
% last to provide context for label that might follow
\nccurve[nodesepA=1pt,nodesepB=0pt,offsetA=0pt,offsetB=0pt,angleA=\theEangleStartCorrected, angleB=\theEangleGiven, ncurv=1.359, linecolor=black, linestyle=dashed]{-}{#2}{midpoint} 
}

%  rEpe  - recursive Edge partial epi
\newcommand{\rEpe}[2][0]{
\setcounter{EangleGiven}{#1}
\setcounter{EangleStartCorrected}{#1-10} %correction required because for nccurve unlike nccircle angle measured at boundary not at centre of node
\setcounter{EangleEndCorrected}{#1+180+10} %correction required because angle measured at boundary not at centre of node
\setcounter{EangleComplementary}{#1 + 180}
\nccircle[angleA=\theEangleComplementary, nodesep=0pt, linestyle=none]{-}{#2}{.4cm} % an invisible circle to hang the midpoint from
\ncput{\pnode(0,0){midpoint}}                                         
\nccurve[nodesepA=1pt,nodesepB=0pt,offsetA=0pt,offsetB=0pt,angleA=\theEangleStartCorrected, angleB=\theEangleGiven, ncurv=1.359, linecolor=black, linestyle=dashed]{-}{#2}{midpoint}
\ncput[nrot=:R,npos=0]{\psline(0,.1)(.075,0)}
\ncput[nrot=:R,npos=0]{\psline(0,.1)(-0.075,0)}
\nccurve[nodesepA=0pt,nodesepB=2pt,offsetA=0pt,offsetB=0pt,angleA=\theEangleComplementary, angleB=\theEangleEndCorrected, ncurv=1.359]{-}{midpoint}{#2}
% 1.359 is e/2 happenchance or algorithmically necessary???
% now draw arrowhead -- dont include in the nccurve because this alters the line position - a strange feature of pstruicks
\ncput[npos=0.9]{\pnode(0,0){yyy}}
\ncline{->}{yyy}{#2}
% repeat from earlier to provide context for label that might follow
\nccurve[nodesepA=1pt,nodesepB=0pt,offsetA=0pt,offsetB=0pt,angleA=\theEangleStartCorrected, angleB=\theEangleGiven, ncurv=1.359, linecolor=black, linestyle=dashed]{-}{#2}{midpoint} 
}

%  rEpme - recursive Edge partial mono epi
\newcommand{\rEpme}[2][0]{
\setcounter{EangleGiven}{#1}
\setcounter{EangleStartCorrected}{#1-10} %correction required because for nccurve unlike nccircle angle measured at boundary not at centre of node
\setcounter{EangleEndCorrected}{#1+180+10} %correction required because angle measured at boundary not at centre of node
\setcounter{EangleComplementary}{#1 + 180}
\nccircle[angleA=\theEangleComplementary, nodesep=0pt, linestyle=none]{-}{#2}{.4cm} % an invisible circle to hang the midpoint from
\ncput{\pnode(0,0){midpoint}}                                         
%\nccurve[nodesepA=0pt,nodesepB=0pt,offsetA=0pt,offsetB=0pt,angleA=\theEangleComplementary, angleB=\theEangleEndCorrected, ncurv=1.359, linestyle=dashed]{->}{xxx}{#2}
\nccurve[nodesepA=0pt,nodesepB=2pt,offsetA=0pt,offsetB=0pt,angleA=\theEangleComplementary, angleB=\theEangleEndCorrected, ncurv=1.359]{-}{midpoint}{#2}
% 1.359 is e/2 happenchance or algorithmically necessary???
% now draw arrowhead -- dont include in the nccurve because this alters the line position - a strange feature of pstruicks
\ncput[npos=0.9]{\pnode(0,0){yyy}}
\ncline{->}{yyy}{#2}
% last so that to provide context for label that might follow
\nccurve[nodesepA=1pt,nodesepB=0pt,offsetA=0pt,offsetB=0pt,angleA=\theEangleStartCorrected, angleB=\theEangleGiven, ncurv=1.359, linecolor=black, linestyle=dashed]{-}{#2}{midpoint} 
}

% rEt - recursive Edge total
\newcommand{\rEt}[2][0]{
\setcounter{EangleGiven}{#1}
\setcounter{EangleStartCorrected}{#1-10} %correction required because for nccurve unlike nccircle angle measured at boundary not at centre of node
\setcounter{EangleEndCorrected}{#1+180+10} %correction required because angle measured at boundary not at centre of node
\setcounter{EangleComplementary}{#1 + 180}
\nccircle[angleA=\theEangleComplementary, nodesep=0pt, linestyle=none]{-}{#2}{.4cm} % an invisible circle to hang the midpoint from
\ncput{\pnode(0,0){midpoint}}                                         
\nccurve[nodesepA=1pt,nodesepB=0pt,offsetA=0pt,offsetB=0pt,angleA=\theEangleStartCorrected, angleB=\theEangleGiven, ncurv=1.359, linecolor=black]{-}{#2}{midpoint}
\ncput[nrot=:R,npos=0]{\psline(0,.1)(.075,0)}
\ncput[nrot=:R,npos=0]{\psline(0,.1)(-0.075,0)}
%\nccurve[nodesepA=0pt,nodesepB=0pt,offsetA=0pt,offsetB=0pt,angleA=\theEangleComplementary, angleB=\theEangleEndCorrected, ncurv=1.359, linestyle=dashed]{->}{xxx}{#2}
\nccurve[nodesepA=0pt,nodesepB=2pt,offsetA=0pt,offsetB=0pt,angleA=\theEangleComplementary, angleB=\theEangleEndCorrected, ncurv=1.359, linestyle=dashed]{-}{midpoint}{#2}
% 1.359 is e/2 happenchance or algorithmically necessary???
% now draw arrowhead -- dont include in the nccurve because this alters the line position - a strange feature of pstruicks
\ncput[npos=0.9]{\pnode(0,0){yyy}}
\ncline{->}{yyy}{#2}
% repeat from earlier to provide context for label that might follow
\nccurve[nodesepA=1pt,nodesepB=0pt,offsetA=0pt,offsetB=0pt,angleA=\theEangleStartCorrected, angleB=\theEangleGiven, ncurv=1.359, linecolor=black]{-}{#2}{midpoint} 
}

%  rEtm  - recursive Edge total mono
\newcommand{\rEtm}[2][0]{
\setcounter{EangleGiven}{#1}
\setcounter{EangleStartCorrected}{#1-10} %correction required because for nccurve unlike nccircle angle measured at boundary not at centre of node
\setcounter{EangleEndCorrected}{#1+180+10} %correction required because angle measured at boundary not at centre of node
\setcounter{EangleComplementary}{#1 + 180}
\nccircle[angleA=\theEangleComplementary, nodesep=0pt, linestyle=none]{-}{#2}{.4cm} % an invisible circle to hang the midpoint from
\ncput{\pnode(0,0){midpoint}}     
\nccurve[nodesepA=0pt,nodesepB=2pt,offsetA=0pt,offsetB=0pt,angleA=\theEangleComplementary, angleB=\theEangleEndCorrected, ncurv=1.359, linestyle=dashed]{-}{midpoint}{#2}
% 1.359 is e/2 happenchance or algorithmically necessary???
% now draw arrowhead -- dont include in the nccurve because this alters the line position - a strange feature of pstruicks
\ncput[npos=0.9]{\pnode(0,0){yyy}}
\ncline{->}{yyy}{#2}
% last to provide context for label that might follow
\nccurve[nodesepA=1pt,nodesepB=0pt,offsetA=0pt,offsetB=0pt,angleA=\theEangleStartCorrected, angleB=\theEangleGiven, ncurv=1.359, linecolor=black]{-}{#2}{midpoint} 
}

%  rEte  - total epi
\newcommand{\rEte}[2][0]{
\setcounter{EangleGiven}{#1}
\setcounter{EangleStartCorrected}{#1-10} %correction required because for nccurve unlike nccircle angle measured at boundary not at centre of node
\setcounter{EangleEndCorrected}{#1+180+10} %correction required because angle measured at boundary not at centre of node
\setcounter{EangleComplementary}{#1 + 180}
\nccircle[angleA=\theEangleComplementary, nodesep=0pt, linestyle=none]{-}{#2}{.4cm} % an invisible circle to hang the midpoint from
\ncput{\pnode(0,0){midpoint}}                                         
\nccurve[nodesepA=1pt,nodesepB=0pt,offsetA=0pt,offsetB=0pt,angleA=\theEangleStartCorrected, angleB=\theEangleGiven, ncurv=1.359, linecolor=black]{-}{#2}{midpoint}
\ncput[nrot=:R,npos=0]{\psline(0,.1)(.075,0)}
\ncput[nrot=:R,npos=0]{\psline(0,.1)(-0.075,0)}
%\nccurve[nodesepA=0pt,nodesepB=0pt,offsetA=0pt,offsetB=0pt,angleA=\theEangleComplementary, angleB=\theEangleEndCorrected, ncurv=1.359, linestyle=dashed]{->}{xxx}{#2}
\nccurve[nodesepA=0pt,nodesepB=2pt,offsetA=0pt,offsetB=0pt,angleA=\theEangleComplementary, angleB=\theEangleEndCorrected, ncurv=1.359]{-}{midpoint}{#2}
% 1.359 is e/2 happenchance or algorithmically necessary???
% now draw arrowhead -- dont include in the nccurve because this alters the line position - a strange feature of pstruicks
\ncput[npos=0.9]{\pnode(0,0){yyy}}
\ncline{->}{yyy}{#2}
% repeat from earlier to provide context for label that might follow
\nccurve[nodesepA=1pt,nodesepB=0pt,offsetA=0pt,offsetB=0pt,angleA=\theEangleStartCorrected, angleB=\theEangleGiven, ncurv=1.359, linecolor=black]{-}{#2}{midpoint} 
}

%  rEtme - recursive Edge total mono epi

\newcommand{\rEtme}[2][0]{
\setcounter{EangleGiven}{#1}
\setcounter{EangleStartCorrected}{#1-10} %correction required because for nccurve unlike nccircle angle measured at boundary not at centre of node
\setcounter{EangleEndCorrected}{#1+180+10} %correction required because angle measured at boundary not at centre of node
\setcounter{EangleComplementary}{#1 + 180}
\nccircle[angleA=\theEangleComplementary, nodesep=0pt, linestyle=none]{-}{#2}{.4cm} % an invisible circle to hang the midpoint from
\ncput{\pnode(0,0){midpoint}}     
\nccurve[nodesepA=0pt,nodesepB=2pt,offsetA=0pt,offsetB=0pt,angleA=\theEangleComplementary, angleB=\theEangleEndCorrected, ncurv=1.359]{-}{midpoint}{#2}
% 1.359 is e/2 happenchance or algorithmically necessary???
% now draw arrowhead -- dont include in the nccurve because this alters the line position - a strange feature of pstruicks
\ncput[npos=0.9]{\pnode(0,0){yyy}}
\ncline{->}{yyy}{#2}
% last to provide context for label that might follow
\nccurve[nodesepA=1pt,nodesepB=0pt,offsetA=0pt,offsetB=0pt,angleA=\theEangleStartCorrected, angleB=\theEangleGiven, ncurv=1.359, linecolor=black]{-}{#2}{midpoint} 
}

%gats.macros.tex

\usepackage{environ}    % also used in ermacros % here used for \NewEnvrion

\newcommand{\gat}[1][U]{
\ensuremath{\mathcal{#1}}}  % used to hav a space in here
\newcommand{\gatw}[1][U]{\gat[#1]\ }  % use this if need trailing space
\newcommand{\ingat}[1][U]{in \gat[#1]}
\newcommand{\isagat}[1][U]{\gat[#1] is a g.a.t.}
\newcommand{\inagat}{in a g.a.t. }

% macro for a generic theory
%\newcommand{\theory}
%{\textit{U}}

\newcommand{\intheory}
{is a derived rule of \gat[U]}

% Macros for GAT rules

\newcommand{\isT}[1]
{#1\mbox{ is a type}}

\newcommand{\ofT}[2]
{#1 \in #2
}

% Macros for GAT rules   <!-- new old -->
\newcommand{\istype}[1]
{#1\mbox{ is a type}}

\newcommand{\oftype}[2]
{#1 \in #2
}

%\context{x}{\Delta}{n}
\newcommand{\context}[3]
{\ofT{#1_1}{#2_1},... \ofT{#1_{#3}}{#2_{#3}(#1_1,...#1_{#3-1})}
}

%\subcontext{x}{\Delta}{i}{k}
\newcommand{\subcontext}[4]
{\ofT{#1_{#3_1}}{#2_{#3_1}},... \ofT{#1_{#3_#4}}{#2_{#3_#4}(#1_1,...#1_{#3_#4-1})}
}

% #schematic context
\newcommand{\schmcon}[3]
{\ofT{#1_1}{#2_1},... \ofT{#1_{#3}}{#2_{#3}}
}
% abbreviated to
\newcommand{\con}[3]
{\schmcon{#1}{#2}{#3}}

% schematic subcontext
%\subcon{x}{\Delta}{i}{k}
\newcommand{\subcon}[4]
{\ofT{#1_{#3_1}}{#2_{#3_1}},... \ofT{#1_{#3_#4}}{#2_{#3_#4}}
}

% permuted context
%\permcon{x}{\Delta}{n}{\sigma}
\newcommand{\permcon}[4]
{\ofT{#1_{#4(1)}}{#2_{#4(1)}},... \ofT{#1_{#4(#3)}}{#2_{#4(#3)}}
}
% permuted term
%\permterm{t}{n}{\sigma}
\newcommand{\permterm}[3]
{
#1_{#3(1)},...#1_{#3(#2)}
}


% Idioms
\newcommand{\xDelta}[1]{\con{x}{\Delta}{#1}}
\newcommand{\xDeltap}[1]{\con{x}{\Delta'}{#1}}
\newcommand{\xOmega}[1]{\con{x}{\Omega}{#1}}
\newcommand{\xOmegap}[1]{\con{x}{\Omega'}{#1}}
\newcommand{\yOmega}[1]{\con{y}{\Omega}{#1}}
\newcommand{\yOmegap}[1]{\con{y}{\Omega'}{#1}}

\newcommand{\xDeltasigma}[1]{\permcon{x}{\Delta}{#1}{\sigma}}
\newcommand{\xDeltapsigma}[1]{\permcon{x}{\Delta'}{#1}{\sigma}}
\newcommand{\xOmegasigma}[1]{\permcon{x}{\Omega}{#1}{\sigma}}
\newcommand{\xOmegapsigma}[1]{\permcon{x}{\Omega'}{#1}{\sigma}}
\newcommand{\yOmegasigma}[1]{\permcon{y}{\Omega}{#1}{\sigma}}
\newcommand{\yOmegapsigma}[1]{\permcon{y}{\Omega'}{#1}{\sigma}}

\newcommand{\xDeltainvsigma}[1]{\permcon{x}{\Delta}{#1}{\sigma^{-1}}}
\newcommand{\xDeltapinvsigma}[1]{\permcon{x}{\Delta'}{#1}{\sigma^{-1}}}
\newcommand{\xOmegainvsigma}[1]{\permcon{x}{\Omega}{#1}{\sigma^{-1}}}
\newcommand{\xOmegapinvsigma}[1]{\permcon{x}{\Omega'}{#1}{\sigma^{-1}}}
\newcommand{\yOmegainvsigma}[1]{\permcon{y}{\Omega}{#1}{\sigma^{-1}}}
\newcommand{\yOmegapinvsigma}[1]{\permcon{y}{\Omega'}{#1}{\sigma^{-1}}}

%Idioms enclosed as tuples
\newcommand{\encxDelta}[1]{\tuple{\con{x}{\Delta}{#1}}}
\newcommand{\encxDeltap}[1]{\tuple{\con{x}{\Delta'}{#1}}}
\newcommand{\encxOmega}[1]{\tuple{\con{x}{\Omega}{#1}}}
\newcommand{\encxOmegap}[1]{\tuple{\con{x}{\Omega'}{#1}}}
\newcommand{\encyOmega}[1]{\tuple{\con{y}{\Omega}{#1}}}
\newcommand{\encyOmegap}[1]{\tuple{\con{y}{\Omega'}{#1}}}

\newcommand{\encxDeltasigma}[1]{\tuple{\permcon{x}{\Delta}{#1}{\sigma}}}
\newcommand{\encxDeltapsigma}[1]{\tuple{\permcon{x}{\Delta'}{#1}{\sigma}}}
\newcommand{\encxOmegasigma}[1]{\tuple{\permcon{x}{\Omega}{#1}{\sigma}}}
\newcommand{\encxOmegapsigma}[1]{\tuple{\permcon{x}{\Omega'}{#1}{\sigma}}}
\newcommand{\encyOmegasigma}[1]{\tuple{\permcon{y}{\Omega}{#1}{\sigma}}}
\newcommand{\encyOmegapsigma}[1]{\tuple{\permcon{y}{\Omega'}{#1}{\sigma}}}

\newcommand{\encxDeltainvsigma}[1]{\tuple{\permcon{x}{\Delta}{#1}{\sigma^{-1}}}}
\newcommand{\encxDeltapinvsigma}[1]{\tuple{\permcon{x}{\Delta'}{#1}{\sigma^{-1}}}}
\newcommand{\encxOmegainvsigma}[1]{\tuple{\permcon{x}{\Omega}{#1}{\sigma^{-1}}}}
\newcommand{\encxOmegapinvsigma}[1]{\tuple{\permcon{x}{\Omega'}{#1}{\sigma^{-1}}}}
\newcommand{\encyOmegainvsigma}[1]{\tuple{\permcon{y}{\Omega}{#1}{\sigma^{-1}}}}
\newcommand{\encyOmegapinvsigma}[1]{\tuple{\permcon{y}{\Omega'}{#1}{\sigma^{-1}}}}

\newcommand{\tstyle}{\vdash}
\newcommand{\gatdisplayrule}[3][]
{
\setlength{\fboxsep}{1pt}       
\setlength{\fboxrule}{0pt}
\fbox{$\displaystyle \frac{#2}{#3\rule[-0.3cm]{0cm}{0cm}}$#1}    %added vertival space using \rule
}
\newcommand{\genericAintroductoryrule} {\gatdisplayrule{\xDelta{n}}{\isT{A(\xn)}}}
\newcommand{\genericfintroductoryrule}  {\gatdisplayrule{\xDelta{n}}{\ofT{f(\xn)}{\Delta}}}

% gat macros developed for cwf paper

% Expressing gats
\newenvironment{gatrules}
{
$$
\begin{array}{l l}
}
{
\end{array}
$$
}
\newcommand{\gatintros}
{
\textbf{Symbol} & \textbf{Introductory\ Rule}                      \\}

\newcommand{\gataxioms}
{\textbf{Axioms}\\}
\newcommand{\gatintro}[3]{\ #1 & #2 \tstyle #3 \\}
\newcommand{\gatlocalintro}[3]{\ #1 & #2 \dashv }
\newcommand{\gataxiom}[2]{\multicolumn{2}{l}{\ \ #1\mbox{,  whenever\ } #2} \\}
\newcommand{\noleft}{\left.\kern-\nulldelimiterspace} % so that no space taken by absent left brace


\newcommand{\gatmultiaxiom}[2]
{\multicolumn{2}{l}{
  \noleft
    \begin{array}{l}
		#1
    \end{array} 
  \right\} \mbox{whenever\ } 	#2 
	}\\}
	
	\newcommand{\axid}[1]{\text{#1}.\ }	

%New context sharing macros
\newcommand{\gatintroducing}[1]{
{\arraycolsep=0pt
  \begin{array}{l}
          #1
  \end{array}} &
}

%*********************************
% \begin{\gatgroup}{context}
%    rules
%  \end{\gatgroup}
%*********************************
\NewEnviron{gatgroup}[1]{%
  \noleft
  {\arraycolsep=0pt
   \begin{array}{l}
\BODY
    \end{array} 
   }
   \ \right\} 
	%\mbox{\ whenever\ } 
	#1
	\vspace{0.1cm} 
}
%*********************************

%*********************************
% \begin{\gatgroupnoshared}
%    rule
%  \end{\gatgroupnoshared}
%*********************************
\NewEnviron{gatgroupnoshared}{%
  {\arraycolsep=0pt
   \begin{array}{l}
\BODY
    \end{array} 
   }
   \ 
	\vspace{0.1cm} 
}
%*********************************

% \gatsingular[width]{context}{conclusion}
\newcommand{\gatsingular}[3][4cm]{
\begin{gatgroupnoshared}
\gatleaf[#1]{#2}{#3} 
\end{gatgroupnoshared}
}

%*********************************
% \gatleaf}[width]{context}{assertion}
%*********************************
\newcommand{\gatleaf}[3][4cm]{%
\makebox[#1]{$#3$ \dotfill} \dotfill \  #2
}
%*********************************
%*********************************
% \gatstandalonesingle}{context}{assertion}
%*********************************
\newcommand{\gatstandalonesingle}[2]{%
#2 \makebox[2.5cm]{\dotfill} \  #1
}
%*********************************

% \gataxiomno{axiomno}
\newcommand{\gataxiomno}[1]{\makebox[0.5cm]{} \axid{#1}}


% metagat.macros.tex

%Meta-theories

%\newcommand{\typ}{\triangleright}
\newcommand{\typ}{\nabla}
\newcommand{\trm}{\tau}
\newcommand{\cross}{\otimes}
\newcommand{\sub}{^*}
\newcommand{\diag}{\delta}

\newcommand{\typeseq}[2]
{\ofT{#1_1}{\typ},... \ofT{#1_{#2}}{\typ(#1_{#2-1})}}

\newcommand{\typeseqcont}[3]
{\ofT{#1_1}{\typ({#2})},... \ofT{#1_{#3}}{\typ(#1_{#3-1})}}

\newcommand{\Ob}{Ob}
\newcommand{\obj}{Ob} % <!-- new old --<
\newcommand{\Hom}{Hom}
\newcommand{\objseq}[2]
{\ofT{#1_1}{\obj},... \ofT{#1_{#2}}{\obj(#1_{#2-1})}}


\def\dottededge{\ncline[linestyle=dotted, nodesep=0.3cm]}
\def\noedge{\ncline[linestyle=none]}
\def\thinedge{\ncline[linewidth=0.4pt]}

\newcommand{\member}[1]
{\ncarc[arcangle=-30,nodesepB=0.03]{->}{\pspred}{\pssucc}
\nbput[labelsep=0.1]{#1}}

\newcommand{\loweraccutemember}[1]
{\ncarc[arcangle=-15,nodesepB=0.03]{->}{\pspred}{\pssucc}
\nbput[labelsep=0.05,npos=0.85]{#1}}

\newcommand{\uppermember}[1]
{\ncarc[arcangle=30,nodesepB=0.03]{->}{\pspred}{\pssucc}\naput{#1}}

\newcommand{\upperaccutemember}[1]
{\ncarc[arcangle=10,nodesepB=0.03]{->}{\pspred}{\pssucc}\naput[npos=0.85]{#1}}

% flexbranch 
% #1 node label
% #2 thislevelsep
% #3 next level sep
% #4 variable (eg x)
% #5 index leter (eg n)
% #6 close parenthesis
% #7 continuation branches
\newcommand{\flexbranch}[7]
{
\pstree[thislevelsep=*#2,nodesep=0.05]
		{\Rnode{#1 1}{\Tr{#4_1 #6}}}
	  {\pstree[thislevelsep=#3]  
				   {\Rnode{#1 2}{\Tr[edge=\dottededge]{#4_{#5} #6}}}
					 {#7}
		}
}

\newcommand{\flexbranchplusleaf}[6]
{
\flexbranch{#1}{#2}{#3}{#4} {#5} {#6}
  {
   %\Rnode{#1 3}{\Tr{#4 #6}}
	 \Tr{\Rnode{#1 3}{#4 #6}}
  }
}

\newcommand{\flexbranchplusarc}[7]
{
\flexbranch{#1}{#2}{#3}{#4} {#5} {#6}
  {
   %\Rnode{#1 3}{\Tr{#4 #6}\member{#7}}
	 \Tr{\Rnode{#1 3}{#4 #6}}\member{#7}
  }
}

\newcommand{\flexbranchinitialarc}[9]
{
\pstree[thislevelsep=*#2,nodesep=0.05]
		{\Rnode{#1 1}{\Tr{#4_#8 #6}}#9}
	  {\pstree[thislevelsep=#3]  
				   {\Rnode{#1 2}{\Tr[edge=\dottededge]{#4_{#5} #6}}}
					 {#7}
		}
}

\newcommand{\equality}[2]
{
\ncline [doubleline=true, nodesep=0.2cm]{#1}{#2}
}
\newcommand{\equalityarc}[2]
{
\ncarc [arcangleA=-30, arcangleB=-20, doubleline=true, nodesep=0.1cm]{#1}{#2}
}

%The following are stylistic so belong in main document not here.
%\usepackage[margin=4.0cm]{geometry} % This shouldn't be here commented out 17 July 2018
%\usepackage{mathptmx}               % This changes font to roman so doesn't belong here
%
\usepackage{amsfonts}
\usepackage{amssymb} % added 08\02\2019 as an experiment. Needed in some instances for \blacksquare
                     % not needed is class is `beamer' but I don't know why not
\usepackage{array}
\usepackage{pstricks}
\usepackage{pst-tree}
\usepackage{pst-plot}
\usepackage{pst-node}
\usepackage{stmaryrd}
\usepackage{amsmath}
\usepackage{verbatim}
\usepackage{graphicx}  
\usepackage{calc}
\usepackage{xifthen}
%\usepackage{xcolor} investigate with beamer
\usepackage{color}
\usepackage{stringstrings}
%\usepackage[small,bf,margin=3pt,format=hang, labelsep=endash,singlelinecheck=false]{caption} %prevuiously justification=justified
%\usepackage{enumerate}
%\usepackage{enumitem}
\usepackage{enumerate}
%\usepackage[shortlabels]{enumitem} %Removed this 28/01/2019 because interfereing with a beamer presentation. 
\usepackage{float}
\usepackage[section]{placeins}
%\setlength{\captionmargin}{5pt}
\usepackage{environ}
\usepackage{multirow}
\usepackage{rotating}
\usepackage{longtable}
\usepackage{afterpage}
\usepackage{needspace}


%DEFINE ENVIRONMENT BLOCK
% Riddle
\newsavebox{\riddlebox}

\newenvironment{erexample}
{\newcommand\colboxcolor{F0F0F0}%was F8F8F8
\begin{lrbox}{\riddlebox}
\begin{minipage}{\dimexpr\columnwidth-2\fboxsep\relax} \textbf{} \\ \itshape}
{\end{minipage}\end{lrbox}%
%\begin{center}
\colorbox[HTML]{\colboxcolor}{\usebox{\riddlebox}}
%\end{center}
}

\newenvironment{erbox}
{\newcommand\colboxcolor{F0F0F0}%was F8F8F8
\begin{lrbox}{\riddlebox}%
\begin{minipage}{\dimexpr\columnwidth-2\fboxsep\relax} }
{\end{minipage}\end{lrbox}%
%\begin{center}
\colorbox[HTML]{\colboxcolor}{\usebox{\riddlebox}}
%\end{center}
}

%\begin{erboxedFigure}{#1 FigureParam}{#2 Label}{#3 Caption}
\NewEnviron{erboxedFigure}[3]{%
\begin{figure}[#1]
\begin{erexample}
\begin{center}
\BODY
\end{center}
\vspace{-0.5cm}
\caption{#3}
\label{#2}
\end{erexample}
\end{figure}
}

\newcommand{\erpictureFolder}[0]{../SharedPictures}

\newcommand{\ercenterPicture}[1]{
\begin{center}
\input{\erpictureFolder/#1}
\end{center}
}


\newlength{\erhalfHt}

%\erinlinePicture{#1 pictureFilename}{#2 pictureHeight}
\newcommand{\erinlinePicture}[2]{
\setlength{\erhalfHt}{#2cm * \real{0.5}}
\raisebox{-\erhalfHt}[\erhalfHt + 0.5cm][\erhalfHt + 0.5cm]{
\input{\erpictureFolder/#1}
} 
}

%\erplainFig{#1 pictureFilename}{#2 figureParam}{#3Caption}
\newcommand{\erplainFig}[3]{
\begin{figure}[#2]
\begin{center}
\input{\erpictureFolder/#1}
\end{center}
\caption{#3}
\label{#1}
\end{figure}
}

%\erboxedFigPicture{#1 pictureFilename}{#2 figureParam}{#3Caption}
\newcommand{\erboxedFigPicture}[3]{
\begin{figure}[#2]
\begin{erexample}
\vspace{-0.5cm}
\begin{center}
\input{\erpictureFolder/#1}
\end{center}
\caption{#3}
\label{#1}
\end{erexample}
\end{figure}
}

%\erLeftSideFig{#1 pictureFilename}{#2 figureParam}{#3Caption}
\newcommand{\erLeftSideFig}[3]{
\begin{figure}[#2]
\begin{erexample}
  \begin{minipage}[c]{0.4\textwidth}
    \caption{#3}
    \label{#1}
  \end{minipage}
  \begin{minipage}[c]{0.5\textwidth}
    \input{\erpictureFolder/#1}
  \end{minipage}
\end{erexample}
\end{figure}
}

%\erbulletedFig{#1 pictureFilename}{#2 figureParam}{#3Caption}
\NewEnviron{erbulletedFig}[3]{%
\begin{figure}[#2]
\begin{erexample}
\vspace{-0.5cm}
\begin{center}
$
\begin{array}{c m{0.25cm} | m{6cm}}
\raisebox{-2.0cm}{
\input{\erpictureFolder/#1}}& & \text{\parbox{6cm}{\raggedright{\footnotesize{
\begin{enumerate}[(i)]
\BODY
\end{enumerate}}}}} \\
\end{array}
$
\end{center}
\caption{#3}
\label{#1}
\end{erexample}
\end{figure} 
}


%\begin{erbulletedDimFig}{#1 pictureFilename}{#2figureParam} {#3Caption} {#4PictureHeight}{#5TextWidth}

\NewEnviron{erbulletedDimFig}[5]{%
\begin{figure}[#2]
\begin{erexample}
\vspace{-0.5cm}
\begin{center}
$
\begin{array}{c m{0.25cm} |  m{#5cm}}
\setlength{\erhalfHt}{#4cm * \real{0.5}}
\raisebox{-\erhalfHt}{
\input{\erpictureFolder/#1}}& & \text{\parbox{#5cm}{\raggedright{\footnotesize{
\begin{enumerate}[(i)]
\BODY
\end{enumerate}}}}} \\
\end{array}
$
\end{center}
\caption{#3}
\label{#1}
\end{erexample}
\end{figure} 
}

%\begin{ernotedModel}{#1 pictureFilename}{#2PictureHeight}{#3PictureWidth}{#4TextWidth}

\NewEnviron{ernotedModel}[4]{%
\begin{center}
$
\begin{array}{m{#3cm} m{1cm} | c m{#4cm}}
\setlength{\erhalfHt}{#2cm * \real{0.5}}
\raisebox{-\erhalfHt}{
\input{\erpictureFolder/#1}}& & & \text{\parbox{#4cm}{\raggedright{\footnotesize{
\BODY
}}}} \\
\end{array}
$
\end{center} 
}

%\begin{ermodelText}{#1 pictureFilename}{#2PictureHeight}{#3PictureWidth}{#4TextWidth}

\NewEnviron{ermodelText}[4]{%
\begin{center}
\begin{tabular}{m{#3cm} m{1cm}  c m{#4cm}}
\setlength{\erhalfHt}{#2cm * \real{0.5}}
\raisebox{-\erhalfHt}{
\input{\erpictureFolder/#1}}& & & \text{\parbox{#4cm}{\raggedright{\small{
\BODY
}}}} \\
\end{tabular}
\end{center} 
}


%\erbulletedModel{#1 pictureFilename}{#2PictureHeight}{#3PictureWidth}{#4TextWidth}

\NewEnviron{erbulletedModel}[4]{%
\begin{center}
$
\begin{array}{m{#3cm} m{1cm} | c m{#4cm}}
\setlength{\erhalfHt}{2cm * \real{0.5}}
\raisebox{-\erhalfHt}{
\input{\erpictureFolder/#1}}& & & \text{\parbox{#4cm}{\raggedright{\footnotesize{
\begin{enumerate}[(i)]
\BODY
\end{enumerate}}}}} \\
\end{array}
$
\end{center} 
}



%\ernotedDimFig{#1 pictureFilename}{#2 figureParam}{#3Caption}{#4PictureHeight}{#5TextWidth}
\NewEnviron{ernotedDimFig}[5]{%
\begin{figure}[#2]
\begin{erexample}
\vspace{-0.5cm}
\begin{center}
$
\begin{array}{c m{0.25cm} | c m{#5cm}}
\setlength{\erhalfHt}{#4cm * \real{0.5}}
\raisebox{-\erhalfHt}{
\input{\erpictureFolder/#1}}& & & \text{\parbox{#5cm}{\raggedright{\footnotesize{
\BODY }}}}\\
\end{array}
$
\end{center}
\caption{#3}
\label{#1}
\end{erexample}
\end{figure} 
}
%\begin{ernotedDimFigPW}{#1 pictureFilename}{#2 figureParam}{#3Caption}{#4PictureHeight}{#5PictureWidth}{#6TextWidth}
\NewEnviron{ernotedDimFigPW}[6]{%
\begin{figure}[#2]
\begin{erexample}
\vspace{-0.5cm}
\begin{center}
$
\begin{array}{>{\centering}m{#5cm} m{0.5cm} | c m{#6cm}}
\setlength{\erhalfHt}{#4cm * \real{0.5}}
\raisebox{-\erhalfHt}{
\centering \input{\erpictureFolder/#1}
}& & & \text{\parbox{#6cm - 0.5cm}{\raggedright{\footnotesize{
\BODY }}}}\\
\end{array}
$ \\
\vspace {0.2cm}
\end{center}
\caption{#3}
\label{#1}
\end{erexample}
\end{figure}
}



\newenvironment{erquote}
{\begin{quote}\itshape}
{\end{quote}}


%
%  erdiagram.tex
%  *************
%  Macros to represent ER diagrams
%  *******************************
% 29/01/2019 Modify so that not reliant on the
%            default fontsize being 10pt by using
%            package anyfontsize and then
%            \fontsize{8}{10}\selectfont to set font to 8pt
% 06/02/2019 Pullback symbol implemented and minor tweaks to positioning 
%            and size of identifier symbol and relationship labels.
%            Accidental forked changes merged on 08/02/2019.
% 15/03/2019 Continuation of 29/01/2019. Need fix fontsize of 
%            ERrelname and ERscope.	 
% ***********************************************************
 \usepackage{anyfontsize}             % 29/01/2019 
  
%\begin{erdiagram}{#1 height}{#2 width} 
% ....
% ....
%\end{erdiagram}
\newenvironment{erdiagram}[2]
{%\pspicture*(-#1,0)(#2,0)
\pspicture*(0,-#1)(#2,0)
%\psgrid
}
{\endpspicture}

\definecolor{lightyellow}{cmyk}{0,0,0.3,0}
\definecolor{verylightgrey}{gray}{0.95}


% \eret{#1 x0} {#2 y0} {#3 x1} {#4 y1} {#5 corner radius} {#6 fill}
\newcommand {\eret}[6]
{ 
\ifthenelse{\equal{#6}{1}}
{\psframe[framearc=#5,fillstyle=solid,fillcolor=lightyellow](#1,#2)(#3,#4)}
{\psframe[framearc=#5,fillstyle=solid,fillcolor=white](#1,#2)(#3,#4)}
}

% et top 
\newcommand {\erettop}[4]
{
%\psframe[linestyle=none,linearc=2pt,cornersize=absolute,fillstyle=solid,fillcolor=lightyellow](#1,#2)(#3,#4)
\psline[linearc=2pt,fillstyle=none,fillcolor=lightyellow](#1,#4)(#1,#2)(#3,#2)(#3,#4)
}

% et bottom 
\newcommand {\eretbtm}[4]
{
%\psframe[linestyle=none,linearc=2pt,cornersize=absolute,fillstyle=solid,fillcolor=lightyellow](#1,#2)(#3,#4)
\psline[linearc=2pt,fillstyle=none,fillcolor=lightyellow](#1,#2)(#1,#4)(#3,#4)(#3,#2)
}

% et bottom left
\newcommand {\eretbl}[4]
{
%\psframe[linestyle=none,linearc=2pt,cornersize=absolute,fillstyle=solid,fillcolor=lightyellow](#1,#2)(#3,#4)
\psline[linearc=2pt,fillstyle=none,fillcolor=lightyellow](#1,#4)(#3,#4)(#3,#2)
}

% et middle left
\newcommand {\eretml}[4]
{
%\psframe[linestyle=none,linearc=2pt,cornersize=absolute,fillstyle=solid,fillcolor=lightyellow](#1,#2)(#3,#4)
\psline[linearc=2pt,fillstyle=none,fillcolor=lightyellow](#1,#2)(#3,#2)(#3,#4)(#1,#4)
}

% et top left
\newcommand {\erettl}[4]
{
%\psframe[linestyle=none,linearc=2pt,cornersize=absolute,fillstyle=solid,fillcolor=lightyellow](#1,#2)(#3,#4)
\psline[linearc=2pt,fillstyle=none,fillcolor=lightyellow](#1,#2)(#3,#2)(#3,#4)
}

% et bottom right
\newcommand {\eretbr}[4]
{
%\psframe[linestyle=none,linearc=2pt,cornersize=absolute,fillstyle=solid,fillcolor=lightyellow](#1,#2)(#3,#4)
\psline[linearc=2pt,fillstyle=none,fillcolor=lightyellow](#1,#2)(#1,#4)(#3,#4)
}

% et middle right
\newcommand {\eretmr}[4]
{
%\psframe[linestyle=none,linearc=2pt,cornersize=absolute,fillstyle=solid,fillcolor=lightyellow](#1,#2)(#3,#4)
\psline[linearc=2pt,fillstyle=none,fillcolor=lightyellow](#3,#4)(#1,#4)(#1,#2)(#3,#2)
}

% et top right
\newcommand {\erettr}[4]
{
\psline[linearc=2pt,fillstyle=none,fillcolor=lightyellow](#1,#4)(#1,#2)(#3,#2)
}

% \ergrp{#1 x0} {#2 y0} {#3 x1} {#4 y1} {#5 corner radius} {#6 fill}
% #5 corner radius is unused!
\newcommand {\ergrp}[6]
{ 
\ifthenelse{\equal{#6}{1}}
{\psframe[fillstyle=solid,fillcolor=verylightgrey](#1,#2)(#3,#4)}
{\psframe[fillstyle=solid,fillcolor=white](#1,#2)(#3,#4)}
}


% \ertext{#1 text}
% 15/03/2019
\newcommand {\erextrasmallitalictext}[1]
{\fontsize{7}{9}\selectfont \textit{#1}}

% 29/01/2019  
\newcommand {\ersmallitalictext}[1]
{\fontsize{8}{10}\selectfont \textit{#1}}

\newcommand {\ermediumitalictext}[1]
{\fontsize{10}{12}\selectfont \textit{#1}}

% \eretname {#1 x left of text} {#2 y top of text} {#3 text}
\newcommand {\olderetname}[3]
{
%shift down 0.1 for height of text the anchor at baseline (B)
\rput[bl]{0}(0,-0.1){\rput[Bl]{0}(#1,#2){\ersmallitalictext{#3}}}
}

% \errelarm {#1 x0} {#2 y0} {#3 x1} {#4 y1} {#5 ismandatory} {#6 isconstructed}
\newcommand {\errelarm}[6]
{
\ifthenelse{\equal{#6}{1}}
{
%%\psline[linewidth=0.5pt,linearc=.05,linestyle=dashed,dash=6pt 6pt]{-}(#1,#2)(#3,#4)}
\ifthenelse{\equal{#5}{1}}
{\psline[linewidth=1.5pt,linearc=.05,linecolor=lightgray]{-}(#1,#2)(#3,#4)}
{\psline[linewidth=1.5pt,linearc=.05,linecolor=lightgray,linestyle=dashed,dash=2pt 2pt]{-}(#1,#2)(#3,#4)}
}
{
\ifthenelse{\equal{#5}{1}}
{\psline[linewidth=0.9pt,linearc=.05]{-}(#1,#2)(#3,#4)}
{\psline[linewidth=0.9pt,linearc=.05,linestyle=dashed,dash=2pt 2pt]{-}(#1,#2)(#3,#4)}
}
}

% \errelangle {#1 x0} {#2 y0} {#3 x1} {#4 y1} {#5 x2} {#6 y2} {#7 ismandatory} {#8 isocnstructed}
\newcommand {\errelangle}[8]
{
\ifthenelse{\equal{#8}{1}}
{
%\psline[linewidth=0.5pt,linearc=.1,linestyle=dashed,dash=6pt 6pt]{-}(#1,#2)(#3,#4)(#5,#6)}
\ifthenelse{\equal{#7}{1}}
{\psline[linewidth=1.5pt,linearc=.05,linecolor=lightgray]{-}(#1,#2)(#3,#4)(#5,#6)}
{\psline[linewidth=1.5pt,linearc=.1,linecolor=lightgray,linestyle=dashed,dash=2pt 2pt]{-}(#1,#2)(#3,#4)(#5,#6)}
}
{
\ifthenelse{\equal{#7}{1}}
{\psline[linewidth=0.9pt,linearc=.1]{-}(#1,#2)(#3,#4)(#5,#6)}
{\psline[linewidth=0.9pt,linearc=.1,linestyle=dashed,dash=2pt 2pt]{-}(#1,#2)(#3,#4)(#5,#6)}
}
}

% \ercrowfoot {#1 x0} {#2 y0} {#3 x11} {#4 y11} {#5 x12} {#6 y12} {#7 x13} {#8 y13} {#9 isconstructed}
\newcommand {\ercrowfoot}[9]
{
\ifthenelse{\equal{#9}{1}}
{
\psline[linewidth=1.5pt,linearc=.05,linecolor=lightgray]{-}(#1,#2)(#3,#4)
\psline[linewidth=1.5pt,linearc=.05,linecolor=lightgray]{-}(#1,#2)(#5,#6)
\psline[linewidth=1.5pt,linearc=.05,linecolor=lightgray]{-}(#1,#2)(#7,#8)
}{
\psline[linewidth=0.9pt,linearc=.05]{-}(#1,#2)(#3,#4)
\psline[linewidth=0.9pt,linearc=.05]{-}(#1,#2)(#5,#6)
\psline[linewidth=0.9pt,linearc=.05]{-}(#1,#2)(#7,#8)
}
}


% \eridcomprel{#1 x1}{#2 x2}{#3 y}
\newcommand {\eridcomprel}[3]
{
\psline[linewidth=0.9pt](#1,#3)(#2,#3)
}

% \eridrefrel{#1 x}{#2 y1}{#3 y2}
\newcommand {\eridrefrel}[3]
{
\psline[linewidth=0.9pt](#1,#2)(#1,#3)
}

% \ertext {#1 x} {#2 y} {#3 text anchor} {#4 text}  PRIVATE
\newcommand {\ertext}[4]
{
\rput[B#3]{0}(#1,#2){\fontsize{8}{10}\selectfont #4}
}

% \eretname {#1 x} {#2 y} {#3 text anchor} {#4 text} 
\newcommand {\eretname}[4]
{
\ertext{#1}{#2}{#3}{#4}
}

% \errelname {#1 x} {#2 y} {#3 text anchor} {#4 text} 
\newcommand {\errelname}[4]
{
\rput[B#3]{0}(#1,#2){\erextrasmallitalictext{#4}}
}


% \erscope {#1 x} {#2 y} {#3 text anchor} {#4 text}  15 March 2019
\newcommand {\erscope}[4]
{
\rput[B#3]{0}(#1,#2){\erextrasmallitalictext{#4}}
}

% \erreletname {#1 x} {#2 y} {#3 text anchor} {#4 text}  15 March 2019
\newcommand {\erreletname}[4]
{
\rput[B#3]{0}(#1,#2){\fontsize{10}{12}\selectfont #4}
}

% \ergroupannotation {#1 x} {#2 y} {#3 text anchor} {#4 text}
\newcommand {\ergroupannotation}[4]
{
\ertext{#1}{#2}{#3}{#4}
}


% \errelseq {#1 x} {#2 y}
\newcommand {\erelseq}[2]
{
}
\newcommand {\erattrmarkermand}
{\fontsize{6}{8}\selectfont $\blacksquare$}
\newcommand {\erattrmarkeropt}
{\fontsize{6}{8}\selectfont \CIRCLE}
\newcommand {\erderattrmarkermand}
{\fontsize{6}{8}\selectfont $\square$}
\newcommand {\erderattrmarkeropt}
{\fontsize{8}{10}\selectfont $\circ$}

% \erattr {#1 x} {#2 y} {#3 ismandatory}{#4 idenitfying} {#5 text}
\newcommand {\erattr}[5]
{
\ifthenelse{\equal{#3}{1}}
{\rput[l]{0}(#1,#2){\erattrmarkermand \ersmallitalictext{\ifthenelse{\equal{#4}{0}}{\underline{#5}}{#5}}}}
{\rput[l]{0}(#1,#2){\erattrmarkeropt \ersmallitalictext{\ifthenelse{\equal{#4}{0}}{\underline{#5}}{#5}}}}
}

\newcommand {\erdattr}[5]
{
\ifthenelse{\equal{#3}{1}}
{\rput[l]{0}(#1,#2){\erderattrmarkermand \ersmallitalictext{\ifthenelse{\equal{#4}{0}}{\underline{#5}}{#5}}}}
{\rput[l]{0}(#1,#2){\erderattrmarkeropt \ersmallitalictext{\ifthenelse{\equal{#4}{0}}{\underline{#5}}{#5}}}}
}


% \erarc {#1 x0} {#2 y0} {#3 x1} {#4 y1} {#5 x2} {#6 y2} {#7 x3} {#8 y3}
\newcommand {\erarc}[8]
{
\psbezier[showpoints=false]{-}(#1,#2) (#3, #4)(#5,#6) (#7, #8)
}

% \erarc {#1 x0} {#2 y0} {#3 x1} {#4 y1} {#5 x2} {#6 y2} {#7 x3} {#8 y3}
\newcommand {\errelseq}[8]
{
\psbezier[showpoints=false]{-}(#1,#2) (#3, #4)(#5,#6) (#7, #8)
}
% \ertrace {#1 trace}   
\newcommand {\ertrace}[1]
{
}
    %beamer aware version
%All these macros are copied from SharedMacros/general.tex which doesnt seem to work with beamer
% Some macros in SharedMacros/general.tex thought to have name clashes with beamer.


\newcommand{\fundep}[3]{#2 \xrightarrow{#1} #3}                                                 
\newcommand{\term}[1]{\textit{#1}} 
\newcommand{\setsuchthat}[2]{\left\{#1 \ \middle|\ #2\right\}}
\newcommand{\set}[1]{\left\{#1\right\}}



\newcommand{\wanton}[1]{#1_1,...#1_n}
\newcommand{\ntuple}[1]{\tuple{\wanton{#1}}}

\newcommand{\xntuple}{\ensuremath{\ntuple{x}}}

% maybe not in general.macros 
\newcommand{\xnset}{\ensuremath{\set{\wanton{x}}}}


%
% othermacros
%

%copied from database literature review
\newcommand{\displaybibentry}[1]
{\begin{framed}
\bibentry{#1}
\end{framed}
}

% used in data tables
\newcommand{\colhead}[1]{\textbf{\textcolor{white}{#1}}}
\definecolor{myblue}{RGB}{71,71,186}
% \vpad gives vertical padding in a tabular
\newcommand{\vpad}[1]{\multicolumn{#1}{c}{}\\[-0.25cm]}

\newcommand{\catMEterm}{category with designated monomorphisms and epimorphisms\ }
\newcommand{\IfSforCwithRCwords}{
If $S$ is a sketch for category \catcw considered as a data specification with requirement $\reqtc$\ }
\newcommand{\IfSforCwithRCwordsvariant}{
If $S$ is a sketch for structured category \catcw and if $S$ is considered as a data specification with requirement $\reqtc$\ }
\newcommand{\IfSforepimonoCwithRCwords}{
If $S$ is a sketch for a category \catcw with designated monomorphisms and epimorphisms considered as a data specification with requirement $\reqtc$\ }
\iffalse
\newcommand{\scmonosketchwording}{
If $S$ is a sketch for such a category
%of a category with finite products and designated monomorphisms and epimorphisms
considered as a data specification
with requirement $\reqtc$\ }
\fi
\newcommand{\IfSforproductepimonoCwithRCwords}{
If $S$ is a sketch for such a category
%category \catcw with finite products and designated monomorphisms and epimorphisms 
considered as a data specification with requirement $\reqtc$\ }

\newcommand{\goodnesscriteria}[1]{\textbf{Goodness Criteria #1:}}


% From the Mathematical Theory of data paper
\newcommand{\ssfd}[2]{\ensuremath{#1 \morph #2}}  % singleton-singleton
\newcommand{\smfd}[2]{\ensuremath{\ssfd{#1}{\set{#2}}}}  % singleton-many
\newcommand{\msfd}[2]{\ensuremath{\ssfd{\set{#1}}{#2}}}  % many-singleton
\newcommand{\mmfd}[2]{\ensuremath{\msfd{#1}{\set{#2}}}}  % many-many



% All these should find a home in SharedMacros eventually 

% Commands for making a bit of vertical space. used when arrows and particularly labels of arrows
% use spec that is otherwise accounted for.
\newcommand{\seeroomup}[1]{\rule{0.1cm}{#1}}
\newcommand{\seeroomdown}[1]{\rule[-#1]{0.1cm}{0.1cm}}
\newcommand{\roomup}[1]{\rule{0cm}{#1}}
\newcommand{\roomdown}[1]{\rule[-#1]{0cm}{0.1cm}}



% DIAGRAMS START HERE
\newcommand{\nakedbinarysourcediagram}[5]{
\begin{array}{c p{0.5cm} c}
             &&   \Rnode{b}{#2}\\[0.01cm]
\Rnode{a}{#1} &&               \\[0.01cm] 
             &&   \Rnode{c}{#3}
\end{array} 
\begin{arrows}
\ncarr{a}{b}
\alabel{#4}
\ncarr{a}{c}
\blabel{#5}
\end{arrows}
}

\newcommand{\binarysourcediagram}[5]{$\nakedbinarysourcediagram{#1}{#2}{#3}{#4}{#5}$}
\newcommand{\fgsourcediagram}{\binarysourcediagram{a}{b}{c}{f}{g}}


\newcommand{\unaryfdrepresentationdiagram}[8]{
$
\begin{array}{c p{0.2cm} c}
\nakedbinarysourcediagram{#1}{#2}{#3}{#4}{#5}&& \Rnode{d}{#6}
\end{array}
\begin{arrows}
\ncarr{d}{b}
\idcomp
\blabel{#7}
\ncarr{d}{c}
\alabel{#8}
\end{arrows}
$
}

\newcommand{\unaryfdrepresentationmappeddiagram}[8]{
$
\begin{array}{c p{0.2cm} c}
\nakedbinarysourcediagram{D(#1)}{D(#2)}{D(#3)}{D(#4)}{D(#5)}&& \Rnode{d}{D(#6)}
\end{array}
\begin{arrows}
\ncarr{b}{d}
\alabel{D(#7)^-1}
\ncarr{d}{c}
\alabel{D(#8)}
\end{arrows}
$
}

\newcommand{\commutativetrianglediagram}[6]{
$
\begin{array}{c p{0.4cm} c p{0.4cm} c}
              && \Rnode{b}{#2}  &&                 \\[0.6cm]
\Rnode{a}{#1} &&                && \Rnode{c}{#3}  
\end{array}
\begin{arrows}
\ncarr{a}{b}
\alabel{#4}
\ncarr{b}{c}
\alabel{#5}
\ncarr{a}{c}
\blabel{#6}
\end{arrows}
$
}

\newcommand{\commutativetrianglediagrammutant}[6]{
$
\begin{array}{c  c  c}
              & \Rnode{b}{#2}  &                 \\[0.85cm]
\Rnode{a}{#1} &                & \Rnode{c}{#3}  
\end{array}
\begin{arrows}
\ncarr{a}{b}
\alabel{#4}[0.15]
\ncarr{b}{c}
\alabel{#5}[0.6]
\ncarr{a}{c}
\blabel{#6}
\end{arrows}
$
}

\newcommand{\epimonosplitdiagram}[3]{
\commutativetrianglediagram{#1}{img(#3)}{#2}{#3_e}{#3_m}{#3}   
}


\iffalse %saved
\begin{array}{c p{2.0cm} c }                
               &&  \Rnode{b1}{#3_1}    \\ [0.75cm]
               &&  \Rnode{b2}{#3_2}    \\ [0.5cm]
\Rnode{a}{#2}  &&                      \\ [-0.5cm]
               &&       \vdots         \\ [0.85cm]
               &&  \Rnode{bn}{#3_{#1}}  
\end{array}
\fi

%nakedmultisourceobjects{n}{a}{b}{f}
\newcommand{\nakedmultisourceobjects}[4]{
\begin{array}{c p{2.0cm} c }
\Rnode{a}{#2}   &&
\begin{array}{c }                
\Rnode{b1}{#3_1}   \\ [0.75cm]
\Rnode{b2}{#3_2}   \\ [0.25cm]
\vdots             \\ [0.35cm]
\Rnode{bn}{#3_{#1}}  
\end{array}
\end{array}
}

% \nakedmultisourcediagram{n}{a}{b}{f}
\newcommand{\nakedmultisourcediagram}[4]{
\nakedmultisourceobjects{#1}{#2}{#3}{#4}
\begin{arrows}
\ncarr{a}{b1}
\alabel{#4_1}[0.5]
\ncarr{a}{b2}
\alabel{#4_2}[0.5][-1]
\ncarr{a}{bn}
\blabel{#4_{#1}}[0.5][-1]
\end{arrows}
}

% \nakedmultisourcepathdiagram{n}{a}{b}{f}
\newcommand{\nakedmultisourcepathdiagram}[4]{
\nakedmultisourceobjects{#1}{#2}{#3}{#4}
\begin{arrows}
\simplepath{a}{b1}
\alabel{#4_1}[0.5]
\simplepath{a}{b2}
\alabel{#4_2}[0.5][-1]
\simplepath{a}{bn}
\blabel{#4_{#1}}[0.5][-1]
\end{arrows}
}


\newcommand{\multisourcediagram}[4]{$\nakedmultisourcediagram{#1}{#2}{#3}{#4}$}
\newcommand{\multisourcepathdiagram}[4]{$\nakedmultisourcepathdiagram{#1}{#2}{#3}{#4}$}



% \multisourcedefinitiondiagram{x}{g}{h}{n}{a}{b}{f}
\newcommand{\monosourcedefinitiondiagram}[7]{
$
\begin{array}{c p{1.5cm} c}
\Rnode{x}{#1} && \nakedmultisourcediagram{#4}{#5}{#6}{#7}
\end{array}
\begin{arrows}
\parallelarrows{x}{a}{#2}{#3}
\end{arrows}
$
}

\newcommand{\fghfactordiagram}[6]
{
\binarysourcediagram{#1}{#2\roomup{0.5cm}}{#3}{#4}{#5}
\begin{arrows}
\ncarr{b}{c}
\alabel{#6}
\end{arrows}
}

\newcommand{\fghpartialfactordiagram}[6]{
\binarysourcediagram{#1}{#2\roomup{0.5cm}}{#3}{#4}{#5}
\begin{arrows}
\ncdarr{b}{c} %dashed arrow
\alabel{#6}
\end{arrows}
}

\newcommand{\fnsourceqnsource}{
$
\begin{array}{c p{0.25cm} c  p{0.25cm} c }
             &&   \Rnode{b1}{b_1} &&              \\[0.4cm]
\Rnode{a}{a} &&                   && \Rnode{c}{c} \\[0.4cm]
             &&   \Rnode{bn}{b_n} &&              
\end{array} 
\begin{arrows}
\ncarr{a}{b1}
\alabel{f_1}
\ncarr{c}{b1}
\blabel{q_1} 
\ncarr{a}{bn}
\blabel{f_n}
\ncarr{c}{bn}
\alabel{q_n}
\end{arrows}
$   
}

\newcommand{\parallelarrows}[4]{
\ncarc[nodesep=2pt,arcangle=10,offset=2pt]{->}{#1}{#2}
\alabel{#3}
\ncarc[nodesep=2pt,arcangle=-10,offset=-2pt]{->}{#1}{#2}
\blabel{#4}
}

\newcommand{\fgparalleldiagram}{
 $
\rule[-0.3cm]{0pt}{0.9cm} %to add vertical space of diagram -- based on lowering diagram 0.3cm and heght 0.9cm
                            % change thickness from 0pt to 1 pt to debug
\begin{array}{c p{0.5cm} c  }
 \Rnode{a}{a}            &&   \Rnode{b}{b}
\end{array} 
\begin{arrows}
\iffalse
\ncarc[nodesep=2pt,arcangle=10,offset=2pt]{->}{a}{b}
\alabel{f}
\ncarc[nodesep=2pt,arcangle=-10,offset=-2pt]{->}{a}{b}
\blabel{g}
\fi
\parallelarrows{a}{b}{f}{g}
\end{arrows}
$  
}

\newcommand{\fgcomposablediagram}[5]{
\mbox{
\roomup{0.45cm}
$
\begin{array}{c p{0.5cm}cp{0.5cm}c}
\Rnode{x}{#1}&&\Rnode{y}{#2}&&\Rnode{z}{#3}
\end{array}
\begin{arrows}
\ncarr{x}{y}
\alabel{#4}
\ncarr{y}{z}
\alabel{#5}
\end{arrows}
$    
}
}



% copied from MToD paper (preamble.tex)
\newcommand{\simplepath}[2]{
\ncline[linestyle=none,linewidth=0.1pt]{#1}{#2}   %was linestyle=dotted
\ncput[npos=0.05]{\pnode{dot#21}}
\ncput[npos=0.27]{\dotnode[dotsize=1pt]{dot#22}}
\ncput[npos=0.50]{\dotnode[dotsize=1pt]{dot#23}}
\ncput[npos=0.80]{\dotnode[dotsize=1pt]{dot#24}}
\ncput[npos=0.975]{\pnode{dot#25}}
\ncline[nodesep=2pt]{->}{dot#21}{dot#22}
\ncline[nodesep=2pt]{->}{dot#22}{dot#23}
\ncline[nodesep=2pt]{->}{dot#24}{dot#25}
\ncline[linestyle=dotted,nodesep=8pt]{dot#23}{dot#24} %was 10pt
}

\renewcommand{\erpictureFolder}[0]{../../SharedPictures}
\setcounter{equation}{0}



\title[John Cartmell]{Mathematical Theory of Data}
%% Which is to say types as they are used in practice in software development and as represented in theory in categories and in syntactic type theories.
%% There is also a subplot concerning representation of context which certain types depend on -- again represented in practice and in theory. 
\author{John Cartmell}
%\institute{ad otium}
\date{Jan 25, 2019}
\bibliographystyle{plainnat}
\usepackage{bibentry}
\nobibliography*

\begin{document}

\begin{frame}
\titlepage
\end{frame}

\iffalse
\begin{frame}{Test}
\begin{description}	[longest label] 
\item<1->[short] Some text. 
\item<2->[longest label] Some text. 
\item<3->[long label] Some text. 
\end{description}
\end{frame}
\fi

\section{Mathematical Theory of Data}

\subsection{Introduction}

\begin{frame}{Relational Normal Form Criteria}
Classic relational database normal form criteria 
\begin{itemize}
    \item from a mathematical perspective are not really normal forms!
    \item they are goodness criteria (GC) that articulate good engineering principles
    \item they include:
        \begin{center}
        \begin{tabular}{p{6.0cm}  l }
         \innerbullet first normal form (1NF)            &\Rnode{A1}{}                       \\
         \innerbullet 2nd normal form (2NF)              &                                   \\
         \innerbullet 3rd normal form (3NF)              &
                      \Rnode{A2}{}\braceLabel{A1}{A2}{Codd 1970,1971}                        \\
         \innerbullet Boyce-Codd normal form (BCNF)      &                                   \\
         \innerbullet 4th normal form (4NF)              & -- Fagin 1977                     \\
         \innerbullet projection-join normal form (5NF)  & -- Fagin 1979                     \\
         \innerbullet inclusion normal form (IN-NF)      & -- Ling and Goh  1992
        \end{tabular}
        \end{center}
\end{itemize}
\end{frame}

\begin{frame}{Introduction}
I wish to show that
\begin{itemize}
\item we can genericise relational database normal form criteria into abstract logical terms,
\item achieve goodness criteria that are generic i.e. can be applied to any data specifications not just to relational schema,
 \item that the classic relational database normal form criteria (2NF, 3NF, BCNF, INC-NF, 4NF, 5NF)  are  consequences of these generic goodness criteria.
\end{itemize}
\end{frame}

\begin{frame}{View}
A data specification  
\begin{itemize}
\item is a  theory (of what is)
\end{itemize}
\medskip
A data specification method 
\begin{itemize}
\item is a method for expressing such a  theory
\item unequivocally it enables definitions of types and certain relationships between these types
\item types are equally types of data and types of real world entity
\end{itemize}
More precisely, 
\begin{itemize}
\item data specification are \underline{presentations} of theories of what is,
\item choice of primitives in a given presentation is choice of which data to be stored or communicated.
\end{itemize}
\end{frame}

\begin{frame}{Goodness Criteria}
I will define two types of goodness criteria
\begin{itemize}
    \item Goodness Criteria of Type 1 -- absence of redundancy in presentation.
    \item Ensures absence of redundancy in data and in data management logic.
    \item Spelt out in more detail in criteria 1A, 1B and so on
    \item  Goodness Criteria of Type 2 -- the theory be the tightest fit to the facts 
    \item Two ways of expressing this. 
    \begin{itemize}
        \item Criteria 2 is that the theory is maximally constrained.
        \item Criteria 2A, 2B etc.  that it be logically complete in some sense.
    \end{itemize}
\end{itemize}
\end{frame}

\begin{frame}{Data Specification as Category}
\begin{itemize}
 \item Not surprising that a data specification can represented as a category or as a sketch of a category
\pause \item
What types of things are there and how are they related? 
\begin{itemize}
\item Data specifications provide the answer to this question in the context of a software development. 
\item Types theories provide the answer in a mathematical context. 
\item Category theory abstracts across both these domains.
\end{itemize}
\pause \item Nor is it surpising if data specifications can be seen in terms
of contextual categories or a sketches for a contextual categories
because as categories model types so contextual categories model types that vary
\pause ... and  our instinct for types and types that vary comes from our linguistic abilities as much as from our mathematics.
\end{itemize}
\end{frame}

\begin{frame}{Coming up}
\begin{center}
\begin{tabular}{p{12cm}}
\begin{itemize}
    \item data specifications 
    \begin{itemize}
        \item as sketches of structured categories of some kind
        \item data instances as certain structure preserving functors to the category of finite sets $\Fin$
    \end{itemize}
    \item database specifications
    \begin{itemize}
         \item category has designated mono sources for some of its objects 
    \end{itemize}
    \item relational database specifications charcterised by
    \begin{itemize}
         \item no use of containment or nesting
         \item no  hierarchical organisation -- said to be flat
         \item no use of pointers 
         \item instead uses foreign keys to represent relationships in the data
    \end{itemize}
\end{itemize} 
\end{tabular}
\end{center}
\end{frame}

\newcommand{\nestedcell}[3]{
\cline {2-2}
\rowcolor{#1} &\multicolumn{1}{|c|}{ \cellcolor{#2}#3}&\\
\cline {2-2}
\rowcolor{#1} & & \\
}

\newcommand{\databasesubtypes}{
\begin{tabular}{| p{0.05cm} c p{0.05cm}|}
\hline
\rowcolor{white}\multicolumn{3}{|l|}{\roomdown{0.25cm}database specific}\\
\nestedcell{white}{lightyellow}{relational(rdb)--SQL}
\nestedcell{white}{lightyellow}{nrdb}
\nestedcell{white}{lightyellow}{hierarchical--IMS}
\nestedcell{white}{lightyellow}{network--CODASYL}
\hline
\end{tabular}	
}

\newcommand{\innermessagingsubtypes}{
\begin{tabular}{p{0.05cm} c  p{0.05cm}}
\hline
\multicolumn{3}{l}{\roomdown{0.25cm}pointer free}\\
\nestedcell{lightyellow}{white}{XML}
\nestedcell{lightyellow}{white}{IDL-PB}
\hline
\end{tabular}	
}

\newcommand{\messagingsubtypes}{
\begin{tabular}{| p{0.05cm} c p{0.05cm}|}
\hline
\rowcolor{white}\multicolumn{3}{|l|}{\roomdown{0.25cm}message specific}\\
\nestedcell{white}{lightyellow}{\innermessagingsubtypes}
\nestedcell{white}{lightyellow}{IDL-CMU}
\hline
\end{tabular}	
}

\begin{frame}{Data specification methods}
\begin{tabular} {|p{0.05cm} c p{0.05cm} c p{0.05cm}|}
\hline
\rowcolor{lightyellow}\multicolumn{5}{|l|}{\roomdown{0.25cm}data specification methods} \\
\rowcolor{lightyellow} &\databasesubtypes && \messagingsubtypes &     \\
\rowcolor{lightyellow} &                  &&                    &     \\
\hline
\end{tabular}
\end{frame}

\newcommand{\bigdownarrow}
{\scalebox{0.3}{
\begin{pspicture}(3,3.5) 
%\psgrid
%\psset{doublesep=2cm} 
\psBigArrow[fillstyle=solid, fillcolor=blue!30,linecolor=blue](2.0,3)(2.0,0)
\end{pspicture}
}}
\newcommand{\biguparrow}
{\scalebox{0.3}{
\begin{pspicture}(3,3.5) 
%\psgrid
%\psset{doublesep=2cm} 
\psBigArrow[fillstyle=solid, fillcolor=red!30,linecolor=red](1.1,0)(1.1,3)
\end{pspicture}
}}
\begin{frame}{Logical and Technological Data Specifications}
\begin{center}
\begin{tabular}{c l}
\cline{1-1}
\multicolumn{1}{|c|}{logical} &  \raisebox{0cm}{\parbox{5cm}{sketch of category of some kind}} \\
\cline{1-1}
\multicolumn{1}{c}{\bigdownarrow} &  \\
\cline{1-1}
\multicolumn{1}{|c|}{technological} & \raisebox{0cm}{\parbox{5cm}{IDL, XML, SQL, etc.}}\\
\cline{1-1}
\end{tabular}
\end{center}
\begin{itemize}
    \item there is a logical data specification i.e. a sketch of a category behind every 
    technological data specification
    \item the generic goodness criteria are defined at the logical level
    \item I will show that a relational data specification passes the BCNF test
    iff its logical specification satisfies certain generic goodness criteria
    \item logical specifications are technology neutral. A single logical data specification can be carried forward into multiple technological specifications i.e. one for each technology or language of interest.
\end{itemize}
\end{frame}

\newcommand{\fourlevelstabular}
{
\begin{tabular}{c}
\cline{1-1}
\multicolumn{1}{|c|}{logical}     \\
\cline{1-1}
\multicolumn{1}{c}{\bigdownarrow} \\
\cline{1-1}
\multicolumn{1}{|c|}{structural}  \\
\cline{1-1} \\[-0.3cm]
\multicolumn{1}{c}{\bigdownarrow} \\
\cline{1-1}
\multicolumn{1}{|c|}{representational} \\
\cline{1-1} \\[-0.3cm]
\multicolumn{1}{c}{\bigdownarrow}   \\
\cline{1-1}
\multicolumn{1}{|c|}{technological} \\
\cline{1-1}
\end{tabular}   
}

\newcommand{\fourlevelstexttabular}
{
\begin{tabular}{p{6cm}}
\onslide<1->\\[-0.2cm]
\onslide<1->sketch of category of some kind \\[0.5cm]
\onslide<2->sketch as above with morphisms distinguished to indicate implementation by containment relationship  \\[0.1cm]
\onslide<3-> sketch with distinguised morphisms plus  foreign keys for some non-distinguished morphisms \\[0.3cm]
\onslide<1->IDL, XML, SQL etc\\
all relationships implemented by pointers, foreign keys or containment
\end{tabular}   
}

\begin{frame}{Different Levels of Data Specification}{different amounts of detail}
\begin{center}
\begin{tabular}{c c}
\fourlevelstabular&\pause\fourlevelstexttabular 
\end{tabular}
\end{center}
\end{frame}

\iffalse
\begin{frame}{Different Kinds of Data Specification}
\begin{center}
\begin{tabular}{c l}
\cline{1-1}
\multicolumn{1}{|c|}{logical} &  \raisebox{0cm}{\parbox{5cm}{sketch of category of some kind}} \\
\cline{1-1}
\\[-0.1cm]
\multicolumn{1}{c}{\bigdownarrow} & \raisebox{0.5cm}{\parbox{6cm}{distinguish morphisms/relationships represented in data by containment}} \\
\cline{1-1}
\multicolumn{1}{|c|}{structural} &\\
\cline{1-1}
\multicolumn{1}{c}{\bigdownarrow} & \raisebox{0.5cm}{\parbox{6cm}{add edges for foreign keys representing non-containment relationships and add path equivalences that define them}} \\
\cline{1-1}
\multicolumn{1}{|c|}{representational} & \\
\cline{1-1}
\\[-0.1cm]
\multicolumn{1}{c}{\bigdownarrow} & \raisebox{0.5cm}{choice of technology} \\
\cline{1-1}
\multicolumn{1}{|c|}{technological} & \raisebox{0cm}{\parbox{5cm}{IDL, XML, SQL}}\\
\cline{1-1}
\end{tabular}
\end{center}
\end{frame}
\fi

\iffalse
\begin{frame}{Methods of Data Specification}
\begin{itemize}
	\item schema of relational database,
	\item structure described by Carnegie-Mellon IDL,
	\item schema of nested relational database,
	\item message structure described by Google protocol buffer IDL,
	\item XML schema language,
	\item ER script.
\end{itemize}
\end{frame}
\fi

\begin{frame}{Data Specifications}
Two kinds of types in play
\begin{itemize}
\item  the \textit{definienda} -- types all of whose instances are \term{particulars}
\begin{itemize}
\item employee, department, student, account, product, order, shipment, delivery, flight, booking and so on
\item molecular structure, atom, bond, element, isotope, reaction, metabolite, mass trace, chromatogram, peak
\item table, column, primary key, foreign key
\item node and edge. 
\end{itemize}
\pause 
\item  the \textit{definiens}  -- types all of whose instances are \term{universals}
\begin{itemize}
       \item string, integer, float, boolean and so on
\end{itemize}
\end{itemize}
\pause
\begin{itemize}
\item in ER modelling 
\begin{itemize}
\item the \textit{definienda} are called \textit{entity types}
\item the \textit{definiens} are called \textit{attribute types} or \textit{domains}.
\end{itemize}
\end{itemize}
\end{frame}


\begin{frame}{Data Specifications}
A data specification is a sketch of a category with some additional structure:
\begin{itemize}
\item that it is a \term{sketch} is crucial because it is only nodes and edges of the sketch for which data is stored and/or communicated, 
\item that there are commutative diagrams is crucial to construction of representational 
specifications from logical specifications.
\item that the category had additional structure is significant:
\begin{itemize}
\item so that we can distinguish structural from non-structural relationships to describe structure nesting and thereby hierarchical data,
\item so that we can give account of database normal forms 
(BCNF, 3NF, 4NF and 5NF),
\item so that we can allow for missing data as represented by NULL values, 
\item so that types of universals can be distinguished from types of particulars.
\end{itemize}
\end{itemize}
\end{frame}

\begin{frame}{Additional Structure}
\resizebox{11.3cm}{!}{
\newcommand{\featurelist}{\begin{tabular}{|l|l l|}
\hline 
\multirow{11}{1.5cm}{category}
                & finitary property        & \\
\cline{2-3}
                & pu-partition             & \\
\cline{2-3}
                & \multirow{2}{3.5cm}{mono-sources}  & \multicolumn{1}{|l|}{cannonical monos}  \\
\cline{3-3}
                &                                    & \multicolumn{1}{|l|}{epi-mono factorisation}   \\
\cline{2-3}
                & finite products          & \\
\cline{2-3} 
                & finite limits            & \\
\cline{2-3}
                & restrictions             & \\
\cline{2-3}
                & \multirow{2}{3.5cm}{distinguished morphisms} & \multicolumn{1}{|l|}{hierarchical}      \\
\cline{3-3}
                &                                             &  \multicolumn{1}{|l|}{non-hierarchical} \\
\cline{3-3}
                &                                             &  \multicolumn{1}{|l|}{pullbacks} \\
\cline{2-3}
                & finite coproducts                           &                                   \\
\hline                
\end{tabular}}
\featurelist
}
\end{frame}

\begin{frame}{Definitions} 
In a category \catc, a  \term{source} is a family of morphisms with common domain: \\
\scalebox{0.65}{
\multisourcediagram{n}{a}{b}{f}
} 
\medskip
Such a source is said to be a \term{mono source}  iff for all $g,h:x \morph a$ in \catcw 
so that \scalebox{0.65}{
\monosourcedefinitiondiagram{x}{g}{h}{n}{a}{b}{f}
} 
in \catcw then if $g \circ f_i = h \circ f_i$, for each $i$,  then $g=h$.
\end{frame}

\begin{frame}{Alternative Definition}{Mono Source Limit Cone}  
In a category \cat{C},
\scalebox{0.65}{


$
\begin{array}{c p{2.0cm} c p{2.0cm} c}				
                   &&	 \Rnode{B1}{B_1}  \\ [0.75cm]
									 &&  \Rnode{B2}{B_2}  \\ [0.5cm]
		\Rnode{A}{A}  &&                    \\ [-0.5cm]
				           &&       \vdots      \\ [0.85cm]
                   &&	 \Rnode{Bn}{B_n}  
\end{array}
$
%\setlength{\arrnodesepA}{7pt}
%\setlength{\arrnodesepB}{8pt}
%\setlength{\arroffsetA}{2pt}
%\setlength{\arroffsetB}{0pt}
\begin{arrows}
\ncarr{A}{B1}
\alabel{f_1}[0.5]
\ncarr{A}{B2}
\alabel{f_2}[0.5][-1]
%\blabel{\vdots}[0.4][-2]  % move up 5pts -- dont know why I need this to get position for vdots
\ncarr{A}{Bn}
\blabel{f_n}[0.5][-1]
\end{arrows}


} is a mono source iff \\
\begin{center}
\scalebox{0.65}{
$
\begin{array}{c p{2.0cm} c p{2.0cm} c}				
                           &&	\Rnode{At}{A}  &&          \Rnode{B1}{B_1}  \\ [0.65cm]
													 &&                &&          \Rnode{B2}{B_2}  \\ [0.5cm]
		\Rnode{Al}{A}          &&                &&                           \\ [0cm]
				                   &&                &&           \vdots      \\ [0.85cm]
                           &&	\Rnode{Ab}{A}  &&          \Rnode{Bn}{B_n}  
\end{array}
$
%\setlength{\arrnodesepA}{7pt}
%\setlength{\arrnodesepB}{8pt}
%\setlength{\arroffsetA}{2pt}
%\setlength{\arroffsetB}{0pt}
\ncarr{Al}{At}
\alabel{id_A}
\ncarr{Al}{Ab}
\blabel{id_A}
\ncarr{At}{B1}
\alabel{f_1}[0.5]
\ncarr{At}{B2}
\alabel{f_2}[0.4][-1]
%\blabel{\vdots}[0.4][-2]  % move up 5pts -- dont know why I need this to get position for vdots
\ncarr{At}{Bn}
\blabel{f_n}[0.3][-2]
\ncarr{Ab}{B1}
\alabel{f_1}[0.3][-1]
\ncarr{Ab}{B2}
\blabel{f_2}[0.3][-1]
\ncarr{Ab}{Bn}
\blabel{f_n}[0.4]
%\alabel{\vdots}[0.4]

} 
is a limit cone.
\end{center}
\end{frame}

\begin{frame}{From here..}
\begin{itemize}
\item I will
\begin{itemize}
   \item give an example of nested tables of data
   \item describe relational model  and other data models 
   %\pause \item touch on my favourite -- ER modelling and ER script
   \pause \item define GCs for sketches of categories as data specifications 
   \pause \item define GCs for sketches of categories with designated monomorphisms and epimorphisms
   \pause \item define an abstract version of the Boyce-Codd normal form (BCNF) criteria 
   \pause \item define GCs for sketches of categories with designated monos and epis and with finite products
  \pause \item  show that if the generic goodness criteria hold then with some additional assumptions we can prove that the BCNF criteria holds
\end{itemize}
\end{itemize}
\end{frame}



\subsection{Literature}

\begin{frame}{Categories and Data Specifications}
\begin{itemize} \footnotesize
\item  \cite{Johnson93} 
\begin{itemize} 
\item describe a category equivalent to an ER schema which they loosely describe as the classifying category of the schema
\item argue that it is/needs to be a lextensive category
\item those objects of the category which are coproducts of morphisms from terminal object represent the attribute type (though they use the term attribute here).
\end{itemize}
\pause \item developed further in \cite{johnson2002REL} and \cite{Johnson2002ERA}
\begin{itemize} \footnotesize
\pause \item The term `EA-sketch' introduced for a sketch representing an ER model 
\pause \item comparison with the relational model of data and with categorical data specifications\cite{piessens1995}
\end{itemize}
\pause \item Cadish and Diskin wrote a preprint with subtitle a Manifesto of Categorizing DataBase Theory in 1995.
\end{itemize}
\end{frame}

\begin{frame}{\cite{Johnson2002ERA}}
\bibentry{Johnson2002ERA}

``An EA sketch E=(G,D,L,C) is a sketch with only finite cones and finite discrete cocones and with a 
specified cone with empty base whose vertex is called 1. Edges with domain 1 are called elements. 
Nodes which are vertices of cocones all of whose injections are elements are called attributes. 
Nodes which are neither attributes, nor 1, are called entitites.''
\end{frame}



\begin{frame}{Category of Partial Maps}
Formalised by Cockett and Lack as 'Restriction Categories'.
There is a locally ordered 2-category associated with a restriction category.
Many diagrams commute upto $\leq$.
Outer Join is pullback.
Inner Join is 2-pullback.
\end{frame}

\begin{frame}{Bibliography}
{\tiny
\bibliography{../../SharedBibliography/temp/bibliography}
}
\end{frame}



\iffalse
\subsection{Categories}

\comingnext{General definitions}

\begin{frame}{Data Specification and Instance}
A \textit{data specification} is a sketch $S$ for some kind of structured category \catc.
i.e. it is a directed graph plus some path equivalences (diagrams in other words) plus 
some additional structure.  \\
\medskip
An \textit{instance} of a data specification is a mapping $D$ from the directed graph
to the category of sets and functions which preserves structure so as
to induce a structure preserving functor $D: \catc \morph \Fin$.
\end{frame}

\begin{frame}{The Requirement}
A \textit{requirement} for a data specification $S$ 
is a set of instances of the sketch $S$ or, equivalently, is a set $R_C$ of functors where for each
$D \in R_C$, $D: \catc \morph \Fin$, where \catcw is the category generated by the sketch $S$.
\end{frame}

\begin{frame}{Goodness Criteria for a Data Specification}
\IfSforCwithRCwordsvariant 
\begin{itemize}
\item 
\textbf {Goodness Criteria of Type 1 :} No redundancy. The sketch $S$ ought to be a minimum sketch for structured category \catcw i.e. there should be no subsketch of $S$ which generates  \catc.
\item
\textbf {Goodness Criteria of Type 2:} \catcw ought to be \textit{maximally constrained} to $\reqtc$.
\end{itemize}
\end{frame}

\begin{frame}{Rationale for Type 1 Criteria}
Must follow the type 1 criteria 
\begin{itemize}
\item to ensure there is no redundancy in the  data that is stored and/or communicated,
\item avoid unnecessary data constraints (such as those that impose uniqueness or referential integrity).
\end{itemize} 
\medskip
One thing to have a system that is out of step with the real world quite another to have a system 
in a state that doesn't correspond to any possible  real world state.
\end{frame}

\begin{frame}{Rationale for Type 2 Criteria}
Try all we can to avoid the possibility of a system state 
that doesn't correspond to any real world state.
\begin{itemize}
\item Good data modelling is good theorising and good theorising is having the tightest theory that fits the facts.
\item Software Engineering: 
\begin{itemize}
\item Document the requirement.
\item Validate data against the requirement. 
\item Restrict degrees of freedom in the data representation.
\item Use declarative style if possible.
\end{itemize}
\end{itemize}
\end{frame}

\begin{frame}{Key definition: \catcw \textit{maximally constrained} to $\reqtc$}
\begin{itemize}


\item  question -- is there a $\catcp$ that extends \catcw and that will do a better job. 

\item Is there a $\catcp$ and an $I: \catc \morph \catcp$  such that 
all instances in the requirement $\reqtc$ uniquely factor though $I$
$$
\begin{array} {c p{2cm} c}
\Rnode{Cp}{C'} && \\ [0.25cm]
             && \Rnode{finset}{\Fin} \\ [0.15cm]
\Rnode{C}{C}  
\end{array}
\begin{arrows}
\ncarr {C}{finset}
\alabel{D}
\ncarr{C}{Cp}
\alabel{I}
\ncarr{Cp}{finset}
\alabel{D'} 
\end{arrows}
$$
\pause  and at least one other instance $F$ of \catcw does not factor through $I$.
\ncarr[-20]{C}{finset}
\blabel{F} 
\end{itemize}
\pause If there is no such $I: \catc \morph \catcp$ then we shall say that 
\catcw is \term{maximally constrained} with respect to $\reqtc$.\\
\medskip
 ...meaning  that structured category \catcw  is the tightest possible fit to facts i.e. to the requirement $R_{\catc}$.
\end{frame}

\begin{frame}{Representational Completeness}
Alternative way of approaching tightest fit:
\begin{itemize}
\item That which is in the data and can be represented in the theory should be represented in the theory.
\item To make precise we can give definitions
 of \textit{representational completeness} wrt $\reqtc$ 
\begin{center}
\begin{tabular}{>{\bfseries}l l} 
2A. & equationally complete   \\
2B. & functionally complete   \\
2C. & referentially complete  \\
2D. & mono complete           \\
2E. & epi complete            \\
2F. & product complete        \\
\end{tabular}
\end{center}
\pause \item In these defintions that \catcw is $x$ complete wrt $\reqtc$ will mean exactly that the set of instances $\reqtc$ are jointly reflective of $x$.
\end{itemize}
\end{frame}

\begin{frame}{Next steps}
For each of
\begin{itemize}
\item categories,
\item categories with designated monomorphisms and epimorphisms,
\item categories with designated monomorphisms and epimorphisms and with finite products
\end{itemize}
\medskip
I am going to
\begin{itemize}
\item define appropriate notions of representational completeness,
\item explore the relationship between representational completeness and maximal constrainedness.
\end{itemize}
{\large Before which -- BCNF that which I want to get to as a consequence of notions of tightest fit.}
\end{frame}


\begin{frame}{BCNF in the abstract (based on Zaniolo 1982 Definition 2)}
If $S$ is a sketch for a category $C$ with designated mono-sources  and finite products
and $S$ is considered as a data specification with requirement $\reqtc$, then it ought to be the case
that if 
\scalebox{0.9}{\multisourcenplusonediagram{n}{a}{b}{x}{c}{y}}
are edges of $S$
and if \msfd{x_1,...x_n}{y}  is a non-trivial functional dependency  between these edges
then \onslide<1>{...}
\onslide<2>{\scalebox{0.8}{\multisourcediagram{n}{a}{b}{x}} is a designated mono-source.}
\end{frame}

\comingnext{Sketches of Categories as Data Specifications}

\begin{frame}{Sketches of Categories as Data Specifications}
By a \term{sketch for a category} we mean a pair $\tuple{G,PE}$ where 
$G$ is a directed graph and $PE$ is a set of path equivalences. \\
\medskip
We interpret  the type 1 goodness criteria that the sketch be minimal as meaning 
\begin{itemize}
\item \goodnessoneA. \\
\medskip
\item \goodnessoneB.
\end{itemize}
\end{frame}

\begin{frame}{Equational Completeness and Goodness Criteria 2A}
\pause \begin{definition}
If $\catc$ is a  category and $\reqtc$ is a set of instances,
 then say that  $\catc$ is \term{equationally complete} with respect 
to the requirement $\reqtc$ iff all path equivalences with respect to $R_C$ are represented in \catcw 
i.e. iff for all diagrams \fgparalleldiagram in $\catc$,  
if in all instances $D \in \reqtc$, $D(f)=D(g)$,  then $f=g$.
\end{definition}
\medskip
\pause In other words:
the set of functors $\reqtc$ is jointly faithful. \\
\medskip
\pause \goodnesscriteria{2A} \IfSforCwithRCwords then \catcw ought to be equationally complete
with respect to $R_C$.
\end{frame}

\begin{frame}{Equational completeness cont. }
Does equational completeness follow from maximal constrainedness?
It does if we assume local finiteness of \catc:
\begin{lemma}[Path Equivalence Representation Lemma]
If $\catc$ is a locally finite category and $\reqtc$ is a set of instances, if $\catc$ 
 is
\textit{maximally constrained} to the requirement $\reqtc$ then it is equationally
complete with respect to $\reqtc$.
\end{lemma}
\begin{proof}
Suppose \fgparalleldiagram  in $\catc$ and that in all instances $D \in \reqtc$, $D(f)=D(g)$. 
Define $\catcp$ to be \catc plus $f=g$. Consider the Hom functor $Hom_{\catc}(a,\_)$. 
This is an instance of $\catc$ which therefore extends to an instance $H$ of $\catcp$.  Then
$Hom_{\catc}(a,f)=H(I(f))=H(I(g))=Hom(a,g)$ and applying to $id_a$ we have that $f=g$.
\end{proof}
\end{frame}

\begin{frame}{Does Criteria 2A follow from Criteria 2 -- Not always}
Suppose \catcw is the category generated by the sketch with directed graph
\begin{displaymath}
\begin{array}{cp{1.4cm}c}
                                    \\[0.1cm]
\Rnode{a}{a}	&& \Rnode{b}{b}     \\[0.25cm]
	            &&  
\end{array}
\begin{arrows}
\ncarr[15]{a}{b}
\alabel{f}[0.35]
\ncarr[-15]{a}{b}
\blabel{g}[0.35]
\ncarr[-70]{a}{b}
\blabel{h'}[0.35]
\ncarr[-70]{b}{a}
\blabel{h}[0.35]
\nccircle[angleA=-90, nodesep=3pt]{->}{b}{.5cm}
\blabel{r}[0.3]
\end{arrows}
\end{displaymath}

subject to the identities
\begin{equation}
\label{fhidentity}
f \circ h = id_a,
\end{equation}
\begin{equation}
\label{ghidentity}
g \circ h = id_a,
\end{equation}
\begin{equation}
\label{rhhpidentity}
r \circ h \circ h' = id_b.
\end{equation}
We can show that for any functor $D:\catc \morph Fin$, $D(f)=D(g)$. 

If $R_C$ is the set of all functors from \catcw into $\Fin$ then with respect to $\reqtc$, Criteria 2 holds but Criteria 2A does not.
\end{frame}

\begin{frame}{Functional Dependencies}
To describe Goodness Criteria 2B I first need to
\begin{itemize}
\item Define what we mean by \term{functional dependency}
-- abstracted and simplified from definition given by Codd 1971.
\item Define what we mean by a functional dependency being \term{represented}
-- inspired by language I found in Zaniolo 1982.
\item State as the criteria that all functional dependencies ought to be represented -- which I think is in
the spirit of Zaniolo's paper. 
\end{itemize}
\end{frame}

\begin{frame}{Definition of Functional Dependency}
\begin{definition}
If \scalebox{0.9}{\fgsourcediagram} in a category \catcw  and if $\reqtc$ is a set of instances of \catcw
then we say that there is a  \term{functional dependency} of $g$ on $f$ with respect to $\reqtc$ iff
for every $D \in \reqtc$, $D(g)$ factors uniquely through $D(f)$. \\
\pause \medskip
i.e.  in every $D$, $D(f)$ is surjective and there is a family of  functions $H_D:D(b) \morph D(c))_{D \in \reqtc}$
such that in each instance $D \in \reqtc$,
\scalebox{0.9}{\fghfactordiagram{\roomup{0.4cm}\roomdown{0.3cm}D(a)}{D(b)}{D(c)}{D(f)}{D(g)}{H_D}} commutes.
\end{definition}

\pause If $H$ is such a functional dependency then we say that $\fundep{H}{f}{g}$ in $\catc$ with respect to $\reqtc$.
\end{frame}

\begin{frame}{Notation Alert}
The notation 
$$
f \morph g
$$
for functional dependency is adapted from relational database theory. 

\begin{itemize}
\item This is not to imply a morphism or 2-cell or anDefinition of y such thing.
\end{itemize}
\end{frame}

\begin{frame}{Definition of Representation of Functional Dependency}
If $\catc$ is a category and $\reqtc$ is a set of instances, 
if \fgsourcediagram in $\catc$ 
and if there is a functional dependency $\fundep{H}{f}{g}$ then say that 
this functional dependency  is \term{represented} in $\catc$ 
iff there exists a morphism $h:b \morph c$ in $\catc$ such that for each instance $D \in \reqtc$, $D(h)=H_D$. \\
\medskip
\pause If \catcw is a requirement and $\reqtc$ a set of instances then \catcw is said to be 
\textit{functionally complete} with respect to $\reqtc$ iff every functional dependency
present in $\reqtc$ is represented in \catc.\\
\medskip
\pause \goodnesscriteria{2B}\IfSforCwithRCwords then \catcw ought to be functionally complete with respect to $\reqtc$.
\end{frame}

\begin{frame}{Representation Lemma for Functional Dependencies}
I can prove this only in the case that \catcw is locally finite.

\begin{lemma}
If $\catc$ is a locally finite category and $\reqtc$ is a set of instances, if $\catc$ is 
\textit{maximally constrained} to the requirement $\reqtc$ then 
\catcw is functionally complete with respect to $\reqtc$.
\end{lemma}
\begin{proof}
Suppose$\fundep{H}{f}{g}$  is a functional dependency with respect to $\reqtc$.
Hence in all instances $D \in \reqtc$, $D(f)$ is surjective.

Extend \catcw to $\catcp$ by formally adding a morphism $\qq{h}$ such that $f \circ \qq{h}=g$. Extend each $D$ to a $D'$ by defining $D(h)=H_D$. 
Because $D(f)$ is surjective $D'$ is unique. 
Define a functor from \catcw to $\Fin$ as  a certain quotient of the coproduct of functor $Hom_{\catc}(a,-)$ with itself. 
Extend to $\catcp$ and demonstrate that there exists $k:b \morph c$ in \catcw such that $f \circ k=g$. 
which shows that $\fundep{H}{f}{g}$ is represented in $\catc$.
\end{proof}
\end{frame}

\begin{frame}{Proof 1.}
We have \fgsourcediagram in \catc.\\
\medskip
Define functor $F: \catc \morph \Fin$ be the coproduct $Hom_{\catc}(a,-) + Hom_{\catc}(a,-)$.
Label the injections $L$ and $R$, respectively so that
for each object $x$ of $\catc$ the diagram
\begin{center}
$
\begin{array}{c p{0.5cm} c p{0.5cm} c  }
\Rnode{h1}{Hom_{\catc}(a,x)}  &&\Rnode{Fx}{F(x)}  &&   \Rnode{h2}{Hom_{\catc}(a,x)}       
\end{array} 
$
\ncarr{h1}{Fx}
\alabel{L_x}
\ncarr{h2}{Fx}
\blabel{R_x}
\end{center}
is a coproduct in $\Fin$.
\medskip
Define $G: \catc \morph \Fin$ as quotient $F/\sim$ where
\begin{align*}
L_x(k_1) \sim_x R_x(k_2) & \mbox{ iff there exists $k:b \morph x$ in $\catc$ such that $k_1 = f \circ k = k_2$,}\\
L_x(k_1) \sim_x L_x(k_2) & \mbox{ iff $k_1 = k_2$,} \\
R_x(k_1) \sim_x R_x(k_2) & \mbox{ iff $k_1 = k_2$.} \\
\end{align*} 
\end{frame}

\begin{frame}{Proof 2.}
We have \simpleunaryfdrepresentationdiagram{a}{b}{c}{f}{g}{\qq{h}}  in \catcp.\\
\medskip
Extend $G$ to $G':\catcp \morph \Fin.$ \\

\medskip
\begin{align*}
&\mbox{Now} & [L_b(f)]                         &= [R_b(f)]                 \\
&\mbox{i.e.}& G'(f)([L_a(id_a)])              &= G'(f)([R_a(id_a)])       \\
&\therefore & G'(\qq{h}) (G'(f)([L_a(id_a)])) &= G'(\qq{h}) (G'(f)([R_a(id_a)])) \\
&\therefore & G'(g)([L_a(id_a)])              &= G'(g)([R_a(id_a)]) & &    \\
&\therefore & [L_c(g)]                        &= [R_c(g)] 
\end{align*}
Therefore there exists $k:b \morph c$ in 
$\catc$ such that $f \circ k = g$ and we have shown as required that the function dependency
$\fundep{H}{f}{g}$ is represented in $\catc$.
\end{frame}

\newcommand{\incdsetup}{$
\begin{array}{c p{0.5cm} c p{0.5cm} c}
             &&\Rnode{bi}{b_i} &&              \\[0.5cm]
\Rnode{a}{a} &&                && \Rnode{c}{c}
\end{array}
\begin{arrows}
\ncarr{a}{bi}\alabel{f_i}
\ncarr{c}{bi}\blabel{q_i}
\end{arrows}
$}

\newcommand{\incdresolution}{$
\begin{array}{c p{0.5cm} c p{0.5cm} c}
             &&\Rnode{bi}{D(b_i)} &&              \\[0.5cm]
\Rnode{a}{D(a)} &&                && \Rnode{c}{D(c)}
\end{array}
\begin{arrows}
\ncarr{a}{bi}\alabel{D(f_i)}
\ncarr{c}{bi}\blabel{D(q_i)}
\ncarr{a}{c}\blabel{J_D}
\end{arrows}
$}

\begin{frame}{Inclusion Dependencies}
In a relational database there is an \textit{inclusion dependency} denoted
$$a[f_1,...f_n] \subseteq c[q_1,..q_n]$$ 
iff
\begin{itemize}
\item $a$ and $c$ are tables
\item $f_1,...f_n$ are columns of table $a$
\item $q_1,...q_n$ are columns of table $c$
\item in all instances of the database
       $\{\tuple{r.f_1,...r.f_n} | r \in \mbox{ rows of } a\} \subseteq \{\tuple{s.q_1,...s.q_n} | s \in \mbox{ rows of } c\}$.
\end{itemize}
\end{frame}

\begin{frame}{Referential Inclusion Dependencies -- Foreign Keys}
$$a[f_1,...f_n] \subseteq c[q_1,..q_n]$$ 
This inclusion dependency is said to be referential iff to each row $r$ of table $a$ there is a unique row $s$ of table $b$
such that for each $i$, $s.q_i=r.f_i$.
\pause
\begin{itemize}
    \item Then $\tuple{f_1,...f_n}$ is said to be a foreign key referencing the key $\tuple{q_1,...q_n}$.
    \item The constraint is called a foreign key constraint or a referential integrity constraint.
    \item It is the way relationships in data are implemented in relational databases.
    \item Programmers use these to navigate the data.
\end{itemize}
\end{frame}


\begin{frame}{Definition of Inclusion Dependencies}
If $\catc$ is a category and $\reqtc$ is a set of instances 
and if
\incdsetup
in $\catc$, for $i$, $1 \leq i \leq n$, then an \term{inclusion dependency} $J$, written $a[f_1,...f_n] \overset{J}{\subseteq} c[q_1,..q_n]$, is a family of functions $J_D)_{D \in \reqtc}$
such that each instance $D \in \reqtc$, $J_D$ is a function 
 such that
 \incdresolution commutes, for $i$, $1 \leq i \leq n$. \\

\medskip
\begin{itemize}
\item If each $J_D$ is the unique such function then the inclusion dependency is said to be referential. 
\end{itemize}
\end{frame}

\begin{frame}{Definition of Referentially Complete}
\begin{definition}
If $\catc$ is a category and $\reqtc$ is a set of instances and if
\fnsourceqnsource
in $\catc$ and if $a[f_1,...f_n] \overset{J}{\subseteq} c[q_1,..q_n]$ is a referential inclusion dependency
with respect  to $\reqtc$ then say that the inclusion dependency $J$ is \term{represented} in $\catc$
iff there exists a morphism $j:a \morph c$ in $\catc$ such that in each instance $D \in \reqtc$, $D(j) = J_D$. 
\end{definition}
If \catcw is a category and $\reqtc$ a set of instances then 
\catcw is \term{referentially complete} with respect to $\reqtc$ 
iff all referential inclusion dependencies present in $\reqtc$ are represented in \catc.
\end{frame}

\begin{frame}{Goodness Criteria 2C}
\goodnesscriteria{2C} \IfSforCwithRCwords 
then \catcw ought to be referentially complete with respect to $\reqtc$.
\end{frame}

\begin{frame}{Representation Lemma for Referential Inclusion Dependencies}
If \catcw is a locally finite category and $\reqtc$ is a set of instances, if \catcw is 
\textit{maximally constrained} to the requirement $\reqtc$ then
every referential inclusion dependency with respect to $\reqtc$ is represented in $\catc$.
\end{frame}

\begin{frame}{Proof}
Suppose $a[f_1,...f_n] \overset{J}{\subseteq} c[q_1,..q_n]$ is a referential inclusion dependency
where for each $i$, $1 \leq i \leq n$,
\scalebox{0.9}{\incdsetup} in \catc.

Extend category \catcw to a category \catcpw such that in \catcpw there is
a morphism $\qq{r}: a \morph c$ such that for each $i$, $1 \leq i \leq n$, 
$\qq{r} \circ q_i = f_i$.
Every $D \in \reqtc$ extends uniquely to functor $D: \catcp \morph \Fin$. Therefore, since \catcw is maximally constrained then every functor $F: \catc \morph \Fin$ can be extended to $F':\catcp \morph \Fin$. \\
\pause Let $H: \catcp \morph \Fin$ be the extension of the hom functor $Hom(a,-): \catc \morph \Fin$ to 
\catcp. \\
\pause For each $i$, we have $H(\qq{r}) \circ H(q_i) = H(f_i)$, \\
\pause \hspace {3cm} i.e. $H(\qq{r}) \circ Hom(a,q_i) = Hom(a,f_i)$. \\
\pause Therefore in particular $H(\qq{r})(id_a) \circ q_i =  f_i$. \\
\pause Hence $H(\qq{r})(id_a):a \morph c$ in \catcw represents the inclusion dependency $J$.
\end{frame}

\begin{frame}{Three Goodness Criteria}{for Categories as Data Specifications}
Definition of Goodness Criteria 2A, 2B and 2C:
\begin{itemize}
\item If \catcw is a category and $\reqtc$ is a set of instances of \catcw then
\medskip
\begin{tabular}{>{\bfseries}l l} 
2A: & \catcw ought to be equationally complete wrt $\reqtc$  \\
2B: & \catcw ought to be  functionally complete wrt $\reqtc$  \\
2C: & \catcw ought to be referentially complete wrt $\reqtc$ \\
\end{tabular}
\pause \item We have shown that if \catcw is locally finite and meets Criteria 2 that it is maximally constrained then it also meets Criteria 2A, 2B and 2C.

\pause \item We will move on to consider categories with designated monomorphisms and epimorphisms
\item First though we will describe a possible improvement to this situation described above.
\end{itemize}
\end{frame}



\subsection{Categories with Designated Monomorphisms and Epimorphims}
\documentclass[xcolor=pst,dvips]{beamer}   %, handout

\usepackage{mathptmx}
\usepackage{amsfonts}
\usepackage{wasysym}
\usepackage{url}
\usepackage{hyperref}

\newcommand{\sharedmacros}{../../SharedMacros}
%\usepackage{imakeidx}
\usepackage{framed}
\makeindex[name=definitions, title=Index of Definitions]
\makeindex[name=lemmas, title=Index of Lemmas]



\newcommand{\seenudgeup}[1]{\rule{0.1cm}{#1}}

\newcommand{\seenudgedown}[1]{\rule[-#1]{0.1cm}{0.1cm}}

\newcommand{\nudgeup}[1]{\rule{0cm}{#1}}

\newcommand{\nudgedown}[1]{\rule[-#1]{0cm}{0.1cm}}

\definecolor{highlight}{cmyk}{0,0,0.7,0}
\newcommand{\commentary}[1]{\marginpar{\footnotesize #1}}
\newcommand{\highlight}[1]{\colorbox{highlight}{#1}}
\newcommand{\whitelight}[1]{\colorbox{white}{#1}}
\newcommand{\term}[1]{\textit{#1}\commentary{\colorbox{lightgray}{\textit{#1}}}\index[definitions]{#1}}
\newcommand{\llabel}[1]{\label{#1}\commentary{\colorbox{pink}{\scriptsize{#1}}}\index[lemmas]{#1}}
\newcommand{\lref}[1]{\ref{#1}\colorbox{pink}{\scriptsize{#1}}\index[lemmas]{#1!use of}}

\newcommand{\daynote}[1]{\commentary{See day notes #1.}}

\newcommand{\newt}[1]{\colorbox{yellow}{#1}}
\newenvironment{newtt}
{  \colorbox{yellow}{$[$ ...} 
}
{  \colorbox{yellow}{... $]$}
}
\newcommand{\oldt}[1]{\colorbox{red}{\sout{#1}}}
\newenvironment{oldtt}
{  \colorbox{red}{$[$ ...} 
}
{  \colorbox{red}{... $]$}
}

\newcommand{\reinstatet}[1]{\colorbox{lime}{#1}}
\newenvironment{reinstatett}
{  \colorbox{lime}{$[$ ...}
}
{  \colorbox{lime}{... $]$}
}

\newcommand{\tbd}{\highlight{TBD}}

%ithprojection function
\newcommand{\proji}[1]{\pi_#1}


\newenvironment{aside}
{\begin{framed}
\textbf{Aside}
}
{
\end{framed}
}

\newenvironment{notebox}[1][Note]
{\begin{framed}
\textbf{#1}
}
{
\end{framed}
}

\newenvironment{categoricalaside}
{\begin{framed}
\textbf{Categorical Aside}
}
{
\end{framed}
}

\newenvironment{noteforfuture}
{\begin{framed}
\textbf{Note For Future}
}
{
\end{framed}
}

\newenvironment{problem}
{\begin{framed}
\textbf{Problem}
}
{
\end{framed}
}

\newenvironment{key}
{
\begin{tabular}{c l p{4cm}}
KEY && \\
\hline
}
{
\end{tabular}
}

\newcommand{\keyentry}[3]{#1 & #2 & #3 \\} 


%quine quote
\newcommand{\qq}[1]{
\left\ulcorner#1\right\urcorner
}

%single quote
\newcommand{\sq}[1]{
\textnormal{\textquotesingle}#1\textnormal{\textquotesingle}
}

%lower quine quote
\newcommand{\lqq}[1]{
\left\llcorner #1\right\lrcorner
}


%from berkley
\newcommand{\langl}{\begin{picture}(4.5,7)
\put(1.1,2.5){\rotatebox{60}{\line(1,0){5.5}}}
\put(1.1,2.5){\rotatebox{300}{\line(1,0){5.5}}}
\end{picture}}
\newcommand{\rangl}{\begin{picture}(4.5,7)
\put(.9,2.5){\rotatebox{120}{\line(1,0){5.5}}}
\put(.9,2.5){\rotatebox{240}{\line(1,0){5.5}}}
\end{picture}}
\newcommand{\lang}{\begin{picture}(5,7)\put(1.1,2.5){\rotatebox{45}{\line(1,0){6.0}}}\put(1.1,2.5){\rotatebox{315}{\line(1,0){6.0}}}\end{picture}}
\newcommand{\rang}{\begin{picture}(5,7)\put(.1,2.5){\rotatebox{135}{\line(1,0){6.0}}}\put(.1,2.5){\rotatebox{225}{\line(1,0){6.0}}}\end{picture}}
%Try sharper tuple brackets -- except gives errors nested in captions so comment out
%\renewcommand{\tuple}[1]{\lang #1 \rang}

\newcommand{\setsuchthat}[2]{\left\{#1 \ \middle|\ #2\right\}}
\newcommand{\set}[1]{\left\{#1\right\}} 

% one to n - wanton
\newcommand{\wanton}[1]{#1_1,...#1_n}
\newcommand{\n}{1...n}
\newcommand{\fn}{\wanton{f}}
\newcommand{\gn}{\wanton{g}}
\newcommand{\pn}{\wanton{p}}
\newcommand{\qn}{\wanton{q}}
\newcommand{\qnprime}{\wanton{q'}}
\newcommand{\tn}{\wanton{t}}
\newcommand{\xn}{\wanton{x}}
\newcommand{\xnp}{\wanton{x'}}
\newcommand{\yn}{\wanton{y}}
\newcommand{\An}{\wanton{A}}
\newcommand{\Bn}{\wanton{B}}
\newcommand{\Cn}{\wanton{C}}
\newcommand{\ntuple}[1]{\tuple{\wanton{#1}}}
\newcommand{\wantom}[2][]{#2_1,...#2_{m#1}}
\newcommand{\m}{1...m}
\newcommand{\mtuple}[1]{\tuple{#1_1,...#1_m}}
\newcommand{\gm}{\wantom{g}}
\newcommand{\qm}{\wantom{q}}
\newcommand{\sm}[1][]{\wantom[#1]{s}}
\newcommand{\smp}{\wantom{s'}}
\newcommand{\ym}{\wantom{y}}
\newcommand{\Bm}{\wantom{B}}
\newcommand {\bntuple}{\ensuremath{\ntuple{b}}}
\newcommand {\fntuple}{\ensuremath{\ntuple{f}}}
\newcommand {\fnptuple}{\ensuremath{\ntuple{f}}}
\newcommand {\pntuple}{\ensuremath{\ntuple{p}}}
\newcommand {\qntuple}{\ensuremath{\ntuple{q}}}
\newcommand {\qnptuple}{\ensuremath{\ntuple{q'}}}
\newcommand {\qmtuple}{\ensuremath{\mtuple{q}}}
\newcommand {\sntuple}{\ensuremath{\ntuple{s}}}
\newcommand {\xntuple}{\ensuremath{\ntuple{x}}}
\newcommand {\xnptuple}{\ensuremath{\ntuple{x'}}}
\newcommand {\ymtuple}{\ensuremath{\mtuple{y}}}
\newcommand{\idef}[1][n]{1 \leq i \leq #1}
\newcommand{\jdef}[1][m]{1 \leq j \leq #1}
\newcommand{\kdef}[1][l]{1 \leq k \leq #1}
\newcommand{\foreachi}[1][n]{for each $i$, $1 \leq i \leq #1$}
\newcommand{\foreachj}[1][m]{for each $j$, $1 \leq j \leq #1$}
\newcommand{\foreachk}[1][l]{for each $k$, $1 \leq k \leq #1$}
\newcommand{\Foreachi}[1][n]{For each $i$, $1 \leq i \leq #1$}
\newcommand{\Foreachj}[1][m]{For each $j$, $1 \leq j \leq #1$}
\newcommand{\Foreachk}[1][l]{For each $k$, $1 \leq k \leq #1$}
\newcommand{\forsomei}[1][n]{for some $i$, $1 \leq i \leq #1$}
\newcommand{\forsomej}[1][m]{for some $j$, $1 \leq j \leq #1$}
\newcommand{\forsomek}[1][l]{for some $k$, $1 \leq k \leq #1$}
\newcommand{\wherei}[1][n]{where $1 \leq i \leq #1$}
\newcommand{\wherej}[1][m]{where $1 \leq j \leq #1$}
\newcommand{\wherek}[1][l]{where $1 \leq k \leq #1$}


\newcommand{\fundep}[3]{#2 \xrightarrow{#1} #3}  %where does this belong? xxxx
% Following used for notes -- indented numbered paras

\newcounter{para}
\newlength{\oldparindent}
\setlength{\oldparindent}{\parindent} % Save \parindent before of change
\newcommand{\ind}{\hspace*{\oldparindent}}
\newcommand\note{
%\setlength{\parskip}{0.5\baselineskip} % Definition of `parskip`
\setlength{\parindent}{0pt}
\par\ind\refstepcounter{para}\thepara.\space
\setlength{\parindent}{\oldparindent}
}


    %causes problems when used with bamer

%ccategories.macros.tex 

% Macros for diagrams in contextual categories and related categories

\usepackage{twoopt}
\usepackage{scalerel} 
\usepackage{xargs}

%\usepackage{mathabx}  %Caused font problems
%\usepackage{MnSymbol}  % caused font problems

\newcommand{\conu}
{\mathbf{C}(U)}

\newcommand{\depu}
{\mathbf{D}(U)}


\newcommand{\reqt}{\textbf{R}}
\newcommand{\reqtc}[1][\catc]{\reqt_{#1}}
\newcommand{\reqtcp}[1][\catcp]{\reqt_{#1}}



\newcommand{\cat}[1]{\textbf{#1}}

\newcommand{\catc}{\cat{C}}
\newcommand{\catcw}{\cat{C}\ }
\newcommand{\catcp}[1][C]{\textbf{#1}'}
\newcommand{\catcpp}[1][C]{\textbf{#1}''}
\newcommand{\obj}[1]{\ensuremath{|\cat{#1}|}}
\newcommand{\ccat}[1][C]{\ensuremath{\mathbb{#1}} }
\newcommand{\ccatc}{contextual category \ccat}
\newcommand{\cobj}[2][]{\ensuremath{|\ccat[#2]|_{#1}}}
\newcommand{\cslice}[2]{\ensuremath{\ccat[#1]_{#2}}}
\newcommand{\csliceobj}[3][]{\ensuremath{|\mathbb{#2}_{#3}|_{#1} }}
\newcommand{\varset}[1][]{\ensuremath{V_{#1} }}
\newcommand{\localvarsets}{\ensuremath{\mathcal{V} }}
\newcommand{\Fam}{\ensuremath{\mathbb{F\mathrm{am}} }}
\newcommand{\Fin}{\ensuremath{\textbf{Fin}} }
\newcommand{\Finp}{\ensuremath{\textbf{Finp}} }
\newcommand{\Po}{\ensuremath{\textbf{Po}} }
\newcommand{\Famslice}[1]{\ensuremath{\mathbb{F\mathrm{am}}_{#1} }}
\newcommand{\Famobj}[1][]{\ensuremath{|\mathbb{F\mathrm{am}}|_{#1} }}
\newcommand{\Famsliceobj}[2][]{\ensuremath{|\mathbb{F\mathrm{am}}_{#2}|_{#1} }}
\newcommand{\morph}{\rightarrow}
\newcommand{\epi}{\twoheadrightarrow}
\newcommand{\base}{\triangleleft}
\newcommand{\comp}{\circ}
\newcommand{\cross}{\otimes}
\newcommand{\pc}[2]{d^{#1}_{#2}}
\newcommand{\sub}{^*}
\newcommand{\diag}{\delta}
\newcommand{\pbase}[1]{\tilde{#1}}
\newcommand{\tuple}[1]{\langle#1\rangle}
\newcommand{\ndidly}{\ensuremath{\Join_n}}

\newcommand{\product}[1]{\bigtimes_{#1}}
\newcommand{\productn}{\product{n}}
\newcommand{\crossx}[3]{#1 \underset{#3}{\cross} #2}
\newcommand{\fibrex}[3]{#1 \underset{#3}{\Join} #2}
\newcommand{\powerset}{\mathcal{P}}
\newcommand{\primeds}[1]{
\ensuremath{\mathcal{P}(#1)} }
\newcommand{\compset}{\ \dot{\circ}\, }

% darrow
%\newcommand{\darrow}{\rightarrowtriangle} %use \smorph instead
\newcommand{\smorph}{\rightarrowtriangle}

 
\newcommand\dhead{\scaleobj{0.6}{\triangleright}}
%\newcommand{\dmorph}{\, \mbox{---} \! \cdot \! \raisebox{1.1pt}{\dhead}}    % dot style
\newcommand{\dmorph}{\, \mbox{---}\kern-1pt\raisebox{1.1pt}{\dhead\kern-1.75pt\dhead}}\,     % double triangle style

% projection tree
%\newcommand{\proj}[2]{proj_{#2}(#1)}

\newcommand{\proj}[2]{
\ensuremath{\mathcal{P}_{#2}(#1)} }

%pstrick supplements for arrows

\newlength{\arrnodesepA}
\newlength{\arrnodesepB}
\newlength{\arroffsetA}
\newlength{\arroffsetB}

%Modified to 2pt from 0pt on 23 July 2018
\newcommand{\arreset}{
\setlength{\arrnodesepA}{2pt}
\setlength{\arrnodesepB}{2pt}
\setlength{\arroffsetA}{0pt}
\setlength{\arroffsetB}{0pt}
}
\arreset

\newcommand{\ncarr}[3][0]{\ncarc[arcangle=#1,nodesepA=\arrnodesepA,nodesepB=\arrnodesepB,offsetA=\arroffsetA,offsetB=\arroffsetB,arrowsize=5pt,arrowinset=0.7]{->}{#2}{#3}}
\newcommand{\ncdarr}[3][0]{\ncarc[linestyle=dashed,arcangle=#1,nodesepA=\arrnodesepA,nodesepB=\arrnodesepB,offsetA=\arroffsetA,offsetB=\arroffsetB,arrowsize=5pt,arrowinset=0.7]{->}{#2}{#3}}
\newcommand{\jcbarr}[4][0]{ % ncbarr is defined in some thridy party package so do not use!\emph{}
\ncarr[#1]{#3}{#4}
\nbput[labelsep=2pt]{\footnotesize $#2$}
}

\newcommand{\ncaarr}[4][0]{
\ncarr[#1]{#3}{#4}
\naput[labelsep=2pt]{\footnotesize $#2$}
}

% \alabel{label}[npos][labelsep_pts]
\newcommandx*\alabel[3][2=0.5,3=2,usedefault]{\naput[labelsep=#3pt,npos=#2]{\footnotesize $#1$}}
% \blabel{label}[npos][labelsep_pts]
\newcommandx*\blabel[3][2=0.5,3=2,usedefault]{\nbput[labelsep=#3pt,npos=#2]{\footnotesize $#1$}}


\newif \ifbars
% to supress display of bars use \barsfalse to swith them on use \barstrue
\barstrue 
% \idcomp mark an arrow as one component of an identifier
\newcommand{\idcomp}{\ifbars{\ncput[npos=0, nrot=:U]{\psline(0.2,-0.075)(0.2,0.075)}}\fi}  %add a bar to a node connection arrow
% pstrick supplements for s-arrows (previous name for d-arrow - should convert}

\newlength{\sarnodesepA}
\newlength{\sarnodesepB}
\newlength{\saroffsetA}
\newlength{\saroffsetB}
\newlength{\sarnodesepAsav}
\newlength{\sarnodesepBsav}

\newcommand{\sarreset}{
\setlength{\sarnodesepA}{0pt}
\setlength{\sarnodesepB}{0pt}
\setlength{\saroffsetA}{0pt}
\setlength{\saroffsetB}{0pt}
}

\sarreset

% sar - S-arrow
\newcommand{\ncsar}[3][0]{
\setlength{\sarnodesepAsav}{\sarnodesepA}
\setlength{\sarnodesepBsav}{\sarnodesepB}
\addtolength{\sarnodesepA}{3pt}
\addtolength{\sarnodesepB}{7pt}
\ncarc[nodesepA=\sarnodesepA,nodesepB=\sarnodesepB,offsetA=\saroffsetA,offsetB=\saroffsetB,arcangle=#1]{-}{#2}{#3}
\ncput[nrot=:R,npos=1]{\pstriangle(0,0)(.2,.2)}
\setlength{\sarnodesepA}{\sarnodesepAsav}
\setlength{\sarnodesepB}{\sarnodesepBsav}
}


% bsar - below labelled S-arrow
\newcommand{\ncbsar}[4][0]{
\ncsar[#1]{#3}{#4}
\nbput[labelsep=2pt]{\footnotesize $#2$}
}
% asar - above labelled S-arrow
\newcommand{\ncasar}[4][0]{
\ncsar[#1]{#3}{#4}
\naput[labelsep=2pt]{\footnotesize $#2$}
}

% OLD cdar - composite dependency arrow - dot tyle
\iffalse
\newcommand{\nccdar}[3][0]{
\setlength{\sarnodesepAsav}{\sarnodesepA}
\setlength{\sarnodesepBsav}{\sarnodesepB}
\addtolength{\sarnodesepA}{3pt}
\addtolength{\sarnodesepB}{11pt}
\ncarc[nodesepA=\sarnodesepA,nodesepB=\sarnodesepB,offsetA=\saroffsetA,offsetB=\saroffsetB,arcangle=#1]{-}{#2}{#3}
\ncput[nrot=:R,npos=1]{\pstriangle(0,0.1)(.2,.2)}
\ncput[nrot=:R,npos=1]{\psdot[dotsize=1pt](-0.0075,0.05)}   %!!
\setlength{\sarnodesepA}{\sarnodesepAsav}
\setlength{\sarnodesepB}{\sarnodesepBsav}
}
\fi

% cdar - composite dependency arrow Mark II - double trangle style
\newcommand{\nccdar}[3][0]{
\setlength{\sarnodesepAsav}{\sarnodesepA}
\setlength{\sarnodesepBsav}{\sarnodesepB}
\addtolength{\sarnodesepA}{3pt}
\addtolength{\sarnodesepB}{13pt}
\ncarc[nodesepA=\sarnodesepA,nodesepB=\sarnodesepB,offsetA=\saroffsetA,offsetB=\saroffsetB,arcangle=#1]{-}{#2}{#3}
\ncput[nrot=:R,npos=1]{\pstriangle(0,0)(.2,.2)}
\ncput[nrot=:R,npos=1]{\pstriangle(0,0.2)(.2,.2)}
\setlength{\sarnodesepA}{\sarnodesepAsav}
\setlength{\sarnodesepB}{\sarnodesepBsav}
}


% bcdar - below labelled composite dependency arrow
\newcommand{\ncbcdar}[4][0]{
\nccdar[#1]{#3}{#4}
\nbput[labelsep=2pt]{\footnotesize $#2$}
}
% acdar - above labelled composite dependency arrow
\newcommand{\ncacdar}[4][0]{
\nccdar[#1]{#3}{#4}
\naput[labelsep=2pt]{\footnotesize $#2$}
}


% rsar - recursive S-arrow
\newcommand{\ncrsar}[2]{
\setlength{\sarnodesepAsav}{\sarnodesepA}
\setlength{\sarnodesepBsav}{\sarnodesepB}
\addtolength{\sarnodesepA}{3pt}
\addtolength{\sarnodesepB}{7pt}
\ncloop[nodesepA=\sarnodesepA,nodesepB=\sarnodesepB,
        offsetA=\saroffsetA,offsetB=\saroffsetB,
        armA=0.7cm,armB=0.6cm,angleA=90,angleB=-90,loopsize=-1,linearc=0.4
				]{-}{#1}{#2}
\ncput[nrot=:R,npos=5]{\pstriangle(0,0)(.2,.2)}
\setlength{\sarnodesepA}{\sarnodesepAsav}
\setlength{\sarnodesepB}{\sarnodesepBsav}
}

% pstrick supplements for multi-arrows

\newlength{\marnodesepA}
\newlength{\marnodesepB}
\newlength{\maroffsetB}
\newlength{\marnodesepBsav}

\newcommand{\marreset}{
\setlength{\marnodesepA}{0pt}
\setlength{\marnodesepB}{0pt}
\setlength{\maroffsetB}{0pt}
}

\marreset

%ncmarr[#1 arcangle1][#2 arcangle2]{#3 name}{#4 domain1}{#5 domain2}{#6 junction}{#7 codomain}
\newcommandtwoopt{\ncmarr}[6][8][8]{%
\ncarc[nodesepA=\marnodesepA,nodesepB=0,arcangle=#1]{-}{#3}{#5}
\ncarc[nodesepB=0,arcangle=-#1]{-}{#4}{#5}
\ncarc[arcangle=#2,nodesepB=\marnodesepB,offsetB=\maroffsetB]{->}{#5}{#6}
}%


\newcommandtwoopt{\nchmarr}[6][8][8]{%
\ncarc[nodesepA=\marnodesepA,nodesepB=0,arcangle=#1]{-}{#3}{#5}
\ncarc[nodesepB=0,arcangle=#1]{-}{#4}{#5}
\ncarc[arcangle=#2,nodesepB=\marnodesepB,offsetB=\maroffsetB]{->}{#5}{#6}
}%

\newcommandtwoopt{\ncamarr}[7][8][8]{%
\ncmarr[#1][#2]{#4}{#5}{#6}{#7}
\naput[npos=.05]{$#3$}
}%
\newcommandtwoopt{\ncbmarr}[7][8][8]{%
\ncmarr[#1][#2]{#4}{#5}{#6}{#7}
\nbput[npos=.05]{$#3$}
}%

\newcommandtwoopt{\ncbhmarr}[7][8][8]{%
\nchmarr[#1][#2]{#4}{#5}{#6}{#7}
\nbput[npos=.05]{$#3$}
}%

\newcommandtwoopt{\ncmarrr}[7][8][8]{
\ncarc[nodesepB=0,arcangle=#1]{-}{#3}{#6}
\ncline[nodesepB=0]{-}{#4}{#6}
\ncarc[nodesepB=0,arcangle=-#1]{-}{#5}{#6}
\ncarc[nodesepA=0,arcangle=#2]{->}{#6}{#7}
}

\newcommandtwoopt{\ncamarrr}[8][8][8]{
\ncmarrr[#1][#2]{#4}{#5}{#6}{#7}{#8}
\naput[npos=.05]{$#3$}
}
\newcommandtwoopt{\ncbmarrr}[8][8][8]{
\ncmarrr[#1][#2]{#4}{#5}{#6}{#7}{#8}
\nbput[npos=.05]{$#3$}
}


% 6 June 2020
% Edges representing attributes and relationship graphs
%  Ep   - partial
%  Epm  - partial mono
%  Epe  - partial epi
%  Epme - partial mono epi
%  Et   - total
%  Etm  - total mono
%  Ete  - total epi
%  Etme - total mono epi
%  recursive edges (use nccircle)
%  rEp   - partial
%  rEpm  - partial mono
%  rEpe  - partial epi
%  rEpme - partial mono epi
%  rEt   - total
%  rEtm  - total mono
%  rEte  - total epi
%  rEtme - total mono epi

\newcounter{EangleA}
\newcounter{EangleB}
\newcounter{EmidangleA}
\newcounter{EmidangleB}

% Ep - Edge partial
\newcommandtwoopt{\Ep}[4][0][0]{
\crowsfootedEdge{#1}{#2}{#3}{#4}{dashed}{dashed}
}



% Epm - Edge partial mono
\newcommandtwoopt{\Epm}[4][0][0]{
\monoEdge{#1}{#2}{#3}{#4}{dashed}{dashed}
}


% Epe - Edge partial epi
\newcommandtwoopt{\Epe}[4][0][0]{
\crowsfootedEdge{#1}{#2}{#3}{#4}{dashed}{solid}
}

% Epme - Edge partial mono epi
\newcommandtwoopt{\Epme}[4][0][0]{
\monoEdge{#1}{#2}{#3}{#4}{dashed}{solid}
}

% Et - Edge total
\newcommandtwoopt{\Et}[4][0][0]{
\crowsfootedEdge{#1}{#2}{#3}{#4}{solid}{dashed}
}

% Etm - Edge total mono
\newcommandtwoopt{\Etm}[4][0][0]{
\monoEdge{#1}{#2}{#3}{#4}{solid}{dashed}
}

% Ete - Edge total epi
\newcommandtwoopt{\Ete}[4][0][0]{
\crowsfootedEdge{#1}{#2}{#3}{#4}{solid}{solid}
}

% Etme - Edge total mono epi
\newcommandtwoopt{\Etme}[4][0][0]{
\monoEdge{#1}{#2}{#3}{#4}{solid}{solid}
}

% crowsfootedEdge - \crowsfootedEdge[angleA][midpointangle]{startnode}{endnode}[startstyle][endstyle]
\newcommand{\crowsfootedEdge}[6]{
\setlength{\sarnodesepAsav}{\sarnodesepA}
\setlength{\sarnodesepBsav}{\sarnodesepB}
\addtolength{\sarnodesepA}{3pt}
\addtolength{\sarnodesepB}{3pt}
\setcounter{EangleA}{ #1 + #2}
\setcounter{EangleB}{180  - #1 + #2}
\setcounter{EmidangleA}{#2}
\setcounter{EmidangleB}{#2 + 180}
\nccurve[nodesepA=\sarnodesepA,nodesepB=\sarnodesepB,offsetA=\saroffsetA,offsetB=\saroffsetB,angleA=\theEangleA, angleB=\theEangleB,linestyle=none,linewidth=0]{->}{#3}{#4}
\ncput[nrot=:R,npos=0]{\psline(0,.1)(.075,0)}
\ncput[nrot=:R,npos=0]{\psline(0,.1)(-0.075,0)}
\ncput{\pnode(0,0){xxx}}
\nccurve[nodesepA=0,nodesepB=\sarnodesepB,offsetA=0,offsetB=\saroffsetB,angleA=\theEmidangleA, angleB=\theEangleB, linestyle=#6]{->}{xxx}{#4}
%the following provides context for any following label
\nccurve[nodesepA=\sarnodesepA,nodesepB=0,offsetA=\saroffsetA,offsetB=0,angleA=\theEangleA, angleB=\theEmidangleB,linestyle=#5]{-}{#3}{xxx}
\setlength{\sarnodesepA}{\sarnodesepAsav}
\setlength{\sarnodesepB}{\sarnodesepBsav}
}

% monoEdge - \monoEdge[angleA][midpointangle]{startnode}{endnode}[startstyle][endstyle]
\newcommand{\monoEdge}[6]{ 
\setlength{\sarnodesepAsav}{\sarnodesepA}
\setlength{\sarnodesepBsav}{\sarnodesepB}
\addtolength{\sarnodesepA}{3pt}
\addtolength{\sarnodesepB}{3pt}
\setcounter{EangleA}{ #1 + #2}
\setcounter{EangleB}{180  - #1 + #2}
\setcounter{EmidangleA}{#2}
\setcounter{EmidangleB}{#2 + 180}
\nccurve[nodesepA=\sarnodesepA,nodesepB=\sarnodesepB,offsetA=\saroffsetA,offsetB=\saroffsetB,angleA=\theEangleA, angleB=\theEangleB,linestyle=none,linewidth=0]{->}{#3}{#4}
\ncput{\pnode(0,0){xxx}}
\nccurve[nodesepA=0,nodesepB=\sarnodesepB,offsetA=0,offsetB=\saroffsetB,angleA=\theEmidangleA, angleB=\theEangleB, linestyle=#6]{->}{xxx}{#4}
%the following provides context for any following label
\nccurve[nodesepA=\sarnodesepA,nodesepB=0,offsetA=\saroffsetA,offsetB=0,angleA=\theEangleA, angleB=\theEmidangleB,linestyle=#5]{-}{#3}{xxx}
\setlength{\sarnodesepA}{\sarnodesepAsav}
\setlength{\sarnodesepB}{\sarnodesepBsav}
}


\newcounter{EangleGiven}
\newcounter{EangleComplementary}
\newcounter{EangleStartCorrected}
\newcounter{EangleEndCorrected}


%  rEp   - recursive Edge partial
\newcommand{\rEp}[2][0]{
\setcounter{EangleGiven}{#1}
\setcounter{EangleStartCorrected}{#1-10} %correction required because for nccurve unlike nccircle angle measured at boundary not at centre of node
\setcounter{EangleEndCorrected}{#1+180+10} %correction required because angle measured at boundary not at centre of node
\setcounter{EangleComplementary}{#1 + 180}
\nccircle[angleA=\theEangleComplementary, nodesep=0pt, linestyle=none]{-}{#2}{.4cm} % an invisible circle to hang the midpoint from
\ncput{\pnode(0,0){midpoint}}                                         
\nccurve[nodesepA=1pt,nodesepB=0pt,offsetA=0pt,offsetB=0pt,angleA=\theEangleStartCorrected, angleB=\theEangleGiven, ncurv=1.359, linecolor=black, linestyle=dashed]{-}{#2}{midpoint}
\ncput[nrot=:R,npos=0]{\psline(0,.1)(.075,0)}
\ncput[nrot=:R,npos=0]{\psline(0,.1)(-0.075,0)}
\nccurve[nodesepA=0pt,nodesepB=2pt,offsetA=0pt,offsetB=0pt,angleA=\theEangleComplementary, angleB=\theEangleEndCorrected, ncurv=1.359, linestyle=dashed]{-}{midpoint}{#2}
% 1.359 is e/2 happenchance or algorithmically necessary???
% now draw arrowhead -- dont include in the nccurve because this alters the line position - a strange feature of pstruicks
\ncput[npos=0.9]{\pnode(0,0){yyy}}
\ncline{->}{yyy}{#2}
% repeat from earlier to provide context for label that might follow
\nccurve[nodesepA=1pt,nodesepB=0pt,offsetA=0pt,offsetB=0pt,angleA=\theEangleStartCorrected, angleB=\theEangleGiven, ncurv=1.359, linecolor=black, linestyle=dashed]{-}{#2}{midpoint} 
} 

%  rEpm  - recursive Edge partial mono
\newcommand{\rEpm}[2][0]{
\setcounter{EangleGiven}{#1}
\setcounter{EangleStartCorrected}{#1-10} %correction required because for nccurve unlike nccircle angle measured at boundary not at centre of node
\setcounter{EangleEndCorrected}{#1+180+10} %correction required because angle measured at boundary not at centre of node
\setcounter{EangleComplementary}{#1 + 180}
\nccircle[angleA=\theEangleComplementary, nodesep=0pt, linestyle=none]{-}{#2}{.4cm} % an invisible circle to hang the midpoint from
\ncput{\pnode(0,0){midpoint}}   
\nccurve[nodesepA=0pt,nodesepB=2pt,offsetA=0pt,offsetB=0pt,angleA=\theEangleComplementary, angleB=\theEangleEndCorrected, ncurv=1.359, linestyle=dashed]{-}{midpoint}{#2}
% 1.359 is e/2 happenchance or algorithmically necessary???
% now draw arrowhead -- dont include in the nccurve because this alters the line position - a strange feature of pstruicks
\ncput[npos=0.9]{\pnode(0,0){yyy}}
\ncline{->}{yyy}{#2}
% last to provide context for label that might follow
\nccurve[nodesepA=1pt,nodesepB=0pt,offsetA=0pt,offsetB=0pt,angleA=\theEangleStartCorrected, angleB=\theEangleGiven, ncurv=1.359, linecolor=black, linestyle=dashed]{-}{#2}{midpoint} 
}

%  rEpe  - recursive Edge partial epi
\newcommand{\rEpe}[2][0]{
\setcounter{EangleGiven}{#1}
\setcounter{EangleStartCorrected}{#1-10} %correction required because for nccurve unlike nccircle angle measured at boundary not at centre of node
\setcounter{EangleEndCorrected}{#1+180+10} %correction required because angle measured at boundary not at centre of node
\setcounter{EangleComplementary}{#1 + 180}
\nccircle[angleA=\theEangleComplementary, nodesep=0pt, linestyle=none]{-}{#2}{.4cm} % an invisible circle to hang the midpoint from
\ncput{\pnode(0,0){midpoint}}                                         
\nccurve[nodesepA=1pt,nodesepB=0pt,offsetA=0pt,offsetB=0pt,angleA=\theEangleStartCorrected, angleB=\theEangleGiven, ncurv=1.359, linecolor=black, linestyle=dashed]{-}{#2}{midpoint}
\ncput[nrot=:R,npos=0]{\psline(0,.1)(.075,0)}
\ncput[nrot=:R,npos=0]{\psline(0,.1)(-0.075,0)}
\nccurve[nodesepA=0pt,nodesepB=2pt,offsetA=0pt,offsetB=0pt,angleA=\theEangleComplementary, angleB=\theEangleEndCorrected, ncurv=1.359]{-}{midpoint}{#2}
% 1.359 is e/2 happenchance or algorithmically necessary???
% now draw arrowhead -- dont include in the nccurve because this alters the line position - a strange feature of pstruicks
\ncput[npos=0.9]{\pnode(0,0){yyy}}
\ncline{->}{yyy}{#2}
% repeat from earlier to provide context for label that might follow
\nccurve[nodesepA=1pt,nodesepB=0pt,offsetA=0pt,offsetB=0pt,angleA=\theEangleStartCorrected, angleB=\theEangleGiven, ncurv=1.359, linecolor=black, linestyle=dashed]{-}{#2}{midpoint} 
}

%  rEpme - recursive Edge partial mono epi
\newcommand{\rEpme}[2][0]{
\setcounter{EangleGiven}{#1}
\setcounter{EangleStartCorrected}{#1-10} %correction required because for nccurve unlike nccircle angle measured at boundary not at centre of node
\setcounter{EangleEndCorrected}{#1+180+10} %correction required because angle measured at boundary not at centre of node
\setcounter{EangleComplementary}{#1 + 180}
\nccircle[angleA=\theEangleComplementary, nodesep=0pt, linestyle=none]{-}{#2}{.4cm} % an invisible circle to hang the midpoint from
\ncput{\pnode(0,0){midpoint}}                                         
%\nccurve[nodesepA=0pt,nodesepB=0pt,offsetA=0pt,offsetB=0pt,angleA=\theEangleComplementary, angleB=\theEangleEndCorrected, ncurv=1.359, linestyle=dashed]{->}{xxx}{#2}
\nccurve[nodesepA=0pt,nodesepB=2pt,offsetA=0pt,offsetB=0pt,angleA=\theEangleComplementary, angleB=\theEangleEndCorrected, ncurv=1.359]{-}{midpoint}{#2}
% 1.359 is e/2 happenchance or algorithmically necessary???
% now draw arrowhead -- dont include in the nccurve because this alters the line position - a strange feature of pstruicks
\ncput[npos=0.9]{\pnode(0,0){yyy}}
\ncline{->}{yyy}{#2}
% last so that to provide context for label that might follow
\nccurve[nodesepA=1pt,nodesepB=0pt,offsetA=0pt,offsetB=0pt,angleA=\theEangleStartCorrected, angleB=\theEangleGiven, ncurv=1.359, linecolor=black, linestyle=dashed]{-}{#2}{midpoint} 
}

% rEt - recursive Edge total
\newcommand{\rEt}[2][0]{
\setcounter{EangleGiven}{#1}
\setcounter{EangleStartCorrected}{#1-10} %correction required because for nccurve unlike nccircle angle measured at boundary not at centre of node
\setcounter{EangleEndCorrected}{#1+180+10} %correction required because angle measured at boundary not at centre of node
\setcounter{EangleComplementary}{#1 + 180}
\nccircle[angleA=\theEangleComplementary, nodesep=0pt, linestyle=none]{-}{#2}{.4cm} % an invisible circle to hang the midpoint from
\ncput{\pnode(0,0){midpoint}}                                         
\nccurve[nodesepA=1pt,nodesepB=0pt,offsetA=0pt,offsetB=0pt,angleA=\theEangleStartCorrected, angleB=\theEangleGiven, ncurv=1.359, linecolor=black]{-}{#2}{midpoint}
\ncput[nrot=:R,npos=0]{\psline(0,.1)(.075,0)}
\ncput[nrot=:R,npos=0]{\psline(0,.1)(-0.075,0)}
%\nccurve[nodesepA=0pt,nodesepB=0pt,offsetA=0pt,offsetB=0pt,angleA=\theEangleComplementary, angleB=\theEangleEndCorrected, ncurv=1.359, linestyle=dashed]{->}{xxx}{#2}
\nccurve[nodesepA=0pt,nodesepB=2pt,offsetA=0pt,offsetB=0pt,angleA=\theEangleComplementary, angleB=\theEangleEndCorrected, ncurv=1.359, linestyle=dashed]{-}{midpoint}{#2}
% 1.359 is e/2 happenchance or algorithmically necessary???
% now draw arrowhead -- dont include in the nccurve because this alters the line position - a strange feature of pstruicks
\ncput[npos=0.9]{\pnode(0,0){yyy}}
\ncline{->}{yyy}{#2}
% repeat from earlier to provide context for label that might follow
\nccurve[nodesepA=1pt,nodesepB=0pt,offsetA=0pt,offsetB=0pt,angleA=\theEangleStartCorrected, angleB=\theEangleGiven, ncurv=1.359, linecolor=black]{-}{#2}{midpoint} 
}

%  rEtm  - recursive Edge total mono
\newcommand{\rEtm}[2][0]{
\setcounter{EangleGiven}{#1}
\setcounter{EangleStartCorrected}{#1-10} %correction required because for nccurve unlike nccircle angle measured at boundary not at centre of node
\setcounter{EangleEndCorrected}{#1+180+10} %correction required because angle measured at boundary not at centre of node
\setcounter{EangleComplementary}{#1 + 180}
\nccircle[angleA=\theEangleComplementary, nodesep=0pt, linestyle=none]{-}{#2}{.4cm} % an invisible circle to hang the midpoint from
\ncput{\pnode(0,0){midpoint}}     
\nccurve[nodesepA=0pt,nodesepB=2pt,offsetA=0pt,offsetB=0pt,angleA=\theEangleComplementary, angleB=\theEangleEndCorrected, ncurv=1.359, linestyle=dashed]{-}{midpoint}{#2}
% 1.359 is e/2 happenchance or algorithmically necessary???
% now draw arrowhead -- dont include in the nccurve because this alters the line position - a strange feature of pstruicks
\ncput[npos=0.9]{\pnode(0,0){yyy}}
\ncline{->}{yyy}{#2}
% last to provide context for label that might follow
\nccurve[nodesepA=1pt,nodesepB=0pt,offsetA=0pt,offsetB=0pt,angleA=\theEangleStartCorrected, angleB=\theEangleGiven, ncurv=1.359, linecolor=black]{-}{#2}{midpoint} 
}

%  rEte  - total epi
\newcommand{\rEte}[2][0]{
\setcounter{EangleGiven}{#1}
\setcounter{EangleStartCorrected}{#1-10} %correction required because for nccurve unlike nccircle angle measured at boundary not at centre of node
\setcounter{EangleEndCorrected}{#1+180+10} %correction required because angle measured at boundary not at centre of node
\setcounter{EangleComplementary}{#1 + 180}
\nccircle[angleA=\theEangleComplementary, nodesep=0pt, linestyle=none]{-}{#2}{.4cm} % an invisible circle to hang the midpoint from
\ncput{\pnode(0,0){midpoint}}                                         
\nccurve[nodesepA=1pt,nodesepB=0pt,offsetA=0pt,offsetB=0pt,angleA=\theEangleStartCorrected, angleB=\theEangleGiven, ncurv=1.359, linecolor=black]{-}{#2}{midpoint}
\ncput[nrot=:R,npos=0]{\psline(0,.1)(.075,0)}
\ncput[nrot=:R,npos=0]{\psline(0,.1)(-0.075,0)}
%\nccurve[nodesepA=0pt,nodesepB=0pt,offsetA=0pt,offsetB=0pt,angleA=\theEangleComplementary, angleB=\theEangleEndCorrected, ncurv=1.359, linestyle=dashed]{->}{xxx}{#2}
\nccurve[nodesepA=0pt,nodesepB=2pt,offsetA=0pt,offsetB=0pt,angleA=\theEangleComplementary, angleB=\theEangleEndCorrected, ncurv=1.359]{-}{midpoint}{#2}
% 1.359 is e/2 happenchance or algorithmically necessary???
% now draw arrowhead -- dont include in the nccurve because this alters the line position - a strange feature of pstruicks
\ncput[npos=0.9]{\pnode(0,0){yyy}}
\ncline{->}{yyy}{#2}
% repeat from earlier to provide context for label that might follow
\nccurve[nodesepA=1pt,nodesepB=0pt,offsetA=0pt,offsetB=0pt,angleA=\theEangleStartCorrected, angleB=\theEangleGiven, ncurv=1.359, linecolor=black]{-}{#2}{midpoint} 
}

%  rEtme - recursive Edge total mono epi

\newcommand{\rEtme}[2][0]{
\setcounter{EangleGiven}{#1}
\setcounter{EangleStartCorrected}{#1-10} %correction required because for nccurve unlike nccircle angle measured at boundary not at centre of node
\setcounter{EangleEndCorrected}{#1+180+10} %correction required because angle measured at boundary not at centre of node
\setcounter{EangleComplementary}{#1 + 180}
\nccircle[angleA=\theEangleComplementary, nodesep=0pt, linestyle=none]{-}{#2}{.4cm} % an invisible circle to hang the midpoint from
\ncput{\pnode(0,0){midpoint}}     
\nccurve[nodesepA=0pt,nodesepB=2pt,offsetA=0pt,offsetB=0pt,angleA=\theEangleComplementary, angleB=\theEangleEndCorrected, ncurv=1.359]{-}{midpoint}{#2}
% 1.359 is e/2 happenchance or algorithmically necessary???
% now draw arrowhead -- dont include in the nccurve because this alters the line position - a strange feature of pstruicks
\ncput[npos=0.9]{\pnode(0,0){yyy}}
\ncline{->}{yyy}{#2}
% last to provide context for label that might follow
\nccurve[nodesepA=1pt,nodesepB=0pt,offsetA=0pt,offsetB=0pt,angleA=\theEangleStartCorrected, angleB=\theEangleGiven, ncurv=1.359, linecolor=black]{-}{#2}{midpoint} 
}

%gats.macros.tex

\usepackage{environ}    % also used in ermacros % here used for \NewEnvrion

\newcommand{\gat}[1][U]{
\ensuremath{\mathcal{#1}}}  % used to hav a space in here
\newcommand{\gatw}[1][U]{\gat[#1]\ }  % use this if need trailing space
\newcommand{\ingat}[1][U]{in \gat[#1]}
\newcommand{\isagat}[1][U]{\gat[#1] is a g.a.t.}
\newcommand{\inagat}{in a g.a.t. }

% macro for a generic theory
%\newcommand{\theory}
%{\textit{U}}

\newcommand{\intheory}
{is a derived rule of \gat[U]}

% Macros for GAT rules

\newcommand{\isT}[1]
{#1\mbox{ is a type}}

\newcommand{\ofT}[2]
{#1 \in #2
}

% Macros for GAT rules   <!-- new old -->
\newcommand{\istype}[1]
{#1\mbox{ is a type}}

\newcommand{\oftype}[2]
{#1 \in #2
}

%\context{x}{\Delta}{n}
\newcommand{\context}[3]
{\ofT{#1_1}{#2_1},... \ofT{#1_{#3}}{#2_{#3}(#1_1,...#1_{#3-1})}
}

%\subcontext{x}{\Delta}{i}{k}
\newcommand{\subcontext}[4]
{\ofT{#1_{#3_1}}{#2_{#3_1}},... \ofT{#1_{#3_#4}}{#2_{#3_#4}(#1_1,...#1_{#3_#4-1})}
}

% #schematic context
\newcommand{\schmcon}[3]
{\ofT{#1_1}{#2_1},... \ofT{#1_{#3}}{#2_{#3}}
}
% abbreviated to
\newcommand{\con}[3]
{\schmcon{#1}{#2}{#3}}

% schematic subcontext
%\subcon{x}{\Delta}{i}{k}
\newcommand{\subcon}[4]
{\ofT{#1_{#3_1}}{#2_{#3_1}},... \ofT{#1_{#3_#4}}{#2_{#3_#4}}
}

% permuted context
%\permcon{x}{\Delta}{n}{\sigma}
\newcommand{\permcon}[4]
{\ofT{#1_{#4(1)}}{#2_{#4(1)}},... \ofT{#1_{#4(#3)}}{#2_{#4(#3)}}
}
% permuted term
%\permterm{t}{n}{\sigma}
\newcommand{\permterm}[3]
{
#1_{#3(1)},...#1_{#3(#2)}
}


% Idioms
\newcommand{\xDelta}[1]{\con{x}{\Delta}{#1}}
\newcommand{\xDeltap}[1]{\con{x}{\Delta'}{#1}}
\newcommand{\xOmega}[1]{\con{x}{\Omega}{#1}}
\newcommand{\xOmegap}[1]{\con{x}{\Omega'}{#1}}
\newcommand{\yOmega}[1]{\con{y}{\Omega}{#1}}
\newcommand{\yOmegap}[1]{\con{y}{\Omega'}{#1}}

\newcommand{\xDeltasigma}[1]{\permcon{x}{\Delta}{#1}{\sigma}}
\newcommand{\xDeltapsigma}[1]{\permcon{x}{\Delta'}{#1}{\sigma}}
\newcommand{\xOmegasigma}[1]{\permcon{x}{\Omega}{#1}{\sigma}}
\newcommand{\xOmegapsigma}[1]{\permcon{x}{\Omega'}{#1}{\sigma}}
\newcommand{\yOmegasigma}[1]{\permcon{y}{\Omega}{#1}{\sigma}}
\newcommand{\yOmegapsigma}[1]{\permcon{y}{\Omega'}{#1}{\sigma}}

\newcommand{\xDeltainvsigma}[1]{\permcon{x}{\Delta}{#1}{\sigma^{-1}}}
\newcommand{\xDeltapinvsigma}[1]{\permcon{x}{\Delta'}{#1}{\sigma^{-1}}}
\newcommand{\xOmegainvsigma}[1]{\permcon{x}{\Omega}{#1}{\sigma^{-1}}}
\newcommand{\xOmegapinvsigma}[1]{\permcon{x}{\Omega'}{#1}{\sigma^{-1}}}
\newcommand{\yOmegainvsigma}[1]{\permcon{y}{\Omega}{#1}{\sigma^{-1}}}
\newcommand{\yOmegapinvsigma}[1]{\permcon{y}{\Omega'}{#1}{\sigma^{-1}}}

%Idioms enclosed as tuples
\newcommand{\encxDelta}[1]{\tuple{\con{x}{\Delta}{#1}}}
\newcommand{\encxDeltap}[1]{\tuple{\con{x}{\Delta'}{#1}}}
\newcommand{\encxOmega}[1]{\tuple{\con{x}{\Omega}{#1}}}
\newcommand{\encxOmegap}[1]{\tuple{\con{x}{\Omega'}{#1}}}
\newcommand{\encyOmega}[1]{\tuple{\con{y}{\Omega}{#1}}}
\newcommand{\encyOmegap}[1]{\tuple{\con{y}{\Omega'}{#1}}}

\newcommand{\encxDeltasigma}[1]{\tuple{\permcon{x}{\Delta}{#1}{\sigma}}}
\newcommand{\encxDeltapsigma}[1]{\tuple{\permcon{x}{\Delta'}{#1}{\sigma}}}
\newcommand{\encxOmegasigma}[1]{\tuple{\permcon{x}{\Omega}{#1}{\sigma}}}
\newcommand{\encxOmegapsigma}[1]{\tuple{\permcon{x}{\Omega'}{#1}{\sigma}}}
\newcommand{\encyOmegasigma}[1]{\tuple{\permcon{y}{\Omega}{#1}{\sigma}}}
\newcommand{\encyOmegapsigma}[1]{\tuple{\permcon{y}{\Omega'}{#1}{\sigma}}}

\newcommand{\encxDeltainvsigma}[1]{\tuple{\permcon{x}{\Delta}{#1}{\sigma^{-1}}}}
\newcommand{\encxDeltapinvsigma}[1]{\tuple{\permcon{x}{\Delta'}{#1}{\sigma^{-1}}}}
\newcommand{\encxOmegainvsigma}[1]{\tuple{\permcon{x}{\Omega}{#1}{\sigma^{-1}}}}
\newcommand{\encxOmegapinvsigma}[1]{\tuple{\permcon{x}{\Omega'}{#1}{\sigma^{-1}}}}
\newcommand{\encyOmegainvsigma}[1]{\tuple{\permcon{y}{\Omega}{#1}{\sigma^{-1}}}}
\newcommand{\encyOmegapinvsigma}[1]{\tuple{\permcon{y}{\Omega'}{#1}{\sigma^{-1}}}}

\newcommand{\tstyle}{\vdash}
\newcommand{\gatdisplayrule}[3][]
{
\setlength{\fboxsep}{1pt}       
\setlength{\fboxrule}{0pt}
\fbox{$\displaystyle \frac{#2}{#3\rule[-0.3cm]{0cm}{0cm}}$#1}    %added vertival space using \rule
}
\newcommand{\genericAintroductoryrule} {\gatdisplayrule{\xDelta{n}}{\isT{A(\xn)}}}
\newcommand{\genericfintroductoryrule}  {\gatdisplayrule{\xDelta{n}}{\ofT{f(\xn)}{\Delta}}}

% gat macros developed for cwf paper

% Expressing gats
\newenvironment{gatrules}
{
$$
\begin{array}{l l}
}
{
\end{array}
$$
}
\newcommand{\gatintros}
{
\textbf{Symbol} & \textbf{Introductory\ Rule}                      \\}

\newcommand{\gataxioms}
{\textbf{Axioms}\\}
\newcommand{\gatintro}[3]{\ #1 & #2 \tstyle #3 \\}
\newcommand{\gatlocalintro}[3]{\ #1 & #2 \dashv }
\newcommand{\gataxiom}[2]{\multicolumn{2}{l}{\ \ #1\mbox{,  whenever\ } #2} \\}
\newcommand{\noleft}{\left.\kern-\nulldelimiterspace} % so that no space taken by absent left brace


\newcommand{\gatmultiaxiom}[2]
{\multicolumn{2}{l}{
  \noleft
    \begin{array}{l}
		#1
    \end{array} 
  \right\} \mbox{whenever\ } 	#2 
	}\\}
	
	\newcommand{\axid}[1]{\text{#1}.\ }	

%New context sharing macros
\newcommand{\gatintroducing}[1]{
{\arraycolsep=0pt
  \begin{array}{l}
          #1
  \end{array}} &
}

%*********************************
% \begin{\gatgroup}{context}
%    rules
%  \end{\gatgroup}
%*********************************
\NewEnviron{gatgroup}[1]{%
  \noleft
  {\arraycolsep=0pt
   \begin{array}{l}
\BODY
    \end{array} 
   }
   \ \right\} 
	%\mbox{\ whenever\ } 
	#1
	\vspace{0.1cm} 
}
%*********************************

%*********************************
% \begin{\gatgroupnoshared}
%    rule
%  \end{\gatgroupnoshared}
%*********************************
\NewEnviron{gatgroupnoshared}{%
  {\arraycolsep=0pt
   \begin{array}{l}
\BODY
    \end{array} 
   }
   \ 
	\vspace{0.1cm} 
}
%*********************************

% \gatsingular[width]{context}{conclusion}
\newcommand{\gatsingular}[3][4cm]{
\begin{gatgroupnoshared}
\gatleaf[#1]{#2}{#3} 
\end{gatgroupnoshared}
}

%*********************************
% \gatleaf}[width]{context}{assertion}
%*********************************
\newcommand{\gatleaf}[3][4cm]{%
\makebox[#1]{$#3$ \dotfill} \dotfill \  #2
}
%*********************************
%*********************************
% \gatstandalonesingle}{context}{assertion}
%*********************************
\newcommand{\gatstandalonesingle}[2]{%
#2 \makebox[2.5cm]{\dotfill} \  #1
}
%*********************************

% \gataxiomno{axiomno}
\newcommand{\gataxiomno}[1]{\makebox[0.5cm]{} \axid{#1}}


% metagat.macros.tex

%Meta-theories

%\newcommand{\typ}{\triangleright}
\newcommand{\typ}{\nabla}
\newcommand{\trm}{\tau}
\newcommand{\cross}{\otimes}
\newcommand{\sub}{^*}
\newcommand{\diag}{\delta}

\newcommand{\typeseq}[2]
{\ofT{#1_1}{\typ},... \ofT{#1_{#2}}{\typ(#1_{#2-1})}}

\newcommand{\typeseqcont}[3]
{\ofT{#1_1}{\typ({#2})},... \ofT{#1_{#3}}{\typ(#1_{#3-1})}}

\newcommand{\Ob}{Ob}
\newcommand{\obj}{Ob} % <!-- new old --<
\newcommand{\Hom}{Hom}
\newcommand{\objseq}[2]
{\ofT{#1_1}{\obj},... \ofT{#1_{#2}}{\obj(#1_{#2-1})}}


\def\dottededge{\ncline[linestyle=dotted, nodesep=0.3cm]}
\def\noedge{\ncline[linestyle=none]}
\def\thinedge{\ncline[linewidth=0.4pt]}

\newcommand{\member}[1]
{\ncarc[arcangle=-30,nodesepB=0.03]{->}{\pspred}{\pssucc}
\nbput[labelsep=0.1]{#1}}

\newcommand{\loweraccutemember}[1]
{\ncarc[arcangle=-15,nodesepB=0.03]{->}{\pspred}{\pssucc}
\nbput[labelsep=0.05,npos=0.85]{#1}}

\newcommand{\uppermember}[1]
{\ncarc[arcangle=30,nodesepB=0.03]{->}{\pspred}{\pssucc}\naput{#1}}

\newcommand{\upperaccutemember}[1]
{\ncarc[arcangle=10,nodesepB=0.03]{->}{\pspred}{\pssucc}\naput[npos=0.85]{#1}}

% flexbranch 
% #1 node label
% #2 thislevelsep
% #3 next level sep
% #4 variable (eg x)
% #5 index leter (eg n)
% #6 close parenthesis
% #7 continuation branches
\newcommand{\flexbranch}[7]
{
\pstree[thislevelsep=*#2,nodesep=0.05]
		{\Rnode{#1 1}{\Tr{#4_1 #6}}}
	  {\pstree[thislevelsep=#3]  
				   {\Rnode{#1 2}{\Tr[edge=\dottededge]{#4_{#5} #6}}}
					 {#7}
		}
}

\newcommand{\flexbranchplusleaf}[6]
{
\flexbranch{#1}{#2}{#3}{#4} {#5} {#6}
  {
   %\Rnode{#1 3}{\Tr{#4 #6}}
	 \Tr{\Rnode{#1 3}{#4 #6}}
  }
}

\newcommand{\flexbranchplusarc}[7]
{
\flexbranch{#1}{#2}{#3}{#4} {#5} {#6}
  {
   %\Rnode{#1 3}{\Tr{#4 #6}\member{#7}}
	 \Tr{\Rnode{#1 3}{#4 #6}}\member{#7}
  }
}

\newcommand{\flexbranchinitialarc}[9]
{
\pstree[thislevelsep=*#2,nodesep=0.05]
		{\Rnode{#1 1}{\Tr{#4_#8 #6}}#9}
	  {\pstree[thislevelsep=#3]  
				   {\Rnode{#1 2}{\Tr[edge=\dottededge]{#4_{#5} #6}}}
					 {#7}
		}
}

\newcommand{\equality}[2]
{
\ncline [doubleline=true, nodesep=0.2cm]{#1}{#2}
}
\newcommand{\equalityarc}[2]
{
\ncarc [arcangleA=-30, arcangleB=-20, doubleline=true, nodesep=0.1cm]{#1}{#2}
}

%The following are stylistic so belong in main document not here.
%\usepackage[margin=4.0cm]{geometry} % This shouldn't be here commented out 17 July 2018
%\usepackage{mathptmx}               % This changes font to roman so doesn't belong here
%
\usepackage{amsfonts}
\usepackage{amssymb} % added 08\02\2019 as an experiment. Needed in some instances for \blacksquare
                     % not needed is class is `beamer' but I don't know why not
\usepackage{array}
\usepackage{pstricks}
\usepackage{pst-tree}
\usepackage{pst-plot}
\usepackage{pst-node}
\usepackage{stmaryrd}
\usepackage{amsmath}
\usepackage{verbatim}
\usepackage{graphicx}  
\usepackage{calc}
\usepackage{xifthen}
%\usepackage{xcolor} investigate with beamer
\usepackage{color}
\usepackage{stringstrings}
%\usepackage[small,bf,margin=3pt,format=hang, labelsep=endash,singlelinecheck=false]{caption} %prevuiously justification=justified
%\usepackage{enumerate}
%\usepackage{enumitem}
\usepackage{enumerate}
%\usepackage[shortlabels]{enumitem} %Removed this 28/01/2019 because interfereing with a beamer presentation. 
\usepackage{float}
\usepackage[section]{placeins}
%\setlength{\captionmargin}{5pt}
\usepackage{environ}
\usepackage{multirow}
\usepackage{rotating}
\usepackage{longtable}
\usepackage{afterpage}
\usepackage{needspace}


%DEFINE ENVIRONMENT BLOCK
% Riddle
\newsavebox{\riddlebox}

\newenvironment{erexample}
{\newcommand\colboxcolor{F0F0F0}%was F8F8F8
\begin{lrbox}{\riddlebox}
\begin{minipage}{\dimexpr\columnwidth-2\fboxsep\relax} \textbf{} \\ \itshape}
{\end{minipage}\end{lrbox}%
%\begin{center}
\colorbox[HTML]{\colboxcolor}{\usebox{\riddlebox}}
%\end{center}
}

\newenvironment{erbox}
{\newcommand\colboxcolor{F0F0F0}%was F8F8F8
\begin{lrbox}{\riddlebox}%
\begin{minipage}{\dimexpr\columnwidth-2\fboxsep\relax} }
{\end{minipage}\end{lrbox}%
%\begin{center}
\colorbox[HTML]{\colboxcolor}{\usebox{\riddlebox}}
%\end{center}
}

%\begin{erboxedFigure}{#1 FigureParam}{#2 Label}{#3 Caption}
\NewEnviron{erboxedFigure}[3]{%
\begin{figure}[#1]
\begin{erexample}
\begin{center}
\BODY
\end{center}
\vspace{-0.5cm}
\caption{#3}
\label{#2}
\end{erexample}
\end{figure}
}

\newcommand{\erpictureFolder}[0]{../SharedPictures}

\newcommand{\ercenterPicture}[1]{
\begin{center}
\input{\erpictureFolder/#1}
\end{center}
}


\newlength{\erhalfHt}

%\erinlinePicture{#1 pictureFilename}{#2 pictureHeight}
\newcommand{\erinlinePicture}[2]{
\setlength{\erhalfHt}{#2cm * \real{0.5}}
\raisebox{-\erhalfHt}[\erhalfHt + 0.5cm][\erhalfHt + 0.5cm]{
\input{\erpictureFolder/#1}
} 
}

%\erplainFig{#1 pictureFilename}{#2 figureParam}{#3Caption}
\newcommand{\erplainFig}[3]{
\begin{figure}[#2]
\begin{center}
\input{\erpictureFolder/#1}
\end{center}
\caption{#3}
\label{#1}
\end{figure}
}

%\erboxedFigPicture{#1 pictureFilename}{#2 figureParam}{#3Caption}
\newcommand{\erboxedFigPicture}[3]{
\begin{figure}[#2]
\begin{erexample}
\vspace{-0.5cm}
\begin{center}
\input{\erpictureFolder/#1}
\end{center}
\caption{#3}
\label{#1}
\end{erexample}
\end{figure}
}

%\erLeftSideFig{#1 pictureFilename}{#2 figureParam}{#3Caption}
\newcommand{\erLeftSideFig}[3]{
\begin{figure}[#2]
\begin{erexample}
  \begin{minipage}[c]{0.4\textwidth}
    \caption{#3}
    \label{#1}
  \end{minipage}
  \begin{minipage}[c]{0.5\textwidth}
    \input{\erpictureFolder/#1}
  \end{minipage}
\end{erexample}
\end{figure}
}

%\erbulletedFig{#1 pictureFilename}{#2 figureParam}{#3Caption}
\NewEnviron{erbulletedFig}[3]{%
\begin{figure}[#2]
\begin{erexample}
\vspace{-0.5cm}
\begin{center}
$
\begin{array}{c m{0.25cm} | m{6cm}}
\raisebox{-2.0cm}{
\input{\erpictureFolder/#1}}& & \text{\parbox{6cm}{\raggedright{\footnotesize{
\begin{enumerate}[(i)]
\BODY
\end{enumerate}}}}} \\
\end{array}
$
\end{center}
\caption{#3}
\label{#1}
\end{erexample}
\end{figure} 
}


%\begin{erbulletedDimFig}{#1 pictureFilename}{#2figureParam} {#3Caption} {#4PictureHeight}{#5TextWidth}

\NewEnviron{erbulletedDimFig}[5]{%
\begin{figure}[#2]
\begin{erexample}
\vspace{-0.5cm}
\begin{center}
$
\begin{array}{c m{0.25cm} |  m{#5cm}}
\setlength{\erhalfHt}{#4cm * \real{0.5}}
\raisebox{-\erhalfHt}{
\input{\erpictureFolder/#1}}& & \text{\parbox{#5cm}{\raggedright{\footnotesize{
\begin{enumerate}[(i)]
\BODY
\end{enumerate}}}}} \\
\end{array}
$
\end{center}
\caption{#3}
\label{#1}
\end{erexample}
\end{figure} 
}

%\begin{ernotedModel}{#1 pictureFilename}{#2PictureHeight}{#3PictureWidth}{#4TextWidth}

\NewEnviron{ernotedModel}[4]{%
\begin{center}
$
\begin{array}{m{#3cm} m{1cm} | c m{#4cm}}
\setlength{\erhalfHt}{#2cm * \real{0.5}}
\raisebox{-\erhalfHt}{
\input{\erpictureFolder/#1}}& & & \text{\parbox{#4cm}{\raggedright{\footnotesize{
\BODY
}}}} \\
\end{array}
$
\end{center} 
}

%\begin{ermodelText}{#1 pictureFilename}{#2PictureHeight}{#3PictureWidth}{#4TextWidth}

\NewEnviron{ermodelText}[4]{%
\begin{center}
\begin{tabular}{m{#3cm} m{1cm}  c m{#4cm}}
\setlength{\erhalfHt}{#2cm * \real{0.5}}
\raisebox{-\erhalfHt}{
\input{\erpictureFolder/#1}}& & & \text{\parbox{#4cm}{\raggedright{\small{
\BODY
}}}} \\
\end{tabular}
\end{center} 
}


%\erbulletedModel{#1 pictureFilename}{#2PictureHeight}{#3PictureWidth}{#4TextWidth}

\NewEnviron{erbulletedModel}[4]{%
\begin{center}
$
\begin{array}{m{#3cm} m{1cm} | c m{#4cm}}
\setlength{\erhalfHt}{2cm * \real{0.5}}
\raisebox{-\erhalfHt}{
\input{\erpictureFolder/#1}}& & & \text{\parbox{#4cm}{\raggedright{\footnotesize{
\begin{enumerate}[(i)]
\BODY
\end{enumerate}}}}} \\
\end{array}
$
\end{center} 
}



%\ernotedDimFig{#1 pictureFilename}{#2 figureParam}{#3Caption}{#4PictureHeight}{#5TextWidth}
\NewEnviron{ernotedDimFig}[5]{%
\begin{figure}[#2]
\begin{erexample}
\vspace{-0.5cm}
\begin{center}
$
\begin{array}{c m{0.25cm} | c m{#5cm}}
\setlength{\erhalfHt}{#4cm * \real{0.5}}
\raisebox{-\erhalfHt}{
\input{\erpictureFolder/#1}}& & & \text{\parbox{#5cm}{\raggedright{\footnotesize{
\BODY }}}}\\
\end{array}
$
\end{center}
\caption{#3}
\label{#1}
\end{erexample}
\end{figure} 
}
%\begin{ernotedDimFigPW}{#1 pictureFilename}{#2 figureParam}{#3Caption}{#4PictureHeight}{#5PictureWidth}{#6TextWidth}
\NewEnviron{ernotedDimFigPW}[6]{%
\begin{figure}[#2]
\begin{erexample}
\vspace{-0.5cm}
\begin{center}
$
\begin{array}{>{\centering}m{#5cm} m{0.5cm} | c m{#6cm}}
\setlength{\erhalfHt}{#4cm * \real{0.5}}
\raisebox{-\erhalfHt}{
\centering \input{\erpictureFolder/#1}
}& & & \text{\parbox{#6cm - 0.5cm}{\raggedright{\footnotesize{
\BODY }}}}\\
\end{array}
$ \\
\vspace {0.2cm}
\end{center}
\caption{#3}
\label{#1}
\end{erexample}
\end{figure}
}



\newenvironment{erquote}
{\begin{quote}\itshape}
{\end{quote}}


%
%  erdiagram.tex
%  *************
%  Macros to represent ER diagrams
%  *******************************
% 29/01/2019 Modify so that not reliant on the
%            default fontsize being 10pt by using
%            package anyfontsize and then
%            \fontsize{8}{10}\selectfont to set font to 8pt
% 06/02/2019 Pullback symbol implemented and minor tweaks to positioning 
%            and size of identifier symbol and relationship labels.
%            Accidental forked changes merged on 08/02/2019.
% 15/03/2019 Continuation of 29/01/2019. Need fix fontsize of 
%            ERrelname and ERscope.	 
% ***********************************************************
 \usepackage{anyfontsize}             % 29/01/2019 
  
%\begin{erdiagram}{#1 height}{#2 width} 
% ....
% ....
%\end{erdiagram}
\newenvironment{erdiagram}[2]
{%\pspicture*(-#1,0)(#2,0)
\pspicture*(0,-#1)(#2,0)
%\psgrid
}
{\endpspicture}

\definecolor{lightyellow}{cmyk}{0,0,0.3,0}
\definecolor{verylightgrey}{gray}{0.95}


% \eret{#1 x0} {#2 y0} {#3 x1} {#4 y1} {#5 corner radius} {#6 fill}
\newcommand {\eret}[6]
{ 
\ifthenelse{\equal{#6}{1}}
{\psframe[framearc=#5,fillstyle=solid,fillcolor=lightyellow](#1,#2)(#3,#4)}
{\psframe[framearc=#5,fillstyle=solid,fillcolor=white](#1,#2)(#3,#4)}
}

% et top 
\newcommand {\erettop}[4]
{
%\psframe[linestyle=none,linearc=2pt,cornersize=absolute,fillstyle=solid,fillcolor=lightyellow](#1,#2)(#3,#4)
\psline[linearc=2pt,fillstyle=none,fillcolor=lightyellow](#1,#4)(#1,#2)(#3,#2)(#3,#4)
}

% et bottom 
\newcommand {\eretbtm}[4]
{
%\psframe[linestyle=none,linearc=2pt,cornersize=absolute,fillstyle=solid,fillcolor=lightyellow](#1,#2)(#3,#4)
\psline[linearc=2pt,fillstyle=none,fillcolor=lightyellow](#1,#2)(#1,#4)(#3,#4)(#3,#2)
}

% et bottom left
\newcommand {\eretbl}[4]
{
%\psframe[linestyle=none,linearc=2pt,cornersize=absolute,fillstyle=solid,fillcolor=lightyellow](#1,#2)(#3,#4)
\psline[linearc=2pt,fillstyle=none,fillcolor=lightyellow](#1,#4)(#3,#4)(#3,#2)
}

% et middle left
\newcommand {\eretml}[4]
{
%\psframe[linestyle=none,linearc=2pt,cornersize=absolute,fillstyle=solid,fillcolor=lightyellow](#1,#2)(#3,#4)
\psline[linearc=2pt,fillstyle=none,fillcolor=lightyellow](#1,#2)(#3,#2)(#3,#4)(#1,#4)
}

% et top left
\newcommand {\erettl}[4]
{
%\psframe[linestyle=none,linearc=2pt,cornersize=absolute,fillstyle=solid,fillcolor=lightyellow](#1,#2)(#3,#4)
\psline[linearc=2pt,fillstyle=none,fillcolor=lightyellow](#1,#2)(#3,#2)(#3,#4)
}

% et bottom right
\newcommand {\eretbr}[4]
{
%\psframe[linestyle=none,linearc=2pt,cornersize=absolute,fillstyle=solid,fillcolor=lightyellow](#1,#2)(#3,#4)
\psline[linearc=2pt,fillstyle=none,fillcolor=lightyellow](#1,#2)(#1,#4)(#3,#4)
}

% et middle right
\newcommand {\eretmr}[4]
{
%\psframe[linestyle=none,linearc=2pt,cornersize=absolute,fillstyle=solid,fillcolor=lightyellow](#1,#2)(#3,#4)
\psline[linearc=2pt,fillstyle=none,fillcolor=lightyellow](#3,#4)(#1,#4)(#1,#2)(#3,#2)
}

% et top right
\newcommand {\erettr}[4]
{
\psline[linearc=2pt,fillstyle=none,fillcolor=lightyellow](#1,#4)(#1,#2)(#3,#2)
}

% \ergrp{#1 x0} {#2 y0} {#3 x1} {#4 y1} {#5 corner radius} {#6 fill}
% #5 corner radius is unused!
\newcommand {\ergrp}[6]
{ 
\ifthenelse{\equal{#6}{1}}
{\psframe[fillstyle=solid,fillcolor=verylightgrey](#1,#2)(#3,#4)}
{\psframe[fillstyle=solid,fillcolor=white](#1,#2)(#3,#4)}
}


% \ertext{#1 text}
% 15/03/2019
\newcommand {\erextrasmallitalictext}[1]
{\fontsize{7}{9}\selectfont \textit{#1}}

% 29/01/2019  
\newcommand {\ersmallitalictext}[1]
{\fontsize{8}{10}\selectfont \textit{#1}}

\newcommand {\ermediumitalictext}[1]
{\fontsize{10}{12}\selectfont \textit{#1}}

% \eretname {#1 x left of text} {#2 y top of text} {#3 text}
\newcommand {\olderetname}[3]
{
%shift down 0.1 for height of text the anchor at baseline (B)
\rput[bl]{0}(0,-0.1){\rput[Bl]{0}(#1,#2){\ersmallitalictext{#3}}}
}

% \errelarm {#1 x0} {#2 y0} {#3 x1} {#4 y1} {#5 ismandatory} {#6 isconstructed}
\newcommand {\errelarm}[6]
{
\ifthenelse{\equal{#6}{1}}
{
%%\psline[linewidth=0.5pt,linearc=.05,linestyle=dashed,dash=6pt 6pt]{-}(#1,#2)(#3,#4)}
\ifthenelse{\equal{#5}{1}}
{\psline[linewidth=1.5pt,linearc=.05,linecolor=lightgray]{-}(#1,#2)(#3,#4)}
{\psline[linewidth=1.5pt,linearc=.05,linecolor=lightgray,linestyle=dashed,dash=2pt 2pt]{-}(#1,#2)(#3,#4)}
}
{
\ifthenelse{\equal{#5}{1}}
{\psline[linewidth=0.9pt,linearc=.05]{-}(#1,#2)(#3,#4)}
{\psline[linewidth=0.9pt,linearc=.05,linestyle=dashed,dash=2pt 2pt]{-}(#1,#2)(#3,#4)}
}
}

% \errelangle {#1 x0} {#2 y0} {#3 x1} {#4 y1} {#5 x2} {#6 y2} {#7 ismandatory} {#8 isocnstructed}
\newcommand {\errelangle}[8]
{
\ifthenelse{\equal{#8}{1}}
{
%\psline[linewidth=0.5pt,linearc=.1,linestyle=dashed,dash=6pt 6pt]{-}(#1,#2)(#3,#4)(#5,#6)}
\ifthenelse{\equal{#7}{1}}
{\psline[linewidth=1.5pt,linearc=.05,linecolor=lightgray]{-}(#1,#2)(#3,#4)(#5,#6)}
{\psline[linewidth=1.5pt,linearc=.1,linecolor=lightgray,linestyle=dashed,dash=2pt 2pt]{-}(#1,#2)(#3,#4)(#5,#6)}
}
{
\ifthenelse{\equal{#7}{1}}
{\psline[linewidth=0.9pt,linearc=.1]{-}(#1,#2)(#3,#4)(#5,#6)}
{\psline[linewidth=0.9pt,linearc=.1,linestyle=dashed,dash=2pt 2pt]{-}(#1,#2)(#3,#4)(#5,#6)}
}
}

% \ercrowfoot {#1 x0} {#2 y0} {#3 x11} {#4 y11} {#5 x12} {#6 y12} {#7 x13} {#8 y13} {#9 isconstructed}
\newcommand {\ercrowfoot}[9]
{
\ifthenelse{\equal{#9}{1}}
{
\psline[linewidth=1.5pt,linearc=.05,linecolor=lightgray]{-}(#1,#2)(#3,#4)
\psline[linewidth=1.5pt,linearc=.05,linecolor=lightgray]{-}(#1,#2)(#5,#6)
\psline[linewidth=1.5pt,linearc=.05,linecolor=lightgray]{-}(#1,#2)(#7,#8)
}{
\psline[linewidth=0.9pt,linearc=.05]{-}(#1,#2)(#3,#4)
\psline[linewidth=0.9pt,linearc=.05]{-}(#1,#2)(#5,#6)
\psline[linewidth=0.9pt,linearc=.05]{-}(#1,#2)(#7,#8)
}
}


% \eridcomprel{#1 x1}{#2 x2}{#3 y}
\newcommand {\eridcomprel}[3]
{
\psline[linewidth=0.9pt](#1,#3)(#2,#3)
}

% \eridrefrel{#1 x}{#2 y1}{#3 y2}
\newcommand {\eridrefrel}[3]
{
\psline[linewidth=0.9pt](#1,#2)(#1,#3)
}

% \ertext {#1 x} {#2 y} {#3 text anchor} {#4 text}  PRIVATE
\newcommand {\ertext}[4]
{
\rput[B#3]{0}(#1,#2){\fontsize{8}{10}\selectfont #4}
}

% \eretname {#1 x} {#2 y} {#3 text anchor} {#4 text} 
\newcommand {\eretname}[4]
{
\ertext{#1}{#2}{#3}{#4}
}

% \errelname {#1 x} {#2 y} {#3 text anchor} {#4 text} 
\newcommand {\errelname}[4]
{
\rput[B#3]{0}(#1,#2){\erextrasmallitalictext{#4}}
}


% \erscope {#1 x} {#2 y} {#3 text anchor} {#4 text}  15 March 2019
\newcommand {\erscope}[4]
{
\rput[B#3]{0}(#1,#2){\erextrasmallitalictext{#4}}
}

% \erreletname {#1 x} {#2 y} {#3 text anchor} {#4 text}  15 March 2019
\newcommand {\erreletname}[4]
{
\rput[B#3]{0}(#1,#2){\fontsize{10}{12}\selectfont #4}
}

% \ergroupannotation {#1 x} {#2 y} {#3 text anchor} {#4 text}
\newcommand {\ergroupannotation}[4]
{
\ertext{#1}{#2}{#3}{#4}
}


% \errelseq {#1 x} {#2 y}
\newcommand {\erelseq}[2]
{
}
\newcommand {\erattrmarkermand}
{\fontsize{6}{8}\selectfont $\blacksquare$}
\newcommand {\erattrmarkeropt}
{\fontsize{6}{8}\selectfont \CIRCLE}
\newcommand {\erderattrmarkermand}
{\fontsize{6}{8}\selectfont $\square$}
\newcommand {\erderattrmarkeropt}
{\fontsize{8}{10}\selectfont $\circ$}

% \erattr {#1 x} {#2 y} {#3 ismandatory}{#4 idenitfying} {#5 text}
\newcommand {\erattr}[5]
{
\ifthenelse{\equal{#3}{1}}
{\rput[l]{0}(#1,#2){\erattrmarkermand \ersmallitalictext{\ifthenelse{\equal{#4}{0}}{\underline{#5}}{#5}}}}
{\rput[l]{0}(#1,#2){\erattrmarkeropt \ersmallitalictext{\ifthenelse{\equal{#4}{0}}{\underline{#5}}{#5}}}}
}

\newcommand {\erdattr}[5]
{
\ifthenelse{\equal{#3}{1}}
{\rput[l]{0}(#1,#2){\erderattrmarkermand \ersmallitalictext{\ifthenelse{\equal{#4}{0}}{\underline{#5}}{#5}}}}
{\rput[l]{0}(#1,#2){\erderattrmarkeropt \ersmallitalictext{\ifthenelse{\equal{#4}{0}}{\underline{#5}}{#5}}}}
}


% \erarc {#1 x0} {#2 y0} {#3 x1} {#4 y1} {#5 x2} {#6 y2} {#7 x3} {#8 y3}
\newcommand {\erarc}[8]
{
\psbezier[showpoints=false]{-}(#1,#2) (#3, #4)(#5,#6) (#7, #8)
}

% \erarc {#1 x0} {#2 y0} {#3 x1} {#4 y1} {#5 x2} {#6 y2} {#7 x3} {#8 y3}
\newcommand {\errelseq}[8]
{
\psbezier[showpoints=false]{-}(#1,#2) (#3, #4)(#5,#6) (#7, #8)
}
% \ertrace {#1 trace}   
\newcommand {\ertrace}[1]
{
}
    %beamer aware version
%All these macros are copied from SharedMacros/general.tex which doesnt seem to work with beamer
% Some macros in SharedMacros/general.tex thought to have name clashes with beamer.


\newcommand{\fundep}[3]{#2 \xrightarrow{#1} #3}                                                 
\newcommand{\term}[1]{\textit{#1}} 
\newcommand{\setsuchthat}[2]{\left\{#1 \ \middle|\ #2\right\}}
\newcommand{\set}[1]{\left\{#1\right\}}



\newcommand{\wanton}[1]{#1_1,...#1_n}
\newcommand{\ntuple}[1]{\tuple{\wanton{#1}}}

\newcommand{\xntuple}{\ensuremath{\ntuple{x}}}

% maybe not in general.macros 
\newcommand{\xnset}{\ensuremath{\set{\wanton{x}}}}


%
% othermacros
%

%copied from database literature review
\newcommand{\displaybibentry}[1]
{\begin{framed}
\bibentry{#1}
\end{framed}
}

% used in data tables
\newcommand{\colhead}[1]{\textbf{\textcolor{white}{#1}}}
\definecolor{myblue}{RGB}{71,71,186}
% \vpad gives vertical padding in a tabular
\newcommand{\vpad}[1]{\multicolumn{#1}{c}{}\\[-0.25cm]}

\newcommand{\catMEterm}{category with designated monomorphisms and epimorphisms\ }
\newcommand{\IfSforCwithRCwords}{
If $S$ is a sketch for category \catcw considered as a data specification with requirement $\reqtc$\ }
\newcommand{\IfSforCwithRCwordsvariant}{
If $S$ is a sketch for structured category \catcw and if $S$ is considered as a data specification with requirement $\reqtc$\ }
\newcommand{\IfSforepimonoCwithRCwords}{
If $S$ is a sketch for a category \catcw with designated monomorphisms and epimorphisms considered as a data specification with requirement $\reqtc$\ }
\iffalse
\newcommand{\scmonosketchwording}{
If $S$ is a sketch for such a category
%of a category with finite products and designated monomorphisms and epimorphisms
considered as a data specification
with requirement $\reqtc$\ }
\fi
\newcommand{\IfSforproductepimonoCwithRCwords}{
If $S$ is a sketch for such a category
%category \catcw with finite products and designated monomorphisms and epimorphisms 
considered as a data specification with requirement $\reqtc$\ }

\newcommand{\goodnesscriteria}[1]{\textbf{Goodness Criteria #1:}}


% From the Mathematical Theory of data paper
\newcommand{\ssfd}[2]{\ensuremath{#1 \morph #2}}  % singleton-singleton
\newcommand{\smfd}[2]{\ensuremath{\ssfd{#1}{\set{#2}}}}  % singleton-many
\newcommand{\msfd}[2]{\ensuremath{\ssfd{\set{#1}}{#2}}}  % many-singleton
\newcommand{\mmfd}[2]{\ensuremath{\msfd{#1}{\set{#2}}}}  % many-many



% All these should find a home in SharedMacros eventually 

% Commands for making a bit of vertical space. used when arrows and particularly labels of arrows
% use spec that is otherwise accounted for.
\newcommand{\seeroomup}[1]{\rule{0.1cm}{#1}}
\newcommand{\seeroomdown}[1]{\rule[-#1]{0.1cm}{0.1cm}}
\newcommand{\roomup}[1]{\rule{0cm}{#1}}
\newcommand{\roomdown}[1]{\rule[-#1]{0cm}{0.1cm}}



% DIAGRAMS START HERE
\newcommand{\nakedbinarysourcediagram}[5]{
\begin{array}{c p{0.5cm} c}
             &&   \Rnode{b}{#2}\\[0.01cm]
\Rnode{a}{#1} &&               \\[0.01cm] 
             &&   \Rnode{c}{#3}
\end{array} 
\begin{arrows}
\ncarr{a}{b}
\alabel{#4}
\ncarr{a}{c}
\blabel{#5}
\end{arrows}
}

\newcommand{\binarysourcediagram}[5]{$\nakedbinarysourcediagram{#1}{#2}{#3}{#4}{#5}$}
\newcommand{\fgsourcediagram}{\binarysourcediagram{a}{b}{c}{f}{g}}


\newcommand{\unaryfdrepresentationdiagram}[8]{
$
\begin{array}{c p{0.2cm} c}
\nakedbinarysourcediagram{#1}{#2}{#3}{#4}{#5}&& \Rnode{d}{#6}
\end{array}
\begin{arrows}
\ncarr{d}{b}
\idcomp
\blabel{#7}
\ncarr{d}{c}
\alabel{#8}
\end{arrows}
$
}

\newcommand{\unaryfdrepresentationmappeddiagram}[8]{
$
\begin{array}{c p{0.2cm} c}
\nakedbinarysourcediagram{D(#1)}{D(#2)}{D(#3)}{D(#4)}{D(#5)}&& \Rnode{d}{D(#6)}
\end{array}
\begin{arrows}
\ncarr{b}{d}
\alabel{D(#7)^-1}
\ncarr{d}{c}
\alabel{D(#8)}
\end{arrows}
$
}

\newcommand{\commutativetrianglediagram}[6]{
$
\begin{array}{c p{0.4cm} c p{0.4cm} c}
              && \Rnode{b}{#2}  &&                 \\[0.6cm]
\Rnode{a}{#1} &&                && \Rnode{c}{#3}  
\end{array}
\begin{arrows}
\ncarr{a}{b}
\alabel{#4}
\ncarr{b}{c}
\alabel{#5}
\ncarr{a}{c}
\blabel{#6}
\end{arrows}
$
}

\newcommand{\commutativetrianglediagrammutant}[6]{
$
\begin{array}{c  c  c}
              & \Rnode{b}{#2}  &                 \\[0.85cm]
\Rnode{a}{#1} &                & \Rnode{c}{#3}  
\end{array}
\begin{arrows}
\ncarr{a}{b}
\alabel{#4}[0.15]
\ncarr{b}{c}
\alabel{#5}[0.6]
\ncarr{a}{c}
\blabel{#6}
\end{arrows}
$
}

\newcommand{\epimonosplitdiagram}[3]{
\commutativetrianglediagram{#1}{img(#3)}{#2}{#3_e}{#3_m}{#3}   
}


\iffalse %saved
\begin{array}{c p{2.0cm} c }                
               &&  \Rnode{b1}{#3_1}    \\ [0.75cm]
               &&  \Rnode{b2}{#3_2}    \\ [0.5cm]
\Rnode{a}{#2}  &&                      \\ [-0.5cm]
               &&       \vdots         \\ [0.85cm]
               &&  \Rnode{bn}{#3_{#1}}  
\end{array}
\fi

%nakedmultisourceobjects{n}{a}{b}{f}
\newcommand{\nakedmultisourceobjects}[4]{
\begin{array}{c p{2.0cm} c }
\Rnode{a}{#2}   &&
\begin{array}{c }                
\Rnode{b1}{#3_1}   \\ [0.75cm]
\Rnode{b2}{#3_2}   \\ [0.25cm]
\vdots             \\ [0.35cm]
\Rnode{bn}{#3_{#1}}  
\end{array}
\end{array}
}

% \nakedmultisourcediagram{n}{a}{b}{f}
\newcommand{\nakedmultisourcediagram}[4]{
\nakedmultisourceobjects{#1}{#2}{#3}{#4}
\begin{arrows}
\ncarr{a}{b1}
\alabel{#4_1}[0.5]
\ncarr{a}{b2}
\alabel{#4_2}[0.5][-1]
\ncarr{a}{bn}
\blabel{#4_{#1}}[0.5][-1]
\end{arrows}
}

% \nakedmultisourcepathdiagram{n}{a}{b}{f}
\newcommand{\nakedmultisourcepathdiagram}[4]{
\nakedmultisourceobjects{#1}{#2}{#3}{#4}
\begin{arrows}
\simplepath{a}{b1}
\alabel{#4_1}[0.5]
\simplepath{a}{b2}
\alabel{#4_2}[0.5][-1]
\simplepath{a}{bn}
\blabel{#4_{#1}}[0.5][-1]
\end{arrows}
}


\newcommand{\multisourcediagram}[4]{$\nakedmultisourcediagram{#1}{#2}{#3}{#4}$}
\newcommand{\multisourcepathdiagram}[4]{$\nakedmultisourcepathdiagram{#1}{#2}{#3}{#4}$}



% \multisourcedefinitiondiagram{x}{g}{h}{n}{a}{b}{f}
\newcommand{\monosourcedefinitiondiagram}[7]{
$
\begin{array}{c p{1.5cm} c}
\Rnode{x}{#1} && \nakedmultisourcediagram{#4}{#5}{#6}{#7}
\end{array}
\begin{arrows}
\parallelarrows{x}{a}{#2}{#3}
\end{arrows}
$
}

\newcommand{\fghfactordiagram}[6]
{
\binarysourcediagram{#1}{#2\roomup{0.5cm}}{#3}{#4}{#5}
\begin{arrows}
\ncarr{b}{c}
\alabel{#6}
\end{arrows}
}

\newcommand{\fghpartialfactordiagram}[6]{
\binarysourcediagram{#1}{#2\roomup{0.5cm}}{#3}{#4}{#5}
\begin{arrows}
\ncdarr{b}{c} %dashed arrow
\alabel{#6}
\end{arrows}
}

\newcommand{\fnsourceqnsource}{
$
\begin{array}{c p{0.25cm} c  p{0.25cm} c }
             &&   \Rnode{b1}{b_1} &&              \\[0.4cm]
\Rnode{a}{a} &&                   && \Rnode{c}{c} \\[0.4cm]
             &&   \Rnode{bn}{b_n} &&              
\end{array} 
\begin{arrows}
\ncarr{a}{b1}
\alabel{f_1}
\ncarr{c}{b1}
\blabel{q_1} 
\ncarr{a}{bn}
\blabel{f_n}
\ncarr{c}{bn}
\alabel{q_n}
\end{arrows}
$   
}

\newcommand{\parallelarrows}[4]{
\ncarc[nodesep=2pt,arcangle=10,offset=2pt]{->}{#1}{#2}
\alabel{#3}
\ncarc[nodesep=2pt,arcangle=-10,offset=-2pt]{->}{#1}{#2}
\blabel{#4}
}

\newcommand{\fgparalleldiagram}{
 $
\rule[-0.3cm]{0pt}{0.9cm} %to add vertical space of diagram -- based on lowering diagram 0.3cm and heght 0.9cm
                            % change thickness from 0pt to 1 pt to debug
\begin{array}{c p{0.5cm} c  }
 \Rnode{a}{a}            &&   \Rnode{b}{b}
\end{array} 
\begin{arrows}
\iffalse
\ncarc[nodesep=2pt,arcangle=10,offset=2pt]{->}{a}{b}
\alabel{f}
\ncarc[nodesep=2pt,arcangle=-10,offset=-2pt]{->}{a}{b}
\blabel{g}
\fi
\parallelarrows{a}{b}{f}{g}
\end{arrows}
$  
}

\newcommand{\fgcomposablediagram}[5]{
\mbox{
\roomup{0.45cm}
$
\begin{array}{c p{0.5cm}cp{0.5cm}c}
\Rnode{x}{#1}&&\Rnode{y}{#2}&&\Rnode{z}{#3}
\end{array}
\begin{arrows}
\ncarr{x}{y}
\alabel{#4}
\ncarr{y}{z}
\alabel{#5}
\end{arrows}
$    
}
}



\renewcommand{\erpictureFolder}[0]{../../SharedPictures}


\usetheme{Szeged}
\usecolortheme{dolphin}

%\setbeamertemplate{navigation symbols}{}

\setcounter{equation}{0}

\title[John Cartmell]{Mathematical Theory of Data}
%% Which is to say types as they are used in practice in software development and as represented in theory in categories and in syntactic type theories.
%% There is also a subplot concerning representation of context which certain types depend on -- again represented in practice and in theory. 
\author{John Cartmell}
\institute{ad otium}
\date{Jan 25, 2019}
\bibliographystyle{plainnat}
\usepackage{bibentry}
\nobibliography*




\begin{document}


\begin{frame}
\titlepage
\end{frame}

\section{\MToDsection}
\subsection{\MToDsubsectionCategorieswithEpiMonos}
\begin{frame}{Definition}
A \term{category with epi-mono splits} is a category such that for all $f:x \morph y$
in \catcw there exists an object $img(f)$, an epimorphism $f_e:x \morph img(f)$
and a monomorphism $f_m:img(f) \morph y$ such that
\scalebox{0.9}{\roomup{1.3cm}\epimonosplitdiagram{x}{y}{f}}
commutes. 

An \term{instance} of a category with epi-mono splits \catcw is a functor from 
\catcw to the category $\Fin$ of finite sets and functions.
\end{frame}

\begin{frame}{Representation of FDs}
\begin{definition}
If $\catc$ is a category and $\reqtc$ is a set of instances, 
if \fgsourcediagram in $\catc$ 
and if there is a functional dependency $\fundep{H}{f}{g}$ then say that 
this functional dependency  is \term{represented} in $\catc$ 
iff there exists an object $d$ of \catc, a monomorphism $q:d \morph b$ and a morphism $w: d \morph c$
such that \scalebox{0.9}{\unaryfdrepresentationdiagram{a}{b}{c}{f}{g}{d}{q}{w}} commutes.
\end{definition}
\end{frame}

\begin{frame}{Representation Lemma for FDs}
{\small Note: I can prove this only in the case that \catcw is locally finite.

No longer restricted to the case that each $D(f)$ is surjective.}

\begin{lemma}
If $\catc$ is a locally finite category with epi-mono splits that is 
\textit{maximally constrained} to a requirement $\reqtc$ then
if $\fundep{H}{f}{g}$  is a functional dependency with respect to $\reqtc$
then $\fundep{H}{f}{g}$ is represented in $\catc$.
\end{lemma}
\begin{proof}
Because 
\scalebox{0.9}{\unaryfdrepresentationdiagram{\roomup{0.6cm}\roomdown{0.4cm}a}{b}{c}{f}{g}{im(f)}{f_m}{w}$\begin{arrows}\ncarr{a}{d}\alabel{f_e}\end{arrows}$} commutes in \catcw  where $w$ represents 
the functional dependency of $g$ on $f_e$ whose existence is given by previous version of this lemma.
\end{proof}
\end{frame}

\begin{frame}{Referential Inclusion Dependencies}
\begin{definition}
If $\catc$ is a category with epi mono splits, if $\reqtc$ is a set of instances 
and if \fnsourceqnsource in $\catc$ and  $\set{q_1,...q_n}$ is a mono-source
 then a \term{referential inclusion dependency} $I$, written $a[f_1,...f_n] \overset{I}{\subseteq} c[q_1,..q_n]$, 
 is a family of functions $I_D)_{D \in \reqtc}$
such that for each instance $D \in \reqtc$, $I_D$ is a function $I_D : D(a) \morph D(c)$ and
for each $i$, $1 \leq i \le n$, $I_D \circ D(q_i) = D(f_i)$.
\end{definition}
\end{frame}


\begin{frame}{Representation of Referential Inclusion Dependencies}
\begin{definition}
If $\catc$ is a category with epi mono splits 
and if $\reqtc$ is a set of instances
and \fnsourceqnsource in $\catc$ 
and if $a[f_1,...f_n] \overset{I}{\subseteq} c[q_1,..q_n]$ is a referential inclusion dependency
with respect  to $\reqtc$ 
then say that the inclusion dependency $I$ is \term{represented} in $\catc$
iff there exists a morphism $i:a \morph c$ in $\catc$ such that in each instance $D \in \reqtc$, $D(i) = I_D$. 
\end{definition}
\end{frame}

\begin{frame}{Epi Mono Goodness Criteria}
\IfSforepimonoCwithRCwords then
\begin{itemize}
\item \goodnesscriteria{2D} the set of instances $\reqtc$ should be jointly reflective of monomorphisms,
\item \goodnesscriteria{2E} the set of instances $\reqtc$ should be jointly reflective of epimorphisms.
\end{itemize}
\end{frame}

\begin{frame}{Knowns and unknowns}
\IfSforepimonoCwithRCwords then
\begin{itemize}
\item Local finiteness and Criteria 2 (maximal constrainedness) implies Criteria 2D (jointly reflective of monomorphisms),
\item Local finiteness and Criteria 2 (maximal constrainedness) implies Criteria 2E (jointly reflective of epimorphisms).
\item Can local finiteness assumption be replaced by the finitary property?
\item Can reverse implications be made - With what additional assumptions do criteria 2A--2E imply maximal constrainedness? 
\end{itemize}
\end{frame}


\iffalse
\begin{frame}{Bibliography}
\bibliography{../../SharedBibliography/temp/bibliography}
\end{frame}
\fi

%\fi

\end{document}

\subsection{Categories with Finite Products and Designated Monomorphisms and Epimorphims}


\comingnext{Sketches of Categories with Designated Monomorphisms and Epimorphisms and Finite Products as Data Specifications}
\renewcommand{\slidecontext}{Sketches of Categories with Designated Mono's and Epi's and Finite Products as Data Specifications}

\begin{frame}{Composition of mono sources}
If $i$ and $j$ are mono sources and $f \in i$ so that
\begin{center}
\scalebox{0.65}{
\setlength{\arraycolsep}{.2cm}
$
\begin{array}{cp{1.5cm}ccp{1.5cm}ccp{1.25cm}c}
             & &         & \dotnode[dotsize=1pt]{b1} & &        &                              && \pnode{bracehigh}  \\ [0.3cm]
						 & &         & \dotnode[dotsize=1pt]{b2} & &        &                                \\ [0.3cm]
\Rnode{a}{a} & & \vdots  &                           & &        &                                \\ [0.02cm]
						 & &         &                           & &        & \dotnode[dotsize=1pt]{x1}      \\ [0.1cm]
             & &         & \Rnode{b}{b}              & & \vdots &                                \\ [0.1cm]
             & &         &                           & &        & \dotnode[dotsize=1pt]{xn}   && \pnode{bracelow}   \\ [0.5cm]
\psbrace[rot=90, nodesepA=-2pt, nodesepB=10pt, braceWidth=1pt, braceWidthInner=3pt](0,0.5)(2.7cm,0.5){i}	
	&  &         & 
\psbrace[rot=90, nodesepA=-2pt, nodesepB=10pt, braceWidth=1pt, braceWidthInner=3pt](0,0.5)(2.7cm,0.5){j} & & \\
\end{array}
$
%\psbrace[rot=0, nodesepA=10pt, braceWidth=1pt, braceWidthInner=3pt, ,ref=lC](bracelow)(bracehigh)
%{$(i \backslash \set{f}) \cup \setsuchthat{f \circ g}{g \in j}$}
\ncarr{b}{x1}
\ncarr{b}{xn}
\ncarr{a}{b1}
\ncarr{a}{b2}
\ncarr{a}{b}
\blabel{f}

}
\end{center}
then $(i \backslash \set{f}) \cup \setsuchthat{f \circ g}{g \in j}$ is a mono source.
\end{frame}

\begin{frame}{Sketches for categories}{with \thirdstructure }
By a sketch for a category with \thirdstructure I shall mean a quintuple
$\tuple{G,PE,M,P,E}$
\begin{itemize}
\item  where $G$ is a directed graph, 
\item  $PE$ is a set of path equivalences, 
\item  $M$ is a set of $G$-sources deemed to be mono-sources,
\item  $P$ is a set of $G$-sources deemed to be product diagrams, 
\item  and $E$ is a set of $G$-paths deemed to be epimorphisms.
\end{itemize}
\medskip
From such a sketch we can generate a category with \thirdstructure.  
\end{frame}

\begin{frame}{Goodness Criteria 1F}{\slidecontext}

In  a nutshell... "don't model products needlessly".\\

\pause In otherwords:\\
\medskip
\IfSforproductepimonoCwithRCwords then

\goodnesscriteria{1F} If $P$ contains the source \scalebox{0.65}{\multisourcediagram{n}{x}{y}{p}} 
then there ought to exist a node $z$ and an edge $f: x \morph z$ in $G$.
\end{frame}

\begin{frame}{Goodness Criteria 2B Revisited}{\slidecontext}
Next:\\
Restate how functional dependencies and their representations are defined in terms of sketches of categories with \thirdstructure.
\end{frame}

\begin{frame}{Functional Dependency Restated}{\slidecontext}
\IfSforproductepimonoCwithRCwords, 
if \scalebox{0.65}{\multisourcepathdiagram{n}{a}{b}{x}} is a path source in $S$ and if
$y: a \morph c$ is a path in $S$
then path $y$ is said to be \term{functionally dependent} on the set of paths $\set{x_1,...x_n}$ with respect to the requirement $\reqtc$
iff the function $D(y)$ factors through $D(\tuple{x_1,... x_n })$

\pause iff in each $D \in \reqtc$ there exists a  (unique)
function $H_D: img(D(\tuple{x_1,... x_n })) \morph D(c)$ \\
\hspace*{\fill}such that \kern14pt
\scalebox{0.65}{\commutativetrianglediagrammutant{D(a)}{img(D(\tuple{x_1,... x_n }))}{D(c)}{D(\xntuple)}{f_D}{D(y)}} commutes.
\end{frame}

\iffalse
\begin{frame} 
CORRECT THIS DIAGRAM
\scalebox{0.9}{ 
$
\begin{array}{cp{2cm}ccp{0.5cm}cc}
                & &         & \Rnode{Eb1}{D(b_1)}& &                            &        \\[0.6cm]
                & &         & \Rnode{Eb2}{D(b_2)}& &                            &        \\[0.6cm]
                & &\vdots   &                     & &                            &        \\[0.2cm]                                              
\Rnode{Ea}{D(a)} & &         & \Rnode{Ebn}{D(b_n)}& & \Rnode{Jnctn}{}            &        \\[1.0cm]
                & &         & \Rnode{Ec}{D(c)}   & &                            &  
\end{array}
\begin{arrows}
\simplepath{Ea}{Eb1}
\alabel{D(x_1)}
\simplepath{Ea}{Eb2}
\alabel{D(x_2)}
\simplepath{Ea}{Ebn}
\alabel{D(x_n)}
\simplepath{Ea}{Ec}
\blabel{D(y)}
\nchpmarr[15][45]{Eb1}{Ebn}{Jnctn}{Ec}
\naput[npos=-0.1]{$h_D$}
\ncarc[arcangle=15]{Eb2}{Jnctn}
%\ncline{h-}{Eb1}{Ebn}
\end{arrows}
$
}
\end{frame}
\fi

\begin{frame}{Notation}
That a path  $y$ is functionally dependent on the set of paths $\set{x_1,...x_n}$ 
is written  \msfd{x_1,...x_n}{y}.
\end{frame}

\begin{frame}{Meeting BCNF}{\slidecontext}
If $S$ is a sketch for a category $C$ with designated mono-sources, epi-mono factorisations  and finite products
and that $S$ is considered as a data specification with requirement $\reqtc$, then 
if $S$ is simple and \catcw is appropriately finite then $S$ as data specification
with requirement $\reqtc$ meets the abstract BCNF criteria:

\pause if  
\multisourcenplusonediagram{n}{a}{b}{x}{c}{y}
are edges of $S$
and if \msfd{x_1,...x_n}{y}  is a non-trivial functional dependency  between these edges then
$x_1,...x_n$ is a designated mono-source.
\end{frame}


\iffalse
\begin{frame}{Referential Inclusion Dependencies}{\slidecontext}
\begin{definition}
If $\catc$ is a category with \thirdstructure, if $\reqtc$ is a set of instances 
and if \fnsourceqnsource in $\catc$ and  $\set{q_1,...q_n}$ is a mono-source
 then a \term{referential inclusion dependency} $I$, 
 written $a[f_1,...f_n] \overset{I}{\subseteq} c[q_1,..q_n]$, 
 is a family of functions $I_D)_{D \in \reqtc}$
such that for each instance $D \in \reqtc$, $I_D$ is a function $I_D : D(a) \morph D(c)$ and
for each $i$, $1 \leq i \le n$, $I_D \circ D(q_i) = D(f_i)$.
\end{definition}
\end{frame}

\begin{frame}{Representation of Referential Inclusion Dependencies}{\slidecontext}
\begin{definition}
If $\catc$ is a category with \thirdstructure
and if $\reqtc$ is a set of instances
and \fnsourceqnsource in $\catc$ 
and if $a[f_1,...f_n] \overset{I}{\subseteq} c[q_1,..q_n]$ is a referential inclusion dependency
with respect  to $\reqtc$ 
then say that the inclusion dependency $I$ is \term{represented} in $\catc$
iff there exists a morphism $i:a \morph c$ in $\catc$ such that in each instance $D \in \reqtc$, $D(i) = I_D$. 
\end{definition}
\end{frame}
\fi

\fi

\subsection{Sketch Monics}


\newcommand{\CEsymboltype}[0]{varchar(2)}
\newcommand{\CEatomicnumbertype}{number(1,1000)}
\newcommand{\CEfloattype}{float}
\newcommand{\CEnametype}{varchar(64)}
\newcommand{\CEvalencynumbertype}{number(-7,7)}



\begin{frame}{Data Specification of the Chemical Elements as a Directed Graph}
\scalebox{0.65}{


$
\begin{array}{c c c p{4.5cm} l}
                        &                        &                            &&\attrtype{\Rnode{symboltype}{\CEsymboltype}  }       \\ [0.7cm]
                        &\etype{\Rnode{element}{chemical\ element}\Rnode{elementR}{}}&  &&                                        \\ [0.3cm]
%												&                        &           &&\attrtype{\Rnode{atomicnumbertypeL}{n}\Rnode{atomicnumbertype}{umber(1,1000)}}\\ [0.55cm]
												&                        &           &&\attrtype{\Rnode{atomicnumbertypeL}{\CEatomicnumbertype}}\\ [0.55cm]
												&                        &\etype{\Rnode{isotope}{isotope}\Rnode{isotopeR}{}}&&                          \\ [0.3cm]
                        &                        &                            &&\attrtype{\Rnode{floattype}{\CEfloattype}}                \\ [0.45cm]
                        &\etype{\Rnode{allotrope}{allotrope}\Rnode{allotropeR}{}}&    &&                                \\ [0.45cm]
												&                        &                            &&\attrtype{\Rnode{nametype}{\CEnametype}}           \\ [1.0cm]
\etype{\Rnode{valency}{valency}\Rnode{valencyR}{}}&      &                            &&\attrtype{\Rnode{valencynumbertype}{\CEvalencynumbertype}}\\
\end{array}
$
\setlength{\arrnodesepA}{7pt}
\setlength{\arrnodesepB}{8pt}
\setlength{\arroffsetB}{7pt}
\ncarr[10]{valency}{element}
\setlength{\arrnodesepB}{7pt}
\setlength{\arroffsetB}{0pt}
\ncarr[-5]{allotrope}{element}
\setlength{\arrnodesepB}{9pt}
\setlength{\arroffsetB}{-5pt}
\ncarr[-5]{isotope}{element}
\setlength{\arroffsetB}{0pt}
\setlength{\arrnodesepB}{3pt}
\setlength{\arroffsetB}{-2pt}
\setlength{\arroffsetA}{5pt}
\ncarr[15]{elementR}{symboltype}
\alabel{symbol}[0.4][0]
\setlength{\arroffsetA}{0pt}
\setlength{\arrnodesepB}{3pt}
\setlength{\arroffsetB}{0pt}
\ncarr[5]{elementR}{atomicnumbertypeL}
\alabel{atomic\,number}[0.4]
\setlength{\arroffsetA}{2pt}
\setlength{\arroffsetB}{-2pt}
\ncarr[5]{isotopeR}{atomicnumbertypeL}
\alabel{neutron\,count}[0.35][0]
\setlength{\arroffsetA}{-2pt}
\ncarr[5]{isotopeR}{floattype}
\alabel{mass}[0.35]
\setlength{\arroffsetA}{2pt}
\ncarr[5]{allotropeR}{floattype}
\alabel{melting\,point}[0.35][-1]
\setlength{\arroffsetA}{-2pt}
\setlength{\arroffsetB}{-4pt}
\ncarr[5]{allotropeR}{nametype}
\blabel{name}[0.3]
\setlength{\arroffsetA}{2pt}
\setlength{\arroffsetB}{-2pt}
\ncarr[5]{valencyR}{valencynumbertype}
\alabel{number}[0.3]
} 
\end{frame}



\begin{frame}{Each entity type has at least on mono source defined}
\begin{tabular}{l}
\scalebox{0.60}{
$
\begin{array}{c p{4.5cm} l}                                                
\etype{\Rnode{element}{chemical\ element}\Rnode{elementR}{}}& &  \attrtype{\Rnode{symboltype}{\CEsymboltype}  } \\
\end{array}
$
\setlength{\arrnodesepA}{7pt}
\setlength{\arrnodesepB}{6pt}
\setlength{\arroffsetB}{-2pt}
\setlength{\arroffsetA}{0pt}
\ncarr[5]{elementR}{symboltype}
\alabel{symbol}[0.4][0]


} \\ [0.6cm]
\scalebox{0.65}{
$
\begin{array}{c p{4.5cm} l}                                                
\etype{\Rnode{element}{chemical\ element}\Rnode{elementR}{}}& & \attrtype{\Rnode{atomicnumbertypeL}{\CEatomicnumbertype}}\\
\end{array}
$
\setlength{\arrnodesepA}{7pt}
\setlength{\arrnodesepB}{6pt}
\setlength{\arroffsetB}{-2pt}
\setlength{\arroffsetA}{0pt}
\ncarr[5]{elementR}{atomicnumbertypeL}
\alabel{atomic\,number}[0.4]


} \\ [0.6cm]
\scalebox{0.65}{
$
\begin{array}{c p{4.5cm} l}                                                
\etype{\Rnode{allotrope}{allotrope}\Rnode{allotropeR}{}}& & \attrtype{\Rnode{nametype}{\CEnametype}} \\
\end{array}
$
\setlength{\arrnodesepA}{7pt}
\setlength{\arrnodesepB}{6pt}
\setlength{\arroffsetB}{-2pt}
\setlength{\arroffsetA}{0pt}
\ncarr[5]{allotropeR}{nametype}
\blabel{name}[0.3]


} \\ [0.6cm]
\scalebox{0.65}{
$
\begin{array}{c p{4.5cm} l}
                                                  & &\etype{\Rnode{element}{chemical\ element}\Rnode{elementR}{}} \\ [0.25cm]
\etype{\Rnode{isotope}{isotope}\Rnode{isotopeR}{}}& &                                                             \\[0.25cm]
                                                  & &\attrtype{\Rnode{atomicnumbertypeL}{\CEatomicnumbertype}} \\
\end{array}
$
\setlength{\arrnodesepA}{7pt}
\setlength{\arrnodesepB}{8pt}
\setlength{\arroffsetA}{2pt}
\setlength{\arroffsetB}{0pt}
\ncarr[10]{isotopeR}{element}
\alabel{of}[0.3]
\setlength{\arroffsetA}{0pt}
\setlength{\arroffsetB}{-3pt}
\ncarr[5]{isotopeR}{atomicnumbertypeL}
\blabel{neutron\,count}[0.35][0]


} \\ [1.1cm]
\scalebox{0.65}{
$
\begin{array}{c p{4.5cm} l}
                                                  & &\etype{\Rnode{element}{chemical\ element}\Rnode{elementR}{}} \\ [0.25cm]
\etype{\Rnode{valency}{valency}\Rnode{valencyR}{}}& &                                                             \\[0.25cm]
                                                  & &\attrtype{\Rnode{valencynumbertype}{\CEvalencynumbertype}}           \\
\end{array}
$
\setlength{\arrnodesepA}{7pt}
\setlength{\arrnodesepB}{8pt}
\setlength{\arroffsetB}{0pt}
\ncarr[10]{valency}{element}
\alabel{of}[0.3]
\setlength{\arroffsetA}{2pt}
\setlength{\arroffsetB}{-3pt}
\ncarr[5]{valencyR}{valencynumbertype}
\blabel{number}[0.3]
} 
\end{tabular}
\end{frame}

\begin{frame}{Composing mono sources}
\begin{itemize}
\item{
In the category there are two further \textit{derived}  mono sources: \\
\vspace{0.5cm}
\begin{tabular}{l}
\scalebox{0.65}{
$
\begin{array}{c p{4.5cm} l}
                                                  & &\attrtype{\Rnode{symboltype}{\CEsymboltype}  }              \\ [0.25cm]
\etype{\Rnode{valency}{valency}\Rnode{valencyR}{}}& &                                                            \\[0.25cm]
                                                  & &\attrtype{\Rnode{valencynumbertype}{\CEvalencynumbertype}}  \\
\end{array}
$
\setlength{\arrnodesepA}{7pt}
\setlength{\arrnodesepB}{8pt}
\setlength{\arroffsetA}{2pt}
\setlength{\arroffsetB}{0pt}
\ncarr[10]{valency}{symboltype}
\alabel{of \circ symbol}[0.3]
\setlength{\arroffsetA}{0pt}
\setlength{\arroffsetB}{-3pt}
\ncarr[5]{valencyR}{valencynumbertype}
\blabel{number}[0.3]

} \\ [1.0cm]
\scalebox{0.65}{
$
\begin{array}{c p{4.5cm} l}
                                                  & &\attrtype{\Rnode{symboltype}{\CEsymboltype}  }            \\ [0.3cm]
\etype{\Rnode{isotope}{isotope}\Rnode{isotopeR}{}}& &                                                          \\[0.3cm]
                                                  & &\attrtype{\Rnode{atomicnumbertypeL}{\CEatomicnumbertype}} \\
\end{array}
$
\setlength{\arrnodesepA}{7pt}
\setlength{\arrnodesepB}{8pt}
\setlength{\arroffsetB}{0pt}
\ncarr[10]{isotopeR}{symboltype}
\alabel{of \circ symbol}[0.3]
\setlength{\arroffsetA}{2pt}
\setlength{\arroffsetB}{-3pt}
\ncarr[5]{isotopeR}{atomicnumbertypeL}
\alabel{neutron\,count}[0.35][0]


} 
\end{tabular}
}
\end{itemize}
\end{frame}

\begin{frame}{Therefore every entity type has at least one candidate key}
\begin{tabular}{l}
\scalebox{0.60}{
$
\begin{array}{c p{4.5cm} l}                                                
\etype{\Rnode{element}{chemical\ element}\Rnode{elementR}{}}& &  \attrtype{\Rnode{symboltype}{\CEsymboltype}  } \\
\end{array}
$
\setlength{\arrnodesepA}{7pt}
\setlength{\arrnodesepB}{6pt}
\setlength{\arroffsetB}{-2pt}
\setlength{\arroffsetA}{0pt}
\ncarr[5]{elementR}{symboltype}
\alabel{symbol}[0.4][0]


} \\ [0.7cm]
\scalebox{0.65}{
$
\begin{array}{c p{4.5cm} l}                                                
\etype{\Rnode{element}{chemical\ element}\Rnode{elementR}{}}& & \attrtype{\Rnode{atomicnumbertypeL}{\CEatomicnumbertype}}\\
\end{array}
$
\setlength{\arrnodesepA}{7pt}
\setlength{\arrnodesepB}{6pt}
\setlength{\arroffsetB}{-2pt}
\setlength{\arroffsetA}{0pt}
\ncarr[5]{elementR}{atomicnumbertypeL}
\alabel{atomic\,number}[0.4]


} \\ [0.7cm]
\scalebox{0.65}{
$
\begin{array}{c p{4.5cm} l}                                                
\etype{\Rnode{allotrope}{allotrope}\Rnode{allotropeR}{}}& & \attrtype{\Rnode{nametype}{\CEnametype}} \\
\end{array}
$
\setlength{\arrnodesepA}{7pt}
\setlength{\arrnodesepB}{6pt}
\setlength{\arroffsetB}{-2pt}
\setlength{\arroffsetA}{0pt}
\ncarr[5]{allotropeR}{nametype}
\blabel{name}[0.3]


} \\ [0.7cm]
\scalebox{0.65}{
$
\begin{array}{c p{4.5cm} l}
                                                  & &\attrtype{\Rnode{symboltype}{\CEsymboltype}  }              \\ [0.25cm]
\etype{\Rnode{valency}{valency}\Rnode{valencyR}{}}& &                                                            \\[0.25cm]
                                                  & &\attrtype{\Rnode{valencynumbertype}{\CEvalencynumbertype}}  \\
\end{array}
$
\setlength{\arrnodesepA}{7pt}
\setlength{\arrnodesepB}{8pt}
\setlength{\arroffsetA}{2pt}
\setlength{\arroffsetB}{0pt}
\ncarr[10]{valency}{symboltype}
\alabel{of \circ symbol}[0.3]
\setlength{\arroffsetA}{0pt}
\setlength{\arroffsetB}{-3pt}
\ncarr[5]{valencyR}{valencynumbertype}
\blabel{number}[0.3]

} \\ [1.1cm]
\scalebox{0.65}{
$
\begin{array}{c p{4.5cm} l}
                                                  & &\attrtype{\Rnode{symboltype}{\CEsymboltype}  }            \\ [0.3cm]
\etype{\Rnode{isotope}{isotope}\Rnode{isotopeR}{}}& &                                                          \\[0.3cm]
                                                  & &\attrtype{\Rnode{atomicnumbertypeL}{\CEatomicnumbertype}} \\
\end{array}
$
\setlength{\arrnodesepA}{7pt}
\setlength{\arrnodesepB}{8pt}
\setlength{\arroffsetB}{0pt}
\ncarr[10]{isotopeR}{symboltype}
\alabel{of \circ symbol}[0.3]
\setlength{\arroffsetA}{2pt}
\setlength{\arroffsetB}{-3pt}
\ncarr[5]{isotopeR}{atomicnumbertypeL}
\alabel{neutron\,count}[0.35][0]


} 
\end{tabular}
\end{frame}




\subsection{Entity-Relationship Notation}
\input{ERA.slides}

\iffalse
\begin{frame}{Bibliography}
\bibliography{../../SharedBibliography/temp/bibliography}
\end{frame}
\fi

\end{document}