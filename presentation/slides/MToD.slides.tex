

\usepackage{mathptmx}
\usepackage{amsfonts}
\usepackage{wasysym}
\usepackage{url}
\usepackage{hyperref}

\newcommand{\sharedmacros}{../../SharedMacros}
%\usepackage{imakeidx}
\makeindex[name=definitions, title=Index of Definitions]
\makeindex[name=lemmas, title=Index of Lemmas]



\newcommand{\commentary}[1]{\marginpar{\footnotesize #1}}
\newcommand{\highlight}[1]{\colorbox{orange}{#1}}
\newcommand{\term}[1]{\textit{#1}\commentary{\colorbox{lightgray}{\textit{#1}}}\index[definitions]{#1}}
\newcommand{\llabel}[1]{\label{#1}\commentary{\colorbox{pink}{\scriptsize{#1}}}\index[lemmas]{#1}}
\newcommand{\lref}[1]{\ref{#1}\colorbox{pink}{\scriptsize{#1}}\index[lemmas]{#1!use of}}

\newcommand{\newt}[1]{\colorbox{yellow}{#1}}
\newenvironment{newtt}
{  \colorbox{yellow}{$[$ ...} 
}
{  \colorbox{yellow}{... $]$}
}
\newcommand{\oldt}[1]{\colorbox{yellow}{\sout{#1}}}
\newenvironment{oldtt}
{  \colorbox{red}{$[$ ...} 
}
{  \colorbox{red}{... $]$}
}

\newcommand{\reinstatet}[1]{\colorbox{lime}{#1}}
\newenvironment{reinstatett}
{  \colorbox{lime}{$[$ ...}
}
{  \colorbox{lime}{... $]$}
}

\newcommand{\tbd}{\highlight{TBD}}

%ithprojection function
\newcommand{\proji}[1]{\pi_#1}



\newenvironment{categoricalaside}
{\begin{framed}
\textbf{Categorical Aside}
}
{
\end{framed}
}

\newenvironment{noteforfuture}
{\begin{framed}
\textbf{Note For Future}
}
{
\end{framed}
}

\newenvironment{problem}
{\begin{framed}
\textbf{Problem}
}
{
\end{framed}
}

%quine quote
\newcommand{\qq}[1]{
\left\ulcorner#1\right\urcorner
}

%single quote
\newcommand{\sq}[1]{
\textnormal{\textquotesingle}#1\textnormal{\textquotesingle}
}

%lower quine quote
\newcommand{\lqq}[1]{
\left\llcorner #1\right\lrcorner
}


%from berkley
\newcommand{\langl}{\begin{picture}(4.5,7)
\put(1.1,2.5){\rotatebox{60}{\line(1,0){5.5}}}
\put(1.1,2.5){\rotatebox{300}{\line(1,0){5.5}}}
\end{picture}}
\newcommand{\rangl}{\begin{picture}(4.5,7)
\put(.9,2.5){\rotatebox{120}{\line(1,0){5.5}}}
\put(.9,2.5){\rotatebox{240}{\line(1,0){5.5}}}
\end{picture}}
\newcommand{\lang}{\begin{picture}(5,7)\put(1.1,2.5){\rotatebox{45}{\line(1,0){6.0}}}\put(1.1,2.5){\rotatebox{315}{\line(1,0){6.0}}}\end{picture}}
\newcommand{\rang}{\begin{picture}(5,7)\put(.1,2.5){\rotatebox{135}{\line(1,0){6.0}}}\put(.1,2.5){\rotatebox{225}{\line(1,0){6.0}}}\end{picture}}
%Try sharper tuple brackets -- except gives errors nested in captions so comment out
%\renewcommand{\tuple}[1]{\lang #1 \rang}

\newcommand{\setsuchthat}[2]{\left\{#1 \ \middle|\ #2\right\}}
\newcommand{\set}[1]{\left\{#1\right\}} 

% one to n - wanton
\newcommand{\wanton}[1]{#1_1,...#1_n}
\newcommand{\fn}{\wanton{f}}
\newcommand{\pn}{\wanton{p}}
\newcommand{\qn}{\wanton{q}}
\newcommand{\qnprime}{\wanton{q'}}
\newcommand{\xn}{\wanton{x}}
\newcommand{\xnp}{\wanton{x'}}
\newcommand{\yn}{\wanton{y}}
\newcommand{\ntuple}[1]{\tuple{\wanton{#1}}}
\newcommand{\wantom}[1]{#1_1,...#1_m}
\newcommand{\mtuple}[1]{\tuple{#1_1,...#1_m}}
\newcommand{\qm}{\wantom{q}}
\newcommand{\ym}{\wantom{y}}
\newcommand {\bntuple}{\ensuremath{\ntuple{b}}}
\newcommand {\fntuple}{\ensuremath{\ntuple{f}}}
\newcommand {\fnptuple}{\ensuremath{\ntuple{f}}}
\newcommand {\pntuple}{\ensuremath{\ntuple{p}}}
\newcommand {\qntuple}{\ensuremath{\ntuple{q}}}
\newcommand {\qnptuple}{\ensuremath{\ntuple{q'}}}
\newcommand {\qmtuple}{\ensuremath{\mtuple{q}}}
\newcommand {\sntuple}{\ensuremath{\ntuple{s}}}
\newcommand {\xntuple}{\ensuremath{\ntuple{x}}}
\newcommand {\xnptuple}{\ensuremath{\ntuple{x'}}}
\newcommand {\ymtuple}{\ensuremath{\mtuple{y}}}
\newcommand{\foreachi}[1][n]{for each $i$, $1 \leq i \leq #1$}
\newcommand{\foreachj}[1][m]{for each $j$, $1 \leq j \leq #1$}
\newcommand{\foreachk}[1][l]{for each $k$, $1 \leq k \leq #1$}

    %causes problems when used with bamer

%ccategories.macros.tex 

% Macros for diagrams in contextual categories and related categories

\usepackage{twoopt}
\usepackage{scalerel} 
\usepackage{xargs}

%\usepackage{mathabx}  %Caused font problems
%\usepackage{MnSymbol}  % caused font problems

\newcommand{\conu}
{\mathbf{C}(U)}

\newcommand{\depu}
{\mathbf{D}(U)}

\newcommand{\cat}[1]{\textbf{#1}}
\newcommand{\obj}[1]{\ensuremath{|\cat{#1}|}}
\newcommand{\ccat}[1][C]{\ensuremath{\mathbb{#1}} }
\newcommand{\ccatc}{contextual category \ccat}
\newcommand{\cobj}[2][]{\ensuremath{|\ccat[#2]|_{#1}}}
\newcommand{\cslice}[2]{\ensuremath{\ccat[#1]_{#2}}}
\newcommand{\csliceobj}[3][]{\ensuremath{|\mathbb{#2}_{#3}|_{#1} }}
\newcommand{\varset}[1][]{\ensuremath{V_{#1} }}
\newcommand{\localvarsets}{\ensuremath{\mathcal{V} }}
\newcommand{\Fam}{\ensuremath{\mathbb{F\mathrm{am}} }}
\newcommand{\Famslice}[1]{\ensuremath{\mathbb{F\mathrm{am}}_{#1} }}
\newcommand{\Famobj}[1][]{\ensuremath{|\mathbb{F\mathrm{am}}|_{#1} }}
\newcommand{\Famsliceobj}[2][]{\ensuremath{|\mathbb{F\mathrm{am}}_{#2}|_{#1} }}
\newcommand{\morph}{\rightarrow}
\newcommand{\epi}{\twoheadrightarrow}
\newcommand{\base}{\triangleleft}
\newcommand{\comp}{\circ}
\newcommand{\cross}{\otimes}
\newcommand{\pc}[2]{d^{#1}_{#2}}
\newcommand{\sub}{^*}
\newcommand{\diag}{\delta}
\newcommand{\pbase}[1]{\tilde{#1}}

\newcommand{\tuple}[1]{\langle#1\rangle}
\newcommand{\ndidly}{\ensuremath{\Join_n}}
\newcommand{\ndidlycospan}{quotiented n-cospan}

\newcommand{\crossx}[3]{#1 \underset{#3}{\cross} #2}
\newcommand{\fibrex}[3]{#1 \underset{#3}{\Join} #2}
\newcommand{\powerset}{\mathcal{P}}
\newcommand{\primeds}[1]{
\ensuremath{\mathcal{P}(#1)} }
\newcommand{\compset}{\ \dot{\circ}\, }

% darrow
%\newcommand{\darrow}{\rightarrowtriangle} %use \smorph instead
\newcommand{\smorph}{\rightarrowtriangle}

 

\newcommand\dhead{\scaleobj{0.6}{\triangleright}}
\newcommand{\dmorph}{\, \mbox{---} \! \cdot \! \raisebox{1.1pt}{\dhead}}

% projection tree
%\newcommand{\proj}[2]{proj_{#2}(#1)}

\newcommand{\proj}[2]{
\ensuremath{\mathcal{P}_{#2}(#1)} }

%pstrick supplements for arrows

\newlength{\arrnodesepA}
\newlength{\arrnodesepB}
\newlength{\arroffsetA}
\newlength{\arroffsetB}

%Modified to 2pt from 0pt on 23 July 2018
\newcommand{\arreset}{
\setlength{\arrnodesepA}{2pt}
\setlength{\arrnodesepB}{2pt}
\setlength{\arroffsetA}{0pt}
\setlength{\arroffsetB}{0pt}
}
\arreset

\newcommand{\ncarr}[3][0]{\ncarc[arcangle=#1,nodesepA=\arrnodesepA,nodesepB=\arrnodesepB,offsetA=\arroffsetA,offsetB=\arroffsetB,arrowsize=5pt,arrowinset=0.7]{->}{#2}{#3}}
\newcommand{\jcbarr}[4][0]{ % ncbarr is defined in some thridy party package so do not use!\emph{}
\ncarr[#1]{#3}{#4}
\nbput[labelsep=2pt]{\footnotesize $#2$}
}

\newcommand{\ncaarr}[4][0]{
\ncarr[#1]{#3}{#4}
\naput[labelsep=2pt]{\footnotesize $#2$}
}

% \alabel{label}[npos][labelsep_pts]
\newcommandx*\alabel[3][2=0.5,3=2,usedefault]{\naput[labelsep=#3pt,npos=#2]{\footnotesize $#1$}}
% \blabel{label}[npos][labelsep_pts]
\newcommandx*\blabel[3][2=0.5,3=2,usedefault]{\nbput[labelsep=#3pt,npos=#2]{\footnotesize $#1$}}

% \idcomp mark an arrow as one component of an identifier
\newcommand{\idcomp}{\ncput[npos=0, nrot=:U]{\psline(0.2,-0.075)(0.2,0.075)}}  %add a bar to a node connection arrow
% pstrick supplements for s-arrows (previous name for d-arrow - should convert}

\newlength{\sarnodesepA}
\newlength{\sarnodesepB}
\newlength{\saroffsetA}
\newlength{\saroffsetB}
\newlength{\sarnodesepAsav}
\newlength{\sarnodesepBsav}

\newcommand{\sarreset}{
\setlength{\sarnodesepA}{0pt}
\setlength{\sarnodesepB}{0pt}
\setlength{\saroffsetA}{0pt}
\setlength{\saroffsetB}{0pt}
}

\sarreset

% sar - S-arrow
\newcommand{\ncsar}[3][0]{
\setlength{\sarnodesepAsav}{\sarnodesepA}
\setlength{\sarnodesepBsav}{\sarnodesepB}
\addtolength{\sarnodesepA}{3pt}
\addtolength{\sarnodesepB}{7pt}
\ncarc[nodesepA=\sarnodesepA,nodesepB=\sarnodesepB,offsetA=\saroffsetA,offsetB=\saroffsetB,arcangle=#1]{-}{#2}{#3}
\ncput[nrot=:R,npos=1]{\pstriangle(0,0)(.2,.2)}
\setlength{\sarnodesepA}{\sarnodesepAsav}
\setlength{\sarnodesepB}{\sarnodesepBsav}
}


% bsar - below labelled S-arrow
\newcommand{\ncbsar}[4][0]{
\ncsar[#1]{#3}{#4}
\nbput[labelsep=2pt]{\footnotesize $#2$}
}
% asar - above labelled S-arrow
\newcommand{\ncasar}[4][0]{
\ncsar[#1]{#3}{#4}
\naput[labelsep=2pt]{\footnotesize $#2$}
}

% cdar - composite dependency arrow
\newcommand{\nccdar}[3][0]{
\setlength{\sarnodesepAsav}{\sarnodesepA}
\setlength{\sarnodesepBsav}{\sarnodesepB}
\addtolength{\sarnodesepA}{3pt}
\addtolength{\sarnodesepB}{11pt}
\ncarc[nodesepA=\sarnodesepA,nodesepB=\sarnodesepB,offsetA=\saroffsetA,offsetB=\saroffsetB,arcangle=#1]{-}{#2}{#3}
\ncput[nrot=:R,npos=1]{\pstriangle(0,0.1)(.2,.2)}
\ncput[nrot=:R,npos=1]{\psdot[dotsize=1pt](-0.0075,0.05)}   %!!
\setlength{\sarnodesepA}{\sarnodesepAsav}
\setlength{\sarnodesepB}{\sarnodesepBsav}
}


% bcdar - below labelled composite dependency arrow
\newcommand{\ncbcdar}[4][0]{
\nccdar[#1]{#3}{#4}
\nbput[labelsep=2pt]{\footnotesize $#2$}
}
% acdar - above labelled composite dependency arrow
\newcommand{\ncacdar}[4][0]{
\nccdar[#1]{#3}{#4}
\naput[labelsep=2pt]{\footnotesize $#2$}
}


% rsar - recursive S-arrow
\newcommand{\ncrsar}[2]{
\setlength{\sarnodesepAsav}{\sarnodesepA}
\setlength{\sarnodesepBsav}{\sarnodesepB}
\addtolength{\sarnodesepA}{3pt}
\addtolength{\sarnodesepB}{7pt}
\ncloop[nodesepA=\sarnodesepA,nodesepB=\sarnodesepB,
        offsetA=\saroffsetA,offsetB=\saroffsetB,
        armA=0.7cm,armB=0.6cm,angleA=90,angleB=-90,loopsize=-1,linearc=0.4
				]{-}{#1}{#2}
\ncput[nrot=:R,npos=5]{\pstriangle(0,0)(.2,.2)}
\setlength{\sarnodesepA}{\sarnodesepAsav}
\setlength{\sarnodesepB}{\sarnodesepBsav}
}

% pstrick supplements for multi-arrows

\newlength{\marnodesepA}
\newlength{\marnodesepB}
\newlength{\maroffsetB}
\newlength{\marnodesepBsav}

\newcommand{\marreset}{
\setlength{\marnodesepA}{0pt}
\setlength{\marnodesepB}{0pt}
\setlength{\maroffsetB}{0pt}
}

\marreset

%ncmarr[#1 arcangle1][#2 arcangle2]{#3 name}{#4 domain1}{#5 domain2}{#6 junction}{#7 codomain}
\newcommandtwoopt{\ncmarr}[6][8][8]{%
\ncarc[nodesepA=\marnodesepA,nodesepB=0,arcangle=#1]{-}{#3}{#5}
\ncarc[nodesepB=0,arcangle=-#1]{-}{#4}{#5}
\ncarc[arcangle=#2,nodesepB=\marnodesepB,offsetB=\maroffsetB]{->}{#5}{#6}
}%


\newcommandtwoopt{\nchmarr}[6][8][8]{%
\ncarc[nodesepA=\marnodesepA,nodesepB=0,arcangle=#1]{-}{#3}{#5}
\ncarc[nodesepB=0,arcangle=#1]{-}{#4}{#5}
\ncarc[arcangle=#2,nodesepB=\marnodesepB,offsetB=\maroffsetB]{->}{#5}{#6}
}%

\newcommandtwoopt{\ncamarr}[7][8][8]{%
\ncmarr[#1][#2]{#4}{#5}{#6}{#7}
\naput[npos=.05]{$#3$}
}%
\newcommandtwoopt{\ncbmarr}[7][8][8]{%
\ncmarr[#1][#2]{#4}{#5}{#6}{#7}
\nbput[npos=.05]{$#3$}
}%

\newcommandtwoopt{\ncbhmarr}[7][8][8]{%
\nchmarr[#1][#2]{#4}{#5}{#6}{#7}
\nbput[npos=.05]{$#3$}
}%

\newcommandtwoopt{\ncmarrr}[7][8][8]{
\ncarc[nodesepB=0,arcangle=#1]{-}{#3}{#6}
\ncline[nodesepB=0]{-}{#4}{#6}
\ncarc[nodesepB=0,arcangle=-#1]{-}{#5}{#6}
\ncarc[nodesepA=0,arcangle=#2]{->}{#6}{#7}
}

\newcommandtwoopt{\ncamarrr}[8][8][8]{
\ncmarrr[#1][#2]{#4}{#5}{#6}{#7}{#8}
\naput[npos=.05]{$#3$}
}
\newcommandtwoopt{\ncbmarrr}[8][8][8]{
\ncmarrr[#1][#2]{#4}{#5}{#6}{#7}{#8}
\nbput[npos=.05]{$#3$}
}

%gats.macros.tex

\usepackage{environ}    % also used in ermacros % here used for \NewEnvrion

\newcommand{\gat}[1][U]{
\ensuremath{\mathcal{#1}}}  % used to hav a space in here
\newcommand{\gatw}[1][U]{\gat[#1]\ }  % use this if need trailing space
\newcommand{\ingat}[1][U]{in \gat[#1]}
\newcommand{\isagat}[1][U]{\gat[#1] is a g.a.t.}
\newcommand{\inagat}{in a g.a.t. }

% macro for a generic theory
%\newcommand{\theory}
%{\textit{U}}

\newcommand{\intheory}
{is a derived rule of \gat[U]}

% Macros for GAT rules

\newcommand{\isT}[1]
{#1\mbox{ is a type}}

\newcommand{\ofT}[2]
{#1 \in #2
}

% Macros for GAT rules   <!-- new old -->
\newcommand{\istype}[1]
{#1\mbox{ is a type}}

\newcommand{\oftype}[2]
{#1 \in #2
}

%\context{x}{\Delta}{n}
\newcommand{\context}[3]
{\ofT{#1_1}{#2_1},... \ofT{#1_{#3}}{#2_{#3}(#1_1,...#1_{#3-1})}
}

%\subcontext{x}{\Delta}{i}{k}
\newcommand{\subcontext}[4]
{\ofT{#1_{#3_1}}{#2_{#3_1}},... \ofT{#1_{#3_#4}}{#2_{#3_#4}(#1_1,...#1_{#3_#4-1})}
}

% #schematic context
\newcommand{\schmcon}[3]
{\ofT{#1_1}{#2_1},... \ofT{#1_{#3}}{#2_{#3}}
}
% abbreviated to
\newcommand{\con}[3]
{\schmcon{#1}{#2}{#3}}

% schematic subcontext
%\subcon{x}{\Delta}{i}{k}
\newcommand{\subcon}[4]
{\ofT{#1_{#3_1}}{#2_{#3_1}},... \ofT{#1_{#3_#4}}{#2_{#3_#4}}
}

% permuted context
%\permcon{x}{\Delta}{n}{\sigma}
\newcommand{\permcon}[4]
{\ofT{#1_{#4(1)}}{#2_{#4(1)}},... \ofT{#1_{#4(#3)}}{#2_{#4(#3)}}
}
% permuted term
%\permterm{t}{n}{\sigma}
\newcommand{\permterm}[3]
{
#1_{#3(1)},...#1_{#3(#2)}
}


% Idioms
\newcommand{\xDelta}[1]{\con{x}{\Delta}{#1}}
\newcommand{\xDeltap}[1]{\con{x}{\Delta'}{#1}}
\newcommand{\xOmega}[1]{\con{x}{\Omega}{#1}}
\newcommand{\xOmegap}[1]{\con{x}{\Omega'}{#1}}
\newcommand{\yOmega}[1]{\con{y}{\Omega}{#1}}
\newcommand{\yOmegap}[1]{\con{y}{\Omega'}{#1}}

\newcommand{\xDeltasigma}[1]{\permcon{x}{\Delta}{#1}{\sigma}}
\newcommand{\xDeltapsigma}[1]{\permcon{x}{\Delta'}{#1}{\sigma}}
\newcommand{\xOmegasigma}[1]{\permcon{x}{\Omega}{#1}{\sigma}}
\newcommand{\xOmegapsigma}[1]{\permcon{x}{\Omega'}{#1}{\sigma}}
\newcommand{\yOmegasigma}[1]{\permcon{y}{\Omega}{#1}{\sigma}}
\newcommand{\yOmegapsigma}[1]{\permcon{y}{\Omega'}{#1}{\sigma}}

\newcommand{\xDeltainvsigma}[1]{\permcon{x}{\Delta}{#1}{\sigma^{-1}}}
\newcommand{\xDeltapinvsigma}[1]{\permcon{x}{\Delta'}{#1}{\sigma^{-1}}}
\newcommand{\xOmegainvsigma}[1]{\permcon{x}{\Omega}{#1}{\sigma^{-1}}}
\newcommand{\xOmegapinvsigma}[1]{\permcon{x}{\Omega'}{#1}{\sigma^{-1}}}
\newcommand{\yOmegainvsigma}[1]{\permcon{y}{\Omega}{#1}{\sigma^{-1}}}
\newcommand{\yOmegapinvsigma}[1]{\permcon{y}{\Omega'}{#1}{\sigma^{-1}}}

%Idioms enclosed as tuples
\newcommand{\encxDelta}[1]{\tuple{\con{x}{\Delta}{#1}}}
\newcommand{\encxDeltap}[1]{\tuple{\con{x}{\Delta'}{#1}}}
\newcommand{\encxOmega}[1]{\tuple{\con{x}{\Omega}{#1}}}
\newcommand{\encxOmegap}[1]{\tuple{\con{x}{\Omega'}{#1}}}
\newcommand{\encyOmega}[1]{\tuple{\con{y}{\Omega}{#1}}}
\newcommand{\encyOmegap}[1]{\tuple{\con{y}{\Omega'}{#1}}}

\newcommand{\encxDeltasigma}[1]{\tuple{\permcon{x}{\Delta}{#1}{\sigma}}}
\newcommand{\encxDeltapsigma}[1]{\tuple{\permcon{x}{\Delta'}{#1}{\sigma}}}
\newcommand{\encxOmegasigma}[1]{\tuple{\permcon{x}{\Omega}{#1}{\sigma}}}
\newcommand{\encxOmegapsigma}[1]{\tuple{\permcon{x}{\Omega'}{#1}{\sigma}}}
\newcommand{\encyOmegasigma}[1]{\tuple{\permcon{y}{\Omega}{#1}{\sigma}}}
\newcommand{\encyOmegapsigma}[1]{\tuple{\permcon{y}{\Omega'}{#1}{\sigma}}}

\newcommand{\encxDeltainvsigma}[1]{\tuple{\permcon{x}{\Delta}{#1}{\sigma^{-1}}}}
\newcommand{\encxDeltapinvsigma}[1]{\tuple{\permcon{x}{\Delta'}{#1}{\sigma^{-1}}}}
\newcommand{\encxOmegainvsigma}[1]{\tuple{\permcon{x}{\Omega}{#1}{\sigma^{-1}}}}
\newcommand{\encxOmegapinvsigma}[1]{\tuple{\permcon{x}{\Omega'}{#1}{\sigma^{-1}}}}
\newcommand{\encyOmegainvsigma}[1]{\tuple{\permcon{y}{\Omega}{#1}{\sigma^{-1}}}}
\newcommand{\encyOmegapinvsigma}[1]{\tuple{\permcon{y}{\Omega'}{#1}{\sigma^{-1}}}}

\newcommand{\tstyle}{\vdash}

% gat macros developed for cwf paper

% Expressing gats
\newenvironment{gatrules}
{
$$
\begin{array}{l l}
}
{
\end{array}
$$
}
\newcommand{\gatintros}
{
\textbf{Symbol} & \textbf{Introductory\ Rule}                      \\}

\newcommand{\gataxioms}
{\textbf{Axioms}\\}
\newcommand{\gatintro}[3]{\ #1 & #2 \tstyle #3 \\}
\newcommand{\gatlocalintro}[3]{\ #1 & #2 \dashv }
\newcommand{\gataxiom}[2]{\multicolumn{2}{l}{\ \ #1\mbox{,  whenever\ } #2} \\}
\newcommand{\noleft}{\left.\kern-\nulldelimiterspace} % so that no space taken by absent left brace


\newcommand{\gatmultiaxiom}[2]
{\multicolumn{2}{l}{
  \noleft
    \begin{array}{l}
		#1
    \end{array} 
  \right\} \mbox{whenever\ } 	#2 
	}\\}
	
	\newcommand{\axid}[1]{\text{#1}.\ }	

%New context sharing macros
\newcommand{\gatintroducing}[1]{
{\arraycolsep=0pt
  \begin{array}{l}
          #1
  \end{array}} &
}

%*********************************
% \begin{\gatgroup}{context}
%    rules
%  \end{\gatgroup}
%*********************************
\NewEnviron{gatgroup}[1]{%
  \noleft
  {\arraycolsep=0pt
   \begin{array}{l}
\BODY
    \end{array} 
   }
   \ \right\} 
	%\mbox{\ whenever\ } 
	#1
	\vspace{0.1cm} 
}
%*********************************

%*********************************
% \begin{\gatgroupnoshared}
%    rule
%  \end{\gatgroupnoshared}
%*********************************
\NewEnviron{gatgroupnoshared}{%
  {\arraycolsep=0pt
   \begin{array}{l}
\BODY
    \end{array} 
   }
   \ 
	\vspace{0.1cm} 
}
%*********************************

% \gatsingular[width]{context}{conclusion}
\newcommand{\gatsingular}[3][4cm]{
\begin{gatgroupnoshared}
\gatleaf[#1]{#2}{#3} 
\end{gatgroupnoshared}
}

%*********************************
% \gatleaf}[width]{context}{assertion}
%*********************************
\newcommand{\gatleaf}[3][4cm]{%
\makebox[#1]{$#3$ \dotfill} \dotfill \  #2
}
%*********************************
%*********************************
% \gatstandalonesingle}{context}{assertion}
%*********************************
\newcommand{\gatstandalonesingle}[2]{%
#2 \makebox[2.5cm]{\dotfill} \  #1
}
%*********************************

% \gataxiomno{axiomno}
\newcommand{\gataxiomno}[1]{\makebox[0.5cm]{} \axid{#1}}


% metagat.macros.tex

%Meta-theories

%\newcommand{\typ}{\triangleright}
\newcommand{\typ}{\nabla}
\newcommand{\trm}{\tau}
\newcommand{\cross}{\otimes}
\newcommand{\sub}{^*}
\newcommand{\diag}{\delta}

\newcommand{\typeseq}[2]
{\ofT{#1_1}{\typ},... \ofT{#1_{#2}}{\typ(#1_{#2-1})}}

\newcommand{\typeseqcont}[3]
{\ofT{#1_1}{\typ({#2})},... \ofT{#1_{#3}}{\typ(#1_{#3-1})}}

\newcommand{\Ob}{Ob}
\newcommand{\obj}{Ob} % <!-- new old --<
\newcommand{\Hom}{Hom}
\newcommand{\objseq}[2]
{\ofT{#1_1}{\obj},... \ofT{#1_{#2}}{\obj(#1_{#2-1})}}


\def\dottededge{\ncline[linestyle=dotted, nodesep=0.3cm]}
\def\noedge{\ncline[linestyle=none]}
\def\thinedge{\ncline[linewidth=0.4pt]}

\newcommand{\member}[1]
{\ncarc[arcangle=-30,nodesepB=0.03]{->}{\pspred}{\pssucc}
\nbput[labelsep=0.1]{#1}}

\newcommand{\loweraccutemember}[1]
{\ncarc[arcangle=-15,nodesepB=0.03]{->}{\pspred}{\pssucc}
\nbput[labelsep=0.05,npos=0.85]{#1}}

\newcommand{\uppermember}[1]
{\ncarc[arcangle=30,nodesepB=0.03]{->}{\pspred}{\pssucc}\naput{#1}}

\newcommand{\upperaccutemember}[1]
{\ncarc[arcangle=10,nodesepB=0.03]{->}{\pspred}{\pssucc}\naput[npos=0.85]{#1}}

% flexbranch 
% #1 node label
% #2 thislevelsep
% #3 next level sep
% #4 variable (eg x)
% #5 index leter (eg n)
% #6 close parenthesis
% #7 continuation branches
\newcommand{\flexbranch}[7]
{
\pstree[thislevelsep=*#2,nodesep=0.05]
		{\Rnode{#1 1}{\Tr{#4_1 #6}}}
	  {\pstree[thislevelsep=#3]  
				   {\Rnode{#1 2}{\Tr[edge=\dottededge]{#4_{#5} #6}}}
					 {#7}
		}
}

\newcommand{\flexbranchplusleaf}[6]
{
\flexbranch{#1}{#2}{#3}{#4} {#5} {#6}
  {
   %\Rnode{#1 3}{\Tr{#4 #6}}
	 \Tr{\Rnode{#1 3}{#4 #6}}
  }
}

\newcommand{\flexbranchplusarc}[7]
{
\flexbranch{#1}{#2}{#3}{#4} {#5} {#6}
  {
   %\Rnode{#1 3}{\Tr{#4 #6}\member{#7}}
	 \Tr{\Rnode{#1 3}{#4 #6}}\member{#7}
  }
}

\newcommand{\flexbranchinitialarc}[9]
{
\pstree[thislevelsep=*#2,nodesep=0.05]
		{\Rnode{#1 1}{\Tr{#4_#8 #6}}#9}
	  {\pstree[thislevelsep=#3]  
				   {\Rnode{#1 2}{\Tr[edge=\dottededge]{#4_{#5} #6}}}
					 {#7}
		}
}

\newcommand{\equality}[2]
{
\ncline [doubleline=true, nodesep=0.2cm]{#1}{#2}
}
\newcommand{\equalityarc}[2]
{
\ncarc [arcangleA=-30, arcangleB=-20, doubleline=true, nodesep=0.1cm]{#1}{#2}
}

\usepackage[margin=4.0cm]{geometry} %was 3cm
\usepackage{mathptmx}
\usepackage{amsfonts}
\usepackage{array}
\usepackage{pstricks}
\usepackage{pst-tree}
\usepackage{pst-plot}
\usepackage{pst-node}
\usepackage{stmaryrd}
\usepackage{amsmath}
\usepackage{verbatim}
\usepackage{graphicx}  
\usepackage{calc}
\usepackage{xifthen}
\usepackage{xcolor}
\usepackage{color}
\usepackage{stringstrings}
%\usepackage[small,bf,margin=3pt,format=hang, labelsep=endash,singlelinecheck=false]{caption} %prevuiously justification=justified
%\usepackage{enumerate}
%\usepackage{enumitem}
\usepackage{enumerate}
\usepackage[shortlabels]{enumitem}
\usepackage{float}
\usepackage[section]{placeins}
%\setlength{\captionmargin}{5pt}
\usepackage{environ}
\usepackage{multirow}
\usepackage{rotating}
\usepackage{longtable}
\usepackage{afterpage}
\usepackage{needspace}


%DEFINE ENVIRONMENT BLOCK
% Riddle
\newsavebox{\riddlebox}

\newenvironment{erexample}
{\newcommand\colboxcolor{F0F0F0}%was F8F8F8
\begin{lrbox}{\riddlebox}
\begin{minipage}{\dimexpr\columnwidth-2\fboxsep\relax} \textbf{} \\ \itshape}
{\end{minipage}\end{lrbox}%
%\begin{center}
\colorbox[HTML]{\colboxcolor}{\usebox{\riddlebox}}
%\end{center}
}

\newenvironment{erbox}
{\newcommand\colboxcolor{F0F0F0}%was F8F8F8
\begin{lrbox}{\riddlebox}%
\begin{minipage}{\dimexpr\columnwidth-2\fboxsep\relax} }
{\end{minipage}\end{lrbox}%
%\begin{center}
\colorbox[HTML]{\colboxcolor}{\usebox{\riddlebox}}
%\end{center}
}

%\begin{erboxedFigure}{#1 FigureParam}{#2 Label}{#3 Caption}
\NewEnviron{erboxedFigure}[3]{%
\begin{figure}[#1]
\begin{erexample}
\begin{center}
\BODY
\end{center}
\vspace{-0.5cm}
\caption{#3}
\label{#2}
\end{erexample}
\end{figure}
}

\newcommand{\erpictureFolder}[0]{../SharedPictures}

\newcommand{\ercenterPicture}[1]{
\begin{center}
\input{\erpictureFolder/#1}
\end{center}
}


\newlength{\erhalfHt}

%\erinlinePicture{#1 pictureFilename}{#2 pictureHeight}
\newcommand{\erinlinePicture}[2]{
\setlength{\erhalfHt}{#2cm * \real{0.5}}
\raisebox{-\erhalfHt}[\erhalfHt + 0.5cm][\erhalfHt + 0.5cm]{
\input{\erpictureFolder/#1}
} 
}

%\erplainFig{#1 pictureFilename}{#2 figureParam}{#3Caption}
\newcommand{\erplainFig}[3]{
\begin{figure}[#2]
\begin{center}
\input{\erpictureFolder/#1}
\end{center}
\caption{#3}
\label{#1}
\end{figure}
}

%\erboxedFigPicture{#1 pictureFilename}{#2 figureParam}{#3Caption}
\newcommand{\erboxedFigPicture}[3]{
\begin{figure}[#2]
\begin{erexample}
\vspace{-0.5cm}
\begin{center}
\input{\erpictureFolder/#1}
\end{center}
\caption{#3}
\label{#1}
\end{erexample}
\end{figure}
}

%\erLeftSideFig{#1 pictureFilename}{#2 figureParam}{#3Caption}
\newcommand{\erLeftSideFig}[3]{
\begin{figure}[#2]
\begin{erexample}
  \begin{minipage}[c]{0.4\textwidth}
    \caption{#3}
    \label{#1}
  \end{minipage}
  \begin{minipage}[c]{0.5\textwidth}
    \input{\erpictureFolder/#1}
  \end{minipage}
\end{erexample}
\end{figure}
}

%\erbulletedFig{#1 pictureFilename}{#2 figureParam}{#3Caption}
\NewEnviron{erbulletedFig}[3]{%
\begin{figure}[#2]
\begin{erexample}
\vspace{-0.5cm}
\begin{center}
$
\begin{array}{c m{0.25cm} | m{6cm}}
\raisebox{-2.0cm}{
\input{\erpictureFolder/#1}}& & \text{\parbox{6cm}{\raggedright{\footnotesize{
\begin{enumerate}[(i)]
\BODY
\end{enumerate}}}}} \\
\end{array}
$
\end{center}
\caption{#3}
\label{#1}
\end{erexample}
\end{figure} 
}


%\begin{erbulletedDimFig}{#1 pictureFilename}{#2figureParam} {#3Caption} {#4PictureHeight}{#5TextWidth}

\NewEnviron{erbulletedDimFig}[5]{%
\begin{figure}[#2]
\begin{erexample}
\vspace{-0.5cm}
\begin{center}
$
\begin{array}{c m{0.25cm} |  m{#5cm}}
\setlength{\erhalfHt}{#4cm * \real{0.5}}
\raisebox{-\erhalfHt}{
\input{\erpictureFolder/#1}}& & \text{\parbox{#5cm}{\raggedright{\footnotesize{
\begin{enumerate}[(i)]
\BODY
\end{enumerate}}}}} \\
\end{array}
$
\end{center}
\caption{#3}
\label{#1}
\end{erexample}
\end{figure} 
}

%\begin{ernotedModel}{#1 pictureFilename}{#2PictureHeight}{#3PictureWidth}{#4TextWidth}

\NewEnviron{ernotedModel}[4]{%
\begin{center}
$
\begin{array}{m{#3cm} m{1cm} | c m{#4cm}}
\setlength{\erhalfHt}{#2cm * \real{0.5}}
\raisebox{-\erhalfHt}{
\input{\erpictureFolder/#1}}& & & \text{\parbox{#4cm}{\raggedright{\footnotesize{
\BODY
}}}} \\
\end{array}
$
\end{center} 
}

%\begin{ermodelText}{#1 pictureFilename}{#2PictureHeight}{#3PictureWidth}{#4TextWidth}

\NewEnviron{ermodelText}[4]{%
\begin{center}
\begin{tabular}{m{#3cm} m{1cm}  c m{#4cm}}
\setlength{\erhalfHt}{#2cm * \real{0.5}}
\raisebox{-\erhalfHt}{
\input{\erpictureFolder/#1}}& & & \text{\parbox{#4cm}{\raggedright{\small{
\BODY
}}}} \\
\end{tabular}
\end{center} 
}


%\erbulletedModel{#1 pictureFilename}{#2PictureHeight}{#3PictureWidth}{#4TextWidth}

\NewEnviron{erbulletedModel}[4]{%
\begin{center}
$
\begin{array}{m{#3cm} m{1cm} | c m{#4cm}}
\setlength{\erhalfHt}{2cm * \real{0.5}}
\raisebox{-\erhalfHt}{
\input{\erpictureFolder/#1}}& & & \text{\parbox{#4cm}{\raggedright{\footnotesize{
\begin{enumerate}[(i)]
\BODY
\end{enumerate}}}}} \\
\end{array}
$
\end{center} 
}



%\ernotedDimFig{#1 pictureFilename}{#2 figureParam}{#3Caption}{#4PictureHeight}{#5TextWidth}
\NewEnviron{ernotedDimFig}[5]{%
\begin{figure}[#2]
\begin{erexample}
\vspace{-0.5cm}
\begin{center}
$
\begin{array}{c m{0.25cm} | c m{#5cm}}
\setlength{\erhalfHt}{#4cm * \real{0.5}}
\raisebox{-\erhalfHt}{
\input{\erpictureFolder/#1}}& & & \text{\parbox{#5cm}{\raggedright{\footnotesize{
\BODY }}}}\\
\end{array}
$
\end{center}
\caption{#3}
\label{#1}
\end{erexample}
\end{figure} 
}
%\begin{ernotedDimFigPW}{#1 pictureFilename}{#2 figureParam}{#3Caption}{#4PictureHeight}{#5PictureWidth}{#6TextWidth}
\NewEnviron{ernotedDimFigPW}[6]{%
\begin{figure}[#2]
\begin{erexample}
\vspace{-0.5cm}
\begin{center}
$
\begin{array}{>{\centering}m{#5cm} m{0.5cm} | c m{#6cm}}
\setlength{\erhalfHt}{#4cm * \real{0.5}}
\raisebox{-\erhalfHt}{
\centering \input{\erpictureFolder/#1}
}& & & \text{\parbox{#6cm - 0.5cm}{\raggedright{\footnotesize{
\BODY }}}}\\
\end{array}
$ \\
\vspace {0.2cm}
\end{center}
\caption{#3}
\label{#1}
\end{erexample}
\end{figure}
}



\newenvironment{erquote}
{\begin{quote}\itshape}
{\end{quote}}


%
%  erdiag
%
  
%\begin{erdiagram}{#1 height}{#2 width} 
% ....
% ....
%\end{erdiagram}
\newenvironment{erdiagram}[2]
{%\pspicture*(-#1,0)(#2,0)
\pspicture*(0,-#1)(#2,0)
%\psgrid
}
{\endpspicture}

\definecolor{lightyellow}{cmyk}{0,0,0.3,0}

% \eret{#1 x0} {#2 y0} {#3 x1} {#4 y1} {#5 corner radius} {#6 fill}
\newcommand {\eret}[6]
{ 
\ifthenelse{\equal{#6}{1}}
{\psframe[framearc=#5,fillstyle=solid,fillcolor=lightyellow](#1,#2)(#3,#4)}
{\psframe[framearc=#5,fillstyle=solid,fillcolor=white](#1,#2)(#3,#4)}
}

% et top 
\newcommand {\erettop}[4]
{
%\psframe[linestyle=none,linearc=2pt,cornersize=absolute,fillstyle=solid,fillcolor=lightyellow](#1,#2)(#3,#4)
\psline[linearc=2pt,fillstyle=none,fillcolor=lightyellow](#1,#4)(#1,#2)(#3,#2)(#3,#4)
}

% et bottom 
\newcommand {\eretbtm}[4]
{
%\psframe[linestyle=none,linearc=2pt,cornersize=absolute,fillstyle=solid,fillcolor=lightyellow](#1,#2)(#3,#4)
\psline[linearc=2pt,fillstyle=none,fillcolor=lightyellow](#1,#2)(#1,#4)(#3,#4)(#3,#2)
}

% et bottom left
\newcommand {\eretbl}[4]
{
%\psframe[linestyle=none,linearc=2pt,cornersize=absolute,fillstyle=solid,fillcolor=lightyellow](#1,#2)(#3,#4)
\psline[linearc=2pt,fillstyle=none,fillcolor=lightyellow](#1,#4)(#3,#4)(#3,#2)
}

% et middle left
\newcommand {\eretml}[4]
{
%\psframe[linestyle=none,linearc=2pt,cornersize=absolute,fillstyle=solid,fillcolor=lightyellow](#1,#2)(#3,#4)
\psline[linearc=2pt,fillstyle=none,fillcolor=lightyellow](#1,#2)(#3,#2)(#3,#4)(#1,#4)
}

% et top left
\newcommand {\erettl}[4]
{
%\psframe[linestyle=none,linearc=2pt,cornersize=absolute,fillstyle=solid,fillcolor=lightyellow](#1,#2)(#3,#4)
\psline[linearc=2pt,fillstyle=none,fillcolor=lightyellow](#1,#2)(#3,#2)(#3,#4)
}

% et bottom right
\newcommand {\eretbr}[4]
{
%\psframe[linestyle=none,linearc=2pt,cornersize=absolute,fillstyle=solid,fillcolor=lightyellow](#1,#2)(#3,#4)
\psline[linearc=2pt,fillstyle=none,fillcolor=lightyellow](#1,#2)(#1,#4)(#3,#4)
}

% et middle right
\newcommand {\eretmr}[4]
{
%\psframe[linestyle=none,linearc=2pt,cornersize=absolute,fillstyle=solid,fillcolor=lightyellow](#1,#2)(#3,#4)
\psline[linearc=2pt,fillstyle=none,fillcolor=lightyellow](#3,#4)(#1,#4)(#1,#2)(#3,#2)
}

% et top right
\newcommand {\erettr}[4]
{
%\psframe[linestyle=none,linearc=2pt,cornersize=absolute,fillstyle=solid,fillcolor=lightyellow](#1,#2)(#3,#4)
\psline[linearc=2pt,fillstyle=none,fillcolor=lightyellow](#1,#4)(#1,#2)(#3,#2)
}

% \ergrp{#1 x0} {#2 y0} {#3 x1} {#4 y1} {#5 corner radius} {#6 fill}
% #5 corner radius is unused!
\newcommand {\ergrp}[6]
{ 
\ifthenelse{\equal{#6}{1}}
{\psframe[fillstyle=solid,fillcolor=lightgray](#1,#2)(#3,#4)}
{\psframe[fillstyle=solid,fillcolor=white](#1,#2)(#3,#4)}
}

% \eretname {#1 x left of text} {#2 y top of text} {#3 text}
\newcommand {\eretname}[3]
{
%shift down 0.1 for height of text the anchor at baseline (B)
\rput[bl]{0}(0,-0.1){\rput[Bl]{0}(#1,#2){\footnotesize \textit{#3}}}
}

% \errelarm {#1 x0} {#2 y0} {#3 x1} {#4 y1} {#5 ismandatory} {#6 isconstructed}
\newcommand {\errelarm}[6]
{
\ifthenelse{\equal{#6}{1}}
{
%%\psline[linewidth=0.5pt,linearc=.05,linestyle=dashed,dash=6pt 6pt]{-}(#1,#2)(#3,#4)}
\ifthenelse{\equal{#5}{1}}
{\psline[linewidth=1.5pt,linearc=.05,linecolor=lightgray]{-}(#1,#2)(#3,#4)}
{\psline[linewidth=1.5pt,linearc=.05,linecolor=lightgray,linestyle=dashed,dash=2pt 2pt]{-}(#1,#2)(#3,#4)}
}
{
\ifthenelse{\equal{#5}{1}}
{\psline[linewidth=0.9pt,linearc=.05]{-}(#1,#2)(#3,#4)}
{\psline[linewidth=0.9pt,linearc=.05,linestyle=dashed,dash=2pt 2pt]{-}(#1,#2)(#3,#4)}
}
}

% \errelangle {#1 x0} {#2 y0} {#3 x1} {#4 y1} {#5 x2} {#6 y2} {#7 ismandatory} {#8 isocnstructed}
\newcommand {\errelangle}[8]
{
\ifthenelse{\equal{#8}{1}}
{
%\psline[linewidth=0.5pt,linearc=.1,linestyle=dashed,dash=6pt 6pt]{-}(#1,#2)(#3,#4)(#5,#6)}
\ifthenelse{\equal{#7}{1}}
{\psline[linewidth=1.5pt,linearc=.05,linecolor=lightgray]{-}(#1,#2)(#3,#4)(#5,#6)}
{\psline[linewidth=1.5pt,linearc=.1,linecolor=lightgray,linestyle=dashed,dash=2pt 2pt]{-}(#1,#2)(#3,#4)(#5,#6)}
}
{
\ifthenelse{\equal{#7}{1}}
{\psline[linewidth=0.9pt,linearc=.1]{-}(#1,#2)(#3,#4)(#5,#6)}
{\psline[linewidth=0.9pt,linearc=.1,linestyle=dashed,dash=2pt 2pt]{-}(#1,#2)(#3,#4)(#5,#6)}
}
}

% \ercrowfoot {#1 x0} {#2 y0} {#3 x11} {#4 y11} {#5 x12} {#6 y12} {#7 x13} {#8 y13} {#9 isconstructed}
\newcommand {\ercrowfoot}[9]
{
\ifthenelse{\equal{#9}{1}}
{
\psline[linewidth=1.5pt,linearc=.05,linecolor=lightgray]{-}(#1,#2)(#3,#4)
\psline[linewidth=1.5pt,linearc=.05,linecolor=lightgray]{-}(#1,#2)(#5,#6)
\psline[linewidth=1.5pt,linearc=.05,linecolor=lightgray]{-}(#1,#2)(#7,#8)
}{
\psline[linewidth=0.9pt,linearc=.05]{-}(#1,#2)(#3,#4)
\psline[linewidth=0.9pt,linearc=.05]{-}(#1,#2)(#5,#6)
\psline[linewidth=0.9pt,linearc=.05]{-}(#1,#2)(#7,#8)
}
}


% \eridcomprel{#1 x1}{#2 x2}{#3 y1}{#4 ymid}{#5 y2}
\newcommand {\eridcomprel}[5]
{
\psline[linewidth=0.9pt](#1,#3)(#1,#5)
\psline[linewidth=0.9pt](#2,#3)(#2,#5)
\psline[linewidth=0.9pt](#1,#4)(#2,#4)
}

% \eridrefrel{#1 x1}{#2 xmid}{#3 x2}{#4 y1}{#5 y2}
\newcommand {\eridrefrel}[5]
{
\psline[linewidth=0.9pt](#1,#4)(#3,#4)
\psline[linewidth=0.9pt](#1,#5)(#3,#5)
\psline[linewidth=0.9pt](#2,#4)(#2,#5)
}


% \errelname {#1 x} {#2 y} {#3 text}
\newcommand {\errelname}[3]
{
\rput[l]{0}(#1,#2){\textit{#3}}
}
% \errelseq {#1 x} {#2 y}
\newcommand {\erelseq}[2]
{
}
% \erattr {#1 x} {#2 y} {#3 ismandatory}{#4 idenitfying} {#5 text}
\newcommand {\erattr}[5]
{
\ifthenelse{\equal{#3}{1}}
{\rput[l]{0}(#1,#2){{\tiny $\square$} {\footnotesize \textit{\ifthenelse{\equal{#4}{0}}{\underline{#5}}{#5}}}}}
{\rput[l]{0}(#1,#2){\footnotesize $\circ$ \textit{\ifthenelse{\equal{#4}{0}}{\underline{#5}}{#5}}}}
}

%\ifthenelse{\equal{#4}{1}}
% \ertext {#1 x} {#2 y} {#3 text anchor} {#4 text}
%{\rput[l]{0}(#1,#2){\footnotesize $\circ$ \underline{\textit{#5}}}}
\newcommand {\ertext}[4]
{
\rput[B#3]{0}(#1,#2){{\footnotesize #4}}
}
% \erarc {#1 x0} {#2 y0} {#3 x1} {#4 y1} {#5 x2} {#6 y2} {#7 x3} {#8 y3}
\newcommand {\erarc}[8]
{
\psbezier[showpoints=false]{-}(#1,#2) (#3, #4)(#5,#6) (#7, #8)
}

% \erarc {#1 x0} {#2 y0} {#3 x1} {#4 y1} {#5 x2} {#6 y2} {#7 x3} {#8 y3}
\newcommand {\errelseq}[8]
{
\psbezier[showpoints=false]{-}(#1,#2) (#3, #4)(#5,#6) (#7, #8)
}
% \ertrace {#1 trace}   
\newcommand {\ertrace}[1]
{
}
    %beamer aware version
%All these macros are copied from SharedMacros/general.tex which doesnt seem to work with beamer
% Some macros in SharedMacros/general.tex thought to have name clashes with beamer.


\newcommand{\fundep}[3]{#2 \xrightarrow{#1} #3}                                                 
\newcommand{\term}[1]{\textit{#1}} 
\newcommand{\setsuchthat}[2]{\left\{#1 \ \middle|\ #2\right\}}
\newcommand{\set}[1]{\left\{#1\right\}}



\newcommand{\wanton}[1]{#1_1,...#1_n}
\newcommand{\ntuple}[1]{\tuple{\wanton{#1}}}

\newcommand{\xntuple}{\ensuremath{\ntuple{x}}}

% maybe not in general.macros 
\newcommand{\xnset}{\ensuremath{\set{\wanton{x}}}}
%
% othermacros
%

% copied and edited from \idcomp to make stronger linestyle
\newcommand{\addedgebar}{
\ncput[npos=0, nrot=:U]{\psline[linewidth=1.25pt](0.2,-0.1)(0.2,0.1)}
}
\newcommand{\addedgedoublebar}{
\ncput[npos=0, nrot=:U]{\psline[linewidth=1.25pt](0.2,-0.1)(0.2,0.1)}
\ncput[npos=0, nrot=:U]{\psline[linewidth=1.25pt](0.3,-0.1)(0.3,0.1)}
}
\newcommand{\addedgetriplebar}{
\ncput[npos=0, nrot=:U]{\psline[linewidth=1.25pt](0.2,-0.1)(0.2,0.1)}
\ncput[npos=0, nrot=:U]{\psline[linewidth=1.25pt](0.3,-0.1)(0.3,0.1)}
\ncput[npos=0, nrot=:U]{\psline[linewidth=1.25pt](0.4,-0.1)(0.4,0.1)}
}

%\newcommand{\addedgebar}{\ifbars{\ncput[npos=0, nrot=:U]{\psline(0.2,-0.075)(0.2,0.075)}}\fi}

%copied from database literature review
\newcommand{\displaybibentry}[1]
{\begin{framed}
\bibentry{#1}
\end{framed}
}

% used in data tables
\newcommand{\colhead}[1]{\textbf{\textcolor{white}{#1}}}
\definecolor{myblue}{RGB}{71,71,186}
\newcommand{\largeAsterisk}{\mathop{\scalebox{1.5}{\raisebox{-0.2ex}{$\ast$}}}}
\newcommand{\fk}[1]{#1$^{\largeAsterisk}$}
\newcommand{\pk}[1]{\underline{#1}}
\newcommand{\seck}[1]{\dashuline{#1}}  % secondary key
\newcommand{\pkfk}[1]{\underline{#1}$^{\largeAsterisk}$} % primary key that is a foreign key
% \vpad gives vertical padding in a tabular
\newcommand{\vpad}[1]{\multicolumn{#1}{c}{}\\[-0.25cm]}
% used in slides
\newcommand{\outerbullet}{{$\color{blue}{\blacktriangleright}$}\ }% please dont remove final space
\newcommand{\innerbullet}{{\footnotesize $\color{blue}{\blacktriangleright}$}\ }% please dont remove final space
\newcommand{\braceLabel}[3]{\psbrace[ref=lC,braceWidth=1pt,braceWidthInner=3pt,braceWidthOuter=3pt](#2)(#1){#3} }

% words words words
\newcommand{\catMEterm}{category with designated monomorphisms and epimorphisms\ }
\newcommand{\IfSforCwithRCwords}{
If $S$ is a sketch for category \catcw considered as a data specification with requirement $\reqtc$\ }
\newcommand{\IfSforCwithRCwordsvariant}{
If $S$ is a sketch for structured category \catcw and if $S$ is considered as a data specification with requirement $\reqtc$\ }
\newcommand{\IfSforepimonoCwithRCwords}{
If $S$ is a sketch for a category \catcw with designated monomorphisms and epimorphisms considered as a data specification with requirement $\reqtc$\ }
\iffalse
\newcommand{\scmonosketchwording}{
If $S$ is a sketch for such a category
%of a category with finite products and designated monomorphisms and epimorphisms
considered as a data specification
with requirement $\reqtc$\ }
\fi
\newcommand{\spacechar}{\ }
\newcommand{\thirdstructure}{designated monomorphisms and epimorphisms and with finite products}
\newcommand{\IfSforproductepimonoCwithRCwords}{
If $S$ is a sketch for  a category with \thirdstructure \spacechar
%category \catcw with finite products and designated monomorphisms and epimorphisms 
considered as a data specification with requirement $\reqtc$\ }

\newcommand{\goodnesscriteria}[1]{\textbf{Goodness Criteria #1:}}

\newcommand{\goodnessoneA}{
\goodnesscriteria{1A} There ought not to be an edge $e$ in $G$ for which there is an equivalent path $p$ which  does not containing $e$
}
\newcommand{\goodnessoneB}{
\goodnesscriteria{1B} 
There ought not exist $d \in PE$ such that $d \in \overline{PE \setminus d}$
}

\newcommand{\goodnessoneC}{
\goodnesscriteria{1C} \\
There ought not exist $m \in M$ such that $m \in \overline{M \setminus m}$
}
\newcommand{\goodnessoneD}{
\goodnesscriteria{1D} \\
There ought not exist $e \in E$ such that $e \in \overline{E \setminus e}$
}




% From the Mathematical Theory of data paper
\newcommand{\ssfd}[2]{\ensuremath{#1 \morph #2}}  % singleton-singleton
\newcommand{\smfd}[2]{\ensuremath{\ssfd{#1}{\set{#2}}}}  % singleton-many
\newcommand{\msfd}[2]{\ensuremath{\ssfd{\set{#1}}{#2}}}  % many-singleton
\newcommand{\mmfd}[2]{\ensuremath{\msfd{#1}{\set{#2}}}}  % many-many



% All these should find a home in SharedMacros eventually 

% Commands for making a bit of vertical space. used when arrows and particularly labels of arrows
% use spec that is otherwise accounted for.
\newcommand{\seeroomup}[1]{\rule{0.1cm}{#1}}
\newcommand{\seeroomdown}[1]{\rule[-#1]{0.1cm}{0.1cm}}
\newcommand{\roomup}[1]{\rule{0cm}{#1}}
\newcommand{\roomdown}[1]{\rule[-#1]{0cm}{0.1cm}}


% BOX DIAGRAMS
\newcommand{\attr}[1]{#1}
\renewcommand{\attr}[1]{\psframebox[linecolor=red,framearc=.1]{#1}}
\newcommand{\attrtype}[1]{#1}
\renewcommand{\attrtype}[1]{\psframebox[linecolor=blue,framearc=.1]{#1}}
\newcommand{\etype}[1]{#1}
\renewcommand{\etype}[1]{\psframebox[linecolor=red,framearc=.1]{#1}}


\newcommand{\regularizetextheight}{\roomup{0.3cm}\roomdown{0.1cm}}

\newcommand{\unarystructurediagramnodes}[3][]{
\rput[tc](2.4,3){\Rnode{#1A}{\psframebox[framesep=10pt]{\regularizetextheight#2}}} 
\rput[tc](2.4,1){\rnode{#1B}{\psframebox[framesep=10pt]{\regularizetextheight#3}}}           
}

\newcommand{\binarystructurediagramnodes}[4][]{
\rput[tr](4.0,3){\Rnode{#1A}{\psframebox[framesep=10pt]{\regularizetextheight#2}}} 
\rput[tr](2.4,1){\rnode{#1B}{\psframebox[framesep=10pt]{\regularizetextheight#3}}}     
\rput[tr](5.6,1){\rnode{#1C}{\psframebox[framesep=10pt]{\regularizetextheight#4}}}        
}

\newcommand{\triplestructurediagramnodes}[5][]{ 
\rput[tc](2.0,3){\rnode{#1A}{\psframebox[framesep=10pt]{\regularizetextheight#2}}}     
\rput[tc](-1.05,1){\rnode{#1B}{\psframebox[framesep=10pt]{\regularizetextheight#3}}}  
\rput[tc](2.0,1){\rnode{#1C}{\psframebox[framesep=10pt]{\regularizetextheight#4}}}   
\rput[tc](5.0,1){\rnode{#1D}{\psframebox[framesep=10pt]{\regularizetextheight#5}}}   
}

\newcommand{\jacksonbinarydiagram}[3]
{
\pspicture(-0.4,0)(5.7,3)  % lower left is 0,0 upper right is 8,3
%\psgrid
\binarystructurediagramnodes{#1}{#2}{#3}
\rput[tr](2.3,0.9){*}
\rput[tr](5.4,0.9){*}
\ncangle[offsetA=-0.5cm, angleA=-90,angleB=90,armB=0.5cm]{A}{B}
\ncangle[offsetA=0.5cm, angleA=-90,angleB=90,armB=0.5cm]{A}{C}
\endpspicture      
}

\newcommand{\bachmanbinarydiagram}[4][]
{
\pspicture(-0.4,0)(5.7,3)  % lower left is 0,0 upper right is 8,3
%\psgrid
\binarystructurediagramnodes[#1]{#2}{#3}{#4}
\ncline[linewidth=3pt]{->}{#1A}{#1B}
\ncline[linewidth=3pt]{->}{#1A}{#1C}
\endpspicture      
}

\newcommand{\unarystructurediagram}[3][]
{
\pspicture(0.9,0)(3.9,3.5)  
%\psgrid
\unarystructurediagramnodes[#1]{#2}{#3}
\endpspicture      
}

\newcommand{\binarystructurediagram}[4][]
{
\pspicture(-0.4,0)(5.7,3)  
%\psgrid
\binarystructurediagramnodes[#1]{#2}{#3}{#4}
\endpspicture      
}

\newcommand{\triplestructurediagram}[5][]
{
\pspicture(-2.5,0)(6.4,3.5)  
%\psgrid
\triplestructurediagramnodes[#1]{#2}{#3}{#4}{#5}
\endpspicture      
}


\newcommand{\binarynetworkdiagramnodes}[3]{ 
\rput[tr](2.4,3){\rnode{A}{\psframebox[framesep=10pt]{\regularizetextheight#1}}}     
\rput[tr](5.6,3){\rnode{B}{\psframebox[framesep=10pt]{\regularizetextheight#2}}} 
\rput[tr](4.0,1){\Rnode{C}{\psframebox[framesep=10pt]{\regularizetextheight#3}}}       
}

\newcommand{\bachmannetworkdiagram}[3]
{
\pspicture(-0.4,0)(5.7,3)  % lower left is 0,0 upper right is 8,3
%\psgrid
\binarynetworkdiagramnodes{#1}{#2}{#3}
\ncline[linewidth=3pt]{->}{A}{C}
\ncline[linewidth=3pt]{->}{B}{C}
\endpspicture      
}

%craft bachman nwtrok share diagram
\newcommand{\doublebinarynetworkdiagramnodes}[6]{ 
\rput[tc](-2.5,3){\rnode{A}{\psframebox[framesep=10pt]{\regularizetextheight#1}}}  
\rput[tc](2.0,3){\rnode{B}{\psframebox[framesep=10pt]{\regularizetextheight#2}}}    
\rput[tc](-4.1,1){\rnode{C}{\psframebox[framesep=10pt]{\regularizetextheight#3}}} 
\rput[tc](-1.05,1){\rnode{D}{\psframebox[framesep=10pt]{\regularizetextheight#4}}}  
\rput[tc](2.0,1){\rnode{E}{\psframebox[framesep=10pt]{\regularizetextheight#5}}}   
\rput[tc](5.0,1){\rnode{F}{\psframebox[framesep=10pt]{\regularizetextheight#6}}}   
}

\newcommand{\doublebachmannetworkdiagram}[6]
{
\pspicture(-5.6,0)(6.5,3)  % lower left is 0,0 upper right is 8,3
%\psgrid
\doublebinarynetworkdiagramnodes{#1}{#2}{#3}{#4}{#5}{#6}
\ncline[linewidth=3pt]{->}{A}{C}
\ncline[linewidth=3pt]{->}{A}{D}
\ncline[linewidth=3pt]{->}{B}{D}
\ncline[linewidth=3pt]{->}{B}{E}
\ncline[linewidth=3pt]{->}{B}{F}
\endpspicture      
}

\newcommand{\doublecategorynetworkdiagram}[6]
{
\pspicture(-5.6,0)(6.5,3)  % lower left is 0,0 upper right is 8,3
%\psgrid
\doublebinarynetworkdiagramnodes{#1}{#2}{#3}{#4}{#5}{#6}
\ncarr{C}{A}
\ncarr{D}{A}
\ncarr{D}{B}
\ncarr{E}{B}
\ncarr{F}{B}
\endpspicture      
}

\newcommand{\mixedcategorynetworkdiagram}[6]
{
\pspicture(-5.6,0)(6.5,3)  % lower left is 0,0 upper right is 8,3
%\psgrid
\doublebinarynetworkdiagramnodes{#1}{#2}{#3}{#4}{#5}{#6}
\ncline[linewidth=2.5pt]{->}{C}{A}
\ncline[linewidth=2.5pt]{->}{D}{A}
\ncarr{D}{B}
\ncline[linewidth=2.5pt]{->}{E}{B}
\ncline[linewidth=2.5pt]{->}{F}{B}
\endpspicture      
}

\newcommand{\contextualcategoryblockstyleexamplekernel}[6]{
\pspicture(-5.6,0)(6.5,3)  % lower left is 0,0 upper right is 8,3
%\psgrid
\doublebinarynetworkdiagramnodes{#1}{#2}{#3}{#4}{#5}{#6}
\ncsar{C}{A}
\ncsar{E}{B}
\ncsar{F}{B}
\endpspicture
}

\newcommand{\contextualcategorynetworkdiagram}[6]
{
\contextualcategoryblockstyleexamplekernel{#1}{#2}{#3}{#4}{#5}{#6}
\ncsar{D}{A}
\ncarr{D}{B}
}

\newcommand{\contextualcategorynetworkdiagramreorganised}[6]
{
\contextualcategoryblockstyleexamplekernel{#1}{#2}{#3}{#4}{#5}{#6}
\ncarr{D}{A}
\ncsar{D}{B}
}


\newcommand{\contextualcategorynetworkdiagramtopologised}[6]
{
\begin{tabular}{c c c}
\scalebox{0.9}{\binarystructurediagram[left]{compound\kern0.1cm}{alias \kern1.2cm}{occurence}}
&&
\scalebox{0.9}{\binarystructurediagram[right]{element\kern0.4cm}{valency \kern0.8cm}{allotrope\kern0.3cm}}
\end{tabular}
\ncangle[offsetA=0.15cm, angleA=0,offsetB=-0.25cm, angleB=180, armB=2.5cm]{->}{leftC}{rightA}
\ncsar{leftB}{leftA}
\ncsar{leftC}{leftA}
\ncsar{rightB}{rightA}
\ncsar{rightC}{rightA}
}

\newcommand{\contextualcategorynetworkdiagramreorganisedtopologised}[6]
{
\begin{tabular}{c c c}
\scalebox{0.9}{\unarystructurediagram[left]{compound\kern0.1cm}{alias \kern1.2cm}}
&&
\scalebox{0.9}{\triplestructurediagram[right]{element\kern0.4cm}{occurence}{valency \kern0.8cm}{allotrope\kern0.3cm}}
\end{tabular}
\ncangle[offsetA=0.15cm, angleA=180,offsetB=-0.25cm, angleB=0, armB=0.9cm]{->}{rightB}{leftA}
\ncsar{leftB}{leftA}
\ncsar{leftC}{leftA}
\ncsar{rightB}{rightA}
\ncsar{rightC}{rightA}
\ncsar{rightD}{rightA}
}

\iffalse
\newcommand{\contextualcategorynetworkdiagramreorganised}[6]
{
\pspicture(-5.6,0)(6.5,3)  % lower left is 0,0 upper right is 8,3
%\psgrid
\doublebinarynetworkdiagramnodes{#1}{#2}{#3}{#4}{#5}{#6}
\ncsar{C}{A}
\ncarr{D}{A}
\ncsar{D}{B} 
\ncsar{E}{B}
\ncsar{F}{B}
\endpspicture      
}
\fi

% Category DIAGRAMS START HERE


\newcommand{\factorisationfdiagram}{
    $
    \begin{array}{c p{1cm} c p{1.0cm} c}
    \Rnode{a}{a}&&\Rnode{Imf}{Im(f)}&&\Rnode{b}{b}
    \end{array}
    \begin{arrows}
    \ncline{->>}{a}{Imf}\alabel{f_e}
    \ncarr{Imf}{b}\alabel{f_m}\idcomp
    \end{arrows}
    $
}
\newcommand{\nakedbinarysourcediagram}[5]{
\begin{array}{c p{0.5cm} c}
             &&   \Rnode{b}{#2}\\[0.01cm]
\Rnode{a}{#1} &&               \\[0.01cm] 
             &&   \Rnode{c}{#3}
\end{array} 
\begin{arrows}
\ncarr{a}{b}
\alabel{#4}
\ncarr{a}{c}
\blabel{#5}
\end{arrows}
}

\newcommand{\binarysourcediagram}[5]{$\nakedbinarysourcediagram{#1}{#2}{#3}{#4}{#5}$}
\newcommand{\fgsourcediagram}{\binarysourcediagram{a}{b}{c}{f}{g}}

%  binary source diagram with arrows pointing SE and SW
% nakedSWSEsourcediagram{prefix}{a}{b}{c}{f}{g}
\newcommand{\nakedSWSEsourcediagram}[6]{
\begin{array}{c c c}
              & \Rnode{#1a}{#2} &               \\[1.0cm] 
\Rnode{#1b}{#3} &               &\Rnode{#1c}{#4}
\end{array} 
\begin{arrows}
\ncarr{#1a}{#1b}
\alabel{#5}
\ncarr{#1a}{#1c}
\blabel{#6}
\end{arrows}
}


%  binary sink diagram with arrows pointing SE and SW
\newcommand{\nakedNWNEsinkdiagram}[5]{
\begin{array}{c c c}
              & \Rnode{a}{#1} &               \\[0.5cm] 
\Rnode{b}{#2} &               &\Rnode{c}{#3}
\end{array} 
\begin{arrows}
\ncarr{b}{a}
\alabel{#4}
\ncarr{c}{a}
\blabel{#5}
\end{arrows}
}

\newcommand{\simpleunaryfdrepresentationdiagram}[6]{
$
\nakedbinarysourcediagram{#1}{#2}{#3}{#4}{#5}
\begin{arrows}
\ncarr{b}{c}
\alabel{#6}
\end{arrows}
$
}

\newcommand{\unaryfdrepresentationdiagram}[8]{
$
\begin{array}{c p{0.2cm} c}
\nakedbinarysourcediagram{#1}{#2}{#3}{#4}{#5}&& \Rnode{d}{#6}
\end{array}
\begin{arrows}
\ncarr{d}{b}
\idcomp
\blabel{#7}
\ncarr{d}{c}
\alabel{#8}
\end{arrows}
$
}

\newcommand{\unaryfdrepresentationmappeddiagram}[8]{
$
\begin{array}{c p{0.2cm} c}
\nakedbinarysourcediagram{D(#1)}{D(#2)}{D(#3)}{D(#4)}{D(#5)}&& \Rnode{d}{D(#6)}
\end{array}
\begin{arrows}
\ncarr{b}{d}
\alabel{D(#7)^-1}
\ncarr{d}{c}
\alabel{D(#8)}
\end{arrows}
$
}

\newcommand{\commutativetrianglediagram}[6]{
$
\begin{array}{c p{0.4cm} c p{0.4cm} c}
              && \Rnode{b}{#2}  &&                 \\[0.6cm]
\Rnode{a}{#1} &&                && \Rnode{c}{#3}  
\end{array}
\begin{arrows}
\ncarr{a}{b}
\alabel{#4}
\ncarr{b}{c}
\alabel{#5}
\ncarr{a}{c}
\blabel{#6}
\end{arrows}
$
}

\newcommand{\commutativetrianglediagrammutant}[6]{
$
\begin{array}{c  c  c}
              & \Rnode{b}{#2}  &                 \\[0.85cm]
\Rnode{a}{#1} &                & \Rnode{c}{#3}  
\end{array}
\begin{arrows}
\ncarr{a}{b}
\alabel{#4}[0.15]
\ncarr{b}{c}
\alabel{#5}[0.6]
\ncarr{a}{c}
\blabel{#6}
\end{arrows}
$
}

\newcommand{\epimonosplitdiagram}[3]{
\commutativetrianglediagram{#1}{img(#3)}{#2}{#3_e}{#3_m}{#3}   
}


\iffalse %saved
\begin{array}{c p{2.0cm} c }                
               &&  \Rnode{b1}{#3_1}    \\ [0.75cm]
               &&  \Rnode{b2}{#3_2}    \\ [0.5cm]
\Rnode{a}{#2}  &&                      \\ [-0.5cm]
               &&       \vdots         \\ [0.85cm]
               &&  \Rnode{bn}{#3_{#1}}  
\end{array}
\fi

%nakedmultisourceobjects{n}{a}{b}
\newcommand{\nakedmultisourceobjects}[3]{
\begin{array}{c p{2.0cm} c }
\Rnode{a}{#2}   &&
\begin{array}{c }                
\Rnode{b1}{#3_1}   \\ [0.75cm]
\Rnode{b2}{#3_2}   \\ [0.25cm]
\vdots             \\ [0.35cm]
\Rnode{bn}{#3_{#1}}  
\end{array}
\end{array}
}

% \nakedmultisourcediagram{n}{a}{b}{f}
\newcommand{\nakedmultisourcediagram}[4]{
\nakedmultisourceobjects{#1}{#2}{#3}
\begin{arrows}
\ncarr{a}{b1}
\alabel{#4_1}[0.5]
\ncarr{a}{b2}
\alabel{#4_2}[0.5][-1]
\ncarr{a}{bn}
\blabel{#4_{#1}}[0.5][-1]
\end{arrows}
}

% \nakedmultisourcepathdiagram{n}{a}{b}{f}
\newcommand{\nakedmultisourcepathdiagram}[4]{
\nakedmultisourceobjects{#1}{#2}{#3}{#4}
\begin{arrows}
\simplepath{a}{b1}
\alabel{#4_1}[0.5]
\simplepath{a}{b2}
\alabel{#4_2}[0.5][-1]
\simplepath{a}{bn}
\blabel{#4_{#1}}[0.5][-1]
\end{arrows}
}


\newcommand{\multisourcediagram}[4]{$\nakedmultisourcediagram{#1}{#2}{#3}{#4}$}
\newcommand{\multisourcepathdiagram}[4]{$\nakedmultisourcepathdiagram{#1}{#2}{#3}{#4}$}


% \monosourcedefinitiondiagram{x}{g}{h}{n}{a}{b}{f}
\newcommand{\monosourcedefinitiondiagram}[7]{
$
\begin{array}{c p{1.5cm} c}
\Rnode{x}{#1} && \nakedmultisourcediagram{#4}{#5}{#6}{#7}
\end{array}
\begin{arrows}
\parallelarrows{x}{a}{#2}{#3}
\end{arrows}
$
}

%\multisourcenplusonediagram{n}{a}{b}{f}{c}{g}
\newcommand{\multisourcenplusonediagram}[6]{
$
\begin{array}{c p{2.0cm} c }
\Rnode{a}{#2}   &&
\begin{array}{c }                
\Rnode{b1}{#3_1}   \\ [0.55cm]
\Rnode{b2}{#3_2}   \\ 
\vdots             \\ 
\Rnode{bn}{#3_{#1}} \\ [0.65cm] 
\Rnode{c}{#5} 
\end{array}
\end{array}
\begin{arrows}
\ncarr{a}{b1}
\alabel{#4_1}[0.6][1]
\ncarr{a}{b2}
\alabel{#4_2}[0.6][0]
\ncarr{a}{bn}
\blabel{#4_{#1}}[0.6][0]
\ncarr{a}{c}\blabel{#6}[0.6][0]
\end{arrows}
$
}

\newcommand{\fghfactordiagram}[6]
{
\binarysourcediagram{#1}{#2\roomup{0.5cm}}{#3}{#4}{#5}
\begin{arrows}
\ncarr{b}{c}
\alabel{#6}
\end{arrows}
}

\newcommand{\fghpartialfactordiagram}[6]{
\binarysourcediagram{#1}{#2\roomup{0.5cm}}{#3}{#4}{#5}
\begin{arrows}
\ncdarr{b}{c} %dashed arrow
\alabel{#6}
\end{arrows}
}

\newcommand{\fnsourceqnsource}{
$
\begin{array}{c p{0.25cm} c  p{0.25cm} c }
             &&   \Rnode{b1}{b_1} &&              \\[0.4cm]
\Rnode{a}{a} &&                   && \Rnode{c}{c} \\[0.4cm]
             &&   \Rnode{bn}{b_n} &&              
\end{array} 
\begin{arrows}
\ncarr{a}{b1}
\alabel{f_1}
\ncarr{c}{b1}
\blabel{q_1} 
\ncarr{a}{bn}
\blabel{f_n}
\ncarr{c}{bn}
\alabel{q_n}
\end{arrows}
$   
}

\newcommand{\parallelarrows}[4]{
\ncarc[nodesep=2pt,arcangle=10,offset=2pt]{->}{#1}{#2}
\alabel{#3}
\ncarc[nodesep=2pt,arcangle=-10,offset=-2pt]{->}{#1}{#2}
\blabel{#4}
}

\newcommand{\paralleldiagram}[4]{
$
\rule[-0.3cm]{0pt}{0.9cm} %to add vertical space of diagram -- based on lowering diagram 0.3cm and heght 0.9cm
                            % change thickness from 0pt to 1 pt to debug
\begin{array}{c p{0.5cm} c}
\Rnode{a}{#1}       &&   \Rnode{b}{#2}
\end{array} 
\begin{arrows}
\parallelarrows{a}{b}{#3}{#4}
\end{arrows}
$
}

\newcommand{\fgparalleldiagram}{
 $
\rule[-0.3cm]{0pt}{0.9cm} %to add vertical space of diagram -- based on lowering diagram 0.3cm and heght 0.9cm
                            % change thickness from 0pt to 1 pt to debug
\begin{array}{c p{0.5cm} c  }
 \Rnode{a}{a}            &&   \Rnode{b}{b}
\end{array} 
\begin{arrows}
\parallelarrows{a}{b}{f}{g}
\end{arrows}
$  
}

\newcommand{\fgcomposablediagram}[5]{
\mbox{
\roomup{0.45cm}
$
\begin{array}{c p{0.5cm}cp{0.5cm}c}
\Rnode{x}{#1}&&\Rnode{y}{#2}&&\Rnode{z}{#3}
\end{array}
\begin{arrows}
\ncarr{x}{y}
\alabel{#4}
\ncarr{y}{z}
\alabel{#5}
\end{arrows}
$    
}
}



% copied from MToD paper (preamble.tex)
\newcommand{\simplepath}[2]{
\ncline[linestyle=none,linewidth=0.1pt]{#1}{#2}   %was linestyle=dotted
\ncput[npos=0.05]{\pnode{dot#21}}
\ncput[npos=0.27]{\dotnode[dotsize=1pt]{dot#22}}
\ncput[npos=0.50]{\dotnode[dotsize=1pt]{dot#23}}
\ncput[npos=0.80]{\dotnode[dotsize=1pt]{dot#24}}
\ncput[npos=0.975]{\pnode{dot#25}}
\ncline[nodesep=2pt]{->}{dot#21}{dot#22}
\ncline[nodesep=2pt]{->}{dot#22}{dot#23}
\ncline[nodesep=2pt]{->}{dot#24}{dot#25}
\ncline[linestyle=dotted,nodesep=8pt]{dot#23}{dot#24} %was 10pt
}

\renewcommand{\erpictureFolder}[0]{../../SharedPictures}
\setcounter{equation}{0}



\title[John Cartmell]{Mathematical Theory of Data}
%% Which is to say types as they are used in practice in software development and as represented in theory in categories and in syntactic type theories.
%% There is also a subplot concerning representation of context which certain types depend on -- again represented in practice and in theory. 
\author{John Cartmell}
%\institute{ad otium}
\date{Jan 25, 2019}
\bibliographystyle{plainnat}
\usepackage{bibentry}
\nobibliography*

\begin{document}

\begin{frame}
\titlepage
\end{frame}

\iffalse
\begin{frame}{Test}
\begin{description}	[longest label] 
\item<1->[short] Some text. 
\item<2->[longest label] Some text. 
\item<3->[long label] Some text. 
\end{description}
\end{frame}
\fi

\section{Mathematical Theory of Data}

\subsection{Introduction}

\begin{frame}{Introduction}
I wish to show that
\begin{itemize}
\item we can genericise database normal form criteria into abstract logical terms,
\item achieve generic goodness criteria that can be applied to all data specifications,
 \item that the classic relational database normal form criteria (2NF, 3NF, BCNF, INC-NF, 4NF, 5NF)  are then consequences of these generic goodness criteria.
\end{itemize}
\end{frame}

\begin{frame}{View}
A data specification  
\begin{itemize}
\item is a  theory (of what is)
\end{itemize}
\medskip
A data specification method 
\begin{itemize}
\item is a method for expressing such a  theory
\item unequivocally it enables definitions of types and certain relationships between these types
\item types are equally types of data and types of real world entity
\end{itemize}
More precisely, 
\begin{itemize}
\item data specification are \underline{presentations} of theories of what is,
\item choice of primitives in a given presentation is choice of which data to be stored or communicated.
\end{itemize}
\end{frame}



\begin{frame}{Goodness Criteria}
I will define two types of goodness criteria
\begin{itemize}
    \item Goodness Criteria of Type 1 -- absence of redundancy in presentation.
    \item Ensures absence of redundancy in data and in data management logic.
    \item Spelt out in more detail in criteria 1A, 1B and so on
    \item  Goodness Criteria of Type 2 -- the theory be the tightest fit to the facts 
    \item Two ways of expressing this. 
    \begin{itemize}
        \item Criteria 2 is that the theory is maximally constrained.
        \item Criteria 2A, 2B etc.  that it be logically complete in some sense.
    \end{itemize}
\end{itemize}
\end{frame}

\begin{frame}{Data Specification as Category}
\begin{itemize}
 \item Not surprising that a data specification can represented as a category or as a sketch of a category
\pause \item
What types of things are there and how are they related? 
\begin{itemize}
\item Data specifications provide the answer to this question in the context of a software development. 
\item Types theories provide the answer in the context of mathematics. 
\item Category theory abstracts across both these domains.
\end{itemize}
\pause \item Nor is it surpising if data specifications can be seen in terms
of contextual categories or a sketches for a contextual categories
because as categories model types so contextual categories model types that vary
and  our instinct for types and types that vary comes from the real world not from mathematics.
\end{itemize}
\end{frame}


\begin{frame}{Overview}
\begin{center}
\begin{tabular}{p{12cm}}
\begin{itemize}
    \item data specifications 
    \begin{itemize}
        \item as sketches of structured categories of some kind
        \item data instances as certain structure preserving functors to the category of finite sets $\Fin$
    \end{itemize}
    \item database specifications
    \begin{itemize}
         \item category has designated mono sources for some of its objects 
    \end{itemize}
    \item relational database specifications charcterised by
    \begin{itemize}
         \item no use of containment or nesting
         \item no  hierarchical organisation -- said to be flat
         \item no use of pointers 
         \item instead uses foreign keys to represent relationships in the data
    \end{itemize}
\end{itemize} 
\end{tabular}
\end{center}
\end{frame}

\begin{frame}{Overview}
\shrinkbox{0.4}{
\begin{pspicture}(-6,-6)(9,9)
 \psgrid
 \psset{fillstyle=solid,opacity=0.5}
 \pscircle[fillcolor=lightyellow]{1.5}
 \psRing[fillcolor=white]{1.5}{3.5}
 \psRing[fillcolor=lightyellow]{3.5}{5}

 \end{pspicture}
}
\end{frame}

\begin{frame}{Normal Forms}
\begin{center}
\begin{tabular}{p{12cm}}
\begin{itemize}
\item data specifications 
\begin{itemize}
    \item as sketches of structured categories of some kind
    \item data instances as certain structure preserving functors to the category of finite sets $\Fin$
\end{itemize}
\item database specifications
\begin{itemize}
     \item category has designated mono sources for some of its objects 
\end{itemize}
\end{itemize} \\
\hdashline
\begin{itemize}
\item relational database normal form criteria  
\begin{itemize}
    \item first normal form (1NF)
    \item 2nd normal form (2NF)
    \item 3rd normal form (3NF)
    \item Boyce-Codd normal form (BCNF)
    \item inclusion dependency normal form (INC-NF)
    \item 4th normal form (4NF)
    \item projection-join normal form (5NF)
\end{itemize}
\end{itemize}
\end{tabular}
\end{center}
\end{frame}

\iffalse
\begin{frame}{Levels of Data Specification}
\begin{itemize}
\item logical   -- a sketch for a category of some kind
\item structural -- a sketch for some kind of category with distinguished morphisms indicating structural relationships
\item representational -- structural PLUS representational indicators for non-structural relationships
\item technological    -- IDL, SQL, XML 
\end{itemize}
\end{frame}
\fi

\newcommand{\bigdownarrow}
{
\scalebox{0.3}
{
\begin{pspicture}(3,3.5) 
%\psgrid
%\psset{doublesep=2cm} 
\psBigArrow[fillstyle=solid, fillcolor=blue!30,linecolor=blue](2.0,3)(2.0,0)
\end{pspicture}
}
}
\begin{frame}{Levels of Data Specification}
\begin{center}
\begin{tabular}{c l}
                              & \raisebox{0cm}{\parbox{5cm}{sketch of category of some kind}}\\
\cline{1-1}
\multicolumn{1}{|c|}{logical} & \\
\cline{1-1}
\multicolumn{1}{c}{\bigdownarrow} & \raisebox{0.5cm}{\parbox{5cm}{distinguish morphisms/relationships represented in data by containment}} \\
\cline{1-1}
\multicolumn{1}{|c|}{structural} &\\
\cline{1-1}
\multicolumn{1}{c}{\bigdownarrow} & \raisebox{0.5cm}{\parbox{5cm}{add edges for foreign keys representing non-containment relationships and add path equivalences that define them}} \\
\cline{1-1}
\multicolumn{1}{|c|}{representational} & \\
\cline{1-1}
\multicolumn{1}{c}{\bigdownarrow} & \raisebox{0.5cm}{choice of technology} \\
\cline{1-1}
\multicolumn{1}{|c|}{technological} & \raisebox{0cm}{\parbox{5cm}{IDL, XML, SQL}}\\
\cline{1-1}
\end{tabular}
\end{center}
\end{frame}


\begin{frame}{Classic Relational Normal Form Criteria}
\begin{itemize}
    \item database normal forms are goodness criteria (GC) based on software engineering principles
    \item relational database normal form criteria  
    \begin{itemize}
        \item first normal form (1NF)
        \item 2nd normal form (2NF)
        \item 3rd normal form (3NF)
        \item Boyce-Codd normal form (BCNF)
        \item inclusion dependency normal form (INC-NF)
        \item 4th normal form (4NF)
        \item projection-join normal form (5NF)
    \end{itemize}
\end{itemize}
\end{frame}


\iffalse
\begin{frame}{Methods of Data Specification}
\begin{itemize}
	\item schema of relational database,
	\item structure described by Carnegie-Mellon IDL,
	\item schema of nested relational database,
	\item message structure described by Google protocol buffer IDL,
	\item XML schema language,
	\item ER script.
\end{itemize}
\end{frame}
\fi

\begin{frame}{Data Specifications}
Two kinds of types in play
\begin{itemize}
\item  the \textit{definienda} -- types all of whose instances are \term{particulars}
\begin{itemize}
\item employee, department, student, account, product, order, shipment, delivery, flight, booking and so on
\item molecular structure, atom, bond, element, isotope, reaction, metabolite, mass trace, chromatogram, peak
\item table, column, primary key, foreign key
\item node and edge. 
\end{itemize}
\pause 
\item  the \textit{definiens}  -- types all of whose instances are \term{universals}
\begin{itemize}
       \item string, integer, float, boolean and so on
\end{itemize}
\end{itemize}
\pause
\begin{itemize}
\item in ER modelling 
\begin{itemize}
\item the \textit{definienda} are called \textit{entity types}
\item the \textit{definiens} are called \textit{attribute types} or \textit{domains}.
\end{itemize}
\end{itemize}
\end{frame}


\begin{frame}{Data Specifications}
A data specification is a sketch of a category with some additional structure:
\begin{itemize}
\item that it is a \term{sketch} is crucial because it is only nodes and edges of the sketch for which data is stored and/or communicated, 
\item that there are commutative diagrams is crucial to construction of physical 
specifications from logical specifications.
\item that the category had additional structure is significant:
\begin{itemize}
\item so that we can give account of database normal forms 
(BCNF, 3NF, 4NF and 5NF),
\item so that we can allow for missing data as represented by NULL values, 
\item so that we can distinguish structural from non-structural relationships to describe structure nesting and thereby hierarchical data,
\item so that types of universals can be distinguished from types of particulars.
\end{itemize}
\end{itemize}
\end{frame}



\iffalse
\begin{frame}{Database Normal Forms}
\begin{itemize}
\item presentations (sketches) should be minimal and avoid redundancy:
\item how to make this precise?
\item in case of relational data model leads to 
   \begin{itemize}
     \item third normal form (3NF)
     \item Boyce-Codd normal form (BCNF)
     \item fourth normal form (4NF)
     \item fifth normal form (5NF)
   \end{itemize}
\end{itemize}
\end{frame}
\fi

\begin{frame}{Relational Database Theory}
\begin{itemize}
\item classic relational database normal form definitions ({\scriptsize 3NF, EKNF, BCNF, 4NF,5NF, INC-NF}) can be abstracted  into a general logical framework

\item such normal forms  examine the fit of a sketch/theory (database schema) to an intended usage

\item we can assume that the intended usage is represented by a full-subcategory of the category $Fun(S,\cat{FinSet})$

\item in such a situation the classic normal forms address the question can the sketch/theory $S$ be improved by addition or removal of morphisms and/or commutative diagrams and/or limit cones.
\item normalisation has dual goal of obtaining as complete a theory as possible and of eliminating redundancy from the sketch.  
\end{itemize}
\end{frame}

\begin{frame}{Normal Forms}
\begin{itemize}
\item IN-NF -- Ling and Goh -- there are no redundant attributes except if absolutely necessary in order to specify a mono source
\end{itemize}
\end{frame}

\iffalse %moved into goodness criteria section and reworded
\begin{frame}{Normalisation}
\begin{definition}
{ \footnotesize
If $T$ is a theory and $W \subset |Mod(T,FinSet)|$ is an intended usage then an interpretation (theory morphism) $I: T \morph T'$ is an improvement of $T$ wrt $W$ iff 
$Mod(I,Finset): Mod(U',Finset) \morph Mod(U,Finset)$ is injective but not surjective
and $W \subseteq img(Mod(I,Finset))$.
i.e. for all models $U \in W$ there exists $U' \in Mod(T,Finset)$ such that $I \circ U'=U$
$
\begin{array} {c p{2cm} c}
\Rnode{T}{T} && \\ [0.25cm]
             && \Rnode{finset}{Finset} \\ [0.25cm]
\Rnode{Tp}{T'}  
\end{array}
$
\ncarr {T}{finset}
\alabel{U}
\ncarr{T}{Tp}
\blabel{I}
\ncarr{Tp}{finset}
\blabel{U'} 
}
\end{definition}

\begin{definition}
If a theory $T$ has no improvement wrt to an intended usage $W$ then $T$ is said to be \textit{optimally formulated} wrt $W$.
\end{definition}
\end{frame}


\begin{frame}{Propositions}
\begin{itemize}
\item If a relational schema $R$ can be normalised to $R'$ then the associated theory $T$ of $R$ can be improved to the associated thery $T'$ of $R'$.

\item If a relational database schema is in normal form then its associated theory is optimally formulated.
\end{itemize}
\end{frame} 
\fi

\begin{frame}{Defining Candidate Keys and/or Identifying Relationships in an EA sketch.  }
\begin{itemize}
\item concept of \textit{candidate keys} used in relational database normal form definitions {\scriptsize (3NF, EKNF, BCNF)}
\item in ER model talk about \textit{identifying} families of relationships
\item in category theory such a key or a family of relationships is a mono source i.e. a to jointly monic family of morphisms
\item mono sources and hence candidate keys can be defined as limit cones
\item more than 99.99 percent of entity modelling uses just mono sources and no other limits
\end{itemize}
\end{frame}

\begin{frame}{Additional Structure}
\resizebox{11.3cm}{!}{
\newcommand{\featurelist}{\begin{tabular}{|l|l l|}
\hline 
\multirow{11}{1.5cm}{category}
                & finitary property        & \\
\cline{2-3}
                & pu-partition             & \\
\cline{2-3}
                & \multirow{2}{3.5cm}{mono-sources}  & \multicolumn{1}{|l|}{cannonical monos}  \\
\cline{3-3}
                &                                    & \multicolumn{1}{|l|}{epi-mono factorisation}   \\
\cline{2-3}
                & finite products          & \\
\cline{2-3} 
                & finite limits            & \\
\cline{2-3}
                & restrictions             & \\
\cline{2-3}
                & \multirow{2}{3.5cm}{distinguished morphisms} & \multicolumn{1}{|l|}{hierarchical}      \\
\cline{3-3}
                &                                             &  \multicolumn{1}{|l|}{non-hierarchical} \\
\cline{3-3}
                &                                             &  \multicolumn{1}{|l|}{pullbacks} \\
\cline{2-3}
                & finite coproducts                           &                                   \\
\hline                
\end{tabular}}
\featurelist
}
\end{frame}

\begin{frame}{How to proceed}
\begin{itemize}
\pause \item I will
\begin{itemize}
   \item give an example of nested tables of data
   \item describe relational model  and other data models 
   \pause \item touch on my favourite -- ER modelling and ER script
   \pause \item describe Boyce-Codd normal form (BCNF)
   \pause \item GCs for sketches of categories as data specifications 
   \pause \item GCs for sketches of categories with designated monos and epis
   \pause \item GCs for sketches of categories with designated monos and epis and with finite products
  \pause \item with some additional assumptions prove BCNF
\end{itemize}
\end{itemize}
\end{frame}



\subsection{Literature}



\begin{frame}{E.F. Codd and the Relational Model of Data}
The following dedication is writ large in Codd's 1990 book:
\begin{quote}
To fellow pilots and aircrew
in the Royal Air Force
during World War II
and the dons at Oxford.
These people were the source of my determination to
fight for what I believed was right during the ten or
more years in which government, industry, and
commerce were strongly opposed to the relational
approach to database management.
\end{quote}
By 2020 Oracle Corporation were the world's second largest software company.
\end{frame}
\begin{frame}{E.F. Codd and the Relational Model of Data}
\displaybibentry{CoddBook1990}
\begin{quote}
The relational model is solidly based on two parts of mathematics: first-
order predicate logic and the theory of relations.
\end{quote} 
\end{frame}

\begin{frame}{E.F. Codd 1970}
\displaybibentry{Codd1970}
Codd introduces the relational model of data and introduces the idea of normal form.
\end{frame}

\begin{frame}{E.F. Codd 1971}
\displaybibentry{Codd1971} 
In 1971, the terms `functional dependency' and  `third normal form' are introduced 
in an IBM techical report published in a now out of print book (also unavailable on Amazon).
\end{frame}

\begin{frame}{Fagin 1977,1979}
\displaybibentry{Fagin1977} 
Fagin introduces `fourth normal form' (4NF) and `multivalued dependencies'.

\displaybibentry{Fagin1979} 
Fagin introduced `projection-join normal form'. This is also known as fifth normal form (5NF).
\end{frame}

\begin{frame}{Zaniola 1962}
\displaybibentry{zaniolo1982}
\begin{itemize}
\item resume of 3NF and BCNF
\item the representation principle (by example)
\item new normal form -- elementary key normal form (EKNF)
\item Bernsteins algorthm, which is known to produce schemas in 3NF, does actually produce EKNF
\end{itemize}
\end{frame}

\begin{frame}{Cartmell 1986}
\displaybibentry{CartmellNetworkDataModel}
\end{frame}

\begin{frame}{Cartmell \& Alderson 1997}
\displaybibentry{CartmellScopePaper}
\end{frame}

\begin{frame}{Categorical Data Specifications 1995}
\displaybibentry{piessens1995} 
Defines data specifications and also MD-sketches.
\end{frame}

\begin{frame}{Diskin and Cadish 1996}
\displaybibentry{Diskin1996DatabaseDesign}
\begin{quote}
...it seems for us that
the current situation in relating category theory with DB theory and practice is very similar to the 16th
century interaction of differential and integral calculi, on one hand, with mechanics and engineering on
the other. Thus, being excited with our discovery, and having in mind the distinctive features of our time
... we have decided to begin propagating our observations with a declarative document in
a manner of a brief manifesto.
\end{quote}
\end{frame}

\begin{frame}{Johnson 1993}
\displaybibentry{Johnson93}
\begin{itemize} \footnotesize
\item describe a category equivalent to an ER schema which they loosely describe as the classifying category of the schema
\item argue that it is/needs to be a lextensive category
\item those objects of the category which are coproducts of morphisms from terminal object represent the attribute type (though they use the term attribute here).
\end{itemize}
\end{frame}

\begin{frame}{Johnson 2001}
\displaybibentry{Johnson2001}

\begin{itemize} \footnotesize{}
\pause \item developed further in  \cite{Johnson2002ERA}
\pause \item The term `EA-sketch' introduced for a sketch representing an ER model 
\end{itemize}
\end{frame}

\begin{frame}{Definitions - Johnstone et al.}

%\begin{definition}{Johnstone et al}
An \textit{EA sketch} is a sketch $\tuple{G,D,L,C}$ where $G$ is a directed graph, $D$ a set of diagrams in $G$, $L$ a set of finite cones and
$C$ a set of finite discrete cocones.
%\end{definition}

If $S$ is an EA-sketch then the theory of $S$ is the lextensive category generated by $S$.

If $S$ is an EA sketch then a model of $S$ is a functor to the category of finite sets preserving finite limits and coproducts.
The category of models is denoted $Mod(S,\cat{FinSet})$.
\end{frame}

\begin{frame}{\cite{Johnson2002ERA}}
\bibentry{Johnson2002ERA}

``An EA sketch E=(G,D,L,C) is a sketch with only finite cones and finite discrete cocones and with a 
specified cone with empty base whose vertex is called 1. Edges with domain 1 are called elements. 
Nodes which are vertices of cocones all of whose injections are elements are called attributes. 
Nodes which are neither attributes, nor 1, are called entitites.''
\end{frame}

\begin{frame}{Category of Partial Maps}
Formalised by Cockett and Lack as 'Restriction Categories'.
There is a locally ordered 2-category associated with a restriction category.
Many diagrams commute upto $\leq$.
Outer Join is pullback.
Inner Join is 2-pullback.
\end{frame}

\begin{frame}{Bibliography}
{\tiny
\bibliography{../../SharedBibliography/temp/bibliography}
}
\end{frame}



\iffalse
\subsection{Categories}

\comingnext{General definitions}

\begin{frame}{Data Specification}
A \textit{data specification} is a sketch $S$ for some kind of structured category \catc.
i.e. it is a directed graph plus some path equivalences (diagrams in other words) plus 
some additional structure.  

An \textit{instance} of a data specification is a mapping $D$ from the directed graph
to the category of sets and functions which preserves structure so as
to induce a structure preserving functor $D: \catc \morph \Fin$.
\end{frame}

\begin{frame}{The Requirement}
A \textit{requirement} for a data specification $S$ 
is a set of instances of the sketch $S$ or, equivalently, is a set $R_C$ of functors where for each
$D \in R_C$, $D: \catc \morph \Fin$, where \catcw is the category generated by the sketch $S$.
\end{frame}

\begin{frame}{Goodness Criteria for a Data Specification}
\IfSforCwithRCwordsvariant 
\begin{itemize}
\item 
\textbf {Goodness Criteria of Type 1 :} No redundancy. The sketch $S$ ought to be a minimum sketch for structured category \catcw i.e. there should be no subsketch of $S$ which generates  \catc.
\item
\textbf {Goodness Criteria of Type 2:} \catcw ought to be \textit{maximally constrained} to $\reqtc$.
\end{itemize}
\end{frame}

\begin{frame}{Rationale for Type 1 Criteria}
Must follow the type 1 criteria 
\begin{itemize}
\item to ensure there is no redundancy in the  data that is stored and/or communicated,
\item avoid unnecessary data constraints (such as those that impose uniqueness or referential integrity).
\end{itemize} 
\medskip
One thing to have a system that is out of step with the real world quite another to have a system 
in a state that doesn't correspond to any possible  real world state.
\end{frame}

\begin{frame}{Rationale for Type 2 Criteria}
Try all we can to avoid the possibility of a system state 
that doesn't correspond to any real world state.
\begin{itemize}
\item Good data modelling is good theorising and good theorising is having the tightest theory that fits the facts.
\item Software Engineering: 
\begin{itemize}
\item Document the requirement.
\item Validate data against the requirement. 
\item Restrict degrees of freedom in the data representation.
\item Use declarative style if possible.
\end{itemize}
\end{itemize}
\end{frame}

\begin{frame}{Key definition: \catcw \textit{maximally constrained} to $\reqtc$}
\begin{itemize}
\item The structured category \catcw ought to be the tightest possible fit to the requirement $R_{\catc}$.

\item The question -- is there a $\catcp$ that extends \catcw and that will do a better job. 

\item Is there a $\catcp$ and an $I: \catc \morph \catcp$  such that 
all instances in the requirement $\reqtc$ factor though $I$
$$
\begin{array} {c p{2cm} c}
\Rnode{Cp}{C'} && \\ [0.25cm]
             && \Rnode{finset}{\Fin} \\ [0.15cm]
\Rnode{C}{C}  
\end{array}
\begin{arrows}
\ncarr {C}{finset}
\alabel{D}
\ncarr{C}{Cp}
\alabel{I}
\ncarr{Cp}{finset}
\alabel{D'} 
\end{arrows}
$$
and
\pause  at least one other instance $F$ of \catcw does not factor through $I$.
\ncarr[-20]{C}{finset}
\blabel{F} 
\end{itemize}
\pause If there is no such $I: \catc \morph \catcp$ then we shall say that 
\catcw is \term{maximally constrained} with respect to $\reqtc$.
\end{frame}

\begin{frame}{Representational Completeness}
Alternative way of approaching tightest fit:
\begin{itemize}
\item That which is in the data and can be represented in the theory should be represented in the theory.
\item To make precise we can give definitions
 of \textit{representational completeness} wrt $\reqtc$ 
\begin{center}
\begin{tabular}{>{\bfseries}l l} 
2A. & equationally complete   \\
2B. & functionally complete   \\
2C. & referentially complete  \\
2D. & mono complete           \\
2E. & epi complete            \\
2F. & product complete        \\
2G. & pullback complete       \\
\end{tabular}
\end{center}
\pause \item In these defintions that \catcw is $x$ complete wrt $\reqtc$ will mean exactly that the set of instances $\reqtc$ are jointly reflective of $x$.
\end{itemize}
\end{frame}

\begin{frame}{Next steps}
For each of
\begin{itemize}
\item categories,
\item categories with designated monomorphisms and epimorphisms,
\item categories with designated monomorphisms and epimorphisms and with finite products
\end{itemize}
\medskip
I am going to
\begin{itemize}
\item define appropriate notions of representational completeness,
\item explore relationship between representational completeness and maximal constrainedness.
\end{itemize}

Herein lies much of the mathematical theory of data but as many or more questions than answers.
\end{frame}


\begin{frame}{BCNF in the abstract (based on Zaniolo 1982 Definition 2)}
If $S$ is a sketch for a category $C$ with designated mono-sources  and finite products
and $S$ is considered as a data specification with requirement $\reqtc$, then it ought to be the case
that if 
\scalebox{0.9}{\multisourcenplusonediagram{n}{a}{b}{x}{c}{y}}
are edges of $S$
and if \msfd{x_1,...x_n}{y}  is a non-trivial functional dependency  between these edges
then \onslide<1>{...}
\onslide<2>{\scalebox{0.8}{\multisourcediagram{n}{a}{b}{x}} is a designated mono-source.}
\end{frame}

\comingnext{Sketches of Categories as Data Specifications}

\begin{frame}{Sketches of Categories as Data Specifications}
By a \term{sketch for a category} we mean a pair $\tuple{G,PE}$ where 
$G$ is a directed graph and $PE$ is a set of path equivalences. \\
\medskip
\goodnessoneA. \\
\medskip
\goodnessoneB.
\end{frame}

\begin{frame}{Equational Completeness}
\begin{definition}
If $\catc$ is a  category and $\reqtc$ is a set of instances,
 then say that  $\catc$ is \term{equationally complete} with respect 
to the requirement $\reqtc$ iff all path equivalences with respect to $R_C$ are represented in \catcw 
i.e. iff for all diagrams \fgparalleldiagram in $\catc$,  
if in all instances $D \in \reqtc$, $D(f)=D(g)$,  then $f=g$.
\end{definition}

In other words:
the set of functors $\reqtc$ is jointly faithful.
\medskip
\goodnesscriteria{2A} \IfSforCwithRCwords then \catcw ought to be equationally complete
with respect to $R_C$.
\end{frame}

\begin{frame}{Equational completeness cont. }
Does equational completeness follow from maximal constrainedness?
It does if we assume local finiteness of \catc:
\begin{lemma}[Path Equivalence Representation Lemma]
If $\catc$ is a locally finite category and $\reqtc$ is a set of instances, if $\catc$ 
 is
\textit{maximally constrained} to the requirement $\reqtc$ then it is equationally
complete with respect to $\reqtc$.
\end{lemma}
\begin{proof}
Suppose \fgparalleldiagram  in $\catc$ and that in all instances $D \in \reqtc$, $D(f)=D(g)$. 
Define $\catcp$ to be \catc plus $f=g$. Consider the Hom functor $Hom_{\catc}(a,\_)$. 
This is an instance of $\catc$ which therefore extends to an instance $H$ of $\catcp$.  Then
$Hom_{\catc}(a,f)=H(I(f))=H(I(g))=Hom(a,g)$ and applying to $id_a$ we have that $f=g$.
\end{proof}
\end{frame}

\begin{frame}{Does Criteria 2A follow from Criteria 2 -- Not always}
Suppose \catcw is the category generated by the sketch with directed graph
\begin{displaymath}
\begin{array}{cp{1.4cm}c}
                                    \\[0.1cm]
\Rnode{a}{a}	&& \Rnode{b}{b}     \\[0.25cm]
	            &&  
\end{array}
\begin{arrows}
\ncarr[15]{a}{b}
\alabel{f}[0.35]
\ncarr[-15]{a}{b}
\blabel{g}[0.35]
\ncarr[-70]{a}{b}
\blabel{h'}[0.35]
\ncarr[-70]{b}{a}
\blabel{h}[0.35]
\nccircle[angleA=-90, nodesep=3pt]{->}{b}{.5cm}
\blabel{r}[0.3]
\end{arrows}
\end{displaymath}

subject to the identities
\begin{equation}
\label{fhidentity}
f \circ h = id_a,
\end{equation}
\begin{equation}
\label{ghidentity}
g \circ h = id_a,
\end{equation}
\begin{equation}
\label{rhhpidentity}
r \circ h \circ h' = id_b.
\end{equation}
We can show that for any functor $D:\catc \morph Fin$, $D(f)=D(g)$. 

If $R_C$ is the set of all functors from \catcw into $\Fin$ then with respect to $\reqtc$, Criteria 2 holds but Criteria 2A does not.
\end{frame}

\begin{frame}{Functional Dependencies}
To describe Goodness Criteria 2B I first need to
\begin{itemize}
\item Define what we mean by \term{functional dependency}
-- abstracted and simplified from definition given by Codd 1971.
\item Define what we mean by a functional dependency being \term{represented}
-- inspired by language I found in Zaniolo 1982.
\item State as the criteria that all functional dependencies ought to be represented -- which I think is in
the spirit of Zaniolo's paper. 
\end{itemize}
\end{frame}

\begin{frame}{Definition of Functional Dependency}
\begin{definition}
If \scalebox{0.9}{\fgsourcediagram} in a category \catcw  and if $\reqtc$ is a set of instances of \catcw
then we say that there is a  \term{functional dependency} of $g$ on $f$ with respect to $\reqtc$ iff
for every $D \in \reqtc$, $D(g)$ factors uniquely through $D(f)$. \\
\pause \medskip
i.e.  in every $D$, $D(f)$ is surjective and there is a family of  functions $H_D:D(b) \morph D(c))_{D \in \reqtc}$
such that in each instance $D \in \reqtc$,
\scalebox{0.9}{\fghfactordiagram{\roomup{0.4cm}\roomdown{0.3cm}D(a)}{D(b)}{D(c)}{D(f)}{D(g)}{H_D}} commutes.
\end{definition}

\pause If $H$ is such a functional dependency then we say that $\fundep{H}{f}{g}$ in $\catc$ with respect to $\reqtc$.
\end{frame}

\begin{frame}{Notation Alert}
The notation 
$$
f \morph g
$$
for functional dependency is adapted from relational database theory. 

\begin{itemize}
\item This is not to imply a morphism or 2-cell or anDefinition of y such thing.
\end{itemize}
\end{frame}

\begin{frame}{Definition of Representation of Functional Dependency}
If $\catc$ is a category and $\reqtc$ is a set of instances, 
if \fgsourcediagram in $\catc$ 
and if there is a functional dependency $\fundep{H}{f}{g}$ then say that 
this functional dependency  is \term{represented} in $\catc$ 
iff there exists a morphism $h:b \morph c$ in $\catc$ such that for each instance $D \in \reqtc$, $D(h)=H_D$.
\medskip
\pause If \catcw is a requirement and $\reqtc$ a set of instances then \catcw is said to be 
\textit{functionally complete} with respect to $\reqtc$ iff every functional dependency
present in $\reqtc$ is represented in \catc.\\
\medskip
\pause \goodnesscriteria{2B}\IfSforCwithRCwords then \catcw ought to be functionally complete with respect to $\reqtc$.
\end{frame}

\begin{frame}{Representation Lemma for Functional Dependencies}
I can prove this only in the case that \catcw is locally finite.

\begin{lemma}
If $\catc$ is a locally finite category and $\reqtc$ is a set of instances, if $\catc$ is 
\textit{maximally constrained} to the requirement $\reqtc$ then 
\catcw is functionally complete with respect to $\reqtc$.
\end{lemma}
\begin{proof}
Suppose$\fundep{H}{f}{g}$  is a functional dependency with respect to $\reqtc$
such that in all instances $D \in \reqtc$, $D(f)$ is surjective.

Extend \catcw to $\catcp$ by formally adding a morphism $h$. Extend each $D$ to a $D'$ by defining $D(h)=H_D$. 
Because $D(f)$ is surjective $D'$ is unique. 
Define a functor from \catcw to $\Fin$ as  a certain quotient of the coproduct of functor $Hom_{\catc}(a,-)$ with itself. 
Extend to $\catcp$ and demonstrate that there exists $k:b \morph c$ in \catcw such that $f \circ k=g$. 
which shows that $\fundep{H}{f}{g}$ is represented in $\catc$.
\end{proof}
\end{frame}

\begin{frame}{Proof 1.}
We have \fgsourcediagram in \catc.\\
\medskip
Define functor $F: \catc \morph \Fin$ be the coproduct $Hom_{\catc}(a,-) + Hom_{\catc}(a,-)$.
Label the injections $L$ and $R$, respectively so that
for each object $x$ of $\catc$ the diagram
\begin{center}
$
\begin{array}{c p{0.5cm} c p{0.5cm} c  }
\Rnode{h1}{Hom_{\catc}(a,x)}  &&\Rnode{Fx}{F(x)}  &&   \Rnode{h2}{Hom_{\catc}(a,x)}       
\end{array} 
$
\ncarr{h1}{Fx}
\alabel{L_x}
\ncarr{h2}{Fx}
\blabel{R_x}
\end{center}
is a coproduct in $\Fin$.
\medskip
Define $G: \catc \morph \Fin$ as quotient $F/\sim$ where
\begin{align*}
L_x(k_1) \sim_x R_x(k_2) & \mbox{ iff there exists $k:b \morph x$ in $\catc$ such that $k_1 = f \circ k = k_2$,}\\
L_x(k_1) \sim_x L_x(k_2) & \mbox{ iff $k_1 = k_2$,} \\
R_x(k_1) \sim_x R_x(k_2) & \mbox{ iff $k_1 = k_2$.} \\
\end{align*} 
\end{frame}

\begin{frame}{Proof 2.}
We have \simpleunaryfdrepresentationdiagram{a}{b}{c}{f}{g}{\qq{h}}  in \catcp.\\
\medskip
Extend $G$ to $G':\catcp \morph \Fin.$ \\

\medskip
\begin{align*}
&\mbox{Now} & [L_b(f)]                         &= [R_b(f)]                 \\
&\mbox{i.e.}& G'(f)([L_a(id_a)])              &= G'(f)([R_a(id_a)])       \\
&\therefore & G'(\qq{h}) (G'(f)([L_a(id_a)])) &= G'(\qq{h}) (G'(f)([R_a(id_a)])) \\
&\therefore & G'(g)([L_a(id_a)])              &= G'(g)([R_a(id_a)]) & &    \\
&\therefore & [L_c(g)]                        &= [R_c(g)] 
\end{align*}
Therefore there exists $k:b \morph c$ in 
$\catc$ such that $f \circ k = g$ and we have shown as required that the function dependency
$\fundep{H}{f}{g}$ is represented in $\catc$.
\end{frame}


\begin{frame}{Definition of Inclusion Dependencies}
If $\catc$ is a category and $\reqtc$ is a set of instances 
and if
\fnsourceqnsource
in $\catc$, then an \term{inclusion dependency} $I$, written $a[f_1,...f_n] \overset{I}{\subseteq} c[q_1,..q_n]$, is a family of functions $I_D)_{D \in \reqtc}$
such that each instance $D \in \reqtc$, $I_D$ is a function $I_D : D(a) \morph D(c)$ such that
for each $i$, $1 \leq i \le n$, $I_D \circ D(q_i) = D(f_i)$.

If each function in this family is the unique such function then the inclusion dependency is said to be referential. Just to be clear the definition is this:
\end{frame}

\begin{frame}{Definition of Referential Inclusion Dependency}
\begin{definition}
If $\catc$ is a category and $\reqtc$ is a set of instances 
and if
\fnsourceqnsource
in $\catc$, then a \term{referential inclusion dependency} $I$, written $a[f_1,...f_n] \overset{I}{\subseteq} c[q_1,..q_n]$, is a family of functions $I_D)_{D \in \reqtc}$
such that each instance $D \in \reqtc$, $I_D$ is a unique function $I_D : D(a) \morph D(c)$ such that
for each $i$, $1 \leq i \le n$, $I_D \circ D(q_i) = D(f_i)$.
\end{definition}
\end{frame}

\begin{frame}{Definition of Referentially Complete}
\begin{definition}
If $\catc$ is a category and $\reqtc$ is a set of instances and if
\fnsourceqnsource
in $\catc$ and if $a[f_1,...f_n] \overset{I}{\subseteq} c[q_1,..q_n]$ is a referential inclusion dependency
with respect  to $\reqtc$ then say that the inclusion dependency $I$ is \term{represented} in $\catc$
iff there exists a morphism $i:a \morph c$ in $\catc$ such that in each instance $D \in \reqtc$, $D(i) = I_D$. 
\end{definition}
If \catcw is a category and $\reqtc$ a set of instances then 
\catcw is \term{referentially complete} with respect to $\reqtc$ 
iff all referential inclusion dependencies present in $\reqtc$ are represented in \catc.
\end{frame}

\begin{frame}{Goodness Criteria 2C}
\goodnesscriteria{2C} \IfSforCwithRCwords 
\catcw ought to be referentially complete with respect to $\reqtc$.

I would like to show that this goodness criteria follows from the criteria of maximal constrainedness
but I have managed to prove this only in the case that \catcw is locally finite. 
\end{frame}

\begin{frame}{Representation Lemma for Referential Inclusion Dependencies}
If \catcw is a locally finite category and $\reqtc$ is a set of instances, if \catcw is 
\textit{maximally constrained} to the requirement $\reqtc$ then
every referential inclusion dependency with respect to $\reqtc$ is represented in $\catc$.
\end{frame}

\begin{frame}{Proof}
Suppose $a[f_1,...f_n] \overset{I}{\subseteq} c[q_1,..q_n]$ is a referential inclusion dependency
where for each $i$, $1 \leq i \leq n$,
\scalebox{0.9}{$
\begin{array}{c p{0.5cm} c p{0.5cm} c}
             &&\Rnode{bi}{b_i} &&              \\[0.5cm]
\Rnode{a}{a} &&                && \Rnode{c}{c}
\end{array}
\begin{arrows}
\ncarr{a}{bi}\alabel{f_i}
\ncarr{c}{bi}\blabel{q_i}
\end{arrows}
$} in \catc.

Extend category \catcw to a category \catcpw such that in \catcpw there is
a morphism $\qq{r}: a \morph c$ such that for each $i$, $1 \leq i \leq n$, 
$\qq{r} \circ q_i = f_i$.
Every $D \in \reqtc$ extends uniquely to functor $D: \catcp \morph \Fin$. Therefore, since \catcw is maximally constrained then every functor $F: \catc \morph \Fin$ can be extended to $F':\catcp \morph \Fin$. \\
\pause Let $H: \catcp \morph \Fin$ be the extension of the hom functor $Hom(a,-): \catc \morph \Fin$ to 
\catcp. \\
\pause For each $i$, we have $H(\qq{r}) \circ H(q_i) = H(f_i)$, \\
\pause \hspace {3cm} i.e. $H(\qq{r}) \circ Hom(a,q_i) = Hom(a,f_i)$. \\
\pause Therefore in particular $H(\qq{r})(id_a) \circ q_i =  f_i$. \\
\pause Hence $H(\qq{r})(id_a):a \morph c$ in \catcw represents the inclusion dependency $I$.
\end{frame}

\begin{frame}{Three Goodness Criteria}{for Categories as Data Specifications}
Definition of Goodness Criteria 2A, 2B and 2C:
\begin{itemize}
\item If \catcw is a category and $\reqtc$ is a set of instances of \catcw then
\medskip
\begin{tabular}{>{\bfseries}l l} 
2A: & \catcw ought to be equationally complete wrt $\reqtc$  \\
2B: & \catcw ought to be  functionally complete wrt $\reqtc$  \\
2C: & \catcw ought to be referentially complete wrt $\reqtc$ \\
\end{tabular}
\pause \item We have shown that if \catcw is locally finite and meets Criteria 2 that it is maximally constrained then it also meets Criteria 2A, 2B and 2C.

\pause \item We will move on to consider categories with designated monomorphisms and epimorphisms
\item First though we will describe a possible improvement to this situation described above.
\end{itemize}
\end{frame}

\begin{frame}{Finitary Property}
\begin{definition}
Define a category \catcw to have the \term{finitary property} iff for all objects $x$ and for all endomorphisms 
$f: x \morph x$ the following are equivalent
\begin{itemize}
\item $f$ is a monomorphism,
\item $f$ is an epimorphism.
\end{itemize}
\end{definition}
\end{frame}

\begin{frame}{Deducing Monos and Epis}
Consider, in any category \catcw, whenever \fgcomposablediagram{x}{y}{z}{f}{g} then
\begin{itemize}
\item $f \circ g$ is a monomorphism implies $f$ is a monomorphism.
\item $f \circ g$ is an epimorphism implies $g$ is an epimorphism.
\end{itemize}
\medskip
Further to this if \catcw has the \term{finitary property} then whenever \fgcomposablediagram{x}{x}{x}{f}{g} 
such that $f \circ g = id_x$ then
\begin{itemize}
\item $f$ is a monomorphism and an epimorphism,
\item $g$ is a monomorphism and an epimorphism.
\end{itemize}
\end{frame}

\begin{frame}
Now return to this example.
\begin{displaymath}
\begin{array}{cp{1.4cm}c}
                                    \\[0.1cm]
\Rnode{a}{a}	&& \Rnode{b}{b}     \\[0.25cm]
	            &&  
\end{array}
\begin{arrows}
\ncarr[15]{a}{b}
\alabel{f}[0.35]
\ncarr[-15]{a}{b}
\blabel{g}[0.35]
\ncarr[-70]{a}{b}
\blabel{h'}[0.35]
\ncarr[-70]{b}{a}
\blabel{h}[0.35]
\nccircle[angleA=-90, nodesep=3pt]{->}{b}{.5cm}
\blabel{r}[0.3]
\end{arrows}
\end{displaymath}

subject to the identities
\begin{equation}
\label{fhidentity}
f \circ h = id_a,
\end{equation}
\begin{equation}
\label{ghidentity}
g \circ h = id_a,
\end{equation}
\begin{equation}
\label{rhhpidentity}
r \circ h \circ h' = id_b.
\end{equation}

If we see this as a sketch for a category with the finitary property then we can deduce that $h$ is a monomorphism
and from this that $f=g$.  It is no longer a counter example (to criteria 2 implies criteria 2A).
\medskip
I am left wondering ...  
\end{frame}



\subsection{Categories with Designated Monomorphisms and Epimorphims}



\begin{frame}{Definition}
\begin{itemize}
\item A \term{\catMEterm} is a triple $\tuple{\catc,M,E}$ where 
\begin{itemize}
\item \catcw is a category,
\item $M$ is a set of designated monomorphisms of \catcw closed under composition and including all identity morphisms,
\item $E$ is a set of designated epimorphisms of \catcw closed under composition and including all identity morphisms.
\end{itemize}
\item Define an instance $F$ of a \catMEterm to be a functor $F: \catc \morph \Fin$ 
that preserves the designated monomorphisms and epimorphisms.

%Note that \catcw may well have other monomorphisms and epimorphisms but these need not be preserved by an instance $F$.

\item A sketch for a \catMEterm is a 4-tuple $\tuple{G,PE_0,M_0,E_0}$ ...
\begin{itemize}
 \item   $M_0$ and $E_0$ are sets of $G$-paths  
\end{itemize}
\end{itemize}
\end{frame}


\begin{frame}{Goodness Type 1 Criteria}
\medskip
\goodnessoneA.
\medskip
\goodnessoneB.

\goodnessoneC. \\
\medskip
\goodnessoneD. \\
\medskip

Definition of closure of set of monomorphics includes
\begin{itemize}
\item $f \in \overline{M}$ and $g \in \overline{M}$ then $f \circ g \in \overline{M}$
\item $f \circ g \in \overline{M}$ then $f \in \overline{M}$
\end{itemize}
and to which we might add the rule that expresses the finitary property.
\end{frame}


\begin{frame}{Revised Definition of Functional Dependency}
\begin{definition}
If \scalebox{0.9}{\fgsourcediagram} in a \catMEterm \catcw  and if $\reqtc$ is a set of instances of \catcw
then we say that there is a  \term{functional dependency} of $g$ on $f$ with respect to $\reqtc$ iff
for every $D \in \reqtc$, $D(g)$ factors through $D(f)$. \\
\medskip
\pause Equivalently there is a family of  functions $H_D:img(D(f)) \morph D(c))_{D \in \reqtc}$
such that in each instance $D \in \reqtc$,
\scalebox{0.9}{\fghfactordiagram{\roomup{0.4cm}\roomdown{0.3cm}D(a)}{img(D(f))}{D(c)}{D(f)}{D(g)}{H_D}} commutes.
\end{definition}

\end{frame}

\newcommand{\representationdiagram}{
\newcommand{\myspacing}{0.15cm}
$
\begin{array}{c p{1.25cm} c c p{0.5cm} c}
            && \Rnode{b}{b}&                &&                \\[\myspacing]
            &&             &                &&\Rnode{d1}{d_1} \\[\myspacing]
            &&             &\Rnode{b1}{b_1} &&                \\[\myspacing]
            &&             &                &&\Rnode{d2}{d_2} \\[\myspacing]
\Rnode{a}{a}&&             &\Rnode{b2}{b_2} &&                \\ %\multirow{2}{5pt}{\vdots}   \\
            &&             &\vdots          &&\vdots          \\
            &&             &\Rnode{bnp}{b_{n-1}} &&           \\[\myspacing]
            &&             &                &&\Rnode{dn}{d_n} \\[\myspacing]
            && \Rnode{c}{c}  
\end{array}
\begin{arrows}
\ncarr[3]{a}{b}\alabel{f}
\ncarr[-3]{a}{c}\blabel{g}
\ncarr[-5]{d1}{b}\blabel{q_1}[0.35]\idcomp
\ncarr{d1}{b1}\blabel{h_1}
\ncarr{d2}{b1}\blabel{q_2}[0.35]\idcomp
\ncarr{d2}{b2}\blabel{h_2}
\ncarr{dn}{bnp}\blabel{q_n}[0.35]\idcomp
\ncarr[5]{dn}{c}\alabel{h_n}
\end{arrows}
$
}

\newcommand{\mappedrepresentationdiagram}{
\newcommand{\myspacing}{0.25cm}
$
\begin{array}{ c p{0.2cm} c p{0.5cm} c}
\Rnode{Imgf}{Img(D(f))} && \Rnode{b}{D(b)}   &&                   \\[\myspacing]
                        &&                   &&\Rnode{d1}{D(d_1)} \\[\myspacing]
                        &&\Rnode{b1}{D(b_1)} &&                   \\[\myspacing]
                        &&                   &&\Rnode{d2}{D(d_2)} \\[\myspacing]
                        &&\Rnode{b2}{D(b_2)} &&                   \\ %\multirow{2}{5pt}{\vdots}   \\
                        &&\vdots             &&\vdots             \\
                        &&\Rnode{bnp}{d(b_{n-1})} &&           \\[\myspacing]
                        &&                   &&\Rnode{dn}{D(d_n)} \\[\myspacing]
\Rnode{c}{D(c)}  
\end{array}
\begin{arrows}
\ncline{H->}{Imgf}{b}
\ncarr{Imgf}{c}\alabel{H_D}
\ncarr[5]{b}{d1}\alabel{D(q_1)^{-1}}[0.5][-3]
\ncarr{d1}{b1}\blabel{D(h_1)}[0.5][-1]
\ncarr{b1}{d2}\alabel{D(q_2)^{-1}}[0.5][-3]
\ncarr{d2}{b2}\blabel{D(h_2)}[0.5][-1]
\ncarr{bnp}{dn}\alabel{D(q_n)^{-1}}[0.5][-3]
\ncarr[5]{dn}{c}\alabel{D(h_n)}[0.5][-1]
\end{arrows}
$
}

\begin{frame}{Definition of Representation of FDs in Category with Designated Monomorphisms}
If $\fundep{H}{f}{g}$ is such a  functional dependency 
then $H$  is \term{represented} in $\catcw$ 
\begin{tabular} {c c}
\multicolumn{1}{p{3.5cm}}{iff there exits} 
                    & \multicolumn{1}{p{4.1cm}}{\onslide<2->{such that for each $D \in \reqtc$}}\\
\scalebox{0.9}{\representationdiagram} 
                    & \onslide<2->{\scalebox{0.85}{\mappedrepresentationdiagram}}               \\
                    & \multicolumn{1}{r}{\onslide<2->{commutes.}}
\end{tabular}
\end{frame}


\begin{frame}{First Sublemma}
\highlight{DONT think this is whorthwhile showing anymore}
\begin{lemma}
If \catcw is a \catMEterm with a finite sketch and if $f$ is any  morphism 
not having an epi-mono factorisation then \catcw can be extended to a category $\catcp$ with finite sketch so that
\begin{itemize} 
\item $f$ has an epi-mono factorisation \factorisationfdiagram in $\catcp$,
\item every morphism $h:im(f) \morph x$ in $\catcp$ factors through $f_m$,
\item every instance of \catcw uniquely extends, upto isomorphism, to an instance of $\catcp$,
\item \catcw is maximally constrained to $\reqtc$ iff $\catcp$ is maximally constrained to $\reqtc$ (extended to $\catcp$).
\end{itemize} 
\end{lemma} 
\end{frame}

\begin{frame}{Second Sublemma}
\highlight{Rename "Sublemma"}
If \catcw is a category with designated epimorphisms then let its split category 
$\catc_s$ be the category \catcw with an additional split morphism for every designated epimorphism.
i.e. for every designated epimorphism $f: a \morph b$ we add a morphism $f_s:b \morph a$
such that $f_s \circ f = id_B$. \\
\medskip
If \catcw is a category with designated monomorphisms and epimorphsims and its split category
is locally finite, if $F: \catcw \morph \Fin$ is a functor which preserves designated monomorphisms
then there is an $F':\catcw \morph Fin$ that preserves designated monomorphisms and epimorphisms
and that extends $F$ i.e. such that $F \hookrightarrow F'$ in the functor category $\Fin^{\catc}$.
\end{frame}

\begin{frame}{Proof of Second SubLemma}
\highlight{Rename "Proof of Sublemma"}
Hand wavy proof as follows:
There is an inclusion functor $I_s:\catc \morph \catc_s$.

Extend $F:\catcw \morph \Fin$ along this inclusion functor  to get a functor 
$F_s:\catc_s \morph \Fin$ and a natural transformation $F \hookrightarrow I_s \circ F_s$. \\
\medskip
Nature of the construction is such that $F_s$ preserves designated monomorphisms because $F$ does.\\
\medskip
$F_s$ preserves designated epimorphisms since any functor preserves split epimorphisms.
Define $F'$ to be $I_s \circ F_s$.
\end{frame}

\begin{frame}{Representation Lemma for FDs}
{\small Note: I can prove this only in the case that \catcw is locally finite.

No longer restricted to the case that each $D(f)$ is surjective.}

\begin{lemma}
If \catcw is a category with designated monomorphisms and epimorphsims and its split category
is locally finite and if \catcw is
\textit{maximally constrained} to a requirement $\reqtc$ then
if $\fundep{H}{f}{g}$  is a functional dependency with respect to $\reqtc$
then $\fundep{H}{f}{g}$ is represented in $\catc$.
\end{lemma} \highlight{Get rid of this proof -- need cut and paste somewhere I can later find it}
\highlight{Also copy later into key result.}
\begin{proof} Extend \catcw to \catcpw so that $f$ has an epi-mono factorisation in \catcpw
and so that there  is a morphism $\qq{h}$ such that
\scalebox{0.9}{\unaryfdrepresentationdiagram{\roomup{0.6cm}\roomdown{0.4cm}a}{b}{c}{f}{g}{im(f)}{f_m}{\qq{h}}$
\begin{arrows}\ncarr{a}{d}\alabel{f_e}\end{arrows}$} 
commutes in \catcp.  
\end{proof}
\end{frame}

\begin{frame}{Proof cont. 1}
We have \fgsourcediagram in \catc.\\
\medskip
Define functor $F: \catc \morph \Fin$ be the coproduct $Hom_{\catc}(a,-) + Hom_{\catc}(a,-)$
as previously with injections $L$ and $R$, respectively so that
for each object $x$ of \catcw the diagram
\begin{center}
$
\begin{array}{c p{0.5cm} c p{0.5cm} c  }
\Rnode{h1}{Hom_{\catc}(a,x)}  &&\Rnode{Fx}{F(x)}  &&   \Rnode{h2}{Hom_{\catc}(a,x)}       
\end{array} 
$
\ncarr{h1}{Fx}
\alabel{L_x}
\ncarr{h2}{Fx}
\blabel{R_x}
\end{center}
is a coproduct in $\Fin$.
\medskip
Define $G: \catc \morph \Fin$ as quotient $F/\sim$ where 
to ensure $G$ preserves designated monomorphisms  $\sim$ is defined as follows
\begin{align*}
L_x(k_1) \sim_x R_x(k_2) & \mbox{ \parbox{6cm}{iff  $k_1=k_2$ and there is a functional dependency 
$f \morph k_1$ in \catcw with respect to $\reqtc$ and  this functional dependency is represented in \catc.}}\\
L_x(k_1) \sim_x L_x(k_2) & \mbox{ iff $k_1 = k_2$,} \\
R_x(k_1) \sim_x R_x(k_2) & \mbox{ iff $k_1 = k_2$.} \\
\end{align*} 
\end{frame}

\begin{frame}{Proof cont 2.}
We have \simpleunaryfdrepresentationdiagram{a}{Im(f)}{c}{f_e}{g}{\qq{h}}  in \catcp.\\
\medskip
Extend $G$ to $G':\catcp \morph \Fin.$ \\
Because $f_m$ is a designated monomorphism and $f_e \circ f_m = f$ then
there is a representation of functional dependency $f \morph f_e$ in \catc.
Hence  we have that $[L_b(f_e)]= [R_b(f_e)]$.
\medskip
\begin{align*}
&\mbox{i.e.}& G'(f_e)([L_a(id_a)])              &= G'(f_e)([R_a(id_a)])       \\
&\therefore & G'(\qq{h}) (G'(f_e)([L_a(id_a)])) &= G'(\qq{h}) (G'(f_e)([R_a(id_a)])) \\
&\therefore & G'(g)([L_a(id_a)])              &= G'(g)([R_a(id_a)]) & &    \\
&\therefore & [L_c(g)] &= [R_c(g)] 
\end{align*}
Therefore by definition of $\sim$ the functional dependency
$\fundep{H}{f}{g}$ is represented in $\catc$.
\end{frame}

\begin{frame}{Epi Mono Type 2 Goodness Criteria}
From local finiteness and maximal constrainedness it follows that all the following goodness criteria hold: 
\begin{itemize}
\item \textbf{2A}:  \catcw ought to be equationally complete wrt $\reqtc$,         
\item \textbf{2B}:  all $\reqtc$ functional depedencies ought to be represented in \catc,
\item \textbf{2C}:  all $\reqtc$ referential inclusion dependencies ought to be represented in \catc, 
\pause \item \textbf{2D}:  if $f$ is injective in every $D \in \reqtc$ then $f$ ought to be a designated monomorphism in \catc,
\pause \item \textbf{2E}:  if $f$ is surjective in every $D \in \reqtc$ then $f$ ought to be a designated epimorphism in \catc.
\end{itemize}
We can also show that \textbf{2B} can be weakened. If all functional dependencies \textbf{upon edges} from sketch $S$ are represented then 
all functional dependencies are represented. \highlight{Too much information}
\end{frame}

\begin{frame}{Obtaining BCNF}
Tp proceed further 
\begin{itemize}
    \item I need work with categories \catcw with designated monomorphisms and epimorphisms
    and epi-mono splits 
    \item I need assume that \catcw is maximally constrained to $\reqtc$ 
    amongst other categories with designated monomorphisms and epimorphisms
    \item I also need assuume that the the underlying category is 
    maximally constrained to $\reqtc$
\end{itemize}
\highlight{CONTINUE FROM THIS POINT}

\medskip
\goodnesscriteria{2B'} Whenever \fgsourcediagram in \catc and  $f$ is a designated epimorphism, if
$f \morph g$ is a functional dependency  wrt $\reqtc$ then there ought to exist a morphism
$h: b \morph c$ in \catcw such that
\simpleunaryfdrepresentationdiagram{a}{b}{c}{f}{g}{h}
commutes.
\end{frame}

\begin{frame}{Tellingly...}
We can show that if
 goodness criteria 2A,  2B' and 2C are met then 
\begin{itemize}
\item If $f$ and $g$ are edges of a sketch of category \catcw 
with designated monomorphisms and epimorphims and if $\fundep{H}{f}{g}$ is a non-trivial  functional dependency wrt set of instances $\reqtc$ then $f$ is a monomorphism. 

\item This moves us into sight of the goal of "explaining the Boyce-Codd Normal Form (BCNF) criteria from first principles". \highlight{Somewhere need to state BCNF.}

\item We cannot obtain this result without Criteria 2B' -- Criteria 2B alone is insufficient.

\item To get closer to BCNF we need add finite products to the mix.
\end{itemize}
\end{frame}


\subsection{Categories with Finite Products and Designated Monomorphisms and Epimorphims}
\documentclass[xcolor=pst,dvips]{beamer}
% For handout  
%\documentclass[handout]{beamer}   
%\usepackage{pgfpages}
%\pgfpagesuselayout{6 on 1}[a4paper,border shrink=5mm]

\usepackage{mathptmx}
\usepackage{amsfonts}
\usepackage{wasysym}
\usepackage{url}
\usepackage{hyperref}
%\usepackage{pst-arrow}


\newcommand{\sharedmacros}{../../SharedMacros}
%\usepackage{imakeidx}
\makeindex[name=definitions, title=Index of Definitions]
\makeindex[name=lemmas, title=Index of Lemmas]



\newcommand{\commentary}[1]{\marginpar{\footnotesize #1}}
\newcommand{\highlight}[1]{\colorbox{orange}{#1}}
\newcommand{\term}[1]{\textit{#1}\commentary{\colorbox{lightgray}{\textit{#1}}}\index[definitions]{#1}}
\newcommand{\llabel}[1]{\label{#1}\commentary{\colorbox{pink}{\scriptsize{#1}}}\index[lemmas]{#1}}
\newcommand{\lref}[1]{\ref{#1}\colorbox{pink}{\scriptsize{#1}}\index[lemmas]{#1!use of}}

\newcommand{\newt}[1]{\colorbox{yellow}{#1}}
\newenvironment{newtt}
{  \colorbox{yellow}{$[$ ...} 
}
{  \colorbox{yellow}{... $]$}
}
\newcommand{\oldt}[1]{\colorbox{yellow}{\sout{#1}}}
\newenvironment{oldtt}
{  \colorbox{red}{$[$ ...} 
}
{  \colorbox{red}{... $]$}
}

\newcommand{\reinstatet}[1]{\colorbox{lime}{#1}}
\newenvironment{reinstatett}
{  \colorbox{lime}{$[$ ...}
}
{  \colorbox{lime}{... $]$}
}

\newcommand{\tbd}{\highlight{TBD}}

%ithprojection function
\newcommand{\proji}[1]{\pi_#1}



\newenvironment{categoricalaside}
{\begin{framed}
\textbf{Categorical Aside}
}
{
\end{framed}
}

\newenvironment{noteforfuture}
{\begin{framed}
\textbf{Note For Future}
}
{
\end{framed}
}

\newenvironment{problem}
{\begin{framed}
\textbf{Problem}
}
{
\end{framed}
}

%quine quote
\newcommand{\qq}[1]{
\left\ulcorner#1\right\urcorner
}

%single quote
\newcommand{\sq}[1]{
\textnormal{\textquotesingle}#1\textnormal{\textquotesingle}
}

%lower quine quote
\newcommand{\lqq}[1]{
\left\llcorner #1\right\lrcorner
}


%from berkley
\newcommand{\langl}{\begin{picture}(4.5,7)
\put(1.1,2.5){\rotatebox{60}{\line(1,0){5.5}}}
\put(1.1,2.5){\rotatebox{300}{\line(1,0){5.5}}}
\end{picture}}
\newcommand{\rangl}{\begin{picture}(4.5,7)
\put(.9,2.5){\rotatebox{120}{\line(1,0){5.5}}}
\put(.9,2.5){\rotatebox{240}{\line(1,0){5.5}}}
\end{picture}}
\newcommand{\lang}{\begin{picture}(5,7)\put(1.1,2.5){\rotatebox{45}{\line(1,0){6.0}}}\put(1.1,2.5){\rotatebox{315}{\line(1,0){6.0}}}\end{picture}}
\newcommand{\rang}{\begin{picture}(5,7)\put(.1,2.5){\rotatebox{135}{\line(1,0){6.0}}}\put(.1,2.5){\rotatebox{225}{\line(1,0){6.0}}}\end{picture}}
%Try sharper tuple brackets -- except gives errors nested in captions so comment out
%\renewcommand{\tuple}[1]{\lang #1 \rang}

\newcommand{\setsuchthat}[2]{\left\{#1 \ \middle|\ #2\right\}}
\newcommand{\set}[1]{\left\{#1\right\}} 

% one to n - wanton
\newcommand{\wanton}[1]{#1_1,...#1_n}
\newcommand{\fn}{\wanton{f}}
\newcommand{\pn}{\wanton{p}}
\newcommand{\qn}{\wanton{q}}
\newcommand{\qnprime}{\wanton{q'}}
\newcommand{\xn}{\wanton{x}}
\newcommand{\xnp}{\wanton{x'}}
\newcommand{\yn}{\wanton{y}}
\newcommand{\ntuple}[1]{\tuple{\wanton{#1}}}
\newcommand{\wantom}[1]{#1_1,...#1_m}
\newcommand{\mtuple}[1]{\tuple{#1_1,...#1_m}}
\newcommand{\qm}{\wantom{q}}
\newcommand{\ym}{\wantom{y}}
\newcommand {\bntuple}{\ensuremath{\ntuple{b}}}
\newcommand {\fntuple}{\ensuremath{\ntuple{f}}}
\newcommand {\fnptuple}{\ensuremath{\ntuple{f}}}
\newcommand {\pntuple}{\ensuremath{\ntuple{p}}}
\newcommand {\qntuple}{\ensuremath{\ntuple{q}}}
\newcommand {\qnptuple}{\ensuremath{\ntuple{q'}}}
\newcommand {\qmtuple}{\ensuremath{\mtuple{q}}}
\newcommand {\sntuple}{\ensuremath{\ntuple{s}}}
\newcommand {\xntuple}{\ensuremath{\ntuple{x}}}
\newcommand {\xnptuple}{\ensuremath{\ntuple{x'}}}
\newcommand {\ymtuple}{\ensuremath{\mtuple{y}}}
\newcommand{\foreachi}[1][n]{for each $i$, $1 \leq i \leq #1$}
\newcommand{\foreachj}[1][m]{for each $j$, $1 \leq j \leq #1$}
\newcommand{\foreachk}[1][l]{for each $k$, $1 \leq k \leq #1$}

    %causes problems when used with bamer

%ccategories.macros.tex 

% Macros for diagrams in contextual categories and related categories

\usepackage{twoopt}
\usepackage{scalerel} 
\usepackage{xargs}

%\usepackage{mathabx}  %Caused font problems
%\usepackage{MnSymbol}  % caused font problems

\newcommand{\conu}
{\mathbf{C}(U)}

\newcommand{\depu}
{\mathbf{D}(U)}

\newcommand{\cat}[1]{\textbf{#1}}
\newcommand{\obj}[1]{\ensuremath{|\cat{#1}|}}
\newcommand{\ccat}[1][C]{\ensuremath{\mathbb{#1}} }
\newcommand{\ccatc}{contextual category \ccat}
\newcommand{\cobj}[2][]{\ensuremath{|\ccat[#2]|_{#1}}}
\newcommand{\cslice}[2]{\ensuremath{\ccat[#1]_{#2}}}
\newcommand{\csliceobj}[3][]{\ensuremath{|\mathbb{#2}_{#3}|_{#1} }}
\newcommand{\varset}[1][]{\ensuremath{V_{#1} }}
\newcommand{\localvarsets}{\ensuremath{\mathcal{V} }}
\newcommand{\Fam}{\ensuremath{\mathbb{F\mathrm{am}} }}
\newcommand{\Famslice}[1]{\ensuremath{\mathbb{F\mathrm{am}}_{#1} }}
\newcommand{\Famobj}[1][]{\ensuremath{|\mathbb{F\mathrm{am}}|_{#1} }}
\newcommand{\Famsliceobj}[2][]{\ensuremath{|\mathbb{F\mathrm{am}}_{#2}|_{#1} }}
\newcommand{\morph}{\rightarrow}
\newcommand{\epi}{\twoheadrightarrow}
\newcommand{\base}{\triangleleft}
\newcommand{\comp}{\circ}
\newcommand{\cross}{\otimes}
\newcommand{\pc}[2]{d^{#1}_{#2}}
\newcommand{\sub}{^*}
\newcommand{\diag}{\delta}
\newcommand{\pbase}[1]{\tilde{#1}}

\newcommand{\tuple}[1]{\langle#1\rangle}
\newcommand{\ndidly}{\ensuremath{\Join_n}}
\newcommand{\ndidlycospan}{quotiented n-cospan}

\newcommand{\crossx}[3]{#1 \underset{#3}{\cross} #2}
\newcommand{\fibrex}[3]{#1 \underset{#3}{\Join} #2}
\newcommand{\powerset}{\mathcal{P}}
\newcommand{\primeds}[1]{
\ensuremath{\mathcal{P}(#1)} }
\newcommand{\compset}{\ \dot{\circ}\, }

% darrow
%\newcommand{\darrow}{\rightarrowtriangle} %use \smorph instead
\newcommand{\smorph}{\rightarrowtriangle}

 

\newcommand\dhead{\scaleobj{0.6}{\triangleright}}
\newcommand{\dmorph}{\, \mbox{---} \! \cdot \! \raisebox{1.1pt}{\dhead}}

% projection tree
%\newcommand{\proj}[2]{proj_{#2}(#1)}

\newcommand{\proj}[2]{
\ensuremath{\mathcal{P}_{#2}(#1)} }

%pstrick supplements for arrows

\newlength{\arrnodesepA}
\newlength{\arrnodesepB}
\newlength{\arroffsetA}
\newlength{\arroffsetB}

%Modified to 2pt from 0pt on 23 July 2018
\newcommand{\arreset}{
\setlength{\arrnodesepA}{2pt}
\setlength{\arrnodesepB}{2pt}
\setlength{\arroffsetA}{0pt}
\setlength{\arroffsetB}{0pt}
}
\arreset

\newcommand{\ncarr}[3][0]{\ncarc[arcangle=#1,nodesepA=\arrnodesepA,nodesepB=\arrnodesepB,offsetA=\arroffsetA,offsetB=\arroffsetB,arrowsize=5pt,arrowinset=0.7]{->}{#2}{#3}}
\newcommand{\jcbarr}[4][0]{ % ncbarr is defined in some thridy party package so do not use!\emph{}
\ncarr[#1]{#3}{#4}
\nbput[labelsep=2pt]{\footnotesize $#2$}
}

\newcommand{\ncaarr}[4][0]{
\ncarr[#1]{#3}{#4}
\naput[labelsep=2pt]{\footnotesize $#2$}
}

% \alabel{label}[npos][labelsep_pts]
\newcommandx*\alabel[3][2=0.5,3=2,usedefault]{\naput[labelsep=#3pt,npos=#2]{\footnotesize $#1$}}
% \blabel{label}[npos][labelsep_pts]
\newcommandx*\blabel[3][2=0.5,3=2,usedefault]{\nbput[labelsep=#3pt,npos=#2]{\footnotesize $#1$}}

% \idcomp mark an arrow as one component of an identifier
\newcommand{\idcomp}{\ncput[npos=0, nrot=:U]{\psline(0.2,-0.075)(0.2,0.075)}}  %add a bar to a node connection arrow
% pstrick supplements for s-arrows (previous name for d-arrow - should convert}

\newlength{\sarnodesepA}
\newlength{\sarnodesepB}
\newlength{\saroffsetA}
\newlength{\saroffsetB}
\newlength{\sarnodesepAsav}
\newlength{\sarnodesepBsav}

\newcommand{\sarreset}{
\setlength{\sarnodesepA}{0pt}
\setlength{\sarnodesepB}{0pt}
\setlength{\saroffsetA}{0pt}
\setlength{\saroffsetB}{0pt}
}

\sarreset

% sar - S-arrow
\newcommand{\ncsar}[3][0]{
\setlength{\sarnodesepAsav}{\sarnodesepA}
\setlength{\sarnodesepBsav}{\sarnodesepB}
\addtolength{\sarnodesepA}{3pt}
\addtolength{\sarnodesepB}{7pt}
\ncarc[nodesepA=\sarnodesepA,nodesepB=\sarnodesepB,offsetA=\saroffsetA,offsetB=\saroffsetB,arcangle=#1]{-}{#2}{#3}
\ncput[nrot=:R,npos=1]{\pstriangle(0,0)(.2,.2)}
\setlength{\sarnodesepA}{\sarnodesepAsav}
\setlength{\sarnodesepB}{\sarnodesepBsav}
}


% bsar - below labelled S-arrow
\newcommand{\ncbsar}[4][0]{
\ncsar[#1]{#3}{#4}
\nbput[labelsep=2pt]{\footnotesize $#2$}
}
% asar - above labelled S-arrow
\newcommand{\ncasar}[4][0]{
\ncsar[#1]{#3}{#4}
\naput[labelsep=2pt]{\footnotesize $#2$}
}

% cdar - composite dependency arrow
\newcommand{\nccdar}[3][0]{
\setlength{\sarnodesepAsav}{\sarnodesepA}
\setlength{\sarnodesepBsav}{\sarnodesepB}
\addtolength{\sarnodesepA}{3pt}
\addtolength{\sarnodesepB}{11pt}
\ncarc[nodesepA=\sarnodesepA,nodesepB=\sarnodesepB,offsetA=\saroffsetA,offsetB=\saroffsetB,arcangle=#1]{-}{#2}{#3}
\ncput[nrot=:R,npos=1]{\pstriangle(0,0.1)(.2,.2)}
\ncput[nrot=:R,npos=1]{\psdot[dotsize=1pt](-0.0075,0.05)}   %!!
\setlength{\sarnodesepA}{\sarnodesepAsav}
\setlength{\sarnodesepB}{\sarnodesepBsav}
}


% bcdar - below labelled composite dependency arrow
\newcommand{\ncbcdar}[4][0]{
\nccdar[#1]{#3}{#4}
\nbput[labelsep=2pt]{\footnotesize $#2$}
}
% acdar - above labelled composite dependency arrow
\newcommand{\ncacdar}[4][0]{
\nccdar[#1]{#3}{#4}
\naput[labelsep=2pt]{\footnotesize $#2$}
}


% rsar - recursive S-arrow
\newcommand{\ncrsar}[2]{
\setlength{\sarnodesepAsav}{\sarnodesepA}
\setlength{\sarnodesepBsav}{\sarnodesepB}
\addtolength{\sarnodesepA}{3pt}
\addtolength{\sarnodesepB}{7pt}
\ncloop[nodesepA=\sarnodesepA,nodesepB=\sarnodesepB,
        offsetA=\saroffsetA,offsetB=\saroffsetB,
        armA=0.7cm,armB=0.6cm,angleA=90,angleB=-90,loopsize=-1,linearc=0.4
				]{-}{#1}{#2}
\ncput[nrot=:R,npos=5]{\pstriangle(0,0)(.2,.2)}
\setlength{\sarnodesepA}{\sarnodesepAsav}
\setlength{\sarnodesepB}{\sarnodesepBsav}
}

% pstrick supplements for multi-arrows

\newlength{\marnodesepA}
\newlength{\marnodesepB}
\newlength{\maroffsetB}
\newlength{\marnodesepBsav}

\newcommand{\marreset}{
\setlength{\marnodesepA}{0pt}
\setlength{\marnodesepB}{0pt}
\setlength{\maroffsetB}{0pt}
}

\marreset

%ncmarr[#1 arcangle1][#2 arcangle2]{#3 name}{#4 domain1}{#5 domain2}{#6 junction}{#7 codomain}
\newcommandtwoopt{\ncmarr}[6][8][8]{%
\ncarc[nodesepA=\marnodesepA,nodesepB=0,arcangle=#1]{-}{#3}{#5}
\ncarc[nodesepB=0,arcangle=-#1]{-}{#4}{#5}
\ncarc[arcangle=#2,nodesepB=\marnodesepB,offsetB=\maroffsetB]{->}{#5}{#6}
}%


\newcommandtwoopt{\nchmarr}[6][8][8]{%
\ncarc[nodesepA=\marnodesepA,nodesepB=0,arcangle=#1]{-}{#3}{#5}
\ncarc[nodesepB=0,arcangle=#1]{-}{#4}{#5}
\ncarc[arcangle=#2,nodesepB=\marnodesepB,offsetB=\maroffsetB]{->}{#5}{#6}
}%

\newcommandtwoopt{\ncamarr}[7][8][8]{%
\ncmarr[#1][#2]{#4}{#5}{#6}{#7}
\naput[npos=.05]{$#3$}
}%
\newcommandtwoopt{\ncbmarr}[7][8][8]{%
\ncmarr[#1][#2]{#4}{#5}{#6}{#7}
\nbput[npos=.05]{$#3$}
}%

\newcommandtwoopt{\ncbhmarr}[7][8][8]{%
\nchmarr[#1][#2]{#4}{#5}{#6}{#7}
\nbput[npos=.05]{$#3$}
}%

\newcommandtwoopt{\ncmarrr}[7][8][8]{
\ncarc[nodesepB=0,arcangle=#1]{-}{#3}{#6}
\ncline[nodesepB=0]{-}{#4}{#6}
\ncarc[nodesepB=0,arcangle=-#1]{-}{#5}{#6}
\ncarc[nodesepA=0,arcangle=#2]{->}{#6}{#7}
}

\newcommandtwoopt{\ncamarrr}[8][8][8]{
\ncmarrr[#1][#2]{#4}{#5}{#6}{#7}{#8}
\naput[npos=.05]{$#3$}
}
\newcommandtwoopt{\ncbmarrr}[8][8][8]{
\ncmarrr[#1][#2]{#4}{#5}{#6}{#7}{#8}
\nbput[npos=.05]{$#3$}
}

%gats.macros.tex

\usepackage{environ}    % also used in ermacros % here used for \NewEnvrion

\newcommand{\gat}[1][U]{
\ensuremath{\mathcal{#1}}}  % used to hav a space in here
\newcommand{\gatw}[1][U]{\gat[#1]\ }  % use this if need trailing space
\newcommand{\ingat}[1][U]{in \gat[#1]}
\newcommand{\isagat}[1][U]{\gat[#1] is a g.a.t.}
\newcommand{\inagat}{in a g.a.t. }

% macro for a generic theory
%\newcommand{\theory}
%{\textit{U}}

\newcommand{\intheory}
{is a derived rule of \gat[U]}

% Macros for GAT rules

\newcommand{\isT}[1]
{#1\mbox{ is a type}}

\newcommand{\ofT}[2]
{#1 \in #2
}

% Macros for GAT rules   <!-- new old -->
\newcommand{\istype}[1]
{#1\mbox{ is a type}}

\newcommand{\oftype}[2]
{#1 \in #2
}

%\context{x}{\Delta}{n}
\newcommand{\context}[3]
{\ofT{#1_1}{#2_1},... \ofT{#1_{#3}}{#2_{#3}(#1_1,...#1_{#3-1})}
}

%\subcontext{x}{\Delta}{i}{k}
\newcommand{\subcontext}[4]
{\ofT{#1_{#3_1}}{#2_{#3_1}},... \ofT{#1_{#3_#4}}{#2_{#3_#4}(#1_1,...#1_{#3_#4-1})}
}

% #schematic context
\newcommand{\schmcon}[3]
{\ofT{#1_1}{#2_1},... \ofT{#1_{#3}}{#2_{#3}}
}
% abbreviated to
\newcommand{\con}[3]
{\schmcon{#1}{#2}{#3}}

% schematic subcontext
%\subcon{x}{\Delta}{i}{k}
\newcommand{\subcon}[4]
{\ofT{#1_{#3_1}}{#2_{#3_1}},... \ofT{#1_{#3_#4}}{#2_{#3_#4}}
}

% permuted context
%\permcon{x}{\Delta}{n}{\sigma}
\newcommand{\permcon}[4]
{\ofT{#1_{#4(1)}}{#2_{#4(1)}},... \ofT{#1_{#4(#3)}}{#2_{#4(#3)}}
}
% permuted term
%\permterm{t}{n}{\sigma}
\newcommand{\permterm}[3]
{
#1_{#3(1)},...#1_{#3(#2)}
}


% Idioms
\newcommand{\xDelta}[1]{\con{x}{\Delta}{#1}}
\newcommand{\xDeltap}[1]{\con{x}{\Delta'}{#1}}
\newcommand{\xOmega}[1]{\con{x}{\Omega}{#1}}
\newcommand{\xOmegap}[1]{\con{x}{\Omega'}{#1}}
\newcommand{\yOmega}[1]{\con{y}{\Omega}{#1}}
\newcommand{\yOmegap}[1]{\con{y}{\Omega'}{#1}}

\newcommand{\xDeltasigma}[1]{\permcon{x}{\Delta}{#1}{\sigma}}
\newcommand{\xDeltapsigma}[1]{\permcon{x}{\Delta'}{#1}{\sigma}}
\newcommand{\xOmegasigma}[1]{\permcon{x}{\Omega}{#1}{\sigma}}
\newcommand{\xOmegapsigma}[1]{\permcon{x}{\Omega'}{#1}{\sigma}}
\newcommand{\yOmegasigma}[1]{\permcon{y}{\Omega}{#1}{\sigma}}
\newcommand{\yOmegapsigma}[1]{\permcon{y}{\Omega'}{#1}{\sigma}}

\newcommand{\xDeltainvsigma}[1]{\permcon{x}{\Delta}{#1}{\sigma^{-1}}}
\newcommand{\xDeltapinvsigma}[1]{\permcon{x}{\Delta'}{#1}{\sigma^{-1}}}
\newcommand{\xOmegainvsigma}[1]{\permcon{x}{\Omega}{#1}{\sigma^{-1}}}
\newcommand{\xOmegapinvsigma}[1]{\permcon{x}{\Omega'}{#1}{\sigma^{-1}}}
\newcommand{\yOmegainvsigma}[1]{\permcon{y}{\Omega}{#1}{\sigma^{-1}}}
\newcommand{\yOmegapinvsigma}[1]{\permcon{y}{\Omega'}{#1}{\sigma^{-1}}}

%Idioms enclosed as tuples
\newcommand{\encxDelta}[1]{\tuple{\con{x}{\Delta}{#1}}}
\newcommand{\encxDeltap}[1]{\tuple{\con{x}{\Delta'}{#1}}}
\newcommand{\encxOmega}[1]{\tuple{\con{x}{\Omega}{#1}}}
\newcommand{\encxOmegap}[1]{\tuple{\con{x}{\Omega'}{#1}}}
\newcommand{\encyOmega}[1]{\tuple{\con{y}{\Omega}{#1}}}
\newcommand{\encyOmegap}[1]{\tuple{\con{y}{\Omega'}{#1}}}

\newcommand{\encxDeltasigma}[1]{\tuple{\permcon{x}{\Delta}{#1}{\sigma}}}
\newcommand{\encxDeltapsigma}[1]{\tuple{\permcon{x}{\Delta'}{#1}{\sigma}}}
\newcommand{\encxOmegasigma}[1]{\tuple{\permcon{x}{\Omega}{#1}{\sigma}}}
\newcommand{\encxOmegapsigma}[1]{\tuple{\permcon{x}{\Omega'}{#1}{\sigma}}}
\newcommand{\encyOmegasigma}[1]{\tuple{\permcon{y}{\Omega}{#1}{\sigma}}}
\newcommand{\encyOmegapsigma}[1]{\tuple{\permcon{y}{\Omega'}{#1}{\sigma}}}

\newcommand{\encxDeltainvsigma}[1]{\tuple{\permcon{x}{\Delta}{#1}{\sigma^{-1}}}}
\newcommand{\encxDeltapinvsigma}[1]{\tuple{\permcon{x}{\Delta'}{#1}{\sigma^{-1}}}}
\newcommand{\encxOmegainvsigma}[1]{\tuple{\permcon{x}{\Omega}{#1}{\sigma^{-1}}}}
\newcommand{\encxOmegapinvsigma}[1]{\tuple{\permcon{x}{\Omega'}{#1}{\sigma^{-1}}}}
\newcommand{\encyOmegainvsigma}[1]{\tuple{\permcon{y}{\Omega}{#1}{\sigma^{-1}}}}
\newcommand{\encyOmegapinvsigma}[1]{\tuple{\permcon{y}{\Omega'}{#1}{\sigma^{-1}}}}

\newcommand{\tstyle}{\vdash}

% gat macros developed for cwf paper

% Expressing gats
\newenvironment{gatrules}
{
$$
\begin{array}{l l}
}
{
\end{array}
$$
}
\newcommand{\gatintros}
{
\textbf{Symbol} & \textbf{Introductory\ Rule}                      \\}

\newcommand{\gataxioms}
{\textbf{Axioms}\\}
\newcommand{\gatintro}[3]{\ #1 & #2 \tstyle #3 \\}
\newcommand{\gatlocalintro}[3]{\ #1 & #2 \dashv }
\newcommand{\gataxiom}[2]{\multicolumn{2}{l}{\ \ #1\mbox{,  whenever\ } #2} \\}
\newcommand{\noleft}{\left.\kern-\nulldelimiterspace} % so that no space taken by absent left brace


\newcommand{\gatmultiaxiom}[2]
{\multicolumn{2}{l}{
  \noleft
    \begin{array}{l}
		#1
    \end{array} 
  \right\} \mbox{whenever\ } 	#2 
	}\\}
	
	\newcommand{\axid}[1]{\text{#1}.\ }	

%New context sharing macros
\newcommand{\gatintroducing}[1]{
{\arraycolsep=0pt
  \begin{array}{l}
          #1
  \end{array}} &
}

%*********************************
% \begin{\gatgroup}{context}
%    rules
%  \end{\gatgroup}
%*********************************
\NewEnviron{gatgroup}[1]{%
  \noleft
  {\arraycolsep=0pt
   \begin{array}{l}
\BODY
    \end{array} 
   }
   \ \right\} 
	%\mbox{\ whenever\ } 
	#1
	\vspace{0.1cm} 
}
%*********************************

%*********************************
% \begin{\gatgroupnoshared}
%    rule
%  \end{\gatgroupnoshared}
%*********************************
\NewEnviron{gatgroupnoshared}{%
  {\arraycolsep=0pt
   \begin{array}{l}
\BODY
    \end{array} 
   }
   \ 
	\vspace{0.1cm} 
}
%*********************************

% \gatsingular[width]{context}{conclusion}
\newcommand{\gatsingular}[3][4cm]{
\begin{gatgroupnoshared}
\gatleaf[#1]{#2}{#3} 
\end{gatgroupnoshared}
}

%*********************************
% \gatleaf}[width]{context}{assertion}
%*********************************
\newcommand{\gatleaf}[3][4cm]{%
\makebox[#1]{$#3$ \dotfill} \dotfill \  #2
}
%*********************************
%*********************************
% \gatstandalonesingle}{context}{assertion}
%*********************************
\newcommand{\gatstandalonesingle}[2]{%
#2 \makebox[2.5cm]{\dotfill} \  #1
}
%*********************************

% \gataxiomno{axiomno}
\newcommand{\gataxiomno}[1]{\makebox[0.5cm]{} \axid{#1}}


% metagat.macros.tex

%Meta-theories

%\newcommand{\typ}{\triangleright}
\newcommand{\typ}{\nabla}
\newcommand{\trm}{\tau}
\newcommand{\cross}{\otimes}
\newcommand{\sub}{^*}
\newcommand{\diag}{\delta}

\newcommand{\typeseq}[2]
{\ofT{#1_1}{\typ},... \ofT{#1_{#2}}{\typ(#1_{#2-1})}}

\newcommand{\typeseqcont}[3]
{\ofT{#1_1}{\typ({#2})},... \ofT{#1_{#3}}{\typ(#1_{#3-1})}}

\newcommand{\Ob}{Ob}
\newcommand{\obj}{Ob} % <!-- new old --<
\newcommand{\Hom}{Hom}
\newcommand{\objseq}[2]
{\ofT{#1_1}{\obj},... \ofT{#1_{#2}}{\obj(#1_{#2-1})}}


\def\dottededge{\ncline[linestyle=dotted, nodesep=0.3cm]}
\def\noedge{\ncline[linestyle=none]}
\def\thinedge{\ncline[linewidth=0.4pt]}

\newcommand{\member}[1]
{\ncarc[arcangle=-30,nodesepB=0.03]{->}{\pspred}{\pssucc}
\nbput[labelsep=0.1]{#1}}

\newcommand{\loweraccutemember}[1]
{\ncarc[arcangle=-15,nodesepB=0.03]{->}{\pspred}{\pssucc}
\nbput[labelsep=0.05,npos=0.85]{#1}}

\newcommand{\uppermember}[1]
{\ncarc[arcangle=30,nodesepB=0.03]{->}{\pspred}{\pssucc}\naput{#1}}

\newcommand{\upperaccutemember}[1]
{\ncarc[arcangle=10,nodesepB=0.03]{->}{\pspred}{\pssucc}\naput[npos=0.85]{#1}}

% flexbranch 
% #1 node label
% #2 thislevelsep
% #3 next level sep
% #4 variable (eg x)
% #5 index leter (eg n)
% #6 close parenthesis
% #7 continuation branches
\newcommand{\flexbranch}[7]
{
\pstree[thislevelsep=*#2,nodesep=0.05]
		{\Rnode{#1 1}{\Tr{#4_1 #6}}}
	  {\pstree[thislevelsep=#3]  
				   {\Rnode{#1 2}{\Tr[edge=\dottededge]{#4_{#5} #6}}}
					 {#7}
		}
}

\newcommand{\flexbranchplusleaf}[6]
{
\flexbranch{#1}{#2}{#3}{#4} {#5} {#6}
  {
   %\Rnode{#1 3}{\Tr{#4 #6}}
	 \Tr{\Rnode{#1 3}{#4 #6}}
  }
}

\newcommand{\flexbranchplusarc}[7]
{
\flexbranch{#1}{#2}{#3}{#4} {#5} {#6}
  {
   %\Rnode{#1 3}{\Tr{#4 #6}\member{#7}}
	 \Tr{\Rnode{#1 3}{#4 #6}}\member{#7}
  }
}

\newcommand{\flexbranchinitialarc}[9]
{
\pstree[thislevelsep=*#2,nodesep=0.05]
		{\Rnode{#1 1}{\Tr{#4_#8 #6}}#9}
	  {\pstree[thislevelsep=#3]  
				   {\Rnode{#1 2}{\Tr[edge=\dottededge]{#4_{#5} #6}}}
					 {#7}
		}
}

\newcommand{\equality}[2]
{
\ncline [doubleline=true, nodesep=0.2cm]{#1}{#2}
}
\newcommand{\equalityarc}[2]
{
\ncarc [arcangleA=-30, arcangleB=-20, doubleline=true, nodesep=0.1cm]{#1}{#2}
}

\usepackage[margin=4.0cm]{geometry} %was 3cm
\usepackage{mathptmx}
\usepackage{amsfonts}
\usepackage{array}
\usepackage{pstricks}
\usepackage{pst-tree}
\usepackage{pst-plot}
\usepackage{pst-node}
\usepackage{stmaryrd}
\usepackage{amsmath}
\usepackage{verbatim}
\usepackage{graphicx}  
\usepackage{calc}
\usepackage{xifthen}
\usepackage{xcolor}
\usepackage{color}
\usepackage{stringstrings}
%\usepackage[small,bf,margin=3pt,format=hang, labelsep=endash,singlelinecheck=false]{caption} %prevuiously justification=justified
%\usepackage{enumerate}
%\usepackage{enumitem}
\usepackage{enumerate}
\usepackage[shortlabels]{enumitem}
\usepackage{float}
\usepackage[section]{placeins}
%\setlength{\captionmargin}{5pt}
\usepackage{environ}
\usepackage{multirow}
\usepackage{rotating}
\usepackage{longtable}
\usepackage{afterpage}
\usepackage{needspace}


%DEFINE ENVIRONMENT BLOCK
% Riddle
\newsavebox{\riddlebox}

\newenvironment{erexample}
{\newcommand\colboxcolor{F0F0F0}%was F8F8F8
\begin{lrbox}{\riddlebox}
\begin{minipage}{\dimexpr\columnwidth-2\fboxsep\relax} \textbf{} \\ \itshape}
{\end{minipage}\end{lrbox}%
%\begin{center}
\colorbox[HTML]{\colboxcolor}{\usebox{\riddlebox}}
%\end{center}
}

\newenvironment{erbox}
{\newcommand\colboxcolor{F0F0F0}%was F8F8F8
\begin{lrbox}{\riddlebox}%
\begin{minipage}{\dimexpr\columnwidth-2\fboxsep\relax} }
{\end{minipage}\end{lrbox}%
%\begin{center}
\colorbox[HTML]{\colboxcolor}{\usebox{\riddlebox}}
%\end{center}
}

%\begin{erboxedFigure}{#1 FigureParam}{#2 Label}{#3 Caption}
\NewEnviron{erboxedFigure}[3]{%
\begin{figure}[#1]
\begin{erexample}
\begin{center}
\BODY
\end{center}
\vspace{-0.5cm}
\caption{#3}
\label{#2}
\end{erexample}
\end{figure}
}

\newcommand{\erpictureFolder}[0]{../SharedPictures}

\newcommand{\ercenterPicture}[1]{
\begin{center}
\input{\erpictureFolder/#1}
\end{center}
}


\newlength{\erhalfHt}

%\erinlinePicture{#1 pictureFilename}{#2 pictureHeight}
\newcommand{\erinlinePicture}[2]{
\setlength{\erhalfHt}{#2cm * \real{0.5}}
\raisebox{-\erhalfHt}[\erhalfHt + 0.5cm][\erhalfHt + 0.5cm]{
\input{\erpictureFolder/#1}
} 
}

%\erplainFig{#1 pictureFilename}{#2 figureParam}{#3Caption}
\newcommand{\erplainFig}[3]{
\begin{figure}[#2]
\begin{center}
\input{\erpictureFolder/#1}
\end{center}
\caption{#3}
\label{#1}
\end{figure}
}

%\erboxedFigPicture{#1 pictureFilename}{#2 figureParam}{#3Caption}
\newcommand{\erboxedFigPicture}[3]{
\begin{figure}[#2]
\begin{erexample}
\vspace{-0.5cm}
\begin{center}
\input{\erpictureFolder/#1}
\end{center}
\caption{#3}
\label{#1}
\end{erexample}
\end{figure}
}

%\erLeftSideFig{#1 pictureFilename}{#2 figureParam}{#3Caption}
\newcommand{\erLeftSideFig}[3]{
\begin{figure}[#2]
\begin{erexample}
  \begin{minipage}[c]{0.4\textwidth}
    \caption{#3}
    \label{#1}
  \end{minipage}
  \begin{minipage}[c]{0.5\textwidth}
    \input{\erpictureFolder/#1}
  \end{minipage}
\end{erexample}
\end{figure}
}

%\erbulletedFig{#1 pictureFilename}{#2 figureParam}{#3Caption}
\NewEnviron{erbulletedFig}[3]{%
\begin{figure}[#2]
\begin{erexample}
\vspace{-0.5cm}
\begin{center}
$
\begin{array}{c m{0.25cm} | m{6cm}}
\raisebox{-2.0cm}{
\input{\erpictureFolder/#1}}& & \text{\parbox{6cm}{\raggedright{\footnotesize{
\begin{enumerate}[(i)]
\BODY
\end{enumerate}}}}} \\
\end{array}
$
\end{center}
\caption{#3}
\label{#1}
\end{erexample}
\end{figure} 
}


%\begin{erbulletedDimFig}{#1 pictureFilename}{#2figureParam} {#3Caption} {#4PictureHeight}{#5TextWidth}

\NewEnviron{erbulletedDimFig}[5]{%
\begin{figure}[#2]
\begin{erexample}
\vspace{-0.5cm}
\begin{center}
$
\begin{array}{c m{0.25cm} |  m{#5cm}}
\setlength{\erhalfHt}{#4cm * \real{0.5}}
\raisebox{-\erhalfHt}{
\input{\erpictureFolder/#1}}& & \text{\parbox{#5cm}{\raggedright{\footnotesize{
\begin{enumerate}[(i)]
\BODY
\end{enumerate}}}}} \\
\end{array}
$
\end{center}
\caption{#3}
\label{#1}
\end{erexample}
\end{figure} 
}

%\begin{ernotedModel}{#1 pictureFilename}{#2PictureHeight}{#3PictureWidth}{#4TextWidth}

\NewEnviron{ernotedModel}[4]{%
\begin{center}
$
\begin{array}{m{#3cm} m{1cm} | c m{#4cm}}
\setlength{\erhalfHt}{#2cm * \real{0.5}}
\raisebox{-\erhalfHt}{
\input{\erpictureFolder/#1}}& & & \text{\parbox{#4cm}{\raggedright{\footnotesize{
\BODY
}}}} \\
\end{array}
$
\end{center} 
}

%\begin{ermodelText}{#1 pictureFilename}{#2PictureHeight}{#3PictureWidth}{#4TextWidth}

\NewEnviron{ermodelText}[4]{%
\begin{center}
\begin{tabular}{m{#3cm} m{1cm}  c m{#4cm}}
\setlength{\erhalfHt}{#2cm * \real{0.5}}
\raisebox{-\erhalfHt}{
\input{\erpictureFolder/#1}}& & & \text{\parbox{#4cm}{\raggedright{\small{
\BODY
}}}} \\
\end{tabular}
\end{center} 
}


%\erbulletedModel{#1 pictureFilename}{#2PictureHeight}{#3PictureWidth}{#4TextWidth}

\NewEnviron{erbulletedModel}[4]{%
\begin{center}
$
\begin{array}{m{#3cm} m{1cm} | c m{#4cm}}
\setlength{\erhalfHt}{2cm * \real{0.5}}
\raisebox{-\erhalfHt}{
\input{\erpictureFolder/#1}}& & & \text{\parbox{#4cm}{\raggedright{\footnotesize{
\begin{enumerate}[(i)]
\BODY
\end{enumerate}}}}} \\
\end{array}
$
\end{center} 
}



%\ernotedDimFig{#1 pictureFilename}{#2 figureParam}{#3Caption}{#4PictureHeight}{#5TextWidth}
\NewEnviron{ernotedDimFig}[5]{%
\begin{figure}[#2]
\begin{erexample}
\vspace{-0.5cm}
\begin{center}
$
\begin{array}{c m{0.25cm} | c m{#5cm}}
\setlength{\erhalfHt}{#4cm * \real{0.5}}
\raisebox{-\erhalfHt}{
\input{\erpictureFolder/#1}}& & & \text{\parbox{#5cm}{\raggedright{\footnotesize{
\BODY }}}}\\
\end{array}
$
\end{center}
\caption{#3}
\label{#1}
\end{erexample}
\end{figure} 
}
%\begin{ernotedDimFigPW}{#1 pictureFilename}{#2 figureParam}{#3Caption}{#4PictureHeight}{#5PictureWidth}{#6TextWidth}
\NewEnviron{ernotedDimFigPW}[6]{%
\begin{figure}[#2]
\begin{erexample}
\vspace{-0.5cm}
\begin{center}
$
\begin{array}{>{\centering}m{#5cm} m{0.5cm} | c m{#6cm}}
\setlength{\erhalfHt}{#4cm * \real{0.5}}
\raisebox{-\erhalfHt}{
\centering \input{\erpictureFolder/#1}
}& & & \text{\parbox{#6cm - 0.5cm}{\raggedright{\footnotesize{
\BODY }}}}\\
\end{array}
$ \\
\vspace {0.2cm}
\end{center}
\caption{#3}
\label{#1}
\end{erexample}
\end{figure}
}



\newenvironment{erquote}
{\begin{quote}\itshape}
{\end{quote}}


%
%  erdiag
%
  
%\begin{erdiagram}{#1 height}{#2 width} 
% ....
% ....
%\end{erdiagram}
\newenvironment{erdiagram}[2]
{%\pspicture*(-#1,0)(#2,0)
\pspicture*(0,-#1)(#2,0)
%\psgrid
}
{\endpspicture}

\definecolor{lightyellow}{cmyk}{0,0,0.3,0}

% \eret{#1 x0} {#2 y0} {#3 x1} {#4 y1} {#5 corner radius} {#6 fill}
\newcommand {\eret}[6]
{ 
\ifthenelse{\equal{#6}{1}}
{\psframe[framearc=#5,fillstyle=solid,fillcolor=lightyellow](#1,#2)(#3,#4)}
{\psframe[framearc=#5,fillstyle=solid,fillcolor=white](#1,#2)(#3,#4)}
}

% et top 
\newcommand {\erettop}[4]
{
%\psframe[linestyle=none,linearc=2pt,cornersize=absolute,fillstyle=solid,fillcolor=lightyellow](#1,#2)(#3,#4)
\psline[linearc=2pt,fillstyle=none,fillcolor=lightyellow](#1,#4)(#1,#2)(#3,#2)(#3,#4)
}

% et bottom 
\newcommand {\eretbtm}[4]
{
%\psframe[linestyle=none,linearc=2pt,cornersize=absolute,fillstyle=solid,fillcolor=lightyellow](#1,#2)(#3,#4)
\psline[linearc=2pt,fillstyle=none,fillcolor=lightyellow](#1,#2)(#1,#4)(#3,#4)(#3,#2)
}

% et bottom left
\newcommand {\eretbl}[4]
{
%\psframe[linestyle=none,linearc=2pt,cornersize=absolute,fillstyle=solid,fillcolor=lightyellow](#1,#2)(#3,#4)
\psline[linearc=2pt,fillstyle=none,fillcolor=lightyellow](#1,#4)(#3,#4)(#3,#2)
}

% et middle left
\newcommand {\eretml}[4]
{
%\psframe[linestyle=none,linearc=2pt,cornersize=absolute,fillstyle=solid,fillcolor=lightyellow](#1,#2)(#3,#4)
\psline[linearc=2pt,fillstyle=none,fillcolor=lightyellow](#1,#2)(#3,#2)(#3,#4)(#1,#4)
}

% et top left
\newcommand {\erettl}[4]
{
%\psframe[linestyle=none,linearc=2pt,cornersize=absolute,fillstyle=solid,fillcolor=lightyellow](#1,#2)(#3,#4)
\psline[linearc=2pt,fillstyle=none,fillcolor=lightyellow](#1,#2)(#3,#2)(#3,#4)
}

% et bottom right
\newcommand {\eretbr}[4]
{
%\psframe[linestyle=none,linearc=2pt,cornersize=absolute,fillstyle=solid,fillcolor=lightyellow](#1,#2)(#3,#4)
\psline[linearc=2pt,fillstyle=none,fillcolor=lightyellow](#1,#2)(#1,#4)(#3,#4)
}

% et middle right
\newcommand {\eretmr}[4]
{
%\psframe[linestyle=none,linearc=2pt,cornersize=absolute,fillstyle=solid,fillcolor=lightyellow](#1,#2)(#3,#4)
\psline[linearc=2pt,fillstyle=none,fillcolor=lightyellow](#3,#4)(#1,#4)(#1,#2)(#3,#2)
}

% et top right
\newcommand {\erettr}[4]
{
%\psframe[linestyle=none,linearc=2pt,cornersize=absolute,fillstyle=solid,fillcolor=lightyellow](#1,#2)(#3,#4)
\psline[linearc=2pt,fillstyle=none,fillcolor=lightyellow](#1,#4)(#1,#2)(#3,#2)
}

% \ergrp{#1 x0} {#2 y0} {#3 x1} {#4 y1} {#5 corner radius} {#6 fill}
% #5 corner radius is unused!
\newcommand {\ergrp}[6]
{ 
\ifthenelse{\equal{#6}{1}}
{\psframe[fillstyle=solid,fillcolor=lightgray](#1,#2)(#3,#4)}
{\psframe[fillstyle=solid,fillcolor=white](#1,#2)(#3,#4)}
}

% \eretname {#1 x left of text} {#2 y top of text} {#3 text}
\newcommand {\eretname}[3]
{
%shift down 0.1 for height of text the anchor at baseline (B)
\rput[bl]{0}(0,-0.1){\rput[Bl]{0}(#1,#2){\footnotesize \textit{#3}}}
}

% \errelarm {#1 x0} {#2 y0} {#3 x1} {#4 y1} {#5 ismandatory} {#6 isconstructed}
\newcommand {\errelarm}[6]
{
\ifthenelse{\equal{#6}{1}}
{
%%\psline[linewidth=0.5pt,linearc=.05,linestyle=dashed,dash=6pt 6pt]{-}(#1,#2)(#3,#4)}
\ifthenelse{\equal{#5}{1}}
{\psline[linewidth=1.5pt,linearc=.05,linecolor=lightgray]{-}(#1,#2)(#3,#4)}
{\psline[linewidth=1.5pt,linearc=.05,linecolor=lightgray,linestyle=dashed,dash=2pt 2pt]{-}(#1,#2)(#3,#4)}
}
{
\ifthenelse{\equal{#5}{1}}
{\psline[linewidth=0.9pt,linearc=.05]{-}(#1,#2)(#3,#4)}
{\psline[linewidth=0.9pt,linearc=.05,linestyle=dashed,dash=2pt 2pt]{-}(#1,#2)(#3,#4)}
}
}

% \errelangle {#1 x0} {#2 y0} {#3 x1} {#4 y1} {#5 x2} {#6 y2} {#7 ismandatory} {#8 isocnstructed}
\newcommand {\errelangle}[8]
{
\ifthenelse{\equal{#8}{1}}
{
%\psline[linewidth=0.5pt,linearc=.1,linestyle=dashed,dash=6pt 6pt]{-}(#1,#2)(#3,#4)(#5,#6)}
\ifthenelse{\equal{#7}{1}}
{\psline[linewidth=1.5pt,linearc=.05,linecolor=lightgray]{-}(#1,#2)(#3,#4)(#5,#6)}
{\psline[linewidth=1.5pt,linearc=.1,linecolor=lightgray,linestyle=dashed,dash=2pt 2pt]{-}(#1,#2)(#3,#4)(#5,#6)}
}
{
\ifthenelse{\equal{#7}{1}}
{\psline[linewidth=0.9pt,linearc=.1]{-}(#1,#2)(#3,#4)(#5,#6)}
{\psline[linewidth=0.9pt,linearc=.1,linestyle=dashed,dash=2pt 2pt]{-}(#1,#2)(#3,#4)(#5,#6)}
}
}

% \ercrowfoot {#1 x0} {#2 y0} {#3 x11} {#4 y11} {#5 x12} {#6 y12} {#7 x13} {#8 y13} {#9 isconstructed}
\newcommand {\ercrowfoot}[9]
{
\ifthenelse{\equal{#9}{1}}
{
\psline[linewidth=1.5pt,linearc=.05,linecolor=lightgray]{-}(#1,#2)(#3,#4)
\psline[linewidth=1.5pt,linearc=.05,linecolor=lightgray]{-}(#1,#2)(#5,#6)
\psline[linewidth=1.5pt,linearc=.05,linecolor=lightgray]{-}(#1,#2)(#7,#8)
}{
\psline[linewidth=0.9pt,linearc=.05]{-}(#1,#2)(#3,#4)
\psline[linewidth=0.9pt,linearc=.05]{-}(#1,#2)(#5,#6)
\psline[linewidth=0.9pt,linearc=.05]{-}(#1,#2)(#7,#8)
}
}


% \eridcomprel{#1 x1}{#2 x2}{#3 y1}{#4 ymid}{#5 y2}
\newcommand {\eridcomprel}[5]
{
\psline[linewidth=0.9pt](#1,#3)(#1,#5)
\psline[linewidth=0.9pt](#2,#3)(#2,#5)
\psline[linewidth=0.9pt](#1,#4)(#2,#4)
}

% \eridrefrel{#1 x1}{#2 xmid}{#3 x2}{#4 y1}{#5 y2}
\newcommand {\eridrefrel}[5]
{
\psline[linewidth=0.9pt](#1,#4)(#3,#4)
\psline[linewidth=0.9pt](#1,#5)(#3,#5)
\psline[linewidth=0.9pt](#2,#4)(#2,#5)
}


% \errelname {#1 x} {#2 y} {#3 text}
\newcommand {\errelname}[3]
{
\rput[l]{0}(#1,#2){\textit{#3}}
}
% \errelseq {#1 x} {#2 y}
\newcommand {\erelseq}[2]
{
}
% \erattr {#1 x} {#2 y} {#3 ismandatory}{#4 idenitfying} {#5 text}
\newcommand {\erattr}[5]
{
\ifthenelse{\equal{#3}{1}}
{\rput[l]{0}(#1,#2){{\tiny $\square$} {\footnotesize \textit{\ifthenelse{\equal{#4}{0}}{\underline{#5}}{#5}}}}}
{\rput[l]{0}(#1,#2){\footnotesize $\circ$ \textit{\ifthenelse{\equal{#4}{0}}{\underline{#5}}{#5}}}}
}

%\ifthenelse{\equal{#4}{1}}
% \ertext {#1 x} {#2 y} {#3 text anchor} {#4 text}
%{\rput[l]{0}(#1,#2){\footnotesize $\circ$ \underline{\textit{#5}}}}
\newcommand {\ertext}[4]
{
\rput[B#3]{0}(#1,#2){{\footnotesize #4}}
}
% \erarc {#1 x0} {#2 y0} {#3 x1} {#4 y1} {#5 x2} {#6 y2} {#7 x3} {#8 y3}
\newcommand {\erarc}[8]
{
\psbezier[showpoints=false]{-}(#1,#2) (#3, #4)(#5,#6) (#7, #8)
}

% \erarc {#1 x0} {#2 y0} {#3 x1} {#4 y1} {#5 x2} {#6 y2} {#7 x3} {#8 y3}
\newcommand {\errelseq}[8]
{
\psbezier[showpoints=false]{-}(#1,#2) (#3, #4)(#5,#6) (#7, #8)
}
% \ertrace {#1 trace}   
\newcommand {\ertrace}[1]
{
}
    %beamer aware version
%All these macros are copied from SharedMacros/general.tex which doesnt seem to work with beamer
% Some macros in SharedMacros/general.tex thought to have name clashes with beamer.


\newcommand{\fundep}[3]{#2 \xrightarrow{#1} #3}                                                 
\newcommand{\term}[1]{\textit{#1}} 
\newcommand{\setsuchthat}[2]{\left\{#1 \ \middle|\ #2\right\}}
\newcommand{\set}[1]{\left\{#1\right\}}



\newcommand{\wanton}[1]{#1_1,...#1_n}
\newcommand{\ntuple}[1]{\tuple{\wanton{#1}}}

\newcommand{\xntuple}{\ensuremath{\ntuple{x}}}

% maybe not in general.macros 
\newcommand{\xnset}{\ensuremath{\set{\wanton{x}}}}
%
% othermacros
%

% copied and edited from \idcomp to make stronger linestyle
\newcommand{\addedgebar}{
\ncput[npos=0, nrot=:U]{\psline[linewidth=1.25pt](0.2,-0.1)(0.2,0.1)}
}
\newcommand{\addedgedoublebar}{
\ncput[npos=0, nrot=:U]{\psline[linewidth=1.25pt](0.2,-0.1)(0.2,0.1)}
\ncput[npos=0, nrot=:U]{\psline[linewidth=1.25pt](0.3,-0.1)(0.3,0.1)}
}
\newcommand{\addedgetriplebar}{
\ncput[npos=0, nrot=:U]{\psline[linewidth=1.25pt](0.2,-0.1)(0.2,0.1)}
\ncput[npos=0, nrot=:U]{\psline[linewidth=1.25pt](0.3,-0.1)(0.3,0.1)}
\ncput[npos=0, nrot=:U]{\psline[linewidth=1.25pt](0.4,-0.1)(0.4,0.1)}
}

%\newcommand{\addedgebar}{\ifbars{\ncput[npos=0, nrot=:U]{\psline(0.2,-0.075)(0.2,0.075)}}\fi}

%copied from database literature review
\newcommand{\displaybibentry}[1]
{\begin{framed}
\bibentry{#1}
\end{framed}
}

% used in data tables
\newcommand{\colhead}[1]{\textbf{\textcolor{white}{#1}}}
\definecolor{myblue}{RGB}{71,71,186}
\newcommand{\largeAsterisk}{\mathop{\scalebox{1.5}{\raisebox{-0.2ex}{$\ast$}}}}
\newcommand{\fk}[1]{#1$^{\largeAsterisk}$}
\newcommand{\pk}[1]{\underline{#1}}
\newcommand{\seck}[1]{\dashuline{#1}}  % secondary key
\newcommand{\pkfk}[1]{\underline{#1}$^{\largeAsterisk}$} % primary key that is a foreign key
% \vpad gives vertical padding in a tabular
\newcommand{\vpad}[1]{\multicolumn{#1}{c}{}\\[-0.25cm]}
% used in slides
\newcommand{\outerbullet}{{$\color{blue}{\blacktriangleright}$}\ }% please dont remove final space
\newcommand{\innerbullet}{{\footnotesize $\color{blue}{\blacktriangleright}$}\ }% please dont remove final space
\newcommand{\braceLabel}[3]{\psbrace[ref=lC,braceWidth=1pt,braceWidthInner=3pt,braceWidthOuter=3pt](#2)(#1){#3} }

% words words words
\newcommand{\catMEterm}{category with designated monomorphisms and epimorphisms\ }
\newcommand{\IfSforCwithRCwords}{
If $S$ is a sketch for category \catcw considered as a data specification with requirement $\reqtc$\ }
\newcommand{\IfSforCwithRCwordsvariant}{
If $S$ is a sketch for structured category \catcw and if $S$ is considered as a data specification with requirement $\reqtc$\ }
\newcommand{\IfSforepimonoCwithRCwords}{
If $S$ is a sketch for a category \catcw with designated monomorphisms and epimorphisms considered as a data specification with requirement $\reqtc$\ }
\iffalse
\newcommand{\scmonosketchwording}{
If $S$ is a sketch for such a category
%of a category with finite products and designated monomorphisms and epimorphisms
considered as a data specification
with requirement $\reqtc$\ }
\fi
\newcommand{\spacechar}{\ }
\newcommand{\thirdstructure}{designated monomorphisms and epimorphisms and with finite products}
\newcommand{\IfSforproductepimonoCwithRCwords}{
If $S$ is a sketch for  a category with \thirdstructure \spacechar
%category \catcw with finite products and designated monomorphisms and epimorphisms 
considered as a data specification with requirement $\reqtc$\ }

\newcommand{\goodnesscriteria}[1]{\textbf{Goodness Criteria #1:}}

\newcommand{\goodnessoneA}{
\goodnesscriteria{1A} There ought not to be an edge $e$ in $G$ for which there is an equivalent path $p$ which  does not containing $e$
}
\newcommand{\goodnessoneB}{
\goodnesscriteria{1B} 
There ought not exist $d \in PE$ such that $d \in \overline{PE \setminus d}$
}

\newcommand{\goodnessoneC}{
\goodnesscriteria{1C} \\
There ought not exist $m \in M$ such that $m \in \overline{M \setminus m}$
}
\newcommand{\goodnessoneD}{
\goodnesscriteria{1D} \\
There ought not exist $e \in E$ such that $e \in \overline{E \setminus e}$
}




% From the Mathematical Theory of data paper
\newcommand{\ssfd}[2]{\ensuremath{#1 \morph #2}}  % singleton-singleton
\newcommand{\smfd}[2]{\ensuremath{\ssfd{#1}{\set{#2}}}}  % singleton-many
\newcommand{\msfd}[2]{\ensuremath{\ssfd{\set{#1}}{#2}}}  % many-singleton
\newcommand{\mmfd}[2]{\ensuremath{\msfd{#1}{\set{#2}}}}  % many-many



% All these should find a home in SharedMacros eventually 

% Commands for making a bit of vertical space. used when arrows and particularly labels of arrows
% use spec that is otherwise accounted for.
\newcommand{\seeroomup}[1]{\rule{0.1cm}{#1}}
\newcommand{\seeroomdown}[1]{\rule[-#1]{0.1cm}{0.1cm}}
\newcommand{\roomup}[1]{\rule{0cm}{#1}}
\newcommand{\roomdown}[1]{\rule[-#1]{0cm}{0.1cm}}


% BOX DIAGRAMS
\newcommand{\attr}[1]{#1}
\renewcommand{\attr}[1]{\psframebox[linecolor=red,framearc=.1]{#1}}
\newcommand{\attrtype}[1]{#1}
\renewcommand{\attrtype}[1]{\psframebox[linecolor=blue,framearc=.1]{#1}}
\newcommand{\etype}[1]{#1}
\renewcommand{\etype}[1]{\psframebox[linecolor=red,framearc=.1]{#1}}


\newcommand{\regularizetextheight}{\roomup{0.3cm}\roomdown{0.1cm}}

\newcommand{\unarystructurediagramnodes}[3][]{
\rput[tc](2.4,3){\Rnode{#1A}{\psframebox[framesep=10pt]{\regularizetextheight#2}}} 
\rput[tc](2.4,1){\rnode{#1B}{\psframebox[framesep=10pt]{\regularizetextheight#3}}}           
}

\newcommand{\binarystructurediagramnodes}[4][]{
\rput[tr](4.0,3){\Rnode{#1A}{\psframebox[framesep=10pt]{\regularizetextheight#2}}} 
\rput[tr](2.4,1){\rnode{#1B}{\psframebox[framesep=10pt]{\regularizetextheight#3}}}     
\rput[tr](5.6,1){\rnode{#1C}{\psframebox[framesep=10pt]{\regularizetextheight#4}}}        
}

\newcommand{\triplestructurediagramnodes}[5][]{ 
\rput[tc](2.0,3){\rnode{#1A}{\psframebox[framesep=10pt]{\regularizetextheight#2}}}     
\rput[tc](-1.05,1){\rnode{#1B}{\psframebox[framesep=10pt]{\regularizetextheight#3}}}  
\rput[tc](2.0,1){\rnode{#1C}{\psframebox[framesep=10pt]{\regularizetextheight#4}}}   
\rput[tc](5.0,1){\rnode{#1D}{\psframebox[framesep=10pt]{\regularizetextheight#5}}}   
}

\newcommand{\jacksonbinarydiagram}[3]
{
\pspicture(-0.4,0)(5.7,3)  % lower left is 0,0 upper right is 8,3
%\psgrid
\binarystructurediagramnodes{#1}{#2}{#3}
\rput[tr](2.3,0.9){*}
\rput[tr](5.4,0.9){*}
\ncangle[offsetA=-0.5cm, angleA=-90,angleB=90,armB=0.5cm]{A}{B}
\ncangle[offsetA=0.5cm, angleA=-90,angleB=90,armB=0.5cm]{A}{C}
\endpspicture      
}

\newcommand{\bachmanbinarydiagram}[4][]
{
\pspicture(-0.4,0)(5.7,3)  % lower left is 0,0 upper right is 8,3
%\psgrid
\binarystructurediagramnodes[#1]{#2}{#3}{#4}
\ncline[linewidth=3pt]{->}{#1A}{#1B}
\ncline[linewidth=3pt]{->}{#1A}{#1C}
\endpspicture      
}

\newcommand{\unarystructurediagram}[3][]
{
\pspicture(0.9,0)(3.9,3.5)  
%\psgrid
\unarystructurediagramnodes[#1]{#2}{#3}
\endpspicture      
}

\newcommand{\binarystructurediagram}[4][]
{
\pspicture(-0.4,0)(5.7,3)  
%\psgrid
\binarystructurediagramnodes[#1]{#2}{#3}{#4}
\endpspicture      
}

\newcommand{\triplestructurediagram}[5][]
{
\pspicture(-2.5,0)(6.4,3.5)  
%\psgrid
\triplestructurediagramnodes[#1]{#2}{#3}{#4}{#5}
\endpspicture      
}


\newcommand{\binarynetworkdiagramnodes}[3]{ 
\rput[tr](2.4,3){\rnode{A}{\psframebox[framesep=10pt]{\regularizetextheight#1}}}     
\rput[tr](5.6,3){\rnode{B}{\psframebox[framesep=10pt]{\regularizetextheight#2}}} 
\rput[tr](4.0,1){\Rnode{C}{\psframebox[framesep=10pt]{\regularizetextheight#3}}}       
}

\newcommand{\bachmannetworkdiagram}[3]
{
\pspicture(-0.4,0)(5.7,3)  % lower left is 0,0 upper right is 8,3
%\psgrid
\binarynetworkdiagramnodes{#1}{#2}{#3}
\ncline[linewidth=3pt]{->}{A}{C}
\ncline[linewidth=3pt]{->}{B}{C}
\endpspicture      
}

%craft bachman nwtrok share diagram
\newcommand{\doublebinarynetworkdiagramnodes}[6]{ 
\rput[tc](-2.5,3){\rnode{A}{\psframebox[framesep=10pt]{\regularizetextheight#1}}}  
\rput[tc](2.0,3){\rnode{B}{\psframebox[framesep=10pt]{\regularizetextheight#2}}}    
\rput[tc](-4.1,1){\rnode{C}{\psframebox[framesep=10pt]{\regularizetextheight#3}}} 
\rput[tc](-1.05,1){\rnode{D}{\psframebox[framesep=10pt]{\regularizetextheight#4}}}  
\rput[tc](2.0,1){\rnode{E}{\psframebox[framesep=10pt]{\regularizetextheight#5}}}   
\rput[tc](5.0,1){\rnode{F}{\psframebox[framesep=10pt]{\regularizetextheight#6}}}   
}

\newcommand{\doublebachmannetworkdiagram}[6]
{
\pspicture(-5.6,0)(6.5,3)  % lower left is 0,0 upper right is 8,3
%\psgrid
\doublebinarynetworkdiagramnodes{#1}{#2}{#3}{#4}{#5}{#6}
\ncline[linewidth=3pt]{->}{A}{C}
\ncline[linewidth=3pt]{->}{A}{D}
\ncline[linewidth=3pt]{->}{B}{D}
\ncline[linewidth=3pt]{->}{B}{E}
\ncline[linewidth=3pt]{->}{B}{F}
\endpspicture      
}

\newcommand{\doublecategorynetworkdiagram}[6]
{
\pspicture(-5.6,0)(6.5,3)  % lower left is 0,0 upper right is 8,3
%\psgrid
\doublebinarynetworkdiagramnodes{#1}{#2}{#3}{#4}{#5}{#6}
\ncarr{C}{A}
\ncarr{D}{A}
\ncarr{D}{B}
\ncarr{E}{B}
\ncarr{F}{B}
\endpspicture      
}

\newcommand{\mixedcategorynetworkdiagram}[6]
{
\pspicture(-5.6,0)(6.5,3)  % lower left is 0,0 upper right is 8,3
%\psgrid
\doublebinarynetworkdiagramnodes{#1}{#2}{#3}{#4}{#5}{#6}
\ncline[linewidth=2.5pt]{->}{C}{A}
\ncline[linewidth=2.5pt]{->}{D}{A}
\ncarr{D}{B}
\ncline[linewidth=2.5pt]{->}{E}{B}
\ncline[linewidth=2.5pt]{->}{F}{B}
\endpspicture      
}

\newcommand{\contextualcategoryblockstyleexamplekernel}[6]{
\pspicture(-5.6,0)(6.5,3)  % lower left is 0,0 upper right is 8,3
%\psgrid
\doublebinarynetworkdiagramnodes{#1}{#2}{#3}{#4}{#5}{#6}
\ncsar{C}{A}
\ncsar{E}{B}
\ncsar{F}{B}
\endpspicture
}

\newcommand{\contextualcategorynetworkdiagram}[6]
{
\contextualcategoryblockstyleexamplekernel{#1}{#2}{#3}{#4}{#5}{#6}
\ncsar{D}{A}
\ncarr{D}{B}
}

\newcommand{\contextualcategorynetworkdiagramreorganised}[6]
{
\contextualcategoryblockstyleexamplekernel{#1}{#2}{#3}{#4}{#5}{#6}
\ncarr{D}{A}
\ncsar{D}{B}
}


\newcommand{\contextualcategorynetworkdiagramtopologised}[6]
{
\begin{tabular}{c c c}
\scalebox{0.9}{\binarystructurediagram[left]{compound\kern0.1cm}{alias \kern1.2cm}{occurence}}
&&
\scalebox{0.9}{\binarystructurediagram[right]{element\kern0.4cm}{valency \kern0.8cm}{allotrope\kern0.3cm}}
\end{tabular}
\ncangle[offsetA=0.15cm, angleA=0,offsetB=-0.25cm, angleB=180, armB=2.5cm]{->}{leftC}{rightA}
\ncsar{leftB}{leftA}
\ncsar{leftC}{leftA}
\ncsar{rightB}{rightA}
\ncsar{rightC}{rightA}
}

\newcommand{\contextualcategorynetworkdiagramreorganisedtopologised}[6]
{
\begin{tabular}{c c c}
\scalebox{0.9}{\unarystructurediagram[left]{compound\kern0.1cm}{alias \kern1.2cm}}
&&
\scalebox{0.9}{\triplestructurediagram[right]{element\kern0.4cm}{occurence}{valency \kern0.8cm}{allotrope\kern0.3cm}}
\end{tabular}
\ncangle[offsetA=0.15cm, angleA=180,offsetB=-0.25cm, angleB=0, armB=0.9cm]{->}{rightB}{leftA}
\ncsar{leftB}{leftA}
\ncsar{leftC}{leftA}
\ncsar{rightB}{rightA}
\ncsar{rightC}{rightA}
\ncsar{rightD}{rightA}
}

\iffalse
\newcommand{\contextualcategorynetworkdiagramreorganised}[6]
{
\pspicture(-5.6,0)(6.5,3)  % lower left is 0,0 upper right is 8,3
%\psgrid
\doublebinarynetworkdiagramnodes{#1}{#2}{#3}{#4}{#5}{#6}
\ncsar{C}{A}
\ncarr{D}{A}
\ncsar{D}{B} 
\ncsar{E}{B}
\ncsar{F}{B}
\endpspicture      
}
\fi

% Category DIAGRAMS START HERE


\newcommand{\factorisationfdiagram}{
    $
    \begin{array}{c p{1cm} c p{1.0cm} c}
    \Rnode{a}{a}&&\Rnode{Imf}{Im(f)}&&\Rnode{b}{b}
    \end{array}
    \begin{arrows}
    \ncline{->>}{a}{Imf}\alabel{f_e}
    \ncarr{Imf}{b}\alabel{f_m}\idcomp
    \end{arrows}
    $
}
\newcommand{\nakedbinarysourcediagram}[5]{
\begin{array}{c p{0.5cm} c}
             &&   \Rnode{b}{#2}\\[0.01cm]
\Rnode{a}{#1} &&               \\[0.01cm] 
             &&   \Rnode{c}{#3}
\end{array} 
\begin{arrows}
\ncarr{a}{b}
\alabel{#4}
\ncarr{a}{c}
\blabel{#5}
\end{arrows}
}

\newcommand{\binarysourcediagram}[5]{$\nakedbinarysourcediagram{#1}{#2}{#3}{#4}{#5}$}
\newcommand{\fgsourcediagram}{\binarysourcediagram{a}{b}{c}{f}{g}}

%  binary source diagram with arrows pointing SE and SW
% nakedSWSEsourcediagram{prefix}{a}{b}{c}{f}{g}
\newcommand{\nakedSWSEsourcediagram}[6]{
\begin{array}{c c c}
              & \Rnode{#1a}{#2} &               \\[1.0cm] 
\Rnode{#1b}{#3} &               &\Rnode{#1c}{#4}
\end{array} 
\begin{arrows}
\ncarr{#1a}{#1b}
\alabel{#5}
\ncarr{#1a}{#1c}
\blabel{#6}
\end{arrows}
}


%  binary sink diagram with arrows pointing SE and SW
\newcommand{\nakedNWNEsinkdiagram}[5]{
\begin{array}{c c c}
              & \Rnode{a}{#1} &               \\[0.5cm] 
\Rnode{b}{#2} &               &\Rnode{c}{#3}
\end{array} 
\begin{arrows}
\ncarr{b}{a}
\alabel{#4}
\ncarr{c}{a}
\blabel{#5}
\end{arrows}
}

\newcommand{\simpleunaryfdrepresentationdiagram}[6]{
$
\nakedbinarysourcediagram{#1}{#2}{#3}{#4}{#5}
\begin{arrows}
\ncarr{b}{c}
\alabel{#6}
\end{arrows}
$
}

\newcommand{\unaryfdrepresentationdiagram}[8]{
$
\begin{array}{c p{0.2cm} c}
\nakedbinarysourcediagram{#1}{#2}{#3}{#4}{#5}&& \Rnode{d}{#6}
\end{array}
\begin{arrows}
\ncarr{d}{b}
\idcomp
\blabel{#7}
\ncarr{d}{c}
\alabel{#8}
\end{arrows}
$
}

\newcommand{\unaryfdrepresentationmappeddiagram}[8]{
$
\begin{array}{c p{0.2cm} c}
\nakedbinarysourcediagram{D(#1)}{D(#2)}{D(#3)}{D(#4)}{D(#5)}&& \Rnode{d}{D(#6)}
\end{array}
\begin{arrows}
\ncarr{b}{d}
\alabel{D(#7)^-1}
\ncarr{d}{c}
\alabel{D(#8)}
\end{arrows}
$
}

\newcommand{\commutativetrianglediagram}[6]{
$
\begin{array}{c p{0.4cm} c p{0.4cm} c}
              && \Rnode{b}{#2}  &&                 \\[0.6cm]
\Rnode{a}{#1} &&                && \Rnode{c}{#3}  
\end{array}
\begin{arrows}
\ncarr{a}{b}
\alabel{#4}
\ncarr{b}{c}
\alabel{#5}
\ncarr{a}{c}
\blabel{#6}
\end{arrows}
$
}

\newcommand{\commutativetrianglediagrammutant}[6]{
$
\begin{array}{c  c  c}
              & \Rnode{b}{#2}  &                 \\[0.85cm]
\Rnode{a}{#1} &                & \Rnode{c}{#3}  
\end{array}
\begin{arrows}
\ncarr{a}{b}
\alabel{#4}[0.15]
\ncarr{b}{c}
\alabel{#5}[0.6]
\ncarr{a}{c}
\blabel{#6}
\end{arrows}
$
}

\newcommand{\epimonosplitdiagram}[3]{
\commutativetrianglediagram{#1}{img(#3)}{#2}{#3_e}{#3_m}{#3}   
}


\iffalse %saved
\begin{array}{c p{2.0cm} c }                
               &&  \Rnode{b1}{#3_1}    \\ [0.75cm]
               &&  \Rnode{b2}{#3_2}    \\ [0.5cm]
\Rnode{a}{#2}  &&                      \\ [-0.5cm]
               &&       \vdots         \\ [0.85cm]
               &&  \Rnode{bn}{#3_{#1}}  
\end{array}
\fi

%nakedmultisourceobjects{n}{a}{b}
\newcommand{\nakedmultisourceobjects}[3]{
\begin{array}{c p{2.0cm} c }
\Rnode{a}{#2}   &&
\begin{array}{c }                
\Rnode{b1}{#3_1}   \\ [0.75cm]
\Rnode{b2}{#3_2}   \\ [0.25cm]
\vdots             \\ [0.35cm]
\Rnode{bn}{#3_{#1}}  
\end{array}
\end{array}
}

% \nakedmultisourcediagram{n}{a}{b}{f}
\newcommand{\nakedmultisourcediagram}[4]{
\nakedmultisourceobjects{#1}{#2}{#3}
\begin{arrows}
\ncarr{a}{b1}
\alabel{#4_1}[0.5]
\ncarr{a}{b2}
\alabel{#4_2}[0.5][-1]
\ncarr{a}{bn}
\blabel{#4_{#1}}[0.5][-1]
\end{arrows}
}

% \nakedmultisourcepathdiagram{n}{a}{b}{f}
\newcommand{\nakedmultisourcepathdiagram}[4]{
\nakedmultisourceobjects{#1}{#2}{#3}{#4}
\begin{arrows}
\simplepath{a}{b1}
\alabel{#4_1}[0.5]
\simplepath{a}{b2}
\alabel{#4_2}[0.5][-1]
\simplepath{a}{bn}
\blabel{#4_{#1}}[0.5][-1]
\end{arrows}
}


\newcommand{\multisourcediagram}[4]{$\nakedmultisourcediagram{#1}{#2}{#3}{#4}$}
\newcommand{\multisourcepathdiagram}[4]{$\nakedmultisourcepathdiagram{#1}{#2}{#3}{#4}$}


% \monosourcedefinitiondiagram{x}{g}{h}{n}{a}{b}{f}
\newcommand{\monosourcedefinitiondiagram}[7]{
$
\begin{array}{c p{1.5cm} c}
\Rnode{x}{#1} && \nakedmultisourcediagram{#4}{#5}{#6}{#7}
\end{array}
\begin{arrows}
\parallelarrows{x}{a}{#2}{#3}
\end{arrows}
$
}

%\multisourcenplusonediagram{n}{a}{b}{f}{c}{g}
\newcommand{\multisourcenplusonediagram}[6]{
$
\begin{array}{c p{2.0cm} c }
\Rnode{a}{#2}   &&
\begin{array}{c }                
\Rnode{b1}{#3_1}   \\ [0.55cm]
\Rnode{b2}{#3_2}   \\ 
\vdots             \\ 
\Rnode{bn}{#3_{#1}} \\ [0.65cm] 
\Rnode{c}{#5} 
\end{array}
\end{array}
\begin{arrows}
\ncarr{a}{b1}
\alabel{#4_1}[0.6][1]
\ncarr{a}{b2}
\alabel{#4_2}[0.6][0]
\ncarr{a}{bn}
\blabel{#4_{#1}}[0.6][0]
\ncarr{a}{c}\blabel{#6}[0.6][0]
\end{arrows}
$
}

\newcommand{\fghfactordiagram}[6]
{
\binarysourcediagram{#1}{#2\roomup{0.5cm}}{#3}{#4}{#5}
\begin{arrows}
\ncarr{b}{c}
\alabel{#6}
\end{arrows}
}

\newcommand{\fghpartialfactordiagram}[6]{
\binarysourcediagram{#1}{#2\roomup{0.5cm}}{#3}{#4}{#5}
\begin{arrows}
\ncdarr{b}{c} %dashed arrow
\alabel{#6}
\end{arrows}
}

\newcommand{\fnsourceqnsource}{
$
\begin{array}{c p{0.25cm} c  p{0.25cm} c }
             &&   \Rnode{b1}{b_1} &&              \\[0.4cm]
\Rnode{a}{a} &&                   && \Rnode{c}{c} \\[0.4cm]
             &&   \Rnode{bn}{b_n} &&              
\end{array} 
\begin{arrows}
\ncarr{a}{b1}
\alabel{f_1}
\ncarr{c}{b1}
\blabel{q_1} 
\ncarr{a}{bn}
\blabel{f_n}
\ncarr{c}{bn}
\alabel{q_n}
\end{arrows}
$   
}

\newcommand{\parallelarrows}[4]{
\ncarc[nodesep=2pt,arcangle=10,offset=2pt]{->}{#1}{#2}
\alabel{#3}
\ncarc[nodesep=2pt,arcangle=-10,offset=-2pt]{->}{#1}{#2}
\blabel{#4}
}

\newcommand{\paralleldiagram}[4]{
$
\rule[-0.3cm]{0pt}{0.9cm} %to add vertical space of diagram -- based on lowering diagram 0.3cm and heght 0.9cm
                            % change thickness from 0pt to 1 pt to debug
\begin{array}{c p{0.5cm} c}
\Rnode{a}{#1}       &&   \Rnode{b}{#2}
\end{array} 
\begin{arrows}
\parallelarrows{a}{b}{#3}{#4}
\end{arrows}
$
}

\newcommand{\fgparalleldiagram}{
 $
\rule[-0.3cm]{0pt}{0.9cm} %to add vertical space of diagram -- based on lowering diagram 0.3cm and heght 0.9cm
                            % change thickness from 0pt to 1 pt to debug
\begin{array}{c p{0.5cm} c  }
 \Rnode{a}{a}            &&   \Rnode{b}{b}
\end{array} 
\begin{arrows}
\parallelarrows{a}{b}{f}{g}
\end{arrows}
$  
}

\newcommand{\fgcomposablediagram}[5]{
\mbox{
\roomup{0.45cm}
$
\begin{array}{c p{0.5cm}cp{0.5cm}c}
\Rnode{x}{#1}&&\Rnode{y}{#2}&&\Rnode{z}{#3}
\end{array}
\begin{arrows}
\ncarr{x}{y}
\alabel{#4}
\ncarr{y}{z}
\alabel{#5}
\end{arrows}
$    
}
}



% copied from MToD paper (preamble.tex)
\newcommand{\simplepath}[2]{
\ncline[linestyle=none,linewidth=0.1pt]{#1}{#2}   %was linestyle=dotted
\ncput[npos=0.05]{\pnode{dot#21}}
\ncput[npos=0.27]{\dotnode[dotsize=1pt]{dot#22}}
\ncput[npos=0.50]{\dotnode[dotsize=1pt]{dot#23}}
\ncput[npos=0.80]{\dotnode[dotsize=1pt]{dot#24}}
\ncput[npos=0.975]{\pnode{dot#25}}
\ncline[nodesep=2pt]{->}{dot#21}{dot#22}
\ncline[nodesep=2pt]{->}{dot#22}{dot#23}
\ncline[nodesep=2pt]{->}{dot#24}{dot#25}
\ncline[linestyle=dotted,nodesep=8pt]{dot#23}{dot#24} %was 10pt
}

\renewcommand{\erpictureFolder}[0]{../../SharedPictures}

\usetheme{Szeged}
\usecolortheme{dolphin}

%\setbeamertemplate{navigation symbols}{}

\setcounter{equation}{0}

\title[John Cartmell]{Mathematical Theory of Data}
%% Which is to say types as they are used in practice in software development and as represented in theory in categories and in syntactic type theories.
%% There is also a subplot concerning representation of context which certain types depend on -- again represented in practice and in theory. 
\author{John Cartmell}
\institute{ad otium}
\date{Jan 25, 2019}
\bibliographystyle{plainnat}
\usepackage{bibentry}
\nobibliography*

\begin{document}
\section{\MToDsection}
\subsection{\MToDsubsectionliteraturewithProductsandEpiMonosCategories}

\begin{frame}{Definitions} 
In a category \catc, a  \term{source} is a family of morphisms with common domain: \\
\scalebox{0.65}{
\multisourcediagram{n}{a}{b}{f}
} 
\medskip
Such a source is said to be a \term{mono source}  iff for all $g,h:x \morph a$ in \catcw 
so that \scalebox{0.65}{
\monosourcedefinitiondiagram{x}{g}{h}{n}{a}{b}{f}
} 
in \catcw then if $g \circ f_i = h \circ f_i$, for each $i$,  then $g=h$.
\end{frame}

\begin{frame}{Mono Source Limit Cone}  %copied from sketchmonics.tex
Lemma: In a category \cat{C}
\scalebox{0.65}{


$
\begin{array}{c p{2.0cm} c p{2.0cm} c}				
                   &&	 \Rnode{B1}{B_1}  \\ [0.75cm]
									 &&  \Rnode{B2}{B_2}  \\ [0.5cm]
		\Rnode{A}{A}  &&                    \\ [-0.5cm]
				           &&       \vdots      \\ [0.85cm]
                   &&	 \Rnode{Bn}{B_n}  
\end{array}
$
%\setlength{\arrnodesepA}{7pt}
%\setlength{\arrnodesepB}{8pt}
%\setlength{\arroffsetA}{2pt}
%\setlength{\arroffsetB}{0pt}
\begin{arrows}
\ncarr{A}{B1}
\alabel{f_1}[0.5]
\ncarr{A}{B2}
\alabel{f_2}[0.5][-1]
%\blabel{\vdots}[0.4][-2]  % move up 5pts -- dont know why I need this to get position for vdots
\ncarr{A}{Bn}
\blabel{f_n}[0.5][-1]
\end{arrows}


} is a mono source iff \\
\begin{center}
\scalebox{0.65}{
$
\begin{array}{c p{2.0cm} c p{2.0cm} c}				
                           &&	\Rnode{At}{A}  &&          \Rnode{B1}{B_1}  \\ [0.65cm]
													 &&                &&          \Rnode{B2}{B_2}  \\ [0.5cm]
		\Rnode{Al}{A}          &&                &&                           \\ [0cm]
				                   &&                &&           \vdots      \\ [0.85cm]
                           &&	\Rnode{Ab}{A}  &&          \Rnode{Bn}{B_n}  
\end{array}
$
%\setlength{\arrnodesepA}{7pt}
%\setlength{\arrnodesepB}{8pt}
%\setlength{\arroffsetA}{2pt}
%\setlength{\arroffsetB}{0pt}
\ncarr{Al}{At}
\alabel{id_A}
\ncarr{Al}{Ab}
\blabel{id_A}
\ncarr{At}{B1}
\alabel{f_1}[0.5]
\ncarr{At}{B2}
\alabel{f_2}[0.4][-1]
%\blabel{\vdots}[0.4][-2]  % move up 5pts -- dont know why I need this to get position for vdots
\ncarr{At}{Bn}
\blabel{f_n}[0.3][-2]
\ncarr{Ab}{B1}
\alabel{f_1}[0.3][-1]
\ncarr{Ab}{B2}
\blabel{f_2}[0.3][-1]
\ncarr{Ab}{Bn}
\blabel{f_n}[0.4]
%\alabel{\vdots}[0.4]

} 
is a limit cone.
\end{center}
\end{frame}

\begin{frame}{Composition of mono sources}
If $i$ and $j$ are mono sources and $f \in i$ so that
\begin{center}
\scalebox{0.65}{
\setlength{\arraycolsep}{.2cm}
$
\begin{array}{cp{1.5cm}ccp{1.5cm}ccp{1.25cm}c}
             & &         & \dotnode[dotsize=1pt]{b1} & &        &                              && \pnode{bracehigh}  \\ [0.3cm]
						 & &         & \dotnode[dotsize=1pt]{b2} & &        &                                \\ [0.3cm]
\Rnode{a}{a} & & \vdots  &                           & &        &                                \\ [0.02cm]
						 & &         &                           & &        & \dotnode[dotsize=1pt]{x1}      \\ [0.1cm]
             & &         & \Rnode{b}{b}              & & \vdots &                                \\ [0.1cm]
             & &         &                           & &        & \dotnode[dotsize=1pt]{xn}   && \pnode{bracelow}   \\ [0.5cm]
\psbrace[rot=90, nodesepA=-2pt, nodesepB=10pt, braceWidth=1pt, braceWidthInner=3pt](0,0.5)(2.7cm,0.5){i}	
	&  &         & 
\psbrace[rot=90, nodesepA=-2pt, nodesepB=10pt, braceWidth=1pt, braceWidthInner=3pt](0,0.5)(2.7cm,0.5){j} & & \\
\end{array}
$
%\psbrace[rot=0, nodesepA=10pt, braceWidth=1pt, braceWidthInner=3pt, ,ref=lC](bracelow)(bracehigh)
%{$(i \backslash \set{f}) \cup \setsuchthat{f \circ g}{g \in j}$}
\ncarr{b}{x1}
\ncarr{b}{xn}
\ncarr{a}{b1}
\ncarr{a}{b2}
\ncarr{a}{b}
\blabel{f}

}
\end{center}
then $(i \backslash \set{f}) \cup \setsuchthat{f \circ g}{g \in j}$ is a mono source.
\end{frame}

\begin{frame}{Sketches for categories with finite products and epi-mono splits.}
By a sketch for a category with finite products and epi-mono splits I shall mean a quintuple
$\tuple{G,PE,M,P,E}$
\begin{itemize}
\item  where $G$ is a directed graph, 
\item  $PE$ is a set of path equivalences, 
\item  $M$ is a set of $G$-sources deemed to be mono-sources,
\item  $P$ is a set of $G$-sources deemed to be product diagrams, 
\item  and $E$ is a set of $G$-paths deemed to be epimorphisms.
\end{itemize}
\medskip
From such a sketch we can generate a category with finite products and designated monomorphisms and epimorphisms.  \pause (unsubstantiated)   
\end{frame}

\begin{frame}
\IfSforproductepimonoCwithRCwords

\goodnesscriteria{1C} If $P$ contains the source \scalebox{0.65}{\multisourcediagram{n}{x}{y}{p}} 
then there ought to exist a node $z$ and an edge $f: x \morph z$ in $G$ such that morphism $f$ does not factor through any of the $p_i$ morphisms 
i.e. such that there exist $g: y_i \morph z$ in \catcw, for some $i$, such that $p_i\circ g = f$ in \catc. 
\end{frame}

\begin{frame}{Goodness Criteria 2B Revisited}
Restate how functional dependencies and their representations are defined in terms of \highlight{sketches} of categories with products and epi mono splits.
\end{frame}

\begin{frame}{Functional Dependency Restated}
\IfSforproductepimonoCwithRCwords, 
if \scalebox{0.65}{\multisourcepathdiagram{n}{a}{b}{x}} is a path source in $S$ and if
$y: a \morph c$ is a path in $S$
then path $y$ is said to be \term{functionally dependent} on the set of paths $\set{x_1,...x_n}$ with respect to the requirement $\reqtc$
iff the function $D(y)$ factors through $D(\tuple{x_1,... x_n })$

i.e. iff in each $D \in \reqtc$ there exists a  (unique)
function $H_D: img(D(\tuple{x_1,... x_n })) \morph D(c)$ 
such that 
\scalebox{0.65}{\commutativetrianglediagrammutant{D(a)}{img(D(\tuple{x_1,... x_n }))}{D(c)}{D(\xntuple)}{f_D}{D(y)}} commutes.
\end{frame}
\begin{frame} 
CORRECT THIS DIAGRAM
\scalebox{0.9}{ 
$
\begin{array}{cp{2cm}ccp{0.5cm}cc}
                & &         & \Rnode{Eb1}{D(b_1)}& &                            &        \\[0.6cm]
                & &         & \Rnode{Eb2}{D(b_2)}& &                            &        \\[0.6cm]
                & &\vdots   &                     & &                            &        \\[0.2cm]                                              
\Rnode{Ea}{D(a)} & &         & \Rnode{Ebn}{D(b_n)}& & \Rnode{Jnctn}{}            &        \\[1.0cm]
                & &         & \Rnode{Ec}{D(c)}   & &                            &  
\end{array}
\begin{arrows}
\simplepath{Ea}{Eb1}
\alabel{D(x_1)}
\simplepath{Ea}{Eb2}
\alabel{D(x_2)}
\simplepath{Ea}{Ebn}
\alabel{D(x_n)}
\simplepath{Ea}{Ec}
\blabel{D(y)}
\nchpmarr[15][45]{Eb1}{Ebn}{Jnctn}{Ec}
\naput[npos=-0.1]{$h_D$}
\ncarc[arcangle=15]{Eb2}{Jnctn}
%\ncline{h-}{Eb1}{Ebn}
\end{arrows}
$
}
\end{frame}

\begin{frame}{Notation}
That a path  $y$ is functionally dependent on the set of paths $\set{x_1,...x_n}$ 
is written  \msfd{x_1,...x_n}{y}.
\end{frame}

\begin{frame}{Main Lemma}
Suppose that $S$ is a sketch for such a category \catcw that 
is maximally constrained to a set of instances $\reqtc$, assume that $S$ is simple and that \catcw is locally finite,
suppose that $x_1,...x_n$ and $y$ are edges
 within sketch $S$ with common codomain, 
if \msfd{x_1,...x_n}{y} is an intransitive functional dependency in \catcw  with respect to $\reqtc$
 then $\xnset$ is a designated mono source in \catc.
\end{frame}

\begin{frame}{Referential Inclusion Dependencies}
\begin{definition}
If $\catc$ is a something or other, if $\reqtc$ is a set of instances 
and if \fnsourceqnsource in $\catc$ and  $\set{q_1,...q_n}$ is a mono-source
 then a \term{referential inclusion dependency} $I$, written $a[f_1,...f_n] \overset{I}{\subseteq} c[q_1,..q_n]$, 
 is a family of functions $I_D)_{D \in \reqtc}$
such that for each instance $D \in \reqtc$, $I_D$ is a function $I_D : D(a) \morph D(c)$ and
for each $i$, $1 \leq i \le n$, $I_D \circ D(q_i) = D(f_i)$.
\end{definition}
\end{frame}

\begin{frame}{Representation of Referential Inclusion Dependencies}
\begin{definition}
If $\catc$ is a something or other
and if $\reqtc$ is a set of instances
and \fnsourceqnsource in $\catc$ 
and if $a[f_1,...f_n] \overset{I}{\subseteq} c[q_1,..q_n]$ is a referential inclusion dependency
with respect  to $\reqtc$ 
then say that the inclusion dependency $I$ is \term{represented} in $\catc$
iff there exists a morphism $i:a \morph c$ in $\catc$ such that in each instance $D \in \reqtc$, $D(i) = I_D$. 
\end{definition}
\end{frame}

%UNUSED
\iffalse
\begin{frame}{Functional Dependency original}
\IfSforproductepimonoCwithRCwords.
 
if for some $n \geq 1$, $a$, $b_{i}, 1 \leq i \leq n$,  and $c$ are nodes and 
if  $x_{i, 1 \leq i \leq n}$, and $y$ are paths such
that for each $i$, $x_i : a \rightarrow b_i$, and such that $y: a \rightarrow c$ 
as shown here:
\begin{center}
\scalebox{4.5}{
\setlength{\arraycolsep}{.2cm}
$
\begin{array}{cp{2cm}cc}
             & &         & \Rnode{b1}{b_1} \\ [0.5cm]
                         & &         & \Rnode{b2}{b_2} \\ [0.6cm]
                         & & \vdots  &                 \\ [0.2cm]
\Rnode{a}{a} & &         & \Rnode{bn}{b_n} \\ [1.0cm]
             & &         & \Rnode{c}{c}   \\
\end{array}
\begin{arrows}
\simplepath{a}{b1}
\alabel{x_1}
\simplepath{a}{b2}
\alabel{x_2}
\simplepath{a}{bn}
\alabel{x_n}
\simplepath{a}{c}
\blabel{y}
\end{arrows}
$
}
\end{center}
 
then path $y$ is said to be \term{functionally dependent} on the set of paths $\{x_1,...x_n\}$ in ???, 
for which  we write  \msfd{x_1,...x_n}{y},
iff
 in each defining instance $E$ of ??? there exists a  partial 
function $f_E: E_{b_1} \times E_{b_n} \rightarrow E_c$ 
\noindent such that 
domain of $f_E \subseteq img(E_{\tuple{x_1,... x_n }})$ 
and  
$E_{\xntuple} \circ f_E = E_y$ 

\begin{center}
$
\begin{array}{cp{2cm}ccp{0.5cm}cc}
                            & &         & \Rnode{Eb1}{E_{b_1}}& &                            &        \\ [0.6cm]
                            & &         & \Rnode{Eb2}{E_{b_2}}& &                            &        \\ [0.6cm]
                            & &\vdots  &                      & &                            &        \\ [0.2cm]                                                
\Rnode{Ea}{E_a} & &         & \Rnode{Ebn}{E_{b_n}}& & \Rnode{Jnctn}{}&  \\ [1.0cm]
                            & &         & \Rnode{Ec}{{E_c}}   & &                            &        \\
\end{array}
\begin{arrows}
\simplepath{Ea}{Eb1}
\alabel{E_{x_1}}
\simplepath{Ea}{Eb2}
\alabel{E_{x_2}}
\simplepath{Ea}{Ebn}
\alabel{E_{x_n}}
\simplepath{Ec}
\blabel{E_y}
\nchmarr[15][45]{Eb1}{Ebn}{Jnctn}{Ec}
\naput[npos=-0.1]{$f_E$}
\ncarc[arcangle=15]{Eb2}{Jnctn}
\end{arrows}
$
\end{center}

\end{frame}
\fi

%UNUSED
\iffalse
\scalebox{0.45}{
\setlength{\arraycolsep}{.2cm}
$
\begin{array}{cp{2cm}cc}
             & &         & \Rnode{b1}{b_1} \\ [0.5cm]
                         & &         & \Rnode{b2}{b_2} \\ [0.6cm]
                         & & \vdots  &                 \\ [0.2cm]
\Rnode{a}{a} & &         & \Rnode{bn}{b_n} \\ [1.0cm]
             & &         & \Rnode{c}{c}   \\
\end{array}
\begin{arrows}
\simplepath{a}{b1}
\alabel{x_1}
\simplepath{a}{b2}
\alabel{x_2}
\simplepath{a}{bn}
\alabel{x_n}
\simplepath{a}{c}
\blabel{y}
\end{arrows}
$
}
\fi
\end{document}
\fi

\subsection{Sketch Monics}


\newcommand{\CEsymboltype}[0]{varchar(2)}
\newcommand{\CEatomicnumbertype}{number(1,1000)}
\newcommand{\CEfloattype}{float}
\newcommand{\CEnametype}{varchar(64)}
\newcommand{\CEvalencynumbertype}{number(-7,7)}


\iffalse % doesnt belong here
\begin{frame}{Definitions - Johnstone et al.}
\begin{definition}{Johnstone et al}
An \textit{EA sketch} is a sketch $\tuple{G,D,L,C}$ where $G$ is a directed graph, $D$ a set of diagrams in $G$, $L$ a set of finite cones and
$C$ a set of finite discrete cocones.
\end{definition}

If $S$ is an EA-sketch then the theory of $S$ is the lextensive category generated by $S$.

If $S$ is an EA sketch then a model of $S$ is a functor to the category of finite sets preserving finite limits and coproducts.
The category of models is denoted $Mod(S,\cat{FinSet})$.
\end{frame}
\fi

\begin{frame}{Data Specification of the Chemical Elements as a Directed Graph}
\scalebox{0.65}{


$
\begin{array}{c c c p{4.5cm} l}
                        &                        &                            &&\attrtype{\Rnode{symboltype}{\CEsymboltype}  }       \\ [0.7cm]
                        &\etype{\Rnode{element}{chemical\ element}\Rnode{elementR}{}}&  &&                                        \\ [0.3cm]
%												&                        &           &&\attrtype{\Rnode{atomicnumbertypeL}{n}\Rnode{atomicnumbertype}{umber(1,1000)}}\\ [0.55cm]
												&                        &           &&\attrtype{\Rnode{atomicnumbertypeL}{\CEatomicnumbertype}}\\ [0.55cm]
												&                        &\etype{\Rnode{isotope}{isotope}\Rnode{isotopeR}{}}&&                          \\ [0.3cm]
                        &                        &                            &&\attrtype{\Rnode{floattype}{\CEfloattype}}                \\ [0.45cm]
                        &\etype{\Rnode{allotrope}{allotrope}\Rnode{allotropeR}{}}&    &&                                \\ [0.45cm]
												&                        &                            &&\attrtype{\Rnode{nametype}{\CEnametype}}           \\ [1.0cm]
\etype{\Rnode{valency}{valency}\Rnode{valencyR}{}}&      &                            &&\attrtype{\Rnode{valencynumbertype}{\CEvalencynumbertype}}\\
\end{array}
$
\setlength{\arrnodesepA}{7pt}
\setlength{\arrnodesepB}{8pt}
\setlength{\arroffsetB}{7pt}
\ncarr[10]{valency}{element}
\setlength{\arrnodesepB}{7pt}
\setlength{\arroffsetB}{0pt}
\ncarr[-5]{allotrope}{element}
\setlength{\arrnodesepB}{9pt}
\setlength{\arroffsetB}{-5pt}
\ncarr[-5]{isotope}{element}
\setlength{\arroffsetB}{0pt}
\setlength{\arrnodesepB}{3pt}
\setlength{\arroffsetB}{-2pt}
\setlength{\arroffsetA}{5pt}
\ncarr[15]{elementR}{symboltype}
\alabel{symbol}[0.4][0]
\setlength{\arroffsetA}{0pt}
\setlength{\arrnodesepB}{3pt}
\setlength{\arroffsetB}{0pt}
\ncarr[5]{elementR}{atomicnumbertypeL}
\alabel{atomic\,number}[0.4]
\setlength{\arroffsetA}{2pt}
\setlength{\arroffsetB}{-2pt}
\ncarr[5]{isotopeR}{atomicnumbertypeL}
\alabel{neutron\,count}[0.35][0]
\setlength{\arroffsetA}{-2pt}
\ncarr[5]{isotopeR}{floattype}
\alabel{mass}[0.35]
\setlength{\arroffsetA}{2pt}
\ncarr[5]{allotropeR}{floattype}
\alabel{melting\,point}[0.35][-1]
\setlength{\arroffsetA}{-2pt}
\setlength{\arroffsetB}{-4pt}
\ncarr[5]{allotropeR}{nametype}
\blabel{name}[0.3]
\setlength{\arroffsetA}{2pt}
\setlength{\arroffsetB}{-2pt}
\ncarr[5]{valencyR}{valencynumbertype}
\alabel{number}[0.3]
} 
\end{frame}

\iffalse %doesnt belong here
\begin{frame}{Relational Database Theory}
\begin{itemize}
\item classic relational database normal form definitions ({\scriptsize 3NF, EKNF, BCNF, 4NF,5NF, INC-NF}) can be transfered into the more general framework
of ER modelling and formalised within the definitional framework of EA sketches

\item such normal forms  examine the fit of a sketch/theory (database schema) to an intended usage

\item we can assume that the intended usage is represented by a full-subcategory of the category $Mod(S,\cat{FinSet})$

\item in such a situation the classic normal forms address the question can the sketch/theory $S$ be improved by addition or removal of morphisms and/or commutative diagrams and/or limit cones.
\item normalisation has dual goal of obtaining as complete a theory as possible and of eliminating redundancy from the sketch.  
\end{itemize}
\end{frame}

\begin{frame}{Normal Forms}
\begin{itemize}
\item IN-NF -- Ling and Goh -- there are no redundant attributes except if absolutely necessary in order to specify a mono source
\end{itemize}
\end{frame}

\begin{frame}{Defining Candidate Keys and/or Identifying Relationships in an EA sketch.  }
\begin{itemize}
\item concept of \textit{candidate keys} used in relational database normal form definitions {\scriptsize (3NF, EKNF, BCNF)}
\item in ER model talk about \textit{identifying} families of relationships
\item in category theory such a key or a family of relationships is a mono source i.e. a to jointly monic family of morphisms
\item mono sources and hence candidate keys can be defined as limit cones
\item more than 99.99 percent of entity modelling uses just mono sources and no other limits
\end{itemize}
\end{frame}
\fi


\begin{frame}{Each entity type has at least on mono source defined}
\begin{tabular}{l}
\scalebox{0.60}{
$
\begin{array}{c p{4.5cm} l}                                                
\etype{\Rnode{element}{chemical\ element}\Rnode{elementR}{}}& &  \attrtype{\Rnode{symboltype}{\CEsymboltype}  } \\
\end{array}
$
\setlength{\arrnodesepA}{7pt}
\setlength{\arrnodesepB}{6pt}
\setlength{\arroffsetB}{-2pt}
\setlength{\arroffsetA}{0pt}
\ncarr[5]{elementR}{symboltype}
\alabel{symbol}[0.4][0]


} \\ [0.6cm]
\scalebox{0.65}{
$
\begin{array}{c p{4.5cm} l}                                                
\etype{\Rnode{element}{chemical\ element}\Rnode{elementR}{}}& & \attrtype{\Rnode{atomicnumbertypeL}{\CEatomicnumbertype}}\\
\end{array}
$
\setlength{\arrnodesepA}{7pt}
\setlength{\arrnodesepB}{6pt}
\setlength{\arroffsetB}{-2pt}
\setlength{\arroffsetA}{0pt}
\ncarr[5]{elementR}{atomicnumbertypeL}
\alabel{atomic\,number}[0.4]


} \\ [0.6cm]
\scalebox{0.65}{
$
\begin{array}{c p{4.5cm} l}                                                
\etype{\Rnode{allotrope}{allotrope}\Rnode{allotropeR}{}}& & \attrtype{\Rnode{nametype}{\CEnametype}} \\
\end{array}
$
\setlength{\arrnodesepA}{7pt}
\setlength{\arrnodesepB}{6pt}
\setlength{\arroffsetB}{-2pt}
\setlength{\arroffsetA}{0pt}
\ncarr[5]{allotropeR}{nametype}
\blabel{name}[0.3]


} \\ [0.6cm]
\scalebox{0.65}{
$
\begin{array}{c p{4.5cm} l}
                                                  & &\etype{\Rnode{element}{chemical\ element}\Rnode{elementR}{}} \\ [0.25cm]
\etype{\Rnode{isotope}{isotope}\Rnode{isotopeR}{}}& &                                                             \\[0.25cm]
                                                  & &\attrtype{\Rnode{atomicnumbertypeL}{\CEatomicnumbertype}} \\
\end{array}
$
\setlength{\arrnodesepA}{7pt}
\setlength{\arrnodesepB}{8pt}
\setlength{\arroffsetA}{2pt}
\setlength{\arroffsetB}{0pt}
\ncarr[10]{isotopeR}{element}
\alabel{of}[0.3]
\setlength{\arroffsetA}{0pt}
\setlength{\arroffsetB}{-3pt}
\ncarr[5]{isotopeR}{atomicnumbertypeL}
\blabel{neutron\,count}[0.35][0]


} \\ [1.1cm]
\scalebox{0.65}{
$
\begin{array}{c p{4.5cm} l}
                                                  & &\etype{\Rnode{element}{chemical\ element}\Rnode{elementR}{}} \\ [0.25cm]
\etype{\Rnode{valency}{valency}\Rnode{valencyR}{}}& &                                                             \\[0.25cm]
                                                  & &\attrtype{\Rnode{valencynumbertype}{\CEvalencynumbertype}}           \\
\end{array}
$
\setlength{\arrnodesepA}{7pt}
\setlength{\arrnodesepB}{8pt}
\setlength{\arroffsetB}{0pt}
\ncarr[10]{valency}{element}
\alabel{of}[0.3]
\setlength{\arroffsetA}{2pt}
\setlength{\arroffsetB}{-3pt}
\ncarr[5]{valencyR}{valencynumbertype}
\blabel{number}[0.3]
} 
\end{tabular}
\end{frame}

\begin{frame}{Composing mono sources}
\begin{itemize}
\item{
In the category there are two further \textit{derived}  mono sources: \\
\vspace{0.5cm}
\begin{tabular}{l}
\scalebox{0.65}{
$
\begin{array}{c p{4.5cm} l}
                                                  & &\attrtype{\Rnode{symboltype}{\CEsymboltype}  }              \\ [0.25cm]
\etype{\Rnode{valency}{valency}\Rnode{valencyR}{}}& &                                                            \\[0.25cm]
                                                  & &\attrtype{\Rnode{valencynumbertype}{\CEvalencynumbertype}}  \\
\end{array}
$
\setlength{\arrnodesepA}{7pt}
\setlength{\arrnodesepB}{8pt}
\setlength{\arroffsetA}{2pt}
\setlength{\arroffsetB}{0pt}
\ncarr[10]{valency}{symboltype}
\alabel{of \circ symbol}[0.3]
\setlength{\arroffsetA}{0pt}
\setlength{\arroffsetB}{-3pt}
\ncarr[5]{valencyR}{valencynumbertype}
\blabel{number}[0.3]

} \\ [1.0cm]
\scalebox{0.65}{
$
\begin{array}{c p{4.5cm} l}
                                                  & &\attrtype{\Rnode{symboltype}{\CEsymboltype}  }            \\ [0.3cm]
\etype{\Rnode{isotope}{isotope}\Rnode{isotopeR}{}}& &                                                          \\[0.3cm]
                                                  & &\attrtype{\Rnode{atomicnumbertypeL}{\CEatomicnumbertype}} \\
\end{array}
$
\setlength{\arrnodesepA}{7pt}
\setlength{\arrnodesepB}{8pt}
\setlength{\arroffsetB}{0pt}
\ncarr[10]{isotopeR}{symboltype}
\alabel{of \circ symbol}[0.3]
\setlength{\arroffsetA}{2pt}
\setlength{\arroffsetB}{-3pt}
\ncarr[5]{isotopeR}{atomicnumbertypeL}
\alabel{neutron\,count}[0.35][0]


} 
\end{tabular}
}
\end{itemize}
\end{frame}

\begin{frame}{Therefore every entity type has at least one candidate key}
\begin{tabular}{l}
\scalebox{0.60}{
$
\begin{array}{c p{4.5cm} l}                                                
\etype{\Rnode{element}{chemical\ element}\Rnode{elementR}{}}& &  \attrtype{\Rnode{symboltype}{\CEsymboltype}  } \\
\end{array}
$
\setlength{\arrnodesepA}{7pt}
\setlength{\arrnodesepB}{6pt}
\setlength{\arroffsetB}{-2pt}
\setlength{\arroffsetA}{0pt}
\ncarr[5]{elementR}{symboltype}
\alabel{symbol}[0.4][0]


} \\ [0.7cm]
\scalebox{0.65}{
$
\begin{array}{c p{4.5cm} l}                                                
\etype{\Rnode{element}{chemical\ element}\Rnode{elementR}{}}& & \attrtype{\Rnode{atomicnumbertypeL}{\CEatomicnumbertype}}\\
\end{array}
$
\setlength{\arrnodesepA}{7pt}
\setlength{\arrnodesepB}{6pt}
\setlength{\arroffsetB}{-2pt}
\setlength{\arroffsetA}{0pt}
\ncarr[5]{elementR}{atomicnumbertypeL}
\alabel{atomic\,number}[0.4]


} \\ [0.7cm]
\scalebox{0.65}{
$
\begin{array}{c p{4.5cm} l}                                                
\etype{\Rnode{allotrope}{allotrope}\Rnode{allotropeR}{}}& & \attrtype{\Rnode{nametype}{\CEnametype}} \\
\end{array}
$
\setlength{\arrnodesepA}{7pt}
\setlength{\arrnodesepB}{6pt}
\setlength{\arroffsetB}{-2pt}
\setlength{\arroffsetA}{0pt}
\ncarr[5]{allotropeR}{nametype}
\blabel{name}[0.3]


} \\ [0.7cm]
\scalebox{0.65}{
$
\begin{array}{c p{4.5cm} l}
                                                  & &\attrtype{\Rnode{symboltype}{\CEsymboltype}  }              \\ [0.25cm]
\etype{\Rnode{valency}{valency}\Rnode{valencyR}{}}& &                                                            \\[0.25cm]
                                                  & &\attrtype{\Rnode{valencynumbertype}{\CEvalencynumbertype}}  \\
\end{array}
$
\setlength{\arrnodesepA}{7pt}
\setlength{\arrnodesepB}{8pt}
\setlength{\arroffsetA}{2pt}
\setlength{\arroffsetB}{0pt}
\ncarr[10]{valency}{symboltype}
\alabel{of \circ symbol}[0.3]
\setlength{\arroffsetA}{0pt}
\setlength{\arroffsetB}{-3pt}
\ncarr[5]{valencyR}{valencynumbertype}
\blabel{number}[0.3]

} \\ [1.1cm]
\scalebox{0.65}{
$
\begin{array}{c p{4.5cm} l}
                                                  & &\attrtype{\Rnode{symboltype}{\CEsymboltype}  }            \\ [0.3cm]
\etype{\Rnode{isotope}{isotope}\Rnode{isotopeR}{}}& &                                                          \\[0.3cm]
                                                  & &\attrtype{\Rnode{atomicnumbertypeL}{\CEatomicnumbertype}} \\
\end{array}
$
\setlength{\arrnodesepA}{7pt}
\setlength{\arrnodesepB}{8pt}
\setlength{\arroffsetB}{0pt}
\ncarr[10]{isotopeR}{symboltype}
\alabel{of \circ symbol}[0.3]
\setlength{\arroffsetA}{2pt}
\setlength{\arroffsetB}{-3pt}
\ncarr[5]{isotopeR}{atomicnumbertypeL}
\alabel{neutron\,count}[0.35][0]


} 
\end{tabular}
\end{frame}




\subsection{Entity-Relationship Notation}

\begin{frame}{Chemical Element DS as Directed Graph}
\scalebox{0.65}{


$
\begin{array}{c c c p{4.5cm} l}
                        &                        &                            &&\attrtype{\Rnode{symboltype}{\CEsymboltype}  }       \\ [0.7cm]
                        &\etype{\Rnode{element}{chemical\ element}\Rnode{elementR}{}}&  &&                                        \\ [0.3cm]
%												&                        &           &&\attrtype{\Rnode{atomicnumbertypeL}{n}\Rnode{atomicnumbertype}{umber(1,1000)}}\\ [0.55cm]
												&                        &           &&\attrtype{\Rnode{atomicnumbertypeL}{\CEatomicnumbertype}}\\ [0.55cm]
												&                        &\etype{\Rnode{isotope}{isotope}\Rnode{isotopeR}{}}&&                          \\ [0.3cm]
                        &                        &                            &&\attrtype{\Rnode{floattype}{\CEfloattype}}                \\ [0.45cm]
                        &\etype{\Rnode{allotrope}{allotrope}\Rnode{allotropeR}{}}&    &&                                \\ [0.45cm]
												&                        &                            &&\attrtype{\Rnode{nametype}{\CEnametype}}           \\ [1.0cm]
\etype{\Rnode{valency}{valency}\Rnode{valencyR}{}}&      &                            &&\attrtype{\Rnode{valencynumbertype}{\CEvalencynumbertype}}\\
\end{array}
$
\setlength{\arrnodesepA}{7pt}
\setlength{\arrnodesepB}{8pt}
\setlength{\arroffsetB}{7pt}
\ncarr[10]{valency}{element}
\setlength{\arrnodesepB}{7pt}
\setlength{\arroffsetB}{0pt}
\ncarr[-5]{allotrope}{element}
\setlength{\arrnodesepB}{9pt}
\setlength{\arroffsetB}{-5pt}
\ncarr[-5]{isotope}{element}
\setlength{\arroffsetB}{0pt}
\setlength{\arrnodesepB}{3pt}
\setlength{\arroffsetB}{-2pt}
\setlength{\arroffsetA}{5pt}
\ncarr[15]{elementR}{symboltype}
\alabel{symbol}[0.4][0]
\setlength{\arroffsetA}{0pt}
\setlength{\arrnodesepB}{3pt}
\setlength{\arroffsetB}{0pt}
\ncarr[5]{elementR}{atomicnumbertypeL}
\alabel{atomic\,number}[0.4]
\setlength{\arroffsetA}{2pt}
\setlength{\arroffsetB}{-2pt}
\ncarr[5]{isotopeR}{atomicnumbertypeL}
\alabel{neutron\,count}[0.35][0]
\setlength{\arroffsetA}{-2pt}
\ncarr[5]{isotopeR}{floattype}
\alabel{mass}[0.35]
\setlength{\arroffsetA}{2pt}
\ncarr[5]{allotropeR}{floattype}
\alabel{melting\,point}[0.35][-1]
\setlength{\arroffsetA}{-2pt}
\setlength{\arroffsetB}{-4pt}
\ncarr[5]{allotropeR}{nametype}
\blabel{name}[0.3]
\setlength{\arroffsetA}{2pt}
\setlength{\arroffsetB}{-2pt}
\ncarr[5]{valencyR}{valencynumbertype}
\alabel{number}[0.3]
} \\
\vspace {0.25cm}
\textit{
\begin{tabular} {c p{0cm} p{2.5cm} l}
the types of entities and       & & the attributes   & the  types  \\
their inter-relationships       & & of the entities     & of the attributes \\
  & &            &
\end{tabular}
}
\end{frame}

\begin{frame}{Chemical Element DS viewed as ER diagram}
\begin{center}
\scalebox{0.85}{
\begin{erdiagram}{4.8}{4.85}

\eret{1}{-1.9}{4}{-0.1}{0.2}{1}\eretname{1.3}{-0.45}{l}{element}
\erattr{1.2}{-0.65}{1}{0}{symbol}
\erattr{1.2}{-0.95}{1}{1}{name}
\erattr{1.2}{-1.25}{1}{1}{atomic number}
\erattr{1.2}{-1.55}{1}{1}{relative atomic mass}
\eret{0.15}{-4.8}{2.25}{-3}{0.2}{1}\eretname{0.36}{-3.35}{l}{allotrope}
\erattr{0.35}{-3.55}{1}{0}{name}
\erattr{0.35}{-3.85}{0}{1}{melting point}
\erattr{0.35}{-4.15}{0}{1}{boiling point}
\erattr{0.35}{-4.45}{0}{1}{density}
\eret{2.75}{-3.9}{4.85}{-3}{0.2}{1}\eretname{2.96}{-3.35}{l}{valency}
\erattr{2.95}{-3.55}{1}{0}{number}

% relationship 
\errelname{2.15}{-2.2}{l}{}\errelarm{2}{-1.9}{2}{-1.975}{1}{0}\errelarm{1.2}{-2.875}{1.2}{-3}{1}{0}\errelangle{2}{-1.975}{2}{-2.05}{1.6}{-2.4}{1}{0}\errelangle{1.6}{-2.4}{1.2}{-2.75}{1.2}{-2.875}{1}{0}\ercrowfoot{1.2}{-2.85}{1.05}{-3}{1.2}{-3}{1.35}{-3}{0}
% relationship 
\errelname{3.15}{-2.2}{l}{}\errelarm{3}{-1.9}{3}{-1.975}{1}{0}\errelarm{3.8}{-2.788}{3.8}{-3}{1}{0}\errelangle{3}{-1.975}{3}{-2.05}{3.4}{-2.313}{1}{0}\errelangle{3.4}{-2.313}{3.8}{-2.575}{3.8}{-2.788}{1}{0}\eridcomprel{3.6999999999999997}{3.9}{-2.75}\ercrowfoot{3.8}{-2.85}{3.65}{-3}{3.8}{-3}{3.95}{-3}{0}
\end{erdiagram}

}
\end{center}

\end{frame}

\begin{frame}{Basic Chemistry viewed as ER diagram}
\begin{center}
\scalebox{0.5}{
\begin{erdiagram}{7.699999999999999}{15.7075}

\eret{1}{-2.4}{4}{-1.5}{0.2}{1}\eretname{1.3}{-1.85}{l}{molecularStructure}
\erattr{1.2}{-2.05}{1}{0}{name}
\eret{1.793}{-4.8}{3.207}{-3.3}{0.2}{1}\eretname{1.934}{-3.65}{l}{atom}
\erattr{1.993}{-3.85}{1}{0}{atomId}
\erattr{1.993}{-4.15}{1}{1}{x}
\erattr{1.993}{-4.45}{1}{1}{y}
\eret{1}{-6.9}{4}{-5.7}{0.2}{1}\eretname{1.3}{-6.05}{l}{bond formed}
\erdattr{1.2}{-6.25}{1}{0}{withAtomId(R2)}
\erattr{1.2}{-6.55}{1}{1}{bondType}
\eret{9.208}{-4.8}{12.208}{-3.3}{0.2}{1}\eretname{9.508}{-3.65}{l}{element}
\erattr{9.407}{-3.85}{1}{0}{symbol}
\erattr{9.407}{-4.15}{1}{1}{name}
\erattr{9.407}{-4.45}{1}{1}{atomic number}
\eret{5.708}{-7.4}{8.708}{-5.9}{0.2}{1}\eretname{6.007}{-6.25}{l}{isotope}
\erattr{5.908}{-6.45}{1}{0}{numberOfNeutrons}
\erattr{5.908}{-6.75}{1}{1}{mass}
\erattr{5.908}{-7.05}{1}{1}{abundancy}
\eret{9.208}{-7.7}{12.208}{-5.9}{0.2}{1}\eretname{9.508}{-6.25}{l}{allotrope}
\erattr{9.407}{-6.45}{1}{0}{name}
\erattr{9.407}{-6.75}{0}{1}{melting point}
\erattr{9.407}{-7.05}{0}{1}{boiling point}
\erattr{9.407}{-7.35}{0}{1}{density}
\eret{12.708}{-6.8}{15.708}{-5.9}{0.2}{1}\eretname{13.008}{-6.25}{l}{valency}
\erattr{12.907}{-6.45}{1}{0}{number}
\eret{0}{-0.2}{15.708}{0.3}{0.2}{1}

% relationship 
\errelname{2.65}{-0.5}{l}{}\errelname{2.65}{-1.35}{l}{..}\errelarm{2.5}{-0.2}{2.5}{-0.85}{1}{0}\errelarm{2.5}{-0.85}{2.5}{-1.5}{1}{0}\ercrowfoot{2.5}{-1.35}{2.35}{-1.5}{2.5}{-1.5}{2.65}{-1.5}{0}
% relationship 
\errelname{10.858}{-0.5}{l}{}\errelname{10.858}{-3.15}{l}{..}\errelarm{10.708}{-0.2}{10.708}{-1.75}{1}{0}\errelarm{10.708}{-1.75}{10.708}{-3.3}{1}{0}\ercrowfoot{10.708}{-3.15}{10.558}{-3.3}{10.708}{-3.3}{10.858}{-3.3}{0}
% relationship 
\errelname{2.65}{-2.7}{l}{}\errelname{2.65}{-3.15}{l}{..}\errelarm{2.5}{-2.4}{2.5}{-2.85}{1}{0}\errelarm{2.5}{-2.85}{2.5}{-3.3}{1}{0}\eridcomprel{2.4}{2.6}{-3.05}\ercrowfoot{2.5}{-3.15}{2.35}{-3.3}{2.5}{-3.3}{2.65}{-3.3}{0}
% relationship 
\errelname{2.65}{-5.1}{l}{}\errelname{2.65}{-5.55}{l}{of}\errelarm{2.5}{-4.8}{2.5}{-5.25}{0}{0}\errelarm{2.5}{-5.25}{2.5}{-5.7}{1}{0}\eridcomprel{2.4}{2.6}{-5.449999999999999}\ercrowfoot{2.5}{-5.55}{2.35}{-5.7}{2.5}{-5.7}{2.65}{-5.7}{0}
% relationship element
\errelname{3.357}{-4.35}{l}{element}\errelarm{3.207}{-4.05}{6.208}{-4.05}{1}{0}\errelarm{6.208}{-4.05}{9.208}{-4.05}{0}{0}\ercrowfoot{3.357}{-4.05}{3.207}{-3.9}{3.207}{-4.05}{3.207}{-4.2}{0}
% relationship with
\errelname{4.15}{-6.6}{l}{with}\errelname{4.15}{-6.15}{l}{with}\errelarm{4}{-6.3}{4.5}{-6.3}{1}{0}\errelarm{4.5}{-6.3}{4}{-6.3}{1}{0}\errelname{5.15}{-6.15}{l}{R2}\eridrefrel{4.25}{-6.199999999999999}{-6.399999999999999}
% relationship 
\errelname{10.108}{-5.1}{l}{}\errelname{7.358}{-5.75}{l}{..}\errelarm{9.958}{-4.8}{9.958}{-4.875}{1}{0}\errelarm{7.208}{-5.687}{7.208}{-5.9}{1}{0}\errelangle{9.958}{-4.875}{9.958}{-4.95}{8.583}{-5.213}{1}{0}\errelangle{8.583}{-5.213}{7.208}{-5.475}{7.208}{-5.687}{1}{0}\eridcomprel{7.1075}{7.307499999999999}{-5.6499999999999995}\ercrowfoot{7.208}{-5.75}{7.057}{-5.9}{7.208}{-5.9}{7.358}{-5.9}{0}
% relationship 
\errelname{10.858}{-5.1}{l}{}\errelname{10.858}{-5.75}{l}{..}\errelarm{10.708}{-4.8}{10.708}{-5.35}{1}{0}\errelarm{10.708}{-5.35}{10.708}{-5.9}{1}{0}\ercrowfoot{10.708}{-5.75}{10.558}{-5.9}{10.708}{-5.9}{10.858}{-5.9}{0}
% relationship 
\errelname{11.608}{-5.1}{l}{}\errelname{14.358}{-5.75}{l}{..}\errelarm{11.458}{-4.8}{11.458}{-4.875}{1}{0}\errelarm{14.208}{-5.687}{14.208}{-5.9}{1}{0}\errelangle{11.458}{-4.875}{11.458}{-4.95}{12.833}{-5.213}{1}{0}\errelangle{12.833}{-5.213}{14.208}{-5.475}{14.208}{-5.687}{1}{0}\eridcomprel{14.1075}{14.3075}{-5.6499999999999995}\ercrowfoot{14.208}{-5.75}{14.058}{-5.9}{14.208}{-5.9}{14.358}{-5.9}{0}
\end{erdiagram}

}
\end{center}
\end{frame}


\begin{frame}{MolecularGeometry ERA diagram  -- Logical }
\begin{center}
\scalebox{0.5}{
\begin{erdiagram}{7}{15.1075}

\eret{1}{-2.6}{3.6}{-1.5}{0.2}{1}\eretname{1.26}{-1.85}{l}{conformation}
\erattr{1.2}{-2.05}{1}{0}{id}
\eret{1.514}{-5.2}{3.086}{-3.7}{0.2}{1}\eretname{1.671}{-4.05}{l}{position}
\erattr{1.714}{-4.25}{1}{1}{x}
\erattr{1.714}{-4.55}{1}{1}{y}
\erattr{1.714}{-4.85}{1}{1}{z}
\eret{7.1}{-2.6}{9.7}{-1.5}{0.2}{1}\eretname{7.36}{-1.85}{l}{molStruct}
\erattr{7.3}{-2.05}{1}{0}{name}
\eret{7.693}{-5.2}{9.108}{-3.7}{0.2}{1}\eretname{7.834}{-4.05}{l}{atom}
\erattr{7.893}{-4.25}{1}{0}{number}
\eret{7.354}{-7}{9.446}{-6.1}{0.2}{1}\eretname{7.563}{-6.45}{l}{bond formed}
\erattr{7.554}{-6.65}{1}{1}{bondType}
\eret{12.108}{-5.2}{15.108}{-3.7}{0.2}{1}\eretname{12.408}{-4.05}{l}{element}
\erattr{12.308}{-4.25}{1}{0}{symbol}
\erattr{12.308}{-4.55}{1}{1}{name}
\erattr{12.308}{-4.85}{1}{1}{atomic number}
\eret{0}{-0.2}{15.108}{0.3}{0.2}{1}

% relationship 
\errelname{2.45}{-0.5}{l}{}\errelname{2.45}{-1.35}{l}{..}\errelarm{2.3}{-0.2}{2.3}{-0.85}{1}{0}\errelarm{2.3}{-0.85}{2.3}{-1.5}{1}{0}\ercrowfoot{2.3}{-1.35}{2.15}{-1.5}{2.3}{-1.5}{2.45}{-1.5}{0}
% relationship 
\errelname{8.55}{-0.5}{l}{}\errelname{8.55}{-1.35}{l}{..}\errelarm{8.4}{-0.2}{8.4}{-0.85}{1}{0}\errelarm{8.4}{-0.85}{8.4}{-1.5}{1}{0}\ercrowfoot{8.4}{-1.35}{8.25}{-1.5}{8.4}{-1.5}{8.55}{-1.5}{0}
% relationship 
\errelname{13.758}{-0.5}{l}{}\errelname{13.758}{-3.55}{l}{..}\errelarm{13.608}{-0.2}{13.608}{-1.95}{1}{0}\errelarm{13.608}{-1.95}{13.608}{-3.7}{1}{0}\ercrowfoot{13.608}{-3.55}{13.458}{-3.7}{13.608}{-3.7}{13.758}{-3.7}{0}
% relationship 
\errelname{2.45}{-2.9}{l}{}\errelname{2.45}{-3.55}{l}{..}\errelarm{2.3}{-2.6}{2.3}{-3.15}{1}{0}\errelarm{2.3}{-3.15}{2.3}{-3.7}{1}{0}\eridcomprel{2.15}{2.4499999999999997}{-3.4000000000000004}\ercrowfoot{2.3}{-3.55}{2.15}{-3.7}{2.3}{-3.7}{2.45}{-3.7}{0}\ercrowfoot{2.3}{-3.55}{2.15}{-3.4}{2.3}{-3.4}{2.45}{-3.4}{0}\ercrowfoot{2.3}{-3.55}{2.15}{-3.7}{2.3}{-3.7}{2.45}{-3.7}{0}
% relationship of
\errelname{3.75}{-2.35}{l}{of}\errelarm{3.6}{-2.05}{5.35}{-2.05}{1}{0}\errelarm{5.35}{-2.05}{7.1}{-2.05}{0}{0}\ercrowfoot{3.75}{-2.05}{3.6}{-1.9}{3.6}{-2.05}{3.6}{-2.2}{0}
% relationship atom
\errelname{3.236}{-4.75}{l}{atom}\erscope{5.139}{-4.75}{l}{d:..=s:..}\errelarm{3.086}{-4.45}{5.389}{-4.45}{1}{0}\errelarm{5.389}{-4.45}{7.693}{-4.45}{0}{0}\ercrowfoot{3.236}{-4.45}{3.086}{-4.3}{3.086}{-4.45}{3.086}{-4.6}{0}\eridrefrel{3.3859999999999997}{-4.3}{-4.6000000000000005}\ercrowfoot{3.236}{-4.45}{3.086}{-4.3}{3.086}{-4.45}{3.086}{-4.6}{0}\ercrowfoot{3.236}{-4.45}{3.386}{-4.3}{3.386}{-4.45}{3.386}{-4.6}{0}
% relationship 
\errelname{8.55}{-2.9}{l}{}\errelname{8.55}{-3.55}{l}{..}\errelarm{8.4}{-2.6}{8.4}{-3.15}{1}{0}\errelarm{8.4}{-3.15}{8.4}{-3.7}{1}{0}\eridcomprel{8.3}{8.5}{-3.45}\ercrowfoot{8.4}{-3.55}{8.25}{-3.7}{8.4}{-3.7}{8.55}{-3.7}{0}
% relationship 
\errelname{8.55}{-5.5}{l}{}\errelname{8.55}{-5.95}{l}{of}\errelarm{8.4}{-5.2}{8.4}{-5.65}{0}{0}\errelarm{8.4}{-5.65}{8.4}{-6.1}{1}{0}\eridcomprel{8.3}{8.5}{-5.85}\ercrowfoot{8.4}{-5.95}{8.25}{-6.1}{8.4}{-6.1}{8.55}{-6.1}{0}
% relationship element
\errelname{9.258}{-4.75}{l}{element}\errelarm{9.108}{-4.45}{10.608}{-4.45}{1}{0}\errelarm{10.608}{-4.45}{12.108}{-4.45}{0}{0}\ercrowfoot{9.258}{-4.45}{9.108}{-4.3}{9.108}{-4.45}{9.108}{-4.6}{0}
% relationship with
\errelname{9.596}{-6.85}{l}{with}\errelname{9.596}{-6.4}{l}{with}\errelarm{9.446}{-6.55}{9.946}{-6.55}{1}{0}\errelarm{9.946}{-6.55}{9.446}{-6.55}{1}{0}\erscope{10.496}{-6.75}{l}{d:of/..=s:of/..}\eridrefrel{9.696375}{-6.45}{-6.6499999999999995}
\end{erdiagram}

}
\end{center}
\end{frame}

\begin{frame}{MolecularGeometry ERA diagram -- Hierarchical}
\begin{center}
\scalebox{0.5}{
\begin{erdiagram}{7.6}{15.8725}

\eret{1}{-2.6}{3.6}{-1.5}{0.2}{1}\eretname{1.26}{-1.85}{l}{conformation}
\erattr{1.2}{-2.05}{1}{0}{id}
\erdattr{1.2}{-2.35}{1}{1}{of\_name(R1)}
\eret{1.019}{-5.5}{3.581}{-3.7}{0.2}{1}\eretname{1.275}{-4.05}{l}{position}
\erdattr{1.219}{-4.25}{1}{0}{atom\_number(R2)}
\erattr{1.219}{-4.55}{1}{1}{x}
\erattr{1.219}{-4.85}{1}{1}{y}
\erattr{1.219}{-5.15}{1}{1}{z}
\eret{7.1}{-2.6}{9.7}{-1.5}{0.2}{1}\eretname{7.36}{-1.85}{l}{molStruct}
\erattr{7.3}{-2.05}{1}{0}{name}
\eret{6.928}{-5.5}{9.873}{-3.7}{0.2}{1}\eretname{7.222}{-4.05}{l}{atom}
\erattr{7.128}{-4.25}{1}{0}{number}
\erdattr{7.128}{-4.55}{1}{1}{element\_symbol(R3)}
\eret{6.75}{-7.6}{10.05}{-6.4}{0.2}{1}\eretname{7.08}{-6.75}{l}{bond formed}
\erdattr{6.95}{-6.95}{1}{0}{with\_atom\_number(R4)}
\erattr{6.95}{-7.25}{1}{1}{bondType}
\eret{12.873}{-5.2}{15.873}{-3.7}{0.2}{1}\eretname{13.173}{-4.05}{l}{element}
\erattr{13.073}{-4.25}{1}{0}{symbol}
\erattr{13.073}{-4.55}{1}{1}{name}
\erattr{13.073}{-4.85}{1}{1}{atomic number}
\eret{0}{-0.2}{15.873}{0.3}{0.2}{1}

% relationship 
\errelname{2.45}{-0.5}{l}{}\errelname{2.45}{-1.35}{l}{..}\errelarm{2.3}{-0.2}{2.3}{-0.85}{1}{0}\errelarm{2.3}{-0.85}{2.3}{-1.5}{1}{0}\ercrowfoot{2.3}{-1.35}{2.15}{-1.5}{2.3}{-1.5}{2.45}{-1.5}{0}
% relationship 
\errelname{8.55}{-0.5}{l}{}\errelname{8.55}{-1.35}{l}{..}\errelarm{8.4}{-0.2}{8.4}{-0.85}{1}{0}\errelarm{8.4}{-0.85}{8.4}{-1.5}{1}{0}\ercrowfoot{8.4}{-1.35}{8.25}{-1.5}{8.4}{-1.5}{8.55}{-1.5}{0}
% relationship 
\errelname{14.523}{-0.5}{l}{}\errelname{14.523}{-3.55}{l}{..}\errelarm{14.373}{-0.2}{14.373}{-1.95}{1}{0}\errelarm{14.373}{-1.95}{14.373}{-3.7}{1}{0}\ercrowfoot{14.373}{-3.55}{14.223}{-3.7}{14.373}{-3.7}{14.523}{-3.7}{0}
% relationship 
\errelname{2.45}{-2.9}{l}{}\errelname{2.45}{-3.55}{l}{..}\errelarm{2.3}{-2.6}{2.3}{-3.15}{1}{0}\errelarm{2.3}{-3.15}{2.3}{-3.7}{1}{0}\eridcomprel{2.15}{2.4499999999999997}{-3.4000000000000004}\ercrowfoot{2.3}{-3.55}{2.15}{-3.7}{2.3}{-3.7}{2.45}{-3.7}{0}\ercrowfoot{2.3}{-3.55}{2.15}{-3.4}{2.3}{-3.4}{2.45}{-3.4}{0}\ercrowfoot{2.3}{-3.55}{2.15}{-3.7}{2.3}{-3.7}{2.45}{-3.7}{0}
% relationship of
\errelname{3.75}{-2.35}{l}{of}\errelname{5.5}{-1.9}{l}{R1}\errelarm{3.6}{-2.05}{5.35}{-2.05}{1}{0}\errelarm{5.35}{-2.05}{7.1}{-2.05}{0}{0}\ercrowfoot{3.75}{-2.05}{3.6}{-1.9}{3.6}{-2.05}{3.6}{-2.2}{0}
% relationship atom
\errelname{3.731}{-4.9}{l}{atom}\errelname{5.404}{-4.45}{l}{R2}\erscope{5.004}{-4.9}{l}{\textasciitilde /..=..}\errelarm{3.581}{-4.6}{5.254}{-4.6}{1}{0}\errelarm{5.254}{-4.6}{6.928}{-4.6}{0}{0}\ercrowfoot{3.731}{-4.6}{3.581}{-4.45}{3.581}{-4.6}{3.581}{-4.75}{0}\eridrefrel{3.8812499999999996}{-4.449999999999999}{-4.75}\ercrowfoot{3.731}{-4.6}{3.581}{-4.45}{3.581}{-4.6}{3.581}{-4.75}{0}\ercrowfoot{3.731}{-4.6}{3.881}{-4.45}{3.881}{-4.6}{3.881}{-4.75}{0}
% relationship 
\errelname{8.55}{-2.9}{l}{}\errelname{8.55}{-3.55}{l}{..}\errelarm{8.4}{-2.6}{8.4}{-3.15}{1}{0}\errelarm{8.4}{-3.15}{8.4}{-3.7}{1}{0}\eridcomprel{8.3}{8.5}{-3.45}\ercrowfoot{8.4}{-3.55}{8.25}{-3.7}{8.4}{-3.7}{8.55}{-3.7}{0}
% relationship 
\errelname{8.55}{-5.8}{l}{}\errelname{8.55}{-6.25}{l}{of}\errelarm{8.4}{-5.5}{8.4}{-5.95}{0}{0}\errelarm{8.4}{-5.95}{8.4}{-6.4}{1}{0}\eridcomprel{8.3}{8.5}{-6.1499999999999995}\ercrowfoot{8.4}{-6.25}{8.25}{-6.4}{8.4}{-6.4}{8.55}{-6.4}{0}
% relationship element
\errelname{10.023}{-4.9}{l}{element}\errelname{11.523}{-4.45}{l}{R3}\errelarm{9.873}{-4.6}{11.373}{-4.6}{1}{0}\errelarm{11.373}{-4.6}{12.873}{-4.6}{0}{0}\ercrowfoot{10.023}{-4.6}{9.873}{-4.45}{9.873}{-4.6}{9.873}{-4.75}{0}
% relationship with
\errelname{10.2}{-7.3}{l}{with}\errelname{10.2}{-6.85}{l}{with}\errelarm{10.05}{-7}{10.55}{-7}{1}{0}\errelarm{10.55}{-7}{10.05}{-7}{1}{0}\errelname{11.2}{-6.85}{l}{R4}\erscope{11.1}{-7.2}{l}{\textasciitilde /of/..=of/..}\eridrefrel{10.3}{-6.8999999999999995}{-7.099999999999999}
\end{erdiagram}

}
\end{center}
\end{frame}

\begin{frame}{MolecularGeometry  ERA diagram -- Relational}
\begin{center}
\scalebox{0.5}{
\begin{erdiagram}{8.5}{15.8725}

\eret{1}{-2.6}{3.6}{-1.5}{0.2}{1}\eretname{1.26}{-1.85}{l}{conformation}
\erattr{1.2}{-2.05}{1}{0}{id}
\erdattr{1.2}{-2.35}{1}{1}{of\_name(R1)}
\eret{0.764}{-5.8}{3.836}{-3.7}{0.2}{1}\eretname{1.071}{-4.05}{l}{position}
\erdattr{0.964}{-4.25}{1}{0}{conformation\_id(D4)}
\erdattr{0.964}{-4.55}{1}{0}{atom\_number(R2)}
\erattr{0.964}{-4.85}{1}{1}{x}
\erattr{0.964}{-5.15}{1}{1}{y}
\erattr{0.964}{-5.45}{1}{1}{z}
\eret{7.1}{-2.6}{9.7}{-1.5}{0.2}{1}\eretname{7.36}{-1.85}{l}{molStruct}
\erattr{7.3}{-2.05}{1}{0}{name}
\eret{6.928}{-5.8}{9.873}{-3.7}{0.2}{1}\eretname{7.222}{-4.05}{l}{atom}
\erdattr{7.128}{-4.25}{1}{0}{molStruct\_name(D5)}
\erattr{7.128}{-4.55}{1}{0}{number}
\erdattr{7.128}{-4.85}{1}{1}{element\_symbol(R3)}
\eret{6.75}{-8.5}{10.05}{-6.7}{0.2}{1}\eretname{7.08}{-7.05}{l}{bond formed}
\erdattr{6.95}{-7.25}{1}{0}{molStruct\_name(D6)}
\erdattr{6.95}{-7.55}{1}{0}{atom\_number(D6)}
\erdattr{6.95}{-7.85}{1}{0}{with\_atom\_number(R4)}
\erattr{6.95}{-8.15}{1}{1}{bondType}
\eret{12.873}{-5.2}{15.873}{-3.7}{0.2}{1}\eretname{13.173}{-4.05}{l}{element}
\erattr{13.073}{-4.25}{1}{0}{symbol}
\erattr{13.073}{-4.55}{1}{1}{name}
\erattr{13.073}{-4.85}{1}{1}{atomic number}
\eret{0}{-0.2}{15.873}{0.3}{0.2}{1}

% relationship 
\errelname{2.45}{-0.5}{l}{}\errelname{2.45}{-1.35}{l}{..}\errelarm{2.3}{-0.2}{2.3}{-0.85}{1}{0}\errelarm{2.3}{-0.85}{2.3}{-1.5}{1}{0}\ercrowfoot{2.3}{-1.35}{2.15}{-1.5}{2.3}{-1.5}{2.45}{-1.5}{0}
% relationship 
\errelname{8.55}{-0.5}{l}{}\errelname{8.55}{-1.35}{l}{..}\errelarm{8.4}{-0.2}{8.4}{-0.85}{1}{0}\errelarm{8.4}{-0.85}{8.4}{-1.5}{1}{0}\ercrowfoot{8.4}{-1.35}{8.25}{-1.5}{8.4}{-1.5}{8.55}{-1.5}{0}
% relationship 
\errelname{14.523}{-0.5}{l}{}\errelname{14.523}{-3.55}{l}{..}\errelarm{14.373}{-0.2}{14.373}{-1.95}{1}{0}\errelarm{14.373}{-1.95}{14.373}{-3.7}{1}{0}\ercrowfoot{14.373}{-3.55}{14.223}{-3.7}{14.373}{-3.7}{14.523}{-3.7}{0}
% relationship 
\errelname{2.45}{-2.9}{l}{}\errelname{2.45}{-3.55}{l}{..}\errelname{2.45}{-3}{l}{D4}\errelarm{2.3}{-2.6}{2.3}{-3.15}{1}{0}\errelarm{2.3}{-3.15}{2.3}{-3.7}{1}{0}\eridcomprel{2.15}{2.4499999999999997}{-3.4000000000000004}\ercrowfoot{2.3}{-3.55}{2.15}{-3.7}{2.3}{-3.7}{2.45}{-3.7}{0}\ercrowfoot{2.3}{-3.55}{2.15}{-3.4}{2.3}{-3.4}{2.45}{-3.4}{0}\ercrowfoot{2.3}{-3.55}{2.15}{-3.7}{2.3}{-3.7}{2.45}{-3.7}{0}
% relationship of
\errelname{3.75}{-2.35}{l}{of}\errelname{5.5}{-1.9}{l}{R1}\errelarm{3.6}{-2.05}{5.35}{-2.05}{1}{0}\errelarm{5.35}{-2.05}{7.1}{-2.05}{0}{0}\ercrowfoot{3.75}{-2.05}{3.6}{-1.9}{3.6}{-2.05}{3.6}{-2.2}{0}
% relationship atom
\errelname{3.986}{-5.05}{l}{atom}\errelname{5.532}{-4.6}{l}{R2}\erscope{5.132}{-5.05}{l}{\textasciitilde /..=..}\errelarm{3.836}{-4.75}{5.382}{-4.75}{1}{0}\errelarm{5.382}{-4.75}{6.928}{-4.75}{0}{0}\ercrowfoot{3.986}{-4.75}{3.836}{-4.6}{3.836}{-4.75}{3.836}{-4.9}{0}\eridrefrel{4.1362499999999995}{-4.6}{-4.9}\ercrowfoot{3.986}{-4.75}{3.836}{-4.6}{3.836}{-4.75}{3.836}{-4.9}{0}\ercrowfoot{3.986}{-4.75}{4.136}{-4.6}{4.136}{-4.75}{4.136}{-4.9}{0}
% relationship 
\errelname{8.55}{-2.9}{l}{}\errelname{8.55}{-3.55}{l}{..}\errelname{8.55}{-3}{l}{D5}\errelarm{8.4}{-2.6}{8.4}{-3.15}{1}{0}\errelarm{8.4}{-3.15}{8.4}{-3.7}{1}{0}\eridcomprel{8.3}{8.5}{-3.45}\ercrowfoot{8.4}{-3.55}{8.25}{-3.7}{8.4}{-3.7}{8.55}{-3.7}{0}
% relationship 
\errelname{8.55}{-6.1}{l}{}\errelname{8.55}{-6.55}{l}{of}\errelname{8.55}{-6.1}{l}{D6}\errelarm{8.4}{-5.8}{8.4}{-6.25}{0}{0}\errelarm{8.4}{-6.25}{8.4}{-6.7}{1}{0}\eridcomprel{8.3}{8.5}{-6.45}\ercrowfoot{8.4}{-6.55}{8.25}{-6.7}{8.4}{-6.7}{8.55}{-6.7}{0}
% relationship element
\errelname{10.023}{-5.05}{l}{element}\errelname{11.523}{-4.6}{l}{R3}\errelarm{9.873}{-4.75}{11.373}{-4.75}{1}{0}\errelarm{11.373}{-4.75}{12.873}{-4.75}{0}{0}\ercrowfoot{10.023}{-4.75}{9.873}{-4.6}{9.873}{-4.75}{9.873}{-4.9}{0}
% relationship with
\errelname{10.2}{-7.9}{l}{with}\errelname{10.2}{-7.45}{l}{with}\errelarm{10.05}{-7.6}{10.55}{-7.6}{1}{0}\errelarm{10.55}{-7.6}{10.05}{-7.6}{1}{0}\errelname{11.2}{-7.45}{l}{R4}\erscope{11.1}{-7.8}{l}{\textasciitilde /of/..=of/..}\eridrefrel{10.3}{-7.5}{-7.699999999999999}
\end{erdiagram}

}
\end{center}
\end{frame}


\begin{frame}{ER modelling Meta-Model}
\scalebox{0.23}{
\begin{erdiagram}{24.550000000000004}{49.63012500000001}

\eret{1.5}{-2.2}{2.833}{-1}{0.2}{1}\ertext{1.633}{-1.35}{l}{diagram}
\erattr{1.7}{-1.55}{0}{1}{deltaw}
\erattr{1.7}{-1.85}{0}{1}{deltah}
\eret{3.333}{-2.8}{6.461}{-1}{0.2}{1}\ertext{3.646}{-1.35}{l}{defaults}
\erattr{3.533}{-1.55}{0}{1}{etwidth}
\erattr{3.533}{-1.85}{0}{1}{etheight}
\erattr{3.533}{-2.15}{0}{1}{etyseparation}
\erattr{3.533}{-2.45}{0}{1}{etydeltaseparation}
\eret{47.04}{-1.9}{49.248}{-1}{0.2}{1}\ertext{47.261}{-1.35}{l}{xml}
\erattr{47.24}{-1.55}{0}{1}{namespaceuri}
\eret{10.461}{-3.75}{26.408}{-1}{0.2}{1}\ertext{10.681}{-1.35}{l}{ENTITY\textunderscore TYPE}
\erattr{10.661}{-1.55}{1}{0}{name}
\erattr{10.661}{-1.85}{0}{1}{description}
\erattr{10.661}{-2.15}{0}{1}{xpath}
\erattr{10.661}{-2.45}{0}{1}{modulename}
\eret{18.786}{-2.15}{20.158}{-1.55}{0.2}{0}\ertext{19.472}{-1.9}{}{absolute}
\eret{12.961}{-3.4}{18.286}{-1.55}{0.2}{0}\ertext{13.109}{-1.9}{l}{entity\textunderscore type\textunderscore like}
\eret{13.211}{-3.15}{15.811}{-2.15}{0.2}{1}\ertext{14.511}{-2.5}{}{entity\textunderscore type}
\eret{16.311}{-3.05}{18.036}{-2.15}{0.2}{1}\ertext{16.483}{-2.5}{l}{group}
\erattr{16.511}{-2.7}{0}{1}{annotation}
\eret{33.213}{-7.95}{46.066}{-6.75}{0.2}{1}\ertext{34.499}{-7.1}{l}{attribute}
\erattr{33.413}{-7.3}{1}{0}{name}
\erattr{33.413}{-7.6}{0}{1}{description}
\eret{8.808}{-6.65}{10.461}{-5.75}{0.2}{1}\ertext{9.635}{-6.1}{}{dependency}\ertext{9.635}{-6.4}{}{group}
\eret{11.461}{-12.05}{27.213}{-5.75}{0.2}{1}\ertext{11.965}{-6.1}{l}{Relationship}
\erattr{11.661}{-6.3}{0}{0}{name}
\erattr{11.661}{-6.6}{0}{1}{description}
\erattr{11.661}{-6.9}{0}{1}{id}
\erattr{11.661}{-7.2}{0}{1}{scope}
\erattr{11.661}{-7.5}{0}{1}{physicalprefix}
\eret{15.461}{-9}{26.463}{-6.1}{0.2}{0}\ertext{15.693}{-6.45}{l}{reference\textunderscore or\textunderscore dependency}
\eret{16.461}{-7.55}{18.113}{-6.95}{0.2}{1}\ertext{17.287}{-7.3}{}{dependency}
\eret{18.613}{-8.45}{24.613}{-6.45}{0.2}{1}\ertext{19.213}{-6.8}{l}{reference}
\erattr{18.813}{-7}{0}{1}{js}
\erattr{18.813}{-7.3}{0}{1}{xpathevaluate}
\eret{18.613}{-10.45}{20.613}{-9.55}{0.2}{0}\ertext{19.613}{-9.9}{}{constructed}\ertext{19.613}{-10.2}{}{relationship}
\eret{15.887}{-10.65}{17.687}{-10.05}{0.2}{0}\ertext{16.787}{-10.4}{}{composition}
\eret{1.385}{-11.95}{7.885}{-9.25}{0.2}{1}\ertext{2.035}{-9.6}{l}{presentation}
\erattr{1.585}{-9.8}{0}{1}{x}
\erattr{1.585}{-10.1}{0}{1}{y}
\erattr{1.585}{-10.4}{0}{1}{h}
\erattr{1.585}{-10.7}{0}{1}{w}
\erattr{1.585}{-11}{0}{1}{deltah}
\erattr{1.585}{-11.3}{0}{1}{deltaw}
\erattr{1.585}{-11.6}{0}{1}{sign}
\eret{9.761}{-15.35}{13.061}{-12.95}{0.2}{1}\ertext{10.091}{-13.3}{l}{path}
\erattr{9.961}{-13.5}{0}{1}{srcsign}
\erattr{9.961}{-13.8}{0}{1}{srcarmlen}
\erattr{9.961}{-14.1}{0}{1}{srcattach}
\erattr{9.961}{-14.4}{0}{1}{destsign}
\erattr{9.961}{-14.7}{0}{1}{destarmlen}
\erattr{9.961}{-15}{0}{1}{destattach}
\eret{13.561}{-13.55}{14.933}{-12.95}{0.2}{1}\ertext{14.247}{-13.3}{}{sequence}
\eret{23.205}{-13.55}{24.757}{-12.95}{0.2}{1}\ertext{23.981}{-13.3}{}{projection}
\eret{24.007}{-14.55}{26.121}{-13.95}{0.2}{1}\ertext{25.064}{-14.3}{}{identifying(1)}
\eret{24.308}{-15.75}{25.82}{-15.15}{0.2}{1}\ertext{25.064}{-15.5}{}{inherited}
\eret{25.121}{-21.1}{27.564}{-16.95}{0.2}{1}\ertext{25.316}{-17.3}{l}{cardinality}
\eret{25.521}{-18.15}{27.033}{-17.55}{0.2}{0}\ertext{26.277}{-17.9}{}{ZeroOrOne}
\eret{25.451}{-19.05}{27.103}{-18.45}{0.2}{0}\ertext{26.277}{-18.8}{}{ExactlyOne}
\eret{25.24}{-19.95}{27.314}{-19.35}{0.2}{0}\ertext{26.277}{-19.7}{}{ZeroOneOrMore}
\eret{25.521}{-20.85}{27.033}{-20.25}{0.2}{0}\ertext{26.277}{-20.6}{}{OneOrMore}
\eret{15.095}{-13.55}{16.607}{-12.95}{0.2}{1}\ertext{15.851}{-13.3}{}{transient}
\eret{16.607}{-16.3}{18.479}{-13.95}{0.2}{1}\ertext{16.757}{-14.3}{l}{initialiser}
\eret{16.857}{-15.15}{18.229}{-14.55}{0.2}{0}\ertext{17.543}{-14.9}{}{pullback}
\eret{16.877}{-16.05}{18.21}{-15.45}{0.2}{0}\ertext{17.543}{-15.8}{}{copy}
\eret{18.479}{-23.85}{24.212}{-17.95}{0.2}{1}\ertext{18.938}{-18.3}{l}{navigation}
\erattr{18.679}{-18.5}{0}{1}{xpathevaluate}
\eret{19.979}{-21.2}{22.279}{-18.85}{0.2}{0}\ertext{20.163}{-19.2}{l}{complex}
\eret{20.229}{-20.05}{22.029}{-19.45}{0.2}{1}\ertext{21.129}{-19.8}{}{join}
\eret{20.229}{-20.95}{22.029}{-20.35}{0.2}{1}\ertext{21.129}{-20.7}{}{aggregate}
\eret{20.379}{-22.7}{21.879}{-22.1}{0.2}{0}\ertext{21.129}{-22.45}{}{component}
\eret{19.297}{-23.6}{20.669}{-23}{0.2}{0}\ertext{19.983}{-23.35}{}{identity}
\eret{21.169}{-23.6}{22.962}{-23}{0.2}{0}\ertext{22.065}{-23.35}{}{theabsolute}
\eret{1.303}{-17.6}{4.653}{-12.85}{0.2}{1}\ertext{1.571}{-13.2}{l}{position}
\erattr{1.503}{-13.4}{0}{1}{d}
\eret{1.703}{-14.65}{4.403}{-13.75}{0.2}{0}\ertext{1.973}{-14.1}{l}{relative}
\erattr{1.903}{-14.3}{0}{1}{ratio}
\eret{2.386}{-15.55}{3.72}{-14.95}{0.2}{0}\ertext{3.053}{-15.3}{}{abs}
\eret{2.386}{-16.45}{3.72}{-15.85}{0.2}{0}\ertext{3.053}{-16.2}{}{local}
\eret{2.386}{-17.35}{3.72}{-16.75}{0.2}{0}\ertext{3.053}{-17.1}{}{default}
\eret{5.153}{-20.6}{7.366}{-12.85}{0.2}{1}\ertext{5.33}{-13.2}{l}{CustomShape}
\eret{5.553}{-14.05}{6.886}{-13.45}{0.2}{0}\ertext{6.22}{-13.8}{}{Top}
\eret{5.553}{-14.95}{6.886}{-14.35}{0.2}{0}\ertext{6.22}{-14.7}{}{TopLeft}
\eret{5.534}{-15.85}{6.906}{-15.25}{0.2}{0}\ertext{6.22}{-15.6}{}{TopRight}
\eret{5.323}{-16.75}{7.116}{-16.15}{0.2}{0}\ertext{6.22}{-16.5}{}{MiddleRight}
\eret{5.393}{-17.65}{7.046}{-17.05}{0.2}{0}\ertext{6.22}{-17.4}{}{MiddleLeft}
\eret{5.393}{-18.55}{7.046}{-17.95}{0.2}{0}\ertext{6.22}{-18.3}{}{BottomLeft}
\eret{5.323}{-19.45}{7.116}{-18.85}{0.2}{0}\ertext{6.22}{-19.2}{}{BottomRight}
\eret{5.553}{-20.35}{6.886}{-19.75}{0.2}{0}\ertext{6.22}{-20.1}{}{Bottom}
\eret{33.122}{-13.85}{35.616}{-8.85}{0.2}{1}\ertext{34.369}{-9.2}{}{implementationOf}
\eret{36.116}{-17.5}{39.42}{-8.85}{0.2}{1}\ertext{36.38}{-9.2}{l}{value\textunderscore type}
\eret{37.116}{-10.05}{38.449}{-9.45}{0.2}{0}\ertext{37.782}{-9.8}{}{boolean}
\eret{37.116}{-10.95}{38.449}{-10.35}{0.2}{0}\ertext{37.782}{-10.7}{}{date}
\eret{37.096}{-11.85}{38.468}{-11.25}{0.2}{0}\ertext{37.782}{-11.6}{}{dateTime}
\eret{37.116}{-12.75}{38.449}{-12.15}{0.2}{0}\ertext{37.782}{-12.5}{}{integer}
\eret{37.116}{-13.65}{38.449}{-13.05}{0.2}{0}\ertext{37.782}{-13.4}{}{float}
\eret{36.395}{-14.55}{39.17}{-13.95}{0.2}{0}\ertext{37.782}{-14.3}{}{nonNegativeInteger}
\eret{36.605}{-15.45}{38.959}{-14.85}{0.2}{0}\ertext{37.782}{-15.2}{}{positiveInteger}
\eret{37.116}{-16.35}{38.449}{-15.75}{0.2}{0}\ertext{37.782}{-16.1}{}{string}
\eret{37.116}{-17.25}{38.449}{-16.65}{0.2}{0}\ertext{37.782}{-17}{}{time}
\eret{39.92}{-9.45}{42.133}{-8.85}{0.2}{1}\ertext{41.026}{-9.2}{}{identifying(2)}
\eret{42.633}{-9.75}{44.005}{-8.85}{0.2}{1}\ertext{42.77}{-9.2}{l}{optional}
\erattr{42.833}{-9.4}{0}{1}{value}
\eret{44.505}{-9.45}{46.158}{-8.85}{0.2}{1}\ertext{45.331}{-9.2}{}{deprecated}
\eret{8.578}{-18.15}{11.578}{-16.95}{0.2}{1}\ertext{8.878}{-17.3}{l}{label}
\erattr{8.778}{-17.5}{0}{1}{xAdjustment}
\erattr{8.778}{-17.8}{0}{1}{yAdjustment}
\eret{12.078}{-19.3}{13.911}{-16.95}{0.2}{1}\ertext{12.224}{-17.3}{l}{src\textunderscore or\textunderscore dest}
\eret{12.328}{-18.15}{13.661}{-17.55}{0.2}{0}\ertext{12.994}{-17.9}{}{ToSrc}
\eret{12.328}{-19.05}{13.661}{-18.45}{0.2}{0}\ertext{12.994}{-18.8}{}{ToDest}
\eret{14.411}{-19.3}{16.244}{-16.95}{0.2}{1}\ertext{14.557}{-17.3}{l}{step}
\eret{14.661}{-18.15}{15.994}{-17.55}{0.2}{0}\ertext{15.327}{-17.9}{}{vstep}
\eret{14.661}{-19.05}{15.994}{-18.45}{0.2}{0}\ertext{15.327}{-18.8}{}{hstep}
\eret{13.494}{-22.25}{17.161}{-20.2}{0.2}{1}\ertext{13.658}{-20.55}{l}{dimension}
\erattr{13.694}{-20.75}{0}{1}{src}
\erattr{13.694}{-21.05}{0}{1}{dest}
\eret{13.744}{-22}{15.077}{-21.4}{0.2}{0}\ertext{14.411}{-21.75}{}{reldim}
\eret{15.577}{-22}{16.911}{-21.4}{0.2}{0}\ertext{16.244}{-21.75}{}{absdim}
\eret{7.594}{-23.65}{9.427}{-20.4}{0.2}{1}\ertext{7.74}{-20.75}{l}{render}
\eret{7.844}{-21.6}{9.177}{-21}{0.2}{0}\ertext{8.51}{-21.35}{}{None}
\eret{7.844}{-22.5}{9.177}{-21.9}{0.2}{0}\ertext{8.51}{-22.25}{}{Split}
\eret{7.844}{-23.4}{9.177}{-22.8}{0.2}{0}\ertext{8.51}{-23.15}{}{NoSplit}
\eret{9.927}{-24.55}{12.561}{-20.4}{0.2}{1}\ertext{10.138}{-20.75}{l}{relative\textunderscore position}
\eret{10.177}{-21.6}{11.51}{-21}{0.2}{0}\ertext{10.844}{-21.35}{}{Right}
\eret{10.177}{-22.5}{11.51}{-21.9}{0.2}{0}\ertext{10.844}{-22.25}{}{Left}
\eret{10.177}{-23.4}{11.51}{-22.8}{0.2}{0}\ertext{10.844}{-23.15}{}{Upside}
\eret{10.158}{-24.3}{11.53}{-23.7}{0.2}{0}\ertext{10.844}{-24.05}{}{Downside}
\eret{46.658}{-12.1}{49.63}{-8.85}{0.2}{1}\ertext{46.895}{-9.2}{l}{xmlStyle(2)}
\eret{47.658}{-10.05}{49.17}{-9.45}{0.2}{0}\ertext{48.414}{-9.8}{}{Attribute}
\eret{47.587}{-10.95}{49.24}{-10.35}{0.2}{0}\ertext{48.414}{-10.7}{}{Element(2)}
\eret{47.447}{-11.85}{49.38}{-11.25}{0.2}{0}\ertext{48.414}{-11.6}{}{Anonymous(2)}
\eret{9.013}{-9.9}{11.256}{-7.55}{0.2}{1}\ertext{9.193}{-7.9}{l}{xmlStyle(1)}
\eret{9.213}{-8.75}{10.866}{-8.15}{0.2}{0}\ertext{10.039}{-8.5}{}{Element(1)}
\eret{9.073}{-9.65}{11.006}{-9.05}{0.2}{0}\ertext{10.039}{-9.4}{}{Anonymous(1)}
\eret{0}{-0.2}{49.63}{0.3}{0.2}{1}

% relationship 
\ertext{16.625}{-0.5}{l}{}\errelarm{16.475}{-0.2}{16.475}{-0.875}{0}{0}\errelarm{16.475}{-0.875}{16.475}{-1.55}{1}{0}\ercrowfoot{16.475}{-1.4}{16.325}{-1.55}{16.475}{-1.55}{16.625}{-1.55}{0}
% relationship 
\ertext{19.622}{-0.5}{l}{}\errelarm{19.472}{-0.2}{19.472}{-0.875}{1}{0}\errelarm{19.472}{-0.875}{19.472}{-1.55}{1}{0}
% relationship presentation
\ertext{2.317}{-0.5}{l}{presentation}\errelarm{2.167}{-0.2}{2.167}{-0.6}{0}{0}\errelarm{2.167}{-0.6}{2.167}{-1}{1}{0}
% relationship 
\ertext{5.047}{-0.5}{l}{}\errelarm{4.897}{-0.2}{4.897}{-0.6}{0}{0}\errelarm{4.897}{-0.6}{4.897}{-1}{1}{0}
% relationship 
\ertext{48.294}{-0.5}{l}{}\errelarm{48.144}{-0.2}{48.144}{-0.6}{0}{0}\errelarm{48.144}{-0.6}{48.144}{-1}{1}{0}
% relationship attributeDefault
\ertext{48.294}{-2.2}{l}{attributeDefault}\errelarm{48.144}{-1.9}{48.144}{-5.375}{0}{0}\errelarm{48.144}{-5.375}{48.144}{-8.85}{1}{0}
% relationship 
\ertext{11.009}{-4.05}{l}{}\errelarm{10.859}{-3.75}{10.859}{-3.825}{0}{0}\errelarm{4.635}{-8.875}{4.635}{-9.25}{1}{0}\errelangle{10.859}{-3.825}{10.859}{-3.9}{7.747}{-6.2}{0}{0}\errelangle{7.747}{-6.2}{4.635}{-8.5}{4.635}{-8.875}{1}{0}
% relationship 
\ertext{11.408}{-4.05}{l}{}\errelarm{11.258}{-3.75}{11.258}{-3.825}{0}{0}\errelarm{9.635}{-5.625}{9.635}{-5.75}{1}{0}\errelangle{11.258}{-3.825}{11.258}{-3.9}{10.446}{-4.7}{0}{0}\errelangle{10.446}{-4.7}{9.635}{-5.5}{9.635}{-5.625}{1}{0}\ercrowfoot{9.635}{-5.6}{9.485}{-5.75}{9.635}{-5.75}{9.785}{-5.75}{0}
% relationship 
\ertext{20.179}{-4.05}{l}{}\ertext{20.179}{-5.6}{l}{..}\errelarm{20.029}{-3.75}{20.029}{-4.75}{0}{0}\errelarm{20.029}{-4.75}{20.029}{-5.75}{1}{0}\eridcomprel{19.928999999999995}{20.128999999999998}{-5.5}\ercrowfoot{20.029}{-5.6}{19.879}{-5.75}{20.029}{-5.75}{20.179}{-5.75}{0}
% relationship 
\ertext{23.368}{-4.05}{l}{}\ertext{39.79}{-6.6}{l}{..}\errelarm{23.218}{-3.75}{23.218}{-3.825}{0}{0}\errelarm{39.64}{-6.538}{39.64}{-6.75}{1}{0}\errelangle{23.218}{-3.825}{23.218}{-3.9}{31.429}{-5.113}{0}{0}\errelangle{31.429}{-5.113}{39.64}{-6.325}{39.64}{-6.538}{1}{0}\eridcomprel{39.53955}{39.73955}{-6.5}\ercrowfoot{39.64}{-6.6}{39.49}{-6.75}{39.64}{-6.75}{39.79}{-6.75}{0}
% relationship 
\ertext{14.868}{-3.7}{l}{}\errelarm{14.718}{-3.4}{14.718}{-3.5}{0}{0}\errelarm{14.718}{-1.4}{14.718}{-1.55}{1}{0}\errelangle{14.718}{-3.5}{14.718}{-3.6}{13.718}{-3.6}{0}{0}\errelangle{14.718}{-1.4}{14.718}{-1.25}{13.718}{-1.25}{1}{0}\errelangle{13.718}{-3.6}{12.718}{-3.6}{12.718}{-2.425}{0}{0}\errelangle{12.718}{-2.425}{12.718}{-1.25}{13.718}{-1.25}{1}{0}\ercrowfoot{14.718}{-1.4}{14.568}{-1.55}{14.718}{-1.55}{14.868}{-1.55}{0}\erarc{14.123}{-1.45}{14.873}{-1.25}{16.373}{-1.25}{17.123}{-1.45}
% relationship xmlRepresentation
\ertext{15.311}{-3.6}{l}{xmlRepresentation}\errelarm{15.161}{-3.15}{15.161}{-3.95}{0}{0}\errelarm{10.807}{-7.5}{10.807}{-7.55}{1}{0}\errelangle{15.161}{-3.95}{15.161}{-4.75}{12.984}{-4.75}{0}{0}\errelangle{10.807}{-7.5}{10.807}{-7.45}{10.807}{-7.45}{1}{0}\errelangle{12.984}{-4.75}{10.807}{-4.75}{10.807}{-6.1}{0}{0}\errelangle{10.807}{-6.1}{10.807}{-7.45}{10.807}{-7.45}{1}{0}
% relationship 
\ertext{34.519}{-8.25}{l}{}\ertext{34.519}{-8.7}{l}{..}\errelarm{34.369}{-7.95}{34.369}{-8.4}{0}{0}\errelarm{34.369}{-8.4}{34.369}{-8.85}{1}{0}
% relationship type
\ertext{37.918}{-8.25}{l}{type}\errelarm{37.768}{-7.95}{37.768}{-8.4}{1}{0}\errelarm{37.768}{-8.4}{37.768}{-8.85}{1}{0}
% relationship 
\ertext{41.176}{-8.25}{l}{}\errelarm{41.026}{-7.95}{41.026}{-8.4}{0}{0}\errelarm{41.026}{-8.4}{41.026}{-8.85}{1}{0}
% relationship 
\ertext{43.469}{-8.25}{l}{}\errelarm{43.319}{-7.95}{43.319}{-8.4}{0}{0}\errelarm{43.319}{-8.4}{43.319}{-8.85}{1}{0}
% relationship 
\ertext{45.481}{-8.25}{l}{}\errelarm{45.331}{-7.95}{45.331}{-8.4}{0}{0}\errelarm{45.331}{-8.4}{45.331}{-8.85}{1}{0}
% relationship xmlStyle
\ertext{45.894}{-8.25}{l}{xmlStyle}\errelarm{45.744}{-7.95}{45.744}{-8.15}{0}{0}\errelarm{47.549}{-8.7}{47.549}{-8.85}{1}{0}\errelangle{45.744}{-8.15}{45.744}{-8.35}{46.647}{-8.45}{0}{0}\errelangle{46.647}{-8.45}{47.549}{-8.55}{47.549}{-8.7}{1}{0}
% relationship 
\ertext{9.785}{-6.95}{l}{}\errelarm{9.635}{-6.65}{9.635}{-6.725}{0}{0}\errelarm{6.26}{-8.925}{6.26}{-9.25}{1}{0}\errelangle{9.635}{-6.725}{9.635}{-6.8}{7.947}{-7.7}{0}{0}\errelangle{7.947}{-7.7}{6.26}{-8.6}{6.26}{-8.925}{1}{0}
% relationship diagram
\ertext{11.705}{-12.35}{r}{diagram}\errelarm{11.855}{-12.05}{11.855}{-12.5}{0}{0}\errelarm{11.855}{-12.5}{11.855}{-12.95}{1}{0}
% relationship 
\ertext{14.397}{-12.35}{l}{}\errelarm{14.247}{-12.05}{14.247}{-12.5}{0}{0}\errelarm{14.247}{-12.5}{14.247}{-12.95}{1}{0}
% relationship 
\ertext{25.214}{-12.35}{l}{}\errelarm{25.064}{-12.05}{25.064}{-13}{0}{0}\errelarm{25.064}{-13}{25.064}{-13.95}{1}{0}
% relationship cardinality
\ertext{26.492}{-12.35}{l}{cardinality}\errelarm{26.342}{-12.05}{26.342}{-14.5}{1}{0}\errelarm{26.342}{-14.5}{26.342}{-16.95}{1}{0}
% relationship type
\ertext{27.363}{-7.94}{l}{type}\errelarm{27.213}{-7.64}{27.313}{-7.64}{1}{0}\errelarm{26.608}{-3.2}{26.408}{-3.2}{0}{0}\errelangle{27.313}{-7.64}{27.413}{-7.64}{27.563}{-7.64}{1}{0}\errelangle{26.608}{-3.2}{26.808}{-3.2}{27.261}{-3.2}{0}{0}\errelangle{27.563}{-7.64}{27.713}{-7.64}{27.713}{-5.42}{1}{0}\errelangle{27.713}{-5.42}{27.713}{-3.2}{27.261}{-3.2}{0}{0}\ercrowfoot{27.363}{-7.64}{27.213}{-7.49}{27.213}{-7.64}{27.213}{-7.79}{0}
% relationship diagonal
\ertext{21.463}{-8.75}{r}{diagonal}\errelarm{21.613}{-8.45}{21.613}{-13.2}{0}{0}\errelarm{21.613}{-13.2}{21.613}{-17.95}{1}{0}
% relationship riser
\ertext{22.063}{-8.75}{l}{riser}\errelarm{21.913}{-8.45}{21.913}{-13.2}{0}{0}\errelarm{21.913}{-13.2}{21.913}{-17.95}{1}{0}
% relationship key
\ertext{23.263}{-8.75}{l}{key}\errelarm{23.113}{-8.45}{23.113}{-13.2}{0}{0}\errelarm{23.113}{-13.2}{23.113}{-17.95}{1}{0}
% relationship 
\ertext{24.131}{-8.75}{l}{}\errelarm{23.981}{-8.45}{23.981}{-10.7}{0}{0}\errelarm{23.981}{-10.7}{23.981}{-12.95}{1}{0}
% relationship inverse
\ertext{24.763}{-7.15}{l}{inverse}\ertext{24.763}{-6.7}{l}{inverse}\errelarm{24.613}{-6.85}{25.113}{-6.85}{0}{0}\ertext{25.263}{-6.85}{l}{\textasciitilde /..=type}\errelarm{25.113}{-6.85}{25.113}{-6.85}{0}{0}\errelarm{25.113}{-6.85}{24.613}{-6.85}{0}{0}
% relationship 
\ertext{20.349}{-10.75}{l}{}\errelarm{20.199}{-10.45}{20.199}{-14.2}{1}{0}\errelarm{20.199}{-14.2}{20.199}{-17.95}{1}{0}
% relationship inverse
\ertext{20.763}{-10.3}{l}{inverse}\ertext{20.763}{-9.85}{l}{inverse}\errelarm{20.613}{-10}{21.113}{-10}{0}{0}\errelarm{21.113}{-10}{21.113}{-10}{0}{0}\errelarm{21.113}{-10}{20.613}{-10}{0}{0}
% relationship 
\ertext{16.637}{-10.95}{l}{}\errelarm{16.487}{-10.65}{16.487}{-10.725}{0}{0}\errelarm{15.851}{-12.9}{15.851}{-12.95}{1}{0}\errelangle{16.487}{-10.725}{16.487}{-10.8}{16.169}{-11.825}{0}{0}\errelangle{16.169}{-11.825}{15.851}{-12.85}{15.851}{-12.9}{1}{0}
% relationship 
\ertext{17.237}{-10.95}{l}{}\ertext{17.693}{-13.8}{l}{..}\errelarm{17.087}{-10.65}{17.087}{-10.725}{0}{0}\errelarm{17.543}{-13.4}{17.543}{-13.95}{1}{0}\errelangle{17.087}{-10.725}{17.087}{-10.8}{17.315}{-11.825}{0}{0}\errelangle{17.315}{-11.825}{17.543}{-12.85}{17.543}{-13.4}{1}{0}
% relationship inverse
\ertext{15.737}{-10.65}{r}{inverse}\ertext{16.311}{-7.1}{r}{of}\ertext{16.311}{-6.8}{r}{inverse}\errelarm{15.887}{-10.35}{15.287}{-10.35}{0}{0}\errelarm{16.261}{-7.25}{16.461}{-7.25}{0}{0}\errelangle{15.287}{-10.35}{14.687}{-10.35}{14.587}{-10.35}{0}{0}\errelangle{16.261}{-7.25}{16.061}{-7.25}{15.274}{-7.25}{0}{0}\errelangle{14.587}{-10.35}{14.487}{-10.35}{14.487}{-8.8}{0}{0}\errelangle{14.487}{-8.8}{14.487}{-7.25}{15.274}{-7.25}{0}{0}
% relationship xl
\ertext{2.51}{-12.25}{l}{xl}\errelarm{2.36}{-11.95}{2.36}{-12.4}{0}{0}\errelarm{2.36}{-12.4}{2.36}{-12.85}{1}{0}
% relationship xc
\ertext{2.9}{-12.25}{l}{xc}\errelarm{2.75}{-11.95}{2.75}{-12.85}{0}{0}\errelarm{2.75}{-12.85}{2.75}{-13.75}{1}{0}
% relationship xr
\ertext{3.29}{-12.25}{l}{xr}\errelarm{3.14}{-11.95}{3.14}{-12.85}{0}{0}\errelarm{3.14}{-12.85}{3.14}{-13.75}{1}{0}
% relationship yt
\ertext{3.68}{-12.25}{l}{yt}\errelarm{3.53}{-11.95}{3.53}{-12.4}{0}{0}\errelarm{3.53}{-12.4}{3.53}{-12.85}{1}{0}
% relationship ym
\ertext{4.07}{-12.25}{l}{ym}\errelarm{3.92}{-11.95}{3.92}{-12.85}{0}{0}\errelarm{3.92}{-12.85}{3.92}{-13.75}{1}{0}
% relationship yb
\ertext{4.525}{-12.25}{l}{yb}\errelarm{4.375}{-11.95}{4.375}{-12.15}{0}{0}\errelarm{4.349}{-13.7}{4.349}{-13.75}{1}{0}\errelangle{4.375}{-12.15}{4.375}{-12.35}{4.362}{-13}{0}{0}\errelangle{4.362}{-13}{4.349}{-13.65}{4.349}{-13.7}{1}{0}
% relationship shape
\ertext{6.41}{-12.25}{l}{shape}\errelarm{6.26}{-11.95}{6.26}{-12.4}{0}{0}\errelarm{6.26}{-12.4}{6.26}{-12.85}{1}{0}
% relationship name
\ertext{7.872}{-12.25}{l}{name}\errelarm{7.722}{-11.95}{7.722}{-12.15}{0}{0}\errelarm{8.235}{-20.35}{8.235}{-20.4}{1}{0}\errelangle{7.722}{-12.15}{7.722}{-12.35}{7.979}{-16.325}{0}{0}\errelangle{7.979}{-16.325}{8.235}{-20.3}{8.235}{-20.35}{1}{0}
% relationship rightOf
\ertext{1.235}{-10.45}{r}{rightOf}\errelarm{1.385}{-10.15}{1.135}{-10.15}{0}{0}\errelarm{10.261}{-3.475}{10.461}{-3.475}{0}{0}\errelangle{1.135}{-10.15}{0.885}{-10.15}{0.785}{-10.15}{0}{0}\errelangle{10.261}{-3.475}{10.061}{-3.475}{5.373}{-3.475}{0}{0}\ertext{0.535}{-7.113}{r}{\textasciitilde /\textasciicircum =\textasciicircum }\errelangle{0.785}{-10.15}{0.685}{-10.15}{0.685}{-6.813}{0}{0}\errelangle{0.685}{-6.813}{0.685}{-3.475}{5.373}{-3.475}{0}{0}\ercrowfoot{1.235}{-10.15}{1.385}{-10}{1.385}{-10.15}{1.385}{-10.3}{0}
% relationship below
\ertext{1.235}{-11.35}{r}{below}\errelarm{1.385}{-11.05}{0.935}{-11.05}{0}{0}\errelarm{10.261}{-3.2}{10.461}{-3.2}{0}{0}\errelangle{0.935}{-11.05}{0.485}{-11.05}{0.435}{-11.05}{0}{0}\errelangle{10.261}{-3.2}{10.061}{-3.2}{5.223}{-3.2}{0}{0}\ertext{0.235}{-7.425}{r}{\textasciitilde /\textasciicircum =\textasciicircum }\errelangle{0.435}{-11.05}{0.385}{-11.05}{0.385}{-7.125}{0}{0}\errelangle{0.385}{-7.125}{0.385}{-3.2}{5.223}{-3.2}{0}{0}\ercrowfoot{1.235}{-11.05}{1.385}{-10.9}{1.385}{-11.05}{1.385}{-11.2}{0}\erarc{4.235}{-9.05}{4.86}{-8.85}{6.11}{-8.85}{6.735}{-9.05}
% relationship 
\ertext{10.241}{-15.65}{l}{}\errelarm{10.091}{-15.35}{10.091}{-16.15}{0}{0}\errelarm{10.091}{-16.15}{10.091}{-16.95}{1}{0}
% relationship inverse
\ertext{11.101}{-15.65}{l}{inverse}\errelarm{10.751}{-15.35}{10.751}{-16.15}{0}{0}\errelarm{10.751}{-16.15}{10.751}{-16.95}{1}{0}
% relationship id
\ertext{11.1}{-15.95}{l}{id}\errelarm{10.85}{-15.35}{10.85}{-16.15}{0}{0}\errelarm{10.85}{-16.15}{10.85}{-16.95}{1}{0}
% relationship scope
\ertext{11.099}{-16.25}{l}{scope}\errelarm{10.949}{-15.35}{10.949}{-16.15}{0}{0}\errelarm{10.949}{-16.15}{10.949}{-16.95}{1}{0}
% relationship align
\ertext{12.056}{-15.65}{l}{align}\errelarm{11.906}{-15.35}{11.906}{-15.55}{0}{0}\errelarm{12.994}{-16.9}{12.994}{-16.95}{1}{0}\errelangle{11.906}{-15.55}{11.906}{-15.75}{12.45}{-16.3}{0}{0}\errelangle{12.45}{-16.3}{12.994}{-16.85}{12.994}{-16.9}{1}{0}
% relationship 
\ertext{12.881}{-15.65}{l}{}\errelarm{12.731}{-15.35}{12.731}{-15.425}{0}{0}\errelarm{15.327}{-16.75}{15.327}{-16.95}{1}{0}\errelangle{12.731}{-15.425}{12.731}{-15.5}{14.029}{-16.025}{0}{0}\errelangle{14.029}{-16.025}{15.327}{-16.55}{15.327}{-16.75}{1}{0}
% relationship 
\ertext{25.214}{-14.85}{l}{}\errelarm{25.064}{-14.55}{25.064}{-14.85}{0}{0}\errelarm{25.064}{-14.85}{25.064}{-15.15}{1}{0}
% relationship along
\ertext{17.019}{-16.6}{r}{along}\errelarm{17.169}{-16.3}{17.169}{-16.6}{1}{0}\errelarm{19.052}{-17.9}{19.052}{-17.95}{1}{0}\errelangle{17.169}{-16.6}{17.169}{-16.9}{18.111}{-17.375}{1}{0}\errelangle{18.111}{-17.375}{19.052}{-17.85}{19.052}{-17.9}{1}{0}
% relationship type
\ertext{18.629}{-15.798}{l}{type}\errelarm{18.479}{-15.948}{19.079}{-15.948}{1}{0}\errelarm{26.608}{-3.063}{26.408}{-3.063}{0}{0}\errelangle{19.079}{-15.948}{19.679}{-15.948}{23.779}{-15.948}{1}{0}\errelangle{26.608}{-3.063}{26.808}{-3.063}{27.343}{-3.063}{0}{0}\errelangle{23.779}{-15.948}{27.879}{-15.948}{27.879}{-9.505}{1}{0}\errelangle{27.879}{-9.505}{27.879}{-3.063}{27.343}{-3.063}{0}{0}\ercrowfoot{18.629}{-15.948}{18.479}{-15.798}{18.479}{-15.948}{18.479}{-16.098}{0}
% relationship projectionrel
\ertext{18.679}{-14.97}{l}{projection}\ertext{18.679}{-15.27}{l}{rel}\errelarm{18.229}{-14.67}{18.529}{-14.67}{0}{0}\errelarm{18.413}{-7.95}{18.613}{-7.95}{0}{0}\ertext{18.671}{-11.61}{l}{\textasciitilde /..=../type}\errelangle{18.529}{-14.67}{18.829}{-14.67}{18.521}{-11.31}{0}{0}\errelangle{18.521}{-11.31}{18.213}{-7.95}{18.413}{-7.95}{0}{0}\ercrowfoot{18.379}{-14.67}{18.229}{-14.52}{18.229}{-14.67}{18.229}{-14.82}{0}
% relationship riser2
\ertext{17.96}{-16.65}{l}{riser2}\errelarm{17.81}{-16.05}{17.81}{-16.3}{1}{0}\errelarm{19.626}{-17.9}{19.626}{-17.95}{1}{0}\errelangle{17.81}{-16.3}{17.81}{-16.55}{18.718}{-17.2}{1}{0}\errelangle{18.718}{-17.2}{19.626}{-17.85}{19.626}{-17.9}{1}{0}
% relationship src
\ertext{24.362}{-22.97}{l}{src}\errelarm{24.212}{-22.67}{24.662}{-22.67}{0}{0}\errelarm{26.608}{-2.238}{26.408}{-2.238}{0}{0}\errelangle{24.662}{-22.67}{25.112}{-22.67}{27.512}{-22.67}{0}{0}\errelangle{26.608}{-2.238}{26.808}{-2.238}{28.36}{-2.238}{0}{0}\errelangle{27.512}{-22.67}{29.912}{-22.67}{29.912}{-12.454}{0}{0}\errelangle{29.912}{-12.454}{29.912}{-2.238}{28.36}{-2.238}{0}{0}\ercrowfoot{24.362}{-22.67}{24.212}{-22.52}{24.212}{-22.67}{24.212}{-22.82}{0}
% relationship dest
\ertext{24.362}{-23.56}{l}{dest}\errelarm{24.212}{-23.26}{24.762}{-23.26}{0}{0}\errelarm{26.608}{-2.1}{26.408}{-2.1}{0}{0}\errelangle{24.762}{-23.26}{25.312}{-23.26}{27.712}{-23.26}{0}{0}\errelangle{26.608}{-2.1}{26.808}{-2.1}{28.46}{-2.1}{0}{0}\errelangle{27.712}{-23.26}{30.112}{-23.26}{30.112}{-12.68}{0}{0}\errelangle{30.112}{-12.68}{30.112}{-2.1}{28.46}{-2.1}{0}{0}\ercrowfoot{24.362}{-23.26}{24.212}{-23.11}{24.212}{-23.26}{24.212}{-23.41}{0}\erarc{18.479}{-17.75}{19.729}{-17.55}{22.229}{-17.55}{23.479}{-17.75}
% relationship 
\ertext{21.279}{-21.5}{l}{}\errelarm{21.129}{-21.2}{21.129}{-21.65}{0}{0}\errelarm{21.129}{-21.65}{21.129}{-22.1}{1}{0}\errelseq{21.189}{-21.65}{20.779}{-21.71}{21.479}{-21.77}{21.069}{-21.83}\ercrowfoot{21.129}{-21.95}{20.979}{-22.1}{21.129}{-22.1}{21.279}{-22.1}{0}
% relationship rel
\ertext{22.029}{-22.7}{l}{rel}\errelarm{21.879}{-22.4}{22.029}{-22.4}{1}{0}\errelarm{27.413}{-10.475}{27.213}{-10.475}{0}{0}\errelangle{22.029}{-22.4}{22.179}{-22.4}{25.429}{-22.4}{1}{0}\errelangle{27.413}{-10.475}{27.613}{-10.475}{28.146}{-10.475}{0}{0}\ertext{28.529}{-16.738}{r}{\textasciitilde /..=src}\errelangle{25.429}{-22.4}{28.679}{-22.4}{28.679}{-16.438}{1}{0}\errelangle{28.679}{-16.438}{28.679}{-10.475}{28.146}{-10.475}{0}{0}\ercrowfoot{22.029}{-22.4}{21.879}{-22.25}{21.879}{-22.4}{21.879}{-22.55}{0}\erarc{1.853}{-12.65}{2.578}{-12.45}{4.028}{-12.45}{4.753}{-12.65}
% relationship to
\ertext{1.553}{-14.5}{r}{to}\errelarm{1.703}{-14.2}{1.103}{-14.2}{1}{0}\errelarm{10.261}{-2.925}{10.461}{-2.925}{0}{0}\errelangle{1.103}{-14.2}{0.503}{-14.2}{0.303}{-14.2}{1}{0}\errelangle{10.261}{-2.925}{10.061}{-2.925}{5.082}{-2.925}{0}{0}\ertext{-0.047}{-8.863}{r}{\textasciitilde /\textasciicircum =\textasciicircum }\errelangle{0.303}{-14.2}{0.103}{-14.2}{0.103}{-8.563}{1}{0}\errelangle{0.103}{-8.563}{0.103}{-2.925}{5.082}{-2.925}{0}{0}\ercrowfoot{1.553}{-14.2}{1.703}{-14.05}{1.703}{-14.2}{1.703}{-14.35}{0}
% relationship destattr
\ertext{32.972}{-9.7}{r}{destattr}\errelarm{33.122}{-9.85}{32.672}{-9.85}{1}{0}\errelarm{33.013}{-7.59}{33.213}{-7.59}{0}{0}\errelangle{32.672}{-9.85}{32.222}{-9.85}{32.197}{-9.85}{1}{0}\errelangle{33.013}{-7.59}{32.813}{-7.59}{32.492}{-7.59}{0}{0}\ertext{33.322}{-9.02}{r}{\textasciitilde /..=rel/type}\errelangle{32.197}{-9.85}{32.172}{-9.85}{32.172}{-8.72}{1}{0}\errelangle{32.172}{-8.72}{32.172}{-7.59}{32.492}{-7.59}{0}{0}\ercrowfoot{32.972}{-9.85}{33.122}{-9.7}{33.122}{-9.85}{33.122}{-10}{0}
% relationship attrOfOrigin
\ertext{32.972}{-10.65}{r}{attrOfOrigin}\errelarm{33.122}{-10.35}{32.622}{-10.35}{1}{0}\errelarm{33.013}{-7.35}{33.213}{-7.35}{0}{0}\errelangle{32.622}{-10.35}{32.122}{-10.35}{32.097}{-10.35}{1}{0}\errelangle{33.013}{-7.35}{32.813}{-7.35}{32.442}{-7.35}{0}{0}\ertext{33.222}{-9.15}{r}{\textasciitilde /..=typeOfOrigin}\errelangle{32.097}{-10.35}{32.072}{-10.35}{32.072}{-8.85}{1}{0}\errelangle{32.072}{-8.85}{32.072}{-7.35}{32.442}{-7.35}{0}{0}\ercrowfoot{32.972}{-10.35}{33.122}{-10.2}{33.122}{-10.35}{33.122}{-10.5}{0}
% relationship typeOfOrigin
\ertext{32.972}{-12.15}{r}{typeOfOrigin}\errelarm{33.122}{-11.85}{32.622}{-11.85}{1}{0}\errelarm{26.608}{-1.55}{26.408}{-1.55}{0}{0}\errelangle{32.622}{-11.85}{32.122}{-11.85}{31.622}{-11.85}{1}{0}\errelangle{26.608}{-1.55}{26.808}{-1.55}{28.965}{-1.55}{0}{0}\ertext{32.272}{-7}{r}{\textasciitilde /\textasciicircum =\textasciicircum }\errelangle{31.622}{-11.85}{31.122}{-11.85}{31.122}{-6.7}{1}{0}\errelangle{31.122}{-6.7}{31.122}{-1.55}{28.965}{-1.55}{0}{0}\ercrowfoot{32.972}{-11.85}{33.122}{-11.7}{33.122}{-11.85}{33.122}{-12}{0}
% relationship rel
\ertext{32.972}{-13.15}{r}{rel}\errelarm{33.122}{-12.85}{32.522}{-12.85}{1}{0}\errelarm{26.663}{-8.42}{26.463}{-8.42}{0}{0}\ertext{29.942}{-11.385}{l}{\textasciitilde /..=../..}\errelangle{32.522}{-12.85}{31.922}{-12.85}{29.392}{-10.635}{1}{0}\errelangle{29.392}{-10.635}{26.863}{-8.42}{26.663}{-8.42}{0}{0}\ercrowfoot{32.972}{-12.85}{33.122}{-12.7}{33.122}{-12.85}{33.122}{-13}{0}
% relationship name
\ertext{9.728}{-18.45}{l}{name}\errelarm{9.578}{-18.15}{9.578}{-18.35}{0}{0}\errelarm{8.785}{-20.35}{8.785}{-20.4}{1}{0}\errelangle{9.578}{-18.35}{9.578}{-18.55}{9.181}{-19.425}{0}{0}\errelangle{9.181}{-19.425}{8.785}{-20.3}{8.785}{-20.35}{1}{0}
% relationship position
\ertext{10.728}{-18.45}{l}{position}\errelarm{10.578}{-18.15}{10.578}{-18.35}{0}{0}\errelarm{11.244}{-20.35}{11.244}{-20.4}{1}{0}\errelangle{10.578}{-18.35}{10.578}{-18.55}{10.911}{-19.425}{0}{0}\errelangle{10.911}{-19.425}{11.244}{-20.3}{11.244}{-20.35}{1}{0}\erarc{9.828}{-16.85}{10.203}{-16.65}{10.953}{-16.65}{11.328}{-16.85}
% relationship 
\ertext{15.172}{-19.6}{l}{}\errelarm{15.022}{-19.3}{15.022}{-19.375}{0}{0}\errelarm{15.052}{-16.75}{15.052}{-16.95}{1}{0}\errelangle{15.022}{-19.375}{15.022}{-19.45}{14.622}{-19.45}{0}{0}\errelangle{15.052}{-16.75}{15.052}{-16.55}{14.637}{-16.55}{1}{0}\errelangle{14.622}{-19.45}{14.222}{-19.45}{14.222}{-18}{0}{0}\errelangle{14.222}{-18}{14.222}{-16.55}{14.637}{-16.55}{1}{0}
% relationship 
\ertext{15.783}{-19.6}{l}{}\errelarm{15.633}{-19.3}{15.633}{-19.75}{0}{0}\errelarm{15.633}{-19.75}{15.633}{-20.2}{1}{0}\erarc{14.786}{-16.85}{14.973}{-16.65}{15.348}{-16.65}{15.536}{-16.85}\erarc{7.777}{-20.2}{8.144}{-20}{8.877}{-20}{9.244}{-20.2}\erarc{47.008}{-8.75}{47.383}{-8.55}{48.133}{-8.55}{48.508}{-8.75}
\end{erdiagram}

}
\end{frame}



\begin{frame}{Data Specification viewed as ER diagram}
\begin{center}
\scalebox{0.5}{
\begin{erdiagram}{8.1}{17.04625}

\eret{1}{-2.4}{4}{-1.5}{0.2}{1}\ertext{1.3}{-1.85}{l}{molecularStructure}
\erattr{1.2}{-2.05}{1}{0}{name}
\eret{0.454}{-5.4}{4.546}{-3.3}{0.2}{1}\ertext{0.863}{-3.65}{l}{atom}
\erdattr{0.654}{-3.85}{1}{0}{molecularStructurename(D3)}
\erattr{0.654}{-4.15}{1}{0}{atomId}
\erdattr{0.654}{-4.45}{1}{1}{elementsymbol(R1)}
\erattr{0.654}{-4.75}{1}{1}{x}
\erattr{0.654}{-5.05}{1}{1}{y}
\eret{0.554}{-8.1}{4.446}{-6.3}{0.2}{1}\ertext{0.943}{-6.65}{l}{bond formed}
\erdattr{0.754}{-6.85}{1}{0}{molecularStructurename(D4)}
\erdattr{0.754}{-7.15}{1}{0}{atomatomId(D4)}
\erdattr{0.754}{-7.45}{1}{0}{withAtomId(R2)}
\erattr{0.754}{-7.75}{1}{1}{bondType}
\eret{10.546}{-4.8}{13.546}{-3.3}{0.2}{1}\ertext{10.846}{-3.65}{l}{element}
\erattr{10.746}{-3.85}{1}{0}{symbol}
\erattr{10.746}{-4.15}{1}{1}{name}
\erattr{10.746}{-4.45}{1}{1}{atomic number}
\eret{7.046}{-7.7}{10.046}{-5.9}{0.2}{1}\ertext{7.346}{-6.25}{l}{isotope}
\erdattr{7.246}{-6.45}{1}{0}{elementsymbol(D5)}
\erattr{7.246}{-6.75}{1}{0}{numberOfNeutrons}
\erattr{7.246}{-7.05}{1}{1}{mass}
\erattr{7.246}{-7.35}{1}{1}{abundancy}
\eret{10.546}{-8}{13.546}{-5.9}{0.2}{1}\ertext{10.846}{-6.25}{l}{allotrope}
\erattr{10.746}{-6.45}{1}{0}{name}
\erdattr{10.746}{-6.75}{1}{1}{elementsymbol(D6)}
\erattr{10.746}{-7.05}{0}{1}{melting point}
\erattr{10.746}{-7.35}{0}{1}{boiling point}
\erattr{10.746}{-7.65}{0}{1}{density}
\eret{14.046}{-7.1}{17.046}{-5.9}{0.2}{1}\ertext{14.346}{-6.25}{l}{valency}
\erdattr{14.246}{-6.45}{1}{0}{elementsymbol(D7)}
\erattr{14.246}{-6.75}{1}{0}{number}
\eret{0}{-0.2}{17.046}{0.3}{0.2}{1}

% relationship 
\ertext{2.65}{-0.5}{l}{}\ertext{2.65}{-1.35}{l}{..}\errelarm{2.5}{-0.2}{2.5}{-0.85}{1}{0}\errelarm{2.5}{-0.85}{2.5}{-1.5}{1}{0}\ercrowfoot{2.5}{-1.35}{2.35}{-1.5}{2.5}{-1.5}{2.65}{-1.5}{0}
% relationship 
\ertext{12.196}{-0.5}{l}{}\ertext{12.196}{-3.15}{l}{..}\errelarm{12.046}{-0.2}{12.046}{-1.75}{1}{0}\errelarm{12.046}{-1.75}{12.046}{-3.3}{1}{0}\ercrowfoot{12.046}{-3.15}{11.896}{-3.3}{12.046}{-3.3}{12.196}{-3.3}{0}
% relationship 
\ertext{2.65}{-2.7}{l}{}\ertext{2.65}{-3.15}{l}{..}\ertext{2.65}{-2.7}{l}{D3}\errelarm{2.5}{-2.4}{2.5}{-2.85}{1}{0}\errelarm{2.5}{-2.85}{2.5}{-3.3}{1}{0}\eridcomprel{2.4}{2.6}{-3.05}\ercrowfoot{2.5}{-3.15}{2.35}{-3.3}{2.5}{-3.3}{2.65}{-3.3}{0}
% relationship 
\ertext{2.65}{-5.7}{l}{}\ertext{2.65}{-6.15}{l}{of}\ertext{2.65}{-5.7}{l}{D4}\errelarm{2.5}{-5.4}{2.5}{-5.85}{0}{0}\errelarm{2.5}{-5.85}{2.5}{-6.3}{1}{0}\eridcomprel{2.4}{2.6}{-6.05}\ercrowfoot{2.5}{-6.15}{2.35}{-6.3}{2.5}{-6.3}{2.65}{-6.3}{0}
% relationship element
\ertext{4.696}{-4.65}{l}{element}\ertext{7.696}{-4.2}{l}{R1}\errelarm{4.546}{-4.35}{7.546}{-4.35}{1}{0}\errelarm{7.546}{-4.35}{10.546}{-4.35}{0}{0}\ercrowfoot{4.696}{-4.35}{4.546}{-4.2}{4.546}{-4.35}{4.546}{-4.5}{0}
% relationship with
\ertext{4.596}{-7.5}{l}{with}\ertext{4.596}{-7.05}{l}{with}\errelarm{4.446}{-7.2}{4.946}{-7.2}{1}{0}\errelarm{4.946}{-7.2}{4.446}{-7.2}{1}{0}\ertext{5.596}{-7.05}{l}{R2}\eridrefrel{4.69625}{-7.1}{-7.299999999999999}
% relationship 
\ertext{11.446}{-5.1}{l}{}\ertext{8.696}{-5.75}{l}{..}\errelarm{11.296}{-4.8}{11.296}{-4.875}{1}{0}\errelarm{8.546}{-5.687}{8.546}{-5.9}{1}{0}\ertext{9.771}{-5.063}{r}{D5}\errelangle{11.296}{-4.875}{11.296}{-4.95}{9.921}{-5.213}{1}{0}\errelangle{9.921}{-5.213}{8.546}{-5.475}{8.546}{-5.687}{1}{0}\eridcomprel{8.446250000000001}{8.64625}{-5.6499999999999995}\ercrowfoot{8.546}{-5.75}{8.396}{-5.9}{8.546}{-5.9}{8.696}{-5.9}{0}
% relationship 
\ertext{12.196}{-5.1}{l}{}\ertext{12.196}{-5.75}{l}{..}\ertext{12.196}{-5.2}{l}{D6}\errelarm{12.046}{-4.8}{12.046}{-5.35}{1}{0}\errelarm{12.046}{-5.35}{12.046}{-5.9}{1}{0}\ercrowfoot{12.046}{-5.75}{11.896}{-5.9}{12.046}{-5.9}{12.196}{-5.9}{0}
% relationship 
\ertext{12.946}{-5.1}{l}{}\ertext{15.696}{-5.75}{l}{..}\errelarm{12.796}{-4.8}{12.796}{-4.875}{1}{0}\errelarm{15.546}{-5.687}{15.546}{-5.9}{1}{0}\ertext{14.321}{-5.063}{l}{D7}\errelangle{12.796}{-4.875}{12.796}{-4.95}{14.171}{-5.213}{1}{0}\errelangle{14.171}{-5.213}{15.546}{-5.475}{15.546}{-5.687}{1}{0}\eridcomprel{15.446250000000001}{15.64625}{-5.6499999999999995}\ercrowfoot{15.546}{-5.75}{15.396}{-5.9}{15.546}{-5.9}{15.696}{-5.9}{0}
\end{erdiagram}

}
\end{center}
\end{frame}
\begin{frame}{ER modelling Meta-Model}
\scalebox{0.23}{
\begin{erdiagram}{24.550000000000004}{49.63012500000001}

\eret{1.5}{-2.2}{2.833}{-1}{0.2}{1}\ertext{1.633}{-1.35}{l}{diagram}
\erattr{1.7}{-1.55}{0}{1}{deltaw}
\erattr{1.7}{-1.85}{0}{1}{deltah}
\eret{3.333}{-2.8}{6.461}{-1}{0.2}{1}\ertext{3.646}{-1.35}{l}{defaults}
\erattr{3.533}{-1.55}{0}{1}{etwidth}
\erattr{3.533}{-1.85}{0}{1}{etheight}
\erattr{3.533}{-2.15}{0}{1}{etyseparation}
\erattr{3.533}{-2.45}{0}{1}{etydeltaseparation}
\eret{47.04}{-1.9}{49.248}{-1}{0.2}{1}\ertext{47.261}{-1.35}{l}{xml}
\erattr{47.24}{-1.55}{0}{1}{namespaceuri}
\eret{10.461}{-3.75}{26.408}{-1}{0.2}{1}\ertext{10.681}{-1.35}{l}{ENTITY\textunderscore TYPE}
\erattr{10.661}{-1.55}{1}{0}{name}
\erattr{10.661}{-1.85}{0}{1}{description}
\erattr{10.661}{-2.15}{0}{1}{xpath}
\erattr{10.661}{-2.45}{0}{1}{modulename}
\eret{18.786}{-2.15}{20.158}{-1.55}{0.2}{0}\ertext{19.472}{-1.9}{}{absolute}
\eret{12.961}{-3.4}{18.286}{-1.55}{0.2}{0}\ertext{13.109}{-1.9}{l}{entity\textunderscore type\textunderscore like}
\eret{13.211}{-3.15}{15.811}{-2.15}{0.2}{1}\ertext{14.511}{-2.5}{}{entity\textunderscore type}
\eret{16.311}{-3.05}{18.036}{-2.15}{0.2}{1}\ertext{16.483}{-2.5}{l}{group}
\erattr{16.511}{-2.7}{0}{1}{annotation}
\eret{33.213}{-7.95}{46.066}{-6.75}{0.2}{1}\ertext{34.499}{-7.1}{l}{attribute}
\erattr{33.413}{-7.3}{1}{0}{name}
\erattr{33.413}{-7.6}{0}{1}{description}
\eret{8.808}{-6.65}{10.461}{-5.75}{0.2}{1}\ertext{9.635}{-6.1}{}{dependency}\ertext{9.635}{-6.4}{}{group}
\eret{11.461}{-12.05}{27.213}{-5.75}{0.2}{1}\ertext{11.965}{-6.1}{l}{Relationship}
\erattr{11.661}{-6.3}{0}{0}{name}
\erattr{11.661}{-6.6}{0}{1}{description}
\erattr{11.661}{-6.9}{0}{1}{id}
\erattr{11.661}{-7.2}{0}{1}{scope}
\erattr{11.661}{-7.5}{0}{1}{physicalprefix}
\eret{15.461}{-9}{26.463}{-6.1}{0.2}{0}\ertext{15.693}{-6.45}{l}{reference\textunderscore or\textunderscore dependency}
\eret{16.461}{-7.55}{18.113}{-6.95}{0.2}{1}\ertext{17.287}{-7.3}{}{dependency}
\eret{18.613}{-8.45}{24.613}{-6.45}{0.2}{1}\ertext{19.213}{-6.8}{l}{reference}
\erattr{18.813}{-7}{0}{1}{js}
\erattr{18.813}{-7.3}{0}{1}{xpathevaluate}
\eret{18.613}{-10.45}{20.613}{-9.55}{0.2}{0}\ertext{19.613}{-9.9}{}{constructed}\ertext{19.613}{-10.2}{}{relationship}
\eret{15.887}{-10.65}{17.687}{-10.05}{0.2}{0}\ertext{16.787}{-10.4}{}{composition}
\eret{1.385}{-11.95}{7.885}{-9.25}{0.2}{1}\ertext{2.035}{-9.6}{l}{presentation}
\erattr{1.585}{-9.8}{0}{1}{x}
\erattr{1.585}{-10.1}{0}{1}{y}
\erattr{1.585}{-10.4}{0}{1}{h}
\erattr{1.585}{-10.7}{0}{1}{w}
\erattr{1.585}{-11}{0}{1}{deltah}
\erattr{1.585}{-11.3}{0}{1}{deltaw}
\erattr{1.585}{-11.6}{0}{1}{sign}
\eret{9.761}{-15.35}{13.061}{-12.95}{0.2}{1}\ertext{10.091}{-13.3}{l}{path}
\erattr{9.961}{-13.5}{0}{1}{srcsign}
\erattr{9.961}{-13.8}{0}{1}{srcarmlen}
\erattr{9.961}{-14.1}{0}{1}{srcattach}
\erattr{9.961}{-14.4}{0}{1}{destsign}
\erattr{9.961}{-14.7}{0}{1}{destarmlen}
\erattr{9.961}{-15}{0}{1}{destattach}
\eret{13.561}{-13.55}{14.933}{-12.95}{0.2}{1}\ertext{14.247}{-13.3}{}{sequence}
\eret{23.205}{-13.55}{24.757}{-12.95}{0.2}{1}\ertext{23.981}{-13.3}{}{projection}
\eret{24.007}{-14.55}{26.121}{-13.95}{0.2}{1}\ertext{25.064}{-14.3}{}{identifying(1)}
\eret{24.308}{-15.75}{25.82}{-15.15}{0.2}{1}\ertext{25.064}{-15.5}{}{inherited}
\eret{25.121}{-21.1}{27.564}{-16.95}{0.2}{1}\ertext{25.316}{-17.3}{l}{cardinality}
\eret{25.521}{-18.15}{27.033}{-17.55}{0.2}{0}\ertext{26.277}{-17.9}{}{ZeroOrOne}
\eret{25.451}{-19.05}{27.103}{-18.45}{0.2}{0}\ertext{26.277}{-18.8}{}{ExactlyOne}
\eret{25.24}{-19.95}{27.314}{-19.35}{0.2}{0}\ertext{26.277}{-19.7}{}{ZeroOneOrMore}
\eret{25.521}{-20.85}{27.033}{-20.25}{0.2}{0}\ertext{26.277}{-20.6}{}{OneOrMore}
\eret{15.095}{-13.55}{16.607}{-12.95}{0.2}{1}\ertext{15.851}{-13.3}{}{transient}
\eret{16.607}{-16.3}{18.479}{-13.95}{0.2}{1}\ertext{16.757}{-14.3}{l}{initialiser}
\eret{16.857}{-15.15}{18.229}{-14.55}{0.2}{0}\ertext{17.543}{-14.9}{}{pullback}
\eret{16.877}{-16.05}{18.21}{-15.45}{0.2}{0}\ertext{17.543}{-15.8}{}{copy}
\eret{18.479}{-23.85}{24.212}{-17.95}{0.2}{1}\ertext{18.938}{-18.3}{l}{navigation}
\erattr{18.679}{-18.5}{0}{1}{xpathevaluate}
\eret{19.979}{-21.2}{22.279}{-18.85}{0.2}{0}\ertext{20.163}{-19.2}{l}{complex}
\eret{20.229}{-20.05}{22.029}{-19.45}{0.2}{1}\ertext{21.129}{-19.8}{}{join}
\eret{20.229}{-20.95}{22.029}{-20.35}{0.2}{1}\ertext{21.129}{-20.7}{}{aggregate}
\eret{20.379}{-22.7}{21.879}{-22.1}{0.2}{0}\ertext{21.129}{-22.45}{}{component}
\eret{19.297}{-23.6}{20.669}{-23}{0.2}{0}\ertext{19.983}{-23.35}{}{identity}
\eret{21.169}{-23.6}{22.962}{-23}{0.2}{0}\ertext{22.065}{-23.35}{}{theabsolute}
\eret{1.303}{-17.6}{4.653}{-12.85}{0.2}{1}\ertext{1.571}{-13.2}{l}{position}
\erattr{1.503}{-13.4}{0}{1}{d}
\eret{1.703}{-14.65}{4.403}{-13.75}{0.2}{0}\ertext{1.973}{-14.1}{l}{relative}
\erattr{1.903}{-14.3}{0}{1}{ratio}
\eret{2.386}{-15.55}{3.72}{-14.95}{0.2}{0}\ertext{3.053}{-15.3}{}{abs}
\eret{2.386}{-16.45}{3.72}{-15.85}{0.2}{0}\ertext{3.053}{-16.2}{}{local}
\eret{2.386}{-17.35}{3.72}{-16.75}{0.2}{0}\ertext{3.053}{-17.1}{}{default}
\eret{5.153}{-20.6}{7.366}{-12.85}{0.2}{1}\ertext{5.33}{-13.2}{l}{CustomShape}
\eret{5.553}{-14.05}{6.886}{-13.45}{0.2}{0}\ertext{6.22}{-13.8}{}{Top}
\eret{5.553}{-14.95}{6.886}{-14.35}{0.2}{0}\ertext{6.22}{-14.7}{}{TopLeft}
\eret{5.534}{-15.85}{6.906}{-15.25}{0.2}{0}\ertext{6.22}{-15.6}{}{TopRight}
\eret{5.323}{-16.75}{7.116}{-16.15}{0.2}{0}\ertext{6.22}{-16.5}{}{MiddleRight}
\eret{5.393}{-17.65}{7.046}{-17.05}{0.2}{0}\ertext{6.22}{-17.4}{}{MiddleLeft}
\eret{5.393}{-18.55}{7.046}{-17.95}{0.2}{0}\ertext{6.22}{-18.3}{}{BottomLeft}
\eret{5.323}{-19.45}{7.116}{-18.85}{0.2}{0}\ertext{6.22}{-19.2}{}{BottomRight}
\eret{5.553}{-20.35}{6.886}{-19.75}{0.2}{0}\ertext{6.22}{-20.1}{}{Bottom}
\eret{33.122}{-13.85}{35.616}{-8.85}{0.2}{1}\ertext{34.369}{-9.2}{}{implementationOf}
\eret{36.116}{-17.5}{39.42}{-8.85}{0.2}{1}\ertext{36.38}{-9.2}{l}{value\textunderscore type}
\eret{37.116}{-10.05}{38.449}{-9.45}{0.2}{0}\ertext{37.782}{-9.8}{}{boolean}
\eret{37.116}{-10.95}{38.449}{-10.35}{0.2}{0}\ertext{37.782}{-10.7}{}{date}
\eret{37.096}{-11.85}{38.468}{-11.25}{0.2}{0}\ertext{37.782}{-11.6}{}{dateTime}
\eret{37.116}{-12.75}{38.449}{-12.15}{0.2}{0}\ertext{37.782}{-12.5}{}{integer}
\eret{37.116}{-13.65}{38.449}{-13.05}{0.2}{0}\ertext{37.782}{-13.4}{}{float}
\eret{36.395}{-14.55}{39.17}{-13.95}{0.2}{0}\ertext{37.782}{-14.3}{}{nonNegativeInteger}
\eret{36.605}{-15.45}{38.959}{-14.85}{0.2}{0}\ertext{37.782}{-15.2}{}{positiveInteger}
\eret{37.116}{-16.35}{38.449}{-15.75}{0.2}{0}\ertext{37.782}{-16.1}{}{string}
\eret{37.116}{-17.25}{38.449}{-16.65}{0.2}{0}\ertext{37.782}{-17}{}{time}
\eret{39.92}{-9.45}{42.133}{-8.85}{0.2}{1}\ertext{41.026}{-9.2}{}{identifying(2)}
\eret{42.633}{-9.75}{44.005}{-8.85}{0.2}{1}\ertext{42.77}{-9.2}{l}{optional}
\erattr{42.833}{-9.4}{0}{1}{value}
\eret{44.505}{-9.45}{46.158}{-8.85}{0.2}{1}\ertext{45.331}{-9.2}{}{deprecated}
\eret{8.578}{-18.15}{11.578}{-16.95}{0.2}{1}\ertext{8.878}{-17.3}{l}{label}
\erattr{8.778}{-17.5}{0}{1}{xAdjustment}
\erattr{8.778}{-17.8}{0}{1}{yAdjustment}
\eret{12.078}{-19.3}{13.911}{-16.95}{0.2}{1}\ertext{12.224}{-17.3}{l}{src\textunderscore or\textunderscore dest}
\eret{12.328}{-18.15}{13.661}{-17.55}{0.2}{0}\ertext{12.994}{-17.9}{}{ToSrc}
\eret{12.328}{-19.05}{13.661}{-18.45}{0.2}{0}\ertext{12.994}{-18.8}{}{ToDest}
\eret{14.411}{-19.3}{16.244}{-16.95}{0.2}{1}\ertext{14.557}{-17.3}{l}{step}
\eret{14.661}{-18.15}{15.994}{-17.55}{0.2}{0}\ertext{15.327}{-17.9}{}{vstep}
\eret{14.661}{-19.05}{15.994}{-18.45}{0.2}{0}\ertext{15.327}{-18.8}{}{hstep}
\eret{13.494}{-22.25}{17.161}{-20.2}{0.2}{1}\ertext{13.658}{-20.55}{l}{dimension}
\erattr{13.694}{-20.75}{0}{1}{src}
\erattr{13.694}{-21.05}{0}{1}{dest}
\eret{13.744}{-22}{15.077}{-21.4}{0.2}{0}\ertext{14.411}{-21.75}{}{reldim}
\eret{15.577}{-22}{16.911}{-21.4}{0.2}{0}\ertext{16.244}{-21.75}{}{absdim}
\eret{7.594}{-23.65}{9.427}{-20.4}{0.2}{1}\ertext{7.74}{-20.75}{l}{render}
\eret{7.844}{-21.6}{9.177}{-21}{0.2}{0}\ertext{8.51}{-21.35}{}{None}
\eret{7.844}{-22.5}{9.177}{-21.9}{0.2}{0}\ertext{8.51}{-22.25}{}{Split}
\eret{7.844}{-23.4}{9.177}{-22.8}{0.2}{0}\ertext{8.51}{-23.15}{}{NoSplit}
\eret{9.927}{-24.55}{12.561}{-20.4}{0.2}{1}\ertext{10.138}{-20.75}{l}{relative\textunderscore position}
\eret{10.177}{-21.6}{11.51}{-21}{0.2}{0}\ertext{10.844}{-21.35}{}{Right}
\eret{10.177}{-22.5}{11.51}{-21.9}{0.2}{0}\ertext{10.844}{-22.25}{}{Left}
\eret{10.177}{-23.4}{11.51}{-22.8}{0.2}{0}\ertext{10.844}{-23.15}{}{Upside}
\eret{10.158}{-24.3}{11.53}{-23.7}{0.2}{0}\ertext{10.844}{-24.05}{}{Downside}
\eret{46.658}{-12.1}{49.63}{-8.85}{0.2}{1}\ertext{46.895}{-9.2}{l}{xmlStyle(2)}
\eret{47.658}{-10.05}{49.17}{-9.45}{0.2}{0}\ertext{48.414}{-9.8}{}{Attribute}
\eret{47.587}{-10.95}{49.24}{-10.35}{0.2}{0}\ertext{48.414}{-10.7}{}{Element(2)}
\eret{47.447}{-11.85}{49.38}{-11.25}{0.2}{0}\ertext{48.414}{-11.6}{}{Anonymous(2)}
\eret{9.013}{-9.9}{11.256}{-7.55}{0.2}{1}\ertext{9.193}{-7.9}{l}{xmlStyle(1)}
\eret{9.213}{-8.75}{10.866}{-8.15}{0.2}{0}\ertext{10.039}{-8.5}{}{Element(1)}
\eret{9.073}{-9.65}{11.006}{-9.05}{0.2}{0}\ertext{10.039}{-9.4}{}{Anonymous(1)}
\eret{0}{-0.2}{49.63}{0.3}{0.2}{1}

% relationship 
\ertext{16.625}{-0.5}{l}{}\errelarm{16.475}{-0.2}{16.475}{-0.875}{0}{0}\errelarm{16.475}{-0.875}{16.475}{-1.55}{1}{0}\ercrowfoot{16.475}{-1.4}{16.325}{-1.55}{16.475}{-1.55}{16.625}{-1.55}{0}
% relationship 
\ertext{19.622}{-0.5}{l}{}\errelarm{19.472}{-0.2}{19.472}{-0.875}{1}{0}\errelarm{19.472}{-0.875}{19.472}{-1.55}{1}{0}
% relationship presentation
\ertext{2.317}{-0.5}{l}{presentation}\errelarm{2.167}{-0.2}{2.167}{-0.6}{0}{0}\errelarm{2.167}{-0.6}{2.167}{-1}{1}{0}
% relationship 
\ertext{5.047}{-0.5}{l}{}\errelarm{4.897}{-0.2}{4.897}{-0.6}{0}{0}\errelarm{4.897}{-0.6}{4.897}{-1}{1}{0}
% relationship 
\ertext{48.294}{-0.5}{l}{}\errelarm{48.144}{-0.2}{48.144}{-0.6}{0}{0}\errelarm{48.144}{-0.6}{48.144}{-1}{1}{0}
% relationship attributeDefault
\ertext{48.294}{-2.2}{l}{attributeDefault}\errelarm{48.144}{-1.9}{48.144}{-5.375}{0}{0}\errelarm{48.144}{-5.375}{48.144}{-8.85}{1}{0}
% relationship 
\ertext{11.009}{-4.05}{l}{}\errelarm{10.859}{-3.75}{10.859}{-3.825}{0}{0}\errelarm{4.635}{-8.875}{4.635}{-9.25}{1}{0}\errelangle{10.859}{-3.825}{10.859}{-3.9}{7.747}{-6.2}{0}{0}\errelangle{7.747}{-6.2}{4.635}{-8.5}{4.635}{-8.875}{1}{0}
% relationship 
\ertext{11.408}{-4.05}{l}{}\errelarm{11.258}{-3.75}{11.258}{-3.825}{0}{0}\errelarm{9.635}{-5.625}{9.635}{-5.75}{1}{0}\errelangle{11.258}{-3.825}{11.258}{-3.9}{10.446}{-4.7}{0}{0}\errelangle{10.446}{-4.7}{9.635}{-5.5}{9.635}{-5.625}{1}{0}\ercrowfoot{9.635}{-5.6}{9.485}{-5.75}{9.635}{-5.75}{9.785}{-5.75}{0}
% relationship 
\ertext{20.179}{-4.05}{l}{}\ertext{20.179}{-5.6}{l}{..}\errelarm{20.029}{-3.75}{20.029}{-4.75}{0}{0}\errelarm{20.029}{-4.75}{20.029}{-5.75}{1}{0}\eridcomprel{19.928999999999995}{20.128999999999998}{-5.5}\ercrowfoot{20.029}{-5.6}{19.879}{-5.75}{20.029}{-5.75}{20.179}{-5.75}{0}
% relationship 
\ertext{23.368}{-4.05}{l}{}\ertext{39.79}{-6.6}{l}{..}\errelarm{23.218}{-3.75}{23.218}{-3.825}{0}{0}\errelarm{39.64}{-6.538}{39.64}{-6.75}{1}{0}\errelangle{23.218}{-3.825}{23.218}{-3.9}{31.429}{-5.113}{0}{0}\errelangle{31.429}{-5.113}{39.64}{-6.325}{39.64}{-6.538}{1}{0}\eridcomprel{39.53955}{39.73955}{-6.5}\ercrowfoot{39.64}{-6.6}{39.49}{-6.75}{39.64}{-6.75}{39.79}{-6.75}{0}
% relationship 
\ertext{14.868}{-3.7}{l}{}\errelarm{14.718}{-3.4}{14.718}{-3.5}{0}{0}\errelarm{14.718}{-1.4}{14.718}{-1.55}{1}{0}\errelangle{14.718}{-3.5}{14.718}{-3.6}{13.718}{-3.6}{0}{0}\errelangle{14.718}{-1.4}{14.718}{-1.25}{13.718}{-1.25}{1}{0}\errelangle{13.718}{-3.6}{12.718}{-3.6}{12.718}{-2.425}{0}{0}\errelangle{12.718}{-2.425}{12.718}{-1.25}{13.718}{-1.25}{1}{0}\ercrowfoot{14.718}{-1.4}{14.568}{-1.55}{14.718}{-1.55}{14.868}{-1.55}{0}\erarc{14.123}{-1.45}{14.873}{-1.25}{16.373}{-1.25}{17.123}{-1.45}
% relationship xmlRepresentation
\ertext{15.311}{-3.6}{l}{xmlRepresentation}\errelarm{15.161}{-3.15}{15.161}{-3.95}{0}{0}\errelarm{10.807}{-7.5}{10.807}{-7.55}{1}{0}\errelangle{15.161}{-3.95}{15.161}{-4.75}{12.984}{-4.75}{0}{0}\errelangle{10.807}{-7.5}{10.807}{-7.45}{10.807}{-7.45}{1}{0}\errelangle{12.984}{-4.75}{10.807}{-4.75}{10.807}{-6.1}{0}{0}\errelangle{10.807}{-6.1}{10.807}{-7.45}{10.807}{-7.45}{1}{0}
% relationship 
\ertext{34.519}{-8.25}{l}{}\ertext{34.519}{-8.7}{l}{..}\errelarm{34.369}{-7.95}{34.369}{-8.4}{0}{0}\errelarm{34.369}{-8.4}{34.369}{-8.85}{1}{0}
% relationship type
\ertext{37.918}{-8.25}{l}{type}\errelarm{37.768}{-7.95}{37.768}{-8.4}{1}{0}\errelarm{37.768}{-8.4}{37.768}{-8.85}{1}{0}
% relationship 
\ertext{41.176}{-8.25}{l}{}\errelarm{41.026}{-7.95}{41.026}{-8.4}{0}{0}\errelarm{41.026}{-8.4}{41.026}{-8.85}{1}{0}
% relationship 
\ertext{43.469}{-8.25}{l}{}\errelarm{43.319}{-7.95}{43.319}{-8.4}{0}{0}\errelarm{43.319}{-8.4}{43.319}{-8.85}{1}{0}
% relationship 
\ertext{45.481}{-8.25}{l}{}\errelarm{45.331}{-7.95}{45.331}{-8.4}{0}{0}\errelarm{45.331}{-8.4}{45.331}{-8.85}{1}{0}
% relationship xmlStyle
\ertext{45.894}{-8.25}{l}{xmlStyle}\errelarm{45.744}{-7.95}{45.744}{-8.15}{0}{0}\errelarm{47.549}{-8.7}{47.549}{-8.85}{1}{0}\errelangle{45.744}{-8.15}{45.744}{-8.35}{46.647}{-8.45}{0}{0}\errelangle{46.647}{-8.45}{47.549}{-8.55}{47.549}{-8.7}{1}{0}
% relationship 
\ertext{9.785}{-6.95}{l}{}\errelarm{9.635}{-6.65}{9.635}{-6.725}{0}{0}\errelarm{6.26}{-8.925}{6.26}{-9.25}{1}{0}\errelangle{9.635}{-6.725}{9.635}{-6.8}{7.947}{-7.7}{0}{0}\errelangle{7.947}{-7.7}{6.26}{-8.6}{6.26}{-8.925}{1}{0}
% relationship diagram
\ertext{11.705}{-12.35}{r}{diagram}\errelarm{11.855}{-12.05}{11.855}{-12.5}{0}{0}\errelarm{11.855}{-12.5}{11.855}{-12.95}{1}{0}
% relationship 
\ertext{14.397}{-12.35}{l}{}\errelarm{14.247}{-12.05}{14.247}{-12.5}{0}{0}\errelarm{14.247}{-12.5}{14.247}{-12.95}{1}{0}
% relationship 
\ertext{25.214}{-12.35}{l}{}\errelarm{25.064}{-12.05}{25.064}{-13}{0}{0}\errelarm{25.064}{-13}{25.064}{-13.95}{1}{0}
% relationship cardinality
\ertext{26.492}{-12.35}{l}{cardinality}\errelarm{26.342}{-12.05}{26.342}{-14.5}{1}{0}\errelarm{26.342}{-14.5}{26.342}{-16.95}{1}{0}
% relationship type
\ertext{27.363}{-7.94}{l}{type}\errelarm{27.213}{-7.64}{27.313}{-7.64}{1}{0}\errelarm{26.608}{-3.2}{26.408}{-3.2}{0}{0}\errelangle{27.313}{-7.64}{27.413}{-7.64}{27.563}{-7.64}{1}{0}\errelangle{26.608}{-3.2}{26.808}{-3.2}{27.261}{-3.2}{0}{0}\errelangle{27.563}{-7.64}{27.713}{-7.64}{27.713}{-5.42}{1}{0}\errelangle{27.713}{-5.42}{27.713}{-3.2}{27.261}{-3.2}{0}{0}\ercrowfoot{27.363}{-7.64}{27.213}{-7.49}{27.213}{-7.64}{27.213}{-7.79}{0}
% relationship diagonal
\ertext{21.463}{-8.75}{r}{diagonal}\errelarm{21.613}{-8.45}{21.613}{-13.2}{0}{0}\errelarm{21.613}{-13.2}{21.613}{-17.95}{1}{0}
% relationship riser
\ertext{22.063}{-8.75}{l}{riser}\errelarm{21.913}{-8.45}{21.913}{-13.2}{0}{0}\errelarm{21.913}{-13.2}{21.913}{-17.95}{1}{0}
% relationship key
\ertext{23.263}{-8.75}{l}{key}\errelarm{23.113}{-8.45}{23.113}{-13.2}{0}{0}\errelarm{23.113}{-13.2}{23.113}{-17.95}{1}{0}
% relationship 
\ertext{24.131}{-8.75}{l}{}\errelarm{23.981}{-8.45}{23.981}{-10.7}{0}{0}\errelarm{23.981}{-10.7}{23.981}{-12.95}{1}{0}
% relationship inverse
\ertext{24.763}{-7.15}{l}{inverse}\ertext{24.763}{-6.7}{l}{inverse}\errelarm{24.613}{-6.85}{25.113}{-6.85}{0}{0}\ertext{25.263}{-6.85}{l}{\textasciitilde /..=type}\errelarm{25.113}{-6.85}{25.113}{-6.85}{0}{0}\errelarm{25.113}{-6.85}{24.613}{-6.85}{0}{0}
% relationship 
\ertext{20.349}{-10.75}{l}{}\errelarm{20.199}{-10.45}{20.199}{-14.2}{1}{0}\errelarm{20.199}{-14.2}{20.199}{-17.95}{1}{0}
% relationship inverse
\ertext{20.763}{-10.3}{l}{inverse}\ertext{20.763}{-9.85}{l}{inverse}\errelarm{20.613}{-10}{21.113}{-10}{0}{0}\errelarm{21.113}{-10}{21.113}{-10}{0}{0}\errelarm{21.113}{-10}{20.613}{-10}{0}{0}
% relationship 
\ertext{16.637}{-10.95}{l}{}\errelarm{16.487}{-10.65}{16.487}{-10.725}{0}{0}\errelarm{15.851}{-12.9}{15.851}{-12.95}{1}{0}\errelangle{16.487}{-10.725}{16.487}{-10.8}{16.169}{-11.825}{0}{0}\errelangle{16.169}{-11.825}{15.851}{-12.85}{15.851}{-12.9}{1}{0}
% relationship 
\ertext{17.237}{-10.95}{l}{}\ertext{17.693}{-13.8}{l}{..}\errelarm{17.087}{-10.65}{17.087}{-10.725}{0}{0}\errelarm{17.543}{-13.4}{17.543}{-13.95}{1}{0}\errelangle{17.087}{-10.725}{17.087}{-10.8}{17.315}{-11.825}{0}{0}\errelangle{17.315}{-11.825}{17.543}{-12.85}{17.543}{-13.4}{1}{0}
% relationship inverse
\ertext{15.737}{-10.65}{r}{inverse}\ertext{16.311}{-7.1}{r}{of}\ertext{16.311}{-6.8}{r}{inverse}\errelarm{15.887}{-10.35}{15.287}{-10.35}{0}{0}\errelarm{16.261}{-7.25}{16.461}{-7.25}{0}{0}\errelangle{15.287}{-10.35}{14.687}{-10.35}{14.587}{-10.35}{0}{0}\errelangle{16.261}{-7.25}{16.061}{-7.25}{15.274}{-7.25}{0}{0}\errelangle{14.587}{-10.35}{14.487}{-10.35}{14.487}{-8.8}{0}{0}\errelangle{14.487}{-8.8}{14.487}{-7.25}{15.274}{-7.25}{0}{0}
% relationship xl
\ertext{2.51}{-12.25}{l}{xl}\errelarm{2.36}{-11.95}{2.36}{-12.4}{0}{0}\errelarm{2.36}{-12.4}{2.36}{-12.85}{1}{0}
% relationship xc
\ertext{2.9}{-12.25}{l}{xc}\errelarm{2.75}{-11.95}{2.75}{-12.85}{0}{0}\errelarm{2.75}{-12.85}{2.75}{-13.75}{1}{0}
% relationship xr
\ertext{3.29}{-12.25}{l}{xr}\errelarm{3.14}{-11.95}{3.14}{-12.85}{0}{0}\errelarm{3.14}{-12.85}{3.14}{-13.75}{1}{0}
% relationship yt
\ertext{3.68}{-12.25}{l}{yt}\errelarm{3.53}{-11.95}{3.53}{-12.4}{0}{0}\errelarm{3.53}{-12.4}{3.53}{-12.85}{1}{0}
% relationship ym
\ertext{4.07}{-12.25}{l}{ym}\errelarm{3.92}{-11.95}{3.92}{-12.85}{0}{0}\errelarm{3.92}{-12.85}{3.92}{-13.75}{1}{0}
% relationship yb
\ertext{4.525}{-12.25}{l}{yb}\errelarm{4.375}{-11.95}{4.375}{-12.15}{0}{0}\errelarm{4.349}{-13.7}{4.349}{-13.75}{1}{0}\errelangle{4.375}{-12.15}{4.375}{-12.35}{4.362}{-13}{0}{0}\errelangle{4.362}{-13}{4.349}{-13.65}{4.349}{-13.7}{1}{0}
% relationship shape
\ertext{6.41}{-12.25}{l}{shape}\errelarm{6.26}{-11.95}{6.26}{-12.4}{0}{0}\errelarm{6.26}{-12.4}{6.26}{-12.85}{1}{0}
% relationship name
\ertext{7.872}{-12.25}{l}{name}\errelarm{7.722}{-11.95}{7.722}{-12.15}{0}{0}\errelarm{8.235}{-20.35}{8.235}{-20.4}{1}{0}\errelangle{7.722}{-12.15}{7.722}{-12.35}{7.979}{-16.325}{0}{0}\errelangle{7.979}{-16.325}{8.235}{-20.3}{8.235}{-20.35}{1}{0}
% relationship rightOf
\ertext{1.235}{-10.45}{r}{rightOf}\errelarm{1.385}{-10.15}{1.135}{-10.15}{0}{0}\errelarm{10.261}{-3.475}{10.461}{-3.475}{0}{0}\errelangle{1.135}{-10.15}{0.885}{-10.15}{0.785}{-10.15}{0}{0}\errelangle{10.261}{-3.475}{10.061}{-3.475}{5.373}{-3.475}{0}{0}\ertext{0.535}{-7.113}{r}{\textasciitilde /\textasciicircum =\textasciicircum }\errelangle{0.785}{-10.15}{0.685}{-10.15}{0.685}{-6.813}{0}{0}\errelangle{0.685}{-6.813}{0.685}{-3.475}{5.373}{-3.475}{0}{0}\ercrowfoot{1.235}{-10.15}{1.385}{-10}{1.385}{-10.15}{1.385}{-10.3}{0}
% relationship below
\ertext{1.235}{-11.35}{r}{below}\errelarm{1.385}{-11.05}{0.935}{-11.05}{0}{0}\errelarm{10.261}{-3.2}{10.461}{-3.2}{0}{0}\errelangle{0.935}{-11.05}{0.485}{-11.05}{0.435}{-11.05}{0}{0}\errelangle{10.261}{-3.2}{10.061}{-3.2}{5.223}{-3.2}{0}{0}\ertext{0.235}{-7.425}{r}{\textasciitilde /\textasciicircum =\textasciicircum }\errelangle{0.435}{-11.05}{0.385}{-11.05}{0.385}{-7.125}{0}{0}\errelangle{0.385}{-7.125}{0.385}{-3.2}{5.223}{-3.2}{0}{0}\ercrowfoot{1.235}{-11.05}{1.385}{-10.9}{1.385}{-11.05}{1.385}{-11.2}{0}\erarc{4.235}{-9.05}{4.86}{-8.85}{6.11}{-8.85}{6.735}{-9.05}
% relationship 
\ertext{10.241}{-15.65}{l}{}\errelarm{10.091}{-15.35}{10.091}{-16.15}{0}{0}\errelarm{10.091}{-16.15}{10.091}{-16.95}{1}{0}
% relationship inverse
\ertext{11.101}{-15.65}{l}{inverse}\errelarm{10.751}{-15.35}{10.751}{-16.15}{0}{0}\errelarm{10.751}{-16.15}{10.751}{-16.95}{1}{0}
% relationship id
\ertext{11.1}{-15.95}{l}{id}\errelarm{10.85}{-15.35}{10.85}{-16.15}{0}{0}\errelarm{10.85}{-16.15}{10.85}{-16.95}{1}{0}
% relationship scope
\ertext{11.099}{-16.25}{l}{scope}\errelarm{10.949}{-15.35}{10.949}{-16.15}{0}{0}\errelarm{10.949}{-16.15}{10.949}{-16.95}{1}{0}
% relationship align
\ertext{12.056}{-15.65}{l}{align}\errelarm{11.906}{-15.35}{11.906}{-15.55}{0}{0}\errelarm{12.994}{-16.9}{12.994}{-16.95}{1}{0}\errelangle{11.906}{-15.55}{11.906}{-15.75}{12.45}{-16.3}{0}{0}\errelangle{12.45}{-16.3}{12.994}{-16.85}{12.994}{-16.9}{1}{0}
% relationship 
\ertext{12.881}{-15.65}{l}{}\errelarm{12.731}{-15.35}{12.731}{-15.425}{0}{0}\errelarm{15.327}{-16.75}{15.327}{-16.95}{1}{0}\errelangle{12.731}{-15.425}{12.731}{-15.5}{14.029}{-16.025}{0}{0}\errelangle{14.029}{-16.025}{15.327}{-16.55}{15.327}{-16.75}{1}{0}
% relationship 
\ertext{25.214}{-14.85}{l}{}\errelarm{25.064}{-14.55}{25.064}{-14.85}{0}{0}\errelarm{25.064}{-14.85}{25.064}{-15.15}{1}{0}
% relationship along
\ertext{17.019}{-16.6}{r}{along}\errelarm{17.169}{-16.3}{17.169}{-16.6}{1}{0}\errelarm{19.052}{-17.9}{19.052}{-17.95}{1}{0}\errelangle{17.169}{-16.6}{17.169}{-16.9}{18.111}{-17.375}{1}{0}\errelangle{18.111}{-17.375}{19.052}{-17.85}{19.052}{-17.9}{1}{0}
% relationship type
\ertext{18.629}{-15.798}{l}{type}\errelarm{18.479}{-15.948}{19.079}{-15.948}{1}{0}\errelarm{26.608}{-3.063}{26.408}{-3.063}{0}{0}\errelangle{19.079}{-15.948}{19.679}{-15.948}{23.779}{-15.948}{1}{0}\errelangle{26.608}{-3.063}{26.808}{-3.063}{27.343}{-3.063}{0}{0}\errelangle{23.779}{-15.948}{27.879}{-15.948}{27.879}{-9.505}{1}{0}\errelangle{27.879}{-9.505}{27.879}{-3.063}{27.343}{-3.063}{0}{0}\ercrowfoot{18.629}{-15.948}{18.479}{-15.798}{18.479}{-15.948}{18.479}{-16.098}{0}
% relationship projectionrel
\ertext{18.679}{-14.97}{l}{projection}\ertext{18.679}{-15.27}{l}{rel}\errelarm{18.229}{-14.67}{18.529}{-14.67}{0}{0}\errelarm{18.413}{-7.95}{18.613}{-7.95}{0}{0}\ertext{18.671}{-11.61}{l}{\textasciitilde /..=../type}\errelangle{18.529}{-14.67}{18.829}{-14.67}{18.521}{-11.31}{0}{0}\errelangle{18.521}{-11.31}{18.213}{-7.95}{18.413}{-7.95}{0}{0}\ercrowfoot{18.379}{-14.67}{18.229}{-14.52}{18.229}{-14.67}{18.229}{-14.82}{0}
% relationship riser2
\ertext{17.96}{-16.65}{l}{riser2}\errelarm{17.81}{-16.05}{17.81}{-16.3}{1}{0}\errelarm{19.626}{-17.9}{19.626}{-17.95}{1}{0}\errelangle{17.81}{-16.3}{17.81}{-16.55}{18.718}{-17.2}{1}{0}\errelangle{18.718}{-17.2}{19.626}{-17.85}{19.626}{-17.9}{1}{0}
% relationship src
\ertext{24.362}{-22.97}{l}{src}\errelarm{24.212}{-22.67}{24.662}{-22.67}{0}{0}\errelarm{26.608}{-2.238}{26.408}{-2.238}{0}{0}\errelangle{24.662}{-22.67}{25.112}{-22.67}{27.512}{-22.67}{0}{0}\errelangle{26.608}{-2.238}{26.808}{-2.238}{28.36}{-2.238}{0}{0}\errelangle{27.512}{-22.67}{29.912}{-22.67}{29.912}{-12.454}{0}{0}\errelangle{29.912}{-12.454}{29.912}{-2.238}{28.36}{-2.238}{0}{0}\ercrowfoot{24.362}{-22.67}{24.212}{-22.52}{24.212}{-22.67}{24.212}{-22.82}{0}
% relationship dest
\ertext{24.362}{-23.56}{l}{dest}\errelarm{24.212}{-23.26}{24.762}{-23.26}{0}{0}\errelarm{26.608}{-2.1}{26.408}{-2.1}{0}{0}\errelangle{24.762}{-23.26}{25.312}{-23.26}{27.712}{-23.26}{0}{0}\errelangle{26.608}{-2.1}{26.808}{-2.1}{28.46}{-2.1}{0}{0}\errelangle{27.712}{-23.26}{30.112}{-23.26}{30.112}{-12.68}{0}{0}\errelangle{30.112}{-12.68}{30.112}{-2.1}{28.46}{-2.1}{0}{0}\ercrowfoot{24.362}{-23.26}{24.212}{-23.11}{24.212}{-23.26}{24.212}{-23.41}{0}\erarc{18.479}{-17.75}{19.729}{-17.55}{22.229}{-17.55}{23.479}{-17.75}
% relationship 
\ertext{21.279}{-21.5}{l}{}\errelarm{21.129}{-21.2}{21.129}{-21.65}{0}{0}\errelarm{21.129}{-21.65}{21.129}{-22.1}{1}{0}\errelseq{21.189}{-21.65}{20.779}{-21.71}{21.479}{-21.77}{21.069}{-21.83}\ercrowfoot{21.129}{-21.95}{20.979}{-22.1}{21.129}{-22.1}{21.279}{-22.1}{0}
% relationship rel
\ertext{22.029}{-22.7}{l}{rel}\errelarm{21.879}{-22.4}{22.029}{-22.4}{1}{0}\errelarm{27.413}{-10.475}{27.213}{-10.475}{0}{0}\errelangle{22.029}{-22.4}{22.179}{-22.4}{25.429}{-22.4}{1}{0}\errelangle{27.413}{-10.475}{27.613}{-10.475}{28.146}{-10.475}{0}{0}\ertext{28.529}{-16.738}{r}{\textasciitilde /..=src}\errelangle{25.429}{-22.4}{28.679}{-22.4}{28.679}{-16.438}{1}{0}\errelangle{28.679}{-16.438}{28.679}{-10.475}{28.146}{-10.475}{0}{0}\ercrowfoot{22.029}{-22.4}{21.879}{-22.25}{21.879}{-22.4}{21.879}{-22.55}{0}\erarc{1.853}{-12.65}{2.578}{-12.45}{4.028}{-12.45}{4.753}{-12.65}
% relationship to
\ertext{1.553}{-14.5}{r}{to}\errelarm{1.703}{-14.2}{1.103}{-14.2}{1}{0}\errelarm{10.261}{-2.925}{10.461}{-2.925}{0}{0}\errelangle{1.103}{-14.2}{0.503}{-14.2}{0.303}{-14.2}{1}{0}\errelangle{10.261}{-2.925}{10.061}{-2.925}{5.082}{-2.925}{0}{0}\ertext{-0.047}{-8.863}{r}{\textasciitilde /\textasciicircum =\textasciicircum }\errelangle{0.303}{-14.2}{0.103}{-14.2}{0.103}{-8.563}{1}{0}\errelangle{0.103}{-8.563}{0.103}{-2.925}{5.082}{-2.925}{0}{0}\ercrowfoot{1.553}{-14.2}{1.703}{-14.05}{1.703}{-14.2}{1.703}{-14.35}{0}
% relationship destattr
\ertext{32.972}{-9.7}{r}{destattr}\errelarm{33.122}{-9.85}{32.672}{-9.85}{1}{0}\errelarm{33.013}{-7.59}{33.213}{-7.59}{0}{0}\errelangle{32.672}{-9.85}{32.222}{-9.85}{32.197}{-9.85}{1}{0}\errelangle{33.013}{-7.59}{32.813}{-7.59}{32.492}{-7.59}{0}{0}\ertext{33.322}{-9.02}{r}{\textasciitilde /..=rel/type}\errelangle{32.197}{-9.85}{32.172}{-9.85}{32.172}{-8.72}{1}{0}\errelangle{32.172}{-8.72}{32.172}{-7.59}{32.492}{-7.59}{0}{0}\ercrowfoot{32.972}{-9.85}{33.122}{-9.7}{33.122}{-9.85}{33.122}{-10}{0}
% relationship attrOfOrigin
\ertext{32.972}{-10.65}{r}{attrOfOrigin}\errelarm{33.122}{-10.35}{32.622}{-10.35}{1}{0}\errelarm{33.013}{-7.35}{33.213}{-7.35}{0}{0}\errelangle{32.622}{-10.35}{32.122}{-10.35}{32.097}{-10.35}{1}{0}\errelangle{33.013}{-7.35}{32.813}{-7.35}{32.442}{-7.35}{0}{0}\ertext{33.222}{-9.15}{r}{\textasciitilde /..=typeOfOrigin}\errelangle{32.097}{-10.35}{32.072}{-10.35}{32.072}{-8.85}{1}{0}\errelangle{32.072}{-8.85}{32.072}{-7.35}{32.442}{-7.35}{0}{0}\ercrowfoot{32.972}{-10.35}{33.122}{-10.2}{33.122}{-10.35}{33.122}{-10.5}{0}
% relationship typeOfOrigin
\ertext{32.972}{-12.15}{r}{typeOfOrigin}\errelarm{33.122}{-11.85}{32.622}{-11.85}{1}{0}\errelarm{26.608}{-1.55}{26.408}{-1.55}{0}{0}\errelangle{32.622}{-11.85}{32.122}{-11.85}{31.622}{-11.85}{1}{0}\errelangle{26.608}{-1.55}{26.808}{-1.55}{28.965}{-1.55}{0}{0}\ertext{32.272}{-7}{r}{\textasciitilde /\textasciicircum =\textasciicircum }\errelangle{31.622}{-11.85}{31.122}{-11.85}{31.122}{-6.7}{1}{0}\errelangle{31.122}{-6.7}{31.122}{-1.55}{28.965}{-1.55}{0}{0}\ercrowfoot{32.972}{-11.85}{33.122}{-11.7}{33.122}{-11.85}{33.122}{-12}{0}
% relationship rel
\ertext{32.972}{-13.15}{r}{rel}\errelarm{33.122}{-12.85}{32.522}{-12.85}{1}{0}\errelarm{26.663}{-8.42}{26.463}{-8.42}{0}{0}\ertext{29.942}{-11.385}{l}{\textasciitilde /..=../..}\errelangle{32.522}{-12.85}{31.922}{-12.85}{29.392}{-10.635}{1}{0}\errelangle{29.392}{-10.635}{26.863}{-8.42}{26.663}{-8.42}{0}{0}\ercrowfoot{32.972}{-12.85}{33.122}{-12.7}{33.122}{-12.85}{33.122}{-13}{0}
% relationship name
\ertext{9.728}{-18.45}{l}{name}\errelarm{9.578}{-18.15}{9.578}{-18.35}{0}{0}\errelarm{8.785}{-20.35}{8.785}{-20.4}{1}{0}\errelangle{9.578}{-18.35}{9.578}{-18.55}{9.181}{-19.425}{0}{0}\errelangle{9.181}{-19.425}{8.785}{-20.3}{8.785}{-20.35}{1}{0}
% relationship position
\ertext{10.728}{-18.45}{l}{position}\errelarm{10.578}{-18.15}{10.578}{-18.35}{0}{0}\errelarm{11.244}{-20.35}{11.244}{-20.4}{1}{0}\errelangle{10.578}{-18.35}{10.578}{-18.55}{10.911}{-19.425}{0}{0}\errelangle{10.911}{-19.425}{11.244}{-20.3}{11.244}{-20.35}{1}{0}\erarc{9.828}{-16.85}{10.203}{-16.65}{10.953}{-16.65}{11.328}{-16.85}
% relationship 
\ertext{15.172}{-19.6}{l}{}\errelarm{15.022}{-19.3}{15.022}{-19.375}{0}{0}\errelarm{15.052}{-16.75}{15.052}{-16.95}{1}{0}\errelangle{15.022}{-19.375}{15.022}{-19.45}{14.622}{-19.45}{0}{0}\errelangle{15.052}{-16.75}{15.052}{-16.55}{14.637}{-16.55}{1}{0}\errelangle{14.622}{-19.45}{14.222}{-19.45}{14.222}{-18}{0}{0}\errelangle{14.222}{-18}{14.222}{-16.55}{14.637}{-16.55}{1}{0}
% relationship 
\ertext{15.783}{-19.6}{l}{}\errelarm{15.633}{-19.3}{15.633}{-19.75}{0}{0}\errelarm{15.633}{-19.75}{15.633}{-20.2}{1}{0}\erarc{14.786}{-16.85}{14.973}{-16.65}{15.348}{-16.65}{15.536}{-16.85}\erarc{7.777}{-20.2}{8.144}{-20}{8.877}{-20}{9.244}{-20.2}\erarc{47.008}{-8.75}{47.383}{-8.55}{48.133}{-8.55}{48.508}{-8.75}
\end{erdiagram}

}
\end{frame}
\begin{frame}{Example -- LCMSMS Data -- Seven Pullback Diagrams}
\scalebox{0.2}{
\begin{erdiagram}{28.45}{55.81075}

\ergrp{0.2}{-3.7}{12.075}{-0.75}{0.2}{1}\ertext{0.15}{-1.05}{r}{module: task}
\eret{0.4}{-3.45}{4.338}{-1.35}{0.2}{1}\ertext{0.794}{-1.7}{l}{task}
\erattr{0.6}{-1.9}{1}{1}{hostnameOfLabsysAppServer}
\erattr{0.6}{-2.2}{1}{1}{tasknumber}
\erattr{0.6}{-2.5}{1}{1}{SOPnumber}
\erattr{0.6}{-2.8}{1}{1}{procedurename}
\erattr{0.6}{-3.1}{1}{1}{location}
\eret{7.338}{-2.55}{11.075}{-1.35}{0.2}{1}\ertext{7.711}{-1.7}{l}{revised\textunderscore labsys\textunderscore task}
\erattr{7.538}{-1.9}{1}{1}{hostnameOfLabsysAppServer}
\erattr{7.538}{-2.2}{1}{1}{tasknumber}
\ergrp{2.088}{-20.6}{10.188}{-5.7}{0.2}{1}\ertext{2.038}{-6}{r}{Documentation only. Not represented in XML nor in rng or ts.}
\eret{2.288}{-8.3}{9.988}{-6.55}{0.2}{1}\ertext{2.428}{-6.9}{l}{sample\textunderscore group}
\erattr{2.488}{-7.1}{1}{0}{alphacode}
\eret{2.538}{-8.05}{5.838}{-7.45}{0.2}{0}\ertext{4.188}{-7.8}{}{test\textunderscore sample\textunderscore group}
\eret{6.338}{-8.05}{9.638}{-7.45}{0.2}{0}\ertext{7.988}{-7.8}{}{shared\textunderscore sample\textunderscore group}
\eret{7.031}{-10.05}{9.744}{-9.15}{0.2}{1}\ertext{7.302}{-9.5}{l}{MSMS\textunderscore component}
\erattr{7.231}{-9.7}{1}{0}{compoundid}
\eret{4.638}{-14.35}{7.638}{-13.15}{0.2}{1}\ertext{4.938}{-13.5}{l}{sample}
\erattr{4.838}{-13.7}{1}{0}{sampleid}
\erattr{4.838}{-14}{1}{1}{isRTreferencepeak}
\eret{4.388}{-17.65}{7.888}{-16.75}{0.2}{1}\ertext{4.738}{-17.1}{l}{required\textunderscore MSMS\textunderscore data}
\erattr{4.588}{-17.3}{1}{0}{compoundid}
\ergrp{12.575}{-26.3}{24.591}{-2.75}{0.2}{1}\ertext{12.525}{-3.05}{r}{chromatogram tower}
\eret{16.575}{-5.75}{19.775}{-3.35}{0.2}{1}\ertext{16.895}{-3.7}{l}{data\textunderscore collection}
\erattr{16.775}{-3.9}{1}{0}{samplelistname}
\erattr{16.775}{-4.2}{1}{1}{instrumentname}
\erattr{16.775}{-4.5}{1}{1}{instrumenttype}
\erattr{16.775}{-4.8}{1}{1}{chromatography}
\erattr{16.775}{-5.1}{1}{1}{collectiontimestamp}
\erattr{16.775}{-5.4}{0}{1}{labsyssubmissionid}
\eret{16.744}{-7.55}{19.606}{-6.65}{0.2}{1}\ertext{17.03}{-7}{l}{sample\textunderscore group(2)}
\erattr{16.944}{-7.2}{1}{1}{methodfilename}
\eret{13.616}{-15.85}{16.734}{-13.15}{0.2}{1}\ertext{13.928}{-13.5}{l}{injection(2)}
\erattr{13.816}{-13.7}{1}{1}{fullsampleid}
\erattr{13.816}{-14}{1}{1}{samplename}
\erattr{13.816}{-14.3}{1}{1}{sampletype}
\erattr{13.816}{-14.6}{1}{1}{rawdatafilename}
\erattr{13.816}{-14.9}{1}{1}{drawerposition}
\erattr{13.816}{-15.2}{1}{1}{wellposition}
\erattr{13.816}{-15.5}{1}{1}{timestamp}
\eret{20.079}{-10.05}{22.271}{-9.15}{0.2}{1}\ertext{21.175}{-9.5}{}{compound(2)}
\eret{20.289}{-11.85}{22.061}{-10.95}{0.2}{1}\ertext{20.466}{-11.3}{l}{trace(2)}
\erattr{20.489}{-11.5}{1}{0}{trace}
\eret{13.959}{-17.35}{16.392}{-16.75}{0.2}{1}\ertext{15.175}{-17.1}{}{component(2)}
\eret{13.425}{-19.75}{16.925}{-18.25}{0.2}{1}\ertext{13.775}{-18.6}{l}{chromatogram}
\erattr{13.625}{-18.8}{1}{1}{extractiontimestamp}
\erattr{13.625}{-19.1}{1}{1}{extractioneventno}
\erattr{13.625}{-19.4}{1}{1}{timeintensitydata}
\eret{13.059}{-23.3}{22.291}{-20.65}{0.2}{1}\ertext{13.271}{-21}{l}{instrument\textunderscore extraction\textunderscore details}
\eret{13.309}{-22.45}{16.785}{-21.25}{0.2}{0}\ertext{13.657}{-21.6}{l}{xevo\textunderscore extraction\textunderscore details}
\erattr{13.509}{-21.8}{1}{1}{xevofunctionno}
\erattr{13.509}{-22.1}{1}{1}{xevocompoundno}
\eret{17.285}{-23.05}{21.541}{-21.25}{0.2}{0}\ertext{17.71}{-21.6}{l}{ab6600\textunderscore extraction\textunderscore details}
\erattr{17.485}{-21.8}{1}{1}{sampleno}
\erattr{17.485}{-22.1}{1}{1}{periodno}
\erattr{17.485}{-22.4}{1}{1}{experimentno}
\erattr{17.485}{-22.7}{1}{1}{numberofdatapoints}
\eret{17.425}{-25.45}{21.401}{-23.95}{0.2}{1}\ertext{17.823}{-24.3}{l}{step\textunderscore size\textunderscore specification}
\erattr{17.625}{-24.5}{1}{0}{startpointno}
\erattr{17.625}{-24.8}{1}{1}{stepsize}
\erattr{17.625}{-25.1}{1}{1}{changepointno}
\ergrp{27.591}{-28.45}{55.561}{-0.75}{0.2}{1}\ertext{27.541}{-1.05}{r}{interpretation tower}
\eret{31.591}{-2.6}{46.927}{-1.2}{0.2}{1}\ertext{33.125}{-1.55}{l}{interpretation\textunderscore session}
\erattr{31.791}{-1.75}{1}{0}{sessionguid}
\eret{29.957}{-4.9}{48.561}{-3.5}{0.2}{1}\ertext{30.069}{-3.85}{l}{interpretation\textunderscore event}
\erattr{30.157}{-4.05}{1}{1}{dateTimeOpened}
\erattr{30.157}{-4.35}{0}{1}{dateTimeSaved}
\erattr{30.157}{-4.65}{0}{1}{dateTimeSubmitted}
\eret{33.957}{-4.65}{38.755}{-3.75}{0.2}{0}\ertext{34.437}{-4.1}{l}{programmed\textunderscore interpretation\textunderscore event}
\erattr{34.157}{-4.3}{1}{1}{programname}
\eret{42.755}{-4.65}{47.411}{-3.75}{0.2}{0}\ertext{43.22}{-4.1}{l}{user\textunderscore interpretation\textunderscore event}
\erattr{42.955}{-4.3}{1}{1}{username}
\eret{33.179}{-8}{39.533}{-6.65}{0.2}{1}\ertext{36.356}{-7}{}{sample\textunderscore group(3)}
\eret{31.389}{-14.5}{34.322}{-12.9}{0.2}{1}\ertext{32.856}{-13.25}{}{injection(3)}
\eret{38.709}{-10.25}{41.002}{-9.15}{0.2}{1}\ertext{39.856}{-9.5}{}{compound(3)}
\eret{38.795}{-12.05}{40.417}{-10.95}{0.2}{1}\ertext{39.606}{-11.3}{}{trace(3)}
\eret{31.389}{-17.5}{34.322}{-16.6}{0.2}{1}\ertext{32.856}{-16.95}{}{component(3)}
\eret{29.179}{-19.85}{35.533}{-18.25}{0.2}{1}\ertext{32.356}{-18.6}{}{chromatogram(2)}
\eret{41.169}{-26.7}{45.996}{-24.95}{0.2}{1}\ertext{41.309}{-25.3}{l}{annotation}
\erattr{41.369}{-25.5}{1}{1}{text}
\eret{41.419}{-26.45}{43.913}{-25.85}{0.2}{0}\ertext{42.666}{-26.2}{}{reject\textunderscore this\textunderscore data}
\eret{44.413}{-26.45}{45.746}{-25.85}{0.2}{0}\ertext{45.08}{-26.2}{}{comment}
\eret{45.57}{-7.05}{47.595}{-6.15}{0.2}{1}\ertext{45.773}{-6.5}{l}{method}
\erattr{45.77}{-6.7}{1}{0}{methodname}
\eret{45.789}{-9.15}{47.376}{-7.95}{0.2}{1}\ertext{45.948}{-8.3}{l}{param}
\erattr{45.989}{-8.5}{1}{1}{name}
\erattr{45.989}{-8.8}{1}{1}{value}
\eret{27.883}{-22.95}{30.828}{-21.45}{0.2}{1}\ertext{28.178}{-21.8}{l}{timepoint}
\erattr{28.083}{-22}{1}{0}{time}
\erattr{28.083}{-22.3}{1}{1}{rawintensity}
\erattr{28.083}{-22.6}{1}{1}{smoothedintensity}
\eret{31.328}{-28.15}{36.715}{-21.45}{0.2}{1}\ertext{31.759}{-21.8}{l}{peak}
\erattr{31.528}{-22}{1}{1}{RT}
\erattr{31.528}{-22.3}{1}{1}{peakArea}
\erattr{31.528}{-22.6}{1}{1}{peakHeight}
\erattr{31.528}{-22.9}{1}{1}{chromatogramNoise}
\erattr{31.528}{-23.2}{1}{1}{startRT}
\erattr{31.528}{-23.5}{1}{1}{endRT}
\erattr{31.528}{-23.8}{1}{1}{startHght}
\erattr{31.528}{-24.1}{1}{1}{endHght}
\erattr{31.528}{-24.4}{1}{1}{peakWidthHalfHeight}
\erattr{31.528}{-24.7}{1}{1}{peakSkew}
\eret{31.578}{-27.4}{33.651}{-26.55}{0.2}{0}\ertext{32.615}{-26.9}{}{selected\textunderscore peak}
\eret{33.751}{-25.6}{35.965}{-25}{0.2}{0}\ertext{34.858}{-25.35}{}{candidate\textunderscore peak}
\eret{37.215}{-24.55}{41.615}{-21.45}{0.2}{1}\ertext{37.655}{-21.8}{l}{status}
\erattr{37.415}{-22}{1}{1}{timestamp}
\eret{0}{-0.2}{55.811}{0.3}{0.2}{1}

% relationship 
\ertext{2.519}{-0.5}{l}{}\errelarm{2.369}{-0.2}{2.369}{-0.775}{0}{0}\errelarm{2.369}{-0.775}{2.369}{-1.35}{1}{0}
% relationship 
\ertext{6.288}{-0.5}{l}{}\errelarm{6.138}{-0.2}{6.138}{-3.375}{0}{0}\errelarm{6.138}{-3.375}{6.138}{-6.55}{1}{0}
% relationship 
\ertext{18.325}{-0.5}{l}{}\errelarm{18.175}{-0.2}{18.175}{-1.775}{0}{0}\errelarm{18.175}{-1.775}{18.175}{-3.35}{1}{0}
% relationship submitto
\ertext{9.356}{-0.5}{l}{submit}\ertext{9.356}{-0.8}{l}{to}\errelarm{9.206}{-0.2}{9.206}{-0.775}{0}{0}\errelarm{9.206}{-0.775}{9.206}{-1.35}{1}{0}
% relationship 
\ertext{39.409}{-0.5}{l}{}\errelarm{39.259}{-0.2}{39.259}{-0.7}{0}{0}\errelarm{39.259}{-0.7}{39.259}{-1.2}{1}{0}
% relationship revisedin
\ertext{11.225}{-2.25}{l}{revised}\ertext{11.225}{-2.55}{l}{in}\errelarm{11.075}{-1.95}{11.675}{-1.95}{0}{0}\errelarm{31.391}{-1.9}{31.591}{-1.9}{0}{0}\ertext{21.583}{-2.225}{r}{\textasciitilde /\textasciicircum =\textasciicircum }\errelangle{11.675}{-1.95}{12.275}{-1.95}{21.733}{-1.925}{0}{0}\errelangle{21.733}{-1.925}{31.191}{-1.9}{31.391}{-1.9}{0}{0}\ercrowfoot{11.225}{-1.95}{11.075}{-1.8}{11.075}{-1.95}{11.075}{-2.1}{0}
% relationship 
\ertext{5.004}{-8.6}{l}{}\ertext{6.288}{-13}{l}{..}\errelarm{4.854}{-8.3}{4.854}{-8.375}{0}{0}\errelarm{6.138}{-12.862}{6.138}{-13.15}{1}{0}\errelangle{4.854}{-8.375}{4.854}{-8.45}{5.496}{-10.513}{0}{0}\errelangle{5.496}{-10.513}{6.138}{-12.575}{6.138}{-12.862}{1}{0}\errelseq{6.198}{-12.625}{5.788}{-12.685}{6.488}{-12.745}{6.078}{-12.805}\eridcomprel{6.0375000000000005}{6.2375}{-12.899999999999999}
% relationship 
\ertext{7.443}{-8.6}{l}{}\ertext{8.538}{-9}{l}{..}\errelarm{7.293}{-8.3}{7.293}{-8.375}{0}{0}\errelarm{8.388}{-8.937}{8.388}{-9.15}{1}{0}\errelangle{7.293}{-8.375}{7.293}{-8.45}{7.84}{-8.587}{0}{0}\errelangle{7.84}{-8.587}{8.388}{-8.725}{8.388}{-8.937}{1}{0}\eridcomprel{8.287500000000001}{8.4875}{-8.899999999999999}
% relationship 
\ertext{6.288}{-14.65}{l}{}\ertext{6.288}{-16.6}{l}{..}\errelarm{6.138}{-14.35}{6.138}{-15.55}{0}{0}\errelarm{6.138}{-15.55}{6.138}{-16.75}{1}{0}\eridcomprel{6.0375000000000005}{6.2375}{-16.5}
% relationship 
\ertext{18.325}{-6.05}{l}{}\ertext{18.325}{-6.5}{l}{..}\errelarm{18.175}{-5.75}{18.175}{-6.2}{0}{0}\errelarm{18.175}{-6.2}{18.175}{-6.65}{1}{0}\errelseq{18.235}{-6.125}{17.825}{-6.185}{18.525}{-6.245}{18.115}{-6.305}\eridcomprel{18.075}{18.275000000000002}{-6.3999999999999995}
% relationship 
\ertext{17.848}{-7.85}{l}{}\ertext{15.325}{-13}{l}{..}\errelarm{17.698}{-7.55}{17.698}{-7.625}{0}{0}\errelarm{15.175}{-12.4}{15.175}{-13.15}{1}{0}\errelangle{17.698}{-7.625}{17.698}{-7.7}{16.436}{-9.675}{0}{0}\errelangle{16.436}{-9.675}{15.175}{-11.65}{15.175}{-12.4}{1}{0}\errelseq{15.235}{-12.625}{14.825}{-12.685}{15.525}{-12.745}{15.115}{-12.805}\eridcomprel{15.075000000000001}{15.275}{-12.899999999999999}
% relationship 
\ertext{18.802}{-7.85}{l}{}\ertext{21.325}{-9}{l}{..}\errelarm{18.652}{-7.55}{18.652}{-7.625}{0}{0}\errelarm{21.175}{-8.862}{21.175}{-9.15}{1}{0}\errelangle{18.652}{-7.625}{18.652}{-7.7}{19.914}{-8.137}{0}{0}\errelangle{19.914}{-8.137}{21.175}{-8.575}{21.175}{-8.862}{1}{0}\errelseq{21.235}{-8.625}{20.825}{-8.685}{21.525}{-8.745}{21.115}{-8.805}\eridcomprel{21.075000000000003}{21.275000000000006}{-8.899999999999999}
% relationship group
\ertext{16.594}{-7.25}{r}{group}\errelarm{16.744}{-6.95}{16.444}{-6.95}{0}{0}\errelarm{10.188}{-7.425}{9.988}{-7.425}{0}{0}\errelangle{16.444}{-6.95}{16.144}{-6.95}{13.266}{-7.188}{0}{0}\errelangle{13.266}{-7.188}{10.388}{-7.425}{10.188}{-7.425}{0}{0}\ercrowfoot{16.594}{-6.95}{16.744}{-6.8}{16.744}{-6.95}{16.744}{-7.1}{0}\eridrefrel{16.493750000000002}{-6.85}{-7.049999999999999}
% relationship RTreference
\ertext{19.756}{-7.67}{l}{RT}\ertext{19.756}{-7.97}{l}{reference}\errelarm{19.606}{-7.37}{20.206}{-7.37}{0}{0}\errelarm{16.934}{-14.77}{16.734}{-14.77}{0}{0}\errelangle{20.206}{-7.37}{20.806}{-7.37}{22.306}{-7.37}{0}{0}\errelangle{16.934}{-14.77}{17.134}{-14.77}{20.47}{-14.77}{0}{0}\ertext{23.956}{-9.87}{l}{\textasciitilde /..=.}\errelangle{22.306}{-7.37}{23.806}{-7.37}{23.806}{-11.07}{0}{0}\errelangle{23.806}{-11.07}{23.806}{-14.77}{20.47}{-14.77}{0}{0}\ercrowfoot{19.756}{-7.37}{19.606}{-7.22}{19.606}{-7.37}{19.606}{-7.52}{0}
% relationship 
\ertext{15.325}{-16.15}{l}{}\ertext{15.325}{-16.6}{l}{..}\errelarm{15.175}{-15.85}{15.175}{-16.3}{0}{0}\errelarm{15.175}{-16.3}{15.175}{-16.75}{1}{0}\errelseq{15.235}{-16.225}{14.825}{-16.285}{15.525}{-16.345}{15.115}{-16.405}\eridcomprel{15.075000000000001}{15.275}{-16.5}
% relationship subject
\ertext{13.466}{-14.8}{r}{subject}\errelarm{13.616}{-14.5}{13.466}{-14.5}{0}{0}\errelarm{7.838}{-13.75}{7.638}{-13.75}{0}{0}\ertext{10.827}{-14.425}{l}{\textasciitilde /..=../group}\errelangle{13.466}{-14.5}{13.316}{-14.5}{10.677}{-14.125}{0}{0}\errelangle{10.677}{-14.125}{8.038}{-13.75}{7.838}{-13.75}{0}{0}\ercrowfoot{13.466}{-14.5}{13.616}{-14.35}{13.616}{-14.5}{13.616}{-14.65}{0}\eridrefrel{13.36625}{-14.399999999999999}{-14.599999999999998}
% relationship 
\ertext{21.325}{-10.35}{l}{}\ertext{21.325}{-10.8}{l}{..}\errelarm{21.175}{-10.05}{21.175}{-10.5}{0}{0}\errelarm{21.175}{-10.5}{21.175}{-10.95}{1}{0}\eridcomprel{21.075000000000003}{21.275000000000006}{-10.7}
% relationship subject
\ertext{19.929}{-9.75}{r}{subject}\errelarm{20.079}{-9.45}{19.779}{-9.45}{0}{0}\errelarm{9.944}{-9.6}{9.744}{-9.6}{0}{0}\ertext{14.661}{-9.825}{r}{\textasciitilde /..=../group}\errelangle{19.779}{-9.45}{19.479}{-9.45}{14.811}{-9.525}{0}{0}\errelangle{14.811}{-9.525}{10.144}{-9.6}{9.944}{-9.6}{0}{0}\ercrowfoot{19.929}{-9.45}{20.079}{-9.3}{20.079}{-9.45}{20.079}{-9.6}{0}\eridrefrel{19.828625000000002}{-9.35}{-9.549999999999999}
% relationship IS
\ertext{22.421}{-9.9}{l}{IS}\errelangle{22.271}{-9.6}{22.271}{-9.6}{22.571}{-9.6}{0}{0}\errelangle{22.271}{-9.87}{22.271}{-9.87}{22.571}{-9.87}{0}{0}\ertext{23.121}{-10.035}{l}{\textasciitilde /..=..}\errelangle{22.571}{-9.6}{22.871}{-9.6}{22.871}{-9.735}{0}{0}\errelangle{22.871}{-9.735}{22.871}{-9.87}{22.571}{-9.87}{0}{0}\ercrowfoot{22.421}{-9.6}{22.271}{-9.45}{22.271}{-9.6}{22.271}{-9.75}{0}
% relationship 
\ertext{15.325}{-17.65}{l}{}\ertext{15.325}{-18.1}{l}{..}\errelarm{15.175}{-17.35}{15.175}{-17.8}{0}{0}\errelarm{15.175}{-17.8}{15.175}{-18.25}{1}{0}\errelseq{15.235}{-17.725}{14.825}{-17.785}{15.525}{-17.845}{15.115}{-17.905}\eridcomprel{15.075000000000001}{15.275}{-18.000000000000004}
% relationship monitored
\ertext{16.542}{-17.17}{l}{monitored}\errelarm{16.392}{-16.87}{16.992}{-16.87}{0}{0}\errelarm{19.879}{-9.87}{20.079}{-9.87}{0}{0}\ertext{19.235}{-13.67}{r}{\textasciitilde /..=../..}\errelangle{16.992}{-16.87}{17.592}{-16.87}{18.635}{-13.37}{0}{0}\errelangle{18.635}{-13.37}{19.679}{-9.87}{19.879}{-9.87}{0}{0}\ercrowfoot{16.542}{-16.87}{16.392}{-16.72}{16.392}{-16.87}{16.392}{-17.02}{0}\eridrefrel{16.6415}{-16.77}{-16.970000000000002}
% relationship 
\ertext{15.325}{-20.05}{l}{}\errelarm{15.175}{-19.75}{15.175}{-19.825}{0}{0}\errelarm{17.675}{-20.438}{17.675}{-20.65}{1}{0}\errelangle{15.175}{-19.825}{15.175}{-19.9}{16.425}{-20.063}{0}{0}\errelangle{16.425}{-20.063}{17.675}{-20.225}{17.675}{-20.438}{1}{0}\eridcomprel{17.575}{17.775000000000002}{-20.400000000000006}
% relationship extracted
\ertext{17.075}{-18.925}{l}{extracted}\errelarm{16.925}{-18.625}{17.975}{-18.625}{0}{0}\errelarm{20.089}{-11.4}{20.289}{-11.4}{0}{0}\ertext{21.057}{-15.313}{r}{\textasciitilde /..=../monitored}\errelangle{17.975}{-18.625}{19.025}{-18.625}{19.457}{-15.013}{0}{0}\errelangle{19.457}{-15.013}{19.889}{-11.4}{20.089}{-11.4}{0}{0}\ercrowfoot{17.075}{-18.625}{16.925}{-18.475}{16.925}{-18.625}{16.925}{-18.775}{0}\eridrefrel{17.175}{-18.525000000000002}{-18.725000000000005}
% relationship 
\ertext{19.563}{-23.35}{l}{}\errelarm{19.413}{-23.05}{19.413}{-23.5}{0}{0}\errelarm{19.413}{-23.5}{19.413}{-23.95}{1}{0}\eridcomprel{19.312875000000002}{19.512875000000005}{-23.70000000000001}
% relationship 
\ertext{39.409}{-2.9}{l}{}\ertext{39.409}{-3.35}{l}{..}\errelarm{39.259}{-2.6}{39.259}{-3.05}{0}{0}\errelarm{39.259}{-3.05}{39.259}{-3.5}{1}{0}\eridcomprel{39.15875}{39.35875}{-3.2499999999999996}
% relationship 
\ertext{34.758}{-5.2}{l}{}\ertext{36.506}{-6.5}{l}{..}\errelarm{34.608}{-4.9}{34.608}{-4.975}{0}{0}\errelarm{36.356}{-6.437}{36.356}{-6.65}{1}{0}\errelangle{34.608}{-4.975}{34.608}{-5.05}{35.482}{-5.637}{0}{0}\errelangle{35.482}{-5.637}{36.356}{-6.225}{36.356}{-6.437}{1}{0}\eridcomprel{36.205625000000005}{36.505625}{-6.35}\ercrowfoot{36.356}{-6.5}{36.206}{-6.65}{36.356}{-6.65}{36.506}{-6.65}{0}\ercrowfoot{36.356}{-6.5}{36.206}{-6.35}{36.356}{-6.35}{36.506}{-6.35}{0}
% relationship 
\ertext{39.409}{-5.2}{l}{}\ertext{46.733}{-6}{l}{..}\errelarm{39.259}{-4.9}{39.259}{-4.975}{0}{0}\errelarm{46.583}{-6.1}{46.583}{-6.15}{1}{0}\errelangle{39.259}{-4.975}{39.259}{-5.05}{42.921}{-5.55}{0}{0}\errelangle{42.921}{-5.55}{46.583}{-6.05}{46.583}{-6.1}{1}{0}
% relationship 
\ertext{44.06}{-5.2}{l}{}\errelarm{43.91}{-4.9}{43.91}{-4.975}{0}{0}\errelarm{45.272}{-24.9}{45.272}{-24.95}{1}{0}\errelangle{43.91}{-4.975}{43.91}{-5.05}{44.591}{-14.95}{0}{0}\errelangle{44.591}{-14.95}{45.272}{-24.85}{45.272}{-24.9}{1}{0}
% relationship baseevent
\ertext{48.711}{-4.92}{l}{base}\ertext{48.711}{-5.22}{l}{event}\errelangle{48.561}{-4.62}{48.561}{-4.62}{48.861}{-4.62}{0}{0}\errelangle{48.561}{-4.34}{48.561}{-4.34}{48.861}{-4.34}{0}{0}\ertext{50.311}{-5.08}{r}{session}\ertext{50.311}{-4.78}{r}{\textasciitilde /..=base}\errelangle{48.861}{-4.62}{49.161}{-4.62}{49.161}{-4.48}{0}{0}\errelangle{49.161}{-4.48}{49.161}{-4.34}{48.861}{-4.34}{0}{0}\ercrowfoot{48.711}{-4.62}{48.561}{-4.47}{48.561}{-4.62}{48.561}{-4.77}{0}
% relationship basesession
\ertext{48.711}{-4.08}{l}{base}\ertext{48.711}{-4.38}{l}{session}\errelarm{48.561}{-3.78}{49.161}{-3.78}{0}{0}\errelarm{47.127}{-2.32}{46.927}{-2.32}{0}{0}\errelangle{49.161}{-3.78}{49.761}{-3.78}{49.811}{-3.78}{0}{0}\errelangle{47.127}{-2.32}{47.326}{-2.32}{48.594}{-2.32}{0}{0}\ertext{49.561}{-3.35}{r}{\textasciitilde /\textasciicircum =\textasciicircum }\errelangle{49.811}{-3.78}{49.861}{-3.78}{49.861}{-3.05}{0}{0}\errelangle{49.861}{-3.05}{49.861}{-2.32}{48.594}{-2.32}{0}{0}\ercrowfoot{48.711}{-3.78}{48.561}{-3.63}{48.561}{-3.78}{48.561}{-3.93}{0}
% relationship subject
\ertext{29.807}{-4.85}{r}{subject}\errelarm{29.957}{-4.55}{29.357}{-4.55}{0}{0}\errelarm{19.975}{-4.55}{19.775}{-4.55}{0}{0}\ertext{24.616}{-4.85}{l}{\textasciitilde /\textasciicircum =\textasciicircum }\errelarm{29.357}{-4.55}{24.466}{-4.55}{0}{0}\errelarm{24.466}{-4.55}{19.975}{-4.55}{0}{0}\ercrowfoot{29.807}{-4.55}{29.957}{-4.4}{29.957}{-4.55}{29.957}{-4.7}{0}\eridrefrel{29.65675}{-4.399999999999999}{-4.699999999999999}\ercrowfoot{29.807}{-4.55}{29.957}{-4.4}{29.957}{-4.55}{29.957}{-4.7}{0}\ercrowfoot{29.807}{-4.55}{29.657}{-4.4}{29.657}{-4.55}{29.657}{-4.7}{0}
% relationship 
\ertext{34.917}{-8.3}{l}{}\ertext{33.006}{-12.75}{l}{..}\errelarm{34.767}{-8}{34.767}{-8.075}{0}{0}\errelarm{32.856}{-12.15}{32.856}{-12.9}{1}{0}\errelangle{34.767}{-8.075}{34.767}{-8.15}{33.811}{-9.775}{0}{0}\errelangle{33.811}{-9.775}{32.856}{-11.4}{32.856}{-12.15}{1}{0}\eridcomprel{32.705625000000005}{33.005625}{-12.599999999999998}\ercrowfoot{32.856}{-12.75}{32.706}{-12.9}{32.856}{-12.9}{33.006}{-12.9}{0}\ercrowfoot{32.856}{-12.75}{32.706}{-12.6}{32.856}{-12.6}{33.006}{-12.6}{0}
% relationship 
\ertext{37.776}{-8.3}{l}{}\ertext{40.006}{-9}{l}{..}\errelarm{37.626}{-8}{37.626}{-8.075}{0}{0}\errelarm{39.856}{-8.937}{39.856}{-9.15}{1}{0}\errelangle{37.626}{-8.075}{37.626}{-8.15}{38.741}{-8.438}{0}{0}\errelangle{38.741}{-8.438}{39.856}{-8.725}{39.856}{-8.937}{1}{0}\eridcomprel{39.705625000000005}{40.005625}{-8.849999999999998}\ercrowfoot{39.856}{-9}{39.706}{-9.15}{39.856}{-9.15}{40.006}{-9.15}{0}\ercrowfoot{39.856}{-9}{39.706}{-8.85}{39.856}{-8.85}{40.006}{-8.85}{0}
% relationship 
\ertext{39.047}{-8.3}{l}{}\errelarm{38.897}{-8}{38.897}{-8.25}{0}{0}\errelarm{44.427}{-24.9}{44.427}{-24.95}{1}{0}\errelangle{38.897}{-8.25}{38.897}{-8.5}{41.397}{-8.5}{0}{0}\errelangle{44.427}{-24.9}{44.427}{-24.85}{44.427}{-24.85}{1}{0}\errelangle{41.397}{-8.5}{43.897}{-8.5}{43.897}{-9.875}{0}{0}\errelangle{44.427}{-24.85}{44.427}{-24.85}{44.427}{-18.05}{1}{0}\errelangle{43.897}{-9.875}{43.897}{-11.25}{44.162}{-11.25}{0}{0}\errelangle{44.162}{-11.25}{44.427}{-11.25}{44.427}{-18.05}{1}{0}
% relationship subject
\ertext{33.029}{-7.625}{r}{subject}\errelarm{33.179}{-7.325}{32.579}{-7.325}{0}{0}\errelarm{19.806}{-7.1}{19.606}{-7.1}{0}{0}\ertext{25.893}{-7.612}{l}{\textasciitilde /..=../subject}\errelangle{32.579}{-7.325}{31.979}{-7.325}{25.993}{-7.212}{0}{0}\errelangle{25.993}{-7.212}{20.006}{-7.1}{19.806}{-7.1}{0}{0}\ercrowfoot{33.029}{-7.325}{33.179}{-7.175}{33.179}{-7.325}{33.179}{-7.475}{0}\eridrefrel{32.878750000000004}{-7.174999999999999}{-7.475}\ercrowfoot{33.029}{-7.325}{33.179}{-7.175}{33.179}{-7.325}{33.179}{-7.475}{0}\ercrowfoot{33.029}{-7.325}{32.879}{-7.175}{32.879}{-7.325}{32.879}{-7.475}{0}
% relationship RTreference
\ertext{39.683}{-8.098}{l}{RT}\ertext{39.683}{-8.398}{l}{reference}\errelarm{39.533}{-7.797}{40.133}{-7.797}{0}{0}\errelarm{34.522}{-14.18}{34.322}{-14.18}{0}{0}\errelangle{40.133}{-7.797}{40.733}{-7.797}{41.733}{-7.797}{0}{0}\errelangle{34.522}{-14.18}{34.722}{-14.18}{38.727}{-14.18}{0}{0}\ertext{42.883}{-9.789}{l}{\textasciitilde /..=.}\errelangle{41.733}{-7.797}{42.733}{-7.797}{42.733}{-10.989}{0}{0}\errelangle{42.733}{-10.989}{42.733}{-14.18}{38.727}{-14.18}{0}{0}\ercrowfoot{39.683}{-7.797}{39.533}{-7.647}{39.533}{-7.797}{39.533}{-7.948}{0}
% relationship basedon
\ertext{39.683}{-7.49}{l}{based}\ertext{39.683}{-7.79}{l}{on}\errelangle{39.533}{-7.19}{39.533}{-7.19}{39.832}{-7.19}{0}{0}\errelangle{39.533}{-6.852}{39.533}{-6.852}{39.832}{-6.852}{0}{0}\ertext{41.383}{-7.621}{r}{event}\ertext{41.383}{-7.321}{r}{\textasciitilde /..=../base}\errelangle{39.832}{-7.19}{40.133}{-7.19}{40.133}{-7.021}{0}{0}\errelangle{40.133}{-7.021}{40.133}{-6.852}{39.832}{-6.852}{0}{0}\ercrowfoot{39.683}{-7.19}{39.533}{-7.04}{39.533}{-7.19}{39.533}{-7.34}{0}
% relationship 
\ertext{33.006}{-14.8}{l}{}\ertext{33.006}{-16.45}{l}{..}\errelarm{32.856}{-14.5}{32.856}{-15.55}{0}{0}\errelarm{32.856}{-15.55}{32.856}{-16.6}{1}{0}\eridcomprel{32.705625000000005}{33.005625}{-16.3}\ercrowfoot{32.856}{-16.45}{32.706}{-16.6}{32.856}{-16.6}{33.006}{-16.6}{0}\ercrowfoot{32.856}{-16.45}{32.706}{-16.3}{32.856}{-16.3}{33.006}{-16.3}{0}
% relationship subject
\ertext{31.239}{-14}{r}{subject}\errelarm{31.389}{-13.7}{30.789}{-13.7}{0}{0}\errelarm{16.934}{-14.5}{16.734}{-14.5}{0}{0}\ertext{23.511}{-14.4}{r}{\textasciitilde /..=../subject}\errelangle{30.789}{-13.7}{30.189}{-13.7}{23.661}{-14.1}{0}{0}\errelangle{23.661}{-14.1}{17.134}{-14.5}{16.934}{-14.5}{0}{0}\ercrowfoot{31.239}{-13.7}{31.389}{-13.55}{31.389}{-13.7}{31.389}{-13.85}{0}\eridrefrel{31.089125000000003}{-13.549999999999999}{-13.85}\ercrowfoot{31.239}{-13.7}{31.389}{-13.55}{31.389}{-13.7}{31.389}{-13.85}{0}\ercrowfoot{31.239}{-13.7}{31.089}{-13.55}{31.089}{-13.7}{31.089}{-13.85}{0}
% relationship basedon
\ertext{34.472}{-13.84}{l}{based}\ertext{34.472}{-14.14}{l}{on}\errelangle{34.322}{-13.54}{34.322}{-13.54}{34.622}{-13.54}{0}{0}\errelangle{34.322}{-13.14}{34.322}{-13.14}{34.622}{-13.14}{0}{0}\ertext{36.172}{-13.94}{r}{on}\ertext{36.172}{-13.64}{r}{\textasciitilde /..=../based}\errelangle{34.622}{-13.54}{34.922}{-13.54}{34.922}{-13.34}{0}{0}\errelangle{34.922}{-13.34}{34.922}{-13.14}{34.622}{-13.14}{0}{0}\ercrowfoot{34.472}{-13.54}{34.322}{-13.39}{34.322}{-13.54}{34.322}{-13.69}{0}
% relationship 
\ertext{39.624}{-10.55}{l}{}\ertext{39.756}{-10.8}{l}{..}\errelarm{39.474}{-10.25}{39.474}{-10.325}{0}{0}\errelarm{39.606}{-10.737}{39.606}{-10.95}{1}{0}\errelangle{39.474}{-10.325}{39.474}{-10.4}{39.54}{-10.462}{0}{0}\errelangle{39.54}{-10.462}{39.606}{-10.525}{39.606}{-10.737}{1}{0}\eridcomprel{39.455625000000005}{39.755625}{-10.649999999999999}\ercrowfoot{39.606}{-10.8}{39.456}{-10.95}{39.606}{-10.95}{39.756}{-10.95}{0}\ercrowfoot{39.606}{-10.8}{39.456}{-10.65}{39.606}{-10.65}{39.756}{-10.65}{0}
% relationship 
\ertext{40.923}{-10.55}{l}{}\errelarm{40.773}{-10.25}{40.773}{-10.325}{0}{0}\errelarm{43.583}{-24.9}{43.583}{-24.95}{1}{0}\errelangle{40.773}{-10.325}{40.773}{-10.4}{40.773}{-12.4}{0}{0}\errelangle{43.583}{-24.9}{43.583}{-24.85}{43.583}{-19.625}{1}{0}\errelangle{40.773}{-12.4}{40.773}{-14.4}{42.178}{-14.4}{0}{0}\errelangle{42.178}{-14.4}{43.583}{-14.4}{43.583}{-19.625}{1}{0}
% relationship subject
\ertext{38.559}{-9.67}{r}{subject}\errelarm{38.709}{-9.37}{38.109}{-9.37}{0}{0}\errelarm{22.471}{-9.6}{22.271}{-9.6}{0}{0}\ertext{31.44}{-9.785}{r}{\textasciitilde /..=../subject}\errelangle{38.109}{-9.37}{37.509}{-9.37}{30.09}{-9.485}{0}{0}\errelangle{30.09}{-9.485}{22.671}{-9.6}{22.471}{-9.6}{0}{0}\ercrowfoot{38.559}{-9.37}{38.709}{-9.22}{38.709}{-9.37}{38.709}{-9.52}{0}\eridrefrel{38.40925000000001}{-9.219999999999999}{-9.52}\ercrowfoot{38.559}{-9.37}{38.709}{-9.22}{38.709}{-9.37}{38.709}{-9.52}{0}\ercrowfoot{38.559}{-9.37}{38.409}{-9.22}{38.409}{-9.37}{38.409}{-9.52}{0}
% relationship IS
\ertext{41.152}{-9.67}{l}{IS}\errelangle{41.002}{-9.37}{41.002}{-9.37}{41.302}{-9.37}{0}{0}\errelangle{41.002}{-9.7}{41.002}{-9.7}{41.302}{-9.7}{0}{0}\ertext{41.852}{-9.835}{l}{\textasciitilde /..=..}\errelangle{41.302}{-9.37}{41.602}{-9.37}{41.602}{-9.535}{0}{0}\errelangle{41.602}{-9.535}{41.602}{-9.7}{41.302}{-9.7}{0}{0}\ercrowfoot{41.152}{-9.37}{41.002}{-9.22}{41.002}{-9.37}{41.002}{-9.52}{0}
% relationship selected
\ertext{41.152}{-10.33}{l}{selected}\errelarm{41.002}{-10.03}{41.102}{-10.03}{0}{0}\errelarm{40.617}{-11.28}{40.417}{-11.28}{0}{0}\errelangle{41.102}{-10.03}{41.202}{-10.03}{41.252}{-10.03}{0}{0}\errelangle{40.617}{-11.28}{40.817}{-11.28}{41.059}{-11.28}{0}{0}\ertext{41.552}{-11.255}{l}{\textasciitilde /..=.}\errelangle{41.252}{-10.03}{41.302}{-10.03}{41.302}{-10.655}{0}{0}\errelangle{41.302}{-10.655}{41.302}{-11.28}{41.059}{-11.28}{0}{0}\ercrowfoot{41.152}{-10.03}{41.002}{-9.88}{41.002}{-10.03}{41.002}{-10.18}{0}
% relationship subject
\ertext{38.645}{-11.47}{r}{subject}\errelarm{38.795}{-11.17}{38.195}{-11.17}{0}{0}\errelarm{22.261}{-11.4}{22.061}{-11.4}{0}{0}\ertext{31.378}{-11.585}{r}{\textasciitilde /..=../subject}\errelangle{38.195}{-11.17}{37.595}{-11.17}{30.028}{-11.285}{0}{0}\errelangle{30.028}{-11.285}{22.461}{-11.4}{22.261}{-11.4}{0}{0}\ercrowfoot{38.645}{-11.17}{38.795}{-11.02}{38.795}{-11.17}{38.795}{-11.32}{0}\eridrefrel{38.494625000000006}{-11.02}{-11.32}\ercrowfoot{38.645}{-11.17}{38.795}{-11.02}{38.795}{-11.17}{38.795}{-11.32}{0}\ercrowfoot{38.645}{-11.17}{38.495}{-11.02}{38.495}{-11.17}{38.495}{-11.32}{0}
% relationship 
\ertext{32.517}{-17.8}{l}{}\ertext{32.506}{-18.1}{l}{..}\errelarm{32.367}{-17.5}{32.367}{-17.575}{0}{0}\errelarm{32.356}{-18.038}{32.356}{-18.25}{1}{0}\errelangle{32.367}{-17.575}{32.367}{-17.65}{32.361}{-17.738}{0}{0}\errelangle{32.361}{-17.738}{32.356}{-17.825}{32.356}{-18.038}{1}{0}\eridcomprel{32.205625000000005}{32.505625}{-17.950000000000003}\ercrowfoot{32.356}{-18.1}{32.206}{-18.25}{32.356}{-18.25}{32.506}{-18.25}{0}\ercrowfoot{32.356}{-18.1}{32.206}{-17.95}{32.356}{-17.95}{32.506}{-17.95}{0}
% relationship 
\ertext{34.179}{-17.8}{l}{}\errelarm{34.029}{-17.5}{34.029}{-17.575}{0}{0}\errelarm{42.738}{-24.9}{42.738}{-24.95}{1}{0}\errelangle{34.029}{-17.575}{34.029}{-17.65}{34.029}{-17.775}{0}{0}\errelangle{42.738}{-24.9}{42.738}{-24.85}{42.738}{-21.375}{1}{0}\errelangle{34.029}{-17.775}{34.029}{-17.9}{38.383}{-17.9}{0}{0}\errelangle{38.383}{-17.9}{42.738}{-17.9}{42.738}{-21.375}{1}{0}
% relationship subject
\ertext{31.239}{-17.35}{r}{subject}\ertext{24.04}{-17.35}{l}{\textasciitilde /..=../subject}\errelarm{31.389}{-17.05}{23.89}{-17.05}{0}{0}\errelarm{23.89}{-17.05}{16.392}{-17.05}{0}{0}\ercrowfoot{31.239}{-17.05}{31.389}{-16.9}{31.389}{-17.05}{31.389}{-17.2}{0}\eridrefrel{31.089125000000003}{-16.900000000000002}{-17.2}\ercrowfoot{31.239}{-17.05}{31.389}{-16.9}{31.389}{-17.05}{31.389}{-17.2}{0}\ercrowfoot{31.239}{-17.05}{31.089}{-16.9}{31.089}{-17.05}{31.089}{-17.2}{0}
% relationship monitored
\ertext{34.472}{-17.08}{l}{monitored}\errelarm{34.322}{-16.78}{34.922}{-16.78}{0}{0}\errelarm{38.509}{-10.03}{38.709}{-10.03}{0}{0}\ertext{37.516}{-13.705}{r}{\textasciitilde /..=../..}\errelangle{34.922}{-16.78}{35.522}{-16.78}{36.916}{-13.405}{0}{0}\errelangle{36.916}{-13.405}{38.309}{-10.03}{38.509}{-10.03}{0}{0}\ercrowfoot{34.472}{-16.78}{34.322}{-16.63}{34.322}{-16.78}{34.322}{-16.93}{0}
% relationship basedon
\ertext{34.472}{-17.575}{l}{based}\ertext{34.472}{-17.875}{l}{on}\errelangle{34.322}{-17.275}{34.322}{-17.275}{34.622}{-17.275}{0}{0}\errelangle{34.322}{-17.05}{34.322}{-17.05}{34.622}{-17.05}{0}{0}\ertext{36.172}{-17.763}{r}{on}\ertext{36.172}{-17.463}{r}{\textasciitilde /..=../based}\errelangle{34.622}{-17.275}{34.922}{-17.275}{34.922}{-17.163}{0}{0}\errelangle{34.922}{-17.163}{34.922}{-17.05}{34.622}{-17.05}{0}{0}\ercrowfoot{34.472}{-17.275}{34.322}{-17.125}{34.322}{-17.275}{34.322}{-17.425}{0}
% relationship timeseries
\ertext{29.964}{-20.15}{l}{time}\ertext{29.964}{-20.45}{l}{series}\ertext{29.506}{-21.3}{l}{..}\errelarm{29.814}{-19.85}{29.814}{-20.05}{0}{0}\errelarm{29.356}{-21.4}{29.356}{-21.45}{1}{0}\errelangle{29.814}{-20.05}{29.814}{-20.25}{29.585}{-20.8}{0}{0}\errelangle{29.585}{-20.8}{29.356}{-21.35}{29.356}{-21.4}{1}{0}
% relationship 
\ertext{32.506}{-20.15}{l}{}\ertext{32.765}{-26.4}{l}{..}\errelarm{32.356}{-19.85}{32.356}{-19.925}{0}{0}\errelarm{32.615}{-26.5}{32.615}{-26.55}{1}{0}\errelangle{32.356}{-19.925}{32.356}{-20}{32.485}{-23.225}{0}{0}\errelangle{32.485}{-23.225}{32.615}{-26.45}{32.615}{-26.5}{1}{0}
% relationship 
\ertext{33.141}{-20.15}{l}{}\ertext{35.008}{-24.85}{l}{..}\errelarm{32.991}{-19.85}{32.991}{-19.925}{0}{0}\errelarm{34.858}{-24.95}{34.858}{-25}{1}{0}\errelangle{32.991}{-19.925}{32.991}{-20}{33.925}{-22.45}{0}{0}\errelangle{33.925}{-22.45}{34.858}{-24.9}{34.858}{-24.95}{1}{0}
% relationship 
\ertext{34.094}{-20.15}{l}{}\ertext{39.565}{-21.3}{l}{..}\errelarm{33.944}{-19.85}{33.944}{-20.1}{0}{0}\errelarm{39.415}{-21.4}{39.415}{-21.45}{1}{0}\errelangle{33.944}{-20.1}{33.944}{-20.35}{36.679}{-20.85}{0}{0}\errelangle{36.679}{-20.85}{39.415}{-21.35}{39.415}{-21.4}{1}{0}
% relationship 
\ertext{35.047}{-20.15}{l}{}\errelarm{34.897}{-19.85}{34.897}{-20.1}{0}{0}\errelarm{41.893}{-24.9}{41.893}{-24.95}{1}{0}\errelangle{34.897}{-20.1}{34.897}{-20.35}{34.897}{-20.4}{0}{0}\errelangle{41.893}{-24.9}{41.893}{-24.85}{41.893}{-22.65}{1}{0}\errelangle{34.897}{-20.4}{34.897}{-20.45}{38.395}{-20.45}{0}{0}\errelangle{38.395}{-20.45}{41.893}{-20.45}{41.893}{-22.65}{1}{0}
% relationship subject
\ertext{29.029}{-18.95}{r}{subject}\errelarm{29.179}{-18.65}{28.579}{-18.65}{0}{0}\errelarm{17.125}{-19}{16.925}{-19}{0}{0}\ertext{22.502}{-19.125}{r}{\textasciitilde /..=../subject}\errelangle{28.579}{-18.65}{27.979}{-18.65}{22.652}{-18.825}{0}{0}\errelangle{22.652}{-18.825}{17.325}{-19}{17.125}{-19}{0}{0}\ercrowfoot{29.029}{-18.65}{29.179}{-18.5}{29.179}{-18.65}{29.179}{-18.8}{0}\eridrefrel{28.878750000000004}{-18.500000000000004}{-18.8}\ercrowfoot{29.029}{-18.65}{29.179}{-18.5}{29.179}{-18.65}{29.179}{-18.8}{0}\ercrowfoot{29.029}{-18.65}{28.879}{-18.5}{28.879}{-18.65}{28.879}{-18.8}{0}
% relationship extracted
\ertext{35.683}{-18.95}{l}{extracted}\ertext{38.645}{-11.68}{r}{chromatograms}\errelarm{35.533}{-18.65}{36.333}{-18.65}{0}{0}\errelarm{38.595}{-11.83}{38.795}{-11.83}{0}{0}\ertext{39.364}{-15.54}{r}{\textasciitilde /..=../monitored}\errelangle{36.333}{-18.65}{37.133}{-18.65}{37.764}{-15.24}{0}{0}\errelangle{37.764}{-15.24}{38.395}{-11.83}{38.595}{-11.83}{0}{0}
% relationship basedon
\ertext{35.683}{-19.75}{l}{based}\ertext{35.683}{-20.05}{l}{on}\errelangle{35.533}{-19.45}{35.533}{-19.45}{35.833}{-19.45}{0}{0}\errelangle{35.533}{-19.05}{35.533}{-19.05}{35.833}{-19.05}{0}{0}\ertext{37.383}{-19.85}{r}{on}\ertext{37.383}{-19.55}{r}{\textasciitilde /..=../based}\errelangle{35.833}{-19.45}{36.133}{-19.45}{36.133}{-19.25}{0}{0}\errelangle{36.133}{-19.25}{36.133}{-19.05}{35.833}{-19.05}{0}{0}\ercrowfoot{35.683}{-19.45}{35.533}{-19.3}{35.533}{-19.45}{35.533}{-19.6}{0}
% relationship lastmodified
\ertext{46.146}{-26.125}{l}{last}\ertext{46.146}{-26.425}{l}{modified}\errelarm{45.996}{-25.825}{46.146}{-25.825}{0}{0}\errelarm{47.127}{-1.48}{46.927}{-1.48}{0}{0}\errelangle{46.146}{-25.825}{46.296}{-25.825}{50.646}{-25.825}{0}{0}\errelangle{47.127}{-1.48}{47.326}{-1.48}{51.161}{-1.48}{0}{0}\ertext{55.596}{-13.953}{r}{\textasciitilde /\textasciicircum =\textasciicircum }\errelangle{50.646}{-25.825}{54.996}{-25.825}{54.996}{-13.653}{0}{0}\errelangle{54.996}{-13.653}{54.996}{-1.48}{51.161}{-1.48}{0}{0}\ercrowfoot{46.146}{-25.825}{45.996}{-25.675}{45.996}{-25.825}{45.996}{-25.975}{0}\erarc{41.652}{-24.75}{42.617}{-24.55}{44.548}{-24.55}{45.514}{-24.75}
% relationship 
\ertext{46.733}{-7.35}{l}{}\errelarm{46.583}{-7.05}{46.583}{-7.5}{0}{0}\errelarm{46.583}{-7.5}{46.583}{-7.95}{1}{0}\eridcomprel{46.482625}{46.682625}{-7.699999999999998}
% relationship previous
\ertext{33.801}{-27.53}{l}{previous}\errelangle{33.651}{-27.23}{33.651}{-27.23}{33.951}{-27.23}{0}{0}\errelangle{33.651}{-26.975}{33.651}{-26.975}{33.951}{-26.975}{0}{0}\ertext{35.501}{-27.703}{r}{on}\ertext{35.501}{-27.403}{r}{\textasciitilde /..=../based}\errelangle{33.951}{-27.23}{34.251}{-27.23}{34.251}{-27.103}{0}{0}\errelangle{34.251}{-27.103}{34.251}{-26.975}{33.951}{-26.975}{0}{0}\ercrowfoot{33.801}{-27.23}{33.651}{-27.08}{33.651}{-27.23}{33.651}{-27.38}{0}
% relationship method
\ertext{41.765}{-22.215}{l}{method}\errelarm{41.615}{-21.915}{42.215}{-21.915}{0}{0}\errelarm{47.795}{-6.69}{47.595}{-6.69}{0}{0}\errelangle{42.215}{-21.915}{42.815}{-21.915}{47.965}{-21.915}{0}{0}\errelangle{47.795}{-6.69}{47.995}{-6.69}{50.555}{-6.69}{0}{0}\ertext{54.715}{-14.603}{r}{\textasciitilde /..=../../../../..}\errelangle{47.965}{-21.915}{53.115}{-21.915}{53.115}{-14.303}{0}{0}\errelangle{53.115}{-14.303}{53.115}{-6.69}{50.555}{-6.69}{0}{0}\ercrowfoot{41.765}{-21.915}{41.615}{-21.765}{41.615}{-21.915}{41.615}{-22.065}{0}
% relationship lastreviewed
\ertext{41.765}{-22.68}{l}{last}\ertext{41.765}{-22.98}{l}{reviewed}\errelarm{41.615}{-22.38}{42.215}{-22.38}{0}{0}\errelarm{47.127}{-2.04}{46.927}{-2.04}{0}{0}\errelangle{42.215}{-22.38}{42.815}{-22.38}{48.815}{-22.38}{0}{0}\errelangle{47.127}{-2.04}{47.326}{-2.04}{51.071}{-2.04}{0}{0}\ertext{55.415}{-12.51}{r}{\textasciitilde /\textasciicircum =\textasciicircum }\errelangle{48.815}{-22.38}{54.815}{-22.38}{54.815}{-12.21}{0}{0}\errelangle{54.815}{-12.21}{54.815}{-2.04}{51.071}{-2.04}{0}{0}\ercrowfoot{41.765}{-22.38}{41.615}{-22.23}{41.615}{-22.38}{41.615}{-22.53}{0}
% relationship lastmodified
\ertext{41.765}{-23.3}{l}{last}\ertext{41.765}{-23.6}{l}{modified}\errelarm{41.615}{-23}{42.215}{-23}{0}{0}\errelarm{47.127}{-1.76}{46.927}{-1.76}{0}{0}\errelangle{42.215}{-23}{42.815}{-23}{49.24}{-23}{0}{0}\errelangle{47.127}{-1.76}{47.326}{-1.76}{51.496}{-1.76}{0}{0}\ertext{56.265}{-12.68}{r}{\textasciitilde /\textasciicircum =\textasciicircum }\errelangle{49.24}{-23}{55.665}{-23}{55.665}{-12.38}{0}{0}\errelangle{55.665}{-12.38}{55.665}{-1.76}{51.496}{-1.76}{0}{0}\ercrowfoot{41.765}{-23}{41.615}{-22.85}{41.615}{-23}{41.615}{-23.15}{0}
% relationship previous
\ertext{41.765}{-24.54}{l}{previous}\errelangle{41.615}{-24.24}{41.615}{-24.24}{41.915}{-24.24}{0}{0}\errelangle{41.615}{-23.775}{41.615}{-23.775}{41.915}{-23.775}{0}{0}\ertext{43.465}{-24.608}{r}{on}\ertext{43.465}{-24.308}{r}{\textasciitilde /..=../based}\errelangle{41.915}{-24.24}{42.215}{-24.24}{42.215}{-24.008}{0}{0}\errelangle{42.215}{-24.008}{42.215}{-23.775}{41.915}{-23.775}{0}{0}\ercrowfoot{41.765}{-24.24}{41.615}{-24.09}{41.615}{-24.24}{41.615}{-24.39}{0}
\end{erdiagram}

}
\end{frame}

\begin{frame}{Croswfoot notation for binary relationships}
\begin{center}
\scalebox{0.9}{
\begin{erdiagram}{1.4}{4.6666}

\eret{0}{-1}{1.333}{-0.4}{0.2}{1}\ertext{0.667}{-0.75}{}{egg}
\eret{3.333}{-1}{4.667}{-0.4}{0.2}{1}\ertext{4}{-0.75}{}{chicken}

% relationship lays
\ertext{3.183}{-1}{r}{lays}\ertext{1.483}{-0.55}{l}{by}\ertext{1.483}{-0.25}{l}{is laid}\errelarm{3.333}{-0.7}{2.333}{-0.7}{0}{0}\errelarm{2.333}{-0.7}{1.333}{-0.7}{1}{0}\ercrowfoot{1.483}{-0.7}{1.333}{-0.55}{1.333}{-0.7}{1.333}{-0.85}{0}
\end{erdiagram}

}
\end{center}
is read as
\begin{center}
\begin{enumerate}
\item Each egg \textit{is laid by} \textbf{exactly one} chicken.
\item Each chicken \textit{lays} \textbf{zero,one or more} eggs.
\end{enumerate}
\end{center}
 \textit{is laid by} therefore represents a partial function with inverse \textit{lays}.
\end{frame}

\begin{frame}{Chicken and Egg}
A further relationship 
\begin{center}
\scalebox{0.9}{
\begin{erdiagram}{1.7999999999999998}{5.066599999999999}

\eret{0.1}{-1.4}{1.433}{-0.4}{0.2}{1}\ertext{0.767}{-0.75}{}{egg}
\eret{3.733}{-1.4}{5.067}{-0.4}{0.2}{1}\ertext{4.4}{-0.75}{}{chicken}

% relationship hatched_from
\ertext{3.583}{-0.55}{r}{from}\ertext{3.583}{-0.25}{r}{hatched}\ertext{1.583}{-0.55}{l}{into}\ertext{1.583}{-0.25}{l}{hatches}\errelarm{3.733}{-0.7}{2.583}{-0.7}{1}{0}\errelarm{2.583}{-0.7}{1.433}{-0.7}{0}{0}
% relationship lays
\ertext{3.583}{-1.3}{r}{lays}\ertext{1.583}{-1.3}{l}{is laid}\ertext{1.583}{-1.6}{l}{by}\errelarm{3.733}{-1}{2.583}{-1}{0}{0}\errelarm{2.583}{-1}{1.433}{-1}{1}{0}\ercrowfoot{1.583}{-1}{1.433}{-0.85}{1.433}{-1}{1.433}{-1.15}{0}
\end{erdiagram}

}
\end{center}
reads
\begin{center}
\begin{enumerate}
\item Each egg \textit{hatches into} \textbf{zero or one} chickens.
\item Each chicken \textit{hatched from} \textbf{exactly one} egg.
\end{enumerate}
\end{center}
 \textit{hatched from} therefore represents an injective function with inverse \textit{hatches into}.
\end{frame}


\begin{frame}{Coproducts and Inheritance in ER}
\begin{itemize}
\pause \item In either formal grammar or in IDL from Carnegie-Melon we may write A ::= A1 \textbar\  A2
\pause \item In an ER model  A is said to generalise A1 and A2, (A1 and A2 are said to inherit from A) and this is represented
(in Barker's book for example) so:
\begin{center}
\scalebox{0.85}{
\begin{erdiagram}{1.45}{4}

\eret{0}{-1.45}{4}{-0}{0.2}{1}\ertext{0.116}{-0.35}{l}{A}
\eret{0.25}{-1.2}{1.75}{-0.6}{0.2}{0}\ertext{1}{-0.95}{}{A1}
\eret{2.25}{-1.2}{3.75}{-0.6}{0.2}{0}\ertext{3}{-0.95}{}{A2}

\end{erdiagram}

}
\end{center}
\pause \item In category theory this situation is represented by a coproductL A = A1 + A2  
\end{itemize}
\end{frame}




\begin{frame}{ER's Binary Relationships as Morphisms}
\footnotesize
\begin{tabular} {l l l l}
cardinality        & inverse cardinality &     &\\
exactly one        & zero, one or many   &     & $f: A \morph B$\\
exactly one        & zero or one         &     & f is injective $f: A \hookrightarrow B$\\
one or many        &                     &     &f is partial $f:A \rightharpoonup B$\\
zero, one or many  &                     &     &\\
zero or one        &                     &     &\\
\end{tabular}
\end{frame}

\iffalse
\begin{frame}{\textit{a priori}s}
\begin{itemize}
\item In language theory, formal grammars have terminals (and non-terminals)
\pause \item In data specifications, we have use of basic types string, integer, float, boolean and so on
              in addition maybe define enumerations 
\pause \item in categorical data types we have \textit{a priori}s i.e coproducts of terminal object
\pause \item In relational data models we have domains
\pause \item in ER models many-one relationships to the basic types are called attributes and are graphically distinct from other relationships
\end{itemize}
\end{frame}
\fi

\begin{frame}{Enumerated Types}
\pause for example
\begin{itemize} 
\pause \item pascal: type Boolean=(True,False)
\pause \item C-M IDL: boolean ::= true | false
\pause \item In an ER+ model:
\scalebox{0.85}{
\begin{erdiagram}{2.45}{3.8166}

\eret{0.1}{-2.45}{3.817}{-1}{0.2}{1}\ertext{0.216}{-1.35}{l}{boolean}
\eret{0.4}{-2.2}{1.733}{-1.6}{0.2}{0}\ertext{1.067}{-1.95}{}{true}
\eret{2.233}{-2.2}{3.567}{-1.6}{0.2}{0}\ertext{2.9}{-1.95}{}{false}
\eret{0}{-0.2}{3.817}{0.3}{0.2}{1}

% relationship 
\ertext{1.217}{-0.5}{l}{}\errelarm{1.067}{-0.2}{1.067}{-0.9}{1}{0}\errelarm{1.067}{-0.9}{1.067}{-1.6}{1}{0}
% relationship 
\ertext{3.05}{-0.5}{l}{}\errelarm{2.9}{-0.2}{2.9}{-0.9}{1}{0}\errelarm{2.9}{-0.9}{2.9}{-1.6}{1}{0}
\end{erdiagram}

}
\end{itemize}
\end{frame}

\begin{frame}{Types and relationships}
The following ER diagram:
\begin{center}
\scalebox{0.9}{
\begin{erdiagram}{1.4}{4.6666}

\eret{0}{-1}{1.333}{-0.4}{0.2}{1}\ertext{0.667}{-0.75}{}{egg}
\eret{3.333}{-1}{4.667}{-0.4}{0.2}{1}\ertext{4}{-0.75}{}{chicken}

% relationship lays
\ertext{3.183}{-1}{r}{lays}\ertext{1.483}{-0.55}{l}{by}\ertext{1.483}{-0.25}{l}{is laid}\errelarm{3.333}{-0.7}{2.333}{-0.7}{0}{0}\errelarm{2.333}{-0.7}{1.333}{-0.7}{1}{0}\ercrowfoot{1.483}{-0.7}{1.333}{-0.55}{1.333}{-0.7}{1.333}{-0.85}{0}
\end{erdiagram}

}
\end{center}
\begin{center}
\begin{enumerate}
\item Can be considered a data specification.
\item Is \textbf{not} a database specification. 
\end{enumerate}
\end{center}
Note: This is the arrow category -- morphisms interpreted by partial functions. 
\end{frame}

\begin{frame}{Identifying Features in Database Specifications}
\begin{itemize}
\item database specifications are data specifications in which types of entity have 
identifying features
\item combination of relationships which can identify an entity unquely
i.e a mono source
\item achieves principle of identity of indiscernibles
\end {itemize}
\end{frame}

\begin{frame}{bar notation}
\begin{itemize}
\item bar across a relationship indicates that it is one of the identifying features
\item if departments identified by department code
\item and employee identified by employee number within department 
\end{itemize}
\end{frame}

\iffalse
\begin{frame}
\begin{center}{ER modelling notation}
(departmentEmployeeComposite here)
\scalebox{0.50}{
\input{departmentEmployeeComposite}
}
\end{center}
\end{frame}
\fi


\begin{frame}{ER modelling notation}
\begin{center}
\scalebox{0.85}{
\begin{erdiagram}{12.2}{8.7}

\eret{1.7}{-3}{3.7}{-1}{0.2}{1}\ertext{1.9}{-1.35}{l}{department}
\erattr{1.9}{-1.55}{1}{0}{deptCode}
\erattr{1.9}{-1.85}{1}{1}{name}
\eret{5.7}{-2}{7.7}{-1}{0.2}{1}\ertext{5.9}{-1.35}{l}{location}
\erattr{5.9}{-1.55}{1}{0}{name}
\eret{1.45}{-6.9}{5.95}{-3.9}{0.2}{1}\ertext{1.9}{-4.25}{l}{employee}
\erattr{1.65}{-4.45}{1}{0}{empCode}
\erattr{1.65}{-4.75}{1}{1}{name}
\erattr{1.65}{-5.05}{1}{1}{startDate}
\eret{0.868}{-12.2}{2.702}{-8.65}{0.2}{1}\ertext{1.015}{-9}{l}{telephone}
\erattr{1.068}{-9.2}{1}{0}{number}
\eret{1.118}{-10.15}{2.452}{-9.55}{0.2}{0}\ertext{1.785}{-9.9}{}{home}
\eret{1.118}{-11.05}{2.452}{-10.45}{0.2}{0}\ertext{1.785}{-10.8}{}{work}
\eret{1.118}{-11.95}{2.452}{-11.35}{0.2}{0}\ertext{1.785}{-11.7}{}{mobile}
\eret{3.552}{-12.15}{5.532}{-10.15}{0.2}{1}\ertext{3.75}{-10.5}{l}{package}
\erattr{3.752}{-10.7}{1}{0}{from}
\erattr{3.752}{-11}{0}{1}{to}
\erattr{3.752}{-11.3}{1}{1}{annualSalary}
\eret{6.132}{-9.6}{7.382}{-8.3}{0.2}{1}\ertext{6.257}{-8.65}{l}{review}
\erattr{6.332}{-8.85}{1}{0}{on}
\eret{0}{-0.2}{8.7}{0.3}{0.2}{1}

% relationship 
\ertext{2.85}{-0.5}{l}{}\errelarm{2.7}{-0.2}{2.7}{-0.6}{1}{0}\errelarm{2.7}{-0.6}{2.7}{-1}{1}{0}\ercrowfoot{2.7}{-0.85}{2.55}{-1}{2.7}{-1}{2.85}{-1}{0}
% relationship 
\ertext{6.85}{-0.5}{l}{}\errelarm{6.7}{-0.2}{6.7}{-0.6}{1}{0}\errelarm{6.7}{-0.6}{6.7}{-1}{1}{0}\ercrowfoot{6.7}{-0.85}{6.55}{-1}{6.7}{-1}{6.85}{-1}{0}
% relationship 
\ertext{2.85}{-3.3}{l}{}\errelarm{2.7}{-3}{2.7}{-3.075}{1}{0}\errelarm{3.7}{-3.688}{3.7}{-3.9}{1}{0}\errelangle{2.7}{-3.075}{2.7}{-3.15}{3.2}{-3.313}{1}{0}\errelangle{3.2}{-3.313}{3.7}{-3.475}{3.7}{-3.688}{1}{0}\eridcomprel{3.6}{3.8000000000000003}{-3.65}\ercrowfoot{3.7}{-3.75}{3.55}{-3.9}{3.7}{-3.9}{3.85}{-3.9}{0}
% relationship location
\ertext{3.85}{-1.35}{l}{location}\errelarm{3.7}{-1.5}{4.7}{-1.5}{1}{0}\errelarm{4.7}{-1.5}{5.7}{-1.5}{0}{0}\ercrowfoot{3.85}{-1.5}{3.7}{-1.35}{3.7}{-1.5}{3.7}{-1.65}{0}
% relationship headOfDept
\ertext{1.55}{-2.183}{r}{headOfDept}\ertext{1.3}{-4.5}{r}{isHeadOf}\errelarm{1.7}{-2.333}{1.1}{-2.333}{1}{0}\errelarm{1.25}{-4.65}{1.45}{-4.65}{0}{0}\errelangle{1.1}{-2.333}{0.5}{-2.333}{0.35}{-2.333}{1}{0}\errelangle{1.25}{-4.65}{1.05}{-4.65}{0.625}{-4.65}{0}{0}\errelangle{0.35}{-2.333}{0.2}{-2.333}{0.2}{-3.492}{1}{0}\errelangle{0.2}{-3.492}{0.2}{-4.65}{0.625}{-4.65}{0}{0}
% relationship 
\ertext{1.935}{-7.2}{l}{}\errelarm{1.785}{-6.9}{1.785}{-7.775}{0}{0}\errelarm{1.785}{-7.775}{1.785}{-8.65}{1}{0}\eridcomprel{1.685}{1.8850000000000002}{-8.4}\ercrowfoot{1.785}{-8.5}{1.635}{-8.65}{1.785}{-8.65}{1.935}{-8.65}{0}
% relationship 
\ertext{4.692}{-7.2}{l}{}\errelarm{4.542}{-6.9}{4.542}{-8.525}{1}{0}\errelarm{4.542}{-8.525}{4.542}{-10.15}{1}{0}\eridcomprel{4.44165}{4.641649999999999}{-9.9}\ercrowfoot{4.542}{-10}{4.392}{-10.15}{4.542}{-10.15}{4.692}{-10.15}{0}
% relationship 
\ertext{5.65}{-7.2}{l}{}\errelarm{5.5}{-6.9}{5.5}{-6.975}{0}{0}\errelarm{6.757}{-8.088}{6.757}{-8.3}{1}{0}\errelangle{5.5}{-6.975}{5.5}{-7.05}{6.128}{-7.463}{0}{0}\errelangle{6.128}{-7.463}{6.757}{-7.875}{6.757}{-8.088}{1}{0}\eridcomprel{6.65665}{6.856649999999999}{-8.05}\ercrowfoot{6.757}{-8.15}{6.607}{-8.3}{6.757}{-8.3}{6.907}{-8.3}{0}
% relationship manager
\ertext{1.3}{-5.25}{r}{manager}\ertext{1.3}{-6.45}{r}{managerOf}\errelangle{1.45}{-5.4}{1.45}{-5.4}{0.85}{-5.4}{1}{0}\errelangle{1.45}{-6.15}{1.45}{-6.15}{0.85}{-6.15}{0}{0}\errelangle{0.85}{-5.4}{0.25}{-5.4}{0.25}{-5.775}{1}{0}\errelangle{0.25}{-5.775}{0.25}{-6.15}{0.85}{-6.15}{0}{0}\ercrowfoot{1.3}{-5.4}{1.45}{-5.25}{1.45}{-5.4}{1.45}{-5.55}{0}
% relationship reviewBy
\ertext{7.532}{-8.579}{l}{reviewBy}\errelarm{7.382}{-8.729}{7.457}{-8.729}{1}{0}\errelarm{6.15}{-6.3}{5.95}{-6.3}{0}{0}\errelangle{7.457}{-8.729}{7.532}{-8.729}{8.032}{-8.729}{1}{0}\errelangle{6.15}{-6.3}{6.35}{-6.3}{7.441}{-6.3}{0}{0}\errelangle{8.032}{-8.729}{8.532}{-8.729}{8.532}{-7.515}{1}{0}\errelangle{8.532}{-7.515}{8.532}{-6.3}{7.441}{-6.3}{0}{0}\ercrowfoot{7.532}{-8.729}{7.382}{-8.579}{7.382}{-8.729}{7.382}{-8.879}{0}
\end{erdiagram}

}
\end{center}
\end{frame}


\begin{frame}{Grammar or ER}
\begin{itemize}
\pause \item Based on syntax given by Brinton (Structure of English Sentence)
\end{itemize}
\begin{center}
\scalebox{0.85}{
\begin{erdiagram}{4.5}{8.6}

\eret{0}{-3.3}{8.6}{-0}{0.2}{1}\ertext{0.264}{-0.35}{l}{verb phrase}
\eret{0.25}{-1.2}{2.65}{-0.6}{0.2}{0}\ertext{1.45}{-0.95}{}{intransitive}
\eret{2.85}{-2.55}{8.35}{-0.6}{0.2}{0}\ertext{3.006}{-0.95}{l}{transitive}
\eret{3.1}{-1.8}{5.5}{-1.2}{0.2}{1}\ertext{4.3}{-1.55}{}{mono transitive}
\eret{5.7}{-1.8}{8.1}{-1.2}{0.2}{1}\ertext{6.9}{-1.55}{}{ditransitive}
\eret{0.25}{-4.5}{2.65}{-3.9}{0.2}{1}\ertext{1.45}{-4.25}{}{verb}
\eret{4.1}{-4.5}{7.1}{-3.9}{0.2}{1}\ertext{5.6}{-4.25}{}{noun phrase}

% relationship head
\ertext{1.3}{-3.6}{r}{head}\errelarm{1.45}{-3.3}{1.45}{-3.6}{1}{0}\errelarm{1.45}{-3.6}{1.45}{-3.9}{1}{0}
% relationship direct_object
\ertext{4.7}{-2.85}{r}{direct}\ertext{4.7}{-3.15}{r}{object}\errelarm{4.85}{-2.55}{4.85}{-3.225}{1}{0}\errelarm{4.85}{-3.225}{4.85}{-3.9}{1}{0}
% relationship indirect_object
\ertext{6.57}{-2.1}{l}{indirect}\ertext{6.57}{-2.4}{l}{object}\errelarm{6.42}{-1.8}{6.42}{-2.85}{1}{0}\errelarm{6.42}{-2.85}{6.42}{-3.9}{1}{0}\erarc{4.35}{-3.7}{4.975}{-3.5}{6.225}{-3.5}{6.85}{-3.7}
\end{erdiagram}

}
\end{center}
\begin{itemize}
\pause \item This is a fragment of either or both of a data specification (ER model) and/or a grammar.
\end{itemize}
\end{frame}


\begin{frame}{Chicken and Egg}
The next ER diagram:
\begin{center}
\scalebox{0.9}{
\begin{erdiagram}{1.7999999999999998}{5.066599999999999}

\eret{0.1}{-1.4}{1.433}{-0.4}{0.2}{1}\ertext{0.767}{-0.75}{}{egg}
\eret{3.733}{-1.4}{5.067}{-0.4}{0.2}{1}\ertext{4.4}{-0.75}{}{chicken}

% relationship hatched_from
\ertext{3.583}{-0.55}{r}{from}\ertext{3.583}{-0.25}{r}{hatched}\ertext{1.583}{-0.55}{l}{into}\ertext{1.583}{-0.25}{l}{hatches}\errelarm{3.733}{-0.7}{2.583}{-0.7}{1}{0}\errelarm{2.583}{-0.7}{1.433}{-0.7}{0}{0}
% relationship lays
\ertext{3.583}{-1.3}{r}{lays}\ertext{1.583}{-1.3}{l}{is laid}\ertext{1.583}{-1.6}{l}{by}\errelarm{3.733}{-1}{2.583}{-1}{0}{0}\errelarm{2.583}{-1}{1.433}{-1}{1}{0}\ercrowfoot{1.583}{-1}{1.433}{-0.85}{1.433}{-1}{1.433}{-1.15}{0}
\end{erdiagram}

}
\end{center}
\begin{center}
\begin{enumerate}
\item Has faults as a data specification.
\item Is still not a database specification. 
\end{enumerate}
\end{center}
\end{frame}




\begin{frame}{Relational Model of Data}
\scalebox{0.6}{
\begin{erdiagram}{10.1}{14.9495}

\eret{4.85}{-2.15}{10.35}{-1.25}{0.2}{1}\ertext{5.4}{-1.6}{l}{table}
\erattr{5.05}{-1.8}{1}{0}{name}
\eret{1.338}{-4.65}{4.112}{-3.75}{0.2}{1}\ertext{1.615}{-4.1}{l}{primary key column}
\erattr{1.538}{-4.3}{1}{1}{seq no}
\eret{6.862}{-4.65}{8.662}{-3.75}{0.2}{1}\ertext{7.042}{-4.1}{l}{column}
\erattr{7.062}{-4.3}{1}{0}{name}
\eret{11.662}{-4.65}{13.462}{-3.75}{0.2}{1}\ertext{11.842}{-4.1}{l}{foreign key}
\erattr{11.862}{-4.3}{1}{0}{name}
\eret{11.175}{-6.8}{13.95}{-6.2}{0.2}{1}\ertext{12.562}{-6.55}{}{foreign key column}
\eret{0}{-0.2}{14.95}{0.3}{0.2}{1}

% relationship all tables
\ertext{7.75}{-0.5}{l}{all tables}\errelarm{7.6}{-0.2}{7.6}{-0.725}{1}{0}\errelarm{7.6}{-0.725}{7.6}{-1.25}{1}{0}\ercrowfoot{7.6}{-1.1}{7.45}{-1.25}{7.6}{-1.25}{7.75}{-1.25}{0}
% relationship 
\ertext{6.1}{-2.45}{l}{}\ertext{2.875}{-3.6}{l}{of}\errelarm{5.95}{-2.15}{5.95}{-2.225}{1}{0}\errelarm{2.725}{-3.463}{2.725}{-3.75}{1}{0}\errelangle{5.95}{-2.225}{5.95}{-2.3}{4.337}{-2.738}{1}{0}\errelangle{4.337}{-2.738}{2.725}{-3.175}{2.725}{-3.463}{1}{0}\errelseq{2.785}{-3.225}{2.375}{-3.285}{3.075}{-3.345}{2.665}{-3.405}\eridcomprel{2.6249999999999987}{2.824999999999999}{-3.5}\ercrowfoot{2.725}{-3.6}{2.575}{-3.75}{2.725}{-3.75}{2.875}{-3.75}{0}
% relationship 
\ertext{7.912}{-2.45}{l}{}\ertext{7.912}{-3.6}{l}{of}\errelarm{7.762}{-2.15}{7.762}{-2.95}{1}{0}\errelarm{7.762}{-2.95}{7.762}{-3.75}{1}{0}\eridcomprel{7.66225}{7.8622499999999995}{-3.5}\ercrowfoot{7.762}{-3.6}{7.612}{-3.75}{7.762}{-3.75}{7.912}{-3.75}{0}
% relationship 
\ertext{9.4}{-2.45}{l}{}\ertext{12.712}{-3.6}{l}{of}\errelarm{9.25}{-2.15}{9.25}{-2.225}{1}{0}\errelarm{12.562}{-3.538}{12.562}{-3.75}{1}{0}\errelangle{9.25}{-2.225}{9.25}{-2.3}{10.906}{-2.813}{1}{0}\errelangle{10.906}{-2.813}{12.562}{-3.325}{12.562}{-3.538}{1}{0}\eridcomprel{12.462250000000001}{12.66225}{-3.5}\ercrowfoot{12.562}{-3.6}{12.412}{-3.75}{12.562}{-3.75}{12.712}{-3.75}{0}
% relationship is
\ertext{4.262}{-4}{l}{is}\ertext{5.137}{-4.5}{l}{\textasciitilde /of=of}\errelarm{4.112}{-4.2}{5.487}{-4.2}{1}{0}\errelarm{5.487}{-4.2}{6.862}{-4.2}{0}{0}\eridrefrel{4.3622499999999995}{-4.1000000000000005}{-4.3}
% relationship 
\ertext{12.712}{-4.95}{l}{}\ertext{12.812}{-6.05}{l}{partof}\errelarm{12.562}{-4.65}{12.562}{-5.425}{1}{0}\errelarm{12.562}{-5.425}{12.562}{-6.2}{1}{0}\eridcomprel{12.462250000000001}{12.66225}{-5.95}\ercrowfoot{12.562}{-6.05}{12.412}{-6.2}{12.562}{-6.2}{12.712}{-6.2}{0}
% relationship to
\ertext{13.612}{-4.05}{l}{to}\errelarm{13.462}{-4.2}{14.062}{-4.2}{1}{0}\errelarm{2.475}{-1.849}{4.85}{-1.849}{0}{0}\errelangle{14.062}{-4.2}{14.662}{-4.2}{14.662}{-6.9}{1}{0}\errelangle{2.475}{-1.849}{0.1}{-1.849}{0.1}{-5.725}{0}{0}\ertext{7.031}{-9.9}{l}{\textasciitilde /\textasciicircum =\textasciicircum }\errelangle{14.662}{-6.9}{14.662}{-9.6}{7.381}{-9.6}{1}{0}\errelangle{7.381}{-9.6}{0.1}{-9.6}{0.1}{-5.725}{0}{0}\ercrowfoot{13.612}{-4.2}{13.462}{-4.05}{13.462}{-4.2}{13.462}{-4.35}{0}
% relationship is
\ertext{11.025}{-6.7}{r}{is}\errelarm{11.175}{-6.4}{11.025}{-6.4}{1}{0}\errelarm{8.812}{-4.349}{8.662}{-4.349}{0}{0}\ertext{8.369}{-5.675}{l}{\textasciitilde /of=partof/of}\errelangle{11.025}{-6.4}{10.875}{-6.4}{9.919}{-5.375}{1}{0}\errelangle{9.919}{-5.375}{8.962}{-4.349}{8.812}{-4.349}{0}{0}\ercrowfoot{11.025}{-6.4}{11.175}{-6.25}{11.175}{-6.4}{11.175}{-6.55}{0}
% relationship to
\ertext{14.1}{-6.4}{l}{to}\errelarm{13.95}{-6.6}{14.2}{-6.6}{1}{0}\errelarm{1.088}{-4.349}{1.338}{-4.349}{0}{0}\errelangle{14.2}{-6.6}{14.45}{-6.6}{14.45}{-7.1}{1}{0}\errelangle{1.088}{-4.349}{0.838}{-4.349}{0.838}{-5.975}{0}{0}\ertext{6.794}{-7.9}{l}{\textasciitilde /of=partof/to}\errelangle{14.45}{-7.1}{14.45}{-7.6}{7.644}{-7.6}{1}{0}\errelangle{7.644}{-7.6}{0.838}{-7.6}{0.838}{-5.975}{0}{0}\ercrowfoot{14.1}{-6.6}{13.95}{-6.45}{13.95}{-6.6}{13.95}{-6.75}{0}\eridrefrel{14.1995}{-6.500000000000001}{-6.7}
\end{erdiagram}

}
\end{frame}

\begin{frame}{Relational Model of Data}
\scalebox{0.6}{
\begin{erdiagram}{11.899999999999999}{16.04475}

\eret{4.85}{-2.15}{10.35}{-1.25}{0.2}{1}\ertext{5.4}{-1.6}{l}{table}
\erattr{5.05}{-1.8}{1}{0}{name}
\eret{0.138}{-5.55}{2.912}{-3.75}{0.2}{1}\ertext{0.415}{-4.1}{l}{primary key column}
\erdattr{0.338}{-4.3}{1}{0}{table name(D2)}
\erdattr{0.338}{-4.6}{1}{0}{is name(R1)}
\erattr{0.338}{-4.9}{1}{1}{seq no}
\erattr{0.338}{-5.2}{1}{1}{seqNo}
\eret{5.662}{-5.25}{9.427}{-3.75}{0.2}{1}\ertext{6.039}{-4.1}{l}{column}
\erdattr{5.862}{-4.3}{1}{0}{table name(D3)}
\erattr{5.862}{-4.6}{1}{0}{name}
\erdattr{5.862}{-4.9}{0}{1}{in primary key is name(R2)}
\eret{12.427}{-5.25}{14.662}{-3.75}{0.2}{1}\ertext{12.651}{-4.1}{l}{foreign key}
\erdattr{12.627}{-4.3}{1}{0}{table name(D4)}
\erattr{12.627}{-4.6}{1}{0}{name}
\erdattr{12.627}{-4.9}{1}{1}{to name(R3)}
\eret{12.045}{-8.6}{15.045}{-6.8}{0.2}{1}\ertext{12.495}{-7.15}{l}{foreign key column}
\erdattr{12.245}{-7.35}{1}{0}{table name(D5)}
\erdattr{12.245}{-7.65}{1}{0}{foreign key name(D5)}
\erdattr{12.245}{-7.95}{1}{0}{to is name(R5)}
\erdattr{12.245}{-8.25}{1}{1}{is name(R4)}
\eret{0}{-0.2}{16.045}{0.3}{0.2}{1}

% relationship all tables
\ertext{7.75}{-0.5}{l}{all tables}\errelarm{7.6}{-0.2}{7.6}{-0.725}{1}{0}\errelarm{7.6}{-0.725}{7.6}{-1.25}{1}{0}\ercrowfoot{7.6}{-1.1}{7.45}{-1.25}{7.6}{-1.25}{7.75}{-1.25}{0}
% relationship 
\ertext{6.1}{-2.45}{l}{}\ertext{1.675}{-3.6}{l}{of}\errelarm{5.95}{-2.15}{5.95}{-2.225}{1}{0}\errelarm{1.525}{-3.463}{1.525}{-3.75}{1}{0}\ertext{3.587}{-2.588}{r}{D2}\errelangle{5.95}{-2.225}{5.95}{-2.3}{3.737}{-2.738}{1}{0}\errelangle{3.737}{-2.738}{1.525}{-3.175}{1.525}{-3.463}{1}{0}\errelseq{1.585}{-3.225}{1.175}{-3.285}{1.875}{-3.345}{1.465}{-3.405}\eridcomprel{1.4249999999999996}{1.6249999999999998}{-3.5}\ercrowfoot{1.525}{-3.6}{1.375}{-3.75}{1.525}{-3.75}{1.675}{-3.75}{0}
% relationship 
\ertext{7.695}{-2.45}{l}{}\ertext{7.695}{-3.6}{l}{of}\ertext{7.695}{-2.8}{l}{D3}\errelarm{7.545}{-2.15}{7.545}{-2.95}{1}{0}\errelarm{7.545}{-2.95}{7.545}{-3.75}{1}{0}\eridcomprel{7.444750000000001}{7.64475}{-3.5}\ercrowfoot{7.545}{-3.6}{7.395}{-3.75}{7.545}{-3.75}{7.695}{-3.75}{0}
% relationship 
\ertext{9.4}{-2.45}{l}{}\ertext{13.695}{-3.6}{l}{of}\errelarm{9.25}{-2.15}{9.25}{-2.225}{1}{0}\errelarm{13.545}{-3.538}{13.545}{-3.75}{1}{0}\ertext{11.547}{-2.663}{l}{D4}\errelangle{9.25}{-2.225}{9.25}{-2.3}{11.397}{-2.813}{1}{0}\errelangle{11.397}{-2.813}{13.545}{-3.325}{13.545}{-3.538}{1}{0}\eridcomprel{13.44475}{13.64475}{-3.5}\ercrowfoot{13.545}{-3.6}{13.395}{-3.75}{13.545}{-3.75}{13.695}{-3.75}{0}
% relationship is
\ertext{3.062}{-4.45}{l}{is}\ertext{4.237}{-4.5}{l}{R1}\ertext{3.937}{-4.95}{l}{\textasciitilde /of=of}\errelarm{2.912}{-4.65}{4.287}{-4.65}{1}{0}\errelarm{4.287}{-4.65}{5.662}{-4.65}{0}{0}\eridrefrel{3.16225}{-4.550000000000001}{-4.75}
% relationship 
\ertext{13.695}{-5.55}{l}{}\ertext{13.795}{-6.65}{l}{partof}\ertext{13.695}{-5.875}{l}{D5}\errelarm{13.545}{-5.25}{13.545}{-6.025}{1}{0}\errelarm{13.545}{-6.025}{13.545}{-6.8}{1}{0}\eridcomprel{13.44475}{13.64475}{-6.55}\ercrowfoot{13.545}{-6.65}{13.395}{-6.8}{13.545}{-6.8}{13.695}{-6.8}{0}
% relationship to
\ertext{14.812}{-4.35}{l}{to}\errelarm{14.662}{-4.5}{15.262}{-4.5}{1}{0}\errelarm{2.475}{-1.849}{4.85}{-1.849}{0}{0}\errelangle{15.262}{-4.5}{15.862}{-4.5}{15.862}{-7.2}{1}{0}\errelangle{2.475}{-1.849}{0.1}{-1.849}{0.1}{-5.875}{0}{0}\ertext{7.631}{-9.75}{l}{R3}\ertext{7.631}{-10.2}{l}{\textasciitilde /\textasciicircum =\textasciicircum }\errelangle{15.862}{-7.2}{15.862}{-9.9}{7.981}{-9.9}{1}{0}\errelangle{7.981}{-9.9}{0.1}{-9.9}{0.1}{-5.875}{0}{0}\ercrowfoot{14.812}{-4.5}{14.662}{-4.35}{14.662}{-4.5}{14.662}{-4.65}{0}
% relationship is
\ertext{11.895}{-7.7}{r}{is}\errelarm{12.045}{-7.4}{11.895}{-7.4}{1}{0}\errelarm{9.577}{-4.749}{9.427}{-4.749}{0}{0}\ertext{10.186}{-6.025}{l}{R4}\ertext{9.186}{-6.375}{l}{\textasciitilde /of=partof/of}\errelangle{11.895}{-7.4}{11.745}{-7.4}{10.736}{-6.075}{1}{0}\errelangle{10.736}{-6.075}{9.727}{-4.749}{9.577}{-4.749}{0}{0}\ercrowfoot{11.895}{-7.4}{12.045}{-7.25}{12.045}{-7.4}{12.045}{-7.55}{0}
% relationship to
\ertext{15.195}{-7.8}{l}{to}\errelarm{15.045}{-8}{15.295}{-8}{1}{0}\errelarm{-0.112}{-4.949}{0.138}{-4.949}{0}{0}\errelangle{15.295}{-8}{15.545}{-8}{15.545}{-8.5}{1}{0}\errelangle{-0.112}{-4.949}{-0.362}{-4.949}{-0.362}{-6.974}{0}{0}\ertext{7.241}{-8.85}{l}{R5}\ertext{6.741}{-9.3}{l}{\textasciitilde /of=partof/to}\errelangle{15.545}{-8.5}{15.545}{-9}{7.591}{-9}{1}{0}\errelangle{7.591}{-9}{-0.362}{-9}{-0.362}{-6.974}{0}{0}\ercrowfoot{15.195}{-8}{15.045}{-7.85}{15.045}{-8}{15.045}{-8.15}{0}\eridrefrel{15.29475}{-7.9}{-8.1}
\end{erdiagram}

}
\end{frame}




\iffalse
\begin{frame}{Bibliography}
\bibliography{../../SharedBibliography/temp/bibliography}
\end{frame}
\fi

\end{document}