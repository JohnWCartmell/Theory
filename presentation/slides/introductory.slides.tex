
\begin{frame}{Introduction}
\begin{itemize}
\item
What types of things are there and how are they related? 
\begin{itemize}
\pause \item Data specifications provide the answer to this question in the context of a software development. 
\pause \item Types theories provide the answer in the context of mathematics. 
\pause \item Category theory abstracts across both these domains.
\end{itemize}
\end{itemize}
\end{frame}

\begin{frame}{Overview}
Data specification method 
\begin{itemize}
\item is a method for expressing a  theory (of what is)
\item unequivocally it enables definitions of types and certain relationships between these types
\item types are equally types of data and types of real world entity
\end{itemize}
\end{frame}

\begin{frame}{Methods of Data Specification}
\begin{itemize}
	\item schema of relational database,\Rnode{i1}{},
	\item structure described by Carnegie-Mellon IDL,
	\item schema of nested relational database,
	\item message structure described by Google protocol buffer IDL,
	\item XML schema language,
	\item ER script.\Rnode{i6}{}
\end{itemize}
\begin{arrows}
\ncarr{i6}{i1}
\end{arrows}
\end{frame}

\begin{frame}{Data Specifications}
\begin{itemize}
\item presentations of theories of what is i.e. an ontology
\item choice of primitives in presentation is choice of which data to be stored or communicated
\item independence of primitives is absence of redundancy in data 
\item engineering driven goodness criteria (3NF, BCNF, 4NF, 5NF) reinterpreted as driving logical completeness of theory with respect to intended instances
\end{itemize}
\end{frame}

\begin{frame}{Data Specifications}
Two kinds of types in play
\begin{itemize}
\item  the \textit{definienda} -- types all of whose instances are \term{particulars}
\begin{itemize}
\item employee, department, student, account, product, order, shipment, delivery, flight, booking and so on
\item molecular structure, atom, bond, element, isotope, reaction, metabolite, mass trace, chromatogram, peak
\item table, column, primary key, foreign key
\item node and edge. 
\end{itemize}
\pause 
\item  the \textit{definiens}  -- types all of whose instances are \term{universals}
\begin{itemize}
       \item string, integer, float, boolean and so on
\end{itemize}
\end{itemize}
\pause
\begin{itemize}
\item in ER modelling 
\begin{itemize}
\item the \textit{definienda} are called \textit{entity types}
\item the \textit{definiens} are called \textit{attribute types} or \textit{domains}.
\end{itemize}
\end{itemize}
\end{frame}

\begin{frame}{Types all of whose instances are Universals}
\begin{itemize}
\item In language theory, formal grammars have terminals (and non-terminals)
\pause \item In data specifications, we have use of basic types string, integer, float, boolean and so on
              in addition maybe define enumerations 
\pause \item in categorical data types we have  i.e coproducts of terminal object
\pause \item In relational data models we have domains
\pause \item in ER models many-one relationships to the basic types are called attributes and are graphically distinct from other relationships.
\end{itemize}
\end{frame}


\begin{frame}{Data Specifications}
A data specification is a sketch of a category with some additional structure:
\begin{itemize}
\item that it is a \term{sketch} is crucial because it is only nodes and edges of the sketch for which data is stored and/or communicated, 
\item that there are commutative diagrams is crucial to construction of physical 
specifications from logical specifications.
\item that the category had additional structure is significant:
\begin{itemize}
\item so that we can give account of database normal forms 
(BCNF, 3NF, 4NF and 5NF),
\item so that we can allow for missing data as represented by NULL values, 
\item so that we can distinguish structural from non-structural relationships to describe structure nesting and thereby hierarchical data,
\item so that types of universals can be distinguished from types of particulars.
\end{itemize}
\end{itemize}
\end{frame}



\begin{frame}{Identifying Features in Database Specifications}
\begin{itemize}
\item database specifications are data specifications in which types of entity have 
identifying features
\item combination of relationships which can identify an entity unquely
i.e a mono source
\item achieves principle of identity of indiscernibles
\end {itemize}
\end{frame}

\begin{frame}{Database Normal Forms}
\begin{itemize}
\item presentations (sketches) should be minimal and avoid redundancy:
\item how to make this precise?
\item in case of relational data model leads to 
   \begin{itemize}
     \item third normal form (3NF)
     \item Boyce-Codd normal form (BCNF)
     \item fourth normal form (4NF)
     \item fifth normal form (5NF)
   \end{itemize}
\end{itemize}
\end{frame}

\begin{frame}{Database Normal Forms}
\begin{itemize}
\item presentations (sketches) should be minimal and avoid redundancy:
\item how to make this precise?
\item in case of relational data model leads to 
   \begin{itemize}
     \item third normal form (3NF)
     \item Boyce-Codd normal form (BCNF)
     \item fourth normal form (4NF)
     \item fifth normal form (5NF)
   \end{itemize}
\end{itemize}
\end{frame}

\begin{frame}{Additional Structure}
\resizebox{11.3cm}{!}{
\newcommand{\featurelist}{\begin{tabular}{|l|l l|}
\hline 
\multirow{11}{1.5cm}{category}
                & finitary property        & \\
\cline{2-3}
                & pu-partition             & \\
\cline{2-3}
                & \multirow{2}{3.5cm}{mono-sources}  & \multicolumn{1}{|l|}{cannonical monos}  \\
\cline{3-3}
                &                                    & \multicolumn{1}{|l|}{epi-mono factorisation}   \\
\cline{2-3}
                & finite products          & \\
\cline{2-3} 
                & finite limits            & \\
\cline{2-3}
                & restrictions             & \\
\cline{2-3}
                & \multirow{2}{3.5cm}{distinguished morphisms} & \multicolumn{1}{|l|}{hierarchical}      \\
\cline{3-3}
                &                                             &  \multicolumn{1}{|l|}{non-hierarchical} \\
\cline{3-3}
                &                                             &  \multicolumn{1}{|l|}{pullbacks} \\
\cline{2-3}
                & finite coproducts                           &                                   \\
\hline                
\end{tabular}}
\featurelist
}
\end{frame}

