
\begin{frame}{Introduction}
I wish to show that
\begin{itemize}
\item we can genericise database normal form criteria into abstract logical terms,
\item achieve generic goodness criteria that can be applied to all data specifications,
 \item that the classic relational database normal form criteria (2NF, 3NF, BCNF, INC-NF, 4NF, 5NF)  are then consequences of these generic goodness criteria.
\end{itemize}
\end{frame}

\begin{frame}{View}
A data specification  
\begin{itemize}
\item is a  theory (of what is)
\end{itemize}
\medskip
A data specification method 
\begin{itemize}
\item is a method for expressing such a  theory
\item unequivocally it enables definitions of types and certain relationships between these types
\item types are equally types of data and types of real world entity
\end{itemize}
More precisely, 
\begin{itemize}
\item data specification are \underline{presentations} of theories of what is,
\item choice of primitives in a given presentation is choice of which data to be stored or communicated.
\end{itemize}
\end{frame}



\begin{frame}{Goodness Criteria}
I will define two types of goodness criteria
\begin{itemize}
    \item Goodness Criteria of Type 1 -- absence of redundancy in presentation.
    \item Ensures absence of redundancy in data and in data management logic.
    \item Spelt out in more detail in criteria 1A, 1B and so on
    \item  Goodness Criteria of Type 2 -- the theory be the tightest fit to the facts 
    \item Two ways of expressing this. 
    \begin{itemize}
        \item Criteria 2 is that the theory is maximally constrained.
        \item Criteria 2A, 2B etc.  that it be logically complete in some sense.
    \end{itemize}
\end{itemize}
\end{frame}

\begin{frame}{Data Specification as Category}
\begin{itemize}
 \item Not surprising that a data specification can represented as a category or as a sketch of a category
\pause \item
What types of things are there and how are they related? 
\begin{itemize}
\item Data specifications provide the answer to this question in the context of a software development. 
\item Types theories provide the answer in the context of mathematics. 
\item Category theory abstracts across both these domains.
\end{itemize}
\pause \item Nor is it surpising if data specifications can be seen in terms
of contextual categories or a sketches for a contextual categories
because as categories model types so contextual categories model types that vary
and  our instinct for types and types that vary comes from the real world not from mathematics.
\end{itemize}
\end{frame}


\begin{frame}{Overview}
\begin{center}
\begin{tabular}{p{12cm}}
\begin{itemize}
    \item data specifications 
    \begin{itemize}
        \item as sketches of structured categories of some kind
        \item data instances as certain structure preserving functors to the category of finite sets $\Fin$
    \end{itemize}
    \item database specifications
    \begin{itemize}
         \item category has designated mono sources for some of its objects 
    \end{itemize}
    \item relational database specifications charcterised by
    \begin{itemize}
         \item no use of containment or nesting
         \item no  hierarchical organisation -- said to be flat
         \item no use of pointers 
         \item instead uses foreign keys to represent relationships in the data
    \end{itemize}
\end{itemize} 
\end{tabular}
\end{center}
\end{frame}

\begin{frame}{Overview}
\shrinkbox{0.4}{
\begin{pspicture}(-6,-6)(9,9)
 \psgrid
 \psset{fillstyle=solid,opacity=0.5}
 \pscircle[fillcolor=lightyellow]{1.5}
 \psRing[fillcolor=white]{1.5}{3.5}
 \psRing[fillcolor=lightyellow]{3.5}{5}

 \end{pspicture}
}
\end{frame}

\begin{frame}{Normal Forms}
\begin{center}
\begin{tabular}{p{12cm}}
\begin{itemize}
\item data specifications 
\begin{itemize}
    \item as sketches of structured categories of some kind
    \item data instances as certain structure preserving functors to the category of finite sets $\Fin$
\end{itemize}
\item database specifications
\begin{itemize}
     \item category has designated mono sources for some of its objects 
\end{itemize}
\end{itemize} \\
\hdashline
\begin{itemize}
\item relational database normal form criteria  
\begin{itemize}
    \item first normal form (1NF)
    \item 2nd normal form (2NF)
    \item 3rd normal form (3NF)
    \item Boyce-Codd normal form (BCNF)
    \item inclusion dependency normal form (INC-NF)
    \item 4th normal form (4NF)
    \item projection-join normal form (5NF)
\end{itemize}
\end{itemize}
\end{tabular}
\end{center}
\end{frame}

\iffalse
\begin{frame}{Levels of Data Specification}
\begin{itemize}
\item logical   -- a sketch for a category of some kind
\item structural -- a sketch for some kind of category with distinguished morphisms indicating structural relationships
\item representational -- structural PLUS representational indicators for non-structural relationships
\item technological    -- IDL, SQL, XML 
\end{itemize}
\end{frame}
\fi

\newcommand{\bigdownarrow}
{
\scalebox{0.3}
{
\begin{pspicture}(3,3.5) 
%\psgrid
%\psset{doublesep=2cm} 
\psBigArrow[fillstyle=solid, fillcolor=blue!30,linecolor=blue](2.0,3)(2.0,0)
\end{pspicture}
}
}
\begin{frame}{Levels of Data Specification}
\begin{center}
\begin{tabular}{c l}
                              & \raisebox{0cm}{\parbox{5cm}{sketch of category of some kind}}\\
\cline{1-1}
\multicolumn{1}{|c|}{logical} & \\
\cline{1-1}
\multicolumn{1}{c}{\bigdownarrow} & \raisebox{0.5cm}{\parbox{5cm}{distinguish morphisms/relationships represented in data by containment}} \\
\cline{1-1}
\multicolumn{1}{|c|}{structural} &\\
\cline{1-1}
\multicolumn{1}{c}{\bigdownarrow} & \raisebox{0.5cm}{\parbox{5cm}{add edges for foreign keys representing non-containment relationships and add path equivalences that define them}} \\
\cline{1-1}
\multicolumn{1}{|c|}{representational} & \\
\cline{1-1}
\multicolumn{1}{c}{\bigdownarrow} & \raisebox{0.5cm}{choice of technology} \\
\cline{1-1}
\multicolumn{1}{|c|}{technological} & \raisebox{0cm}{\parbox{5cm}{IDL, XML, SQL}}\\
\cline{1-1}
\end{tabular}
\end{center}
\end{frame}


\begin{frame}{Classic Relational Normal Form Criteria}
\begin{itemize}
    \item database normal forms are goodness criteria (GC) based on software engineering principles
    \item relational database normal form criteria  
    \begin{itemize}
        \item first normal form (1NF)
        \item 2nd normal form (2NF)
        \item 3rd normal form (3NF)
        \item Boyce-Codd normal form (BCNF)
        \item inclusion dependency normal form (INC-NF)
        \item 4th normal form (4NF)
        \item projection-join normal form (5NF)
    \end{itemize}
\end{itemize}
\end{frame}


\iffalse
\begin{frame}{Methods of Data Specification}
\begin{itemize}
	\item schema of relational database,
	\item structure described by Carnegie-Mellon IDL,
	\item schema of nested relational database,
	\item message structure described by Google protocol buffer IDL,
	\item XML schema language,
	\item ER script.
\end{itemize}
\end{frame}
\fi

\begin{frame}{Data Specifications}
Two kinds of types in play
\begin{itemize}
\item  the \textit{definienda} -- types all of whose instances are \term{particulars}
\begin{itemize}
\item employee, department, student, account, product, order, shipment, delivery, flight, booking and so on
\item molecular structure, atom, bond, element, isotope, reaction, metabolite, mass trace, chromatogram, peak
\item table, column, primary key, foreign key
\item node and edge. 
\end{itemize}
\pause 
\item  the \textit{definiens}  -- types all of whose instances are \term{universals}
\begin{itemize}
       \item string, integer, float, boolean and so on
\end{itemize}
\end{itemize}
\pause
\begin{itemize}
\item in ER modelling 
\begin{itemize}
\item the \textit{definienda} are called \textit{entity types}
\item the \textit{definiens} are called \textit{attribute types} or \textit{domains}.
\end{itemize}
\end{itemize}
\end{frame}


\begin{frame}{Data Specifications}
A data specification is a sketch of a category with some additional structure:
\begin{itemize}
\item that it is a \term{sketch} is crucial because it is only nodes and edges of the sketch for which data is stored and/or communicated, 
\item that there are commutative diagrams is crucial to construction of physical 
specifications from logical specifications.
\item that the category had additional structure is significant:
\begin{itemize}
\item so that we can give account of database normal forms 
(BCNF, 3NF, 4NF and 5NF),
\item so that we can allow for missing data as represented by NULL values, 
\item so that we can distinguish structural from non-structural relationships to describe structure nesting and thereby hierarchical data,
\item so that types of universals can be distinguished from types of particulars.
\end{itemize}
\end{itemize}
\end{frame}



\iffalse
\begin{frame}{Database Normal Forms}
\begin{itemize}
\item presentations (sketches) should be minimal and avoid redundancy:
\item how to make this precise?
\item in case of relational data model leads to 
   \begin{itemize}
     \item third normal form (3NF)
     \item Boyce-Codd normal form (BCNF)
     \item fourth normal form (4NF)
     \item fifth normal form (5NF)
   \end{itemize}
\end{itemize}
\end{frame}
\fi

\begin{frame}{Relational Database Theory}
\begin{itemize}
\item classic relational database normal form definitions ({\scriptsize 3NF, EKNF, BCNF, 4NF,5NF, INC-NF}) can be abstracted  into a general logical framework

\item such normal forms  examine the fit of a sketch/theory (database schema) to an intended usage

\item we can assume that the intended usage is represented by a full-subcategory of the category $Fun(S,\cat{FinSet})$

\item in such a situation the classic normal forms address the question can the sketch/theory $S$ be improved by addition or removal of morphisms and/or commutative diagrams and/or limit cones.
\item normalisation has dual goal of obtaining as complete a theory as possible and of eliminating redundancy from the sketch.  
\end{itemize}
\end{frame}

\begin{frame}{Normal Forms}
\begin{itemize}
\item IN-NF -- Ling and Goh -- there are no redundant attributes except if absolutely necessary in order to specify a mono source
\end{itemize}
\end{frame}

\iffalse %moved into goodness criteria section and reworded
\begin{frame}{Normalisation}
\begin{definition}
{ \footnotesize
If $T$ is a theory and $W \subset |Mod(T,FinSet)|$ is an intended usage then an interpretation (theory morphism) $I: T \morph T'$ is an improvement of $T$ wrt $W$ iff 
$Mod(I,Finset): Mod(U',Finset) \morph Mod(U,Finset)$ is injective but not surjective
and $W \subseteq img(Mod(I,Finset))$.
i.e. for all models $U \in W$ there exists $U' \in Mod(T,Finset)$ such that $I \circ U'=U$
$
\begin{array} {c p{2cm} c}
\Rnode{T}{T} && \\ [0.25cm]
             && \Rnode{finset}{Finset} \\ [0.25cm]
\Rnode{Tp}{T'}  
\end{array}
$
\ncarr {T}{finset}
\alabel{U}
\ncarr{T}{Tp}
\blabel{I}
\ncarr{Tp}{finset}
\blabel{U'} 
}
\end{definition}

\begin{definition}
If a theory $T$ has no improvement wrt to an intended usage $W$ then $T$ is said to be \textit{optimally formulated} wrt $W$.
\end{definition}
\end{frame}


\begin{frame}{Propositions}
\begin{itemize}
\item If a relational schema $R$ can be normalised to $R'$ then the associated theory $T$ of $R$ can be improved to the associated thery $T'$ of $R'$.

\item If a relational database schema is in normal form then its associated theory is optimally formulated.
\end{itemize}
\end{frame} 
\fi

\begin{frame}{Defining Candidate Keys and/or Identifying Relationships in an EA sketch.  }
\begin{itemize}
\item concept of \textit{candidate keys} used in relational database normal form definitions {\scriptsize (3NF, EKNF, BCNF)}
\item in ER model talk about \textit{identifying} families of relationships
\item in category theory such a key or a family of relationships is a mono source i.e. a to jointly monic family of morphisms
\item mono sources and hence candidate keys can be defined as limit cones
\item more than 99.99 percent of entity modelling uses just mono sources and no other limits
\end{itemize}
\end{frame}

\begin{frame}{Additional Structure}
\resizebox{11.3cm}{!}{
\newcommand{\featurelist}{\begin{tabular}{|l|l l|}
\hline 
\multirow{11}{1.5cm}{category}
                & finitary property        & \\
\cline{2-3}
                & pu-partition             & \\
\cline{2-3}
                & \multirow{2}{3.5cm}{mono-sources}  & \multicolumn{1}{|l|}{cannonical monos}  \\
\cline{3-3}
                &                                    & \multicolumn{1}{|l|}{epi-mono factorisation}   \\
\cline{2-3}
                & finite products          & \\
\cline{2-3} 
                & finite limits            & \\
\cline{2-3}
                & restrictions             & \\
\cline{2-3}
                & \multirow{2}{3.5cm}{distinguished morphisms} & \multicolumn{1}{|l|}{hierarchical}      \\
\cline{3-3}
                &                                             &  \multicolumn{1}{|l|}{non-hierarchical} \\
\cline{3-3}
                &                                             &  \multicolumn{1}{|l|}{pullbacks} \\
\cline{2-3}
                & finite coproducts                           &                                   \\
\hline                
\end{tabular}}
\featurelist
}
\end{frame}

\begin{frame}{How to proceed}
\begin{itemize}
\pause \item I will
\begin{itemize}
   \item give an example of nested tables of data
   \item describe relational model  and other data models 
   \pause \item touch on my favourite -- ER modelling and ER script
   \pause \item describe Boyce-Codd normal form (BCNF)
   \pause \item GCs for sketches of categories as data specifications 
   \pause \item GCs for sketches of categories with designated monos and epis
   \pause \item GCs for sketches of categories with designated monos and epis and with finite products
  \pause \item with some additional assumptions prove BCNF
\end{itemize}
\end{itemize}
\end{frame}

