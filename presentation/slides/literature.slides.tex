
\begin{frame}{Categories and Data Specifications}
\begin{itemize} \footnotesize
\item  \cite{Johnson93} 
\begin{itemize} 
\item describe a category equivalent to an ER schema which they loosely describe as the classifying category of the schema
\item argue that it is/needs to be a lextensive category
\item those objects of the category which are coproducts of morphisms from terminal object represent the attribute type (though they use the term attribute here).
\end{itemize}
\pause \item developed further in \cite{johnson2002REL} and \cite{Johnson2002ERA}
\begin{itemize} \footnotesize
\pause \item The term `EA-sketch' introduced for a sketch representing an ER model 
\pause \item comparison with the relational model of data and with categorical data specifications\cite{piessens1995}
\end{itemize}
\pause \item Cadish and Diskin wrote a preprint with subtitle a Manifesto of Categorizing DataBase Theory in 1995.
\end{itemize}
\end{frame}

\begin{frame}{\cite{Johnson2002ERA}}
\bibentry{Johnson2002ERA}

``An EA sketch E=(G,D,L,C) is a sketch with only finite cones and finite discrete cocones and with a 
specified cone with empty base whose vertex is called 1. Edges with domain 1 are called elements. 
Nodes which are vertices of cocones all of whose injections are elements are called attributes. 
Nodes which are neither attributes, nor 1, are called entitites.''
\end{frame}



\begin{frame}{Category of Partial Maps}
Formalised by Cockett and Lack as 'Restriction Categories'.
There is a locally ordered 2-category associated with a restriction category.
Many diagrams commute upto $\leq$.
Outer Join is pullback.
Inner Join is 2-pullback.
\end{frame}

\begin{frame}{Bibliography}
{\tiny
\bibliography{../../SharedBibliography/temp/bibliography}
}
\end{frame}

