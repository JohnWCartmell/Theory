


\begin{frame}{E.F. Codd and the Relational Model of Data}
The following dedication is writ large in Codd's 1990 book:
\begin{quote}
To fellow pilots and aircrew
in the Royal Air Force
during World War II
and the dons at Oxford.
These people were the source of my determination to
fight for what I believed was right during the ten or
more years in which government, industry, and
commerce were strongly opposed to the relational
approach to database management.
\end{quote}
By 2020 Oracle Corporation were the world's second largest software company.
\end{frame}
\begin{frame}{E.F. Codd and the Relational Model of Data}
\displaybibentry{CoddBook1990}
\begin{quote}
The relational model is solidly based on two parts of mathematics: first-
order predicate logic and the theory of relations.
\end{quote} 
\end{frame}

\begin{frame}{E.F. Codd 1970}
\displaybibentry{Codd1970}
Codd introduces the relational model of data and introduces the idea of normal form.
\end{frame}

\begin{frame}{E.F. Codd 1971}
\displaybibentry{Codd1971} 
In 1971, the terms `functional dependency' and  `third normal form' are introduced 
in an IBM techical report published in a now out of print book (also unavailable on Amazon).
\end{frame}

\begin{frame}{Fagin 1977,1979}
\displaybibentry{Fagin1977} 
Fagin introduces `fourth normal form' (4NF) and `multivalued dependencies'.

\displaybibentry{Fagin1979} 
Fagin introduced `projection-join normal form'. This is also known as fifth normal form (5NF).
\end{frame}

\begin{frame}{Zaniola 1962}
\displaybibentry{zaniolo1982}
\begin{itemize}
\item resume of 3NF and BCNF
\item the representation principle (by example)
\item new normal form -- elementary key normal form (EKNF)
\item Bernsteins algorthm, which is known to produce schemas in 3NF, does actually produce EKNF
\end{itemize}
\end{frame}

\begin{frame}{Cartmell 1986}
\displaybibentry{CartmellNetworkDataModel}
\end{frame}

\begin{frame}{Cartmell \& Alderson 1997}
\displaybibentry{CartmellScopePaper}
\end{frame}

\begin{frame}{Categorical Data Specifications 1995}
\displaybibentry{piessens1995} 
Defines data specifications and also MD-sketches.
\end{frame}

\begin{frame}{Diskin and Cadish 1996}
\displaybibentry{Diskin1996DatabaseDesign}
\begin{quote}
...it seems for us that
the current situation in relating category theory with DB theory and practice is very similar to the 16th
century interaction of differential and integral calculi, on one hand, with mechanics and engineering on
the other. Thus, being excited with our discovery, and having in mind the distinctive features of our time
... we have decided to begin propagating our observations with a declarative document in
a manner of a brief manifesto.
\end{quote}
\end{frame}

\begin{frame}{Johnson 1993}
\displaybibentry{Johnson93}
\begin{itemize} \footnotesize
\item describe a category equivalent to an ER schema which they loosely describe as the classifying category of the schema
\item argue that it is/needs to be a lextensive category
\item those objects of the category which are coproducts of morphisms from terminal object represent the attribute type (though they use the term attribute here).
\end{itemize}
\end{frame}

\begin{frame}{Johnson 2001}
\displaybibentry{Johnson2001}

\begin{itemize} \footnotesize{}
\pause \item developed further in  \cite{Johnson2002ERA}
\pause \item The term `EA-sketch' introduced for a sketch representing an ER model 
\end{itemize}
\end{frame}

\begin{frame}{Definitions - Johnstone et al.}

%\begin{definition}{Johnstone et al}
An \textit{EA sketch} is a sketch $\tuple{G,D,L,C}$ where $G$ is a directed graph, $D$ a set of diagrams in $G$, $L$ a set of finite cones and
$C$ a set of finite discrete cocones.
%\end{definition}

If $S$ is an EA-sketch then the theory of $S$ is the lextensive category generated by $S$.

If $S$ is an EA sketch then a model of $S$ is a functor to the category of finite sets preserving finite limits and coproducts.
The category of models is denoted $Mod(S,\cat{FinSet})$.
\end{frame}

\begin{frame}{\cite{Johnson2002ERA}}
\bibentry{Johnson2002ERA}

``An EA sketch E=(G,D,L,C) is a sketch with only finite cones and finite discrete cocones and with a 
specified cone with empty base whose vertex is called 1. Edges with domain 1 are called elements. 
Nodes which are vertices of cocones all of whose injections are elements are called attributes. 
Nodes which are neither attributes, nor 1, are called entitites.''
\end{frame}

\begin{frame}{Category of Partial Maps}
Formalised by Cockett and Lack as 'Restriction Categories'.
There is a locally ordered 2-category associated with a restriction category.
Many diagrams commute upto $\leq$.
Outer Join is pullback.
Inner Join is 2-pullback.
\end{frame}

\begin{frame}{Bibliography}
{\tiny
\bibliography{../../SharedBibliography/temp/bibliography}
}
\end{frame}

