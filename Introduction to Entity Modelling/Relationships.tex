
\section{Relationships}
\label{Relationships}
\mynote
A defining feature of entity modelling is the description
 (the modelling) of the relationships that may or may not  exist between things according as to the types of things that they are. 
\mynote 
We need to say something about this term `relationship' as it is used in
entity modelling.
As a starting point consider these two definitions that I found online. 

\begin{erquote}
a relationship between two things is the way in which they are connected
\end{erquote}
and 
\begin{erquote}
a relationship is a connection, association, or involvement between two things.
\end{erquote}

At first reading these two dictionary entries seem to be in accord and indeed 
they both describe what would more specifically be called \textit{binary relationships} involving as they do exactly two things in relationship with each other.
A more careful reading reveals a significant difference between the two because \textit{friendship} as an abstraction satisfies the first definition but not the second whereas a particular friendship such as that between \textit{Don Quixote} and \textit{Sancho Panza} satisfies the second  but not the first. 
To distinguish the two meanings some authors use the term \textit{relationship type} for the first sense and reserve the term \textit{relationship} itself for the second. Here we will stick with the \commentary{re-jiggle}single word relationship, which, for us, can therefore have two distinct meanings\footnote{interpretations?}. If we need to emphasise the second of the two we may speak of instances of relationships rather than relationships. When speaking in this way we might say that there is a relationship called `friendship' between people and that there is an instance of this relationship between \textit{Don Quixote} and \textit{Sancho Panza}.

\mynote
Just for the record, in another dictionary we find 
\begin{erquote}
In logic and mathematics relationship is another name for relation.
\end{erquote}
and in yet another there is a definition of mathematical relation as a subset of the cartesian product of two sets 
(evidently, therefore, a definition of a binary relation)\footnote{Chambers definition here}\footnote{Need to say more here because this latter definition is a relation in a model of logic not in logic itself}. 

\mynote
Friendship,  marriage, adjacency, ownership and succession are all examples of binary relationships. There are binary relationships too  between parent and child, between teacher and student and between a planet and the star that it orbits.   
The following are also binary relationships:
being employed by, in the sense that a person might be employed by a particular employee,
managing, in the sense that an employee might manage a project. 
There are binary relationships too between a sales order and the customer making the order and between a debit card and  the bank that issues the card. As a final example, a banking transaction has binary relationships both with the account debited and with the account credited. 

\mynote The most important feature of relationships as the term is used  in this book is that the existence of a relationship instance between any two particular entities should be a matter of black and white. For example between any particular Jack and any particular Jill we must take it that there either is or is not a friendship between them. There can be no equivocation or matter of degree --- either the relationship is or is not. This is a point of difference, by the way, between our terminology and that of Chen --- Chen's relationships may have attributes (properties) assigned to them, our relationships may not which is why they are, mathematically, binary relations. We will see that such (black and white) binary relationships are all we need for modelling and that many-many relationships are not needed either. 

\mynote 
Every binary relationship can be described from two different points of view. 
For the planet-star relationship the two points of view are
\begin{itemize}
\item every planet orbits a star,
\item every star may be orbited by planets.
\end{itemize}

In these descriptions the language is slightly ambiguous and questions arise.
Does every planet orbit exactly one star or may it orbit multiple stars? Are there  stars having no planets orbiting them?

In an entity model ambiguities such as these are removed. 
The planet-star relationship would be expressed in more detail as
\begin{itemize}
\item every planet must be orbiting exactly one star,\footnote{You may disagree with this if your cosmology is more nuanced than mine.}
\item every star may be orbited by one or more planets.
\end{itemize}

Note that these two prescriptions describe the very same relationship from opposite ends. 
They are complementary points of view, a fact which  can be expressed like this:
\begin{itemize}
\item a planet is orbiting a star if and only if the star is orbited by the planet.
\end{itemize}
\subsection*{Chen's Notation}
\mynote
Chen introduces diagrammatic representations of entities and relationships in his 
1976 paper. There is a great deal to find in this paper and it has been very influential and therefore is historically important but his diagram style in many peoples opinion is a poor notation and less practical than other notations which were subsequently published many of which were then summarised by Chen himself in his 1983 paper. 
Of these, our preferred notation is the Barker-Ellis notation and it is a variation on this that is used in this book. 
Because it is frequently seen, at least academically, it is appropriate
to describe the Chen notation for relationships and then to contrast with the Barker-Ellis style.
\mynote
In his introduction to entity relationship diagramming, Chen gives an example of an entity relationship diagram describing the types of entities relevant to the organisation of an imaginary manufacturing company.
In it he gives the example of a relationship between an employee and the department in which they work. In his diamond notation Chen draws the relationship like this:
\begin{center}
\scalebox{0.9}{\input{\handCraftedImagesFolder/ChenStyleDepartmentEmployee.tex}}
\end{center}

\noindent In this diagram, the numeric $1$ that labels the left hand line in this diagram indicates that every employee is in exactly one department and the corresponding annotation N on the right side indicates the fact that there are many employees within each department. \\

\noindent This relationship can therefore be described from  two complementary points of view as
\begin{align}
\label{employingText}&\mbox{every department may be employing one or more employees,} \\
\label{employedByText}&\mbox{every employee must be employed by exactly one department.}
\end{align}


\subsection*{Barker-Ellis Notation}
\noindent In the notation used in this book, which is based on the Barker-Ellis notation,  this diamond and its two connectors are replaced by a single line. The fact that there
are multiple employees within a department is represented by drawing a crows foot at the employee end of the line. The relationship is drawn like so:
\definecolor{lightyellow}{cmyk}{0,0,0,0}

\begin{gather}
\label{employsRelationshipDiagram}
\raisebox{-1cm}{\begin{erdiagram}{2.3}{7.999999999999999}

\eret{0.1}{-1.9}{2.4}{-0.4}{0.2}{1}\eretname{1.25}{-0.75}{}{}\eretname{1.25}{-1.05}{}{DEPARTMENT}
\eret{5.7}{-1.9}{8}{-0.4}{0.2}{1}\eretname{6.85}{-0.75}{}{}\eretname{6.85}{-1.05}{}{EMPLOYEE}

% relationship employing
\errelname{2.55}{-1}{l}{employing}\errelname{5.55}{-1.45}{r}{employed by}\errelarm{2.4}{-1.15}{4.05}{-1.15}{0}{0}\errelarm{4.05}{-1.15}{5.699}{-1.15}{1}{0}\ercrowfoot{5.55}{-1.15}{5.7}{-1}{5.7}{-1.15}{5.7}{-1.3}{0}
\end{erdiagram}
}
\end{gather}

\noindent In this form the relationship can be read directionally
 --- left to right it
reads as  (\ref{employingText}), above,  
right to left it reads as (\ref{employedByText}). 

\noindent
We can think of there being two directional relationships here inseparably
connected (meta-related, in fact) as the poles of a magnet.
The right to left reading expresses a many-one directional relationship and this is otherwise known as a functional relationship.

 
\subsection*{Recap -- Relationships Looked at Directionally}

\noindent
To be clear, every binary relationship can be thought of as consisting of a pair of directional relationships each one of which is one view of it. \commentary{subsection rejig required}
\mynote Strictly speaking, directional relationships are not, as their name suggests, types of relationships, but are ways of looking at or considering binary relationships. Each directional relationship is a binary relationship viewed at or considered from one end or the other. Each binary relationship can be considered as a directional relationship in two different ways.
This is how binary relationships are described in the entity modelling meta-model given later.

\mynote A directional relationship which is many-one is said to be a functional relationship.
In mathematics we also talk of functional relationships but also just say function.
\mynote
In mathematical notation this would indicate a function and be 
represented by an arrow: 
\begin{equation}
\label{employedByFunction}
employed\,by : employee \longrightarrow department
\end{equation}
\begin{noteforfuture}
give planet and star and maybe other examples here.
\end{noteforfuture}

\subsection*{Mathematical Inverse}
\mynote Mathematically speaking, every function has an inverse which generally is many-valued and which therefore is represented as a separate function whose target is a power set i.e. a set of subsets. In our case here the inverse to the function \textit{employed by} will in some mathematical circles by represented by a function
\begin{equation}
\label{employingFunction}
employing : department \longrightarrow \mathcal{P}(employee).\footnote{Where $\mathcal{P}$ represents power set so that $\mathcal{P}(x)$ is the set of subsets of set $x$.}
\end{equation}

\noindent Let me recap this illustration. The function signatures (\ref{employedByFunction}) and (\ref{employingFunction}) express the same 
information as the words do in (\ref{employedByText}) and (\ref{employingText}).
Diagram (\ref{employsRelationshipDiagram}) encapsulates the same information and also expresses  the fact that (\ref{employingText}) and (\ref{employedByText}) are complementary 
i.e. that an employee works for a particular department if an only if that particular department employs the employee.
Finally, this complementarity may be expressed in the mathematical language of
sets and functions by asserting  function (\ref{employingFunction}) to be the inverse to function (\ref{employedByFunction}).


\subsubsection*{Notations for Many-many Binary Relationships}
In Chen's paper the relationship between employees and projects (whose meaning is that
an employee is assigned to work on a project) is depicted like this.
\begin{gather}
\label{projectWorkerChenStyle}
\raisebox{-0.9cm}{\scalebox{0.9}{
\input{\handCraftedImagesFolder/ChenStyleProjectWorker.tex}
}}
\end{gather}
On this diagram there is an annotation $N$ which documents that an employee may work on many projects (we are to understand that $N >= 0$). There is also an annotation $M$ to document that
a project has many employees assigned to it, $M$ in fact, for some $M >= 0$. \\
\noindent Such a relationship is called a many-many relationship.  \\
\noindent In Barker-Ellis notation which is the basis for the notation used in this book the diamond and its two connectors are replaced by a single line. The fact that the relationship is many-many is represented in the diagram by having a crows foot at each end of the line like so:

\begin{gather}
\label{employeeProjectWorker}
\raisebox{-1cm}{
\begin{erdiagram}{2.3}{7.4}

\eret{0.1}{-1.9}{2.2}{-0.4}{0.2}{1}\eretname{1.15}{-0.75}{}{}\eretname{1.15}{-1.05}{}{EMPLOYEE}
\eret{5.3}{-1.9}{7.4}{-0.4}{0.2}{1}\eretname{6.35}{-0.75}{}{}\eretname{6.35}{-1.05}{}{PROJECT}

% relationship assigned to
\errelname{2.35}{-1}{l}{assigned to}\errelname{5.15}{-1.45}{r}{assigned}\errelarm{2.2}{-1.15}{3.75}{-1.15}{0}{0}\errelarm{3.75}{-1.15}{5.3}{-1.15}{0}{0}\ercrowfoot{5.15}{-1.15}{5.3}{-1}{5.3}{-1.15}{5.3}{-1.3}{0}\ercrowfoot{2.35}{-1.15}{2.2}{-1}{2.2}{-1.15}{2.2}{-1.3}{0}
\end{erdiagram}

}
\end{gather}

\noindent As before, the relationship is labelled at both ends. Each text label gives the name of the relationship from the point of view of the type of entity at that end of the line.

\subsection*{Elimination of Many-many Relationships}
\noindent In fact, many-many relationships are infrequently used in entity models and never used at all when modelling for the purpose of data specification.
Consider that we can make a thing out of the fact of the assignment of an employee to a project. This means that we can entify the many-many relationship and remodel the relationship (\ref{employeeProjectWorker}) like this:\commentary{squeeze this diagram and flip labels}
\begin{gather}
\label{employeeProjectWorkerMediated}
\raisebox{-1cm}{
\begin{erdiagram}{1.1}{10.974499999999999}

\eret{0.1}{-1}{1.6}{-0.4}{0.2}{1}\eretname{0.85}{-0.75}{}{employee}
\eret{4.1}{-1}{6.975}{-0.4}{0.2}{1}\eretname{5.537}{-0.75}{}{project assignment}
\eret{9.474}{-1}{10.974}{-0.4}{0.2}{1}\eretname{10.224}{-0.75}{}{project}

% relationship assigning
\errelname{3.95}{-0.55}{r}{assigning}\errelname{1.75}{-1}{l}{subject of}\errelarm{4.1}{-0.7}{2.849}{-0.7}{1}{0}\errelarm{2.849}{-0.7}{1.6}{-0.7}{0}{0}\ercrowfoot{3.95}{-0.7}{4.1}{-0.55}{4.1}{-0.7}{4.1}{-0.85}{0}
% relationship to work on
\errelname{7.125}{-0.55}{l}{to work on}\errelname{9.324}{-1}{r}{resourced by}\errelarm{6.974}{-0.7}{8.224}{-0.7}{1}{0}\errelarm{8.224}{-0.7}{9.474}{-0.7}{0}{0}\ercrowfoot{7.125}{-0.7}{6.975}{-0.55}{6.975}{-0.7}{6.975}{-0.85}{0}
\end{erdiagram}

}
\end{gather}
\noindent
In (\ref{employeeProjectWorkerMediated}) the  many-many relationship depicted in (\ref{employeeProjectWorker}) is now mediated by the conjunction of  type
\textit{project assignment} and two functional relationships. Because we can `'entify' many-many relationships in this way and mediate them by entity types and functional relationships they can be eliminated from entity modelling. Only functional relationships have direct representations in data, however represented, and for this reason it is usual to eliminate many-many relationships from entity models in favour of functional relationships. Since the remaining relationships are functional, in one direction or the other, so it is that in most entity models types are related solely by functional relationships and their inverses. In this form, entity modelling models the world in functional relationships, a fact that isn't usually articulated.

\mynote For the sake of completeness, I should mention that in situations
such as (\ref{employeeProjectWorkerMediated}) above, the mediating entity or indeed its type is often referred to as an \textit{intersection entity}.

\subsection*{Mathematical Notation}
\mynote
Before I describe how, in the Barker-Ellis style, the representations of functional relationships are styled to give additional information about them
 it is appropriate to bring to mind corresponding mathematical notations.\footnote{This book promotes my favourite notation but I think it important to describe other notations as well especially those used in mathematics.}

\mynote Functional relationships in mathematics that are total are usually depicted by the arrow symbol $\longrightarrow$ so the assertion that $f:A \longrightarrow B$ in mathematics generally means that $f$ is a functional relationship from $A$ to $B$ which is defined everywhere i.e. is total.\footnote{The more abstract use of $\longrightarrow$ in category theory is a notable exception and is entirely germane as we will discuss later.} 
For a functional relationship that is not known to be total 
then a partial arrow symbol $\longrightharpoonup$ is used instead so in this case it might be asserted that $f:A \longrightharpoonup B$.
Finally, the notation $f: A \longtwoheadrightarrow B$ can be used to assert that $f$ is a total functional relationship that is surjective.  

\noindent All of these notations have equivalents in the Barker-Ellis style, as we will now see.

\subsection*{Barker-Ellis Styling of Relationships}
\mynote We have already seen above examples of Barker-Ellis style depictions of relationships in which functional relationships of A to B
are shown like this:
\begin{center}
A\,\barkerEllisA\,B
\end{center}
\noindent for example as here
\begin{center}
\begin{erdiagram}{1}{5.744000000000001}

\eret{0.1}{-0.8}{1.532}{-0.2}{0.2}{1}\eretname{0.816}{-0.55}{}{student}
\eret{4.032}{-0.8}{5.744}{-0.2}{0.2}{1}\eretname{4.888}{-0.55}{}{professor}

% relationship advised by
\errelname{1.682}{-0.35}{l}{advised by}\errelname{3.882}{-0.8}{r}{advises}\errelarm{1.531}{-0.5}{1.781}{-0.5}{0}{0}\errelarm{1.781}{-0.5}{2.781}{-0.5}{0}{0}\errelarm{2.781}{-0.5}{3.781}{-0.5}{0}{0}\errelarm{3.781}{-0.5}{4.031}{-0.5}{0}{0}\ercrowfoot{1.682}{-0.5}{1.532}{-0.35}{1.532}{-0.5}{1.532}{-0.65}{0}
\end{erdiagram}

\end{center}
If however a functional relationship is known to be total then the left half line is drawn as a solid line instead of a dashed line like so \barkerEllisC
% \begin{center}
% A\,\barkerEllisC\,B
% \end{center}
for example as here 
\begin{center}
\begin{erdiagram}{1.5499999999999998}{5.884250000000001}

\eret{0.1}{-0.95}{1.532}{-0.35}{0.2}{1}\eretname{0.816}{-0.7}{}{student}
\eret{4.032}{-0.95}{5.884}{-0.35}{0.2}{1}\eretname{4.958}{-0.7}{}{department}

% relationship majoring_in
\errelname{1.682}{-0.5}{l}{in}\errelname{1.682}{-0.2}{l}{majoring}\errelname{3.882}{-0.95}{r}{providing}\errelname{3.882}{-1.25}{r}{specialization}\errelname{3.882}{-1.55}{r}{for}\errelarm{1.531}{-0.649}{1.781}{-0.649}{1}{0}\errelarm{1.781}{-0.649}{2.781}{-0.649}{1}{0}\errelarm{2.781}{-0.649}{3.781}{-0.649}{0}{0}\errelarm{3.781}{-0.649}{4.031}{-0.649}{0}{0}\ercrowfoot{1.682}{-0.65}{1.532}{-0.5}{1.532}{-0.65}{1.532}{-0.8}{0}
\end{erdiagram}

\end{center}
In either case if the map is known to be surjective then the right hand half line is solidified and looks as this \barkerEllisB or \barkerEllisD
% \begin{center}
% A\,\barkerEllisB\,B
% \end{center}
or this
% \begin{center}
% A\,\barkerEllisD\,B
% \end{center}
as for example in the relationship between a professor and the department they are a part of 
\begin{center}
\begin{erdiagram}{1}{6.164750000000001}

\eret{0.1}{-0.8}{1.812}{-0.2}{0.2}{1}\eretname{0.956}{-0.55}{}{professor}
\eret{4.312}{-0.8}{6.165}{-0.2}{0.2}{1}\eretname{5.239}{-0.55}{}{department}

% relationship on staff of
\errelname{1.962}{-0.35}{l}{on staff of}\errelname{4.162}{-0.8}{r}{staffed by}\errelarm{1.812}{-0.5}{2.062}{-0.5}{1}{0}\errelarm{2.062}{-0.5}{3.062}{-0.5}{1}{0}\errelarm{3.062}{-0.5}{4.062}{-0.5}{1}{0}\errelarm{4.062}{-0.5}{4.312}{-0.5}{1}{0}\ercrowfoot{1.962}{-0.5}{1.812}{-0.35}{1.812}{-0.5}{1.812}{-0.65}{0}
\end{erdiagram}

\end{center}
\footnote{In these examples we have assumed (and then documented) that departments \textbf{must} have professors and \textbf{may} have students.}

If in any of these cases it is known in addition  that the map is injective i.e. that two distinct instances of the A type cannot be related to the vary same instance of the B type then the crowsfoot is omitted and so the relationship is depicted by a line that
may be dashed or half dashed as the case might be, the possibilities being:
A\,\barkerEllisE\,B, 
A\,\barkerEllisF\,B, 
A\,\barkerEllisG\,B or 
A\,\barkerEllisH\,B.

\commentary{maybe omit these four depictions}
Note that in this final case the relationship in question establishes a 1-1 correspondence between types A and B.

\begin{noteforfuture}
have an example of some sort here with a variety of cardinalities or refer forward to the manufacturing example?
\end{noteforfuture}

\subsection*{Diagramming Relationships}
The next thing to point out is that multiple relationships can be depicted on a single diagram and the relationship lines can be routed about as convenient for the drawing as here:
\begin{center}
\begin{erdiagram}{3.3}{7.09825}

\eret{2.6}{-0.7}{4.453}{-0.1}{0.2}{1}\eretname{3.526}{-0.45}{}{department}
\eret{0.704}{-2.8}{2.136}{-2.2}{0.2}{1}\eretname{1.42}{-2.55}{}{student}
\eret{4.636}{-2.8}{6.348}{-2.2}{0.2}{1}\eretname{5.492}{-2.55}{}{professor}

% relationship provides specialization for
\errelname{3.068}{-1}{r}{provides specialization for}\errelname{1.27}{-2.05}{r}{majors in}\errelarm{3.217}{-0.7}{3.217}{-0.899}{0}{0}\errelarm{3.217}{-0.899}{3.217}{-1.1}{0}{0}\errelarm{3.217}{-1.1}{2.318}{-1.525}{0}{0}\errelarm{2.318}{-1.525}{1.42}{-1.949}{1}{0}\errelarm{1.42}{-1.949}{1.42}{-2.074}{1}{0}\errelarm{1.42}{-2.074}{1.42}{-2.199}{1}{0}\ercrowfoot{1.42}{-2.05}{1.27}{-2.2}{1.42}{-2.2}{1.57}{-2.2}{0}
% relationship staffed by
\errelname{3.985}{-1}{l}{staffed by}\errelname{5.642}{-2.05}{l}{on staff of}\errelarm{3.835}{-0.7}{3.835}{-0.899}{1}{0}\errelarm{3.835}{-0.899}{3.835}{-1.1}{1}{0}\errelarm{3.835}{-1.1}{4.663}{-1.525}{1}{0}\errelarm{4.663}{-1.525}{5.492}{-1.949}{1}{0}\errelarm{5.492}{-1.949}{5.492}{-2.074}{1}{0}\errelarm{5.492}{-2.074}{5.492}{-2.199}{1}{0}\ercrowfoot{5.492}{-2.05}{5.342}{-2.2}{5.492}{-2.2}{5.642}{-2.2}{0}
% relationship advised by
\errelname{2.286}{-2.35}{l}{advised by}\errelname{4.486}{-2.8}{r}{advises}\errelarm{2.136}{-2.499}{2.386}{-2.499}{0}{0}\errelarm{2.386}{-2.499}{3.386}{-2.499}{0}{0}\errelarm{3.386}{-2.499}{4.386}{-2.499}{0}{0}\errelarm{4.386}{-2.499}{4.636}{-2.499}{0}{0}\ercrowfoot{2.286}{-2.5}{2.136}{-2.35}{2.136}{-2.5}{2.136}{-2.65}{0}
\end{erdiagram}

\end{center}

This diagram is then said to be an entity relationship diagram, in this case in the Barker Ellis style. In this book there will be many of these smaller examples but we can imagine most of these to be fragments of some larger diagram which is the backbone to an entity model.
In a later section Structured Entity Modelling we will describe additional meanings that we ascribe to relationships dependent on their orientations on the diagram. These are then the ascribed meanings in the remainder of this book. 

\subsection*{Ternary Relationships}
For the sake of completeness I should mention that Chen also represents ternary relationships and gives an 
example of a ternary relationship between a project, a supplier and a part which he shows like this:


\begin{center}
\scalebox{0.9}{\input{\handCraftedImagesFolder/ChenStyleSupplierProjectPart.tex}}
\end{center}

In the Barker-Ellis style there are no ternary relationships nor are any needed since ternary relationships can be represented as a combination of binary relationships and entity types. The fact that all relationships are binary and are represented by lines only rather than diamonds  is of practical importance in that more details can be fitted onto diagrams.

\mynote
When it comes to describing how this ternary relationship of Chen's can be represented in terms of binary many-one relationships then I have to guess a bit at what Chen intended this ternary relationship to mean in the context of his hypothetical manufacturing company. 
Having made my guess I have represented this ternary relationship by type \textit{project supply option} 
in the Barker-Ellis style diagram given  in figure \ref{ChenManufacturingBarkerEllisStyle} in the next section. I hope that you agree that this Barker-Ellis style diagram leaves less to the imagination than Chen's diagram did. 

\subsection*{Phraseology}
\mynote 
We don't have a simple name for the relationship between a planet and the star it orbits. Instead we have  relative terms `orbits' and `orbited by' which describe each of the parties in the relationship from the point of view of the other. This dual naming of binary relationships is the way that relationships are named in an entity model. 
This is where a diagram comes in handy. 
\mynote
From this naming and diagramming of relationships there is a phraseology i.e. a particular mode of expression,  described, for example, by Barker in his book. Two phrases can be generated to describe a relationship. The grammatically of the phrase generated depends on a carefully choice of label for each end of the relationship. \commentary{Spell this out.}
