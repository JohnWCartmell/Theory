
\section{Communicating Relationships}
\subsection{Recap}
\mynote
Earlier, in section \ref{sub:conveying_facts_and_storing_of_data}, we looked at examples of the use
of references to entitites, and thereby referentials,  
to convey instances of relationships.  
On the surface of it it is very simple ---
to convey instances of relationships we reference two entities and assert a relationship between them
as in this example:

\begin{equation}
\label{JupiterIoPernickity}
\mbox{\textit{The planet \rdash{Jupiter} 
\uwave{is orbited} by the moon \rdot{Io}}}
\end{equation}

in which a single referential references one entity, a second referential references a second entity
and a relationship instance is asserted.  
In this simple example each of these references to entities contains  a single referential 
but we have already seen examples where multiple referentials are required to reference an entity. For example
\begin{itemize}
\item Hot Springs, Arkansas;
\item the route from Hot Springs, Arkansas to Jacksonville, Alabahma;
\item the conjunction of Mars and Jupiter on July 4, 929;
\item Shakespeare’s Twelfth Night (i.e. the play Twelfth Night by playwright William Shakespeare);
\item Touchstone from Shakespeare’s As You Like It;
\item the role of Touchstone in the production of Shakespeare’s As You Like It
performed April – May 1975, at Oxford Playhouse.
\end{itemize}

We have seen also how in some references, for example the final one in the list above,  there are less referentials than is implied by a simple sum of the parts  and that this is because of overlaps (for which we used the term collapsed referentials) and that these overlaps come about because of the equivalence of comparable paths of identifying relationships.

For example an instance of a dramatic role being played by an actor might be commmunicated as follows:
\begin{erquote}
\parbox{8.7cm}{\linespread{1.5}\normalsize the role of \rdot{Touchstone} in the production of \mbox{\rdot{Shakespeare}'s} \\
 \rdot{As You Like It} performed \mbox{\rdot{April – May 1975},} at \\
\rdot{Oxford Playhouse} was \uwave{played by} \rdash{Bob Hoskins},
}
\end{erquote}
which, by the way, is a true fact. 
In this example, referentials to reference the dramatic role are dotted underlined --- there are five of them.
 The relationship is wavy underlined. Referentials to reference the actor are dash underlined   --- there is only one of them.
 The relationship being instanced is this one
\begin{equation}
\scalebox{0.9}{\begin{erdiagram}{1.72}{6}

\eret{0.1}{-1.22}{1.6}{-0.3}{0.2}{1}\eretname{0.85}{-0.66}{}{dramatic}\eretname{0.85}{-0.96}{}{role}
\eret{4.5}{-1.22}{6}{-0.3}{0.2}{1}\eretname{5.25}{-0.81}{}{actor}

% relationship played_by
\errelname{1.75}{-0.61}{l}{by}\errelname{1.75}{-0.31}{l}{played}\errelname{4.35}{-1.06}{r}{the}\errelname{4.35}{-1.36}{r}{player}\errelname{4.35}{-1.66}{r}{of}\errelarm{1.6}{-0.76}{3.05}{-0.76}{1}{0}\errelarm{3.05}{-0.76}{4.5}{-0.76}{0}{0}\errelid{3.05}{-0.85}{}{r5}\ercrowfoot{1.75}{-0.76}{1.6}{-0.61}{1.6}{-0.76}{1.6}{-0.91}{0}
\end{erdiagram}
}
\end{equation}
which was described earlier in figure \ref{dramaticArts1..diagram}.

\subsection{What More is to be Said}
\mynote
We see that whenever a relationship instance is to be conveyed two sets of referentials are required --- one to reference the first entity and the other to reference the second entity.  
\mynote
Now we have to consider the possibility that sometimes, \textit{a priori}, these two sets of referentials overlap. 
In other words that, \textit{a priori}, some of the referentials for the two distinct entities coincide.
Shlaer and Long, in a very much underrated paper, 
explore this topic in the context of relational database structure and refer to the then shared coincident referential as a \textit{collapsed referential}. 
\mynote
We look at cases where there are collapsed referentials, and also at some other cases which exhibit different characteristics by which 
referentials that one might expect to be present are simply not needed --- in such cases we shall 
adapt the coinage of Shlaer and Long and say that there are \textit{absent referentials}.
Just as we saw in the previous section, we see different characteristics\commentary{better word than behaviour} emerge\commentary{exhibited?} in the presence of comparable paths of relationships
which are either equivalent or near equivalent.

\mynote
This is an important topic that requires careful artriculation
and without knowledge of this topic one has not a full understanding
of data both as it is conveyed in common langauge and as it is communicated and stored in information systems.

\subsection{An Example in which Referentials Collapse}
\mynote 
To illustrate the discussion that follows, 
imagine we model the plots of plays and in doing so consider \commentary{globalise this somehow, summarise/compare the plots of many plays}
love, reciprocated or not, to be a many-many relationship between characters like so
\begin{equation}
\label{dramaticArtsLoves..relationship}
\scalebox{0.9}{\begin{erdiagram}{1.32}{5.5}

\eret{0.2}{-1.12}{1.7}{-0.2}{0.2}{1}\eretname{0.95}{-0.71}{}{character}
\eret{4}{-1.12}{5.5}{-0.2}{0.2}{1}\eretname{4.75}{-0.71}{}{character }

% relationship in love_with
\errelname{1.85}{-0.51}{l}{with}\errelname{1.85}{-0.21}{l}{in love}\errelname{3.85}{-0.96}{r}{loved}\errelname{3.85}{-1.26}{r}{by}\errelarm{1.7}{-0.66}{2.85}{-0.66}{0}{0}\errelarm{2.85}{-0.66}{4}{-0.66}{0}{0}\errelid{2.85}{-0.75}{}{r2}\ercrowfoot{3.85}{-0.66}{4}{-0.51}{4}{-0.66}{4}{-0.81}{0}\ercrowfoot{1.85}{-0.66}{1.7}{-0.51}{1.7}{-0.66}{1.7}{-0.81}{0}
\end{erdiagram}
}
\end{equation}
and consider how instances of this relationship might properly be conveyed.
\mynote
As we have said above, and according to the model we have assumed (see figure \ref{dramaticArts1..diagram}),
  characters from plays need be identified by three referentials
 --- the name of the character, the title of the play and the name of the playwright.
With this being so, it naively implies that 
if we wish to communicate an instance of the relationship, 
(\ref{dramaticArtsLoves..relationship}), 
then we will be required to communicate six referentials
 -- three to identify one party in the relationship and three to identify the other party;
so, if I follow this approach, I would convey a particular relationship instance like this:
\begin{equation}
\text{\parbox{9cm}{\textit{The character named Antonio in Shakespeare's play Twelfth Night loves the character named Sebastian in Shakespeare play Twelfth Night.}}}
\end{equation} 
But there is a reason why the shorter
\begin{equation}
\text{\parbox{9cm}{\textit{The character named Antonio in Shakespeare's play Twelfth Night loves the character named Sebastian.}}}
\end{equation} 
is correct and it is  important we can understannd and articulate this reason.

The reason is because the relationship `loves' in this context 
is, \textit{a priori}, a relationship which is internal to the plot of a single individual play ---
it is an intra-play relationship rather than an inter-play relationship.
Because it is intra-play, when we convey an instance then the referentials that refer to the play are shared between lover and loved As shown in figure \ref{SebastianLovesAntonio}. The title of the play ad the name of the playwright are coincident. 

\begin{erboxedFigure}{H}{SebastianLovesAntonio}
{Referencing two different characters using just four referentials. This is an example of what Shlaer and Lang (?) called collapsed referentials. What we see is that referential `Shakespeare' contributes to two different references. Likewise does the referential `Twelfth Night'.
These are examples of shared referentals --- they contribute, as indicated, to both reference 1 and reference 2. The naive requirement for six referentials collapses to there being four.}
\newcommand{\dashRefOne}{2pt 2pt}
\newcommand{\dashRelationship}{1pt 0pt}
\newcommand{\dashRefTwo}{1pt 1pt}
\begin{tabular}{l}
\Rnode{w1}{\rdash{Antonio}} in \Rnode{w2}{\rdot{\rdash{Shakepeare}}}'s \Rnode{w3}{\rdot{\rdash{Twelfth Night}}} \Rnode{w4}{\rline{loves}}  \Rnode{w5}{\rdot{Sebastian}} \\[1.4cm]
\kern1.2cm\Rnode{ref1}{\textit{reference 1}}
\kern0.75cm\Rnode{rel}{\textit{relationship}}
\kern0.6cm\Rnode{ref2}{\textit{reference 2}} \\[0.5cm]
\syntag{\dashRefOne}{ref1}{0.9}{w1}{0}
\syntag{\dashRefOne}{ref1}{0.9}{w2}{-0.2}
\syntag{\dashRefOne}{ref1}{0.9}{w3}{-0.2}
\syntag{\dashRelationship}{rel} {0.7} {w4}{0}
\syntag{\dashRefTwo}{ref2}{0.4}{w2}{0.2}
\syntag{\dashRefTwo}{ref2}{0.4}{w3}{0.3}
\syntag{\dashRefTwo}{ref2}{0.4}{w5}{0}
\end{tabular}
\end{erboxedFigure}

An instances of the `loves' relationship is conveyed by a total of four referentials
 not the six that might have been required
if the relationship were global in scope as would have been the case if a
 character from one play could love a character from an entirely different play.

\mynote 
Now we need to drill into this statement made above that `the loves relationship is intra-play rather than inter-play'. What that means is that whenever one character loves another character then the loved character is a character in the same play as the lover character.  We can say that more concisely in symbols
if we label the relationship between lover and loved as `r' and if we label the relationship between a character and the play  they are a character within as `d1'
then 
\begin{equation}
r \circ d1 \leq d1
\end{equation}

which is used as a shorthand for saying
\begin{quote}
for all instances x and y of type character such that $y \in r(x)$,
\begin{equation*}
d(x) = d(y)
\end{equation*}  
\end{quote}
In the sense given earlier the following triangle of relationships(in which d1 is depicted twice over)  is near-commutative
 \begin{equation*}
\scalebox{0.9}{\begin{erdiagram}{3.42}{5.41665}

\eret{2.2}{-0.7}{3.533}{-0.1}{0.2}{1}\eretname{2.867}{-0.45}{}{play}
\eret{0.117}{-3.22}{1.617}{-2.3}{0.2}{1}\eretname{0.867}{-2.81}{}{character}
\eret{3.917}{-3.22}{5.417}{-2.3}{0.2}{1}\eretname{4.667}{-2.81}{}{character }

% relationship about
\errelname{2.494}{-1}{r}{about}\errelname{0.717}{-2.15}{r}{in}\errelarm{2.644}{-0.7}{2.644}{-0.899}{0}{0}\errelarm{2.644}{-0.899}{2.644}{-1.1}{0}{0}\errelarm{2.644}{-1.1}{1.755}{-1.487}{0}{0}\errelarm{1.755}{-1.487}{0.866}{-1.874}{1}{0}\errelarm{0.866}{-1.874}{0.866}{-2.087}{1}{0}\errelarm{0.866}{-2.087}{0.866}{-2.3}{1}{0}\errelid{1.756}{-1.578}{}{d1}\eridcomprel{0.7666499999999999}{0.9666499999999999}{-2.05}\ercrowfoot{0.867}{-2.15}{0.717}{-2.3}{0.867}{-2.3}{1.017}{-2.3}{0}
% relationship about 
\errelname{3.239}{-1}{l}{about }\errelname{4.817}{-2.15}{l}{in }\errelarm{3.088}{-0.7}{3.088}{-0.899}{0}{0}\errelarm{3.088}{-0.899}{3.088}{-1.1}{0}{0}\errelarm{3.088}{-1.1}{3.877}{-1.487}{0}{0}\errelarm{3.877}{-1.487}{4.666}{-1.874}{1}{0}\errelarm{4.666}{-1.874}{4.666}{-2.087}{1}{0}\errelarm{4.666}{-2.087}{4.666}{-2.3}{1}{0}\errelid{3.878}{-1.578}{}{d1}\eridcomprel{4.56665}{4.766649999999999}{-2.05}
% relationship in love_with
\errelname{1.767}{-2.61}{l}{with}\errelname{1.767}{-2.31}{l}{in love}\errelname{3.767}{-3.06}{r}{loved}\errelname{3.767}{-3.36}{r}{by}\errelarm{1.616}{-2.76}{2.766}{-2.76}{0}{0}\errelarm{2.766}{-2.76}{3.916}{-2.76}{0}{0}\errelid{2.767}{-2.85}{}{r}\ercrowfoot{3.767}{-2.76}{3.917}{-2.61}{3.917}{-2.76}{3.917}{-2.91}{0}\ercrowfoot{1.767}{-2.76}{1.617}{-2.61}{1.617}{-2.76}{1.617}{-2.91}{0}
\end{erdiagram}
}
\end{equation*}

I shall say that this diagram provides the scope for relationship (\ref{dramaticArtsLoves..relationship}). 

We can reinforce what was said earlier, equivalent and near-equivalent paths
have  a bearing on how instances are communicated and for this reason it is important 
that they are documented in entity models. Without knowledge of these equivalences,
we cannot infer from the model how relationship instances can properly be communicated 
i.e. how communicated in a way that avoids redundancy of information,
nor can we tell, for that matter, how they can be best represented in data.  

We can reinforce the earlier point that equivalent and near-equivalent paths significantly impact how instances are communicated. For this reason, it is crucial to document them in entity models. Without an understanding of these equivalences, we cannot determine from the model how relationship instances should be communicated in a way that avoids redundancy. Moreover, we cannot establish how they should be optimally represented in data.


 \subsection{Referentials That Do Not Collapse Though They Could}
 An example of a relationship between characters that is global in scope 
 is the relationship of one character being modelled on or resembling or being inspired 
 by another character from the dramatic arts.
 To convey an instance of this relationship, being as the relationship is global in scope,  does indeed require
 six referentials --- three for the one party and three for the other --- as in this example:
\begin{erquote}
\parbox{9.0cm}{\linespread{1.3}\normalsize
\rdash{Romeo} from \rdash{Shakespeare}'s \rdash{Romeo and Juliet}
          \uwave{is modelled on} \rdot{Pyramus} from \rdot{Ovid}'s \rdot{Metamorphoses}.}
\end{erquote}
which, interpreted from the perspective of the dramatic arts that we have given above, does indeed contain six referentials. 
The first three, shown with dashed underlining, reference one character.
The verb of the sentence is the name of the relationship, and, 
the final three referentials, with dotted underlining, 
reference a second character from a different play.\footnote{I am bending the facts a bit here 
in that Metamophoses isn't strictly a play - its a bunch of stories.} 
There are no shared referentials.


\section{Discussion --- The Airport Gate Example}
\label{AirportGateExample}

\mynote 
This example is based on a message that I received on my phone on the day that I returned from a recent (at time of writing) holiday. The message read:
\begin{equation}
\label{LH2502PhoneMessage}
\text{\parbox{9cm}{\textit{
Your flight LH2502 from Munich to Manchester on 14 August 2024 at 15:55 will depart from gate L06.}}}
\end{equation}
Consider, this message contains multiple referentials and individually or combined these make reference to
multiple entities:
\begin{itemize}
	\item the flight number, LH2502, makes reference to an \textit{airline service},\footnote{which, from one point of view at least, is a little odd because the name flight number purports to reference a flight. It isn't so odd though because on any given day flight numbers do reference flights.}
	\item each airport name, Munich, respectively, Manchester, makes reference to an \textit{airport},
	\item the combination of flight number, LH2502, and date, 14 August 2024, make reference to a \textit{flight},
	\item the combination of the airport the flight is identified as being from, Munich, and the gate number, 
	L06, identifies a \textit{gate}.
\end{itemize}
\mynote 
These referentials, the names (Munich, Manchester), numbers (LH2502, L06) and the date (14 August 2024)
I can understand as the values of attributes of the various referenced entitites. 
The types of these entities (airline service, airport, flight and gate), the relationships between them and the attributes employed
(flight number, date of departure, airport name and gate number) I can arrange on a diagram like this:
\begin{equation}
\label{boardingGate1}
\raisebox{-1.5cm}{\begin{erdiagram}{3.8}{9.100000000000001}

\eret{0.5}{-1.7}{3.2}{-0.7}{0.2}{1}\eretname{0.77}{-1.05}{l}{airline service}
\erCoreAttribute{0.7}{-1.25}{1}{0}{flight number}{}
\eret{0.5}{-3.8}{3.2}{-2.8}{0.2}{1}\eretname{0.77}{-3.15}{l}{flight}
\erCoreAttribute{0.7}{-3.35}{1}{0}{date of departure}{}
\eret{6.2}{-1.7}{8.9}{-0.7}{0.2}{1}\eretname{6.47}{-1.05}{l}{airport}
\erCoreAttribute{6.4}{-1.25}{1}{0}{name}{}
\eret{6.2}{-3.8}{8.9}{-2.8}{0.2}{1}\eretname{6.47}{-3.15}{l}{gate}
\erCoreAttribute{6.4}{-3.35}{1}{0}{number}{}

% relationship scheduled_as
\errelname{1.7}{-2}{r}{scheduled}\errelname{1.7}{-2.3}{r}{as}\errelname{2}{-2.65}{l}{of}\errelarm{1.85}{-1.7}{1.85}{-2.25}{0}{0}\errelarm{1.85}{-2.25}{1.85}{-2.8}{1}{0}\eridcomprel{1.75}{1.9500000000000002}{-2.55}\ercrowfoot{1.85}{-2.65}{1.7}{-2.8}{1.85}{-2.8}{2}{-2.8}{0}
% relationship departing_from
\errelname{3.35}{-0.883}{l}{from}\errelname{3.35}{-0.583}{l}{departing}\errelname{6.05}{-0.883}{r}{for}\errelname{6.05}{-0.583}{r}{airport}\errelname{6.05}{-0.283}{r}{departure}\errelarm{3.2}{-1.033}{4.7}{-1.033}{1}{0}\errelarm{4.7}{-1.033}{6.2}{-1.033}{0}{0}\ercrowfoot{3.35}{-1.033}{3.2}{-0.883}{3.2}{-1.033}{3.2}{-1.183}{0}
% relationship going_to
\errelname{3.35}{-1.667}{l}{going}\errelname{3.35}{-1.967}{l}{to}\errelname{6.05}{-1.667}{r}{arrival}\errelname{6.05}{-1.967}{r}{airport}\errelname{6.05}{-2.267}{r}{for}\errelarm{3.2}{-1.366}{4.7}{-1.366}{1}{0}\errelarm{4.7}{-1.366}{6.2}{-1.366}{0}{0}\ercrowfoot{3.35}{-1.367}{3.2}{-1.217}{3.2}{-1.367}{3.2}{-1.517}{0}
% relationship leaving_from
\errelname{3.35}{-3.15}{l}{from}\errelname{3.35}{-2.85}{l}{leaving}\errelname{6.05}{-3.6}{r}{used by}\errelarm{3.2}{-3.3}{4.7}{-3.3}{0}{0}\errelarm{4.7}{-3.3}{6.2}{-3.3}{0}{0}\ercrowfoot{3.35}{-3.3}{3.2}{-3.15}{3.2}{-3.3}{3.2}{-3.45}{0}\eridrefrel{3.45}{-3.1999999999999997}{-3.4}
% relationship having
\errelname{7.7}{-2}{l}{having}\errelname{7.4}{-2.65}{r}{at}\errelarm{7.55}{-1.7}{7.55}{-2.25}{0}{0}\errelarm{7.55}{-2.25}{7.55}{-2.8}{1}{0}\eridcomprel{7.450000000000001}{7.65}{-2.55}\ercrowfoot{7.55}{-2.65}{7.4}{-2.8}{7.55}{-2.8}{7.7}{-2.8}{0}
\end{erdiagram}
}
\end{equation}
Like so many examples this diagram doesn't have the full generality needed to be descriptive of all air transport situations (what about airports with multiple terminals? what about code sharing flights? what about change of gauge?\footnote{You might be intersted in looking up use of this term `change of gauge' in relation to air transport
 --- it describes a way of operating an airline service that falls outside the reality described by my diagram here. The term is borrowed (airquotes) from its use describing a reality that might be faced by a rail transport system. }). Nonetheless this is a useful example and it has some very interesting features and has instances of impactful patterns that recur over and again in modelling situations.

\mynote
The underlining of the flight number attribute in the representation
\raisebox{-0.5cm}{\begin{erdiagram}{1.1}{2.7}

\eret{0}{-1.1}{2.7}{-0.1}{0.2}{1}\eretname{0.27}{-0.45}{l}{airline service}
\erCoreAttribute{0.2}{-0.65}{1}{0}{flight number}{}

\end{erdiagram}
} of the airline service type
on diagram (\ref{boardingGate1}), and the absence of other underlined attributes, is interpreted as meaning that:
\begin{equation}
\label{airlineServiceIdentification}
\text{\parbox{9cm}{\textit{
Each airline service can be uniquely identified or referenced by its flight number.}}}
\end{equation}
Just to be be absolutely clear what this means --- it means no two distinct airline services have the same flight number. Thinking for a moment about the mathematical expression of this --- it means that the flight number attribute, which we know like all attributes can be thought of mathematically as a function, is a function that is total and is injective.\footnote{
If $f: A \longrightarrow B$ is a function then the function is injective iff for all $x,y \in A$,
$f(x) = f(y)$ implies $x=y$. 
}
 \mynote
Whereas the flight number of an airline service is unique the date of departure of a
flight is definitely not -- many flights leave each day. Instead flights are uniquely identified by the combination of their date of departure and the airline service that they are an instance of.
 To document this on the diagram we underline the name attribute and put a bar, like this \barkerEllisJ, through the relationship that contributes to the identification and provides context so that
 the type flight on the diagram appears like this:
 \begin{equation}
 \label{boardingGate0A}
\raisebox{-1.5cm}{\begin{erdiagram}{2.3}{2.9000000000000004}

\eret{0}{-2.3}{2.7}{-1.3}{0.2}{1}\eretname{0.27}{-1.65}{l}{flight}
\erCoreAttribute{0.2}{-1.85}{1}{0}{date of departure}{}
\eret{0}{-0.2}{2.9}{0.3}{0.2}{1}

% relationship scheduled as
\errelname{1.5}{-0.5}{l}{scheduled as}\errelname{1.5}{-1.15}{l}{of}\errelarm{1.35}{-0.2}{1.35}{-0.75}{0}{0}\errelarm{1.35}{-0.75}{1.35}{-1.3}{1}{0}\eridcomprel{1.25}{1.4500000000000002}{-1.05}\ercrowfoot{1.35}{-1.15}{1.2}{-1.3}{1.35}{-1.3}{1.5}{-1.3}{0}
\end{erdiagram}
}
\end{equation}

In summary, this part of the diagram conveys 
\begin{equation}
\label{airlineFlightIdentification}
\text{\parbox{9cm}{\textit{
Each flight can be uniquely identified or referenced by its date of departure in the context of the airline service that it is an instance of.}}}
\end{equation}

\mynote
Because of (\ref{airlineServiceIdentification}), that a airline service is identified or referenced by flight number,  we can fill out the detail in (\ref{airlineFlightIdentification}) and deduce:
\begin{equation}
\label{airlineFlightNetIdentification}
\text{\parbox{9cm}{\textit{
Each flight can be uniquely identified or referenced by its date of departure along with the flight number of the airline service that it is an instance of}}}
\end{equation}
and that is how we were able to, nay, were expected to, interpret the flight number LH2502, 
and date, 14 August 2024, in the original phone message (\ref{LH2502PhoneMessage}) --- as identifying a flight.\footnote{Because flight number got involved in the identification of flights indirectly, 
 mediated by the  relationship of a flight to an airline service, 
 and because multiple level of this sort of thing are common, flight number in such a situation is sometimes said to be cascaded.}

\mynote
Looking at the right hand side of diagram (\ref{boardingGate1}),
the name attribute of an airport is underlined to indicate that
it is an identifying attribute and so, in the absence of other identifying attributes and relationships,
that:
\begin{equation}
\label{airportIdentification}
\text{\parbox{9cm}{\textit{
Each airport can be uniquely identified or referenced by its name.}}}
\end{equation}
and if we inspect the representation of the type \textit{gate} then we see that it says:
\begin{equation}
\label{gateIdentification}
\text{\parbox{9cm}{\textit{
Each gate can be uniquely identified or referenced by its gate number in the context of the airport that it is located at.}}}
\end{equation}

\mynote
Because of (\ref{airportIdentification}), that a airport is identified or referenced by its name,  we can fill out the detail in (\ref{gateIdentification}) and deduce that:
\begin{equation}
\label{gateNetIdentification}
\text{\parbox{9cm}{\textit{
Each gate can be uniquely identified or referenced by its gate number along with the name of the airport it is located at.}}}
\end{equation}
\subsubsection{Discussion --- scope of relationship}
This doesn't yet give a full explanation of the original phone message 
(\ref{LH2502PhoneMessage}) though because
in the message there are two airport names. 
I was expected to know somehow which of the two airports named along with the my gate number L06
was the location of my gate.  Was it L06 at Manchester or was it L06 at Munich? How was I supposed to know? 

\mynote The answer to this question involves knowing something about the arrangement
of the concepts and relationships shown in diagram (\ref{boardingGate1}) 
and this something that I needed to know, 
and all of us would have known,
is not currently represented in the diagram.  
This missing something is an example of a phenomena  
that is massively understudied and unreported. It is something that mathematicians, particularly those versed in category theory, come across all the time but which entity modellers, data modellers and programmers literally have no words for and therefore it remains largely unobserved and wholly unremarked even though it is extremely impactful.

\mynote 
Why didn't the message say gate L06 of Munich airport?
That's because everybody knows and so I am expected to know that my flight will be leaving from a gate at the same airport as the airline service that I have booked is departing from. 
Diagram (\ref{boardingGate1}) doesn't express this fact. 
Later in this book we introduce the concept of relationship scope and 
suggest an annotation that could be added to the diagram to rectify this shortcoming.

It is to the impoverishment of data specifiers everywhere (and surely this includes all programmers) that these phenomena are not in the core common syllabus of computer science. 







 
