
\section{Communicating Relationships}
\subsection{What this section is about}
\mynote
Earlier, in section \ref{sub:conveying_facts_and_storing_of_data}, we looked at examples of the use
of references to entitites, and thereby referentials,  
to convey instances of relationships.  
On the surface of it it is very simple ---
to convey instances of relationships we reference two entities and assert a relationship between them
as in this example:

\begin{equation}
\label{JupiterIoPernickity}
\mbox{\textit{The planet \rdash{Jupiter} 
\uwave{is orbited} by the moon \rdot{Io}}}
\end{equation}

in which a single referential references one entity, a second referential references a second entity
and a relationship instance is asserted.  
In this simple example each of these references to entities contains  a single referential 
but we have already seen examples where multiple referentials are required to reference an entity. For example
\begin{itemize}
\item Hot Springs, Arkansas;
\item the route from Hot Springs, Arkansas to Jacksonville, Alabahma;
\item the conjunction of Mars and Jupiter on July 4, 929;
\item Shakespeare’s Twelfth Night (i.e. the play Twelfth Night by playwright William Shakespeare);
\item Touchstone from Shakespeare’s As You Like It;
\item the role of Touchstone in the production of Shakespeare’s As You Like It
performed April – May 1975, at Oxford Playhouse.
\end{itemize}

We have seen also how in some references, for example the final one in the list above,  there are less referentials than is implied by a simple sum of the parts  and that this is because of overlaps (for which we used the term collapsed referentials) and that these overlaps come about because of the equivalence of comparable paths of identifying relationships.

Now we have to consider the possibility that when relationship instances are to be conveyed, so that multiple referentials are required to reference the first entity in the relationship and also multiple referentials are rqeuired to reference the second entity, that, \textit{a priori}, these two sets of referentials overlap. 
In other words that, \textit{a priori}, referentials to the two distinct entities coincide and collapse into a single referential. 

We look at cases where this is so, and at yet other cases in which 
referentials that one might expect to be present are simply not needed --- in such cases we shall say that there are \textit{absent referentials}.
As we saw in the previous section, we see different behaviours\commentary{better word than behaviour} emerge in the presence of comparable paths of relationships
which are equivalent or near equivalent.

\subsection {Recap}
Consider how we convey an instance of this relationship:
\begin{equation}
\begin{erdiagram}{1.72}{6}

\eret{0.1}{-1.22}{1.6}{-0.3}{0.2}{1}\eretname{0.85}{-0.66}{}{dramatic}\eretname{0.85}{-0.96}{}{role}
\eret{4.5}{-1.22}{6}{-0.3}{0.2}{1}\eretname{5.25}{-0.81}{}{actor}

% relationship played_by
\errelname{1.75}{-0.61}{l}{by}\errelname{1.75}{-0.31}{l}{played}\errelname{4.35}{-1.06}{r}{the}\errelname{4.35}{-1.36}{r}{player}\errelname{4.35}{-1.66}{r}{of}\errelarm{1.6}{-0.76}{3.05}{-0.76}{1}{0}\errelarm{3.05}{-0.76}{4.5}{-0.76}{0}{0}\errelid{3.05}{-0.85}{}{r5}\ercrowfoot{1.75}{-0.76}{1.6}{-0.61}{1.6}{-0.76}{1.6}{-0.91}{0}
\end{erdiagram}

\end{equation}

we reference an actor entity, by name according to figure ref{drammaticArts1},
and we convey a dramatic role entity by means of five referentials as we have seen above.
In this way we get statements such as
\begin{erquote}

\parbox{9.0cm}{\linespread{1.5}\normalsize\rdash{Bob Hoskins} was \uwave{the player of} the role of \rdot{Touchstone} in the production of \mbox{\rdot{Shakespeare}'s} \rdot{As You Like It} performed \mbox{\rdot{April – May 1975},} at \rdot{Oxford Playhouse}.
}
\end{erquote}
which, by the way, is a true fact. 
In this example, referentials to reference the actor are dash underlined. The relationship is wavy underlined.
Referentials to reference the dramatic role are dotted underlined.
\subsection{An Example in which Referentials Collapse}
All this is straightforward. But things are not always so straightforward because sometimes referentials can overlap. 
Shlaer and Long explore this topic in the context of relational database structure and refer to them as \textit{collapsed referentials}. This is an important topic and the subject isn't complete if we don't 
consider these and so here is an example that contains such `collapsed referentials'.

As we have said above and according to the model we have assumed,
  characters from plays need be identified by three referentials
 --- the name of the character, the title of the play and the name of the playwright.

\mynote 
That there are three referentials to reference a character naively implies that if we wish to communicate an instance of relationship between two different characters then we will be required to communicate six referentials
 -- three to identify one party in the relationship and three to identify the other party. 
 we will see shortly  --- this is the significant bit --- that this is not always the case and so not what we instinctively do, when we have \textit{a priori} knowledge of the relationship being communicated is limited in scope. 
 If we know the relationship is confined to individual playwrights or is internal to individual play 
 or script then clearly we do not 
  communicate the playwright and play title twice over. The number of referentials may in such a case collapse to four:
   name of playwright, title of play and name of character.
Shlaer-Lang in the context of data representation of entities and in a very much underrated paper, coined the term \textit{collapsed referentials} in
their description of this phenomena.  We shall use instead the term \textit{shared referential} for a referential that contributes to multiple references.

\mynote 
In the  the play Twelfth Night by William Shakespeare, Antonio loves Sebastian.

For the discussion that follows suppose love to be a many-many relationship between characters.

A naive approach would suggest that six referentials are required to convey an instance of love
between characters --- three referentials to reference the lover character and three to reference the loved.
If I follow the naive approach I would convey the relationship instance like this:
\begin{equation}
\text{\parbox{9cm}{\textit{The character named Antonio in Shakespeare's play Twelfth Night loves the character named Sebastian in Shakespeare play Twelfth Night.}}}
\end{equation} 
But there is a reason why the shorter
\begin{equation}
\text{\parbox{9cm}{\textit{The character named Antonio in Shakespeare's play Twelfth Night loves the character named Sebastian.}}}
\end{equation} 
is correct and it is  important we understand this reason.
What is occuring here is another instance of the same basic phenomenon we saw before in the context of the referencing of entities --- less referentials are required than we first expect from a naive reading.  

This reason in this case is because the relationship `loves' in this context 
is, \textit{a priori}, a relationship which is internal to the plot of a single individual play ---
it is an intra-play relationship rather than an inter-play relationship.
Because it is intra-play, when we covey an instance then the referentials that refer to the play are shared between lover and loved As shown in figure \ref{SebastianLovesAntonio}.


\begin{erboxedFigure}{H}{SebastianLovesAntonio}
{Referencing two different characters using just four referentials. This is an example of what Shlaer and Lang (?) called collapsed referentials. What we see is that referential `Shakespeare' contributes to two different references. Likewise does the referential `Twelfth Night'.
These are examples of shared referentals --- they contribute, as indicated, to both reference 1 and reference 2.}
\newcommand{\dashRefOne}{2pt 2pt}
\newcommand{\dashRelationship}{1pt 0pt}
\newcommand{\dashRefTwo}{1pt 1pt}
\begin{tabular}{l}
\Rnode{w1}{\rdash{Antonio}} in \Rnode{w2}{\rdot{\rdash{Shakepeare}}}'s \Rnode{w3}{\rdot{\rdash{Twelfth Night}}} \Rnode{w4}{\rline{loves}}  \Rnode{w5}{\rdot{Sebastian}} \\[1.4cm]
\kern1.2cm\Rnode{ref1}{\textit{reference 1}}
\kern0.75cm\Rnode{rel}{\textit{relationship}}
\kern0.6cm\Rnode{ref2}{\textit{reference 2}} \\[0.5cm]
\syntag{\dashRefOne}{ref1}{0.9}{w1}{0}
\syntag{\dashRefOne}{ref1}{0.9}{w2}{-0.2}
\syntag{\dashRefOne}{ref1}{0.9}{w3}{-0.2}
\syntag{\dashRelationship}{rel} {0.7} {w4}{0}
\syntag{\dashRefTwo}{ref2}{0.4}{w2}{0.2}
\syntag{\dashRefTwo}{ref2}{0.4}{w3}{0.3}
\syntag{\dashRefTwo}{ref2}{0.4}{w5}{0}
\end{tabular}
\end{erboxedFigure}

An instances of the `loves' relationship is conveyed by a total of four referentials
 not the six that might have been required
if the relationship were global in scope (as would have been the case if a
 character from one play could love  a character from an entirely different play).
 An example of a relationship between characters that is global in scope 
 is the relationship of one character being modelled on or resembling or being inspired 
 by another character from the dramatic arts.
 To convey an instance of this relationship, being as the relationship is global in scope,  does indeed require
 six referentials --- three for the one party and three for the other --- as in this example:
\begin{erquote}
\parbox{9.0cm}{\linespread{1.3}\normalsize
\rdash{Romeo} from \rdash{Shakespeare}'s \rdash{Romeo and Juliet}
          \uwave{is modelled on} \rdot{Pyramus} from \rdot{Ovid}'s \rdot{Metamorphoses}.}
\end{erquote}
which, interpreted from the perspective of the dramatic arts that we have given above, does indeed contain six referentials. 
The first three, shown with dashed underlining, reference one character.
The verb of the sentence is the name of the relationship, and, 
the final three referentials, with dotted underlining, 
reference a second character from a different play.\footnote{I am bending the facts a bit here 
in that Metamophoses isn't strictly a play - its a bunch of stories.} 
There are no shared referentials.









 
