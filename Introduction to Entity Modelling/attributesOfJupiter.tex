
\subsubsection{Data Model}
\begin{tabular}{ | l | l |}
\hline
\multicolumn{2}{|l|}{Orbital Characteristics}  \\
\hline
Semi-major axis&5.2038 AU (778.479 million km) \\
\hline
Eccentricity&0.0489  \\
\hline
Inclination to ecliptic&1.303° to ecliptic\\
\hline
Longitude of ascending node&100.464° \\
\hline
Time of perihelion&January 21, 2023 \\
\hline
Argument of perihelion&273.867° \\
\hline
\end{tabular}

\begin{tabular}{ | l | l |}
\hline
\multicolumn{2}{|l|}{Physical Characteristics}  \\
\hline
Equatorial radius&71492 km  \\
Polar radius&66854 km  \\
Mass&1.8982×$10^27$ kg  \\
Moment of inertia factor&0.2756±0.0006  \\
Sidereal rotation period&9.9250 hours  \\
North pole right ascension&268.057° \\
North pole declination&64.495° \\
Absolute magnitude (H)&−9.4 \\
\end{tabular}


\subsubsection{Workings}
\begin{tabular}{ | l | l | p{3cm}|}
\hline
\multicolumn{3}{l}{|Orbital Characteristics|}  \\
\hline
Aphelion&5.4570 AU (816.363 million km)       & \multirow{2}{2cm}{either these two}\\
Perihelion&4.9506 AU (740.595 million km)  &    \\
\hline
Semi-major axis&5.2038 AU (778.479 million km) & \multirow{2}{2cm}{or these two} \\
Eccentricity&0.0489  & \\
\hline
Orbital period (sidereal) & 11.862 yr  & calc(semi-major-axis)\\
Orbital period (synodic)&398.88 d      & calc(sideral orbital period)\\
\hline
Average orbital speed&13.06 km/s       & calc(sideral orbital period) \\
\hline
Mean anomaly&20.020° & either this or Time of perihelion
this required reference date \\
\hline
Inclination&
1.303° to ecliptic,& \multirow{2}{2cm}{either of these three}\\
&6.09° to Sun's equator &, \\
&0.32° to invariable plane  & \\
\hline
Longitude of ascending node&100.464° & \\
\hline
Time of perihelion&January 21, 2023 & either this or Mean anomaly \\
\hline
Argument of perihelion&273.867° & \\
\hline
Known satellites&95 (as of 2023) & random\\
\hline
Physical characteristics & \\
\hline
Mean radius&69911 km & calculated\\
Equatorial radius&71492 km & \\
Polar radius&66854 km & \\
Flattening&0.06487 & calculated\\
Surface area&6.1469×1010 km2 & calculated\\
Volume&1.4313×1015 km3 & calculated\\
Mass&1.8982×1027 kg & \\
Mean density&1.326 g/cm3 & calculated\\
Surface gravity&24.79 m/s2 & calculated\\
Moment of inertia factor&0.2756±0.0006 & \\
Escape velocity&59.5 km/s & calculated\\
Synodic rotation period&9.9258 h & calculate from sideral rotation period \\
Sidereal rotation period&9.9250 hours  & \\
Equatorial rotation velocity&12.6 km/s & calculated\\
Axial tilt&3.13° (to orbit) & calculate this\\
North pole right ascension&268.057°; 17h 52m 14s & \\
North pole declination&64.495° & \\
Albedo  
0.503 (Bond)  & \\
0.538 (geometric) & can be calculated from diameter and absolute magnitude\\
Temperature&88 K (−185 °C)  & (blackbody temperature) \\
Surface temp.&min mean max  \\
1 bar& 165 K & \\
0.1 bar&78 K, 128 K& \\
Apparent magnitude&−2.94[16] to −1.66 & \\
Absolute magnitude (H)&−9.4 & \\
Angular diameter&29.8" to 50.1" & \\
\hline
Atmosphere & \\
\hline
Surface pressure&200–600 kPa (30–90 psi) & calculated\\
Scale height&27 km (17 mi)  &  calculate from atmosphere\\
\end{tabular}