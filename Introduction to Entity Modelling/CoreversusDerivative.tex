
\section{Core versus Derivative \small and goodness criteria for entity models}
\label{CoreversusDerivative}
\mynote Entity models comprise a set of definitions of types, relationships and attributes. 
and one point of view is that this set should be a minimal set for the job at hand and that none of these definitions should be derivative of the remainder. For example
\begin{itemize}
\item we might model that a bicycle is related to its front wheel and that a bicycle is related to its back wheel but we will not then need to model,  and therefore ought not to model, that the front wheel of a bicycle is related to its back wheel because though there is such a relationship it is implied by or mediated by the other two. It is derivative.
\item temperature in degrees farenheit (a possible attribute) and temperature in degrees centigrade (another possible attribute) cannot both be considered core -- one must be considered core and the other considered derivative,
\item age (a possible attribute) should be considered derivative of date of birth (another attribute, assuming dates are universals),
\item unequivocally, grandparenthood (a candidate many-many binary relationship) should be considered derivative of parenthood (another many-many relationship),
\item likewise, compatriot, as a binary relationship, is derivative of country of origin (another binary relationship),\footnote{This begs an interesting point --- is it ever right to model a symmetric, transitive, reflexive (or anti-reflexive) relationship --- consider this from goodness criteria}
\item father's surname (a candidate attribute of a person) \commentary{find better example?} should be considered derivative to fatherhood (a relationship) and surname (an attribute), 
\item ancestorship is recursively derivative of parenthood. \commentary{\textit{recursively derivative}}
\end{itemize}

If this point of view is adhered to then the entity model documents a conceptual core.

\mynote
Another possible point of view is that derivative attributes, relationships and attributes may be included in an entiy model provided thay are clearly delineated and the method of derivation or inference from the core
is clearly specified.\footnote{Current book has `core' and `constructed'. ER scipt has `constructed'. Seems to me now that terms `core' and `derivative' work better.
I have also had in mind to use the word `mediated'. Toolbuild had `derived' relationships but it now seems to me that `derivative' gets closer to the fact that they are not `core'. }

This second point of view is necessary when entity models are used as data models i.e. as data specifications and leads us to the idea that a model can have data cores that extend the conceptual core. We will come back to this later. That a data core has no derivatives other than those prescribed becomes the basis of Codd's famous normal form criteria for goodness of database specifications. 

\mynote Please note that the prescription that an entity model or its delineated core be should free of redundancy as I have explained it above is a general goodness criteria for entity models. Though it isn't usual for it to be articulated as I have here, it is a  criteria that is tacitly understood --- it is followed by most practioners, most of the time. 

\mynote
Now for an example that breaks everything we have said so far. This model fails tyhe goodness criteria as described so far
\begin{equation}
\label{nearestShop}
\raisebox{-1.5cm}{\begin{erdiagram}{3.4999999999999996}{7.3}

\eret{0.1}{-0.9}{1.9}{-0}{0.2}{1}\eretname{0.28}{-0.35}{l}{person}
\erCoreAttribute{0.3}{-0.55}{1}{0}{name}{}
\eret{2.7}{-0.9}{4.5}{-0}{0.2}{1}\eretname{2.88}{-0.35}{l}{shopType}
\erCoreAttribute{2.9}{-0.55}{1}{0}{name}{}
\eret{5.3}{-0.9}{7.1}{-0}{0.2}{1}\eretname{5.48}{-0.35}{l}{shop}
\erCoreAttribute{5.5}{-0.55}{1}{0}{name}{}
\eret{1.954}{-3.5}{5.246}{-2.6}{0.2}{1}\eretname{2.283}{-2.95}{l}{nearestShop}
\erCoreAttribute{2.154}{-3.15}{1}{1}{distance}{}

% relationship has
\errelname{0.85}{-1.2}{r}{has}\errelname{2.248}{-2.45}{r}{to}\errelarm{1}{-0.899}{1}{-1.099}{1}{0}\errelarm{1}{-1.099}{1}{-1.299}{1}{0}\errelarm{1}{-1.299}{1.723}{-1.649}{1}{0}\errelarm{1.723}{-1.649}{2.447}{-1.999}{1}{0}\errelarm{2.447}{-1.999}{2.447}{-2.3}{1}{0}\errelarm{2.447}{-2.3}{2.447}{-2.599}{1}{0}\eridcomprel{2.3475375}{2.5475375000000002}{-2.3499999999999996}\ercrowfoot{2.448}{-2.45}{2.298}{-2.6}{2.448}{-2.6}{2.598}{-2.6}{0}
% relationship used as
\errelname{3.45}{-1.2}{r}{used as}\errelname{3.45}{-2.45}{r}{ofType}\errelarm{3.6}{-0.899}{3.6}{-1.449}{1}{0}\errelarm{3.6}{-1.449}{3.6}{-2.3}{1}{0}\errelarm{3.6}{-2.3}{3.6}{-2.599}{1}{0}\eridcomprel{3.5}{3.7}{-2.3499999999999996}\ercrowfoot{3.6}{-2.45}{3.45}{-2.6}{3.6}{-2.6}{3.75}{-2.6}{0}
% relationship being
\errelname{6.05}{-1.2}{r}{being}\errelname{4.388}{-2.45}{r}{is}\errelarm{6.2}{-0.899}{6.2}{-1.099}{1}{0}\errelarm{6.2}{-1.099}{6.2}{-1.299}{1}{0}\errelarm{6.2}{-1.299}{5.393}{-1.649}{1}{0}\errelarm{5.393}{-1.649}{4.587}{-1.999}{1}{0}\errelarm{4.587}{-1.999}{4.587}{-2.3}{1}{0}\errelarm{4.587}{-2.3}{4.587}{-2.599}{1}{0}\ercrowfoot{4.588}{-2.45}{4.438}{-2.6}{4.588}{-2.6}{4.738}{-2.6}{0}
% relationship type
\errelname{5.15}{-0.75}{r}{type}\errelarm{5.3}{-0.449}{4.9}{-0.449}{1}{0}\errelarm{4.9}{-0.449}{4.5}{-0.449}{0}{0}\ercrowfoot{5.15}{-0.45}{5.3}{-0.3}{5.3}{-0.45}{5.3}{-0.6}{0}
\end{erdiagram}
}
\end{equation}\commentary{ditto}.

We are in  ajm with this model. I need xxx so that I can specify it to be identifying.
But xxx is derivative. It can be derived as a compose b.
This situation turns out to be fatal unless we allow derivative relationships to be allowed in a model. Defined as derivative so that the core is still core. Maybe a way of drawing what we need in this case is:
\begin{equation}
\label{zanioloExample2}
\raisebox{-1.5cm}{\begin{erdiagram}{4.6}{4}

\eret{0.5}{-1}{3}{-0.1}{0.2}{1}\eretname{0.75}{-0.45}{l}{area}
\erCoreAttribute{0.7}{-0.65}{1}{0}{code}{}
\eret{0.5}{-2.8}{3}{-1.9}{0.2}{1}\eretname{0.75}{-2.25}{l}{place}
\erCoreAttribute{0.7}{-2.45}{1}{0}{name}{}
\eret{0.5}{-4.6}{3}{-3.7}{0.2}{1}\eretname{0.75}{-4.05}{l}{telephone}
\erCoreAttribute{0.7}{-4.25}{1}{0}{number}{}

% relationship 
\errelname{1.9}{-1.3}{l}{}\errelname{1.9}{-1.75}{l}{within}\errelarm{1.75}{-0.999}{1.75}{-1.45}{0}{0}\errelarm{1.75}{-1.45}{1.75}{-1.9}{0}{0}\ercrowfoot{1.75}{-1.75}{1.6}{-1.9}{1.75}{-1.9}{1.9}{-1.9}{0}
% relationship 
\errelname{1.9}{-3.1}{l}{}\errelname{1.9}{-3.55}{l}{at}\errelarm{1.75}{-2.8}{1.75}{-3.25}{0}{0}\errelarm{1.75}{-3.25}{1.75}{-3.699}{0}{0}\ercrowfoot{1.75}{-3.55}{1.6}{-3.7}{1.75}{-3.7}{1.9}{-3.7}{0}
% relationship area
\errelname{3.15}{-4.45}{l}{area}\errelarm{3}{-4.149}{3.3}{-4.149}{0}{0}\errelarm{3.3}{-4.149}{3.6}{-4.149}{0}{0}\errelarm{3.6}{-4.149}{3.65}{-4.149}{0}{0}\errelarm{3.65}{-4.149}{3.7}{-4.149}{0}{0}\errelarm{3.7}{-4.149}{3.7}{-2.349}{0}{0}\errelarm{3.7}{-2.349}{3.7}{-0.549}{0}{0}\errelarm{3.7}{-0.549}{3.65}{-0.549}{0}{0}\errelarm{3.65}{-0.549}{3.6}{-0.549}{0}{0}\errelarm{3.6}{-0.549}{3.3}{-0.549}{0}{0}\errelarm{3.3}{-0.549}{3}{-0.549}{0}{0}\ercrowfoot{3.15}{-4.15}{3}{-4}{3}{-4.15}{3}{-4.3}{0}\eridrefrel{3.25}{-4.05}{-4.249999999999999}
\end{erdiagram}
}
\end{equation}\commentary{redraw this diagram with an exclamation mark next to the derivative relationship}.


\begin{noteforfuture}
Move Goodness of an Entity Model section HERE from tutorial ??
four finger flight example seems a bit strange. 
Becomes less strange if given earlier as an example?
\end{noteforfuture}


\subsection{Zaniolo's Example and the Nearest Shop}
This situation described in this 
model here is a classic from relational database literature. The situation was described by Zaniolo
in 19xx.
\begin{equation}
\label{zanioloExample2}
\raisebox{-1.5cm}{\begin{erdiagram}{4.6}{4}

\eret{0.5}{-1}{3}{-0.1}{0.2}{1}\eretname{0.75}{-0.45}{l}{area}
\erCoreAttribute{0.7}{-0.65}{1}{0}{code}{}
\eret{0.5}{-2.8}{3}{-1.9}{0.2}{1}\eretname{0.75}{-2.25}{l}{place}
\erCoreAttribute{0.7}{-2.45}{1}{0}{name}{}
\eret{0.5}{-4.6}{3}{-3.7}{0.2}{1}\eretname{0.75}{-4.05}{l}{telephone}
\erCoreAttribute{0.7}{-4.25}{1}{0}{number}{}

% relationship 
\errelname{1.9}{-1.3}{l}{}\errelname{1.9}{-1.75}{l}{within}\errelarm{1.75}{-0.999}{1.75}{-1.45}{0}{0}\errelarm{1.75}{-1.45}{1.75}{-1.9}{0}{0}\ercrowfoot{1.75}{-1.75}{1.6}{-1.9}{1.75}{-1.9}{1.9}{-1.9}{0}
% relationship 
\errelname{1.9}{-3.1}{l}{}\errelname{1.9}{-3.55}{l}{at}\errelarm{1.75}{-2.8}{1.75}{-3.25}{0}{0}\errelarm{1.75}{-3.25}{1.75}{-3.699}{0}{0}\ercrowfoot{1.75}{-3.55}{1.6}{-3.7}{1.75}{-3.7}{1.9}{-3.7}{0}
% relationship area
\errelname{3.15}{-4.45}{l}{area}\errelarm{3}{-4.149}{3.3}{-4.149}{0}{0}\errelarm{3.3}{-4.149}{3.6}{-4.149}{0}{0}\errelarm{3.6}{-4.149}{3.65}{-4.149}{0}{0}\errelarm{3.65}{-4.149}{3.7}{-4.149}{0}{0}\errelarm{3.7}{-4.149}{3.7}{-2.349}{0}{0}\errelarm{3.7}{-2.349}{3.7}{-0.549}{0}{0}\errelarm{3.7}{-0.549}{3.65}{-0.549}{0}{0}\errelarm{3.65}{-0.549}{3.6}{-0.549}{0}{0}\errelarm{3.6}{-0.549}{3.3}{-0.549}{0}{0}\errelarm{3.3}{-0.549}{3}{-0.549}{0}{0}\ercrowfoot{3.15}{-4.15}{3}{-4}{3}{-4.15}{3}{-4.3}{0}\eridrefrel{3.25}{-4.05}{-4.249999999999999}
\end{erdiagram}
}
\end{equation}
It has characteristics which are fundamental in shaping relational data theory and we will return to this subject later. \commentary{I have thought of placing an exclamation mark to distinguish derived relationships.}

\mynote
The following example exhibits the same features in a slightly more general way.\footnote{I found the situation modelled here described on Wikipedia in an article discussing TNF and BCNF normal forms. I haven't been able to trace it back any other sources.}
\begin{equation}
\label{nearestShop}
\raisebox{-1.5cm}{\begin{erdiagram}{3.4999999999999996}{7.3}

\eret{0.1}{-0.9}{1.9}{-0}{0.2}{1}\eretname{0.28}{-0.35}{l}{person}
\erCoreAttribute{0.3}{-0.55}{1}{0}{name}{}
\eret{2.7}{-0.9}{4.5}{-0}{0.2}{1}\eretname{2.88}{-0.35}{l}{shopType}
\erCoreAttribute{2.9}{-0.55}{1}{0}{name}{}
\eret{5.3}{-0.9}{7.1}{-0}{0.2}{1}\eretname{5.48}{-0.35}{l}{shop}
\erCoreAttribute{5.5}{-0.55}{1}{0}{name}{}
\eret{1.954}{-3.5}{5.246}{-2.6}{0.2}{1}\eretname{2.283}{-2.95}{l}{nearestShop}
\erCoreAttribute{2.154}{-3.15}{1}{1}{distance}{}

% relationship has
\errelname{0.85}{-1.2}{r}{has}\errelname{2.248}{-2.45}{r}{to}\errelarm{1}{-0.899}{1}{-1.099}{1}{0}\errelarm{1}{-1.099}{1}{-1.299}{1}{0}\errelarm{1}{-1.299}{1.723}{-1.649}{1}{0}\errelarm{1.723}{-1.649}{2.447}{-1.999}{1}{0}\errelarm{2.447}{-1.999}{2.447}{-2.3}{1}{0}\errelarm{2.447}{-2.3}{2.447}{-2.599}{1}{0}\eridcomprel{2.3475375}{2.5475375000000002}{-2.3499999999999996}\ercrowfoot{2.448}{-2.45}{2.298}{-2.6}{2.448}{-2.6}{2.598}{-2.6}{0}
% relationship used as
\errelname{3.45}{-1.2}{r}{used as}\errelname{3.45}{-2.45}{r}{ofType}\errelarm{3.6}{-0.899}{3.6}{-1.449}{1}{0}\errelarm{3.6}{-1.449}{3.6}{-2.3}{1}{0}\errelarm{3.6}{-2.3}{3.6}{-2.599}{1}{0}\eridcomprel{3.5}{3.7}{-2.3499999999999996}\ercrowfoot{3.6}{-2.45}{3.45}{-2.6}{3.6}{-2.6}{3.75}{-2.6}{0}
% relationship being
\errelname{6.05}{-1.2}{r}{being}\errelname{4.388}{-2.45}{r}{is}\errelarm{6.2}{-0.899}{6.2}{-1.099}{1}{0}\errelarm{6.2}{-1.099}{6.2}{-1.299}{1}{0}\errelarm{6.2}{-1.299}{5.393}{-1.649}{1}{0}\errelarm{5.393}{-1.649}{4.587}{-1.999}{1}{0}\errelarm{4.587}{-1.999}{4.587}{-2.3}{1}{0}\errelarm{4.587}{-2.3}{4.587}{-2.599}{1}{0}\ercrowfoot{4.588}{-2.45}{4.438}{-2.6}{4.588}{-2.6}{4.738}{-2.6}{0}
% relationship type
\errelname{5.15}{-0.75}{r}{type}\errelarm{5.3}{-0.449}{4.9}{-0.449}{1}{0}\errelarm{4.9}{-0.449}{4.5}{-0.449}{0}{0}\ercrowfoot{5.15}{-0.45}{5.3}{-0.3}{5.3}{-0.45}{5.3}{-0.6}{0}
\end{erdiagram}
}
\end{equation}

\subsection*{Other Goodness Criteria}
\mynote
There are other goodness criteria and one of these is violated by Chen in his 1983 English paper.

in Chen's figure 17, his  relationship type PERFORMED AT which he inherits from Teorey and Fry
is a cartesian product of JOB TYPE with PLANT. 
If further analysis doesn't come up with any reason to keep it then it should be rem,oved from the entity model. A look at Teorey and Fry would be useful.



