\section{Communicating Entity Instances}
\label{CommunicatingEntityInstances}


\mynote to convey an entity we need to convey how that entity stands relative to our context i.e. how it is related,
to convey the relationship we need convey attributes,
 relationship are communicated soley using the values of attributes,

\mynote
Whereas the names of moons and planets are unique in the context of the solar system then the names of characters in Shakespeare's plays are not. Shakespeare's characters are named uniquely only within the context of individual plays. To document this on an entity relationship diagram we underline the name attribute and put a bar through the relationship that contributes to the identification like this \barkerEllisJ, for example, as we illustrate twice over in figure \ref{playwrightPlayCharacterModel}.


\begin{erboxedFigure} {H}{playwrightPlayCharacterModel}
{
caption blah blah blah
 }
\begin{erdiagram}{2.4999999999999996}{11.15}

\eret{0.1}{-2.1}{1.85}{-1.2}{0.2}{1}\eretname{0.425}{-1.55}{l}{playwright}
\erCoreAttribute{0.3}{-1.75}{1}{0}{name}{}
\eret{4.75}{-2.1}{6.5}{-1.2}{0.2}{1}\eretname{5.075}{-1.55}{l}{play}
\erCoreAttribute{4.95}{-1.75}{1}{0}{title}{}
\eret{9.4}{-2.1}{11.15}{-1.2}{0.2}{1}\eretname{9.725}{-1.55}{l}{character}
\erCoreAttribute{9.6}{-1.75}{1}{0}{name}{}
\eret{0}{-0.2}{11.15}{0.3}{0.2}{1}

% relationship 
\errelname{1.125}{-0.5}{l}{}\errelarm{0.975}{-0.2}{0.975}{-0.7}{1}{0}\errelarm{0.975}{-0.7}{0.975}{-1.2}{0}{0}\ercrowfoot{0.975}{-1.05}{0.825}{-1.2}{0.975}{-1.2}{1.125}{-1.2}{0}
% relationship 
\errelname{5.775}{-0.5}{l}{}\errelarm{5.625}{-0.2}{5.625}{-0.7}{1}{0}\errelarm{5.625}{-0.7}{5.625}{-1.2}{0}{0}\ercrowfoot{5.625}{-1.05}{5.475}{-1.2}{5.625}{-1.2}{5.775}{-1.2}{0}
% relationship 
\errelname{10.425}{-0.5}{l}{}\errelarm{10.27}{-0.2}{10.27}{-0.7}{1}{0}\errelarm{10.27}{-0.7}{10.27}{-1.2}{0}{0}\ercrowfoot{10.275}{-1.05}{10.125}{-1.2}{10.275}{-1.2}{10.425}{-1.2}{0}
% relationship written by
\errelname{4.6}{-1.95}{r}{written by}\errelname{2}{-1.5}{l}{author of}\errelarm{4.75}{-1.65}{3.3}{-1.65}{1}{0}\errelarm{3.3}{-1.65}{1.85}{-1.65}{1}{0}\ercrowfoot{4.6}{-1.65}{4.75}{-1.5}{4.75}{-1.65}{4.75}{-1.8}{0}\eridrefrel{4.5}{-1.5499999999999998}{-1.75}
% relationship within
\errelname{9.25}{-1.95}{r}{within}\errelname{6.65}{-1.5}{l}{has}\errelarm{9.4}{-1.65}{7.95}{-1.65}{1}{0}\errelarm{7.95}{-1.65}{6.5}{-1.65}{1}{0}\ercrowfoot{9.25}{-1.65}{9.4}{-1.5}{9.4}{-1.65}{9.4}{-1.8}{0}\eridrefrel{9.15}{-1.5499999999999998}{-1.75}
\end{erdiagram}

\end{erboxedFigure}

In this diagram we have shown characters being identified by name and by play 
and plays being identified by title and by playwright. Playwrights we have indicated are identified by their names.\commentary{cascaded identifier}



\mynote 
From this section most interest and significance  in how relationship instances are/can be communicated but this question cannot be separated from how entitites are communicted and how entity identitites are communicated.

\mynote I will tend talk abouty how they are commuincated in data but everything that is siad applies to how they are communicated generally including nay especially in natural language. There is something deeply logical and linguistic about this but it is not ususally consided as a part of formal logic.

\mynote Rationale: What is universally true need not be communicated. Only that which is particular and the relation of the particular with the universal need be communicated. \commentary{Need to find a home for this -- I would be sad to lose it.}

\mynote An entity model systematically describes all that can be known of an entity
in terms of core functional relationships with universals and core functional relationships with particulars i.e. in terms of core attributes and core outgoing directional relationships.

\mynote But of all this which  may be known of an entity all we can communicate of an entity is its functional relationships with universals. Other relationships must be communicated indirectly via derivative (i.e. non-core) such relationships with universals. 

\mynote For this reason a systematic way of identifying and referencing each particular type  of entity in an entity model become significant if the model is to be used as any kind of data or message specification. There is some subtlety to identification and referencing as we will explain.

\subsection*{Communicating Relationships}
\mynote To communicate the value of a relationship is to communicate the identity of the relationship and to identify (i.e. to communicate the identity of) the related entity. 

\mynote The related entity is a particular we are required therefore to have ways of identifying particulars and communicating these identifications.  
Only universals can be communicated so each type of entity is required 
to have one or more identifying attributes. 

\mynote Every entity may be identified by the values of related universals i.e. values that can be attributed to it. These attributions  may be core or derivative.

\mynote Since these relationships with other particulars themself need communication then ultimately every entity can be identified by a set of attributes each one of which may be core or may be derivative.  

\mynote
These attributes of a particular are called identifying attributes and subsequent use of these as attributes in a communication scheme in order to identify an entity as a related entity are known as referential attributes. 

\mynote For a long time now identifying attributes have been called key attributes and referential attributes have been called foreign key attributes. 

\mynote If you think that each outgoing directional relationship requires a distinct set of referential attributes to support its communication then you would be wrong. 
There is a possibility of `collapsed referentials' (a phrase coined in Shlaer and Long) whereby a single referential attribute may support communication of two or more distinct relationships. If you haven't understood this collapsed referentials concept then you have not  fully and completely understood the nature of data. 

\mynote In the presence of collapsed of referentials the number of bits of information required to communicate all the relationships of an entity is less that the sum of the bits required for their individual communication. 

\mynote the resolved set of attributes

\mynote conceptual core versus data core

\begin{oldtt}
\mynote To communicate the value of an attribute is to communicate its identity as an attribute and the universal that is its value. Both the identity of an attribute and its value are universals and communication of universals is a given i.e. does not require further explanation. 
\end{oldtt}

\begin{noteforfuture}
I imagine describing regular communication schemes... and why there can be a choice of schemes for relationships because of the choice of identifying features of target entity. 
\end{noteforfuture}
\begin{noteforfuture}
I imagine describing the communication of outgoing directional relationships of an entity by way of referentials and subsequently explaining that referentials can collapse which yields the term collapsed referentials.
\end{noteforfuture}