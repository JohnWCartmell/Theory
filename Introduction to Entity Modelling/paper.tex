\documentclass[10pt,a4paper]{article}

%\usepackage{geometry}
\usepackage{pstricks}
\usepackage{pst-node}
\usepackage{pst-tree}
\usepackage{stmaryrd}
\usepackage{amsmath}
\usepackage{amssymb}
\usepackage{verbatim}
\usepackage{enumerate}
\usepackage{calc}
\usepackage{float} % for H option for tables and figures
\usepackage{changepage} % used for adjustwidth
\usepackage{framed}
\usepackage{wasysym}

\usepackage{ulem}   %added because \sout was not striking out

\newcommand{\SharedMacros}{../SharedMacros}
\newcommand{\SharedText}{../SharedText}

%\usepackage{enumitem} % to be able to [resume] an enumeration but clashes with [(i)] !!
\usepackage[margin=4.0cm]{geometry} %was 3cm
\usepackage{mathptmx}
\usepackage{amsfonts}
\usepackage{array}
\usepackage{pstricks}
\usepackage{pst-tree}
\usepackage{pst-plot}
\usepackage{pst-node}
\usepackage{stmaryrd}
\usepackage{amsmath}
\usepackage{verbatim}
\usepackage{graphicx}  
\usepackage{calc}
\usepackage{xifthen}
\usepackage{xcolor}
\usepackage{color}
\usepackage{stringstrings}
%\usepackage[small,bf,margin=3pt,format=hang, labelsep=endash,singlelinecheck=false]{caption} %prevuiously justification=justified
%\usepackage{enumerate}
%\usepackage{enumitem}
\usepackage{enumerate}
\usepackage[shortlabels]{enumitem}
\usepackage{float}
\usepackage[section]{placeins}
%\setlength{\captionmargin}{5pt}
\usepackage{environ}
\usepackage{multirow}
\usepackage{rotating}
\usepackage{longtable}
\usepackage{afterpage}
\usepackage{needspace}


%DEFINE ENVIRONMENT BLOCK
% Riddle
\newsavebox{\riddlebox}

\newenvironment{erexample}
{\newcommand\colboxcolor{F0F0F0}%was F8F8F8
\begin{lrbox}{\riddlebox}
\begin{minipage}{\dimexpr\columnwidth-2\fboxsep\relax} \textbf{} \\ \itshape}
{\end{minipage}\end{lrbox}%
%\begin{center}
\colorbox[HTML]{\colboxcolor}{\usebox{\riddlebox}}
%\end{center}
}

\newenvironment{erbox}
{\newcommand\colboxcolor{F0F0F0}%was F8F8F8
\begin{lrbox}{\riddlebox}%
\begin{minipage}{\dimexpr\columnwidth-2\fboxsep\relax} }
{\end{minipage}\end{lrbox}%
%\begin{center}
\colorbox[HTML]{\colboxcolor}{\usebox{\riddlebox}}
%\end{center}
}

%\begin{erboxedFigure}{#1 FigureParam}{#2 Label}{#3 Caption}
\NewEnviron{erboxedFigure}[3]{%
\begin{figure}[#1]
\begin{erexample}
\begin{center}
\BODY
\end{center}
\vspace{-0.5cm}
\caption{#3}
\label{#2}
\end{erexample}
\end{figure}
}

\newcommand{\erpictureFolder}[0]{../SharedPictures}

\newcommand{\ercenterPicture}[1]{
\begin{center}
\input{\erpictureFolder/#1}
\end{center}
}


\newlength{\erhalfHt}

%\erinlinePicture{#1 pictureFilename}{#2 pictureHeight}
\newcommand{\erinlinePicture}[2]{
\setlength{\erhalfHt}{#2cm * \real{0.5}}
\raisebox{-\erhalfHt}[\erhalfHt + 0.5cm][\erhalfHt + 0.5cm]{
\input{\erpictureFolder/#1}
} 
}

%\erplainFig{#1 pictureFilename}{#2 figureParam}{#3Caption}
\newcommand{\erplainFig}[3]{
\begin{figure}[#2]
\begin{center}
\input{\erpictureFolder/#1}
\end{center}
\caption{#3}
\label{#1}
\end{figure}
}

%\erboxedFigPicture{#1 pictureFilename}{#2 figureParam}{#3Caption}
\newcommand{\erboxedFigPicture}[3]{
\begin{figure}[#2]
\begin{erexample}
\vspace{-0.5cm}
\begin{center}
\input{\erpictureFolder/#1}
\end{center}
\caption{#3}
\label{#1}
\end{erexample}
\end{figure}
}

%\erLeftSideFig{#1 pictureFilename}{#2 figureParam}{#3Caption}
\newcommand{\erLeftSideFig}[3]{
\begin{figure}[#2]
\begin{erexample}
  \begin{minipage}[c]{0.4\textwidth}
    \caption{#3}
    \label{#1}
  \end{minipage}
  \begin{minipage}[c]{0.5\textwidth}
    \input{\erpictureFolder/#1}
  \end{minipage}
\end{erexample}
\end{figure}
}

%\erbulletedFig{#1 pictureFilename}{#2 figureParam}{#3Caption}
\NewEnviron{erbulletedFig}[3]{%
\begin{figure}[#2]
\begin{erexample}
\vspace{-0.5cm}
\begin{center}
$
\begin{array}{c m{0.25cm} | m{6cm}}
\raisebox{-2.0cm}{
\input{\erpictureFolder/#1}}& & \text{\parbox{6cm}{\raggedright{\footnotesize{
\begin{enumerate}[(i)]
\BODY
\end{enumerate}}}}} \\
\end{array}
$
\end{center}
\caption{#3}
\label{#1}
\end{erexample}
\end{figure} 
}


%\begin{erbulletedDimFig}{#1 pictureFilename}{#2figureParam} {#3Caption} {#4PictureHeight}{#5TextWidth}

\NewEnviron{erbulletedDimFig}[5]{%
\begin{figure}[#2]
\begin{erexample}
\vspace{-0.5cm}
\begin{center}
$
\begin{array}{c m{0.25cm} |  m{#5cm}}
\setlength{\erhalfHt}{#4cm * \real{0.5}}
\raisebox{-\erhalfHt}{
\input{\erpictureFolder/#1}}& & \text{\parbox{#5cm}{\raggedright{\footnotesize{
\begin{enumerate}[(i)]
\BODY
\end{enumerate}}}}} \\
\end{array}
$
\end{center}
\caption{#3}
\label{#1}
\end{erexample}
\end{figure} 
}

%\begin{ernotedModel}{#1 pictureFilename}{#2PictureHeight}{#3PictureWidth}{#4TextWidth}

\NewEnviron{ernotedModel}[4]{%
\begin{center}
$
\begin{array}{m{#3cm} m{1cm} | c m{#4cm}}
\setlength{\erhalfHt}{#2cm * \real{0.5}}
\raisebox{-\erhalfHt}{
\input{\erpictureFolder/#1}}& & & \text{\parbox{#4cm}{\raggedright{\footnotesize{
\BODY
}}}} \\
\end{array}
$
\end{center} 
}

%\begin{ermodelText}{#1 pictureFilename}{#2PictureHeight}{#3PictureWidth}{#4TextWidth}

\NewEnviron{ermodelText}[4]{%
\begin{center}
\begin{tabular}{m{#3cm} m{1cm}  c m{#4cm}}
\setlength{\erhalfHt}{#2cm * \real{0.5}}
\raisebox{-\erhalfHt}{
\input{\erpictureFolder/#1}}& & & \text{\parbox{#4cm}{\raggedright{\small{
\BODY
}}}} \\
\end{tabular}
\end{center} 
}


%\erbulletedModel{#1 pictureFilename}{#2PictureHeight}{#3PictureWidth}{#4TextWidth}

\NewEnviron{erbulletedModel}[4]{%
\begin{center}
$
\begin{array}{m{#3cm} m{1cm} | c m{#4cm}}
\setlength{\erhalfHt}{2cm * \real{0.5}}
\raisebox{-\erhalfHt}{
\input{\erpictureFolder/#1}}& & & \text{\parbox{#4cm}{\raggedright{\footnotesize{
\begin{enumerate}[(i)]
\BODY
\end{enumerate}}}}} \\
\end{array}
$
\end{center} 
}



%\ernotedDimFig{#1 pictureFilename}{#2 figureParam}{#3Caption}{#4PictureHeight}{#5TextWidth}
\NewEnviron{ernotedDimFig}[5]{%
\begin{figure}[#2]
\begin{erexample}
\vspace{-0.5cm}
\begin{center}
$
\begin{array}{c m{0.25cm} | c m{#5cm}}
\setlength{\erhalfHt}{#4cm * \real{0.5}}
\raisebox{-\erhalfHt}{
\input{\erpictureFolder/#1}}& & & \text{\parbox{#5cm}{\raggedright{\footnotesize{
\BODY }}}}\\
\end{array}
$
\end{center}
\caption{#3}
\label{#1}
\end{erexample}
\end{figure} 
}
%\begin{ernotedDimFigPW}{#1 pictureFilename}{#2 figureParam}{#3Caption}{#4PictureHeight}{#5PictureWidth}{#6TextWidth}
\NewEnviron{ernotedDimFigPW}[6]{%
\begin{figure}[#2]
\begin{erexample}
\vspace{-0.5cm}
\begin{center}
$
\begin{array}{>{\centering}m{#5cm} m{0.5cm} | c m{#6cm}}
\setlength{\erhalfHt}{#4cm * \real{0.5}}
\raisebox{-\erhalfHt}{
\centering \input{\erpictureFolder/#1}
}& & & \text{\parbox{#6cm - 0.5cm}{\raggedright{\footnotesize{
\BODY }}}}\\
\end{array}
$ \\
\vspace {0.2cm}
\end{center}
\caption{#3}
\label{#1}
\end{erexample}
\end{figure}
}



\newenvironment{erquote}
{\begin{quote}\itshape}
{\end{quote}}



%ccategories.macros.tex 

% Macros for diagrams in contextual categories and related categories

\usepackage{twoopt}
\usepackage{scalerel} 
\usepackage{xargs}

%\usepackage{mathabx}  %Caused font problems
%\usepackage{MnSymbol}  % caused font problems

\newcommand{\conu}
{\mathbf{C}(U)}

\newcommand{\depu}
{\mathbf{D}(U)}

\newcommand{\cat}[1]{\textbf{#1}}
\newcommand{\obj}[1]{\ensuremath{|\cat{#1}|}}
\newcommand{\ccat}[1][C]{\ensuremath{\mathbb{#1}} }
\newcommand{\ccatc}{contextual category \ccat}
\newcommand{\cobj}[2][]{\ensuremath{|\ccat[#2]|_{#1}}}
\newcommand{\cslice}[2]{\ensuremath{\ccat[#1]_{#2}}}
\newcommand{\csliceobj}[3][]{\ensuremath{|\mathbb{#2}_{#3}|_{#1} }}
\newcommand{\varset}[1][]{\ensuremath{V_{#1} }}
\newcommand{\localvarsets}{\ensuremath{\mathcal{V} }}
\newcommand{\Fam}{\ensuremath{\mathbb{F\mathrm{am}} }}
\newcommand{\Famslice}[1]{\ensuremath{\mathbb{F\mathrm{am}}_{#1} }}
\newcommand{\Famobj}[1][]{\ensuremath{|\mathbb{F\mathrm{am}}|_{#1} }}
\newcommand{\Famsliceobj}[2][]{\ensuremath{|\mathbb{F\mathrm{am}}_{#2}|_{#1} }}
\newcommand{\morph}{\rightarrow}
\newcommand{\epi}{\twoheadrightarrow}
\newcommand{\base}{\triangleleft}
\newcommand{\comp}{\circ}
\newcommand{\cross}{\otimes}
\newcommand{\pc}[2]{d^{#1}_{#2}}
\newcommand{\sub}{^*}
\newcommand{\diag}{\delta}
\newcommand{\pbase}[1]{\tilde{#1}}

\newcommand{\tuple}[1]{\langle#1\rangle}
\newcommand{\ndidly}{\ensuremath{\Join_n}}
\newcommand{\ndidlycospan}{quotiented n-cospan}

\newcommand{\crossx}[3]{#1 \underset{#3}{\cross} #2}
\newcommand{\fibrex}[3]{#1 \underset{#3}{\Join} #2}
\newcommand{\powerset}{\mathcal{P}}
\newcommand{\primeds}[1]{
\ensuremath{\mathcal{P}(#1)} }
\newcommand{\compset}{\ \dot{\circ}\, }

% darrow
%\newcommand{\darrow}{\rightarrowtriangle} %use \smorph instead
\newcommand{\smorph}{\rightarrowtriangle}

 

\newcommand\dhead{\scaleobj{0.6}{\triangleright}}
\newcommand{\dmorph}{\, \mbox{---} \! \cdot \! \raisebox{1.1pt}{\dhead}}

% projection tree
%\newcommand{\proj}[2]{proj_{#2}(#1)}

\newcommand{\proj}[2]{
\ensuremath{\mathcal{P}_{#2}(#1)} }

%pstrick supplements for arrows

\newlength{\arrnodesepA}
\newlength{\arrnodesepB}
\newlength{\arroffsetA}
\newlength{\arroffsetB}

%Modified to 2pt from 0pt on 23 July 2018
\newcommand{\arreset}{
\setlength{\arrnodesepA}{2pt}
\setlength{\arrnodesepB}{2pt}
\setlength{\arroffsetA}{0pt}
\setlength{\arroffsetB}{0pt}
}
\arreset

\newcommand{\ncarr}[3][0]{\ncarc[arcangle=#1,nodesepA=\arrnodesepA,nodesepB=\arrnodesepB,offsetA=\arroffsetA,offsetB=\arroffsetB,arrowsize=5pt,arrowinset=0.7]{->}{#2}{#3}}
\newcommand{\jcbarr}[4][0]{ % ncbarr is defined in some thridy party package so do not use!\emph{}
\ncarr[#1]{#3}{#4}
\nbput[labelsep=2pt]{\footnotesize $#2$}
}

\newcommand{\ncaarr}[4][0]{
\ncarr[#1]{#3}{#4}
\naput[labelsep=2pt]{\footnotesize $#2$}
}

% \alabel{label}[npos][labelsep_pts]
\newcommandx*\alabel[3][2=0.5,3=2,usedefault]{\naput[labelsep=#3pt,npos=#2]{\footnotesize $#1$}}
% \blabel{label}[npos][labelsep_pts]
\newcommandx*\blabel[3][2=0.5,3=2,usedefault]{\nbput[labelsep=#3pt,npos=#2]{\footnotesize $#1$}}

% \idcomp mark an arrow as one component of an identifier
\newcommand{\idcomp}{\ncput[npos=0, nrot=:U]{\psline(0.2,-0.075)(0.2,0.075)}}  %add a bar to a node connection arrow
% pstrick supplements for s-arrows (previous name for d-arrow - should convert}

\newlength{\sarnodesepA}
\newlength{\sarnodesepB}
\newlength{\saroffsetA}
\newlength{\saroffsetB}
\newlength{\sarnodesepAsav}
\newlength{\sarnodesepBsav}

\newcommand{\sarreset}{
\setlength{\sarnodesepA}{0pt}
\setlength{\sarnodesepB}{0pt}
\setlength{\saroffsetA}{0pt}
\setlength{\saroffsetB}{0pt}
}

\sarreset

% sar - S-arrow
\newcommand{\ncsar}[3][0]{
\setlength{\sarnodesepAsav}{\sarnodesepA}
\setlength{\sarnodesepBsav}{\sarnodesepB}
\addtolength{\sarnodesepA}{3pt}
\addtolength{\sarnodesepB}{7pt}
\ncarc[nodesepA=\sarnodesepA,nodesepB=\sarnodesepB,offsetA=\saroffsetA,offsetB=\saroffsetB,arcangle=#1]{-}{#2}{#3}
\ncput[nrot=:R,npos=1]{\pstriangle(0,0)(.2,.2)}
\setlength{\sarnodesepA}{\sarnodesepAsav}
\setlength{\sarnodesepB}{\sarnodesepBsav}
}


% bsar - below labelled S-arrow
\newcommand{\ncbsar}[4][0]{
\ncsar[#1]{#3}{#4}
\nbput[labelsep=2pt]{\footnotesize $#2$}
}
% asar - above labelled S-arrow
\newcommand{\ncasar}[4][0]{
\ncsar[#1]{#3}{#4}
\naput[labelsep=2pt]{\footnotesize $#2$}
}

% cdar - composite dependency arrow
\newcommand{\nccdar}[3][0]{
\setlength{\sarnodesepAsav}{\sarnodesepA}
\setlength{\sarnodesepBsav}{\sarnodesepB}
\addtolength{\sarnodesepA}{3pt}
\addtolength{\sarnodesepB}{11pt}
\ncarc[nodesepA=\sarnodesepA,nodesepB=\sarnodesepB,offsetA=\saroffsetA,offsetB=\saroffsetB,arcangle=#1]{-}{#2}{#3}
\ncput[nrot=:R,npos=1]{\pstriangle(0,0.1)(.2,.2)}
\ncput[nrot=:R,npos=1]{\psdot[dotsize=1pt](-0.0075,0.05)}   %!!
\setlength{\sarnodesepA}{\sarnodesepAsav}
\setlength{\sarnodesepB}{\sarnodesepBsav}
}


% bcdar - below labelled composite dependency arrow
\newcommand{\ncbcdar}[4][0]{
\nccdar[#1]{#3}{#4}
\nbput[labelsep=2pt]{\footnotesize $#2$}
}
% acdar - above labelled composite dependency arrow
\newcommand{\ncacdar}[4][0]{
\nccdar[#1]{#3}{#4}
\naput[labelsep=2pt]{\footnotesize $#2$}
}


% rsar - recursive S-arrow
\newcommand{\ncrsar}[2]{
\setlength{\sarnodesepAsav}{\sarnodesepA}
\setlength{\sarnodesepBsav}{\sarnodesepB}
\addtolength{\sarnodesepA}{3pt}
\addtolength{\sarnodesepB}{7pt}
\ncloop[nodesepA=\sarnodesepA,nodesepB=\sarnodesepB,
        offsetA=\saroffsetA,offsetB=\saroffsetB,
        armA=0.7cm,armB=0.6cm,angleA=90,angleB=-90,loopsize=-1,linearc=0.4
				]{-}{#1}{#2}
\ncput[nrot=:R,npos=5]{\pstriangle(0,0)(.2,.2)}
\setlength{\sarnodesepA}{\sarnodesepAsav}
\setlength{\sarnodesepB}{\sarnodesepBsav}
}

% pstrick supplements for multi-arrows

\newlength{\marnodesepA}
\newlength{\marnodesepB}
\newlength{\maroffsetB}
\newlength{\marnodesepBsav}

\newcommand{\marreset}{
\setlength{\marnodesepA}{0pt}
\setlength{\marnodesepB}{0pt}
\setlength{\maroffsetB}{0pt}
}

\marreset

%ncmarr[#1 arcangle1][#2 arcangle2]{#3 name}{#4 domain1}{#5 domain2}{#6 junction}{#7 codomain}
\newcommandtwoopt{\ncmarr}[6][8][8]{%
\ncarc[nodesepA=\marnodesepA,nodesepB=0,arcangle=#1]{-}{#3}{#5}
\ncarc[nodesepB=0,arcangle=-#1]{-}{#4}{#5}
\ncarc[arcangle=#2,nodesepB=\marnodesepB,offsetB=\maroffsetB]{->}{#5}{#6}
}%


\newcommandtwoopt{\nchmarr}[6][8][8]{%
\ncarc[nodesepA=\marnodesepA,nodesepB=0,arcangle=#1]{-}{#3}{#5}
\ncarc[nodesepB=0,arcangle=#1]{-}{#4}{#5}
\ncarc[arcangle=#2,nodesepB=\marnodesepB,offsetB=\maroffsetB]{->}{#5}{#6}
}%

\newcommandtwoopt{\ncamarr}[7][8][8]{%
\ncmarr[#1][#2]{#4}{#5}{#6}{#7}
\naput[npos=.05]{$#3$}
}%
\newcommandtwoopt{\ncbmarr}[7][8][8]{%
\ncmarr[#1][#2]{#4}{#5}{#6}{#7}
\nbput[npos=.05]{$#3$}
}%

\newcommandtwoopt{\ncbhmarr}[7][8][8]{%
\nchmarr[#1][#2]{#4}{#5}{#6}{#7}
\nbput[npos=.05]{$#3$}
}%

\newcommandtwoopt{\ncmarrr}[7][8][8]{
\ncarc[nodesepB=0,arcangle=#1]{-}{#3}{#6}
\ncline[nodesepB=0]{-}{#4}{#6}
\ncarc[nodesepB=0,arcangle=-#1]{-}{#5}{#6}
\ncarc[nodesepA=0,arcangle=#2]{->}{#6}{#7}
}

\newcommandtwoopt{\ncamarrr}[8][8][8]{
\ncmarrr[#1][#2]{#4}{#5}{#6}{#7}{#8}
\naput[npos=.05]{$#3$}
}
\newcommandtwoopt{\ncbmarrr}[8][8][8]{
\ncmarrr[#1][#2]{#4}{#5}{#6}{#7}{#8}
\nbput[npos=.05]{$#3$}
}

%gats.macros.tex

\usepackage{environ}    % also used in ermacros % here used for \NewEnvrion

\newcommand{\gat}[1][U]{
\ensuremath{\mathcal{#1}}}  % used to hav a space in here
\newcommand{\gatw}[1][U]{\gat[#1]\ }  % use this if need trailing space
\newcommand{\ingat}[1][U]{in \gat[#1]}
\newcommand{\isagat}[1][U]{\gat[#1] is a g.a.t.}
\newcommand{\inagat}{in a g.a.t. }

% macro for a generic theory
%\newcommand{\theory}
%{\textit{U}}

\newcommand{\intheory}
{is a derived rule of \gat[U]}

% Macros for GAT rules

\newcommand{\isT}[1]
{#1\mbox{ is a type}}

\newcommand{\ofT}[2]
{#1 \in #2
}

% Macros for GAT rules   <!-- new old -->
\newcommand{\istype}[1]
{#1\mbox{ is a type}}

\newcommand{\oftype}[2]
{#1 \in #2
}

%\context{x}{\Delta}{n}
\newcommand{\context}[3]
{\ofT{#1_1}{#2_1},... \ofT{#1_{#3}}{#2_{#3}(#1_1,...#1_{#3-1})}
}

%\subcontext{x}{\Delta}{i}{k}
\newcommand{\subcontext}[4]
{\ofT{#1_{#3_1}}{#2_{#3_1}},... \ofT{#1_{#3_#4}}{#2_{#3_#4}(#1_1,...#1_{#3_#4-1})}
}

% #schematic context
\newcommand{\schmcon}[3]
{\ofT{#1_1}{#2_1},... \ofT{#1_{#3}}{#2_{#3}}
}
% abbreviated to
\newcommand{\con}[3]
{\schmcon{#1}{#2}{#3}}

% schematic subcontext
%\subcon{x}{\Delta}{i}{k}
\newcommand{\subcon}[4]
{\ofT{#1_{#3_1}}{#2_{#3_1}},... \ofT{#1_{#3_#4}}{#2_{#3_#4}}
}

% permuted context
%\permcon{x}{\Delta}{n}{\sigma}
\newcommand{\permcon}[4]
{\ofT{#1_{#4(1)}}{#2_{#4(1)}},... \ofT{#1_{#4(#3)}}{#2_{#4(#3)}}
}
% permuted term
%\permterm{t}{n}{\sigma}
\newcommand{\permterm}[3]
{
#1_{#3(1)},...#1_{#3(#2)}
}


% Idioms
\newcommand{\xDelta}[1]{\con{x}{\Delta}{#1}}
\newcommand{\xDeltap}[1]{\con{x}{\Delta'}{#1}}
\newcommand{\xOmega}[1]{\con{x}{\Omega}{#1}}
\newcommand{\xOmegap}[1]{\con{x}{\Omega'}{#1}}
\newcommand{\yOmega}[1]{\con{y}{\Omega}{#1}}
\newcommand{\yOmegap}[1]{\con{y}{\Omega'}{#1}}

\newcommand{\xDeltasigma}[1]{\permcon{x}{\Delta}{#1}{\sigma}}
\newcommand{\xDeltapsigma}[1]{\permcon{x}{\Delta'}{#1}{\sigma}}
\newcommand{\xOmegasigma}[1]{\permcon{x}{\Omega}{#1}{\sigma}}
\newcommand{\xOmegapsigma}[1]{\permcon{x}{\Omega'}{#1}{\sigma}}
\newcommand{\yOmegasigma}[1]{\permcon{y}{\Omega}{#1}{\sigma}}
\newcommand{\yOmegapsigma}[1]{\permcon{y}{\Omega'}{#1}{\sigma}}

\newcommand{\xDeltainvsigma}[1]{\permcon{x}{\Delta}{#1}{\sigma^{-1}}}
\newcommand{\xDeltapinvsigma}[1]{\permcon{x}{\Delta'}{#1}{\sigma^{-1}}}
\newcommand{\xOmegainvsigma}[1]{\permcon{x}{\Omega}{#1}{\sigma^{-1}}}
\newcommand{\xOmegapinvsigma}[1]{\permcon{x}{\Omega'}{#1}{\sigma^{-1}}}
\newcommand{\yOmegainvsigma}[1]{\permcon{y}{\Omega}{#1}{\sigma^{-1}}}
\newcommand{\yOmegapinvsigma}[1]{\permcon{y}{\Omega'}{#1}{\sigma^{-1}}}

%Idioms enclosed as tuples
\newcommand{\encxDelta}[1]{\tuple{\con{x}{\Delta}{#1}}}
\newcommand{\encxDeltap}[1]{\tuple{\con{x}{\Delta'}{#1}}}
\newcommand{\encxOmega}[1]{\tuple{\con{x}{\Omega}{#1}}}
\newcommand{\encxOmegap}[1]{\tuple{\con{x}{\Omega'}{#1}}}
\newcommand{\encyOmega}[1]{\tuple{\con{y}{\Omega}{#1}}}
\newcommand{\encyOmegap}[1]{\tuple{\con{y}{\Omega'}{#1}}}

\newcommand{\encxDeltasigma}[1]{\tuple{\permcon{x}{\Delta}{#1}{\sigma}}}
\newcommand{\encxDeltapsigma}[1]{\tuple{\permcon{x}{\Delta'}{#1}{\sigma}}}
\newcommand{\encxOmegasigma}[1]{\tuple{\permcon{x}{\Omega}{#1}{\sigma}}}
\newcommand{\encxOmegapsigma}[1]{\tuple{\permcon{x}{\Omega'}{#1}{\sigma}}}
\newcommand{\encyOmegasigma}[1]{\tuple{\permcon{y}{\Omega}{#1}{\sigma}}}
\newcommand{\encyOmegapsigma}[1]{\tuple{\permcon{y}{\Omega'}{#1}{\sigma}}}

\newcommand{\encxDeltainvsigma}[1]{\tuple{\permcon{x}{\Delta}{#1}{\sigma^{-1}}}}
\newcommand{\encxDeltapinvsigma}[1]{\tuple{\permcon{x}{\Delta'}{#1}{\sigma^{-1}}}}
\newcommand{\encxOmegainvsigma}[1]{\tuple{\permcon{x}{\Omega}{#1}{\sigma^{-1}}}}
\newcommand{\encxOmegapinvsigma}[1]{\tuple{\permcon{x}{\Omega'}{#1}{\sigma^{-1}}}}
\newcommand{\encyOmegainvsigma}[1]{\tuple{\permcon{y}{\Omega}{#1}{\sigma^{-1}}}}
\newcommand{\encyOmegapinvsigma}[1]{\tuple{\permcon{y}{\Omega'}{#1}{\sigma^{-1}}}}

\newcommand{\tstyle}{\vdash}

% gat macros developed for cwf paper

% Expressing gats
\newenvironment{gatrules}
{
$$
\begin{array}{l l}
}
{
\end{array}
$$
}
\newcommand{\gatintros}
{
\textbf{Symbol} & \textbf{Introductory\ Rule}                      \\}

\newcommand{\gataxioms}
{\textbf{Axioms}\\}
\newcommand{\gatintro}[3]{\ #1 & #2 \tstyle #3 \\}
\newcommand{\gatlocalintro}[3]{\ #1 & #2 \dashv }
\newcommand{\gataxiom}[2]{\multicolumn{2}{l}{\ \ #1\mbox{,  whenever\ } #2} \\}
\newcommand{\noleft}{\left.\kern-\nulldelimiterspace} % so that no space taken by absent left brace


\newcommand{\gatmultiaxiom}[2]
{\multicolumn{2}{l}{
  \noleft
    \begin{array}{l}
		#1
    \end{array} 
  \right\} \mbox{whenever\ } 	#2 
	}\\}
	
	\newcommand{\axid}[1]{\text{#1}.\ }	

%New context sharing macros
\newcommand{\gatintroducing}[1]{
{\arraycolsep=0pt
  \begin{array}{l}
          #1
  \end{array}} &
}

%*********************************
% \begin{\gatgroup}{context}
%    rules
%  \end{\gatgroup}
%*********************************
\NewEnviron{gatgroup}[1]{%
  \noleft
  {\arraycolsep=0pt
   \begin{array}{l}
\BODY
    \end{array} 
   }
   \ \right\} 
	%\mbox{\ whenever\ } 
	#1
	\vspace{0.1cm} 
}
%*********************************

%*********************************
% \begin{\gatgroupnoshared}
%    rule
%  \end{\gatgroupnoshared}
%*********************************
\NewEnviron{gatgroupnoshared}{%
  {\arraycolsep=0pt
   \begin{array}{l}
\BODY
    \end{array} 
   }
   \ 
	\vspace{0.1cm} 
}
%*********************************

% \gatsingular[width]{context}{conclusion}
\newcommand{\gatsingular}[3][4cm]{
\begin{gatgroupnoshared}
\gatleaf[#1]{#2}{#3} 
\end{gatgroupnoshared}
}

%*********************************
% \gatleaf}[width]{context}{assertion}
%*********************************
\newcommand{\gatleaf}[3][4cm]{%
\makebox[#1]{$#3$ \dotfill} \dotfill \  #2
}
%*********************************
%*********************************
% \gatstandalonesingle}{context}{assertion}
%*********************************
\newcommand{\gatstandalonesingle}[2]{%
#2 \makebox[2.5cm]{\dotfill} \  #1
}
%*********************************

% \gataxiomno{axiomno}
\newcommand{\gataxiomno}[1]{\makebox[0.5cm]{} \axid{#1}}


% metagat.macros.tex

%Meta-theories

%\newcommand{\typ}{\triangleright}
\newcommand{\typ}{\nabla}
\newcommand{\trm}{\tau}
\newcommand{\cross}{\otimes}
\newcommand{\sub}{^*}
\newcommand{\diag}{\delta}

\newcommand{\typeseq}[2]
{\ofT{#1_1}{\typ},... \ofT{#1_{#2}}{\typ(#1_{#2-1})}}

\newcommand{\typeseqcont}[3]
{\ofT{#1_1}{\typ({#2})},... \ofT{#1_{#3}}{\typ(#1_{#3-1})}}

\newcommand{\Ob}{Ob}
\newcommand{\obj}{Ob} % <!-- new old --<
\newcommand{\Hom}{Hom}
\newcommand{\objseq}[2]
{\ofT{#1_1}{\obj},... \ofT{#1_{#2}}{\obj(#1_{#2-1})}}


\def\dottededge{\ncline[linestyle=dotted, nodesep=0.3cm]}
\def\noedge{\ncline[linestyle=none]}
\def\thinedge{\ncline[linewidth=0.4pt]}

\newcommand{\member}[1]
{\ncarc[arcangle=-30,nodesepB=0.03]{->}{\pspred}{\pssucc}
\nbput[labelsep=0.1]{#1}}

\newcommand{\loweraccutemember}[1]
{\ncarc[arcangle=-15,nodesepB=0.03]{->}{\pspred}{\pssucc}
\nbput[labelsep=0.05,npos=0.85]{#1}}

\newcommand{\uppermember}[1]
{\ncarc[arcangle=30,nodesepB=0.03]{->}{\pspred}{\pssucc}\naput{#1}}

\newcommand{\upperaccutemember}[1]
{\ncarc[arcangle=10,nodesepB=0.03]{->}{\pspred}{\pssucc}\naput[npos=0.85]{#1}}

% flexbranch 
% #1 node label
% #2 thislevelsep
% #3 next level sep
% #4 variable (eg x)
% #5 index leter (eg n)
% #6 close parenthesis
% #7 continuation branches
\newcommand{\flexbranch}[7]
{
\pstree[thislevelsep=*#2,nodesep=0.05]
		{\Rnode{#1 1}{\Tr{#4_1 #6}}}
	  {\pstree[thislevelsep=#3]  
				   {\Rnode{#1 2}{\Tr[edge=\dottededge]{#4_{#5} #6}}}
					 {#7}
		}
}

\newcommand{\flexbranchplusleaf}[6]
{
\flexbranch{#1}{#2}{#3}{#4} {#5} {#6}
  {
   %\Rnode{#1 3}{\Tr{#4 #6}}
	 \Tr{\Rnode{#1 3}{#4 #6}}
  }
}

\newcommand{\flexbranchplusarc}[7]
{
\flexbranch{#1}{#2}{#3}{#4} {#5} {#6}
  {
   %\Rnode{#1 3}{\Tr{#4 #6}\member{#7}}
	 \Tr{\Rnode{#1 3}{#4 #6}}\member{#7}
  }
}

\newcommand{\flexbranchinitialarc}[9]
{
\pstree[thislevelsep=*#2,nodesep=0.05]
		{\Rnode{#1 1}{\Tr{#4_#8 #6}}#9}
	  {\pstree[thislevelsep=#3]  
				   {\Rnode{#1 2}{\Tr[edge=\dottededge]{#4_{#5} #6}}}
					 {#7}
		}
}

\newcommand{\equality}[2]
{
\ncline [doubleline=true, nodesep=0.2cm]{#1}{#2}
}
\newcommand{\equalityarc}[2]
{
\ncarc [arcangleA=-30, arcangleB=-20, doubleline=true, nodesep=0.1cm]{#1}{#2}
}

%
%  erdiag
%
  
%\begin{erdiagram}{#1 height}{#2 width} 
% ....
% ....
%\end{erdiagram}
\newenvironment{erdiagram}[2]
{%\pspicture*(-#1,0)(#2,0)
\pspicture*(0,-#1)(#2,0)
%\psgrid
}
{\endpspicture}

\definecolor{lightyellow}{cmyk}{0,0,0.3,0}

% \eret{#1 x0} {#2 y0} {#3 x1} {#4 y1} {#5 corner radius} {#6 fill}
\newcommand {\eret}[6]
{ 
\ifthenelse{\equal{#6}{1}}
{\psframe[framearc=#5,fillstyle=solid,fillcolor=lightyellow](#1,#2)(#3,#4)}
{\psframe[framearc=#5,fillstyle=solid,fillcolor=white](#1,#2)(#3,#4)}
}

% et top 
\newcommand {\erettop}[4]
{
%\psframe[linestyle=none,linearc=2pt,cornersize=absolute,fillstyle=solid,fillcolor=lightyellow](#1,#2)(#3,#4)
\psline[linearc=2pt,fillstyle=none,fillcolor=lightyellow](#1,#4)(#1,#2)(#3,#2)(#3,#4)
}

% et bottom 
\newcommand {\eretbtm}[4]
{
%\psframe[linestyle=none,linearc=2pt,cornersize=absolute,fillstyle=solid,fillcolor=lightyellow](#1,#2)(#3,#4)
\psline[linearc=2pt,fillstyle=none,fillcolor=lightyellow](#1,#2)(#1,#4)(#3,#4)(#3,#2)
}

% et bottom left
\newcommand {\eretbl}[4]
{
%\psframe[linestyle=none,linearc=2pt,cornersize=absolute,fillstyle=solid,fillcolor=lightyellow](#1,#2)(#3,#4)
\psline[linearc=2pt,fillstyle=none,fillcolor=lightyellow](#1,#4)(#3,#4)(#3,#2)
}

% et middle left
\newcommand {\eretml}[4]
{
%\psframe[linestyle=none,linearc=2pt,cornersize=absolute,fillstyle=solid,fillcolor=lightyellow](#1,#2)(#3,#4)
\psline[linearc=2pt,fillstyle=none,fillcolor=lightyellow](#1,#2)(#3,#2)(#3,#4)(#1,#4)
}

% et top left
\newcommand {\erettl}[4]
{
%\psframe[linestyle=none,linearc=2pt,cornersize=absolute,fillstyle=solid,fillcolor=lightyellow](#1,#2)(#3,#4)
\psline[linearc=2pt,fillstyle=none,fillcolor=lightyellow](#1,#2)(#3,#2)(#3,#4)
}

% et bottom right
\newcommand {\eretbr}[4]
{
%\psframe[linestyle=none,linearc=2pt,cornersize=absolute,fillstyle=solid,fillcolor=lightyellow](#1,#2)(#3,#4)
\psline[linearc=2pt,fillstyle=none,fillcolor=lightyellow](#1,#2)(#1,#4)(#3,#4)
}

% et middle right
\newcommand {\eretmr}[4]
{
%\psframe[linestyle=none,linearc=2pt,cornersize=absolute,fillstyle=solid,fillcolor=lightyellow](#1,#2)(#3,#4)
\psline[linearc=2pt,fillstyle=none,fillcolor=lightyellow](#3,#4)(#1,#4)(#1,#2)(#3,#2)
}

% et top right
\newcommand {\erettr}[4]
{
%\psframe[linestyle=none,linearc=2pt,cornersize=absolute,fillstyle=solid,fillcolor=lightyellow](#1,#2)(#3,#4)
\psline[linearc=2pt,fillstyle=none,fillcolor=lightyellow](#1,#4)(#1,#2)(#3,#2)
}

% \ergrp{#1 x0} {#2 y0} {#3 x1} {#4 y1} {#5 corner radius} {#6 fill}
% #5 corner radius is unused!
\newcommand {\ergrp}[6]
{ 
\ifthenelse{\equal{#6}{1}}
{\psframe[fillstyle=solid,fillcolor=lightgray](#1,#2)(#3,#4)}
{\psframe[fillstyle=solid,fillcolor=white](#1,#2)(#3,#4)}
}

% \eretname {#1 x left of text} {#2 y top of text} {#3 text}
\newcommand {\eretname}[3]
{
%shift down 0.1 for height of text the anchor at baseline (B)
\rput[bl]{0}(0,-0.1){\rput[Bl]{0}(#1,#2){\footnotesize \textit{#3}}}
}

% \errelarm {#1 x0} {#2 y0} {#3 x1} {#4 y1} {#5 ismandatory} {#6 isconstructed}
\newcommand {\errelarm}[6]
{
\ifthenelse{\equal{#6}{1}}
{
%%\psline[linewidth=0.5pt,linearc=.05,linestyle=dashed,dash=6pt 6pt]{-}(#1,#2)(#3,#4)}
\ifthenelse{\equal{#5}{1}}
{\psline[linewidth=1.5pt,linearc=.05,linecolor=lightgray]{-}(#1,#2)(#3,#4)}
{\psline[linewidth=1.5pt,linearc=.05,linecolor=lightgray,linestyle=dashed,dash=2pt 2pt]{-}(#1,#2)(#3,#4)}
}
{
\ifthenelse{\equal{#5}{1}}
{\psline[linewidth=0.9pt,linearc=.05]{-}(#1,#2)(#3,#4)}
{\psline[linewidth=0.9pt,linearc=.05,linestyle=dashed,dash=2pt 2pt]{-}(#1,#2)(#3,#4)}
}
}

% \errelangle {#1 x0} {#2 y0} {#3 x1} {#4 y1} {#5 x2} {#6 y2} {#7 ismandatory} {#8 isocnstructed}
\newcommand {\errelangle}[8]
{
\ifthenelse{\equal{#8}{1}}
{
%\psline[linewidth=0.5pt,linearc=.1,linestyle=dashed,dash=6pt 6pt]{-}(#1,#2)(#3,#4)(#5,#6)}
\ifthenelse{\equal{#7}{1}}
{\psline[linewidth=1.5pt,linearc=.05,linecolor=lightgray]{-}(#1,#2)(#3,#4)(#5,#6)}
{\psline[linewidth=1.5pt,linearc=.1,linecolor=lightgray,linestyle=dashed,dash=2pt 2pt]{-}(#1,#2)(#3,#4)(#5,#6)}
}
{
\ifthenelse{\equal{#7}{1}}
{\psline[linewidth=0.9pt,linearc=.1]{-}(#1,#2)(#3,#4)(#5,#6)}
{\psline[linewidth=0.9pt,linearc=.1,linestyle=dashed,dash=2pt 2pt]{-}(#1,#2)(#3,#4)(#5,#6)}
}
}

% \ercrowfoot {#1 x0} {#2 y0} {#3 x11} {#4 y11} {#5 x12} {#6 y12} {#7 x13} {#8 y13} {#9 isconstructed}
\newcommand {\ercrowfoot}[9]
{
\ifthenelse{\equal{#9}{1}}
{
\psline[linewidth=1.5pt,linearc=.05,linecolor=lightgray]{-}(#1,#2)(#3,#4)
\psline[linewidth=1.5pt,linearc=.05,linecolor=lightgray]{-}(#1,#2)(#5,#6)
\psline[linewidth=1.5pt,linearc=.05,linecolor=lightgray]{-}(#1,#2)(#7,#8)
}{
\psline[linewidth=0.9pt,linearc=.05]{-}(#1,#2)(#3,#4)
\psline[linewidth=0.9pt,linearc=.05]{-}(#1,#2)(#5,#6)
\psline[linewidth=0.9pt,linearc=.05]{-}(#1,#2)(#7,#8)
}
}


% \eridcomprel{#1 x1}{#2 x2}{#3 y1}{#4 ymid}{#5 y2}
\newcommand {\eridcomprel}[5]
{
\psline[linewidth=0.9pt](#1,#3)(#1,#5)
\psline[linewidth=0.9pt](#2,#3)(#2,#5)
\psline[linewidth=0.9pt](#1,#4)(#2,#4)
}

% \eridrefrel{#1 x1}{#2 xmid}{#3 x2}{#4 y1}{#5 y2}
\newcommand {\eridrefrel}[5]
{
\psline[linewidth=0.9pt](#1,#4)(#3,#4)
\psline[linewidth=0.9pt](#1,#5)(#3,#5)
\psline[linewidth=0.9pt](#2,#4)(#2,#5)
}


% \errelname {#1 x} {#2 y} {#3 text}
\newcommand {\errelname}[3]
{
\rput[l]{0}(#1,#2){\textit{#3}}
}
% \errelseq {#1 x} {#2 y}
\newcommand {\erelseq}[2]
{
}
% \erattr {#1 x} {#2 y} {#3 ismandatory}{#4 idenitfying} {#5 text}
\newcommand {\erattr}[5]
{
\ifthenelse{\equal{#3}{1}}
{\rput[l]{0}(#1,#2){{\tiny $\square$} {\footnotesize \textit{\ifthenelse{\equal{#4}{0}}{\underline{#5}}{#5}}}}}
{\rput[l]{0}(#1,#2){\footnotesize $\circ$ \textit{\ifthenelse{\equal{#4}{0}}{\underline{#5}}{#5}}}}
}

%\ifthenelse{\equal{#4}{1}}
% \ertext {#1 x} {#2 y} {#3 text anchor} {#4 text}
%{\rput[l]{0}(#1,#2){\footnotesize $\circ$ \underline{\textit{#5}}}}
\newcommand {\ertext}[4]
{
\rput[B#3]{0}(#1,#2){{\footnotesize #4}}
}
% \erarc {#1 x0} {#2 y0} {#3 x1} {#4 y1} {#5 x2} {#6 y2} {#7 x3} {#8 y3}
\newcommand {\erarc}[8]
{
\psbezier[showpoints=false]{-}(#1,#2) (#3, #4)(#5,#6) (#7, #8)
}

% \erarc {#1 x0} {#2 y0} {#3 x1} {#4 y1} {#5 x2} {#6 y2} {#7 x3} {#8 y3}
\newcommand {\errelseq}[8]
{
\psbezier[showpoints=false]{-}(#1,#2) (#3, #4)(#5,#6) (#7, #8)
}
% \ertrace {#1 trace}   
\newcommand {\ertrace}[1]
{
}

\usepackage{amsthm} % added 7th April 2018
% theorems.macros.tex

\newtheorem{theorem}{Theorem}[section]
\newtheorem{observation}[theorem]{Observation}
\newtheorem{lemma}[theorem]{Lemma}

\newtheorem{proposition}[theorem]{Proposition}
\newtheorem{corollary}[theorem]{Corollary}
\newtheorem{conjecture}[theorem]{Conjecture}
\newtheorem{numbereddefinition}[theorem]{Definition}

\newenvironment{definition}[1][Definition]{\begin{trivlist}
\item[\hskip \labelsep {\bfseries #1}]}{\end{trivlist}}
\newenvironment{examples}[1][Examples]{\begin{trivlist}
\item[\hskip \labelsep {\bfseries #1}]}{\end{trivlist}}
\newenvironment{example}[1][Example]{\begin{trivlist}
\item[\hskip \labelsep {\bfseries #1}]}{\end{trivlist}}
\newenvironment{remark}[1][Remark]{\begin{trivlist}
\item[\hskip \labelsep {\bfseries #1}]}{\end{trivlist}}

\newenvironment{tageqn}[1]
{
\begin{equation}
\stepcounter{equation}
\label{#1}
\tag{\theequation --#1}
}
{
\end{equation}
}

\newenvironment{axiom}[1]
{
\begin{equation}
\label{#1}
\tag{#1}
}
{
\end{equation}
}

% when the tag is required different from the label eg when has math symbols can use:
\newenvironment{axiomtagged}[2]
{
\begin{equation}
\label{#1}
\tag{#2}
}
{
\end{equation}
}

%visible label
\newcommand{\vlabel}[2][]{\label{#2}#1(\textit{#2}):}





% 
% general.macros.v2
%
% Rename macros that conflict with beamer 31 Aug 2022
% Don't assume an index                   31 Aug 2022

\usepackage{changepage} % used for adjustwidth

\iffalse % 31 Aug 2022
\usepackage{imakeidx}
\usepackage{framed}
\makeindex[name=definitions, title=Index of Definitions]
\makeindex[name=lemmas, title=Index of Lemmas]
\fi

\definecolor{highlight}{cmyk}{0,0,0.7,0}
\newcommand{\commentary}[1]{\marginpar{\footnotesize #1}}
\newcommand{\highlight}[1]{\colorbox{highlight}{#1}}
\newcommand{\whitelight}[1]{\colorbox{white}{#1}}
\newcommand{\term}[1]{\textit{#1}\commentary{\colorbox{lightgray}{\textit{#1}}}\index[definitions]{#1}}
\newcommand{\llabel}[1]{\label{#1}\commentary{\colorbox{pink}{\scriptsize{#1}}}\index[lemmas]{#1}}
\newcommand{\lref}[1]{\ref{#1}\colorbox{pink}{\scriptsize{#1}}\index[lemmas]{#1!use of}}

\newcommand{\daynote}[1]{\commentary{See day notes #1.}}

\newcommand{\newt}[1]{\colorbox{yellow}{#1}}
\newenvironment{newtt}
{  \colorbox{yellow}{$[$ ...} 
}
{  \colorbox{yellow}{... $]$}
}
\newcommand{\oldt}[1]{\colorbox{yellow}{\sout{#1}}}
\newenvironment{oldtt}
{  \colorbox{red}{$[$ ...} 
}
{  \colorbox{red}{... $]$}
}

\newcommand{\reinstatet}[1]{\colorbox{lime}{#1}}
\newenvironment{reinstatett}
{  \colorbox{lime}{$[$ ...}
}
{  \colorbox{lime}{... $]$}
}

\newcommand{\tbd}{\highlight{TBD}}

%ithprojection function
\newcommand{\proji}[1]{\pi_#1}


\newenvironment{aside}
{\begin{framed}
\textbf{Aside}
}
{
\end{framed}
}

\newenvironment{notebox}[1][Note]
{\begin{framed}
\textbf{#1}
}
{
\end{framed}
}

\newenvironment{categoricalaside}
{\begin{framed}
\textbf{Categorical Aside}
}
{
\end{framed}
}

\newenvironment{noteforfuture}
{\begin{framed}
\textbf{Note For Future}
}
{
\end{framed}
}

\newenvironment{myproblem}       %31 Aug 2022
{\begin{framed}
\textbf{Problem}
}
{
\end{framed}
}

\newenvironment{key}
{
\begin{tabular}{c l p{4cm}}
KEY && \\
\hline
}
{
\end{tabular}
}

%  31 Aug 2022
\NewEnviron{tightquote} %italic text indented left and right hand side
{\begin{adjustwidth}{1.5cm}{1.5cm}
\textit{
\BODY
}
\end{adjustwidth}
}

\newcommand{\keyentry}[3]{#1 & #2 & #3 \\} 


%quine quote
\newcommand{\qq}[1]{
\left\ulcorner#1\right\urcorner
}

%single quote
\newcommand{\sq}[1]{
\textnormal{\textquotesingle}#1\textnormal{\textquotesingle}
}

%lower quine quote
\newcommand{\lqq}[1]{
\left\llcorner #1\right\lrcorner
}


%from berkley
\newcommand{\langl}{\begin{picture}(4.5,7)
\put(1.1,2.5){\rotatebox{60}{\line(1,0){5.5}}}
\put(1.1,2.5){\rotatebox{300}{\line(1,0){5.5}}}
\end{picture}}
\newcommand{\rangl}{\begin{picture}(4.5,7)
\put(.9,2.5){\rotatebox{120}{\line(1,0){5.5}}}
\put(.9,2.5){\rotatebox{240}{\line(1,0){5.5}}}
\end{picture}}
\newcommand{\lang}{\begin{picture}(5,7)\put(1.1,2.5){\rotatebox{45}{\line(1,0){6.0}}}\put(1.1,2.5){\rotatebox{315}{\line(1,0){6.0}}}\end{picture}}
\newcommand{\rang}{\begin{picture}(5,7)\put(.1,2.5){\rotatebox{135}{\line(1,0){6.0}}}\put(.1,2.5){\rotatebox{225}{\line(1,0){6.0}}}\end{picture}}
%Try sharper tuple brackets -- except gives errors nested in captions so comment out
%\renewcommand{\tuple}[1]{\lang #1 \rang}

\newcommand{\setsuchthat}[2]{\left\{#1 \ \middle|\ #2\right\}}
\newcommand{\set}[1]{\left\{#1\right\}} 

% one to n - wanton
\newcommand{\wanton}[1]{#1_1,...#1_n}
\newcommand{\n}{1...n}
\newcommand{\fn}{\wanton{f}}
\newcommand{\gn}{\wanton{g}}
\newcommand{\pn}{\wanton{p}}
\newcommand{\qn}{\wanton{q}}
\newcommand{\qnprime}{\wanton{q'}}
\newcommand{\tn}{\wanton{t}}
\newcommand{\xn}{\wanton{x}}
\newcommand{\xnp}{\wanton{x'}}
\newcommand{\yn}{\wanton{y}}
\newcommand{\An}{\wanton{A}}
\newcommand{\Bn}{\wanton{B}}
\newcommand{\Cn}{\wanton{C}}
\newcommand{\ntuple}[1]{\tuple{\wanton{#1}}}
\newcommand{\wantom}[2][]{#2_1,...#2_{m#1}}
\newcommand{\m}{1...m}
\newcommand{\mtuple}[1]{\tuple{#1_1,...#1_m}}
\newcommand{\gm}{\wantom{g}}
\newcommand{\qm}{\wantom{q}}
\newcommand{\sm}[1][]{\wantom[#1]{s}}
\newcommand{\smp}{\wantom{s'}}
\newcommand{\ym}{\wantom{y}}
\newcommand{\Bm}{\wantom{B}}
\newcommand {\bntuple}{\ensuremath{\ntuple{b}}}
\newcommand {\fntuple}{\ensuremath{\ntuple{f}}}
\newcommand {\fnptuple}{\ensuremath{\ntuple{f}}}
\newcommand {\pntuple}{\ensuremath{\ntuple{p}}}
\newcommand {\qntuple}{\ensuremath{\ntuple{q}}}
\newcommand {\qnptuple}{\ensuremath{\ntuple{q'}}}
\newcommand {\qmtuple}{\ensuremath{\mtuple{q}}}
\newcommand {\sntuple}{\ensuremath{\ntuple{s}}}
\newcommand {\xntuple}{\ensuremath{\ntuple{x}}}
\newcommand {\xnptuple}{\ensuremath{\ntuple{x'}}}
\newcommand {\ymtuple}{\ensuremath{\mtuple{y}}}
\newcommand{\idef}[1][n]{1 \leq i \leq #1}
\newcommand{\jdef}[1][m]{1 \leq j \leq #1}
\newcommand{\kdef}[1][l]{1 \leq k \leq #1}
\newcommand{\foreachi}[1][n]{for each $i$, $1 \leq i \leq #1$}
\newcommand{\foreachj}[1][m]{for each $j$, $1 \leq j \leq #1$}
\newcommand{\foreachk}[1][l]{for each $k$, $1 \leq k \leq #1$}
\newcommand{\Foreachi}[1][n]{For each $i$, $1 \leq i \leq #1$}
\newcommand{\Foreachj}[1][m]{For each $j$, $1 \leq j \leq #1$}
\newcommand{\Foreachk}[1][l]{For each $k$, $1 \leq k \leq #1$}
\newcommand{\forsomei}[1][n]{for some $i$, $1 \leq i \leq #1$}
\newcommand{\forsomej}[1][m]{for some $j$, $1 \leq j \leq #1$}
\newcommand{\forsomek}[1][l]{for some $k$, $1 \leq k \leq #1$}
\newcommand{\wherei}[1][n]{where $1 \leq i \leq #1$}
\newcommand{\wherej}[1][m]{where $1 \leq j \leq #1$}
\newcommand{\wherek}[1][l]{where $1 \leq k \leq #1$}


\newcommand{\fundep}[3]{#2 \xrightarrow{#1} #3}  %where does this belong? xxxx
% Following used for notes -- indented numbered paras

\newcounter{para}
\newlength{\oldparindent}
\setlength{\oldparindent}{\parindent} % Save \parindent before of change
\newcommand{\ind}{\hspace*{\oldparindent}}

\newcommand\mynote{                                                 % renamed 31 Aug 2022
%\setlength{\parskip}{0.5\baselineskip} % Definition of `parskip`
\setlength{\parindent}{0pt}
\par\ind\refstepcounter{para}\thepara.\space
\setlength{\parindent}{\oldparindent}
}



%%%%%%%%%%%%%%%%%%%%%%%%%%%%%%%%%
% alternate.beamer.macros.tex
%%%%%%%%%%%%%%%%%%%%%%%%%%%%%%%%%
% This file contains implmentation of macros that are also implmented in
% beamer.macros.tex 
% The implementations here should be used for when papers are being built
% i.e. for non-presentations i.e. for papers.
%%%%%%%%%%%%%%%%%%%%%%%%%%%%%%%%%%%%%%%%%%%%%%%%%%%%%%%%%%%%%%%%%%%%%%%%%%%%%
% Commands to control level of detail in pictures
\newcounter{levelofdetail}
\newcommand{\waitfor}[2]{\ifnum #1 > \value{levelofdetail} \else #2 \fi}
% Setting a high number for level of detail shows most detail
\setcounter{levelofdetail}{10}
% In presentations waitfor is implemented using onslide as follows
%\renewcommand{\waitfor}[2]{\onslide<#1->{#2}} 
%Because of this outside of a waitfor level of detail is 1
% frist level programmed with waitfor should be level 2
%%%%%%%%%%%%%%%%%%%%%%%%%%%%%%%%%%%%%%%%%%%%%%%%%%%%%%%%%%%%%%%%%%%%%%%%%%%%

%indexedsets.macros.tex

% Macros for sets and families of sets
\newlength{\xl}
\newlength{\yb}
\newlength{\xr}
\newlength{\yt}
\newlength{\ytm}
\newlength{\ybm}
\newlength{\dotxl}
\newlength{\dotxr}
\newlength{\dotym}
\newlength{\basex} 
\newlength{\basey} 
\newlength{\childx} 
\newlength{\childy}
\newcommand{\putthreeset}[5][0]{
  \setlength{\xl}{-1.6cm * \real{#2}}
  \setlength{\xr}{1.8cm * \real{#2}}
  \setlength{\yt}{0.55cm * \real{#2}}
  \setlength{\ytm}{0.75cm * \real{#2}}
  \setlength{\yb}{-0.55cm * \real{#2}}
  \setlength{\ybm}{-0.80cm * \real{#2}}
  \setlength{\dotxl}{-1cm * \real{#2}}
  \setlength{\dotxr}{0.9cm * \real{#2}}
  \setlength{\dotym}{0.15cm * \real{#2}}
  %
  \rput{#1}(#3,#4){        
           {\psccurve%[showpoints=true]
                     (\xl ,\yt)(\xl,\yb)(0,\ybm )(\xr,\yb)(\xr,\yt) (0,\ytm)  }
            \dotnode[dotscale=0.4](\dotxl,0){#5l}
            \dotnode[dotscale=0.4](0,\dotym){#5m}
            \dotnode[dotscale=0.4](\dotxr,0){#5r}
            \pnode(0,\ybm){#5c}
           }
}
\newcommand{\puttwoset}[5][0]{
  \setlength{\xl}{-1.0cm * \real{#2}}
  \setlength{\xr}{1cm * \real{#2}}
  \setlength{\yt}{0.55cm * \real{#2}}
  \setlength{\ytm}{0.75cm * \real{#2}}
  \setlength{\yb}{-0.55cm * \real{#2}}
  \setlength{\ybm}{-0.80cm * \real{#2}}
  \setlength{\dotxl}{-0.75cm * \real{#2}}
  \setlength{\dotxr}{0.25cm * \real{#2}}
  %
  \rput{#1}(#3,#4){        
           {\psccurve%[showpoints=true]
                     (\xl ,\yt)(\xl,\yb)(0,\ybm )(\xr,\yb)(\xr,\yt) (0,\ytm)  }
            
            \dotnode[dotscale=0.4](\dotxl,0){#5l} 
            \dotnode[dotscale=0.4](\dotxr,0){#5r}
            \pnode(0,\ybm){#5c}
           }
}

%\putfamilyOfSets[#1 rotation]{#2 basescale}{#3 childscale}{#4 x}{#5 y}{#6 childoffset}{#7nodeprefix}
\newcommand{\putfamilyOfSets}[7][0]{
  \setlength{\basex}{#4}
  \setlength{\basey}{#5}
  \putthreeset[#1]{#2}{\basex}{\basey}{#7BASE} 
  %child 1
  \setlength{\childx} {#4 - (4cm * \real{#3})}
  \setlength{\childy} {#5 + #6}
  \putthreeset[#1]{#3}{\childx}{\childy}{L}
  %child 2
  \setlength{\childy}{\childy + 0.5cm}
  \putthreeset[#1]{#3}{#4}{\childy}{M}
  %child 3
  \setlength{\childx} {#4 + (4cm * \real{#3})}
  \setlength{\childy}{\childy - 0.5cm}
  \putthreeset[#1]{#3}{\childx}{\childy}{R}
  \ncline[nodesep=3pt]{|->}{#7BASEl}{Lc}
  \ncline[nodesep=3pt]{|->}{#7BASEm}{Mc}
  \ncline[nodesep=3pt]{|->}{#7BASEr}{Rc}
}

%putFunction[#1 rotation]{#2 basescale}{#3 childscale}{#4 x}{#5 y}{#6 childoffset}{#7nodeprefix}
\newcommand{\putFunction}[7][0]{
  \setlength{\basex}{#4}
  \setlength{\basey}{#5}
  \putthreeset[#1]{#2}{\basex}{\basey}{#7BASE} 
  %child 1
  %\setlength{\childx} {#4 - (4cm * \real{#3})}
	\setlength{\childx} {#4 }
  \setlength{\childy} {#5 + #6}
  \putthreeset[#1]{#3}{\childx}{\childy}{DEST}
  \ncline[nodesep=3pt]{|->}{#7BASEl}{DESTl}
  \ncline[nodesep=3pt]{|->}{#7BASEm}{DESTm}
  \ncline[nodesep=3pt]{|->}{#7BASEr}{DESTm}
}

\newcommand{\mylinethickness}{0.3pt}
\newcommand{\midheight}{2.325pt}
\newcommand{\myline}[1]{\rule[\midheight]{#1}{\mylinethickness}}
\newcommand{\backtostart}{\rule{-0.24cm}{0pt}}
\newcommand{\crowsfoot}{$>$\backtostart\myline{5.5pt}}
\newcommand{\dash}{\myline{2.4pt}}
\newcommand{\twodash}{\dash\ \dash}
\newcommand{\threedash}{\dash\ \dash\ \dash}
\newcommand{\Ahalfsolid}{\myline{20.3pt}}
\newcommand{\Bhalfsolid}{\myline{13.75pt}}
\newcommand{\AAhalfsolid}{\myline{17pt}}
\newcommand{\barkerEllisA}{\crowsfoot\ \twodash\ \threedash}
\newcommand{\barkerEllisB}{\crowsfoot\ \twodash\ \Bhalfsolid}
\newcommand{\barkerEllisC}{$>$\backtostart\AAhalfsolid\ \threedash}
\newcommand{\barkerEllisD}{$>$\backtostart\Ahalfsolid\Bhalfsolid}
\newcommand{\barkerEllisE}{\dash\ \twodash\ \threedash}
\newcommand{\barkerEllisF}{\dash\ \twodash\ \Bhalfsolid}
\newcommand{\barkerEllisG}{\Bhalfsolid\ \threedash}
\newcommand{\barkerEllisH}{\AAhalfsolid\Bhalfsolid}

\usepackage{hyperref}
\setcounter{equation}{0}

\newcommand{\ImagesFolder}{images}
\renewcommand{\erpictureFolder}[0]{images}
\newcommand{\handCraftedImagesFolder}{handCraftedImages}

\newcommand{\longrightharpoonup}
{-\kern-5pt\rightharpoonup}

\newcommand{\longtwoheadrightarrow}
{-\kern-5pt\twoheadrightarrow}

\newcommand{\chenboxhalfwidth}{1.1}
\newcommand{\chenboxquarterwidth}{0.55}
\newcommand{\chenboxhalfheight}{0.75}
\newcommand{\chenboxquarterheight}{0.375}

% \chenbox{label}{text}
\newcommand{\chenbox}[2]{
	\rput[l](0,\chenboxhalfheight){\pnode{#1N}}
	\rput[l](\chenboxhalfwidth,\chenboxhalfheight){\pnode{#1NE}}
	\rput[l](-\chenboxhalfwidth,\chenboxhalfheight){\pnode{#1NW}}
	\rput[l](0,-\chenboxhalfheight){\pnode{#1S}}
	\rput[l](\chenboxhalfwidth,-\chenboxhalfheight){\pnode{#1SE}}
	\rput[l](-\chenboxhalfwidth,-\chenboxhalfheight){\pnode{#1SW}}
	\rput[l](\chenboxquarterwidth,-\chenboxhalfheight){\pnode{#1SSE}}
	\rput[l](-\chenboxquarterwidth,-\chenboxhalfheight){\pnode{#1SSW}}
	\rput[l](\chenboxhalfwidth,0){\pnode{#1E}}
	\rput[l](-\chenboxhalfwidth,0){\pnode{#1W}}
  	\psframe[fillstyle=solid,fillcolor=white]
  	     (-\chenboxhalfwidth,-\chenboxhalfheight)(\chenboxhalfwidth,\chenboxhalfheight)
  \rput(0,0){\scriptsize #2}
}

% \chendiamond{label}{text}
\newcommand{\chendiamond}[2]{
	\rput[l](0,\chenboxhalfheight){\pnode{#1N}}
	\rput[l](\chenboxquarterwidth,\chenboxquarterheight){\pnode{#1NE}}
	\rput[l](-\chenboxquarterwidth,\chenboxquarterheight){\pnode{#1NW}}
	\rput[l](0,-\chenboxhalfheight){\pnode{#1S}}
	\rput[l](\chenboxhalfwidth,0){\pnode{#1E}}
	\rput[l](-\chenboxhalfwidth,0){\pnode{#1W}}
  	\psdiamond[fillstyle=solid,fillcolor=white](0,0)(\chenboxhalfwidth,\chenboxhalfheight) 
  	\rput(0,0){\scriptsize #2}
}

\newcommand{\chenvaluetyperadius}{0.75}

% \chenvaluetype{label}{text}
\newcommand{\chenvaluetype}[2]{
    \rput[l](0,\chenvaluetyperadius){\pnode{#1N}}
	\pscircle[fillstyle=solid,fillcolor=white](0,0){\chenvaluetyperadius}
	\rput(0,0){\scriptsize #2}
}
% macro used for tagging referential and relationship words
%\syntag{dash}{tagnode}{armBcm}{wordnode}{offsetcm}
%dash=1pt 0pt    for solid line 
%    =1pt 1pt    for dotted line
%    =5pt 3pt    for default dashed style
%    or any other specified dashed style
\newcommand{\syntag}[5]
{\ncangle[nodesepA=6pt, 
          nodesepB=2pt,
          linestyle=dashed,
          dash=#1, 
          offsetA=#5cm, 
          angleA=-90, 
          angleB=90, 
          armB=#3cm, 
          linearc=.2]{#4}{#2}}

\title{Introduction to Entity Modelling}
\author{John Cartmell}
\begin{document}
\maketitle
 
\newcommand{\mysection}[1]{\underline{\hyperref[#1]{#1}}}
\section*{Overview}
\label{Overview}
\addcontentsline{toc}{section}{\nameref{Overview}}
\begin{tabular}{l l p{7cm}}
1 & \mysection{Perspective} & Entity modelling form of concept modelling, more narrowly for describing the structuring of data.\\
\hline
2 & \mysection{EntityModels} & Entities, particulars and universals, attribute and relationship, entity model as structured document, ER and ERA diagrams, Chen origins, diamonds, Barker-Ellis notation, single composite thing, the absolute, parts, hierarchies, structured entity models.\\
\hline
3 & \mysection{Relationships} & Relationship and relationship instance, binary relationships, directional relationships, Chen diamond notation, Barker-Ellis notation, many-many relationships, functional relationships, ternary relationships. \\
\hline
4 & \mysection{Entity Relationship Diagramming} & Generalities TBD. Chen manufacturing example, wikipedia entry `part number'.
                                     Goodland and Slater Car Hire Business on SSADM.\\
\hline
5 & \mysection{PathsofRelationships} & Paths of Relationships. This could move before attributes.\\
\hline
6 & \mysection{Attributes} & attributes as functional relationships, value types.
\\
\hline
7 & \mysection{IdentifyingFeatures} &  set of identifying features\\
\hline
8 & \mysection{DramaticArtsExample} &  single model loosely based on David Hay\\
\hline
9 & \mysection{ReferencingEntities} &  with section(s) on communicating relationships\\
\hline
10 & \mysection{AirportGateExample} &  Discussion --- The Airport Gate example\\

\hline
11 & \mysection{Scope} & The Scope Concept from book.   Schlaer Lang example. \\
\hline
12 & \mysection{CommunicatingConjunctions} & Sharing referentials across conjuncts --- 
                                                 leading to structure of tables\\
13 & \mysection{Conclusion} & I have demonstrated the ubiquity of commutative diagrams of relationships
and indicated how by recognising then,  relational database design can be \textit{right first time}. \\
\hline
13 & \mysection{CoreversusDerivative} &  Core versus derivative, conceptual core, data cores, goodness criteria. The Core and Derivatives? \\
\hline
14 & \mysection{DataModelling}& Database and messge structure, conceptual, logical, physical, Codd oriented history, goodness criteria, normal forms, methodology improvement.\\
\hline
15 & \mysection{TypeInheritance} & specialisation and generalisation, species and genera, meta physics, nestedbox notation. Described as sub-types and super-types in Barker's book and likewise but mentioned as non-standard in SSADM book. Single inheritance versus multiple inheritance.   \\
\hline
16 & \mysection{StructuredEntityModelling} & Chen, PCTE, composition relationships,  top-down style, simple meta-model.\\
\hline
17 & \mysection{TheAbsolute} & Some metaphysics.\\
\hline
18 & DistinguishingCompositionandReference&The Distinction Between Composition and Reference from tutorial part one\\
\hline
\end{tabular}
 \section{Perspective}
\label{Perspective}
At its broadest, entity modelling is a technique and a notation for describing and communicating what is in the world and we set out to present it here from first principles. More usually, a narower view, its purpose is the structuring of data to be stored in information systems; in this book we describe an extended notation and describe its use in information systems development but we do not take this use as the starting point for the presentation nor as a subsequent raison d'etre because to do either would be to obscure the wider view that the purpose of an entity model is to provide a framework for knowledge and that entity modelling is a form of conceptual modelling — a technique for the elaboration of concepts. We describe the technique from first principles and relate to other meta-conceptual systems. 

We extend the notation from that commonly used in the development of information systems by the introduction of various kinds of constraints. This is done in a way strongly influenced by consideration of conceptual patterns brought to the fore in the branch of mathematics known as category theory. We shall demonstrate the significant benefits and transformative effect
of uptake into information systems development methodology of the abstractions and the thinking behind this extended notation. This amounts to a utilisation of the basic concepts of category theory. I cannot over emphasise the significant gains to be made.

To illustrate at the outset what we mean by conceptual modelling, consider the experience of reading into a new subject area and finding terms which seemingly have specific patterns of usage, and, it must be assumed, contextual meanings, but which patterns and meanings are unfamiliar to us. In so reading we are drawn into a systematic and iterative arrangement and a classification of the unfamiliar terms; in this process we will likely distinguish terms for individual things, for types or classes of things, for relations between things and also quantitative and adjectival terms. In this way it is inevitable that we will construct some sort of conceptual model. 
Entity modelling is a particular technique for expressing such models and, indeed, it is a technique and a notation used by information scientists seeking to represent and computerise the sometimes unfamiliar domains in which they work. When asked whether they understand a particular topic, an entity modeller might well affirm they do so only if they can sketch an entity model that frames the topic.

Thinking about understanding a new area and its language, though ‘things themselves’ are the subjects of the text, the language is of the types of things, the relations among the different types of things and the properties that can meaningfully be attributed to them. 

The premises of entity modelling have in them a pragmatic answer to the question what is knowledge? Knowledge, according to the premises of entity modelling, is knowledge about things. For any ‘thing’ the knowledge that we can have of it is a conjunction of elementary pieces each of which is either the fact of an attribution, by which is meant a property a thing has inherent in itself, or else the fact of a relationship with another thing. More precisely, we can have knowledge of the type of a thing and knowing its type is to know both the kind of attributions which may be made of it and the relationships in which it may participate. This is the theory of knowledge according to the entity modeller; it is also the basis of information modelling and therefore it is a pragmatic view of what knowledge is — it is that which can be represented as information in a structured form suitable for representation in a computer system or in a single computer program.

Frequently, computer systems and individual computer programs have as their subjects, everyday if not concrete and physical things, things such as people, accounts, orders, contracts, airline bookings, and so on; other computer programs have as their subjects the structures of molecules, languages, stellar processes or computer programs themselves, or things such as mathematical propositions, relationships in general, not particular, types of things as distinct from the things themselves, and so on. These are the sorts of entities that we are concerned with in this book and various points each such type appear as an entity type within an entity model for illustrative purposes. Other entity types we present in models will include knots within quipus, atoms within molecular structures, adjectives within the sentence structure of the English. In this book all these types of things appear in specific entity models and each entity model describes a particular domain of discourse. 
In each specific case this domain of discourse is the context within which there are types of things defined by the relations between one another. In illustrative fragments we have chickens and their eggs and bicycles and their wheels. 
In some cases we lift examples from the literature of relational data theory and get a deeper understanding
of them by looking at them through the lens of entity relationship modelling.




 

\section{Entity Models}
\label{EntityModels}
\mynote The word `entity' as it is used in this book is used in its most general sense,  the sense in which it just means `thing'. The entities that are being modelled in entity modelling can be just about any `things' at all --- and so we could just as well speak of `modelling things' as `entity modelling' but that the term entity modelling
has come to mean a particular way of modelling things in which things are described by describing their types in terms of the binary relationships that things of the type may participate in.

\mynote There is a proviso to this. 
Things that are known from the beginning and unchanging from one context to another, things that can be said to be `universal' are excluded from being subjects of our modelling. We do not model, therefore, whole numbers, nor real numbers, nor truth values 
(the things known to programmers as `booleans') but take these as universal givens and, generally, 
we do not model language characters and language character sequences 
(known to computer programmers as `strings') with the richness that this would require -- we take these too to be universal and given. 

\mynote In philosophy the non-universal things that are our entities are referred to as particulars and so we can say that the entity types within an entity model represent types of particular things i.e. those types all of whose instances are particulars.  

\mynote Now I am able to explain that, in entity modelling, the term attribute is adopted as a specific term meaning a relationship between a particular on the one hand and a universal on the other. Likewise, the term relationship 
in its primary use, in entity modelling, is reserved specifically for relationships between particulars i.e. between entities. 

\mynote Having explained this much we can now say that an entity model posits a collection of entity types, relationships and attributes. These are the E, the R, and the A of entity modelling and give rise to the acronyms ER and ERA and the description of entity modelling as ER modelling or ERA modelling.
Each entity model can be thought of as a structured document containing definitions of such  
E, R and A i.e. of such types, relationships and attributes. This is much  as a dictionary or glossary has entries giving definitions of (certain) nouns, verbs, adjectives as well as other parts of speech. In some circles the term data dictionary is used. Like an architect's design, or an electrician's circuit, much of an entity model can be usefully represented on one or more diagrams. The diagrams are often known as entity-relationship (ER) diagrams or sometimes as  entity-relationship-attribute (ERA) diagrams. 
These diagrams in one or other of the available styles are the visible and recognisable face of entity modelling. They originate with the diagrams of Chen from his 1976 paper which, most significantly, has the title ``The Entity-Relationship Model---Toward a Unified View of Data''. His style of diagrams are recognisable by their use of diamond shaped boxes to represent relationships. In this book our diagrams are more structured. They use Barker-Ellis style diagrams and, as we come to shortly, these are recognisable by their use of optionally half dashed lines to represent relationships and their optional use of crowsfeet on the ends of these lines to represent multiplicity. 

\mynote Entity modelling may have as its subject about any thing but it is never about modelling everything.
An entity model serves to describe a particular domain of discourse. In the general case it describes the entities in this domain in relation not just to each other but to other given types representing that which is universal. 
\mynote
Now a very  important point ---
\textit{any entity model describes the structure of a single composite thing which all the individual things that it describes are part of}. Now it is not usual to speak of entity models in this way which is why I draw attention to this and if you are already familiar with entity modelling you may find this controversial. Bear with me.
\mynote
Like other things this composite thing is described by defining the type of thing that it is. This composite is usually in the background in entity modelling but in our entity models and in our diagrams we bring it to the foreground and represent it as special type on the diagram. 
This, in context, represents the whole of everything.\footnote{Another way of saying this is that in the internal logic of the model it represents the whole of everything. This, in the internal logic is one and the same with the absolute. Subtly, an entity model determines that which is to be considered absolute.} As we come to later and for good reason we like to call this whole of everything \textit{the absolute}. \commentary{fine tune this discussion}  This composite thing  is the subject of an entity model and it can be a small thing such as a single molecular structure so that the whole of everything consists of atoms and covalent bonds or, maybe, the state of a game of chess so that the whole of everything is the board and the positions of pieces on the board. It could be the parse tree of a single grammatical sentence 
so that the individual entities are syntactic constituents. 
It could be a company such as a manufacturing company and all its suppliers, customers, employees, departments, projects and premises. 
It could be an online business and all of its products, customers and sales or
a science company and all of its laboratories, equipment, samples, analytic instruments, assignations, data records and  scientific reports. 
\mynote
Describing a composite thing we are led to its parts. Often we find that these parts are composites and themselves have parts. The result is a hierarchy of things. Many of the  entity models in this book describe such hierarchies abnd for this reason we call them structured entity models.


 
\section{Relationships}
\mynote
A defining feature of entity modelling is the description
 (the modelling) of the relationships that can possible  exist between things according as to the types of things that they are. 
\mynote 
We need to say something about this term `relationship' as it is used in
entity modelling.
As a starting point consider these two definitions that I found online. 

\begin{erquote}
a relationship between two things is the way in which they are connected
\end{erquote}
and 
\begin{erquote}
a relationship is a connection, association, or involvement between two things.
\end{erquote}

At first reading these two dictionary entries seem to be in accord and indeed 
they both describe what would more specifically be called \textit{binary relationships} involving as they do exactly two things in relationship with each other.
A more careful reading reveals a significant difference between the two because \textit{friendship} as an abstraction satisfies the first definition but not the second whereas a particular friendship such as that between \textit{Don Quixote} and \textit{Sancho Panza} satisfies the second  but not the first. 
To distinguish the two meanings some authors use the term \textit{relationship type} for the first sense and reserve the term \textit{relationship} itself for the second. Here we will stick with the single word relationship, which, for us, can therefore have two distinct meanings. If we need to emphasise the second meaning we may speak of instances of relationships. When speaking in this way we might say that their is a relationship called `friendship' between people and that there is an instance of this relationship between \textit{Don Quixote} and \textit{Sancho Panza}.

\mynote
Just for the record, in another dictionary we find 
\begin{erquote}
In logic and mathematics relationship is another name for relation.
\end{erquote}
and in yet another there is a definition of mathematical relation as a subset of the cartesian product of two sets 
(evidently, therefore, a definition of a binary relation)\footnote{Chambers definition here}\footnote{Need to say more here because this latter definition is a relation in a model of logic not in logic itself}. 

\mynote
Friendship,  marriage, adjacency, ownership and succession are all examples of binary relationships. There are binary relationships too  between parent and child, between teacher and student and between a planet and the star that it orbits.   
The following are also binary relationships:
being employed by, in the sense that a person might be employed by a particular employee,
managing, in the sense that an employee might manage a project. 
There are binary relationships between a sales order and the customer making the order and between a debit card and  the bank that issues the card. As a final example, a banking transaction has binary relationships both with the account debited and with the account credited. 

\mynote The most important feature of relationships as the term is used  in this book is that the existence of a relationship instance between any two particular entities should be a matter of black and white. For example between any particular Jack and any particular Jill we must take it that there either is or is not a friendship between them. There can be no equivocation or matter of degree --- either the relationship is or is not. This is a point of difference, by the way, between our terminology and that of Chen --- Chen's relationships may have attributes (properties) assigned to them, our relationships may not which is why they are, mathematically, binary relations. We will see that such (black and white) binary relationships are all we need for modelling and that many-many relationships are not needed either. 

\mynote 
Every binary relationship can be described from two different points of view. 
For the planet-star relationship the two points of view are
\begin{itemize}
\item every planet orbits a star,
\item every star may be orbited by planets.
\end{itemize}

In these descriptions the language is slightly ambiguous and questions arise.
Does every planet orbit exactly one star or may it orbit multiple stars? Do any stars have no planets orbiting them?

In an entity model ambiguities such as these are removed. 
The planet-star relationship would be expressed in more detail as
\begin{itemize}
\item every planet must be orbiting exactly one star,\footnote{You may disagree with this if your cosmology is more nuanced than mine.}
\item every star may be orbited by one or more planets.
\end{itemize}

Note that these two prescriptions describe the very same relationship from opposite ends. That they are complementary points of view I can express like this:

\begin{itemize}
\item a planet is orbiting a star if and only if the star is orbited by the planet.
\end{itemize}

\mynote
In his introduction to entity relationship diagramming, Chen gives an example of an entity relationship diagram describing the types of entities relevant to the organisation of an imaginary manufacturing company.
In it he gives the example of a relationship between an employee and the department in which they work. 

\noindent In his diamond notation Chen draws the relationship like this:
\begin{center}
\begin{pspicture}(-7,-1.2)(7,1.2)
%\psgrid


\chendiamond{de}{\begin{tabular}{c}DEPT-\\EMP\end{tabular}}
\rput[l](-3.5,0){
   \chenbox{d}{DEPARTMENT}
	}
\rput[l](3.5,0){
   \chenbox{e}{EMPLOYEE}
	}
\ncline{-}{dE}{deW}
\nbput{\scriptsize 1}	
\ncline{-}{deE}{eW}
\nbput{\scriptsize N}

\end{pspicture}
\end{center}

\noindent In the Chen notation, the numeric $1$ labelling the left hand line in this diagram indicates that every employee is in exactly one department and the corresponding annotation N on the right side indicates the fact that there are many employees within each department. \\

\noindent This relationship can be described from  two complementary points of view as
\begin{align}
\label{employing}&\mbox{every department may be employing one or more employees,} \\
\label{employedby}&\mbox{every employee must be employed by exactly one department.}
\end{align}

\noindent In the notation used in this book, which is based on the Barker-Ellis notation,  this diamond and its two connectors are replaced by a single line. The fact that there
are multiple employees within a department is represented on the diagram by drawing a crows foot at the employee end of the line and the relationship is drawn like so:
\definecolor{lightyellow}{cmyk}{0,0,0,0}

\begin{gather}
\label{employsRelationship}
\raisebox{-1cm}{\begin{erdiagram}{2.3}{7.999999999999999}

\eret{0.1}{-1.9}{2.4}{-0.4}{0.2}{1}\eretname{1.25}{-0.75}{}{}\eretname{1.25}{-1.05}{}{DEPARTMENT}
\eret{5.7}{-1.9}{8}{-0.4}{0.2}{1}\eretname{6.85}{-0.75}{}{}\eretname{6.85}{-1.05}{}{EMPLOYEE}

% relationship employing
\errelname{2.55}{-1}{l}{employing}\errelname{5.55}{-1.45}{r}{employed by}\errelarm{2.4}{-1.15}{4.05}{-1.15}{0}{0}\errelarm{4.05}{-1.15}{5.699}{-1.15}{1}{0}\ercrowfoot{5.55}{-1.15}{5.7}{-1}{5.7}{-1.15}{5.7}{-1.3}{0}
\end{erdiagram}
}
\end{gather}

\noindent In this form the relationship can be read directionally
 --- left to right it
reads as  (\ref{employing}), above,  right to left it reads as (\ref{employedby}). 
We can think of there being two directional relationships here inseparably
connected (meta-related, in fact) as the poles of a magnet.

\noindent The right to left reading expresses a many-one directional relationship and this is otherwise known as a functional relationship. In mathematical notation this would indicate a function and be 
represented by an arrow: 
\begin{equation}
employed\,by : employee \longrightarrow department
\end{equation}
\noindent
To be clear, every binary relationship can be thought of as consisting of a pair of directional relationships each one of which is one view of it. 

\mynote Mathematically speaking, every function has an inverse which generally is many-valued and which therefore is represented as a separate function whose target is a power set i.e. a set of subsets. In our case here the inverse to the function \textit{employed by} will in some mathematical circles by represented by a function
\begin{equation}
employing : department \longrightarrow \mathcal{P}(employee).\footnote{Where $\mathcal{P}$ represents power set so that $\mathcal{P}(x)$ is the set of subsets of set $x$.}
\end{equation}

\noindent Let me recap this illustration. The function signatures (3) and (4) express the same information as the words do in (1) and (2).
Diagram (\ref{employsRelationship}) encapsulates the same information and also expresses  the fact that (1) and (2) are complementary i.e. that
an employee works for a particular department if an only if that particular department employs the employee.
Finally, this complementarity may be expressed in the mathematical language of
sets and functions by asserting  function (4) to be the inverse to function (3).   

\subsubsection*{Many-many Binary Relationships}
In Chen's paper the relationship between employees and projects (whose meaning is that
an employee is assigned to work on a project) is depicted like this.
\begin{gather}
\label{projectWorkerChenStyle}
\raisebox{-1cm}{
\begin{pspicture}(-7,-1.2)(7,1.2)
%\psgrid
\chendiamond{pw}{\rput(0,-0.05){\begin{tabular}{c}PROJECT-\\WORKER\end{tabular}}}
\rput[l](-3.5,0){
   \chenbox{e}{EMPLOYEE}
	}
\rput[l](3.5,0){
   \chenbox{p}{PROJECT}
	}
\ncline{-}{eE}{pwW}
\nbput{\scriptsize M}	
\ncline{-}{pwE}{pW}
\nbput{\scriptsize N}
\end{pspicture}
}
\end{gather}
On this diagram there is an annotation $N$ which documents that an employee may work on many projects (we are to understand that $N >= 0$). There is also an annotation $M$ to document that
a project has many employees assigned to it, $M$ in fact, for some $M >= 0$. \\
\noindent Such a relationship is called a many-many relationship.  \\
\noindent In Barker-Ellis notation which is the basis for the notation used in this book the diamond and its two connectors are replaced by a single line. The fact that the relationship is many-many is represented in the diagram by having a crows foot at each end of the line like so:

\begin{gather}
\label{projectWorkerOurStyle}
\raisebox{-1cm}{
\begin{erdiagram}{2.3}{7.4}

\eret{0.1}{-1.9}{2.2}{-0.4}{0.2}{1}\eretname{1.15}{-0.75}{}{}\eretname{1.15}{-1.05}{}{EMPLOYEE}
\eret{5.3}{-1.9}{7.4}{-0.4}{0.2}{1}\eretname{6.35}{-0.75}{}{}\eretname{6.35}{-1.05}{}{PROJECT}

% relationship works on
\errelname{2.35}{-1}{l}{works on}\errelname{5.15}{-1.45}{r}{has as}\errelname{5.15}{-1.75}{r}{workers}\errelarm{2.2}{-1.15}{3.75}{-1.15}{1}{0}\errelarm{3.75}{-1.15}{5.3}{-1.15}{1}{0}\ercrowfoot{5.15}{-1.15}{5.3}{-1}{5.3}{-1.15}{5.3}{-1.3}{0}\ercrowfoot{2.35}{-1.15}{2.2}{-1}{2.2}{-1.15}{2.2}{-1.3}{0}
\end{erdiagram}

}
\end{gather}

\noindent Also notice that the relationship is labelled at both ends. Each text label gives the name of the relationship from the point of view of the type of entity at that end of the line.
\noindent In fact, many-many relationships are infrequently used in entity models and never used at all when modelling data.
Consider that we can make a thing out of the fact of the assignment of an employee to a project. This means that we can entify the many-many relationship and model the relationship like this:

\begin{gather}
\label{projectWorkerOurStyleMediated}
\raisebox{-1cm}{
\begin{erdiagram}{2.1}{12.6}

\eret{0.1}{-1.7}{2.2}{-0.4}{0.2}{1}\eretname{1.15}{-0.75}{}{}\eretname{1.15}{-1.05}{}{EMPLOYEE}
\eret{5.3}{-1.7}{7.4}{-0.4}{0.2}{1}\eretname{6.35}{-0.75}{}{PROJECT}\eretname{6.35}{-1.05}{}{WORKER}\eretname{6.35}{-1.35}{}{ASSIGNMENT}
\eret{10.5}{-1.7}{12.6}{-0.4}{0.2}{1}\eretname{11.55}{-0.75}{}{}\eretname{11.55}{-1.05}{}{PROJECT}

% relationship assigning
\errelname{5.15}{-0.9}{r}{assigning}\errelname{2.35}{-0.9}{l}{subject of}\errelarm{5.3}{-1.05}{3.75}{-1.05}{1}{0}\errelarm{3.75}{-1.05}{2.2}{-1.05}{0}{0}\ercrowfoot{5.15}{-1.05}{5.3}{-0.9}{5.3}{-1.05}{5.3}{-1.2}{0}
% relationship to
\errelname{7.55}{-0.9}{l}{to}\errelname{10.35}{-0.9}{r}{resourced by}\errelarm{7.4}{-1.05}{8.95}{-1.05}{1}{0}\errelarm{8.95}{-1.05}{10.5}{-1.05}{0}{0}\ercrowfoot{7.55}{-1.05}{7.4}{-0.9}{7.4}{-1.05}{7.4}{-1.2}{0}
\end{erdiagram}

}
\end{gather}

The original many-many relationship is now mediated by the entity type PROJECT WORKER ASSIGNMENT and two functional relationships. Because we can 'entify' many-relationships and mediate them by other entity types and relationships they can be eliminated from entity modelling. Because functional relationships have direct representations, in data however represented, then it is usual to eliminate many-many relationships from entity models. Since the remaining relationships are functional, in one direction or the other, so it is that in most entity models types are related solely by functional relationships and their inverses. Entity modelling in this form models the world in functional relationships.

\mynote For the sake of completeness I should mention that in such cases the mediating entity or indeed its type is often referred to as an \textit{intersection entity}. 

\mynote Strictly speaking, directional relationships are not, as their name suggests, types of relationships, but are ways of looking at or considering binary relationships. Each directional relationship is a binary relationship viewed at or considered from one end or the other. Each binary relationship can be considered as a directional relationship in two different ways.
This is how binary relationships are described in the entity modelling meta-model given later.
\mynote Functional relationships in mathematics that are total are usually depicted by the arrow symbol $\longrightarrow$.

The partial arrow symbol $\rightharpoonup$ may be used if not know to be total. 

\mynote We have already seen above examples of Barker-Ellis style depictions of relationships in which functional relationships of A to B
are shown like this:
\begin{center}
A\,\barkerEllisA\,B
\end{center}
If however a functional relationship is known to be total then the left half line is drawn as a solid line instead of a dashed line like so

\begin{center}
A\,\barkerEllisB\,B
\end{center}
In either case if the map is known to be surjective then the right hand half line is solidified and looks as this
\begin{center}
A\,\barkerEllisC\,B
\end{center}
or this
\begin{center}
A\,\barkerEllisD\,B
\end{center}

If in any of these four cases it is known in addition that that the map is injective then the crowsfoot is omitted and so it is depicted is one of
\begin{center}
A\,\barkerEllisE\,B \\
A\,\barkerEllisF\,B \\
A\,\barkerEllisG\,B \\
A\,\barkerEllisH\,B 
\end{center}

Note that in this final case the relationship in question establishes a 1-1 correspondence between types A and B.

\subsection*{Ternary Relationships}
For the sake of completeness I should mention that Chen also represents ternary relationships and gives an 
example of a ternary relationship between a project, a supplier and a part which he shows like this:
\begin{center}
\begin{pspicture}(-7,-1.2)(7,4.4)
%\psgrid

\chendiamond{spp}{\rput(0,-0.15){\begin{tabular}{c}SUPP-PROJ-\\PART\end{tabular}}}
\rput[l](0,2.6){
   \chenbox{supp}{SUPPLIER}
	}
\rput[l](-3.5,0){
   \chenbox{prj}{PROJECT}
	}
\rput[l](3.5,0){
   \chenbox{prt}{PART}
	}	
	
\ncline{-}{suppS}{sppN}
\nbput{\footnotesize N}
\ncline{-}{prjE}{sppW}
\nbput{\footnotesize M}	
\ncline{-}{sppE}{prtW}
\nbput{\footnotesize P}
\end{pspicture}
\end{center}

In the Barker-Ellis style there are no ternary relationships nor are any needed since ternary relationships can be represented as a combination of binary relationships and entity types. The fact that all relationships are binary and are represented by lines only rather than diamonds  is of practical importance in that more details can be fitted onto diagrams.
 %\iffalse
 \section{Example --- Chen Manufacturing Company}
\label{ChenManufacturingCompany}
In figure \ref{ChenManufacturingExample} I have reproduced an example entity relationship diagram  given in Chen's 1976 paper. This example documents principle types of entity in the organisation of an imaginary manufacturing company.  I wanted to redraw
this diagram in the Barker-Ellis style used in this book. To do this I needed to make some educated guesses about the meaning of some of Chen's relationships since use of the Barker-Ellis style  required me to descriptively label the relationships. Having made these guesses I drew the diagram shown
 in figure \ref{chenManufacturingCo..diagram}. 

\begin{erboxedFigure} {H}{ChenManufacturingExample}{
In effect this is the first ever entity relationship diagram. It is an entity relationship diagram in the Chen style from his seminal paper of 1976 and it is there described as being an analysis of information in a manufacturing firm. In this diagram
diamonds represent what we describe here as relationships and boxes represent types of entities.  
For completeness we should mention that in Chen's 1976 terminology the boxes were said to be entity sets and the diamonds were said to be relationship sets but by 1983 in a paper published in that year Chen is instead using the terms `entity type' and `relationship type'. While sympathising with this terminology we think it correct to to use the term `relationship' rather than 'relationship type' because this brings us closer to the terminology of formal logic.
}
\begin{center}
\scalebox{0.80}{\begin{pspicture}(-4.4,-5.80)(10.2,5.5)
%\psgrid

\rput[l](-3.0,5){
   \chenbox{dept}{DEPARTMENT}
}

\rput(-3,2.5){
	\chendiamond{de}{\rput(0,-0.05){\begin{tabular}{c}DEPT-\\EMP\end{tabular}}}
}

\rput[l](-3.0,0){
   \chenbox{emp}{EMPLOYEE}
	}
	
\rput(-3,-2.5){
	\chendiamond{ed}{\rput(0,-0.05){\begin{tabular}{c}EMP-\\DEP\end{tabular}}}
}

\rput[l](-3.0,-5){
   \chenbox{dpndt}{DEPENDENT}
	}
	
\rput(0,1.0){
	\chendiamond{pw}{\rput(0,-0.05){\begin{tabular}{c}PROJECT-\\WORKER\end{tabular}}}
}

\rput(0,-1){
	\chendiamond{pm}{\rput(0,-0.05){\begin{tabular}{c}PROJECT-\\MANAGER\end{tabular}}}
}

\rput[l](3.0,0){
   \chenbox{prj}{PROJECT}
	}

\ncline{-}{empNE}{pwW}
\naput{\scriptsize M}	
\ncline{-}{pwE}{prjNW}
\naput{\scriptsize N}

\ncline{-}{empSE}{pmW}
\nbput{\scriptsize 1}	
\ncline{-}{pmE}{prjSW}
\nbput{\scriptsize N}


\rput(6,0){
	\chendiamond{pp}{\rput(0,-0.05){\begin{tabular}{c}PROJECT-\\PART\end{tabular}}}
}

\rput[l](9,0){
   \chenbox{prt}{PART}
	}
\ncline{-}{prjE}{ppW}
\nbput{\scriptsize M}	
\ncline{-}{ppE}{prtW}
\nbput{\scriptsize N}

\rput(9,-2.5){
	\chendiamond{c}{\rput(0,-0.05){\begin{tabular}{c}COMPONENT\end{tabular}}}
}

\rput[l](6.0,5){
   \chenbox{supp}{SUPPLIER}
	}
	
\rput(6,2.5){
	\chendiamond{spp}{\rput(0,-0.05){\begin{tabular}{c}SUPP-PROJ-\\PART\end{tabular}}}
}

\ncline{-}{deptS}{deN}
\nbput{\scriptsize 1}	
\ncline{-}{deS}{empN}
\nbput{\scriptsize N}

\ncline{-}{empS}{edN}
\nbput{\scriptsize 1}	
\ncline{-}{edS}{dpndtN}
\nbput{\scriptsize ??}

\ncline{-}{suppS}{sppN}
\nbput{\scriptsize 1}	
\ncline{-}{sppE}{prtNW}     %  prtNW
\naput{\scriptsize P}
\ncline{-}{sppW}{prjNE}     %  prjNE
\nbput{\scriptsize M}

\ncline{-}{prtSSE}{cNE}     
\naput{\scriptsize N}	
\ncline{-}{prtSSW}{cNW}
\nbput{\scriptsize M}


\end{pspicture}}
\end{center}
\end{erboxedFigure}

\begin{erboxedFigure} {H}{chenManufacturingCo..diagram}{
Chen's 1976 entity relationship diagram (shown in figure \ref{ChenManufacturingExample}) 
redrawn as a structured entity model using the Barker-Ellis notation. 
 The rail at the top of the diagram represents the whole of the model which in this case is the manufacturing company in question.
I have had to guess at the meanings of some of Chen's relationships in figure \ref{ChenManufacturingExample}
 in order to label them meaningfully in this diagram. 
Having made my guess I have represented  the ternary relationship in Chen's diagram by type \textit{project supply option} and type \textit{part used on project}. 
I hope that you agree that this Barker-Ellis style diagram leaves less to the imagination than Chen's original diagram did.
}
%\begin{center}
\scalebox{0.95}{\begin{erdiagram}{6.3}{12.383}

\eret{0.8}{-1.6}{3.1}{-1}{0.2}{1}\eretname{1.95}{-1.35}{}{department}
\eret{0.8}{-3.5}{3.1}{-2.7}{0.2}{1}\eretname{1.95}{-3.05}{}{employee}
\eret{0.8}{-5.45}{3.1}{-4.85}{0.2}{1}\eretname{1.95}{-5.2}{}{dependent}
\eret{5.1}{-1.6}{8.1}{-1}{0.2}{1}\eretname{6.6}{-1.35}{}{project}
\eret{3.85}{-6.3}{6.502}{-5.1}{0.2}{1}\eretname{5.176}{-5.45}{}{project}\eretname{5.176}{-5.75}{}{worker}\eretname{5.176}{-6.05}{}{assignment}
\eret{7.374}{-3.75}{9.226}{-2.85}{0.2}{1}\eretname{8.3}{-3.2}{}{part used}\eretname{8.3}{-3.5}{}{on project}
\eret{7.413}{-6.25}{11.187}{-5.65}{0.2}{1}\eretname{9.3}{-6}{}{part supply option}
\eret{8.6}{-1.6}{9.811}{-1}{0.2}{1}\eretname{9.205}{-1.35}{}{part}
\eret{10.311}{-1.6}{12.133}{-1}{0.2}{1}\eretname{11.222}{-1.35}{}{supplier}
\eret{0}{-0.25}{12.383}{0.25}{0.2}{1}\eretname{4.279}{-0.2}{l}{Chen '76 Manufacturing Company}

% relationship 
\errelname{2.1}{-0.55}{l}{}\errelname{2.1}{-0.85}{l}{..}\errelarm{1.95}{-0.25}{1.95}{-0.625}{0}{0}\errelarm{1.95}{-0.625}{1.95}{-1}{1}{0}\ercrowfoot{1.95}{-0.85}{1.8}{-1}{1.95}{-1}{2.1}{-1}{0}
% relationship 
\errelname{6.75}{-0.55}{l}{}\errelname{6.75}{-0.85}{l}{..}\errelarm{6.6}{-0.25}{6.6}{-0.625}{0}{0}\errelarm{6.6}{-0.625}{6.6}{-1}{1}{0}\ercrowfoot{6.6}{-0.85}{6.45}{-1}{6.6}{-1}{6.75}{-1}{0}
% relationship 
\errelname{9.356}{-0.55}{l}{}\errelname{9.356}{-0.85}{l}{..}\errelarm{9.205}{-0.25}{9.205}{-0.625}{0}{0}\errelarm{9.205}{-0.625}{9.205}{-1}{1}{0}\ercrowfoot{9.205}{-0.85}{9.055}{-1}{9.205}{-1}{9.356}{-1}{0}
% relationship 
\errelname{11.372}{-0.55}{l}{}\errelname{11.372}{-0.85}{l}{..}\errelarm{11.22}{-0.25}{11.22}{-0.625}{0}{0}\errelarm{11.22}{-0.625}{11.22}{-1}{1}{0}\ercrowfoot{11.222}{-0.85}{11.072}{-1}{11.222}{-1}{11.372}{-1}{0}
% relationship employing
\errelname{1.8}{-1.9}{r}{employing}\errelname{1.8}{-2.55}{r}{employed by}\errelarm{1.95}{-1.6}{1.95}{-2.15}{0}{0}\errelarm{1.95}{-2.15}{1.95}{-2.7}{1}{0}\ercrowfoot{1.95}{-2.55}{1.8}{-2.7}{1.95}{-2.7}{2.1}{-2.7}{0}
% relationship depended on by
\errelname{1.8}{-3.8}{r}{depended on by}\errelname{1.8}{-4.7}{r}{depending on}\errelarm{1.95}{-3.5}{1.95}{-4.175}{0}{0}\errelarm{1.95}{-4.175}{1.95}{-4.85}{1}{0}\ercrowfoot{1.95}{-4.7}{1.8}{-4.85}{1.95}{-4.85}{2.1}{-4.85}{0}
% relationship subject_of
\errelname{2.675}{-3.8}{l}{subject}\errelname{2.675}{-4.1}{l}{of}\errelname{4.575}{-4.95}{r}{of}\errelname{4.575}{-4.65}{r}{assignment}\errelarm{2.525}{-3.5}{2.525}{-3.825}{0}{0}\errelarm{2.525}{-3.825}{2.525}{-4.15}{0}{0}\errelarm{2.525}{-4.15}{3.625}{-4.312}{0}{0}\errelarm{3.625}{-4.312}{4.725}{-4.475}{1}{0}\errelarm{4.725}{-4.475}{4.725}{-4.787}{1}{0}\errelarm{4.725}{-4.787}{4.725}{-5.1}{1}{0}\eridcomprel{4.625325}{4.825324999999999}{-4.85}\ercrowfoot{4.725}{-4.95}{4.575}{-5.1}{4.725}{-5.1}{4.875}{-5.1}{0}
% relationship resourced_by
\errelname{5.751}{-1.9}{l}{resourced}\errelname{5.751}{-2.2}{l}{by}\errelname{5.751}{-4.95}{l}{to}\errelname{5.751}{-4.65}{l}{assignment}\errelarm{5.6}{-1.6}{5.6}{-3.349}{0}{0}\errelarm{5.6}{-3.349}{5.6}{-5.1}{1}{0}\eridcomprel{5.50065}{5.7006499999999996}{-4.85}\ercrowfoot{5.601}{-4.95}{5.451}{-5.1}{5.601}{-5.1}{5.751}{-5.1}{0}
% relationship requires
\errelname{7.5}{-1.9}{l}{requires}\errelname{7.687}{-2.7}{r}{use by}\errelarm{7.35}{-1.6}{7.35}{-1.8}{0}{0}\errelarm{7.35}{-1.8}{7.35}{-2}{0}{0}\errelarm{7.35}{-2}{7.593}{-2.212}{0}{0}\errelarm{7.593}{-2.212}{7.836}{-2.425}{1}{0}\errelarm{7.836}{-2.425}{7.836}{-2.637}{1}{0}\errelarm{7.836}{-2.637}{7.836}{-2.85}{1}{0}\eridcomprel{7.7368749999999995}{7.936874999999999}{-2.6}\ercrowfoot{7.837}{-2.7}{7.687}{-2.85}{7.837}{-2.85}{7.987}{-2.85}{0}
% relationship managed by
\errelname{4.95}{-1.15}{r}{managed by}\errelname{3.25}{-3.264}{l}{managing}\errelarm{5.1}{-1.3}{4.5}{-1.3}{1}{0}\errelarm{4.5}{-1.3}{3.899}{-1.3}{1}{0}\errelarm{3.899}{-1.3}{3.774}{-2.132}{1}{0}\errelarm{3.774}{-2.132}{3.649}{-2.964}{0}{0}\errelarm{3.649}{-2.964}{3.374}{-2.964}{0}{0}\errelarm{3.374}{-2.964}{3.099}{-2.964}{0}{0}\ercrowfoot{4.95}{-1.3}{5.1}{-1.15}{5.1}{-1.3}{5.1}{-1.45}{0}
% relationship able to be_sourced via
\errelname{8.45}{-4.05}{l}{able to be}\errelname{8.45}{-4.35}{l}{sourced via}\errelname{8.15}{-5.5}{r}{supply of}\errelname{8.15}{-5.2}{r}{option for}\errelarm{8.299}{-3.75}{8.299}{-4.699}{0}{0}\errelarm{8.299}{-4.699}{8.299}{-5.649}{1}{0}\eridcomprel{8.2}{8.399999999999999}{-5.3999999999999995}\ercrowfoot{8.3}{-5.5}{8.15}{-5.65}{8.3}{-5.65}{8.45}{-5.65}{0}
% relationship subject_of
\errelname{9.356}{-1.9}{l}{subject}\errelname{9.356}{-2.2}{l}{of}\errelname{8.913}{-2.7}{l}{use of}\errelarm{9.205}{-1.6}{9.205}{-1.8}{0}{0}\errelarm{9.205}{-1.8}{9.205}{-2}{0}{0}\errelarm{9.205}{-2}{8.984}{-2.212}{0}{0}\errelarm{8.984}{-2.212}{8.763}{-2.425}{1}{0}\errelarm{8.763}{-2.425}{8.763}{-2.637}{1}{0}\errelarm{8.763}{-2.637}{8.763}{-2.85}{1}{0}\eridcomprel{8.663124999999999}{8.863124999999998}{-2.6}\ercrowfoot{8.763}{-2.7}{8.613}{-2.85}{8.763}{-2.85}{8.913}{-2.85}{0}
% relationship able to_provide
\errelname{11.372}{-1.9}{l}{able to}\errelname{11.372}{-2.2}{l}{provide}\errelname{10.394}{-5.5}{l}{aquire from}\errelname{10.394}{-5.2}{l}{option to}\errelarm{11.22}{-1.6}{11.22}{-1.95}{0}{0}\errelarm{11.22}{-1.95}{11.22}{-2.3}{0}{0}\errelarm{11.22}{-2.3}{10.73}{-3.624}{0}{0}\errelarm{10.73}{-3.624}{10.24}{-4.949}{1}{0}\errelarm{10.24}{-4.949}{10.24}{-5.3}{1}{0}\errelarm{10.24}{-5.3}{10.24}{-5.649}{1}{0}\eridcomprel{10.143625}{10.343625}{-5.3999999999999995}\ercrowfoot{10.244}{-5.5}{10.094}{-5.65}{10.244}{-5.65}{10.394}{-5.65}{0}
\end{erdiagram}
}
%\end{center}
\end{erboxedFigure}
As a footnote to figures \ref{ChenManufacturingExample} and \ref{chenManufacturingCo..diagram}, 
note that the  meaning of the term  `part' can itself be itself be a point of confusion ---
does it mean an actual physical part or  a part design?
I found a discussion of the ambiguity of the term  on wikipedia which stated that `part' as something that had a `part number' ususally meant `part design' rather than an instantiation of that design:
\begin{erquote}
As a part number is an identifier of a part design (independent of its instantiations), a serial number is a unique identifier of a particular instantiation of that part design. In other words, a part number identifies any particular (physical) part as being made to that one unique design; a serial number, when used, identifies a particular (physical) part (one physical instance), as differentiated from the next unit that was stamped, machined, or extruded right after it. This distinction is not always clear, as natural language blurs it by typically referring to both part designs, and particular instantiations of those designs, by the same word, ``part(s)''. Thus if you buy a muffler of P/N 12345 today, and another muffler of P/N 12345 next Tuesday, you have bought ``two copies of the same part'', or``two parts'', depending on the sense implied.
\end{erquote}


 \section{Goodland Vehicle Hire Company}
\label{GoodlandVehicleHireCompany}

\mynote This is an example from a book on SSADM by Goodland and Slater.
\mynote It is a diagram specifying the logical data structure underlying the business operations of a hypothetical vehicle rental company that leases vans and trucks, and sometimes drivers, to its customers. 
\mynote
This example is developed and elaborated on throughout the book as an example of how a computer system may be analysed, documented, respecified and extended with the goal of better supporting business.
\mynote Hyopthetically this company has many local offices and both drivers and vehicles are based at local offices. \commentary{improvise some text to go with this as a series of independent mynotes}
\begin{erboxedFigure} {H}{SSADMCarHireExample}{
Figure 3.35, from Goodland and Slater SSADM book page 106. See also fig. 4.13, pg 159 and fig 4.58 page 213.
Consider simplifying and having a many-many betwixt payment and booking. 
Later will have a version as in the book with allocated payment as an intersection entity. 
Can we avoid needing bars on relationships in this way?
}

\begin{center}
\scalebox{0.95}{\begin{erdiagram}{6.26}{12.438649999999999}

\eret{1.6}{-1.92}{3.6}{-1}{0.2}{1}\eretname{2.6}{-1.35}{}{customer}
\eret{0.433}{-4.14}{1.767}{-3.12}{0.2}{1}\eretname{1.1}{-3.47}{}{payment}
\eret{0.344}{-6.26}{1.856}{-5.34}{0.2}{1}\eretname{1.1}{-5.69}{}{allocated}\eretname{1.1}{-5.99}{}{payment}
\eret{3.367}{-5.32}{5.067}{-3.12}{0.2}{1}\eretname{4.217}{-3.47}{}{booking}\eretname{4.217}{-3.77}{}{/invoice}
\eret{7.35}{-2.22}{8.683}{-1.3}{0.2}{1}\eretname{8.017}{-1.65}{}{local}\eretname{8.017}{-1.95}{}{office}
\eret{11.067}{-4.54}{12.439}{-3.12}{0.2}{1}\eretname{11.753}{-3.47}{}{vehicle}\eretname{11.753}{-3.77}{}{category}
\eret{7.35}{-4.04}{8.683}{-3.02}{0.2}{1}\eretname{8.017}{-3.37}{}{vehicle}
\eret{0}{-0.2}{12.439}{0.3}{0.2}{1}

% relationship 
\errelname{2.75}{-0.5}{l}{}\errelarm{2.6}{-0.2}{2.6}{-0.6}{1}{0}\errelarm{2.6}{-0.6}{2.6}{-1}{0}{0}\ercrowfoot{2.6}{-0.85}{2.45}{-1}{2.6}{-1}{2.75}{-1}{0}
% relationship sender of
\errelname{1.95}{-2.22}{r}{sender of}\errelname{0.95}{-2.97}{r}{sent by}\errelarm{2.1}{-1.92}{2.1}{-2.12}{0}{0}\errelarm{2.1}{-2.12}{2.1}{-2.32}{0}{0}\errelarm{2.1}{-2.32}{1.6}{-2.594}{0}{0}\errelarm{1.6}{-2.594}{1.1}{-2.87}{1}{0}\errelarm{1.1}{-2.87}{1.1}{-2.995}{1}{0}\errelarm{1.1}{-2.995}{1.1}{-3.12}{1}{0}\ercrowfoot{1.1}{-2.97}{0.95}{-3.12}{1.1}{-3.12}{1.25}{-3.12}{0}
% relationship maker of
\errelname{3.25}{-2.22}{l}{maker of}\errelname{4.197}{-2.97}{l}{made by}\errelarm{3.1}{-1.92}{3.1}{-2.12}{1}{0}\errelarm{3.1}{-2.12}{3.1}{-2.32}{1}{0}\errelarm{3.1}{-2.32}{3.573}{-2.594}{1}{0}\errelarm{3.573}{-2.594}{4.046}{-2.87}{1}{0}\errelarm{4.046}{-2.87}{4.046}{-2.995}{1}{0}\errelarm{4.046}{-2.995}{4.046}{-3.12}{1}{0}\ercrowfoot{4.047}{-2.97}{3.897}{-3.12}{4.047}{-3.12}{4.197}{-3.12}{0}
% relationship split into
\errelname{0.95}{-4.44}{r}{split into}\errelname{0.95}{-5.19}{r}{part of}\errelarm{1.1}{-4.14}{1.1}{-4.74}{0}{0}\errelarm{1.1}{-4.74}{1.1}{-5.34}{1}{0}\ercrowfoot{1.1}{-5.19}{0.95}{-5.34}{1.1}{-5.34}{1.25}{-5.34}{0}
% relationship made to
\errelname{2.006}{-6.1}{l}{made to}\errelname{3.217}{-4.07}{r}{for by}\errelname{3.217}{-3.77}{r}{paid}\errelarm{1.856}{-5.8}{2.006}{-5.8}{1}{0}\errelarm{2.006}{-5.8}{2.156}{-5.8}{1}{0}\errelarm{2.156}{-5.8}{2.561}{-5.01}{1}{0}\errelarm{2.561}{-5.01}{2.966}{-4.22}{0}{0}\errelarm{2.966}{-4.22}{3.166}{-4.22}{0}{0}\errelarm{3.166}{-4.22}{3.366}{-4.22}{0}{0}\ercrowfoot{2.006}{-5.8}{1.856}{-5.65}{1.856}{-5.8}{1.856}{-5.95}{0}
% relationship from_and to
\errelname{5.217}{-3.74}{l}{and to}\errelname{5.217}{-3.44}{l}{from}\errelname{7.2}{-1.61}{r}{start and end of}\errelarm{5.066}{-3.89}{5.416}{-3.89}{1}{0}\errelarm{5.416}{-3.89}{5.766}{-3.89}{1}{0}\errelarm{5.766}{-3.89}{6.358}{-2.825}{1}{0}\errelarm{6.358}{-2.825}{6.949}{-1.76}{0}{0}\errelarm{6.949}{-1.76}{7.149}{-1.76}{0}{0}\errelarm{7.149}{-1.76}{7.35}{-1.76}{0}{0}\ercrowfoot{5.217}{-3.89}{5.067}{-3.74}{5.067}{-3.89}{5.067}{-4.04}{0}
% relationship user of
\errelname{5.217}{-4.685}{l}{user of}\errelname{7.2}{-3.38}{r}{used for}\errelarm{5.066}{-4.385}{5.566}{-4.385}{1}{0}\errelarm{5.566}{-4.385}{6.066}{-4.385}{1}{0}\errelarm{6.066}{-4.385}{6.508}{-3.957}{1}{0}\errelarm{6.508}{-3.957}{6.95}{-3.53}{0}{0}\errelarm{6.95}{-3.53}{7.15}{-3.53}{0}{0}\errelarm{7.15}{-3.53}{7.35}{-3.53}{0}{0}\ercrowfoot{5.217}{-4.385}{5.067}{-4.235}{5.067}{-4.385}{5.067}{-4.535}{0}
% relationship requiring
\errelname{5.217}{-5.18}{l}{requiring}\errelname{10.917}{-4.485}{r}{required}\errelname{10.917}{-4.785}{r}{for}\errelarm{5.066}{-4.88}{6.066}{-4.88}{1}{0}\errelarm{6.066}{-4.88}{7.066}{-4.88}{1}{0}\errelarm{7.066}{-4.88}{8.191}{-4.532}{1}{0}\errelarm{8.191}{-4.532}{9.316}{-4.185}{0}{0}\errelarm{9.316}{-4.185}{10.19}{-4.185}{0}{0}\errelarm{10.19}{-4.185}{11.06}{-4.185}{0}{0}\ercrowfoot{5.217}{-4.88}{5.067}{-4.73}{5.067}{-4.88}{5.067}{-5.03}{0}
% relationship classified by
\errelname{8.833}{-3.38}{l}{classified by}\errelname{10.917}{-3.846}{r}{classifier of}\errelarm{8.683}{-3.53}{8.933}{-3.53}{1}{0}\errelarm{8.933}{-3.53}{9.183}{-3.53}{1}{0}\errelarm{9.183}{-3.53}{9.874}{-3.538}{1}{0}\errelarm{9.874}{-3.538}{10.56}{-3.546}{0}{0}\errelarm{10.56}{-3.546}{10.81}{-3.546}{0}{0}\errelarm{10.81}{-3.546}{11.06}{-3.546}{0}{0}\ercrowfoot{8.833}{-3.53}{8.683}{-3.38}{8.683}{-3.53}{8.683}{-3.68}{0}
\end{erdiagram}
}
\end{center}
\end{erboxedFigure}
 

\section{Attributes}
\label{Attributes} 
\footnote{ChatGPT: Overall Comment
There is slight conceptual repetition — for instance, the idea that attributes relate particulars to universals is revisited several times.} 
\subsection{Data and Attribution}
\mynote
This section concerns the simplest aspects of things 
to which data can be meaningfully attributed. 
In entity modelling, these simplest aspects are known as \textit{attributes}.
We have discussed entity types and relationships at length, and now  we describe this  third vital concept.
Working together, the concepts of entity type, relationship and attribute
are the foundation of entity modelling, making it a methodology that  bridges the gap between the world as we understand it (or some domain therein) and the structure of data, which data, after-all, purports to describe some part of what is. 

\mynote
By way of explanation, 
consider that when we call to mind \textit{data}, 
then we think of names, quantities, monetary values, 
addresses, dates, temperatures, geographical coordinates, and so on. 
Such items of data as these convey information only within specific contexts and when attributed 
to subjects at hand. 
A temperature, a colour, a price, a height, a distance — such items of data 
tell us nothing less they be the temperature, the colour, the price, the height, or the 
distance of \textit{some aspect} of \textit{some thing}. 
They tell us nothing, therefore, lest they be attributed to an entity.

\mynote
A product in a catalogue or on a website  might have a full price and a sale price
and the given values will be monetary amounts.
The product has a name 
and it is likely that in the small print there is a product identifier.
Within the scope of this website, `product' is a type of entity
and `product identifier', `product name',  `full price' and `sale price' are attributes;
each attribute is constrained, in reality, in what can be meaningfully attributed to it ---
it wouldn't make sense to say `size: red' or `colour: large' and
it  would be outside the bounds of expectation 
for a price to be given as `squillions' or `rainy' or as `10 miles'.

\mynote 
Each attribute within an entity model defines a named aspect of a type of entity and
also defines what can be meaningfully attributed to this aspect; 
for example, it makes sense to  attribute a planet within the solar system with
 orbital eccentricity
  --- the details we find on Wikipedia, for example --- 
 but it doesn't make sense to attribute your stay in a hotel with orbital eccentricity. 
 Instead your stay in a hotel may be attributed with arrival date and a departure date.
Attributes in an entity model are named and doubly constrained:
firstly they are constrained to the type of entity to which they apply, 
as aspects of entities of the type (planetary orbit rather than stay in a hotel);
secondly they are constrained in what it is that may be attributed to this aspect 
(a pure number, in the case of planetary eccentricity, 
but in other cases a colour or a date or, in
some cases, free text).

\mynote
In an entity model each type of entity has an enumerated list of named attributes. These attributes may be displayed on entity relationship diagram or they might be separately documented as, for example,
 like this:
\begin{verbatim}
product 
   => product identifier: text,
      product name: text,
      full price: monetary amount,
      sale_price: monetary amount
\end{verbatim}

What is important here is that within the context of an entity type there is a list of attributes, each attribute has a name and each attribute has a type of value specified (in this case either text or monetary amount). In this stylised layout the \verb!=>! can be read as introducing a list of attributes and the colon can be read aloud as "is" or as "is of type". 

For a more technical example, in some context I might chose to describe the orbit of any planet around the sun by the following attributes
\begin{verbatim}
planet 
   => semi-major axis: distance,
      eccentricity : real number between 0 and 1,
      inclination to ecliptic: angular distance,
      longitude of ascending node: angular distance,
      argument of perihelion: angular distance,
      time of perihelion: point in time
\end{verbatim}

The attributes \textit{semi-major axis} and \textit{eccentricity}, taken together, describe the shape and size of the ellipse traced out by the planet in orbit. The other three parameters describe the plane that the ellipse lies in. 
From the first two parameters the distance of the closest approach to the sun --- the  \textit{perihelion} --- can be calculated, as too can the furthest distance away from the sun reached by the planet 
--- the \textit{aphelion}. In fact for any planet, knowedge of \textit{perihelion}  and \textit{aphelion} gives us the size and shape of the orbit --- in other words,  
\textit{semi-major axis} and \textit{eccentricity} 
can be calculated from  \textit{aphelion}  \textit{perihelion}. 
\begin{table}[H]
\small 
\setlength{\tabcolsep}{3pt}
\begin{tabular}{|l| 
  >{\centering\arraybackslash}m{0.9cm} | 
  c | 
  >{\centering\arraybackslash}m{1.45cm} | 
  >{\centering\arraybackslash}m{1.8cm} | 
  >{\centering\arraybackslash}m{1.6cm} | 
  >{\centering\arraybackslash}m{1.8cm}|}
\hline
\small Name & 
\small Semi-major axis \newline (AU) & 
\small Eccentricity &
\small Inclination \newline to ecliptic (°) & 
\small Longitude\newline of ascending\newline node \newline (°) & 
\small Argument\newline of \newline perihelion (°) & 
\small Time of \newline perihelion \\
\hline
Mercury & 0.387 & 0.205630 & 7.0049 & 48.331 & 29.124 & 2024-Dec-25 \\
Venus   & 0.723 & 0.006772 & 3.3947 & 76.680 & 54.884 & 2023-Dec-31 \\
Earth   & 1.000 & 0.016710 & 0.0000 & 0.000 & 102.937 & 2023-Jan-04 \\
\hline
\end{tabular}
\caption{Parameters describing the planetary orbits of the first three 
planets. A fuller version of this table is given in appendix \ref{PlanetaryOrbits}
}
\end{table}

What this means is that I could have chosen a different set of attributes for my entity type `planet'. 
For example, among the four
attributes \textit{perihelion}, \textit{aphelion}, \textit{semi-mean axis} and \textit{eccentricity},  any two can be used to compute the other two. The idea has been to present a \textit{core set}. This matters. When data is stored or communicated  then 
we need only store or transmit such a core set  --- any values that can be derived from them need not be sent. We will return to this topic later.

\subsection{Types of Particulars and Types of Universals}
In entity modelling, the word \textit{type} is used in two distinct senses, each playing a different role in how we describe the world:
\begin{itemize}
\item An entity type classifies particular things—for example, the entity type Planet includes Earth, Mars, and Jupiter as instances. These are concrete, identifiable entities.
\item An attribute type, by contrast, is typically a determinable: a general kind of quality or measurement. Its instances are determinates, such as red, green, 6.4 km/s, or 23.5°C. These are not things in themselves, but possible values that can be attributed to entities.
\end{itemize}
This distinction helps clarify how attributes function in a model. An attribute identifies a particular aspect of an entity (such as its colour or its orbital period), and it is constrained by a type that defines what kind of value may be associated with that aspect (e.g. a colour, a duration, or a number).

In philosophy, a distinction is drawn between \textit{particulars} and \textit{universals}. 
Numbers, colours and  other determinate quantities are generally judged to be universals. 
With this distinction in mind then we can characterise the two kinds of types that we are
talking about as
\begin{itemize}
\item types all of whose instances are particulars (entity types), and
\item types all of whose instances are universals (attribute types).  
\end{itemize} 

\subsection*{Diagramming Attributes}
\mynote
Users of entity modelling may decide whether or not to include attributes on diagrams. Chen's original paper did not show attributes on diagrams  but in 1984
he published a paper that did show attributes diagrammatically represented. 
It is not clear that he expected this style to be used wholesale. 
\mynote
In this last mentioned 1984 paper Chen presents an example diagram 
which I have reproduced in figure  \ref{ChenStyleProjectWorkerAttributed} 
and which shows a graphical representation of attributes associated with the PROJECT-WORKER relationship (\ref{projectWorkerChenStyle}) that was previously given as an example relationship in his 1976 paper.

\begin{erboxedFigure} {H}{ChenStyleProjectWorkerAttributed}{From Chen 1984. In this diagram NUMBER, NAME, NUMBER-OF-YEARS, DATE and DOLLAR-AMOUNT must all be understood as types all of whose instances are universals. Chen uses the term `value type' for such types;
thus our universal things are what he calls values. In the same way it is 
the practice in programming to refer to names, numbers, dates and so on, as values but it is more enlightening to use  the term universal, as we do here, and to reflect on  the dichotomy between things that we treat as universals and things that we treat as particulars. }
\begin{pspicture}(-6,-4.0)(5.1,1.2)
%\psgrid
\chendiamond{pw}{\rput(0,-0.05){\begin{tabular}{c}PROJECT-\\WORKER\end{tabular}}}
\rput[l](-3.5,0){
   \chenbox{e}{EMPLOYEE}
	}
\rput[l](3.5,0){
   \chenbox{p}{PROJECT}
	}
%  relationship connectors
\ncline{-}{eE}{pwW}
\nbput{\scriptsize M}	
\ncline{-}{pwE}{pW}
\nbput{\scriptsize N}
%Attributes
\rput[l](-5.2,-3.0){
	\chenvaluetype{NUMBER}{NUMBER}
	}
\ncline{->}{eSSW}{NUMBERN}
\ncput*{\footnotesize EMP\#}
\rput[l](-3.5,-3.0){
	\chenvaluetype{NAME}{NAME}
	}
\ncline{->}{eS}{NAMEN}
\ncput*{\setlength{\tabcolsep}{0pt}\footnotesize\begin{tabular}{c}EMP-\\NAME\\[-0.1cm]\end{tabular}}
\rput[l](-1.8,-3.0){
	\chenvaluetype{YEARS}{\begin{tabular}{c}\\[-0.075cm]NUMBER-\\OF-\\YEARS\end{tabular}}
	}
\ncline{->}{eSSE}{YEARSN}
\ncput*{\footnotesize AGE}
%
\rput[l](0,-3.0){
	\chenvaluetype{DATE}{DATE}
	}
\ncline{->}{pwS}{DATEN}
\ncput*{\footnotesize\begin{tabular}{c}STARTING-\\DATE\\[-0.1cm]\end{tabular}}
%
\rput[l](2.6,-3.0){
	\chenvaluetype{NUMBER}{NUMBER}
	}
\ncline{->}{pSSW}{NUMBERN}
\ncput*{\footnotesize PROJ\#}
\rput[l](4.4,-3.0){
	\chenvaluetype{YEARS}{\begin{tabular}{c}DOLLAR-\\AMOUNT\end{tabular}}
	}
\ncline{->}{pSSE}{YEARSN}
\ncput*{\footnotesize BUDGET}

\end{pspicture}
\end{erboxedFigure}

The same model is shown in our variant of the Barker-Ellis notation
 in the diagram shown in figure \ref{employeeProjectWorkerMediatedAttributed}.
As in the Chen diagram, diagrams in Barker's book use a hash symbol to foreground key attributes though it  is positioned differently to this use by Chen. In our style as shown here the key attributes are underlined as will be explained in the next section.\footnote{Our use of underlining is consistent with relational notations used in theory papers such as Zaniola and in descriptions of relations in Chen 1976.}
\begin{erboxedFigure} {H}{employeeProjectWorkerMediatedAttributed}
{The example from figure \ref{ChenStyleProjectWorkerAttributed} recast in the our Barker-Ellis style. 
In this style the value types of attributes are not shown on the diagram but are expected to be 
separately documented. }
\begin{erdiagram}{2}{12.0745}

\eret{0.1}{-1.6}{2.1}{-0.1}{0.2}{1}\eretname{0.45}{-0.45}{l}{employee}
\erCoreAttribute{0.3}{-0.65}{1}{0}{emp-no}{}
\erCoreAttribute{0.3}{-0.95}{1}{1}{emp-name}{}
\erCoreAttribute{0.3}{-1.25}{1}{1}{age}{}
\eret{4.6}{-1.6}{7.575}{-0.1}{0.2}{1}\eretname{5.047}{-0.45}{l}{project assignment}
\erCoreAttribute{4.8}{-0.65}{1}{1}{starting-date}{}
\eret{10.075}{-1.6}{12.075}{-0.1}{0.2}{1}\eretname{10.425}{-0.45}{l}{project}
\erCoreAttribute{10.275}{-0.65}{1}{0}{proj-no}{}
\erCoreAttribute{10.275}{-0.95}{1}{1}{budget}{}

% relationship assigning
\errelname{4.45}{-0.7}{r}{assigning}\errelname{2.25}{-1.15}{l}{subject of}\errelarm{4.6}{-0.85}{3.349}{-0.85}{1}{0}\errelarm{3.349}{-0.85}{2.1}{-0.85}{0}{0}\ercrowfoot{4.45}{-0.85}{4.6}{-0.7}{4.6}{-0.85}{4.6}{-1}{0}\eridrefrel{4.35}{-0.75}{-0.95}
% relationship to work on
\errelname{7.725}{-0.7}{l}{to work on}\errelname{9.925}{-1.15}{r}{resourced by}\errelarm{7.574}{-0.85}{8.824}{-0.85}{1}{0}\errelarm{8.824}{-0.85}{10.07}{-0.85}{0}{0}\ercrowfoot{7.725}{-0.85}{7.575}{-0.7}{7.575}{-0.85}{7.575}{-1}{0}\eridrefrel{7.8245000000000005}{-0.75}{-0.95}
\end{erdiagram}

\end{erboxedFigure}

Review\footnote{Chatgpt:Diagramming Section:
This is rich but perhaps slightly dense. One option could be to move some of the historical commentary (about Chen’s evolving use of diagrams) to footnotes or an appendix if you’re aiming for a smoother read. But if your audience values the historical depth, then the current form is great.}


\subsection{Descriptions and Definitions}\footnote{“That is my definition, and if you don’t like it... well, I have others.”}

\mynote
In the literature the notion of \textit{an attribute} is variously described.
In the Goodland SSADM book an attribute is defined as
\begin{erquote}
  the smallest discrete component of the system information that is meaningful,
\end{erquote}
In  Shlaer Mellor we find an attribute is described as being
the abstraction of a \textit{single} characteristic possessed by all the entities
of a type. Whereas in Richard Barker's book we find that an attribute is 
\begin{erquote}
any detail that serves to qualify, identify, classify, quantify or express an  entity,
\end{erquote}

\mynote
Chen in his  1976 paper uses the term \textit{value set} for the sets of values of what I have variously
referred to as attribute types and more prosaically \textit{types all of whose instances are universals}.
This then enables Chen to observe that \textit{An attribute can be formally defined as a function which maps...into a value set} and also \textit{Since an attribute is defined as a function, it maps an entity in an entity
set to a single value in a value set}.

\mynote
Chen's formal definition is an important insight.
It means that in entity modelling, the term \textit{attribute} is used as a specific term
for a named functional relationship between a type all of whose instances are particulars  and a type all of whose instances are universals of some kind.
The universals in question are often said to be \textit{values} but 
in philosopy  are said to be \textit{determinates}. 

\mynote
It is ironic that although attributes relate entities to values—
that is, particulars to determinates, and are therefore, very much relationships, as the term is properly used, 
they are not called such in entity modelling terminology.
It breaks the duck test, but that’s just how it is:
attributes are called attributes,
and the term relationship is reserved---both in entity modelling and in these pages---
for relationships between between types of entities;
so it comes about that the principal components of an entity model are its entity types,
their relationships, and their attributes.

\subsection{Further Characterisations}
\mynote  From what we have said, as stored within information systems
then the individually represented items --- the 
names, colours, quantities, the monetary amounts, the dates, etc --- in the language 
of entity modelling, are said to be the values of attributes. Thus an actual name 
like ``John Smith'' is said to the value of a ‘name’ attribute of a ‘person’ entityand if 
`person' entities may be attributed with ‘date of birth’ to
then on a diagram this appears like this
\ercenterPicture{personNameDobAttribute}

\mynote 
In a situation where for some persons data of birth may not be known then the attribute
may be defined to be \textit{optional}\footnote{With regard to the attribute as a functional relationship
then this is equivalent to asserting that the functional relationship is partial not total.}
Usually this will be represented on the diagram somehow. I would represent it like this
\ercenterPicture{personMandatoryNameAttribute}

\mynote
In a situation where `person' entitites can be uniquly identified by name then the `name' attribute
iss said to be identifying and in my diagrams this is shown by underlining the attribute as here	
\ercenterPicture{personIdentifyingNameAttribute}

On the otherhand, if it is necessary to give both a person's name \textit{and} their date of birth to uniquely identify 
them  then we underline
both of these attributes on the diagram:
\ercenterPicture{personIdentifyingNameDobAttribute}
Generally, systems will hold and communicate many different attributes of each type of entity and
these attributes are shown beneath the identifying attributes:
\ercenterPicture{personAttributes}

\mynote 
From a practical perspective, 
it is clear that computer programs are effective only in so long as the data items 
they manipulate are intended and understood as attributes of subject entities. It 
follows that to have an effective information system we must first have agreed types 
of subject entity and also what may be attributed to entities  of these types. 
In this agreement we agree the data content of the program or system i.e.  
its subject matter. But what is seen to be at first sight an attribute of type of entity
as, say, `phone number' of the person entity,  may
subsequently found to belong somewhere else in the model  ---   if a person can have multiple phone numbers then ‘phone number’ is 
not an attribute  of a ‘person’ entity type \textit{per se} but an attribute of a 
separate but related entity as here 
\ercenterPicture{personPhone}


\subsection{Relationships between Universals}
\mynote A final point on universals. In addition to the relationships and attributes as so far described there will exist many relationships between universals themselves
for example there will be logical and arithmetical relationships. When we construct an entity model these universal relationships are taken as givens --- we do not set out to model them.
For example, temperature values are comparable, and we can speak of one being higher than another.
\commentary{revise chatgpt suggested text}
 These kinds of relationships between values (universals) — like comparisons, \newtt{unit conversions} and
 \newtt{arithmetical operations} 
 \oldtt{units, or derived quantities} — are typically considered background knowledge \oldtt{or}
 \newtt{to be} handled by the application logic, not modelled in the entity structure itself.



\subsection{What are the Givens?}

I ask you to go along with the proposition that numbers are \textit{givens} and \textit{universal}, though I admit there is room for discussion as to what exactly we mean by \textit{number}. We need to distinguish between natural numbers ($0,1,2,...$), integers ($...,-2,-1,0,1,2,3,...$), and real (decimal) numbers, as used in pricing, measurement, proportions, and so on. 

Moreover, in information systems, numbers are typically finite in extent, constrained by 32- or 64-bit representations (or similar schemes). These technical limits impose restrictions both on the magnitudes of integral values and on the granularity and precision of real-number representations.

Likewise, I ask you to consider alphabetic characters and textual values as givens and universals—though,  they vary widely across the world and across systems. We won’t be delving into that variety here.

When it comes to defining the values we take as givens, different technologies—such as XML Schema, IDL, and SQL—define their own sets of primitive types. Modelling notations may adopt one or another of these foundational systems. A certain attribute might be modelled conceptually as having values of type \textit{real}; or, more pragmatically, it might be modelled as type \textit{float64} to reflect its implementation in a particular system.


\subsection{Styles of Attribute Type Definition}

There are also questions of style.

One such consideration is the treatment of units. Units of measurement are crucial—so how should they be handled? One option is to include the unit in the attribute name, as in:

\begin{erquote}
heightInCm: float64
\end{erquote}

Alternatively, one might define custom types that incorporate the unit, allowing for a more abstract naming scheme:

\begin{erquote}
height: numberOfCm,\\
width: numberOfCm
\end{erquote}

I will probably favour this latter style.

A further abstraction is to conceptualise \textit{length} as a universal, without committing to a specific unit. For example:

\begin{erquote}
height: length
\end{erquote}
where we expect the length concept to subsume the unit of measurement in some as yet unspecified way. 

Another stylistic question arises in the context of scientific measurement: how should accuracy and precision be represented? Can a single attribute contain both a measured magnitude and an associated uncertainty? Or should these be modelled as distinct attributes?
















 %\fi
 \section{Identifying Features}
\label{IdentifyingFeatures}
\subsection{The Question of Identity}
\mynote
The focus of this section is the identification of entities ---  we ask, regarding any particular type, how can entities of the type be distinguished from one other 
--- what distinguishing features do they have? 

For there to be an answer to this question then we must assume that no two entities of the given type are alike in all their features. This in fact  is a philosophical position 
--- it is called the principle of the identity of indiscernibles. 
While this principle is contested in philosophy, in entity modelling we tend to assume it holds.
Thus we work under the assumption that no two entities are exactly the same in all their features. 

From this assumption, it follows that for any type of entity there is a minimum set of features sufficient to distinguish any two entities of the type. Any such set of features is said to be identifying as it gives us a way of identifying and therefore referencing entities of the type. 

 Generally, 
 the identifying features will consist of a combination of attributes and relationships, 
 with the relationships being functional relationships to other entities. 
Though it is the set as a whole that is \textit{identifying} 
informally we shall say each of the attributes and relationships in the set is identifying. 

Such a set has a simple mathematical characterisation, which we come to later following various
examples  of identifying features and the notation used to represent them.

\subsection{The Significance of Identifying Features}

\mynote 
It is desirable that an entity model documents identifying features for the types of entities it describes and  it is  absolutely necessary for most types in those models that are to be used as data specifications as we will come to in a later section. 

\mynote
The reason for this is that identifying features give us concrete ways of identifying entities within a context and thereby enables us to describe the relationships between entities. Consequently, they provide methods for communicating  relationships and this ultimately facilitates the storage of representations of entities in data systems.

\subsection{Notation}
 \mynote
Various conventions have been used diagrammatically to convey that an attribute is identifying. In Barker's book the \# symbol is placed to the left of the attribute to indicate as much. In diagrams in this book the name of the attribute is underlined and this is consistent with papers on relational theory in which  the names of columns are underlined to show that they are key columns.

\mynote Identifying relationships on the other hand, both in this book and in Barker's book 
are distinguished by drawing a bar across them so that they like this: \barkerEllisJ\ or like this: \barkerEllisK.



 
\subsection{Examples}
\mynote
Flight number, payment card number, country code, international bank account number (IBAN), 
international standard book number (ISBN), social security number, passport number, part number and many like attributes are designed specifically for the purpose of enabling unique identification of entities of a particular type so that each entity of the type can be referenced. Such attributes are said to be identifying attributes and are underlined 
on diagrams to indicate as much. Attributes 
such as ISBN and IBAN are designed to be globally unique.  
Others such as:
\begin{itemize}
\item
car registration plate number and passport number have values that are unique, and are therefore identifying, only with respect to particular issuing authorities, for example at a national level,
\item part numbers may be unique with respect to particular part suppliers or manufacturers,
\item airport terminal numbers are unique and therefore identifying with respect to the airports they are located at.
\end{itemize}
In these bulleted cases the underlining of the identifying attribute is combined with the barring of the context providing relationship as shown here in fragments in which, in the absence of other detail, types \textit{passport}, \textit{part} and \textit{terminal} are shown to be uniquely identifying  in relation
to \textit{country}, \textit{supplier} and \textit{airport}, respectively, 
and in which these later types are shown to have instances uniquely identified by name:

\begin{tabular} {m{3.5cm} m{3.5cm} m{3.5cm}}
\begin{erdiagram}{4.5}{2.9000000000000004}

\eret{0.1}{-1}{2.8}{-0.1}{0.2}{1}\eretname{0.37}{-0.45}{l}{country}
\erCoreAttribute{0.3}{-0.65}{1}{0}{name}{}
\eret{0.1}{-4.5}{2.8}{-2.1}{0.2}{1}\eretname{0.37}{-2.45}{l}{passport}
\erCoreAttribute{0.3}{-2.65}{1}{0}{passport number}{}
\erCoreAttribute{0.3}{-2.95}{1}{1}{date of issue}{}
\erCoreAttribute{0.3}{-3.25}{1}{1}{valid until date}{}
\erCoreAttribute{0.3}{-3.55}{1}{1}{family name}{}
\erCoreAttribute{0.3}{-3.85}{1}{1}{given name}{}
\erCoreAttribute{0.3}{-4.15}{1}{1}{date of birth}{}

% relationship having_issued
\errelname{1.3}{-1.3}{r}{having}\errelname{1.3}{-1.6}{r}{issued}\errelname{1.6}{-1.95}{l}{issued by}\errelarm{1.45}{-0.999}{1.45}{-1.549}{0}{0}\errelarm{1.45}{-1.549}{1.45}{-2.099}{1}{0}\eridcomprel{1.35}{1.5500000000000003}{-1.8499999999999996}\ercrowfoot{1.45}{-1.95}{1.3}{-2.1}{1.45}{-2.1}{1.6}{-2.1}{0}
\end{erdiagram}
 &
\begin{erdiagram}{3.5999999999999996}{2.9}

\eret{0.1}{-1.3}{2.5}{-0.1}{0.2}{1}\eretname{0.34}{-0.45}{l}{supplier}
\erCoreAttribute{0.3}{-0.65}{1}{0}{name}{}
\erCoreAttribute{0.3}{-0.95}{1}{1}{address}{}
\eret{0.1}{-3.6}{2.5}{-2.4}{0.2}{1}\eretname{0.34}{-2.75}{l}{part}
\erCoreAttribute{0.3}{-2.95}{1}{0}{part number}{}
\erCoreAttribute{0.3}{-3.25}{1}{1}{description}{}

% relationship supplying
\errelname{1.15}{-1.6}{r}{supplying}\errelname{1.45}{-2.25}{l}{supplied by}\errelarm{1.3}{-1.3}{1.3}{-1.85}{0}{0}\errelarm{1.3}{-1.85}{1.3}{-2.4}{1}{0}\eridcomprel{1.2}{1.4000000000000001}{-2.15}\ercrowfoot{1.3}{-2.25}{1.15}{-2.4}{1.3}{-2.4}{1.45}{-2.4}{0}
\end{erdiagram}
 &
\begin{erdiagram}{3.3}{2.4}

\eret{0.1}{-1.3}{1.8}{-0.1}{0.2}{1}\eretname{0.27}{-0.45}{l}{airport}
\erCoreAttribute{0.3}{-0.65}{1}{0}{name}{}
\erCoreAttribute{0.3}{-0.95}{1}{1}{country}{}
\eret{0.1}{-3.3}{1.8}{-2.4}{0.2}{1}\eretname{0.27}{-2.75}{l}{terminal}
\erCoreAttribute{0.3}{-2.95}{1}{0}{number}{}

% relationship having
\errelname{0.8}{-1.6}{r}{having}\errelname{1.1}{-2.25}{l}{a facility of}\errelarm{0.95}{-1.3}{0.95}{-1.85}{0}{0}\errelarm{0.95}{-1.85}{0.95}{-2.4}{1}{0}\eridcomprel{0.85}{1.05}{-2.15}\ercrowfoot{0.95}{-2.25}{0.8}{-2.4}{0.95}{-2.4}{1.1}{-2.4}{0}
\end{erdiagram}

\end{tabular}

\subsubsection{Two or more identifying attributes}
For some types, the values of more than one attribute need be examined to distinguish entities of the type ---  as with a multi-dimensional coordinate system a number of attributes provide values that  taken together are unique 
and enable an entity to be uniquely identified and referenced.  
In such a case it is a number of attributes, latitude and longitude, for example, rather than an individual attribute, which as a set  can be said to be identifying. 
This is depicted by underlining each attribute that is in the set.
This is  shown in the following example
in which spot heights on maps are shown to be identifiable by the combination of latitude and longitude within the context of the map:
 \begin{equation}
 \raisebox{-0.5cm}{\begin{erdiagram}{3.9}{2.8}

\eret{0}{-1.3}{2.4}{-0.1}{0.2}{1}\eretname{0.24}{-0.45}{l}{map}
\erCoreAttribute{0.2}{-0.65}{1}{0}{title}{}
\erCoreAttribute{0.2}{-0.95}{1}{1}{scale}{}
\eret{0}{-3.9}{2.4}{-2.4}{0.2}{1}\eretname{0.24}{-2.75}{l}{spot height}
\erCoreAttribute{0.2}{-2.95}{1}{0}{latitude}{}
\erCoreAttribute{0.2}{-3.25}{1}{0}{longitude}{}
\erCoreAttribute{0.2}{-3.55}{1}{1}{altitude}{}

% relationship depicts
\errelname{1.05}{-1.6}{r}{depicts}\errelname{1.35}{-2.25}{l}{depicted on}\errelarm{1.2}{-1.3}{1.2}{-1.85}{0}{0}\errelarm{1.2}{-1.85}{1.2}{-2.4}{1}{0}\eridcomprel{1.0999999999999999}{1.3}{-2.15}\ercrowfoot{1.2}{-2.25}{1.05}{-2.4}{1.2}{-2.4}{1.35}{-2.4}{0}
\end{erdiagram}
}
 \end{equation} 
\mynote 
For a second example, 
consider modelling the type \textit{person} whose instances, of course, are people.
Now people are tricky to identify from a few well defined attributes but for some purposes at least the attributes full name, date of birth, and  home address might be considered sufficient. In an entity model this would be shown by underlining these three atributes like this:

\begin{equation}
\label{personAttributes2}
\raisebox{-1.5cm}{\begin{erdiagram}{2.7}{3}

\eret{0}{-2.7}{3}{-0}{0.2}{1}\eretname{0.3}{-0.35}{l}{person}
\erCoreAttribute{0.2}{-0.55}{1}{0}{name}{}
\erCoreAttribute{0.2}{-0.85}{1}{0}{address}{}
\erCoreAttribute{0.2}{-1.15}{1}{0}{date of birth}{}
\erCoreAttribute{0.2}{-1.45}{1}{1}{nationality}{}
\erCoreAttribute{0.2}{-1.75}{0}{1}{passport number}{}
\erCoreAttribute{0.2}{-2.05}{0}{1}{is married}{}
\erCoreAttribute{0.2}{-2.35}{0}{1}{height}{}

\end{erdiagram}
}
\end{equation}

\subsubsection{Intersection Entities}
\mynote \commentary{move to beginning and refer back.}
For some types in some circumstances there are no attributes and only relationships in the identifying set. 
We saw an example earlier
in figure \ref{employeeProjectWorkerMediatedAttributed}. In that figure there
is an entity type
\textit{project assignment} which implements a many-many relationship between employees and projects. 
Such entity types as this are common in the Barker-Ellis notation 
standing as they do in positions that would be occupied many-many relationships
 and depicted by  diamonds in Chen's notation. 
 Because the type `project assignment' implements a many-many relationship, we can be sure that entities of the type  are uniquely identified by 
 identifying the employee that they assign in combination with the project they assign to and this means that the relationships \textit{assigning} and \textit{to work on}, in combination, are identifying features of the type. This has been indicated in 
figure  \ref{employeeProjectWorkerMediatedAttributed} by the bars drawn across these two relationships. As I mentioned before such types as this are sometimes said to be intersection entities.

\subsection {Mathematical Characterisation}
\mynote From a mathematical perspective, 
each feature of an entity type can be represented by a function. A set of features is suffiecient to uniquely identify entitities of a type if and only if when thought  of in this way the set is \textit{jointly injective}.   
See subsection \ref{JointlyInjective} for a defintion of this term.

\subsection{Summary}
\mynote A set of features of a type of entity i.e. a set of attributes and outgoing direction We say that al relationships, is said to be \textit{identifying} provided that the values of these features,
taken together, is guaranteed to uniquely identify an entity of the type i.e. to be such that no two distinct entities of the type can have identical values for all of the features. 

\mynote In the simplest cases a single attribute will suffice. 
Social security number, part no, 
passport number and like attributes are designed to uniquely identify entities of the type from others of the type. 
An example of this 
in  shown in figure \ref{boardingPass2} in which entities of type \textit{airline route} are uniquely identified by a \textit{flight number} attribute.
This is
indicated in the diagram  by the underlining of the name of this attribute where it appears on the diagram.\footnote{Aside: There is a trivial difference here from Barker's notation because he distinguishes the identifying attributes with a \# symbol where we here use  underlining.} 


\mynote A final point, as we have said, identifying attributes are of particular significance with regard to 
 how relationships are represented in data and therefore how communicated and stored.
They are also significant in evaluating the goodness of an entity model, as we will see later.





 

 \section{Referencing Entities}
\label{ReferencingEntities}

In this section we explore the significance of the identifying features
of a type and the manner in which knowledge of these features enables us to 
reference entities and, in turn, to convey, communicate and store instances of relationships. 
There is more to this than initially meets the eye and we explore the topic
by way of examples. 

\subsection{The Simplest Case --- Planets and their Moons as Named Entities}
\mynote
In the simplest examples entities of a type have names associated with them and these are unique and can be  quoted to reference entities of the type. 
Name attributes in these cases are archetypal examples of identifying attributes. 

We have already seen an example of the role of the naming of both planets and moons in the example given earlier for a phraseology for the communication of relationships
 in which we presented this statement:
\begin{equation}
\label{JupiterIoShort}
\mbox{\textit{Jupiter is orbited by Io.}}
\end{equation}
as a representation of an instance of the binary relationship
\begin{gather}
\label{planetMoonSecondInstance}
\raisebox{-1cm}{
\begin{erdiagram}{1.4}{6.9}

\eret{0.1}{-1}{1.85}{-0.4}{0.2}{1}\eretname{0.975}{-0.75}{}{planet}
\eret{5.15}{-1}{6.9}{-0.4}{0.2}{1}\eretname{6.025}{-0.75}{}{moon}

% relationship orbited by
\errelname{2}{-0.55}{l}{orbited by}\errelname{5}{-1}{r}{orbiting}\errelarm{1.85}{-0.7}{3.5}{-0.7}{0}{0}\errelarm{3.5}{-0.7}{5.15}{-0.7}{1}{0}\ercrowfoot{5}{-0.7}{5.15}{-0.55}{5.15}{-0.7}{5.15}{-0.85}{0}
\end{erdiagram}

}
\end{gather}
for in the context of statement (\ref{JupiterIoShort}), Jupiter is the name of, and therefore  a reference to, a planet and  Io is the name of, and therefore a reference to, a moon i.e. a natural satellite.\footnote{
Note that within the context of the solar system the name Io  doesn't uniquely identify this moon of Jupiter amongst all named solar system objects since it is also the name of an asteroid that orbits within the asteroid belt.
The names of moons are only unique as names of moons. 
Of course in other contexts, both Jupiter and Io are the names of mythological figures.
} 
We understand  (\ref{JupiterIoShort}), therefore, to be shorthand for the 
more pernickerty 
\begin{equation}
\mbox{\textit{The planet with name `Jupiter' is orbited by the moon with name `Io'.}}
\end{equation}

\mynote So planets and moons are named which is to say they have name attributes.
\commentary{Rewrite this paragraph.}
Of course name attributes are archetypal examples of identifying attributes --- we name things so that we can identify them in context. In such a situation the name of an entity, i.e. the value of the name attribute,  can be used in context to definitively identify the entity.
Quoting this value, i.e. using the name, is one way of referencing an entity of the type from within the context of another.
 \mynote
Focusing on just this aspect of the solar system we  draw the model shown in figure
\ref{planetMoonModel}
containing the relationship (\ref{planetMoonSecondInstance}) and showing name attributes of both moons and planets and
underlining these to showing that each of them is unique and therefore identifying. 
This is the entity model (albeit of a very small universe) that explains the communication (\ref{JupiterIoShort}) or equally, and as we shall see later, the structure of the data table of planets and moons given earlier.
\begin{erboxedFigure} {H}{planetMoonModel}
{
caption blah blah blah
 }
\begin{erdiagram}{2.4999999999999996}{6.9}

\eret{0.1}{-2.1}{1.85}{-1.2}{0.2}{1}\eretname{0.425}{-1.55}{l}{planet}
\erCoreAttribute{0.3}{-1.75}{1}{0}{name}{}
\eret{5.15}{-2.1}{6.9}{-1.2}{0.2}{1}\eretname{5.475}{-1.55}{l}{moon}
\erCoreAttribute{5.35}{-1.75}{1}{0}{name}{}
\eret{0}{-0.2}{6.9}{0.3}{0.2}{1}

% relationship 
\errelname{1.125}{-0.5}{l}{}\errelarm{0.975}{-0.2}{0.975}{-0.7}{1}{0}\errelarm{0.975}{-0.7}{0.975}{-1.2}{0}{0}\ercrowfoot{0.975}{-1.05}{0.825}{-1.2}{0.975}{-1.2}{1.125}{-1.2}{0}
% relationship 
\errelname{6.175}{-0.5}{l}{}\errelarm{6.025}{-0.2}{6.025}{-0.7}{1}{0}\errelarm{6.025}{-0.7}{6.025}{-1.2}{0}{0}\ercrowfoot{6.025}{-1.05}{5.875}{-1.2}{6.025}{-1.2}{6.175}{-1.2}{0}
% relationship orbited by
\errelname{2}{-1.5}{l}{orbited by}\errelname{5}{-1.95}{r}{orbiting}\errelarm{1.85}{-1.65}{3.5}{-1.65}{0}{0}\errelarm{3.5}{-1.65}{5.15}{-1.65}{1}{0}\ercrowfoot{5}{-1.65}{5.15}{-1.5}{5.15}{-1.65}{5.15}{-1.8}{0}
\end{erdiagram}

\end{erboxedFigure}
\mynote
We are very much stating the obvious of course but there is a pattern here and it is repeated over and over when instances of relationships are communicated  and the same pattern occurs more formally  and uniformly when instances of relationships are communicated in software systems and/or stored in data.

\begin{noteforfuture}
would like an example with two or three identifying attributes but --- cannot think of one. The ones I have given earlier are not very convincing.
\end{noteforfuture}

\subsection{Context and the Referencing of Entities}
\mynote
This section is about the way that entities may be referenced using identifying features
 when one these of features is a relationship.
We look at how,  in this situation, the referencing 
--- the nature of the reference ---   
is impacted by the context within which the reference is being made. 
We see the reference as a phrase quoting the values of identifying attributes
and with the number quoted depending on 
the context within which the reference is made.
What we see contributes to a later analysis --- the analysis of  how 
instances of a relationship can be conveyed, communicated or stored and
how this is affected by shared knowledge we might have about 
the scope of  the relationship.\footnote{Speak of scope beforehand because need to keep interest in what seems not interesting. Add footnote about scope at first use (here?). Probably explain scope in next section because scope affects everything not just ``data''.} \footnote{if explained before could here say ``Here is that scope concept again. Explanation comes in next section.''}

\mynote
 We use as an example the referencing of characters from  plays
 within the context of  the  dramatic arts  as modelled earlier in figure xxx.
Since there are only a few specific details of figure xxx --- three in fact ---
that concern us here we pull these out as details  (a), (b) and (c) in
figure \ref{dramaticArtsIdentificationDetails} . 
%\vspace{-0.2cm} % Dont know why I need this but I do.
\begin{erboxedFigure}{H}{dramaticArtsIdentificationDetails}
{Details from the model of the dramatic arts that was presented in figure xxx.
(a), (b) and (c) show the identifying features of, respectively, types character, play and playwright. 
}
\vspace{-0.7cm}% Dont know why I need this but I do.
\begin{tabular}{ccccc}
(a) 
\raisebox{-1.5cm}{\begin{erdiagram}{2.4}{2.7}

\eret{0}{-2.4}{2.7}{-1.4}{0.2}{1}\eretname{0.27}{-1.75}{l}{character}
\erCoreAttribute{0.2}{-1.95}{1}{0}{name}{}
\eret{0}{-0.25}{2.7}{0.25}{0.2}{1}\eretname{1.095}{-0.2}{l}{play}

% relationship containing
\errelname{1.2}{-0.55}{r}{containing}\errelname{1.5}{-1.25}{l}{within}\errelarm{1.35}{-0.25}{1.35}{-0.825}{0}{0}\errelarm{1.35}{-0.825}{1.35}{-1.4}{1}{0}\eridcomprel{1.25}{1.4500000000000002}{-1.15}\ercrowfoot{1.35}{-1.25}{1.2}{-1.4}{1.35}{-1.4}{1.5}{-1.4}{0}
\end{erdiagram}
}
 && (b) \kern-0.35cm
\raisebox{-1.5cm}{\begin{erdiagram}{2.4}{2.7}

\eret{0}{-2.4}{2.7}{-1.4}{0.2}{1}\eretname{0.27}{-1.75}{l}{play}
\erCoreAttribute{0.2}{-1.95}{1}{0}{title}{}
\eret{0}{-0.25}{2.7}{0.25}{0.2}{1}\eretname{0.713}{-0.2}{l}{playwright}

% relationship writer_of
\errelname{1.2}{-0.55}{r}{writer}\errelname{1.2}{-0.85}{r}{of}\errelname{1.5}{-1.25}{l}{by}\errelname{1.5}{-0.95}{l}{written}\errelarm{1.35}{-0.25}{1.35}{-0.825}{1}{0}\errelarm{1.35}{-0.825}{1.35}{-1.4}{1}{0}\eridcomprel{1.25}{1.4500000000000002}{-1.15}\ercrowfoot{1.35}{-1.25}{1.2}{-1.4}{1.35}{-1.4}{1.5}{-1.4}{0}
\end{erdiagram}
}
 &&  (c) 
\raisebox{-1.5cm}{\begin{erdiagram}{2.4}{2.7}

\eret{0}{-2.4}{2.7}{-1.4}{0.2}{1}\eretname{0.27}{-1.75}{l}{playwright}
\erCoreAttribute{0.2}{-1.95}{1}{0}{name}{}
\eret{0}{-0.25}{2.7}{0.25}{0.2}{1}\eretname{0.521}{-0.2}{l}{dramatic arts}

% relationship includes_works by
\errelname{1.2}{-0.55}{r}{includes}\errelname{1.2}{-0.85}{r}{works by}\errelname{1.5}{-1.25}{l}{within}\errelname{1.5}{-0.95}{l}{known}\errelarm{1.35}{-0.25}{1.35}{-0.825}{0}{0}\errelarm{1.35}{-0.825}{1.35}{-1.4}{1}{0}\eridcomprel{1.25}{1.4500000000000002}{-1.15}\ercrowfoot{1.35}{-1.25}{1.2}{-1.4}{1.35}{-1.4}{1.5}{-1.4}{0}
\end{erdiagram}
}
\end{tabular}
\end{erboxedFigure}

\mynote 
Detail (a) can be paraphrased as: 
\begin{equation}
\label{characterReferenceFromPlay}
\text{\parbox{9cm}{\textit{within the context of a play, a character within it can be referenced by name}}}\\
\end{equation}

So in this stated context the value of a single identifying attribute suffices. 
Accordingly in a stage direction such as this one:  
\begin{equation*}
\text{\textit{Enter Sebastian.}}
\end{equation*}
as we are soundly rooted in the context of a play, in this case in William Shakespeare's Twelfth Night, say,
though the name of the character isn't unique, not so even in the plays of Shakespeare, the name alone
 --- the value of a single identifying attribute --- suffices.
This is the simplest of observations
but it is worth drawing attention to it  
because it is an an example of a phenomenon  that is significant
and impactful on the way that 
relationships are conveyed, communicated and stored.
Examples like this will be used to illuminate a  mechanism frequently found at work in the
structuring of data.

\begin{exerciseforreader}
Describe the phraseology 
at work to produce  (\ref{characterReferenceFromPlay}) from detail (a). 
Use the same phraseology to algorithically paraphrase details (b) and then detail (c).
What mistakes of english arise?
What would a phraseology need take account of to correct these mistakes.
What would a more flexibly phraseology need take into account.
\end{exerciseforreader}
 
\mynote %49
On the other hand, details (a) and (b) taken together imply: 
\begin{equation}
\label{characterReferenceFromPlaywright}
\text{\parbox{9cm}{\textit{within the context of a playwright
(say in a discussion of the works of a playwright), 
a character within the works of the playwright 
can be referenced by the title of a play 
and a name of a character within that play. }}}\\
\end{equation}
So in answer to a question``Who is the Shakespeare character who is a twin brother and is believed lost at sea?''
someone might answer with 
\begin{equation*}
\label{antonioReferenceFromPlaywright}
\text{\parbox{9cm}{\textit{Sebastian from Twelfth Night. }}}\\
\end{equation*}
using both name of the character and title of the play and, 
in accordance with (\ref{characterReferenceFromPlaywright}),
using the values of two identifying attributes.
\mynote 
Likewise, from (a), (b) and (c) taken together we have that
\begin{equation} 
\label{absoluteCharacterReferencing}
\text{\parbox{9cm}{\textit{without context, a character within a play 
may be referenced by the name of character, the title of a play and the name of the playwright who wrote the play.}}}
\end{equation}
so in answer to the question ``who is the linguist character who teaches diction and manners to a flower girl?''
 then the answer might be the name of a character, the title of a play and the name of the author as in 
 \begin{equation*}
\label{HenryHigginsReferenceFromAbsolute}
\text{\parbox{9cm}{\textit{Henry Higgins from George Bernard Shaw's Pygmalion. }}}\\
\end{equation*}
in accordance with (\ref{absoluteCharacterReferencing}) the values of three identifying attributes have been used.

\subsection{Referentials}
\mynote
In this book we will refer to the values of identifying attributes as referentials.
In the examples above 
and assuming  the model of the dramtics arts previously given in figure xxx,
 `Sebastian', `Twelfth Night' and `Shakespeare' are all referentials 
 as to are `Henry Higgins', `George Bernard Shaw'  and `Pygmalion' and
we saw that depending on context one, two or three referentials were required
 to identify an entities of type character.
\mynote 
Later in this book we will introduce attributes whose values are referentials 
for the purpose of referencing one entity from another.
In general, it requires multiple referentials to reference an entity (one for every identifying
feature of the entity) and so, in general, multiple referential attributes are used to represent relationships.\footnote{If you are already familiar with the relational model of data then you ought recognise that I am speaking of foreign keys here. 
I am avoiding use of the term `foreign key' though --- I don't find it very helpful --- its use is an accident of history and some standards (which) have avoided use of it and have used the term referential attribute or column (which?) instead. } 

\subsection{Communicating Relationships}

\mynote As we have said above and according to the model we have assumed,
  characters from plays need be identified by three referentials
 --- the name of the character, the title of the play and the name of the playwright.
It follows, on the surface of it at least, that conveying a instance of 
relationship instance between characters will require six referentials
--- three to reference one party to the relationship and three to reference the other.

An example that supports this view is found in this statement\footnote{You will have to give me a bit of leeway regarding the correctness or otherwise of this statement --- it is the structure that I interested in.}
\begin{equation}
\text{\parbox{9cm}{\textit{Romeo from Shakespeare's Romeo and Juliet
                           is modelled on Pyramus from Ovid's Metamorphoses.}}}
\end{equation}
which, interpreted from the perspective of the dramatic arts that we have given above, does indeed contain six referentials as shown here but abbreviated to fit on a single line:

\begin{tabular}{l}
Rom from Shake's Romeo & Juliet
                           is modelled on Pyr from Ovid's Metamorph.
\end{tabular}

\mynote % was 89 communicating entitites
Naively this implies that if we wish to communicate an instance of relationship between two different characters then we will be required to communicate six referentials
 -- three to identify one party in the relationship and three to identify the other party. 
 we will see shortly  --- and this is very significant --- that this is not always the case.

\mynote
 This is not the case, and so not what we instinctively do, when we have \textit{a priori} knowledge of the relationship being communicated is limited in scope. If we know the relationship is confined to individual playwrights or is internal to individual play or script then clearly we do not 
  communicate the playwright and play title twice over. The number of referentials may in such a case collapse to four: name of playwright, title of play and name of character.
Shlair-Lang in a very much underrated paper coined the term \textit{collapsed referentials} for this phenomena in the context of data representation of models represented in entity relationship notation.

\mynote was %90 communicating entitites
So we have, in the context of the plays of William Shakespeare,
\begin{equation}
\label{AntonioLovesSebastian2}
\mbox{\textit{The character Antonio in Twelfth Night loves Sebastian.}}
\end{equation} 

You might think that a pedantic elaboration of (\label{AntonioLovesSebastian}) would be:
\begin{equation}
\text{\parbox{9cm}{\textit{The character named Antonio in Shakespeare's play Twelfth Night loves the character named Sebastian in Shakespeare play Twelfth Night.}}}
\end{equation} 
but there is a reason why the shorter
\begin{equation}
\text{\parbox{9cm}{\textit{The character named Antonio in Shakespeare's play Twelfth Night loves the character named Sebastian.}}}
\end{equation} 
is equally unambiguous. This is because the relationship `loves' in this context is a relationship which is internal to the plot of a single individual play. 
It is an intra-play relationship rather than an inter-play relationship.
Reading from left to right we establish that we are making a statement about 
Antonio from Twelfth Night  and so must look to establish Sebastian as a character from the same play i.e. also from Twelfth Night. This is an example of a relationship which is not global in scope (it is local to the play) and therefore describing instances of this fictional loves relationship
doesn't require as many referential attributes as would otherwise be the case if it was global in scope. 

The relationship is conveyed by a total of  4 referential attributes
(Shakespeare, Twelfth Night, Antonio and Sebastian) not the six that might have been required
if the relationship was global in scope so that characters from one play might be described as being in love with a character from a different play.

\newcommand{\dashRefOne}{2pt 2pt}
\newcommand{\dashRelationship}{1pt 0pt}
\newcommand{\dashRefTwo}{2pt 2pt}
\begin{erboxedFigure}{H}{antonioloves}
{
The four referentials make reference to two different characters. The six referentials that would have been required in the general case collapse to four. 
 }
\begin{tabular}{l}
\Rnode{w1}{\dotuline{Sebastian}} in \Rnode{w2}{\dotuline{Shakepeare}}'s \Rnode{w3}{\dotuline{Twelfth Night}} \Rnode{w4}{\uline{loves}}  \Rnode{w5}{\dotuline{Antonio}} \\[1.4cm]
\kern1.2cm\Rnode{ref1}{\textit{reference 1}}
\kern0.75cm\Rnode{rel}{\textit{relationship}}
\kern0.6cm\Rnode{ref2}{\textit{reference 2}} \\[0.5cm]
\syntag{\dashRefOne}{ref1}{0.9}{w1}{0}
\syntag{\dashRefOne}{ref1}{0.9}{w2}{-0.2}
\syntag{\dashRefOne}{ref1}{0.9}{w3}{-0.2}
\syntag{\dashRelationship}{rel} {0.7} {w4}{0}
\syntag{\dashRefTwo}{ref2}{0.4}{w2}{0.2}
\syntag{\dashRefTwo}{ref2}{0.4}{w3}{0.3}
\syntag{\dashRefTwo}{ref2}{0.4}{w5}{0}
\end{tabular}
\end{erboxedFigure}



On the other hand contrast with
\begin{equation}
\mbox{\textit{Romeo from Shakespeare's Romeo and Juliet is modelled on Pyramus from Ovid's 
Metamorphoses.}}
\end{equation}

in this there is a relationship instance conveyed by six referentials. This is because the resemblance relationship to is not an internal relationship. It is between characters and a character from one play may be perceived as resembling a character from a different play.

In the first example we say that there are collapsed referentials. 

None of these referentials collapse. The models of the two relationships
look the same but behind the scenes they are different. They have different scopes and we come back to talking about these later.

Antonio and Twelfth Night refer
Sebastian and Twelfth Night refer.

Twelfth Night doubles up.

Now two plays are referenced.



\subsection{Philosophy}
\mynote
There are one, two or three attribute values to reference a character from a play namely
the name of character, maybe, the title of the play and, maybe, the name of the playwright. 
These values are universals so up to three universals to reference a play. 
Such universals as these are what we use to reference entities 
and therefore to communicate relationship instances.
\mynote
We think particulars --- entities and relationships --- but we speak universals.
\mynote
Further, universals are the stuff of data though its meaning is in the 
particulars --- entities and relationships --- it describes. 

\newpage


 

 
\section{Scope}
\label{Scope}
If we can imagine examining all binary relationships and all their possible instances then we would find in this examination that a very  good number of  binary relationships have instances that  do not reach as far across the landscape of entities as seems possible from an examination of the types of entity involved. 

\begin{noteforfuture}Need an example here. It would be best for it to be an observation or an enhancement regarding an earlier example. Could it be from Barker?
\end{noteforfuture}

We use the term \textit{scope} and speak of the scope of a relationship as a way of articulating this extent or range across which instances a relationship may reach.  

Its scope is a characteristic of a relationship. A relationship may be broad in scope or narrow in scope. The broader it is in in scope the more bits of information are needed to communicate its instances. 

In a programming or database context an understanding of scope comes an understanding of scope errors i.e. violations of scope. 

\begin{noteforfuture}
Schlaer Mellor example. Documented as a danger. Here we don't see it as a danger but something to get ahead of. To document scope up front when relationships are documented in an entity model.
\end{noteforfuture}
 \section{Discussion --- The Airport Gate Example}
\label{AirportGateExample}

\mynote 
This example is based on a message that I received on my phone on the day that I returned from a recent (at time of writing) holiday. The message read:
\begin{equation}
\label{LH2502PhoneMessage}
\text{\parbox{9cm}{\textit{
Your flight LH2502 from Munich to Manchester on 14 August 2024 at 15:55 will depart from gate L06.}}}
\end{equation}
Consider, this message contains multiple referentials and individually or combined these make reference to
multiple entities:
\begin{itemize}
	\item the flight number, LH2502, makes reference to an \textit{airline service},\footnote{which, from one point of view at least, is a little odd because the name flight number purports to reference a flight. It isn't so odd though because on any given day flight numbers do reference flights.}
	\item each airport name, Munich, respectively, Manchester, makes reference to an \textit{airport},
	\item the combination of flight number, LH2502, and date, 14 August 2024, make reference to a \textit{flight},
	\item the combination of the airport the flight is identified as being from, Munich, and the gate number, 
	L06, identifies a \textit{gate}.
\end{itemize}
\mynote 
These referentials, the names (Munich, Manchester), numbers (LH2502, L06) and the date (14 August 2024)
I can understand as the values of attributes of the various referenced entitites. 
The types of these entities (airline service, airport, flight and gate), the relationships between them and the attributes employed
(flight number, date of departure, airport name and gate number) I can arrange on a diagram like this:
\begin{equation}
\label{boardingGate1}
\raisebox{-1.5cm}{\begin{erdiagram}{3.8}{9.100000000000001}

\eret{0.5}{-1.7}{3.2}{-0.7}{0.2}{1}\eretname{0.77}{-1.05}{l}{airline service}
\erCoreAttribute{0.7}{-1.25}{1}{0}{flight number}{}
\eret{0.5}{-3.8}{3.2}{-2.8}{0.2}{1}\eretname{0.77}{-3.15}{l}{flight}
\erCoreAttribute{0.7}{-3.35}{1}{0}{date of departure}{}
\eret{6.2}{-1.7}{8.9}{-0.7}{0.2}{1}\eretname{6.47}{-1.05}{l}{airport}
\erCoreAttribute{6.4}{-1.25}{1}{0}{name}{}
\eret{6.2}{-3.8}{8.9}{-2.8}{0.2}{1}\eretname{6.47}{-3.15}{l}{gate}
\erCoreAttribute{6.4}{-3.35}{1}{0}{number}{}

% relationship scheduled_as
\errelname{1.7}{-2}{r}{scheduled}\errelname{1.7}{-2.3}{r}{as}\errelname{2}{-2.65}{l}{of}\errelarm{1.85}{-1.7}{1.85}{-2.25}{0}{0}\errelarm{1.85}{-2.25}{1.85}{-2.8}{1}{0}\eridcomprel{1.75}{1.9500000000000002}{-2.55}\ercrowfoot{1.85}{-2.65}{1.7}{-2.8}{1.85}{-2.8}{2}{-2.8}{0}
% relationship departing_from
\errelname{3.35}{-0.883}{l}{from}\errelname{3.35}{-0.583}{l}{departing}\errelname{6.05}{-0.883}{r}{for}\errelname{6.05}{-0.583}{r}{airport}\errelname{6.05}{-0.283}{r}{departure}\errelarm{3.2}{-1.033}{4.7}{-1.033}{1}{0}\errelarm{4.7}{-1.033}{6.2}{-1.033}{0}{0}\ercrowfoot{3.35}{-1.033}{3.2}{-0.883}{3.2}{-1.033}{3.2}{-1.183}{0}
% relationship going_to
\errelname{3.35}{-1.667}{l}{going}\errelname{3.35}{-1.967}{l}{to}\errelname{6.05}{-1.667}{r}{arrival}\errelname{6.05}{-1.967}{r}{airport}\errelname{6.05}{-2.267}{r}{for}\errelarm{3.2}{-1.366}{4.7}{-1.366}{1}{0}\errelarm{4.7}{-1.366}{6.2}{-1.366}{0}{0}\ercrowfoot{3.35}{-1.367}{3.2}{-1.217}{3.2}{-1.367}{3.2}{-1.517}{0}
% relationship leaving_from
\errelname{3.35}{-3.15}{l}{from}\errelname{3.35}{-2.85}{l}{leaving}\errelname{6.05}{-3.6}{r}{used by}\errelarm{3.2}{-3.3}{4.7}{-3.3}{0}{0}\errelarm{4.7}{-3.3}{6.2}{-3.3}{0}{0}\ercrowfoot{3.35}{-3.3}{3.2}{-3.15}{3.2}{-3.3}{3.2}{-3.45}{0}\eridrefrel{3.45}{-3.1999999999999997}{-3.4}
% relationship having
\errelname{7.7}{-2}{l}{having}\errelname{7.4}{-2.65}{r}{at}\errelarm{7.55}{-1.7}{7.55}{-2.25}{0}{0}\errelarm{7.55}{-2.25}{7.55}{-2.8}{1}{0}\eridcomprel{7.450000000000001}{7.65}{-2.55}\ercrowfoot{7.55}{-2.65}{7.4}{-2.8}{7.55}{-2.8}{7.7}{-2.8}{0}
\end{erdiagram}
}
\end{equation}
Like so many examples this diagram doesn't have the full generality needed to be descriptive of all air transport situations (what about airports with multiple terminals? what about code sharing flights? what about change of gauge?\footnote{You might be intersted in looking up use of this term `change of gauge' in relation to air transport
 --- it describes a way of operating an airline service that falls outside the reality described by my diagram here. The term is borrowed (airquotes) from its use describing a reality that might be faced by a rail transport system. }). Nonetheless this is a useful example and it has some very interesting features and has instances of impactful patterns that recur over and again in modelling situations.

\mynote
The underlining of the flight number attribute in the representation
\raisebox{-0.5cm}{\begin{erdiagram}{1.1}{2.7}

\eret{0}{-1.1}{2.7}{-0.1}{0.2}{1}\eretname{0.27}{-0.45}{l}{airline service}
\erCoreAttribute{0.2}{-0.65}{1}{0}{flight number}{}

\end{erdiagram}
} of the airline service type
on diagram (\ref{boardingGate1}), and the absence of other underlined attributes, is interpreted as meaning that:
\begin{equation}
\label{airlineServiceIdentification}
\text{\parbox{9cm}{\textit{
Each airline service can be uniquely identified or referenced by its flight number.}}}
\end{equation}
Just to be be absolutely clear what this means --- it means no two distinct airline services have the same flight number. Thinking for a moment about the mathematical expression of this --- it means that the flight number attribute, which we know like all attributes can be thought of mathematically as a function, is a function that is total and is injective.\footnote{
If $f: A \longrightarrow B$ is a function then the function is injective iff for all $x,y \in A$,
$f(x) = f(y)$ implies $x=y$. 
}
 \mynote
Whereas the flight number of an airline service is unique the date of departure of a
flight is definitely not -- many flights leave each day. Instead flights are uniquely identified by the combination of their date of departure and the airline service that they are an instance of.
 To document this on the diagram we underline the name attribute and put a bar, like this \barkerEllisJ, through the relationship that contributes to the identification and provides context so that
 the type flight on the diagram appears like this:
 \begin{equation}
 \label{boardingGate0A}
\raisebox{-1.5cm}{\begin{erdiagram}{2.3}{2.9000000000000004}

\eret{0}{-2.3}{2.7}{-1.3}{0.2}{1}\eretname{0.27}{-1.65}{l}{flight}
\erCoreAttribute{0.2}{-1.85}{1}{0}{date of departure}{}
\eret{0}{-0.2}{2.9}{0.3}{0.2}{1}

% relationship scheduled as
\errelname{1.5}{-0.5}{l}{scheduled as}\errelname{1.5}{-1.15}{l}{of}\errelarm{1.35}{-0.2}{1.35}{-0.75}{0}{0}\errelarm{1.35}{-0.75}{1.35}{-1.3}{1}{0}\eridcomprel{1.25}{1.4500000000000002}{-1.05}\ercrowfoot{1.35}{-1.15}{1.2}{-1.3}{1.35}{-1.3}{1.5}{-1.3}{0}
\end{erdiagram}
}
\end{equation}

In summary, this part of the diagram conveys 
\begin{equation}
\label{airlineFlightIdentification}
\text{\parbox{9cm}{\textit{
Each flight can be uniquely identified or referenced by its date of departure in the context of the airline service that it is an instance of.}}}
\end{equation}

\mynote
Because of (\ref{airlineServiceIdentification}), that a airline service is identified or referenced by flight number,  we can fill out the detail in (\ref{airlineFlightIdentification}) and deduce:
\begin{equation}
\label{airlineFlightNetIdentification}
\text{\parbox{9cm}{\textit{
Each flight can be uniquely identified or referenced by its date of departure along with the flight number of the airline service that it is an instance of}}}
\end{equation}
and that is how we were able to, nay, were expected to, interpret the flight number LH2502, 
and date, 14 August 2024, in the original phone message (\ref{LH2502PhoneMessage}) --- as identifying a flight.\footnote{Because flight number got involved in the identification of flights indirectly, 
 mediated by the  relationship of a flight to an airline service, 
 and because multiple level of this sort of thing are common, flight number in such a situation is sometimes said to be cascaded.}

\mynote
Looking at the right hand side of diagram (\ref{boardingGate1}),
the name attribute of an airport is underlined to indicate that
it is an identifying attribute and so, in the absence of other identifying attributes and relationships,
that:
\begin{equation}
\label{airportIdentification}
\text{\parbox{9cm}{\textit{
Each airport can be uniquely identified or referenced by its name.}}}
\end{equation}
and if we inspect the representation of the type \textit{gate} then we see that it says:
\begin{equation}
\label{gateIdentification}
\text{\parbox{9cm}{\textit{
Each gate can be uniquely identified or referenced by its gate number in the context of the airport that it is located at.}}}
\end{equation}

\mynote
Because of (\ref{airportIdentification}), that a airport is identified or referenced by its name,  we can fill out the detail in (\ref{gateIdentification}) and deduce that:
\begin{equation}
\label{gateNetIdentification}
\text{\parbox{9cm}{\textit{
Each gate can be uniquely identified or referenced by its gate number along with the name of the airport it is located at.}}}
\end{equation}
\subsubsection{Discussion --- scope of relationship}
This doesn't yet give a full explanation of the original phone message 
(\ref{LH2502PhoneMessage}) though because
in the message there are two airport names. 
I was expected to know somehow which of the two airports named along with the my gate number L06
was the location of my gate.  Was it L06 at Manchester or was it L06 at Munich? How was I supposed to know? 

\mynote The answer to this question involves knowing something about the arrangement
of the concepts and relationships shown in diagram (\ref{boardingGate1}) 
and this something that I needed to know, 
and all of us would have known,
is not currently represented in the diagram.  
This missing something is an example of a phenomena  
that is massively understudied and unreported. It is something that mathematicians, particularly those versed in category theory, come across all the time but which entity modellers, data modellers and programmers literally have no words for and therefore it remains largely unobserved and wholly unremarked even though it is extremely impactful.

\mynote 
Why didn't the message say gate L06 of Munich airport?
That's because everybody knows and so I am expected to know that my flight will be leaving from a gate at the same airport as the airline service that I have booked is departing from. 
Diagram (\ref{boardingGate1}) doesn't express this fact. 
Later in this book we introduce the concept of relationship scope and 
suggest an annotation that could be added to the diagram to rectify this shortcoming.

It is to the impoverishment of data specifiers everywhere (and surely this includes all programmers) that these phenomena are not in the core common syllabus of computer science. 

 
\section{Core versus Derivative \small and goodness criteria for entity models}
\label{CoreversusDerivative}
\mynote Entity models comprise a set of definitions of types, relationships and attributes. 
and one point of view is that this set should be a minimal set for the job at hand and that none of these definitions should be derivative of the remainder. For example
\begin{itemize}
\item we might model that a bicycle is related to its front wheel and that a bicycle is related to its back wheel but we will not then need to model,  and therefore ought not to model, that the front wheel of a bicycle is related to its back wheel because though there is such a relationship it is implied by or mediated by the other two. It is derivative.
\item temperature in degrees farenheit (a possible attribute) and temperature in degrees centigrade (another possible attribute) cannot both be considered core -- one must be considered core and the other considered derivative,
\item age (a possible attribute) should be considered derivative of date of birth (another attribute, assuming dates are universals),
\item unequivocally, grandparenthood (a candidate many-many binary relationship) should be considered derivative of parenthood (another many-many relationship),
\item likewise, compatriot, as a binary relationship, is derivative of country of origin (another binary relationship),\footnote{This begs an interesting point --- is it ever right to model a symmetric, transitive, reflexive (or anti-reflexive) relationship --- consider this from goodness criteria}
\item father's surname (a candidate attribute of a person) \commentary{find better example?} should be considered derivative to fatherhood (a relationship) and surname (an attribute), 
\item ancestorship is recursively derivative of parenthood. \commentary{\textit{recursively derivative}}
\end{itemize}

If this point of view is adhered to then the entity model documents a conceptual core.

\mynote
Another possible point of view is that derivative attributes, relationships and attributes may be included in an entiy model provided thay are clearly delineated and the method of derivation or inference from the core
is clearly specified.\footnote{Current book has `core' and `constructed'. ER scipt has `constructed'. Seems to me now that terms `core' and `derivative' work better.
I have also had in mind to use the word `mediated'. Toolbuild had `derived' relationships but it now seems to me that `derivative' gets closer to the fact that they are not `core'. }

This second point of view is necessary when entity models are used as data models i.e. as data specifications and leads us to the idea that a model can have data cores that extend the conceptual core. We will come back to this later. That a data core has no derivatives other than those prescribed becomes the basis of Codd's famous normal form criteria for goodness of database specifications. 

\mynote Please note that the prescription that an entity model or its delineated core be should free of redundancy as I have explained it above is a general goodness criteria for entity models. Though it isn't usual for it to be articulated as I have here, it is a  criteria that is tacitly understood --- it is followed by most practioners, most of the time. 

\mynote
Now for an example that breaks everything we have said so far. This model fails tyhe goodness criteria as described so far
\begin{equation}
\label{nearestShop}
\raisebox{-1.5cm}{\begin{erdiagram}{3.4999999999999996}{7.3}

\eret{0.1}{-0.9}{1.9}{-0}{0.2}{1}\eretname{0.28}{-0.35}{l}{person}
\erCoreAttribute{0.3}{-0.55}{1}{0}{name}{}
\eret{2.7}{-0.9}{4.5}{-0}{0.2}{1}\eretname{2.88}{-0.35}{l}{shopType}
\erCoreAttribute{2.9}{-0.55}{1}{0}{name}{}
\eret{5.3}{-0.9}{7.1}{-0}{0.2}{1}\eretname{5.48}{-0.35}{l}{shop}
\erCoreAttribute{5.5}{-0.55}{1}{0}{name}{}
\eret{1.954}{-3.5}{5.246}{-2.6}{0.2}{1}\eretname{2.283}{-2.95}{l}{nearestShop}
\erCoreAttribute{2.154}{-3.15}{1}{1}{distance}{}

% relationship has
\errelname{0.85}{-1.2}{r}{has}\errelname{2.248}{-2.45}{r}{to}\errelarm{1}{-0.899}{1}{-1.099}{1}{0}\errelarm{1}{-1.099}{1}{-1.299}{1}{0}\errelarm{1}{-1.299}{1.723}{-1.649}{1}{0}\errelarm{1.723}{-1.649}{2.447}{-1.999}{1}{0}\errelarm{2.447}{-1.999}{2.447}{-2.3}{1}{0}\errelarm{2.447}{-2.3}{2.447}{-2.599}{1}{0}\eridcomprel{2.3475375}{2.5475375000000002}{-2.3499999999999996}\ercrowfoot{2.448}{-2.45}{2.298}{-2.6}{2.448}{-2.6}{2.598}{-2.6}{0}
% relationship used as
\errelname{3.45}{-1.2}{r}{used as}\errelname{3.45}{-2.45}{r}{ofType}\errelarm{3.6}{-0.899}{3.6}{-1.449}{1}{0}\errelarm{3.6}{-1.449}{3.6}{-2.3}{1}{0}\errelarm{3.6}{-2.3}{3.6}{-2.599}{1}{0}\eridcomprel{3.5}{3.7}{-2.3499999999999996}\ercrowfoot{3.6}{-2.45}{3.45}{-2.6}{3.6}{-2.6}{3.75}{-2.6}{0}
% relationship being
\errelname{6.05}{-1.2}{r}{being}\errelname{4.388}{-2.45}{r}{is}\errelarm{6.2}{-0.899}{6.2}{-1.099}{1}{0}\errelarm{6.2}{-1.099}{6.2}{-1.299}{1}{0}\errelarm{6.2}{-1.299}{5.393}{-1.649}{1}{0}\errelarm{5.393}{-1.649}{4.587}{-1.999}{1}{0}\errelarm{4.587}{-1.999}{4.587}{-2.3}{1}{0}\errelarm{4.587}{-2.3}{4.587}{-2.599}{1}{0}\ercrowfoot{4.588}{-2.45}{4.438}{-2.6}{4.588}{-2.6}{4.738}{-2.6}{0}
% relationship type
\errelname{5.15}{-0.75}{r}{type}\errelarm{5.3}{-0.449}{4.9}{-0.449}{1}{0}\errelarm{4.9}{-0.449}{4.5}{-0.449}{0}{0}\ercrowfoot{5.15}{-0.45}{5.3}{-0.3}{5.3}{-0.45}{5.3}{-0.6}{0}
\end{erdiagram}
}
\end{equation}\commentary{ditto}.

We are in  ajm with this model. I need xxx so that I can specify it to be identifying.
But xxx is derivative. It can be derived as a compose b.
This situation turns out to be fatal unless we allow derivative relationships to be allowed in a model. Defined as derivative so that the core is still core. Maybe a way of drawing what we need in this case is:
\begin{equation}
\label{zanioloExample2}
\raisebox{-1.5cm}{\begin{erdiagram}{4.6}{4}

\eret{0.5}{-1}{3}{-0.1}{0.2}{1}\eretname{0.75}{-0.45}{l}{area}
\erCoreAttribute{0.7}{-0.65}{1}{0}{code}{}
\eret{0.5}{-2.8}{3}{-1.9}{0.2}{1}\eretname{0.75}{-2.25}{l}{place}
\erCoreAttribute{0.7}{-2.45}{1}{0}{name}{}
\eret{0.5}{-4.6}{3}{-3.7}{0.2}{1}\eretname{0.75}{-4.05}{l}{telephone}
\erCoreAttribute{0.7}{-4.25}{1}{0}{number}{}

% relationship 
\errelname{1.9}{-1.3}{l}{}\errelname{1.9}{-1.75}{l}{within}\errelarm{1.75}{-0.999}{1.75}{-1.45}{0}{0}\errelarm{1.75}{-1.45}{1.75}{-1.9}{0}{0}\ercrowfoot{1.75}{-1.75}{1.6}{-1.9}{1.75}{-1.9}{1.9}{-1.9}{0}
% relationship 
\errelname{1.9}{-3.1}{l}{}\errelname{1.9}{-3.55}{l}{at}\errelarm{1.75}{-2.8}{1.75}{-3.25}{0}{0}\errelarm{1.75}{-3.25}{1.75}{-3.699}{0}{0}\ercrowfoot{1.75}{-3.55}{1.6}{-3.7}{1.75}{-3.7}{1.9}{-3.7}{0}
% relationship area
\errelname{3.15}{-4.45}{l}{area}\errelarm{3}{-4.149}{3.3}{-4.149}{0}{0}\errelarm{3.3}{-4.149}{3.6}{-4.149}{0}{0}\errelarm{3.6}{-4.149}{3.65}{-4.149}{0}{0}\errelarm{3.65}{-4.149}{3.7}{-4.149}{0}{0}\errelarm{3.7}{-4.149}{3.7}{-2.349}{0}{0}\errelarm{3.7}{-2.349}{3.7}{-0.549}{0}{0}\errelarm{3.7}{-0.549}{3.65}{-0.549}{0}{0}\errelarm{3.65}{-0.549}{3.6}{-0.549}{0}{0}\errelarm{3.6}{-0.549}{3.3}{-0.549}{0}{0}\errelarm{3.3}{-0.549}{3}{-0.549}{0}{0}\ercrowfoot{3.15}{-4.15}{3}{-4}{3}{-4.15}{3}{-4.3}{0}\eridrefrel{3.25}{-4.05}{-4.249999999999999}
\end{erdiagram}
}
\end{equation}\commentary{redraw this diagram with an exclamation mark next to the derivative relationship}.


\begin{noteforfuture}
Move Goodness of an Entity Model section HERE from tutorial ??
four finger flight example seems a bit strange. 
Becomes less strange if given earlier as an example?
\end{noteforfuture}


\subsection{Zaniolo's Example and the Nearest Shop}
This situation described in this 
model here is a classic from relational database literature. The situation was described by Zaniolo
in 19xx.
\begin{equation}
\label{zanioloExample2}
\raisebox{-1.5cm}{\begin{erdiagram}{4.6}{4}

\eret{0.5}{-1}{3}{-0.1}{0.2}{1}\eretname{0.75}{-0.45}{l}{area}
\erCoreAttribute{0.7}{-0.65}{1}{0}{code}{}
\eret{0.5}{-2.8}{3}{-1.9}{0.2}{1}\eretname{0.75}{-2.25}{l}{place}
\erCoreAttribute{0.7}{-2.45}{1}{0}{name}{}
\eret{0.5}{-4.6}{3}{-3.7}{0.2}{1}\eretname{0.75}{-4.05}{l}{telephone}
\erCoreAttribute{0.7}{-4.25}{1}{0}{number}{}

% relationship 
\errelname{1.9}{-1.3}{l}{}\errelname{1.9}{-1.75}{l}{within}\errelarm{1.75}{-0.999}{1.75}{-1.45}{0}{0}\errelarm{1.75}{-1.45}{1.75}{-1.9}{0}{0}\ercrowfoot{1.75}{-1.75}{1.6}{-1.9}{1.75}{-1.9}{1.9}{-1.9}{0}
% relationship 
\errelname{1.9}{-3.1}{l}{}\errelname{1.9}{-3.55}{l}{at}\errelarm{1.75}{-2.8}{1.75}{-3.25}{0}{0}\errelarm{1.75}{-3.25}{1.75}{-3.699}{0}{0}\ercrowfoot{1.75}{-3.55}{1.6}{-3.7}{1.75}{-3.7}{1.9}{-3.7}{0}
% relationship area
\errelname{3.15}{-4.45}{l}{area}\errelarm{3}{-4.149}{3.3}{-4.149}{0}{0}\errelarm{3.3}{-4.149}{3.6}{-4.149}{0}{0}\errelarm{3.6}{-4.149}{3.65}{-4.149}{0}{0}\errelarm{3.65}{-4.149}{3.7}{-4.149}{0}{0}\errelarm{3.7}{-4.149}{3.7}{-2.349}{0}{0}\errelarm{3.7}{-2.349}{3.7}{-0.549}{0}{0}\errelarm{3.7}{-0.549}{3.65}{-0.549}{0}{0}\errelarm{3.65}{-0.549}{3.6}{-0.549}{0}{0}\errelarm{3.6}{-0.549}{3.3}{-0.549}{0}{0}\errelarm{3.3}{-0.549}{3}{-0.549}{0}{0}\ercrowfoot{3.15}{-4.15}{3}{-4}{3}{-4.15}{3}{-4.3}{0}\eridrefrel{3.25}{-4.05}{-4.249999999999999}
\end{erdiagram}
}
\end{equation}
It has characteristics which are fundamental in shaping relational data theory and we will return to this subject later. \commentary{I have thought of placing an exclamation mark to distinguish derived relationships.}

\mynote
The following example exhibits the same features in a slightly more general way.\footnote{I found the situation modelled here described on Wikipedia in an article discussing TNF and BCNF normal forms. I haven't been able to trace it back any other sources.}
\begin{equation}
\label{nearestShop}
\raisebox{-1.5cm}{\begin{erdiagram}{3.4999999999999996}{7.3}

\eret{0.1}{-0.9}{1.9}{-0}{0.2}{1}\eretname{0.28}{-0.35}{l}{person}
\erCoreAttribute{0.3}{-0.55}{1}{0}{name}{}
\eret{2.7}{-0.9}{4.5}{-0}{0.2}{1}\eretname{2.88}{-0.35}{l}{shopType}
\erCoreAttribute{2.9}{-0.55}{1}{0}{name}{}
\eret{5.3}{-0.9}{7.1}{-0}{0.2}{1}\eretname{5.48}{-0.35}{l}{shop}
\erCoreAttribute{5.5}{-0.55}{1}{0}{name}{}
\eret{1.954}{-3.5}{5.246}{-2.6}{0.2}{1}\eretname{2.283}{-2.95}{l}{nearestShop}
\erCoreAttribute{2.154}{-3.15}{1}{1}{distance}{}

% relationship has
\errelname{0.85}{-1.2}{r}{has}\errelname{2.248}{-2.45}{r}{to}\errelarm{1}{-0.899}{1}{-1.099}{1}{0}\errelarm{1}{-1.099}{1}{-1.299}{1}{0}\errelarm{1}{-1.299}{1.723}{-1.649}{1}{0}\errelarm{1.723}{-1.649}{2.447}{-1.999}{1}{0}\errelarm{2.447}{-1.999}{2.447}{-2.3}{1}{0}\errelarm{2.447}{-2.3}{2.447}{-2.599}{1}{0}\eridcomprel{2.3475375}{2.5475375000000002}{-2.3499999999999996}\ercrowfoot{2.448}{-2.45}{2.298}{-2.6}{2.448}{-2.6}{2.598}{-2.6}{0}
% relationship used as
\errelname{3.45}{-1.2}{r}{used as}\errelname{3.45}{-2.45}{r}{ofType}\errelarm{3.6}{-0.899}{3.6}{-1.449}{1}{0}\errelarm{3.6}{-1.449}{3.6}{-2.3}{1}{0}\errelarm{3.6}{-2.3}{3.6}{-2.599}{1}{0}\eridcomprel{3.5}{3.7}{-2.3499999999999996}\ercrowfoot{3.6}{-2.45}{3.45}{-2.6}{3.6}{-2.6}{3.75}{-2.6}{0}
% relationship being
\errelname{6.05}{-1.2}{r}{being}\errelname{4.388}{-2.45}{r}{is}\errelarm{6.2}{-0.899}{6.2}{-1.099}{1}{0}\errelarm{6.2}{-1.099}{6.2}{-1.299}{1}{0}\errelarm{6.2}{-1.299}{5.393}{-1.649}{1}{0}\errelarm{5.393}{-1.649}{4.587}{-1.999}{1}{0}\errelarm{4.587}{-1.999}{4.587}{-2.3}{1}{0}\errelarm{4.587}{-2.3}{4.587}{-2.599}{1}{0}\ercrowfoot{4.588}{-2.45}{4.438}{-2.6}{4.588}{-2.6}{4.738}{-2.6}{0}
% relationship type
\errelname{5.15}{-0.75}{r}{type}\errelarm{5.3}{-0.449}{4.9}{-0.449}{1}{0}\errelarm{4.9}{-0.449}{4.5}{-0.449}{0}{0}\ercrowfoot{5.15}{-0.45}{5.3}{-0.3}{5.3}{-0.45}{5.3}{-0.6}{0}
\end{erdiagram}
}
\end{equation}


 \section{Communicating Entity Instances}
\label{CommunicatingEntityInstances}

\begin{newtt}
\mynote
Sometimes in a situation being modelled, entities of a type are named and then in the model
we can represent name as an attribute associated with entities of the type.  
Such name attributes are archetypal examples of identifying attributes --- we name things so that we can identify them in context. In such a situation the name of an entity, i.e. the value of the name attribute,  can be used in context to definitively identify the entity.
Quoting this value, i.e. using the name, is one way of referencing an entity of the type from within the context of another entity. This is illustrated in  an example given earlier in which we presented this statement
\begin{equation}
\label{JupiterIoShort}
\mbox{\textit{Jupiter is orbited by Io.}}
\end{equation}
as a representation of an instance of the binary relationship
\begin{gather}
\label{planetMoonSecondInstance}
\raisebox{-1cm}{
\begin{erdiagram}{1.4}{6.9}

\eret{0.1}{-1}{1.85}{-0.4}{0.2}{1}\eretname{0.975}{-0.75}{}{planet}
\eret{5.15}{-1}{6.9}{-0.4}{0.2}{1}\eretname{6.025}{-0.75}{}{moon}

% relationship orbited by
\errelname{2}{-0.55}{l}{orbited by}\errelname{5}{-1}{r}{orbiting}\errelarm{1.85}{-0.7}{3.5}{-0.7}{0}{0}\errelarm{3.5}{-0.7}{5.15}{-0.7}{1}{0}\ercrowfoot{5}{-0.7}{5.15}{-0.55}{5.15}{-0.7}{5.15}{-0.85}{0}
\end{erdiagram}

}
\end{gather}

In the context of statement (\ref{JupiterIoShort}), Jupiter is the name of, and therefore  a reference to, a planet and  Io is the name of, and therefore a reference to, a moon i.e. a natural satellite.  
We understand  (\ref{JupiterIoShort}), therefore, to be shorthand for the 
more pernickerty 
\begin{equation}
\mbox{\textit{The planet with name `Jupiter' is orbited by the moon with name `Io'.}}
\end{equation}

\mynote
We are very much stating the obvious of course but there is a pattern here and it is repeated over and over when instances of relationships are communicated  and the same pattern occurs more formally  and uniformly when instances of relationships are communicated and/or stored in data.

\mynote
Note that within the context of the solar system the name Io  doesn't uniquely identify this moon of Jupiter amongst all named solar system objects since it is also the name of an asteroid that orbits within the asteroid belt.
The names of moons are only unique as names of moons. 
Of course in other contexts, both Jupiter and Io are the names of mythological figures. 

\mynote
 Not just name attributes but identifying attributes more generally provide us with ways of 
 referencing entities and thereby communicating instances of relationships. 
 This is often the most pragmatic reason for requiring types to have associated identifying feature. 

 \mynote
Focusing on just this aspect of the solar system we  draw the model shown in figure
\ref{planetMoonModel}
containing the relationship () and showing name attributes of both moons and planets and
underlining these to showing that each of them is unique and therefore identifying. 

\begin{erboxedFigure} {H}{planetMoonModel}
{
caption blah blah blah
 }
\begin{erdiagram}{2.4999999999999996}{6.9}

\eret{0.1}{-2.1}{1.85}{-1.2}{0.2}{1}\eretname{0.425}{-1.55}{l}{planet}
\erCoreAttribute{0.3}{-1.75}{1}{0}{name}{}
\eret{5.15}{-2.1}{6.9}{-1.2}{0.2}{1}\eretname{5.475}{-1.55}{l}{moon}
\erCoreAttribute{5.35}{-1.75}{1}{0}{name}{}
\eret{0}{-0.2}{6.9}{0.3}{0.2}{1}

% relationship 
\errelname{1.125}{-0.5}{l}{}\errelarm{0.975}{-0.2}{0.975}{-0.7}{1}{0}\errelarm{0.975}{-0.7}{0.975}{-1.2}{0}{0}\ercrowfoot{0.975}{-1.05}{0.825}{-1.2}{0.975}{-1.2}{1.125}{-1.2}{0}
% relationship 
\errelname{6.175}{-0.5}{l}{}\errelarm{6.025}{-0.2}{6.025}{-0.7}{1}{0}\errelarm{6.025}{-0.7}{6.025}{-1.2}{0}{0}\ercrowfoot{6.025}{-1.05}{5.875}{-1.2}{6.025}{-1.2}{6.175}{-1.2}{0}
% relationship orbited by
\errelname{2}{-1.5}{l}{orbited by}\errelname{5}{-1.95}{r}{orbiting}\errelarm{1.85}{-1.65}{3.5}{-1.65}{0}{0}\errelarm{3.5}{-1.65}{5.15}{-1.65}{1}{0}\ercrowfoot{5}{-1.65}{5.15}{-1.5}{5.15}{-1.65}{5.15}{-1.8}{0}
\end{erdiagram}

\end{erboxedFigure}

\mynote
Whereas the names of moons and planets are unique in the context of the solar system then the names of characters in Shakespeare's plays are not. Shakespeare's characters are named uniquely only within the context of individual plays. To document this on an entity relationship diagram we underline the name attribute and put a bar through the relationship that contributes to the identification like this \barkerEllisJ, for example, as we illustrate twice over in figure \ref{playwrightPlayCharacterModel}.


\begin{erboxedFigure} {H}{playwrightPlayCharacterModel}
{
caption blah blah blah
 }
\begin{erdiagram}{2.4999999999999996}{11.15}

\eret{0.1}{-2.1}{1.85}{-1.2}{0.2}{1}\eretname{0.425}{-1.55}{l}{playwright}
\erCoreAttribute{0.3}{-1.75}{1}{0}{name}{}
\eret{4.75}{-2.1}{6.5}{-1.2}{0.2}{1}\eretname{5.075}{-1.55}{l}{play}
\erCoreAttribute{4.95}{-1.75}{1}{0}{title}{}
\eret{9.4}{-2.1}{11.15}{-1.2}{0.2}{1}\eretname{9.725}{-1.55}{l}{character}
\erCoreAttribute{9.6}{-1.75}{1}{0}{name}{}
\eret{0}{-0.2}{11.15}{0.3}{0.2}{1}

% relationship 
\errelname{1.125}{-0.5}{l}{}\errelarm{0.975}{-0.2}{0.975}{-0.7}{1}{0}\errelarm{0.975}{-0.7}{0.975}{-1.2}{0}{0}\ercrowfoot{0.975}{-1.05}{0.825}{-1.2}{0.975}{-1.2}{1.125}{-1.2}{0}
% relationship 
\errelname{5.775}{-0.5}{l}{}\errelarm{5.625}{-0.2}{5.625}{-0.7}{1}{0}\errelarm{5.625}{-0.7}{5.625}{-1.2}{0}{0}\ercrowfoot{5.625}{-1.05}{5.475}{-1.2}{5.625}{-1.2}{5.775}{-1.2}{0}
% relationship 
\errelname{10.425}{-0.5}{l}{}\errelarm{10.27}{-0.2}{10.27}{-0.7}{1}{0}\errelarm{10.27}{-0.7}{10.27}{-1.2}{0}{0}\ercrowfoot{10.275}{-1.05}{10.125}{-1.2}{10.275}{-1.2}{10.425}{-1.2}{0}
% relationship written by
\errelname{4.6}{-1.95}{r}{written by}\errelname{2}{-1.5}{l}{author of}\errelarm{4.75}{-1.65}{3.3}{-1.65}{1}{0}\errelarm{3.3}{-1.65}{1.85}{-1.65}{1}{0}\ercrowfoot{4.6}{-1.65}{4.75}{-1.5}{4.75}{-1.65}{4.75}{-1.8}{0}\eridrefrel{4.5}{-1.5499999999999998}{-1.75}
% relationship within
\errelname{9.25}{-1.95}{r}{within}\errelname{6.65}{-1.5}{l}{has}\errelarm{9.4}{-1.65}{7.95}{-1.65}{1}{0}\errelarm{7.95}{-1.65}{6.5}{-1.65}{1}{0}\ercrowfoot{9.25}{-1.65}{9.4}{-1.5}{9.4}{-1.65}{9.4}{-1.8}{0}\eridrefrel{9.15}{-1.5499999999999998}{-1.75}
\end{erdiagram}

\end{erboxedFigure}

In this diagram we have shown characters being identified by name and by play 
and plays being identified by title and by playwright. Playwrights we have indicated are identified by their names.\commentary{cascaded identifier}

\mynote What this diagram implies is that characters from plays need be identified by three referentials - the name of the character, the title of the play and the name of the playwright.
Example.
\mynote Naively this implies that if we wish to communicate a instance of relationship between two different plays then we will be required to communicate six referentials
 -- three to identify one party in the relationship and three to identify the other party. This is not the case, and so not what we instinctively do, when we have \textit{a priori} knowledge of the relationship being communicated is limited in scope. If we know the relationship is confined to individual playwrights or is internal to individual play or script then clearly we do not have to communicate the playwright and play title twice over. The number of referentials may in such a case collapse to four: name of playwright, title of play and name of character.
Shlair-Lang in a very much underrated paper coined the term \textit{collapsed referentials} for this phenomena in the context of data representation of models represented in entity relationship notation.

\mynote 
So we have, in the context of the plays of William Shakespeare,
\begin{equation}
\label{AntonioLovesSebastian}
\mbox{\textit{The character Antonio in Twelfth Night loves Sebastian.}}
\end{equation} 

You might think that a pedantic elaboration of (\label{AntonioLovesSebastian}) would be:
\begin{equation}
\text{\parbox{9cm}{\textit{The character named Antonio in Shakespeare's play Twelfth Night loves the character named Sebastian in Shakespeare play Twelfth Night.}}}
\end{equation} 
but there is a reason why the shorter
\begin{equation}
\text{\parbox{9cm}{\textit{The character named Antonio in Shakespeare's play Twelfth Night loves the character named Sebastian.}}}
\end{equation} 
is equally unambiguous. This is because the relationship `loves' in this context is a relationship which is internal to the plot of a single individual play. 
It is an intra-play relationship rather than an inter-play relationship.
Reading from left to right we establish that we are making a statement about 
Antonio from Twelfth Night  and so must look to establish Sebastian as a character from the same play i.e. also from Twelfth Night. This is an example of a relationship which is not global in scope (it is local to the play) and therefore describing instances of this fictional loves relationship
doesn't require as many referential attributes as would otherwise be the case if it was global in scope. 

The relationship is conveyed by a total of  4 referential attributes
(Shakespeare, Twelfth Night, Antonio and Sebastian) not the six that might have been required
if the relationship was global in scope so that characters from one play might be described as being in love with a character from a different play.

On the other hand contrast with
\begin{equation}
\mbox{\textit{Romeo from Shakespeare's Romeo and Juliet resembles Pyramus from Ovid's 
Metamorphoses.}}
\end{equation}

in this there is a relationship instance conveyed by six referentials. This is because the resemblance relationship to is not an internal relationship. It is between characters and a character from one play may be perceived as resembling a character from a different play.

In the first example we say that there are collapsed referentials. 

None of these referentials collapse. The models of the two relationships
look the same but behind the scenes they are different. They have different scopes and we come back to talking about these later.

Antonio and Twelfth Night refer
Sebastian and Twelfth Night refer.

Twelfth Night doubles up.

Now two plays are referenced 
 \end{newtt}

\mynote From this section most interest and significance  in how relationship instances are/can be communicated but this question cannot be separated from how entitites are communicted and how entity identitites are communicated.

\mynote I will tend talk abouty how they are commuincated in data but everything that is siad applies to how they are communicated generally including nay especially in natural language. There is something deeply logical and linguistic about this but it is not ususally consided as a part of formal logic.

\mynote Rationale: What is universally true need not be communicated. Only that which is particular and the relation of the particular with the universal need be communicated. \commentary{Need to find a home for this -- I would be sad to lose it.}

\mynote An entity model systematically describes all that can be known of an entity
in terms of core functional relationships with universals and core functional relationships with particulars i.e. in terms of core attributes and core outgoing directional relationships.

\mynote But of all this which  may be known of an entity all we can communicate of an entity is its functional relationships with universals. Other relationships must be communicated indirectly via derivative (i.e. non-core) such relationships with universals. 

\mynote For this reason a systematic way of identifying and referencing each particular type  of entity in an entity model become significant if the model is to be used as any kind of data or message specification. There is some subtlety to identification and referencing as we will explain.

\subsection*{Communicating Relationships}
\mynote To communicate the value of a relationship is to communicate the identity of the relationship and  to identify (i.e. to communicated the identity of) the related entity. 

\mynote The related entity is a particular we are required therefore to have ways of identifying particulars and communicating these identifications.  
Only universals can be communicated so each type of entity is required 
to have one or more identifying attributes. 

\mynote Every entity may be identified by the values of related universals i.e. values that can be attributed to it. These attributions  may be core or derivative.

\mynote Since these relationships with other particulars themself need communication then ultimately every entity can be identified by a set of attributes each one of which may be core or may be derivative.  

\mynote
These attributes of a particular are called identifying attributes and subsequent use of these as attributes in a communication scheme in order to identify an entity as a related entity are known as referential attributes. 

\mynote For a long time now identifying attributes have been called key attributes and referential attributes have been called foreign key attributes. 

\mynote If you think that each outgoing directional relationship requires a distinct set of referential attributes to support its communication then you would be wrong. 
There is a possibility of `collapsed referentials' (a phrase coined in Shlaer and Long) whereby a single referential attribute may support communication of two or more distinct relationships. If you haven't understood this collapsed referentials concept then you have not  fully and completely understood the nature of data. 

\mynote In the presence of collapsed of referentials the number of bits of information required to communicate all the relationships of an entity is less that the sum of the bits required for their individual communication. 

\mynote the resolved set of attributes

\mynote conceptual core versus data core

\begin{oldtt}
\mynote To communicate the value of an attribute is to communicate its identity as an attribute and the universal that is its value. Both the identity of an attribute and its value are universals and communication of universals is a given i.e. does not require further explanation. 
\end{oldtt}

\begin{noteforfuture}
I imagine describing regular communication schemes... and why there can be a choice of schemes for relationships because of the choice of identifying features of target entity. 
\end{noteforfuture}
\begin{noteforfuture}
I imagine describing the communication of outgoing directional relationships of an entity by way of referentials and subsequently explaining that referentials can collapse which yields the term collapsed referentials.
\end{noteforfuture}
 %\iffalse
 \section{Data Modelling}
The term ‘Data Modelling’ we use to cover both database design and also the specification of the structure of messages in quite a broad sense. Entity modelling is most commonly positioned as a precursor to relational data modelling but in fact is equally suitable as method of specifying hierarchical data structures — and it is significant that supporting relational, hierarchical and network models of data from a single specification was the big idea behind the notation as originally proposed by Chen in the seminal paper ‘Unified Model of Data’ in 1976. This is particularly well illustrated in the Barker Entity Modelling book.

ER modelling can be used to specify data at three distinct, increasingly prescriptive levels:

conceptual — entities and relationships only,
logical — entities, relationships and non-referential attributes,
physical — as logical but with message structure completed by the addition of attributes said to be referential and whose purpose is the representation of relationships.
At the third and most prescriptive level, that of a physical model, an entity model is precisely an abstract data model and such models can be classified as relational or hierarchical; most significantly, and in accord with the proposal of Chen, one of each can be generated algorithmically from a well-formulated logical model. Looking back, it seemed that hierarchical structured data took a back seat for a while during the theoretical development and popularisation of the relational model of data but it has made a come back subsequently through widespread adoption of the structured markup language XML. Via appropriate physical models both relational Data Definition Language (DDL) and hierarchical XML schemas (DTD, XSD or the like) can be generated automatically from a single logical ER model. \footnote{We will return to this theme later but it has been said that relational data models generated in this way will naturally tend to be well-formulated data models (i.e. to be in normal form). This is definitely not the case unless account is taken of reference scope constraints as described here in later sections. A logical ER model is agnostic between hierarchical and relational.}

Historically, E.F.Codd's meta theory that was presented as the relational model of data by Codd in 1970,4 emerged fully formed — the meta concepts of table, column and primary key are defined as is that of a foreign key enabling one table to cross reference the rows of another. His is a theory of what data is and this theory came to underpin the majority of corporate databases. Each such database, in accord with Codd's prescriptions, holds a meta-description of its own units of storage — the tables, columns and keys — what their names are and how they fit together to enable navigation through the data; this description is the core of what is described as a relational schema. The development of the relational model of data in the first place was strongly influenced by the predicate calculus representation of formal logic but arguably this meta-mathematics that influenced Codd was overtaken in mathematical imagination by later 20th century meta-mathematics in the form of type theory and category theory; these are more diagrammatic in form and lead not to the relational model of data but to versions of the binary entity relationship model such as is promoted in this book. It is these other meta-mathematical disciplines that influence this presentation here and these lead to significant improvements in relational design methodology. Paradoxically, each such improvement in relational design methodology undermines the pre-eminence enjoyed by the relational model.

Codd has described various tests of goodness of a schema, applicable, it must be remembered, only with cognisance to the possibilities among the data that it is designed to hold i.e. the intended usage. In the first instance three tests were described and successively a schema said to be in 1st normal form, 2nd normal form or 3rd normal form depending on its success in passing the tests. A process for fixing deficient schemas is described as normalisation of the schema. Normalisation is therefore a method for converting or transforming one relational schema into another that is deemed more suitable for a purpose at hand.

Subsequently, the relations of Codd's model are more abstractly presented, as either entities or as n-ary relationships, in Chen's entity-relationship model of data; in the approach of Chen there is emphasis on a diagrammatic representation of the model. Chen describes a method for constructing a relational schema (in the sense of Codd) from an entity-relationship schema (ER-schema). He states that normalisation of the relational schema might be required after construction from an ER-schema — though why this might be is not explained. We will explain in a later section the fundamental reason why this is so but also why it need not be so.

After Chen's 1976 paper, coming into and through the 1980's, came the development, concurrently, of Computer Aided Software Engineering (CASE) tools, including Meta-CASE tools, and semi-formalised and, in some instances, standardised official methodologies and notations supporting structured systems analysis and development. Universally in the methodologies from this time the terms entity and relationship introduced in Chen's paper were retained within a logical modelling phase and Chen's transformation step into relational database design, inclusive of a normalisation step, is likewise retained. Though the terms and the overall shape of the process is retained the concepts behind these terms are subtly shifted. Most noticeably relationships are now binary relationships and at an early stage in these methodologies many-many relationships are eliminated in favour of many-one relationships. At this point there has been a conceptual volte face for a many-one binary relationship, implementation considerations aside, is a thinly disguised and abstracted pointer between records of a file, such as in a VSAM file system, or a link between records in the network database model, and it can be conceptualised, abstractly, as a function between sets of like-typed entities which has lead some authors to describe a functional model of data. The entity-relationship diagrams of these software analysis methods and the accompanying CASE tools that emerged in the 80's bear more resemblance to notation that preceded the work of Codd and Chen such as Bachman's data structure diagrams from 1973 than to the diagrams of Chen. Among the many, and as summarised in the book of Rosemary Rock-Evans, there are three variants of binary entity relationship diagram that stand out, those found, respectively, in SSADM/Barker-Ellis (now adopted by Oracle), in Clive Finkelstein and James Martin's Information Engineering, and in IDEF.

In some instances, software methodologies and supporting CASE tools introduced an intermediate step between the ER model and the relational model naming the intermediary model the physical design model to contrast with the logically descriptive model that precedes it in the software development life-cycle. By a significant methodological improvement described in later sections we follow this approach but are able to eliminate the normalisation step.
\begin{figure}[H]
\small
\begin{center}
\setlength{\tabcolsep}{2pt}
\begin{tabular}{ p{1.4cm}  p{2.2cm}  p{1.5cm} p{1.5cm} p{1.5cm} p{1.5cm}  p{1.25cm}}
\raisebox{-0.8cm}{\parbox{1.4cm}{logical er~model}}& \textit{Chen~transform (automatic)} $\xrightarrow{\hspace*{1.75cm}}$ &
\raisebox{-0.8cm}{\parbox{1.4cm}{physical er model}}& \textit{manually normalise} $\xrightarrow{\hspace*{1.5cm}}$ &
\raisebox{-0.8cm}{\parbox{1.4cm}{physical er model}}& \textit{code generate} $\xrightarrow{\hspace*{1.5cm}}$ &  \raisebox{-0.8cm}{\parbox{1.25cm}{relational schema}} 
\end{tabular}
\end{center}
\caption{Traditional methodology for relational data design includes a manual normalisation step.}
\end{figure}

It is noteworthy that in these methodologies the normalisation step is present in order to achieve the goodness of the physical data model as prescribed by Codd in his normal form prescriptions. In the methodology described in this book we achieve the goodness of the final physical design, i.e. Codd's normal forms, by enabling suitable and pertinent real world conditions to be expressed at the logical level and supporting an automatic transformation to physical models that take advantage of these conditions so as to be able to meet the goodness prescriptions.\footnote{We also describe the features of the logical model that determine whether in addition to Codd's third normal form the goodness condition known as the Boyce-Codd normal form is also be met by the physical model}


 \section{Type Inheritance}
\label{TypeInheritance}




\subsection*{To do}
\mynote 
Can I preface this with something more practical to get a better blend? 

Can I talk about programmking languages? 

terminology subtype and supertype Barker?

non-standard part of the SSADM method.

Use in car example from SSADM

Inheritance is used in the example in the file systemn structured entity modelling example which is germane to history.

\subsection*{Types of Entity Specific and General}
Entity modelling has the questions ‘What is?’ and ‘What can be said of it?’ at its heart. We can define it to be the process of defining what can be predicated of entities and it necessarily seems to embrace, in passing, many questions of philosophy, specifically of Ontology, the branch of metaphysics dealing with the nature of being.

In ancient times, Aristotle had used the Greek word ousia (being) to describe the subject, to which predicates are ascribed.1 Traditionally in translation the Greek ousia has been rendered as substance, a term with broad connotations whereas the Latin word ens which is, as is also the Greek ousia, the present participle of the verb to be, yielded our modern English word entity defined as:
\begin{erquote}
entity n. things existence, as opposed to its qualities or relations; thing that has real existence. (Concise Oxford Dictionary)
\end{erquote}
or as:
\begin{erquote}
entity n. (the quality of having) a single separate and independent existence. (Longman Dictionary of Contemporary English).
\end{erquote}
The term entity, as we use it here in the term ‘entity modelling’, was introduced into information science by Chen in 1966. An entity for our purposes is simply something about which it is possible to have knowledge and which can be counted. Thus we can have fictional entities such as characters from a book or entities whose state of existence we can debate such as the number zero or the transcendental number $\pi$.

In Aristotelian ontology, as outlined in Categories, there are ten genera of being. The first genus of being, ousia, is of two types: primary and secondary. Primary ousia are individual things — they are our entities, secondary ousia are classes of things, or the genera and species of things
 — for us these are entity types.\footnote{The other genera of being
  (quantity, quality, relation, place, time, position, state, action, affection)
  are properties inherent in the primary ousia.}

Though Metaphysics has ancient origins, there is an intersection of concerns via data modelling between it and the most practical and modern disciplines of software engineering and the programming of computer based systems. The overlap exists because metaphysics has as its subject matter what is most general about things in general — not just physical properties of physical things — and the software development discipline starts with the representation of the very same generalities. So whereas most computer programs have as their subjects, everyday if not concrete and physical things, things such as people, accounts, orders, contracts, airline bookings, and so on; other computer programs have as their subjects the structures of molecules, languages, stellar processes or programs themselves, or such as mathematical propositions, relationships in general, not particular, types of things as distinct from the things themselves, and so on. 

\subsection*{Generic types Aristotle's Categories}
In the statement ‘man and the apes are descendent from a common ancestor’— it is clear that singular ‘man’ is being used to denote a specific type which in biology, and as far back as Aristotle's Categories, is termed a species, and that plural ‘apes’ is denoting a more general class of thing which following Aristotle we might call a genus or in biology are said to be ‘a classification of a higher rank’ and which classifications includes genera, families, orders and kingdoms. As a reader one does not need know the individual species of apes but can infer that a multiplicity of species is implied — simply by noting the use of the plural form. In entity modelling both such specific and general types of thing are represented; these are species and genera not just in the strict sense used in biological nomenclature but in the more general sense that we find in translations of Aristotle's Categories, a species is a specific type such as the type ‘man’ and a particular individual man is said to be predicated by that species. A genus, on the other hand, is a more general type such as ‘animal’ or ‘ape’.
\begin{erquote}
For if any one should render an account of what a primary ousia is, he would render a more instructive account by stating the species than by stating the genus. Thus, he would give a more instructive account of an individual man by stating that he was a man than by stating that he was an animal, for the former description is peculiar to the individual in a greater degree, while the latter is too general.
\end{erquote}
In Aristotle's description both the individual man and the species man are predicated by the genus animal, and so to, by example, the species ox is predicated by the genus animal. From what we have already said this we will represent thus:
\begin{center}
\begin{erdiagram}{1.45}{3.6666}

\eret{0}{-1.45}{3.667}{-0}{0.2}{1}\eretname{0.116}{-0.35}{l}{Animal}
\eret{0.25}{-1.2}{1.583}{-0.6}{0.2}{0}\eretname{0.917}{-0.95}{}{man}
\eret{2.083}{-1.2}{3.417}{-0.6}{0.2}{0}\eretname{2.75}{-0.95}{}{ox}

\end{erdiagram}

\end{center}



In entity modelling the term entity type is used for both genera and species —ox, man and animal are all entity types as shown in the fragment above.

In some ways, such diagrams as these show similarity to Venn diagrams. This one can be so interpreted as showing the set of men and the set of oxen included in the set of animals. However, to the entity modeller, there is no set of all men, nor of all oxen, nor of all animals for the question is not ‘what exists?’ but ‘what types of things exist and what can be said of them?’. The diagram can be interpreted as saying ‘what can be said of animal’ can be said of ‘man’ and of ‘ox’ also.

Aristotle says it like this:
\begin{erquote}
Whenever one thing is predicated of another as a subject, all things said of what is predicated will be said of the subject also. For example, man is predicated of the individual man, and animal of man; so animal will be predicated of the individual man also — for the individual man is both a man and an animal.
\end{erquote}
Subsequently, after the time of Aristotle, used in biological nomenclature the term genus adopted a more specific meaning, in contradistinction to use of the term species for the lowest rank in the system — individuals of the same species varying in minor ways and able to interbreed — the term genus became used for a group of related species, the genera strictly forming just the second rank in a multi-ranked system.

\subsection*{Types of Particles in Physics}
Another example is given by the types of particle discussed in Feynman's Lecture Notes on Physics:
\begin{erquote}
Particles which interfere with a positive sign are called Bose particles and those which interfere with a negative sign are called Fermi particles. The Bose particles are the photon, the mesons, and the graviton. The Fermi particles are the electron, the muon, the neutrinos, the nucleons, and the baryons.
\end{erquote}
As in the usage ‘man and the apes’ singular and plural terms are used in this passage to distinguish between specific types of things and more general classes of things. The author's respective use of singular and plural terms inform as to which are the fundamental types of particle, the species, and which the related families, the genera. In diagramming this in an entity model, instead of retaining the plural form, so to speak, we may capitalise the genera and so arrive this diagram:
\begin{center}
\begin{erdiagram}{4.699999999999999}{6.199999999999999}

\eret{0}{-4.7}{6.2}{-0}{0.2}{1}\eretname{0.376}{-0.35}{l}{Particle}
\eret{0.25}{-2.05}{5.95}{-0.6}{0.2}{0}\eretname{0.366}{-0.95}{l}{Bose Particle}
\eret{0.5}{-1.8}{1.9}{-1.2}{0.2}{1}\eretname{1.2}{-1.55}{}{photon}
\eret{2.4}{-1.8}{3.8}{-1.2}{0.2}{1}\eretname{3.1}{-1.55}{}{Meson}
\eret{4.3}{-1.8}{5.7}{-1.2}{0.2}{1}\eretname{5}{-1.55}{}{graviton}
\eret{0.25}{-4.45}{5.95}{-2.35}{0.2}{0}\eretname{0.418}{-2.7}{l}{Fermi Particle}
\eret{2.4}{-3.3}{3.8}{-2.7}{0.2}{1}\eretname{3.1}{-3.05}{}{electron}
\eret{4.3}{-3.3}{5.7}{-2.7}{0.2}{1}\eretname{5}{-3.05}{}{muon}
\eret{0.5}{-4.2}{1.9}{-3.6}{0.2}{1}\eretname{1.2}{-3.95}{}{Nucleon}
\eret{2.4}{-4.2}{3.8}{-3.6}{0.2}{1}\eretname{3.1}{-3.95}{}{Neutrino}
\eret{4.3}{-4.2}{5.7}{-3.6}{0.2}{1}\eretname{5}{-3.95}{}{Baryon}

\end{erdiagram}

\end{center}

\subsection*{Types of Words in Linguistics}
In linguistics the entity type ‘word’ is generally represented as a generalisation of more specific types often referred to as word classes. These include noun, verb, adjective and so on as shown in figure \ref{wordclassesnested} and these are illustrated in table \ref{wordclassestable}.

The basis for the recognition of word classes in linguistics is the observation that certain words can be freely interchanged in sentences without altering the acceptability of the sentence grammatically. For example we can apply substitutions replacing various words of an example sentence such as I received beautiful flowers for my birthday by randomly chosen other words and some will deliver equally grammatical sentences and some will not. For example we can replace ‘beautiful’ by ‘ugly’ or ‘red’ or ‘expensive’ without losing sentence structure whereas we cannot replace by ‘the’ or ‘very’ or ‘at’. So the words ‘beautiful’,‘ugly’, ‘red’, ‘expensive’ are of the same class adjective and the words ‘the’, ‘very’, ‘at’ are not of this class — in fact the word ‘the’ is classed as a determiner, the word ‘very’ as a degree word and the word ‘at’ as a preposition.
\begin{table}
\begin{tabular}{ l l l}
class&	abbreviation &	examples \\
noun&	N	&athelete, house, race, record, stream, water  \\
pronoun&	&Pro	I, you, he, she, we, they  \\
determiner&	&Det	a, the  \\
verb&	    & V	arrive, run, set  \\
auxiliary&	& Aux	had, will  \\
preposition&	&Prep	at, by, from, in, to  \\
prepositional specifier &	Pspec	& close, right, straight, three seconds  \\
adjective	   & A	    &fierce, long, new, red, right, rosy, silk, young  \\
general adverb &	Adv	&abruptly, brightly, clearly, quickly  \\
degree adverb&	Deg	    &more, most, quite, rather, so, too, very  \\
\end{tabular}
\caption{Types of word and abbreviations used.}
\label{wordclassestable}
\end{table}


It is common to use abbreviations to identify the word classes; as to how many there are then it has to be said that they cannot be enumerated unequivocally; linguist C.C Fries defined nineteen types as the nineteen parts of speech of English in 1952 (he also distinguished content bearing types of word: nouns, verbs, adjectives and adverbs from function types such as prepositions, determiners and coordinating conjunctions). For our purposes here we will use the classes and the abbreviations shown in table \ref{wordclassestable}.

\begin{figure}
\begin{center}
\begin{erdiagram}{2.55}{7.5999}

\eret{0}{-2.55}{7.6}{-0}{0.2}{1}\eretname{0.204}{-0.35}{l}{Word}
\eret{0.25}{-1.2}{1.583}{-0.6}{0.2}{0}\eretname{0.917}{-0.95}{}{noun}
\eret{2.083}{-1.2}{3.417}{-0.6}{0.2}{0}\eretname{2.75}{-0.95}{}{verb}
\eret{3.917}{-1.2}{5.517}{-0.6}{0.2}{0}\eretname{4.717}{-0.95}{}{adjective}
\eret{6.017}{-1.2}{7.35}{-0.6}{0.2}{0}\eretname{6.683}{-0.95}{}{adverb}
\eret{0.25}{-2.3}{1.583}{-1.5}{0.2}{0}\eretname{0.917}{-1.85}{}{proper}\eretname{0.917}{-2.15}{}{name}
\eret{2.083}{-2.1}{3.417}{-1.5}{0.2}{0}\eretname{2.75}{-1.85}{}{pronoun}
\eret{3.917}{-2.1}{5.517}{-1.5}{0.2}{0}\eretname{4.717}{-1.85}{}{determiner}
\eret{6.017}{-2.1}{7.35}{-1.5}{0.2}{0}\eretname{6.683}{-1.85}{}{degree}

\end{erdiagram}

\end{center}
\caption{Word Classes shown using the nested box notation.}
\label{wordclassesnested}
 \end{figure}
On Naming of Types
In biological nomenclature the species name alone is not always enough to uniquely identify a type of thing less it be used alongside of the genus name — this system of naming, using the genus name alongside the species name, being called binomial and having been introduced by Linnaeus. Resonant and antecedent to this can be seen in the very first section of Aristotle's Categories which is the about the proper delineation of the of types of things:

Things are said to be named ‘equivocally’ when, though they have a common name, the definition corresponding with the name differs for each. Thus, a real man and a figure in a picture can both lay claim to the name ‘animal’; yet these are equivocally so named, for, though they have a common name, the definition corresponding with the name differs for each. For should any one define in what sense each is an animal, his definition in the one case will be appropriate to that case only.
The appropriate diagram to fit Aristotle's text would seem to be this:
\begin{center}
\begin{erdiagram}{2.3}{5.1666}

\eret{0}{-2.3}{2.333}{-0}{0.2}{1}\eretname{0.184}{-0.35}{l}{Figure}
\eret{0.25}{-2.05}{2.083}{-0.6}{0.2}{0}\eretname{0.366}{-0.95}{l}{Animal}
\eret{0.5}{-1.8}{1.833}{-1.2}{0.2}{1}\eretname{1.167}{-1.55}{}{man}
\eret{2.833}{-2.3}{5.167}{-0}{0.2}{1}\eretname{3.017}{-0.35}{l}{Real Thing}
\eret{3.083}{-2.05}{4.917}{-0.6}{0.2}{0}\eretname{3.199}{-0.95}{l}{Animal }
\eret{3.333}{-1.8}{4.667}{-1.2}{0.2}{1}\eretname{4}{-1.55}{}{man }

\end{erdiagram}

\end{center}

With such a configuration of types you might suppose a trinomial notation is required — or... simply a diagram and the ability to point at it. And this is the point of such diagrams or at least a very good part of it. 
 \section{Structured Entity Modelling}
\label{StructuredEntityModelling}
Chen's paper introduced the idea of entities being dependent on binary relationships with others for both their identification and their existence:
\begin{erquote}
Theoretically, any kind of relationship may be used to identify entities. For simplicity, we shall restrict ourselves to the use of only one kind of relationship: the binary relationships with 1:n mapping in which the existence of the n entities on one side of the relationship depends on the existence of one entity on the other side of the relationship. For example, one employee may have n ( = 0, 1, 2, . . .) dependants, and the existence of the dependants depends on the existence of the corresponding employee. This method of identification of entities by relationships with other entities can be applied recursively until the entities which can be identified by their own attribute values are reached. For example, the primary key of a department in a company may consist of the department number and the primary key of the division, which in turn consists of the division number and the name of the company.
\end{erquote}

Following PCTE8 we use the term composition relationship for Chen's binary relationships with 1:n mapping in which the existence of the n entities on one side ... depends on the existence of one entity on the other side and we use the term reference relationship for binary relationships which are neither composition relationships nor their inverses. We shall also describe the inverses of composition relationships as being dependency relationships. \commentary{contextual relationships?}Earlier than this a similar distinction had been made by the designers of the CAIS9 specification but in which the two kinds of relationship were distinguished as primary and secondary — their rationale for the distinction was as follows:
\begin{erquote}
[Entities] and relationships may form a general graph or bowl of spaghetti. However, this raises various practical problems of deletion and garbage collection, long term naming, and unconnected sub-graphs. CAIS therefore designates certain relationships as primary (and all others as secondary) and requires that all [Entities] and primary relationships in the database form a single tree structure.
\end{erquote}
This distinction between composition and reference made by both CAIS and then PCTE served the goal of modelling computer file systems within a database framework, see figure \ref{filesystem2} for example.
\commentary{Need mention UML also}

\begin{figure}[H]
\begin{center}
\begin{erdiagram}{5.300000000000001}{7.335949999999999}

\eret{2.8}{-1.6}{4.133}{-1}{0.2}{1}\eretname{3.467}{-1.35}{}{drive}
\eret{0.697}{-4.2}{6.736}{-2.8}{0.2}{1}\eretname{0.809}{-3.15}{l}{entry}
\eret{0.947}{-3.95}{2.281}{-3.35}{0.2}{0}\eretname{1.614}{-3.7}{}{file}
\eret{2.781}{-3.95}{4.114}{-3.35}{0.2}{0}\eretname{3.447}{-3.7}{}{folder}
\eret{4.614}{-3.95}{5.986}{-3.35}{0.2}{0}\eretname{5.3}{-3.7}{}{shortcut}
\eret{0}{-0.2}{7.336}{0.3}{0.2}{1}

% relationship 
\errelname{3.617}{-0.5}{l}{}\errelarm{3.466}{-0.2}{3.466}{-0.6}{0}{0}\errelarm{3.466}{-0.6}{3.466}{-1}{0}{0}\ercrowfoot{3.467}{-0.85}{3.317}{-1}{3.467}{-1}{3.617}{-1}{0}
% relationship root
\errelname{3.617}{-1.9}{l}{root}\errelarm{3.466}{-1.6}{3.456}{-2.1}{0}{0}\errelarm{3.456}{-2.1}{3.447}{-2.975}{0}{0}\errelarm{3.447}{-2.975}{3.447}{-3.35}{0}{0}
% relationship 
\errelname{3.464}{-4.25}{l}{}\errelarm{3.313}{-3.95}{3.313}{-4.45}{0}{0}\errelarm{3.313}{-4.45}{3.313}{-4.95}{0}{0}\errelarm{3.313}{-4.95}{1.663}{-4.95}{0}{0}\errelarm{1.663}{-4.95}{0.013}{-4.95}{0}{0}\errelarm{0.013}{-4.95}{0.013}{-3.575}{0}{0}\errelarm{0.013}{-3.575}{0.013}{-2.2}{0}{0}\errelarm{0.013}{-2.2}{1.11}{-2.2}{0}{0}\errelarm{1.11}{-2.2}{2.207}{-2.2}{0}{0}\errelarm{2.207}{-2.2}{2.207}{-2.5}{0}{0}\errelarm{2.207}{-2.5}{2.207}{-2.8}{0}{0}\ercrowfoot{2.207}{-2.65}{2.057}{-2.8}{2.207}{-2.8}{2.357}{-2.8}{0}
% relationship to
\errelname{6.136}{-3.95}{l}{to}\errelarm{5.985}{-3.65}{6.585}{-3.65}{0}{0}\errelarm{6.585}{-3.65}{7.185}{-3.65}{0}{0}\errelarm{7.185}{-3.65}{7.235}{-3.65}{0}{0}\errelarm{7.235}{-3.65}{7.285}{-3.65}{0}{0}\errelarm{7.285}{-3.65}{7.285}{-3.435}{0}{0}\errelarm{7.285}{-3.435}{7.285}{-3.22}{0}{0}\errelarm{7.285}{-3.22}{7.21}{-3.22}{0}{0}\errelarm{7.21}{-3.22}{7.135}{-3.22}{0}{0}\errelarm{7.135}{-3.22}{6.935}{-3.22}{0}{0}\errelarm{6.935}{-3.22}{6.735}{-3.22}{0}{0}\ercrowfoot{6.136}{-3.65}{5.986}{-3.5}{5.986}{-3.65}{5.986}{-3.8}{0}
\end{erdiagram}

\end{center}
\caption{An ER model of folder system modelling the hierarchical structure as a recursive composition relationship and shortcuts as reference relationships.}
\label{filesystem2}
\end{figure}

In this presentation we shall not assume that all composition relationships are identifying nor, vice-versa, that only composition relationships may be identifying. To depict ER-schemas we use a variant of the Barker-Ellis notation. \commentary{concept of identifying not yet defined.}
Figure \ref{entityRelationalMetaModel1} is a meta-model of this notation — it is an ER schema describing ER schemas.

In cases where we wish to distinguish composition relationships from reference relationships then we draw the diagram top down: an anonymous root entity type (the ‘absolute’) is introduced at the top of the diagram, relationships leaving the lower edges of boxes are composition relationships and they always meet the top edge of the box representing the dependent type, reference relationships meet boxes from one side or the other. We note that there is a structural resemblance to diagrams drawn by Bachman. To summarise, for composition relationships the crows feet point down; at this point the notation converges with that of SSADM for which one explanation says: ‘there are no dead crows’. Our diagrams also have reference relationships and for these the crows feet are pointing sideways (the crows, presumably, at rest). The entity types which have the least numbers of instances occur at the top of our diagrams whereas in what seems an odd choice they occur to the bottom right in the diagrams style promoted in Barker's Entity Modelling book.

\begin{figure}[H]
\begin{center}
\begin{erdiagram}{4.15}{8.492750000000001}

\eret{0.4}{-0.9}{8.193}{-0}{0.2}{1}\ertext{1.179}{-0.35}{l}{entity type}
\erattr{0.6}{-0.55}{1}{1}{name}
\eret{0.548}{-2.5}{2.2}{-1.9}{0.2}{1}\ertext{1.374}{-2.25}{}{identifier}
\eret{3.7}{-4.15}{8.045}{-1.9}{0.2}{1}\ertext{3.88}{-2.25}{l}{RELATIONSHIP}
\eret{3.95}{-3.4}{5.462}{-2.5}{0.2}{0}\ertext{4.101}{-2.85}{l}{attribute}
\erattr{4.15}{-3.05}{1}{1}{name}
\eret{5.862}{-3.9}{7.795}{-2.7}{0.2}{0}\ertext{6.206}{-3.05}{l}{relationship}
\erattr{6.062}{-3.25}{1}{1}{name}
\erattr{6.062}{-3.55}{1}{1}{optional?}

% relationship identified by
\ertext{1.224}{-1.2}{r}{identified by}\errelarm{1.374}{-0.9}{1.374}{-1.4}{1}{0}\errelarm{1.374}{-1.4}{1.374}{-1.9}{1}{0}\errelseq{1.434}{-1.45}{1.024}{-1.51}{1.724}{-1.57}{1.314}{-1.63}\ercrowfoot{1.374}{-1.75}{1.224}{-1.9}{1.374}{-1.9}{1.524}{-1.9}{0}
% relationship attributes
\ertext{4.556}{-1.2}{r}{attributes}\errelarm{4.706}{-0.9}{4.706}{-1.7}{0}{0}\errelarm{4.706}{-1.7}{4.706}{-2.5}{1}{0}\ercrowfoot{4.706}{-2.35}{4.556}{-2.5}{4.706}{-2.5}{4.856}{-2.5}{0}
% relationship outgoing
\ertext{6.292}{-1.2}{r}{outgoing}\ertext{6.292}{-2.55}{r}{source}\errelarm{6.442}{-0.9}{6.442}{-1.8}{0}{0}\errelarm{6.442}{-1.8}{6.442}{-2.7}{1}{0}\ercrowfoot{6.442}{-2.55}{6.292}{-2.7}{6.442}{-2.7}{6.592}{-2.7}{0}
% relationship incoming
\ertext{7.365}{-1.2}{l}{incoming}\ertext{7.365}{-2.55}{l}{destination}\errelarm{7.215}{-0.9}{7.215}{-1.8}{0}{0}\errelarm{7.215}{-1.8}{7.215}{-2.7}{1}{0}\ercrowfoot{7.215}{-2.55}{7.065}{-2.7}{7.215}{-2.7}{7.365}{-2.7}{0}
% relationship is
\ertext{2.35}{-2.05}{l}{is}\ertext{3.55}{-2.05}{r}{is identifier}\errelarm{2.2}{-2.2}{2.95}{-2.2}{1}{0}\errelarm{2.95}{-2.2}{3.7}{-2.2}{0}{0}
\end{erdiagram}

\end{center}
\caption{The logical ER meta-model. A simple version of the logical ER model of a logical ER model.}
\label{entityRelationalMetaModel1}
\end{figure}

\begin{noteforfuture}
above diagram questionable:
\begin{enumerate}
\item I haven't figured out where identifying features will be defined.
\item I need a version of this diagram that shows the absolute.
\item reintroduce earlier meta model diagram? One without attributes?
\item relationship becomes directional relationship
\item replace cardinality by optionality plus many-valuedness ?
\end{enumerate}
\end{noteforfuture}







 \section{The Absolute}
\label{TheAbsolute}
Within structured entioty modelling.
Now  I want to discuss that the representation within an entity model of the whole of everything should in fact be referred to as `the absolute'.
\mynote
We use the term ‘the Absolute’, which has a varied history in metaphysical writing, to mean the whole of everything and, equally, the context for everything. In giving two meanings we are following a rich tradition — for definitions abound. The term was central to much of the philosophy of G.W.F Hegel — in Shorter Logic section 87, among a number of other definitions, we find:
\begin{erquote}
...the Absolute is the Nought...The Nothing which the Buddhists make the universal principle, as well as the final aim and goal of everything, is the same abstraction.
\end{erquote}
whereas in Phenomonology of Spirit in section 20 we find:
\begin{erquote}
The True is the whole.
\end{erquote}
and in section 75 of the same we find:
\begin{erquote}
...the Absolute alone is true, truth alone is absolute.
\end{erquote}
\mynote
Some sense can be made of these statements when we consider from a logical perpective and from the point of view of information theory. We can illustrate by considering the seemingly puzzling fact that in some programming languages (such as ML) there is an in-built concept with the name ‘Unit’ and described as a singleton type, whereas in some other earlier programming languages (Algol68, C) the name given to the very same concept has been ‘void’. The apparently striking disparity in naming, the one versus the zero, has come about as a result of the concept being named on the one hand on the basis of the number of things predicated by the type (i.e the number of things we can say are of that type), which is exactly one, and on the other hand based on the number of bits (binary digits) of information carried in communicating a member of the type, which is precisely zero. We get the name ‘unit’ from one point of view and the name ‘void’ from the other. In this way we can say of the Absolute that it is the whole of everything. From it being the whole we can say there is only one of the type. From there being only one of the type we can say that it's information content is zero. If you present to me the absolute you present me with nothing. Like the true it can be assumed in all contexts for it carries nothing new with it. This is the logic of the absolute.

\mynote The whole of a modelling situation can be considered a single composite and this is both ‘the ultimate whole’ when considered as a composite and equally ‘the absolute’ when considered as a context. If what we have said above can be summarised as saying that there is a duality between context and composition then in this duality ‘the whole’ and ‘the absolute’ are duals: they are the same logical entities.

\mynote Another useage that we have, is to speak of concepts that are absolute. What we usually mean by saying of a concept that it is abolute is that it does not vary — that it is not relative to the context in which it appears. As we seek to construct models of usage and thereby a conceptual model we can expect to find a dichotomy of relative and absolute terms: some terms, such as father, daughter, length, colour, that vary in so much as they reference different items in different contexts; and terms, such as the earth, the pole star, the London Times that are absolute or constant in what they reference. Whereas relative terms are conceptualised as relationships or as quantitative or adjectival attributes of subject entities, absolute terms cannot be so interpreted unless we posit the existence of a singular entity and then interpret the absolute terms as relational to or as attributive to this singular entity. In this way the matter is finessed for we can say of a relative term such as father that it varies as the person varies — different subject persons having different fathers — and we can say of an absolute term pole star that it varies depending on the singular entity — the Absolute — which is to say that it does not vary at all.
Can I talk about programming languages? 
terminology subtype and supertype Barker?
non-standard part of the SSADM method.
Use in car example from SSADM

\section{The Distinction Between Composition and Reference}
\label{DistinctionBetweenCompositionAndReference}

The entity modelling notation in one form or another is a part of the core syllabus in the information sciences. Invariably, though, no distinction is made between composition and reference.\footnote{
The one exception to this would be in teaching of the UML notation wherein there is a further classification of composition relationships resulting in three subclasses of the core relationship concept rather than two as here.} This is a weakness currently; students are not being provided with the best conceptual tools for database design and without these tools there remains a database normalisation step which is not properly explained and has the feel of a dark art. The concept of relationship scope is the key missing concept and it is introduced in the sections which follow. Before this however we revisit the distinction between composition and reference and ask ‘is this a real distinction?’.

Consider these two superficially similar types of relationship:
\begin{itemize}
\item the relationship between a play and the characters within the play,
\item the relationship between a play and performances of that play.
\end{itemize}
The first of these would generally be classified as a composition relationship for we can say that a play is in part composed of all the characters within it, whereas the second would generally be classified as a reference relationship for we would not say that a play is in part composed of all of it's performances. For this reason an entity model describing just these three entity types, play, performance and character, contains both vertical composition relationships and an orthogonal reference relationship, as shown below in figure \ref{modelPerformancePlayCharacter}.

\begin{erboxedFigure}{H}{modelPerformancePlayCharacter}{Composition and reference}
\begin{erbulletedModel}{modelPerformancePlayCharacter}{3}{4.5}{5}
\item a play is composed of one or more characters
\item a performance is a performance of exactly one play
\end{erbulletedModel}
\end{erboxedFigure}

Since it is a distinction rarely made many readers may be sceptical of whether there is a credible distinction between composition and reference; in the circumstances such doubts are reasonable and much of this chapter will be devoted to examples and implications of the distinction. So far much weight has rested on appeal to a sense of what constitutes a part and of what parts something can reasonably be said to be composed. There is another way of thinking about it though. We said in the introduction that entity modelling was concerned with what could be known of an entity; now, another way of asking what can be known of an entity is to ask what description can be given of an entity or what of an entity can be communicated.

If focusing on parts and composition doesn't clarify the distinction between composition and reference or, for that matter, to convince of the credibility of the distinction, then another ways of clarifying relies on a focus on full description or communication and this in turns leads to the idea of copying the full description of an entity - for to communicate an entity is to copy it in some way from source to destination.

Therefore we ask what would be communicated in a full description of a play and we answer that surely it would include a full description of each of the characters? The play-characters relationship therefore passes the full description test and is classified as a composition relationship. The play-performances relationship on the other hand fails this same test - it is not necessary to describe every performance of a play in order to fully describe the play - it fails the full description test.

The matter will not rest however - there are many relationships which can be modelled either way and then models containing them are subtly different and are appropriate in different circumstances.

\erboxedFigPicture{filesystem2}{H}{
The folder example is an excellent example of a composition relationship. I cannot delete a folder on my computer without deleting all the folders and files contained within it (of course I can move the contained items first and then delete the parent folder). Shortcuts are different - I can delete a shortcut to a file or folder without deleting the file or folder. Therefore the relationship between a shortcut and that which it is a short cut to is a reference relationship.}
\commentary{This diagram needs work done on it.... I would expect as a minimum  an exclusion arc.}


\section{Barker Airline Example}
\label{BarkerAirlineExample}

\commentary{Not sure how this develops and where it should be.}
\mynote Work on this example from page 3-13 of Barker.
\commentary{flight needs data  of departure (time of departure not required)}
\erboxedFigPicture{boardingPass2}{H}
{This example is based on an example developed in the Barker book. I have simplified in some areas.}


\begin{noteforfuture}
In this figure \ref{boardingPass2} example aircraft registration number is an absent referentials from
boarding pass. Referentials can collapse but they can also be totally absented. Maybe call them indirect referentials. 
\end{noteforfuture}

\commentary{airline route versus flight reminds me of Saussure discussion re: trains}



\section*{Notes}
\mynote I  have one source file per section.
\mynote I  use \verb'\commentary' to add markups in the margin.

\mynote Beef up the `Scope' section futher. 
Drop the SSADM book customer,payment,allocated payment,invoice,booking,vehicle and vehicle category in earlier as well. Then have the two subdiagrams in this scope section 3and comment on the scope of these.
Reproduce the entire such ssadm example. Put in as a second example ERD in the current example erd section.
\begin{noteforfuture}
shlaerlang use the term \textit{collapsed referentials}.
an I use this as a section title? That would be good.
\end{noteforfuture}

\mynote rationale -- when it comes to one order for the introduction of terms rather than another -- there are a few considerations which likely conflict
\begin{itemize}
	\item  present conceptual modelling before data modelling
	\item  present relational data modelling before structured entity modelling
	\item rationalise structured entity modelling from point of view of hierarchical data specification. 
	\item  present identifying features as part of conceptual modelling because
	it is applicable to conceptual modelling though it really comes into its own
	later applied in data modelling
	because there is then more value to it.
	\item present discussions about the comminiucation of relationship instances before 
		getting on to data modelling
	\item   present goodness criteria as part of conceptual modelling
	or present the core and its derivatives as part of conceptual modelling then goodness criteria as part of data model where value is stated.
	\item similarly present scope as part of conceptual modelling because it is part of understanding concepts then show its value in data modelling? 
	\item in goodness section in conceptual modelling bit discuss absence of referential attributes to entities in scope of model and out of scope of model.
	\item and dont model a referential attribute in preference to a relationship.
	\item in data modelling section reintroduce referential attributes. 
	\item somewhere have a section on entity modelling without diagrams. Can get almost most of the most significant advantages of entity modelling without using diagrams. This might be a  introducing xml and ERScript without hgaving to worry about diagrams.
	\item example of modelling a boolean international flight in which an airport is located.
\end{itemize}

\mynote somewhere --- exclusion arcs
\mynote embed a few more examples Check out SSADM book page 213. 
\mynote meta model example given in section 9 needs to have other meta-models before it and needs the very idea to be explained.

\begin{notebox}
Important point. When I omit attributes from a diagram and depict just the entity types and relationships then I omit bars from the relationships unless the set of identifying features of the entity type consists entirely of relationships. Then the bars are shown.
Thus in figure \ref{chenManufacturingCo..diagram} I show bars only on relationships from intersection entities. This is entirely and utterly satisfying and frees me up from some doubt. My doubt has been how could I show identifying relationships when they are only identifying in the presence of 
mutual identifying attributes. Relief! 
\end{notebox}

\begin{noteforfuture}
One of the goals of this book is to get a deper understanding of data and in turn to be able to be more surefooted designing and navigating data systems --- their data, their user interfaces and their code.
\end{noteforfuture}

\begin{noteforfuture}
Entity models explain the particular in terms of the universal.
\end{noteforfuture}

\begin{noteforfuture}
The referecing entities section. I would like an example with two or three identifying attributes but --- cannot think of one. The ones I have given earlier are not very convincing.
\end{noteforfuture}

\begin{noteforfuture}
 Sally Shlaer and Stephen J. Mellor use the term referential attribute
in their book \textit{OBJECT-ORIENTED SYSTEMS ANALYSIS --- Modeling the World in Data}.
\end{noteforfuture}



\begin{noteforfuture}
\begin{itemize}
\item referential attributes represent data not concept.
\item relationships model concept and context not data. 
\end{itemize}
\end{noteforfuture}

\begin{noteforfuture}
When communicating a plurality of instances of relationships is less that the sum of the parts.
communication( R1 + R2) less than or equal communication (X) + communication of (Y).=
\end{noteforfuture}

\begin{noteforfuture}
 The Distinction between Composition and Reference \\
In structured entity modelling the distinction is made between composition relationships and reference relationships and this does such and such. Is this a real distinction? ETC ETC as per.

Diagrams expressing Scopes.\\
Document scope by drawing this diagram and asserting that it commutes. Or document by stating the equivalence of two paths. 
\end{noteforfuture}

\begin{noteforfuture}
 The airline flight example. Could model terminals that `support international flights' and also
 model characteristic of a flight whether it is international or not. Instead of having such boolean attribute of flight could instead model country in which a airport is located. 
\end{noteforfuture}

\begin{noteforfuture}
when we model data then we impose structure and that in some ways limits what we can say.
free text has no such limitation.
data may have elements of free text within it. Thus a description of a play may have individual lines of text attributed to charcaters in a structured way. To store the script of a play as a database. a novel can be stored as a blob or it can be broken down into chapeters one blob per chapter. Some element of structure is now recogised. none the less 'the data' is not very datalike rather that is is text like. an accounts databse will be much more datlike but may well have text elements embedded within it. The dream of structuralism is that all be understood as pure structure. A sentence can be parsed and its parse tree is more datalike and less text like.
\end{noteforfuture}

\begin{noteforfuture}
For discussion of universals in  the context of mereology see A.J.Cotnoir in my data/database literature review. In particular
\begin{erquote}
Universals are typically said to be ‘wholly located wherever they are instantiated’.
\end{erquote}
\end{noteforfuture}

\begin{noteforfuture}
18th Sept 2024. I am thinking about what comes first the section on data or the section of structured
entity modelling. 
Currently I have the data section followed by type inheritance followed by structured entity modelling. 

However I have just been trying to fraw a diagram showing Chen's ternary \verb'SUPP-PROJ-PART' relation as a Barker-style diagram. I first drew it as a network with three composition relationships meeting at the intersection entity. Then I scrapped that diagram and drew it agin with one composition and two references. In fact when I look at how I have represented intersection entities in Barker stykle rendering of Chen's manufacturing example (figure \ref{chenManufacturingCo..diagram}) then indeed I see that I it is a network diagram in that the three intersection entitites have two incoming composition relationships. 

Having been exercised by this i.e. wondering what choices to make I wonder whether the way to proceed is put structured entity modelling before data modelling.

Describe structured entity modelling with multiple incoming compositions if it seems right.

Describe data modelling for hierarchical data as eliminating multiple incoming compositions.

Then as an example give a \verb'SUPP-PROJ-PART' as an example of choising an elimination.

Giving a second version of the chen manufacturing model in figure \ref{chenManufacturingCo..diagram}
which is hierarchical.

During all this a second thought occurs to me. Since there is a choice of how network structure is eliminated could support either way of representing the data in a hierarchic schema such as xml.
 Need a diagram but cant include jpg see photos folder \verb'whiteboard_dual_elimination.jpg'.

A further point.
Could keep the order Data, Structured Entity Modelling as I curren;ty have it and add a sectiuon on hierarchical modelling. Just don't like that as a section title as hierarchical modelling is for messages, xml and program structure but is a turn off if it suggests it is restricted to hierarchical databases.

\end{noteforfuture}

\subsection{Notes regarding referencing entities and first cut}
\section{Notes on Identity and Reference}

\begin{noteforfuture}
I might be able to get to the equivalence of paths by examining
a noun phrase involving seven referentials.

dramatic role is referenced by reference to production reference to play
\end{noteforfuture}

 

 \subsubsection {How Referential Attributes are Accumulated}
 \mynote
 In the examples that we have given above, values of attributes are quoted to reference entities and these values we are calling referentials.  
\mynote
The attributes whose values are quoted to reference a particular entity are
an aggregation of
\begin{itemize}
\item the
identifying attributes that are defined in the entity model for that type of entity aggregated with 
\item
the attributes we need to quote to reference all those entitites reached from the given entity by an identifying relationship, if there are any.
\end{itemize}





\mynote
Because of the second bullet point the description above is recusive. 
To make sense as a definition 
we need to be sure that there are not any loops in the network of identifying relationships.  

\mynote
Without recursion: 
The attributes whose values are quoted to reference a particular entity are 
\begin{itemize}
\item the identifying attributes that are defined in the entity model 
for that type of entity
aggregated with 
\item the identifying attributes of all those entities reachable by paths of identifying relationships.
\end{itemize}  

\mynote 
We need to refine these descriptions to take account of 
equivalent paths of identifying relationships for if
entities of a type can be reached by distict but comparable paths that are known to be equivalent then there is no point in 
referencing entitites of the type twice over. 
Aggregation of attributes is then combined with elimination of known duplicates.
Another way of thinking and speaking about elimination of duplicates is to think and speak of known duplicate referentials 
merging, or collapsing, resulting in referentials that fulfil multiple roles. This we illustrated in figure xxx. 

\mynote 
Naive Algorithm for Accumulation of Identifying Attributes:
 Another way of looking at the matter is based simply on the observation that the referential attributes involved in identifying an entity  type $a$, without
 context,  correspond to the identifying attributes of the entity type $a$ union the set of identifying attributes reaced 
 via paths 
 where  $\langle r_1,...r_n, p \rangle$ where $n \geq 1$ and where
$\overset{r_1}{a \rightpitchfork \hspace{-0.35em} -  \cdot} \overset{r_2}{\rightpitchfork \hspace{-0.35em} -} \cdot ... \overset{r_n}{\rightpitchfork \hspace{-0.35em} -} b$ is a path of single-valued relationships, where 
$r_i$ is identifying for each $i > 1$ and where $p$ is an identifying attribute of the destination entity type $b$ of the
relationship $r_n$. 

This observation is followed up in Barker's book and in the SSADM method as a first cut definition of relational table stuctutre and is incorparted in Chen's transformation from an entity model to a relational design.

In the eventuality that two of these paths are comparable then whether or not the identifying attributes of the destination type need to be double counted, so to speak,
depends on whether or not the comparabe paths are equivalent.
\mynote
This naive approach is flawed just because it doesdn't take accout os euqivalent paths that might exist within the set of entity types and relationships and which ought to be specified in the model as equivalent paths.

\mynote 
Adjusted Algorithm for Accumulation of Identifying Attributes:
\commentary{referential accumulation algorithm}
Describe the algorithm.
Next need to refine this algorithm to take account of commutative relationships. 
To do use define paths of identifying relationships.

\mynote
Mathematical Formulation:
Accumulate into the referential attributes the identifying attributes of the subject entity type
with all the referential attributes of all entity types reached by traversing an identifying relationship.
$$RefAttr_A=IdAtrt_A \cup \bigcup_{I:A \morph B}RefAttr_B$$

\mynote
This example XML is relational in structure. A different way of structuring this message
is this
\begin{verbatim}
<PstlAdr>
  <Building no=120 postcode="60690-0834">
    <Street name="South LaSalle Street">
      <Town name="Chicago">
        <CountrySubdivision name="IL">
            <Country name="US"/>
        </CountrySubdivision>
      </Town>
    </Street>
  </Building>
</PstlAdr>
\end{verbatim}

\begin{noteforfuture}
Note that an address seems to be an entity type with 5 attributes ---
number, street name, name of city, name of state, name of country.
Addresses are communicated as if these quantities were attributes. That these attributes are referentials. In deep structure they belong to other entitites. In surface structure they appear as referentials implementing references to other entities.
\end{noteforfuture}

\begin{noteforfuture}
I am drawing Barker figure 3-18 which has a missing but relevant relationship from flight to aircraft.
I have drawn the missing relationship on my copy of the book.

Barker's bullet points that describe what is printed on the boarding pass and why.

I feel like saying that by the same logic that he a
states  in the final bullet point
\textit{as the unique identifier of the flight also includes the relationship to the airline route, we also need the flight number.}
leads to an additional but missing bullet point
\textit{as the unique identifier to the seat also includes the relationship to the aircraft, we also need the registration number (of thew aircraft)}.

The intererst is two fold.
\begin{enumerate}
		\item What is printed on the boarding pass matches the primary key.
		The primary key doesn't need the aircraft registartion number because of the relationship from an flight to an aircraft and because if the fact that the diagram so formed commutes. This is a great example. Everybody knows that the seat the b.p. referes to is on the aircraft that is assigned to fly the flight. Great example. Also see meta data remark below.
		\item Second point of interest and possibly a bit wacky.
		How does the relationship to an aircraft get implmented. In the relationsal database it is stored. But how about on the ground. Well we get the riught pllane because it is parked at the gate and because we check in and are ushered to the correct gate. Therefore there is a relationship between a flight and a gate which is for sakes of argument implemented in information andthere is a relationship between a agte and an aircraft at a point of time (it is parked there). At a point of time the relationship between flight and aircraft is implmented physically in the real world by a combination of physical occurences of relationships. When you localise there is unique aircraft and that must be what is related to. Once we have folloowed the instruction "go to gate 47" there is only one gate (locally) and once the aircraft has parked there is only one aircraft. Not sure this can be carried through and v. abstrcat but maybe keep an eye open for this. 
	\end{enumerate}
Regarding bullet 1 above,
	What 'everybody knows' is meta-data (in fact it is universal) and it is prefereble to represent meta-data once than to represent particular instance (the registration number of the aircraft) many times over. 


\end{noteforfuture}

Quote from Bertrand Russell --- The Problems of Philosopy Annotated from my copy and then found as a quote online.
\begin{quote}
“The world of universals, therefore, may also be described as the world of being. The world of being is unchangeable, rigid, exact, delightful to the mathematician, the logician, the builder of metaphysical systems, and all who love perfection more than life. The world of existence is fleeting, vague, without sharp boundaries, without any clear plan or arrangement, but it contains all thoughts and feelings, all the data of sense, and all physical objects, everything that can do either good or harm, everything that makes any difference to the value of life and the world. According to our temperaments, we shall prefer the contemplation of the one or of the other. The one we do not prefer will probably seem to us a pale shadow of the one we prefer, and hardly worthy to be regarded as in any sense real. But the truth is that both have the same claim on our impartial attention, both are real, and both are important to the metaphysician.”
\end{quote}

\mynote
Bearing in mind that I have said that entities, and relationship instances, are particulars
and that the values of attributes, referentials included, are universals then, arguably,
 what we have been examining in this section, 
 is the mechanics behind the following slogan:
\begin{erquote}
We think particulars, but we speak universals.
\end{erquote}
\begin{noteforfuture}
\mynote
A follow up slogan for the coming section on data is
\begin{erquote}
Universals are the stuff of data though its meaning is in the 
particulars --- entities and relationships --- that the universals describe. 
\end{erquote}
\end{noteforfuture}

\subsection{old material from communicating relationships}

\begin{notebox}[Note on Mathematical Terminology]
Remember that we are saying that two paths $x$ and $y$ are \textit{comparable} when they
both have the same source and destination entity types i.e. when there are entity types 
$A$ and $B$ so that $x$ and $y$ are both paths of relationships from $A$ to $B$.
Comparable paths are sometimes equivalent in which case we write
$$ x = y $$
or sometimes either of the two paths might be \textit{subordinate} to the other in which case we case write
$x \leq y$
or 
$y \leq x$, as appropriate.
In some mathematical writing and in similar situations if $x$ and $y$ are distinct and if neither  
$x \leq y$ nor $y \leq x$ then it is said that $x$ and $y$ are incomparable and this is written as
$$x\ ||\ y$$.

In our context we cannot use the term \textit{incomparable} in this way.
We shall use the term \textit{unrelated} instead.

In summary we say that path $x$ is unrelated to path $y$ if the paths are not equivalent paths
and neither $x \leq y$ nor $y \leq x$.

In this case we have paths $d1$ and $r \circ d2$ and we are sayting that if message xxx is an appropriate format then it must be the case that
\begin{equation}
r \circ d2\ ||\ d1
\end{equation}
\end{notebox}

\subsubsection{new diagrams}
\begin{tabular}{ l p{4cm}}
collapsed referential &
\begin{equation}
\label{shlaerLang-DeptStudentProfessor..collapsedReferential..diagram}
\scalebox{0.9}{\begin{erdiagram}{3.9}{7.7516375}

\eret{3}{-1}{5.01}{-0.1}{0.2}{1}\eretname{3.201}{-0.45}{l}{department}
\erCoreAttribute{3.2}{-0.65}{1}{0}{name}{}
\eret{1.27}{-3.4}{2.74}{-2.5}{0.2}{1}\eretname{1.417}{-2.85}{l}{student}
\erCoreAttribute{1.47}{-3.05}{1}{0}{name}{}
\eret{5.24}{-3.4}{7.002}{-2.5}{0.2}{1}\eretname{5.416}{-2.85}{l}{professor}
\erCoreAttribute{5.44}{-3.05}{1}{0}{name}{}

% relationship providing specialization for
\errelname{3.52}{-1.3}{r}{providing specialization for}\errelname{1.855}{-2.35}{r}{majoring in}\errelarm{3.669}{-0.999}{3.669}{-1.2}{0}{0}\errelarm{3.669}{-1.2}{3.669}{-1.4}{0}{0}\errelarm{3.669}{-1.4}{2.837}{-1.737}{0}{0}\errelarm{2.837}{-1.737}{2.004}{-2.075}{1}{0}\errelarm{2.004}{-2.075}{2.004}{-2.287}{1}{0}\errelarm{2.004}{-2.287}{2.004}{-2.5}{1}{0}\errelid{2.837}{-1.828}{}{d1}\eridcomprel{1.9047625}{2.1047625}{-2.25}\ercrowfoot{2.005}{-2.35}{1.855}{-2.5}{2.005}{-2.5}{2.155}{-2.5}{0}
% relationship staffed by
\errelname{4.49}{-1.3}{l}{staffed by}\errelname{6.271}{-2.35}{l}{on staff of}\errelarm{4.339}{-0.999}{4.339}{-1.2}{0}{0}\errelarm{4.339}{-1.2}{4.339}{-1.4}{0}{0}\errelarm{4.339}{-1.4}{5.23}{-1.737}{0}{0}\errelarm{5.23}{-1.737}{6.12}{-2.075}{1}{0}\errelarm{6.12}{-2.075}{6.12}{-2.287}{1}{0}\errelarm{6.12}{-2.287}{6.12}{-2.5}{1}{0}\errelid{5.23}{-1.828}{}{d2}\eridcomprel{6.0207625}{6.220762499999999}{-2.25}\ercrowfoot{6.121}{-2.35}{5.971}{-2.5}{6.121}{-2.5}{6.271}{-2.5}{0}
% relationship advised_by
\errelname{2.89}{-3.25}{l}{advised}
\errelname{2.89}{-3.55}{l}{by}
\errelname{2.89}{-3.85}{l}{(inter-department)} %hand edit
%\errelname{2.89}{-3.85}{l}{($r \circ d2 \leq d1$)} %hand edit
\errelname{5.09}{-2.8}{r}{advising}\errelarm{2.739}{-2.95}{2.989}{-2.95}{0}{0}\errelarm{2.989}{-2.95}{3.989}{-2.95}{0}{0}\errelarm{3.989}{-2.95}{4.989}{-2.95}{0}{0}\errelarm{4.989}{-2.95}{5.239}{-2.95}{0}{0}\errelid{3.99}{-3.04}{}{r}\ercrowfoot{2.89}{-2.95}{2.74}{-2.8}{2.74}{-2.95}{2.74}{-3.1}{0}
\end{erdiagram}
}
\end{equation}
\\
nonCollapsedReferential &
\begin{equation}
\label{shlaerLang-DeptStudentProfessor..nonCollapsedReferential..diagram}
\scalebox{0.9}{\begin{erdiagram}{3.9}{7.7516375}

\eret{3}{-1}{5.01}{-0.1}{0.2}{1}\eretname{3.201}{-0.45}{l}{department}
\erCoreAttribute{3.2}{-0.65}{1}{0}{name}{}
\eret{1.27}{-3.4}{2.74}{-2.5}{0.2}{1}\eretname{1.417}{-2.85}{l}{student}
\erCoreAttribute{1.47}{-3.05}{1}{0}{name}{}
\eret{5.24}{-3.4}{7.002}{-2.5}{0.2}{1}\eretname{5.416}{-2.85}{l}{professor}
\erCoreAttribute{5.44}{-3.05}{1}{0}{name}{}

% relationship providing specialization for
\errelname{3.52}{-1.3}{r}{providing specialization for}\errelname{1.855}{-2.35}{r}{majoring in}\errelarm{3.669}{-0.999}{3.669}{-1.2}{0}{0}\errelarm{3.669}{-1.2}{3.669}{-1.4}{0}{0}\errelarm{3.669}{-1.4}{2.837}{-1.737}{0}{0}\errelarm{2.837}{-1.737}{2.004}{-2.075}{1}{0}\errelarm{2.004}{-2.075}{2.004}{-2.287}{1}{0}\errelarm{2.004}{-2.287}{2.004}{-2.5}{1}{0}\errelid{2.837}{-1.828}{}{d1}\eridcomprel{1.9047625}{2.1047625}{-2.25}\ercrowfoot{2.005}{-2.35}{1.855}{-2.5}{2.005}{-2.5}{2.155}{-2.5}{0}
% relationship staffed by
\errelname{4.49}{-1.3}{l}{staffed by}\errelname{6.271}{-2.35}{l}{on staff of}\errelarm{4.339}{-0.999}{4.339}{-1.2}{0}{0}\errelarm{4.339}{-1.2}{4.339}{-1.4}{0}{0}\errelarm{4.339}{-1.4}{5.23}{-1.737}{0}{0}\errelarm{5.23}{-1.737}{6.12}{-2.075}{1}{0}\errelarm{6.12}{-2.075}{6.12}{-2.287}{1}{0}\errelarm{6.12}{-2.287}{6.12}{-2.5}{1}{0}\errelid{5.23}{-1.828}{}{d2}\eridcomprel{6.0207625}{6.220762499999999}{-2.25}\ercrowfoot{6.121}{-2.35}{5.971}{-2.5}{6.121}{-2.5}{6.271}{-2.5}{0}
% relationship advised_by
\errelname{2.89}{-3.25}{l}{advised}
\errelname{2.89}{-3.55}{l}{by}
\errelname{2.89}{-3.85}{l}{(intra-department)} %hand edit
\errelname{5.09}{-2.8}{r}{advising}\errelarm{2.739}{-2.95}{2.989}{-2.95}{0}{0}\errelarm{2.989}{-2.95}{3.989}{-2.95}{0}{0}\errelarm{3.989}{-2.95}{4.989}{-2.95}{0}{0}\errelarm{4.989}{-2.95}{5.239}{-2.95}{0}{0}\errelid{3.99}{-3.04}{}{r}\ercrowfoot{2.89}{-2.95}{2.74}{-2.8}{2.74}{-2.95}{2.74}{-3.1}{0}
\end{erdiagram}
}
\end{equation}
\\
absent Referential &
\begin{equation}
\label{shlaerLang-DeptStudentProfessor..absentReferential..diagram}
\scalebox{0.9}{\begin{erdiagram}{3.9}{7.7516375}

\eret{3}{-1}{5.01}{-0.1}{0.2}{1}\eretname{3.201}{-0.45}{l}{department}
\erCoreAttribute{3.2}{-0.65}{1}{0}{name}{}
\eret{1.27}{-3.4}{2.74}{-2.5}{0.2}{1}\eretname{1.417}{-2.85}{l}{student}
\erCoreAttribute{1.47}{-3.05}{1}{0}{name}{}
\eret{5.24}{-3.4}{7.002}{-2.5}{0.2}{1}\eretname{5.416}{-2.85}{l}{professor}
\erCoreAttribute{5.44}{-3.05}{1}{0}{name}{}

% relationship providing specialization for
\errelname{3.52}{-1.3}{r}{providing specialization for}\errelname{1.855}{-2.35}{r}{majoring in}\errelarm{3.669}{-0.999}{3.669}{-1.2}{0}{0}\errelarm{3.669}{-1.2}{3.669}{-1.4}{0}{0}\errelarm{3.669}{-1.4}{2.837}{-1.825}{0}{0}\errelarm{2.837}{-1.825}{2.004}{-2.25}{1}{0}\errelarm{2.004}{-2.25}{2.004}{-2.375}{1}{0}\errelarm{2.004}{-2.375}{2.004}{-2.5}{1}{0}\errelid{2.837}{-1.915}{}{d1}\ercrowfoot{2.005}{-2.35}{1.855}{-2.5}{2.005}{-2.5}{2.155}{-2.5}{0}
% relationship staffed by
\errelname{4.49}{-1.3}{l}{staffed by}\errelname{6.271}{-2.35}{l}{on staff of}\errelarm{4.339}{-0.999}{4.339}{-1.2}{0}{0}\errelarm{4.339}{-1.2}{4.339}{-1.4}{0}{0}\errelarm{4.339}{-1.4}{5.23}{-1.737}{0}{0}\errelarm{5.23}{-1.737}{6.12}{-2.075}{1}{0}\errelarm{6.12}{-2.075}{6.12}{-2.287}{1}{0}\errelarm{6.12}{-2.287}{6.12}{-2.5}{1}{0}\errelid{5.23}{-1.828}{}{d2}\eridcomprel{6.0207625}{6.220762499999999}{-2.25}\ercrowfoot{6.121}{-2.35}{5.971}{-2.5}{6.121}{-2.5}{6.271}{-2.5}{0}
% relationship advised_by
\errelname{2.89}{-3.25}{l}{advised}
\errelname{2.89}{-3.55}{l}{by}
\errelname{2.89}{-3.85}{l}{(intra-department)}
\errelname{5.09}{-2.8}{r}{advising}\errelarm{2.739}{-2.95}{2.989}{-2.95}{0}{0}\errelarm{2.989}{-2.95}{3.989}{-2.95}{0}{0}\errelarm{3.989}{-2.95}{4.989}{-2.95}{0}{0}\errelarm{4.989}{-2.95}{5.239}{-2.95}{0}{0}\errelid{3.99}{-3.04}{}{r}\ercrowfoot{2.89}{-2.95}{2.74}{-2.8}{2.74}{-2.95}{2.74}{-3.1}{0}
\end{erdiagram}
}
\end{equation}
\\
non absent Referential &
\begin{equation}
\label{shlaerLang-DeptStudentProfessor..nonAbsentReferential..diagram}
\scalebox{0.9}{\begin{erdiagram}{3.9}{7.7516375}

\eret{3}{-1}{5.01}{-0.1}{0.2}{1}\eretname{3.201}{-0.45}{l}{department}
\erCoreAttribute{3.2}{-0.65}{1}{0}{name}{}
\eret{1.27}{-3.4}{2.74}{-2.5}{0.2}{1}\eretname{1.417}{-2.85}{l}{student}
\erCoreAttribute{1.47}{-3.05}{1}{0}{name}{}
\eret{5.24}{-3.4}{7.002}{-2.5}{0.2}{1}\eretname{5.416}{-2.85}{l}{professor}
\erCoreAttribute{5.44}{-3.05}{1}{0}{name}{}

% relationship providing specialization for
\errelname{3.52}{-1.3}{r}{providing specialization for}\errelname{1.855}{-2.35}{r}{majoring in}\errelarm{3.669}{-0.999}{3.669}{-1.2}{0}{0}\errelarm{3.669}{-1.2}{3.669}{-1.4}{0}{0}\errelarm{3.669}{-1.4}{2.837}{-1.825}{0}{0}\errelarm{2.837}{-1.825}{2.004}{-2.25}{1}{0}\errelarm{2.004}{-2.25}{2.004}{-2.375}{1}{0}\errelarm{2.004}{-2.375}{2.004}{-2.5}{1}{0}\errelid{2.837}{-1.915}{}{d1}\ercrowfoot{2.005}{-2.35}{1.855}{-2.5}{2.005}{-2.5}{2.155}{-2.5}{0}
% relationship staffed by
\errelname{4.49}{-1.3}{l}{staffed by}\errelname{6.271}{-2.35}{l}{on staff of}\errelarm{4.339}{-0.999}{4.339}{-1.2}{0}{0}\errelarm{4.339}{-1.2}{4.339}{-1.4}{0}{0}\errelarm{4.339}{-1.4}{5.23}{-1.737}{0}{0}\errelarm{5.23}{-1.737}{6.12}{-2.075}{1}{0}\errelarm{6.12}{-2.075}{6.12}{-2.287}{1}{0}\errelarm{6.12}{-2.287}{6.12}{-2.5}{1}{0}\errelid{5.23}{-1.828}{}{d2}\eridcomprel{6.0207625}{6.220762499999999}{-2.25}\ercrowfoot{6.121}{-2.35}{5.971}{-2.5}{6.121}{-2.5}{6.271}{-2.5}{0}
% relationship advised_by
\errelname{2.89}{-3.25}{l}{$advised$}
\errelname{2.89}{-3.55}{l}{by}
\errelname{2.89}{-3.85}{l}{intra-department}
%\errelname{2.89}{-3.85}{l}{$r \circ d2 \leq d1$}
\errelname{5.09}{-2.8}{r}{advising}\errelarm{2.739}{-2.95}{2.989}{-2.95}{0}{0}\errelarm{2.989}{-2.95}{3.989}{-2.95}{0}{0}\errelarm{3.989}{-2.95}{4.989}{-2.95}{0}{0}\errelarm{4.989}{-2.95}{5.239}{-2.95}{0}{0}\errelid{3.99}{-3.04}{}{r}\ercrowfoot{2.89}{-2.95}{2.74}{-2.8}{2.74}{-2.95}{2.74}{-3.1}{0}
\end{erdiagram}
}
\end{equation}
\end{tabular}



Tabular data.

\setlength{\tabcolsep}{3pt} % Default value: 6pt
{\footnotesize
\begin{tabular}{|l | l|l| l| l| l|}
Touchstone&Shakespeare&As You Like It&April-May 1975&Oxford Playhouse&Bob Hoskins
\end{tabular}
}

\subsubsection{Regarding Air Travel Example}
Like so many examples this diagram doesn't have the full generality needed to be descriptive of all air transport situations (what about airports with multiple terminals? what about code sharing flights? what about change of gauge?\footnote{You might be intersted in looking up use of this term `change of gauge' in relation to air transport
 --- it describes a way of operating an airline service that falls outside the reality described by my diagram here. The term is borrowed (airquotes) from its use describing a reality that might be faced by a rail transport system. }). Nonetheless this is a useful example and it has some very interesting features and has instances of impactful patterns that recur over and again in modelling situations.

%\fi
\end{document}
