

\section{Attributes}
\label{Attributes}
\mynote I have said earlier that in entity modelling, the term attribute is adopted as a specific term meaning a relationship between a particular on the one hand and a universal on the other. Actually it would be more accurate to say that the term attribute is used for a \textit{functional} relationship between a type of particulars and a type of universals,
 i.e. between a type all of whose instances are particulars a type all of whose instances are universals.
When this functional relationship is evaluated at a particular entity then the result is said to be the value of the attribute. For this reason the types which I describe as \textit{types all of whose instances are universals} are called \textit{value sets} by Chen in his  1976 paper.\footnote{
In Chen 1076 we find \textit{An attribute can be formally defined as a function which maps...into a value set} and also \textit{Since an attribute is defined as a function, it maps an entity in an entity set to a single value in a value set}.} \commentary{what does Barker call these types?}

\mynote
 It is a curious thing but though attributes relate entities to  values i.e. particulars to universals
 and therefore truly are relationships they are not called relationships in entity modelling terminology. It breaks the duck test but this is just how it is. 

\mynote Since attributes are called attributes (even though they are relationships!) then the term relationship is reserved, in entity modelling terminology and subsequently in these pages, for relationships between particulars i.e. between types of entities. 
 The principle components of an entity model are its entity types and their relationships and attributes.

\mynote A final point on universals. In addition to the relationships and attributes as so far described there will exist many relationships between universals 
for example there will be logical and arithmetical relationships. When we construct an entity model these universal relationships are taken as givens --- we do not set out to model them.

\subsection*{Diagramming Attributes}
\mynote
The use of attributes alongside entities and relationships is an essential part of entity modelling especially in the context of data modelling. Users of entity modelling may decide whether or not to include attributes on diagrams. Chen's original paper did not show attributes on diagrams  but in 1984
he published a paper that did show attributes diagramatically represented. It is not clear that he expected this style to be used wholesale. 
\mynote
In this last mentioned 1984 paper Chen presents an example diagram 
which I have reproduced in figure  \ref{ChenStyleProjectWorkerAttributed} 
and which shows a graphical representation of attributes associated with the PROJECT-WORKER relationship (\ref{projectWorkerChenStyle}) that was previously given as an example relationship in his 1976 paper.

\begin{erboxedFigure} {H}{ChenStyleProjectWorkerAttributed}{From Chen 1984. In this diagram NUMBER, NAME, NUMBER-OF-YEARS, DATE and DOLLAR-AMOUNT must all be understood as types all of whose instances are universals. Chen uses the term `value type' for such types;
thus our universal things are what he calls values. In the same way it is 
the practice in programming to refer to names, numbers, dates and so on, as values but it is more enlightening to use  the term universal, as we do here, and to reflect on  the dichotomy between things that we treat as universals and things that we treat as particulars. }
\input{\handCraftedImagesFolder/ChenStyleProjectWorkerAttributed.tex}
\end{erboxedFigure}

The same model is shown in our variant of the Barker-Ellis notation
 in the diagram shown in figure \ref{employeeProjectWorkerMediatedAttributed}.
As in the Chen diagram, diagrams in Barker's book use a hash symbol to foreground key attributes though it  is positioned differently to this use by Chen. In our style as shown here the key attributes are underlined as will be explained in the next section.\footnote{Our use of underlining is consistent with relational notations used in theory papers such as Zaniola and in descriptions of relations in Chen 1976.}
\begin{erboxedFigure} {H}{employeeProjectWorkerMediatedAttributed}
{The example from figure \ref{ChenStyleProjectWorkerAttributed} recast in the our Barker-Ellis style. 
In this style the value types of attributes are not shown on the diagram but are expected to be 
separately documented. }
\begin{erdiagram}{2.3}{14.299999999999999}

\eret{0.1}{-1.9}{2.6}{-0.4}{0.2}{1}\eretname{0.5}{-0.75}{l}{EMPLOYEE}
\erCoreAttribute{0.3}{-0.95}{1}{1}{EMP\#}{}
\erCoreAttribute{0.3}{-1.25}{1}{1}{EMP-NAME}{}
\erCoreAttribute{0.3}{-1.55}{1}{1}{AGE}{}
\eret{5.7}{-1.9}{8.7}{-0.4}{0.2}{1}\eretname{6.15}{-0.75}{l}{PROJECT-}\eretname{6.15}{-1.05}{l}{WORKER-}\eretname{6.15}{-1.35}{l}{ASSIGNMENT}
\erCoreAttribute{5.9}{-1.55}{1}{1}{STARTING-DATE}{}
\eret{11.8}{-1.9}{14.3}{-0.4}{0.2}{1}\eretname{12.2}{-0.75}{l}{PROJECT}
\erCoreAttribute{12}{-0.95}{1}{1}{PROJ\#}{}
\erCoreAttribute{12}{-1.25}{1}{1}{BUDGET}{}

% relationship assigning
\errelname{5.55}{-1}{r}{assigning}\errelname{2.75}{-1}{l}{subject of}\errelarm{5.7}{-1.15}{4.15}{-1.15}{1}{0}\errelarm{4.15}{-1.15}{2.6}{-1.15}{0}{0}\ercrowfoot{5.55}{-1.15}{5.7}{-1}{5.7}{-1.15}{5.7}{-1.3}{0}
% relationship to
\errelname{8.85}{-1}{l}{to}\errelname{11.65}{-1}{r}{resourced by}\errelarm{8.7}{-1.15}{10.25}{-1.15}{1}{0}\errelarm{10.25}{-1.15}{11.79}{-1.15}{0}{0}\ercrowfoot{8.85}{-1.15}{8.7}{-1}{8.7}{-1.15}{8.7}{-1.3}{0}
\end{erdiagram}

\end{erboxedFigure}


 

