

\section{Attributes}
\label{Attributes} 
\footnote{ChatGPT: Overall Comment
There is slight conceptual repetition — for instance, the idea that attributes relate particulars to universals is revisited several times.} 
\subsection{Data and Attribution}
\mynote
This section concerns the simplest aspects of things 
to which data can be meaningfully attributed. 
In entity modelling, these simplest aspects are known as \textit{attributes}.
We have discussed entity types and relationships at length, and now  we describe this  third vital concept.
Working together, the concepts of entity type, relationship and attribute
are the foundation of entity modelling, and establish it as a methodology that  bridges the gap between the world as we understand it (or some domain therein) and the structure of data, which data, after-all, purports to describe some part of what is.\footnote{Of course, data can be mistaken, fictional, or illustrative; nonetheless, by its form, it purports to describe some part of what is. } 

\mynote
By way of explanation, 
consider that when we call to mind \textit{data}, 
then we think of names, quantities, monetary values, 
addresses, dates, temperatures, geographical coordinates, and so on. 
Such items of data as these convey information only within specific contexts and when attributed 
to subjects at hand. 
A temperature, a colour, a price, a height, a distance — such items of data 
tell us nothing less they be the temperature, the colour, the price, the height, or the 
distance of \textit{some aspect} of \textit{some thing}. 
They tell us nothing, therefore, lest they be attributed to an entity.

\mynote
A product in a catalogue or on a website  might have a full price and a sale price
and the given values will be monetary amounts.
The product has a name 
and it is likely that in the small print there is a product identifier.
Within the scope of this website, `product' is a type of entity
and `product identifier', `product name',  `full price' and `sale price' are attributes;
each attribute is constrained, in reality, in what can be meaningfully attributed to it ---
it wouldn't make sense to say `size: red' or `colour: large' and
it  would be outside the bounds of expectation 
for a price to be given as `squillions' or `rainy' or as `10 miles'.

\mynote 
Each attribute within an entity model defines a named aspect of a type of entity and
also defines what can be meaningfully attributed to this aspect; 
for example, it makes sense to  attribute a planet within the solar system with
 orbital eccentricity
  --- the details we find on Wikipedia, for example --- 
 but it doesn't make sense to attribute your stay in a hotel with orbital eccentricity. 
 Instead your stay in a hotel may be attributed with arrival date and a departure date.
Attributes in an entity model are named and doubly constrained:
firstly they are constrained to the type of entity to which they apply, 
as aspects of entities of the type (planetary orbit rather than stay in a hotel);
secondly they are constrained in what it is that may be attributed to this aspect 
(a pure number, in the case of planetary eccentricity, 
but in other cases a colour or a date or, in
some cases, free text).

\mynote
In an entity model each type of entity has an enumerated list of named attributes. These attributes may be displayed on entity relationship diagram or they might be separately documented as, for example,
 like this:
\begin{verbatim}
product 
   => product identifier: text,
      product name: text,
      full price: monetary amount,
      sale_price: monetary amount
\end{verbatim}

What is important here is that within the context of an entity type there is a list of attributes, each attribute has a name and each attribute has a type of value specified (in this case either text or monetary amount). In this stylised layout the \verb!=>! can be read as introducing a list of attributes and the colon can be read aloud as "is" or as "is of type". 

For a more technical example, in some context I might chose to describe the orbit of any planet around the sun by the following attributes
\begin{verbatim}
planet 
   => semi-major axis: distance,
      eccentricity : real number between 0 and 1,
      inclination to ecliptic: angular distance,
      longitude of ascending node: angular distance,
      argument of perihelion: angular distance,
      time of perihelion: point in time
\end{verbatim}

The attributes \textit{semi-major axis} and \textit{eccentricity}, taken together, describe the shape and size of the ellipse traced out by the planet in orbit. The other three parameters describe the plane that the ellipse lies in. 
From the first two parameters the distance of the closest approach to the sun --- the  \textit{perihelion} --- can be calculated, as too can the furthest distance away from the sun reached by the planet 
--- the \textit{aphelion}. In fact for any planet, knowedge of \textit{perihelion}  and \textit{aphelion} gives us the size and shape of the orbit --- in other words,  
\textit{semi-major axis} and \textit{eccentricity} 
can be calculated from  \textit{aphelion}  \textit{perihelion}. 
\begin{table}[H]
\small 
\setlength{\tabcolsep}{3pt}
\begin{tabular}{|l| 
  >{\centering\arraybackslash}m{0.9cm} | 
  c | 
  >{\centering\arraybackslash}m{1.45cm} | 
  >{\centering\arraybackslash}m{1.8cm} | 
  >{\centering\arraybackslash}m{1.6cm} | 
  >{\centering\arraybackslash}m{1.8cm}|}
\hline
\small Name & 
\small Semi-major axis \newline (AU) & 
\small Eccentricity &
\small Inclination \newline to ecliptic (°) & 
\small Longitude\newline of ascending\newline node \newline (°) & 
\small Argument\newline of \newline perihelion (°) & 
\small Time of \newline perihelion \\
\hline
Mercury & 0.387 & 0.205630 & 7.0049 & 48.331 & 29.124 & 2024-Dec-25 \\
Venus   & 0.723 & 0.006772 & 3.3947 & 76.680 & 54.884 & 2023-Dec-31 \\
Earth   & 1.000 & 0.016710 & 0.0000 & 0.000 & 102.937 & 2023-Jan-04 \\
\hline
\end{tabular}
\caption{Parameters describing the planetary orbits of the first three 
planets. A fuller version of this table is given in appendix \ref{PlanetaryOrbits}
}
\end{table}

What this means is that I could have chosen a different set of attributes for my entity type `planet'. 
For example, among the four
attributes \textit{perihelion}, \textit{aphelion}, \textit{semi-mean axis} and \textit{eccentricity},  any two can be used to compute the other two. The idea has been to present a \textit{core set}. This matters. When data is stored or communicated  then 
we need only store or transmit such a core set  --- any values that can be derived from them need not be sent. We will return to this topic later.

\subsection{Types of Particulars and Types of Universals}
In entity modelling, the word \textit{type} is used in two distinct senses, each playing a different role in how we describe the world:
\begin{itemize}
\item An entity type classifies particular things—for example, the entity type Planet includes Earth, Mars, and Jupiter as instances. These are concrete, identifiable entities.
\item An attribute type, by contrast, is typically a determinable: a general kind of quality or measurement. Its instances are determinates, such as red, green, 6.4 km/s, or 23.5°C. These are not things in themselves, but possible values that can be attributed to entities.
\end{itemize}
This distinction helps clarify how attributes function in a model. An attribute identifies a particular aspect of an entity (such as its colour or its orbital period), and it is constrained by a type that defines what kind of value may be associated with that aspect (e.g. a colour, a duration, or a number).

In philosophy, a distinction is drawn between \textit{particulars} and \textit{universals}. 
Numbers, colours and  other determinate quantities are generally judged to be universals. 
With this distinction in mind then we can characterise the two kinds of types that we are
talking about as
\begin{itemize}
\item types all of whose instances are particulars (entity types), and
\item types all of whose instances are universals (attribute types).  
\end{itemize} 

\subsection*{Diagramming Attributes}
\mynote
Users of entity modelling may decide whether to include attributes on diagrams or whether to document them
separately.
Chen's original paper did not show attributes on diagrams  but in 1984
he published a paper that did. In this  paper Chen presents an example diagram 
which I reproduce in figure  \ref{ChenStyleProjectWorkerAttributed}. 
It shows  attributes associated with the PROJECT-WORKER relationship (\ref{projectWorkerChenStyle}) that was previously given as an example relationship in his 1976 paper.

\begin{erboxedFigure} {H}{ChenStyleProjectWorkerAttributed}{From Chen 1984. In this diagram NUMBER, NAME, NUMBER-OF-YEARS, DATE and DOLLAR-AMOUNT must all be understood as types all of whose instances are universals. Chen uses the term `value type' for such types;
thus our universal things are what he calls values just as in programming it is is the practice 
to refer to names, numbers, dates and so on, as values. }
\input{\handCraftedImagesFolder/ChenStyleProjectWorkerAttributed.tex}
\end{erboxedFigure}

This Chen diagram is redrawn in the Barker-Ellis style
 in the diagram shown in figure \ref{employeeProjectWorkerMediatedAttributed}.
As in the Chen diagram, diagrams in Barker's book use a hash symbol to foreground key attributes. In this book the key attributes are distinguished by underlining them as will be explained in the next section.\footnote{Our use of underlining is consistent with relational notations used in theory papers such as Zaniola and in descriptions of relations in Chen 1976.}
\begin{erboxedFigure} {H}{employeeProjectWorkerMediatedAttributed}
{The example from figure \ref{ChenStyleProjectWorkerAttributed} recast in the our Barker-Ellis style. 
In this style the value types of attributes are not shown on the diagram but are expected to be 
separately documented. }
\begin{erdiagram}{2.3}{14.299999999999999}

\eret{0.1}{-1.9}{2.6}{-0.4}{0.2}{1}\eretname{0.5}{-0.75}{l}{EMPLOYEE}
\erCoreAttribute{0.3}{-0.95}{1}{1}{EMP\#}{}
\erCoreAttribute{0.3}{-1.25}{1}{1}{EMP-NAME}{}
\erCoreAttribute{0.3}{-1.55}{1}{1}{AGE}{}
\eret{5.7}{-1.9}{8.7}{-0.4}{0.2}{1}\eretname{6.15}{-0.75}{l}{PROJECT-}\eretname{6.15}{-1.05}{l}{WORKER-}\eretname{6.15}{-1.35}{l}{ASSIGNMENT}
\erCoreAttribute{5.9}{-1.55}{1}{1}{STARTING-DATE}{}
\eret{11.8}{-1.9}{14.3}{-0.4}{0.2}{1}\eretname{12.2}{-0.75}{l}{PROJECT}
\erCoreAttribute{12}{-0.95}{1}{1}{PROJ\#}{}
\erCoreAttribute{12}{-1.25}{1}{1}{BUDGET}{}

% relationship assigning
\errelname{5.55}{-1}{r}{assigning}\errelname{2.75}{-1}{l}{subject of}\errelarm{5.7}{-1.15}{4.15}{-1.15}{1}{0}\errelarm{4.15}{-1.15}{2.6}{-1.15}{0}{0}\ercrowfoot{5.55}{-1.15}{5.7}{-1}{5.7}{-1.15}{5.7}{-1.3}{0}
% relationship to
\errelname{8.85}{-1}{l}{to}\errelname{11.65}{-1}{r}{resourced by}\errelarm{8.7}{-1.15}{10.25}{-1.15}{1}{0}\errelarm{10.25}{-1.15}{11.79}{-1.15}{0}{0}\ercrowfoot{8.85}{-1.15}{8.7}{-1}{8.7}{-1.15}{8.7}{-1.3}{0}
\end{erdiagram}

\end{erboxedFigure}

Review\footnote{Chatgpt:Diagramming Section:
This is rich but perhaps slightly dense. One option could be to move some of the historical commentary (about Chen’s evolving use of diagrams) to footnotes or an appendix if you’re aiming for a smoother read. But if your audience values the historical depth, then the current form is great.}


\subsection{Descriptions and Definitions}\footnote{“That is my definition, and if you don’t like it... well, I have others.”}

\mynote
In the literature the notion of \textit{an attribute} is variously described.
In the Goodland SSADM book an attribute is defined as
\begin{erquote}
  the smallest discrete component of the system information that is meaningful,
\end{erquote}
In  Shlaer Mellor we find an attribute is described as being
the abstraction of a \textit{single} characteristic possessed by all the entities
of a type. Whereas in Richard Barker's book we find that an attribute is 
\begin{erquote}
any detail that serves to qualify, identify, classify, quantify or express an  entity,
\end{erquote}

\mynote
Chen in his  1976 paper uses the term \textit{value set} for the sets of values of what I have variously
referred to as attribute types and more prosaically \textit{types all of whose instances are universals}.
This then enables Chen to observe that 
\begin{erquote}
\textit{An attribute can be formally defined as a function which maps...into a value set}.
\end{erquote}
 and to say also that
 \begin{erquote}
 \textit{Since an attribute is defined as a function, it maps an entity in an entity
set to a single value in a value set}.
\end{erquote}

\mynote
Chen's formal definition is an important insight.
It means that in entity modelling, the term \textit{attribute} is used as a specific term
for a named functional relationship between a type all of whose instances are particulars  and a type all of whose instances are universals of some kind.
The universals in question are often said to be \textit{values} but 
in philosophy  are said to be \textit{determinates}. 

\mynote
It is ironic that although attributes relate entities to values—
that is, particulars to determinates, and are therefore, very much relationships, if the term is properly used, 
they are not called this in entity modelling.
It breaks the duck test, but that's just how it is:
attributes are called attributes,
and the term relationship is reserved---both in entity modelling and in these pages---
for relationships between between types of entities;
it comes about that the principal components of an entity model are its entity types,
their relationships, and their attributes.

\subsection{Notation}
\mynote  As stored within information systems
then the individually represented items --- the 
names, colours, quantities, the monetary amounts, the dates, etc --- in the language 
of entity modelling, are said to be the values of attributes. Thus an actual name 
like ``John Smith'' is said to the value of a ‘name’ attribute of a ‘person’ entity and if 
`person' entities may be attributed with ‘date of birth’ also
then on a diagram this is represented like this
\ercenterPicture{personNameDobAttribute}

\mynote 
In a situation where for some persons date of birth may not be known then the attribute
may be defined to be \textit{optional}.\footnote{With regard to the attribute as a functional relationship
then this is equivalent to asserting that the functional relationship is partial not total.}
Usually this will be represented on the diagram somehow. I would represent it like this
\ercenterPicture{personMandatoryNameAttribute}

\mynote
In a situation where `person' entities can be uniquely identified by name then the `name' attribute
is said to be identifying and in my diagrams this is shown by underlining the attribute as here	
\ercenterPicture{personIdentifyingNameAttribute}

On the other hand, if it is necessary to give both a person's name \textit{and} their date of birth to uniquely identify 
them  then we underline
both of these attributes on the diagram:
\ercenterPicture{personIdentifyingNameDobAttribute}
Generally, systems will hold and communicate many different attributes of each type of entity and
these attributes are shown beneath the identifying attributes:
\ercenterPicture{personAttributes}

\mynote 
From a practical perspective, 
it is clear that computer programs are effective only in so long as the data items 
they manipulate are intended and understood as attributes of subject entities. It 
follows that to have an effective information system we must first have agreed types 
of subject entity and also what may be attributed to entities  of these types. 
In this agreement we agree the data content of the program or system i.e.  
its subject matter. But what is seen to be at first sight an attribute of type of entity
as, say, `phone number' of the person entity,  may
subsequently found to belong somewhere else in the model  ---   if a person can have multiple phone numbers then ‘phone number’ is 
not an attribute  of a ‘person’ entity type \textit{per se} but an attribute of a 
separate but related entity as here 
\ercenterPicture{personPhone}


\subsection{Relationships between Universals}
\mynote A final point on universals. In addition to the relationships and attributes as so far described there will exist many relationships between universals themselves
for example there will be logical and arithmetical relationships. When we construct an entity model these universal relationships are taken as givens --- we do not set out to model them.
For example, temperature values are comparable, and we can speak of one being higher than another.
\commentary{revise chatgpt suggested text}
 These kinds of relationships between values (universals) — like comparisons, \newtt{unit conversions} and
 \newtt{arithmetical operations} 
 \oldtt{units, or derived quantities} — are typically considered background knowledge \oldtt{or}
 \newtt{to be} handled by the application logic, not modelled in the entity structure itself.



\subsection{What are the Givens?}

I ask you to go along with the proposition that numbers are \textit{givens} and \textit{universal}, though I admit there is room for discussion as to what exactly we mean by \textit{number}. We need to distinguish between natural numbers ($0,1,2,...$), integers ($...,-2,-1,0,1,2,3,...$), and real (decimal) numbers, as used in pricing, measurement, proportions, and so on. 

Moreover, in information systems, numbers are typically finite in extent, constrained by 32- or 64-bit representations (or similar schemes). These technical limits impose restrictions both on the magnitudes of integral values and on the granularity and precision of real-number representations.

Likewise, I ask you to consider alphabetic characters and textual values as givens and universals—though,  they vary widely across the world and across systems. We won’t be delving into that variety here.

When it comes to defining the values we take as givens, different technologies—such as XML Schema, IDL, and SQL—define their own sets of primitive types. Modelling notations may adopt one or another of these foundational systems. A certain attribute might be modelled conceptually as having values of type \textit{real}; or, more pragmatically, it might be modelled as type \textit{float64} to reflect its implementation in a particular system.


\subsection{Styles of Attribute Type Definition}

There are also questions of style.

One such consideration is the treatment of units. Units of measurement are crucial—so how should they be handled? One option is to include the unit in the attribute name, as in:

\begin{erquote}
heightInCm: float64
\end{erquote}

Alternatively, one might define custom types that incorporate the unit, allowing for a more abstract naming scheme:

\begin{erquote}
height: numberOfCm,\\
width: numberOfCm
\end{erquote}


A further possibility is to conceptualise \textit{length} as a universal, without committing to a specific unit. If we follow this approach then we will write
\begin{erquote}
height: length
\end{erquote}
and expect the length concept to subsume the unit of measurement in some way. 

A question arises in the context of scientific measurement: how should accuracy and precision be represented? Can a single attribute contain both a measured magnitude and an associated uncertainty? Or should these be modelled as distinct attributes?






