

\section{Entity Models}
\label{EntityModels}
\mynote The word `entity' as it is used in this book is used in its most general sense,  the sense in which it just means `thing'. The entities that are being modelled in entity modelling can be just about any `things' at all --- and so we could just as well speak of `modelling things' as `entity modelling' but that the term entity modelling
has come to mean a particular way of modelling things in which things are described by enumerating the binary relationships that things of each type may participate in.

\mynote There is a proviso to this. 
Things that are known from the beginning and unchanging from one context to another, things that can be said to be `universal' are excluded from being subjects of our modelling. We do not model, therefore, whole numbers, nor real numbers, nor truth values 
(the things known to programmers as `booleans') but take these as universal givens and, generally, 
we do not model language characters and language character sequences 
(known to computer programmers as `strings') with the richness that this would require --- we take these too to be universal and given. 

\mynote In philosophy the non-universal things that are our entities are referred to as \textit{particulars} and so we can say that the entity types within an entity model represent types of particular things i.e. those types all of whose instances are particulars.  

\mynote Now I am able to explain, that, in entity modelling, the term attribute is adopted as a specific term meaning a relationship between a particular on the one hand and a universal on the other. 
Likewise, the term relationship 
in its primary use in entity modelling, is reserved specifically for relationships between particulars 
i.e. between entities. 

\mynote Having explained this much we can now say that an entity model posits a collection of entity types, relationships and attributes. These are the E, the R, and the A of entity modelling and give rise to the acronyms ER and ERA and the description of entity modelling as ER modelling or ERA modelling.
Each entity model can be thought of as a structured document containing definitions of such  
E, R and A i.e. of such types, relationships and attributes. 
This is much  as a dictionary or glossary has entries giving definitions of (certain) nouns, verbs, adjectives as well as other parts of speech and, indeed, in some circles the term data dictionary is used. Like an architect's design, or an electrician's circuit, much of an entity model can be usefully represented on one or more diagrams. The diagrams are often known as entity-relationship (ER) diagrams or sometimes as  entity-relationship-attribute (ERA) diagrams. 
These diagrams in one or other of the available styles are the visible and recognisable face of entity modelling. They originate with the diagrams of Chen from his 1976 paper which, most significantly, has the title ``The Entity-Relationship Model---Toward a Unified View of Data''. His style of diagrams are recognisable by their use of diamond shaped boxes to represent relationships. In this book our diagrams are more structured. They use Barker-Ellis style diagrams and, as we come to shortly, these are recognisable by their use of optionally half dashed lines to represent relationships and their optional use of crowsfeet on the ends of these lines to represent multiplicity. 

\mynote Entity modelling may have as its subject just about any thing but it is never about modelling everything.
Instead, an entity model serves to describe a particular domain of discourse. In the general case it describes the entities in this domain in relation not just to each other but to other given types representing that which is universal. 
\mynote
Now a very  important point ---
\textit{any entity model describes the structure of a single composite thing which all the individual things that it describes are part of}. Now it is not usual to speak of entity models in this way which is why I draw attention to this and if you are already familiar with entity modelling you may find this controversial. Bear with me.
\mynote
Like other things this composite thing is described by defining the type of thing that it is. This composite is usually in the background in entity modelling but in our entity models and in our diagrams we bring it to the foreground and represent it as special type on the diagram. 
This, in context, represents the whole of everything.\footnote{Another way of saying this is that in the internal logic of the model it represents the whole of everything. This, in the internal logic is one and the same with the absolute. Subtly, an entity model determines that which is to be considered absolute.} As we come to later and for good reason we like to call this whole of everything \textit{the absolute}. \commentary{fine tune this discussion}  This composite thing  is the subject of an entity model and it can be a small thing such as a single molecular structure so that the whole of everything consists of atoms and covalent bonds or, maybe, the state of a game of chess so that the whole of everything is the board and the positions of pieces on the board. It could be the parse tree of a single grammatical sentence 
so that the individual entities are syntactic constituents. 
It could be a company such as a manufacturing company and all its suppliers, customers, employees, departments, projects and premises. 
It could be an online business and all of its products, customers and sales or
a science company and all of its laboratories, equipment, samples, analytic instruments, assignations, data records and  scientific reports. 
\mynote
Describing a composite thing we are led to its parts. Often we find that these parts are composites and themselves have parts. The result is a hierarchy of things. Many of the  entity models in this book describe such hierarchies and such models we refer to as  \textit{structured entity models}.

