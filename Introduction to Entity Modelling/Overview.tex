
\newcommand{\mysection}[1]{\underline{\hyperref[#1]{#1}}}
\section*{Overview}
\label{Overview}
\addcontentsline{toc}{section}{\nameref{Overview}}
\begin{tabular}{l l p{7cm}}
1 & \mysection{Perspective} & Entity modelling form of concept modelling, more narrowly for describing the structuring of data.\\
\hline
2 & \mysection{EntityModels} & Entities, particulars and universals, attribute and relationship, entity model as structured document, ER and ERA diagrams, Chen origins, diamonds, Barker-Ellis notation, single composite thing, the absolute, parts, hierarchies, structured entity models.\\
\hline
3 & \mysection{Relationships} & Relationship and relationship instance, binary relationships, directional relationships, Chen diamond notation, Barker-Ellis notation, many-many relationships, functional relationships, ternary relationships. \\
\hline
4 & \mysection{Entity Relationship Diagramming} & Generalities TBD. Chen manufacturing example, wikipedia entry `part number'.
                                     Goodland and Slater Car Hire Business on SSADM.\\
\hline
5 & \mysection{PathsofRelationships} & Paths of Relationships. This could move before attributes.\\
\hline
6 & \mysection{Attributes} & attributes as functional relationships, value types.
\\
\hline
7 & \mysection{IdentifyingFeatures} &  set of identifying features\\
\hline
8 & \mysection{DramaticArtsExample} &  single model loosely based on David Hay\\
\hline
9 & \mysection{ReferencingEntities} &  with section(s) on communicating relationships\\
\hline
10 & \mysection{AirportGateExample} &  Discussion --- The Airport Gate example\\

\hline
11 & \mysection{Scope} & The Scope Concept from book.   Schlaer Lang example. \\
\hline
12 & \mysection{CommunicatingConjunctions} & Sharing referentials across conjuncts --- 
                                                 leading to structure of tables\\
13 & \mysection{Conclusion} & I have demonstrated the ubiquity of commutative diagrams of relationships
and indicated how by recognising then,  relational database design can be \textit{right first time}. \\
\hline
13 & \mysection{CoreversusDerivative} &  Core versus derivative, conceptual core, data cores, goodness criteria. The Core and Derivatives? \\
\hline
14 & \mysection{DataModelling}& Database and messge structure, conceptual, logical, physical, Codd oriented history, goodness criteria, normal forms, methodology improvement.\\
\hline
15 & \mysection{TypeInheritance} & specialisation and generalisation, species and genera, meta physics, nestedbox notation. Described as sub-types and super-types in Barker's book and likewise but mentioned as non-standard in SSADM book. Single inheritance versus multiple inheritance.   \\
\hline
16 & \mysection{StructuredEntityModelling} & Chen, PCTE, composition relationships,  top-down style, simple meta-model.\\
\hline
17 & \mysection{TheAbsolute} & Some metaphysics.\\
\hline
18 & DistinguishingCompositionandReference&The Distinction Between Composition and Reference from tutorial part one\\
\hline
\end{tabular}
\section*{Notes}
\mynote I  have one source file per section.
\mynote I  use \verb'\commentary' to add markups in the margin.

\mynote Beef up the `Scope' section futher. 
Drop the SSADM book customer,payment,allocated payment,invoice,booking,vehicle and vehicle category in earlier as well. Then have the two subdiagrams in this scope section 3and comment on the scope of these.
Reproduce the entire such ssadm example. Put in as a second example ERD in the current example erd section.
\begin{noteforfuture}
shlaerlang use the term \textit{collapsed referentials}.
an I use this as a section title? That would be good.
\end{noteforfuture}

\mynote rationale -- when it comes to one order for the introduction of terms rather than another -- there are a few considerations which likely conflict
\begin{itemize}
	\item  present conceptual modelling before data modelling
	\item  present relational data modelling before structured entity modelling
	\item rationalise structured entity modelling from point of view of hierarchical data specification. 
	\item  present identifying features as part of conceptual modelling because
	it is applicable to conceptual modelling though it really comes into its own
	later applied in data modelling
	because there is then more value to it.
	\item present discussions about the comminiucation of relationship instances before 
		getting on to data modelling
	\item   present goodness criteria as part of conceptual modelling
	or present the core and its derivatives as part of conceptual modelling then goodness criteria as part of data model where value is stated.
	\item similarly present scope as part of conceptual modelling because it is part of understanding concepts then show its value in data modelling? 
	\item in goodness section in conceptual modelling bit discuss absence of referential attributes to entities in scope of model and out of scope of model.
	\item and dont model a referential attribute in preference to a relationship.
	\item in data modelling section reintroduce referential attributes. 
	\item somewhere have a section on entity modelling without diagrams. Can get almost most of the most significant advantages of entity modelling without using diagrams. This might be a  introducing xml and ERScript without hgaving to worry about diagrams.
	\item example of modelling a boolean international flight in which an airport is located.
\end{itemize}

\mynote somewhere --- exclusion arcs
\mynote embed a few more examples Check out SSADM book page 213. 
\mynote meta model example given in section 9 needs to have other meta-models before it and needs the very idea to be explained.

\begin{notebox}
Important point. When I omit attributes from a diagram and depict just the entity types and relationships then I omit bars from the relationships unless the set of identifying features of the entity type consists entirely of relationships. Then the bars are shown.
Thus in figure \ref{chenManufacturingCo..diagram} I show bars only on relationships from intersection entities. This is entirely and utterly satisfying and frees me up from some doubt. My doubt has been how could I show identifying relationships when they are only identifying in the presence of 
mutual identifying attributes. Relief! 
\end{notebox}

