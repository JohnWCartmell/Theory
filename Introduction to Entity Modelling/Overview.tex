
\newcommand{\mysection}[1]{\underline{\hyperref[#1]{#1}}}
\section*{Overview}
\label{Overview}
\addcontentsline{toc}{section}{\nameref{Overview}}
\begin{tabular}{l l p{7cm}}
1 & \mysection{Perspective} & Entity modelling form of concept modelling, more narrowly for describing the structuring of data.\\
\hline
2 & \mysection{EntityModels} & Entities, particulars and universals, attribute and relationship, entity model as structured document, ER and ERA diagrams, Chen origins, diamonds, Barker-Ellis notation, single composite thing, the absolute, parts, hierarchies, structured entity models.\\
\hline
3 & \mysection{Relationships} & Relationship and relationship instance, binary relationships, directional relationships, Chen diamond notation, Barker-Ellis notation, many-many relationships, functional relationships, ternary relationships. \\
\hline
4 & \mysection{Entity Relationship Diagramming} & Generalities TBD. Chen manufacturing example, wikipedia entry `part number'.
                                     Goodland and Slater Car Hire Business on SSADM.\\
\hline
5 & \mysection{PathsofRelationships} & Paths of relationships. Comparable paths, equivalent and near-equivalent paths. Commuting diagrams.\\
\hline
6 & \mysection{Attributes} & attributes as functional relationships, value types.
\\
\hline
7 & \mysection{IdentifyingFeatures} &  set of identifying features, dramatic arts example.\\
\hline
8 & \mysection{ReferencingEntities} &  referentials and the accumulation of identifying attributes, collapsed referentials.\\
\hline
9 & \mysection{CommunicatingRelationships} & equivalent and near equivalent paths and collapsed referentials. \\
\hline
10 & \mysection{AirportGateExample} &  Discussion --- The Airport Gate example\\
\hline
11 & \mysection{Scope} & Placeholder on the Scope Concept from book.   Schlaer Lang example. \\
\hline
12 & \mysection{CommunicatingConjunctions} & This is about communicating sets of relationships. 
Sharing referentials across different relationships --- 
                                                 leads to structure of tables???\\
13 & \mysection{Conclusion} & I have demonstrated the ubiquity of commutative diagrams of relationships
and indicated how by recognising them,  relational database design can be \textit{right first time}. \\
\hline
13 & \mysection{CoreversusDerivative} &  Core versus derivative, conceptual core, data cores, goodness criteria. The Core and Derivatives? \\
\hline
14 & \mysection{DataModelling}& Database and messge structure, conceptual, logical, physical, Codd oriented history, goodness criteria, normal forms, methodology improvement.\\
\hline
15 & \mysection{TypeInheritance} & specialisation and generalisation, species and genera, meta physics, nestedbox notation. Described as sub-types and super-types in Barker's book and likewise but mentioned as non-standard in SSADM book. Single inheritance versus multiple inheritance.   \\
\hline
16 & \mysection{StructuredEntityModelling} & Chen, PCTE, composition relationships,  top-down style, simple meta-model.\\
\hline
17 & \mysection{TheAbsolute} & Some metaphysics.\\
\hline
18 & DistinguishingCompositionandReference&The Distinction Between Composition and Reference from tutorial part one\\
\hline
\end{tabular}