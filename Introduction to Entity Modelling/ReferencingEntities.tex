
\newcommand{\BarkerEllisIdentiyingRelationship}{\barkerEllisJ}

\section{Referencing Entities}
\label{ReferencingEntities}\commentary{\highlight{remove near-equivalence}}

We now explore how the arrangement of identifying features of a type of entity
 guides us in phrasing references to entities of that type, 
 and how this, in turn, influences the ways we convey, communicate, and store instances of relationships.
There is more to this than initially meets the eye.
Through examples, we will see that comparable paths that are equivalent (as described in Section~\ref{comparisonOfPaths}) affect how we phrase references.
Consequently, equivalence between paths has a direct impact on both the structure of data and on how simple, everyday facts are expressed.

This section is divided into subsections, each addressing a characteristic configuration of identifying relationships and attributes relevant to referencing entities of a type.
We begin with the simplest cases and proceed to examples involving branching chains of identifying relationships, and, in some cases, comparable or equivalent paths.
At each step, we examine how these affect the way entities are referenced.

\subsection{Identifying Relationships and Nested References}
\begin{worktt}The act of referencing mirrors the structure of identification itself: when an entity’s identity depends on another, so too does the form of \workt{its reference}. 
\end{worktt}
\noindent
\begin{worktt}
In what follows, references that depend on other references will generally be called \textit{subordinate references}.
Occasionally, when one subordinate reference is literally contained within another in phrasing or structure, we will describe it as \textit{nested}. \commentary{think of implementing this}
\end{worktt}

We begin with a simple example involving two entity types, $A$ and $B$, connected by an identifying relationship
$A$ \BarkerEllisIdentiyingRelationship\ $B$.
In such a situation, a stand-alone reference to an entity of type $A$ includes, nested within it, a subordinate reference to an entity of type $B$.

\subsubsection{Example 1 — Referencing an Entity of Type \textit{Play}}

Consider the type \textit{play} from the model of the dramatic arts introduced earlier in Figure~\ref{dramaticArts1..diagram}.
Its identifying features are its attribute \textit{title} and its identifying relationship \textit{written\_by}, relating each play to a playwright, as shown here:
\begin{equation*}
\begin{erdiagram}{1.7}{6.024038749999999}

\eret{0.1}{-1.2}{1.433}{-0.3}{0.2}{1}\eretname{0.233}{-0.65}{l}{play}
\erCoreAttribute{0.3}{-0.85}{1}{0}{title}{}
\eret{4.233}{-1.2}{5.924}{-0.3}{0.2}{1}\eretname{4.402}{-0.65}{l}{playwright}
\erCoreAttribute{4.433}{-0.85}{1}{0}{name}{}

% relationship written_by
\errelname{1.583}{-0.6}{l}{by}\errelname{1.583}{-0.3}{l}{written}\errelname{4.083}{-1.05}{r}{the}\errelname{4.083}{-1.35}{r}{author}\errelname{4.083}{-1.65}{r}{of}\errelarm{1.433}{-0.75}{2.833}{-0.75}{1}{0}\errelarm{2.833}{-0.75}{4.233}{-0.75}{0}{0}\errelid{2.833}{-0.84}{}{r1}\ercrowfoot{1.583}{-0.75}{1.433}{-0.6}{1.433}{-0.75}{1.433}{-0.9}{0}\eridrefrel{1.6833}{-0.65}{-0.85}
\end{erdiagram}

\end{equation*}
From this we see that a play is identified by its title together with a reference to the playwright it was written by.
Thus we may refer to:
\begin{equation}
\mbox{the play \textit{Twelfth Night} by playwright William Shakespeare}
\end{equation}
which can also be phrased as
\begin{equation}
\mbox{\textit{Shakespeare’s Twelfth Night}.}
\end{equation}

Either way, explicitly or implicitly, there is a reference to a play containing within it a nested reference to a playwright.
The nested reference may be omitted when the playwright is clear from context—for example, when listing Shakespeare’s comedies:
\textit{Twelfth Night, As You Like It, A Midsummer Night’s Dream}, and so on.
In general, however,
\begin{itemize}
\item without contextual restriction,
\begin{equation}
\label{absolutePlayReferencing}
\text{\parbox{9cm}{
a play is referenced by a combination of the \uline{title} of the play and the \uline{name} of the playwright it is \uwave{written by.}
}}
\end{equation}
\end{itemize}
Identifying attributes are underlined to highlight the number required in the reference.


\begin{erboxedFigure}{H}{TwelfthNight}
{
An ER analysis of a two level reference to the play Twelfth Night. This is an example of a reference with another reference nested within it.
}
\newcommand{\dashRefOne}{2pt 2pt}
\newcommand{\dashRelationship}{1pt 0pt}
\newcommand{\dashRefTwo}{1pt 1pt}
\newcommand{\synLabel}[3]
{
  \Rnode{#1}{\parbox[t]{#2cm}{\textit{#3}}}
}
\begin{tabular}{l}
the 
\Rnode{et}{\uline{play}}
\Rnode{attrvalue}{\rdash{Twelfth Night}}
\Rnode{relname}{\uwave{by}}
\Rnode{nestedref}{\rdot{playwright Shakespeare}} \\[1.5cm]

\synLabel{tagET}{1}{name of entity type}
\kern0.35cm\synLabel{tagAV}{1.65}{value of identifying attribute}
\kern0.35cm\synLabel{tagRN}{1.625}{name of identifying relationship}
\kern0.5cm\synLabel{tagNestedRef}{1.95}{\kern0.5cmnested \\reference to entity of type playwright}\\[0.5cm]
\syntag{\dashRefOne}{tagET}{0.9}{et}{0}
\syntag{\dashRefOne}{tagAV}{0.9}{attrvalue}{-0.5}
\syntag{\dashRefOne}{tagRN}{0.9}{relname}{0}
\syntag{\dashRefTwo}{tagNestedRef}{0.9}{nestedref}{0}
\end{tabular}
\end{erboxedFigure}

\subsection{Linear Arrangements of Identifying Relationships}\commentary{Russian Doll references}

When a type has emanating from it a simple path of identifying relationships,
when the arrangement of identifying relationships is linear i.e. there is no branching of such relationships,
then references to the type have nested references which in turn have nested references within in them and so on; you could say that these are Russian-doll-structured references --- finitely, each reference bar one has one other reference nested within it.

\subsubsection{Example 2 --- Postal Addresses}

A classic example is the postal address. \commentary{when you see an envelope with an address on it it becoming to be much more data-like} \commentary{Example with an address on an envelope.}
In practice there is no uniformity  but traditional postal addresses however
represented, are references having multiple levels of nesting within them.
For instance a reference to a particular house will usually have nested within it a reference to a street, usually within this there is a reference to a town or city and
as discussed this may contain a reference to a state or province.  
The multiple levels of nesting of references follows from the
the chaining of identifying relationships in an arrangement
of entity types and relationships like this one in which there are four levels of nesting
\begin{equation*}
\begin{erdiagram}{1.51}{11.7914675}

\eret{0.1}{-1.51}{1.6}{-0.3}{0.2}{1}\eretname{0.25}{-0.65}{l}{building}
\erCoreAttribute{0.3}{-0.85}{1}{0}{number}{}
\erCoreAttribute{0.3}{-1.15}{1}{0}{postcode}{}
\eret{2.6}{-1.51}{4.1}{-0.3}{0.2}{1}\eretname{2.75}{-0.65}{l}{street}
\erCoreAttribute{2.8}{-0.85}{1}{0}{name}{}
\eret{5.1}{-1.51}{6.6}{-0.3}{0.2}{1}\eretname{5.25}{-0.65}{l}{town}
\erCoreAttribute{5.3}{-0.85}{1}{0}{name}{}
\eret{7.6}{-1.51}{9.291}{-0.3}{0.2}{1}\eretname{7.769}{-0.65}{l}{country}\eretname{7.769}{-0.95}{l}{subdivision}
\erCoreAttribute{7.8}{-1.15}{1}{0}{name}{}
\eret{10.291}{-1.51}{11.791}{-0.3}{0.2}{1}\eretname{10.441}{-0.65}{l}{country}
\erCoreAttribute{10.491}{-0.85}{1}{0}{name}{}

% relationship in
\errelname{1.75}{-0.705}{l}{in}\errelarm{1.6}{-0.905}{2.1}{-0.905}{1}{0}\errelarm{2.1}{-0.905}{2.6}{-0.905}{1}{0}\ercrowfoot{1.75}{-0.905}{1.6}{-0.755}{1.6}{-0.905}{1.6}{-1.055}{0}\eridrefrel{1.85}{-0.805}{-1.0050000000000001}
% relationship in
\errelname{4.25}{-0.705}{l}{in}\errelarm{4.1}{-0.905}{4.6}{-0.905}{1}{0}\errelarm{4.6}{-0.905}{5.1}{-0.905}{1}{0}\ercrowfoot{4.25}{-0.905}{4.1}{-0.755}{4.1}{-0.905}{4.1}{-1.055}{0}\eridrefrel{4.35}{-0.805}{-1.0050000000000001}
% relationship in
\errelname{6.75}{-0.705}{l}{in}\errelarm{6.6}{-0.905}{7.1}{-0.905}{1}{0}\errelarm{7.1}{-0.905}{7.6}{-0.905}{1}{0}\ercrowfoot{6.75}{-0.905}{6.6}{-0.755}{6.6}{-0.905}{6.6}{-1.055}{0}\eridrefrel{6.85}{-0.805}{-1.0050000000000001}
% relationship in
\errelname{9.441}{-0.705}{l}{in}\errelarm{9.291}{-0.905}{9.791}{-0.905}{1}{0}\errelarm{9.791}{-0.905}{10.29}{-0.905}{1}{0}\ercrowfoot{9.441}{-0.905}{9.291}{-0.755}{9.291}{-0.905}{9.291}{-1.055}{0}\eridrefrel{9.5414675}{-0.805}{-1.0050000000000001}
\end{erdiagram}

\end{equation*}
\mynote
What I say regarding nesting of references in spoken English also applies, though arguably with some exceptions, when references are communicated in data.
The following has been given as an example of a message structure for a postal address and is represented as XML
\begin{verbatim}
<PstlAdr>
  <StrtNm>South LaSalle Street</StrtNm>
  <BldgNb>120</BldgNb>
  <PstlCd>60690-0834</PstlCd>
  <TwnNm>Chicago</TwnNm>
  <CtrySubDvsn>IL</CtrySubDvsn>
  <Ctry>US</Ctry>
</PstlAdr>
\end{verbatim}

\subsubsection{Example 3 --- Referencing Characters from Plays}
\label{exampleReferencingCharacters}
\mynote
 We use as an example the referencing of characters from  plays
 within the context of  the  dramatic arts
   as modelled earlier in figure \ref{dramaticArts1..diagram}.
We focus on the details shown in this fragment \commentary{should I use a linear variant this diagram?}
\begin{equation*}
\begin{erdiagram}{4.1}{5.138263749999999}

\eret{0.1}{-1.5}{1.791}{-0.6}{0.2}{1}\eretname{0.269}{-0.95}{l}{playwright}
\erCoreAttribute{0.3}{-1.15}{1}{0}{name}{}
\eret{3.591}{-1.5}{4.924}{-0.6}{0.2}{1}\eretname{3.724}{-0.95}{l}{play}
\erCoreAttribute{3.791}{-1.15}{1}{0}{title}{}
\eret{3.477}{-4}{5.038}{-3.1}{0.2}{1}\eretname{3.633}{-3.45}{l}{character}
\erCoreAttribute{3.677}{-3.65}{1}{0}{name}{}

% relationship about
\errelname{4.407}{-1.8}{l}{about}\errelname{4.407}{-2.95}{l}{in}\errelarm{4.257}{-1.5}{4.257}{-2.3}{1}{0}\errelarm{4.257}{-2.3}{4.257}{-3.1}{1}{0}\errelid{4.257}{-2.39}{}{d1}\eridcomprel{4.15738875}{4.357388749999999}{-2.85}\ercrowfoot{4.257}{-2.95}{4.107}{-3.1}{4.257}{-3.1}{4.407}{-3.1}{0}
% relationship written_by
\errelname{3.441}{-0.9}{r}{by}\errelname{3.441}{-0.6}{r}{written}\errelname{1.941}{-1.35}{l}{the}\errelname{1.941}{-1.65}{l}{author}\errelname{1.941}{-1.95}{l}{of}\errelarm{3.59}{-1.049}{2.69}{-1.049}{1}{0}\errelarm{2.69}{-1.049}{1.79}{-1.049}{0}{0}\errelid{2.691}{-1.14}{}{r1}\ercrowfoot{3.441}{-1.05}{3.591}{-0.9}{3.591}{-1.05}{3.591}{-1.2}{0}\eridrefrel{3.34073875}{-0.9499999999999998}{-1.15}
\end{erdiagram}

\end{equation*}

these are all the details that are relevant to the identification of entities of type character.


\mynote
From inspecting the representation of the character type and in particular its identifying features then we see that:
\begin{itemize}
\item within the context of a specific play, 
\begin{equation} 
\label{absoluteCharacterReferencing}
\text{\parbox{9cm}{
a character may be referenced by their \uline{name}.}}
\end{equation}
Accordingly in the text of a play characters are referenced 
by name as for example in stage directions such as this one:  
\begin{equation*}
\text{\textit{Enter \rdash{Sebastian}.}}
\end{equation*}
\item within the context of a specific playwright (but with no particular play in mind), 
\begin{equation} 
\label{playwrightRelativeCharacterReferencing}
\text{\parbox{9cm}{
 a character from within a play 
may be referenced by their \uline{name} along with the \uline{title} of the play they are \uwave{in}.}}
\end{equation}
So in a discussion of our favourite Shakespearean characters  we might 
make reference to
\begin{equation*}
\text{\textit{\rdash{Sebastian} from \rdash{Twelfth Night}.}}
\end{equation*}
\item
without any context at all, 
\begin{equation} 
\label{absoluteCharacterReferencing}
\text{\parbox{9cm}{a character from within a play 
may be referenced by their \uline{name} along with the \uline{title} of the play they are \uwave{in} and the \uline{name} of the playwright it is \uwave{written by}.}}
\end{equation}
Thus, without context, three referentials are required for the referencing of a character
as  when I reference the following character:
 \begin{equation}
\label{TouchstoneReferenceFromAbsolute}
\text{\parbox{9cm}{\textit{\rdash{Touchstone} from \rdash{Shakespeare}'s \rdash{As You Like It}. }}}
\end{equation}
\end{itemize}

\textbf{Methodologically}
Each of these three referentials corresponds to 
a path through diagram xxx from the left hand side of the diagram to the right as we see here. There are exactly three paths from left to right across the diagram. These are
\begin{align*}
d1/r1/a3 & Shakespeare \\
d1/a2    & As You Like It\\
a3       & Touchstone
\end{align*}

and each one corresponds to a referential within example
(\ref{TouchstoneReferenceFromAbsolute}) as follows
\begin{tabular}{l l}
\textbf{Path}     & \textbf{Referential} \\
d1/r1/a3 & Shakespeare \\
d1/a2    & As You Like It\\
a3       & Touchstone
\end{tabular}

At a first approximation in the absence of comparable paths in the pattern of pertinent identifying relationships the referentials required to reference entities of a type exactly correspond to such paths from left to right across the identifier diagram.


\subsection{Arrangements with Branching Paths of Identifying Relationships}
If there is a branch in the arrangement of identifying relationships emanating from a type then references to entities at the branch point will contain multiple  references nested within them; 
the Russian-doll analogy breaks down but a nested boxes analogy holds up --- we can open a box and find multiple boxes nested inside it. 


\subsubsection{Example 4 --- Productions of Plays}

\label{exampleReferencingProductions}

For a second example we move on to another area within the overall model of figure \ref{dramaticArts1..diagram}. From the following detail 
\begin{equation*}
\begin{erdiagram}{1.3}{8.2118725}

\eret{0.1}{-1.2}{1.5}{-0.3}{0.2}{1}\eretname{0.8}{-0.65}{}{play}\ergroupannotation{0.55}{-0.9}{l}{...}
\eret{3.2}{-1.2}{4.854}{-0.3}{0.2}{1}\eretname{3.365}{-0.65}{l}{production}
\erCoreAttribute{3.4}{-0.85}{1}{0}{season}{}
\eret{6.504}{-1.2}{7.904}{-0.3}{0.2}{1}\eretname{7.204}{-0.65}{}{venue}\ergroupannotation{6.979}{-0.9}{l}{...}

% relationship of
\errelname{3.05}{-0.6}{r}{of}\errelname{1.65}{-0.6}{l}{in}\errelname{1.65}{-0.3}{l}{given}\errelarm{3.2}{-0.75}{2.35}{-0.75}{1}{0}\errelarm{2.35}{-0.75}{1.5}{-0.75}{0}{0}\errelid{2.35}{-0.84}{}{r2}\ercrowfoot{3.05}{-0.75}{3.2}{-0.6}{3.2}{-0.75}{3.2}{-0.9}{0}\eridrefrel{2.95}{-0.65}{-0.85}
% relationship at
\errelname{5.004}{-0.6}{l}{at}\errelname{6.354}{-0.6}{r}{of}\errelname{6.354}{-0.3}{r}{location}\errelname{6.354}{-0}{r}{the}\errelarm{4.853}{-0.75}{5.678}{-0.75}{1}{0}\errelarm{5.678}{-0.75}{6.503}{-0.75}{0}{0}\errelid{5.679}{-0.84}{}{r3}\ercrowfoot{5.004}{-0.75}{4.854}{-0.6}{4.854}{-0.75}{4.854}{-0.9}{0}\eridrefrel{5.1035725}{-0.65}{-0.85}
\end{erdiagram}

\end{equation*}
regarding the production entity type  we see that
\begin{itemize}
  \item
  without any context, 
  \begin{equation} 
\label{absoluteCharacterReferencing}
\text{\parbox{9cm}{a production can be referenced by referencing the play it is a production \uwave{of} 
  and by giving the \underline{name} of the venue  the production is \uwave{at} along with the \underline{season} 
  over which it plays.}}
\end{equation}
\end{itemize}
We have already seen that to reference a play requires two referentials and so
we can count up the number of referentials required to reference a production
as one for the venue, two for the play and one for the season so four referentials in all. 

As an example consider the following prescription
  \begin{equation} 
\label{referenceToAsYouLikeItProduction}
\text{\parbox{9cm}{a production of \mbox{\rdash{Shakespeare}'s} \rdash{As You Like It}
 performed \mbox{\rdash{April – May 1975},} at \rdash{Oxford Playhouse}.}
 }\\
\end{equation}
As is to be expected in this description there are four referentials present.\footnote{
I found the literal description online as ``a production of the play As You Like It (by William Shakespeare), April – May 1975, at Oxford Playhouse''.  
} 


\subsection{Arrangements with Comparable Paths of Identifying Relationships}
When a type has multiple paths of identifying relationships emanating from it and when these paths intersect in the sense that there is a type in common between the paths (in addition to the
branching point) then in some cases the nested boxes analogy breaks down
because we are sometimes led to reference having nested references which overlap and are not totally independent. We give two examples one where the analogy holds up and one where it does not. The key characteristic is the presence of equivalent paths.
\commentary {Find  a home or use or rephrasing for the next para}
We need beware though because referencing doesn't always work out by simple aggregation. Because sometimes,
in the words of Shlaer and Lang, referentials collapse. \commentary{We are saying the same thing in two different ways which is confusing}
To see this we are going to move on to another area within our model and look at how dramatic roles are identified and referenced.

\begin{worktt}
Bring diagrams for the here talking showing comparable paths intersectiong. 
Potentially show the paths and the possibly commuting squares.
\end{worktt}


 \subsubsection{Example 5 --- Dramatic Role from Figure \ref{dramaticArts1..diagram}}
\mynote
In the following detail
\begin{equation*}
\begin{erdiagram}{3.8000000000000003}{5.5333000000000006}

\eret{3.6}{-0.9}{5.4}{-0}{0.2}{1}\eretname{4.5}{-0.35}{}{production}\ergroupannotation{4.45}{-0.6}{l}{...}
\eret{3.6}{-3.4}{5.4}{-2.5}{0.2}{1}\eretname{4.5}{-2.85}{}{dramatic}\eretname{4.5}{-3.15}{}{role}
\eret{0.1}{-3.4}{1.9}{-2.5}{0.2}{1}\eretname{1}{-2.85}{}{character}\ergroupannotation{0.95}{-3.1}{l}{...}

% relationship cast_with
\errelname{4.35}{-1.2}{r}{cast}\errelname{4.35}{-1.5}{r}{with}\errelname{4.65}{-2.35}{l}{in}\errelarm{4.5}{-0.9}{4.5}{-1.7}{1}{0}\errelarm{4.5}{-1.7}{4.5}{-2.5}{1}{0}\errelid{4.5}{-1.79}{}{d2}\eridcomprel{4.4}{4.6}{-2.25}\ercrowfoot{4.5}{-2.35}{4.35}{-2.5}{4.5}{-2.5}{4.65}{-2.5}{0}
% relationship the_portrayal_of
\errelname{3.45}{-2.8}{r}{of}\errelname{3.45}{-2.5}{r}{portrayal}\errelname{3.45}{-2.2}{r}{the}\errelname{2.05}{-3.25}{l}{portrayed}\errelname{2.05}{-3.55}{l}{by}\errelarm{3.6}{-2.95}{2.75}{-2.95}{1}{0}\errelarm{2.75}{-2.95}{1.9}{-2.95}{0}{0}\errelid{2.75}{-3.04}{}{r4}\ercrowfoot{3.45}{-2.95}{3.6}{-2.8}{3.6}{-2.95}{3.6}{-3.1}{0}\eridrefrel{3.35}{-2.85}{-3.0500000000000003}
\end{erdiagram}

\end{equation*}
taken from the model of the dramatic arts, as shown in figure \ref{dramaticArts1..diagram}, 
we see the type \textit{dramatic role} 
with  two identifying relationships emanating from it and having no identifying attributes;
this detail informs us that
\begin{equation}
\label{DramaticRoleReferencing}
\text{\parbox{9cm}{\textit{to reference a dramatic role it is necessary  
to reference both a character and  a production.}}}
\end{equation}

Now, as we have already discussed, \textit{character}s from plays need be referenced using three referentials
(section \ref{exampleReferencingCharacters})
and \textit{production}s need be referenced using four (section \ref{exampleReferencingProductions})
and so, from this, we can expect that, without context, a reference to
a dramatic role will require seven referentials
 --- three for the character and four for the production. 
So we will need seven, right?
 Well no, rather surprisingly, not so --- only five referentials are required. 
 What happened to the other two? Can you figure this out, can you answer the question?
In what follows, we will explore the answer from a number of points of view;
one of them is mathematical in nature --- this point of view may or may not suit you depending on your background.

First though, that only five referentials are needed can be seen from this reference to a dramatic role in which I have dash underlined the referentials:
\begin{equation}
\label{theRoleOfTouchstone}
\text{
\parbox{8.0cm}{the role of \rdash{Touchstone} in the production of \mbox{\rdash{Shakespeare}'s} \rdash{As You Like It} performed \mbox{\rdash{April – May 1975},} at \rdash{Oxford Playhouse}.
}
}
\end{equation}

To see why five only we have to spell out our declaration, (\ref{DramaticRoleReferencing}), in more detail
and make something clear that couldn't be inferred from the diagram alone,  namely that
\begin{equation}
\label{DramaticRoleReferencingRevised}
\text{\parbox{9cm}{to reference a dramatic role it is necessary 
to both reference a character \textbf{from a play} and to reference a production \textbf{of that same play}.}}
\end{equation}
In this restatement of (\ref{DramaticRoleReferencing}) as (\ref{DramaticRoleReferencingRevised}) we make clear that in a reference to a dramatic role, the referenced character and the referenced production will be in relationship with the very same play
and that, consequently,  in referencing a dramatic role, the referencing of the character and the referencing of
the production will not be independent in that
each of these two references --- that to a character and that to a production ---
must require a subordinate reference to the very same play;  
these two, required subordinate references collapse to a single reference(see figure \ref{theRoleOfTouchstoneAnalysed}) --- 
the subordinate reference to a play need not be duplicated. 
Finally, as we have seen, a reference to a play consist of two referentials;
by requiring this subordinate reference only once, we save two referentials, which explains why only five  are required, not seven.

If we were to violate prescription (\ref{DramaticRoleReferencingRevised})
 --- for instance, by claiming that in a given season an actor appeared in a production
of  Ibsen's \textit{A Doll's House} at the Royal Exchange, Manchester, playing the role of Touchstone from Shakespeare's \textit{As You like It} --- the result would not merely be incorrect but nonsensical. If a data structure allowed such prescriptions as this then it would be said that the defining structure was not in normal-form ---  which in this case at least is as an innocent-looking way for saying admits of pure nonsense.

Situations like this were noticed long ago by Shlaer and Lang, 
who coined the term ‘collapsed referentials’.\footnote{Note that we are talking about referentials 
which \textit{a priori} collapse,
rather than contingently as when we a particular marriage might be identified as being between parties \textbf{of the same parish}}. They were working with tabular structures, whereas here we are exploring the phenomenon through the lens of phrasal structure. We will be going into formal detail, but for now the concept gives a sense of why the counting of referentials isn’t always straightforward. Later we will return to this idea and also introduce the related notion of ‘absent referentials’.

\begin{erboxedFigure}{H}{theRoleOfTouchstoneAnalysed}
{
A reference to a dramatic role has two nested references within it. 
One to an entity of type character: 
\begin{equation*}
\text{
T. in  Shakespeare's As You Like It
}
\end{equation*}
and one to an entity of type production, namely that of:
\begin{equation*}
\text{
\parbox{9.0cm}{Shakespeare's As You Like It performed Spring '75 at the OP.}
}
\end{equation*}
 In this example the nested references overlap. The overlap is Shakespeare's As You Like It --- a reference to a play that is subordinate to both of the two nested references.
}
\newcommand{\dashRefOne}{2pt 2pt}
\newcommand{\dashRelationship}{1pt 0pt}
\newcommand{\dashRefTwo}{1pt 1pt}
\begin{tabular}{l}
the role of 
\Rnode{w1}{\rdash{T}} in 
\Rnode{w2}{\rdot{\rdash{Shakespeare}}}’s 
\Rnode{w3}{\rdot{\rdash{As You Like It}}} performed 
\Rnode{w4}{\rdot{Spring '75}}, at 
\Rnode{w5}{\rdot{OP}} \\[1.4cm]
\kern2cm\Rnode{ref1}{\parbox[t]{1.95cm}{\textit{reference to entity of type character}}}
\kern3.0cm\Rnode{ref2}{\parbox[t]{1.95cm}{\textit{reference to entity of type production}}} \\[0.5cm]
\syntag{\dashRefOne}{ref1}{0.9}{w1}{0}
\syntag{\dashRefOne}{ref1}{0.9}{w2}{-0.2}
\syntag{\dashRefOne}{ref1}{0.9}{w3}{-0.2}
\syntag{\dashRefTwo}{ref2}{0.4}{w2}{0.2}
\syntag{\dashRefTwo}{ref2}{0.4}{w3}{0.3}
\syntag{\dashRefTwo}{ref2}{0.4}{w4}{0.3}
\syntag{\dashRefTwo}{ref2}{0.4}{w5}{0}
\end{tabular}
\end{erboxedFigure}


\mynote
\begin{worktt}
Now a crucial observation,\commentary{This might move in situ after the observation if we have the square upfront (see apricot)} 
We can consider that a character and a production can include reference to ``the same play'' injunction  just because in this diagram 
there are two comparable paths from the bottom right to the top left, from entity type dramatic role to entity type play, 
\end{worktt}

\begin{equation}
\label{dramaticArtsPortrayalScopeFragment..diagram}
\begin{erdiagram}{4.5}{4.892135}

\eret{0.1}{-1.4}{1.433}{-0.5}{0.2}{1}\eretname{0.233}{-0.85}{l}{play}
\erCoreAttribute{0.3}{-1.05}{1}{0}{title}{}
\eret{0.036}{-3.9}{1.498}{-3}{0.2}{1}\eretname{0.182}{-3.35}{l}{character}
\erCoreAttribute{0.236}{-3.55}{1}{0}{name}{}
\eret{3.133}{-1.4}{4.787}{-0.5}{0.2}{1}\eretname{3.299}{-0.85}{l}{production}
\erCoreAttribute{3.333}{-1.05}{1}{0}{season}{}
\eret{3.428}{-3.9}{4.892}{-3}{0.2}{1}\eretname{4.16}{-3.35}{}{dramatic}\eretname{4.16}{-3.65}{}{role}

% relationship about
\errelname{0.917}{-1.7}{l}{about}\errelname{0.917}{-2.85}{l}{in}\errelarm{0.766}{-1.4}{0.766}{-2.2}{0}{0}\errelarm{0.766}{-2.2}{0.766}{-3}{1}{0}\errelid{0.767}{-2.29}{}{d1}\ercrowfoot{0.767}{-2.85}{0.617}{-3}{0.767}{-3}{0.917}{-3}{0}
% relationship cast_with
\errelname{4.01}{-1.7}{r}{cast}\errelname{4.01}{-2}{r}{with}\errelname{4.31}{-2.85}{l}{in}\errelarm{4.16}{-1.4}{4.16}{-2.2}{0}{0}\errelarm{4.16}{-2.2}{4.16}{-3}{1}{0}\errelid{4.16}{-2.29}{}{d2}\ercrowfoot{4.16}{-2.85}{4.01}{-3}{4.16}{-3}{4.31}{-3}{0}
% relationship of
\errelname{2.983}{-0.8}{r}{of}\errelname{1.583}{-0.8}{l}{in}\errelname{1.583}{-0.5}{l}{given}\errelarm{3.133}{-0.95}{2.283}{-0.95}{1}{0}\errelarm{2.283}{-0.95}{1.433}{-0.95}{0}{0}\errelid{2.283}{-1.04}{}{r2}\ercrowfoot{2.983}{-0.95}{3.133}{-0.8}{3.133}{-0.95}{3.133}{-1.1}{0}\eridrefrel{2.8833}{-0.85}{-1.05}
% relationship the_portrayal_of
\errelname{3.278}{-3.75}{r}{the}\errelname{3.278}{-4.05}{r}{portrayal}\errelname{3.278}{-4.35}{r}{of}\errelname{1.648}{-3.3}{l}{by}\errelname{1.648}{-3}{l}{portrayed}\errelarm{3.428}{-3.45}{2.462}{-3.45}{1}{0}\errelarm{2.462}{-3.45}{1.497}{-3.45}{0}{0}\errelid{2.463}{-3.54}{}{r4}\ercrowfoot{3.278}{-3.45}{3.428}{-3.3}{3.428}{-3.45}{3.428}{-3.6}{0}\eridrefrel{3.1780375}{-3.35}{-3.5500000000000003}
\end{erdiagram}

\end{equation}
of taken from figure \ref{dramaticArts1..diagram}.

The ``in the same play'' injunction is necessary 
and appropriate just because 
the two comparable paths f are equivalent.


\begin{reinstatett}
\mynote
However we look at it, if xxxx then The conclusion is that an entity model is more than is shown in the diagram of relationships. It needs include additional information... such as, in the declaration above,
 is captured by qualifying as ``in the same play''. Lets come back to this in a moment. First a contrasting example.
 \end{reinstatett} 

\subsubsection {Example 6 --- Route City State}

\mynote 
Some comparable paths are not equivalent and when they are not equivalent then subreferences do not as a matter of course collapse. For example consider the following reference
 \begin{equation*}
 \mbox{the route from Hot Springs, Arkansas to Jacksonville, Alabama.}
\end{equation*}

The types, relationships and identifying features that are instantiated here
I would sketch out as follows

\iffalse
\begin{equation*}
\begin{erdiagram}{6.299999999999999}{2.55}

\eret{0.3}{-1}{2.05}{-0.1}{0.2}{1}\eretname{0.475}{-0.45}{l}{state}
\erCoreAttribute{0.5}{-0.65}{1}{0}{name}{}
\eret{0.3}{-3.8}{2.05}{-2.9}{0.2}{1}\eretname{0.475}{-3.25}{l}{city}
\erCoreAttribute{0.5}{-3.45}{1}{0}{name}{}
\eret{0.3}{-6.3}{2.05}{-5.7}{0.2}{1}\eretname{1.175}{-6.05}{}{route}

% relationship location_of
\errelname{1.325}{-1.3}{l}{location}\errelname{1.325}{-1.6}{l}{of}\errelname{1.025}{-2.75}{r}{within}\errelname{1.025}{-2.45}{r}{located}\errelarm{1.175}{-0.999}{1.175}{-1.95}{0}{0}\errelarm{1.175}{-1.95}{1.175}{-2.9}{1}{0}\errelid{1.175}{-2.04}{}{d1}\eridcomprel{1.075}{1.2750000000000001}{-2.65}\ercrowfoot{1.175}{-2.75}{1.025}{-2.9}{1.175}{-2.9}{1.325}{-2.9}{0}
% relationship start_of
\errelname{0.733}{-4.1}{r}{start}\errelname{0.733}{-4.4}{r}{of}\errelname{0.733}{-5.55}{r}{from}\errelarm{0.883}{-3.8}{0.883}{-4.75}{0}{0}\errelarm{0.883}{-4.75}{0.883}{-5.699}{1}{0}\errelid{0.883}{-4.84}{}{d2}\eridcomprel{0.7833333333333333}{0.9833333333333333}{-5.449999999999999}\ercrowfoot{0.883}{-5.55}{0.733}{-5.7}{0.883}{-5.7}{1.033}{-5.7}{0}
% relationship end_of
\errelname{1.617}{-4.1}{l}{end}\errelname{1.617}{-4.4}{l}{of}\errelname{1.617}{-5.55}{l}{to}\errelarm{1.466}{-3.8}{1.466}{-4.75}{0}{0}\errelarm{1.466}{-4.75}{1.466}{-5.699}{1}{0}\errelid{1.467}{-4.84}{}{d3}\eridcomprel{1.3666666666666665}{1.5666666666666667}{-5.449999999999999}\ercrowfoot{1.467}{-5.55}{1.317}{-5.7}{1.467}{-5.7}{1.617}{-5.7}{0}
\end{erdiagram}

\end{equation*}
\fi
\begin{equation*}
\begin{erdiagram}{2.5}{10.25}

\eret{8}{-2.5}{9.75}{-1.3}{0.2}{1}\eretname{8.175}{-1.65}{l}{state}
\erCoreAttribute{8.2}{-1.85}{1}{0}{name}{}
\eret{4.05}{-2.5}{5.8}{-1.3}{0.2}{1}\eretname{4.225}{-1.65}{l}{city}
\erCoreAttribute{4.25}{-1.85}{1}{0}{name}{}
\eret{0.1}{-2.5}{1.85}{-1.3}{0.2}{1}\eretname{0.975}{-1.65}{}{route}

% relationship located_within
\errelname{5.95}{-1.75}{l}{within}\errelname{5.95}{-1.45}{l}{located}\errelarm{5.8}{-1.9}{6.9}{-1.9}{1}{0}\errelarm{6.9}{-1.9}{8}{-1.9}{0}{0}\errelid{6.9}{-1.99}{}{d1}\ercrowfoot{5.95}{-1.9}{5.8}{-1.75}{5.8}{-1.9}{5.8}{-2.05}{0}\eridrefrel{6.050000000000001}{-1.7999999999999998}{-2}
% relationship from
\errelname{2}{-1.51}{l}{from}\errelarm{1.85}{-1.66}{2.95}{-1.66}{1}{0}\errelarm{2.95}{-1.66}{4.05}{-1.66}{0}{0}\errelid{2.95}{-1.75}{}{d2}\ercrowfoot{2}{-1.66}{1.85}{-1.51}{1.85}{-1.66}{1.85}{-1.81}{0}\eridrefrel{2.1}{-1.56}{-1.7600000000000002}
% relationship to
\errelname{2}{-2.44}{l}{to}\errelarm{1.85}{-2.14}{2.95}{-2.14}{1}{0}\errelarm{2.95}{-2.14}{4.05}{-2.14}{0}{0}\errelid{2.95}{-2.23}{}{d3}\ercrowfoot{2}{-2.14}{1.85}{-1.99}{1.85}{-2.14}{1.85}{-2.29}{0}\eridrefrel{2.1}{-2.04}{-2.24}
\end{erdiagram}

\end{equation*}


\mynote 
As before in example () the reference () has two nested references and these themselves have a common nested reference namely
\begin{equation*}
\mbox{the state Arkansas}
\end{equation*}
The difference between this example and the previous one is that the commonality of the secondary nested reference is not  \textit{a priori}
 common to the two primaries --- in this example

\mynote
WE noted earlier that commonality of the secondary nested references was guaranteed in the earlier example because of the equivalence of two paths or, in other words the commutivity of a certain diagram (diagram ()).
In this second example commonality is not guaranteed. Equally we can say that the following two comparable paths are not equivalent:

Another ways of saying this is by saying that the following diagram does not commute:
\iffalse
\begin{equation*}
\begin{erdiagram}{5.699999999999999}{5.6}

\eret{2.3}{-0.7}{3.8}{-0.1}{0.2}{1}\eretname{3.05}{-0.45}{}{state}
\eret{1}{-3.2}{2.5}{-2.6}{0.2}{1}\eretname{1.75}{-2.95}{}{city}
\eret{3.6}{-3.2}{5.1}{-2.6}{0.2}{1}\eretname{4.35}{-2.95}{}{city }
\eret{2.175}{-5.7}{3.925}{-5.1}{0.2}{1}\eretname{3.05}{-5.45}{}{route}

% relationship location_of
\errelname{1.6}{-2.45}{r}{within}\errelname{1.6}{-2.15}{r}{located}\errelarm{2.8}{-0.7}{2.8}{-0.899}{0}{0}\errelarm{2.8}{-0.899}{2.8}{-1.1}{0}{0}\errelarm{2.8}{-1.1}{2.275}{-1.637}{0}{0}\errelarm{2.275}{-1.637}{1.749}{-2.175}{1}{0}\errelarm{1.749}{-2.175}{1.749}{-2.387}{1}{0}\errelarm{1.749}{-2.387}{1.749}{-2.599}{1}{0}\errelid{2.275}{-1.728}{}{d1}\eridcomprel{1.6499999999999997}{1.8499999999999999}{-2.3499999999999996}\ercrowfoot{1.75}{-2.45}{1.6}{-2.6}{1.75}{-2.6}{1.9}{-2.6}{0}
% relationship location_of 
\errelname{4.5}{-2.45}{l}{within}\errelname{4.5}{-2.15}{l}{located}\errelarm{3.3}{-0.7}{3.3}{-0.899}{0}{0}\errelarm{3.3}{-0.899}{3.3}{-1.1}{0}{0}\errelarm{3.3}{-1.1}{3.824}{-1.637}{0}{0}\errelarm{3.824}{-1.637}{4.35}{-2.175}{1}{0}\errelarm{4.35}{-2.175}{4.35}{-2.387}{1}{0}\errelarm{4.35}{-2.387}{4.35}{-2.599}{1}{0}\errelid{3.825}{-1.728}{}{d2}\eridcomprel{4.25}{4.449999999999999}{-2.3499999999999996}\ercrowfoot{4.35}{-2.45}{4.2}{-2.6}{4.35}{-2.6}{4.5}{-2.6}{0}
% relationship start_of
\errelname{2.55}{-4.95}{r}{from}\errelarm{1.749}{-3.199}{1.749}{-3.4}{0}{0}\errelarm{1.749}{-3.4}{1.749}{-3.599}{0}{0}\errelarm{1.749}{-3.599}{2.224}{-4.137}{0}{0}\errelarm{2.224}{-4.137}{2.699}{-4.675}{1}{0}\errelarm{2.699}{-4.675}{2.699}{-4.887}{1}{0}\errelarm{2.699}{-4.887}{2.699}{-5.1}{1}{0}\errelid{2.225}{-4.227}{}{d3}\eridcomprel{2.5999999999999996}{2.8}{-4.85}\ercrowfoot{2.7}{-4.95}{2.55}{-5.1}{2.7}{-5.1}{2.85}{-5.1}{0}
% relationship end_of
\errelname{3.55}{-4.95}{l}{to}\errelarm{4.35}{-3.199}{4.35}{-3.4}{0}{0}\errelarm{4.35}{-3.4}{4.35}{-3.599}{0}{0}\errelarm{4.35}{-3.599}{3.874}{-4.137}{0}{0}\errelarm{3.874}{-4.137}{3.399}{-4.675}{1}{0}\errelarm{3.399}{-4.675}{3.399}{-4.887}{1}{0}\errelarm{3.399}{-4.887}{3.399}{-5.1}{1}{0}\errelid{3.875}{-4.227}{}{d4}\eridcomprel{3.2999999999999994}{3.4999999999999996}{-4.85}\ercrowfoot{3.4}{-4.95}{3.25}{-5.1}{3.4}{-5.1}{3.55}{-5.1}{0}
\end{erdiagram}

\end{equation*}
\fi

\begin{equation*}
\begin{erdiagram}{4.3}{8.9}

\eret{7.3}{-2.5}{8.8}{-1.5}{0.2}{1}\eretname{8.05}{-1.85}{}{state}
\eret{3.7}{-1.2}{5.2}{-0.2}{0.2}{1}\eretname{4.45}{-0.55}{}{city}
\eret{3.7}{-3.9}{5.2}{-2.9}{0.2}{1}\eretname{4.45}{-3.25}{}{city }
\eret{0.1}{-2.5}{1.6}{-1.5}{0.2}{1}\eretname{0.85}{-1.85}{}{route}

% relationship located_within
\errelname{5.35}{-0.55}{l}{within}\errelname{5.35}{-0.25}{l}{located}\errelarm{5.2}{-0.7}{5.45}{-0.7}{1}{0}\errelarm{5.45}{-0.7}{5.7}{-0.7}{1}{0}\errelarm{5.7}{-0.7}{6.3}{-1.25}{1}{0}\errelarm{6.3}{-1.25}{6.9}{-1.8}{0}{0}\errelarm{6.9}{-1.8}{7.1}{-1.8}{0}{0}\errelarm{7.1}{-1.8}{7.3}{-1.8}{0}{0}\errelid{6.3}{-1.34}{}{d1}\ercrowfoot{5.35}{-0.7}{5.2}{-0.55}{5.2}{-0.7}{5.2}{-0.85}{0}\eridrefrel{5.45}{-0.6}{-0.7999999999999999}
% relationship located_within
\errelname{5.35}{-3.7}{l}{located}\errelname{5.35}{-4}{l}{within}\errelarm{5.2}{-3.4}{5.45}{-3.4}{1}{0}\errelarm{5.45}{-3.4}{5.7}{-3.4}{1}{0}\errelarm{5.7}{-3.4}{6.3}{-2.8}{1}{0}\errelarm{6.3}{-2.8}{6.9}{-2.2}{0}{0}\errelarm{6.9}{-2.2}{7.1}{-2.2}{0}{0}\errelarm{7.1}{-2.2}{7.3}{-2.2}{0}{0}\errelid{6.3}{-2.89}{}{d2}\ercrowfoot{5.35}{-3.4}{5.2}{-3.25}{5.2}{-3.4}{5.2}{-3.55}{0}\eridrefrel{5.45}{-3.3}{-3.5}
% relationship from
\errelname{1.75}{-1.65}{l}{from}\errelarm{1.6}{-1.8}{1.85}{-1.8}{1}{0}\errelarm{1.85}{-1.8}{2.1}{-1.8}{1}{0}\errelarm{2.1}{-1.8}{2.7}{-1.25}{1}{0}\errelarm{2.7}{-1.25}{3.3}{-0.7}{0}{0}\errelarm{3.3}{-0.7}{3.5}{-0.7}{0}{0}\errelarm{3.5}{-0.7}{3.7}{-0.7}{0}{0}\errelid{2.7}{-1.34}{}{d3}\ercrowfoot{1.75}{-1.8}{1.6}{-1.65}{1.6}{-1.8}{1.6}{-1.95}{0}\eridrefrel{1.85}{-1.7}{-1.9000000000000001}
% relationship to
\errelname{1.75}{-2.5}{l}{to}\errelarm{1.6}{-2.2}{1.85}{-2.2}{1}{0}\errelarm{1.85}{-2.2}{2.1}{-2.2}{1}{0}\errelarm{2.1}{-2.2}{2.7}{-2.8}{1}{0}\errelarm{2.7}{-2.8}{3.3}{-3.4}{0}{0}\errelarm{3.3}{-3.4}{3.5}{-3.4}{0}{0}\errelarm{3.5}{-3.4}{3.7}{-3.4}{0}{0}\errelid{2.7}{-2.89}{}{d4}\ercrowfoot{1.75}{-2.2}{1.6}{-2.05}{1.6}{-2.2}{1.6}{-2.35}{0}\eridrefrel{1.85}{-2.1}{-2.3000000000000003}
\end{erdiagram}

\end{equation*}

\textbf{Contingency versus \textit{a priori}}

Though in this model there are routes between cities in different states in some cases cities may well be in the same state nonettheless the reference to the route may well collapse the shared referential.
\begin{equation}
\mbox{the route from Hot Springs, Arkansas to Jacksonville, Arkansas}
\end{equation}
may be rephrased with collapsed referentials for the state such as by saying
\begin{equation}
\mbox{the route from Hot Springs, Arkansas to Jacksonville in the same state.}
\end{equation}
or even 
\begin{equation}
\mbox{the route in Arkansas from Hot Springs to Jacksonville.}
\end{equation} 

Since the collapsing only is possible in some cases we might say that it is \textit{contingent} rather than \textit{a priori}. The latter come about because of equivalent paths, the former are accidental, so to speak.


\subsection{The Methodological Derivation Of Referentials}
\subsubsection{Example 3 Revisited --- Referencing Characters from Plays}
\begin{worktt} 
\begin{equation*}
\begin{erdiagram}{3.0500000000000003}{6.799999999999999}

\eretsides{4.7}{-1.65}{4.75}{-0.35}\eretname{4.725}{-0.3}{}{playwright}
\eretsides{2.65}{-2.45}{2.7}{-0.35}\eretname{2.675}{-0.3}{}{play}
\eretsides{0.6}{-3.05}{0.65}{-0.35}\eretname{0.625}{-0.3}{}{character}
\eretsides{6.75}{-2.85}{6.8}{-0.35}\eretname{6.775}{-0.3}{}{*}

% relationship name
\errelname{4.9}{-0.85}{l}{name}\errelarm{4.749}{-1}{5.749}{-1}{0}{0}\errelarm{5.749}{-1}{6.749}{-1}{0}{0}\errelid{5.75}{-1.09}{}{a3}\ercrowfoot{4.9}{-1}{4.75}{-0.85}{4.75}{-1}{4.75}{-1.15}{0}\eridrefrel{4.999999999999999}{-0.9}{-1.1}
% relationship written_by
\errelname{2.85}{-0.85}{l}{by}\errelname{2.85}{-0.55}{l}{written}\errelarm{2.699}{-1}{3.699}{-1}{1}{0}\errelarm{3.699}{-1}{4.699}{-1}{0}{0}\errelid{3.7}{-1.09}{}{r1}\ercrowfoot{2.85}{-1}{2.7}{-0.85}{2.7}{-1}{2.7}{-1.15}{0}\eridrefrel{2.9499999999999997}{-0.9}{-1.1}
% relationship title
\errelname{2.85}{-1.88}{l}{title}\errelarm{2.699}{-2.03}{4.725}{-2.03}{0}{0}\errelarm{4.725}{-2.03}{6.749}{-2.03}{0}{0}\errelid{4.725}{-2.12}{}{a2}\ercrowfoot{2.85}{-2.03}{2.7}{-1.88}{2.7}{-2.03}{2.7}{-2.18}{0}\eridrefrel{2.9499999999999997}{-1.9300000000000002}{-2.1300000000000003}
% relationship in
\errelname{0.8}{-1.213}{l}{in}\errelarm{0.65}{-1.362}{1.65}{-1.362}{1}{0}\errelarm{1.65}{-1.362}{2.65}{-1.362}{0}{0}\errelid{1.65}{-1.453}{}{d1}\ercrowfoot{0.8}{-1.363}{0.65}{-1.213}{0.65}{-1.363}{0.65}{-1.513}{0}\eridrefrel{0.9}{-1.2625000000000002}{-1.4625000000000004}
% relationship name
\errelname{0.8}{-2.495}{l}{name}\errelarm{0.65}{-2.645}{3.699}{-2.645}{0}{0}\errelarm{3.699}{-2.645}{6.749}{-2.645}{0}{0}\errelid{3.7}{-2.735}{}{a1}\ercrowfoot{0.8}{-2.645}{0.65}{-2.495}{0.65}{-2.645}{0.65}{-2.795}{0}\eridrefrel{0.9}{-2.545}{-2.745}
\end{erdiagram}

\end{equation*}
\end{worktt}

\subsubsection{Example 4 Revisited --- Referencing Productions of Plays}
\begin{worktt}
We should point out that each of the four referentials corresponds to a path from left to right across the  diagram shown in figure \ref{productionIdentificationScheme..diagram}.

\begin{erboxedFigure}{H}{productionIdentificationScheme..diagram}
{Identification scheme diagram for entity type production. 
The paths from left to right across this diagram determine the four referentials 
that must be provided to reference an entity of type production
according to the model in figure \ref{dramaticArts1..diagram}.
It is necessary, therefore,  to provide a total of four referentials
to correspond to the paths
\begin{center}
\begin{tabular}{p{3cm}}
$r2 \comp r1 \comp a3$ \\
$r2 \comp a2$ \\
$a4$          \\
$r3 \comp a5$ \\
\end{tabular}
\end{center}
}
 \begin{equation*}
\begin{erdiagram}{3.0900000000000003}{10.4}

\eret{0.1}{-2.3}{1.6}{-1.1}{0.2}{1}\eretname{0.25}{-1.45}{l}{production}
\erCoreAttribute{0.3}{-1.65}{1}{0}{season}{}
\eret{4.5}{-1.27}{6}{-0.35}{0.2}{1}\eretname{4.65}{-0.7}{l}{play}
\erCoreAttribute{4.7}{-0.9}{1}{0}{title}{}
\eret{8.9}{-1.27}{10.4}{-0.35}{0.2}{1}\eretname{9.05}{-0.7}{l}{playwright}
\erCoreAttribute{9.1}{-0.9}{1}{0}{name}{}
\eret{4.5}{-3.09}{6}{-2.17}{0.2}{1}\eretname{4.65}{-2.52}{l}{venue}
\erCoreAttribute{4.7}{-2.72}{1}{0}{name}{}

% relationship of
\errelname{1.75}{-1.31}{l}{of}\errelarm{1.6}{-1.46}{1.975}{-1.46}{1}{0}\errelarm{1.975}{-1.46}{2.35}{-1.46}{1}{0}\errelarm{2.35}{-1.46}{3.224}{-1.135}{1}{0}\errelarm{3.224}{-1.135}{4.1}{-0.81}{0}{0}\errelarm{4.1}{-0.81}{4.3}{-0.81}{0}{0}\errelarm{4.3}{-0.81}{4.5}{-0.81}{0}{0}\errelid{3.225}{-1.225}{}{r2}\ercrowfoot{1.75}{-1.46}{1.6}{-1.31}{1.6}{-1.46}{1.6}{-1.61}{0}\eridrefrel{1.85}{-1.3599999999999999}{-1.56}
% relationship at
\errelname{1.75}{-2.24}{l}{at}\errelarm{1.6}{-1.94}{1.975}{-1.94}{1}{0}\errelarm{1.975}{-1.94}{2.35}{-1.94}{1}{0}\errelarm{2.35}{-1.94}{3.224}{-2.285}{1}{0}\errelarm{3.224}{-2.285}{4.1}{-2.63}{0}{0}\errelarm{4.1}{-2.63}{4.3}{-2.63}{0}{0}\errelarm{4.3}{-2.63}{4.5}{-2.63}{0}{0}\errelid{3.225}{-2.375}{}{r3}\ercrowfoot{1.75}{-1.94}{1.6}{-1.79}{1.6}{-1.94}{1.6}{-2.09}{0}\eridrefrel{1.85}{-1.8399999999999999}{-2.04}
% relationship written_by
\errelname{6.15}{-0.66}{l}{by}\errelname{6.15}{-0.36}{l}{written}\errelarm{6}{-0.81}{7.45}{-0.81}{1}{0}\errelarm{7.45}{-0.81}{8.9}{-0.81}{0}{0}\errelid{7.45}{-0.9}{}{r1}\ercrowfoot{6.15}{-0.81}{6}{-0.66}{6}{-0.81}{6}{-0.96}{0}\eridrefrel{6.25}{-0.7100000000000001}{-0.91}
\end{erdiagram}

\end{equation*}
\end{erboxedFigure}

The referentials that instantiate these paths in 
the reference to a production of 
\textit{As You Like It} in (\ref{referenceToAsYouLikeItProduction}) 
are\\
\newline
\begin{center}
\begin{tabular}{l l}
\textbf{Path}          & \textbf{Referential} \\
$r2 \comp r1 \comp a3$ & ``Shakespeare''       \\
$r2\comp a2$           & ``As You Like It''   \\
$a4$                   & ``April-May 1975''   \\
$r3 \comp a5$          & ``Oxford Playhouse''
\end{tabular}
\end{center}
\end{worktt}

\subsubsection{Example 5 Revisited --- Referencing Dramatic Roles}
\begin{worktt} 
\begin{erboxedFigure}{H}{dramaticArtsRole..identificationScheme..diagram}
{Identification Scheme diagram for the dramatic role type.
from the model in figure \ref{dramaticArts1..diagram} ---
There are seven ways of getting from the left hand side of this diagram to the right hand side. Naively, this implies seven referentials are required to reference a dramatic role, these corresponding to the following paths from left to right across the diagram:
\begin{displaymath}
\begin{array}{l c l}
\text{path 1:} &  & r4 \comp a1 \\
\text{path 2:} & & r4 \comp d1 \comp a2 \\
\text{path 3:} & & r4 \comp d1 \comp r1 \comp a3 \\
\text{path 4:} &  & d2 \comp d1 \comp a2 \\
\text{path 5:} &  & d2 \comp d1 \comp r1 \comp a3 \\
\text{path 6:} &  & d2 \comp a4 \\
\text{path 7:} &  & d2 \comp r3 \comp a5
\end{array}
\end{displaymath}


Could speak of collapsed referentials.
Could later add a section on algebra and derived two equivalences from one equivalence.
} %end of caption
\begin{equation*}
\begin{erdiagram}{4}{11.974836250000001}

\eret{3.574}{-1.07}{5.074}{-0.15}{0.2}{1}\eretname{3.724}{-0.5}{l}{character}
\erCoreAttribute{3.774}{-0.7}{1}{0}{name}{}
\eret{3.574}{-3.02}{5.074}{-1.82}{0.2}{1}\eretname{3.724}{-2.17}{l}{production}
\erCoreAttribute{3.774}{-2.37}{1}{0}{season}{}
\eret{0.1}{-2.1}{1.564}{-0.9}{0.2}{1}\eretname{0.832}{-1.25}{}{dramatic}\eretname{0.832}{-1.55}{}{role}
\eret{7.084}{-2.1}{8.384}{-0.9}{0.2}{1}\eretname{7.214}{-1.25}{l}{play}
\erCoreAttribute{7.284}{-1.45}{1}{0}{title}{}
\eret{10.284}{-1.968}{11.975}{-1.048}{0.2}{1}\eretname{10.453}{-1.398}{l}{playwright}
\erCoreAttribute{10.484}{-1.598}{1}{0}{name}{}
\eret{7.084}{-4}{8.384}{-3}{0.2}{1}\eretname{7.214}{-3.35}{l}{venue}
\erCoreAttribute{7.284}{-3.55}{1}{0}{name}{}

% relationship in
\errelname{5.224}{-0.46}{l}{in}\errelarm{5.074}{-0.61}{5.349}{-0.61}{1}{0}\errelarm{5.349}{-0.61}{5.624}{-0.61}{1}{0}\errelarm{5.624}{-0.61}{6.154}{-0.935}{1}{0}\errelarm{6.154}{-0.935}{6.684}{-1.26}{0}{0}\errelarm{6.684}{-1.26}{6.884}{-1.26}{0}{0}\errelarm{6.884}{-1.26}{7.084}{-1.26}{0}{0}\errelid{6.154}{-1.025}{}{d1}\ercrowfoot{5.224}{-0.61}{5.074}{-0.46}{5.074}{-0.61}{5.074}{-0.76}{0}\eridrefrel{5.324097500000001}{-0.5100000000000001}{-0.7100000000000001}
% relationship of
\errelname{5.224}{-2.03}{l}{of}\errelarm{5.074}{-2.18}{5.349}{-2.18}{1}{0}\errelarm{5.349}{-2.18}{5.624}{-2.18}{1}{0}\errelarm{5.624}{-2.18}{6.154}{-1.96}{1}{0}\errelarm{6.154}{-1.96}{6.684}{-1.74}{0}{0}\errelarm{6.684}{-1.74}{6.884}{-1.74}{0}{0}\errelarm{6.884}{-1.74}{7.084}{-1.74}{0}{0}\errelid{6.154}{-2.05}{}{r2}\ercrowfoot{5.224}{-2.18}{5.074}{-2.03}{5.074}{-2.18}{5.074}{-2.33}{0}\eridrefrel{5.324097500000001}{-2.08}{-2.2800000000000002}
% relationship at
\errelname{5.224}{-2.96}{l}{at}\errelarm{5.074}{-2.66}{5.349}{-2.66}{1}{0}\errelarm{5.349}{-2.66}{5.624}{-2.66}{1}{0}\errelarm{5.624}{-2.66}{6.154}{-3.08}{1}{0}\errelarm{6.154}{-3.08}{6.684}{-3.5}{0}{0}\errelarm{6.684}{-3.5}{6.884}{-3.5}{0}{0}\errelarm{6.884}{-3.5}{7.084}{-3.5}{0}{0}\errelid{6.154}{-3.17}{}{r3}\ercrowfoot{5.224}{-2.66}{5.074}{-2.51}{5.074}{-2.66}{5.074}{-2.81}{0}\eridrefrel{5.324097500000001}{-2.56}{-2.7600000000000002}
% relationship in
\errelname{1.714}{-2.04}{l}{in}\errelarm{1.564}{-1.74}{1.839}{-1.74}{1}{0}\errelarm{1.839}{-1.74}{2.114}{-1.74}{1}{0}\errelarm{2.114}{-1.74}{2.644}{-2.08}{1}{0}\errelarm{2.644}{-2.08}{3.174}{-2.42}{0}{0}\errelarm{3.174}{-2.42}{3.374}{-2.42}{0}{0}\errelarm{3.374}{-2.42}{3.574}{-2.42}{0}{0}\errelid{2.644}{-2.17}{}{d2}\ercrowfoot{1.714}{-1.74}{1.564}{-1.59}{1.564}{-1.74}{1.564}{-1.89}{0}\eridrefrel{1.8140975000000004}{-1.64}{-1.84}
% relationship the_portrayal_of
\errelname{1.714}{-1.01}{l}{of}\errelname{1.714}{-0.71}{l}{portrayal}\errelname{1.714}{-0.41}{l}{the}\errelarm{1.564}{-1.26}{1.839}{-1.26}{1}{0}\errelarm{1.839}{-1.26}{2.114}{-1.26}{1}{0}\errelarm{2.114}{-1.26}{2.644}{-0.935}{1}{0}\errelarm{2.644}{-0.935}{3.174}{-0.61}{0}{0}\errelarm{3.174}{-0.61}{3.374}{-0.61}{0}{0}\errelarm{3.374}{-0.61}{3.574}{-0.61}{0}{0}\errelid{2.644}{-1.025}{}{r4}\ercrowfoot{1.714}{-1.26}{1.564}{-1.11}{1.564}{-1.26}{1.564}{-1.41}{0}\eridrefrel{1.8140975000000004}{-1.16}{-1.36}
% relationship written_by
\errelname{8.534}{-1.35}{l}{by}\errelname{8.534}{-1.05}{l}{written}\errelarm{8.384}{-1.5}{8.659}{-1.5}{1}{0}\errelarm{8.659}{-1.5}{8.934}{-1.5}{1}{0}\errelarm{8.934}{-1.5}{9.409}{-1.504}{1}{0}\errelarm{9.409}{-1.504}{9.884}{-1.508}{0}{0}\errelarm{9.884}{-1.508}{10.08}{-1.508}{0}{0}\errelarm{10.08}{-1.508}{10.28}{-1.508}{0}{0}\errelid{9.409}{-1.594}{}{r1}\ercrowfoot{8.534}{-1.5}{8.384}{-1.35}{8.384}{-1.5}{8.384}{-1.65}{0}\eridrefrel{8.634097500000001}{-1.4}{-1.6}
\end{erdiagram}

\end{equation*}
\end{erboxedFigure}
\mynote

\textbf{Mathematically}
\begin{tabular}{l l l}
& r4/a1 &    \\
& \multirow[t]{2}{1cm}{r4/d1=d2/r2} & a2  \\
&                                 & r1/a3 \\
& d2/r2    & as above\\
& d2/a4 &            \\
& d2/r3/a5 &      
\end{tabular}

\end{worktt}



\subsection{Entity Models and Equivalent Paths}
\begin{worktt}
Path equivalences are important to ensuring entity models fully explain; the question arises how should
path equivalences be expressed in an entity model. Well we are not being presriptive here about the precise way diagrams and accompanying information should be represented. Here are sokme thoughts though and they reflect the software ttols that the author has used in the past.
\end{worktt} 
\mynote \commentary{Somewhere happen}
The recognition and explicit statement of such path equivalences are essential steps in closing the gap between modelling and data definition. They enable the model to serve not merely as a structural schema but as a full explanation of how references are to be phrased and how facts are to be expressed.
\mynote
In these examples we have shown that, however expressed,  the specifications of equivalent paths are
an important supplement to entity relationship diagrams. 
An entity model in the main part consists of at least an entity relationship diagram  
plus a statement of path equivalences.

\begin{reinstatett}
\begin{reinstatett}
REcap.

\begin{oldtt}
\mynote
There are two paths of identifying relationships between entity type \textit{dramatic role} and type \textit{play}.
One of them proceeds via type \textit{character} and consists of relationship \textit{r4} followed by relationship \textit{d1} as follows:

\begin{equation}
\label{dramaticArtsPath1..diagram}
\scalebox{0.95}{\begin{erdiagram}{1.4700000000000002}{10.4}

\eret{8.9}{-1.47}{10.4}{-0.55}{0.2}{1}\eretname{9.65}{-1.06}{}{play}
\eret{4.5}{-1.47}{6}{-0.55}{0.2}{1}\eretname{5.25}{-1.06}{}{character}
\eret{0.1}{-1.47}{1.6}{-0.55}{0.2}{1}\eretname{0.85}{-1.06}{}{dramatic}\eretname{0.85}{-1.36}{}{role}

% relationship in
\errelname{6.15}{-0.86}{l}{in}\errelarm{6}{-1.01}{7.45}{-1.01}{1}{0}\errelarm{7.45}{-1.01}{8.9}{-1.01}{0}{0}\errelid{7.45}{-1.1}{}{d1}\ercrowfoot{6.15}{-1.01}{6}{-0.86}{6}{-1.01}{6}{-1.16}{0}\eridrefrel{6.25}{-0.91}{-1.11}
% relationship the_portrayal_of
\errelname{1.75}{-0.86}{l}{of}\errelname{1.75}{-0.56}{l}{portrayal}\errelname{1.75}{-0.26}{l}{the}\errelarm{1.6}{-1.01}{3.05}{-1.01}{1}{0}\errelarm{3.05}{-1.01}{4.5}{-1.01}{0}{0}\errelid{3.05}{-1.1}{}{r4}\ercrowfoot{1.75}{-1.01}{1.6}{-0.86}{1.6}{-1.01}{1.6}{-1.16}{0}\eridrefrel{1.85}{-0.91}{-1.11}
\end{erdiagram}
}
\end{equation}
The fact that relationship d? is identifying the reason that a reference to a role has nested within it a reference to a character. The fact that relationship d?? is identifying is the reason that a reference to a character has a reference to a play nested within it and hence why a reference to a role has nested reference to a character which in turn has a nested reference to a play.


The other path proceeds via type \textit{production} and consists of relationship \textit{d2} followed by relationship \textit{r2}:
\begin{equation}
\label{dramaticArtsPath2..diagram}
\scalebox{0.95}{\begin{erdiagram}{0.9700000000000001}{10.4}

\eret{8.9}{-0.97}{10.4}{-0.05}{0.2}{1}\eretname{9.05}{-0.4}{l}{play}
\eret{4.5}{-0.97}{6}{-0.05}{0.2}{1}\eretname{4.65}{-0.4}{l}{production}
\eret{0.1}{-0.97}{1.6}{-0.05}{0.2}{1}\eretname{0.85}{-0.4}{}{dramatic}\eretname{0.85}{-0.7}{}{role}

% relationship of
\errelname{6.15}{-0.36}{l}{of}\errelarm{6}{-0.51}{7.45}{-0.51}{1}{0}\errelarm{7.45}{-0.51}{8.9}{-0.51}{0}{0}\errelid{7.45}{-0.6}{}{r2}\ercrowfoot{6.15}{-0.51}{6}{-0.36}{6}{-0.51}{6}{-0.66}{0}\eridrefrel{6.25}{-0.41000000000000003}{-0.61}
% relationship in
\errelname{1.75}{-0.36}{l}{in}\errelarm{1.6}{-0.51}{3.05}{-0.51}{1}{0}\errelarm{3.05}{-0.51}{4.5}{-0.51}{0}{0}\errelid{3.05}{-0.6}{}{d2}\ercrowfoot{1.75}{-0.51}{1.6}{-0.36}{1.6}{-0.51}{1.6}{-0.66}{0}\eridrefrel{1.85}{-0.41000000000000003}{-0.61}
\end{erdiagram}
}
\end{equation}
The fact that relationship d? is identifying the reason that a reference to a role has nested within it a reference to a production. The fact that relationship d?? is identifying is the reason that a reference to a production has a reference to a play nested within it and hence why a reference to a role has nested reference to a production which in turn has a nested reference to a play.
\end{oldtt}


As we have seen these two nested references to a play are shared in our example and in fact they are shared in any reference to a role because of
the equivalence of these two paths. 
The simple existence of these comparable paths in the diagram of figure \ref{dramaticArts1..diagram}
does not imply their equivalence.

\commentary{fully understand this is one part of closing the gap that we spoke about in the introduction}
\commentary{To fully understand how references to dramatic roles are phrased we need to understand the equivalence of these two paths. If we don't fully undersatnd then the entity model cannot guide us fully in the phrasing of facts 
regarding dramatic role sshould bew represented in data. }
To understand the referencing of roles we need understand the equivalence of these two paths. 
So significant is this that the equivalence ought to be specified in the entity model in some way.
For now we can imagine annotating the diagram in some way to indicate that these two paths are equivalent 
or else specifying in a separate document or in a separate diagram. 
One way of specifying is by writing the equation
\begin{equation}
\label{dramaticArtsDramaticRolePathEquivalence}
r4 \circ d1 = d2 \circ r2
\end{equation}

Another way is to take diagram ((\ref{dramaticArtsPortrayalScopeFragment..diagram})), 
to flip it  around, and to draw  in this reduced form like this
\begin{equation}
\label{dramaticArtsPortrayalScopeAppearance1}
\begin{erdiagram}{3}{5.6010975}

\erettl{0}{-0.7}{1.654}{-0.1}\eretname{0.827}{-0.45}{}{production}
\erettr{4.154}{-0.7}{5.487}{-0.1}\eretname{4.82}{-0.45}{}{play}
\eretbl{-0.213}{-2.4}{1.867}{-1.8}\eretname{0.827}{-2.15}{}{dramatic role}
\eretbr{4.039}{-2.4}{5.601}{-1.8}\eretname{4.82}{-2.15}{}{character}

% relationship 
\errelname{0.977}{-1}{l}{}\errelname{0.977}{-1.65}{l}{in}\errelarm{0.826}{-0.7}{0.826}{-1.25}{1}{0}\errelarm{0.826}{-1.25}{0.826}{-1.799}{1}{0}\eridcomprel{0.7267862500000001}{0.92678625}{-1.5499999999999998}\ercrowfoot{0.827}{-1.65}{0.677}{-1.8}{0.827}{-1.8}{0.977}{-1.8}{0}
% relationship of
\errelname{1.804}{-0.25}{l}{of}\errelarm{1.653}{-0.4}{2.903}{-0.4}{1}{0}\errelarm{2.903}{-0.4}{4.153}{-0.4}{0}{0}\ercrowfoot{1.804}{-0.4}{1.654}{-0.25}{1.654}{-0.4}{1.654}{-0.55}{0}\eridrefrel{1.9035725000000001}{-0.30000000000000004}{-0.5}
% relationship 
\errelname{4.97}{-1}{l}{}\errelname{4.97}{-1.65}{l}{in}\errelarm{4.82}{-0.7}{4.82}{-1.25}{1}{0}\errelarm{4.82}{-1.25}{4.82}{-1.799}{1}{0}\eridcomprel{4.7202225}{4.9202224999999995}{-1.5499999999999998}\ercrowfoot{4.82}{-1.65}{4.67}{-1.8}{4.82}{-1.8}{4.97}{-1.8}{0}
% relationship the_portrayal_of
\errelname{2.017}{-2.4}{l}{the}\errelname{2.017}{-2.7}{l}{portrayal}\errelname{2.017}{-3}{l}{of}\errelarm{1.866}{-2.099}{2.953}{-2.099}{1}{0}\errelarm{2.953}{-2.099}{4.039}{-2.099}{0}{0}\ercrowfoot{2.017}{-2.1}{1.867}{-1.95}{1.867}{-2.1}{1.867}{-2.25}{0}\eridrefrel{2.116731875}{-1.9999999999999996}{-2.1999999999999997}
\end{erdiagram}

\end{equation}
and to assert that this diagram commutes.

This is our preferred diagrammatic way of presenting equivalent paths in this book and we return to this subject in a later section where we refer to such a diagram as this as a scope diagram. 
We also in the habit of writing equations such as (\ref{dramaticArtsDramaticRolePathEquivalence}) on the entity relationships diagram itself.
We can think of dramatic roles coming into being when productions of plays are being cast. What the commutivity of this diagram expresses is when a production of a play is being cast then it is all characters from the play that are being cast not characters from plays in general. 
The relationship ``the portray of'', we say, is restricted in scope.
\end{reinstatett}

\end{reinstatett}

\mynote
I don't expect the reader to be a mathematician but, for those readers that are, an entity model in the main part is a presentation 
(or, equivalently, a sketch) of a category of some kind. 


