\section{Referencing Entities}
\label{ReferencingEntities}

In this section we explore the significance of the identifying features
of a type and the manner in which knowledge of these features, 
be they attributes or relationships, enables us to 
reference entities and, in turn, 
to convey, communicate and store instances of relationships. 
There is more to this than initially meets the eye and we explore the topic
by way of examples.  

\subsection{Entities that are Named}
\mynote \commentary{Possibly ditch this example and just use the 
dramatic arts example which could be moved here.}
In the simplest examples, entities of a type have names associated with them
 and these names are unique and can be  quoted to reference entities of the type. 
 In these cases an entity model will represent the entity type with a name attribute
and mark it as identifying. 
Examples of such types are ubiquitous and you will see plenty in these pages but to mention two, because we have already come across them, in a model that includes the 
entity types \textit{planet} and  \textit{moon}  these types will be represented as having 
a name attribute and these will be marked as identifying as in this fragment here:
\begin{equation}
\label{planetMoonModel}
\begin{erdiagram}{1.0999999999999999}{6.9}

\eret{0.1}{-1}{1.85}{-0.1}{0.2}{1}\eretname{0.425}{-0.45}{l}{planet}
\erCoreAttribute{0.3}{-0.65}{1}{0}{name}{}
\eret{5.15}{-1}{6.9}{-0.1}{0.2}{1}\eretname{5.475}{-0.45}{l}{moon}
\erCoreAttribute{5.35}{-0.65}{1}{0}{name}{}

% relationship orbited by
\errelname{2}{-0.4}{l}{orbited by}\errelname{5}{-0.85}{r}{orbiting}\errelarm{1.85}{-0.549}{3.5}{-0.549}{0}{0}\errelarm{3.5}{-0.549}{5.15}{-0.549}{1}{0}\ercrowfoot{5}{-0.55}{5.15}{-0.4}{5.15}{-0.55}{5.15}{-0.7}{0}
\end{erdiagram}

\end{equation}
This tells us that we can reference planets by name, as in 
\begin{equation}
\label{theplanetJupiter}
\mbox{\textit{the planet with name Jupiter}}
\end{equation}
and, likewise, moons, so that we can make reference to 
\begin{equation}
\label{themoonIo}
\mbox{\textit{the moon with name Io}}.
\end{equation}

Note that there is a standard phraseology 
at play in (\ref{theplanetJupiter}) and (\ref{themoonIo}) and 
being used to reference entities using the values of their singular identifying attributes. 
To illustrate various characteristics of this referencing, 
we will underline the values of identifying attributes in example phrases 
and we will use different styles of underlining to distinguish different entities that are being referenced. We see this in this statement: 
\begin{equation}
\label{JupiterIoPernickity}
\mbox{\textit{The planet with name \rdash{Jupiter} 
is orbited by the moon with name \rdot{Io}.}}
\end{equation}
This is an example of a wider standard phraseology, one  for conveying relationship instances,
and one that builds on the standard phraseology for referencing entities.
There is a bit more to this than meets the eye, though, as we shall see.
\mynote
When the values of identifying attributes are used in referencing then we will speak of them as \textit{referentials}. 
Accordingly, we speak of the terms `Jupiter' and `Io' in statement (\ref{JupiterIoPernickity}) as referentials.

In the context of statement (\ref{JupiterIoPernickity}), Jupiter is the name of, and therefore  a reference to, a planet and  Io is the name of, and therefore a reference to, a moon (aka a natural satellite).
Note though that within the context of the solar system the name Io  doesn't uniquely identify this moon of Jupiter amongst all named solar system objects since it is also the name of an asteroid that orbits within the asteroid belt. This  illustrates that 
the names of moons are only unique as names of moons. This is very much the meaning of the underlining of the name attribute in fragment (\ref{JupiterIoPernickity}). 

\mynote
We are very much stating the obvious of course but there is a pattern here and it is repeated over and over when instances of relationships are communicated  and the same pattern occurs more formally  and uniformly when instances of relationships are communicated in software systems and/or stored in data.

We will come back to the phrasing of relationship instances in a monent but first we must look at what happens when some of the identifying features of an entity are relationships.

\subsection{Context and the Referencing of Entities}
\mynote
In this section we look at the way that entities may be referenced using identifying features
 when one these of features is a relationship.
We examine how,  in this situation, the referencing 
--- the nature of the reference ---   
is impacted by the context within which the reference is being made. 
The reference is  a noun phrase quoting the values of identifying attributes
and with the number quoted depending on 
the context within which the reference is made.
What we see contributes to a later analysis --- the analysis of  how 
instances of a relationship can be conveyed, communicated or stored and
how this is affected by shared knowledge we might have about 
the scope of  the relationship.\footnote{Speak of scope beforehand because need to keep interest in what seems not interesting. Add footnote about scope at first use (here?). Probably explain scope in next section because scope affects everything not just ``data''.} \footnote{if explained before could here say ``Here is that scope concept again. Explanation comes in next section.''}
\subsubsection{Referencing Characters from Plays}
\mynote
 We use as an example the referencing of characters from  plays
 within the context of  the  dramatic arts
   as modelled earlier in figure \ref{dramaticArts1}.
\mynote
Consider this fragment 
\begin{equation*}
\begin{erdiagram}{4.1}{5.0752749999999995}

\eret{0.1}{-1.5}{1.753}{-0.6}{0.2}{1}\eretname{0.265}{-0.95}{l}{playwright}
\erCoreAttribute{0.3}{-1.15}{1}{0}{name}{}
\eret{3.553}{-1.5}{4.886}{-0.6}{0.2}{1}\eretname{3.686}{-0.95}{l}{play}
\erCoreAttribute{3.753}{-1.15}{1}{0}{title}{}
\eret{3.463}{-4}{4.975}{-3.1}{0.2}{1}\eretname{3.614}{-3.45}{l}{character}
\erCoreAttribute{3.663}{-3.65}{1}{0}{name}{}

% relationship about
\errelname{4.369}{-1.8}{l}{about}\errelname{4.369}{-2.95}{l}{in}\errelarm{4.219}{-1.5}{4.219}{-2.3}{1}{0}\errelarm{4.219}{-2.3}{4.219}{-3.1}{0}{0}\eridcomprel{4.11915}{4.31915}{-2.85}\ercrowfoot{4.219}{-2.95}{4.069}{-3.1}{4.219}{-3.1}{4.369}{-3.1}{0}
% relationship written_by
\errelname{3.403}{-1.35}{r}{written}\errelname{3.403}{-1.65}{r}{by}\errelname{1.903}{-0.9}{l}{of}\errelname{1.903}{-0.6}{l}{author}\errelname{1.903}{-0.3}{l}{the}\errelarm{3.552}{-1.049}{2.652}{-1.049}{1}{0}\errelarm{2.652}{-1.049}{1.752}{-1.049}{0}{0}\ercrowfoot{3.403}{-1.05}{3.553}{-0.9}{3.553}{-1.05}{3.553}{-1.2}{0}\eridrefrel{3.3025}{-0.9499999999999998}{-1.15}
\end{erdiagram}

\end{equation*}
of figure \ref{dramaticArts1} which picks out just the details which are
relevant to the identification of character entities.

From this part of the model we see that playwrights are identified by name and that plays are identified by their titles in the context of the playwright they are written by. Therefore
\begin{itemize} \commentary{rewrite this reference phrase}
\item without context, 
\begin{equation} 
\label{absolutePlayReferencing}
\text{\parbox{9cm}{\textit{a  play 
may be referenced by the \rdash{title} of the play and the \rdash{name} of the playwright it is written by.}}}
\end{equation}
\end{itemize}
Note that I will continue to underline  referentials so as to keep drawing attention to the number of them required in any given context.

\mynote
From inspecting the representation of the character type and in particular its identifying features then we see that:
\begin{itemize}
\item within the context of a play, \commentary{also rewrite this reference phrase}
\begin{equation} 
\label{absoluteCharacterReferencing}
\text{\parbox{9cm}{\textit{
a character may be referenced by the name of character.}}}
\end{equation}
Accordingly in the text of a play characters are referenced 
by name as for example in a stage direction such as this one:  
\begin{equation*}
\text{\textit{Enter Sebastian.}}
\end{equation*}
\item within the context of a particular playwright, \commentary{and this one}
\begin{equation} 
\label{playwrightRelativeCharacterReferencing}
\text{\parbox{9cm}{\textit{
 a character within a play 
may be referenced by the name of character and the title of the play they are in.}}}
\end{equation}
So in a discussion of favourite characters from Shakespeare we might 
make reference to
\begin{equation*}
\text{\textit{\rdash{Sebastian} from \rdash{Twelfth Night}.}}
\end{equation*}
\item
without context, 
\begin{equation} 
\label{absoluteCharacterReferencing}
\text{\parbox{9cm}{\textit{a character within a play 
may be referenced by the name of character, the title of a play and the name of the playwright who wrote the play.}}}
\end{equation}
Thus, without context, three referentials are required for the referencing of a character
as for instance when I reference a memorable comic character such as:
 \begin{equation*}
\label{TounchstoneReferenceFromAbsolute}
\text{\parbox{9cm}{\textit{\rdash{Touchstone} from \rdash{Shakespeare}'s \rdash{As You Like It}. }}}
\end{equation*}
\end{itemize}

\begin{notebox}
In this example there were two identifying relationships arranged end to end. It was by looking at all the entity types involved, directly and indirectly through these identifying relationships,
 that I gathered the identifying attributes whose values would be required as referentials. 
I mention this because there is the beginnings of a simple algorithm here by which entity models can be transformed to definitions of tabular structure.
\end{notebox}

\subsubsection{Referencing Productions of Plays}
From this detail 
\begin{equation*}
\begin{erdiagram}{1.3}{7.7691}

\eret{0.1}{-1.2}{1.433}{-0.3}{0.2}{1}\eretname{0.767}{-0.65}{}{play}
\eret{3.133}{-1.2}{4.786}{-0.3}{0.2}{1}\eretname{3.299}{-0.65}{l}{production}
\erCoreAttribute{3.333}{-0.85}{1}{0}{season}{}
\eret{6.436}{-1.2}{7.769}{-0.3}{0.2}{1}\eretname{7.102}{-0.65}{}{venue}

% relationship of
\errelname{2.983}{-0.6}{r}{of}\errelname{1.583}{-0.6}{l}{in}\errelname{1.583}{-0.3}{l}{given}\errelarm{3.133}{-0.75}{2.283}{-0.75}{1}{0}\errelarm{2.283}{-0.75}{1.433}{-0.75}{0}{0}\ercrowfoot{2.983}{-0.75}{3.133}{-0.6}{3.133}{-0.75}{3.133}{-0.9}{0}\eridrefrel{2.8833}{-0.65}{-0.85}
% relationship at
\errelname{4.936}{-0.6}{l}{at}\errelname{6.286}{-0.6}{r}{of}\errelname{6.286}{-0.3}{r}{location}\errelname{6.286}{-0}{r}{the}\errelarm{4.785}{-0.75}{5.61}{-0.75}{1}{0}\errelarm{5.61}{-0.75}{6.435}{-0.75}{0}{0}\ercrowfoot{4.936}{-0.75}{4.786}{-0.6}{4.786}{-0.75}{4.786}{-0.9}{0}\eridrefrel{5.0358}{-0.65}{-0.85}
\end{erdiagram}

\end{equation*}
regarding the production entity type in  of figure \ref{dramaticArts1} we see that
\begin{itemize}
  \item
  without context, a \textit{production} can be referenced by referencing the play it is a production of, 
  the venue where the production takes place and the season 
  over which it plays. 
  We can count up the number of referentials required
  as one for the venue, two for the play and one for the season so four referentials in all. 
  I looked up a production of a play that I remember going to in 1975
  to use as an example. 
  It was
\begin{erquote}
\parbox{9cm}{a production of \mbox{\rdash{Shakespeare}'s} \rdash{As You Like It}
 performed \mbox{\rdash{April – May 1975},} at \rdash{Oxford Playhouse}}\\
\end{erquote}
as expected there are four referentials, I have underlined them, in the description.\footnote{
The exact description that I found online was ``a production of the play As You Like It (by William Shakespeare), April – May 1975, at Oxford Playhouse''.  
} 
\end{itemize}

 \subsubsection{Referencing Dramatic Roles}
 \mynote
Sometimes, as is the case with so called intersection entity (types), 
all identifying features of the type are relationships. 
This is such an example but it is an example with a difference.
\mynote
In figure \ref{dramaticArts1}, the entity type \textit{dramatic role} is
depicted as an intersection entity as it is shown having
two identifying relationships and no identifying attributes.
The relevant detail is  this
From this detail 
\begin{equation*}
\begin{erdiagram}{3.8000000000000003}{5.5333000000000006}

\eret{0.1}{-3.4}{1.9}{-2.5}{0.2}{1}\eretname{1}{-2.85}{}{character}\ergroupannotation{0.95}{-3.1}{l}{...}
\eret{3.6}{-0.9}{5.4}{-0}{0.2}{1}\eretname{4.5}{-0.35}{}{production}\ergroupannotation{4.45}{-0.6}{l}{...}
\eret{3.6}{-3.4}{5.4}{-2.5}{0.2}{1}\eretname{4.5}{-2.85}{}{dramatic}\eretname{4.5}{-3.15}{}{role}

% relationship cast_with
\errelname{4.35}{-1.2}{r}{cast}\errelname{4.35}{-1.5}{r}{with}\errelname{4.65}{-2.35}{l}{in}\errelarm{4.5}{-0.9}{4.5}{-1.7}{1}{0}\errelarm{4.5}{-1.7}{4.5}{-2.5}{1}{0}\eridcomprel{4.4}{4.6}{-2.25}\ercrowfoot{4.5}{-2.35}{4.35}{-2.5}{4.5}{-2.5}{4.65}{-2.5}{0}
% relationship the_portrayal_of
\errelname{3.45}{-2.8}{r}{of}\errelname{3.45}{-2.5}{r}{portrayal}\errelname{3.45}{-2.2}{r}{the}\errelname{2.05}{-3.25}{l}{portrayed}\errelname{2.05}{-3.55}{l}{by}\errelarm{3.6}{-2.95}{2.75}{-2.95}{1}{0}\errelarm{2.75}{-2.95}{1.9}{-2.95}{0}{0}\ercrowfoot{3.45}{-2.95}{3.6}{-2.8}{3.6}{-2.95}{3.6}{-3.1}{0}\eridrefrel{3.35}{-2.85}{-3.0500000000000003}
\end{erdiagram}

\end{equation*}
Obviously, to reference a dramatic role it is necessary to reference
both a character from a play and a production. 
We have already discussed how characters from plays need be referenced using three referentials
and  how productions need be referenced using four referentials.  
\mynote
Now we can expect that, without context, a reference to
a dramatic role will require seven referentials
 --- three for the character and four for the production. 
So we will need seven, right?
 Well no, rather surprisingly, only five referentials are required. 
 What happended to the other two? Can you see why this is? Can you explain it?

That only five referentials are needed can be seen from this reference to a dramatic role:
\begin{erquote}
\parbox{9.0cm}{the role of \rdash{Touchstone} in the production of \mbox{\rdash{Shakespeare}'s} \rdash{As You Like It} performed \mbox{\rdash{April – May 1975},} at \rdash{Oxford Playhouse}.
}
\end{erquote}

There is a phenomenon at work here and it is of some significance. 
We describe this phenomenon in a number of ways in the remainder of this section and in the next. 
I don't expect the reader to be a mathematician but, for those readers that are, there is a quick way of saying what is happening --- there is a commutative diagram of functional relationships at play and, in cases such as this, this makes all the difference to how we reference entities.

\begin{noteforfuture}
I might be able to get to the equivalence of paths by examining
a noun phrase involvinf seven referentials.

dramatic role is referenced by reference to production reference to play
\end{noteforfuture}

\subsubsection{Algorithm for accumulating referentials}
\commentary{referential accumulation algorithm}

Describe the algorithm.
Accumulate into the referential attributes the identifying attributes of the subject entity type
with all the referential attributes of all entity types reached by travering an identifying relationship.

Next need to refine this algorithm to take account of commutative relationships. 

To do use define paths of identifying relationships.

\subsection{Communicating Relationships}

\mynote 
Now on the surface of it it is very simple
to convey instances of relationships --- we reference two entitites and assert a relationship between them. 
To convey an instance of this relationship://
XXXX actor to dramatic role
we reference an actor entity, by name according to figure ref{drammaticArts1},
and we convey a dramatic role entity by means of five referentials as we have seen above.
In this way we get statements such as
\begin{erquote}

\parbox{9.0cm}{\linespread{1.5}\normalsize\rdash{Bob Hoskins} was \uwave{the player of} the role of \rdot{Touchstone} in the production of \mbox{\rdot{Shakespeare}'s} \rdot{As You Like It} performed \mbox{\rdot{April – May 1975},} at \rdot{Oxford Playhouse}.
}
\end{erquote}
which, by the way, is true  --- Bob Hoskins did play this role --- I was there.
In this example, referentials to reference the actor are dash underlined. The relationship is wavy underlined.
Referentials to reference the dramatic role are dotted underlined.

All this is strightforward. But things are not always so straightforward because sometimes referentials can overlap. 
Shlaer and Long explore this topic in the context of relational database structure and refer to them as \textit{collapsed referentials}. This is an important topic and the subject isn't complete if we dont 
consder these and so here is an example that contains such `collapsed referentials'.

As we have said above and according to the model we have assumed,
  characters from plays need be identified by three referentials
 --- the name of the character, the title of the play and the name of the playwright.

\mynote 
That there three referentials to reference a character naively implies that if we wish to communicate an instance of relationship between two different characters then we will be required to communicate six referentials
 -- three to identify one party in the relationship and three to identify the other party. 
 we will see shortly  --- this is the significant bit --- that this is not always the case and so not what we instinctively do, when we have \textit{a priori} knowledge of the relationship being communicated is limited in scope. 
 If we know the relationship is confined to individual playwrights or is internal to individual play 
 or script then clearly we do not 
  communicate the playwright and play title twice over. The number of referentials may in such a case collapse to four:
   name of playwright, title of play and name of character.
Shlaer-Lang in the context of data representation of entities and in a very much underrated paper, coined the term \textit{collapsed referentials} in
their description of this phenomena.  We shall use instead the term \textit{shared referential} for a referential that contributes to multiple references.

\mynote 
In the  the play Twelfth Night by William Shakespeare, Antonio loves Sebastian.

For the discussion that follows suppose love to be a many-many relationship between characters.

A naive approach would suggest that six referentials are required to convey an instance of love
between characters --- three referentials to reference the lover character and three to reference the loved.
If I follow the naive approach I would convey the relationship instance like this:
\begin{equation}
\text{\parbox{9cm}{\textit{The character named Antonio in Shakespeare's play Twelfth Night loves the character named Sebastian in Shakespeare play Twelfth Night.}}}
\end{equation} 
But there is a reason why the shorter
\begin{equation}
\text{\parbox{9cm}{\textit{The character named Antonio in Shakespeare's play Twelfth Night loves the character named Sebastian.}}}
\end{equation} 
is correct and it is  important we understand this reason.
What is at play here is another instance of the same basic phenomenon we saw before in the context of the referencing of entitites --- less referentials are required than we first expect from a naive reading.  

This reason in this case is because the relationship `loves' in this context 
is, \textit{a priori}, a relationship which is internal to the plot of a single individual play ---
it is an intra-play relationship rather than an inter-play relationship.
Because it is intra-play, when we covey an instance then the referentials that refer to the play are shared between lover and loved As shown in figure \ref{SebastianLovesAntonio}.


\begin{erboxedFigure}{H}{SebastianLovesAntonio}
{Referencing two different characters using just four referentials. This is an example of what Shlaer and Lang (?) called collapsed referentials. What we see is that referential `Shakespeare' contributes to two different references. Likewise does the referential `Twelfth Night'.
These are examples of shared referentals --- they contribute, as indicated, to both reference 1 and reference 2.}
\newcommand{\dashRefOne}{2pt 2pt}
\newcommand{\dashRelationship}{1pt 0pt}
\newcommand{\dashRefTwo}{1pt 1pt}
\begin{tabular}{l}
\Rnode{w1}{\rdash{Antonio}} in \Rnode{w2}{\rdot{\rdash{Shakepeare}}}'s \Rnode{w3}{\rdot{\rdash{Twelfth Night}}} \Rnode{w4}{\rline{loves}}  \Rnode{w5}{\rdot{Sebastian}} \\[1.4cm]
\kern1.2cm\Rnode{ref1}{\textit{reference 1}}
\kern0.75cm\Rnode{rel}{\textit{relationship}}
\kern0.6cm\Rnode{ref2}{\textit{reference 2}} \\[0.5cm]
\syntag{\dashRefOne}{ref1}{0.9}{w1}{0}
\syntag{\dashRefOne}{ref1}{0.9}{w2}{-0.2}
\syntag{\dashRefOne}{ref1}{0.9}{w3}{-0.2}
\syntag{\dashRelationship}{rel} {0.7} {w4}{0}
\syntag{\dashRefTwo}{ref2}{0.4}{w2}{0.2}
\syntag{\dashRefTwo}{ref2}{0.4}{w3}{0.3}
\syntag{\dashRefTwo}{ref2}{0.4}{w5}{0}
\end{tabular}
\end{erboxedFigure}

An instances of the `loves' relationship is conveyed by a total of four referentials
 not the six that might have been required
if the relationship were global in scope (as would have been the case if a
 character from one play could love  a character from an entirely different play).
 An example of a relationship between characters that is global in scope 
 is the relationship of one character being modelled on or resembling or being inspired 
 by another character from the dramatic arts.
 To convey an instance of this relationship, being as the relationship is gloabl in scope,  does indeed require
 six referentials --- three for the one party and three for the other --- as in this example:
\begin{erquote}
\parbox{9.0cm}{\linespread{1.3}\normalsize
\rdash{Romeo} from \rdash{Shakespeare}'s \rdash{Romeo and Juliet}
          \uwave{is modelled on} \rdot{Pyramus} from \rdot{Ovid}'s \rdot{Metamorphoses}.}
\end{erquote}
which, interpreted from the perspective of the dramatic arts that we have given above, does indeed contain six referentials. 
The first three, shown with dashed underlining, reference one character.
The verb of the sentence is the name of the relationship, and, 
the final three referentials, with dotted underlining, 
reference a second charater from a different play.\footnote{I am bending the facts a bit here 
in that Metamophoses isn't strictly a play - its a bunch of stories.} 
There are no shared referentials.


\subsection{Philosophy}
\mynote
Bearing in mind that I have said that entities, and relationship instances, are particulars
and that the values of attributes, referentials included, are universals then, arguably,
 what we have been examining in this section, 
 is the mechanics behind the following slogan:
\begin{erquote}
We think particulars, but we speak universals.
\end{erquote}
\begin{noteforfuture}
\mynote
A follow up slogan for the coming section on data is
\begin{erquote}
Universals are the stuff of data though its meaning is in the 
particulars --- entities and relationships --- it describes. 
\end{erquote}
\end{noteforfuture}





 
