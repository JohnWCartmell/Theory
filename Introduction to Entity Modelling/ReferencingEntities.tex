\section{Referencing Entities}
\label{ReferencingEntities}

In this section we explore the significance of the identifying features
of a type and the manner in which knowledge of these features enables us to 
reference entities and, in turn, to convey, communicate and store instances of relationships. 
There is more to this than initially meets the eye and we explore the topic
by way of examples. 

\subsection{The Simplest Case --- Planets and their Moons as Named Entities}
\mynote
In the simplest examples entities of a type have names associated with them and these are unique and can be  quoted to reference entities of the type. 
Name attributes in these cases are archetypal examples of identifying attributes. 

We have already seen an example of the role of the naming of both planets and moons in the example given earlier for a phraseology for the communication of relationships
 in which we presented this statement:
\begin{equation}
\label{JupiterIoShort}
\mbox{\textit{Jupiter is orbited by Io.}}
\end{equation}
as a representation of an instance of the binary relationship
\begin{gather}
\label{planetMoonSecondInstance}
\raisebox{-1cm}{
\begin{erdiagram}{1.4}{6.9}

\eret{0.1}{-1}{1.85}{-0.4}{0.2}{1}\eretname{0.975}{-0.75}{}{planet}
\eret{5.15}{-1}{6.9}{-0.4}{0.2}{1}\eretname{6.025}{-0.75}{}{moon}

% relationship orbited by
\errelname{2}{-0.55}{l}{orbited by}\errelname{5}{-1}{r}{orbiting}\errelarm{1.85}{-0.7}{3.5}{-0.7}{0}{0}\errelarm{3.5}{-0.7}{5.15}{-0.7}{1}{0}\ercrowfoot{5}{-0.7}{5.15}{-0.55}{5.15}{-0.7}{5.15}{-0.85}{0}
\end{erdiagram}

}
\end{gather}
for in the context of statement (\ref{JupiterIoShort}), Jupiter is the name of, and therefore  a reference to, a planet and  Io is the name of, and therefore a reference to, a moon i.e. a natural satellite.\footnote{
Note that within the context of the solar system the name Io  doesn't uniquely identify this moon of Jupiter amongst all named solar system objects since it is also the name of an asteroid that orbits within the asteroid belt.
The names of moons are only unique as names of moons. 
Of course in other contexts, both Jupiter and Io are the names of mythological figures.
} 
We understand  (\ref{JupiterIoShort}), therefore, to be shorthand for the 
more pernickerty 
\begin{equation}
\mbox{\textit{The planet with name `Jupiter' is orbited by the moon with name `Io'.}}
\end{equation}

\mynote So planets and moons are named which is to say they have name attributes.
\commentary{Rewrite this paragraph.}
Of course name attributes are archetypal examples of identifying attributes --- we name things so that we can identify them in context. In such a situation the name of an entity, i.e. the value of the name attribute,  can be used in context to definitively identify the entity.
Quoting this value, i.e. using the name, is one way of referencing an entity of the type from within the context of another.
 \mynote
Focusing on just this aspect of the solar system we  draw the model shown in figure
\ref{planetMoonModel}
containing the relationship (\ref{planetMoonSecondInstance}) and showing name attributes of both moons and planets and
underlining these to showing that each of them is unique and therefore identifying. 
This is the entity model (albeit of a very small universe) that explains the communication (\ref{JupiterIoShort}) or equally, and as we shall see later, the structure of the data table of planets and moons given earlier.
\begin{erboxedFigure} {H}{planetMoonModel}
{
caption blah blah blah
 }
\begin{erdiagram}{2.4999999999999996}{6.9}

\eret{0.1}{-2.1}{1.85}{-1.2}{0.2}{1}\eretname{0.425}{-1.55}{l}{planet}
\erCoreAttribute{0.3}{-1.75}{1}{0}{name}{}
\eret{5.15}{-2.1}{6.9}{-1.2}{0.2}{1}\eretname{5.475}{-1.55}{l}{moon}
\erCoreAttribute{5.35}{-1.75}{1}{0}{name}{}
\eret{0}{-0.2}{6.9}{0.3}{0.2}{1}

% relationship 
\errelname{1.125}{-0.5}{l}{}\errelarm{0.975}{-0.2}{0.975}{-0.7}{1}{0}\errelarm{0.975}{-0.7}{0.975}{-1.2}{0}{0}\ercrowfoot{0.975}{-1.05}{0.825}{-1.2}{0.975}{-1.2}{1.125}{-1.2}{0}
% relationship 
\errelname{6.175}{-0.5}{l}{}\errelarm{6.025}{-0.2}{6.025}{-0.7}{1}{0}\errelarm{6.025}{-0.7}{6.025}{-1.2}{0}{0}\ercrowfoot{6.025}{-1.05}{5.875}{-1.2}{6.025}{-1.2}{6.175}{-1.2}{0}
% relationship orbited by
\errelname{2}{-1.5}{l}{orbited by}\errelname{5}{-1.95}{r}{orbiting}\errelarm{1.85}{-1.65}{3.5}{-1.65}{0}{0}\errelarm{3.5}{-1.65}{5.15}{-1.65}{1}{0}\ercrowfoot{5}{-1.65}{5.15}{-1.5}{5.15}{-1.65}{5.15}{-1.8}{0}
\end{erdiagram}

\end{erboxedFigure}
\mynote
We are very much stating the obvious of course but there is a pattern here and it is repeated over and over when instances of relationships are communicated  and the same pattern occurs more formally  and uniformly when instances of relationships are communicated in software systems and/or stored in data.

\begin{noteforfuture}
would like an example with two or three identifying attributes but --- cannot think of one. The ones I have given earlier are not very convincing.
\end{noteforfuture}

\subsection{Context and the Referencing of Entities}
\mynote
This section is about how entities are referenced using identifying features when one these features is a relationship.
We look at how,  in this situation, the referencing 
--- how the referencing is done ---   
is impacted by the context within which the reference is being made. 
What we see contributes 
later to the analysis of how 
instances of a relationship can be conveyed, communicated or stored and
how this is affected by shared knowledge we might have about 
the scope of  the relationship.\footnote{Speak of scope beforehand because need to keep interest in what seems not interesting. Add footnote about scope at first use (here?). Probably explain scope in next section because scope affects everything not just ``data''.} \footnote{if explained before could here say ``Here is that scope concept again. Explanation comes in next section.''}

\mynote
 We use as an example the referencing of characters from a play 
 assuming the  dramatic arts to be as modelled earlier as shown in figure xxx.
There are only a few specific details --- three in fact ---
that concern us here. These three details are shown in 
\ref{dramaticArtsIdentificationDetails} (a), (b) and (c). 
%\vspace{-0.2cm} % Dont know why I need this but I do.
\begin{erboxedFigure}{H}{dramaticArtsIdentificationDetails}
{Details from the model of the dramatic arts that was presented in figure xxx.
(a), (b) and (c) show the identifying features of, respectively, types character, play and playwright. 
}
\vspace{-0.7cm}% Dont know why I need this but I do.
\begin{tabular}{ccccc}
(a) 
\raisebox{-1.5cm}{\begin{erdiagram}{2.4}{2.7}

\eret{0}{-2.4}{2.7}{-1.4}{0.2}{1}\eretname{0.27}{-1.75}{l}{character}
\erCoreAttribute{0.2}{-1.95}{1}{0}{name}{}
\eret{0}{-0.25}{2.7}{0.25}{0.2}{1}\eretname{1.095}{-0.2}{l}{play}

% relationship containing
\errelname{1.2}{-0.55}{r}{containing}\errelname{1.5}{-1.25}{l}{within}\errelarm{1.35}{-0.25}{1.35}{-0.825}{0}{0}\errelarm{1.35}{-0.825}{1.35}{-1.4}{1}{0}\eridcomprel{1.25}{1.4500000000000002}{-1.15}\ercrowfoot{1.35}{-1.25}{1.2}{-1.4}{1.35}{-1.4}{1.5}{-1.4}{0}
\end{erdiagram}
}
 && (b) \kern-0.35cm
\raisebox{-1.5cm}{\begin{erdiagram}{2.4}{2.7}

\eret{0}{-2.4}{2.7}{-1.4}{0.2}{1}\eretname{0.27}{-1.75}{l}{play}
\erCoreAttribute{0.2}{-1.95}{1}{0}{title}{}
\eret{0}{-0.25}{2.7}{0.25}{0.2}{1}\eretname{0.713}{-0.2}{l}{playwright}

% relationship writer_of
\errelname{1.2}{-0.55}{r}{writer}\errelname{1.2}{-0.85}{r}{of}\errelname{1.5}{-1.25}{l}{by}\errelname{1.5}{-0.95}{l}{written}\errelarm{1.35}{-0.25}{1.35}{-0.825}{1}{0}\errelarm{1.35}{-0.825}{1.35}{-1.4}{1}{0}\eridcomprel{1.25}{1.4500000000000002}{-1.15}\ercrowfoot{1.35}{-1.25}{1.2}{-1.4}{1.35}{-1.4}{1.5}{-1.4}{0}
\end{erdiagram}
}
 &&  (c) 
\raisebox{-1.5cm}{\begin{erdiagram}{2.4}{2.7}

\eret{0}{-2.4}{2.7}{-1.4}{0.2}{1}\eretname{0.27}{-1.75}{l}{playwright}
\erCoreAttribute{0.2}{-1.95}{1}{0}{name}{}
\eret{0}{-0.25}{2.7}{0.25}{0.2}{1}\eretname{0.521}{-0.2}{l}{dramatic arts}

% relationship includes_works by
\errelname{1.2}{-0.55}{r}{includes}\errelname{1.2}{-0.85}{r}{works by}\errelname{1.5}{-1.25}{l}{within}\errelname{1.5}{-0.95}{l}{known}\errelarm{1.35}{-0.25}{1.35}{-0.825}{0}{0}\errelarm{1.35}{-0.825}{1.35}{-1.4}{1}{0}\eridcomprel{1.25}{1.4500000000000002}{-1.15}\ercrowfoot{1.35}{-1.25}{1.2}{-1.4}{1.35}{-1.4}{1.5}{-1.4}{0}
\end{erdiagram}
}
\end{tabular}
\end{erboxedFigure}

\mynote
Considered the way that types character, play and playwright are identified as summarised in
\ref{dramaticArtsIdentificationDetails} (a), (b) and (c). 
\footnote{With a minor exception there is a standard phraseology in use here. Exercise for the reader What exception is being used and how might this be accounted for?}


\mynote 
Detail (a) can be paraphrased as: 
\begin{equation}
\label{characterReferenceFromPlay}
\text{\parbox{9cm}{\textit{within the context of a play, a character within it can be referenced by name}}}\\
\end{equation}

Example, in a stage direction, 
\begin{equation*}
\text{\textit{Enter Sebastian.}}
\end{equation*}
though the name of the character isn't unique, not so even in the plays of Shakespeare, the name of the character alone suffices in the stage direction. This is the simplest thing  and it isn't usual to draw attention to it but it is worth it
  --- drawing attention to it --- 
 because it is an an example of something that is significant
 and impactful on the way that 
 relationships are conveyed, communicated and stored
 and therefore to the structure and mechanism of data. 

 \begin{noteforfuture}
 stage directions, when they reference characters, reference characters from the very same play that they are directions within
 \end{noteforfuture}

  \begin{noteforfuture}
 who is your favourite character?\\
 xyz from Pygmalion\\
 which pygmalion?\\
 pygmalion by GBS --- my favourite character from all of drama is 
 xyz from Pymalion by GBS.
 \end{noteforfuture}
 
\mynote 
Details (b) and (c) can be paraphrased likewise
and then details (a) and (b) taken together imply: 
\begin{equation}
\label{characterReferenceFromPlaywright}
\text{\parbox{9cm}{\textit{within the context of a playwright
(say in a discussion of the works of a playwright), 
a character within the works of the playwright 
can be referenced by the title of a play 
and a name of a character within that play. }}}\\
\end{equation}
\mynote 
Likewise, from (a), (b) and (c) taken together we have that
\begin{equation} 
\label{absoluteCharacterReferencing}
\text{\parbox{9cm}{\textit{without context, a character within a play 
may be referenced by the name of character, the title of a play and the name of the playwright who wrote the play.}}}
\end{equation}


\begin{oldtt}

\begin{gather}
\text{\parbox{9cm}{\textit{within the context of a play, a character within it can be referenced by name,}}} \label{aWords}\\
\text{\parbox{9cm}{\textit{within the context of a playwright, a play written by them can be referenced by title,}}}\label{bWords} \\
\text{\parbox{9cm}{\textit{a playwright known to the dramatic arts can be referenced by name.}}}\label{cWords}
\end{gather}

ILLUSTRATION

\mynote
In accord with detail (a) if a reference to a character is made  in the context of a play,
say in the stage direction, then the character is referenced by name 
\begin{equation*}
\text{\textit{Enter Sebastian.}}
\end{equation*}

From (b) and  (c) we have that
\begin{equation}
\text{\parbox{9cm}{\textit{without context, a play 
may be referenced by the title of the play and the name of the playwright,}}} 
\label{absolutePlayReferencing}
\end{equation}

From (b) and  (c) we have that
\begin{equation}
\text{\parbox{9cm}{\textit{within the context of a playwright, a character in a play written by them can be referenced by title of the play and name of the character,}}}
\label{playwrightRelativeCharacterReferencing}
\end{equation}

\mynote 
If the reference to a character is made in the context of a particular playwright but independently of any particular play then the reference will need to reference a play by title and reference the character by name. 

If the reference to a character is made in the context of dramatical studies or performance in general rather than in the context of a particular playwright then three referentials will be given 


In the context of a discussion of Shakespeare's play Twelfth Night we might say that
\begin{equation}
\label{AntonioLovesSebastian}
\mbox{\textit{The character Antonio loves Sebastian.}}
\end{equation} 
meaning
\begin{equation}
\label{AntonioLovesSebastianPedantic}
\mbox{\textit{The character name Antonio loves the character named Sebastian.}}
\end{equation} 

\mynote 
So we have, in the context of the plays of William Shakespeare,
\begin{equation}
\label{AntonioLovesSebastian}
\mbox{\textit{The character Antonio in Twelfth Night loves Sebastian.}}
\end{equation} 

This actor has played xxx in yyy, ddd in ccc, www in mmm.
\end{oldtt}
\subsection{Context and Referencing Plays and Playwrights --- Getting to universals}
\newpage
\subsection{Nested Context and Referencing.}
\mynote Title alone is not used because generally it is not unique.
There are two plays jean-Jacques Rousseau's "Pygmalion"
George Bernard Shaw's play of the same name.

\subsection{When communicating a plurarity of instances of relationships is less that the sum of the parts }
\subsection{Philosophy}
We think particulars --- entities and relationships --- but we speak universals.


Three universals to reference a character. Name of playwright, title of play, name of character.

\subsection{Somewhere}
\mynote
referential attributes represent data not concept.
\commentary{maybe in a new subsection on cascading identifiers}
\mynote
relationships model concept and context not data.

\mynote What are the attributes of a play that enable it toi be referenced? Well, in one sense at least, they are the title of the play and the name of author. What are the attributes which enable a character to be identified? Name of character, title of play and name of author of play. In one sense these are clearly attributes of a play? In an other sense they are not. How this is is discussed in the nest section.  

\mynote to convey an entity we need to convey how that entity stands relative to our context i.e. how it is related,
to convey the relationship we need convey attributes,
 relationship are communicated soley using the values of attributes,
 
