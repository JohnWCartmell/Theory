\section{Referencing Entities}
\label{ReferencingEntities}

In this section we explore the significance of the identifying features
of a type and the manner in which knowledge of these features enables us to 
reference entities and, in turn, to convey, communicate and store instances of relationships. 
There is more to this than initially meets the eye and we explore the topic
by way of examples. 

\subsection{The Simplest Case --- Planets and their Moons as Named Entities}
\mynote
In the simplest examples entities of a type have names associated with them and these are unique and can be  quoted to reference entities of the type. 
Name attributes in these cases are archetypal examples of identifying attributes. 

We have already seen an example of the role of the naming of both planets and moons in the example given earlier for a phraseology for the communication of relationships
 in which we presented this statement:
\begin{equation}
\label{JupiterIoShort}
\mbox{\textit{Jupiter is orbited by Io.}}
\end{equation}
as a representation of an instance of the binary relationship
\begin{gather}
\label{planetMoonSecondInstance}
\raisebox{-1cm}{
\begin{erdiagram}{1.4}{6.9}

\eret{0.1}{-1}{1.85}{-0.4}{0.2}{1}\eretname{0.975}{-0.75}{}{planet}
\eret{5.15}{-1}{6.9}{-0.4}{0.2}{1}\eretname{6.025}{-0.75}{}{moon}

% relationship orbited by
\errelname{2}{-0.55}{l}{orbited by}\errelname{5}{-1}{r}{orbiting}\errelarm{1.85}{-0.7}{3.5}{-0.7}{0}{0}\errelarm{3.5}{-0.7}{5.15}{-0.7}{1}{0}\ercrowfoot{5}{-0.7}{5.15}{-0.55}{5.15}{-0.7}{5.15}{-0.85}{0}
\end{erdiagram}

}
\end{gather}
for in the context of statement (\ref{JupiterIoShort}), Jupiter is the name of, and therefore  a reference to, a planet and  Io is the name of, and therefore a reference to, a moon i.e. a natural satellite.\footnote{
Note that within the context of the solar system the name Io  doesn't uniquely identify this moon of Jupiter amongst all named solar system objects since it is also the name of an asteroid that orbits within the asteroid belt.
The names of moons are only unique as names of moons. 
Of course in other contexts, both Jupiter and Io are the names of mythological figures.
} 
We understand  (\ref{JupiterIoShort}), therefore, to be shorthand for the 
more pernickerty 
\begin{equation}
\mbox{\textit{The planet with name `Jupiter' is orbited by the moon with name `Io'.}}
\end{equation}

\mynote So planets and moons are named which is to say they have name attributes.
\commentary{Rewrite this paragraph.}
Of course name attributes are archetypal examples of identifying attributes --- we name things so that we can identify them in context. In such a situation the name of an entity, i.e. the value of the name attribute,  can be used in context to definitively identify the entity.
Quoting this value, i.e. using the name, is one way of referencing an entity of the type from within the context of another.
 \mynote
Focusing on just this aspect of the solar system we  draw the model shown in figure
\ref{planetMoonModel}
containing the relationship (\ref{planetMoonSecondInstance}) and showing name attributes of both moons and planets and
underlining these to showing that each of them is unique and therefore identifying. 
This is the entity model (albeit of a very small universe) that explains the communication (\ref{JupiterIoShort}) or equally, and as we shall see later, the structure of the data table of planets and moons given earlier.
\begin{erboxedFigure} {H}{planetMoonModel}
{
caption blah blah blah
 }
\begin{erdiagram}{2.4999999999999996}{6.9}

\eret{0.1}{-2.1}{1.85}{-1.2}{0.2}{1}\eretname{0.425}{-1.55}{l}{planet}
\erCoreAttribute{0.3}{-1.75}{1}{0}{name}{}
\eret{5.15}{-2.1}{6.9}{-1.2}{0.2}{1}\eretname{5.475}{-1.55}{l}{moon}
\erCoreAttribute{5.35}{-1.75}{1}{0}{name}{}
\eret{0}{-0.2}{6.9}{0.3}{0.2}{1}

% relationship 
\errelname{1.125}{-0.5}{l}{}\errelarm{0.975}{-0.2}{0.975}{-0.7}{1}{0}\errelarm{0.975}{-0.7}{0.975}{-1.2}{0}{0}\ercrowfoot{0.975}{-1.05}{0.825}{-1.2}{0.975}{-1.2}{1.125}{-1.2}{0}
% relationship 
\errelname{6.175}{-0.5}{l}{}\errelarm{6.025}{-0.2}{6.025}{-0.7}{1}{0}\errelarm{6.025}{-0.7}{6.025}{-1.2}{0}{0}\ercrowfoot{6.025}{-1.05}{5.875}{-1.2}{6.025}{-1.2}{6.175}{-1.2}{0}
% relationship orbited by
\errelname{2}{-1.5}{l}{orbited by}\errelname{5}{-1.95}{r}{orbiting}\errelarm{1.85}{-1.65}{3.5}{-1.65}{0}{0}\errelarm{3.5}{-1.65}{5.15}{-1.65}{1}{0}\ercrowfoot{5}{-1.65}{5.15}{-1.5}{5.15}{-1.65}{5.15}{-1.8}{0}
\end{erdiagram}

\end{erboxedFigure}
\mynote
We are very much stating the obvious of course but there is a pattern here and it is repeated over and over when instances of relationships are communicated  and the same pattern occurs more formally  and uniformly when instances of relationships are communicated in software systems and/or stored in data.

\begin{noteforfuture}
would like an example with two or three identifying attributes but --- cannot think of one. The ones I have given earlier are not very convincing.
\end{noteforfuture}

\subsection{Referencing within and without context -- Plays and Characters}
Consider the different contexts we might have for referencing a character of a play.

\mynote
If the reference is made in the context of a play then we will simply name the character as is the name attribute of character is identifying as shown here.

\begin{equation}
\label{characterAbsolute}
\raisebox{-1.5cm}{\begin{erdiagram}{2.3}{2.7}

\eret{0}{-2.3}{2.7}{-1.3}{0.2}{1}\eretname{0.27}{-1.65}{l}{character}
\erCoreAttribute{0.2}{-1.85}{1}{0}{name}{}
\eret{0}{-0.2}{2.7}{0.3}{0.2}{1}

% relationship having
\errelname{1.5}{-0.5}{l}{having}\errelname{1.5}{-1.15}{l}{within}\errelarm{1.35}{-0.2}{1.35}{-0.75}{0}{0}\errelarm{1.35}{-0.75}{1.35}{-1.3}{1}{0}\eridcomprel{1.25}{1.4500000000000002}{-1.05}\ercrowfoot{1.35}{-1.15}{1.2}{-1.3}{1.35}{-1.3}{1.5}{-1.3}{0}
\end{erdiagram}
}
\end{equation}
The absolute in this diagram is representing the context of the discussion in which we name the character and this context, we say, is of a particular play i.e. a particular dramatic work.
\mynote 
If the reference to a character is made in the context of a particular playwright but independently of any particular play then the reference will need to reference a play by title and reference the character by name. There will be two referentials as in Antonio from Twelfth Night in a discussion of William Shakespeare. 
\begin{equation}
\label{characterPlayAbsolute}
\raisebox{-1.5cm}{\begin{erdiagram}{4.199999999999999}{2.7}

\eret{0}{-2.3}{2.7}{-1.3}{0.2}{1}\eretname{0.27}{-1.65}{l}{play}
\erCoreAttribute{0.2}{-1.85}{1}{0}{title}{}
\eret{0}{-4.2}{2.7}{-3.2}{0.2}{1}\eretname{0.27}{-3.55}{l}{character}
\erCoreAttribute{0.2}{-3.75}{1}{0}{name}{}
\eret{0}{-0.2}{2.7}{0.3}{0.2}{1}

% relationship author of
\errelname{1.5}{-0.5}{l}{author of}\errelname{1.5}{-1.15}{l}{context}\errelarm{1.35}{-0.2}{1.35}{-0.75}{0}{0}\errelarm{1.35}{-0.75}{1.35}{-1.3}{1}{0}\eridcomprel{1.25}{1.4500000000000002}{-1.05}\ercrowfoot{1.35}{-1.15}{1.2}{-1.3}{1.35}{-1.3}{1.5}{-1.3}{0}
% relationship having
\errelname{1.5}{-2.6}{l}{having}\errelname{1.5}{-3.05}{l}{within}\errelarm{1.35}{-2.3}{1.35}{-2.75}{0}{0}\errelarm{1.35}{-2.75}{1.35}{-3.199}{1}{0}\eridcomprel{1.25}{1.4500000000000002}{-2.9499999999999997}\ercrowfoot{1.35}{-3.05}{1.2}{-3.2}{1.35}{-3.2}{1.5}{-3.2}{0}
\end{erdiagram}
}
\end{equation}
Finally if the reference to a character is made in the context of dramatical studies or performance in general rather than in the context of a particular playwright then three referentials will be given corresponding to this model 
\begin{equation}
\label{characterPlayPlaywrightAbsolute}
\raisebox{-1.5cm}{\begin{erdiagram}{6.099999999999999}{2.7}

\eret{0}{-2.3}{2.7}{-1.3}{0.2}{1}\eretname{0.27}{-1.65}{l}{playwright}
\erCoreAttribute{0.2}{-1.85}{1}{0}{name}{}
\eret{0}{-4.2}{2.7}{-3.2}{0.2}{1}\eretname{0.27}{-3.55}{l}{play}
\erCoreAttribute{0.2}{-3.75}{1}{0}{title}{}
\eret{0}{-6.1}{2.7}{-5.1}{0.2}{1}\eretname{0.27}{-5.45}{l}{character}
\erCoreAttribute{0.2}{-5.65}{1}{0}{name}{}
\eret{0}{-0.2}{2.7}{0.3}{0.2}{1}

% relationship study of
\errelname{1.5}{-0.5}{l}{study of}\errelname{1.5}{-1.15}{l}{context}\errelarm{1.35}{-0.2}{1.35}{-0.75}{0}{0}\errelarm{1.35}{-0.75}{1.35}{-1.3}{1}{0}\eridcomprel{1.25}{1.4500000000000002}{-1.05}\ercrowfoot{1.35}{-1.15}{1.2}{-1.3}{1.35}{-1.3}{1.5}{-1.3}{0}
% relationship author of
\errelname{1.5}{-2.6}{l}{author of}\errelname{1.5}{-3.05}{l}{by}\errelarm{1.35}{-2.3}{1.35}{-2.75}{0}{0}\errelarm{1.35}{-2.75}{1.35}{-3.199}{1}{0}\eridcomprel{1.25}{1.4500000000000002}{-2.9499999999999997}\ercrowfoot{1.35}{-3.05}{1.2}{-3.2}{1.35}{-3.2}{1.5}{-3.2}{0}
% relationship having
\errelname{1.5}{-4.5}{l}{having}\errelname{1.5}{-4.95}{l}{within}\errelarm{1.35}{-4.199}{1.35}{-4.649}{0}{0}\errelarm{1.35}{-4.649}{1.35}{-5.099}{1}{0}\eridcomprel{1.25}{1.4500000000000002}{-4.849999999999999}\ercrowfoot{1.35}{-4.95}{1.2}{-5.1}{1.35}{-5.1}{1.5}{-5.1}{0}
\end{erdiagram}
}
\end{equation}





In the context of a discussion of Shakespeare's play Twelfth Night we might say that
\begin{equation}
\label{AntonioLovesSebastian}
\mbox{\textit{The character Antonio loves Sebastian.}}
\end{equation} 
meaning
\begin{equation}
\label{AntonioLovesSebastianPedantic}
\mbox{\textit{The character name Antonio loves the character named Sebastian.}}
\end{equation} 
We are using the name attribute of type character of a play and relying on the fact that characters within plays are uniquely named. At this point we can represent the state of affairs we are depending on by underlining the name attribute of type character and drawing a bar across the within relationship as follows:
\begin{equation}
\label{personAttributes2}
\raisebox{-1.5cm}{\begin{erdiagram}{2.7}{3}

\eret{0}{-2.7}{3}{-0}{0.2}{1}\eretname{0.3}{-0.35}{l}{person}
\erCoreAttribute{0.2}{-0.55}{1}{0}{name}{}
\erCoreAttribute{0.2}{-0.85}{1}{0}{address}{}
\erCoreAttribute{0.2}{-1.15}{1}{0}{date of birth}{}
\erCoreAttribute{0.2}{-1.45}{1}{1}{nationality}{}
\erCoreAttribute{0.2}{-1.75}{0}{1}{passport number}{}
\erCoreAttribute{0.2}{-2.05}{0}{1}{is married}{}
\erCoreAttribute{0.2}{-2.35}{0}{1}{height}{}

\end{erdiagram}
}
\end{equation}


\mynote 
So we have, in the context of the plays of William Shakespeare,
\begin{equation}
\label{AntonioLovesSebastian}
\mbox{\textit{The character Antonio in Twelfth Night loves Sebastian.}}
\end{equation} 


This actor has played xxx in yyy, ddd in ccc, www in mmm.

\subsection{Context and Referencing Plays and Playwrights --- Getting to universals}
Characters and plays first because more convincing.

\subsection{Nested Context and Referencing.}
\mynote Title alone is not used because generally it is not unique.
There are two plays jean-Jacques Rousseau's "Pygmalion"
George Bernard Shaw's play of the same name.

Three universals to reference a character. Name of playwright, title of play, name of character.

\subsection{Somewhere}
\mynote
referential attributes represent data not concept.
\commentary{maybe in a new subsection on cascading identifiers}
\mynote
relationships model concept and context not data.
 
