

\section{Attributes}
\label{Attributes} 

\subsection{Data and Attribution}
\mynote
This section concerns the simplest aspects of things 
to which data can be meaningfully attributed. 
In entity modelling, these simplest aspects are known as \textit{attributes}.
We have  discussed entity types and relationships at length, and now  we describe this  third vital concept.
Working together, the concepts of entity type, relationship and attribute
are the foundation of entity modelling, making it a methodology that  bridges the gap between the world as we understand it (or some domain therein) and the structure of data, which data, after-all, purports to describe some part of what is. 

\mynote
By way of explanation, 
consider that when we call to mind \textit{data}, 
then we think of names, quantities, monetary values, 
addresses, dates, temperatures, geographical coordinates, and so on. 
Such items of data as these convey information only within specific contexts and when attributed 
to subjects at hand. 
A temperature, a colour, a price, a height, a distance — such items of data 
tell us nothing less they be the temperature, the colour, the price, the height, or the 
distance of \textit{some aspect} of \textit{some thing}. 
They tell us nothing, therefore, less they be attributed to an entity.

\mynote
A product in a catalogue or on a website  might have a full price and a sale price
and the given values will be monetary amounts.
The product has a name 
and it is likely that in the small print there is a product identifier.
Within the scope of this website, `product' is a type of entity
and `product identifier', `product name',  `full price' and `sale price' are attributes;
each attribute is constrained, in reality, in what can be meaningful attributed to it ---
it wouldn't make sense to say `size: red' or `colour: large' and
it  would be outside the bounds of expectation 
for a price to be given as `squillions' or `rainy' or as `10 miles'.

\mynote 
Each attribute within an entity model defines a named aspect of a type of entity and
also defines what can be meaningfully attributed to this aspect; 
for example, it makes sense to  attribute a planet within the solar system with
 orbital eccentricity
  --- the details we find on Wikipedia, for example --- 
 but it doesn't make sense to attribute your stay in a hotel with orbital eccentricity. 
 Instead your stay in a hotel may be attributed with arrival date and a departure date.

\mynote 
Attributes in an entity model are named and doubly constrained:
firstly the are constrained to the type of entity to which they apply, 
as aspects of entities of the type (planetary orbit rather than stay in a hotel);
secondly they are constrained in what it is that may be attributed to this aspect 
(a pure number, in the case of planetary eccentricity, 
but in other cases a colour or a date or, in
some cases, free text).

\mynote
In an entity model each type of entity has an enumerated list of named attributes. These attributes may be displayed on entity relationship diagram or they might be separately documented as, for example,
 like this:
\begin{verbatim}
product 
   => product identifier: text,
      product name: text,
      full price: monetary amount,
      sale_price: monetary amount
\end{verbatim}

What is important here is that within the context of an entity type there is a list of attributes, each attribute has a name and each attribute has a type of value specified (in this case either text or monetary amount). In this stylised layout the \verb!=>! can be read as introducing a list of attributes and the colon can be read aloud as "is" or as "is of type". 


For a more technical example, in some context I might chose to describe the orbit of any planet around the sun by the following attributes
\begin{verbatim}
planet 
   => aphelion: distance,
      eccentricity : real number between 0 and 1,
      mean anomaly: angular distance,
      inclination to ecliptic: angular distance,
      longitude of ascending node: angular distance
\end{verbatim}

The attribute \textit{aphelion} here represents the furthest distance from the sun. 
I do not need to include the closest distance to the sun --- the  \textit{perihelion} --- because
this can be calculated from \textit{aphelion} and \textit{eccentricity}.

Note that I could have chosen a different set of attributes. For example, among the three 
attributes \textit{perihelion}, \textit{aphelion} and \textit{eccentricity},  any two will suffice
to compute the third. There are many other attributes of  planetary orbits, but  the idea here has been to present a \textit{core set}. This matters. When data is stored or communicated  then 
we need only store or transmit such a core set  --- any values that can be derived from them need not be sent. We will return to this topic later.

\subsection{Types of Particulars and Types of Universals}
In entity modelling, the word \textit{type} is used in two distinct senses, each playing a different role in how we describe the world:
\begin{itemize}
\item An entity type classifies particular things—for example, the entity type Planet includes Earth, Mars, and Jupiter as instances. These are concrete, identifiable entities.
\item An attribute type, by contrast, is typically a determinable: a general kind of quality or measurement. Its instances are determinates, such as red, green, 6.4 km/s, or 23.5°C. These are not things in themselves, but possible values that can be attributed to entities.
\end{itemize}
This distinction helps clarify how attributes function in a model. An attribute identifies a particular aspect of an entity (such as its colour or its orbital period), and it is constrained by a type that defines what kind of value may be associated with that aspect (e.g. a colour, a duration, or a number).

In philosophy, a distinction is drawn between \textit{particulars} and \textit{universals}. 
Numbers, colours and  other determinate quantities are generally judged to be universals. 
We can say, then, that the two kinds of types that we are
talking about are 
\begin{itemize}
\item types all of whose instances are particulars (entity types), and
\item types all of whose instances are universals (attribute types).  
\end{itemize} 

\subsection*{Diagramming Attributes}
\mynote
Users of entity modelling may decide whether or not to include attributes on diagrams. Chen's original paper did not show attributes on diagrams  but in 1984
he published a paper that did show attributes diagrammatically represented. 
It is not clear that he expected this style to be used wholesale. 
\mynote
In this last mentioned 1984 paper Chen presents an example diagram 
which I have reproduced in figure  \ref{ChenStyleProjectWorkerAttributed} 
and which shows a graphical representation of attributes associated with the PROJECT-WORKER relationship (\ref{projectWorkerChenStyle}) that was previously given as an example relationship in his 1976 paper.

\begin{erboxedFigure} {H}{ChenStyleProjectWorkerAttributed}{From Chen 1984. In this diagram NUMBER, NAME, NUMBER-OF-YEARS, DATE and DOLLAR-AMOUNT must all be understood as types all of whose instances are universals. Chen uses the term `value type' for such types;
thus our universal things are what he calls values. In the same way it is 
the practice in programming to refer to names, numbers, dates and so on, as values but it is more enlightening to use  the term universal, as we do here, and to reflect on  the dichotomy between things that we treat as universals and things that we treat as particulars. }
\begin{pspicture}(-6,-4.0)(5.1,1.2)
%\psgrid
\chendiamond{pw}{\rput(0,-0.05){\begin{tabular}{c}PROJECT-\\WORKER\end{tabular}}}
\rput[l](-3.5,0){
   \chenbox{e}{EMPLOYEE}
	}
\rput[l](3.5,0){
   \chenbox{p}{PROJECT}
	}
%  relationship connectors
\ncline{-}{eE}{pwW}
\nbput{\scriptsize M}	
\ncline{-}{pwE}{pW}
\nbput{\scriptsize N}
%Attributes
\rput[l](-5.2,-3.0){
	\chenvaluetype{NUMBER}{NUMBER}
	}
\ncline{->}{eSSW}{NUMBERN}
\ncput*{\footnotesize EMP\#}
\rput[l](-3.5,-3.0){
	\chenvaluetype{NAME}{NAME}
	}
\ncline{->}{eS}{NAMEN}
\ncput*{\setlength{\tabcolsep}{0pt}\footnotesize\begin{tabular}{c}EMP-\\NAME\\[-0.1cm]\end{tabular}}
\rput[l](-1.8,-3.0){
	\chenvaluetype{YEARS}{\begin{tabular}{c}\\[-0.075cm]NUMBER-\\OF-\\YEARS\end{tabular}}
	}
\ncline{->}{eSSE}{YEARSN}
\ncput*{\footnotesize AGE}
%
\rput[l](0,-3.0){
	\chenvaluetype{DATE}{DATE}
	}
\ncline{->}{pwS}{DATEN}
\ncput*{\footnotesize\begin{tabular}{c}STARTING-\\DATE\\[-0.1cm]\end{tabular}}
%
\rput[l](2.6,-3.0){
	\chenvaluetype{NUMBER}{NUMBER}
	}
\ncline{->}{pSSW}{NUMBERN}
\ncput*{\footnotesize PROJ\#}
\rput[l](4.4,-3.0){
	\chenvaluetype{YEARS}{\begin{tabular}{c}DOLLAR-\\AMOUNT\end{tabular}}
	}
\ncline{->}{pSSE}{YEARSN}
\ncput*{\footnotesize BUDGET}

\end{pspicture}
\end{erboxedFigure}

The same model is shown in our variant of the Barker-Ellis notation
 in the diagram shown in figure \ref{employeeProjectWorkerMediatedAttributed}.
As in the Chen diagram, diagrams in Barker's book use a hash symbol to foreground key attributes though it  is positioned differently to this use by Chen. In our style as shown here the key attributes are underlined as will be explained in the next section.\footnote{Our use of underlining is consistent with relational notations used in theory papers such as Zaniola and in descriptions of relations in Chen 1976.}
\begin{erboxedFigure} {H}{employeeProjectWorkerMediatedAttributed}
{The example from figure \ref{ChenStyleProjectWorkerAttributed} recast in the our Barker-Ellis style. 
In this style the value types of attributes are not shown on the diagram but are expected to be 
separately documented. }
\begin{erdiagram}{2}{12.0745}

\eret{0.1}{-1.6}{2.1}{-0.1}{0.2}{1}\eretname{0.45}{-0.45}{l}{employee}
\erCoreAttribute{0.3}{-0.65}{1}{0}{emp-no}{}
\erCoreAttribute{0.3}{-0.95}{1}{1}{emp-name}{}
\erCoreAttribute{0.3}{-1.25}{1}{1}{age}{}
\eret{4.6}{-1.6}{7.575}{-0.1}{0.2}{1}\eretname{5.047}{-0.45}{l}{project assignment}
\erCoreAttribute{4.8}{-0.65}{1}{1}{starting-date}{}
\eret{10.075}{-1.6}{12.075}{-0.1}{0.2}{1}\eretname{10.425}{-0.45}{l}{project}
\erCoreAttribute{10.275}{-0.65}{1}{0}{proj-no}{}
\erCoreAttribute{10.275}{-0.95}{1}{1}{budget}{}

% relationship assigning
\errelname{4.45}{-0.7}{r}{assigning}\errelname{2.25}{-1.15}{l}{subject of}\errelarm{4.6}{-0.85}{3.349}{-0.85}{1}{0}\errelarm{3.349}{-0.85}{2.1}{-0.85}{0}{0}\ercrowfoot{4.45}{-0.85}{4.6}{-0.7}{4.6}{-0.85}{4.6}{-1}{0}\eridrefrel{4.35}{-0.75}{-0.95}
% relationship to work on
\errelname{7.725}{-0.7}{l}{to work on}\errelname{9.925}{-1.15}{r}{resourced by}\errelarm{7.574}{-0.85}{8.824}{-0.85}{1}{0}\errelarm{8.824}{-0.85}{10.07}{-0.85}{0}{0}\ercrowfoot{7.725}{-0.85}{7.575}{-0.7}{7.575}{-0.85}{7.575}{-1}{0}\eridrefrel{7.8245000000000005}{-0.75}{-0.95}
\end{erdiagram}

\end{erboxedFigure}



\subsection{Definitions}\footnote{“That is my definition, and if you don’t like it... well, I have others.”}

\mynote
In entity modelling, the term \textit{attribute} is used as a specific term
for a named functional relationship between a type all of whose instances are particulars  and a type all of whose instances are universals and which is said to be the type of the attribute.

\mynote
In the literature the notion of \textit{an attribute} is variously defined.
In the Goodland SSADM book an attribute is defined as
\begin{erquote}
  the smallest discrete component of the system information that is meaningful\footnote{\cite{SSADM}},
\end{erquote}
In  Shlaer Mellor we find an attribute is 
``the abstraction of a \textit{single} characteristic possessed by all the entities''
of a type. Whereas in Richard Barker's book we find that an attribute is
\begin{erquote}
any detail that serves to qualify, identify, classify, quantify or express an  entity,
\end{erquote}


For this reason the types which I describe as \textit{types all of whose instances are universals} are called \textit{value sets} by Chen in his  1976 paper.\footnote{
In Chen 1076 we find \textit{An attribute can be formally defined as a function which maps...into a value set} and also \textit{Since an attribute is defined as a function, it maps an entity in an entity set to a single value in a value set}.} 
To these we can add our definition of an attribute being a functional relationship of some kind that is distinct from the functional relationships between types of particulars.

\mynote
Chen's formal definition is an important insight.
It means that in entity modelling, the term \textit{attribute} is used as a specific term
for a named functional relationship between a type all of whose instances are particulars  and a type all of whose instances are universals of some kind.
The universals in question are often said to be \textit{values} but 
in philosopy  are said to be \textit{determinates}. 

The exact border between which things to consider to be entities and which to consider determinates is not a precise. 

and so we could say, more specifically, that an attribute is a named functional relationship between types of entities and types of determinates.

It is ironic that although attributes relate entities to values—
that is, particulars to determinates, and are therefore, in a sense, relationships,
they are not called such in entity modelling terminology.
It breaks the duck test, but that’s just how it is:
attributes are called attributes,
and the term relationship is reserved---both in entity modelling and in these pages---
for relationships between types of particulars; that is, between types of entities
and so the principal components of an entity model are its entity types,
their relationships, and their attributes.

\section{Uused Notes}

The detrminates can be more opne ended if in our modelling we can introduce new 
types of determinates just enumerating their values. 


\mynote
The weather forecast for a particular day next week at midday 
might be forecast as being sunny or cloudy or rainy with a temperature in degrees 
centigrade or farenheit along with predicted mean wind speed and gust wind speed
in miles per hour or kilometers per hour.

\mynote There is a particular entity, the entity according to its type has aspects.
Each aspect has some elementary data assiated with it -- a date, an numeric amount, a number of inches or centimetres, a temperature, a speed (mph or kph). 

\mynote
But some attributions do not make sense. Not every thing can be meaningfully attributed with colour or dimensions or date. 
Famously, not every thing has a price. 

\mynote
The product on the shelf cannot meaningfully be said to be rainy or sunny or cloudy.
No aspect of next weeks weather has length and width and height. 
\mynote 
Alongside the definition of types of entity and the relationships 
comes the definition of the aspects of the entitites to which elementaty data
can be attributed. Each such aspect is defined as an attribute. Attributes are defined to apply to types of entity. They are named.  
The elementary types of data include numbers, names, labels, identifiers, dates, times of day and so on.   
\mynote 
Entity models have a role in defining the meaningfulness and the extent of data
 by defining which attributions make sense for which types of entity.
\mynote 
Entity models define the extent and the meaningfulness of data relating to subject entities by defining which attributions make sense according to their type.
\mynote
These meaningful aspects of things, that are generally numerical or textual in nature, 
are defined in entity modelling by defining for every entity type in the  model 
 a set of named attributions that can be made of entitites of the type. 





\subsection{A curious thing}
\mynote
It is a curious thing that although attributes relate entities to values
i.e. particulars to determinates and therefore truly are relationships they do not
go by this name in entity modelling terminology. It breaks the duck test but
this is just how it is.
So attributes are called attributes (even though they
are relationships!) and the term relationship is reserved, in entity modelling
terminology and subsequently in these pages, for relationships between types of
particulars i.e. between types of entities. The principle components of an entity
model are its entity types and their relationships and attributes.

\begin{reinstatet}

\end{reinstatet}

\mynote 
To use entity modelling in this way in the description and construction of information systems, 
we require, alongside of the meta-concepts of \textit{entity type} and \textit{relationship}, a
third and vital meta-concept — that of \textit{attribute}. 

\subsection{Introduction}
\mynote  As stored within information systems then the individually represented items — the 
names, colours, quantities, the monetary amounts, the dates, etc — in the language 
of entity modelling, are said to be the values of attributes. Thus an actual name 
like "John Smith" is said to the value of a ‘name’ attribute of a ‘person’ entity.

We may express that information about a person is communicated or stored as 
message with two components name and data of birth we might say that a ‘person’ entity may be attributed a ‘name’ and a ‘date of birth’ 
on show an entity model diagram representing type ‘person’
with ‘name’ and ‘date of birth’ attribute annotations, like this:
\ercenterPicture{personNameDobAttribute}

\mynote Alternatively to say 
that date of birth is optional
we may use a circle in place of the square:
\ercenterPicture{personMandatoryNameAttribute}
To show that the name attribute within a message is the identifying attribute we underline the annotation in the diagram: 	
\ercenterPicture{personIdentifyingNameAttribute}

\mynote 
If it is necessary to give both a person's name \textit{and} their data of birth to uniquely identify 
them  then we underline
both of these attributes on the diagram:
\ercenterPicture{personIdentifyingNameDobAttribute}
Generally, systems will hold and communicate many different attributes of each type of entity and
these attributes are shown beneath the identifying attributes:
\ercenterPicture{personAttributes}

\mynote 
It is clear that computer programs are effective only in so long as the data items 
they manipulate are intended and understood as attributes of subject entities. It 
follows that to have an effective information system we must first have agreed types 
of subject entity and also what may be attributed to entities  of these types. In this agreement we agree the data content of the program or system i.e.  
its subject matter.

\mynote 
In a message about a person two or more phone numbers may be communicated
but it is a rule of entity modelling that for a single attribute an entity may only be attributed a single value. 
For this reason if a person can have multiple phone numbers then ‘phone number’ is 
not an attribute  of a ‘person’ entity type \textit{per se} but an attribute of a ‘phone’ entity type that 
stands in relation to (is owned by) the
‘person’ entity:
\ercenterPicture{personPhone}



More specifically, an attribute is a named functional relationship between a type of entity and some type other than a type of entity, whose instances, rather, are universals of some kind. These universals have been called in philosophy \textit{determinates}. In entity modelling the types of determinates are defined by enumerating their values. 

\mynote
In an undertaking where the goal is to set out \textit{what is}, then when a definition is given, it is best if it is given in term of things that have come before.  Attributes, as we come to describe them now, are a third element within entity modelling and they connect entities, and, indirectly relationships, 
with things  which must come before and are taken as primitive and universal. 

\mynote 
When we reference an instance then we reference by type and by quoting identifying features.
\mynote 
This is true of particulars and remains true of determinants
if we assume that a determinant is a type with no distinguishing features and
therefore a single instance.
\mynote
Attributes and relationships are differentiating features. 
Determinants are featureless types. We can have different representations of
determinants just as we have different synonyms for entity types.
\mynote 
How does this look in langauges like ML?

\mynote 
 What kind of things are Sunday, Monday, Tuesday, Wednesday, Thursday, Friday, and Saturday? They are \textit{days of the week}, obviously.

Likewise:
\begin{itemize}
  \item 0,1,2,3,... are the \textit{natural numbers}
  \item true, false are \textit{truth values};
  \item red, amber, green are the \textit{traffic light colours};
  \item clubs, diamonds, hearts, spades are the \textit{suits} in a standard pack of cards;
  \item metre, kilogram, second, ampere, kelvin, mole, candela are \textit{SI units};
  \item dot and dash are \textit{code elements} in morse code;
  \item doh, ray, me, fah, sol, la, te are the \textit{solfège syllables};
  \item 0, 1 are \textit{binary digits};
  \item 0, 1, 2, 3, 4, 5, 6, 7, 8, 9 are \textit{decimal digits};
  \item a, b, c, d, ... z are the \textit{lower case latin characters}.
\end{itemize}

\mynote  Wherever we use a number we agree that it is the same number as any other place we use it. We can say the same thing about the days of the week or
the truth values true and false. 
So what exactly are  numbers, truth values and days of the week? In what if any sense do they exist? 
\begin{notebox}[theory]
In a theory of data paper. Model categories with coproducts. Take away
the stuff about types of universals. Define the existence of sets of joint monics. 
Define that there may not be circularities in the set of monics. Deduce that there must be at least be objects which represent universals.

More precisely. If there are model instances in which an object has multiple instances,
n say, then there must be objects which are coproducts of at least n copies of the terminal object. Tempting to try and write this up. 
\end{notebox}

\mynote For these types of things, \textit{days of the week} 
\mynote I don't have a term for \textit{attribute type}. 
\mynote Sometimes the term \textit{domain} is used.
\mynote Chen uses the term \textit{value-set}.

\mynote
Though this is so, it is not usual to explain it in this way; my reason for doing so is that
I want to avoid falling into a trap that most authors fall into ---
it is my opinion that they short-change the reader when they come to describe what is meant by the term; instead, by defining the term \textit{attribute} properly and thinking about it properly 
we get a more coherent picture of entity modelling 
which on the one hands relates backwards into mathematics, the theory of knowledge and formal logic and on the other
points forward into the structure of data and program code. 

\mynote The background for the proper definition can be found in philosophical writings and probable dates back to Aristotle; for what is expressed in data is facts or knowledge or, if it be in error, at least has the form of facts or knowledge. Abstractly, at least, this form and its workings are an abstraction of the workings of ordinary language and  we can benefit from theories of knowledge informing our understanding of data.

\subsection{Particulars and Universals}  
\mynote 
In discussing attributes, we make use of the philosophical distinction between particulars and universals. 
The usual way of expressing it is that 
\begin{itemize}
\item	particulars are individual things like particular persons, tables, books, events and so on, 
\item universals are shared properties, types or qualities such as green, triangular, human.
\end{itemize}
\mynote 
On the distinction between particulars and universals, philosopher and mathematician Bertrand Russell in his classic 1912 introduction to philosophy
writes that
\begin{erquote}
When we examine common words, we find that, broadly speaking, proper names stand for particulars, while other substantives, adjectives, prepositions, and verbs stand for universals. Pronouns stand for particulars, but are ambiguous: it is only by the context or the circumstances that we know what particulars they stand for.\footnote{page 53. He also observes that
\textit{the word `now' stands for a particular, namely the present moment; but like pronouns, it stands for an ambiguous particular, because the present is always changing.}}
\end{erquote}

He writes that it can be seen that
\begin{erquote}
 no sentence can be made up without at least one word which denotes a universal...
\end{erquote}
and this  leads to 
\begin{erquote}
... all truths involve universals, and all knowledge of truths involves acquaintance with universals.
\end{erquote}
Later, regarding the words we use for universals, he says that, generally
\begin{erquote}
adjectives and common nouns express qualities or properties of single things, whereas prepositions and verbs tend to express relations between two or more things.\footnote{page 54. Later still on page 59 he seeks to establish the proposition that
all \textit{a priori} knowledge deals exclusively with relations between universals.
}
\end{erquote}


The values which are 

\mynote 
Numbers, whole numbers and floating point or real numbers, are universals;
 so to are characters and sequences of characters and so are the truth values (`true' and `false'). 
The types whose instances are universals include therefore
\begin{itemize}
	\item the type \textit{integer} of all whole numbers,
	\item the type \textit{float} of all floating point numbers,
	\item the type \textit{latinchar} of all latin characters,
	\item the type \textit{unicode} of all unicode characters,
	\item the type \textit{boolean} of truth values,
	\item the type \textit{string} of all finite sequences of characters.
\end{itemize}

\mynote By arrangement we can include
\begin{itemize}
	\item the type of colours of trafic lights \textit{red}, \textit{amber}, \textit{red};
	\item the type \textit{suit} whose instances are \textit{clubs}, \textit{diamonds}, \textit{hearts}, \textit{spades}.
\end{itemize}

\mynote We might also include
\begin{itemize}
	\item the type \textit{length} (is this the same as spatial distance)
	\item the type \textit{duration}
	\item the type \textit{date}
	\item the type \textit{time}
	\item the type \textit{temperature}
	\item the type \textit{angular separation} (i.e. a number of degrees or radians)
	\item \textit{force}
	\item \textit{pressure}
	\item \textit{speed}
	\item \textit{acceleration}
	\item \textit{colour}
	\item \textit{gps coordinate}
\end{itemize}



An instance of an attribute is sometimes said to be an \textit{attribution}.
\mynote
The idea is that universals are things we have a shared language for before we start modelling. 
\mynote 


\mynote I have said earlier that in entity modelling, the term attribute is adopted as a specific term meaning a relationship between a particular on the one hand and a universal on the other. Actually it would be more accurate to say that the term attribute is used for a \textit{functional} relationship between a type of particulars and a type of universals,
 i.e. between a type all of whose instances are particulars a type all of whose instances are universals.
When this functional relationship is evaluated at a particular entity then the result is said to be the value of the attribute. For this reason the types which I describe as \textit{types all of whose instances are universals} are called \textit{value sets} by Chen in his  1976 paper.\footnote{
In Chen 1076 we find \textit{An attribute can be formally defined as a function which maps...into a value set} and also \textit{Since an attribute is defined as a function, it maps an entity in an entity set to a single value in a value set}.} 

\mynote
 It is a curious thing that  although attributes relate entities to  values i.e. particulars to universals
 and therefore truly are relationships they do not go by this name in entity modelling terminology. It breaks the duck test but this is just how it is.  So attributes are called attributes (even though they are relationships!) andn the term relationship is reserved, in entity modelling terminology and subsequently in these pages, for relationships between types of particulars i.e. between types of entities. 
 The principle components of an entity model are its entity types and their relationships and attributes.

\mynote A final point on universals. In addition to the relationships and attributes as so far described there will exist many relationships between universals themselves
for example there will be logical and arithmetical relationships. When we construct an entity model these universal relationships are taken as givens --- we do not set out to model them.





