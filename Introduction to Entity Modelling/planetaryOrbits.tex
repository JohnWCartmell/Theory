
\section{Planetary Orbits}
\label{PlanetaryOrbits}

\begin{table}[H]
\small 
\setlength{\tabcolsep}{3pt}
\begin{tabular}{|l| 
  >{\centering\arraybackslash}m{0.9cm} | 
  c | 
  >{\centering\arraybackslash}m{1.45cm} | 
  >{\centering\arraybackslash}m{1.8cm} | 
  >{\centering\arraybackslash}m{1.6cm} | 
  >{\centering\arraybackslash}m{1.8cm}|}
\hline
\small Name & 
\small Semi-major axis \newline (AU) & 
\small Eccentricity &
\small Inclination \newline to ecliptic (°) & 
\small Longitude\newline of ascending\newline node \newline (°) & 
\small Argument\newline of \newline perihelion (°) & 
\small Time of \newline perihelion \\
\hline
Mercury & 0.387 & 0.205630 & 7.0049 & 48.331 & 29.124 & 2024-Dec-25 \\
Venus   & 0.723 & 0.006772 & 3.3947 & 76.680 & 54.884 & 2023-Dec-31 \\
Earth   & 1.000 & 0.016710 & 0.0000 & 0.000 & 102.937 & 2023-Jan-04 \\
Mars    & 1.524 & 0.093400 & 1.8506 & 49.558 & 286.502 & 2022-Jun-21 \\
Jupiter & 5.203  & 0.048498 & 1.3030 & 100.464 & 273.867 & 2023-Jan-20 \\
Saturn  & 9.555  & 0.055508 & 2.4852 & 113.665 & 339.392 & 2003-Jul-26 \\
Uranus  & 19.22  & 0.046295 & 0.7730 & 74.006 & 96.998 & 1966-Apr-14 \\
Neptune & 30.11  & 0.008988 & 1.7700 & 131.784 & 273.187 & 2042-Sep-05 \\
Pluto   & 39.48  & 0.248807 & 17.1400 & 110.299 & 113.834 & 1989-Sep-05 \\
\hline
\end{tabular}
\caption{Tables of planetary orbits}
\end{table}



\iffalse %some of my working out of the data model

\begin{table}[H]
\begin{minipage}{\textwidth}  %minipage so as to catch footnotes
\begin{tabular}{ | l | l | p{3cm}|}
\hline
\multicolumn{3}{|l|}{Orbital Characteristics}  \\
\hline
Aphelion&5.4570 AU (816.363 million km)        & \multirow{4}{3cm}{any two of these 
                                                                   four to specify the size and shape of the elipse}\\
Perihelion&4.9506 AU (740.595 million km)      &  \\
Semi-major axis\footnote{could 
instead use either of orbital period(synodic or sideral) or average orbital speed} 
                                               &5.2038 AU (778.479 million km) &  \\
Eccentricity&0.0489                            &  \\
\hline
Inclination to ecliptic\footnote{could instead use `inclination to sun\textquotesingle s equator' or `inclination to invariable plane'}& 1.303° & \multirow{2}{3cm}{these two specify the 
                                                               position of the orbital plane in space 
                                                                }\\
Longitude of ascending node&100.464° & \\[1cm]
\hline
Argument of perihelion&273.867° & This specifies the position of the elipse within the orbital plane\\
\hline
Time of perihelion\footnote{could instead use `mean anomaly' on a reference time} &January 21, 2023 & this positions the planet within its orbit \\
\hline
\end{tabular}
\end{minipage}
\caption{Collected orbital parameters illustrated for Jupiter}
\end{table}

\begin{table}[h]
\begin{tabular}{ | l | l | p{3cm}|}
\hline
Mean radius&69911 km & calculated\\
Equatorial radius&71492 km & \\
Polar radius&66854 km & \\
Flattening&0.06487 & calculated\\
Surface area&6.1469×1010 km2 & calculated\\
Volume&1.4313×1015 km3 & calculated\\
Mass&1.8982×1027 kg & \\
Mean density&1.326 g/cm3 & calculated\\
Surface gravity&24.79 m/s2 & calculated\\
Moment of inertia factor&0.2756±0.0006 & \\
Escape velocity&59.5 km/s & calculated\\
Synodic rotation period&9.9258 h & calculate from sideral rotation period \\
Sidereal rotation period&9.9250 hours  & \\
Equatorial rotation velocity&12.6 km/s & calculated\\
Axial tilt&3.13° (to orbit) & calculate this\\
North pole right ascension&268.057°; 17h 52m 14s & \\
North pole declination&64.495° & \\
Albedo&&\\  
&0.503 (Bond)  & \\
&0.538 (geometric) & can be calculated from diameter and absolute magnitude\\
Temperature&88 K (−185 °C)  & (blackbody temperature) \\
Surface temp.&min mean max  &\\
1 bar& 165 K & \\
0.1 bar&78 K, 128 K& \\
Apparent magnitude&−2.94[16] to −1.66 & \\
Absolute magnitude (H)&−9.4 & \\
Angular diameter&29.8" to 50.1" & \\
\hline
\end{tabular}
\caption{Collection of Physical Characteristics of Jupiter}
\end{table}

\begin{table}[h]
\begin{tabular}{ | l | l |}
\hline
Equatorial radius&71492 km  \\
Polar radius&66854 km  \\
Mass&1.8982×$10^27$ kg  \\
Sidereal rotation period&9.9250 hours  \\
North pole right ascension&268.057° \\
North pole declination&64.495° \\
Absolute magnitude (H)&−9.4 \\
Moment of inertia factor&0.2756±0.0006  \\
\hline
\end{tabular}
\caption{Possible data model for physical characteristics illustrated for Jupiter}
\end{table}
\fi

