\begin{noteforfuture}
 The Distinction between Composition and Reference \\
In structured entity modelling the distinction is made between composition relationships and reference relationships and this does such and such. Is this a real distinction? ETC ETC as per.

Diagrams expressing Scopes.\\
Document scope by drawing this diagram and asserting that it commutes. Or document by stating the equivalence of two paths. 
\end{noteforfuture}

\begin{noteforfuture}
 The airline flight example. Could model terminals that `support international flights' and also
 model characteristic of a flight whether it is international or not. Instead of having such boolean attribute of flight could instead model country in which a airport is located. 
\end{noteforfuture}

\begin{noteforfuture}
when we model data then we impose structure and that in some ways limits what we can say.
free text has no such limitation.
data may have elements of free text within it. Thus a description of a play may have individual lines of text attributed to charcaters in a structured way. To store the script of a play as a database. a novel can be stored as a blob or it can be broken down into chapeters one blob per chapter. Some element of structure is now recogised. none the less 'the data' is not very datalike rather that is is text like. an accounts databse will be much more datlike but may well have text elements embedded within it. The dream of structuralism is that all be understood as pure structure. A sentence can be parsed and its parse tree is more datalike and less text like.
\end{noteforfuture}

\begin{noteforfuture}
For discussion of universals in  the context of mereology see A.J.Cotnoir in my data/database literature review. In particular
\begin{erquote}
Universals are typically said to be ‘wholly located wherever they are instantiated’.
\end{erquote}
\end{noteforfuture}