\section{Dust Jacket}

Entity modelling is widely understood in software development—but too rarely used effectively. The reason is simple: the traditional gap between modelling and data definition has always demanded a corrective step called data normalisation. This book shows that the gap need never exist. By applying carefully chosen conceptual distinctions from category theory, the transition from entity modelling to data definition can be made seamless—without requiring any mathematical background.

What has been accepted as normal practice—tedious, repetitive, and expensive data normalisation—can be consigned to the past. This book presents entity modelling as it should have always been: clear, rigorous, and directly applicable.

\section{Perspective}

\label{Perspective}

\begin{newtt}
At its broadest, entity modelling is a technique and a notation for describing and communicating what is in the world. Taking a narrower view, its purpose is the conceptualisation of those things whose data are to be stored in information systems; we describe its use in the development of such systems whilst keeping the wider perspective that the purpose of an entity model is to provide a framework for knowledge, and that entity modelling is a form of conceptual modelling — a technique for the elaboration of concepts.\footnote{The wider truth is that the act of defining data is not separate from the act of understanding the world it represents.}

In the software development context, entity modelling is widely understood but less often used as a precursor to data definition. This book aims to change that by presenting a version of entity modelling in which the transition to data definition is seamless. This seamlessness is achieved through the careful consideration and application of conceptual distinctions drawn from a branch of mathematics known as category theory — without requiring the reader to have any particular mathematical background.

Historically, a stumbling block to the wider adoption of entity modelling has been the gap between modelling and data definition. Traditional methodologies have attempted to bridge this gap using a corrective step called data normalisation. The approach described in this book eliminates that gap entirely, rendering data normalisation unnecessary. It is both perplexing and regrettable that entity modelling has not previously been presented in this way — a situation that, I believe, represents a significant waste of time and money.

To some extent, any entity model provides an explanation of, or a guide to, the phrasing of facts about the things it describes, whether the phrasing be in common language (English in the examples in this book) or in data. However, historically, entity modelling has generally provided only partial explanations. This has been the very gap between modelling and data definition.

Of the many available, we focus on one particular flavour — the Barker–Ellis variant of the entity modelling notation. We augment this to give it additional firepower, specifically so that it is capable, very precisely, of fully guiding us to and fully explaining how we properly phrase facts about things or properly represent them in data — thereby closing the gap.
\end{newtt}


For good reason, we will make mention of mathematical concepts at a number of points in this book
but please be reassured --- my idea is that it is ok for you to skip any of these sections. 

To illustrate at the outset what we mean by conceptual modelling, consider the experience of reading into a new subject area and finding terms which seemingly have specific patterns of usage, and, it must be assumed, contextual meanings, but which patterns and meanings are unfamiliar to us. In so reading we are drawn into a systematic and iterative arrangement and a classification of the unfamiliar terms; in this process we will likely distinguish terms for individual things, for types or classes of things, for relations between things and also quantitative and adjectival terms. In this way it is inevitable that we will construct some sort of conceptual model. 
Entity modelling is a particular technique for formalising such models and, indeed, it is a technique and a notation used by information scientists seeking to represent and computerise the sometimes unfamiliar domains in which they work. When asked whether they understand a particular topic, an entity modeller might well affirm they do so only if they can sketch an entity model that frames the topic.

Thinking about understanding a new area and its language, though ‘things themselves’ are the subjects of the text, the language is of the types of things, the relations among the different types of things and the properties that can meaningfully be attributed to them. 

The premises of entity modelling have in them a pragmatic answer to the question what is knowledge? Knowledge, according to the premises of entity modelling, is knowledge about things. For any ‘thing’ the knowledge that we can have of it is a conjunction of elementary pieces each of which is either the fact of an attribution, by which is meant a property a thing has inherent in itself, or else the fact of a relationship with another thing. More precisely, we can have knowledge of the type of a thing and knowing its type is to know both the kind of attributions which may be made of it and the relationships in which it may participate. This is the theory of knowledge according to the entity modeller; it is also the basis of information modelling and therefore it is a pragmatic view of what knowledge is — it is that which can be represented as information in a structured form suitable for representation in a computer system or in a single computer program.

Frequently, computer systems and individual computer programs have as their subjects, everyday if not concrete and physical things, things such as people, accounts, orders, contracts, airline bookings, and so on; other computer programs have as their subjects the structures of molecules, languages, stellar processes or computer programs themselves; or things such as mathematical propositions, relationships in general, not particular, types of things as distinct from the things themselves, and so on. 
These are the sorts of entities that we are concerned with in this book and various points each such type appears as an entity type within an entity model for illustrative purposes. Other entity types we present in models will include knots within quipus, atoms within molecular structures, adjectives within the sentence structure of the English. 
In this book all these types of things appear in specific entity models and each entity model describes a particular domain of discourse. 
In each specific case this domain of discourse is the context within which there are types of things defined by the relations between them. 
In illustrative fragments we have chickens and their eggs and bicycles and their wheels. 
In some cases we lift examples from the literature of relational data theory and get a deeper understanding of them by looking at them through the lens of entity relationship modelling.



