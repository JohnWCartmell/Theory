\section{Entity Relationship Diagramming}
\label{EntityRelationshipDiagramming}

\subsection{Entity Models}

\mynote
An entity model emumerates types of things and the binary relationships these things may participate in and
most often entity models --- their types and relationships --- are presented as entity relationships diagrams
as described in this section and as are going play a big part of this book. 
Equally the nub of an entity model can be represented either as pure data 
or else in IDL-style langauge.\footnote{There have been many flavours of Interface Definition Language (IDL) the earliest being from Carnegie-Mellon in the 60's (70's)??} 
Before we say more about the diagrams, to give a bit of balance between, on the one hand, the diagrams and,
on the other, the information that they communicate, here is a representation in XML of the underlying model described by diagram ().\footnote{Note that in this XML representation there is some internal syntax used that resembles the syntax used in some variants of IDL and more fundamentally, as occurs in the syntax of regular expressions.} 

\begin{verbatim}
shlaerMellorDeptStudentProfessor0 surface xml needed here
To get this would need to convert this example into surface + presentation
\end{verbatim}

\subsection{Diagrams of Relationships}
\mynote
An entity relationship (ER) diagram is a presentation of an entity model.
It shows boxes for entity types and connections between types for relationships.
\mynote
The relative positions of the boxes and lines is generally reckoned to be immaterial. Therefore there 
are lots of different ways of representing the same entities and relationshoips on a diagram. 
\subsection{Example --- Chen Manufacturing Company}
\label{ChenManufacturingCompany}
In figure \ref{ChenManufacturingExample} I have reproduced an example entity relationship diagram  given in Chen's 1976 paper. This example documents principle types of entity in the organisation of an imaginary manufacturing company.  I wanted to redraw
this diagram in the Barker-Ellis style used in this book. To do this I needed to make some educated guesses about the meaning of some of Chen's relationships since use of the Barker-Ellis style  required me to descriptively label the relationships. Having made these guesses I drew the diagram shown
 in figure \ref{chenManufacturingCo..diagram}. 

\begin{erboxedFigure} {H}{ChenManufacturingExample}{
In effect this is the first ever entity relationship diagram. It is an entity relationship diagram in the Chen style from his seminal paper of 1976 and it is there described as being an analysis of information in a manufacturing firm. In this diagram
diamonds represent what we describe here as relationships and boxes represent types of entities.  
For completeness we should mention that in Chen's 1976 terminology the boxes were said to be entity sets and the diamonds were said to be relationship sets but by 1983 in a paper published in that year Chen is instead using the terms `entity type' and `relationship type'. While sympathising with this terminology we think it correct to to use the term `relationship' rather than 'relationship type' because this brings us closer to the terminology of formal logic.
}
\begin{center}
\scalebox{0.80}{\begin{pspicture}(-4.4,-5.80)(10.2,5.5)
%\psgrid

\rput[l](-3.0,5){
   \chenbox{dept}{DEPARTMENT}
}

\rput(-3,2.5){
	\chendiamond{de}{\rput(0,-0.05){\begin{tabular}{c}DEPT-\\EMP\end{tabular}}}
}

\rput[l](-3.0,0){
   \chenbox{emp}{EMPLOYEE}
	}
	
\rput(-3,-2.5){
	\chendiamond{ed}{\rput(0,-0.05){\begin{tabular}{c}EMP-\\DEP\end{tabular}}}
}

\rput[l](-3.0,-5){
   \chenbox{dpndt}{DEPENDENT}
	}
	
\rput(0,1.0){
	\chendiamond{pw}{\rput(0,-0.05){\begin{tabular}{c}PROJECT-\\WORKER\end{tabular}}}
}

\rput(0,-1){
	\chendiamond{pm}{\rput(0,-0.05){\begin{tabular}{c}PROJECT-\\MANAGER\end{tabular}}}
}

\rput[l](3.0,0){
   \chenbox{prj}{PROJECT}
	}

\ncline{-}{empNE}{pwW}
\naput{\scriptsize M}	
\ncline{-}{pwE}{prjNW}
\naput{\scriptsize N}

\ncline{-}{empSE}{pmW}
\nbput{\scriptsize 1}	
\ncline{-}{pmE}{prjSW}
\nbput{\scriptsize N}


\rput(6,0){
	\chendiamond{pp}{\rput(0,-0.05){\begin{tabular}{c}PROJECT-\\PART\end{tabular}}}
}

\rput[l](9,0){
   \chenbox{prt}{PART}
	}
\ncline{-}{prjE}{ppW}
\nbput{\scriptsize M}	
\ncline{-}{ppE}{prtW}
\nbput{\scriptsize N}

\rput(9,-2.5){
	\chendiamond{c}{\rput(0,-0.05){\begin{tabular}{c}COMPONENT\end{tabular}}}
}

\rput[l](6.0,5){
   \chenbox{supp}{SUPPLIER}
	}
	
\rput(6,2.5){
	\chendiamond{spp}{\rput(0,-0.05){\begin{tabular}{c}SUPP-PROJ-\\PART\end{tabular}}}
}

\ncline{-}{deptS}{deN}
\nbput{\scriptsize 1}	
\ncline{-}{deS}{empN}
\nbput{\scriptsize N}

\ncline{-}{empS}{edN}
\nbput{\scriptsize 1}	
\ncline{-}{edS}{dpndtN}
\nbput{\scriptsize ??}

\ncline{-}{suppS}{sppN}
\nbput{\scriptsize 1}	
\ncline{-}{sppE}{prtNW}     %  prtNW
\naput{\scriptsize P}
\ncline{-}{sppW}{prjNE}     %  prjNE
\nbput{\scriptsize M}

\ncline{-}{prtSSE}{cNE}     
\naput{\scriptsize N}	
\ncline{-}{prtSSW}{cNW}
\nbput{\scriptsize M}


\end{pspicture}}
\end{center}
\end{erboxedFigure}

\begin{erboxedFigure} {H}{chenManufacturingCo..diagram}{
Chen's 1976 entity relationship diagram (shown in figure \ref{ChenManufacturingExample}) 
redrawn as a structured entity model using the Barker-Ellis notation. 
 The rail at the top of the diagram represents the whole of the model which in this case is the manufacturing company in question.
I have had to guess at the meanings of some of Chen's relationships in figure \ref{ChenManufacturingExample}
 in order to label them meaningfully in this diagram. 
Having made my guess I have represented  the ternary relationship in Chen's diagram by type \textit{project supply option} and type \textit{part used on project}. 
I hope that you agree that this Barker-Ellis style diagram leaves less to the imagination than Chen's original diagram did.
}
%\begin{center}
\scalebox{0.95}{\begin{erdiagram}{6.3}{12.383}

\eret{0.8}{-1.6}{3.1}{-1}{0.2}{1}\eretname{1.95}{-1.35}{}{department}
\eret{0.8}{-3.5}{3.1}{-2.7}{0.2}{1}\eretname{1.95}{-3.05}{}{employee}
\eret{0.8}{-5.45}{3.1}{-4.85}{0.2}{1}\eretname{1.95}{-5.2}{}{dependent}
\eret{5.1}{-1.6}{8.1}{-1}{0.2}{1}\eretname{6.6}{-1.35}{}{project}
\eret{3.85}{-6.3}{6.502}{-5.1}{0.2}{1}\eretname{5.176}{-5.45}{}{project}\eretname{5.176}{-5.75}{}{worker}\eretname{5.176}{-6.05}{}{assignment}
\eret{7.374}{-3.75}{9.226}{-2.85}{0.2}{1}\eretname{8.3}{-3.2}{}{part used}\eretname{8.3}{-3.5}{}{on project}
\eret{7.413}{-6.25}{11.187}{-5.65}{0.2}{1}\eretname{9.3}{-6}{}{part supply option}
\eret{8.6}{-1.6}{9.811}{-1}{0.2}{1}\eretname{9.205}{-1.35}{}{part}
\eret{10.311}{-1.6}{12.133}{-1}{0.2}{1}\eretname{11.222}{-1.35}{}{supplier}
\eret{0}{-0.25}{12.383}{0.25}{0.2}{1}\eretname{4.279}{-0.2}{l}{Chen '76 Manufacturing Company}

% relationship 
\errelname{2.1}{-0.55}{l}{}\errelname{2.1}{-0.85}{l}{..}\errelarm{1.95}{-0.25}{1.95}{-0.625}{0}{0}\errelarm{1.95}{-0.625}{1.95}{-1}{1}{0}\ercrowfoot{1.95}{-0.85}{1.8}{-1}{1.95}{-1}{2.1}{-1}{0}
% relationship 
\errelname{6.75}{-0.55}{l}{}\errelname{6.75}{-0.85}{l}{..}\errelarm{6.6}{-0.25}{6.6}{-0.625}{0}{0}\errelarm{6.6}{-0.625}{6.6}{-1}{1}{0}\ercrowfoot{6.6}{-0.85}{6.45}{-1}{6.6}{-1}{6.75}{-1}{0}
% relationship 
\errelname{9.356}{-0.55}{l}{}\errelname{9.356}{-0.85}{l}{..}\errelarm{9.205}{-0.25}{9.205}{-0.625}{0}{0}\errelarm{9.205}{-0.625}{9.205}{-1}{1}{0}\ercrowfoot{9.205}{-0.85}{9.055}{-1}{9.205}{-1}{9.356}{-1}{0}
% relationship 
\errelname{11.372}{-0.55}{l}{}\errelname{11.372}{-0.85}{l}{..}\errelarm{11.22}{-0.25}{11.22}{-0.625}{0}{0}\errelarm{11.22}{-0.625}{11.22}{-1}{1}{0}\ercrowfoot{11.222}{-0.85}{11.072}{-1}{11.222}{-1}{11.372}{-1}{0}
% relationship employing
\errelname{1.8}{-1.9}{r}{employing}\errelname{1.8}{-2.55}{r}{employed by}\errelarm{1.95}{-1.6}{1.95}{-2.15}{0}{0}\errelarm{1.95}{-2.15}{1.95}{-2.7}{1}{0}\ercrowfoot{1.95}{-2.55}{1.8}{-2.7}{1.95}{-2.7}{2.1}{-2.7}{0}
% relationship depended on by
\errelname{1.8}{-3.8}{r}{depended on by}\errelname{1.8}{-4.7}{r}{depending on}\errelarm{1.95}{-3.5}{1.95}{-4.175}{0}{0}\errelarm{1.95}{-4.175}{1.95}{-4.85}{1}{0}\ercrowfoot{1.95}{-4.7}{1.8}{-4.85}{1.95}{-4.85}{2.1}{-4.85}{0}
% relationship subject_of
\errelname{2.675}{-3.8}{l}{subject}\errelname{2.675}{-4.1}{l}{of}\errelname{4.575}{-4.95}{r}{of}\errelname{4.575}{-4.65}{r}{assignment}\errelarm{2.525}{-3.5}{2.525}{-3.825}{0}{0}\errelarm{2.525}{-3.825}{2.525}{-4.15}{0}{0}\errelarm{2.525}{-4.15}{3.625}{-4.312}{0}{0}\errelarm{3.625}{-4.312}{4.725}{-4.475}{1}{0}\errelarm{4.725}{-4.475}{4.725}{-4.787}{1}{0}\errelarm{4.725}{-4.787}{4.725}{-5.1}{1}{0}\eridcomprel{4.625325}{4.825324999999999}{-4.85}\ercrowfoot{4.725}{-4.95}{4.575}{-5.1}{4.725}{-5.1}{4.875}{-5.1}{0}
% relationship resourced_by
\errelname{5.751}{-1.9}{l}{resourced}\errelname{5.751}{-2.2}{l}{by}\errelname{5.751}{-4.95}{l}{to}\errelname{5.751}{-4.65}{l}{assignment}\errelarm{5.6}{-1.6}{5.6}{-3.349}{0}{0}\errelarm{5.6}{-3.349}{5.6}{-5.1}{1}{0}\eridcomprel{5.50065}{5.7006499999999996}{-4.85}\ercrowfoot{5.601}{-4.95}{5.451}{-5.1}{5.601}{-5.1}{5.751}{-5.1}{0}
% relationship requires
\errelname{7.5}{-1.9}{l}{requires}\errelname{7.687}{-2.7}{r}{use by}\errelarm{7.35}{-1.6}{7.35}{-1.8}{0}{0}\errelarm{7.35}{-1.8}{7.35}{-2}{0}{0}\errelarm{7.35}{-2}{7.593}{-2.212}{0}{0}\errelarm{7.593}{-2.212}{7.836}{-2.425}{1}{0}\errelarm{7.836}{-2.425}{7.836}{-2.637}{1}{0}\errelarm{7.836}{-2.637}{7.836}{-2.85}{1}{0}\eridcomprel{7.7368749999999995}{7.936874999999999}{-2.6}\ercrowfoot{7.837}{-2.7}{7.687}{-2.85}{7.837}{-2.85}{7.987}{-2.85}{0}
% relationship managed by
\errelname{4.95}{-1.15}{r}{managed by}\errelname{3.25}{-3.264}{l}{managing}\errelarm{5.1}{-1.3}{4.5}{-1.3}{1}{0}\errelarm{4.5}{-1.3}{3.899}{-1.3}{1}{0}\errelarm{3.899}{-1.3}{3.774}{-2.132}{1}{0}\errelarm{3.774}{-2.132}{3.649}{-2.964}{0}{0}\errelarm{3.649}{-2.964}{3.374}{-2.964}{0}{0}\errelarm{3.374}{-2.964}{3.099}{-2.964}{0}{0}\ercrowfoot{4.95}{-1.3}{5.1}{-1.15}{5.1}{-1.3}{5.1}{-1.45}{0}
% relationship able to be_sourced via
\errelname{8.45}{-4.05}{l}{able to be}\errelname{8.45}{-4.35}{l}{sourced via}\errelname{8.15}{-5.5}{r}{supply of}\errelname{8.15}{-5.2}{r}{option for}\errelarm{8.299}{-3.75}{8.299}{-4.699}{0}{0}\errelarm{8.299}{-4.699}{8.299}{-5.649}{1}{0}\eridcomprel{8.2}{8.399999999999999}{-5.3999999999999995}\ercrowfoot{8.3}{-5.5}{8.15}{-5.65}{8.3}{-5.65}{8.45}{-5.65}{0}
% relationship subject_of
\errelname{9.356}{-1.9}{l}{subject}\errelname{9.356}{-2.2}{l}{of}\errelname{8.913}{-2.7}{l}{use of}\errelarm{9.205}{-1.6}{9.205}{-1.8}{0}{0}\errelarm{9.205}{-1.8}{9.205}{-2}{0}{0}\errelarm{9.205}{-2}{8.984}{-2.212}{0}{0}\errelarm{8.984}{-2.212}{8.763}{-2.425}{1}{0}\errelarm{8.763}{-2.425}{8.763}{-2.637}{1}{0}\errelarm{8.763}{-2.637}{8.763}{-2.85}{1}{0}\eridcomprel{8.663124999999999}{8.863124999999998}{-2.6}\ercrowfoot{8.763}{-2.7}{8.613}{-2.85}{8.763}{-2.85}{8.913}{-2.85}{0}
% relationship able to_provide
\errelname{11.372}{-1.9}{l}{able to}\errelname{11.372}{-2.2}{l}{provide}\errelname{10.394}{-5.5}{l}{aquire from}\errelname{10.394}{-5.2}{l}{option to}\errelarm{11.22}{-1.6}{11.22}{-1.95}{0}{0}\errelarm{11.22}{-1.95}{11.22}{-2.3}{0}{0}\errelarm{11.22}{-2.3}{10.73}{-3.624}{0}{0}\errelarm{10.73}{-3.624}{10.24}{-4.949}{1}{0}\errelarm{10.24}{-4.949}{10.24}{-5.3}{1}{0}\errelarm{10.24}{-5.3}{10.24}{-5.649}{1}{0}\eridcomprel{10.143625}{10.343625}{-5.3999999999999995}\ercrowfoot{10.244}{-5.5}{10.094}{-5.65}{10.244}{-5.65}{10.394}{-5.65}{0}
\end{erdiagram}
}
%\end{center}
\end{erboxedFigure}
As a footnote to figures \ref{ChenManufacturingExample} and \ref{chenManufacturingCo..diagram}, 
note that the  meaning of the term  `part' can itself be itself be a point of confusion ---
does it mean an actual physical part or  a part design?
I found a discussion of the ambiguity of the term  on wikipedia which stated that `part' as something that had a `part number' ususally meant `part design' rather than an instantiation of that design:
\begin{erquote}
As a part number is an identifier of a part design (independent of its instantiations), a serial number is a unique identifier of a particular instantiation of that part design. In other words, a part number identifies any particular (physical) part as being made to that one unique design; a serial number, when used, identifies a particular (physical) part (one physical instance), as differentiated from the next unit that was stamped, machined, or extruded right after it. This distinction is not always clear, as natural language blurs it by typically referring to both part designs, and particular instantiations of those designs, by the same word, ``part(s)''. Thus if you buy a muffler of P/N 12345 today, and another muffler of P/N 12345 next Tuesday, you have bought ``two copies of the same part'', or``two parts'', depending on the sense implied.
\end{erquote}

\subsection{Example --- Goodland Vehicle Hire Company}
\label{GoodlandVehicleHireCompany}

\mynote As a further example in figure \ref{goodlandSSADMcarHire..diagram}  I have reproduced an example from a book on SSADM\footnote{SSADM is an acronym for Structured Systems Analysis and Design Methodology.} by Goodland and Slater. It is, say the authors, a diagram specifying the logical data structure underlying the business operations of a hypothetical vehicle rental company that leases vans and trucks, and sometimes drivers, to its customers. 
\mynote
The example is developed and further elaborated at different points in the SSADM book as part of an extended example
of how to document, analyse, specify and support improved business systems. 

\mynote Hypothetically this company has many local offices and both drivers and vehicles are based at local offices. \commentary{improvise some text to go with this as a series of independent mynotes}

\begin{figure} [H]
\begin{tabular}{c}
\begin{erexample}
\scalebox{0.95}{\epsfbox{\handCraftedImagesFolder/goodlandVariantA.flex.eps}}
\end{erexample}\\
 \\[1cm]
\begin{erexample}
\scalebox{0.93}{\begin{erdiagram}{8.020000000000001}{12.671999999999999}

\eret{1.6}{-2.22}{3.6}{-1.3}{0.2}{1}\eretname{2.6}{-1.65}{}{customer}
\eret{0.433}{-4.74}{1.767}{-3.72}{0.2}{1}\eretname{1.1}{-4.07}{}{payment}
\eret{0.344}{-6.86}{1.856}{-5.94}{0.2}{1}\eretname{1.1}{-6.29}{}{allocated}\eretname{1.1}{-6.59}{}{payment}
\eret{3.367}{-7.52}{5.067}{-3.72}{0.2}{1}\eretname{4.217}{-4.07}{}{booking}\eretname{4.217}{-4.37}{}{/invoice}
\eret{7.35}{-2.16}{10.25}{-1.3}{0.2}{1}\eretname{8.8}{-1.65}{}{local}\eretname{8.8}{-1.95}{}{office}
\eret{11.3}{-7.52}{12.672}{-5.92}{0.2}{1}\eretname{11.986}{-6.27}{}{vehicle}\eretname{11.986}{-6.57}{}{category}
\eret{7.35}{-4.72}{8.45}{-3.97}{0.2}{1}\eretname{7.9}{-4.32}{}{driver}
\eret{8.55}{-6.27}{9.65}{-5.27}{0.2}{1}\eretname{9.1}{-5.62}{}{vehicle}
\eret{0}{-0.25}{12.672}{0.25}{0.2}{1}\eretname{6.336}{-0.2}{l}{}

% relationship 
\errelname{2.75}{-0.55}{l}{}\errelarm{2.6}{-0.25}{2.6}{-0.775}{0}{0}\errelarm{2.6}{-0.775}{2.6}{-1.3}{0}{0}\ercrowfoot{2.6}{-1.15}{2.45}{-1.3}{2.6}{-1.3}{2.75}{-1.3}{0}
% relationship 
\errelname{8.95}{-0.55}{l}{}\errelarm{8.799}{-0.25}{8.799}{-0.775}{0}{0}\errelarm{8.799}{-0.775}{8.799}{-1.3}{0}{0}\ercrowfoot{8.8}{-1.15}{8.65}{-1.3}{8.8}{-1.3}{8.95}{-1.3}{0}
% relationship 
\errelname{12.136}{-0.55}{l}{}\errelarm{11.98}{-0.25}{11.98}{-3.085}{0}{0}\errelarm{11.98}{-3.085}{11.98}{-5.92}{0}{0}\ercrowfoot{11.986}{-5.77}{11.836}{-5.92}{11.986}{-5.92}{12.136}{-5.92}{0}
% relationship sender of
\errelname{1.95}{-2.52}{r}{sender of}\errelname{0.95}{-3.57}{r}{sent by}\errelarm{2.1}{-2.22}{2.1}{-2.42}{0}{0}\errelarm{2.1}{-2.42}{2.1}{-2.62}{0}{0}\errelarm{2.1}{-2.62}{1.6}{-3.045}{0}{0}\errelarm{1.6}{-3.045}{1.1}{-3.47}{1}{0}\errelarm{1.1}{-3.47}{1.1}{-3.595}{1}{0}\errelarm{1.1}{-3.595}{1.1}{-3.72}{1}{0}\ercrowfoot{1.1}{-3.57}{0.95}{-3.72}{1.1}{-3.72}{1.25}{-3.72}{0}
% relationship maker of
\errelname{3.25}{-2.52}{l}{maker of}\errelname{4.197}{-3.57}{l}{made by}\errelarm{3.1}{-2.22}{3.1}{-2.42}{0}{0}\errelarm{3.1}{-2.42}{3.1}{-2.62}{0}{0}\errelarm{3.1}{-2.62}{3.573}{-3.045}{0}{0}\errelarm{3.573}{-3.045}{4.046}{-3.47}{1}{0}\errelarm{4.046}{-3.47}{4.046}{-3.595}{1}{0}\errelarm{4.046}{-3.595}{4.046}{-3.72}{1}{0}\ercrowfoot{4.047}{-3.57}{3.897}{-3.72}{4.047}{-3.72}{4.197}{-3.72}{0}
% relationship split into
\errelname{0.95}{-5.04}{r}{split into}\errelname{0.95}{-5.79}{r}{part of}\errelarm{1.1}{-4.74}{1.1}{-5.34}{0}{0}\errelarm{1.1}{-5.34}{1.1}{-5.939}{1}{0}\ercrowfoot{1.1}{-5.79}{0.95}{-5.94}{1.1}{-5.94}{1.25}{-5.94}{0}
% relationship made to
\errelname{2.006}{-6.7}{l}{made to}\errelname{3.217}{-5.47}{r}{for by}\errelname{3.217}{-5.17}{r}{paid}\errelarm{1.856}{-6.399}{2.006}{-6.399}{1}{0}\errelarm{2.006}{-6.399}{2.156}{-6.399}{1}{0}\errelarm{2.156}{-6.399}{2.561}{-6.01}{1}{0}\errelarm{2.561}{-6.01}{2.966}{-5.62}{0}{0}\errelarm{2.966}{-5.62}{3.166}{-5.62}{0}{0}\errelarm{3.166}{-5.62}{3.366}{-5.62}{0}{0}\ercrowfoot{2.006}{-6.4}{1.856}{-6.25}{1.856}{-6.4}{1.856}{-6.55}{0}
% relationship from_and to
\errelname{5.167}{-4.31}{l}{and to}\errelname{5.167}{-4.01}{l}{from}\errelname{7.2}{-1.58}{r}{start and end of}\errelarm{5.066}{-4.385}{5.516}{-4.385}{1}{0}\errelarm{5.516}{-4.385}{5.966}{-4.385}{1}{0}\errelarm{5.966}{-4.385}{6.283}{-3.057}{1}{0}\errelarm{6.283}{-3.057}{6.6}{-1.73}{0}{0}\errelarm{6.6}{-1.73}{6.975}{-1.73}{0}{0}\errelarm{6.975}{-1.73}{7.35}{-1.73}{0}{0}\ercrowfoot{5.217}{-4.385}{5.067}{-4.235}{5.067}{-4.385}{5.067}{-4.535}{0}
% relationship driven by
\errelname{5.167}{-5.284}{l}{driven by}\errelname{7.2}{-4.195}{r}{for}\errelname{7.2}{-3.895}{r}{driver}\errelarm{5.066}{-5.059}{5.816}{-5.059}{0}{0}\errelarm{5.816}{-5.059}{6.566}{-5.059}{0}{0}\errelarm{6.566}{-5.059}{6.758}{-4.702}{0}{0}\errelarm{6.758}{-4.702}{6.949}{-4.345}{0}{0}\errelarm{6.949}{-4.345}{7.149}{-4.345}{0}{0}\errelarm{7.149}{-4.345}{7.349}{-4.345}{0}{0}\ercrowfoot{5.217}{-5.06}{5.067}{-4.909}{5.067}{-5.06}{5.067}{-5.21}{0}
% relationship user of
\errelname{5.217}{-6.13}{l}{user of}\errelname{8.475}{-5.755}{r}{for}\errelname{8.475}{-5.455}{r}{used}\errelarm{5.066}{-5.904}{6.808}{-5.904}{0}{0}\errelarm{6.808}{-5.904}{8.549}{-5.904}{0}{0}\ercrowfoot{5.217}{-5.905}{5.067}{-5.755}{5.067}{-5.905}{5.067}{-6.055}{0}
% relationship requiring
\errelname{5.217}{-7.345}{l}{requiring}\errelname{11.15}{-7.42}{r}{required}\errelname{11.15}{-7.72}{r}{for}\errelarm{5.066}{-7.12}{6.066}{-7.12}{1}{0}\errelarm{6.066}{-7.12}{8.308}{-7.12}{1}{0}\errelarm{8.308}{-7.12}{10.42}{-7.12}{0}{0}\errelarm{10.42}{-7.12}{11.29}{-7.12}{0}{0}\ercrowfoot{5.217}{-7.12}{5.067}{-6.97}{5.067}{-7.12}{5.067}{-7.27}{0}
% relationship employer_of
\errelname{7.745}{-2.36}{r}{employer}\errelname{7.745}{-2.66}{r}{of}\errelname{7.97}{-3.87}{l}{at}\errelname{7.97}{-3.57}{l}{employed}\errelarm{7.844}{-2.16}{7.844}{-3.065}{0}{0}\errelarm{7.844}{-3.065}{7.844}{-3.97}{1}{0}\ercrowfoot{7.845}{-3.82}{7.695}{-3.97}{7.845}{-3.97}{7.995}{-3.97}{0}
% relationship base of
\errelname{9.36}{-2.46}{l}{base of}\errelname{9.36}{-5.12}{l}{at}\errelname{9.36}{-4.82}{l}{based}\errelarm{9.209}{-2.16}{9.209}{-3.715}{0}{0}\errelarm{9.209}{-3.715}{9.209}{-5.27}{1}{0}\ercrowfoot{9.21}{-5.12}{9.06}{-5.27}{9.21}{-5.27}{9.36}{-5.27}{0}
% relationship classified by
\errelname{9.8}{-5.62}{l}{classified by}\errelname{11.15}{-6.86}{r}{classifier of}\errelarm{9.649}{-5.77}{9.899}{-5.77}{1}{0}\errelarm{9.899}{-5.77}{10.14}{-5.77}{1}{0}\errelarm{10.14}{-5.77}{10.47}{-6.165}{1}{0}\errelarm{10.47}{-6.165}{10.79}{-6.56}{0}{0}\errelarm{10.79}{-6.56}{11.04}{-6.56}{0}{0}\errelarm{11.04}{-6.56}{11.29}{-6.56}{0}{0}\ercrowfoot{9.8}{-5.77}{9.65}{-5.62}{9.65}{-5.77}{9.65}{-5.92}{0}
\end{erdiagram}
}
\end{erexample}
\end{tabular}
\label{goodlandSSADMcarHireTwofold}
\caption{An example of two different ways of laying out the same entity types and relationships.
The upper layout is the more straightforward. The boxes representing entity types have been sized for the convenience of the layout of the lines representing relationships. The lower layout is en example of a top down diagram and is in a style that we promote in this book and which we refer to as a structured entity model.
Most of the diagrams in this bool follow this style.
 The entity types and relationships shown here are from  the Goodland Car Hire example. Goodland and Slater's  layout is different again as shown in figure 3.35 on pg 106 of their book. See also in their book, their figure 4.13 on pg. 159 and their figure 4.58 on pg. 213.
Consider simplifying and having a many-many betwixt payment and booking. }
\end{figure}

\footnote{Later will have a version as in the book with allocated payment as an intersection entity. 
Can we avoid needing bars on relationships in this way?}
\newpage



