\section{Identifying Features}
\label{IdentifyingFeatures}
\subsection{The Question of Identity}
\mynote
The subject that we are concerned with in this section 
is the identification of entities ---  regarding any particular type, how can entities of the type be distinguished from each other? The assumption that we make is that no two entities of a given type are alike in all their features. In actual fact this is a philosophical position that is called the principle of the identity of indiscernibles and it is a contested principle but in entity modelling we tend to assume the principle holds and, accordingly, that no two entities are exactly the same in all their features. 

Any set of features of a type, whether the whole set or a subset, if they suffice for the purpose of distinguishing the entities of the type then they are said to be identifying. Such a set of features gives us a way of identifying and therefore referencing entities of the type. 

 Generally, 
 the identifying features will be a combined set of attributes and relationships, 
 the relationships being functional relationships with other entities. 
Though it is the set is that is \textit{identifying}, 
informally, we shall say each of the attributes and relationships in the set is identifying. 
Such a set has a simple mathematical characterisation which we come to later after giving examples both of identifying features in various situations and of the notation that we use.

\subsection{The Significance of Identifying Features}

\mynote 
It is desirable that an entity model documents identifying features for the types of entities it describes and  it is  absolutely necessary for most types in those models that are to be used as data specifications as we will come to in a later section. 

\mynote
The reason for this is that identifying features give us concrete ways of identifying entities within a context and thereby enables us to describe the relationships between entities. Consequently, they provide methods for communicating  relationships and this ultimately facilitates the storage of representations of entities in data systems.




\subsection{Notation}
 \mynote
Various conventions have been used diagrammatically to convey that an attribute is identifying. In Barker's book the \# symbol is placed to the left of the attribute to indicate as much. In diagrams in this book the name of the attribute is underlined and this is consistent with papers on relational theory in which  the names of columns are underlined to show that they are key columns.

\mynote Identifying relationships on the other hand, both in this book and in Barker's book 
are distinguished by drawing a bar across them so that they like this: \barkerEllisJ\ or like this: \barkerEllisK.



 
\subsection{Examples}
\mynote
Flight number, payment card number, country code, international bank account number (IBAN), 
international standard book number (ISBN), social security number, passport number, part number and many like attributes are designed specifically for the purpose of enabling unique identification of entities of a particular type so that each entity of the type can be referenced. Such attributes are said to be identifying attributes and are underlined 
on diagrams to indicate as much. Attributes 
such as ISBN and IBAN are designed to be globally unique.  
Others such as:
\begin{itemize}
\item
car registration plate number and passport number have values that are unique, and are therefore identifying, only with respect to particular issuing authorities, for example at a national level,
\item part numbers may be unique with respect to particular part suppliers or manufacturers,
\item airport terminal numbers are unique and therefore identifying with respect to the airports they are located at.
\end{itemize}
In these bulleted cases the underlining of the identifying attribute is combined with the barring of the context providing relationship as shown here in fragments in which, in the absence of other detail, types \textit{passport}, \textit{part} and \textit{terminal} are shown to be uniquely identifying  in relation
to \textit{country}, \textit{supplier} and \textit{airport}, respectively, 
and in which these later types are shown to have instances uniquely identified by name:

\begin{tabular} {m{3.5cm} m{3.5cm} m{3.5cm}}
\begin{erdiagram}{4.5}{2.9000000000000004}

\eret{0.1}{-1}{2.8}{-0.1}{0.2}{1}\eretname{0.37}{-0.45}{l}{country}
\erCoreAttribute{0.3}{-0.65}{1}{0}{name}{}
\eret{0.1}{-4.5}{2.8}{-2.1}{0.2}{1}\eretname{0.37}{-2.45}{l}{passport}
\erCoreAttribute{0.3}{-2.65}{1}{0}{passport number}{}
\erCoreAttribute{0.3}{-2.95}{1}{1}{date of issue}{}
\erCoreAttribute{0.3}{-3.25}{1}{1}{valid until date}{}
\erCoreAttribute{0.3}{-3.55}{1}{1}{family name}{}
\erCoreAttribute{0.3}{-3.85}{1}{1}{given name}{}
\erCoreAttribute{0.3}{-4.15}{1}{1}{date of birth}{}

% relationship having_issued
\errelname{1.3}{-1.3}{r}{having}\errelname{1.3}{-1.6}{r}{issued}\errelname{1.6}{-1.95}{l}{issued by}\errelarm{1.45}{-0.999}{1.45}{-1.549}{0}{0}\errelarm{1.45}{-1.549}{1.45}{-2.099}{1}{0}\eridcomprel{1.35}{1.5500000000000003}{-1.8499999999999996}\ercrowfoot{1.45}{-1.95}{1.3}{-2.1}{1.45}{-2.1}{1.6}{-2.1}{0}
\end{erdiagram}
 &
\begin{erdiagram}{3.5999999999999996}{2.9}

\eret{0.1}{-1.3}{2.5}{-0.1}{0.2}{1}\eretname{0.34}{-0.45}{l}{supplier}
\erCoreAttribute{0.3}{-0.65}{1}{0}{name}{}
\erCoreAttribute{0.3}{-0.95}{1}{1}{address}{}
\eret{0.1}{-3.6}{2.5}{-2.4}{0.2}{1}\eretname{0.34}{-2.75}{l}{part}
\erCoreAttribute{0.3}{-2.95}{1}{0}{part number}{}
\erCoreAttribute{0.3}{-3.25}{1}{1}{description}{}

% relationship supplying
\errelname{1.15}{-1.6}{r}{supplying}\errelname{1.45}{-2.25}{l}{supplied by}\errelarm{1.3}{-1.3}{1.3}{-1.85}{0}{0}\errelarm{1.3}{-1.85}{1.3}{-2.4}{1}{0}\eridcomprel{1.2}{1.4000000000000001}{-2.15}\ercrowfoot{1.3}{-2.25}{1.15}{-2.4}{1.3}{-2.4}{1.45}{-2.4}{0}
\end{erdiagram}
 &
\begin{erdiagram}{3.3}{2.4}

\eret{0.1}{-1.3}{1.8}{-0.1}{0.2}{1}\eretname{0.27}{-0.45}{l}{airport}
\erCoreAttribute{0.3}{-0.65}{1}{0}{name}{}
\erCoreAttribute{0.3}{-0.95}{1}{1}{country}{}
\eret{0.1}{-3.3}{1.8}{-2.4}{0.2}{1}\eretname{0.27}{-2.75}{l}{terminal}
\erCoreAttribute{0.3}{-2.95}{1}{0}{number}{}

% relationship having
\errelname{0.8}{-1.6}{r}{having}\errelname{1.1}{-2.25}{l}{a facility of}\errelarm{0.95}{-1.3}{0.95}{-1.85}{0}{0}\errelarm{0.95}{-1.85}{0.95}{-2.4}{1}{0}\eridcomprel{0.85}{1.05}{-2.15}\ercrowfoot{0.95}{-2.25}{0.8}{-2.4}{0.95}{-2.4}{1.1}{-2.4}{0}
\end{erdiagram}

\end{tabular}

\subsubsection{Two or more identifying attributes}
For some types, the values of more than one attribute need be examined to distinguish entities of the type ---  as with a multi-dimensional coordinate system a number of attributes provide values that  taken together are unique 
and enable an entity to be uniquely identified and referenced.  
In such a case it is a number of attributes, latitude and longitude, for example, rather than an individual attribute, which as a set  can be said to be identifying. 
This is depicted by underlining each attribute that is in the set.
This is  shown in the following example
in which spot heights on maps are shown to be identifiable by the combination of latitude and longitude within the context of the map:
 \begin{equation}
 \raisebox{-0.5cm}{\begin{erdiagram}{3.9}{2.8}

\eret{0}{-1.3}{2.4}{-0.1}{0.2}{1}\eretname{0.24}{-0.45}{l}{map}
\erCoreAttribute{0.2}{-0.65}{1}{0}{title}{}
\erCoreAttribute{0.2}{-0.95}{1}{1}{scale}{}
\eret{0}{-3.9}{2.4}{-2.4}{0.2}{1}\eretname{0.24}{-2.75}{l}{spot height}
\erCoreAttribute{0.2}{-2.95}{1}{0}{latitude}{}
\erCoreAttribute{0.2}{-3.25}{1}{0}{longitude}{}
\erCoreAttribute{0.2}{-3.55}{1}{1}{altitude}{}

% relationship depicts
\errelname{1.05}{-1.6}{r}{depicts}\errelname{1.35}{-2.25}{l}{depicted on}\errelarm{1.2}{-1.3}{1.2}{-1.85}{0}{0}\errelarm{1.2}{-1.85}{1.2}{-2.4}{1}{0}\eridcomprel{1.0999999999999999}{1.3}{-2.15}\ercrowfoot{1.2}{-2.25}{1.05}{-2.4}{1.2}{-2.4}{1.35}{-2.4}{0}
\end{erdiagram}
}
 \end{equation} 
\mynote 
For a second example, 
consider modelling the type \textit{person} whose instances, of course, are people.
Now people are tricky to identify from a few well defined attributes but for some purposes at least the attributes full name, date of birth, and  home address might be considered sufficient. In an entity model this would be shown by underlining these three atributes like this:

\begin{equation}
\label{personAttributes2}
\raisebox{-1.5cm}{\begin{erdiagram}{2.7}{3}

\eret{0}{-2.7}{3}{-0}{0.2}{1}\eretname{0.3}{-0.35}{l}{person}
\erCoreAttribute{0.2}{-0.55}{1}{0}{name}{}
\erCoreAttribute{0.2}{-0.85}{1}{0}{address}{}
\erCoreAttribute{0.2}{-1.15}{1}{0}{date of birth}{}
\erCoreAttribute{0.2}{-1.45}{1}{1}{nationality}{}
\erCoreAttribute{0.2}{-1.75}{0}{1}{passport number}{}
\erCoreAttribute{0.2}{-2.05}{0}{1}{is married}{}
\erCoreAttribute{0.2}{-2.35}{0}{1}{height}{}

\end{erdiagram}
}
\end{equation}

\subsubsection{Intersection Entities}
\mynote
For some types in some circumstances there are no attributes and only relationships in the identifying set. 
We saw an example earlier
in figure \ref{employeeProjectWorkerMediatedAttributed}. In that figure there
is an entity type
\textit{project assignment} which implements a many-many relationship between employees and projects. 
Such entity types as this are common in the Barker-Ellis notation 
standing as they do in positions that would be occupied many-many relationships
 and depicted by  diamonds in Chen's notation. 
 Because the type `project assignment' implements a many-many relationship, we can be sure that entities of the type  are uniquely identified by 
 identifying the employee that they assign in combination with the project they assign to and this means that the relationships \textit{assigning} and \textit{to work on}, in combination, are identifying features of the type. This has been indicated in 
figure  \ref{employeeProjectWorkerMediatedAttributed} by the bars drawn across these two relationships. As I mentioned before such types as this are sometimes said to be intersection entities.


\subsection {Mathematical Characterisation}
\mynote From a mathematical perspective, 
each feature of an entity type can be represented by a function. A set of features is suffiecient to uniquely identify entitities of a type if and only if when thought  of in this way the set is jointly injective.   



 

\subsection{Summary}
\mynote A set of features of a type of entity i.e. a set of attributes and outgoing direction We say that al relationships, is said to be \textit{identifying} provided that the values of these features,
taken together, is guaranteed to uniquely identify an entity of the type i.e. to be such that no two distinct entities of the type can have identical values for all of the features. 

\mynote In the simplest cases a single attribute will suffice. 
Social security number, part no, 
passport number and like attributes are designed to uniquely identify entities of the type from others of the type. 
An example of this 
in  shown in figure \ref{boardingPass2} in which entities of type \textit{airline route} are uniquely identified by a \textit{flight number} attribute.
This is
indicated in the diagram  by the underlining of the name of this attribute where it appears on the diagram.\footnote{Aside: There is a trivial difference here from Barker's notation because he distinguishes the identifying attributes with a \# symbol where we here use  underlining.} 


\mynote A final point, as we have said, identifying attributes are of particular significance with regard to 
 how relationships are represented in data and therefore how communicated and stored.
They are also significant in evaluating the goodness of an entity model, as we will see later.





 
