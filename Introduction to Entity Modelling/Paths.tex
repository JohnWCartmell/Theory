


\section{Paths of Relationships}
\label{PathsofRelationships}


\mynote
In the terminology of Ellis Ellis, HC. A Refined Model for Definition of System Requirements. Database Journal, 12(3):2--9, 1982. , whereever in an entity model there is a path of single-valued relationships:
\begin{center}
%No idea why the rightpitchform command no longer exists
\newcommand{\rightpitchfork}{\text{\raisebox{0pt}{\rotatebox[origin=c]{-90}{\scalebox{0.7}{$\pitchfork$}}}}}
$\overset{r_1}{a \hspace{0.3em} \rightpitchfork \hspace{-0.35em} -  \cdot} \overset{r_2}{\rightpitchfork \hspace{-0.35em} -} \cdot ... \overset{r_n}{\rightpitchfork \hspace{-0.35em} -} b$

\end{center}
then the destination entity type b is said to be in the logical horizon of the source entity type a. In programming, equivalently, we might say that it was possible to navigate from any one a to a definitive other b. 


\mynote
Now if there are two such navigation paths between entity type a (the source) and entity type b (the destination) then a question naturally arises as to whether following one path is equivalent to following the other i.e whether starting at any entity of type a we arrive at the same destination entity of type b regardless of which of the two paths we follow. In an abstract mathematical setting, diagrams showing such equivalent paths are said to be commutative diagrams and methods of reasoning using such diagrams is the starting point of category theory.


