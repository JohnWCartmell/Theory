
\section{Scope}
\label{Scope}
If we can imagine examining all binary relationships and all their possible instances then we would find in this examination that a very  good number of  binary relationships have instances that  do not reach as far across the landscape of entities as would seem possible from an examination of the types of entity involved.

\mynote As an example consider the example given earlier of a triangle of relationships involving students, professors and departments. 
Now consider from the point of view of an individual student. 
According to the diagram this student may or may not have an advisor. Lets suppose that a certain student does have an advisor. Now related to this student there are potentially two different departments (i) the department that the student majors in (ii) the department that their advisor is on the staff of. Let us suppose that the diagram is part of an entity model of University X and suppose university X has a rule that students are advised by professors from the departments that they major in. Since the context for the diagram is University X then these two departments related to an individual student entity are bound to be identical. To reflect this property, in this book, we say that the triangle is a scope triangle.\footnote{Without wanting to put-off non-mathematicians I also need to point out that there is a mathematically algebraic way of describing the property that this triangle of relationships has in the context of University X. For now we can say that it a near-commuting triangle of relationships and leave it at that.}   


\begin{noteforfuture}Need an example here. It would be best for it to be an observation or an enhancement regarding an earlier example. Could it be from Barker?
\end{noteforfuture}

We use the term \textit{scope} and speak of the scope of a relationship as a way of articulating the  extent or range across which instances a relationship may reach.  

Its scope is a characteristic of a relationship. A relationship may be broad in scope or narrow in scope. The broader it is in in scope the more bits of information are needed to communicate its instances. 

In a programming or database context an understanding of scope comes an understanding of scope errors i.e. violations of scope. 

Exercises: Which of the following triangles can we expect to be near-commuting:
a) b) c) d)

\begin{noteforfuture}
Schlaer Mellor example. Documented as a danger. Here we don't see it as a danger but something to get ahead of. To document scope up front when relationships are documented in an entity model.
\end{noteforfuture}