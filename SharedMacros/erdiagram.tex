%
%  erdiagram.tex
%  *************
%  Macros to represent ER diagrams
%  *******************************
% 29/01/2019 Modify so that not reliant on the
%            default fontsize being 10pt by using
%            package anyfontsize and then
%            \fontsize{8}{10}\selectfont to set font to 8pt
% 06/02/2019 Pullback symbol implemented and minor tweaks to positioning 
%            and size of identifier symbol and relationship labels.
%            Accidental forked changes merged on 08/02/2019.
% 15/03/2019 Continuation of 29/01/2019. Need fix fontsize of 
%            ERrelname and ERscope.	 
% 22/09/2022 Restructure the representation of attributes.
%            This is to enable hierarchical and relational attributes to
%            be animated in Beamer slides. See erdiagram.beamer.renewals.tex.
% 23/09/2022 Add a representation \errelid. Share body with \errelname.
%            See also  erdiagram.beamer.renewals.tex.
% 11/10/2022 Add erAttribute so that display of core attributes can be 'switched off'.
% 25/08/2024 Try changing symbols for attributes using circle and half circle for
%            mandatory and optional and solid for core and hoolow for derivative.
% ***********************************************************
 \usepackage{anyfontsize}             % 29/01/2019 
  
%\begin{erdiagram}{#1 height}{#2 width} 
% ....
% ....
%\end{erdiagram}
\newenvironment{erdiagram}[2]
{%\pspicture*(-#1,0)(#2,0)
\pspicture*(0,-#1)(#2,0)
%\psgrid
}
{\endpspicture}

\definecolor{lightyellow}{cmyk}{0,0,0.3,0}
\definecolor{verylightgrey}{gray}{0.95}


% \eret{#1 x0} {#2 y0} {#3 x1} {#4 y1} {#5 corner radius} {#6 fill}
\newcommand {\eret}[6]
{ 
\ifthenelse{\equal{#6}{1}}
{\psframe[framearc=#5,fillstyle=solid,fillcolor=lightyellow](#1,#2)(#3,#4)}
{\psframe[framearc=#5,fillstyle=solid,fillcolor=white](#1,#2)(#3,#4)}
}

% et top 
\newcommand {\erettop}[4]
{
%\psframe[linestyle=none,linearc=2pt,cornersize=absolute,fillstyle=solid,fillcolor=lightyellow](#1,#2)(#3,#4)
\psline[linearc=2pt,fillstyle=none,fillcolor=lightyellow](#1,#4)(#1,#2)(#3,#2)(#3,#4)
}

% et bottom 
\newcommand {\eretbtm}[4]
{
%\psframe[linestyle=none,linearc=2pt,cornersize=absolute,fillstyle=solid,fillcolor=lightyellow](#1,#2)(#3,#4)
\psline[linearc=2pt,fillstyle=none,fillcolor=lightyellow](#1,#2)(#1,#4)(#3,#4)(#3,#2)
}

% et bottom left
\newcommand {\eretbl}[4]
{
%\psframe[linestyle=none,linearc=2pt,cornersize=absolute,fillstyle=solid,fillcolor=lightyellow](#1,#2)(#3,#4)
\psline[linearc=2pt,fillstyle=none,fillcolor=lightyellow](#1,#4)(#3,#4)(#3,#2)
}

% et middle left
\newcommand {\eretml}[4]
{
%\psframe[linestyle=none,linearc=2pt,cornersize=absolute,fillstyle=solid,fillcolor=lightyellow](#1,#2)(#3,#4)
\psline[linearc=2pt,fillstyle=none,fillcolor=lightyellow](#1,#2)(#3,#2)(#3,#4)(#1,#4)
}

% et top left
\newcommand {\erettl}[4]
{
%\psframe[linestyle=none,linearc=2pt,cornersize=absolute,fillstyle=solid,fillcolor=lightyellow](#1,#2)(#3,#4)
\psline[linearc=2pt,fillstyle=none,fillcolor=lightyellow](#1,#2)(#3,#2)(#3,#4)
}

% et bottom right
\newcommand {\eretbr}[4]
{
%\psframe[linestyle=none,linearc=2pt,cornersize=absolute,fillstyle=solid,fillcolor=lightyellow](#1,#2)(#3,#4)
\psline[linearc=2pt,fillstyle=none,fillcolor=lightyellow](#1,#2)(#1,#4)(#3,#4)
}

% et middle right
\newcommand {\eretmr}[4]
{
%\psframe[linestyle=none,linearc=2pt,cornersize=absolute,fillstyle=solid,fillcolor=lightyellow](#1,#2)(#3,#4)
\psline[linearc=2pt,fillstyle=none,fillcolor=lightyellow](#3,#4)(#1,#4)(#1,#2)(#3,#2)
}

% et top right
\newcommand {\erettr}[4]
{
\psline[linearc=2pt,fillstyle=none,fillcolor=lightyellow](#1,#4)(#1,#2)(#3,#2)
}

% \ergrp{#1 x0} {#2 y0} {#3 x1} {#4 y1} {#5 corner radius} {#6 fill}
% #5 corner radius is unused!
\newcommand {\ergrp}[6]
{ 
\ifthenelse{\equal{#6}{1}}
{\psframe[fillstyle=solid,fillcolor=verylightgrey](#1,#2)(#3,#4)}
{\psframe[fillstyle=solid,fillcolor=white](#1,#2)(#3,#4)}
}


% \ertext{#1 text}
% 15/03/2019
\newcommand {\erextrasmallitalictext}[1]
{\fontsize{7}{9}\selectfont \textit{#1}}

% 29/01/2019  
\newcommand {\ersmallitalictext}[1]
{\fontsize{8}{10}\selectfont \textit{#1}}

\newcommand {\ermediumitalictext}[1]
{\fontsize{10}{12}\selectfont \textit{#1}}

% \eretname {#1 x left of text} {#2 y top of text} {#3 text}
\newcommand {\olderetname}[3]
{
%shift down 0.1 for height of text the anchor at baseline (B)
\rput[bl]{0}(0,-0.1){\rput[Bl]{0}(#1,#2){\ersmallitalictext{#3}}}
}

% \errelarm {#1 x0} {#2 y0} {#3 x1} {#4 y1} {#5 ismandatory} {#6 isconstructed}
\newcommand {\errelarm}[6]
{
\ifthenelse{\equal{#6}{1}}
{
%%\psline[linewidth=0.5pt,linearc=.05,linestyle=dashed,dash=6pt 6pt]{-}(#1,#2)(#3,#4)}
\ifthenelse{\equal{#5}{1}}
{\psline[linewidth=1.5pt,linearc=.05,linecolor=lightgray]{-}(#1,#2)(#3,#4)}
{\psline[linewidth=1.5pt,linearc=.05,linecolor=lightgray,linestyle=dashed,dash=2pt 2pt]{-}(#1,#2)(#3,#4)}
}
{
\ifthenelse{\equal{#5}{1}}
{\psline[linewidth=0.9pt,linearc=.05]{-}(#1,#2)(#3,#4)}
{\psline[linewidth=0.9pt,linearc=.05,linestyle=dashed,dash=2pt 2pt]{-}(#1,#2)(#3,#4)}
}
}

% \errelangle {#1 x0} {#2 y0} {#3 x1} {#4 y1} {#5 x2} {#6 y2} {#7 ismandatory} {#8 isocnstructed}
\newcommand {\errelangle}[8]
{
\ifthenelse{\equal{#8}{1}}
{
%\psline[linewidth=0.5pt,linearc=.1,linestyle=dashed,dash=6pt 6pt]{-}(#1,#2)(#3,#4)(#5,#6)}
\ifthenelse{\equal{#7}{1}}
{\psline[linewidth=1.5pt,linearc=.05,linecolor=lightgray]{-}(#1,#2)(#3,#4)(#5,#6)}
{\psline[linewidth=1.5pt,linearc=.1,linecolor=lightgray,linestyle=dashed,dash=2pt 2pt]{-}(#1,#2)(#3,#4)(#5,#6)}
}
{
\ifthenelse{\equal{#7}{1}}
{\psline[linewidth=0.9pt,linearc=.1]{-}(#1,#2)(#3,#4)(#5,#6)}
{\psline[linewidth=0.9pt,linearc=.1,linestyle=dashed,dash=2pt 2pt]{-}(#1,#2)(#3,#4)(#5,#6)}
}
}

% \ercrowfoot {#1 x0} {#2 y0} {#3 x11} {#4 y11} {#5 x12} {#6 y12} {#7 x13} {#8 y13} {#9 isconstructed}
\newcommand {\ercrowfoot}[9]
{
\ifthenelse{\equal{#9}{1}}
{
\psline[linewidth=1.5pt,linearc=.05,linecolor=lightgray]{-}(#1,#2)(#3,#4)
\psline[linewidth=1.5pt,linearc=.05,linecolor=lightgray]{-}(#1,#2)(#5,#6)
\psline[linewidth=1.5pt,linearc=.05,linecolor=lightgray]{-}(#1,#2)(#7,#8)
}{
\psline[linewidth=0.9pt,linearc=.05]{-}(#1,#2)(#3,#4)
\psline[linewidth=0.9pt,linearc=.05]{-}(#1,#2)(#5,#6)
\psline[linewidth=0.9pt,linearc=.05]{-}(#1,#2)(#7,#8)
}
}


% \eridcomprel{#1 x1}{#2 x2}{#3 y}
\newcommand {\eridcomprel}[3]
{
\psline[linewidth=0.9pt](#1,#3)(#2,#3)
}

% \eridrefrel{#1 x}{#2 y1}{#3 y2}
\newcommand {\eridrefrel}[3]
{
\psline[linewidth=0.9pt](#1,#2)(#1,#3)
}

% \ertext {#1 x} {#2 y} {#3 text anchor} {#4 text}  PRIVATE
\newcommand {\ertext}[4]
{
\rput[B#3]{0}(#1,#2){\fontsize{8}{10}\selectfont #4}
}

% \eretname {#1 x} {#2 y} {#3 text anchor} {#4 text} 
\newcommand {\eretname}[4]
{
\ertext{#1}{#2}{#3}{#4}
}

%private
\newcommand {\errelnamebody}[4]
{
\rput[B#3]{0}(#1,#2){\erextrasmallitalictext{#4}}
}

% \errelname {#1 x} {#2 y} {#3 text anchor} {#4 text} 
\newcommand {\errelname}[4]
{
\errelnamebody{#1}{#2}{#3}{#4}    
}

%private
\newcommand {\errelidbody}[4]
{
\rput[B#3]{0}(#1,#2){\erextrasmallitalictext{#4}}
}

% \errelid {#1 x} {#2 y} {#3 text anchor} {#4 text} 
\newcommand {\errelid}[4]
{
\errelidbody{#1}{#2}{#3}{#4}    
}


%private
% \erscopebody {#1 x} {#2 y} {#3 text anchor} {#4 text}  15 March 2019
\newcommand {\erscopebody}[4]
{
\rput[B#3]{0}(#1,#2){\erextrasmallitalictext{#4}}
}

% \erscope {#1 x} {#2 y} {#3 text anchor} {#4 text}  15 March 2019
\newcommand {\erscope}[4]
{
  \erscopebody{#1}{#2}{#3}{#4}
}

% \erreletname {#1 x} {#2 y} {#3 text anchor} {#4 text}  15 March 2019
\newcommand {\erreletname}[4]
{
\rput[B#3]{0}(#1,#2){\fontsize{10}{12}\selectfont #4}
}

% \ergroupannotation {#1 x} {#2 y} {#3 text anchor} {#4 text}
\newcommand {\ergroupannotation}[4]
{
\ertext{#1}{#2}{#3}{#4}
}


% \errelseq {#1 x} {#2 y}
\newcommand {\erelseq}[2]
{
}
\newcommand {\erattrmarkermand}
{\fontsize{6}{8}\selectfont $\CIRCLE$}
%{\fontsize{6}{8}\selectfont $\blacksquare$}
\newcommand {\erattrmarkeropt}
{\fontsize{6}{8}\selectfont \RIGHTCIRCLE} %package wasysym
%{\fontsize{6}{8}\selectfont \CIRCLE} %package wasysym
\newcommand {\erderattrmarkermand}
{\fontsize{6}{8}\selectfont $\circle$}
%{\fontsize{6}{8}\selectfont $\square$}
\newcommand {\erderattrmarkeropt}
{\fontsize{8}{10}\selectfont $\rightcircle$}
%{\fontsize{8}{10}\selectfont $\circ$}

% \erattr {#1 x} {#2 y} {#3 ismandatory}{#4 identifying} {#5 text}
\newcommand {\erattr}[5]
{
\ifthenelse{\equal{#3}{1}}
{\rput[l]{0}(#1,#2){\erattrmarkermand \ersmallitalictext{\ifthenelse{\equal{#4}{0}}{\underline{#5}}{#5}}}}
{\rput[l]{0}(#1,#2){\erattrmarkeropt \ersmallitalictext{\ifthenelse{\equal{#4}{0}}{\underline{#5}}{#5}}}}
}

\newcommand{\erCoreAttribute}[5]{
\erattr{#1}{#2}{#3}{#4}{#5}
}

% Deprecated \erdattr
\newcommand {\erdattr}[5]
{
\ifthenelse{\equal{#3}{1}}
{\rput[l]{0}(#1,#2){\erderattrmarkermand \ersmallitalictext{\ifthenelse{\equal{#4}{0}}{\underline{#5}}{#5}}}}
{\rput[l]{0}(#1,#2){\erderattrmarkeropt \ersmallitalictext{\ifthenelse{\equal{#4}{0}}{\underline{#5}}{#5}}}}
}

%\erHierarchicalAttribute {#1 x} {#2 y} {#3 ismandatory}{#4 identifying} {#5 name} {#6 annotation}
\newcommand {\erHierarchicalAttribute}[6]
{
\erRelorHierAttribute{#1}{#2}{#3}{#4}{#5}{#6}
}

%\erRelationalAttribute {#1 x} {#2 y} {#3 ismandatory}{#4 identifying} {#5 name} {#6 annotation}
\newcommand {\erRelationalAttribute}[6]
{
\erRelorHierAttribute{#1}{#2}{#3}{#4}{#5}{#6}
}

%\erRelorHierAttribute {#1 x} {#2 y} {#3 ismandatory}{#4 identifying} {#5 name} {#6 annotation}
\newcommand {\erRelorHierAttribute}[6]
{
\ifthenelse{\equal{#3}{1}}
{\rput[l]{0}(#1,#2){\erderattrmarkermand \ersmallitalictext{\ifthenelse{\equal{#4}{0}}{\underline{#5}}{#5}#6}}}
{\rput[l]{0}(#1,#2){\erderattrmarkeropt \ersmallitalictext{\ifthenelse{\equal{#4}{0}}{\underline{#5}}{#5}#6}}}
}






% \erarc {#1 x0} {#2 y0} {#3 x1} {#4 y1} {#5 x2} {#6 y2} {#7 x3} {#8 y3}
\newcommand {\erarc}[8]
{
\psbezier[showpoints=false]{-}(#1,#2) (#3, #4)(#5,#6) (#7, #8)
}

% \erarc {#1 x0} {#2 y0} {#3 x1} {#4 y1} {#5 x2} {#6 y2} {#7 x3} {#8 y3}
\newcommand {\errelseq}[8]
{
\psbezier[showpoints=false]{-}(#1,#2) (#3, #4)(#5,#6) (#7, #8)
}
% \ertrace {#1 trace}   
\newcommand {\ertrace}[1]
{
}
