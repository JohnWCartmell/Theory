%
%  erdiag
%
  
%\begin{erdiagram}{#1 height}{#2 width} 
% ....
% ....
%\end{erdiagram}
\newenvironment{erdiagram}[2]
{%\pspicture*(-#1,0)(#2,0)
\pspicture*(0,-#1)(#2,0)
%\psgrid
}
{\endpspicture}

\definecolor{lightyellow}{cmyk}{0,0,0.3,0}

% \eret{#1 x0} {#2 y0} {#3 x1} {#4 y1} {#5 corner radius} {#6 fill}
\newcommand {\eret}[6]
{ 
\ifthenelse{\equal{#6}{1}}
{\psframe[framearc=#5,fillstyle=solid,fillcolor=lightyellow](#1,#2)(#3,#4)}
{\psframe[framearc=#5,fillstyle=solid,fillcolor=white](#1,#2)(#3,#4)}
}

% et top 
\newcommand {\erettop}[4]
{
%\psframe[linestyle=none,linearc=2pt,cornersize=absolute,fillstyle=solid,fillcolor=lightyellow](#1,#2)(#3,#4)
\psline[linearc=2pt,fillstyle=none,fillcolor=lightyellow](#1,#4)(#1,#2)(#3,#2)(#3,#4)
}

% et bottom 
\newcommand {\eretbtm}[4]
{
%\psframe[linestyle=none,linearc=2pt,cornersize=absolute,fillstyle=solid,fillcolor=lightyellow](#1,#2)(#3,#4)
\psline[linearc=2pt,fillstyle=none,fillcolor=lightyellow](#1,#2)(#1,#4)(#3,#4)(#3,#2)
}

% et bottom left
\newcommand {\eretbl}[4]
{
%\psframe[linestyle=none,linearc=2pt,cornersize=absolute,fillstyle=solid,fillcolor=lightyellow](#1,#2)(#3,#4)
\psline[linearc=2pt,fillstyle=none,fillcolor=lightyellow](#1,#4)(#3,#4)(#3,#2)
}

% et middle left
\newcommand {\eretml}[4]
{
%\psframe[linestyle=none,linearc=2pt,cornersize=absolute,fillstyle=solid,fillcolor=lightyellow](#1,#2)(#3,#4)
\psline[linearc=2pt,fillstyle=none,fillcolor=lightyellow](#1,#2)(#3,#2)(#3,#4)(#1,#4)
}

% et top left
\newcommand {\erettl}[4]
{
%\psframe[linestyle=none,linearc=2pt,cornersize=absolute,fillstyle=solid,fillcolor=lightyellow](#1,#2)(#3,#4)
\psline[linearc=2pt,fillstyle=none,fillcolor=lightyellow](#1,#2)(#3,#2)(#3,#4)
}

% et bottom right
\newcommand {\eretbr}[4]
{
%\psframe[linestyle=none,linearc=2pt,cornersize=absolute,fillstyle=solid,fillcolor=lightyellow](#1,#2)(#3,#4)
\psline[linearc=2pt,fillstyle=none,fillcolor=lightyellow](#1,#2)(#1,#4)(#3,#4)
}

% et middle right
\newcommand {\eretmr}[4]
{
%\psframe[linestyle=none,linearc=2pt,cornersize=absolute,fillstyle=solid,fillcolor=lightyellow](#1,#2)(#3,#4)
\psline[linearc=2pt,fillstyle=none,fillcolor=lightyellow](#3,#4)(#1,#4)(#1,#2)(#3,#2)
}

% et top right
\newcommand {\erettr}[4]
{
%\psframe[linestyle=none,linearc=2pt,cornersize=absolute,fillstyle=solid,fillcolor=lightyellow](#1,#2)(#3,#4)
\psline[linearc=2pt,fillstyle=none,fillcolor=lightyellow](#1,#4)(#1,#2)(#3,#2)
}

% \ergrp{#1 x0} {#2 y0} {#3 x1} {#4 y1} {#5 corner radius} {#6 fill}
% #5 corner radius is unused!
\newcommand {\ergrp}[6]
{ 
\ifthenelse{\equal{#6}{1}}
{\psframe[fillstyle=solid,fillcolor=lightgray](#1,#2)(#3,#4)}
{\psframe[fillstyle=solid,fillcolor=white](#1,#2)(#3,#4)}
}

% \eretname {#1 x left of text} {#2 y top of text} {#3 text}
\newcommand {\eretname}[3]
{
%shift down 0.1 for height of text the anchor at baseline (B)
\rput[bl]{0}(0,-0.1){\rput[Bl]{0}(#1,#2){\footnotesize \textit{#3}}}
}

% \errelarm {#1 x0} {#2 y0} {#3 x1} {#4 y1} {#5 ismandatory} {#6 isconstructed}
\newcommand {\errelarm}[6]
{
\ifthenelse{\equal{#6}{1}}
{
%%\psline[linewidth=0.5pt,linearc=.05,linestyle=dashed,dash=6pt 6pt]{-}(#1,#2)(#3,#4)}
\ifthenelse{\equal{#5}{1}}
{\psline[linewidth=1.5pt,linearc=.05,linecolor=lightgray]{-}(#1,#2)(#3,#4)}
{\psline[linewidth=1.5pt,linearc=.05,linecolor=lightgray,linestyle=dashed,dash=2pt 2pt]{-}(#1,#2)(#3,#4)}
}
{
\ifthenelse{\equal{#5}{1}}
{\psline[linewidth=0.9pt,linearc=.05]{-}(#1,#2)(#3,#4)}
{\psline[linewidth=0.9pt,linearc=.05,linestyle=dashed,dash=2pt 2pt]{-}(#1,#2)(#3,#4)}
}
}

% \errelangle {#1 x0} {#2 y0} {#3 x1} {#4 y1} {#5 x2} {#6 y2} {#7 ismandatory} {#8 isocnstructed}
\newcommand {\errelangle}[8]
{
\ifthenelse{\equal{#8}{1}}
{
%\psline[linewidth=0.5pt,linearc=.1,linestyle=dashed,dash=6pt 6pt]{-}(#1,#2)(#3,#4)(#5,#6)}
\ifthenelse{\equal{#7}{1}}
{\psline[linewidth=1.5pt,linearc=.05,linecolor=lightgray]{-}(#1,#2)(#3,#4)(#5,#6)}
{\psline[linewidth=1.5pt,linearc=.1,linecolor=lightgray,linestyle=dashed,dash=2pt 2pt]{-}(#1,#2)(#3,#4)(#5,#6)}
}
{
\ifthenelse{\equal{#7}{1}}
{\psline[linewidth=0.9pt,linearc=.1]{-}(#1,#2)(#3,#4)(#5,#6)}
{\psline[linewidth=0.9pt,linearc=.1,linestyle=dashed,dash=2pt 2pt]{-}(#1,#2)(#3,#4)(#5,#6)}
}
}

% \ercrowfoot {#1 x0} {#2 y0} {#3 x11} {#4 y11} {#5 x12} {#6 y12} {#7 x13} {#8 y13} {#9 isconstructed}
\newcommand {\ercrowfoot}[9]
{
\ifthenelse{\equal{#9}{1}}
{
\psline[linewidth=1.5pt,linearc=.05,linecolor=lightgray]{-}(#1,#2)(#3,#4)
\psline[linewidth=1.5pt,linearc=.05,linecolor=lightgray]{-}(#1,#2)(#5,#6)
\psline[linewidth=1.5pt,linearc=.05,linecolor=lightgray]{-}(#1,#2)(#7,#8)
}{
\psline[linewidth=0.9pt,linearc=.05]{-}(#1,#2)(#3,#4)
\psline[linewidth=0.9pt,linearc=.05]{-}(#1,#2)(#5,#6)
\psline[linewidth=0.9pt,linearc=.05]{-}(#1,#2)(#7,#8)
}
}


% \eridcomprel{#1 x1}{#2 x2}{#3 y1}{#4 ymid}{#5 y2}
\newcommand {\eridcomprel}[5]
{
\psline[linewidth=0.9pt](#1,#3)(#1,#5)
\psline[linewidth=0.9pt](#2,#3)(#2,#5)
\psline[linewidth=0.9pt](#1,#4)(#2,#4)
}

% \eridrefrel{#1 x1}{#2 xmid}{#3 x2}{#4 y1}{#5 y2}
\newcommand {\eridrefrel}[5]
{
\psline[linewidth=0.9pt](#1,#4)(#3,#4)
\psline[linewidth=0.9pt](#1,#5)(#3,#5)
\psline[linewidth=0.9pt](#2,#4)(#2,#5)
}


% \errelname {#1 x} {#2 y} {#3 text}
\newcommand {\errelname}[3]
{
\rput[l]{0}(#1,#2){\textit{#3}}
}
% \errelseq {#1 x} {#2 y}
\newcommand {\erelseq}[2]
{
}
% \erattr {#1 x} {#2 y} {#3 ismandatory}{#4 idenitfying} {#5 text}
\newcommand {\erattr}[5]
{
\ifthenelse{\equal{#3}{1}}
{\rput[l]{0}(#1,#2){{\tiny $\square$} {\footnotesize \textit{\ifthenelse{\equal{#4}{0}}{\underline{#5}}{#5}}}}}
{\rput[l]{0}(#1,#2){\footnotesize $\circ$ \textit{\ifthenelse{\equal{#4}{0}}{\underline{#5}}{#5}}}}
}

%\ifthenelse{\equal{#4}{1}}
% \ertext {#1 x} {#2 y} {#3 text anchor} {#4 text}
%{\rput[l]{0}(#1,#2){\footnotesize $\circ$ \underline{\textit{#5}}}}
\newcommand {\ertext}[4]
{
\rput[B#3]{0}(#1,#2){{\footnotesize #4}}
}
% \erarc {#1 x0} {#2 y0} {#3 x1} {#4 y1} {#5 x2} {#6 y2} {#7 x3} {#8 y3}
\newcommand {\erarc}[8]
{
\psbezier[showpoints=false]{-}(#1,#2) (#3, #4)(#5,#6) (#7, #8)
}

% \erarc {#1 x0} {#2 y0} {#3 x1} {#4 y1} {#5 x2} {#6 y2} {#7 x3} {#8 y3}
\newcommand {\errelseq}[8]
{
\psbezier[showpoints=false]{-}(#1,#2) (#3, #4)(#5,#6) (#7, #8)
}
% \ertrace {#1 trace}   
\newcommand {\ertrace}[1]
{
}
