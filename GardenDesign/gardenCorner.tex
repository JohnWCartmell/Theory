\documentclass[a4paper,landscape]{article}
\usepackage{geometry}
\geometry{margin=1.2cm}
\usepackage{pstricks}

\usepackage{pst-plot}
\usepackage{graphicx} % for \scalebox
\usepackage{xfp}      % for floating point arithmetic

\newdimen\pictureHeight
\pictureHeight=10cm

\newdimen\pictureWidth
\pictureWidth=25cm


\newdimen\gardenWidth
\newdimen\gardenHeight

\newcount\Scale
\Scale=40
\def\scaled#1{\dimexpr #1mm/\Scale *10 \relax}



\makeatletter
\newcommand{\cmActual}[1]{%
  \fpeval{round((\strip@pt#1)*40*25.4/72.27/10,1)}cm
}
\makeatother


\gardenWidth=\scaled{923}   % #1 = 923
\gardenHeight=\scaled{381}  % #1 = 381

\newdimen\negGardenWidth
\negGardenWidth= -\gardenWidth 
\newdimen\negGardenHeight
\negGardenHeight= -\gardenHeight


%Concrete Slab
\newdimen\xTopRightSlab \xTopRightSlab=\scaled{-611}
\newdimen\yTopRightSlab \yTopRightSlab=\scaled{-5} %?????????????
\newdimen\lengthSlab \lengthSlab=\scaled{200} %?????
\newdimen\widthSlab \widthSlab=\scaled{180} %?????
\newdimen\xBottomLeftSlab \xBottomLeftSlab=\dimexpr \xTopRightSlab - \lengthSlab \relax
\newdimen\yBottomLeftSlab \yBottomLeftSlab=\dimexpr \yTopRightSlab - \widthSlab  \relax

% Post One
\newdimen\xTopRightPostOne   \xTopRightPostOne=\scaled{-188}
\newdimen\yTopRightPostOne   \yTopRightPostOne=\scaled{4}
\newdimen\xBottomLeftPostOne \xBottomLeftPostOne=\dimexpr \xTopRightPostOne - \scaled{8} \relax
\newdimen\yBottomLeftPostOne \yBottomLeftPostOne=\dimexpr \yTopRightPostOne - \scaled{8} \relax
% Post Two
\newdimen\xTopRightPostTwo   \xTopRightPostTwo=\dimexpr \xBottomLeftPostOne - \scaled{181} \relax
\newdimen\yTopRightPostTwo   \yTopRightPostTwo=\scaled{5}
\newdimen\xBottomLeftPostTwo \xBottomLeftPostTwo=\dimexpr \xTopRightPostTwo - \scaled{10} \relax
\newdimen\yBottomLeftPostTwo \yBottomLeftPostTwo=\dimexpr \yTopRightPostTwo - \scaled{10} \relax
% Post Three
\newdimen\xTopRightPostThree   \xTopRightPostThree=\dimexpr \xBottomLeftPostTwo - \scaled{180} \relax
\newdimen\yTopRightPostThree   \yTopRightPostThree=\scaled{5}
\newdimen\xBottomLeftPostThree \xBottomLeftPostThree=\dimexpr \xTopRightPostThree - \scaled{10} \relax
\newdimen\yBottomLeftPostThree \yBottomLeftPostThree=\dimexpr \yTopRightPostThree - \scaled{10} \relax
% Post Four
\newdimen\xTopRightPostFour   \xTopRightPostFour=\dimexpr \xBottomLeftPostThree - \scaled{181} \relax
\newdimen\yTopRightPostFour   \yTopRightPostFour=\scaled{5}
\newdimen\xBottomLeftPostFour \xBottomLeftPostFour=\dimexpr \xTopRightPostFour - \scaled{10} \relax
\newdimen\yBottomLeftPostFour \yBottomLeftPostFour=\dimexpr \yTopRightPostFour - \scaled{10} \relax
% Post Five
\newdimen\xTopRightPostFive  \xTopRightPostFive=\dimexpr \xBottomLeftPostFour - \scaled{153} \relax
\newdimen\yTopRightPostFive   \yTopRightPostFive=\scaled{5}
\newdimen\xBottomLeftPostFive \xBottomLeftPostFive=\dimexpr \xTopRightPostFive - \scaled{10} \relax
\newdimen\yBottomLeftPostFive \yBottomLeftPostFive=\dimexpr \yTopRightPostFive - \scaled{10} \relax


% Outer End Wall
\newdimen\xTopRightOuterEndWall  \xTopRightOuterEndWall=\scaled{0} 
\newdimen\yTopRightOuterEndWall   \yTopRightOuterEndWall=\scaled{-5}
\newdimen\xBottomLeftOuterEndWall \xBottomLeftOuterEndWall=\dimexpr \xTopRightOuterEndWall - \scaled{477} \relax  %conjectured
\newdimen\yBottomLeftOuterEndWall \yBottomLeftOuterEndWall=\dimexpr \yTopRightOuterEndWall - \scaled{10} \relax  %100mm blocks
% Inner End Wall
\newdimen\xTopRightInnerEndWall  \xTopRightInnerEndWall=\scaled{0} 
\newdimen\yTopRightInnerEndWall   \yTopRightInnerEndWall=\dimexpr \yBottomLeftOuterEndWall -\scaled{7} \relax % want 50mm gap  but forced out to 70mm by concrete
\newdimen\xBottomLeftInnerEndWall \xBottomLeftInnerEndWall=\dimexpr \xTopRightInnerEndWall - \scaled{477} \relax  %conjectured
\newdimen\yBottomLeftInnerEndWall \yBottomLeftInnerEndWall=\dimexpr \yTopRightInnerEndWall - \scaled{10} \relax  %100mm bricks

% Key Point
\newdimen\xKeyPoint \xKeyPoint=-\scaled{447}
\newdimen\yKeyPoint \yKeyPoint=-\scaled{368}

% Lower Wall
\newdimen\lowerInnerInnerRadius \lowerInnerInnerRadius=\dimexpr \yBottomLeftInnerEndWall - \yKeyPoint \relax
\newdimen\lowerInnerOuterRadius \lowerInnerOuterRadius=\dimexpr \lowerInnerInnerRadius + \scaled{10} \relax  % 10cm brick
\newdimen\lowerOuterInnerRadius \lowerOuterInnerRadius=\dimexpr \lowerInnerOuterRadius + \scaled{5}  \relax  % 5cm gap
\newdimen\lowerOuterOuterRadius \lowerOuterOuterRadius=\dimexpr \lowerOuterInnerRadius + \scaled{10} \relax  % 10cm brick
\newdimen\xLowerCentre  \xLowerCentre=\dimexpr  \xKeyPoint + \lowerInnerInnerRadius \relax
\newdimen\yLowerCentre \yLowerCentre=\scaled{-368}

% Upper Wall
\newdimen\xUpperCentre \xUpperCentre=\xLowerCentre
\newdimen\yUpperCentre \yUpperCentre=\yLowerCentre

\newdimen\upperInnerInnerRadius \upperInnerInnerRadius=\dimexpr \lowerInnerInnerRadius + \scaled{136} \relax
\newdimen\upperInnerOuterRadius \upperInnerOuterRadius=\dimexpr \upperInnerInnerRadius + \scaled{10} \relax % 10cm brick
\newdimen\upperOuterInnerRadius \upperOuterInnerRadius=\dimexpr \upperInnerOuterRadius + \scaled{5}  \relax  % 5cm gap
\newdimen\upperOuterOuterRadius \upperOuterOuterRadius=\dimexpr \upperOuterInnerRadius + \scaled{10} \relax  % 10cm brick


\begin{document}

xxx

\vspace{2cm}



\begin{center}
\begin{pspicture}(-\scaled{1000},-\scaled{440})(\scaled{40},\scaled{40})
    \psaxes[Ox=-24,Oy=-10]{->}(-24,-10)(0,0)(-24,-10)
% Overall rectangle (923 × 381 cm real → 23.075 × 9.525 drawing)
    \psframe[linewidth=0.25pt](-\scaled{923},-\scaled{368})(0,0)

%Concrete Slab
    \psframe[linewidth=0.15pt, fillstyle=solid, fillcolor=lightgray](\xTopRightSlab,\yTopRightSlab)(\xBottomLeftSlab,\yBottomLeftSlab)
% Post One
    \psframe[linewidth=0.15pt, fillstyle=solid, fillcolor=lightgray](\xTopRightPostOne,\yTopRightPostOne)(\xBottomLeftPostOne,\yBottomLeftPostOne)
% Post Two
    \psframe[linewidth=0.15pt, fillstyle=solid, fillcolor=lightgray](\xTopRightPostTwo,\yTopRightPostTwo)(\xBottomLeftPostTwo,\yBottomLeftPostTwo)
% Post Three
    \psframe[linewidth=0.15pt, fillstyle=solid, fillcolor=lightgray](\xTopRightPostThree,\yTopRightPostThree)(\xBottomLeftPostThree,\yBottomLeftPostThree)
% Post Four
    \psframe[linewidth=0.15pt, fillstyle=solid, fillcolor=lightgray](\xTopRightPostFour,\yTopRightPostFour)(\xBottomLeftPostFour,\yBottomLeftPostFour)
% Post Five
    \psframe[linewidth=0.15pt, fillstyle=solid, fillcolor=lightgray](\xTopRightPostFive,\yTopRightPostFive)(\xBottomLeftPostFive,\yBottomLeftPostFive)
% Key Point
    \pscircle(\xKeyPoint,\yKeyPoint){0.05}
    \uput[45](\xKeyPoint,\yKeyPoint){\cmActual{\xKeyPoint}}
% Key Point Projection To y (House)
    \pscircle(0,\yKeyPoint){0.05}
% y distance label
    \rput[t]{90}(0.1,-5.0){-----136cm-----}
% Label Distance of Lower Key point to Upper Key Point
    \rput[t](-13,-9.35){-----136cm-----}

% Outer End Wall
\psframe[linewidth=0.15pt, fillstyle=solid, fillcolor=lightgray](\xTopRightOuterEndWall,\yTopRightOuterEndWall)(\xBottomLeftOuterEndWall,\yBottomLeftOuterEndWall)
% Inner End Wall
\psframe[linewidth=0.15pt, fillstyle=solid, fillcolor=lightgray](\xTopRightInnerEndWall,\yTopRightInnerEndWall)(\xBottomLeftInnerEndWall,\yBottomLeftInnerEndWall)

%centre point for curving walls
    \pscircle(\xLowerCentre,\yLowerCentre){0.05}
    \uput[45](\xLowerCentre,\yLowerCentre){\cmActual{\xLowerCentre}}
%lower wall
    \psarc[linewidth=1pt](\xLowerCentre,\yLowerCentre){\lowerInnerInnerRadius}{90}{180} 
    \psarc[linewidth=1pt](\xLowerCentre,\yLowerCentre){\lowerInnerOuterRadius}{104}{180} 
    \psarc[linewidth=1pt](\xLowerCentre,\yLowerCentre){\lowerOuterInnerRadius}{90}{180}
    \psarc[linewidth=1pt](\xLowerCentre,\yLowerCentre){\lowerOuterOuterRadius}{101.5}{180} 
%upper wall
    \psarc[linewidth=1pt](\xUpperCentre,\yUpperCentre){\upperInnerInnerRadius}{134}{180} 
    \psarc[linewidth=1pt](\xUpperCentre,\yUpperCentre){\upperInnerOuterRadius}{135}{180} 
    \psarc[linewidth=1pt](\xUpperCentre,\yUpperCentre){\upperOuterInnerRadius}{135.3}{180}
    \psarc[linewidth=1pt](\xUpperCentre,\yUpperCentre){\upperOuterOuterRadius}{136.3}{180} 
\end{pspicture}
\end{center}

\end{document}




