
\note \label{ccgatequivalence}From my thesis, 
\begin{point}
there is a category $\catGAT$ of generalised algebraic theories and interpretations,
\end{point}
\begin{point}
there is a category $\catCon$ of contextual categories,
\end{point}
\begin{point}
there is a functor $\ccat[C]: \catGAT \morph \catCon$,
\end{point}
\begin{point}
there is a functor $\gat[U]:\catCon \morph \catGAT$,
\end{point}
\begin{point}
the functor $\ccat[C]$ is an equivalence with inverse $\gat[U]$.
\end{point}

\note
The proof that categories $\catGAT$ and $\catCon$ are equivalent  is entirely trivial but runs to more than 50 pages. I have always interpreted this as meaning that generalised algebraic theories and contextual categories are more or less the same thing but if this is considered from the point of view of foundations then we have to tread carefully.

\note 
In section \ref{sectioninwhichinstantiationisdefined}, below, we define the notion of 
an `instantiation of  generalised algebraic theory $\gat[U]$ in  contextual category \catc'. Such an instantiation we also say is an \highlight{internal $\gat[U]$-structure in contextual category $\catc$}. 

\note
The category of internal $\gat[U]$-structures is the category whose objects
are pairs $\tuple{C,I}$ where $C$ is a contextual category and $I$ is an instantiation of the theory $U$ in the contextual category $C$ and whose morphisms between
 $\tuple{C,I}$ and $\tuple{C',I'}$ are pairs $\tuple{F, \eta}$ where
$F: C \morph C'$ is a contextual functor and $\eta: I \circ F \morph  I'$ is a natural transformation. 

\note 
As  noted in para \ref{ccgatequivalence}, to every generalised algebraic theory $\gat[U]$ there  is a contextual category $\CofU$ corresponding to $\gat[U]$. This category has as objects equivalence classes of contexts and realisations (as defined 
in \cite{Cartmell78} and  \cite{Cartmell86}). 
From these definitions it also follows that there is a trivial instantiation
of $\gat[U]$ in  $\CofU$. This is an initial object in
the category of internal $\gat[U]$-structures.

\note 
An alternative definition and one that is offered  in my thesis (though the terminology is different) is that an internal $\gat[U]$-structure in a contextual category \catcw is precisely 
\highlight{a contextual functor from the contextual} \highlight{category $\CofU$ to \catc}. 
Meta-mathematically the two definitions are equivalent\footnote{The proof that they are equivalent 
definitions is entirely straightforward though I didn't write it up my thesis. I did once write out the proof of a  corresponding lemma in regard to single-sorted algebraic theories; this was in my Msc dissertation.}.
\note
In this way, the category of internal $\gat[U]$-structures  is isomorphic to the coslice category
$\CofU \downarrow \catCon$. Needless to say this has an initial object which is the identity functor on  $\CofU$.
If\ $\gat[U]$ is a considered a type theory (whatever that is) then this initial object is what I believe Vladimir refers
to as the term model when speaking of the initiality conjecture. It is the contextual category
$\CofU$ along with the trivial instantiation of $\gat[U]$ in $\CofU$.

\note 
Instantiations of a theory $U$ in the contextual category $\Fam$ are said to be $U$-algebras. 
\newcommand{\hatU}{\rule{0pt}{12pt}\hat {\gat[U]}}
\note 
From the details of
section \ref{sectioninwhichinstantiationisdefined} 
in which instantiations are both defined and characterised 
it follows that \highlight{to every gat $\gat[U]$} there is \highlight{ a theory 
of internal $\gat[U]$-structures}. We shall denote this theory as \highlight{$\hatU$}.

\note
Examples of $\hatU$ in practice are given for the theory of monoids 
(section \ref{monoidsinstantiationexample})  and 
for the theory of monoids (section \ref{categoriesinstantiationexample}). 
\note 
The instantiations of $\hatU$  in $\Fam$ consist of  internal $\gat[U]$-structures  i.e. they consist of contextual categories \catcw along with particular instantiations $I$ of
the theory $\gat[U]$ in the contextual category \catc. \\
\highlight{The category of $\hatU$-algebras is (isomorphic to) the category of internal $\gatU$-structures.}
\note
An instantiation of $\hatU$ in an arbitrary contextual category
consists of  an internal internal $\gat[U]$-structure. This sounds a bit crazy but it isn't -- there are after all categories internal to other categories and it isn't much of a stretch to suppose these internal categories have internal $\gat[U]$'s inside of them. 

\note 
Definition of initial $U$-algebras. From my thesis:
\begin{tightquote}
Consider for a moment. Every theory $\gat[U]$ has a minimal model denoted $\KU$ built out of the closed terms of \gat[U]. Alternatively this minimal model is described just in terms of the structure $\CofU$. For example
if $1 \base A$ in $\CofU$ then 
$\KU(A)=Hom(1,A)$, otherwise if $1 \base A_1 \base ... \base A_n \base A$ in $\CofU$
then if $a_1 \in \KU(A_1)$, ... if $a_n \in \KU(A_n)(a_1,...a_{n-1})$ then 
$\KU(A)(a_1,...a_n)=\setsuchthat{a\in Hom_{\CofU}(1,A)}{a \circ p_A = a_n}$. \\
\end{tightquote} 

Followed by :
\begin{tightquote}
Now, the free U-algebras are the algebras $I$-$alg(\KUp)$ for $I: \gat[U] \morph \gat[U']$ an extension of $\gat[U]$ by constants alone. The finitely generated free U-algebras are those algebras where $\gat[U']$ is an extension by finitely many constants. \\
\end{tightquote}

\note
\label{termmodelEQfreealgebra}For any generalised algebraic theory $\gat[U]$ we have two different 
and therefore isomprophic descriptions of the initial object of the category of internal $\gatU$-structures:\\
\highlightpara{
\begin{equation}
K_{\hat{U}} \cong \bigtuple{\CofU, I_{triv}}
\end{equation}
}
where $I_{triv}$ is the trivial instantiation of $\gatU$ in $\CofU$.

\note
Is para \ref{termmodelEQfreealgebra} relevant
to the initiality conjecture or to formal (machine checked) theory?  It can be summarised 
by saying that the term model of a theory $\gatU$ is the initial algebra of a theory $\hatU$.
Is that useful? Another way of looking at it (is it helpful?) is that the open terms and types
of $\gatU$ correspond to the closed terms of $\hatU$. 
Is use of the word combinator appropraite here?

\note 
Note that if $\gat[U]$ is a single-sorted or many-sorted algebraic theory then 
$\hatU$ is generalised algebraic 
and so there is no equivalent of \ref{termmodelEQfreealgebra}., above, except in the generalised algebraic context.

\note As a worked example, we show in lemma \ref{internalmonoidlemma}, below, that the generalised algebraic theory of internal monoids can be expressed (using notation to be introduced in section 
\ref{contextualnotation}) as 
the theory of contextual categories plus:
\begin{gatrules}
\gatintros
\gatintroducing{M}
\ofT{M}{Ob} \\
\gatintroducing{unit}
\ofT{unit}{Hom(1,M)} \\
\gatintroducing{mult}
\ofT{mult}{Hom(\crossx{M}{M}{1},1)} \\
\gataxioms
\gatintroducing{ \gataxiomno{1} }
\tuple{p_M \circ unit,id_M} \circ mult =id_M \\
\gatintroducing{ \gataxiomno{2} }
\tuple{id_M,p_M \circ unit} \circ mult =id_M \\
\gatintroducing{ \gataxiomno{3} }
(\crossx{mult}{ id_M}{1}) \circ mult = (\crossx{id_M}{mult}{1}) \circ mult
\end{gatrules}

This is in agreement with Barr and Wells \cite{BarrandWells}, page 232, where they describe
monoids internal to  a category\footnote{Generally we would be thinking of a category with finite products including terminal object here, though as they say not all products need be available in the category for there to be an internal monoid.}
as an example of a finite product (FP) sketch.

\note As a second worked example  we derive the generalised algebraic theory of internal categories
 in lemma \ref{internalcategorylemma}.

\section{Shortcomings}

\note From my 1986 paper (\cite{Cartmell86})based on my thesis (\cite{Cartmell78}):
\begin{tightquote}
Every interpretation $I:U \morph U'$ induces a contextual functor 
$C(I):C(U) \morph C(U')$. Composition with $C(I)$ is a functor from 
$ConFunc(C(U'),Fam)$ to
$ConFunc(C(U),Fam)$. It is the functor $\Ialg:\Upalg \morph \Ualg$. Those functors
between categories of models which are induced in this way are called generalised
algebraic functors. We can show that all such functors have left adjoints.
This is
equivalent to a known generalisation of Lawvere \cite{LawvereAlgebraicTheories}'s theorem that all algebraic
functors have a left adjoint. 
\end{tightquote}
A year or too back, Jonathan Sterling asked me about the proof of this. The question threw me and I couldn't give a good answer and therefore I think there is work to be done in this area unless someone has cleaned up in the meantime.
\note
Sketches: You would think that we could give a definition for a sketch of contextual category 
similar to  definitions such as those of linear sketch, finite product (FP) sketch, finite discrete (FD) sketch and finite limit (FL) sketch. Such a CC sketch would be syntax-free way a way of defining a generalised algebraic theory and it would parallel and benefit from treatments given to other notions of theory versus category with additional structure. 
\note
Generators and relations: On the other hand we should be able to construct contextual categories by generators and  
relations as Maclane describes the construction of categories by generators and relations
[page 51,52 Categories for the Working Mathematician]. Maybe it isn't  straightforward  though because of the inductive nature of the types/objects. 
\note
Lawvere defines a functor which he calls algebraic-structure as a  left adjoint  to the algebraic-semantics functor which in turn is defined as the functor which takes a theory to its concrete category of algebras. 
The algebraic-structure functor is an answer to questions of the form ... what is the best way of modelling such and such algebraically? 
Conceptually this is most interesting and Lawvere gives lots of examples. 

Lawvere's algebraic structure functor generalises to the case of many-sorted algebraic theories but I don't see how we could generalise to the case of generalised algebraic theories. 




