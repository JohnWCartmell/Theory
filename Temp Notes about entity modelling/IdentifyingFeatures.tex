\section{Identifying Features}
\label{IdentifyingFeatures}

\mynote We say that a set of features of a type of entity i.e. a set of attributes and outgoing directional relationships, is identifying provided that the values of these features is guaranteed to uniquely identify an entity of the type i.e. to be such that no two distinct entities of the type can have identical values for all of the features.

\mynote Very often the set of identifying features of a type of entity will be a singleton set containing a single attribute. An example of this 
in  shown in figure \ref{BoardingPass2} in which entities of type \textit{airline route} are uniquely identified by the \textit{flight number} attribute. This is
indicated in the diagram  by the underlining of the name of this attribute where it appears on the diagram.\footnote{Aside: There is a trivial difference here from Barker's notation because he distinguishes the identifying attributes with a \# symbol where we here use  underlining.} 

\mynote In other cases the set of identifying features will consist of a number of attributes. In such case  the values of these attributes taken together, uniquely identify entities of the type. Context is most significant though.
If the context of an entity model is an individal map thenb a spot height is identified by a combination of latitude and longitude as shown here by underlining the two attributes latitude and longitude: 

\begin{center}
\begin{erdiagram}{2.85}{2.9}

\eret{0.1}{-0.25}{2.9}{0.15}{0.2}{1}\eretname{1.5}{-0.2}{}{map}
\eret{0.3}{-2.85}{2.7}{-1.35}{0.2}{1}\eretname{0.54}{-1.7}{l}{spot height}
\erCoreAttribute{0.5}{-1.9}{1}{0}{latitude}{}
\erCoreAttribute{0.5}{-2.2}{1}{0}{longitude}{}
\erCoreAttribute{0.5}{-2.5}{1}{1}{altitude}{}

% relationship 
\errelname{1.65}{-0.55}{l}{}\errelarm{1.5}{-0.25}{1.5}{-0.8}{0}{0}\errelarm{1.5}{-0.8}{1.5}{-1.35}{1}{0}\ercrowfoot{1.5}{-1.2}{1.35}{-1.35}{1.5}{-1.35}{1.65}{-1.35}{0}
\end{erdiagram}

\end{center}

However if the context is that of a catalogue of maps then latitude and longitude only identify within the context of the map on which the spot height is depicted. The outgoing contextual relationship \textit{depicted on}
has to be included in the set of identifying features for this type of entity. In the Barker-Ellis notation this is shown by drawing a bar across the relationship as shown here:  

\begin{center}
\begin{erdiagram}{3.9}{2.8}

\eret{0}{-1.3}{2.4}{-0.1}{0.2}{1}\eretname{0.24}{-0.45}{l}{map}
\erCoreAttribute{0.2}{-0.65}{1}{0}{title}{}
\erCoreAttribute{0.2}{-0.95}{1}{1}{scale}{}
\eret{0}{-3.9}{2.4}{-2.4}{0.2}{1}\eretname{0.24}{-2.75}{l}{spot height}
\erCoreAttribute{0.2}{-2.95}{1}{0}{latitude}{}
\erCoreAttribute{0.2}{-3.25}{1}{0}{longitude}{}
\erCoreAttribute{0.2}{-3.55}{1}{1}{altitude}{}

% relationship depicts
\errelname{1.05}{-1.6}{r}{depicts}\errelname{1.35}{-2.25}{l}{depicted on}\errelarm{1.2}{-1.3}{1.2}{-1.85}{0}{0}\errelarm{1.2}{-1.85}{1.2}{-2.4}{1}{0}\eridcomprel{1.0999999999999999}{1.3}{-2.15}\ercrowfoot{1.2}{-2.25}{1.05}{-2.4}{1.2}{-2.4}{1.35}{-2.4}{0}
\end{erdiagram}

\end{center}

\mynote Work on this example from page 3-13 of Barker.
\commentary{flight needs data  of departure (time of departure not required)}

\erboxedFigPicture{boardingPass2.tex}{H}
{This example is based on an example developed in the Barker book. I have simplified in some areas.}

\commentary{airline route versus flight reminds me of sassure discussion re: trains}

\begin{noteforfuture}
For discussion of universals in  the context of mereology see A.J.Cotnoir in my data/database literature review. In particular
\begin{erquote}
Universals are typically said to be ‘wholly located wherever they are instantiated’.
\end{erquote}
\end{noteforfuture}

\begin{noteforfuture}
unreal identities - range from ISBNs and such to system-specific identifiers. part numbers.
\end{noteforfuture}



 
