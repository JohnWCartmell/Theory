
\section{Scope}
\label{Scope}
If we can imagine examining all binary relationships and all their possible instances then we would find in this examination that a very  good number of  binary relationships have instances that  do not reach as far across the landscape of entities as seems possible from an examination of the types of entity involved. 

\begin{noteforfuture}Need an example here. It would be best for it to be an observation or an enhancement regarding an earlier example. Could it be from Barker?
\end{noteforfuture}

We use the term \textit{scope} and speak of the scope of a relationship as a way of articulating this extent or range across which instances a relationship may reach.  

Its scope is a characteristic of a relationship. A relationship may be broad in scope or narrow in scope. The broader it is in in scope the more bits of information are needed to communicate its instances. 

In a programming or database context an understanding of scope comes an understanding of scope errors i.e. violations of scope. 

\begin{noteforfuture}
Schlaer Mellor example. Documented as a danger. Here we don't see it as a danger but something to get ahead of. To document scope up front when relationships are documented in an entity model.
\end{noteforfuture}