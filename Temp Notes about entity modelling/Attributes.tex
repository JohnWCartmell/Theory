

\section{Attributes}
\label{Attributes}
\mynote I have said earlier that in entity modelling, the term attribute is adopted as a specific term meaning a relationship between a particular on the one hand and a universal on the other. Actually it would be more accurate to say that the term attribute is used for a \textit{functional} relationship between a type of particulars and a type of universals i.e. a type all of whose instances are universals. When this functional relationship is evaluated at a particular entity then the result is said to be the value of the attribute. For this reason the types which I describe as types all of whose instances are universals are called value types by Chen in his  1976 paper.

\mynote
In Chen's 1984 paper he presents as an example the diagram shown in figure 
\ref{AttributesExampleFromChen1984} which shows a graphical representation of attributes associated with the PROJECT-WORKER relationship (\ref{projectWorkerChenStyle}) that was an example in his 1976 paper.
\begin{figure}
\begin{pspicture}(-7,-4.0)(7,1.2)
%\psgrid
\chendiamond{pw}{\rput(0,-0.05){\begin{tabular}{c}PROJECT-\\WORKER\end{tabular}}}
\rput[l](-3.5,0){
   \chenbox{e}{EMPLOYEE}
	}
\rput[l](3.5,0){
   \chenbox{p}{PROJECT}
	}
%  relationship connectors
\ncline{-}{eE}{pwW}
\nbput{\scriptsize M}	
\ncline{-}{pwE}{pW}
\nbput{\scriptsize N}
%Attributes
\rput[l](-5.2,-3.0){
	\chenvaluetype{NUMBER}{NUMBER}
	}
\ncline{->}{eSSW}{NUMBERN}
\ncput*{\footnotesize EMP\#}
\rput[l](-3.5,-3.0){
	\chenvaluetype{NAME}{NAME}
	}
\ncline{->}{eS}{NAMEN}
\ncput*{\setlength{\tabcolsep}{0pt}\footnotesize\begin{tabular}{c}EMP-\\NAME\\[-0.1cm]\end{tabular}}
\rput[l](-1.8,-3.0){
	\chenvaluetype{YEARS}{\begin{tabular}{c}\\[-0.075cm]NUMBER-\\OF-\\YEARS\end{tabular}}
	}
\ncline{->}{eSSE}{YEARSN}
\ncput*{\footnotesize AGE}
%
\rput[l](0,-3.0){
	\chenvaluetype{DATE}{DATE}
	}
\ncline{->}{pwS}{DATEN}
\ncput*{\footnotesize\begin{tabular}{c}STARTING-\\DATE\\[-0.1cm]\end{tabular}}
%
\rput[l](2.6,-3.0){
	\chenvaluetype{NUMBER}{NUMBER}
	}
\ncline{->}{pSSW}{NUMBERN}
\ncput*{\footnotesize PROJ\#}
\rput[l](4.4,-3.0){
	\chenvaluetype{YEARS}{\begin{tabular}{c}DOLLAR-\\AMOUNT\end{tabular}}
	}
\ncline{->}{pSSE}{YEARSN}
\ncput*{\footnotesize BUDGET}

\end{pspicture}
\caption{From Chen 1984. In this diagram NUMBER, NAME, NUMBER-OF-YEARS, DATE and DOLLAR-AMOUNT must all be understood as types all of whose instances are universals. Chen uses the term `value type' for such types. For him our universal things are values. We  say that names, numbers, dates etc are values.}
\label{AttributesExampleFromChen1984}
\end{figure}

\begin{figure}
\begin{erdiagram}{2.3}{14.299999999999999}

\eret{0.1}{-1.9}{2.6}{-0.4}{0.2}{1}\eretname{0.5}{-0.75}{l}{EMPLOYEE}
\erCoreAttribute{0.3}{-0.95}{1}{1}{EMP\#}{}
\erCoreAttribute{0.3}{-1.25}{1}{1}{EMP-NAME}{}
\erCoreAttribute{0.3}{-1.55}{1}{1}{AGE}{}
\eret{5.7}{-1.9}{8.7}{-0.4}{0.2}{1}\eretname{6.15}{-0.75}{l}{PROJECT-}\eretname{6.15}{-1.05}{l}{WORKER-}\eretname{6.15}{-1.35}{l}{ASSIGNMENT}
\erCoreAttribute{5.9}{-1.55}{1}{1}{STARTING-DATE}{}
\eret{11.8}{-1.9}{14.3}{-0.4}{0.2}{1}\eretname{12.2}{-0.75}{l}{PROJECT}
\erCoreAttribute{12}{-0.95}{1}{1}{PROJ\#}{}
\erCoreAttribute{12}{-1.25}{1}{1}{BUDGET}{}

% relationship assigning
\errelname{5.55}{-1}{r}{assigning}\errelname{2.75}{-1}{l}{subject of}\errelarm{5.7}{-1.15}{4.15}{-1.15}{1}{0}\errelarm{4.15}{-1.15}{2.6}{-1.15}{0}{0}\ercrowfoot{5.55}{-1.15}{5.7}{-1}{5.7}{-1.15}{5.7}{-1.3}{0}
% relationship to
\errelname{8.85}{-1}{l}{to}\errelname{11.65}{-1}{r}{resourced by}\errelarm{8.7}{-1.15}{10.25}{-1.15}{1}{0}\errelarm{10.25}{-1.15}{11.79}{-1.15}{0}{0}\ercrowfoot{8.85}{-1.15}{8.7}{-1}{8.7}{-1.15}{8.7}{-1.3}{0}
\end{erdiagram}

\caption{The example from Chen 1984 recast in the Barker-Ellis style. In this style the value types of attributes are not shown on the diagram but are expected to be 
separately documented. }
\label{projectWorkerAttributesOurStyle}
\end{figure}

\mynote
In standard entity modelling terminology an attribute relates entity to a value but whilst they relate things they are not considered to be relationships. If you ask why not then the answer is that it is just so.
 
\mynote Note that we do not set out to model types of universals and the relationships between them in an entity model. 

\mynote Since relationships of this second kind are called attributes, the term relationship is reserved, in entity modelling terminology and subsequently in these pages, for relationships of the first kind.  The principle components of an entity model are its entity types and their attributes and relationships.


 

