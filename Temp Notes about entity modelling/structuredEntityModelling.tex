\section{Structured Entity Modelling}
Chen's paper introduced the idea of entities being dependent on binary relationships with others for both their identification and their existence:
\begin{quote}
Theoretically, any kind of relationship may be used to identify entities. For simplicity, we shall restrict ourselves to the use of only one kind of relationship: the binary relationships with 1:n mapping in which the existence of the n entities on one side of the relationship depends on the existence of one entity on the other side of the relationship. For example, one employee may have n ( = 0, 1, 2, . . .) dependants, and the existence of the dependants depends on the existence of the corresponding employee. This method of identification of entities by relationships with other entities can be applied recursively until the entities which can be identified by their own attribute values are reached. For example, the primary key of a department in a company may consist of the department number and the primary key of the division, which in turn consists of the division number and the name of the company.
\end{quote}

Following PCTE8 we use the term composition relationship for Chen's binary relationships with 1:n mapping in which the existence of the n entities on one side ... depends on the existence of one entity on the other side and we use the term reference relationship for binary relationships which are neither composition relationships nor their inverses. We shall also describe the inverses of composition relationships as being dependency relationships. Earlier than this a similar distinction had been made by the designers of the CAIS9 specification but in which the two kinds of relationship were distinguished as primary and secondary — their rationale for the distinction was as follows10:

[Entities] and relationships may form a general graph or bowl of spaghetti. However, this raises various practical problems of deletion and garbage collection, long term naming, and unconnected sub-graphs. CAIS therefore designates certain relationships as primary (and all others as secondary) and requires that all [Entities] and primary relationships in the database form a single tree structure.
This distinction between composition and reference made by both CAIS and then PCTE served the goal of modelling computer file systems within a database framework, see figure 9 for example.

Figure 9An ER model of folder system modelling the hierarchical structure as a recursive composition relationship and shortcuts as reference relationships.
In this presentation we shall not assume that all composition relationships are identifying nor, vice-versa, that only composition relationships may be identifying. To depict ER-schemas we use a variant of the Barker-Ellis notation. Figure 10 is a meta-model of this notation — it is an ER schema describing ER schemas.

In cases where we wish to distinguish composition relationships from reference relationships then we draw the diagram top down: an anonymous root entity type (the ‘absolute’) is introduced at the top of the diagram, relationships leaving the lower edges of boxes are composition relationships and they always meet the top edge of the box representing the dependent type, reference relationships meet boxes from one side or the other. We note that there is a structural resemblance to diagrams drawn by Bachman. To summarise, for composition relationships the crows feet point down; at this point the notation converges with that of SSADM for which one explanation says: ‘there are no dead crows’. Our diagrams also have reference relationships and for these the crows feet are pointing sideways (the crows, presumably, at rest). The entity types which have the least numbers of instances occur at the top of our diagrams whereas in what seems an odd choice they occur to the bottom right in the diagrams style promoted in Barker's Entity Modelling book.


Figure 10The logical ER meta-model. A simple version of the logical ER model of a logical ER model.

\subsection*{The Absolute}

Now  I want to discuss that the representation within an entity model of the whole of everything should in fact be referred to as `the absolute'.

We use the term ‘the Absolute’, which has a varied history in metaphysical writing, to mean the whole of everything and, equally, the context for everything. In giving two meanings we are following a rich tradition — for definitions abound. The term was central to much of the philosophy of G.W.F Hegel — in Shorter Logic section 87, among a number of other definitions, we find:

...the Absolute is the Nought...The Nothing which the Buddhists make the universal principle, as well as the final aim and goal of everything, is the same abstraction.
whereas in Phenomonology of Spirit in section 20 we find:

The True is the whole.
and in section 75 of the same we find:

...the Absolute alone is true, truth alone is absolute.
Some sense can be made of these statements when we consider from a logical perpective and from the point of view of information theory. We can illustrate by considering the seemingly puzzling fact that in some programming languages (such as ML) there is an in-built concept with the name ‘Unit’ and described as a singleton type, whereas in some other earlier programming languages (Algol68, C) the name given to the very same concept has been ‘void’. The apparently striking disparity in naming, the one versus the zero, has come about as a result of the concept being named on the one hand on the basis of the number of things predicated by the type (i.e the number of things we can say are of that type), which is exactly one, and on the other hand based on the number of bits (binary digits) of information carried in communicating a member of the type, which is precisely zero. We get the name ‘unit’ from one point of view and the name ‘void’ from the other. In this way we can say of the Absolute that it is the whole of everything. From it being the whole we can say there is only one of the type. From there being only one of the type we can say that it's information content is zero. If you present to me the absolute you present me with nothing. Like the true it can be assumed in all contexts for it carries nothing new with it. This is the logic of the absolute.

The whole of a modelling situation can be considered a single composite and this is both ‘the ultimate whole’ when considered as a composite and equally ‘the absolute’ when considered as a context. If what we have said above can be summarised as saying that there is a duality between context and composition then in this duality ‘the whole’ and ‘the absolute’ are duals: they are the same logical entities.

Another useage that we have, is to speak of concepts that are absolute. What we usually mean by saying of a concept that it is abolute is that it does not vary — that it is not relative to the context in which it appears. As we seek to construct models of usage and thereby a conceptual model we can expect to find a dichotomy of relative and absolute terms: some terms, such as father, daughter, length, colour, that vary in so much as they reference different items in different contexts; and terms, such as the earth, the pole star, the London Times that are absolute or constant in what they reference. Whereas relative terms are conceptualised as relationships or as quantitative or adjectival attributes of subject entities, absolute terms cannot be so interpreted unless we posit the existence of a singular entity and then interpret the absolute terms as relational to or as attributive to this singular entity. In this way the matter is finessed for we can say of a relative term such as father that it varies as the person varies — different subject persons having different fathers — and we can say of an absolute term pole star that it varies depending on the singular entity — the Absolute — which is to say that it does not vary at all

Perhaps now

The Distinction between Composition and Reference
In structured entity modelling the distinctin is made between composition relationships and reference relationships and this does such and such. Is this a real distinction? ETC ETC as per.

The Scope concept

Diagrams expressing Scopes.

