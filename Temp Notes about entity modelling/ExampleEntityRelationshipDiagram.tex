

\section{An Example Entity Relationship Diagram}
In figure \ref{ChenManufacturingExample} we reproduce an example entity relationship diagram  given in Chen's 1976 paper. This example documents principle types of entity in the organisation of an imaginary manufacturing company. The intended meaning of some of the relationships shown in Chen's diagram is not all clear from the diagram nor from the surrounding text. I wanted to redraw
this diagram in the Barker-Ellis style because that is the style used in this book, but to do this I needed to make some guesses about the meaning of some of Chen's relationships because the Barker-Ellis style has required me to descriptively label the relationships. Having made these guesses I drew the diagram shown
 in figure \ref{ChenManufacturingBarkerEllisStyle}. 

\begin{figure} [H]
\begin{center}
\begin{pspicture}(-4.4,-1.2)(10,5.5)
%\psgrid

\rput[l](-3.0,5){
   \chenbox{dept}{DEPARTMENT}
}

\rput(-3,2.5){
	\chendiamond{de}{\rput(0,-0.05){\begin{tabular}{c}DEPT-\\EMP\end{tabular}}}
}

\rput[l](-3.0,0){
   \chenbox{emp}{EMPLOYEE}
	}
	
\rput(-3,-2.5){
	\chendiamond{ed}{\rput(0,-0.05){\begin{tabular}{c}EMP-\\DEP\end{tabular}}}
}

\rput[l](-3.0,-5){
   \chenbox{dpndt}{DEPENDENT}
	}

\rput(0,0){
	\chendiamond{pw}{\rput(0,-0.05){\begin{tabular}{c}PROJECT-\\WORKER\end{tabular}}}
}
\rput[l](3.0,0){
   \chenbox{prj}{PROJECT}
	}

\ncline{-}{empE}{pwW}
\nbput{\scriptsize M}	
\ncline{-}{pwE}{prjW}
\nbput{\scriptsize N}

\rput(6,0){
	\chendiamond{pp}{\rput(0,-0.05){\begin{tabular}{c}PROJECT-\\PART\end{tabular}}}
}

\rput[l](9,0){
   \chenbox{prt}{PART}
	}
\ncline{-}{prjE}{ppW}
\nbput{\scriptsize M}	
\ncline{-}{ppE}{prtW}
\nbput{\scriptsize N}

\rput(9,-2.5){
	\chendiamond{c}{\rput(0,-0.05){\begin{tabular}{c}COMPONENT\end{tabular}}}
}

\rput[l](6.0,5){
   \chenbox{supp}{SUPPLIER}
	}
	
\rput(6,2.5){
	\chendiamond{spp}{\rput(0,-0.05){\begin{tabular}{c}SUPP-PROJ-\\PART\end{tabular}}}
}

\ncline{-}{deptS}{deN}
\nbput{\scriptsize 1}	
\ncline{-}{deS}{empN}
\nbput{\scriptsize N}

\ncline{-}{empS}{edN}
\nbput{\scriptsize 1}	
\ncline{-}{edS}{dpndtN}
\nbput{\scriptsize ??}



\end{pspicture}
\end{center}
\caption{
In effect this is the first ever entity relationship diagram. It is an entity relationship diagram in the Chen style from his seminal paper of 1976 and it is there described as being an analysis of information in a manufacturing firm. In this diagram
diamonds represent what we describe here as relationships and boxes represent types of entities.  
For completeness we should mention that in Chen's 1976 terminology the boxes were said to be entity sets and the diamonds were said to be relationship sets but by 1983 in a paper published in that year Chen is instead using the terms `entity type' and `relationship type'. While sympathising with this terminology we think it correct to to use the term `relationship' rather than 'relationship type' because this brings us closer to the terminology of formal logic.
}
\label{ChenManufacturingExample}
\end{figure}

\begin{figure} [H]
\begin{center}
\begin{erdiagram}{6.3}{12.383}

\eret{0.8}{-1.6}{3.1}{-1}{0.2}{1}\eretname{1.95}{-1.35}{}{department}
\eret{0.8}{-3.5}{3.1}{-2.7}{0.2}{1}\eretname{1.95}{-3.05}{}{employee}
\eret{0.8}{-5.45}{3.1}{-4.85}{0.2}{1}\eretname{1.95}{-5.2}{}{dependent}
\eret{5.1}{-1.6}{8.1}{-1}{0.2}{1}\eretname{6.6}{-1.35}{}{project}
\eret{3.85}{-6.3}{6.502}{-5.1}{0.2}{1}\eretname{5.176}{-5.45}{}{project}\eretname{5.176}{-5.75}{}{worker}\eretname{5.176}{-6.05}{}{assignment}
\eret{7.374}{-3.75}{9.226}{-2.85}{0.2}{1}\eretname{8.3}{-3.2}{}{part used}\eretname{8.3}{-3.5}{}{on project}
\eret{7.413}{-6.25}{11.187}{-5.65}{0.2}{1}\eretname{9.3}{-6}{}{part supply option}
\eret{8.6}{-1.6}{9.811}{-1}{0.2}{1}\eretname{9.205}{-1.35}{}{part}
\eret{10.311}{-1.6}{12.133}{-1}{0.2}{1}\eretname{11.222}{-1.35}{}{supplier}
\eret{0}{-0.25}{12.383}{0.25}{0.2}{1}\eretname{4.279}{-0.2}{l}{Chen '76 Manufacturing Company}

% relationship 
\errelname{2.1}{-0.55}{l}{}\errelname{2.1}{-0.85}{l}{..}\errelarm{1.95}{-0.25}{1.95}{-0.625}{0}{0}\errelarm{1.95}{-0.625}{1.95}{-1}{1}{0}\ercrowfoot{1.95}{-0.85}{1.8}{-1}{1.95}{-1}{2.1}{-1}{0}
% relationship 
\errelname{6.75}{-0.55}{l}{}\errelname{6.75}{-0.85}{l}{..}\errelarm{6.6}{-0.25}{6.6}{-0.625}{0}{0}\errelarm{6.6}{-0.625}{6.6}{-1}{1}{0}\ercrowfoot{6.6}{-0.85}{6.45}{-1}{6.6}{-1}{6.75}{-1}{0}
% relationship 
\errelname{9.356}{-0.55}{l}{}\errelname{9.356}{-0.85}{l}{..}\errelarm{9.205}{-0.25}{9.205}{-0.625}{0}{0}\errelarm{9.205}{-0.625}{9.205}{-1}{1}{0}\ercrowfoot{9.205}{-0.85}{9.055}{-1}{9.205}{-1}{9.356}{-1}{0}
% relationship 
\errelname{11.372}{-0.55}{l}{}\errelname{11.372}{-0.85}{l}{..}\errelarm{11.22}{-0.25}{11.22}{-0.625}{0}{0}\errelarm{11.22}{-0.625}{11.22}{-1}{1}{0}\ercrowfoot{11.222}{-0.85}{11.072}{-1}{11.222}{-1}{11.372}{-1}{0}
% relationship employing
\errelname{1.8}{-1.9}{r}{employing}\errelname{1.8}{-2.55}{r}{employed by}\errelarm{1.95}{-1.6}{1.95}{-2.15}{0}{0}\errelarm{1.95}{-2.15}{1.95}{-2.7}{1}{0}\ercrowfoot{1.95}{-2.55}{1.8}{-2.7}{1.95}{-2.7}{2.1}{-2.7}{0}
% relationship depended on by
\errelname{1.8}{-3.8}{r}{depended on by}\errelname{1.8}{-4.7}{r}{depending on}\errelarm{1.95}{-3.5}{1.95}{-4.175}{0}{0}\errelarm{1.95}{-4.175}{1.95}{-4.85}{1}{0}\ercrowfoot{1.95}{-4.7}{1.8}{-4.85}{1.95}{-4.85}{2.1}{-4.85}{0}
% relationship subject_of
\errelname{2.675}{-3.8}{l}{subject}\errelname{2.675}{-4.1}{l}{of}\errelname{4.575}{-4.95}{r}{of}\errelname{4.575}{-4.65}{r}{assignment}\errelarm{2.525}{-3.5}{2.525}{-3.825}{0}{0}\errelarm{2.525}{-3.825}{2.525}{-4.15}{0}{0}\errelarm{2.525}{-4.15}{3.625}{-4.312}{0}{0}\errelarm{3.625}{-4.312}{4.725}{-4.475}{1}{0}\errelarm{4.725}{-4.475}{4.725}{-4.787}{1}{0}\errelarm{4.725}{-4.787}{4.725}{-5.1}{1}{0}\eridcomprel{4.625325}{4.825324999999999}{-4.85}\ercrowfoot{4.725}{-4.95}{4.575}{-5.1}{4.725}{-5.1}{4.875}{-5.1}{0}
% relationship resourced_by
\errelname{5.751}{-1.9}{l}{resourced}\errelname{5.751}{-2.2}{l}{by}\errelname{5.751}{-4.95}{l}{to}\errelname{5.751}{-4.65}{l}{assignment}\errelarm{5.6}{-1.6}{5.6}{-3.349}{0}{0}\errelarm{5.6}{-3.349}{5.6}{-5.1}{1}{0}\eridcomprel{5.50065}{5.7006499999999996}{-4.85}\ercrowfoot{5.601}{-4.95}{5.451}{-5.1}{5.601}{-5.1}{5.751}{-5.1}{0}
% relationship requires
\errelname{7.5}{-1.9}{l}{requires}\errelname{7.687}{-2.7}{r}{use by}\errelarm{7.35}{-1.6}{7.35}{-1.8}{0}{0}\errelarm{7.35}{-1.8}{7.35}{-2}{0}{0}\errelarm{7.35}{-2}{7.593}{-2.212}{0}{0}\errelarm{7.593}{-2.212}{7.836}{-2.425}{1}{0}\errelarm{7.836}{-2.425}{7.836}{-2.637}{1}{0}\errelarm{7.836}{-2.637}{7.836}{-2.85}{1}{0}\eridcomprel{7.7368749999999995}{7.936874999999999}{-2.6}\ercrowfoot{7.837}{-2.7}{7.687}{-2.85}{7.837}{-2.85}{7.987}{-2.85}{0}
% relationship managed by
\errelname{4.95}{-1.15}{r}{managed by}\errelname{3.25}{-3.264}{l}{managing}\errelarm{5.1}{-1.3}{4.5}{-1.3}{1}{0}\errelarm{4.5}{-1.3}{3.899}{-1.3}{1}{0}\errelarm{3.899}{-1.3}{3.774}{-2.132}{1}{0}\errelarm{3.774}{-2.132}{3.649}{-2.964}{0}{0}\errelarm{3.649}{-2.964}{3.374}{-2.964}{0}{0}\errelarm{3.374}{-2.964}{3.099}{-2.964}{0}{0}\ercrowfoot{4.95}{-1.3}{5.1}{-1.15}{5.1}{-1.3}{5.1}{-1.45}{0}
% relationship able to be_sourced via
\errelname{8.45}{-4.05}{l}{able to be}\errelname{8.45}{-4.35}{l}{sourced via}\errelname{8.15}{-5.5}{r}{supply of}\errelname{8.15}{-5.2}{r}{option for}\errelarm{8.299}{-3.75}{8.299}{-4.699}{0}{0}\errelarm{8.299}{-4.699}{8.299}{-5.649}{1}{0}\eridcomprel{8.2}{8.399999999999999}{-5.3999999999999995}\ercrowfoot{8.3}{-5.5}{8.15}{-5.65}{8.3}{-5.65}{8.45}{-5.65}{0}
% relationship subject_of
\errelname{9.356}{-1.9}{l}{subject}\errelname{9.356}{-2.2}{l}{of}\errelname{8.913}{-2.7}{l}{use of}\errelarm{9.205}{-1.6}{9.205}{-1.8}{0}{0}\errelarm{9.205}{-1.8}{9.205}{-2}{0}{0}\errelarm{9.205}{-2}{8.984}{-2.212}{0}{0}\errelarm{8.984}{-2.212}{8.763}{-2.425}{1}{0}\errelarm{8.763}{-2.425}{8.763}{-2.637}{1}{0}\errelarm{8.763}{-2.637}{8.763}{-2.85}{1}{0}\eridcomprel{8.663124999999999}{8.863124999999998}{-2.6}\ercrowfoot{8.763}{-2.7}{8.613}{-2.85}{8.763}{-2.85}{8.913}{-2.85}{0}
% relationship able to_provide
\errelname{11.372}{-1.9}{l}{able to}\errelname{11.372}{-2.2}{l}{provide}\errelname{10.394}{-5.5}{l}{aquire from}\errelname{10.394}{-5.2}{l}{option to}\errelarm{11.22}{-1.6}{11.22}{-1.95}{0}{0}\errelarm{11.22}{-1.95}{11.22}{-2.3}{0}{0}\errelarm{11.22}{-2.3}{10.73}{-3.624}{0}{0}\errelarm{10.73}{-3.624}{10.24}{-4.949}{1}{0}\errelarm{10.24}{-4.949}{10.24}{-5.3}{1}{0}\errelarm{10.24}{-5.3}{10.24}{-5.649}{1}{0}\eridcomprel{10.143625}{10.343625}{-5.3999999999999995}\ercrowfoot{10.244}{-5.5}{10.094}{-5.65}{10.244}{-5.65}{10.394}{-5.65}{0}
\end{erdiagram}

\end{center}
\caption{
Chen's 1976 entity relationship diagram (shown in figure \ref{ChenManufacturingExample}) redrawn as a structured entity model using the Barker-Ellis notation. 
I have to guess some of the meaning of Chen's relationships in order to label them meaning fully in this diagram. The rail at the top of the diagram represents the whole of the model which in this case is the manufacturing company in question.
}
\label{ChenManufacturingBarkerEllisStyle}
\end{figure}

\subsection{Background Wikipedia entry for `Part number'}
Has this para headed ``the part design versus instantiations of it''
As a part number is an identifier of a part design (independent of its instantiations), a serial number is a unique identifier of a particular instantiation of that part design. In other words, a part number identifies any particular (physical) part as being made to that one unique design; a serial number, when used, identifies a particular (physical) part (one physical instance), as differentiated from the next unit that was stamped, machined, or extruded right after it. This distinction is not always clear, as natural language blurs it by typically referring to both part designs, and particular instantiations of those designs, by the same word, "part(s)". Thus if you buy a muffler of P/N 12345 today, and another muffler of P/N 12345 next Tuesday, you have bought "two copies of the same part", or "two parts", depending on the sense implied.

