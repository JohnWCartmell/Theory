\section{Type Inheritance}
\label{TypeInheritance}




\subsection*{To do}
\mynote 
Can I preface this with something more practical to get a better blend? 

Can I talk about programmking languages? 

terminology subtype and supertype Barker?

non-standard part of the SSADM method.

Use in car example from SSADM

Inheritance is used in the example in the file systemn structured entity modelling example which is germane to history.

\subsection*{Types of Entity Specific and General}
Entity modelling has the questions ‘What is?’ and ‘What can be said of it?’ at its heart. We can define it to be the process of defining what can be predicated of entities and it necessarily seems to embrace, in passing, many questions of philosophy, specifically of Ontology, the branch of metaphysics dealing with the nature of being.

In ancient times, Aristotle had used the Greek word ousia (being) to describe the subject, to which predicates are ascribed.1 Traditionally in translation the Greek ousia has been rendered as substance, a term with broad connotations whereas the Latin word ens which is, as is also the Greek ousia, the present participle of the verb to be, yielded our modern English word entity defined as:
\begin{erquote}
entity n. things existence, as opposed to its qualities or relations; thing that has real existence. (Concise Oxford Dictionary)
\end{erquote}
or as:
\begin{erquote}
entity n. (the quality of having) a single separate and independent existence. (Longman Dictionary of Contemporary English).
\end{erquote}
The term entity, as we use it here in the term ‘entity modelling’, was introduced into information science by Chen in 1966. An entity for our purposes is simply something about which it is possible to have knowledge and which can be counted. Thus we can have fictional entities such as characters from a book or entities whose state of existence we can debate such as the number zero or the transcendental number $\pi$.

In Aristotelian ontology, as outlined in Categories, there are ten genera of being. The first genus of being, ousia, is of two types: primary and secondary. Primary ousia are individual things — they are our entities, secondary ousia are classes of things, or the genera and species of things
 — for us these are entity types.\footnote{The other genera of being
  (quantity, quality, relation, place, time, position, state, action, affection)
  are properties inherent in the primary ousia.}

Though Metaphysics has ancient origins, there is an intersection of concerns via data modelling between it and the most practical and modern disciplines of software engineering and the programming of computer based systems. The overlap exists because metaphysics has as its subject matter what is most general about things in general — not just physical properties of physical things — and the software development discipline starts with the representation of the very same generalities. So whereas most computer programs have as their subjects, everyday if not concrete and physical things, things such as people, accounts, orders, contracts, airline bookings, and so on; other computer programs have as their subjects the structures of molecules, languages, stellar processes or programs themselves, or such as mathematical propositions, relationships in general, not particular, types of things as distinct from the things themselves, and so on. 

\subsection*{Generic types Aristotle's Categories}
In the statement ‘man and the apes are descendent from a common ancestor’— it is clear that singular ‘man’ is being used to denote a specific type which in biology, and as far back as Aristotle's Categories, is termed a species, and that plural ‘apes’ is denoting a more general class of thing which following Aristotle we might call a genus or in biology are said to be ‘a classification of a higher rank’ and which classifications includes genera, families, orders and kingdoms. As a reader one does not need know the individual species of apes but can infer that a multiplicity of species is implied — simply by noting the use of the plural form. In entity modelling both such specific and general types of thing are represented; these are species and genera not just in the strict sense used in biological nomenclature but in the more general sense that we find in translations of Aristotle's Categories, a species is a specific type such as the type ‘man’ and a particular individual man is said to be predicated by that species. A genus, on the other hand, is a more general type such as ‘animal’ or ‘ape’.
\begin{erquote}
For if any one should render an account of what a primary ousia is, he would render a more instructive account by stating the species than by stating the genus. Thus, he would give a more instructive account of an individual man by stating that he was a man than by stating that he was an animal, for the former description is peculiar to the individual in a greater degree, while the latter is too general.
\end{erquote}
In Aristotle's description both the individual man and the species man are predicated by the genus animal, and so to, by example, the species ox is predicated by the genus animal. From what we have already said this we will represent thus:
\begin{center}
\begin{erdiagram}{1.45}{3.6666}

\eret{0}{-1.45}{3.667}{-0}{0.2}{1}\eretname{0.116}{-0.35}{l}{Animal}
\eret{0.25}{-1.2}{1.583}{-0.6}{0.2}{0}\eretname{0.917}{-0.95}{}{man}
\eret{2.083}{-1.2}{3.417}{-0.6}{0.2}{0}\eretname{2.75}{-0.95}{}{ox}

\end{erdiagram}

\end{center}



In entity modelling the term entity type is used for both genera and species —ox, man and animal are all entity types as shown in the fragment above.

In some ways, such diagrams as these show similarity to Venn diagrams. This one can be so interpreted as showing the set of men and the set of oxen included in the set of animals. However, to the entity modeller, there is no set of all men, nor of all oxen, nor of all animals for the question is not ‘what exists?’ but ‘what types of things exist and what can be said of them?’. The diagram can be interpreted as saying ‘what can be said of animal’ can be said of ‘man’ and of ‘ox’ also.

Aristotle says it like this:
\begin{erquote}
Whenever one thing is predicated of another as a subject, all things said of what is predicated will be said of the subject also. For example, man is predicated of the individual man, and animal of man; so animal will be predicated of the individual man also — for the individual man is both a man and an animal.
\end{erquote}
Subsequently, after the time of Aristotle, used in biological nomenclature the term genus adopted a more specific meaning, in contradistinction to use of the term species for the lowest rank in the system — individuals of the same species varying in minor ways and able to interbreed — the term genus became used for a group of related species, the genera strictly forming just the second rank in a multi-ranked system.

\subsection*{Types of Particles in Physics}
Another example is given by the types of particle discussed in Feynman's Lecture Notes on Physics:
\begin{erquote}
Particles which interfere with a positive sign are called Bose particles and those which interfere with a negative sign are called Fermi particles. The Bose particles are the photon, the mesons, and the graviton. The Fermi particles are the electron, the muon, the neutrinos, the nucleons, and the baryons.
\end{erquote}
As in the usage ‘man and the apes’ singular and plural terms are used in this passage to distinguish between specific types of things and more general classes of things. The author's respective use of singular and plural terms inform as to which are the fundamental types of particle, the species, and which the related families, the genera. In diagramming this in an entity model, instead of retaining the plural form, so to speak, we may capitalise the genera and so arrive this diagram:
\begin{center}
\begin{erdiagram}{4.699999999999999}{6.199999999999999}

\eret{0}{-4.7}{6.2}{-0}{0.2}{1}\eretname{0.376}{-0.35}{l}{Particle}
\eret{0.25}{-2.05}{5.95}{-0.6}{0.2}{0}\eretname{0.366}{-0.95}{l}{Bose Particle}
\eret{0.5}{-1.8}{1.9}{-1.2}{0.2}{1}\eretname{1.2}{-1.55}{}{photon}
\eret{2.4}{-1.8}{3.8}{-1.2}{0.2}{1}\eretname{3.1}{-1.55}{}{Meson}
\eret{4.3}{-1.8}{5.7}{-1.2}{0.2}{1}\eretname{5}{-1.55}{}{graviton}
\eret{0.25}{-4.45}{5.95}{-2.35}{0.2}{0}\eretname{0.418}{-2.7}{l}{Fermi Particle}
\eret{2.4}{-3.3}{3.8}{-2.7}{0.2}{1}\eretname{3.1}{-3.05}{}{electron}
\eret{4.3}{-3.3}{5.7}{-2.7}{0.2}{1}\eretname{5}{-3.05}{}{muon}
\eret{0.5}{-4.2}{1.9}{-3.6}{0.2}{1}\eretname{1.2}{-3.95}{}{Nucleon}
\eret{2.4}{-4.2}{3.8}{-3.6}{0.2}{1}\eretname{3.1}{-3.95}{}{Neutrino}
\eret{4.3}{-4.2}{5.7}{-3.6}{0.2}{1}\eretname{5}{-3.95}{}{Baryon}

\end{erdiagram}

\end{center}

\subsection*{Types of Words in Linguistics}
In linguistics the entity type ‘word’ is generally represented as a generalisation of more specific types often referred to as word classes. These include noun, verb, adjective and so on as shown in figure \ref{wordclassesnested} and these are illustrated in table \ref{wordclassestable}.

The basis for the recognition of word classes in linguistics is the observation that certain words can be freely interchanged in sentences without altering the acceptability of the sentence grammatically. For example we can apply substitutions replacing various words of an example sentence such as I received beautiful flowers for my birthday by randomly chosen other words and some will deliver equally grammatical sentences and some will not. For example we can replace ‘beautiful’ by ‘ugly’ or ‘red’ or ‘expensive’ without losing sentence structure whereas we cannot replace by ‘the’ or ‘very’ or ‘at’. So the words ‘beautiful’,‘ugly’, ‘red’, ‘expensive’ are of the same class adjective and the words ‘the’, ‘very’, ‘at’ are not of this class — in fact the word ‘the’ is classed as a determiner, the word ‘very’ as a degree word and the word ‘at’ as a preposition.
\begin{table}
\begin{tabular}{ l l l}
class&	abbreviation &	examples \\
noun&	N	&athelete, house, race, record, stream, water  \\
pronoun&	&Pro	I, you, he, she, we, they  \\
determiner&	&Det	a, the  \\
verb&	    & V	arrive, run, set  \\
auxiliary&	& Aux	had, will  \\
preposition&	&Prep	at, by, from, in, to  \\
prepositional specifier &	Pspec	& close, right, straight, three seconds  \\
adjective	   & A	    &fierce, long, new, red, right, rosy, silk, young  \\
general adverb &	Adv	&abruptly, brightly, clearly, quickly  \\
degree adverb&	Deg	    &more, most, quite, rather, so, too, very  \\
\end{tabular}
\caption{Types of word and abbreviations used.}
\label{wordclassestable}
\end{table}


It is common to use abbreviations to identify the word classes; as to how many there are then it has to be said that they cannot be enumerated unequivocally; linguist C.C Fries defined nineteen types as the nineteen parts of speech of English in 1952 (he also distinguished content bearing types of word: nouns, verbs, adjectives and adverbs from function types such as prepositions, determiners and coordinating conjunctions). For our purposes here we will use the classes and the abbreviations shown in table \ref{wordclassestable}.

\begin{figure}
\begin{center}
\begin{erdiagram}{2.55}{7.5999}

\eret{0}{-2.55}{7.6}{-0}{0.2}{1}\eretname{0.204}{-0.35}{l}{Word}
\eret{0.25}{-1.2}{1.583}{-0.6}{0.2}{0}\eretname{0.917}{-0.95}{}{noun}
\eret{2.083}{-1.2}{3.417}{-0.6}{0.2}{0}\eretname{2.75}{-0.95}{}{verb}
\eret{3.917}{-1.2}{5.517}{-0.6}{0.2}{0}\eretname{4.717}{-0.95}{}{adjective}
\eret{6.017}{-1.2}{7.35}{-0.6}{0.2}{0}\eretname{6.683}{-0.95}{}{adverb}
\eret{0.25}{-2.3}{1.583}{-1.5}{0.2}{0}\eretname{0.917}{-1.85}{}{proper}\eretname{0.917}{-2.15}{}{name}
\eret{2.083}{-2.1}{3.417}{-1.5}{0.2}{0}\eretname{2.75}{-1.85}{}{pronoun}
\eret{3.917}{-2.1}{5.517}{-1.5}{0.2}{0}\eretname{4.717}{-1.85}{}{determiner}
\eret{6.017}{-2.1}{7.35}{-1.5}{0.2}{0}\eretname{6.683}{-1.85}{}{degree}

\end{erdiagram}

\end{center}
\caption{Word Classes shown using the nested box notation.}
\label{wordclassesnested}
 \end{figure}
On Naming of Types
In biological nomenclature the species name alone is not always enough to uniquely identify a type of thing less it be used alongside of the genus name — this system of naming, using the genus name alongside the species name, being called binomial and having been introduced by Linnaeus. Resonant and antecedent to this can be seen in the very first section of Aristotle's Categories which is the about the proper delineation of the of types of things:

Things are said to be named ‘equivocally’ when, though they have a common name, the definition corresponding with the name differs for each. Thus, a real man and a figure in a picture can both lay claim to the name ‘animal’; yet these are equivocally so named, for, though they have a common name, the definition corresponding with the name differs for each. For should any one define in what sense each is an animal, his definition in the one case will be appropriate to that case only.
The appropriate diagram to fit Aristotle's text would seem to be this:
\begin{center}
\begin{erdiagram}{2.3}{5.1666}

\eret{0}{-2.3}{2.333}{-0}{0.2}{1}\eretname{0.184}{-0.35}{l}{Figure}
\eret{0.25}{-2.05}{2.083}{-0.6}{0.2}{0}\eretname{0.366}{-0.95}{l}{Animal}
\eret{0.5}{-1.8}{1.833}{-1.2}{0.2}{1}\eretname{1.167}{-1.55}{}{man}
\eret{2.833}{-2.3}{5.167}{-0}{0.2}{1}\eretname{3.017}{-0.35}{l}{Real Thing}
\eret{3.083}{-2.05}{4.917}{-0.6}{0.2}{0}\eretname{3.199}{-0.95}{l}{Animal }
\eret{3.333}{-1.8}{4.667}{-1.2}{0.2}{1}\eretname{4}{-1.55}{}{man }

\end{erdiagram}

\end{center}

With such a configuration of types you might suppose a trinomial notation is required — or... simply a diagram and the ability to point at it. And this is the point of such diagrams or at least a very good part of it.