\section{Data Modelling}\commentary{Speak also of Data Specification}
\label{DataModelling}
\mynote This term Data Modelling we use to cover both database design and also the specification of the structure of messages in quite a broad sense. Entity modelling is most commonly positioned as a precursor to relational data modelling but in fact is equally suitable as method of specifying hierarchical data structures — and it is significant that supporting relational, hierarchical and network models of data from a single specification was the big idea behind the notation as originally proposed by Chen in the seminal paper ‘Unified Model of Data’ in 1976. This is also a point made and illustrated in the Barker Entity Modelling book.

\mynote ER modelling can be used to specify data at three distinct, increasingly prescriptive levels:

\begin{tabular}{l p{8.5cm}}
conceptual &— entities and relationships only,\\
logical &— entities, relationships and non-referential attributes,\\
physical &— as logical but with message structure completed by the addition of attributes said to be referential and whose purpose is the representation of relationships.\\
\end{tabular}\commentary{consider whether I still want to use these terms}\commentary{I will need explain referential attributes in the earlier section. Good challenge.}

At the third and most prescriptive level, that of a physical model, an entity model is precisely an abstract data model and such models can be classified as relational or hierarchical; most significantly, and in accord with the proposal of Chen, one of each can be generated algorithmically from a well-formulated logical model. Looking back, it seemed that hierarchical structured data took a back seat for a while during the theoretical development and popularisation of the relational model of data but it has made a come back subsequently through widespread adoption of the structured markup language XML. Via appropriate physical models both relational Data Definition Language (DDL) and hierarchical XML schemas (DTD, XSD or the like) can be generated automatically from a single logical ER model.\footnote{We will return to this theme later but it has been said that relational data models generated in this way will naturally tend to be well-formulated data models (i.e. to be in normal form). This is definitely not the case unless account is taken of reference scope constraints as described here in later sections. A logical ER model is agnostic between hierarchical and relational.}

\mynote Historically, E.F.Codd's meta theory that was presented as the relational model of data by Codd in 1970,4 emerged fully formed — the meta concepts of table, column and primary key are defined as is that of a foreign key enabling one table to cross reference the rows of another. His is a theory of what data is and this theory came to underpin the majority of corporate databases. Each such database, in accord with Codd's prescriptions, holds a meta-description of its own storage units — the tables, columns and keys — what their names are and how they fit together to enable navigation through the data; this description is the core of what is described as a relational schema. The development of the relational model of data in the first place was strongly influenced by the predicate calculus representation of formal logic but arguably this meta-mathematics that influenced Codd was overtaken in mathematical imagination by later 20th century meta-mathematics in the form of type theory and category theory; these are more diagrammatic in form and lead not to the relational model of data but to versions of the binary entity relationship model such as is promoted in this book. It is these other meta-mathematical disciplines that influence this presentation here and these lead to significant improvements in relational design methodology. Paradoxically, each such improvement in relational design methodology undermines the pre-eminence enjoyed by the relational model.

\mynote Codd has described various tests of goodness of a schema, applicable, it must be remembered, only with cognisance to the possibilities among the data that it is designed to hold i.e. the intended usage. In the first instance three tests were described and successively a schema said to be in 1st normal form, 2nd normal form or 3rd normal form depending on its success in passing the tests. A process for fixing deficient schemas is described as normalisation of the schema. Normalisation is therefore a method for converting or transforming one relational schema into another that is deemed more suitable for a purpose at hand.

\mynote Subsequently, the relations of Codd's model are more abstractly presented, as either entities or as n-ary relationships, in Chen's entity-relationship model of data; in the approach of Chen there is emphasis on a diagrammatic representation of the model. Chen describes a method for constructing a relational schema (in the sense of Codd) from an entity-relationship schema (ER-schema). He states that normalisation of the relational schema might be required after construction from an ER-schema — though why this might be is not explained. We will explain in a later section the fundamental reason why this is so but also why it need not be so.

After Chen's 1976 paper, coming into and through the 1980's, came the development, concurrently, of Computer Aided Software Engineering (CASE) tools, including Meta-CASE tools, and semi-formalised and, in some instances, standardised official methodologies and notations supporting structured systems analysis and development. Universally in the methodologies from this time the terms entity and relationship introduced in Chen's paper were retained within a logical modelling phase and Chen's transformation step into relational database design, inclusive of a normalisation step, is likewise retained. Though the terms and the overall shape of the process is retained the concepts behind these terms are subtly shifted. Most noticeably relationships are now binary relationships and at an early stage in these methodologies many-many relationships are eliminated in favour of many-one relationships. At this point there has been a conceptual volte face for a many-one binary relationship, implementation considerations aside, is a thinly disguised and abstracted pointer between records of a file, such as in a VSAM file system, or a link between records in the network database model, and it can be conceptualised, abstractly, as a function between sets of like-typed entities which has lead some authors to describe a functional model of data. The entity-relationship diagrams of these software analysis methods and the accompanying CASE tools that emerged in the 80's bear more resemblance to notation that preceded the work of Codd and Chen such as Bachman's data structure diagrams from 1973 than to the diagrams of Chen. Among the many, \commentary{also mention the versions in Chen's later english paper} and as summarised in the book of Rosemary Rock-Evans, there are three variants of binary entity relationship diagram that stand out, those found, respectively, in SSADM/Barker-Ellis (now adopted by Oracle), in Clive Finkelstein and James Martin's Information Engineering, and in IDEF.

\mynote In some instances, software methodologies and supporting CASE tools introduced an intermediate step between the ER model and the relational model naming the intermediary model the physical design model to contrast with the logically descriptive model that precedes it in the software development life-cycle. By a significant methodological improvement described in later sections we follow this approach but are able to eliminate the normalisation step.
\begin{figure}[H]
\small
\begin{center}
\setlength{\tabcolsep}{2pt}
\begin{tabular}{ p{1.4cm}  p{2.2cm}  p{1.5cm} p{1.5cm} p{1.5cm} p{1.5cm}  p{1.25cm}}
\raisebox{-0.8cm}{\parbox{1.4cm}{logical er~model}}& \textit{Chen~transform (automatic)} $\xrightarrow{\hspace*{1.75cm}}$ &
\raisebox{-0.8cm}{\parbox{1.4cm}{physical er model}}& \textit{manually normalise} $\xrightarrow{\hspace*{1.5cm}}$ &
\raisebox{-0.8cm}{\parbox{1.4cm}{physical er model}}& \textit{code generate} $\xrightarrow{\hspace*{1.5cm}}$ &  \raisebox{-0.8cm}{\parbox{1.25cm}{relational schema}} 
\end{tabular}
\end{center}
\caption{Traditional methodology for relational data design includes a manual normalisation step.}
\end{figure}

\mynote It is noteworthy that in these methodologies the normalisation step is present in order to achieve the goodness of the physical data model as prescribed by Codd in his normal form prescriptions. In the methodology described in this book we achieve the goodness of the final physical design, i.e. Codd's normal forms, by enabling suitable and pertinent real world conditions to be expressed at the logical level and supporting an automatic transformation to physical models that take advantage of these conditions so as to be able to meet the goodness prescriptions.\footnote{We also describe the features of the logical model that determine whether in addition to Codd's third normal form the goodness condition known as the Boyce-Codd normal form is also be met by the physical model}

