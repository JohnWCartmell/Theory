\section{Communication and Identifying Features}

\mynote An entity model systematically describes all that can be known of an entity
in terms of core functional relationships with universals and core functional relationships with particulars i.e. in terms of core attributes and core outgoing directional relationships.

\mynote But of all this which  may be known of an entity all we can communicate of an entity is its functional relationships with universals. Other relationships must be communicated indirectly via derivative (i.e. non-core) such relationships with universals. 

\mynote For this reason a systematic way of identifying and referencing each particular type  of entity in an entity model become significant if the model is to be used as any kind of data or message specification. There is some subtlety to identification and referencing as we will explain.

\subsection*{Identifying Features}
\mynote We say that a set of features of a type of entity i.e. a set of attributes and outgoing directional relationships, is identifying provided that the values of these features is guaranteed to uniquely identify an entity of the type i.e. to be such that no two distinct entities of the type can have identical values for all of the features.

\mynote Very often the set of identifying features of a type of entity will be a singleton set containing a single attribute. An example of this 
in  shown in figure \ref{BoardingPass2} in which entities of type \textit{airline route} are uniquely identified by the \textit{flight number} attribute. This is
indicated in the diagram  by the underlining of the name of this attribute where it appears on the diagram.\footnote{Aside: There is a trivial difference here from Barker's notation because he distinguishes the identifying attributes with a \# symbol where we here use  underlining.} 

\mynote In other cases the set of identifying features will consist of a number of attributes. In such case  the values of these attributes taken together, uniquely identify entities of the type. This is the case with the entity type xxx shown in the example in figure \ref{yyy}. In this example attributes xxx and yyyy taken together unqiely identify entitties of typoe zzz. 

\mynote Simple examples. This might be based on the ssadm example if I construct that first. Can also look at the example from Chen also.

\mynote Work on this example from page 3-13 of Barker.
\commentary{flight needs data  of departure (time of departure not required)}
\begin{figure}[H]
\begin{center}
\begin{erdiagram}{6.4}{9.5}

\eret{0.5}{-2.2}{3.1}{-1.2}{0.2}{1}\eretname{0.76}{-1.55}{l}{airline route}
\erCoreAttribute{0.7}{-1.75}{1}{0}{flight number}{}
\eret{0.5}{-4.3}{3.1}{-3.3}{0.2}{1}\eretname{0.76}{-3.65}{l}{flight}
\erCoreAttribute{0.7}{-3.85}{1}{0}{date of departure}{}
\eret{5.5}{-4.3}{8.5}{-3.3}{0.2}{1}\eretname{5.8}{-3.65}{l}{aircraft}
\erCoreAttribute{5.7}{-3.85}{1}{0}{registration number}{}
\eret{0.5}{-6.4}{3.1}{-5.4}{0.2}{1}\eretname{1.8}{-5.75}{}{boarding pass}
\eret{6}{-6.4}{8}{-5.4}{0.2}{1}\eretname{6.2}{-5.75}{l}{seat}
\erCoreAttribute{6.2}{-5.95}{1}{0}{number}{}

% relationship scheduled_as
\errelname{1.65}{-2.5}{r}{scheduled}\errelname{1.65}{-2.8}{r}{as}\errelname{1.95}{-3.15}{l}{of}\errelarm{1.8}{-2.2}{1.8}{-2.75}{0}{0}\errelarm{1.8}{-2.75}{1.8}{-3.3}{1}{0}\eridcomprel{1.7}{1.9000000000000001}{-3.0500000000000003}\ercrowfoot{1.8}{-3.15}{1.65}{-3.3}{1.8}{-3.3}{1.95}{-3.3}{0}
% relationship boarded_using
\errelname{1.65}{-4.6}{r}{boarded}\errelname{1.65}{-4.9}{r}{using}\errelname{1.95}{-5.25}{l}{issued for}\errelarm{1.8}{-4.3}{1.8}{-4.85}{0}{0}\errelarm{1.8}{-4.85}{1.8}{-5.4}{1}{0}\eridcomprel{1.7}{1.9000000000000001}{-5.15}\ercrowfoot{1.8}{-5.25}{1.65}{-5.4}{1.8}{-5.4}{1.95}{-5.4}{0}
% relationship scheduled_against
\errelname{3.25}{-3.65}{l}{against}\errelname{3.25}{-3.35}{l}{scheduled}\errelname{5.35}{-4.1}{r}{allocated to}\errelarm{3.1}{-3.8}{4.3}{-3.8}{0}{0}\errelarm{4.3}{-3.8}{5.5}{-3.8}{0}{0}\ercrowfoot{3.25}{-3.8}{3.1}{-3.65}{3.1}{-3.8}{3.1}{-3.95}{0}
% relationship made up of
\errelname{7.15}{-4.6}{l}{made up of}\errelname{6.85}{-5.25}{r}{on}\errelarm{7}{-4.3}{7}{-4.85}{0}{0}\errelarm{7}{-4.85}{7}{-5.4}{1}{0}\eridcomprel{6.9}{7.1}{-5.15}\ercrowfoot{7}{-5.25}{6.85}{-5.4}{7}{-5.4}{7.15}{-5.4}{0}
% relationship issued for
\errelname{3.25}{-5.75}{l}{issued for}\errelname{5.85}{-6.2}{r}{used via}\errelarm{3.1}{-5.9}{4.55}{-5.9}{1}{0}\errelarm{4.55}{-5.9}{6}{-5.9}{0}{0}\ercrowfoot{3.25}{-5.9}{3.1}{-5.75}{3.1}{-5.9}{3.1}{-6.05}{0}\eridrefrel{3.35}{-5.800000000000001}{-6}
\end{erdiagram}

\caption{This example is based on an example developed in the Barker book. I have simplified in some areas.}
\label{BoardingPass2}
\end{center}
\end{figure}

\subsection*{Communicating Relationships}
\mynote To communicate the value of a relationship is to communicate the identity of the relationship and  to identify (i.e. to communicated the identity of) the related entity. 

\mynote The related entity is a particular we are required therefore to have ways of identifying particulars and communicating these identifications.  
Only universals can be communicated so each type of entity is required 
to have one or more identifying attributes. 

\mynote Every entity may be identified by the values of related universals i.e. values that can be attributed to it. These attributions  may be core or derivative.


\mynote Since these relationships with other particulars themself need communication then ultimately every entity can be identified by a set of attributes each one of which may be core or may be derivative.  

\mynote
These attributes of a particular are called identifying attributes and subsequent use of these as attributes in a communication scheme in order to identify an entity as a related entity are known as referential attributes. 

\mynote For a long time now identifying attributes have been called key attributes and referential attributes have been called foreign key attributes. 

\mynote If you think that each outgoing directional relationship requires a distinct set of referential attributes to support its communication then you would be wrong. 
There is a possibility of `collapsed referentials' (a phrase coined in Shlaer and Long) whereby a single referential attribute may support communication of two or more distinct relationships. If you haven't understood this collapsed referentials concept then you have not  fully and completely understood the nature of data. 

\mynote In the presence of collapsed of referentials the number of bits of information required to communicate all the relationships of an entity is less that the sum of the bits required for their individual communication. 

\mynote the resolved set of attributes

\mynote conceptual core versus data core


\begin{oldtt}
\mynote To communicate the value of an attribute is to communicate its identity as an attribute and the universal that is its value. Both the identity of an attribute and its value are universals and communication of universals is a given i.e. does not require further explanation. 
\end{oldtt}



\begin{noteforfuture}
I imagine describing regular communication schemes... and why there can be a choice of schemes for relationships because of the choice of identifying features of target entity. 
\end{noteforfuture}
\begin{noteforfuture}
I imagine describing the communication of outgoing directional relationships of an entity by way of referentials and subsequently explaining that referentials can collapse which yields the term collapsed referentials.
\end{noteforfuture}
\begin{noteforfuture}
For discussion of universals in  the context of mereology see A.J.Cotnoir in my data/database literature review. In particular
\begin{erquote}
Universals are typically said to be ‘wholly located wherever they are instantiated’.
\end{erquote}
\end{noteforfuture}



 
