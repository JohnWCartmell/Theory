
\section{Entity Types}

\mynote The word `entity' as it is used in this book is used in its most general sense,  the sense in which it just means `thing' --- the entities that are being modelled in entity modelling can be just any old 'things' at all --- and so we could just as well speak of `modelling things' as `entity modelling' but that the term entity modelling
has come to mean a particular way of modelling things in which things are described by describing their types and the types of relationships between them (and, strictly speaking, between them and types of universals, as we will come to).

\mynote There is a proviso to this. We do not set out to model types whose instances are all known from the beginning (though these types are available and relevant). Rather we set out to model types whose instances (things) are open-ended and may vary from context to context. These are the subject entity types within an entity model. The whole entity model is expected to be realised in many contexts and the instances of any given subject type are expected to vary from context to context.
 
\mynote
Conversely, things that are known from the beginning and unchanging from one context to another, things that can besaid to be `universal' are excluded from being subjects of our modelling. We do not model, therefore, whole nunbers, nor real nunbers, nor truth values (booleans) but take these as universal givens and, generally, we do not model language characters and language character sequences (strings) with all the richness that this would imply and we take these too to be universal and given. 

\mynote
In philosophy the non-universal things that are our entitites are called particulars and so we can say that subject entity types within an entity model represent types of things all of whose instances are particulars. 

\mynote An `entity model' enumerates words that are to be used for types of things and also words to be used for types of relationship between things. Each type of relationship defines a possible relationship between things of one particular subject entity type and things of another particular type which is either a subject entity type or a given type of universals.  

\mynote Mathematically speaking an entity model is not much more than a labelled directed graph and for this reason they are most usually presented as diagrams.

\mynote We have indicated that subject entity types are related in the entity model to 
other subject types but also to given types of universals. 
Relationships of the second kind are referred to as attributes. The term relationship is reserved, in entity modelling terminology and subsequenly in these pages, for relationships of the first kind.  The principle components within an entity model are entity types, relationships\footnotes{Some authors would say entity types, relationship types and attributes and though I have sympathy with this terminology I shall stick to relationship where others perhaps correctly have said relationship type}  and attributes. 

\mynote When entity models are presented in diagrams as directed graphs
they are called ER diagrams (E for entity, R for relationship) or sometimes
ERA diagrams (where the A is for attribute). 

\mynote
The presence of attributes and therefore the involvement of a given types of universals
is an optional feature of entity modelling. 

\mynote Though entity modelling may be about just about any thing it is never about modelling everything.
An entity model serves to describe a domain of discourse. In the general case it describes the entities in this domain in relation not just to each other but to other given types which are types which serve to represent that which is universal. 


\mynote  
 
\section{Relationships}

\mynote Sometimes relationships are easy to name, sometimes not. 

\mynote 

parent--child relationship also described as the parenthood relationship

succedes--succeeded-by relationship also described as the succession relationship

married to relationship also described as the matrimony relationship

owns--owned-by relationship also described as the ownership relationship

references--referenced-by relationship described as the referentiality relationship?

composed-of--part-of relationship described as the composition relationship


\mynote We might model that a bicycle is related to its front wheel and that a bicycle is related to its back wheel but we will not then need to model,  and therefore ought not to model, that the front wheel of a bicycle is related to its back wheel because though there is such a relationship it is implied by the other two.  

\mynote xxx
  
  