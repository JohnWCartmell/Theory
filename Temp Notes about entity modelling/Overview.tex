
\newcommand{\mysection}[1]{\underline{\hyperref[#1]{#1}}}
\section*{Overview}
\begin{tabular}{l l p{7cm}}
1 & \mysection{Perspective} & Entity modelling form of concept modelling, more narrowly for describing the structuring of data.\\
\hline
2 & \mysection{EntityModels} & Entities, particulars and universals, attribute and relationship, entity model as structured document, ER and ERA diagrams, Chen origins, diamonds, Barker-Ellis notation, single composite thing, the absolute, parts, hierarchies, structured entity models.\\
\hline
3 & \mysection{Relationships} & Relationship and relationship instance, binary relationships, directional relationships, Chen diamond notation, Barker-Ellis notation, many-many relationships, functional relationships, ternary relationships. \\
\hline
4 & \mysection{AnExampleEntityRelationshipDiagram} & Chen manufacturing example, wikipedia entry `part number'.\\
\hline
5 & \mysection{Attributes} & attributes as functional relationships, value types.
\\
\hline
6 & \mysection{CoreversusDerivative} &  Core versus derivative, conceptual core, data cores, goodness criteria. The Core and Derivatives?\\\hline
?? & \mysection{Identifying Features} &  communication of entities\\
\hline
7 & \mysection{Scope} & How far a relationship extends across the landscape of entities. Schlaer Lang example. Auxiliary to the type of a directional relationship. relationship x relates entitties of type A to entitites of type B that clause. EG. \textit{that are in the same department}. \\
\hline
8 & \mysection{DataModelling}& Database and messge structure, conceptual, logical, physical, Codd oriented history, goodness criteria, normal forms, methodology improvement.\\
\hline
9 & \mysection{TypeInheritance} & specialisation and generalisation, species and genera, meta physics, nestedbox notation. Described as sub-types and super-types in Barker's book and likewise but mentioned as non-standard in SSADM book. Single inheritance versus multiple inheritance.   \\
\hline
10 & \mysection{StructuredEntityModelling} & Chen, PCTE, composition relationships,  top-down style, simple meta-model.\\
\hline
11 & \mysection{TheAbsolute} & Some metaphysics.\\
\hline
12 & DistinguishingCompositionandReference&The Distinction Between Composition and Reference from tutorial part one\\
\hline
13 & ScopeRevisited &The Scope Concept from tutorial part one\\
\hline
14 & DiagramsExpressingScopes &from tutorial part one\\
\hline
\end{tabular}
\section*{Notes}
\mynote I should concentrate on document structure.
\mynote I should have one source file per section.
\mynote I should use commentaries to identify places where rewrites or new examples are required. 

\mynote Beef up the `Scope' section futher. 
Drop the SSADM book customer,payment,allocated payment,invoice,booking,vehicle and vehicle category in earlier as well. Then have the two subdiagrams in this scope section 3and comment on the scope of these.
Reproduce the entire such ssadm example. Put in as a second example ERD in the current example erd section.
\begin{noteforfuture}
shlaerlang use the term \textit{collapsed referentials}.
an I use this as a section title? That would be good.
\end{noteforfuture}


\mynote where do I cover Identifying Features i.e. jointly monomorphics ?\\
One option is to cover attributes that are injective and then in a later section to discuss sets of features that are monomorphic. Explain use of the word `key'. 

\mynote rationale -- when it comes to one order for the introduction of terms rather than another -- there are a few considerations which likely conflict
\begin{itemize}
	\item  present conceptual modelling before data modelling
	\item  present relational data modelling before structured entity modelling
	\item rationalise structured entity modelling from point of view of hierarchical data specification. 
	\item  present identifying features as part of conceptual modelling because
	it is applicable to conceptual modelling or
	\item  present identifying features as a part of data modelling
	because there is then more value to it.
	\item   present goodness criteria as part of conceptual modelling
	or present the core and its derivatives as part of conceptual modelling then goodness criteria as part of data model where value is stated.
	\item similarly present scope as part of conceptual modelling because it si part of understanding concepts then show its value in data modelling? 
	\item in goodness section in conceptual modelling bit discuss absence of referential attributes to entities in scope of model and out of scope of model.
	\item and dont model a referential attribute in preference to a relationship.
	\item in data modelling section reintroduce referential attributes. 
	\item somewhere have a section on entity modelling without diagrams. Can get almost most of the most significant advantages of entity modelling without using diagrams. This might be a  introducing xml and ERScript without hgaving to worry about diagrams.
\end{itemize}

\mynote embed a few more examples Check out SSADM book page 213. 
\mynote meta model example given in section 9 needs to have other meta-models before it and needs the very idea to be explained.
