\section{Perspective}
\label{Perspective}
At its broadest, entity modelling is a technique and a notation for describing and communicating what is in the world and we set out to present it here from first principles. More usually, a narower view, its purpose is the structuring of data to be stored in information systems; we extend the notation and describe this use in information systems development but we do not to take this use as the starting point for the presentation nor as a subsequent raison d'etre because to do either would be to obscure the wider view that the purpose of an entity model is to provide a framework for knowledge and that entity modelling is a form of conceptual modelling — a technique for the elaboration of concepts. We describe the technique from first principles and relate to other meta-conceptual systems. We extend the notation from that commonly used in the development of information systems by the introduction of various kinds of constraints. This is done in a way strongly influenced by consideration of conceptual patterns brought to the fore in the branch of mathematics known as category theory. We shall demonstrate that the extended notation has significant benefits for information systems development.\commentary{Strenthen this sentence. Theory of data?} I cannot over emphasise this point.

To illustrate at the outset what we mean by conceptual modelling, consider the experience of reading into a new subject area and finding terms which seemingly have specific patterns of usage, and, it must be assumed, contextual meanings, but which patterns and meanings are unfamiliar to us. In so reading we are drawn into a systematic and iterative arrangement and a classification of the unfamiliar terms; in this process we will likely distinguish terms for individual things, for types or classes of things, for relations between things and also quantitative and adjectival terms. The mental process we follow will build for us a conceptual model. Entity modelling is a particular technique for expressing such models and, indeed, it is a technique and a notation used by information scientists seeking to represent and computerise the sometimes unfamiliar domains in which they work. When asked whether they understand a particular topic, an entity modeller might well affirm they do so only if they can sketch an entity model that frames the topic.

Thinking about understanding a new area and its language, though ‘things themselves’ are the subjects of the text, the language is of the types of things, the relations among the different types of things and the properties that can meaningfully be attributed to them. 

The premises of Entity Modelling have in them a pragmatic answer to the question what is knowledge? Knowledge, according to the premises of Entity Modelling, is knowledge about things. For any ‘thing’ the knowledge that we can have of it is a sum of elementary pieces each of which is either the fact of an attribution, by which is meant a property a thing has inherent in itself, or else the fact of a relationship with another thing. More precisely, we can have knowledge of the type of a thing and knowing its type is to know both the kind of attributions which may be made of it and the relationships in which it may participate. This is the theory of knowledge according to the entity modeller; it is also the basis of information modelling and therefore it is a pragmatic view of what knowledge is — it is that which can be represented as information in a structured form suitable for representation in a computer system or in a single computer program.

Frequently, computer systems and individual computer programs have as their subjects, everyday if not concrete and physical things, things such as people, accounts, orders, contracts, airline bookings, and so on; other computer programs have as their subjects the structures of molecules, languages, stellar processes or even programs themselves, or such as mathematical propositions, relationships in general, not particular, types of things as distinct from the things themselves, and so on. These are the sorts of entities that we are concerned with in this book and various points each such type appear as an entity type within an entity model for illustrative purposes in this book. Other entity types we present in models will include knots within quipus, atioms within molecular structures, adjectives within the sentence structure of the English. In this book all these types of things appear in specific entity models and each entity model describes a particular domain of discourse. This domain of discourse is the perspective within which there are types of things entity types defined by the relations between one another. In illustrative fragments we also have chickens and their eggs and bicycles and their wheels. In some cases we lift examples from the literature of relational data theory and understand them better by looking at them through the lens of entity relationship modelling.



