\section{The Absolute}
\label{TheAbsolute}
Within structured entioty modelling.
Now  I want to discuss that the representation within an entity model of the whole of everything should in fact be referred to as `the absolute'.
\mynote
We use the term ‘the Absolute’, which has a varied history in metaphysical writing, to mean the whole of everything and, equally, the context for everything. In giving two meanings we are following a rich tradition — for definitions abound. The term was central to much of the philosophy of G.W.F Hegel — in Shorter Logic section 87, among a number of other definitions, we find:
\begin{erquote}
...the Absolute is the Nought...The Nothing which the Buddhists make the universal principle, as well as the final aim and goal of everything, is the same abstraction.
\end{erquote}
whereas in Phenomonology of Spirit in section 20 we find:
\begin{erquote}
The True is the whole.
\end{erquote}
and in section 75 of the same we find:
\begin{erquote}
...the Absolute alone is true, truth alone is absolute.
\end{erquote}
\mynote
Some sense can be made of these statements when we consider from a logical perpective and from the point of view of information theory. We can illustrate by considering the seemingly puzzling fact that in some programming languages (such as ML) there is an in-built concept with the name ‘Unit’ and described as a singleton type, whereas in some other earlier programming languages (Algol68, C) the name given to the very same concept has been ‘void’. The apparently striking disparity in naming, the one versus the zero, has come about as a result of the concept being named on the one hand on the basis of the number of things predicated by the type (i.e the number of things we can say are of that type), which is exactly one, and on the other hand based on the number of bits (binary digits) of information carried in communicating a member of the type, which is precisely zero. We get the name ‘unit’ from one point of view and the name ‘void’ from the other. In this way we can say of the Absolute that it is the whole of everything. From it being the whole we can say there is only one of the type. From there being only one of the type we can say that it's information content is zero. If you present to me the absolute you present me with nothing. Like the true it can be assumed in all contexts for it carries nothing new with it. This is the logic of the absolute.

\mynote The whole of a modelling situation can be considered a single composite and this is both ‘the ultimate whole’ when considered as a composite and equally ‘the absolute’ when considered as a context. If what we have said above can be summarised as saying that there is a duality between context and composition then in this duality ‘the whole’ and ‘the absolute’ are duals: they are the same logical entities.

\mynote Another useage that we have, is to speak of concepts that are absolute. What we usually mean by saying of a concept that it is abolute is that it does not vary — that it is not relative to the context in which it appears. As we seek to construct models of usage and thereby a conceptual model we can expect to find a dichotomy of relative and absolute terms: some terms, such as father, daughter, length, colour, that vary in so much as they reference different items in different contexts; and terms, such as the earth, the pole star, the London Times that are absolute or constant in what they reference. Whereas relative terms are conceptualised as relationships or as quantitative or adjectival attributes of subject entities, absolute terms cannot be so interpreted unless we posit the existence of a singular entity and then interpret the absolute terms as relational to or as attributive to this singular entity. In this way the matter is finessed for we can say of a relative term such as father that it varies as the person varies — different subject persons having different fathers — and we can say of an absolute term pole star that it varies depending on the singular entity — the Absolute — which is to say that it does not vary at all.
