	
\note
Peter Dybjer \cite{dybjer96}introduces \term{categories with families} (\term{cwfs}) as a notion of model for a basic framework of dependent types and presents a generalised algebraic theory of such categories with families. \\
\note
In \cite{CastellanClairambaultDybjer2019} 
Castellan, Clairambault and Dybjer introduce the notion of \term{contextuality} of a cwf.


With regard to the  requirement that a cwf be contextual  they write:
\begin{tightquote}
\textit{
Although this requirement will be used in some of our equivalence theorems,
it is not part of our definition of cwf. The reason is that unlike the other
parts of the definition of cwfs, it does not correspond to an inference rule of
dependent type theory, and it is not expressed in the language of generalized
algebraic theories. However, the free cwf is contextual.}
\end{tightquote}
In spite of this it is arguable that there is a  generalised algebraic theory of contextual cwfs;
it is an argument worth having and so we start it here.\\

\note
I have also argued that there is a generalised algebraic theory of contextual categories (\cite{Cartmell86}).
This uses  the insight of Vladimir Voevodsky (\cite{Voevodsky14C}).
He  introduced an $s$-operator along with three axioms 
into the description of contextual categories\footnote{Though Voevodsky objects to the name contextual category and uses the name c-system as an alternative.}.  
He showed that the pullback structure of contextual categories is derivable 
from the s-operator and its associated axioms.  \\
\note
Contextual cwfs are equivalent to contextual categories in a way that we discuss here. 
In the presentation  we depart from the notation of Dybjer and use an infix operator $^*$ to represent substitution into types and terms. This notation then matches the notation used in contextual categories (\cite{Cartmell86}) and in metaGat algebras (\cite{Cartmell14B}). \\
\note
The argument that there are generalised algebraic theories of both contextual categories and contextual cwfs follows from an argument regarding the existence of a generalised algebraic theory of certain
tree structures. We get this argument out of the way first. \\

\note 
By  the generic term \term{tree} is meant a partially ordered set (poset) $(T, <)$ such that for each $t \in T$, the set $\set{s \in T : s < t}$ is well-ordered by the relation $<$.
In this discussion we restrict ourselves to rooted $\omega$-trees i.e. trees for which the set $\set{s \in T : s < t}$
is finite for all $t \in T$ and for which there is a least element in the partial ordering. 
In the original definition
of contextual categories (\cite{Cartmell78,Cartmell86},) there is defined to be such a tree-structure on the objects of the category. In a contextual category the root of the tree of objects is also  terminal object $1$
in the category and if object $y$ covers object $x$ in the partial ordering\footnote{A element $y$ is said \textit{to cover} 
an element $x$ in a partial ordering $<$ iff $x<y$ and there does not exist $w$ such that $x < w$ and $w < y$.
A more usual notation would be to write $x \lessdot y$ to assert that $y$ covers $x$ in partial ordering $<$. }
then we write $x \base y$ (we use this in preference to the more usual $x \lessdot y$). 

We define the rank (sometimes called the grade) of an element $t \in T$ to be the cardinality
of the set $\setsuchthat{s \in T}{s < t}$. If we define the set $T_i$ to be the set of elements of a tree
of rank $i$ then we have that $T= \bigcup_{i \in N}T_i$. 
The theory presented in figure \ref{theoryoftrees} is a generalised algebraic theory of rooted $\omega$-trees.\\
\newcommand{\Sz}{T_0}
\newcommand{\ofS}[1]{\ofT{#1}{\Sz}}
\newcommand{\Si}[1]{T_{#1}}
\newcommand{\ofSi}[3]{\ofT{#1}{\Si{#2}(#3)}}
\begin{figure}[H]
\caption{Theory of rooted $\omega$-trees (after elision of unnecessary variables)}
\label{theoryoftrees}
\begin{tabular}{>{\itshape}l l}
Symbol & \itshape{Introductory Rule} \\
$\Sz  $&$\isT{\Sz}$\\
$\Si{1} $&$\ofS{x_0} \tstyle \isT{\Si{1}(x_0)} $\\
$\Si{2} $&$\ofS{x_0},\ofSi{x_1}{1}{x_0} \tstyle \isT{\Si{2}(x_1)} $\\
\vdots  \\
$\Si{n} $&$\ofS{x_0},\ofSi{x_1}{1}{x_0}, \hdots \ofSi{x_{n-1}}{n}{ x_{n-2}} \tstyle \isT{\Si{n}(x_{n-1})} $\\
\vdots   \\
$root$ & $\ofT{root}{\Sz}$ \\
\itshape{Axioms}:  \\
$\ofT{x,y}{\Sz} \tstyle x = y$
\end{tabular} \\
\end{figure}
\newcommand{\bbin}[1]{
\raisebox{-0.5em}{$\stackrel{\displaystyle{\in}} {\scriptstyle{#1}}$}
}
\newcommand{\ofTn}[3]{
#1 \bbin{#2} #3}

\note
Modelled as a generalised algebraic theory, the theory of categories with families (cwfs) extends the theory of categories with terminal object by the following:



\begin{figure} [H]
\caption{The theory of categories with families (cwfs).}
\label{theorycwf}
\begin{gatrules}
\gatintros
\gatintroducing{ Ty \\Tm  }
\begin{gatgroup}{\ofT{x}{Ob}}
  \gatleaf{}{\isT{Ty(x)}} \\
	\gatleaf{\ofT{y}{Ty(x)}}
             {\isT{Tm(y)}}
\end{gatgroup} \\

\gatintroducing{ ^* \\^*}
\begin{gatgroup}{\ofT{z}{Ty(y)},\ \ofT{f}{Hom(x,y)},\ \ofT{x,y}{Ob}}
    \gatleaf  {}{\ofT{f^*z}{Ty(x)}} \\
    \gatleaf  {\ofT{t}{Tm(z)}} {\ofT{f^*t}{Tm(f^*z)} }
\end{gatgroup} \\

\gatintroducing{.\\p\\q}
\begin{gatgroup}{\ofT{y}{Ty(x)},\ \ofT{x}{Ob}}
\gatleaf{}{\ofT{x.y}{Ob}} \\
\gatleaf{}{\ofT{p_y}{Hom(x.y,x)}} \\
\gatleaf{}{\ofT{q_y}{Tm(p_y^*y)}}
\end{gatgroup} \\

\gatintroducing{\tuple{,}}
\gatsingular{\ofT{t}{Tm(f^*z)},\ \ofT{z}{Ty(y)},\ \ofT{f}{Hom(x,y)},\ \ofT{x,y}{Ob}}{\ofT{\tuple{f,t}}{Hom(x,y.z)}}  \\

\gataxioms 
\gatintroducing{  \gataxiomno{1} \\   \gataxiomno{2} }
\begin{gatgroup}{\ofT{y}{Ty(x)},\ \ofT{x}{Ob}}   
						\gatleaf{}{ id_x^*y=y }  \\
						\gatleaf{\ofT{t}{Tm(y)}}{id_x^*t=t } 
\end{gatgroup} \\

\gatintroducing{  \gataxiomno{3} \\   \gataxiomno{4}}
\begin{gatgroup}{\ofT{z}{Ty(y)},\ \ofT{f}{Hom(w,x)},\ \ofT{g}{Hom(x,y)},\ \ofT{w,x,y}{Ob}}
    \gatleaf{}{f^*g^*z=(f \circ g)^*z} \\
    \gatleaf{\ofT{t}{Tm(y)}} { f^*g^*t=(f \circ g)^*t}
\end{gatgroup} \\

\gatintroducing{ \gataxiomno{5} \\  \gataxiomno{6} }
\begin{gatgroup}{\ofT{t}{Tm(f^*z)},\ \ofT{z}{Ty(y)},\ \ofT{f}{Hom(x,y)},\ \ofT{x,y}{Ob}}   
						\gatleaf{}{ \tuple{f,t}\circ p_z=f }  \\
						\gatleaf{}{\tuple{f,t}^* q_z=t } 
\end{gatgroup} \\

\gatintroducing{  \gataxiomno{7}}
\gatsingular{\ofT{y}{Ty(x)},\ \ofT{x}{Ob}}
            {\tuple{p_x,q_y}=id_{x.y}} \\

\gatintroducing{  \gataxiomno{8}}
\gatsingular{\ofT{t}{Tm(g^*z)},\ \ofT{z}{Ty(y)},\ \ofT{f}{Hom(w,x)},\ \ofT{g}{Hom(x,y)},\ \ofT{w,x,y}{Ob}}{f \circ \tuple{g,t} =\tuple{f \circ g,f*t}} \\

\end{gatrules}
\end{figure}

\begin{figure} [H]
\caption{Illustration of axioms  $\tuple{f,t}\circ p_y = f$ and   $\tuple{f,t}^*q_y=t$.}
\vspace{0.1cm}
\begin{tabular} {c p{1cm} c}
\(\displaystyle
\begin{array}{ c p{1.0cm} c p{1.0cm} c}
       && \termballoon{Tm.psy}{q_y} && \\
\termballoon{Tm.fsy}{t} &&  && \emptyballoon{Tm.y}\\ [0cm]
       && \typeballoon{Ty.xpy}{p_y^*y} && \\
\typeballoon{Ty.x}{f^*y} &&  && \typeballoon{Ty.xp}{y}\\ [0.5cm]
  && \Rnode{xpy}{x'.y} &&   \\ [0.5cm]
\Rnode{x}{x} && && \Rnode{xp}{x'}  
\end{array}
\)
\ncarr{x}{xpy}
\alabel{\tuple{f,t}}
\ncarr{xpy}{xp}
\alabel{p_y}
\ncarr{x}{xp}
\blabel{f}
\termtether{Tm.psy}{Ty.xpy.point}
\termtether{Tm.fsy}{Ty.x.point}
\termtether{Tm.y}{Ty.xp.point}
\typetether{x}
\typetether{xpy}
\typetether{xp}
\ncline[nodesep=6pt]{|->}{Tm.psy}{Tm.fsy}
\blabel{\tuple{f,t}^*}
\ncline[nodesep=6pt]{|->}{Ty.xpy}{Ty.x}
\blabel{\tuple{f,t}^*}
\ncline[nodesep=6pt]{|->}{Tm.y}{Tm.psy}
\blabel{p_y^*}
\ncline[nodesep=6pt]{|->}{Ty.xp}{Ty.xpy}
\blabel{p_y^*}
\vspace{0.3cm}
%&& 
\begin{minipage}{6cm}
Axiom 5, $\tuple{f,t}\circ p_y = f$, corresponds to the commutivity of the lower triangle of contexts and realisations shown in this diagram.\\

Assume that $t \in Tm(f^*y)$.
From axioms 3 and 5 we have  that $\tuple{f,t}^*(p^*y)=f^*y$
and therefore from the rule introducing $\tuple{}$ we have that $\tuple{f,t}^*q_y \in Tm(f^*y)$. 
Axiom 6 tells us that in fact $\tuple{f,t}^*q_y=t$.
\end{minipage} 
\end{tabular}
\end{figure} 

\note In a contextual category \catcw, for any objects $x$ and $y$ such that $x \base y$ define
the set of sections\footnote{I called these arrows of $y$ and denoted $Arr(y)$ in my thesis.} of $y$ denoted $Sect(y)$ to be the subset of $Hom(x,y)$ of morphisms
$s:x \morph y$ in \catcw such that $s \circ p_y = id_x$. 

Likewise in a cwf \catcw define the set of sections of a 
type $y \in Ty(x)$, denoted $Sect(y)$,
to be the subset of morphisms $s: x \morph x.y$ in \catcw such that $s \circ p_y = id_x$. \\

\note 
Background: From the pullback structure of  a contextual category it follows that for any
object $x$ there is a diagonal morphism\footnote{So called because it is a local version of the diagonal 
morphism $\delta_x:x \morph x \times x$  defined in any category with products of pairs.}
$\delta_x : x \morph p(x)^*x$ which is a section (i.e. such that $\delta_x \circ p_{p_x^*x} = id_x$) and such that $\delta_x \circ q(p_x,x)=p_x$ as illustrated in this diagram.

\vspace{3mm}
\begin{center}
\begin{equation*}
\label{pullback}
\begin{array}{cp{0.5cm}cp{1.2cm}c}
\Rnode{x0}{x} &&                     &&           \\ [0.7cm]
             &&\Rnode{p_xstarx}{p_x^*x} && \Rnode{x1}{x}\\ [1.2cm]
             &&\Rnode{x2}{x}         && \Rnode{xp}{x_p}
\end{array}
\end{equation*}
\ncbsar{p_{p_x \sub x}}{p_xstarx}{x2}
\ncsar{x1}{xp}
\alabel{p_x}
\ncaarr{q(p_x,x)}{p_xstarx}{x1}
\ncsar{x2}{xp}
\blabel{p_x}
\setlength{\arrnodesepA}{3pt}
\jcbarr[-35]{id_x}{x0}{x2}
\ncaarr[35]{id_x}{x0}{x1}
\psset{linestyle=dashed}
\ncaarr{\delta_x}{x0}{p_xstarx}
\end{center}

\note To construct a cwf from a contextual category we take the base category of the cwf to be the underlying category of the contextual category. We define the set of types $Ty(x)$ of an object 
$x$ of \catcw to be the set of objects $y$ of \catcw such that $x \base y$ in \catc. 
If $x$ is an object of \catcw and if $y$ is in $Ty(x)$, i.e. if $x \base y$ in \catc, then
we define Tm(y), the set of terms of type $y$, to be the set $Sect(y)$.

The morphism $p_y: x.y \morph x$ required by the definition of cwf is the morphisms $p: y \morph x$ provided, whenever $x \base y$, by the definition of contextual category.
Similarly, the term $q_y \in Tm(p_y ^* y)$ required by the definition of cwf is provided by the 
section $\delta_y$ of the contextual category. \\

\note To any term $t \in Tm(y)$  of a cwf \catcw there corresponds a morphism of the base category which is a section
and which we denote $sect_y(t)$  defined as follows. 
If $x$ is an object of the base category and if 
$y$ is a type over $x$ i.e. if $y \in Ty(x)$, if $t$ is a term, $t \in Tm(y)$ 
then we define $sect_y(t): x \morph x.y$
to be the morphism $<id_x,t>$ which is defined because $id_x : x \morph x$,\ $y \in Ty(x)$ 
and $t \in Tm(y)$.
$<id_x,t>$ is a section because $\tuple{id_x,t} \circ p_y=id_x$ by cwf axiom 5. \\

\note 
\cite{CastellanClairambaultDybjer2019} define contextuality for a cwf.
Their definition implies, if a cwf is contextual,  there is a rooted $\omega$-tree structure on the objects of the base category 
such that for all objects $x$ and $w$ in the base category such that $x \base w$ there exists
a type $y \in Tm(x)$ such that $w=x.y$.
For such an object $x.y$, to every section $s \in Sect(x.y)$ i.e. to
every morphism $s: x \morph x.y$ such that $s \circ p_y=id_x$ there exists a term $t \in Tm(y)$ such that $sect_y(t)=s$.
We shall denote this term $term_y(s)$. Functions $term_y$ and $sect_y$ are inverses and establish an isomorphism between
the set of sections of $x.y$ and the set of terms $Tm(y)$. $term_y(s)$ is defined to be $s*q_y$. \\

\begin{figure}[H]
\caption{Illustration of $term(s_y)$ -- the term $s_y^*q_y$ associated with a section $s_y$.} 
\vspace{0.1cm}
\begin{tabular} {c p{0.1cm} l}
\(\displaystyle
\begin{array}{  c p{1.0cm} c}
\termballoon{Tm.psy}{q_y}    && \\
                             && \termballoon{Tm.y}{s_y^*q_y}\\ [0cm]
\typeballoon{Ty.xy}{p_y^*y} && \\
                             && \typeballoon{Ty.x}{y}\\ [0.5cm]
\Rnode{xy}{x.y}              &&   \\ [0.5cm]
                             && \Rnode{x}{x}  
\end{array}
\)
\ncarr{xy}{x}
\blabel{p_y}
\ncarr[-20]{x}{xy}
\blabel{s_y}
\termtether{Tm.psy}{Ty.xy.point}
\termtether{Tm.y}{Ty.x.point}
\typetether{xy}
\typetether{x}
\ncarc[nodesep=6pt, arcangle=20]{|->}{Tm.psy}{Tm.y}
\alabel{s_y^*}
\ncarc[nodesep=6pt, arcangle=20]{|->}{Ty.xy}{Ty.x}
\alabel{s_y^*}
&& 
\begin{minipage}{8cm}

If $s_y: x \morph x.y$ is a section in a cwf \catcw then
$s_y^*q_y \in Tm(y)$ because $s_y^*p_y^*y=(s_y \circ p_y)^*y=y$ because $s_y \circ p_y=id_x$ and
     $id_y^*y=y$.\\
		
If $t \in Tm(y)$ then 

$term_y(sect_y(t))= \tuple{id_x,t}^*q_y = t$ by cwf axiom 6. \\
and if $s : x \morph x.y$ is a section then \\
$sect_y(term_y(s))=\tuple{id_x,s}^*q_y=s$ also by cwf axiom 6. \\

Therefore functions $sect_y$ and $term_y$ are inverses and establish an isomorphism between the
set of terms $Tm(y)$ and the set of sections $Sect(x.y)$ in any cwf \catcw for any object $x$ of the base category and for any type $y \in Ty(x)$.
\end{minipage} 
\end{tabular}
\end{figure}


