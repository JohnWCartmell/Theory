

\newcommand{\qstarstructure}{q$^*$-structure}
\newcommand{\sqstarstructure}{sq$^*$-structure}
\note
If \catcw is a contextual cwf then to every object $x$ of \catcw other than the specified terminal object there is  uniquely an object $x_p$ of \catcw and a type $x_t \in Ty(x_p)$ such that $x=x_p.x_t$.
There is thus a tree-structure (strictly speaking, an $\omega$-tree-structure)  on the objects of the base category of any contextual cwf. This tree is rooted at the terminal object
and along with the morphisms $p_{y_t}: y \morph x$, for $y_t \in Ty(x)$ and for
 $x$ any object of the base category comprise the structure of a tree category.
Further than this we show in the next lemma that the base category comprises a contextual category


\begin{figure}
\caption{The definition of $f^*z$ and $q(f,z)$.}
\label{qdefinition}
\vspace{0.5cm}
\begin{tabular} {c c}
\(\displaystyle
\begin{array}{ c l p{1.0cm} c p{0.25cm} l }
             & \hspace{-0.8cm}\termballoon{Tm.psfszt}{q_{f^*z_t}} &&      &&  \\ [0.5cm]
             & \hspace{-0.65cm}\typeballoon[1.4cm][0.05]{Ty.xfstarzt}{(p_{f^*z_t} \circ f)^*z_t} &&   &&  \\ [1.25cm]
\typeballoon{Ty.x}{f^*z_t} 
             &  && && \hspace{-0.525cm}\typeballoon{Ty.y}{z_t}\\ [-0.5cm]
             & \Rnode{xfstarzt}{x.f^*z_t} && \Rnode{yzt}{y.z_t}  &&\\ [1.0cm]
\Rnode{x}{x} & && && \Rnode{y}{y}  
\end{array}
\)
\ncsar{xfstarzt}{x}
\alabel{p_{f^*z_t}}
\ncsar{yzt}{y}
\alabel{p_{z_t}}
\ncarr{x}{y}
\blabel{f}
\ncline[linestyle=dashed,nodesep=5pt]{->}{xfstarzt}{yzt}
\alabel{\tuple{p_{f^*z_t} \circ f,q_{f^*z_t}}}
\termtether{Tm.psfszt}{Ty.xfstarzt.point}
\typetether{x}
\typetether{xfstarzt}
\typetether{y}
\ncline[nodesep=6pt]{|->}{Ty.x}{Ty.xfstarzt}
\alabel{(p_{f^*z_t})^*}
\ncline[nodesep=6pt]{|->}{Ty.y}{Ty.xfstarzt}
\blabel{(p_{f^*z_t}\circ f)^*}
\vspace{0.5cm}
&
\hspace{-1.8cm}
\begin{minipage}{6.5cm}
\small
For $f: x \morph y$ and for any object $z$ of the form $y.z_t$ for some type $z_t \in Ty(y)$
define the object $f^*z$ using the cwf operators $.$ and $^*$ as the object $x.f^*z_t$. \\

The morphism $p_z:f^*z \morph x$ is defined in terms of the cwf operator $p$ as $p_{f^*z_t}$. \\

By definition, $q_{f^*z_t} \in Tm({p_{f^*z_t}}^*f^*z_t)$ and therefore 
$q_{f^*z_t} \in Tm((p_{f^*z_t}\circ f)^*z_t)$ because ${p_{f^*z_t}}^*f^*z_t=(p_{f^*z_t}\circ f)^*z_t$.
Therefore we can define $q(f,y.z_t): x.f^*z_t \morph y.z_t$ to be the cwf tuple $\tuple{p_{f^*z_t} \circ f,q_{f^*z_t}}$.
\end{minipage} 
\end{tabular}
\end{figure}
%\end{notebox} 

\begin{figure}
\caption{The definition of $s$.}
\label{sdefinition}
\begin{tabular} {c c}
\(\displaystyle
\hspace{-2cm}
\begin{array}{c  p{1.0cm}  c  c }
\hspace{3.0cm}\termballoon{Tm.fsqy}{f^*q_{y_t}}                      
                              && \termballoon{Tm.psy}{q_{y_t}}               \\ [0.6cm]
\hspace{4.0cm}\typeballoon[1.2cm][0.05]{Ty.x}{(f\circ p_{y_t})^*y_t} 
                              && \typeballoon{Ty.y}{p_{y_t}^*y_t}            \\ [0.15cm]
\Rnode{xfpsy}{x.(f \circ p_{y_t})^*y_t}        &&                           \\ [1.5cm]
\hspace{4.0cm} \Rnode{x}{x} &&\Rnode{y}{y_p.y_t}                            \\ [1.0cm]
                    && \Rnode{yp}{y_p}  
\end{array}
\)
\ncsar{y}{yp}
\alabel{p_{y_t}}
\ncarr{x}{y}
\blabel{f}
\ncsar{xfpsy}{x}
\alabel{p_{f\circ p_{y_t})^*y_t}}[0.25][0]
\ncarc[linestyle=dashed,nodesep=5pt,arcangle=30]{->}{x}{xfpsy}
\alabel{\tuple{id_x,f^*q_{y_t}}}
\termtether{Tm.psy}{Ty.y.point}
\termtether{Tm.fsqy}{Ty.x.point}
\termtether{Tm.y}{Ty.xp.point}
\typetether{x}
\typetether{y}
%\typetether{yp}
\ncline[nodesep=6pt]{|->}{Tm.psy}{Tm.fsqy}
\blabel{f^*}
\ncline[nodesep=6pt]{|->}{Ty.y}{Ty.x}
\blabel{f^*}
&
\hspace{-1.5cm}
\begin{minipage}{6.5cm}
\small
For $f: x \morph y$ in the base category of a contextual cwf 
we define a morphism $s(f)$ which in contextual category terms is a  morphism
$s(f): x \morph (x \circ p_y)*y$ such that $s(f) \circ p_{(x \circ p_y)*y}=id_x$. \\

Suppose $y$ is an object $y_p.y_t$, for some object $y_p$ and for some type $y_t \in Ty(y_p)$,
then in terms of the cwf operators we need define $s(f):x \morph  (f\circ p_{y_t})^*y_t$.

From the cwf definition we have that $f^*q_{y_t} \in Tm(f^*{p_{y_t}}^*y_t)$.
Therefore $f^*q_{y_t} \in Tm(f\circ p_{y_t})^*y_t)$ since $f^*{p_{y_t}}^*y_t=(f\circ p_{y_t})^*y_t$. \\

From the cwf definition of the $\tuple{}$ operator  it follows that
$\tuple{id_x,f^*q_{y_t}}:x \morph (f\circ p_{y_t})^*y_t$ in the base category. 
We define $s(f)$ to be $\tuple{id_x,f^*q_{y_t}}$.
\end{minipage} 
\end{tabular}
\end{figure}


\begin{lemma}
If \catcw is a contextual cwf then if the base category considered as a tree-structured category is augmented  by operations $^*$, $q$ and $s$ defined by
\begin{enumerate} [(i)]
\item for all $f: x \morph y$ in the base category \catcw, for objects $z$ such that $y \base z$ in \catcw, i.e such that
$z=z_t.y$ for some type $z_t \in Ty(y)$, define the object $f^*z$ 
and define the  morphism $q(f,z):f^*z \morph z$  using the $.$, $^*$ $p$, $q$ and $\tuple{}$ operations of the cwf \catcw   by
\begin{align}
f^*z    & = x.(f^*z_t) \\
q(f,z) & =\tuple{p_{f^*z_t} \circ f,q_{f^*z_t}}
\end{align}
That $\tuple{p_{f^*z_t} \circ f,q_{f^*z_t}}$ is defined 
and $\tuple{p_{f^*z_t} \circ f,q_{f^*z_t}}: f^*z \morph z$ is outlined in figure \ref{qdefinition}.
\item 
for all $f : x \morph y$ in the base category \catcw where $y=y_p.y_t$ for some type $y_t \in Ty(y_p)$ 
define\footnote{Note that $s(f) = sect(f*q_y)$ which corresponds to the way that 
$s$ can be recovered from a $\delta^*$-structure as $s(f)=f^*\delta_y$.} $s(f):x \morph (f\circ p_y)^*y$,
which is to say in terms of cwf operators that $s(f):x \morph x.(f \circ p_{y_t})^*y_t$, by
\begin{equation}
s(f) = \tuple{id_x,f^*q_{y_t}}
\end{equation}
That  $\tuple{id_x,f^*q_{y_t}}$ is well-defined and that $\tuple{id_x,f^*q_{y_t}}:x \morph x.(f \circ p_{y_t})^*y_t$
is outlined in figure \ref{sdefinition}.
\end{enumerate}
then  the operations $^*$, $q$ and $s$ 
meet the conditions of axioms (q1) -- (q5) and (s1) -- s(3) of the Voevodsky presentation of the axioms of a contextual category.
\end{lemma}
\begin{proof}
\begin{enumerate}[(i)]
\item 
Axiom (q1) requires,  
for all $f: x \morph y$ in the base category \catcw, for objects $z$ such that $y \base z$ in \catcw, i.e. for all objects $y.z_t$ where $z_t \in Ty(y)$ in \catcw
, the commutivity of this diagram:
\vspace{3mm}
\begin{center}
\begin{displaymath}
\begin{array}{cp{.9cm}c}
\Rnode{fstarz}{f^*z} & & \Rnode{z}{z}\\ [1.2cm]
\Rnode{x}{x}         & & \Rnode{y}{y}
\end{array}
\end{displaymath}
\ncbsar{p_{f \sub z}}{fstarz}{x}
\jcbarr{f}{x}{y}
\ncaarr{q(f,z)}{fstarz}{z}
\ncasar{p_z}{z}{y}
\end{center}
i.e. the commutivity of this diagram written in terms of the cwf operators:
\vspace{3mm}
\begin{center}
\begin{displaymath}
\begin{array}{cp{.9cm}c}
\Rnode{fstarz}{x.(f^*z_t)} & & \Rnode{z}{y.z_t}\\ [1.2cm]
\Rnode{x}{x}         & & \Rnode{y}{y}
\end{array}
.
\end{displaymath}
\ncbsar{p_{f^*z_t}}{fstarz}{x}
\jcbarr{f}{x}{y}
\ncaarr{\tuple{p_{f^*z} \circ f,q_{f^*z}}}{fstarz}{z}
\ncasar{p_{z_t}}{z}{y}
\end{center}

This commutes  by cwf axiom 5.

\item
Axioms (q2) and (q3) require that 
for all objects $x$ of \catcw, for all objects $y$ of \catcw such that $x \base y$ in \cat{C}, i.e. for all $y=x.y_t$ for some object
$y_t \in Ty(x)$, 
\begin{align}
id_x^*y   & =y \\
q(id_x,y) & = id_y
\end{align}
which when written in terms of the cwf operators is to say that we need show
\begin{align}
x.(id_x^*y_t)   & = x.y_t \label{q2rewritten}\\
\tuple{p_{id_x^*y_t} \circ id_x,q_{id_x^*y_t}} & = id_y \label{q3rewritten}
\end{align}
(\ref{q2rewritten}) follows from cwf axiom 1.

(\ref{q3rewritten})  simplies to
\begin{equation}
\tuple{p_{y_t},q_{y_t}}=id_{x.y_t}
\end{equation}
and this follows by cwf axiom 7.  
\item
Axioms (q4) and (q5) require that 
for all morphisms $f:w \morph x$ and $g:x \morph y$ in \catcw, for all objects $z$ of \catcw such that $y \base z$ in \cat{C}, i.e. for all $z=y.z_t$ for some object
$z_t \in Ty(y)$,

\begin{equation}
(f \circ g)^*z =  f^* (g ^* z)
\end{equation} 
and
\begin{equation}
q(f \circ g,z) = q(f,g^*z) \circ q(g,z)
\end{equation}

which when written in terms of the cwf operators is to say that we need show

\begin{equation}
w.(f \circ g)^*z_t =  w.(f^* (g ^* z_t))  \label{q4rewritten}
\end{equation}

and 

\begin{equation}
\tuple{p_{(f \circ g)^*z_t} \circ (f \circ g),q_{(f \circ g)^*z_t}} =  \tuple{p_{f^*g^*z_t} \circ f,q_{f^*g^*z_t}} \circ \tuple{p_{g^*z_t} \circ g,q_{g^*z_t}} \label{q5rewritten}
\end{equation}

(\ref{q4rewritten}) follows from cwf axiom 3.

The right hand side of equation (\ref{q5rewritten}) can be rewritten by use of the cwf axiom that says $f \circ \tuple{g,z}=\tuple{f\circ g,f^*z}$ as
\begin{equation*}
\tuple{
    \tuple{p_{f^*g^*z_t} \circ f,q_{f^*g^*z_t}} \circ p_{g^*z_t} \circ g,
		\tuple{p_{f^*g^*z_t} \circ f,q_{f^*g^*z_t}} ^* q_{g^*z_t}
}
\end{equation*} 
which simplies using axioms $\tuple{\alpha,\beta}\circ p = \alpha$ and $\tuple{\alpha,\beta}\circ q = \beta$
to 
\begin{equation*}
\tuple{
    p_{f^*g^*z_t} \circ f  \circ g,
		q_{f^*g^*z_t}
}
\end{equation*}
which by axiom $f^*g^*\alpha=(f \circ g)^* \alpha$ simplies to the left hand side of (\ref{q5rewritten}).
\item
Axiom (s1) 
\begin{equation}
s(f) \circ p_{f\sub p_y \sub y}=id_x
\end{equation}
That is we are rewquired to show $s(f)$ is a section but s(f) is defined as $sect(q_{etc})$ 
and we have shown SOMEWHERE such morphisms are sections.

\item Axiom (s2) Assume $f:x \morph y$  require to show that 

\begin{equation}
s(f) \circ q( f \circ p_y     ,y)=f
\end{equation}
Assume 
 $y=y_p.y_t$.
then rewriting above in terms of cwf operations we are required to show that 
\begin{equation}
\tuple{id_x,f^*q_{y_t}}\circ \tuple{p_{(f \circ p_{y_t})*y_t}\circ f \circ p_{y_t},
                                    q_{(f \circ p_{y_t})^*y_t}
																		} = f
\end{equation}
which we can show as follows:
\begin{align*}
\tuple{id_x,f^*q_{y_t}}\circ \tuple{p_{(f \circ p_{y_t})*y_t}&\circ f \circ p_{y_t}, q_{(f \circ p_{y_t})^*y_t}} \\
=& \tuple{\tuple{id_x,f^*q_{y_t}} \circ p_{(f \circ p_{y_t})*y_t}\circ f \circ p_{y_t}, 
				  \tuple{id_x,f^*q_{y_t}} ^* q_{(f \circ p_{y_t})^*y_t}
         } && \mbox{ by cwf axiom 8} \\
=& \tuple{f \circ p_{y_t},f^*q_{y_t}} && \mbox{by cwf axioms 5 and 6.} \\
=& f \circ \tuple{p_{y_t},q_{y_t}}    && \mbox{by cwf axiom 8.} \\
=& f                                  && \mbox{by cwf axiom 7.}
\end{align*}
\item 
axiom (s3) requires that for

\begin{center}
\begin{displaymath}
\begin{array}{c p{.9cm} c p{.9cm} c}
\Rnode{w}{w}&& \Rnode{g*z}{g \sub z} && \Rnode{z}{z} \\ [1.2cm]
            && \Rnode{x}{x}  && \Rnode{y}{y} \\ [0.2cm]
\end{array}
\end{displaymath}
\jcbarr{f}{w}{g*z}
\jcbarr{g}{x}{y}
\ncaarr{q(g,z)}{g*z}{z}
\ncasar{}{g*z}{x}
\ncasar{}{z}{y}
\end{center}

\noindent in \cat{C} then

\begin{equation}
s(f \circ q(g,z))=s(f)
\end{equation}

Suppose that $z=y.z_t$ for some $z_t \in Ty(y)$ then in terms of the cwf operators we need show that:
\begin{equation}
\tuple{id_w,(f \circ \tuple{p_{g^*z_t} \circ g, q_{g^*z_t}} )^* q_{z_t}} = \tuple{id_w,f^*q_{g^*z_t}}
\end{equation}

which we can show as follows
\begin{align*}
\tuple{id_w,(f \circ \tuple{p_{g^*z_t} \circ g, q_{g^*z_t}} )^* q_{z_t}}
    &= \tuple{id_w,f ^* \tuple{p_{g^*z_t} \circ g, q_{g^*z_t}} ^* q_{z_t}} &&\mbox{ by ccf functoriality}\\
		&= \tuple{id_w,f ^*  q_{g^*z_t}}  &&\mbox{ by ccf axiom 6.}\\
\end{align*}

\end{enumerate}
\end{proof}





