



\note
The theory of rooted $\omega$-trees that was presented in figure \ref{theoryoftrees} can be extended variously to a generalised algebraic theories of 
contextual categories with families and a generalised algebraic theory of contextual categories.
A common intermediate we have presented in \cite{Cartmell2018A} as the theory of tree structures categories.
This is shown in figure \ref{theoryoftreestructuredcategories}.   The theory of tree structured categories has
a sort of objects $Ob_n$ for each $n \geq 0$ and a sort of morphisms $Hom_{n,m}$ for each $n \geq 0$, for each $m \geq 0$. It has operator symbols $id_n$, $p_n$ and $t_n$ for each  $n \geq 0$ and it has
a family of operator symbols for composition $\circ_{n,m,p}$ for each $n,m,p \geq 0$. These 
symbols are introduced by rules presented in figure \ref{theoryoftreestructuredcategories} after ellipsis
of unnecessary variables. It has families of axioms in place of individual axioms. 

\note
To present the generalised algebraic theory of tree structured categories it becomes necessary to introduce a shorthand. We use 
$\ofTn{x}{n}{T}$ for the context $\ofS{x_0},\ofSi{x_1}{1}{x_0}, \hdots \ofSi{x_n}{n}{x_0,x_1,\hdots x_{n-1}} $. \\

\noindent Using this shorthand,  the family of sorts $T_{n+1}$, for $n \geq 0$,  in the theory of trees is introduced by: \\

\vspace{0.03cm} 
\begin{tabular}{>{\itshape}l l}
Symbol & \itshape{Introductory Rule} \\
$\Si{n+1}, n \geq 0 $ & $\ofTn{x}{n}{T}    \tstyle \isT{\Si{n+1}(x)} $\\
\end{tabular} \\
\vspace{.1cm}  \\

\begin{figure}[H]
\caption{Theory of tree structured categories}
\label{theoryoftreestructuredcategories}

\newcommand{\Obi}[1]{\Ob_{#1}}
\newcommand{\Homij}[2]{\Hom_{#1,#2}}
%\newcommand{\ofObi}[2]{\bbar{#1}\bbin{#2}{\Bbar{\Ob}}}
\newcommand{\ofObi}[2]{#1 \bbin{#2}{\Ob}}
%\newcommand{\HomijBar}[4]{\Homij{#1}{#2}(\bbar{#3},\bbar{#4})}
\newcommand{\HomijBar}[4]{\Homij{#1}{#2}(#3,#4)}
\newcommand{\ofHomij}[5]{\ofT{#1}{\HomijBar{#2}{#3}{#4}{#5}}}
%\newcommand{\HomijBarFt}[4]{\Homij{#1}{#2}(\bbar{#3},\ft{\overline{#4}})}
\newcommand{\HomijBarFt}[4]{\Homij{#1}{#2}(#3,\ft{#4})}
\newcommand{\ofHomiBarFt}[4]{\ofT{#1}{\HomijBarFt{#2}{#2-1}{#3}{#4}}}

\begin{tabular}{>{\itshape}l l}
Symbol & \itshape{Introductory Rule} \\[0.1cm]
$\Obi{0}  $&$\isT{\Obi{0}}$\\[0.2cm]
$\Obi{n+1} $&$\ofTn{x}{n}{\Ob} \    \tstyle \isT{\Obi{n+1}(x)} $\\ [0.25cm]
$\Homij{n}{m} $ &$\ofObi{x}{n} ,\  \ofObi{y}{m} \ \tstyle \isT{\HomijBar{n}{m}{x}{y}} $\\ [0.25cm]
$\circ$ & $\ofObi{x}{n}, \  \ofObi{y}{m}, \ \ofObi{z}{p}, \ \ofHomij{f}{n}{m}{x}{y},\ofHomij{g}{m}{p}{y}{z} \tstyle \ofHomij{\circ(f,g)}{n}{p}{x}{z}$ \\ [0.25cm]
$id_n   $   & $ \ofObi{x}{n} \tstyle \ofHomij{id_n(x)}{n}{n}{x}{x} $\\ [0.25cm]
$p_n   $   & $ \ofObi{x}{n} \tstyle \ofHomij{p_n(x)}{n}{n-1}{x}{x_{n-1}} $\\ [0.25cm]
$1     $   & $\ofT{1}\Obi{0} $\\         [0.25cm]
$t_n   $   & $ \ofObi{x}{n} \tstyle t_n(x) \in \Homij{n}{0}(x,1)$\\ [0.25cm]
\end{tabular} \\
\vspace{.1cm}  \\
\vspace{.03cm} \\
\begin{tabular}{l}
\itshape{Axioms} \\

$\circ(id(x),f)=f \mbox{,  whenever\ } \ofObi{x}{n} ,\  \ofObi{y}{m} , \ \ofHomij{f}{n}{m}{x}{y}  $\\ [0.25cm]
$\circ(f,id(y))=f \mbox{,  whenever\ } \ofObi{x}{n} ,\  \ofObi{y}{m} , \ \ofHomij{f}{n}{m}{x}{y}  $\\ [0.25cm]
$ \circ(\circ(f,g),h) = \circ(f,\circ(g,h)),$ \ whenever \\ [0.2cm]
$ \ \ \ \ \ \ \ \ \ \ \ \ \ \ \ 
\ofObi{w}{l} , \ \ofObi{x}{n} , \ \ofObi{y}{m} , \ \ofObi{z}{p}, \ 
 \ofHomij{f}{l}{n}{w}{x}, \ofHomij{g}{n}{m}{x}{y}, \ofHomij{h}{m}{p}{y}{z}$ \\ [0.25cm]
$x = y, $ whenever $x,y \in \Obi{0} $ \\ [0.25cm]
$f = t_n(x),  $ whenever $ \ofObi{x}{n}, \ f \in \Homij{n}{0}(x,1)$\\ [0.25cm]
\end{tabular}  \\
\end{figure}
\note
Following this style the operators and axioms given earlier in figure \ref{theorycwf} can be recast as
families of sort symbols, operators and axioms as a extension to the theory of tree structured categories.
What results is a  generalised algebraic theory of contextual categories with families.




