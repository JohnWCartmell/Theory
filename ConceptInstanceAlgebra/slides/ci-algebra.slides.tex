
\usepackage{mathptmx}
\usepackage{amsfonts}
\usepackage{wasysym}
\usepackage{url}
\usepackage{hyperref}

\newcommand{\SharedMacros}{../../SharedMacros}
\newcommand{\SharedText}{../../SharedText}
%\usepackage{imakeidx}
\usepackage{framed}
\makeindex[name=definitions, title=Index of Definitions]
\makeindex[name=lemmas, title=Index of Lemmas]



\newcommand{\seenudgeup}[1]{\rule{0.1cm}{#1}}

\newcommand{\seenudgedown}[1]{\rule[-#1]{0.1cm}{0.1cm}}

\newcommand{\nudgeup}[1]{\rule{0cm}{#1}}

\newcommand{\nudgedown}[1]{\rule[-#1]{0cm}{0.1cm}}

\definecolor{highlight}{cmyk}{0,0,0.7,0}
\newcommand{\commentary}[1]{\marginpar{\footnotesize #1}}
\newcommand{\highlight}[1]{\colorbox{highlight}{#1}}
\newcommand{\whitelight}[1]{\colorbox{white}{#1}}
\newcommand{\term}[1]{\textit{#1}\commentary{\colorbox{lightgray}{\textit{#1}}}\index[definitions]{#1}}
\newcommand{\llabel}[1]{\label{#1}\commentary{\colorbox{pink}{\scriptsize{#1}}}\index[lemmas]{#1}}
\newcommand{\lref}[1]{\ref{#1}\colorbox{pink}{\scriptsize{#1}}\index[lemmas]{#1!use of}}

\newcommand{\daynote}[1]{\commentary{See day notes #1.}}

\newcommand{\newt}[1]{\colorbox{yellow}{#1}}
\newenvironment{newtt}
{  \colorbox{yellow}{$[$ ...} 
}
{  \colorbox{yellow}{... $]$}
}
\newcommand{\oldt}[1]{\colorbox{red}{\sout{#1}}}
\newenvironment{oldtt}
{  \colorbox{red}{$[$ ...} 
}
{  \colorbox{red}{... $]$}
}

\newcommand{\reinstatet}[1]{\colorbox{lime}{#1}}
\newenvironment{reinstatett}
{  \colorbox{lime}{$[$ ...}
}
{  \colorbox{lime}{... $]$}
}

\newcommand{\tbd}{\highlight{TBD}}

%ithprojection function
\newcommand{\proji}[1]{\pi_#1}


\newenvironment{aside}
{\begin{framed}
\textbf{Aside}
}
{
\end{framed}
}

\newenvironment{notebox}[1][Note]
{\begin{framed}
\textbf{#1}
}
{
\end{framed}
}

\newenvironment{categoricalaside}
{\begin{framed}
\textbf{Categorical Aside}
}
{
\end{framed}
}

\newenvironment{noteforfuture}
{\begin{framed}
\textbf{Note For Future}
}
{
\end{framed}
}

\newenvironment{problem}
{\begin{framed}
\textbf{Problem}
}
{
\end{framed}
}

\newenvironment{key}
{
\begin{tabular}{c l p{4cm}}
KEY && \\
\hline
}
{
\end{tabular}
}

\newcommand{\keyentry}[3]{#1 & #2 & #3 \\} 


%quine quote
\newcommand{\qq}[1]{
\left\ulcorner#1\right\urcorner
}

%single quote
\newcommand{\sq}[1]{
\textnormal{\textquotesingle}#1\textnormal{\textquotesingle}
}

%lower quine quote
\newcommand{\lqq}[1]{
\left\llcorner #1\right\lrcorner
}


%from berkley
\newcommand{\langl}{\begin{picture}(4.5,7)
\put(1.1,2.5){\rotatebox{60}{\line(1,0){5.5}}}
\put(1.1,2.5){\rotatebox{300}{\line(1,0){5.5}}}
\end{picture}}
\newcommand{\rangl}{\begin{picture}(4.5,7)
\put(.9,2.5){\rotatebox{120}{\line(1,0){5.5}}}
\put(.9,2.5){\rotatebox{240}{\line(1,0){5.5}}}
\end{picture}}
\newcommand{\lang}{\begin{picture}(5,7)\put(1.1,2.5){\rotatebox{45}{\line(1,0){6.0}}}\put(1.1,2.5){\rotatebox{315}{\line(1,0){6.0}}}\end{picture}}
\newcommand{\rang}{\begin{picture}(5,7)\put(.1,2.5){\rotatebox{135}{\line(1,0){6.0}}}\put(.1,2.5){\rotatebox{225}{\line(1,0){6.0}}}\end{picture}}
%Try sharper tuple brackets -- except gives errors nested in captions so comment out
%\renewcommand{\tuple}[1]{\lang #1 \rang}

\newcommand{\setsuchthat}[2]{\left\{#1 \ \middle|\ #2\right\}}
\newcommand{\set}[1]{\left\{#1\right\}} 

% one to n - wanton
\newcommand{\wanton}[1]{#1_1,...#1_n}
\newcommand{\n}{1...n}
\newcommand{\fn}{\wanton{f}}
\newcommand{\gn}{\wanton{g}}
\newcommand{\pn}{\wanton{p}}
\newcommand{\qn}{\wanton{q}}
\newcommand{\qnprime}{\wanton{q'}}
\newcommand{\tn}{\wanton{t}}
\newcommand{\xn}{\wanton{x}}
\newcommand{\xnp}{\wanton{x'}}
\newcommand{\yn}{\wanton{y}}
\newcommand{\An}{\wanton{A}}
\newcommand{\Bn}{\wanton{B}}
\newcommand{\Cn}{\wanton{C}}
\newcommand{\ntuple}[1]{\tuple{\wanton{#1}}}
\newcommand{\wantom}[2][]{#2_1,...#2_{m#1}}
\newcommand{\m}{1...m}
\newcommand{\mtuple}[1]{\tuple{#1_1,...#1_m}}
\newcommand{\gm}{\wantom{g}}
\newcommand{\qm}{\wantom{q}}
\newcommand{\sm}[1][]{\wantom[#1]{s}}
\newcommand{\smp}{\wantom{s'}}
\newcommand{\ym}{\wantom{y}}
\newcommand{\Bm}{\wantom{B}}
\newcommand {\bntuple}{\ensuremath{\ntuple{b}}}
\newcommand {\fntuple}{\ensuremath{\ntuple{f}}}
\newcommand {\fnptuple}{\ensuremath{\ntuple{f}}}
\newcommand {\pntuple}{\ensuremath{\ntuple{p}}}
\newcommand {\qntuple}{\ensuremath{\ntuple{q}}}
\newcommand {\qnptuple}{\ensuremath{\ntuple{q'}}}
\newcommand {\qmtuple}{\ensuremath{\mtuple{q}}}
\newcommand {\sntuple}{\ensuremath{\ntuple{s}}}
\newcommand {\xntuple}{\ensuremath{\ntuple{x}}}
\newcommand {\xnptuple}{\ensuremath{\ntuple{x'}}}
\newcommand {\ymtuple}{\ensuremath{\mtuple{y}}}
\newcommand{\idef}[1][n]{1 \leq i \leq #1}
\newcommand{\jdef}[1][m]{1 \leq j \leq #1}
\newcommand{\kdef}[1][l]{1 \leq k \leq #1}
\newcommand{\foreachi}[1][n]{for each $i$, $1 \leq i \leq #1$}
\newcommand{\foreachj}[1][m]{for each $j$, $1 \leq j \leq #1$}
\newcommand{\foreachk}[1][l]{for each $k$, $1 \leq k \leq #1$}
\newcommand{\Foreachi}[1][n]{For each $i$, $1 \leq i \leq #1$}
\newcommand{\Foreachj}[1][m]{For each $j$, $1 \leq j \leq #1$}
\newcommand{\Foreachk}[1][l]{For each $k$, $1 \leq k \leq #1$}
\newcommand{\forsomei}[1][n]{for some $i$, $1 \leq i \leq #1$}
\newcommand{\forsomej}[1][m]{for some $j$, $1 \leq j \leq #1$}
\newcommand{\forsomek}[1][l]{for some $k$, $1 \leq k \leq #1$}
\newcommand{\wherei}[1][n]{where $1 \leq i \leq #1$}
\newcommand{\wherej}[1][m]{where $1 \leq j \leq #1$}
\newcommand{\wherek}[1][l]{where $1 \leq k \leq #1$}


\newcommand{\fundep}[3]{#2 \xrightarrow{#1} #3}  %where does this belong? xxxx
% Following used for notes -- indented numbered paras

\newcounter{para}
\newlength{\oldparindent}
\setlength{\oldparindent}{\parindent} % Save \parindent before of change
\newcommand{\ind}{\hspace*{\oldparindent}}
\newcommand\note{
%\setlength{\parskip}{0.5\baselineskip} % Definition of `parskip`
\setlength{\parindent}{0pt}
\par\ind\refstepcounter{para}\thepara.\space
\setlength{\parindent}{\oldparindent}
}


    %causes problems when used with bamer

%ccategories.macros.tex 

% Macros for diagrams in contextual categories and related categories

\usepackage{twoopt}
\usepackage{scalerel} 
\usepackage{xargs}

%\usepackage{mathabx}  %Caused font problems
%\usepackage{MnSymbol}  % caused font problems

\newcommand{\conu}
{\mathbf{C}(U)}

\newcommand{\depu}
{\mathbf{D}(U)}


\newcommand{\reqt}{\textbf{R}}
\newcommand{\reqtc}[1][\catc]{\reqt_{#1}}
\newcommand{\reqtcp}[1][\catcp]{\reqt_{#1}}



\newcommand{\cat}[1]{\textbf{#1}}

\newcommand{\catc}{\cat{C}}
\newcommand{\catcw}{\cat{C}\ }
\newcommand{\catcp}[1][C]{\textbf{#1}'}
\newcommand{\catcpp}[1][C]{\textbf{#1}''}
\newcommand{\obj}[1]{\ensuremath{|\cat{#1}|}}
\newcommand{\ccat}[1][C]{\ensuremath{\mathbb{#1}} }
\newcommand{\ccatc}{contextual category \ccat}
\newcommand{\cobj}[2][]{\ensuremath{|\ccat[#2]|_{#1}}}
\newcommand{\cslice}[2]{\ensuremath{\ccat[#1]_{#2}}}
\newcommand{\csliceobj}[3][]{\ensuremath{|\mathbb{#2}_{#3}|_{#1} }}
\newcommand{\varset}[1][]{\ensuremath{V_{#1} }}
\newcommand{\localvarsets}{\ensuremath{\mathcal{V} }}
\newcommand{\Fam}{\ensuremath{\mathbb{F\mathrm{am}} }}
\newcommand{\Fin}{\ensuremath{\textbf{Fin}} }
\newcommand{\Finp}{\ensuremath{\textbf{Finp}} }
\newcommand{\Po}{\ensuremath{\textbf{Po}} }
\newcommand{\Famslice}[1]{\ensuremath{\mathbb{F\mathrm{am}}_{#1} }}
\newcommand{\Famobj}[1][]{\ensuremath{|\mathbb{F\mathrm{am}}|_{#1} }}
\newcommand{\Famsliceobj}[2][]{\ensuremath{|\mathbb{F\mathrm{am}}_{#2}|_{#1} }}
\newcommand{\morph}{\rightarrow}
\newcommand{\epi}{\twoheadrightarrow}
\newcommand{\base}{\triangleleft}
\newcommand{\comp}{\circ}
\newcommand{\cross}{\otimes}
\newcommand{\pc}[2]{d^{#1}_{#2}}
\newcommand{\sub}{^*}
\newcommand{\diag}{\delta}
\newcommand{\pbase}[1]{\tilde{#1}}
\newcommand{\tuple}[1]{\langle#1\rangle}
\newcommand{\ndidly}{\ensuremath{\Join_n}}

\newcommand{\product}[1]{\bigtimes_{#1}}
\newcommand{\productn}{\product{n}}
\newcommand{\crossx}[3]{#1 \underset{#3}{\cross} #2}
\newcommand{\fibrex}[3]{#1 \underset{#3}{\Join} #2}
\newcommand{\powerset}{\mathcal{P}}
\newcommand{\primeds}[1]{
\ensuremath{\mathcal{P}(#1)} }
\newcommand{\compset}{\ \dot{\circ}\, }

% darrow
%\newcommand{\darrow}{\rightarrowtriangle} %use \smorph instead
\newcommand{\smorph}{\rightarrowtriangle}

 
\newcommand\dhead{\scaleobj{0.6}{\triangleright}}
%\newcommand{\dmorph}{\, \mbox{---} \! \cdot \! \raisebox{1.1pt}{\dhead}}    % dot style
\newcommand{\dmorph}{\, \mbox{---}\kern-1pt\raisebox{1.1pt}{\dhead\kern-1.75pt\dhead}}\,     % double triangle style

% projection tree
%\newcommand{\proj}[2]{proj_{#2}(#1)}

\newcommand{\proj}[2]{
\ensuremath{\mathcal{P}_{#2}(#1)} }

%pstrick supplements for arrows

\newlength{\arrnodesepA}
\newlength{\arrnodesepB}
\newlength{\arroffsetA}
\newlength{\arroffsetB}

%Modified to 2pt from 0pt on 23 July 2018
\newcommand{\arreset}{
\setlength{\arrnodesepA}{2pt}
\setlength{\arrnodesepB}{2pt}
\setlength{\arroffsetA}{0pt}
\setlength{\arroffsetB}{0pt}
}
\arreset

\newcommand{\ncarr}[3][0]{\ncarc[arcangle=#1,nodesepA=\arrnodesepA,nodesepB=\arrnodesepB,offsetA=\arroffsetA,offsetB=\arroffsetB,arrowsize=5pt,arrowinset=0.7]{->}{#2}{#3}}
\newcommand{\ncdarr}[3][0]{\ncarc[linestyle=dashed,arcangle=#1,nodesepA=\arrnodesepA,nodesepB=\arrnodesepB,offsetA=\arroffsetA,offsetB=\arroffsetB,arrowsize=5pt,arrowinset=0.7]{->}{#2}{#3}}
\newcommand{\jcbarr}[4][0]{ % ncbarr is defined in some thridy party package so do not use!\emph{}
\ncarr[#1]{#3}{#4}
\nbput[labelsep=2pt]{\footnotesize $#2$}
}

\newcommand{\ncaarr}[4][0]{
\ncarr[#1]{#3}{#4}
\naput[labelsep=2pt]{\footnotesize $#2$}
}

% \alabel{label}[npos][labelsep_pts]
\newcommandx*\alabel[3][2=0.5,3=2,usedefault]{\naput[labelsep=#3pt,npos=#2]{\footnotesize $#1$}}
% \blabel{label}[npos][labelsep_pts]
\newcommandx*\blabel[3][2=0.5,3=2,usedefault]{\nbput[labelsep=#3pt,npos=#2]{\footnotesize $#1$}}


\newif \ifbars
% to supress display of bars use \barsfalse to swith them on use \barstrue
\barstrue 
% \idcomp mark an arrow as one component of an identifier
\newcommand{\idcomp}{\ifbars{\ncput[npos=0, nrot=:U]{\psline(0.2,-0.075)(0.2,0.075)}}\fi}  %add a bar to a node connection arrow
% pstrick supplements for s-arrows (previous name for d-arrow - should convert}

\newlength{\sarnodesepA}
\newlength{\sarnodesepB}
\newlength{\saroffsetA}
\newlength{\saroffsetB}
\newlength{\sarnodesepAsav}
\newlength{\sarnodesepBsav}

\newcommand{\sarreset}{
\setlength{\sarnodesepA}{0pt}
\setlength{\sarnodesepB}{0pt}
\setlength{\saroffsetA}{0pt}
\setlength{\saroffsetB}{0pt}
}

\sarreset

% sar - S-arrow
\newcommand{\ncsar}[3][0]{
\setlength{\sarnodesepAsav}{\sarnodesepA}
\setlength{\sarnodesepBsav}{\sarnodesepB}
\addtolength{\sarnodesepA}{3pt}
\addtolength{\sarnodesepB}{7pt}
\ncarc[nodesepA=\sarnodesepA,nodesepB=\sarnodesepB,offsetA=\saroffsetA,offsetB=\saroffsetB,arcangle=#1]{-}{#2}{#3}
\ncput[nrot=:R,npos=1]{\pstriangle(0,0)(.2,.2)}
\setlength{\sarnodesepA}{\sarnodesepAsav}
\setlength{\sarnodesepB}{\sarnodesepBsav}
}


% bsar - below labelled S-arrow
\newcommand{\ncbsar}[4][0]{
\ncsar[#1]{#3}{#4}
\nbput[labelsep=2pt]{\footnotesize $#2$}
}
% asar - above labelled S-arrow
\newcommand{\ncasar}[4][0]{
\ncsar[#1]{#3}{#4}
\naput[labelsep=2pt]{\footnotesize $#2$}
}

% OLD cdar - composite dependency arrow - dot tyle
\iffalse
\newcommand{\nccdar}[3][0]{
\setlength{\sarnodesepAsav}{\sarnodesepA}
\setlength{\sarnodesepBsav}{\sarnodesepB}
\addtolength{\sarnodesepA}{3pt}
\addtolength{\sarnodesepB}{11pt}
\ncarc[nodesepA=\sarnodesepA,nodesepB=\sarnodesepB,offsetA=\saroffsetA,offsetB=\saroffsetB,arcangle=#1]{-}{#2}{#3}
\ncput[nrot=:R,npos=1]{\pstriangle(0,0.1)(.2,.2)}
\ncput[nrot=:R,npos=1]{\psdot[dotsize=1pt](-0.0075,0.05)}   %!!
\setlength{\sarnodesepA}{\sarnodesepAsav}
\setlength{\sarnodesepB}{\sarnodesepBsav}
}
\fi

% cdar - composite dependency arrow Mark II - double trangle style
\newcommand{\nccdar}[3][0]{
\setlength{\sarnodesepAsav}{\sarnodesepA}
\setlength{\sarnodesepBsav}{\sarnodesepB}
\addtolength{\sarnodesepA}{3pt}
\addtolength{\sarnodesepB}{13pt}
\ncarc[nodesepA=\sarnodesepA,nodesepB=\sarnodesepB,offsetA=\saroffsetA,offsetB=\saroffsetB,arcangle=#1]{-}{#2}{#3}
\ncput[nrot=:R,npos=1]{\pstriangle(0,0)(.2,.2)}
\ncput[nrot=:R,npos=1]{\pstriangle(0,0.2)(.2,.2)}
\setlength{\sarnodesepA}{\sarnodesepAsav}
\setlength{\sarnodesepB}{\sarnodesepBsav}
}


% bcdar - below labelled composite dependency arrow
\newcommand{\ncbcdar}[4][0]{
\nccdar[#1]{#3}{#4}
\nbput[labelsep=2pt]{\footnotesize $#2$}
}
% acdar - above labelled composite dependency arrow
\newcommand{\ncacdar}[4][0]{
\nccdar[#1]{#3}{#4}
\naput[labelsep=2pt]{\footnotesize $#2$}
}


% rsar - recursive S-arrow
\newcommand{\ncrsar}[2]{
\setlength{\sarnodesepAsav}{\sarnodesepA}
\setlength{\sarnodesepBsav}{\sarnodesepB}
\addtolength{\sarnodesepA}{3pt}
\addtolength{\sarnodesepB}{7pt}
\ncloop[nodesepA=\sarnodesepA,nodesepB=\sarnodesepB,
        offsetA=\saroffsetA,offsetB=\saroffsetB,
        armA=0.7cm,armB=0.6cm,angleA=90,angleB=-90,loopsize=-1,linearc=0.4
				]{-}{#1}{#2}
\ncput[nrot=:R,npos=5]{\pstriangle(0,0)(.2,.2)}
\setlength{\sarnodesepA}{\sarnodesepAsav}
\setlength{\sarnodesepB}{\sarnodesepBsav}
}

% pstrick supplements for multi-arrows

\newlength{\marnodesepA}
\newlength{\marnodesepB}
\newlength{\maroffsetB}
\newlength{\marnodesepBsav}

\newcommand{\marreset}{
\setlength{\marnodesepA}{0pt}
\setlength{\marnodesepB}{0pt}
\setlength{\maroffsetB}{0pt}
}

\marreset

%ncmarr[#1 arcangle1][#2 arcangle2]{#3 name}{#4 domain1}{#5 domain2}{#6 junction}{#7 codomain}
\newcommandtwoopt{\ncmarr}[6][8][8]{%
\ncarc[nodesepA=\marnodesepA,nodesepB=0,arcangle=#1]{-}{#3}{#5}
\ncarc[nodesepB=0,arcangle=-#1]{-}{#4}{#5}
\ncarc[arcangle=#2,nodesepB=\marnodesepB,offsetB=\maroffsetB]{->}{#5}{#6}
}%


\newcommandtwoopt{\nchmarr}[6][8][8]{%
\ncarc[nodesepA=\marnodesepA,nodesepB=0,arcangle=#1]{-}{#3}{#5}
\ncarc[nodesepB=0,arcangle=#1]{-}{#4}{#5}
\ncarc[arcangle=#2,nodesepB=\marnodesepB,offsetB=\maroffsetB]{->}{#5}{#6}
}%

\newcommandtwoopt{\ncamarr}[7][8][8]{%
\ncmarr[#1][#2]{#4}{#5}{#6}{#7}
\naput[npos=.05]{$#3$}
}%
\newcommandtwoopt{\ncbmarr}[7][8][8]{%
\ncmarr[#1][#2]{#4}{#5}{#6}{#7}
\nbput[npos=.05]{$#3$}
}%

\newcommandtwoopt{\ncbhmarr}[7][8][8]{%
\nchmarr[#1][#2]{#4}{#5}{#6}{#7}
\nbput[npos=.05]{$#3$}
}%

\newcommandtwoopt{\ncmarrr}[7][8][8]{
\ncarc[nodesepB=0,arcangle=#1]{-}{#3}{#6}
\ncline[nodesepB=0]{-}{#4}{#6}
\ncarc[nodesepB=0,arcangle=-#1]{-}{#5}{#6}
\ncarc[nodesepA=0,arcangle=#2]{->}{#6}{#7}
}

\newcommandtwoopt{\ncamarrr}[8][8][8]{
\ncmarrr[#1][#2]{#4}{#5}{#6}{#7}{#8}
\naput[npos=.05]{$#3$}
}
\newcommandtwoopt{\ncbmarrr}[8][8][8]{
\ncmarrr[#1][#2]{#4}{#5}{#6}{#7}{#8}
\nbput[npos=.05]{$#3$}
}


% 6 June 2020
% Edges representing attributes and relationship graphs
%  Ep   - partial
%  Epm  - partial mono
%  Epe  - partial epi
%  Epme - partial mono epi
%  Et   - total
%  Etm  - total mono
%  Ete  - total epi
%  Etme - total mono epi
%  recursive edges (use nccircle)
%  rEp   - partial
%  rEpm  - partial mono
%  rEpe  - partial epi
%  rEpme - partial mono epi
%  rEt   - total
%  rEtm  - total mono
%  rEte  - total epi
%  rEtme - total mono epi

\newcounter{EangleA}
\newcounter{EangleB}
\newcounter{EmidangleA}
\newcounter{EmidangleB}

% Ep - Edge partial
\newcommandtwoopt{\Ep}[4][0][0]{
\crowsfootedEdge{#1}{#2}{#3}{#4}{dashed}{dashed}
}



% Epm - Edge partial mono
\newcommandtwoopt{\Epm}[4][0][0]{
\monoEdge{#1}{#2}{#3}{#4}{dashed}{dashed}
}


% Epe - Edge partial epi
\newcommandtwoopt{\Epe}[4][0][0]{
\crowsfootedEdge{#1}{#2}{#3}{#4}{dashed}{solid}
}

% Epme - Edge partial mono epi
\newcommandtwoopt{\Epme}[4][0][0]{
\monoEdge{#1}{#2}{#3}{#4}{dashed}{solid}
}

% Et - Edge total
\newcommandtwoopt{\Et}[4][0][0]{
\crowsfootedEdge{#1}{#2}{#3}{#4}{solid}{dashed}
}

% Etm - Edge total mono
\newcommandtwoopt{\Etm}[4][0][0]{
\monoEdge{#1}{#2}{#3}{#4}{solid}{dashed}
}

% Ete - Edge total epi
\newcommandtwoopt{\Ete}[4][0][0]{
\crowsfootedEdge{#1}{#2}{#3}{#4}{solid}{solid}
}

% Etme - Edge total mono epi
\newcommandtwoopt{\Etme}[4][0][0]{
\monoEdge{#1}{#2}{#3}{#4}{solid}{solid}
}

% crowsfootedEdge - \crowsfootedEdge[angleA][midpointangle]{startnode}{endnode}[startstyle][endstyle]
\newcommand{\crowsfootedEdge}[6]{
\setlength{\sarnodesepAsav}{\sarnodesepA}
\setlength{\sarnodesepBsav}{\sarnodesepB}
\addtolength{\sarnodesepA}{3pt}
\addtolength{\sarnodesepB}{3pt}
\setcounter{EangleA}{ #1 + #2}
\setcounter{EangleB}{180  - #1 + #2}
\setcounter{EmidangleA}{#2}
\setcounter{EmidangleB}{#2 + 180}
\nccurve[nodesepA=\sarnodesepA,nodesepB=\sarnodesepB,offsetA=\saroffsetA,offsetB=\saroffsetB,angleA=\theEangleA, angleB=\theEangleB,linestyle=none,linewidth=0]{->}{#3}{#4}
\ncput[nrot=:R,npos=0]{\psline(0,.1)(.075,0)}
\ncput[nrot=:R,npos=0]{\psline(0,.1)(-0.075,0)}
\ncput{\pnode(0,0){xxx}}
\nccurve[nodesepA=0,nodesepB=\sarnodesepB,offsetA=0,offsetB=\saroffsetB,angleA=\theEmidangleA, angleB=\theEangleB, linestyle=#6]{->}{xxx}{#4}
%the following provides context for any following label
\nccurve[nodesepA=\sarnodesepA,nodesepB=0,offsetA=\saroffsetA,offsetB=0,angleA=\theEangleA, angleB=\theEmidangleB,linestyle=#5]{-}{#3}{xxx}
\setlength{\sarnodesepA}{\sarnodesepAsav}
\setlength{\sarnodesepB}{\sarnodesepBsav}
}

% monoEdge - \monoEdge[angleA][midpointangle]{startnode}{endnode}[startstyle][endstyle]
\newcommand{\monoEdge}[6]{ 
\setlength{\sarnodesepAsav}{\sarnodesepA}
\setlength{\sarnodesepBsav}{\sarnodesepB}
\addtolength{\sarnodesepA}{3pt}
\addtolength{\sarnodesepB}{3pt}
\setcounter{EangleA}{ #1 + #2}
\setcounter{EangleB}{180  - #1 + #2}
\setcounter{EmidangleA}{#2}
\setcounter{EmidangleB}{#2 + 180}
\nccurve[nodesepA=\sarnodesepA,nodesepB=\sarnodesepB,offsetA=\saroffsetA,offsetB=\saroffsetB,angleA=\theEangleA, angleB=\theEangleB,linestyle=none,linewidth=0]{->}{#3}{#4}
\ncput{\pnode(0,0){xxx}}
\nccurve[nodesepA=0,nodesepB=\sarnodesepB,offsetA=0,offsetB=\saroffsetB,angleA=\theEmidangleA, angleB=\theEangleB, linestyle=#6]{->}{xxx}{#4}
%the following provides context for any following label
\nccurve[nodesepA=\sarnodesepA,nodesepB=0,offsetA=\saroffsetA,offsetB=0,angleA=\theEangleA, angleB=\theEmidangleB,linestyle=#5]{-}{#3}{xxx}
\setlength{\sarnodesepA}{\sarnodesepAsav}
\setlength{\sarnodesepB}{\sarnodesepBsav}
}


\newcounter{EangleGiven}
\newcounter{EangleComplementary}
\newcounter{EangleStartCorrected}
\newcounter{EangleEndCorrected}


%  rEp   - recursive Edge partial
\newcommand{\rEp}[2][0]{
\setcounter{EangleGiven}{#1}
\setcounter{EangleStartCorrected}{#1-10} %correction required because for nccurve unlike nccircle angle measured at boundary not at centre of node
\setcounter{EangleEndCorrected}{#1+180+10} %correction required because angle measured at boundary not at centre of node
\setcounter{EangleComplementary}{#1 + 180}
\nccircle[angleA=\theEangleComplementary, nodesep=0pt, linestyle=none]{-}{#2}{.4cm} % an invisible circle to hang the midpoint from
\ncput{\pnode(0,0){midpoint}}                                         
\nccurve[nodesepA=1pt,nodesepB=0pt,offsetA=0pt,offsetB=0pt,angleA=\theEangleStartCorrected, angleB=\theEangleGiven, ncurv=1.359, linecolor=black, linestyle=dashed]{-}{#2}{midpoint}
\ncput[nrot=:R,npos=0]{\psline(0,.1)(.075,0)}
\ncput[nrot=:R,npos=0]{\psline(0,.1)(-0.075,0)}
\nccurve[nodesepA=0pt,nodesepB=2pt,offsetA=0pt,offsetB=0pt,angleA=\theEangleComplementary, angleB=\theEangleEndCorrected, ncurv=1.359, linestyle=dashed]{-}{midpoint}{#2}
% 1.359 is e/2 happenchance or algorithmically necessary???
% now draw arrowhead -- dont include in the nccurve because this alters the line position - a strange feature of pstruicks
\ncput[npos=0.9]{\pnode(0,0){yyy}}
\ncline{->}{yyy}{#2}
% repeat from earlier to provide context for label that might follow
\nccurve[nodesepA=1pt,nodesepB=0pt,offsetA=0pt,offsetB=0pt,angleA=\theEangleStartCorrected, angleB=\theEangleGiven, ncurv=1.359, linecolor=black, linestyle=dashed]{-}{#2}{midpoint} 
} 

%  rEpm  - recursive Edge partial mono
\newcommand{\rEpm}[2][0]{
\setcounter{EangleGiven}{#1}
\setcounter{EangleStartCorrected}{#1-10} %correction required because for nccurve unlike nccircle angle measured at boundary not at centre of node
\setcounter{EangleEndCorrected}{#1+180+10} %correction required because angle measured at boundary not at centre of node
\setcounter{EangleComplementary}{#1 + 180}
\nccircle[angleA=\theEangleComplementary, nodesep=0pt, linestyle=none]{-}{#2}{.4cm} % an invisible circle to hang the midpoint from
\ncput{\pnode(0,0){midpoint}}   
\nccurve[nodesepA=0pt,nodesepB=2pt,offsetA=0pt,offsetB=0pt,angleA=\theEangleComplementary, angleB=\theEangleEndCorrected, ncurv=1.359, linestyle=dashed]{-}{midpoint}{#2}
% 1.359 is e/2 happenchance or algorithmically necessary???
% now draw arrowhead -- dont include in the nccurve because this alters the line position - a strange feature of pstruicks
\ncput[npos=0.9]{\pnode(0,0){yyy}}
\ncline{->}{yyy}{#2}
% last to provide context for label that might follow
\nccurve[nodesepA=1pt,nodesepB=0pt,offsetA=0pt,offsetB=0pt,angleA=\theEangleStartCorrected, angleB=\theEangleGiven, ncurv=1.359, linecolor=black, linestyle=dashed]{-}{#2}{midpoint} 
}

%  rEpe  - recursive Edge partial epi
\newcommand{\rEpe}[2][0]{
\setcounter{EangleGiven}{#1}
\setcounter{EangleStartCorrected}{#1-10} %correction required because for nccurve unlike nccircle angle measured at boundary not at centre of node
\setcounter{EangleEndCorrected}{#1+180+10} %correction required because angle measured at boundary not at centre of node
\setcounter{EangleComplementary}{#1 + 180}
\nccircle[angleA=\theEangleComplementary, nodesep=0pt, linestyle=none]{-}{#2}{.4cm} % an invisible circle to hang the midpoint from
\ncput{\pnode(0,0){midpoint}}                                         
\nccurve[nodesepA=1pt,nodesepB=0pt,offsetA=0pt,offsetB=0pt,angleA=\theEangleStartCorrected, angleB=\theEangleGiven, ncurv=1.359, linecolor=black, linestyle=dashed]{-}{#2}{midpoint}
\ncput[nrot=:R,npos=0]{\psline(0,.1)(.075,0)}
\ncput[nrot=:R,npos=0]{\psline(0,.1)(-0.075,0)}
\nccurve[nodesepA=0pt,nodesepB=2pt,offsetA=0pt,offsetB=0pt,angleA=\theEangleComplementary, angleB=\theEangleEndCorrected, ncurv=1.359]{-}{midpoint}{#2}
% 1.359 is e/2 happenchance or algorithmically necessary???
% now draw arrowhead -- dont include in the nccurve because this alters the line position - a strange feature of pstruicks
\ncput[npos=0.9]{\pnode(0,0){yyy}}
\ncline{->}{yyy}{#2}
% repeat from earlier to provide context for label that might follow
\nccurve[nodesepA=1pt,nodesepB=0pt,offsetA=0pt,offsetB=0pt,angleA=\theEangleStartCorrected, angleB=\theEangleGiven, ncurv=1.359, linecolor=black, linestyle=dashed]{-}{#2}{midpoint} 
}

%  rEpme - recursive Edge partial mono epi
\newcommand{\rEpme}[2][0]{
\setcounter{EangleGiven}{#1}
\setcounter{EangleStartCorrected}{#1-10} %correction required because for nccurve unlike nccircle angle measured at boundary not at centre of node
\setcounter{EangleEndCorrected}{#1+180+10} %correction required because angle measured at boundary not at centre of node
\setcounter{EangleComplementary}{#1 + 180}
\nccircle[angleA=\theEangleComplementary, nodesep=0pt, linestyle=none]{-}{#2}{.4cm} % an invisible circle to hang the midpoint from
\ncput{\pnode(0,0){midpoint}}                                         
%\nccurve[nodesepA=0pt,nodesepB=0pt,offsetA=0pt,offsetB=0pt,angleA=\theEangleComplementary, angleB=\theEangleEndCorrected, ncurv=1.359, linestyle=dashed]{->}{xxx}{#2}
\nccurve[nodesepA=0pt,nodesepB=2pt,offsetA=0pt,offsetB=0pt,angleA=\theEangleComplementary, angleB=\theEangleEndCorrected, ncurv=1.359]{-}{midpoint}{#2}
% 1.359 is e/2 happenchance or algorithmically necessary???
% now draw arrowhead -- dont include in the nccurve because this alters the line position - a strange feature of pstruicks
\ncput[npos=0.9]{\pnode(0,0){yyy}}
\ncline{->}{yyy}{#2}
% last so that to provide context for label that might follow
\nccurve[nodesepA=1pt,nodesepB=0pt,offsetA=0pt,offsetB=0pt,angleA=\theEangleStartCorrected, angleB=\theEangleGiven, ncurv=1.359, linecolor=black, linestyle=dashed]{-}{#2}{midpoint} 
}

% rEt - recursive Edge total
\newcommand{\rEt}[2][0]{
\setcounter{EangleGiven}{#1}
\setcounter{EangleStartCorrected}{#1-10} %correction required because for nccurve unlike nccircle angle measured at boundary not at centre of node
\setcounter{EangleEndCorrected}{#1+180+10} %correction required because angle measured at boundary not at centre of node
\setcounter{EangleComplementary}{#1 + 180}
\nccircle[angleA=\theEangleComplementary, nodesep=0pt, linestyle=none]{-}{#2}{.4cm} % an invisible circle to hang the midpoint from
\ncput{\pnode(0,0){midpoint}}                                         
\nccurve[nodesepA=1pt,nodesepB=0pt,offsetA=0pt,offsetB=0pt,angleA=\theEangleStartCorrected, angleB=\theEangleGiven, ncurv=1.359, linecolor=black]{-}{#2}{midpoint}
\ncput[nrot=:R,npos=0]{\psline(0,.1)(.075,0)}
\ncput[nrot=:R,npos=0]{\psline(0,.1)(-0.075,0)}
%\nccurve[nodesepA=0pt,nodesepB=0pt,offsetA=0pt,offsetB=0pt,angleA=\theEangleComplementary, angleB=\theEangleEndCorrected, ncurv=1.359, linestyle=dashed]{->}{xxx}{#2}
\nccurve[nodesepA=0pt,nodesepB=2pt,offsetA=0pt,offsetB=0pt,angleA=\theEangleComplementary, angleB=\theEangleEndCorrected, ncurv=1.359, linestyle=dashed]{-}{midpoint}{#2}
% 1.359 is e/2 happenchance or algorithmically necessary???
% now draw arrowhead -- dont include in the nccurve because this alters the line position - a strange feature of pstruicks
\ncput[npos=0.9]{\pnode(0,0){yyy}}
\ncline{->}{yyy}{#2}
% repeat from earlier to provide context for label that might follow
\nccurve[nodesepA=1pt,nodesepB=0pt,offsetA=0pt,offsetB=0pt,angleA=\theEangleStartCorrected, angleB=\theEangleGiven, ncurv=1.359, linecolor=black]{-}{#2}{midpoint} 
}

%  rEtm  - recursive Edge total mono
\newcommand{\rEtm}[2][0]{
\setcounter{EangleGiven}{#1}
\setcounter{EangleStartCorrected}{#1-10} %correction required because for nccurve unlike nccircle angle measured at boundary not at centre of node
\setcounter{EangleEndCorrected}{#1+180+10} %correction required because angle measured at boundary not at centre of node
\setcounter{EangleComplementary}{#1 + 180}
\nccircle[angleA=\theEangleComplementary, nodesep=0pt, linestyle=none]{-}{#2}{.4cm} % an invisible circle to hang the midpoint from
\ncput{\pnode(0,0){midpoint}}     
\nccurve[nodesepA=0pt,nodesepB=2pt,offsetA=0pt,offsetB=0pt,angleA=\theEangleComplementary, angleB=\theEangleEndCorrected, ncurv=1.359, linestyle=dashed]{-}{midpoint}{#2}
% 1.359 is e/2 happenchance or algorithmically necessary???
% now draw arrowhead -- dont include in the nccurve because this alters the line position - a strange feature of pstruicks
\ncput[npos=0.9]{\pnode(0,0){yyy}}
\ncline{->}{yyy}{#2}
% last to provide context for label that might follow
\nccurve[nodesepA=1pt,nodesepB=0pt,offsetA=0pt,offsetB=0pt,angleA=\theEangleStartCorrected, angleB=\theEangleGiven, ncurv=1.359, linecolor=black]{-}{#2}{midpoint} 
}

%  rEte  - total epi
\newcommand{\rEte}[2][0]{
\setcounter{EangleGiven}{#1}
\setcounter{EangleStartCorrected}{#1-10} %correction required because for nccurve unlike nccircle angle measured at boundary not at centre of node
\setcounter{EangleEndCorrected}{#1+180+10} %correction required because angle measured at boundary not at centre of node
\setcounter{EangleComplementary}{#1 + 180}
\nccircle[angleA=\theEangleComplementary, nodesep=0pt, linestyle=none]{-}{#2}{.4cm} % an invisible circle to hang the midpoint from
\ncput{\pnode(0,0){midpoint}}                                         
\nccurve[nodesepA=1pt,nodesepB=0pt,offsetA=0pt,offsetB=0pt,angleA=\theEangleStartCorrected, angleB=\theEangleGiven, ncurv=1.359, linecolor=black]{-}{#2}{midpoint}
\ncput[nrot=:R,npos=0]{\psline(0,.1)(.075,0)}
\ncput[nrot=:R,npos=0]{\psline(0,.1)(-0.075,0)}
%\nccurve[nodesepA=0pt,nodesepB=0pt,offsetA=0pt,offsetB=0pt,angleA=\theEangleComplementary, angleB=\theEangleEndCorrected, ncurv=1.359, linestyle=dashed]{->}{xxx}{#2}
\nccurve[nodesepA=0pt,nodesepB=2pt,offsetA=0pt,offsetB=0pt,angleA=\theEangleComplementary, angleB=\theEangleEndCorrected, ncurv=1.359]{-}{midpoint}{#2}
% 1.359 is e/2 happenchance or algorithmically necessary???
% now draw arrowhead -- dont include in the nccurve because this alters the line position - a strange feature of pstruicks
\ncput[npos=0.9]{\pnode(0,0){yyy}}
\ncline{->}{yyy}{#2}
% repeat from earlier to provide context for label that might follow
\nccurve[nodesepA=1pt,nodesepB=0pt,offsetA=0pt,offsetB=0pt,angleA=\theEangleStartCorrected, angleB=\theEangleGiven, ncurv=1.359, linecolor=black]{-}{#2}{midpoint} 
}

%  rEtme - recursive Edge total mono epi

\newcommand{\rEtme}[2][0]{
\setcounter{EangleGiven}{#1}
\setcounter{EangleStartCorrected}{#1-10} %correction required because for nccurve unlike nccircle angle measured at boundary not at centre of node
\setcounter{EangleEndCorrected}{#1+180+10} %correction required because angle measured at boundary not at centre of node
\setcounter{EangleComplementary}{#1 + 180}
\nccircle[angleA=\theEangleComplementary, nodesep=0pt, linestyle=none]{-}{#2}{.4cm} % an invisible circle to hang the midpoint from
\ncput{\pnode(0,0){midpoint}}     
\nccurve[nodesepA=0pt,nodesepB=2pt,offsetA=0pt,offsetB=0pt,angleA=\theEangleComplementary, angleB=\theEangleEndCorrected, ncurv=1.359]{-}{midpoint}{#2}
% 1.359 is e/2 happenchance or algorithmically necessary???
% now draw arrowhead -- dont include in the nccurve because this alters the line position - a strange feature of pstruicks
\ncput[npos=0.9]{\pnode(0,0){yyy}}
\ncline{->}{yyy}{#2}
% last to provide context for label that might follow
\nccurve[nodesepA=1pt,nodesepB=0pt,offsetA=0pt,offsetB=0pt,angleA=\theEangleStartCorrected, angleB=\theEangleGiven, ncurv=1.359, linecolor=black]{-}{#2}{midpoint} 
}

%gats.macros.tex

\usepackage{environ}    % also used in ermacros % here used for \NewEnvrion

\newcommand{\gat}[1][U]{
\ensuremath{\mathcal{#1}}}  % used to hav a space in here
\newcommand{\gatw}[1][U]{\gat[#1]\ }  % use this if need trailing space
\newcommand{\ingat}[1][U]{in \gat[#1]}
\newcommand{\isagat}[1][U]{\gat[#1] is a g.a.t.}
\newcommand{\inagat}{in a g.a.t. }

% macro for a generic theory
%\newcommand{\theory}
%{\textit{U}}

\newcommand{\intheory}
{is a derived rule of \gat[U]}

% Macros for GAT rules

\newcommand{\isT}[1]
{#1\mbox{ is a type}}

\newcommand{\ofT}[2]
{#1 \in #2
}

% Macros for GAT rules   <!-- new old -->
\newcommand{\istype}[1]
{#1\mbox{ is a type}}

\newcommand{\oftype}[2]
{#1 \in #2
}

%\context{x}{\Delta}{n}
\newcommand{\context}[3]
{\ofT{#1_1}{#2_1},... \ofT{#1_{#3}}{#2_{#3}(#1_1,...#1_{#3-1})}
}

%\subcontext{x}{\Delta}{i}{k}
\newcommand{\subcontext}[4]
{\ofT{#1_{#3_1}}{#2_{#3_1}},... \ofT{#1_{#3_#4}}{#2_{#3_#4}(#1_1,...#1_{#3_#4-1})}
}

% #schematic context
\newcommand{\schmcon}[3]
{\ofT{#1_1}{#2_1},... \ofT{#1_{#3}}{#2_{#3}}
}
% abbreviated to
\newcommand{\con}[3]
{\schmcon{#1}{#2}{#3}}

% schematic subcontext
%\subcon{x}{\Delta}{i}{k}
\newcommand{\subcon}[4]
{\ofT{#1_{#3_1}}{#2_{#3_1}},... \ofT{#1_{#3_#4}}{#2_{#3_#4}}
}

% permuted context
%\permcon{x}{\Delta}{n}{\sigma}
\newcommand{\permcon}[4]
{\ofT{#1_{#4(1)}}{#2_{#4(1)}},... \ofT{#1_{#4(#3)}}{#2_{#4(#3)}}
}
% permuted term
%\permterm{t}{n}{\sigma}
\newcommand{\permterm}[3]
{
#1_{#3(1)},...#1_{#3(#2)}
}


% Idioms
\newcommand{\xDelta}[1]{\con{x}{\Delta}{#1}}
\newcommand{\xDeltap}[1]{\con{x}{\Delta'}{#1}}
\newcommand{\xOmega}[1]{\con{x}{\Omega}{#1}}
\newcommand{\xOmegap}[1]{\con{x}{\Omega'}{#1}}
\newcommand{\yOmega}[1]{\con{y}{\Omega}{#1}}
\newcommand{\yOmegap}[1]{\con{y}{\Omega'}{#1}}

\newcommand{\xDeltasigma}[1]{\permcon{x}{\Delta}{#1}{\sigma}}
\newcommand{\xDeltapsigma}[1]{\permcon{x}{\Delta'}{#1}{\sigma}}
\newcommand{\xOmegasigma}[1]{\permcon{x}{\Omega}{#1}{\sigma}}
\newcommand{\xOmegapsigma}[1]{\permcon{x}{\Omega'}{#1}{\sigma}}
\newcommand{\yOmegasigma}[1]{\permcon{y}{\Omega}{#1}{\sigma}}
\newcommand{\yOmegapsigma}[1]{\permcon{y}{\Omega'}{#1}{\sigma}}

\newcommand{\xDeltainvsigma}[1]{\permcon{x}{\Delta}{#1}{\sigma^{-1}}}
\newcommand{\xDeltapinvsigma}[1]{\permcon{x}{\Delta'}{#1}{\sigma^{-1}}}
\newcommand{\xOmegainvsigma}[1]{\permcon{x}{\Omega}{#1}{\sigma^{-1}}}
\newcommand{\xOmegapinvsigma}[1]{\permcon{x}{\Omega'}{#1}{\sigma^{-1}}}
\newcommand{\yOmegainvsigma}[1]{\permcon{y}{\Omega}{#1}{\sigma^{-1}}}
\newcommand{\yOmegapinvsigma}[1]{\permcon{y}{\Omega'}{#1}{\sigma^{-1}}}

%Idioms enclosed as tuples
\newcommand{\encxDelta}[1]{\tuple{\con{x}{\Delta}{#1}}}
\newcommand{\encxDeltap}[1]{\tuple{\con{x}{\Delta'}{#1}}}
\newcommand{\encxOmega}[1]{\tuple{\con{x}{\Omega}{#1}}}
\newcommand{\encxOmegap}[1]{\tuple{\con{x}{\Omega'}{#1}}}
\newcommand{\encyOmega}[1]{\tuple{\con{y}{\Omega}{#1}}}
\newcommand{\encyOmegap}[1]{\tuple{\con{y}{\Omega'}{#1}}}

\newcommand{\encxDeltasigma}[1]{\tuple{\permcon{x}{\Delta}{#1}{\sigma}}}
\newcommand{\encxDeltapsigma}[1]{\tuple{\permcon{x}{\Delta'}{#1}{\sigma}}}
\newcommand{\encxOmegasigma}[1]{\tuple{\permcon{x}{\Omega}{#1}{\sigma}}}
\newcommand{\encxOmegapsigma}[1]{\tuple{\permcon{x}{\Omega'}{#1}{\sigma}}}
\newcommand{\encyOmegasigma}[1]{\tuple{\permcon{y}{\Omega}{#1}{\sigma}}}
\newcommand{\encyOmegapsigma}[1]{\tuple{\permcon{y}{\Omega'}{#1}{\sigma}}}

\newcommand{\encxDeltainvsigma}[1]{\tuple{\permcon{x}{\Delta}{#1}{\sigma^{-1}}}}
\newcommand{\encxDeltapinvsigma}[1]{\tuple{\permcon{x}{\Delta'}{#1}{\sigma^{-1}}}}
\newcommand{\encxOmegainvsigma}[1]{\tuple{\permcon{x}{\Omega}{#1}{\sigma^{-1}}}}
\newcommand{\encxOmegapinvsigma}[1]{\tuple{\permcon{x}{\Omega'}{#1}{\sigma^{-1}}}}
\newcommand{\encyOmegainvsigma}[1]{\tuple{\permcon{y}{\Omega}{#1}{\sigma^{-1}}}}
\newcommand{\encyOmegapinvsigma}[1]{\tuple{\permcon{y}{\Omega'}{#1}{\sigma^{-1}}}}

\newcommand{\tstyle}{\vdash}
\newcommand{\gatdisplayrule}[3][]
{
\setlength{\fboxsep}{1pt}       
\setlength{\fboxrule}{0pt}
\fbox{$\displaystyle \frac{#2}{#3\rule[-0.3cm]{0cm}{0cm}}$#1}    %added vertival space using \rule
}
\newcommand{\genericAintroductoryrule} {\gatdisplayrule{\xDelta{n}}{\isT{A(\xn)}}}
\newcommand{\genericfintroductoryrule}  {\gatdisplayrule{\xDelta{n}}{\ofT{f(\xn)}{\Delta}}}

% gat macros developed for cwf paper

% Expressing gats
\newenvironment{gatrules}
{
$$
\begin{array}{l l}
}
{
\end{array}
$$
}
\newcommand{\gatintros}
{
\textbf{Symbol} & \textbf{Introductory\ Rule}                      \\}

\newcommand{\gataxioms}
{\textbf{Axioms}\\}
\newcommand{\gatintro}[3]{\ #1 & #2 \tstyle #3 \\}
\newcommand{\gatlocalintro}[3]{\ #1 & #2 \dashv }
\newcommand{\gataxiom}[2]{\multicolumn{2}{l}{\ \ #1\mbox{,  whenever\ } #2} \\}
\newcommand{\noleft}{\left.\kern-\nulldelimiterspace} % so that no space taken by absent left brace


\newcommand{\gatmultiaxiom}[2]
{\multicolumn{2}{l}{
  \noleft
    \begin{array}{l}
		#1
    \end{array} 
  \right\} \mbox{whenever\ } 	#2 
	}\\}
	
	\newcommand{\axid}[1]{\text{#1}.\ }	

%New context sharing macros
\newcommand{\gatintroducing}[1]{
{\arraycolsep=0pt
  \begin{array}{l}
          #1
  \end{array}} &
}

%*********************************
% \begin{\gatgroup}{context}
%    rules
%  \end{\gatgroup}
%*********************************
\NewEnviron{gatgroup}[1]{%
  \noleft
  {\arraycolsep=0pt
   \begin{array}{l}
\BODY
    \end{array} 
   }
   \ \right\} 
	%\mbox{\ whenever\ } 
	#1
	\vspace{0.1cm} 
}
%*********************************

%*********************************
% \begin{\gatgroupnoshared}
%    rule
%  \end{\gatgroupnoshared}
%*********************************
\NewEnviron{gatgroupnoshared}{%
  {\arraycolsep=0pt
   \begin{array}{l}
\BODY
    \end{array} 
   }
   \ 
	\vspace{0.1cm} 
}
%*********************************

% \gatsingular[width]{context}{conclusion}
\newcommand{\gatsingular}[3][4cm]{
\begin{gatgroupnoshared}
\gatleaf[#1]{#2}{#3} 
\end{gatgroupnoshared}
}

%*********************************
% \gatleaf}[width]{context}{assertion}
%*********************************
\newcommand{\gatleaf}[3][4cm]{%
\makebox[#1]{$#3$ \dotfill} \dotfill \  #2
}
%*********************************
%*********************************
% \gatstandalonesingle}{context}{assertion}
%*********************************
\newcommand{\gatstandalonesingle}[2]{%
#2 \makebox[2.5cm]{\dotfill} \  #1
}
%*********************************

% \gataxiomno{axiomno}
\newcommand{\gataxiomno}[1]{\makebox[0.5cm]{} \axid{#1}}


% metagat.macros.tex

%Meta-theories

%\newcommand{\typ}{\triangleright}
\newcommand{\typ}{\nabla}
\newcommand{\trm}{\tau}
\newcommand{\cross}{\otimes}
\newcommand{\sub}{^*}
\newcommand{\diag}{\delta}

\newcommand{\typeseq}[2]
{\ofT{#1_1}{\typ},... \ofT{#1_{#2}}{\typ(#1_{#2-1})}}

\newcommand{\typeseqcont}[3]
{\ofT{#1_1}{\typ({#2})},... \ofT{#1_{#3}}{\typ(#1_{#3-1})}}

\newcommand{\Ob}{Ob}
\newcommand{\obj}{Ob} % <!-- new old --<
\newcommand{\Hom}{Hom}
\newcommand{\objseq}[2]
{\ofT{#1_1}{\obj},... \ofT{#1_{#2}}{\obj(#1_{#2-1})}}


\def\dottededge{\ncline[linestyle=dotted, nodesep=0.3cm]}
\def\noedge{\ncline[linestyle=none]}
\def\thinedge{\ncline[linewidth=0.4pt]}

\newcommand{\member}[1]
{\ncarc[arcangle=-30,nodesepB=0.03]{->}{\pspred}{\pssucc}
\nbput[labelsep=0.1]{#1}}

\newcommand{\loweraccutemember}[1]
{\ncarc[arcangle=-15,nodesepB=0.03]{->}{\pspred}{\pssucc}
\nbput[labelsep=0.05,npos=0.85]{#1}}

\newcommand{\uppermember}[1]
{\ncarc[arcangle=30,nodesepB=0.03]{->}{\pspred}{\pssucc}\naput{#1}}

\newcommand{\upperaccutemember}[1]
{\ncarc[arcangle=10,nodesepB=0.03]{->}{\pspred}{\pssucc}\naput[npos=0.85]{#1}}

% flexbranch 
% #1 node label
% #2 thislevelsep
% #3 next level sep
% #4 variable (eg x)
% #5 index leter (eg n)
% #6 close parenthesis
% #7 continuation branches
\newcommand{\flexbranch}[7]
{
\pstree[thislevelsep=*#2,nodesep=0.05]
		{\Rnode{#1 1}{\Tr{#4_1 #6}}}
	  {\pstree[thislevelsep=#3]  
				   {\Rnode{#1 2}{\Tr[edge=\dottededge]{#4_{#5} #6}}}
					 {#7}
		}
}

\newcommand{\flexbranchplusleaf}[6]
{
\flexbranch{#1}{#2}{#3}{#4} {#5} {#6}
  {
   %\Rnode{#1 3}{\Tr{#4 #6}}
	 \Tr{\Rnode{#1 3}{#4 #6}}
  }
}

\newcommand{\flexbranchplusarc}[7]
{
\flexbranch{#1}{#2}{#3}{#4} {#5} {#6}
  {
   %\Rnode{#1 3}{\Tr{#4 #6}\member{#7}}
	 \Tr{\Rnode{#1 3}{#4 #6}}\member{#7}
  }
}

\newcommand{\flexbranchinitialarc}[9]
{
\pstree[thislevelsep=*#2,nodesep=0.05]
		{\Rnode{#1 1}{\Tr{#4_#8 #6}}#9}
	  {\pstree[thislevelsep=#3]  
				   {\Rnode{#1 2}{\Tr[edge=\dottededge]{#4_{#5} #6}}}
					 {#7}
		}
}

\newcommand{\equality}[2]
{
\ncline [doubleline=true, nodesep=0.2cm]{#1}{#2}
}
\newcommand{\equalityarc}[2]
{
\ncarc [arcangleA=-30, arcangleB=-20, doubleline=true, nodesep=0.1cm]{#1}{#2}
}

%The following are stylistic so belong in main document not here.
%\usepackage[margin=4.0cm]{geometry} % This shouldn't be here commented out 17 July 2018
%\usepackage{mathptmx}               % This changes font to roman so doesn't belong here
%
\usepackage{amsfonts}
\usepackage{amssymb} % added 08\02\2019 as an experiment. Needed in some instances for \blacksquare
                     % not needed is class is `beamer' but I don't know why not
\usepackage{array}
\usepackage{pstricks}
\usepackage{pst-tree}
\usepackage{pst-plot}
\usepackage{pst-node}
\usepackage{stmaryrd}
\usepackage{amsmath}
\usepackage{verbatim}
\usepackage{graphicx}  
\usepackage{calc}
\usepackage{xifthen}
%\usepackage{xcolor} investigate with beamer
\usepackage{color}
\usepackage{stringstrings}
%\usepackage[small,bf,margin=3pt,format=hang, labelsep=endash,singlelinecheck=false]{caption} %prevuiously justification=justified
%\usepackage{enumerate}
%\usepackage{enumitem}
\usepackage{enumerate}
%\usepackage[shortlabels]{enumitem} %Removed this 28/01/2019 because interfereing with a beamer presentation. 
\usepackage{float}
\usepackage[section]{placeins}
%\setlength{\captionmargin}{5pt}
\usepackage{environ}
\usepackage{multirow}
\usepackage{rotating}
\usepackage{longtable}
\usepackage{afterpage}
\usepackage{needspace}


%DEFINE ENVIRONMENT BLOCK
% Riddle
\newsavebox{\riddlebox}

\newenvironment{erexample}
{\newcommand\colboxcolor{F0F0F0}%was F8F8F8
\begin{lrbox}{\riddlebox}
\begin{minipage}{\dimexpr\columnwidth-2\fboxsep\relax} \textbf{} \\ \itshape}
{\end{minipage}\end{lrbox}%
%\begin{center}
\colorbox[HTML]{\colboxcolor}{\usebox{\riddlebox}}
%\end{center}
}

\newenvironment{erbox}
{\newcommand\colboxcolor{F0F0F0}%was F8F8F8
\begin{lrbox}{\riddlebox}%
\begin{minipage}{\dimexpr\columnwidth-2\fboxsep\relax} }
{\end{minipage}\end{lrbox}%
%\begin{center}
\colorbox[HTML]{\colboxcolor}{\usebox{\riddlebox}}
%\end{center}
}

%\begin{erboxedFigure}{#1 FigureParam}{#2 Label}{#3 Caption}
\NewEnviron{erboxedFigure}[3]{%
\begin{figure}[#1]
\begin{erexample}
\begin{center}
\BODY
\end{center}
\vspace{-0.5cm}
\caption{#3}
\label{#2}
\end{erexample}
\end{figure}
}

\newcommand{\erpictureFolder}[0]{../SharedPictures}

\newcommand{\ercenterPicture}[1]{
\begin{center}
\input{\erpictureFolder/#1}
\end{center}
}


\newlength{\erhalfHt}

%\erinlinePicture{#1 pictureFilename}{#2 pictureHeight}
\newcommand{\erinlinePicture}[2]{
\setlength{\erhalfHt}{#2cm * \real{0.5}}
\raisebox{-\erhalfHt}[\erhalfHt + 0.5cm][\erhalfHt + 0.5cm]{
\input{\erpictureFolder/#1}
} 
}

%\erplainFig{#1 pictureFilename}{#2 figureParam}{#3Caption}
\newcommand{\erplainFig}[3]{
\begin{figure}[#2]
\begin{center}
\input{\erpictureFolder/#1}
\end{center}
\caption{#3}
\label{#1}
\end{figure}
}

%\erboxedFigPicture{#1 pictureFilename}{#2 figureParam}{#3Caption}
\newcommand{\erboxedFigPicture}[3]{
\begin{figure}[#2]
\begin{erexample}
\vspace{-0.5cm}
\begin{center}
\input{\erpictureFolder/#1}
\end{center}
\caption{#3}
\label{#1}
\end{erexample}
\end{figure}
}

%\erLeftSideFig{#1 pictureFilename}{#2 figureParam}{#3Caption}
\newcommand{\erLeftSideFig}[3]{
\begin{figure}[#2]
\begin{erexample}
  \begin{minipage}[c]{0.4\textwidth}
    \caption{#3}
    \label{#1}
  \end{minipage}
  \begin{minipage}[c]{0.5\textwidth}
    \input{\erpictureFolder/#1}
  \end{minipage}
\end{erexample}
\end{figure}
}

%\erbulletedFig{#1 pictureFilename}{#2 figureParam}{#3Caption}
\NewEnviron{erbulletedFig}[3]{%
\begin{figure}[#2]
\begin{erexample}
\vspace{-0.5cm}
\begin{center}
$
\begin{array}{c m{0.25cm} | m{6cm}}
\raisebox{-2.0cm}{
\input{\erpictureFolder/#1}}& & \text{\parbox{6cm}{\raggedright{\footnotesize{
\begin{enumerate}[(i)]
\BODY
\end{enumerate}}}}} \\
\end{array}
$
\end{center}
\caption{#3}
\label{#1}
\end{erexample}
\end{figure} 
}


%\begin{erbulletedDimFig}{#1 pictureFilename}{#2figureParam} {#3Caption} {#4PictureHeight}{#5TextWidth}

\NewEnviron{erbulletedDimFig}[5]{%
\begin{figure}[#2]
\begin{erexample}
\vspace{-0.5cm}
\begin{center}
$
\begin{array}{c m{0.25cm} |  m{#5cm}}
\setlength{\erhalfHt}{#4cm * \real{0.5}}
\raisebox{-\erhalfHt}{
\input{\erpictureFolder/#1}}& & \text{\parbox{#5cm}{\raggedright{\footnotesize{
\begin{enumerate}[(i)]
\BODY
\end{enumerate}}}}} \\
\end{array}
$
\end{center}
\caption{#3}
\label{#1}
\end{erexample}
\end{figure} 
}

%\begin{ernotedModel}{#1 pictureFilename}{#2PictureHeight}{#3PictureWidth}{#4TextWidth}

\NewEnviron{ernotedModel}[4]{%
\begin{center}
$
\begin{array}{m{#3cm} m{1cm} | c m{#4cm}}
\setlength{\erhalfHt}{#2cm * \real{0.5}}
\raisebox{-\erhalfHt}{
\input{\erpictureFolder/#1}}& & & \text{\parbox{#4cm}{\raggedright{\footnotesize{
\BODY
}}}} \\
\end{array}
$
\end{center} 
}

%\begin{ermodelText}{#1 pictureFilename}{#2PictureHeight}{#3PictureWidth}{#4TextWidth}

\NewEnviron{ermodelText}[4]{%
\begin{center}
\begin{tabular}{m{#3cm} m{1cm}  c m{#4cm}}
\setlength{\erhalfHt}{#2cm * \real{0.5}}
\raisebox{-\erhalfHt}{
\input{\erpictureFolder/#1}}& & & \text{\parbox{#4cm}{\raggedright{\small{
\BODY
}}}} \\
\end{tabular}
\end{center} 
}


%\erbulletedModel{#1 pictureFilename}{#2PictureHeight}{#3PictureWidth}{#4TextWidth}

\NewEnviron{erbulletedModel}[4]{%
\begin{center}
$
\begin{array}{m{#3cm} m{1cm} | c m{#4cm}}
\setlength{\erhalfHt}{2cm * \real{0.5}}
\raisebox{-\erhalfHt}{
\input{\erpictureFolder/#1}}& & & \text{\parbox{#4cm}{\raggedright{\footnotesize{
\begin{enumerate}[(i)]
\BODY
\end{enumerate}}}}} \\
\end{array}
$
\end{center} 
}



%\ernotedDimFig{#1 pictureFilename}{#2 figureParam}{#3Caption}{#4PictureHeight}{#5TextWidth}
\NewEnviron{ernotedDimFig}[5]{%
\begin{figure}[#2]
\begin{erexample}
\vspace{-0.5cm}
\begin{center}
$
\begin{array}{c m{0.25cm} | c m{#5cm}}
\setlength{\erhalfHt}{#4cm * \real{0.5}}
\raisebox{-\erhalfHt}{
\input{\erpictureFolder/#1}}& & & \text{\parbox{#5cm}{\raggedright{\footnotesize{
\BODY }}}}\\
\end{array}
$
\end{center}
\caption{#3}
\label{#1}
\end{erexample}
\end{figure} 
}
%\begin{ernotedDimFigPW}{#1 pictureFilename}{#2 figureParam}{#3Caption}{#4PictureHeight}{#5PictureWidth}{#6TextWidth}
\NewEnviron{ernotedDimFigPW}[6]{%
\begin{figure}[#2]
\begin{erexample}
\vspace{-0.5cm}
\begin{center}
$
\begin{array}{>{\centering}m{#5cm} m{0.5cm} | c m{#6cm}}
\setlength{\erhalfHt}{#4cm * \real{0.5}}
\raisebox{-\erhalfHt}{
\centering \input{\erpictureFolder/#1}
}& & & \text{\parbox{#6cm - 0.5cm}{\raggedright{\footnotesize{
\BODY }}}}\\
\end{array}
$ \\
\vspace {0.2cm}
\end{center}
\caption{#3}
\label{#1}
\end{erexample}
\end{figure}
}



\newenvironment{erquote}
{\begin{quote}\itshape}
{\end{quote}}


%
%  erdiagram.tex
%  *************
%  Macros to represent ER diagrams
%  *******************************
% 29/01/2019 Modify so that not reliant on the
%            default fontsize being 10pt by using
%            package anyfontsize and then
%            \fontsize{8}{10}\selectfont to set font to 8pt
% 06/02/2019 Pullback symbol implemented and minor tweaks to positioning 
%            and size of identifier symbol and relationship labels.
%            Accidental forked changes merged on 08/02/2019.
% 15/03/2019 Continuation of 29/01/2019. Need fix fontsize of 
%            ERrelname and ERscope.	 
% ***********************************************************
 \usepackage{anyfontsize}             % 29/01/2019 
  
%\begin{erdiagram}{#1 height}{#2 width} 
% ....
% ....
%\end{erdiagram}
\newenvironment{erdiagram}[2]
{%\pspicture*(-#1,0)(#2,0)
\pspicture*(0,-#1)(#2,0)
%\psgrid
}
{\endpspicture}

\definecolor{lightyellow}{cmyk}{0,0,0.3,0}
\definecolor{verylightgrey}{gray}{0.95}


% \eret{#1 x0} {#2 y0} {#3 x1} {#4 y1} {#5 corner radius} {#6 fill}
\newcommand {\eret}[6]
{ 
\ifthenelse{\equal{#6}{1}}
{\psframe[framearc=#5,fillstyle=solid,fillcolor=lightyellow](#1,#2)(#3,#4)}
{\psframe[framearc=#5,fillstyle=solid,fillcolor=white](#1,#2)(#3,#4)}
}

% et top 
\newcommand {\erettop}[4]
{
%\psframe[linestyle=none,linearc=2pt,cornersize=absolute,fillstyle=solid,fillcolor=lightyellow](#1,#2)(#3,#4)
\psline[linearc=2pt,fillstyle=none,fillcolor=lightyellow](#1,#4)(#1,#2)(#3,#2)(#3,#4)
}

% et bottom 
\newcommand {\eretbtm}[4]
{
%\psframe[linestyle=none,linearc=2pt,cornersize=absolute,fillstyle=solid,fillcolor=lightyellow](#1,#2)(#3,#4)
\psline[linearc=2pt,fillstyle=none,fillcolor=lightyellow](#1,#2)(#1,#4)(#3,#4)(#3,#2)
}

% et bottom left
\newcommand {\eretbl}[4]
{
%\psframe[linestyle=none,linearc=2pt,cornersize=absolute,fillstyle=solid,fillcolor=lightyellow](#1,#2)(#3,#4)
\psline[linearc=2pt,fillstyle=none,fillcolor=lightyellow](#1,#4)(#3,#4)(#3,#2)
}

% et middle left
\newcommand {\eretml}[4]
{
%\psframe[linestyle=none,linearc=2pt,cornersize=absolute,fillstyle=solid,fillcolor=lightyellow](#1,#2)(#3,#4)
\psline[linearc=2pt,fillstyle=none,fillcolor=lightyellow](#1,#2)(#3,#2)(#3,#4)(#1,#4)
}

% et top left
\newcommand {\erettl}[4]
{
%\psframe[linestyle=none,linearc=2pt,cornersize=absolute,fillstyle=solid,fillcolor=lightyellow](#1,#2)(#3,#4)
\psline[linearc=2pt,fillstyle=none,fillcolor=lightyellow](#1,#2)(#3,#2)(#3,#4)
}

% et bottom right
\newcommand {\eretbr}[4]
{
%\psframe[linestyle=none,linearc=2pt,cornersize=absolute,fillstyle=solid,fillcolor=lightyellow](#1,#2)(#3,#4)
\psline[linearc=2pt,fillstyle=none,fillcolor=lightyellow](#1,#2)(#1,#4)(#3,#4)
}

% et middle right
\newcommand {\eretmr}[4]
{
%\psframe[linestyle=none,linearc=2pt,cornersize=absolute,fillstyle=solid,fillcolor=lightyellow](#1,#2)(#3,#4)
\psline[linearc=2pt,fillstyle=none,fillcolor=lightyellow](#3,#4)(#1,#4)(#1,#2)(#3,#2)
}

% et top right
\newcommand {\erettr}[4]
{
\psline[linearc=2pt,fillstyle=none,fillcolor=lightyellow](#1,#4)(#1,#2)(#3,#2)
}

% \ergrp{#1 x0} {#2 y0} {#3 x1} {#4 y1} {#5 corner radius} {#6 fill}
% #5 corner radius is unused!
\newcommand {\ergrp}[6]
{ 
\ifthenelse{\equal{#6}{1}}
{\psframe[fillstyle=solid,fillcolor=verylightgrey](#1,#2)(#3,#4)}
{\psframe[fillstyle=solid,fillcolor=white](#1,#2)(#3,#4)}
}


% \ertext{#1 text}
% 15/03/2019
\newcommand {\erextrasmallitalictext}[1]
{\fontsize{7}{9}\selectfont \textit{#1}}

% 29/01/2019  
\newcommand {\ersmallitalictext}[1]
{\fontsize{8}{10}\selectfont \textit{#1}}

\newcommand {\ermediumitalictext}[1]
{\fontsize{10}{12}\selectfont \textit{#1}}

% \eretname {#1 x left of text} {#2 y top of text} {#3 text}
\newcommand {\olderetname}[3]
{
%shift down 0.1 for height of text the anchor at baseline (B)
\rput[bl]{0}(0,-0.1){\rput[Bl]{0}(#1,#2){\ersmallitalictext{#3}}}
}

% \errelarm {#1 x0} {#2 y0} {#3 x1} {#4 y1} {#5 ismandatory} {#6 isconstructed}
\newcommand {\errelarm}[6]
{
\ifthenelse{\equal{#6}{1}}
{
%%\psline[linewidth=0.5pt,linearc=.05,linestyle=dashed,dash=6pt 6pt]{-}(#1,#2)(#3,#4)}
\ifthenelse{\equal{#5}{1}}
{\psline[linewidth=1.5pt,linearc=.05,linecolor=lightgray]{-}(#1,#2)(#3,#4)}
{\psline[linewidth=1.5pt,linearc=.05,linecolor=lightgray,linestyle=dashed,dash=2pt 2pt]{-}(#1,#2)(#3,#4)}
}
{
\ifthenelse{\equal{#5}{1}}
{\psline[linewidth=0.9pt,linearc=.05]{-}(#1,#2)(#3,#4)}
{\psline[linewidth=0.9pt,linearc=.05,linestyle=dashed,dash=2pt 2pt]{-}(#1,#2)(#3,#4)}
}
}

% \errelangle {#1 x0} {#2 y0} {#3 x1} {#4 y1} {#5 x2} {#6 y2} {#7 ismandatory} {#8 isocnstructed}
\newcommand {\errelangle}[8]
{
\ifthenelse{\equal{#8}{1}}
{
%\psline[linewidth=0.5pt,linearc=.1,linestyle=dashed,dash=6pt 6pt]{-}(#1,#2)(#3,#4)(#5,#6)}
\ifthenelse{\equal{#7}{1}}
{\psline[linewidth=1.5pt,linearc=.05,linecolor=lightgray]{-}(#1,#2)(#3,#4)(#5,#6)}
{\psline[linewidth=1.5pt,linearc=.1,linecolor=lightgray,linestyle=dashed,dash=2pt 2pt]{-}(#1,#2)(#3,#4)(#5,#6)}
}
{
\ifthenelse{\equal{#7}{1}}
{\psline[linewidth=0.9pt,linearc=.1]{-}(#1,#2)(#3,#4)(#5,#6)}
{\psline[linewidth=0.9pt,linearc=.1,linestyle=dashed,dash=2pt 2pt]{-}(#1,#2)(#3,#4)(#5,#6)}
}
}

% \ercrowfoot {#1 x0} {#2 y0} {#3 x11} {#4 y11} {#5 x12} {#6 y12} {#7 x13} {#8 y13} {#9 isconstructed}
\newcommand {\ercrowfoot}[9]
{
\ifthenelse{\equal{#9}{1}}
{
\psline[linewidth=1.5pt,linearc=.05,linecolor=lightgray]{-}(#1,#2)(#3,#4)
\psline[linewidth=1.5pt,linearc=.05,linecolor=lightgray]{-}(#1,#2)(#5,#6)
\psline[linewidth=1.5pt,linearc=.05,linecolor=lightgray]{-}(#1,#2)(#7,#8)
}{
\psline[linewidth=0.9pt,linearc=.05]{-}(#1,#2)(#3,#4)
\psline[linewidth=0.9pt,linearc=.05]{-}(#1,#2)(#5,#6)
\psline[linewidth=0.9pt,linearc=.05]{-}(#1,#2)(#7,#8)
}
}


% \eridcomprel{#1 x1}{#2 x2}{#3 y}
\newcommand {\eridcomprel}[3]
{
\psline[linewidth=0.9pt](#1,#3)(#2,#3)
}

% \eridrefrel{#1 x}{#2 y1}{#3 y2}
\newcommand {\eridrefrel}[3]
{
\psline[linewidth=0.9pt](#1,#2)(#1,#3)
}

% \ertext {#1 x} {#2 y} {#3 text anchor} {#4 text}  PRIVATE
\newcommand {\ertext}[4]
{
\rput[B#3]{0}(#1,#2){\fontsize{8}{10}\selectfont #4}
}

% \eretname {#1 x} {#2 y} {#3 text anchor} {#4 text} 
\newcommand {\eretname}[4]
{
\ertext{#1}{#2}{#3}{#4}
}

% \errelname {#1 x} {#2 y} {#3 text anchor} {#4 text} 
\newcommand {\errelname}[4]
{
\rput[B#3]{0}(#1,#2){\erextrasmallitalictext{#4}}
}


% \erscope {#1 x} {#2 y} {#3 text anchor} {#4 text}  15 March 2019
\newcommand {\erscope}[4]
{
\rput[B#3]{0}(#1,#2){\erextrasmallitalictext{#4}}
}

% \erreletname {#1 x} {#2 y} {#3 text anchor} {#4 text}  15 March 2019
\newcommand {\erreletname}[4]
{
\rput[B#3]{0}(#1,#2){\fontsize{10}{12}\selectfont #4}
}

% \ergroupannotation {#1 x} {#2 y} {#3 text anchor} {#4 text}
\newcommand {\ergroupannotation}[4]
{
\ertext{#1}{#2}{#3}{#4}
}


% \errelseq {#1 x} {#2 y}
\newcommand {\erelseq}[2]
{
}
\newcommand {\erattrmarkermand}
{\fontsize{6}{8}\selectfont $\blacksquare$}
\newcommand {\erattrmarkeropt}
{\fontsize{6}{8}\selectfont \CIRCLE}
\newcommand {\erderattrmarkermand}
{\fontsize{6}{8}\selectfont $\square$}
\newcommand {\erderattrmarkeropt}
{\fontsize{8}{10}\selectfont $\circ$}

% \erattr {#1 x} {#2 y} {#3 ismandatory}{#4 idenitfying} {#5 text}
\newcommand {\erattr}[5]
{
\ifthenelse{\equal{#3}{1}}
{\rput[l]{0}(#1,#2){\erattrmarkermand \ersmallitalictext{\ifthenelse{\equal{#4}{0}}{\underline{#5}}{#5}}}}
{\rput[l]{0}(#1,#2){\erattrmarkeropt \ersmallitalictext{\ifthenelse{\equal{#4}{0}}{\underline{#5}}{#5}}}}
}

\newcommand {\erdattr}[5]
{
\ifthenelse{\equal{#3}{1}}
{\rput[l]{0}(#1,#2){\erderattrmarkermand \ersmallitalictext{\ifthenelse{\equal{#4}{0}}{\underline{#5}}{#5}}}}
{\rput[l]{0}(#1,#2){\erderattrmarkeropt \ersmallitalictext{\ifthenelse{\equal{#4}{0}}{\underline{#5}}{#5}}}}
}


% \erarc {#1 x0} {#2 y0} {#3 x1} {#4 y1} {#5 x2} {#6 y2} {#7 x3} {#8 y3}
\newcommand {\erarc}[8]
{
\psbezier[showpoints=false]{-}(#1,#2) (#3, #4)(#5,#6) (#7, #8)
}

% \erarc {#1 x0} {#2 y0} {#3 x1} {#4 y1} {#5 x2} {#6 y2} {#7 x3} {#8 y3}
\newcommand {\errelseq}[8]
{
\psbezier[showpoints=false]{-}(#1,#2) (#3, #4)(#5,#6) (#7, #8)
}
% \ertrace {#1 trace}   
\newcommand {\ertrace}[1]
{
}
    %beamer awar{}e version
\usepackage{imakeidx}
\usepackage{framed}
\makeindex[name=definitions, title=Index of Definitions]
\makeindex[name=lemmas, title=Index of Lemmas]



\newcommand{\seenudgeup}[1]{\rule{0.1cm}{#1}}

\newcommand{\seenudgedown}[1]{\rule[-#1]{0.1cm}{0.1cm}}

\newcommand{\nudgeup}[1]{\rule{0cm}{#1}}

\newcommand{\nudgedown}[1]{\rule[-#1]{0cm}{0.1cm}}

\definecolor{highlight}{cmyk}{0,0,0.7,0}
\newcommand{\commentary}[1]{\marginpar{\footnotesize #1}}
\newcommand{\highlight}[1]{\colorbox{highlight}{#1}}
\newcommand{\whitelight}[1]{\colorbox{white}{#1}}
\newcommand{\term}[1]{\textit{#1}\commentary{\colorbox{lightgray}{\textit{#1}}}\index[definitions]{#1}}
\newcommand{\llabel}[1]{\label{#1}\commentary{\colorbox{pink}{\scriptsize{#1}}}\index[lemmas]{#1}}
\newcommand{\lref}[1]{\ref{#1}\colorbox{pink}{\scriptsize{#1}}\index[lemmas]{#1!use of}}

\newcommand{\daynote}[1]{\commentary{See day notes #1.}}

\newcommand{\newt}[1]{\colorbox{yellow}{#1}}
\newenvironment{newtt}
{  \colorbox{yellow}{$[$ ...} 
}
{  \colorbox{yellow}{... $]$}
}
\newcommand{\oldt}[1]{\colorbox{red}{\sout{#1}}}
\newenvironment{oldtt}
{  \colorbox{red}{$[$ ...} 
}
{  \colorbox{red}{... $]$}
}

\newcommand{\reinstatet}[1]{\colorbox{lime}{#1}}
\newenvironment{reinstatett}
{  \colorbox{lime}{$[$ ...}
}
{  \colorbox{lime}{... $]$}
}

\newcommand{\tbd}{\highlight{TBD}}

%ithprojection function
\newcommand{\proji}[1]{\pi_#1}


\newenvironment{aside}
{\begin{framed}
\textbf{Aside}
}
{
\end{framed}
}

\newenvironment{notebox}[1][Note]
{\begin{framed}
\textbf{#1}
}
{
\end{framed}
}

\newenvironment{categoricalaside}
{\begin{framed}
\textbf{Categorical Aside}
}
{
\end{framed}
}

\newenvironment{noteforfuture}
{\begin{framed}
\textbf{Note For Future}
}
{
\end{framed}
}

\newenvironment{problem}
{\begin{framed}
\textbf{Problem}
}
{
\end{framed}
}

\newenvironment{key}
{
\begin{tabular}{c l p{4cm}}
KEY && \\
\hline
}
{
\end{tabular}
}

\newcommand{\keyentry}[3]{#1 & #2 & #3 \\} 


%quine quote
\newcommand{\qq}[1]{
\left\ulcorner#1\right\urcorner
}

%single quote
\newcommand{\sq}[1]{
\textnormal{\textquotesingle}#1\textnormal{\textquotesingle}
}

%lower quine quote
\newcommand{\lqq}[1]{
\left\llcorner #1\right\lrcorner
}


%from berkley
\newcommand{\langl}{\begin{picture}(4.5,7)
\put(1.1,2.5){\rotatebox{60}{\line(1,0){5.5}}}
\put(1.1,2.5){\rotatebox{300}{\line(1,0){5.5}}}
\end{picture}}
\newcommand{\rangl}{\begin{picture}(4.5,7)
\put(.9,2.5){\rotatebox{120}{\line(1,0){5.5}}}
\put(.9,2.5){\rotatebox{240}{\line(1,0){5.5}}}
\end{picture}}
\newcommand{\lang}{\begin{picture}(5,7)\put(1.1,2.5){\rotatebox{45}{\line(1,0){6.0}}}\put(1.1,2.5){\rotatebox{315}{\line(1,0){6.0}}}\end{picture}}
\newcommand{\rang}{\begin{picture}(5,7)\put(.1,2.5){\rotatebox{135}{\line(1,0){6.0}}}\put(.1,2.5){\rotatebox{225}{\line(1,0){6.0}}}\end{picture}}
%Try sharper tuple brackets -- except gives errors nested in captions so comment out
%\renewcommand{\tuple}[1]{\lang #1 \rang}

\newcommand{\setsuchthat}[2]{\left\{#1 \ \middle|\ #2\right\}}
\newcommand{\set}[1]{\left\{#1\right\}} 

% one to n - wanton
\newcommand{\wanton}[1]{#1_1,...#1_n}
\newcommand{\n}{1...n}
\newcommand{\fn}{\wanton{f}}
\newcommand{\gn}{\wanton{g}}
\newcommand{\pn}{\wanton{p}}
\newcommand{\qn}{\wanton{q}}
\newcommand{\qnprime}{\wanton{q'}}
\newcommand{\tn}{\wanton{t}}
\newcommand{\xn}{\wanton{x}}
\newcommand{\xnp}{\wanton{x'}}
\newcommand{\yn}{\wanton{y}}
\newcommand{\An}{\wanton{A}}
\newcommand{\Bn}{\wanton{B}}
\newcommand{\Cn}{\wanton{C}}
\newcommand{\ntuple}[1]{\tuple{\wanton{#1}}}
\newcommand{\wantom}[2][]{#2_1,...#2_{m#1}}
\newcommand{\m}{1...m}
\newcommand{\mtuple}[1]{\tuple{#1_1,...#1_m}}
\newcommand{\gm}{\wantom{g}}
\newcommand{\qm}{\wantom{q}}
\newcommand{\sm}[1][]{\wantom[#1]{s}}
\newcommand{\smp}{\wantom{s'}}
\newcommand{\ym}{\wantom{y}}
\newcommand{\Bm}{\wantom{B}}
\newcommand {\bntuple}{\ensuremath{\ntuple{b}}}
\newcommand {\fntuple}{\ensuremath{\ntuple{f}}}
\newcommand {\fnptuple}{\ensuremath{\ntuple{f}}}
\newcommand {\pntuple}{\ensuremath{\ntuple{p}}}
\newcommand {\qntuple}{\ensuremath{\ntuple{q}}}
\newcommand {\qnptuple}{\ensuremath{\ntuple{q'}}}
\newcommand {\qmtuple}{\ensuremath{\mtuple{q}}}
\newcommand {\sntuple}{\ensuremath{\ntuple{s}}}
\newcommand {\xntuple}{\ensuremath{\ntuple{x}}}
\newcommand {\xnptuple}{\ensuremath{\ntuple{x'}}}
\newcommand {\ymtuple}{\ensuremath{\mtuple{y}}}
\newcommand{\idef}[1][n]{1 \leq i \leq #1}
\newcommand{\jdef}[1][m]{1 \leq j \leq #1}
\newcommand{\kdef}[1][l]{1 \leq k \leq #1}
\newcommand{\foreachi}[1][n]{for each $i$, $1 \leq i \leq #1$}
\newcommand{\foreachj}[1][m]{for each $j$, $1 \leq j \leq #1$}
\newcommand{\foreachk}[1][l]{for each $k$, $1 \leq k \leq #1$}
\newcommand{\Foreachi}[1][n]{For each $i$, $1 \leq i \leq #1$}
\newcommand{\Foreachj}[1][m]{For each $j$, $1 \leq j \leq #1$}
\newcommand{\Foreachk}[1][l]{For each $k$, $1 \leq k \leq #1$}
\newcommand{\forsomei}[1][n]{for some $i$, $1 \leq i \leq #1$}
\newcommand{\forsomej}[1][m]{for some $j$, $1 \leq j \leq #1$}
\newcommand{\forsomek}[1][l]{for some $k$, $1 \leq k \leq #1$}
\newcommand{\wherei}[1][n]{where $1 \leq i \leq #1$}
\newcommand{\wherej}[1][m]{where $1 \leq j \leq #1$}
\newcommand{\wherek}[1][l]{where $1 \leq k \leq #1$}


\newcommand{\fundep}[3]{#2 \xrightarrow{#1} #3}  %where does this belong? xxxx
% Following used for notes -- indented numbered paras

\newcounter{para}
\newlength{\oldparindent}
\setlength{\oldparindent}{\parindent} % Save \parindent before of change
\newcommand{\ind}{\hspace*{\oldparindent}}
\newcommand\note{
%\setlength{\parskip}{0.5\baselineskip} % Definition of `parskip`
\setlength{\parindent}{0pt}
\par\ind\refstepcounter{para}\thepara.\space
\setlength{\parindent}{\oldparindent}
}


         % beamer safe version

%%%%%%%%%%%%%%%%%%%%%%%%%%%%%%%%%
% alternate.beamer.macros.tex
%%%%%%%%%%%%%%%%%%%%%%%%%%%%%%%%%
% macros here for use with beamer
% nonBeamer.macros has alternate versions for use in papers
% I am confused why I reimplmented highlight macro

\newcommand{\waitfor}[2]{\onslide<#1->{#2}}   %enables different version of waitfor in standard documents
%
%  erdiagram.beamer.renewal.tex
%  ****************************
%
%  Renew macros defined in erdiagram so that 
%             core attributes
%             hierarchical attributes
%             relational attribuites 
%             annotations identifying relationships ali=ongside above attribute names
%  appear successively as the presentation advances. 
%
% Created 22 Sep 2022.
% core attributes catered for 11 Oct 2022
\newcommand{\coreAttributesSlideRange}{0}
\newcommand{\scopesSlideRange}{0}
\newcommand{\referenceImplementorAttributesSlideRange}{0}
\newcommand{\referenceImplementorAttributeswithRelIdAnnotationSlideRange}{0}
\newcommand{\dependencyImplementorAttributesSlideRange}{0}
\newcommand{\dependencyImplementorAttributeswithRelIdAnnotationSlideRange}{0}
\newcommand{\relIdSlideRange}{0}


% By default there is no animation
\newcommand{\erDisplayAllWithoutAnimation}
{
\renewcommand{\coreAttributesSlideRange}{1-}
\renewcommand{\scopesSlideRange}{1-}
\renewcommand{\referenceImplementorAttributesSlideRange}{0}
\renewcommand{\referenceImplementorAttributeswithRelIdAnnotationSlideRange}{1-}
\renewcommand{\dependencyImplementorAttributesSlideRange}{0}
\renewcommand{\dependencyImplementorAttributeswithRelIdAnnotationSlideRange}{1-}
\renewcommand{\relIdSlideRange}{1-}
}
\erDisplayAllWithoutAnimation %set the default

% Can reset to a five slide animation
% Slide 1 - shows logical model without scopes and without relationship ids
% Slide 2 - adds scopes -- these indicate equivalent paths i.e. commutative diagrams
% Slide 3 - reveals attributes required in a hierarchical implementation such as XML or Googles protocol buff IDL
% Silde 4 - reveals attributes required in a relational implementation
% slide 5 - annotates relationships with identifiers and tags attributes to
%            indicate the relationships which they are there to implement
\newcommand{\erDisplayFiveSlideAnimation}
{
\renewcommand{\coreAttributesSlideRange}{1-}
\renewcommand{\scopesSlideRange}{2}
\renewcommand{\referenceImplementorAttributesSlideRange}{3-4}
\renewcommand{\referenceImplementorAttributeswithRelIdAnnotationSlideRange}{5}
\renewcommand{\dependencyImplementorAttributesSlideRange}{4}
\renewcommand{\dependencyImplementorAttributeswithRelIdAnnotationSlideRange}{5}
\renewcommand{\relIdSlideRange}{5}
}

% Can reset to an early start five slide animation
% Slide 1 - shows model without attributes without scopes and without relationship ids
% Slide 2 - adds scopes -- these indicate equivalent paths i.e. commutative diagrams
% Slide 3 - adds core attributes
% Slide 4 - reveals attributes required in a hierarchical implementation such as XML or Googles protocol buff IDL
% Silde 5 - reveals attributes required in a relational implementation
\newcommand{\erDisplayEarlyStartFiveSlideAnimation}
{
\renewcommand{\coreAttributesSlideRange}{3-}
\renewcommand{\scopesSlideRange}{2-3}
\renewcommand{\referenceImplementorAttributesSlideRange}{4-}
\renewcommand{\referenceImplementorAttributeswithRelIdAnnotationSlideRange}{0}
\renewcommand{\dependencyImplementorAttributesSlideRange}{5-}
\renewcommand{\dependencyImplementorAttributeswithRelIdAnnotationSlideRange}{0}
\renewcommand{\relIdSlideRange}{0}
}

% Can reset to a two slide animation of the logical model
% Slide 1 - shows logical model without scopes and without relationship ids
% Slide 2 - adds scopes -- these indicate equivalent paths i.e. commutative diagrams
\newcommand{\erDisplayTwoSlideAnimation}
{
\renewcommand{\coreAttributesSlideRange}{1-}
\renewcommand{\scopesSlideRange}{2}
\renewcommand{\referenceImplementorAttributesSlideRange}{0}
\renewcommand{\referenceImplementorAttributeswithRelIdAnnotationSlideRange}{0}
\renewcommand{\dependencyImplementorAttributesSlideRange}{0}
\renewcommand{\dependencyImplementorAttributeswithRelIdAnnotationSlideRange}{0}
\renewcommand{\relIdSlideRange}{0}
}

% Can reset to a pure logical model 
% Slide 1 - shows logical model without scopes and without relationship ids
\newcommand{\erDisplayPureLogicalWithoutAnimation}
{

\renewcommand{\coreAttributesSlideRange}{1-}
\renewcommand{\scopesSlideRange}{0}
\renewcommand{\referenceImplementorAttributesSlideRange}{0}
\renewcommand{\referenceImplementorAttributeswithRelIdAnnotationSlideRange}{0}
\renewcommand{\dependencyImplementorAttributesSlideRange}{0}
\renewcommand{\dependencyImplementorAttributeswithRelIdAnnotationSlideRange}{0}
\renewcommand{\relIdSlideRange}{0}
}

% Can reset to a pure abstract model 
% Slide 1 - shows logical model without attributes without scopes and without relationship ids
\newcommand{\erDisplayPureLogicalWithoutAttributesWithoutAnimation}
{
\renewcommand{\coreAttributesSlideRange}{0}
\renewcommand{\scopesSlideRange}{0}
\renewcommand{\referenceImplementorAttributesSlideRange}{0}
\renewcommand{\referenceImplementorAttributeswithRelIdAnnotationSlideRange}{0}
\renewcommand{\dependencyImplementorAttributesSlideRange}{0}
\renewcommand{\dependencyImplementorAttributeswithRelIdAnnotationSlideRange}{0}
\renewcommand{\relIdSlideRange}{0}
}


\renewcommand{\erCoreAttribute}[5]{
\only<\coreAttributesSlideRange>
{
\erattr{#1}{#2}{#3}{#4}{#5}
}
}

%\erHierarchicalAttribute {#1 x} {#2 y} {#3 ismandatory}{#4 identifying} {#5 name} {#6 annotation}
% Attributes required in a hierarchical implementation
% These are reference implementor attributes
\renewcommand {\erHierarchicalAttribute}[6] 
{
\only<\referenceImplementorAttributesSlideRange>
          {\erRelorHierAttribute{#1}{#2}{#3}{#4}{#5}{}}
\only<\referenceImplementorAttributeswithRelIdAnnotationSlideRange>
          {\erRelorHierAttribute{#1}{#2}{#3}{#4}{#5}{#6}}
}

%\erRelationalAttribute {#1 x} {#2 y} {#3 ismandatory}{#4 identifying} {#5 name} {#6 annotation}
% Additional attributes required in a relational implementation
% These are depdendency implementor attributes
\renewcommand {\erRelationalAttribute}[6]   %These are dependency implementor attributes
{
\only<\dependencyImplementorAttributesSlideRange>
        {\erRelorHierAttribute{#1}{#2}{#3}{#4}{#5}{}}
\only<\dependencyImplementorAttributeswithRelIdAnnotationSlideRange>
        {\erRelorHierAttribute{#1}{#2}{#3}{#4}{#5}{#6}}
}

% \errelid {#1 x} {#2 y} {#3 text anchor} {#4 text} 
\renewcommand {\errelid}[4]
{
\only<\relIdSlideRange>{\errelidbody{#1}{#2}{#3}{#4}}    
}

% \erscope {#1 x} {#2 y} {#3 text anchor} {#4 text}  15 March 2019
\renewcommand {\erscope}[4]
{
  \only<\scopesSlideRange>{\erscopebody{#1}{#2}{#3}{#4}}
}


%%
% beamermacros
%

%re-impmentation of highlight for beamer
%\newcommand<>\highlightbox[2]{%
%  \alt#3{\makebox[\dimexpr\width-2\fboxsep]{\colorbox{#1}{#2}}}{#2}%
%}
%By copying above from internet
\definecolor{highlightcolor}{cmyk}{0,0,0.7,0}
\newcommand<>{\highlight}[1]{%
  \alt#2{\makebox[\dimexpr\width-2\fboxsep]{\colorbox{highlightcolor}{#1}}}{#1}%
}


  % this has a beamer aware highlight command but nor sure why I needed it
%%All these macros are copied from SharedMacros/general.tex which doesnt seem to work with beamer
% Some macros in SharedMacros/general.tex thought to have name clashes with beamer.


\newcommand{\fundep}[3]{#2 \xrightarrow{#1} #3}                                                 
\newcommand{\term}[1]{\textit{#1}} 
\newcommand{\setsuchthat}[2]{\left\{#1 \ \middle|\ #2\right\}}
\newcommand{\set}[1]{\left\{#1\right\}}



\newcommand{\wanton}[1]{#1_1,...#1_n}
\newcommand{\ntuple}[1]{\tuple{\wanton{#1}}}

\newcommand{\xntuple}{\ensuremath{\ntuple{x}}}

% maybe not in general.macros 
\newcommand{\xnset}{\ensuremath{\set{\wanton{x}}}}


%%
% othermacros
%

%copied from database literature review
\newcommand{\displaybibentry}[1]
{\begin{framed}
\bibentry{#1}
\end{framed}
}

% used in data tables
\newcommand{\colhead}[1]{\textbf{\textcolor{white}{#1}}}
\definecolor{myblue}{RGB}{71,71,186}
% \vpad gives vertical padding in a tabular
\newcommand{\vpad}[1]{\multicolumn{#1}{c}{}\\[-0.25cm]}

\newcommand{\catMEterm}{category with designated monomorphisms and epimorphisms\ }
\newcommand{\IfSforCwithRCwords}{
If $S$ is a sketch for category \catcw considered as a data specification with requirement $\reqtc$\ }
\newcommand{\IfSforCwithRCwordsvariant}{
If $S$ is a sketch for structured category \catcw and if $S$ is considered as a data specification with requirement $\reqtc$\ }
\newcommand{\IfSforepimonoCwithRCwords}{
If $S$ is a sketch for a category \catcw with designated monomorphisms and epimorphisms considered as a data specification with requirement $\reqtc$\ }
\iffalse
\newcommand{\scmonosketchwording}{
If $S$ is a sketch for such a category
%of a category with finite products and designated monomorphisms and epimorphisms
considered as a data specification
with requirement $\reqtc$\ }
\fi
\newcommand{\IfSforproductepimonoCwithRCwords}{
If $S$ is a sketch for such a category
%category \catcw with finite products and designated monomorphisms and epimorphisms 
considered as a data specification with requirement $\reqtc$\ }

\newcommand{\goodnesscriteria}[1]{\textbf{Goodness Criteria #1:}}


% From the Mathematical Theory of data paper
\newcommand{\ssfd}[2]{\ensuremath{#1 \morph #2}}  % singleton-singleton
\newcommand{\smfd}[2]{\ensuremath{\ssfd{#1}{\set{#2}}}}  % singleton-many
\newcommand{\msfd}[2]{\ensuremath{\ssfd{\set{#1}}{#2}}}  % many-singleton
\newcommand{\mmfd}[2]{\ensuremath{\msfd{#1}{\set{#2}}}}  % many-many



% All these should find a home in SharedMacros eventually 

% Commands for making a bit of vertical space. used when arrows and particularly labels of arrows
% use spec that is otherwise accounted for.
\newcommand{\seeroomup}[1]{\rule{0.1cm}{#1}}
\newcommand{\seeroomdown}[1]{\rule[-#1]{0.1cm}{0.1cm}}
\newcommand{\roomup}[1]{\rule{0cm}{#1}}
\newcommand{\roomdown}[1]{\rule[-#1]{0cm}{0.1cm}}



% DIAGRAMS START HERE
\newcommand{\nakedbinarysourcediagram}[5]{
\begin{array}{c p{0.5cm} c}
             &&   \Rnode{b}{#2}\\[0.01cm]
\Rnode{a}{#1} &&               \\[0.01cm] 
             &&   \Rnode{c}{#3}
\end{array} 
\begin{arrows}
\ncarr{a}{b}
\alabel{#4}
\ncarr{a}{c}
\blabel{#5}
\end{arrows}
}

\newcommand{\binarysourcediagram}[5]{$\nakedbinarysourcediagram{#1}{#2}{#3}{#4}{#5}$}
\newcommand{\fgsourcediagram}{\binarysourcediagram{a}{b}{c}{f}{g}}


\newcommand{\unaryfdrepresentationdiagram}[8]{
$
\begin{array}{c p{0.2cm} c}
\nakedbinarysourcediagram{#1}{#2}{#3}{#4}{#5}&& \Rnode{d}{#6}
\end{array}
\begin{arrows}
\ncarr{d}{b}
\idcomp
\blabel{#7}
\ncarr{d}{c}
\alabel{#8}
\end{arrows}
$
}

\newcommand{\unaryfdrepresentationmappeddiagram}[8]{
$
\begin{array}{c p{0.2cm} c}
\nakedbinarysourcediagram{D(#1)}{D(#2)}{D(#3)}{D(#4)}{D(#5)}&& \Rnode{d}{D(#6)}
\end{array}
\begin{arrows}
\ncarr{b}{d}
\alabel{D(#7)^-1}
\ncarr{d}{c}
\alabel{D(#8)}
\end{arrows}
$
}

\newcommand{\commutativetrianglediagram}[6]{
$
\begin{array}{c p{0.4cm} c p{0.4cm} c}
              && \Rnode{b}{#2}  &&                 \\[0.6cm]
\Rnode{a}{#1} &&                && \Rnode{c}{#3}  
\end{array}
\begin{arrows}
\ncarr{a}{b}
\alabel{#4}
\ncarr{b}{c}
\alabel{#5}
\ncarr{a}{c}
\blabel{#6}
\end{arrows}
$
}

\newcommand{\commutativetrianglediagrammutant}[6]{
$
\begin{array}{c  c  c}
              & \Rnode{b}{#2}  &                 \\[0.85cm]
\Rnode{a}{#1} &                & \Rnode{c}{#3}  
\end{array}
\begin{arrows}
\ncarr{a}{b}
\alabel{#4}[0.15]
\ncarr{b}{c}
\alabel{#5}[0.6]
\ncarr{a}{c}
\blabel{#6}
\end{arrows}
$
}

\newcommand{\epimonosplitdiagram}[3]{
\commutativetrianglediagram{#1}{img(#3)}{#2}{#3_e}{#3_m}{#3}   
}


\iffalse %saved
\begin{array}{c p{2.0cm} c }                
               &&  \Rnode{b1}{#3_1}    \\ [0.75cm]
               &&  \Rnode{b2}{#3_2}    \\ [0.5cm]
\Rnode{a}{#2}  &&                      \\ [-0.5cm]
               &&       \vdots         \\ [0.85cm]
               &&  \Rnode{bn}{#3_{#1}}  
\end{array}
\fi

%nakedmultisourceobjects{n}{a}{b}{f}
\newcommand{\nakedmultisourceobjects}[4]{
\begin{array}{c p{2.0cm} c }
\Rnode{a}{#2}   &&
\begin{array}{c }                
\Rnode{b1}{#3_1}   \\ [0.75cm]
\Rnode{b2}{#3_2}   \\ [0.25cm]
\vdots             \\ [0.35cm]
\Rnode{bn}{#3_{#1}}  
\end{array}
\end{array}
}

% \nakedmultisourcediagram{n}{a}{b}{f}
\newcommand{\nakedmultisourcediagram}[4]{
\nakedmultisourceobjects{#1}{#2}{#3}{#4}
\begin{arrows}
\ncarr{a}{b1}
\alabel{#4_1}[0.5]
\ncarr{a}{b2}
\alabel{#4_2}[0.5][-1]
\ncarr{a}{bn}
\blabel{#4_{#1}}[0.5][-1]
\end{arrows}
}

% \nakedmultisourcepathdiagram{n}{a}{b}{f}
\newcommand{\nakedmultisourcepathdiagram}[4]{
\nakedmultisourceobjects{#1}{#2}{#3}{#4}
\begin{arrows}
\simplepath{a}{b1}
\alabel{#4_1}[0.5]
\simplepath{a}{b2}
\alabel{#4_2}[0.5][-1]
\simplepath{a}{bn}
\blabel{#4_{#1}}[0.5][-1]
\end{arrows}
}


\newcommand{\multisourcediagram}[4]{$\nakedmultisourcediagram{#1}{#2}{#3}{#4}$}
\newcommand{\multisourcepathdiagram}[4]{$\nakedmultisourcepathdiagram{#1}{#2}{#3}{#4}$}



% \multisourcedefinitiondiagram{x}{g}{h}{n}{a}{b}{f}
\newcommand{\monosourcedefinitiondiagram}[7]{
$
\begin{array}{c p{1.5cm} c}
\Rnode{x}{#1} && \nakedmultisourcediagram{#4}{#5}{#6}{#7}
\end{array}
\begin{arrows}
\parallelarrows{x}{a}{#2}{#3}
\end{arrows}
$
}

\newcommand{\fghfactordiagram}[6]
{
\binarysourcediagram{#1}{#2\roomup{0.5cm}}{#3}{#4}{#5}
\begin{arrows}
\ncarr{b}{c}
\alabel{#6}
\end{arrows}
}

\newcommand{\fghpartialfactordiagram}[6]{
\binarysourcediagram{#1}{#2\roomup{0.5cm}}{#3}{#4}{#5}
\begin{arrows}
\ncdarr{b}{c} %dashed arrow
\alabel{#6}
\end{arrows}
}

\newcommand{\fnsourceqnsource}{
$
\begin{array}{c p{0.25cm} c  p{0.25cm} c }
             &&   \Rnode{b1}{b_1} &&              \\[0.4cm]
\Rnode{a}{a} &&                   && \Rnode{c}{c} \\[0.4cm]
             &&   \Rnode{bn}{b_n} &&              
\end{array} 
\begin{arrows}
\ncarr{a}{b1}
\alabel{f_1}
\ncarr{c}{b1}
\blabel{q_1} 
\ncarr{a}{bn}
\blabel{f_n}
\ncarr{c}{bn}
\alabel{q_n}
\end{arrows}
$   
}

\newcommand{\parallelarrows}[4]{
\ncarc[nodesep=2pt,arcangle=10,offset=2pt]{->}{#1}{#2}
\alabel{#3}
\ncarc[nodesep=2pt,arcangle=-10,offset=-2pt]{->}{#1}{#2}
\blabel{#4}
}

\newcommand{\fgparalleldiagram}{
 $
\rule[-0.3cm]{0pt}{0.9cm} %to add vertical space of diagram -- based on lowering diagram 0.3cm and heght 0.9cm
                            % change thickness from 0pt to 1 pt to debug
\begin{array}{c p{0.5cm} c  }
 \Rnode{a}{a}            &&   \Rnode{b}{b}
\end{array} 
\begin{arrows}
\iffalse
\ncarc[nodesep=2pt,arcangle=10,offset=2pt]{->}{a}{b}
\alabel{f}
\ncarc[nodesep=2pt,arcangle=-10,offset=-2pt]{->}{a}{b}
\blabel{g}
\fi
\parallelarrows{a}{b}{f}{g}
\end{arrows}
$  
}

\newcommand{\fgcomposablediagram}[5]{
\mbox{
\roomup{0.45cm}
$
\begin{array}{c p{0.5cm}cp{0.5cm}c}
\Rnode{x}{#1}&&\Rnode{y}{#2}&&\Rnode{z}{#3}
\end{array}
\begin{arrows}
\ncarr{x}{y}
\alabel{#4}
\ncarr{y}{z}
\alabel{#5}
\end{arrows}
$    
}
}


%
% copied from MToD paper (preamble.tex)
\newcommand{\simplepath}[2]{
\ncline[linestyle=none,linewidth=0.1pt]{#1}{#2}   %was linestyle=dotted
\ncput[npos=0.05]{\pnode{dot#21}}
\ncput[npos=0.27]{\dotnode[dotsize=1pt]{dot#22}}
\ncput[npos=0.50]{\dotnode[dotsize=1pt]{dot#23}}
\ncput[npos=0.80]{\dotnode[dotsize=1pt]{dot#24}}
\ncput[npos=0.975]{\pnode{dot#25}}
\ncline[nodesep=2pt]{->}{dot#21}{dot#22}
\ncline[nodesep=2pt]{->}{dot#22}{dot#23}
\ncline[nodesep=2pt]{->}{dot#24}{dot#25}
\ncline[linestyle=dotted,nodesep=8pt]{dot#23}{dot#24} %was 10pt
}
%


\newcommand{\stringtype}{text}
\newcommand{\numbertype}{number}

\newcommand{\dgsrcedge}
{
\setlength{\arroffsetA}{3pt}
\setlength{\arroffsetB}{3pt}
\ncarr[5]{edge}{node} 
\arreset  
}
\newcommand{\structuraldgsrcedge}
{
\setlength{\saroffsetA}{3pt}
\setlength{\saroffsetB}{3pt}
\ncsar[5]{edge}{node} 
\sarreset 
}
\newcommand{\dgtargetedge}
{
\setlength{\arroffsetA}{-3pt}
\setlength{\arroffsetB}{-3pt}
\ncarr[-5]{edge}{node} 
\arreset   
}
\newcommand{\dgbasic}
{
\begin{array}{c}
\Rnode{node}{node}  \\[2cm]
\Rnode{edge}{edge}       
\end{array}
\begin{arrows}
\dgsrcedge
\alabel{src}
\dgtargetedge
\blabel{trg}
\end{arrows}    
}

% dgbasiclabelled{labeltype}
\newcommand{\dgbasiclabelled}[1]
{
\begin{array}{cp{0.75cm}c}
%rule [raise-height]{width}{height}
\dgbasic   &&  \Rnode{labeltypelhs}{\rule[0cm]{0cm}{0.3cm}}\Rnode{labeltype}{#1} 
\end{array}
\begin{arrows}
\ncarr{node}{labeltypelhs}
\alabel{label}
\ncarr{edge}{labeltypelhs}
\blabel{label}
\end{arrows}       
}

\newcommand{\setoflabelleddgs}
{
\begin{array}{cp{1.0cm} : p{0.5cm}c}
\setofdg   &&&
\begin{array}{l}
\Rnode{text}{}\stringtype \\[1cm]
\Rnode{number}{}\numbertype 
\end{array}
\end{array}
\begin{arrows}
\ncarr{dg}{text}
\alabel{name}[0.3]
\ncarr{node}{number}
\alabel{label}[0.3]
\ncarr{edge}{number}
\blabel{label}[0.3]   
\end{arrows}
}


\newcommand{\structuraldgbasic}
{
\begin{array}{c}
\Rnode{node}{node}  \\[2cm]
\Rnode{edge}{edge}       
\end{array}
\begin{arrows}
\structuraldgsrcedge
\alabel{src}
\dgtargetedge
\blabel{trg}
\end{arrows}    
}

\newcommand{\nodepartof}
{
\ncarr{node}{dg}     
}
\newcommand{\structuralnodepartof}
{
\ncsar{node}{dg}     
}

\newcommand{\setofdg}
{
\begin{array}{c}
\rnode{dg}{dg} \\[2cm]
\dgbasic
\begin{arrows}
\nodepartof
\end{arrows}
\end{array}
}


\newcommand{\structuralsetofdg}
{
\begin{array}{c}
\rnode{dg}{dg} \\[2cm]
\structuraldgbasic
\begin{arrows}
\structuralnodepartof
\end{arrows}
\end{array}
}


\newcommand{\dgnumericallylabelleddetail}
{
\begin{array} {c}
\begin{array} {p{1.5cm} c}
     & \Rnode{abs}{\ \ 1\ \ }  
\end{array} \\[1.0cm]
\dgbasiclabelled{number}  
\end{array}
\begin{arrows}
%\setlength{\arroffsetA}{3pt}
\setlength{\arroffsetA}{-6pt}
\setlength{\arroffsetB}{-12pt}
\ncarr{abs}{labeltype}
\alabel{0}
\setlength{\arroffsetA}{3pt}
\setlength{\arroffsetB}{0pt}
\ncarr{abs}{labeltype}
\alabel{1}
\setlength{\arroffsetA}{9pt}
\setlength{\arroffsetB}{12pt}
\ncarr{abs}{labeltype}
\alabel{2 \hdots}
\end{arrows}
}



\newcommand{\dgabsuniquelylabelled}
{
\begin{array}{cp{0.75cm}c}
\dgbasic   &&  \Rnode{l}{l} 
\end{array}
\begin{arrows}
\ncarr{node}{l}
\alabel{label}
\idcomp
\ncarr{edge}{l}
\blabel{label}
\idcomp
\end{arrows}
}


\newcommand{\dglocallyuniquelylabelleddirectedgraph}
{
\begin{array}{cp{0.75cm}c}
\dgbasic   &&  \Rnode{l}{l} 
\end{array}
\begin{arrows}
\ncarr{node}{l}
\alabel{label}
\idcomp
\ncarr{edge}{l}
\blabel{label}
\idcomp
\dgsrcedge  % redrawn so that I can bar it with \idcomp
\idcomp
\end{arrows}
}


\newcommand{\setdgexitsuniquelylabelled}
{
\setoflabelleddgs
%
\begin{arrows}
%redraw arrows and bar using \idcomp
\ncarr{dg}{text}
\idcomp
\ncarr{node}{number}
\idcomp
\ncarr{edge}{number}
\idcomp
\dgsrcedge  % redrawn so that I can bar it with \idcomp
\idcomp
\nodepartof  % repeat so thatI can bar with \idcomp
\idcomp
\end{arrows}
}




\newcommand{\structuralsetofsgincludingabs}
{
\begin{array}{cp{0.75cm}c}
\Rnode{abs}{abs}                       \\[1cm]
\structuralsetofdg   &&  \Rnode{u}{u} 
\end{array}
\begin{arrows}
\ncsar{dg}{abs}
\ncarr{dg}{u}
\alabel{name}
\idcomp
\ncarr{node}{u}
\alabel{label}
\idcomp
\ncarr{edge}{u}
\blabel{label}
\idcomp
\structuraldgsrcedge  % redrawn so that I can bar it with \idcomp
\idcomp
\structuralnodepartof  % repeat so that I can bar with \idcomp
\idcomp
\end{arrows}
}






\renewcommand{\erpictureFolder}[0]{../../SharedPictures}
\setcounter{equation}{0}


%\usepackage{arydshln} % vertical dashed lines between columns of an array

\title[John Cartmell]{Concept-instance algebra}
%% Which is to say types as they are used in practice in software development and as represented in theory in categories and in syntactic type theories.
%% There is also a subplot concerning representation of context which certain types depend on -- again represented in practice and in theory. 
\subtitle{Retelling B-systems}
%subtitle{... and making the trivial trivially trivial}
\author{John Cartmell}
\institute{\\}
\date{Aug 15, 2022}
\bibliographystyle{plainnat}
\usepackage{framed}
\usepackage{bibentry}
\usepackage{colortbl}
\usepackage{ulem}   % for \dashuline{dashing} for seconday key
\usepackage{listings}
\lstset{%
  escapeinside={(*}{*)},%
}
\usepackage{arydshln} % vertical dashed lines between columns of an array
\usepackage{pst-arrow} %for \bigarrow
\usepackage{hhline}

%Redefine the \Fin macro to be category of sets
%\renewcommand{\Fin}{\Set

\newcommand{\CItreemode}{D}  % so that trees are top down
\newcommand{\CItreesep}{2cm}  % horizontal spearation of limbs of top down tree

\newcommand{\slidecontext}{Introduction} % to be redefined before use

\newcommand{\comingnext}[1]{
\begin{frame}{COMING NEXT}
\begin{center}
\Large #1
\end{center}
\end{frame}
}


\begin{document}
\begin{frame}
\titlepage
\nobibliography{../../SharedBibliography/temp/bibliography}
\end{frame}


\iffalse
\begin{frame}{test}
\begin{itemize}
\only<1->{\item Only 1-  One}
\only<2->{\item Only 2-  Two}
\onslide<3->{\item Onslide 3- Three}
\onslide<4->{\item Onslide 4- Four}
\end{itemize}
\end{frame}
\fi


\begin{frame}{Relational Normal Form Criteria}
Classic relational database normal form criteria 
\begin{itemize}
    \item from a mathematical perspective are not really normal forms!
    \item they are goodness criteria (GC) that articulate good engineering principles
    \item they include:
        \begin{center}
        \begin{tabular}{p{6.0cm}  l }
         \innerbullet first normal form (1NF)            &\Rnode{A1}{}                       \\
         \innerbullet 2nd normal form (2NF)              &                                   \\
         \innerbullet 3rd normal form (3NF)              &
                      \Rnode{A2}{}\braceLabel{A1}{A2}{Codd 1970,1971}                        \\
         \innerbullet Boyce-Codd normal form (BCNF)      &                                   \\
         \innerbullet 4th normal form (4NF)              & -- Fagin 1977                     \\
         \innerbullet projection-join normal form (5NF)  & -- Fagin 1979                     \\
         \innerbullet inclusion normal form (IN-NF)      & -- Ling and Goh  1992
        \end{tabular}
        \end{center}
\end{itemize}
\end{frame}

\begin{frame}{Introduction}
I wish to show that
\begin{itemize}
\item we can genericise relational database normal form criteria into abstract logical terms,
\item achieve goodness criteria that are generic i.e. can be applied to any data specifications not just to relational schema,
 \item that the classic relational database normal form criteria (2NF, 3NF, BCNF, INC-NF, 4NF, 5NF)  are  consequences of these generic goodness criteria.
\end{itemize}
\end{frame}

\begin{frame}{View}
A data specification  
\begin{itemize}
\item is a  theory (of what is)
\end{itemize}
\medskip
A data specification method 
\begin{itemize}
\item is a method for expressing such a  theory
\item unequivocally it enables definitions of types and certain relationships between these types
\item types are equally types of data and types of real world entity
\end{itemize}
More precisely, 
\begin{itemize}
\item data specification are \underline{presentations} of theories of what is,
\item choice of primitives in a given presentation is choice of which data to be stored or communicated.
\end{itemize}
\end{frame}

\begin{frame}{Goodness Criteria}
I will define two types of goodness criteria
\begin{itemize}
    \item Goodness Criteria of Type 1 -- absence of redundancy in presentation.
    \item Ensures absence of redundancy in data and in data management logic.
    \item Spelt out in more detail in criteria 1A, 1B and so on
    \item  Goodness Criteria of Type 2 -- the theory be the tightest fit to the facts 
    \item Two ways of expressing this. 
    \begin{itemize}
        \item Criteria 2 is that the theory is maximally constrained.
        \item Criteria 2A, 2B etc.  that it be logically complete in some sense.
    \end{itemize}
\end{itemize}
\end{frame}

\begin{frame}{Data Specification as Category}
\begin{itemize}
 \item Not surprising that a data specification can represented as a category or as a sketch of a category
\pause \item
What types of things are there and how are they related? 
\begin{itemize}
\item Data specifications provide the answer to this question in the context of a software development. 
\item Types theories provide the answer in a mathematical context. 
\item Category theory abstracts across both these domains.
\end{itemize}
\pause \item Nor is it surpising if data specifications can be seen in terms
of contextual categories or a sketches for a contextual categories
because as categories model types so contextual categories model types that vary
\pause ... and  our instinct for types and types that vary comes from our linguistic abilities as much as from our mathematics.
\end{itemize}
\end{frame}

\begin{frame}{Coming up}
\begin{center}
\begin{tabular}{p{12cm}}
\begin{itemize}
    \item data specifications 
    \begin{itemize}
        \item as sketches of structured categories of some kind
        \item data instances as certain structure preserving functors to the category of finite sets $\Fin$
    \end{itemize}
    \item database specifications
    \begin{itemize}
         \item category has designated mono sources for some of its objects 
    \end{itemize}
    \item relational database specifications charcterised by
    \begin{itemize}
         \item no use of containment or nesting
         \item no  hierarchical organisation -- said to be flat
         \item no use of pointers 
         \item instead uses foreign keys to represent relationships in the data
    \end{itemize}
\end{itemize} 
\end{tabular}
\end{center}
\end{frame}

\newcommand{\nestedcell}[3]{
\cline {2-2}
\rowcolor{#1} &\multicolumn{1}{|c|}{ \cellcolor{#2}#3}&\\
\cline {2-2}
\rowcolor{#1} & & \\
}

\newcommand{\databasesubtypes}{
\begin{tabular}{| p{0.05cm} c p{0.05cm}|}
\hline
\rowcolor{white}\multicolumn{3}{|l|}{\roomdown{0.25cm}database specific}\\
\nestedcell{white}{lightyellow}{relational(rdb)--SQL}
\nestedcell{white}{lightyellow}{nrdb}
\nestedcell{white}{lightyellow}{hierarchical--IMS}
\nestedcell{white}{lightyellow}{network--CODASYL}
\hline
\end{tabular}	
}

\newcommand{\innermessagingsubtypes}{
\begin{tabular}{p{0.05cm} c  p{0.05cm}}
\hline
\multicolumn{3}{l}{\roomdown{0.25cm}pointer free}\\
\nestedcell{lightyellow}{white}{XML}
\nestedcell{lightyellow}{white}{IDL-PB}
\hline
\end{tabular}	
}

\newcommand{\messagingsubtypes}{
\begin{tabular}{| p{0.05cm} c p{0.05cm}|}
\hline
\rowcolor{white}\multicolumn{3}{|l|}{\roomdown{0.25cm}message specific}\\
\nestedcell{white}{lightyellow}{\innermessagingsubtypes}
\nestedcell{white}{lightyellow}{IDL-CMU}
\hline
\end{tabular}	
}

\begin{frame}{Data specification methods}
\begin{tabular} {|p{0.05cm} c p{0.05cm} c p{0.05cm}|}
\hline
\rowcolor{lightyellow}\multicolumn{5}{|l|}{\roomdown{0.25cm}data specification methods} \\
\rowcolor{lightyellow} &\databasesubtypes && \messagingsubtypes &     \\
\rowcolor{lightyellow} &                  &&                    &     \\
\hline
\end{tabular}
\end{frame}

\newcommand{\bigdownarrow}
{\scalebox{0.3}{
\begin{pspicture}(3,3.5) 
%\psgrid
%\psset{doublesep=2cm} 
\psBigArrow[fillstyle=solid, fillcolor=blue!30,linecolor=blue](2.0,3)(2.0,0)
\end{pspicture}
}}
\newcommand{\biguparrow}
{\scalebox{0.3}{
\begin{pspicture}(3,3.5) 
%\psgrid
%\psset{doublesep=2cm} 
\psBigArrow[fillstyle=solid, fillcolor=red!30,linecolor=red](1.1,0)(1.1,3)
\end{pspicture}
}}
\begin{frame}{Logical and Technological Data Specifications}
\begin{center}
\begin{tabular}{c l}
\cline{1-1}
\multicolumn{1}{|c|}{logical} &  \raisebox{0cm}{\parbox{5cm}{sketch of category of some kind}} \\
\cline{1-1}
\multicolumn{1}{c}{\bigdownarrow} &  \\
\cline{1-1}
\multicolumn{1}{|c|}{technological} & \raisebox{0cm}{\parbox{5cm}{IDL, XML, SQL, etc.}}\\
\cline{1-1}
\end{tabular}
\end{center}
\begin{itemize}
    \item there is a logical data specification i.e. a sketch of a category behind every 
    technological data specification
    \item the generic goodness criteria are defined at the logical level
    \item I will show that a relational data specification passes the BCNF test
    iff its logical specification satisfies certain generic goodness criteria
    \item logical specifications are technology neutral. A single logical data specification can be carried forward into multiple technological specifications i.e. one for each technology or language of interest.
\end{itemize}
\end{frame}

\newcommand{\fourlevelstabular}
{
\begin{tabular}{c}
\cline{1-1}
\multicolumn{1}{|c|}{logical}     \\
\cline{1-1}
\multicolumn{1}{c}{\bigdownarrow} \\
\cline{1-1}
\multicolumn{1}{|c|}{structural}  \\
\cline{1-1} \\[-0.3cm]
\multicolumn{1}{c}{\bigdownarrow} \\
\cline{1-1}
\multicolumn{1}{|c|}{representational} \\
\cline{1-1} \\[-0.3cm]
\multicolumn{1}{c}{\bigdownarrow}   \\
\cline{1-1}
\multicolumn{1}{|c|}{technological} \\
\cline{1-1}
\end{tabular}   
}

\newcommand{\fourlevelstexttabular}
{
\begin{tabular}{p{6cm}}
\onslide<1->\\[-0.2cm]
\onslide<1->sketch of category of some kind \\[0.5cm]
\onslide<2->sketch as above with morphisms distinguished to indicate implementation by containment relationship  \\[0.1cm]
\onslide<3-> sketch with distinguised morphisms plus  foreign keys for some non-distinguished morphisms \\[0.3cm]
\onslide<1->IDL, XML, SQL etc\\
all relationships implemented by pointers, foreign keys or containment
\end{tabular}   
}

\begin{frame}{Different Levels of Data Specification}{different amounts of detail}
\begin{center}
\begin{tabular}{c c}
\fourlevelstabular&\pause\fourlevelstexttabular 
\end{tabular}
\end{center}
\end{frame}

\iffalse
\begin{frame}{Different Kinds of Data Specification}
\begin{center}
\begin{tabular}{c l}
\cline{1-1}
\multicolumn{1}{|c|}{logical} &  \raisebox{0cm}{\parbox{5cm}{sketch of category of some kind}} \\
\cline{1-1}
\\[-0.1cm]
\multicolumn{1}{c}{\bigdownarrow} & \raisebox{0.5cm}{\parbox{6cm}{distinguish morphisms/relationships represented in data by containment}} \\
\cline{1-1}
\multicolumn{1}{|c|}{structural} &\\
\cline{1-1}
\multicolumn{1}{c}{\bigdownarrow} & \raisebox{0.5cm}{\parbox{6cm}{add edges for foreign keys representing non-containment relationships and add path equivalences that define them}} \\
\cline{1-1}
\multicolumn{1}{|c|}{representational} & \\
\cline{1-1}
\\[-0.1cm]
\multicolumn{1}{c}{\bigdownarrow} & \raisebox{0.5cm}{choice of technology} \\
\cline{1-1}
\multicolumn{1}{|c|}{technological} & \raisebox{0cm}{\parbox{5cm}{IDL, XML, SQL}}\\
\cline{1-1}
\end{tabular}
\end{center}
\end{frame}
\fi

\iffalse
\begin{frame}{Methods of Data Specification}
\begin{itemize}
	\item schema of relational database,
	\item structure described by Carnegie-Mellon IDL,
	\item schema of nested relational database,
	\item message structure described by Google protocol buffer IDL,
	\item XML schema language,
	\item ER script.
\end{itemize}
\end{frame}
\fi

\begin{frame}{Data Specifications}
Two kinds of types in play
\begin{itemize}
\item  the \textit{definienda} -- types all of whose instances are \term{particulars}
\begin{itemize}
\item employee, department, student, account, product, order, shipment, delivery, flight, booking and so on
\item molecular structure, atom, bond, element, isotope, reaction, metabolite, mass trace, chromatogram, peak
\item table, column, primary key, foreign key
\item node and edge. 
\end{itemize}
\pause 
\item  the \textit{definiens}  -- types all of whose instances are \term{universals}
\begin{itemize}
       \item string, integer, float, boolean and so on
\end{itemize}
\end{itemize}
\pause
\begin{itemize}
\item in ER modelling 
\begin{itemize}
\item the \textit{definienda} are called \textit{entity types}
\item the \textit{definiens} are called \textit{attribute types} or \textit{domains}.
\end{itemize}
\end{itemize}
\end{frame}


\begin{frame}{Data Specifications}
A data specification is a sketch of a category with some additional structure:
\begin{itemize}
\item that it is a \term{sketch} is crucial because it is only nodes and edges of the sketch for which data is stored and/or communicated, 
\item that there are commutative diagrams is crucial to construction of representational 
specifications from logical specifications.
\item that the category had additional structure is significant:
\begin{itemize}
\item so that we can distinguish structural from non-structural relationships to describe structure nesting and thereby hierarchical data,
\item so that we can give account of database normal forms 
(BCNF, 3NF, 4NF and 5NF),
\item so that we can allow for missing data as represented by NULL values, 
\item so that types of universals can be distinguished from types of particulars.
\end{itemize}
\end{itemize}
\end{frame}

\begin{frame}{Additional Structure}
\resizebox{11.3cm}{!}{
\newcommand{\featurelist}{\begin{tabular}{|l|l l|}
\hline 
\multirow{11}{1.5cm}{category}
                & finitary property        & \\
\cline{2-3}
                & pu-partition             & \\
\cline{2-3}
                & \multirow{2}{3.5cm}{mono-sources}  & \multicolumn{1}{|l|}{cannonical monos}  \\
\cline{3-3}
                &                                    & \multicolumn{1}{|l|}{epi-mono factorisation}   \\
\cline{2-3}
                & finite products          & \\
\cline{2-3} 
                & finite limits            & \\
\cline{2-3}
                & restrictions             & \\
\cline{2-3}
                & \multirow{2}{3.5cm}{distinguished morphisms} & \multicolumn{1}{|l|}{hierarchical}      \\
\cline{3-3}
                &                                             &  \multicolumn{1}{|l|}{non-hierarchical} \\
\cline{3-3}
                &                                             &  \multicolumn{1}{|l|}{pullbacks} \\
\cline{2-3}
                & finite coproducts                           &                                   \\
\hline                
\end{tabular}}
\featurelist
}
\end{frame}

\begin{frame}{Definitions} 
In a category \catc, a  \term{source} is a family of morphisms with common domain: \\
\scalebox{0.65}{
\multisourcediagram{n}{a}{b}{f}
} 
\medskip
Such a source is said to be a \term{mono source}  iff for all $g,h:x \morph a$ in \catcw 
so that \scalebox{0.65}{
\monosourcedefinitiondiagram{x}{g}{h}{n}{a}{b}{f}
} 
in \catcw then if $g \circ f_i = h \circ f_i$, for each $i$,  then $g=h$.
\end{frame}

\begin{frame}{Alternative Definition}{Mono Source Limit Cone}  
In a category \cat{C},
\scalebox{0.65}{


$
\begin{array}{c p{2.0cm} c p{2.0cm} c}				
                   &&	 \Rnode{B1}{B_1}  \\ [0.75cm]
									 &&  \Rnode{B2}{B_2}  \\ [0.5cm]
		\Rnode{A}{A}  &&                    \\ [-0.5cm]
				           &&       \vdots      \\ [0.85cm]
                   &&	 \Rnode{Bn}{B_n}  
\end{array}
$
%\setlength{\arrnodesepA}{7pt}
%\setlength{\arrnodesepB}{8pt}
%\setlength{\arroffsetA}{2pt}
%\setlength{\arroffsetB}{0pt}
\begin{arrows}
\ncarr{A}{B1}
\alabel{f_1}[0.5]
\ncarr{A}{B2}
\alabel{f_2}[0.5][-1]
%\blabel{\vdots}[0.4][-2]  % move up 5pts -- dont know why I need this to get position for vdots
\ncarr{A}{Bn}
\blabel{f_n}[0.5][-1]
\end{arrows}


} is a mono source iff \\
\begin{center}
\scalebox{0.65}{
$
\begin{array}{c p{2.0cm} c p{2.0cm} c}				
                           &&	\Rnode{At}{A}  &&          \Rnode{B1}{B_1}  \\ [0.65cm]
													 &&                &&          \Rnode{B2}{B_2}  \\ [0.5cm]
		\Rnode{Al}{A}          &&                &&                           \\ [0cm]
				                   &&                &&           \vdots      \\ [0.85cm]
                           &&	\Rnode{Ab}{A}  &&          \Rnode{Bn}{B_n}  
\end{array}
$
%\setlength{\arrnodesepA}{7pt}
%\setlength{\arrnodesepB}{8pt}
%\setlength{\arroffsetA}{2pt}
%\setlength{\arroffsetB}{0pt}
\ncarr{Al}{At}
\alabel{id_A}
\ncarr{Al}{Ab}
\blabel{id_A}
\ncarr{At}{B1}
\alabel{f_1}[0.5]
\ncarr{At}{B2}
\alabel{f_2}[0.4][-1]
%\blabel{\vdots}[0.4][-2]  % move up 5pts -- dont know why I need this to get position for vdots
\ncarr{At}{Bn}
\blabel{f_n}[0.3][-2]
\ncarr{Ab}{B1}
\alabel{f_1}[0.3][-1]
\ncarr{Ab}{B2}
\blabel{f_2}[0.3][-1]
\ncarr{Ab}{Bn}
\blabel{f_n}[0.4]
%\alabel{\vdots}[0.4]

} 
is a limit cone.
\end{center}
\end{frame}

\begin{frame}{From here..}
\begin{itemize}
\item I will
\begin{itemize}
   \item give an example of nested tables of data
   \item describe relational model  and other data models 
   %\pause \item touch on my favourite -- ER modelling and ER script
   \pause \item define GCs for sketches of categories as data specifications 
   \pause \item define GCs for sketches of categories with designated monomorphisms and epimorphisms
   \pause \item define an abstract version of the Boyce-Codd normal form (BCNF) criteria 
   \pause \item define GCs for sketches of categories with designated monos and epis and with finite products
  \pause \item  show that if the generic goodness criteria hold then with some additional assumptions we can prove that the BCNF criteria holds
\end{itemize}
\end{itemize}
\end{frame}



\iffalse %DON'T USE THESE SLIDES

\iffalse
\begin{frame}{Concept, Context, Instance (how to think)}
All that there is...
\begin{itemize}
\item ...\textbf{is} in context.
\item ... \textbf{is} concept and instance of concept.
\end{itemize}
Only...
\begin{itemize}
\item ... concept provides context.
\end{itemize}
There is...
\begin{itemize}
\item a root concept that represents the absolute or the whole.
\end{itemize}
and to every concept $x$ there are...
\begin{itemize}
\item ...other concepts and their instances for which $x$ provides context.
\end{itemize}
\iffalse{
By the way...
\begin{itemize}
\item a concept whose context is provided by the absolute can be said to be an absolute concept.  
All of its instances can also be said to be absolute.
\end{itemize}
}\fi
\end{frame}

\begin{frame}{Concepts with a dependency on context}
\begin{center}
\raisebox{-0.5cm}{
\pspicture(0,-0.1)(1.1,1)
\psline(0,0)(0,1)(1,1)(1,0)(0,0)
\psline (0,0)(1,1)
%\psline(0,0.5)(1,0.5)
%\psline(0.5,0)(0.5,1)
\endpspicture
}
\end{center}
\begin{itemize}
\item \textit{"there are two triangles and these have six sides"}, 
\item  we are understanding \textit{side} to be a concept that 
depends on \textit{triangle} for context,
%\item a side, therefore, is a dependent type of thing
% -- it is some thing to be held in the mind
%in the context of some other thing,
\item write $$triangle \base side\ \ \ \ \ \ \ $$
\end{itemize}
\end{frame}

\begin{frame}{Opposite Side}

\textit{"in the context of an angle of a triangle, the opposite side is a side of the same triangle"}

\medskip
Write 
$$oppositeSide \in inst(angle \cross side)$$
because
\begin{itemize}
\item $angle \cross side$ represents a $side$ in the context of an $angle$ of the same $triangle$
\end{itemize}
if we assume
$$triangle \base angle$$
 and 
$$triangle \base side$$
so that
$$angle \base (angle \cross side)$$
\end{frame}
\fi

\fi %Dont use the slides ABOVE
\iffalse


\begin{frame}{Reminder}
In a contextual category,
\begin{itemize}
\item there is a well-founded partial ordering $<$ between objects
such that there is  a rooted $\omega$-tree of objects,
\item if $y$ covers $x$ in the partial order then write $x \base y$,
\item whenever $x \base y$ in the partial order then there is a distinguished
morphism $p_y:y \morph x$ which I write as $y \smorph x$,
\item the root object is a terminal object in the category,
\item the root object represents the empty context, 
\item non-root objects are at the same time both contexts and types
because the syntactic \underline{difference beween contexts and types} is 
just that - \underline{a syntactic difference}. 
\end{itemize}
\end{frame}



\begin{frame}{Concept-Instance Algebra -  Overview}
In a concept-instance algebra there are concepts and instances and there are operations $^*$, $\crossx{}{}{}$ and $\delta$
and these must satisfy various axioms.
\begin{itemize}
\item there is a rooted $\omega$-tree of concepts,
\item for each non-root concept there is a associated set of instances (instances of that concept),
\item operation $^*$ enables  concepts and instances to be particularised (syntactically this is substitution),
\item  operation $\crossx{}{}{}$ enables the positing in-scope concepts and instances (syntactially this is sometimes called weaking),
\item operation $\delta$ is an expression of self (as for example the identity morphisms in a category).
\end{itemize}
\end{frame}

\begin{frame}{Summary of Operations -- $^*$, $\cross$ and $\delta$}

In the following summary assume $x_c \base x$ and $y_c \base y$ in the tree of concepts of an algebra $A$. \\
\medskip
\scalebox{0.9}{%scope for tabcolsep
\setlength\tabcolsep{2pt}
\begin{tabular}{|l|l|p{3cm}|l|}
\hline
            &     reads as                      & defined whenever    & and is such that \\
\hhline{|=|=|=|=|}
$f^*y$      & particularise $y$ to $f$  & $f \in inst(x)$\newline \rule{0.8cm}{0pt} and $x<y$   &\begin{tabular}[t] {l}
                                                                              $x \base y$ then $x_c \base f^*y$\\
                                                                              $x \ll y$ then $f^*y_c \base f^*y$
                                                                           \end{tabular}                              \\
\hline
$f^*g$       & particularise $g$ to $f$ & if also $g\in inst(y)$      & $f^*g \in inst(f^*y)$                          \\
\hline
$x \cross y$ &posit $y$ in context $x$  & $x_c < y$                      &\begin{tabular} {l}
                                                                              $x_c \base y$ then $x \base (x \cross y)$\\
                                                                              $x_c \ll y$ then $(x \cross y_c) \base (x \cross y)$
                                                                           \end{tabular}                              \\
\hline
$x \cross g$ &posit $g$ in context $x$  & if also $g\in inst(y)$            & $x \cross g \in inst(x \cross y)$            \\
\hline
$\delta_x$   & this $x$                 &                         & $\delta_x \in inst(x \cross x)$              \\
\hline
\end{tabular}
}
\pause Thats all folks
\end{frame}

\begin{frame}{Concepts and their Contexts}
\begin{itemize}
\item In a concept-instance algebra $A$ concepts are equally contexts.
\item If $a \base b \base c$ in the tree of concepts of algebra $A$ 
then $a$ is to be understood as the context for concept $b$ and $b$ is to be understood as the context for concept $c$,
\item For example some concepts in cricket
\pstree[treemode=\CItreemode,levelsep=*0.65cm,treesep=\CItreesep,nodesep=0.05]
{
    \Tr{\circ}
}
{
    \pstree [levelsep=*0.85cm]
    {
		\Tr{match} 
	}
	{		  
		\pstree [levelsep=*0.85cm]
		{
				   \Tr{side} %\uppermember {homeSide}
		}
		{
					\Tr{player} %\uppermember {captain}
		}
	    \pstree [levelsep=*0.85cm]
		{
			\Tr{innings} 
		}
		{		  
		    \pstree [levelsep=*0.85cm]
			{
					   \Tr{over} 
			}
			{   
					   \Tr{delivery} 
			}			
		}		
	}	
}
\end{itemize} 
\end{frame}

\begin{frame}{Concepts, Contexts  and Instances}
\begin{itemize}
\item Instances of concepts are to be understood as instances in context: If $x_c \base x$  in algebra $A$ and if $i \in inst_A(x)$ thn $i$ is to be considered an instance of $x$ in context $x_c$.
\onslide<2-> {\item Thus $homeSide \in inst(side)$ and $match \base side$ implies  $match$ is context for $homeSide$}
\onslide<4-> {\item and $captain \in inst(player)$ and $side \base player$ implies $side$ is context for $captain$}.
\end{itemize}
\onslide<3->{\begin{displaymath}
\pstree[treemode=\CItreemode,levelsep=*0.65cm,treesep=\CItreesep,nodesep=0.05]
{
    \Tr{\circ}
}
{
    \pstree [levelsep=*0.85cm]
    {
		\Tr{match} 
	}
	{		  
		\pstree [levelsep=*0.85cm]
		{
				   \Tr{side} \member {homeSide}
		}
		{
					\Tr{player} \waitfor{5}{\member {captain}}
		}
	    \pstree [levelsep=*0.85cm]
		{
			\Tr{innings} 
		}
		{		  
		    \pstree [levelsep=*0.85cm]
			{
					   \Tr{over} 
			}
			{   
					   \Tr{delivery} 
			}			
		}		
	}	
}
\end{displaymath}}
\end{frame}






\begin{frame}{Particularisation $^*$}
In a concept-instance algebra $A$
\begin{itemize}
\item concepts and their instances can be particularised. The particularisation of concept $y$ to instance $f$ is written as $f^*y$. 
\item If $f$ is an instance of $x$ and $x < y$ then $f^*y$ is defined and assuming $x_c \base x$ then 
\end{itemize}
\medskip
\pause
\begin{tabular} {c p{0.2cm} c}
\onslide<2->{(a) if $x \base y$ then $x_c \base f^*y$}   
&&   
\onslide<5->{(b) if $y_c \base y$ and $x <y_c$ then $f^*y_c \base f^*y$} \\
\onslide<3->{$\displaystyle
% I coded it this way with $ and displaystyle rather than with displaymath because uit is the only way that i caould
% find that enabled me to include in a tabular display
\pstree[treemode=\CItreemode,levelsep=*0.65cm,treesep=\CItreesep,nodesep=0.05]
{
	\Tr{x_c}
}
{
   \pstree
	{
	   \Tr{x}\member{f}
	}
	{
		\Tr{y} 
	}
	\Tr{f \sub y}
}
$}
&&
\onslide<6->{$\displaystyle
% I coded it this way with $ and displaystyle rather than with displaymath because uit is the only way that i caould
% find that enabled me to include in a tabular display
\pstree[treemode=\CItreemode,levelsep=*0.65cm,treesep=\CItreesep,nodesep=0.05]
{
	\Tr{x}
}
{
   	\pstree[levelsep=*2.0cm]
	{
	   \Tr{y}\member{f}
	}
	{
		\pstree[levelsep=*0.75cm]
	   	{
	     	\Tr[edge=\dottededge]{z_c}
	   	}
	   	{
			\Tr{z}
	   	} 
	}
	\pstree[nodesep=0, levelsep=*0.85cm] %latest added levelsep
	{
	   \Tr{\ .}
	}
	{
		\pstree[nodesep=0.05,levelsep=*0.75cm]
	   	{
	     	\Tr[edge=\dottededge]{f^*z_c}
	   	}
	   	{
			\Tr{f^*z}
	   	} 
	}
}
$} \\
\end{tabular}
\end{frame}

\begin{frame}{Particularisation cont.}
\begin{itemize}
\item The particularisation of instance $g$ to instance $f$ is written as $f^*g$.
\item If $f \in inst(x)$ and  $g \in inst(y)$
then if $f^*y$ is defined, i.e. if $x < y$, then $f^*g$ is defined and $f^*g \in inst(f^*y)$.
\medskip
\pause
\item In the cricket example 
\begin{displaymath}
\pstree[treemode=\CItreemode,levelsep=*0.65cm,treesep=\CItreesep,nodesep=0.05]
{
    \Tr{\circ}
}
{
    \pstree [levelsep=*0.85cm]
    {
		\Tr{match} 
	}
	{		  
		\pstree [levelsep=*0.85cm]
		{
				   \Tr{side} \member {home}
		}
		{
					\Tr{player} \member {captain}
		}
	    \Tr{home^*player\kern-1cm} \uppermember{home^*captain}		
	}	
}
\end{displaymath}
\end{itemize}
\end{frame}

\iffalse
\begin{frame}{Particularisation cont. (2)}
At this point it follows that if $y \base z_1 ... \base z_m \base z$ and $g$ is a instance of $z$ then we have:
\begin{displaymath}
\pstree[treemode=\CItreemode,levelsep=*0.65cm,treesep=\CItreesep,nodesep=0.05]
 {
    %\Tr{\circ}
		\Tr{x}
 }
 {%\flexbranch{Lxn}{1cm}{1cm}{x}{n}{}
    {\pstree
		   {\Tr{y}\member{f}}
			 {
			 \flexbranchplusarc{Ly}{1cm}{1cm}{z}{m}{}{g}
			 }
		 \flexbranchplusarc{Lfy}{1cm}{1.3cm}{f \sub z}{m}{} {f \sub g}
		}
}
\end{displaymath}
\vspace{0.5cm} 
\end{frame}


\begin{frame}{Example}
\begin{displaymath}
\pstree[treemode=\CItreemode,levelsep=*0.65cm,treesep=\CItreesep,nodesep=0.05]
{
    \Tr{\circ}
}
{
    \pstree [levelsep=*0.85cm]
    {
		\Tr{match} 
	}
	{		  
		\pstree [levelsep=*0.85cm]
		{
				   \Tr{side} \member {home}
		}
		{
					\Tr{player} \member {captain}
		}
	    \Tr{home^*player\kern-1cm} \uppermember{home^*captain}		
	}	
}
\end{displaymath}
\end{frame}
\fi








\begin{frame}{$x \cross y$ -- Posit concept $y$ in context $x$}
\begin{itemize}
\item If $x$ and $y$ are concepts of $A$ such that $x_c \base x$ and $x_c < y$
%if x_c is context for x and x_c is part of the context of y then
 then $x \cross y$ is defined and is a concept of $A$.
\end{itemize}
\pause
\begin{tabular} {c p{0.2cm} c}
\onslide<2->{(a) if $x_c \base y$ then $x \base x \cross y$}   
&&
\onslide<5->{(b) if $y_c \base y$ and $x_c < y_c$then $ x \cross y_c  \base x \cross y$} \\[0.5cm]
\onslide<3->{\input{../content/positintro.concepts.caseone}}
&&
\onslide<6->{$\displaystyle
% I coded it this way with $ and displaystyle rather than with displaymath because uit is the only way that i caould
% find that enabled me to include in a tabular display
\pstree[treemode=\CItreemode,levelsep=*0.65cm,treesep=\CItreesep,nodesep=0.05]
{
	\Tr{x_c}
}
{
   	\pstree[levelsep=*2.0cm]
	{
	   \Tr{x}
	}
	{
	\onslide<7>{
		\pstree[levelsep=*0.75cm]
	   	{
	     	\Tr[edge=\dottededge]{x \cross y_c}  
	     	              %use nudgeup to counter balances the vert space of the subscript w
	   	}
	   	{
			\Tr{x \cross y}
	   	} 
	   	      }
	}
	\pstree[nodesep=0, levelsep=*0.85cm] %latest added levelsep
	{
	   \Tr{\ .}
	}
	{
		\pstree[nodesep=0.05,levelsep=*0.75cm]
	   	{
	     	\Tr[edge=\dottededge]{y_c}
	   	}
	   	{
			\Tr{y}
	   	} 
	}
}
$} \\
\end{tabular}
\end{frame}

\begin{frame} {$x \cross g$ -- Posit instance $g$ in context $x$}
\begin{itemize}
\item If $g \in inst(y)$ in $A$ and $x \cross y$ is defined then $x \cross g$ is defined and $x \cross g \in inst(x \cross y)$.
\medskip
\pause
\item In the cricket example, a side can be posited in the context of an innings.
\pause
\onslide<5->{\item The batting side is an instance such concept.}
\begin{displaymath}
\pstree[treemode=\CItreemode,levelsep=*0.65cm,treesep=\CItreesep,nodesep=0.05]
{
    \Tr{\circ}
}
{
    \pstree [levelsep=*0.85cm]
    {
		\Tr{match} 
	}
	{		  
		\Tr{side} 
	    \pstree [levelsep=*0.85cm]
		{
			\Tr{innings} 
		}
		{		  
			\pause \Tr{innings \cross side} \pause \uppermember {battingSide}		
		}		
	}	
}
\end{displaymath}
\end{itemize}
\end{frame}

\begin{frame}{Further instance example -- introduce the bowler}
\begin{itemize}
\item The bowler is defined in the context of an over and is (an instance of) a fielding side player.
\end{itemize}
\medskip
\pause
\begin{displaymath}
\pstree[treemode=\CItreemode,levelsep=*0.65cm,treesep=0.5cm,nodesep=0.05]
%{
%    \Tr{\circ}
%}
%{
    %\pstree [levelsep=*0.85cm]
    {
		\Tr{match} 
	}
	{		  
		%\Tr{side} 
	    \pstree [levelsep=*0.85cm]
		{
			\Tr{innings} 
		}
		{		  
			\pstree [levelsep=*0.85cm]
			{
				\Tr{innings \cross side} \member {fieldingSide}
			}
			{
				\Tr{innings \cross player}
			}
			%\Tr{fieldingSide^*(innings \cross player)}	
			\pstree
			{
             	\Tr{over}
			}
			{
				\Tr{over \cross (fieldingSide^*(innings \cross player))} \uppermember{bowler}
			}	
		}		
	}	
%}
\end{displaymath}
\end{frame}








\begin{frame}{self $delta$}
If $x$ is a concept of $A$  then $\delta_x$ is an instance of $x \cross x$
\vspace{0.5cm}
\begin{displaymath}
\pstree[treemode=R,levelsep=*0.65cm,treesep=1cm,nodesep=0.05]
{
   \Tr{x}
}
{
%\pstree[treemode=R,levelsep=*0.5cm,treesep=1cm,nodesep=0.05]
\pstree [levelsep=*0.85cm]
 {
    \Tr{y}
 }
 {%\flexbranch{Lxn}{1.3cm}{1.5cm}{x}{n}{}
   {
	   \Tr{y \cross_x y} \member{\diag{y}}
	}
 }
}
\end{displaymath}
\vspace{0.3cm} 
\end{frame}


\fi
\newcommand{\objaxiom}[3]{\onslide<3->{#1&=#2&&\mbox{#3}}}
\newcommand{\instaxiom}[3]{\onslide<4->{#1&=#2&&\mbox{#3}}}
\newcommand{\otheraxiom}[3]{\onslide<5->{#1&=#2&&\mbox{#3}}}
\begin{frame}{Axioms}
6 pairs of axioms plus 2 others.
\pause \small \begin{align*}
\objaxiom{(f^*g)^*(f^*z) }{ f^*(g^*z)}{$f \in inst(x),\ g \in inst(y),\ x < y$ and  $y < z$}\\
\instaxiom{(f^*g)^*(f^*h) }{ f^*(g^*h)        
}{as above and $h \in inst(z)$ }\\
\objaxiom{(x \cross y) \cross (x \cross z) }{ x \cross (y \cross z)
}{$x_c < y$, where $x_c \base x$, and $x_c < z$} \\
\instaxiom{(x \cross y) \cross (x \cross g) }{ x \cross (y \cross g)
}{as above and $g \in inst(z)$}\\
\objaxiom{f^*(x \cross y) }{ y               
}{ $f \in inst(x)$ and $x_c < y$, where $x_c \base x$}\\
\instaxiom{f^*(x \cross g) }{ g               
}{as above and $g \in inst(y)$}\\
\objaxiom{f^*y \cross f^*z }{ f^*(y \cross z)  %problem here  
}{$f \in inst(x)$, $x < z$, $x < y$ and $y_c <z$ where $y_c \base y$}\\
\instaxiom{f^*y \cross f^*g }{ f^*(y \cross g)    
}{as above  and $g \in inst(z)$}\\
\objaxiom{(x \cross g)^*(x \cross z)}{x \cross(g^*z) 
}{$g \in inst(y)$ and $x_c < y$ and $y < z$, where $x_c \base x$}\\
\instaxiom{(x \cross g)^*(x \cross h)}{x \cross(g^*h) 
}{as above and  $h \in inst(z)$}\\
\objaxiom{\delta_x ^*(x \cross y)}{y                 
}{$x < y$}\\
\instaxiom{\delta_x ^*(x \cross g)}{g                 
}{$x < y$ and $g\in inst(y)$}\\
\otheraxiom{f \sub \diag(x) }{ f                       
}{$f \in inst(x)$}\\
\otheraxiom{f \sub \diag(y) }{ \diag(f \sub y)         
}{$f \in inst(x)$ and  $x < y$.}
\end{align*}
\end{frame}
\iffalse

% \mmCIalg mathmode CIalg
\newcommand{\mmCIalg}{\cat{CI}\mhyphen\cat{alg}}

\newcommand{\makespace}{\nudgedown{14pt}\nudgeup{20pt}} % vertical space for labels on arrows

\newcommand{\gatconarrows}{
\ncarr{gat}{con}
\alabel{C}
\ncarr{con}{gat}
\alabel{U}
}

\newcommand{\gatconequiv}
{
\begin{array}{c p{1.5cm} c}
\makespace\Rnode{gat}{\cat{GAT}} && \Rnode{con}{\cat{Con}}
\end{array}
\begin{arrows}
\setlength{\arroffsetA}{5pt}
\setlength{\arroffsetB}{5pt}
\gatconarrows
\end{arrows}
}
\newcommand{\gatciarrows}{
\ncarr{gat}{cialg}
\alabel{C}
\ncarr{cialg}{gat}
\alabel{U}
}

\newcommand{\gatciequiv}
{
\begin{array}{c p{1.5cm} c}
\makespace\Rnode{gat}{\cat{GAT}} && \Rnode{cialg}{\mmCIalg}
\end{array}
\begin{arrows}
\setlength{\arroffsetA}{5pt}
\setlength{\arroffsetB}{5pt}
\gatciarrows
\end{arrows}
}

\newcommand{\conciarrows}
{
\ncarr{con}{cialg}
\alabel{CI}
\ncarr{cialg}{con}
\alabel{CC}
}

\newcommand{\concialgequiv}
{\begin{array}{c p{1.5cm} c}
\makespace\Rnode{con}{\cat{Con}}  &&  \Rnode{cialg}{\mmCIalg}
\end{array}
\begin{arrows}
\setlength{\arroffsetA}{5pt}
\setlength{\arroffsetB}{5pt}
\conciarrows
\end{arrows}
}

\newcommand{\UCIequivalences}
{
\begin{array}{c p{1.5cm} c p{1.5cm} c}
\makespace\Rnode{gat}{\cat{GAT}} && \Rnode{con}{\cat{Con}}  &&  \Rnode{cialg}{\mmCIalg}
\end{array}
\begin{arrows}
\setlength{\arroffsetA}{5pt}
\setlength{\arroffsetB}{5pt}
\gatconarrows
\conciarrows
\end{arrows}
}

\begin{frame}{Equivalent Categories}
\begin{itemize}
\item There are equivalences of categories  $\cat{GAT} \simeq \mmCIalg \simeq \cat{Con}$.
\item The equivalence $\gatciequiv$ I proved in 1976 in a draft (non-extant) of my thesis.
\medskip
\item The equivalence $\gatconequiv$ is proved in my 1978 thesis and is summarised in my 1986 paper.
\medskip
\item The equivalence $\concialgequiv$ is an exercise in equational reasoning (passed up 1976)
-- I'm going to discuss this next.
\end{itemize}
\end{frame}


\newcommand{\inningsCrossSide}{\crossx{innings\kern-0.3cm}{\kern-0.3cm side}{match}}
\newcommand{\inningsCrossPlayer}{\crossx{innings\kern-0.3cm}{\kern-0.3cm player}{match}}
\newcommand{\fieldingSidePlayer}{fieldingSide ^* (\inningsCrossPlayer)}
\newcommand{\battingSidePlayer}{battingSide ^* (\inningsCrossPlayer)}
\newcommand{\overCrossFieldingSidePlayer}{\crossx{over\kern-0.4cm}{\kern-0.4cm(\fieldingSidePlayer)}{innings}}
\newcommand{\deliveryCrossBattingSidePlayer}{\crossx{delivery\kern-0.4cm}{\kern-0.4cm (\battingSidePlayer)}{innings}}

\begin{frame}{Introducing the $\crossx{x}{y}{w}$ and $\crossx{x}{g}{w}$ Notation}
\pause
\begin{itemize}
\item $\crossx{x}{y}{w}$ is defined whenever $w < x$ and $w<y$.
\pause
\item Suppose $w \base x_1 \base ... \base x_n $
and $w \base y_1 \base ... \base y_m$, 
\begin{center}
$\displaystyle
% I coded it this way with $ and displaystyle rather than with displaymath because uit is the only way that i caould
% find that enabled me to include in a tabular display
\pstree[treemode=\CItreemode,levelsep=*0.65cm,treesep=\CItreesep,nodesep=0.05]
{
	\Tr{w}
}
{
   	\pstree[levelsep=*0.85cm]
	{
	   \Tr{x_1}
	}
	{   \pstree[levelsep=*0.65cm]
	    {
	    	\Tr[edge=\dottededge]{x_n}
	    }
	    {
            \onslide<4->
            {
				\pstree[nodesep=0.05,levelsep=*0.85cm]
			   	{
			     	\Tr{\crossx{x_n}{y_1}{w}} 
			   	}
			   	{
					\Tr[edge=\dottededge]{\crossx{x_n}{y_m}{w}}
			   	}
		   	}
	   	} 
	}
	\pstree[nodesep=0, levelsep=*0.95cm] %latest added levelsep
	{
	   \Tr{y_1}
	}
	{
			\Tr[edge=\dottededge]{y_m}
	}
}
$
\end{center}
\onslide<5->{\item Define $\crossx{x_n}{y_i}{w}=x_n \cross ( x_{n-1} \cross ... (x_1 \cross y_i)...)$.}
\onslide<6->{\item Similarly, if $g \in inst(y_i)$ then define $\crossx{x_n}{g}{w}=x_n \cross ( x_{n-1} \cross ... (x_1 \cross g)...)$}.
\end{itemize}
\end{frame}

\begin{frame}{Introducing $\delta_{x_n,x_i}$ Notation}
\begin{itemize}
\item Suppose $1 \base x_1 \base x_2 ... \base x_n$ and $1 \leq i \leq n$,
\begin{center}
$\displaystyle
% I coded it this way with $ and displaystyle rather than with displaymath because uit is the only way that i caould
% find that enabled me to include in a tabular display
\pstree[treemode=\CItreemode,levelsep=*0.65cm,treesep=0.1cm,nodesep=0.05, treenodesize=0.75cm]
{
	\TR{x_{i-1}}
}
{
   	\pstree[levelsep=*0.85cm]
	{
	   \TR{x_i}
	}
	{   \Tn
	    \pstree[levelsep=*0.85cm]
	    {
	    	\TR{x_{i+1}}
	    }
	    {       
				\pstree[nodesep=0.05,levelsep=*0.85cm]
			   	{
			     	\TR[edge=\dottededge]{x_n} 
			   	}
			   	{
			   	    \onslide<2->{
			   	    \Tn   
			   	    \TR{\crossx{x_n}{x_i}{x_{i-1}}}
			   	    \onslide<2>{\uppermember{\crossx{x_n}{\delta_{x_i}}{x_i}}}
			   	    \onslide<3->{\uppermember{\delta_{x_n,x_i}=\crossx{x_n}{\delta_{x_i}}{x_i}}}
			   	               }
			   	}
	   	} 
	    \TR{\crossx{x_i}{x_i}{x_{i-1}}}\uppermember{\delta_{x_i}}
	}
}
$
\end{center}
\onslide<4>{\item define 
$\delta_{x_n,x_i} \in inst(\crossx{x_n}{x_i}{x_{i-1}})$ by:
\begin{tabular}{c c}
$\delta_{x_n,x_n}=\delta_{x_n}$ & $\delta_{x_n,x_i}=\crossx{x_n}{\delta_{x_{n-1}}}{x_{i-1}}$
\end{tabular}
}
\end{itemize}
\end{frame}


\fi

\begin{frame}{Construction of CI-algebra}
\begin{itemize}
\item The tree of concepts of $CI(C)$ is defined to be the tree of objects of $C$.

\item If $A \base B$ in $C$ then $inst(B)$ in $CI(C)$ is defined to be the set of sections of
 $B$ in $C$ i.e. the set $\setsuchthat{f: A \morph B}{B \circ p_B = id_A}$

\item Operations $^*$, $\cross$ are defined in $CI(C)$ using the pullbacks and dependency morphisms of $C$.  
 There are details of suitable  operations $^*$ and $\cross$ defined in section 3 of my paper
 "Instances of Generalised Algebraic Theories in Contextual Categories" on ResearchGate.
 \item If $A$ is an object of $C$ then define $\delta_A$ in algebra $CI(C)$ to be the section
 $s(id_A)$, where $s$ is Voevodsy's `s' operator.
 \item Proving that the fourteen axioms hold is an exercise in simple equational algebra. Some of the work is done, in passing, in lemmas 5.1, 5.2, 5.11, 5.14 and 5.17 of my "Instances..." paper.
\end{itemize}
\end{frame}

\begin{frame}{CC -- Construction of Contextual Category}
\begin{itemize}
\item From a concept-instance algebra $A$ we can construct a contextual category $CC(A)$.
\item the tree of objects of the category $CC(A)$ is defined to be the tree of concepts of the CI-algebra.
\item if $x$ and $y_n$ are concepts and $1 \base y_1 \base y_2 ... \base y_n$ in $A$
then $Hom_{CC(A)}(x,y_n)$ is defined to be the set of n-tuples $\tuple{\fn}$ of instances of $A$ 
where 
\begin{align*}
f_1 \in in&st(\crossx{x}{y_1}{1}),                                 \\ 
f_2 \in in&st(\fonestar(\crossx{x}{y_2}{1})),                      \\
&\vdots                                                            \\
f_n \in in&st(\fnonestar...\ftwostar\fonestar(\crossx{x}{y_n}{1})) \\
\end{align*}
\end{itemize}
\end{frame}

\begin{frame}{Identity and Dependency Morphisms in $CC(A)$}
If $1 \base x_1 \base x_2 ... \base x_n$ in $CC(A)$
then 
\begin{itemize}
\item define the identity morphism $id_{x_n}:x_n \morph x_n$  
to be the n-tuple:
$$\tuple{\delta_{x_n,x_1},...\delta_{x_n,x_n}}$$
\item define the dependency morphism $p_{x_n}:x_n \smorph x_{n-1}$ to be the n-tuple:
$$\tuple{\delta_{x_n,x_1},...\delta_{x_n,x_{n-1}}}$$
\end{itemize}
\end{frame}


\iffalse{
\begin{frame}{Composition of morphisms}
Composition of morphisms is defined as follows.

$$\tuple{f_1,...f_n}\circ \tuple{g_1,...g_m} 
=\tuple{\fnvectorstar(\crossx{x}{g_1}{1}),...\fnvectorstar(\crossx{x}{g_m}{1})}
$$
\end{frame}
}\fi

\begin{frame}{$\circ$, $f^*x$ and $q(f,x)$ and $s(f)$}
\newcommand{\pullbackobject}{\fnvectorstar(\crossx{x}{y}{1})}
\newcommand{\deltaterm}{\delta_{\pullbackobject,x}}

Define composition by
\begin{equation} 
\tuple{f_1,...f_n}\circ \tuple{g_1,...g_m} 
=\tuple{\fnvectorstar(\crossx{x}{g_1}{1}),...\fnvectorstar(\crossx{x}{g_m}{1})},
\end{equation}
define the $^*$ operator by
\begin{equation} 
\tuple{\fn}^*y=\pullbackobject,
\end{equation}
define the $q$ operator by
\begin{equation} 
q(\tuple{\fn},y)=
\tuple{{\beta}^*f_1,...{\beta}^*f_n,\beta},
\end{equation}
where 
$\beta = \deltaterm$,

and define the $s$ operator by 
\begin{equation} 
s(\tuple{\fn})=f_n.
\end{equation}

Can prove that $CC(A)$ is a contextual category by showing that the operators
$\circ$,$id$, $^*$, $q$, $s$ are well-defined, properly typed and satisfy the axioms discovered by Vladimir Voevodsky and documented on my ResearchGate page
www.researchgate.net/profile/John-Cartmell
as 
"Generalised Algebraic Axiomatisations of Contextual Categories".
\end{frame}





\newcommand{\ofOb}[1]{\ofT{#1}{\Ob}}
\newcommand{\ofHom}[2]{\ofT{#1}{\Hom(#2)}}

\begin{frame}{Generalised Algebraic Theories}
Generalised algebraic theories are expressed in a syntax that
\begin{itemize}
\item involves rules of these forms\\
\begin{tabular}{c p{1cm} c}
\gatdisplayrule{\xDelta{n}}{\isT{\Delta}}   & \gatdisplayrule{\xDelta{n}}{\ofT{t}{\Delta}}\\
\gatdisplayrule{\xDelta{n}}{\Delta=\Delta'} & \gatdisplayrule{\xDelta{n}}{t=\ofT{t'}{\Delta}} 
\end{tabular}
\item and meta-rules which I call principles of derivation each enabling a rule of one of the above forms to be derived from a number of other previously derived rules
\begin{itemize}
\item LI1...LI7 -- Laws of Identity,
\item T1 -- Transfer Law,
\item CF1, CF2(a) and CF2(b) -- Cut-free versions of term and type substitution laws,
\item SI1 and SI2 -- Substitution of and into Identicals.
\end{itemize} 
\end{itemize}
\end{frame}


\begin{frame}{Generalised Algebraic Theories}
... are defined to consist of
\begin{itemize}
\item A set of sort symbols each with an introductory rule. 
A sort symbol $A$ must have an introductory rule of the form \genericAintroductoryrule and this rule must be well-typed.
\item A set of operator symbols each with an introductory rule. 
An operator symbol $f$ must have an introductory rule of this form   \genericfintroductoryrule and this rule  must be well-typed,
\item a set of well-typed axioms. Each axiom is of one of these two forms
\begin{tabular}{c p{1cm} c}
\gatdisplayrule{\xDelta{n}}{\Delta=\Delta'} & \gatdisplayrule{\xDelta{n}}{t=\ofT{t'}{\Delta}} 
\end{tabular}
\end{itemize}
\end{frame}

\begin{frame}{Theory of Categories}
\footnotesize
\newcommand{\associativitypremisereversed}
       {\begin{array}[t]{l}\ofT{f}{Hom(z_1,z_2)},\,\ofT{g}{Hom(z_2,z_3)},\,\ofT{h}{Hom(z_3,z_4)},\\
                       \hspace{3.5cm}\ofT{z_1,z_2,z_3,z_4}{Ob}
        \end{array}
       }
\renewcommand{\gatintros}
{
\textbf{Sym.} & \textbf{Introductory\ Rule}                      \\}
\renewcommand{\gataxioms}
{\textbf{Ax.}\\}
% \gataxiomno{axiomno}
\renewcommand{\gataxiomno}[1]{\makebox[0.15cm]{} \axid{#1}}

\begin{gatrules}
\gatintros
\gatintroducing{Ob}
\isT{Ob} \\
\gatintroducing{Hom}
  \gatsingular{\ofT{x_1,x_2}{Ob}}{\isT{Hom(x_1,x_2)}} \\	
\gatintroducing{id}
  \gatsingular{\ofT{w}{Ob}}{\ofT{id_w}{Hom(w,w)}} \\	
\gatintroducing{\circ}
  \gatsingular{\ofT{f}{Hom(x_1,x_2)},\ \ofT{g}{Hom(x_2,x_3)},\ \ofT{x_1,x_2,x_3}{Ob}, }{\ofT{\circ(f,g)}{Hom(x_1,x_3)}} \\  
\gataxioms
\gatintroducing{  \gataxiomno{1} \\   \gataxiomno{2}}
\begin{gatgroup}{\ofT{f}{Hom(x_1,x_2)},\ \ofT{x_1,x_2}{Ob}}
    \gatleaf{}{id_{x_1} \circ f = f} \\
    \gatleaf{}{f \circ id_{x_2} = f}
\end{gatgroup} \\
\gatintroducing{ \gataxiomno{3} }
\gatsingular{\associativitypremisereversed}{(f \circ g) \circ h = f \circ (g \circ h)} 
\end{gatrules}
\end{frame}

\iffalse
\begin{frame}{Theory of Cricket}
\footnotesize

% In this example I have had to fine tune the various widths.
% I haven't been able to find a way of getting the dotfill to fill
% out the entire width of a containing column of an array.
\begin{displaymath}
\begin{array}{l}
\isT{match} \\
\begin{gatgroup}{\ofT{m}{match}}
  \gatleaf[8.0cm]{}{\isT{innings(m)}} \\
  \gatleaf[8cm]{}{\isT{side(m)}}\\
  \gatleaf[8cm]{}{\ofT{homeSide(m)}{side}}\\
  \begin{gatgroup}{\ofT{s}{side(m)}}
    \gatleaf[6cm]{} {\isT{player(s)}} \\
    \gatleaf[6cm]{} {\ofT{captain(s)}{player(s)}}
  \end{gatgroup} \\
    \begin{gatgroup}{\ofT{i}{innings(m)}}
    \gatleaf[6cm]{}{\ofT{fieldingSide(i)}{side}} \\
    \gatleaf[6cm]{}{\ofT{battingSide(i)}{side}} \\
    \gatleaf[6cm]{}{\isT{over(i)}} \\
    \begin{gatgroup}{\ofT{o} {over(i)}}
      \gatleaf[5.8cm]{}{\ofT{bowler(o)}{player(fieldingSide(i))}} \\
      \gatleaf[5.8cm]{}{\isT{delivery(o)}} \\
      \gatleaf[5.8cm]{}{\ofT{facingBatter(d)}{player(battingSide(i))}} \\
      \makebox[5.8cm][r]{\hspace{2cm} \dotfill where $\ofT{d}{delivery(o)}$  }
    \end{gatgroup} 
  \end{gatgroup} 
\end{gatgroup}
\end{array}
\end{displaymath}


\end{frame}
\fi



\newcommand{\inst}{i} 
\newcommand{\cpt}{c} 
\begin{frame}{Generalised Algebraic Theory of Concept Instance Algebras}
\begin{itemize}
\item There is a generalised algebraic theory of concept instance algebras.
\item sorts representing levels of concepts are $\cpt_1, \cpt_2 ...$.
\item sort $\cpt_{n}$ is introduced by:
\gatdisplayrule{\ofT{x_1}{\cpt_1},... \ofT{x_{n-1}}{\cpt_{n-1}(x_1,...x_{n-2})}} {\isT{\cpt_{n}(x_1,...x_{n-1})}}
\item sorts representing instances are $\inst_1, \inst_2, ....$.
\item sort $\inst_{n}$ is introduced by:
\gatdisplayrule{\context{x}{\cpt}{n}}{\isT{\inst_{n}(x_1,...x_n)}}
\item Similarly 
\begin{itemize}
\item countably many $^*$, $\cross$ and $\delta$ operations,
\item countably many axioms for each of the axioms given earlier.
\end{itemize}
\end{itemize}

\end{frame}

\begin{frame}{Relationship with Contextual Categories \& Generalised Algebraic Theories}
There are equivalences of categories between
\begin{itemize}
\item the category of generalised algebraic theories
\end{itemize}
and each of the following
\begin{itemize}
\item the category of concept instance algebras,
\item the category of contextual categories,
\item the category of contextual categories with families
\end{itemize}
but not with 
\begin{itemize}
\item the category of categories with families.
\end{itemize}
\end{frame}


\iffalse
\begin{frame}
\begin{displaymath}
\begin{array}{c p{0.1cm} c}
                     && \Rnode{root}{\circ}     \\[0.4cm]
                     && \Rnode{C0}{\cpt_0}         \\[0.4cm]
\Rnode{I0}{\inst_0}      && \Rnode{C1}{\cpt_1}         \\[0.4cm]
\Rnode{I1}{\inst_1}      && \Rnode{C2}{\cpt_2}         \\[0.25cm]
\Rnode{I2}{\inst_2}      && \vdots                  \\[0.25cm]
\vdots			     && \Rnode{Ci}{\cpt_i}         \\[0.4cm]
\Rnode{Ii}{\inst_i}      && \Rnode{Csi}{\cpt_{i+1}}    \\[0.25cm]
\Rnode{Isi}{\inst_{i+1}} && \vdots                  \\[0.25cm]
\vdots               &&
\end{array}
\begin{arrows}
\ncline[nodesep=4pt]{C0}{root}
\ncline[nodesep=4pt]{C1}{C0}
\ncline[nodesep=4pt]{C2}{C1}
\ncline[nodesep=4pt]{Csi}{Ci}
\ncline[nodesep=4pt]{I0}{C0}
\ncline[nodesep=4pt]{I1}{C1}
\ncline[nodesep=4pt]{I2}{C2}
\ncline[nodesep=4pt]{Ii}{Ci}
\ncline[nodesep=4pt]{Isi}{Csi}
\end{arrows}
\end{displaymath}
\end{frame}
\fi

\begin{frame}{Now as a Concept-Instance Algebra}
 \def\dedge{\ncline[linestyle=dotted]}
$$
\pstree[treemode=\CItreemode, treefit=loose,treenodesize=0.25cm,levelsep=0.65cm,treesep=0.7cm,nodesep=2pt]
{
  \Tr{\circ}
}
{
  \pstree
  {
     \Tr{\cpt_0}
  }
  {
    \Tr{\inst_0}
	\pstree
	{
	     \Tr{\cpt_1}
	}
	{
      \Tr{\inst_1}
  	  \pstree
	  {
	     \Tr{\cpt_2}
	  }
	  {  
		 \Tr{\inst_2}
		 \pstree%[levelsep=*0.75cm]
		 {
		    \Tr[edge=\dedge]{\cpt_i} 
		 }
		 {  
	        \Tr{\inst_i}
	        \pstree[levelsep=0.75cm,treesep=1.5cm] 
			{
			   \Tr{\cpt_{i+1}}
			}
			{
			   \Tr{\inst_{i+1}}
			   \Tr[edge=\dedge]{\ \ \ \ \ \nudgeup{0.3cm} \ \ \ \ \ \ \ \ } 
			}
		 }
	  }
	}
  }
}
$$
$$_0\cross_0$$
$$\qq{_0\cross_0}$$
$$\qq{\kern-3pt_0\cross_0\kern-3pt}$$
$$\qq{\cross_{0,0}} \in inst(\cpt_0 \cross (\cpt_0 \cross \cpt_0))$$
$$\qq{\delta_0} \in inst({\delta_{\cpt_0}}^*(\qq{\cross_0}^* (\cpt_0 \cross ( \cpt_0 \cross \inst_0))))$$
\end{frame}

\begin{frame}{As a Contextual Category}
 \def\dedge{\ncline[linestyle=dotted]}
 \def\sedge{\ncksar}
\begin{displaymath}
\pstree[edge=\sedge, treemode=U, treefit=loose,treenodesize=0.25cm,levelsep=0.8cm,treesep=0.7cm,nodesep=2pt]
{
  \Tr{1}
}
{
  \pstree
  {
     \Tr{\cpt_0}
  }
  {
    \Tr{\inst_0}
	\pstree
	{
	     \Tr{\cpt_1}
	}
	{
      \Tr{\inst_1}
  	  \pstree
	  {
	     \Tr{\cpt_2}
	  }
	  {  
		 \Tr{\inst_2}
		 \pstree%[levelsep=*0.75cm]
		 {
		    \Tr[edge=\dedge]{\cpt_i} 
		 }
		 {  
	        \Tr{\inst_i}
	        \pstree[levelsep=0.75cm,treesep=1.5cm] 
			{
			   \Tr{\cpt_{i+1}}
			}
			{
			   \Tr{\inst_{i+1}}
			   \Tr[edge=\dedge]{\ \ \ \ \ \nudgeup{0.3cm} \ \ \ \ \ \ \ \ } 
			}
		 }
	  }
	}
  }
}
\end{displaymath}
\end{frame}

\begin{frame}{Richard Garner...}
Describes a monad on the category $\Set^C$, where $C$ is this category 
 \def\dedge{\ncline[linestyle=dotted]}
 \def\backarrow{\nckarr}
$$
  \pstree[edge=\backarrow, treemode=U, treefit=loose,treenodesize=0.25cm,levelsep=0.8cm,treesep=0.7cm,nodesep=2pt]
  {
     \Tr{\cpt_0}
  }
  {
    \Tr{\inst_0}
	\pstree
	{
	     \Tr{\cpt_1}
	}
	{
      \Tr{\inst_1}
  	  \pstree
	  {
	     \Tr{\cpt_2}
	  }
	  {  
		 \Tr{\inst_2}
		 \pstree%[levelsep=*0.75cm]
		 {
		    \Tr[edge=\dedge]{\cpt_i} 
		 }
		 {  
	        \Tr{\inst_i}
	        \pstree[levelsep=0.75cm,treesep=1.5cm] 
			{
			   \Tr{\cpt_{i+1}}
			}
			{
			   \Tr{\inst_{i+1}}
			   \Tr[edge=\dedge]{\ \ \ \ \ \nudgeup{0.3cm} \ \ \ \ \ \ \ \ } 
			}
		 }
	  }
	}
  }
$$ and defines the algebras as equivalent to B-systems.
\end{frame}


\begin{frame}{Fungible Algebras}
Q. Are the following types of structure fungible
\begin{itemize}
\item concept instance algebras,
\item contextual categories,
\item contextual categories with families.
\end{itemize}
A. Depends on the matter at hand. If we are looking for Set-like instances of theories then they are, 
more generally they are not.

\begin{itemize}
\item concept-instance algebras and contextual categories are $\Sigma Id$-fungible
\end{itemize}
\end{frame}


\iffalse

\begin{frame}{Cricket as Data Specification}
\begin{tabular}{l l}
\raisebox{-4cm}{\scalebox{0.75}{\begin{erdiagram}{11.549999999999999}{9.25375}

\eret{2.9}{-2}{6.9}{-1.1}{0.2}{1}\eretname{3.3}{-1.45}{l}{match}
\erCoreAttribute{3.1}{-1.65}{1}{0}{id}{}
\eret{0.146}{-4.75}{3.774}{-2.95}{0.2}{1}\eretname{0.509}{-3.3}{l}{innings}
\erRelationalAttribute{0.346}{-3.5}{1}{0}{match\textunderscore id}{(D2)}
\erCoreAttribute{0.346}{-3.8}{1}{0}{number}{}
\erHierarchicalAttribute{0.346}{-4.1}{1}{1}{battingSide\textunderscore name}{(R1)}
\erHierarchicalAttribute{0.346}{-4.4}{1}{1}{fieldingSide\textunderscore name}{(R2)}
\eret{6.774}{-4.75}{9.254}{-2.95}{0.2}{1}\eretname{7.022}{-3.3}{l}{side}
\erRelationalAttribute{6.974}{-3.5}{1}{0}{match\textunderscore id}{(D3)}
\erCoreAttribute{6.974}{-3.8}{1}{0}{name}{}
\eret{0.438}{-7.75}{3.483}{-5.95}{0.2}{1}\eretname{0.742}{-6.3}{l}{over}
\erRelationalAttribute{0.638}{-6.5}{1}{0}{match\textunderscore id}{(D4)}
\erRelationalAttribute{0.638}{-6.8}{1}{0}{innings\textunderscore number}{(D4)}
\erCoreAttribute{0.638}{-7.1}{1}{0}{number}{}
\erHierarchicalAttribute{0.638}{-7.4}{1}{1}{bowler\textunderscore number}{(R3)}
\eret{6.81}{-7.75}{9.218}{-5.95}{0.2}{1}\eretname{7.051}{-6.3}{l}{player}
\erRelationalAttribute{7.01}{-6.5}{1}{0}{match\textunderscore id}{(D5)}
\erRelationalAttribute{7.01}{-6.8}{1}{0}{side\textunderscore name}{(D5)}
\erCoreAttribute{7.01}{-7.1}{1}{0}{number}{}
\erCoreAttribute{7.01}{-7.4}{1}{1}{name}{}
\eret{0.119}{-11.05}{3.801}{-8.95}{0.2}{1}\eretname{0.487}{-9.3}{l}{delivery}
\erRelationalAttribute{0.319}{-9.5}{1}{0}{match\textunderscore id}{(D6)}
\erRelationalAttribute{0.319}{-9.8}{1}{0}{innings\textunderscore number}{(D6)}
\erRelationalAttribute{0.319}{-10.1}{1}{0}{over\textunderscore number}{(D6)}
\erCoreAttribute{0.319}{-10.4}{1}{0}{number}{}
\erHierarchicalAttribute{0.319}{-10.7}{1}{1}{facingBatter\textunderscore number}{(R4)}
\eret{0}{-0.2}{9.254}{0.3}{0.2}{1}

% relationship 
\errelname{5.05}{-0.5}{l}{}\errelarm{4.9}{-0.2}{4.9}{-0.65}{1}{0}\errelarm{4.9}{-0.65}{4.9}{-1.1}{1}{0}\ercrowfoot{4.9}{-0.95}{4.75}{-1.1}{4.9}{-1.1}{5.05}{-1.1}{0}
% relationship 
\errelname{4.083}{-2.3}{r}{}\errelarm{4.233}{-2}{4.233}{-2.075}{1}{0}\errelarm{1.96}{-2.738}{1.96}{-2.95}{1}{0}\errelid{2.947}{-2.188}{r}{D2}\errelangle{4.233}{-2.075}{4.233}{-2.15}{3.097}{-2.338}{1}{0}\errelangle{3.097}{-2.338}{1.96}{-2.525}{1.96}{-2.738}{1}{0}\eridcomprel{1.86}{2.06}{-2.7}\ercrowfoot{1.96}{-2.8}{1.81}{-2.95}{1.96}{-2.95}{2.11}{-2.95}{0}
% relationship 
\errelname{5.717}{-2.3}{l}{}\errelarm{5.567}{-2}{5.567}{-2.075}{1}{0}\errelarm{8.014}{-2.738}{8.014}{-2.95}{1}{0}\errelid{6.94}{-2.188}{l}{D3}\errelangle{5.567}{-2.075}{5.567}{-2.15}{6.79}{-2.338}{1}{0}\errelangle{6.79}{-2.338}{8.014}{-2.525}{8.014}{-2.738}{1}{0}\eridcomprel{7.91375}{8.11375}{-2.7}\ercrowfoot{8.014}{-2.8}{7.864}{-2.95}{8.014}{-2.95}{8.164}{-2.95}{0}
% relationship 
\errelname{2.11}{-5.05}{l}{}\errelid{2.11}{-5.2}{l}{D4}\errelarm{1.96}{-4.75}{1.96}{-5.35}{1}{0}\errelarm{1.96}{-5.35}{1.96}{-5.95}{1}{0}\eridcomprel{1.86}{2.06}{-5.699999999999999}\ercrowfoot{1.96}{-5.8}{1.81}{-5.95}{1.96}{-5.95}{2.11}{-5.95}{0}
% relationship battingSide
\errelname{3.924}{-3.25}{l}{battingSide}\errelid{5.424}{-3.25}{l}{R1}\erscope{5.024}{-3.7}{l}{d:..=s:..}\errelarm{3.774}{-3.4}{5.274}{-3.4}{1}{0}\errelarm{5.274}{-3.4}{6.774}{-3.4}{0}{0}\ercrowfoot{3.924}{-3.4}{3.774}{-3.25}{3.774}{-3.4}{3.774}{-3.55}{0}
% relationship fieldingSide
\errelname{3.924}{-4.15}{l}{fieldingSide}\errelid{5.424}{-4.15}{l}{R2}\erscope{5.024}{-4.6}{l}{d:..=s:..}\errelarm{3.774}{-4.3}{5.274}{-4.3}{1}{0}\errelarm{5.274}{-4.3}{6.774}{-4.3}{0}{0}\ercrowfoot{3.924}{-4.3}{3.774}{-4.15}{3.774}{-4.3}{3.774}{-4.45}{0}
% relationship 
\errelname{8.164}{-5.05}{l}{}\errelid{8.164}{-5.2}{l}{D5}\errelarm{8.014}{-4.75}{8.014}{-5.35}{1}{0}\errelarm{8.014}{-5.35}{8.014}{-5.95}{1}{0}\eridcomprel{7.91375}{8.11375}{-5.699999999999999}\ercrowfoot{8.014}{-5.8}{7.864}{-5.95}{8.014}{-5.95}{8.164}{-5.95}{0}
% relationship 
\errelname{2.11}{-8.05}{l}{}\errelid{2.11}{-8.2}{l}{D6}\errelarm{1.96}{-7.75}{1.96}{-8.35}{1}{0}\errelarm{1.96}{-8.35}{1.96}{-8.95}{1}{0}\eridcomprel{1.86}{2.06}{-8.7}\ercrowfoot{1.96}{-8.8}{1.81}{-8.95}{1.96}{-8.95}{2.11}{-8.95}{0}
% relationship bowler
\errelname{3.633}{-6.7}{l}{bowler}\errelid{5.296}{-6.7}{l}{R3}\erscope{4.096}{-7.15}{l}{d:..=s:../fieldingSide}\errelarm{3.483}{-6.85}{5.146}{-6.85}{1}{0}\errelarm{5.146}{-6.85}{6.81}{-6.85}{0}{0}\ercrowfoot{3.633}{-6.85}{3.483}{-6.7}{3.483}{-6.85}{3.483}{-7}{0}
% relationship facingBatter
\errelname{3.951}{-10.3}{l}{facingBatter}\errelarm{3.801}{-10}{4.401}{-10}{1}{0}\errelarm{6.61}{-7.57}{6.81}{-7.57}{0}{0}\errelid{5.556}{-8.635}{r}{R4}\erscope{8.256}{-8.935}{r}{d:..=s:../../battingSide}\errelangle{4.401}{-10}{5.001}{-10}{5.706}{-8.785}{1}{0}\errelangle{5.706}{-8.785}{6.41}{-7.57}{6.61}{-7.57}{0}{0}\ercrowfoot{3.951}{-10}{3.801}{-9.85}{3.801}{-10}{3.801}{-10.15}{0}
\end{erdiagram}
} }
& \onslide*<1-1>{Logical or Conceptual} \onslide*<2-2>{Physical...Hierarchical} \onslide*<3-3>{Physical...Relational}
\end{tabular}
\end{frame}
\fi

\begin{frame}{Ideas next}
generalised algebraic functors - left adjoints

generalised algebraic functors - monadic

maybe refer to a contemplation of the absolute the abstract

move final Ryle quote here - is it too grand to suppose that concept-instance algebras might provide such a grammar?
\end{frame}

\end{document}