
\usepackage{mathptmx}
\usepackage{amsfonts}
\usepackage{wasysym}
\usepackage{url}
\usepackage{hyperref}

\newcommand{\SharedMacros}{../../SharedMacros}
\newcommand{\SharedText}{../../SharedText}
%\usepackage{imakeidx}
\makeindex[name=definitions, title=Index of Definitions]
\makeindex[name=lemmas, title=Index of Lemmas]



\newcommand{\commentary}[1]{\marginpar{\footnotesize #1}}
\newcommand{\highlight}[1]{\colorbox{orange}{#1}}
\newcommand{\term}[1]{\textit{#1}\commentary{\colorbox{lightgray}{\textit{#1}}}\index[definitions]{#1}}
\newcommand{\llabel}[1]{\label{#1}\commentary{\colorbox{pink}{\scriptsize{#1}}}\index[lemmas]{#1}}
\newcommand{\lref}[1]{\ref{#1}\colorbox{pink}{\scriptsize{#1}}\index[lemmas]{#1!use of}}

\newcommand{\newt}[1]{\colorbox{yellow}{#1}}
\newenvironment{newtt}
{  \colorbox{yellow}{$[$ ...} 
}
{  \colorbox{yellow}{... $]$}
}
\newcommand{\oldt}[1]{\colorbox{yellow}{\sout{#1}}}
\newenvironment{oldtt}
{  \colorbox{red}{$[$ ...} 
}
{  \colorbox{red}{... $]$}
}

\newcommand{\reinstatet}[1]{\colorbox{lime}{#1}}
\newenvironment{reinstatett}
{  \colorbox{lime}{$[$ ...}
}
{  \colorbox{lime}{... $]$}
}

\newcommand{\tbd}{\highlight{TBD}}

%ithprojection function
\newcommand{\proji}[1]{\pi_#1}



\newenvironment{categoricalaside}
{\begin{framed}
\textbf{Categorical Aside}
}
{
\end{framed}
}

\newenvironment{noteforfuture}
{\begin{framed}
\textbf{Note For Future}
}
{
\end{framed}
}

\newenvironment{problem}
{\begin{framed}
\textbf{Problem}
}
{
\end{framed}
}

%quine quote
\newcommand{\qq}[1]{
\left\ulcorner#1\right\urcorner
}

%single quote
\newcommand{\sq}[1]{
\textnormal{\textquotesingle}#1\textnormal{\textquotesingle}
}

%lower quine quote
\newcommand{\lqq}[1]{
\left\llcorner #1\right\lrcorner
}


%from berkley
\newcommand{\langl}{\begin{picture}(4.5,7)
\put(1.1,2.5){\rotatebox{60}{\line(1,0){5.5}}}
\put(1.1,2.5){\rotatebox{300}{\line(1,0){5.5}}}
\end{picture}}
\newcommand{\rangl}{\begin{picture}(4.5,7)
\put(.9,2.5){\rotatebox{120}{\line(1,0){5.5}}}
\put(.9,2.5){\rotatebox{240}{\line(1,0){5.5}}}
\end{picture}}
\newcommand{\lang}{\begin{picture}(5,7)\put(1.1,2.5){\rotatebox{45}{\line(1,0){6.0}}}\put(1.1,2.5){\rotatebox{315}{\line(1,0){6.0}}}\end{picture}}
\newcommand{\rang}{\begin{picture}(5,7)\put(.1,2.5){\rotatebox{135}{\line(1,0){6.0}}}\put(.1,2.5){\rotatebox{225}{\line(1,0){6.0}}}\end{picture}}
%Try sharper tuple brackets -- except gives errors nested in captions so comment out
%\renewcommand{\tuple}[1]{\lang #1 \rang}

\newcommand{\setsuchthat}[2]{\left\{#1 \ \middle|\ #2\right\}}
\newcommand{\set}[1]{\left\{#1\right\}} 

% one to n - wanton
\newcommand{\wanton}[1]{#1_1,...#1_n}
\newcommand{\fn}{\wanton{f}}
\newcommand{\pn}{\wanton{p}}
\newcommand{\qn}{\wanton{q}}
\newcommand{\qnprime}{\wanton{q'}}
\newcommand{\xn}{\wanton{x}}
\newcommand{\xnp}{\wanton{x'}}
\newcommand{\yn}{\wanton{y}}
\newcommand{\ntuple}[1]{\tuple{\wanton{#1}}}
\newcommand{\wantom}[1]{#1_1,...#1_m}
\newcommand{\mtuple}[1]{\tuple{#1_1,...#1_m}}
\newcommand{\qm}{\wantom{q}}
\newcommand{\ym}{\wantom{y}}
\newcommand {\bntuple}{\ensuremath{\ntuple{b}}}
\newcommand {\fntuple}{\ensuremath{\ntuple{f}}}
\newcommand {\fnptuple}{\ensuremath{\ntuple{f}}}
\newcommand {\pntuple}{\ensuremath{\ntuple{p}}}
\newcommand {\qntuple}{\ensuremath{\ntuple{q}}}
\newcommand {\qnptuple}{\ensuremath{\ntuple{q'}}}
\newcommand {\qmtuple}{\ensuremath{\mtuple{q}}}
\newcommand {\sntuple}{\ensuremath{\ntuple{s}}}
\newcommand {\xntuple}{\ensuremath{\ntuple{x}}}
\newcommand {\xnptuple}{\ensuremath{\ntuple{x'}}}
\newcommand {\ymtuple}{\ensuremath{\mtuple{y}}}
\newcommand{\foreachi}[1][n]{for each $i$, $1 \leq i \leq #1$}
\newcommand{\foreachj}[1][m]{for each $j$, $1 \leq j \leq #1$}
\newcommand{\foreachk}[1][l]{for each $k$, $1 \leq k \leq #1$}

    %causes problems when used with bamer

%ccategories.macros.tex 

% Macros for diagrams in contextual categories and related categories

\usepackage{twoopt}
\usepackage{scalerel} 
\usepackage{xargs}

%\usepackage{mathabx}  %Caused font problems
%\usepackage{MnSymbol}  % caused font problems

\newcommand{\conu}
{\mathbf{C}(U)}

\newcommand{\depu}
{\mathbf{D}(U)}

\newcommand{\cat}[1]{\textbf{#1}}
\newcommand{\obj}[1]{\ensuremath{|\cat{#1}|}}
\newcommand{\ccat}[1][C]{\ensuremath{\mathbb{#1}} }
\newcommand{\ccatc}{contextual category \ccat}
\newcommand{\cobj}[2][]{\ensuremath{|\ccat[#2]|_{#1}}}
\newcommand{\cslice}[2]{\ensuremath{\ccat[#1]_{#2}}}
\newcommand{\csliceobj}[3][]{\ensuremath{|\mathbb{#2}_{#3}|_{#1} }}
\newcommand{\varset}[1][]{\ensuremath{V_{#1} }}
\newcommand{\localvarsets}{\ensuremath{\mathcal{V} }}
\newcommand{\Fam}{\ensuremath{\mathbb{F\mathrm{am}} }}
\newcommand{\Famslice}[1]{\ensuremath{\mathbb{F\mathrm{am}}_{#1} }}
\newcommand{\Famobj}[1][]{\ensuremath{|\mathbb{F\mathrm{am}}|_{#1} }}
\newcommand{\Famsliceobj}[2][]{\ensuremath{|\mathbb{F\mathrm{am}}_{#2}|_{#1} }}
\newcommand{\morph}{\rightarrow}
\newcommand{\epi}{\twoheadrightarrow}
\newcommand{\base}{\triangleleft}
\newcommand{\comp}{\circ}
\newcommand{\cross}{\otimes}
\newcommand{\pc}[2]{d^{#1}_{#2}}
\newcommand{\sub}{^*}
\newcommand{\diag}{\delta}
\newcommand{\pbase}[1]{\tilde{#1}}

\newcommand{\tuple}[1]{\langle#1\rangle}
\newcommand{\ndidly}{\ensuremath{\Join_n}}
\newcommand{\ndidlycospan}{quotiented n-cospan}

\newcommand{\crossx}[3]{#1 \underset{#3}{\cross} #2}
\newcommand{\fibrex}[3]{#1 \underset{#3}{\Join} #2}
\newcommand{\powerset}{\mathcal{P}}
\newcommand{\primeds}[1]{
\ensuremath{\mathcal{P}(#1)} }
\newcommand{\compset}{\ \dot{\circ}\, }

% darrow
%\newcommand{\darrow}{\rightarrowtriangle} %use \smorph instead
\newcommand{\smorph}{\rightarrowtriangle}

 

\newcommand\dhead{\scaleobj{0.6}{\triangleright}}
\newcommand{\dmorph}{\, \mbox{---} \! \cdot \! \raisebox{1.1pt}{\dhead}}

% projection tree
%\newcommand{\proj}[2]{proj_{#2}(#1)}

\newcommand{\proj}[2]{
\ensuremath{\mathcal{P}_{#2}(#1)} }

%pstrick supplements for arrows

\newlength{\arrnodesepA}
\newlength{\arrnodesepB}
\newlength{\arroffsetA}
\newlength{\arroffsetB}

%Modified to 2pt from 0pt on 23 July 2018
\newcommand{\arreset}{
\setlength{\arrnodesepA}{2pt}
\setlength{\arrnodesepB}{2pt}
\setlength{\arroffsetA}{0pt}
\setlength{\arroffsetB}{0pt}
}
\arreset

\newcommand{\ncarr}[3][0]{\ncarc[arcangle=#1,nodesepA=\arrnodesepA,nodesepB=\arrnodesepB,offsetA=\arroffsetA,offsetB=\arroffsetB,arrowsize=5pt,arrowinset=0.7]{->}{#2}{#3}}
\newcommand{\jcbarr}[4][0]{ % ncbarr is defined in some thridy party package so do not use!\emph{}
\ncarr[#1]{#3}{#4}
\nbput[labelsep=2pt]{\footnotesize $#2$}
}

\newcommand{\ncaarr}[4][0]{
\ncarr[#1]{#3}{#4}
\naput[labelsep=2pt]{\footnotesize $#2$}
}

% \alabel{label}[npos][labelsep_pts]
\newcommandx*\alabel[3][2=0.5,3=2,usedefault]{\naput[labelsep=#3pt,npos=#2]{\footnotesize $#1$}}
% \blabel{label}[npos][labelsep_pts]
\newcommandx*\blabel[3][2=0.5,3=2,usedefault]{\nbput[labelsep=#3pt,npos=#2]{\footnotesize $#1$}}

% \idcomp mark an arrow as one component of an identifier
\newcommand{\idcomp}{\ncput[npos=0, nrot=:U]{\psline(0.2,-0.075)(0.2,0.075)}}  %add a bar to a node connection arrow
% pstrick supplements for s-arrows (previous name for d-arrow - should convert}

\newlength{\sarnodesepA}
\newlength{\sarnodesepB}
\newlength{\saroffsetA}
\newlength{\saroffsetB}
\newlength{\sarnodesepAsav}
\newlength{\sarnodesepBsav}

\newcommand{\sarreset}{
\setlength{\sarnodesepA}{0pt}
\setlength{\sarnodesepB}{0pt}
\setlength{\saroffsetA}{0pt}
\setlength{\saroffsetB}{0pt}
}

\sarreset

% sar - S-arrow
\newcommand{\ncsar}[3][0]{
\setlength{\sarnodesepAsav}{\sarnodesepA}
\setlength{\sarnodesepBsav}{\sarnodesepB}
\addtolength{\sarnodesepA}{3pt}
\addtolength{\sarnodesepB}{7pt}
\ncarc[nodesepA=\sarnodesepA,nodesepB=\sarnodesepB,offsetA=\saroffsetA,offsetB=\saroffsetB,arcangle=#1]{-}{#2}{#3}
\ncput[nrot=:R,npos=1]{\pstriangle(0,0)(.2,.2)}
\setlength{\sarnodesepA}{\sarnodesepAsav}
\setlength{\sarnodesepB}{\sarnodesepBsav}
}


% bsar - below labelled S-arrow
\newcommand{\ncbsar}[4][0]{
\ncsar[#1]{#3}{#4}
\nbput[labelsep=2pt]{\footnotesize $#2$}
}
% asar - above labelled S-arrow
\newcommand{\ncasar}[4][0]{
\ncsar[#1]{#3}{#4}
\naput[labelsep=2pt]{\footnotesize $#2$}
}

% cdar - composite dependency arrow
\newcommand{\nccdar}[3][0]{
\setlength{\sarnodesepAsav}{\sarnodesepA}
\setlength{\sarnodesepBsav}{\sarnodesepB}
\addtolength{\sarnodesepA}{3pt}
\addtolength{\sarnodesepB}{11pt}
\ncarc[nodesepA=\sarnodesepA,nodesepB=\sarnodesepB,offsetA=\saroffsetA,offsetB=\saroffsetB,arcangle=#1]{-}{#2}{#3}
\ncput[nrot=:R,npos=1]{\pstriangle(0,0.1)(.2,.2)}
\ncput[nrot=:R,npos=1]{\psdot[dotsize=1pt](-0.0075,0.05)}   %!!
\setlength{\sarnodesepA}{\sarnodesepAsav}
\setlength{\sarnodesepB}{\sarnodesepBsav}
}


% bcdar - below labelled composite dependency arrow
\newcommand{\ncbcdar}[4][0]{
\nccdar[#1]{#3}{#4}
\nbput[labelsep=2pt]{\footnotesize $#2$}
}
% acdar - above labelled composite dependency arrow
\newcommand{\ncacdar}[4][0]{
\nccdar[#1]{#3}{#4}
\naput[labelsep=2pt]{\footnotesize $#2$}
}


% rsar - recursive S-arrow
\newcommand{\ncrsar}[2]{
\setlength{\sarnodesepAsav}{\sarnodesepA}
\setlength{\sarnodesepBsav}{\sarnodesepB}
\addtolength{\sarnodesepA}{3pt}
\addtolength{\sarnodesepB}{7pt}
\ncloop[nodesepA=\sarnodesepA,nodesepB=\sarnodesepB,
        offsetA=\saroffsetA,offsetB=\saroffsetB,
        armA=0.7cm,armB=0.6cm,angleA=90,angleB=-90,loopsize=-1,linearc=0.4
				]{-}{#1}{#2}
\ncput[nrot=:R,npos=5]{\pstriangle(0,0)(.2,.2)}
\setlength{\sarnodesepA}{\sarnodesepAsav}
\setlength{\sarnodesepB}{\sarnodesepBsav}
}

% pstrick supplements for multi-arrows

\newlength{\marnodesepA}
\newlength{\marnodesepB}
\newlength{\maroffsetB}
\newlength{\marnodesepBsav}

\newcommand{\marreset}{
\setlength{\marnodesepA}{0pt}
\setlength{\marnodesepB}{0pt}
\setlength{\maroffsetB}{0pt}
}

\marreset

%ncmarr[#1 arcangle1][#2 arcangle2]{#3 name}{#4 domain1}{#5 domain2}{#6 junction}{#7 codomain}
\newcommandtwoopt{\ncmarr}[6][8][8]{%
\ncarc[nodesepA=\marnodesepA,nodesepB=0,arcangle=#1]{-}{#3}{#5}
\ncarc[nodesepB=0,arcangle=-#1]{-}{#4}{#5}
\ncarc[arcangle=#2,nodesepB=\marnodesepB,offsetB=\maroffsetB]{->}{#5}{#6}
}%


\newcommandtwoopt{\nchmarr}[6][8][8]{%
\ncarc[nodesepA=\marnodesepA,nodesepB=0,arcangle=#1]{-}{#3}{#5}
\ncarc[nodesepB=0,arcangle=#1]{-}{#4}{#5}
\ncarc[arcangle=#2,nodesepB=\marnodesepB,offsetB=\maroffsetB]{->}{#5}{#6}
}%

\newcommandtwoopt{\ncamarr}[7][8][8]{%
\ncmarr[#1][#2]{#4}{#5}{#6}{#7}
\naput[npos=.05]{$#3$}
}%
\newcommandtwoopt{\ncbmarr}[7][8][8]{%
\ncmarr[#1][#2]{#4}{#5}{#6}{#7}
\nbput[npos=.05]{$#3$}
}%

\newcommandtwoopt{\ncbhmarr}[7][8][8]{%
\nchmarr[#1][#2]{#4}{#5}{#6}{#7}
\nbput[npos=.05]{$#3$}
}%

\newcommandtwoopt{\ncmarrr}[7][8][8]{
\ncarc[nodesepB=0,arcangle=#1]{-}{#3}{#6}
\ncline[nodesepB=0]{-}{#4}{#6}
\ncarc[nodesepB=0,arcangle=-#1]{-}{#5}{#6}
\ncarc[nodesepA=0,arcangle=#2]{->}{#6}{#7}
}

\newcommandtwoopt{\ncamarrr}[8][8][8]{
\ncmarrr[#1][#2]{#4}{#5}{#6}{#7}{#8}
\naput[npos=.05]{$#3$}
}
\newcommandtwoopt{\ncbmarrr}[8][8][8]{
\ncmarrr[#1][#2]{#4}{#5}{#6}{#7}{#8}
\nbput[npos=.05]{$#3$}
}

%gats.macros.tex

\usepackage{environ}    % also used in ermacros % here used for \NewEnvrion

\newcommand{\gat}[1][U]{
\ensuremath{\mathcal{#1}}}  % used to hav a space in here
\newcommand{\gatw}[1][U]{\gat[#1]\ }  % use this if need trailing space
\newcommand{\ingat}[1][U]{in \gat[#1]}
\newcommand{\isagat}[1][U]{\gat[#1] is a g.a.t.}
\newcommand{\inagat}{in a g.a.t. }

% macro for a generic theory
%\newcommand{\theory}
%{\textit{U}}

\newcommand{\intheory}
{is a derived rule of \gat[U]}

% Macros for GAT rules

\newcommand{\isT}[1]
{#1\mbox{ is a type}}

\newcommand{\ofT}[2]
{#1 \in #2
}

% Macros for GAT rules   <!-- new old -->
\newcommand{\istype}[1]
{#1\mbox{ is a type}}

\newcommand{\oftype}[2]
{#1 \in #2
}

%\context{x}{\Delta}{n}
\newcommand{\context}[3]
{\ofT{#1_1}{#2_1},... \ofT{#1_{#3}}{#2_{#3}(#1_1,...#1_{#3-1})}
}

%\subcontext{x}{\Delta}{i}{k}
\newcommand{\subcontext}[4]
{\ofT{#1_{#3_1}}{#2_{#3_1}},... \ofT{#1_{#3_#4}}{#2_{#3_#4}(#1_1,...#1_{#3_#4-1})}
}

% #schematic context
\newcommand{\schmcon}[3]
{\ofT{#1_1}{#2_1},... \ofT{#1_{#3}}{#2_{#3}}
}
% abbreviated to
\newcommand{\con}[3]
{\schmcon{#1}{#2}{#3}}

% schematic subcontext
%\subcon{x}{\Delta}{i}{k}
\newcommand{\subcon}[4]
{\ofT{#1_{#3_1}}{#2_{#3_1}},... \ofT{#1_{#3_#4}}{#2_{#3_#4}}
}

% permuted context
%\permcon{x}{\Delta}{n}{\sigma}
\newcommand{\permcon}[4]
{\ofT{#1_{#4(1)}}{#2_{#4(1)}},... \ofT{#1_{#4(#3)}}{#2_{#4(#3)}}
}
% permuted term
%\permterm{t}{n}{\sigma}
\newcommand{\permterm}[3]
{
#1_{#3(1)},...#1_{#3(#2)}
}


% Idioms
\newcommand{\xDelta}[1]{\con{x}{\Delta}{#1}}
\newcommand{\xDeltap}[1]{\con{x}{\Delta'}{#1}}
\newcommand{\xOmega}[1]{\con{x}{\Omega}{#1}}
\newcommand{\xOmegap}[1]{\con{x}{\Omega'}{#1}}
\newcommand{\yOmega}[1]{\con{y}{\Omega}{#1}}
\newcommand{\yOmegap}[1]{\con{y}{\Omega'}{#1}}

\newcommand{\xDeltasigma}[1]{\permcon{x}{\Delta}{#1}{\sigma}}
\newcommand{\xDeltapsigma}[1]{\permcon{x}{\Delta'}{#1}{\sigma}}
\newcommand{\xOmegasigma}[1]{\permcon{x}{\Omega}{#1}{\sigma}}
\newcommand{\xOmegapsigma}[1]{\permcon{x}{\Omega'}{#1}{\sigma}}
\newcommand{\yOmegasigma}[1]{\permcon{y}{\Omega}{#1}{\sigma}}
\newcommand{\yOmegapsigma}[1]{\permcon{y}{\Omega'}{#1}{\sigma}}

\newcommand{\xDeltainvsigma}[1]{\permcon{x}{\Delta}{#1}{\sigma^{-1}}}
\newcommand{\xDeltapinvsigma}[1]{\permcon{x}{\Delta'}{#1}{\sigma^{-1}}}
\newcommand{\xOmegainvsigma}[1]{\permcon{x}{\Omega}{#1}{\sigma^{-1}}}
\newcommand{\xOmegapinvsigma}[1]{\permcon{x}{\Omega'}{#1}{\sigma^{-1}}}
\newcommand{\yOmegainvsigma}[1]{\permcon{y}{\Omega}{#1}{\sigma^{-1}}}
\newcommand{\yOmegapinvsigma}[1]{\permcon{y}{\Omega'}{#1}{\sigma^{-1}}}

%Idioms enclosed as tuples
\newcommand{\encxDelta}[1]{\tuple{\con{x}{\Delta}{#1}}}
\newcommand{\encxDeltap}[1]{\tuple{\con{x}{\Delta'}{#1}}}
\newcommand{\encxOmega}[1]{\tuple{\con{x}{\Omega}{#1}}}
\newcommand{\encxOmegap}[1]{\tuple{\con{x}{\Omega'}{#1}}}
\newcommand{\encyOmega}[1]{\tuple{\con{y}{\Omega}{#1}}}
\newcommand{\encyOmegap}[1]{\tuple{\con{y}{\Omega'}{#1}}}

\newcommand{\encxDeltasigma}[1]{\tuple{\permcon{x}{\Delta}{#1}{\sigma}}}
\newcommand{\encxDeltapsigma}[1]{\tuple{\permcon{x}{\Delta'}{#1}{\sigma}}}
\newcommand{\encxOmegasigma}[1]{\tuple{\permcon{x}{\Omega}{#1}{\sigma}}}
\newcommand{\encxOmegapsigma}[1]{\tuple{\permcon{x}{\Omega'}{#1}{\sigma}}}
\newcommand{\encyOmegasigma}[1]{\tuple{\permcon{y}{\Omega}{#1}{\sigma}}}
\newcommand{\encyOmegapsigma}[1]{\tuple{\permcon{y}{\Omega'}{#1}{\sigma}}}

\newcommand{\encxDeltainvsigma}[1]{\tuple{\permcon{x}{\Delta}{#1}{\sigma^{-1}}}}
\newcommand{\encxDeltapinvsigma}[1]{\tuple{\permcon{x}{\Delta'}{#1}{\sigma^{-1}}}}
\newcommand{\encxOmegainvsigma}[1]{\tuple{\permcon{x}{\Omega}{#1}{\sigma^{-1}}}}
\newcommand{\encxOmegapinvsigma}[1]{\tuple{\permcon{x}{\Omega'}{#1}{\sigma^{-1}}}}
\newcommand{\encyOmegainvsigma}[1]{\tuple{\permcon{y}{\Omega}{#1}{\sigma^{-1}}}}
\newcommand{\encyOmegapinvsigma}[1]{\tuple{\permcon{y}{\Omega'}{#1}{\sigma^{-1}}}}

\newcommand{\tstyle}{\vdash}

% gat macros developed for cwf paper

% Expressing gats
\newenvironment{gatrules}
{
$$
\begin{array}{l l}
}
{
\end{array}
$$
}
\newcommand{\gatintros}
{
\textbf{Symbol} & \textbf{Introductory\ Rule}                      \\}

\newcommand{\gataxioms}
{\textbf{Axioms}\\}
\newcommand{\gatintro}[3]{\ #1 & #2 \tstyle #3 \\}
\newcommand{\gatlocalintro}[3]{\ #1 & #2 \dashv }
\newcommand{\gataxiom}[2]{\multicolumn{2}{l}{\ \ #1\mbox{,  whenever\ } #2} \\}
\newcommand{\noleft}{\left.\kern-\nulldelimiterspace} % so that no space taken by absent left brace


\newcommand{\gatmultiaxiom}[2]
{\multicolumn{2}{l}{
  \noleft
    \begin{array}{l}
		#1
    \end{array} 
  \right\} \mbox{whenever\ } 	#2 
	}\\}
	
	\newcommand{\axid}[1]{\text{#1}.\ }	

%New context sharing macros
\newcommand{\gatintroducing}[1]{
{\arraycolsep=0pt
  \begin{array}{l}
          #1
  \end{array}} &
}

%*********************************
% \begin{\gatgroup}{context}
%    rules
%  \end{\gatgroup}
%*********************************
\NewEnviron{gatgroup}[1]{%
  \noleft
  {\arraycolsep=0pt
   \begin{array}{l}
\BODY
    \end{array} 
   }
   \ \right\} 
	%\mbox{\ whenever\ } 
	#1
	\vspace{0.1cm} 
}
%*********************************

%*********************************
% \begin{\gatgroupnoshared}
%    rule
%  \end{\gatgroupnoshared}
%*********************************
\NewEnviron{gatgroupnoshared}{%
  {\arraycolsep=0pt
   \begin{array}{l}
\BODY
    \end{array} 
   }
   \ 
	\vspace{0.1cm} 
}
%*********************************

% \gatsingular[width]{context}{conclusion}
\newcommand{\gatsingular}[3][4cm]{
\begin{gatgroupnoshared}
\gatleaf[#1]{#2}{#3} 
\end{gatgroupnoshared}
}

%*********************************
% \gatleaf}[width]{context}{assertion}
%*********************************
\newcommand{\gatleaf}[3][4cm]{%
\makebox[#1]{$#3$ \dotfill} \dotfill \  #2
}
%*********************************
%*********************************
% \gatstandalonesingle}{context}{assertion}
%*********************************
\newcommand{\gatstandalonesingle}[2]{%
#2 \makebox[2.5cm]{\dotfill} \  #1
}
%*********************************

% \gataxiomno{axiomno}
\newcommand{\gataxiomno}[1]{\makebox[0.5cm]{} \axid{#1}}


% metagat.macros.tex

%Meta-theories

%\newcommand{\typ}{\triangleright}
\newcommand{\typ}{\nabla}
\newcommand{\trm}{\tau}
\newcommand{\cross}{\otimes}
\newcommand{\sub}{^*}
\newcommand{\diag}{\delta}

\newcommand{\typeseq}[2]
{\ofT{#1_1}{\typ},... \ofT{#1_{#2}}{\typ(#1_{#2-1})}}

\newcommand{\typeseqcont}[3]
{\ofT{#1_1}{\typ({#2})},... \ofT{#1_{#3}}{\typ(#1_{#3-1})}}

\newcommand{\Ob}{Ob}
\newcommand{\obj}{Ob} % <!-- new old --<
\newcommand{\Hom}{Hom}
\newcommand{\objseq}[2]
{\ofT{#1_1}{\obj},... \ofT{#1_{#2}}{\obj(#1_{#2-1})}}


\def\dottededge{\ncline[linestyle=dotted, nodesep=0.3cm]}
\def\noedge{\ncline[linestyle=none]}
\def\thinedge{\ncline[linewidth=0.4pt]}

\newcommand{\member}[1]
{\ncarc[arcangle=-30,nodesepB=0.03]{->}{\pspred}{\pssucc}
\nbput[labelsep=0.1]{#1}}

\newcommand{\loweraccutemember}[1]
{\ncarc[arcangle=-15,nodesepB=0.03]{->}{\pspred}{\pssucc}
\nbput[labelsep=0.05,npos=0.85]{#1}}

\newcommand{\uppermember}[1]
{\ncarc[arcangle=30,nodesepB=0.03]{->}{\pspred}{\pssucc}\naput{#1}}

\newcommand{\upperaccutemember}[1]
{\ncarc[arcangle=10,nodesepB=0.03]{->}{\pspred}{\pssucc}\naput[npos=0.85]{#1}}

% flexbranch 
% #1 node label
% #2 thislevelsep
% #3 next level sep
% #4 variable (eg x)
% #5 index leter (eg n)
% #6 close parenthesis
% #7 continuation branches
\newcommand{\flexbranch}[7]
{
\pstree[thislevelsep=*#2,nodesep=0.05]
		{\Rnode{#1 1}{\Tr{#4_1 #6}}}
	  {\pstree[thislevelsep=#3]  
				   {\Rnode{#1 2}{\Tr[edge=\dottededge]{#4_{#5} #6}}}
					 {#7}
		}
}

\newcommand{\flexbranchplusleaf}[6]
{
\flexbranch{#1}{#2}{#3}{#4} {#5} {#6}
  {
   %\Rnode{#1 3}{\Tr{#4 #6}}
	 \Tr{\Rnode{#1 3}{#4 #6}}
  }
}

\newcommand{\flexbranchplusarc}[7]
{
\flexbranch{#1}{#2}{#3}{#4} {#5} {#6}
  {
   %\Rnode{#1 3}{\Tr{#4 #6}\member{#7}}
	 \Tr{\Rnode{#1 3}{#4 #6}}\member{#7}
  }
}

\newcommand{\flexbranchinitialarc}[9]
{
\pstree[thislevelsep=*#2,nodesep=0.05]
		{\Rnode{#1 1}{\Tr{#4_#8 #6}}#9}
	  {\pstree[thislevelsep=#3]  
				   {\Rnode{#1 2}{\Tr[edge=\dottededge]{#4_{#5} #6}}}
					 {#7}
		}
}

\newcommand{\equality}[2]
{
\ncline [doubleline=true, nodesep=0.2cm]{#1}{#2}
}
\newcommand{\equalityarc}[2]
{
\ncarc [arcangleA=-30, arcangleB=-20, doubleline=true, nodesep=0.1cm]{#1}{#2}
}

\usepackage[margin=4.0cm]{geometry} %was 3cm
\usepackage{mathptmx}
\usepackage{amsfonts}
\usepackage{array}
\usepackage{pstricks}
\usepackage{pst-tree}
\usepackage{pst-plot}
\usepackage{pst-node}
\usepackage{stmaryrd}
\usepackage{amsmath}
\usepackage{verbatim}
\usepackage{graphicx}  
\usepackage{calc}
\usepackage{xifthen}
\usepackage{xcolor}
\usepackage{color}
\usepackage{stringstrings}
%\usepackage[small,bf,margin=3pt,format=hang, labelsep=endash,singlelinecheck=false]{caption} %prevuiously justification=justified
%\usepackage{enumerate}
%\usepackage{enumitem}
\usepackage{enumerate}
\usepackage[shortlabels]{enumitem}
\usepackage{float}
\usepackage[section]{placeins}
%\setlength{\captionmargin}{5pt}
\usepackage{environ}
\usepackage{multirow}
\usepackage{rotating}
\usepackage{longtable}
\usepackage{afterpage}
\usepackage{needspace}


%DEFINE ENVIRONMENT BLOCK
% Riddle
\newsavebox{\riddlebox}

\newenvironment{erexample}
{\newcommand\colboxcolor{F0F0F0}%was F8F8F8
\begin{lrbox}{\riddlebox}
\begin{minipage}{\dimexpr\columnwidth-2\fboxsep\relax} \textbf{} \\ \itshape}
{\end{minipage}\end{lrbox}%
%\begin{center}
\colorbox[HTML]{\colboxcolor}{\usebox{\riddlebox}}
%\end{center}
}

\newenvironment{erbox}
{\newcommand\colboxcolor{F0F0F0}%was F8F8F8
\begin{lrbox}{\riddlebox}%
\begin{minipage}{\dimexpr\columnwidth-2\fboxsep\relax} }
{\end{minipage}\end{lrbox}%
%\begin{center}
\colorbox[HTML]{\colboxcolor}{\usebox{\riddlebox}}
%\end{center}
}

%\begin{erboxedFigure}{#1 FigureParam}{#2 Label}{#3 Caption}
\NewEnviron{erboxedFigure}[3]{%
\begin{figure}[#1]
\begin{erexample}
\begin{center}
\BODY
\end{center}
\vspace{-0.5cm}
\caption{#3}
\label{#2}
\end{erexample}
\end{figure}
}

\newcommand{\erpictureFolder}[0]{../SharedPictures}

\newcommand{\ercenterPicture}[1]{
\begin{center}
\input{\erpictureFolder/#1}
\end{center}
}


\newlength{\erhalfHt}

%\erinlinePicture{#1 pictureFilename}{#2 pictureHeight}
\newcommand{\erinlinePicture}[2]{
\setlength{\erhalfHt}{#2cm * \real{0.5}}
\raisebox{-\erhalfHt}[\erhalfHt + 0.5cm][\erhalfHt + 0.5cm]{
\input{\erpictureFolder/#1}
} 
}

%\erplainFig{#1 pictureFilename}{#2 figureParam}{#3Caption}
\newcommand{\erplainFig}[3]{
\begin{figure}[#2]
\begin{center}
\input{\erpictureFolder/#1}
\end{center}
\caption{#3}
\label{#1}
\end{figure}
}

%\erboxedFigPicture{#1 pictureFilename}{#2 figureParam}{#3Caption}
\newcommand{\erboxedFigPicture}[3]{
\begin{figure}[#2]
\begin{erexample}
\vspace{-0.5cm}
\begin{center}
\input{\erpictureFolder/#1}
\end{center}
\caption{#3}
\label{#1}
\end{erexample}
\end{figure}
}

%\erLeftSideFig{#1 pictureFilename}{#2 figureParam}{#3Caption}
\newcommand{\erLeftSideFig}[3]{
\begin{figure}[#2]
\begin{erexample}
  \begin{minipage}[c]{0.4\textwidth}
    \caption{#3}
    \label{#1}
  \end{minipage}
  \begin{minipage}[c]{0.5\textwidth}
    \input{\erpictureFolder/#1}
  \end{minipage}
\end{erexample}
\end{figure}
}

%\erbulletedFig{#1 pictureFilename}{#2 figureParam}{#3Caption}
\NewEnviron{erbulletedFig}[3]{%
\begin{figure}[#2]
\begin{erexample}
\vspace{-0.5cm}
\begin{center}
$
\begin{array}{c m{0.25cm} | m{6cm}}
\raisebox{-2.0cm}{
\input{\erpictureFolder/#1}}& & \text{\parbox{6cm}{\raggedright{\footnotesize{
\begin{enumerate}[(i)]
\BODY
\end{enumerate}}}}} \\
\end{array}
$
\end{center}
\caption{#3}
\label{#1}
\end{erexample}
\end{figure} 
}


%\begin{erbulletedDimFig}{#1 pictureFilename}{#2figureParam} {#3Caption} {#4PictureHeight}{#5TextWidth}

\NewEnviron{erbulletedDimFig}[5]{%
\begin{figure}[#2]
\begin{erexample}
\vspace{-0.5cm}
\begin{center}
$
\begin{array}{c m{0.25cm} |  m{#5cm}}
\setlength{\erhalfHt}{#4cm * \real{0.5}}
\raisebox{-\erhalfHt}{
\input{\erpictureFolder/#1}}& & \text{\parbox{#5cm}{\raggedright{\footnotesize{
\begin{enumerate}[(i)]
\BODY
\end{enumerate}}}}} \\
\end{array}
$
\end{center}
\caption{#3}
\label{#1}
\end{erexample}
\end{figure} 
}

%\begin{ernotedModel}{#1 pictureFilename}{#2PictureHeight}{#3PictureWidth}{#4TextWidth}

\NewEnviron{ernotedModel}[4]{%
\begin{center}
$
\begin{array}{m{#3cm} m{1cm} | c m{#4cm}}
\setlength{\erhalfHt}{#2cm * \real{0.5}}
\raisebox{-\erhalfHt}{
\input{\erpictureFolder/#1}}& & & \text{\parbox{#4cm}{\raggedright{\footnotesize{
\BODY
}}}} \\
\end{array}
$
\end{center} 
}

%\begin{ermodelText}{#1 pictureFilename}{#2PictureHeight}{#3PictureWidth}{#4TextWidth}

\NewEnviron{ermodelText}[4]{%
\begin{center}
\begin{tabular}{m{#3cm} m{1cm}  c m{#4cm}}
\setlength{\erhalfHt}{#2cm * \real{0.5}}
\raisebox{-\erhalfHt}{
\input{\erpictureFolder/#1}}& & & \text{\parbox{#4cm}{\raggedright{\small{
\BODY
}}}} \\
\end{tabular}
\end{center} 
}


%\erbulletedModel{#1 pictureFilename}{#2PictureHeight}{#3PictureWidth}{#4TextWidth}

\NewEnviron{erbulletedModel}[4]{%
\begin{center}
$
\begin{array}{m{#3cm} m{1cm} | c m{#4cm}}
\setlength{\erhalfHt}{2cm * \real{0.5}}
\raisebox{-\erhalfHt}{
\input{\erpictureFolder/#1}}& & & \text{\parbox{#4cm}{\raggedright{\footnotesize{
\begin{enumerate}[(i)]
\BODY
\end{enumerate}}}}} \\
\end{array}
$
\end{center} 
}



%\ernotedDimFig{#1 pictureFilename}{#2 figureParam}{#3Caption}{#4PictureHeight}{#5TextWidth}
\NewEnviron{ernotedDimFig}[5]{%
\begin{figure}[#2]
\begin{erexample}
\vspace{-0.5cm}
\begin{center}
$
\begin{array}{c m{0.25cm} | c m{#5cm}}
\setlength{\erhalfHt}{#4cm * \real{0.5}}
\raisebox{-\erhalfHt}{
\input{\erpictureFolder/#1}}& & & \text{\parbox{#5cm}{\raggedright{\footnotesize{
\BODY }}}}\\
\end{array}
$
\end{center}
\caption{#3}
\label{#1}
\end{erexample}
\end{figure} 
}
%\begin{ernotedDimFigPW}{#1 pictureFilename}{#2 figureParam}{#3Caption}{#4PictureHeight}{#5PictureWidth}{#6TextWidth}
\NewEnviron{ernotedDimFigPW}[6]{%
\begin{figure}[#2]
\begin{erexample}
\vspace{-0.5cm}
\begin{center}
$
\begin{array}{>{\centering}m{#5cm} m{0.5cm} | c m{#6cm}}
\setlength{\erhalfHt}{#4cm * \real{0.5}}
\raisebox{-\erhalfHt}{
\centering \input{\erpictureFolder/#1}
}& & & \text{\parbox{#6cm - 0.5cm}{\raggedright{\footnotesize{
\BODY }}}}\\
\end{array}
$ \\
\vspace {0.2cm}
\end{center}
\caption{#3}
\label{#1}
\end{erexample}
\end{figure}
}



\newenvironment{erquote}
{\begin{quote}\itshape}
{\end{quote}}


%
%  erdiag
%
  
%\begin{erdiagram}{#1 height}{#2 width} 
% ....
% ....
%\end{erdiagram}
\newenvironment{erdiagram}[2]
{%\pspicture*(-#1,0)(#2,0)
\pspicture*(0,-#1)(#2,0)
%\psgrid
}
{\endpspicture}

\definecolor{lightyellow}{cmyk}{0,0,0.3,0}

% \eret{#1 x0} {#2 y0} {#3 x1} {#4 y1} {#5 corner radius} {#6 fill}
\newcommand {\eret}[6]
{ 
\ifthenelse{\equal{#6}{1}}
{\psframe[framearc=#5,fillstyle=solid,fillcolor=lightyellow](#1,#2)(#3,#4)}
{\psframe[framearc=#5,fillstyle=solid,fillcolor=white](#1,#2)(#3,#4)}
}

% et top 
\newcommand {\erettop}[4]
{
%\psframe[linestyle=none,linearc=2pt,cornersize=absolute,fillstyle=solid,fillcolor=lightyellow](#1,#2)(#3,#4)
\psline[linearc=2pt,fillstyle=none,fillcolor=lightyellow](#1,#4)(#1,#2)(#3,#2)(#3,#4)
}

% et bottom 
\newcommand {\eretbtm}[4]
{
%\psframe[linestyle=none,linearc=2pt,cornersize=absolute,fillstyle=solid,fillcolor=lightyellow](#1,#2)(#3,#4)
\psline[linearc=2pt,fillstyle=none,fillcolor=lightyellow](#1,#2)(#1,#4)(#3,#4)(#3,#2)
}

% et bottom left
\newcommand {\eretbl}[4]
{
%\psframe[linestyle=none,linearc=2pt,cornersize=absolute,fillstyle=solid,fillcolor=lightyellow](#1,#2)(#3,#4)
\psline[linearc=2pt,fillstyle=none,fillcolor=lightyellow](#1,#4)(#3,#4)(#3,#2)
}

% et middle left
\newcommand {\eretml}[4]
{
%\psframe[linestyle=none,linearc=2pt,cornersize=absolute,fillstyle=solid,fillcolor=lightyellow](#1,#2)(#3,#4)
\psline[linearc=2pt,fillstyle=none,fillcolor=lightyellow](#1,#2)(#3,#2)(#3,#4)(#1,#4)
}

% et top left
\newcommand {\erettl}[4]
{
%\psframe[linestyle=none,linearc=2pt,cornersize=absolute,fillstyle=solid,fillcolor=lightyellow](#1,#2)(#3,#4)
\psline[linearc=2pt,fillstyle=none,fillcolor=lightyellow](#1,#2)(#3,#2)(#3,#4)
}

% et bottom right
\newcommand {\eretbr}[4]
{
%\psframe[linestyle=none,linearc=2pt,cornersize=absolute,fillstyle=solid,fillcolor=lightyellow](#1,#2)(#3,#4)
\psline[linearc=2pt,fillstyle=none,fillcolor=lightyellow](#1,#2)(#1,#4)(#3,#4)
}

% et middle right
\newcommand {\eretmr}[4]
{
%\psframe[linestyle=none,linearc=2pt,cornersize=absolute,fillstyle=solid,fillcolor=lightyellow](#1,#2)(#3,#4)
\psline[linearc=2pt,fillstyle=none,fillcolor=lightyellow](#3,#4)(#1,#4)(#1,#2)(#3,#2)
}

% et top right
\newcommand {\erettr}[4]
{
%\psframe[linestyle=none,linearc=2pt,cornersize=absolute,fillstyle=solid,fillcolor=lightyellow](#1,#2)(#3,#4)
\psline[linearc=2pt,fillstyle=none,fillcolor=lightyellow](#1,#4)(#1,#2)(#3,#2)
}

% \ergrp{#1 x0} {#2 y0} {#3 x1} {#4 y1} {#5 corner radius} {#6 fill}
% #5 corner radius is unused!
\newcommand {\ergrp}[6]
{ 
\ifthenelse{\equal{#6}{1}}
{\psframe[fillstyle=solid,fillcolor=lightgray](#1,#2)(#3,#4)}
{\psframe[fillstyle=solid,fillcolor=white](#1,#2)(#3,#4)}
}

% \eretname {#1 x left of text} {#2 y top of text} {#3 text}
\newcommand {\eretname}[3]
{
%shift down 0.1 for height of text the anchor at baseline (B)
\rput[bl]{0}(0,-0.1){\rput[Bl]{0}(#1,#2){\footnotesize \textit{#3}}}
}

% \errelarm {#1 x0} {#2 y0} {#3 x1} {#4 y1} {#5 ismandatory} {#6 isconstructed}
\newcommand {\errelarm}[6]
{
\ifthenelse{\equal{#6}{1}}
{
%%\psline[linewidth=0.5pt,linearc=.05,linestyle=dashed,dash=6pt 6pt]{-}(#1,#2)(#3,#4)}
\ifthenelse{\equal{#5}{1}}
{\psline[linewidth=1.5pt,linearc=.05,linecolor=lightgray]{-}(#1,#2)(#3,#4)}
{\psline[linewidth=1.5pt,linearc=.05,linecolor=lightgray,linestyle=dashed,dash=2pt 2pt]{-}(#1,#2)(#3,#4)}
}
{
\ifthenelse{\equal{#5}{1}}
{\psline[linewidth=0.9pt,linearc=.05]{-}(#1,#2)(#3,#4)}
{\psline[linewidth=0.9pt,linearc=.05,linestyle=dashed,dash=2pt 2pt]{-}(#1,#2)(#3,#4)}
}
}

% \errelangle {#1 x0} {#2 y0} {#3 x1} {#4 y1} {#5 x2} {#6 y2} {#7 ismandatory} {#8 isocnstructed}
\newcommand {\errelangle}[8]
{
\ifthenelse{\equal{#8}{1}}
{
%\psline[linewidth=0.5pt,linearc=.1,linestyle=dashed,dash=6pt 6pt]{-}(#1,#2)(#3,#4)(#5,#6)}
\ifthenelse{\equal{#7}{1}}
{\psline[linewidth=1.5pt,linearc=.05,linecolor=lightgray]{-}(#1,#2)(#3,#4)(#5,#6)}
{\psline[linewidth=1.5pt,linearc=.1,linecolor=lightgray,linestyle=dashed,dash=2pt 2pt]{-}(#1,#2)(#3,#4)(#5,#6)}
}
{
\ifthenelse{\equal{#7}{1}}
{\psline[linewidth=0.9pt,linearc=.1]{-}(#1,#2)(#3,#4)(#5,#6)}
{\psline[linewidth=0.9pt,linearc=.1,linestyle=dashed,dash=2pt 2pt]{-}(#1,#2)(#3,#4)(#5,#6)}
}
}

% \ercrowfoot {#1 x0} {#2 y0} {#3 x11} {#4 y11} {#5 x12} {#6 y12} {#7 x13} {#8 y13} {#9 isconstructed}
\newcommand {\ercrowfoot}[9]
{
\ifthenelse{\equal{#9}{1}}
{
\psline[linewidth=1.5pt,linearc=.05,linecolor=lightgray]{-}(#1,#2)(#3,#4)
\psline[linewidth=1.5pt,linearc=.05,linecolor=lightgray]{-}(#1,#2)(#5,#6)
\psline[linewidth=1.5pt,linearc=.05,linecolor=lightgray]{-}(#1,#2)(#7,#8)
}{
\psline[linewidth=0.9pt,linearc=.05]{-}(#1,#2)(#3,#4)
\psline[linewidth=0.9pt,linearc=.05]{-}(#1,#2)(#5,#6)
\psline[linewidth=0.9pt,linearc=.05]{-}(#1,#2)(#7,#8)
}
}


% \eridcomprel{#1 x1}{#2 x2}{#3 y1}{#4 ymid}{#5 y2}
\newcommand {\eridcomprel}[5]
{
\psline[linewidth=0.9pt](#1,#3)(#1,#5)
\psline[linewidth=0.9pt](#2,#3)(#2,#5)
\psline[linewidth=0.9pt](#1,#4)(#2,#4)
}

% \eridrefrel{#1 x1}{#2 xmid}{#3 x2}{#4 y1}{#5 y2}
\newcommand {\eridrefrel}[5]
{
\psline[linewidth=0.9pt](#1,#4)(#3,#4)
\psline[linewidth=0.9pt](#1,#5)(#3,#5)
\psline[linewidth=0.9pt](#2,#4)(#2,#5)
}


% \errelname {#1 x} {#2 y} {#3 text}
\newcommand {\errelname}[3]
{
\rput[l]{0}(#1,#2){\textit{#3}}
}
% \errelseq {#1 x} {#2 y}
\newcommand {\erelseq}[2]
{
}
% \erattr {#1 x} {#2 y} {#3 ismandatory}{#4 idenitfying} {#5 text}
\newcommand {\erattr}[5]
{
\ifthenelse{\equal{#3}{1}}
{\rput[l]{0}(#1,#2){{\tiny $\square$} {\footnotesize \textit{\ifthenelse{\equal{#4}{0}}{\underline{#5}}{#5}}}}}
{\rput[l]{0}(#1,#2){\footnotesize $\circ$ \textit{\ifthenelse{\equal{#4}{0}}{\underline{#5}}{#5}}}}
}

%\ifthenelse{\equal{#4}{1}}
% \ertext {#1 x} {#2 y} {#3 text anchor} {#4 text}
%{\rput[l]{0}(#1,#2){\footnotesize $\circ$ \underline{\textit{#5}}}}
\newcommand {\ertext}[4]
{
\rput[B#3]{0}(#1,#2){{\footnotesize #4}}
}
% \erarc {#1 x0} {#2 y0} {#3 x1} {#4 y1} {#5 x2} {#6 y2} {#7 x3} {#8 y3}
\newcommand {\erarc}[8]
{
\psbezier[showpoints=false]{-}(#1,#2) (#3, #4)(#5,#6) (#7, #8)
}

% \erarc {#1 x0} {#2 y0} {#3 x1} {#4 y1} {#5 x2} {#6 y2} {#7 x3} {#8 y3}
\newcommand {\errelseq}[8]
{
\psbezier[showpoints=false]{-}(#1,#2) (#3, #4)(#5,#6) (#7, #8)
}
% \ertrace {#1 trace}   
\newcommand {\ertrace}[1]
{
}
    %beamer awar{}e version
% 
% general.macros.v2
%
% Rename macros that conflict with beamer 31 Aug 2022
% Don't assume an index                   31 Aug 2022

\usepackage{changepage} % used for adjustwidth

\iffalse % 31 Aug 2022
\usepackage{imakeidx}
\usepackage{framed}
\makeindex[name=definitions, title=Index of Definitions]
\makeindex[name=lemmas, title=Index of Lemmas]
\fi

\definecolor{highlight}{cmyk}{0,0,0.7,0}
\newcommand{\commentary}[1]{\marginpar{\footnotesize #1}}
\newcommand{\highlight}[1]{\colorbox{highlight}{#1}}
\newcommand{\whitelight}[1]{\colorbox{white}{#1}}
\newcommand{\term}[1]{\textit{#1}\commentary{\colorbox{lightgray}{\textit{#1}}}\index[definitions]{#1}}
\newcommand{\llabel}[1]{\label{#1}\commentary{\colorbox{pink}{\scriptsize{#1}}}\index[lemmas]{#1}}
\newcommand{\lref}[1]{\ref{#1}\colorbox{pink}{\scriptsize{#1}}\index[lemmas]{#1!use of}}

\newcommand{\daynote}[1]{\commentary{See day notes #1.}}

\newcommand{\newt}[1]{\colorbox{yellow}{#1}}
\newenvironment{newtt}
{  \colorbox{yellow}{$[$ ...} 
}
{  \colorbox{yellow}{... $]$}
}
\newcommand{\oldt}[1]{\colorbox{yellow}{\sout{#1}}}
\newenvironment{oldtt}
{  \colorbox{red}{$[$ ...} 
}
{  \colorbox{red}{... $]$}
}

\newcommand{\reinstatet}[1]{\colorbox{lime}{#1}}
\newenvironment{reinstatett}
{  \colorbox{lime}{$[$ ...}
}
{  \colorbox{lime}{... $]$}
}

\newcommand{\tbd}{\highlight{TBD}}

%ithprojection function
\newcommand{\proji}[1]{\pi_#1}


\newenvironment{aside}
{\begin{framed}
\textbf{Aside}
}
{
\end{framed}
}

\newenvironment{notebox}[1][Note]
{\begin{framed}
\textbf{#1}
}
{
\end{framed}
}

\newenvironment{categoricalaside}
{\begin{framed}
\textbf{Categorical Aside}
}
{
\end{framed}
}

\newenvironment{noteforfuture}
{\begin{framed}
\textbf{Note For Future}
}
{
\end{framed}
}

\newenvironment{myproblem}       %31 Aug 2022
{\begin{framed}
\textbf{Problem}
}
{
\end{framed}
}

\newenvironment{key}
{
\begin{tabular}{c l p{4cm}}
KEY && \\
\hline
}
{
\end{tabular}
}

%  31 Aug 2022
\NewEnviron{tightquote} %italic text indented left and right hand side
{\begin{adjustwidth}{1.5cm}{1.5cm}
\textit{
\BODY
}
\end{adjustwidth}
}

\newcommand{\keyentry}[3]{#1 & #2 & #3 \\} 


%quine quote
\newcommand{\qq}[1]{
\left\ulcorner#1\right\urcorner
}

%single quote
\newcommand{\sq}[1]{
\textnormal{\textquotesingle}#1\textnormal{\textquotesingle}
}

%lower quine quote
\newcommand{\lqq}[1]{
\left\llcorner #1\right\lrcorner
}


%from berkley
\newcommand{\langl}{\begin{picture}(4.5,7)
\put(1.1,2.5){\rotatebox{60}{\line(1,0){5.5}}}
\put(1.1,2.5){\rotatebox{300}{\line(1,0){5.5}}}
\end{picture}}
\newcommand{\rangl}{\begin{picture}(4.5,7)
\put(.9,2.5){\rotatebox{120}{\line(1,0){5.5}}}
\put(.9,2.5){\rotatebox{240}{\line(1,0){5.5}}}
\end{picture}}
\newcommand{\lang}{\begin{picture}(5,7)\put(1.1,2.5){\rotatebox{45}{\line(1,0){6.0}}}\put(1.1,2.5){\rotatebox{315}{\line(1,0){6.0}}}\end{picture}}
\newcommand{\rang}{\begin{picture}(5,7)\put(.1,2.5){\rotatebox{135}{\line(1,0){6.0}}}\put(.1,2.5){\rotatebox{225}{\line(1,0){6.0}}}\end{picture}}
%Try sharper tuple brackets -- except gives errors nested in captions so comment out
%\renewcommand{\tuple}[1]{\lang #1 \rang}

\newcommand{\setsuchthat}[2]{\left\{#1 \ \middle|\ #2\right\}}
\newcommand{\set}[1]{\left\{#1\right\}} 

% one to n - wanton
\newcommand{\wanton}[1]{#1_1,...#1_n}
\newcommand{\n}{1...n}
\newcommand{\fn}{\wanton{f}}
\newcommand{\gn}{\wanton{g}}
\newcommand{\pn}{\wanton{p}}
\newcommand{\qn}{\wanton{q}}
\newcommand{\qnprime}{\wanton{q'}}
\newcommand{\tn}{\wanton{t}}
\newcommand{\xn}{\wanton{x}}
\newcommand{\xnp}{\wanton{x'}}
\newcommand{\yn}{\wanton{y}}
\newcommand{\An}{\wanton{A}}
\newcommand{\Bn}{\wanton{B}}
\newcommand{\Cn}{\wanton{C}}
\newcommand{\ntuple}[1]{\tuple{\wanton{#1}}}
\newcommand{\wantom}[2][]{#2_1,...#2_{m#1}}
\newcommand{\m}{1...m}
\newcommand{\mtuple}[1]{\tuple{#1_1,...#1_m}}
\newcommand{\gm}{\wantom{g}}
\newcommand{\qm}{\wantom{q}}
\newcommand{\sm}[1][]{\wantom[#1]{s}}
\newcommand{\smp}{\wantom{s'}}
\newcommand{\ym}{\wantom{y}}
\newcommand{\Bm}{\wantom{B}}
\newcommand {\bntuple}{\ensuremath{\ntuple{b}}}
\newcommand {\fntuple}{\ensuremath{\ntuple{f}}}
\newcommand {\fnptuple}{\ensuremath{\ntuple{f}}}
\newcommand {\pntuple}{\ensuremath{\ntuple{p}}}
\newcommand {\qntuple}{\ensuremath{\ntuple{q}}}
\newcommand {\qnptuple}{\ensuremath{\ntuple{q'}}}
\newcommand {\qmtuple}{\ensuremath{\mtuple{q}}}
\newcommand {\sntuple}{\ensuremath{\ntuple{s}}}
\newcommand {\xntuple}{\ensuremath{\ntuple{x}}}
\newcommand {\xnptuple}{\ensuremath{\ntuple{x'}}}
\newcommand {\ymtuple}{\ensuremath{\mtuple{y}}}
\newcommand{\idef}[1][n]{1 \leq i \leq #1}
\newcommand{\jdef}[1][m]{1 \leq j \leq #1}
\newcommand{\kdef}[1][l]{1 \leq k \leq #1}
\newcommand{\foreachi}[1][n]{for each $i$, $1 \leq i \leq #1$}
\newcommand{\foreachj}[1][m]{for each $j$, $1 \leq j \leq #1$}
\newcommand{\foreachk}[1][l]{for each $k$, $1 \leq k \leq #1$}
\newcommand{\Foreachi}[1][n]{For each $i$, $1 \leq i \leq #1$}
\newcommand{\Foreachj}[1][m]{For each $j$, $1 \leq j \leq #1$}
\newcommand{\Foreachk}[1][l]{For each $k$, $1 \leq k \leq #1$}
\newcommand{\forsomei}[1][n]{for some $i$, $1 \leq i \leq #1$}
\newcommand{\forsomej}[1][m]{for some $j$, $1 \leq j \leq #1$}
\newcommand{\forsomek}[1][l]{for some $k$, $1 \leq k \leq #1$}
\newcommand{\wherei}[1][n]{where $1 \leq i \leq #1$}
\newcommand{\wherej}[1][m]{where $1 \leq j \leq #1$}
\newcommand{\wherek}[1][l]{where $1 \leq k \leq #1$}


\newcommand{\fundep}[3]{#2 \xrightarrow{#1} #3}  %where does this belong? xxxx
% Following used for notes -- indented numbered paras

\newcounter{para}
\newlength{\oldparindent}
\setlength{\oldparindent}{\parindent} % Save \parindent before of change
\newcommand{\ind}{\hspace*{\oldparindent}}

\newcommand\mynote{                                                 % renamed 31 Aug 2022
%\setlength{\parskip}{0.5\baselineskip} % Definition of `parskip`
\setlength{\parindent}{0pt}
\par\ind\refstepcounter{para}\thepara.\space
\setlength{\parindent}{\oldparindent}
}


         % beamer safe version

%%%%%%%%%%%%%%%%%%%%%%%%%%%%%%%%%
% alternate.beamer.macros.tex
%%%%%%%%%%%%%%%%%%%%%%%%%%%%%%%%%
% macros here for use with beamer
% nonBeamer.macros has alternate versions for use in papers
% I am confused why I reimplmented highlight macro

\newcommand{\waitfor}[2]{\onslide<#1->{#2}}   %enables different version of waitfor in standard documents
%%
% beamermacros
%

%re-impmentation of highlight for beamer
%\newcommand<>\highlightbox[2]{%
%  \alt#3{\makebox[\dimexpr\width-2\fboxsep]{\colorbox{#1}{#2}}}{#2}%
%}
%By copying above from internet
\definecolor{highlightcolor}{cmyk}{0,0,0.7,0}
\newcommand<>{\highlight}[1]{%
  \alt#2{\makebox[\dimexpr\width-2\fboxsep]{\colorbox{highlightcolor}{#1}}}{#1}%
}


  % this has a beamer aware highlight command but nor sure why I needed it
%%All these macros are copied from SharedMacros/general.tex which doesnt seem to work with beamer
% Some macros in SharedMacros/general.tex thought to have name clashes with beamer.


\newcommand{\fundep}[3]{#2 \xrightarrow{#1} #3}                                                 
\newcommand{\term}[1]{\textit{#1}} 
\newcommand{\setsuchthat}[2]{\left\{#1 \ \middle|\ #2\right\}}
\newcommand{\set}[1]{\left\{#1\right\}}



\newcommand{\wanton}[1]{#1_1,...#1_n}
\newcommand{\ntuple}[1]{\tuple{\wanton{#1}}}

\newcommand{\xntuple}{\ensuremath{\ntuple{x}}}

% maybe not in general.macros 
\newcommand{\xnset}{\ensuremath{\set{\wanton{x}}}}
%%
% othermacros
%

% copied and edited from \idcomp to make stronger linestyle
\newcommand{\addedgebar}{
\ncput[npos=0, nrot=:U]{\psline[linewidth=1.25pt](0.2,-0.1)(0.2,0.1)}
}
\newcommand{\addedgedoublebar}{
\ncput[npos=0, nrot=:U]{\psline[linewidth=1.25pt](0.2,-0.1)(0.2,0.1)}
\ncput[npos=0, nrot=:U]{\psline[linewidth=1.25pt](0.3,-0.1)(0.3,0.1)}
}
\newcommand{\addedgetriplebar}{
\ncput[npos=0, nrot=:U]{\psline[linewidth=1.25pt](0.2,-0.1)(0.2,0.1)}
\ncput[npos=0, nrot=:U]{\psline[linewidth=1.25pt](0.3,-0.1)(0.3,0.1)}
\ncput[npos=0, nrot=:U]{\psline[linewidth=1.25pt](0.4,-0.1)(0.4,0.1)}
}

%\newcommand{\addedgebar}{\ifbars{\ncput[npos=0, nrot=:U]{\psline(0.2,-0.075)(0.2,0.075)}}\fi}

%copied from database literature review
\newcommand{\displaybibentry}[1]
{\begin{framed}
\bibentry{#1}
\end{framed}
}

% used in data tables
\newcommand{\colhead}[1]{\textbf{\textcolor{white}{#1}}}
\definecolor{myblue}{RGB}{71,71,186}
\newcommand{\largeAsterisk}{\mathop{\scalebox{1.5}{\raisebox{-0.2ex}{$\ast$}}}}
\newcommand{\fk}[1]{#1$^{\largeAsterisk}$}
\newcommand{\pk}[1]{\underline{#1}}
\newcommand{\seck}[1]{\dashuline{#1}}  % secondary key
\newcommand{\pkfk}[1]{\underline{#1}$^{\largeAsterisk}$} % primary key that is a foreign key
% \vpad gives vertical padding in a tabular
\newcommand{\vpad}[1]{\multicolumn{#1}{c}{}\\[-0.25cm]}
% used in slides
\newcommand{\outerbullet}{{$\color{blue}{\blacktriangleright}$}\ }% please dont remove final space
\newcommand{\innerbullet}{{\footnotesize $\color{blue}{\blacktriangleright}$}\ }% please dont remove final space
\newcommand{\braceLabel}[3]{\psbrace[ref=lC,braceWidth=1pt,braceWidthInner=3pt,braceWidthOuter=3pt](#2)(#1){#3} }

% words words words
\newcommand{\catMEterm}{category with designated monomorphisms and epimorphisms\ }
\newcommand{\IfSforCwithRCwords}{
If $S$ is a sketch for category \catcw considered as a data specification with requirement $\reqtc$\ }
\newcommand{\IfSforCwithRCwordsvariant}{
If $S$ is a sketch for structured category \catcw and if $S$ is considered as a data specification with requirement $\reqtc$\ }
\newcommand{\IfSforepimonoCwithRCwords}{
If $S$ is a sketch for a category \catcw with designated monomorphisms and epimorphisms considered as a data specification with requirement $\reqtc$\ }
\iffalse
\newcommand{\scmonosketchwording}{
If $S$ is a sketch for such a category
%of a category with finite products and designated monomorphisms and epimorphisms
considered as a data specification
with requirement $\reqtc$\ }
\fi
\newcommand{\spacechar}{\ }
\newcommand{\thirdstructure}{designated monomorphisms and epimorphisms and with finite products}
\newcommand{\IfSforproductepimonoCwithRCwords}{
If $S$ is a sketch for  a category with \thirdstructure \spacechar
%category \catcw with finite products and designated monomorphisms and epimorphisms 
considered as a data specification with requirement $\reqtc$\ }

\newcommand{\goodnesscriteria}[1]{\textbf{Goodness Criteria #1:}}

\newcommand{\goodnessoneA}{
\goodnesscriteria{1A} There ought not to be an edge $e$ in $G$ for which there is an equivalent path $p$ which  does not containing $e$
}
\newcommand{\goodnessoneB}{
\goodnesscriteria{1B} 
There ought not exist $d \in PE$ such that $d \in \overline{PE \setminus d}$
}

\newcommand{\goodnessoneC}{
\goodnesscriteria{1C} \\
There ought not exist $m \in M$ such that $m \in \overline{M \setminus m}$
}
\newcommand{\goodnessoneD}{
\goodnesscriteria{1D} \\
There ought not exist $e \in E$ such that $e \in \overline{E \setminus e}$
}




% From the Mathematical Theory of data paper
\newcommand{\ssfd}[2]{\ensuremath{#1 \morph #2}}  % singleton-singleton
\newcommand{\smfd}[2]{\ensuremath{\ssfd{#1}{\set{#2}}}}  % singleton-many
\newcommand{\msfd}[2]{\ensuremath{\ssfd{\set{#1}}{#2}}}  % many-singleton
\newcommand{\mmfd}[2]{\ensuremath{\msfd{#1}{\set{#2}}}}  % many-many



% All these should find a home in SharedMacros eventually 

% Commands for making a bit of vertical space. used when arrows and particularly labels of arrows
% use spec that is otherwise accounted for.
\newcommand{\seeroomup}[1]{\rule{0.1cm}{#1}}
\newcommand{\seeroomdown}[1]{\rule[-#1]{0.1cm}{0.1cm}}
\newcommand{\roomup}[1]{\rule{0cm}{#1}}
\newcommand{\roomdown}[1]{\rule[-#1]{0cm}{0.1cm}}


% BOX DIAGRAMS
\newcommand{\attr}[1]{#1}
\renewcommand{\attr}[1]{\psframebox[linecolor=red,framearc=.1]{#1}}
\newcommand{\attrtype}[1]{#1}
\renewcommand{\attrtype}[1]{\psframebox[linecolor=blue,framearc=.1]{#1}}
\newcommand{\etype}[1]{#1}
\renewcommand{\etype}[1]{\psframebox[linecolor=red,framearc=.1]{#1}}


\newcommand{\regularizetextheight}{\roomup{0.3cm}\roomdown{0.1cm}}

\newcommand{\unarystructurediagramnodes}[3][]{
\rput[tc](2.4,3){\Rnode{#1A}{\psframebox[framesep=10pt]{\regularizetextheight#2}}} 
\rput[tc](2.4,1){\rnode{#1B}{\psframebox[framesep=10pt]{\regularizetextheight#3}}}           
}

\newcommand{\binarystructurediagramnodes}[4][]{
\rput[tr](4.0,3){\Rnode{#1A}{\psframebox[framesep=10pt]{\regularizetextheight#2}}} 
\rput[tr](2.4,1){\rnode{#1B}{\psframebox[framesep=10pt]{\regularizetextheight#3}}}     
\rput[tr](5.6,1){\rnode{#1C}{\psframebox[framesep=10pt]{\regularizetextheight#4}}}        
}

\newcommand{\triplestructurediagramnodes}[5][]{ 
\rput[tc](2.0,3){\rnode{#1A}{\psframebox[framesep=10pt]{\regularizetextheight#2}}}     
\rput[tc](-1.05,1){\rnode{#1B}{\psframebox[framesep=10pt]{\regularizetextheight#3}}}  
\rput[tc](2.0,1){\rnode{#1C}{\psframebox[framesep=10pt]{\regularizetextheight#4}}}   
\rput[tc](5.0,1){\rnode{#1D}{\psframebox[framesep=10pt]{\regularizetextheight#5}}}   
}

\newcommand{\jacksonbinarydiagram}[3]
{
\pspicture(-0.4,0)(5.7,3)  % lower left is 0,0 upper right is 8,3
%\psgrid
\binarystructurediagramnodes{#1}{#2}{#3}
\rput[tr](2.3,0.9){*}
\rput[tr](5.4,0.9){*}
\ncangle[offsetA=-0.5cm, angleA=-90,angleB=90,armB=0.5cm]{A}{B}
\ncangle[offsetA=0.5cm, angleA=-90,angleB=90,armB=0.5cm]{A}{C}
\endpspicture      
}

\newcommand{\bachmanbinarydiagram}[4][]
{
\pspicture(-0.4,0)(5.7,3)  % lower left is 0,0 upper right is 8,3
%\psgrid
\binarystructurediagramnodes[#1]{#2}{#3}{#4}
\ncline[linewidth=3pt]{->}{#1A}{#1B}
\ncline[linewidth=3pt]{->}{#1A}{#1C}
\endpspicture      
}

\newcommand{\unarystructurediagram}[3][]
{
\pspicture(0.9,0)(3.9,3.5)  
%\psgrid
\unarystructurediagramnodes[#1]{#2}{#3}
\endpspicture      
}

\newcommand{\binarystructurediagram}[4][]
{
\pspicture(-0.4,0)(5.7,3)  
%\psgrid
\binarystructurediagramnodes[#1]{#2}{#3}{#4}
\endpspicture      
}

\newcommand{\triplestructurediagram}[5][]
{
\pspicture(-2.5,0)(6.4,3.5)  
%\psgrid
\triplestructurediagramnodes[#1]{#2}{#3}{#4}{#5}
\endpspicture      
}


\newcommand{\binarynetworkdiagramnodes}[3]{ 
\rput[tr](2.4,3){\rnode{A}{\psframebox[framesep=10pt]{\regularizetextheight#1}}}     
\rput[tr](5.6,3){\rnode{B}{\psframebox[framesep=10pt]{\regularizetextheight#2}}} 
\rput[tr](4.0,1){\Rnode{C}{\psframebox[framesep=10pt]{\regularizetextheight#3}}}       
}

\newcommand{\bachmannetworkdiagram}[3]
{
\pspicture(-0.4,0)(5.7,3)  % lower left is 0,0 upper right is 8,3
%\psgrid
\binarynetworkdiagramnodes{#1}{#2}{#3}
\ncline[linewidth=3pt]{->}{A}{C}
\ncline[linewidth=3pt]{->}{B}{C}
\endpspicture      
}

%craft bachman nwtrok share diagram
\newcommand{\doublebinarynetworkdiagramnodes}[6]{ 
\rput[tc](-2.5,3){\rnode{A}{\psframebox[framesep=10pt]{\regularizetextheight#1}}}  
\rput[tc](2.0,3){\rnode{B}{\psframebox[framesep=10pt]{\regularizetextheight#2}}}    
\rput[tc](-4.1,1){\rnode{C}{\psframebox[framesep=10pt]{\regularizetextheight#3}}} 
\rput[tc](-1.05,1){\rnode{D}{\psframebox[framesep=10pt]{\regularizetextheight#4}}}  
\rput[tc](2.0,1){\rnode{E}{\psframebox[framesep=10pt]{\regularizetextheight#5}}}   
\rput[tc](5.0,1){\rnode{F}{\psframebox[framesep=10pt]{\regularizetextheight#6}}}   
}

\newcommand{\doublebachmannetworkdiagram}[6]
{
\pspicture(-5.6,0)(6.5,3)  % lower left is 0,0 upper right is 8,3
%\psgrid
\doublebinarynetworkdiagramnodes{#1}{#2}{#3}{#4}{#5}{#6}
\ncline[linewidth=3pt]{->}{A}{C}
\ncline[linewidth=3pt]{->}{A}{D}
\ncline[linewidth=3pt]{->}{B}{D}
\ncline[linewidth=3pt]{->}{B}{E}
\ncline[linewidth=3pt]{->}{B}{F}
\endpspicture      
}

\newcommand{\doublecategorynetworkdiagram}[6]
{
\pspicture(-5.6,0)(6.5,3)  % lower left is 0,0 upper right is 8,3
%\psgrid
\doublebinarynetworkdiagramnodes{#1}{#2}{#3}{#4}{#5}{#6}
\ncarr{C}{A}
\ncarr{D}{A}
\ncarr{D}{B}
\ncarr{E}{B}
\ncarr{F}{B}
\endpspicture      
}

\newcommand{\mixedcategorynetworkdiagram}[6]
{
\pspicture(-5.6,0)(6.5,3)  % lower left is 0,0 upper right is 8,3
%\psgrid
\doublebinarynetworkdiagramnodes{#1}{#2}{#3}{#4}{#5}{#6}
\ncline[linewidth=2.5pt]{->}{C}{A}
\ncline[linewidth=2.5pt]{->}{D}{A}
\ncarr{D}{B}
\ncline[linewidth=2.5pt]{->}{E}{B}
\ncline[linewidth=2.5pt]{->}{F}{B}
\endpspicture      
}

\newcommand{\contextualcategoryblockstyleexamplekernel}[6]{
\pspicture(-5.6,0)(6.5,3)  % lower left is 0,0 upper right is 8,3
%\psgrid
\doublebinarynetworkdiagramnodes{#1}{#2}{#3}{#4}{#5}{#6}
\ncsar{C}{A}
\ncsar{E}{B}
\ncsar{F}{B}
\endpspicture
}

\newcommand{\contextualcategorynetworkdiagram}[6]
{
\contextualcategoryblockstyleexamplekernel{#1}{#2}{#3}{#4}{#5}{#6}
\ncsar{D}{A}
\ncarr{D}{B}
}

\newcommand{\contextualcategorynetworkdiagramreorganised}[6]
{
\contextualcategoryblockstyleexamplekernel{#1}{#2}{#3}{#4}{#5}{#6}
\ncarr{D}{A}
\ncsar{D}{B}
}


\newcommand{\contextualcategorynetworkdiagramtopologised}[6]
{
\begin{tabular}{c c c}
\scalebox{0.9}{\binarystructurediagram[left]{compound\kern0.1cm}{alias \kern1.2cm}{occurence}}
&&
\scalebox{0.9}{\binarystructurediagram[right]{element\kern0.4cm}{valency \kern0.8cm}{allotrope\kern0.3cm}}
\end{tabular}
\ncangle[offsetA=0.15cm, angleA=0,offsetB=-0.25cm, angleB=180, armB=2.5cm]{->}{leftC}{rightA}
\ncsar{leftB}{leftA}
\ncsar{leftC}{leftA}
\ncsar{rightB}{rightA}
\ncsar{rightC}{rightA}
}

\newcommand{\contextualcategorynetworkdiagramreorganisedtopologised}[6]
{
\begin{tabular}{c c c}
\scalebox{0.9}{\unarystructurediagram[left]{compound\kern0.1cm}{alias \kern1.2cm}}
&&
\scalebox{0.9}{\triplestructurediagram[right]{element\kern0.4cm}{occurence}{valency \kern0.8cm}{allotrope\kern0.3cm}}
\end{tabular}
\ncangle[offsetA=0.15cm, angleA=180,offsetB=-0.25cm, angleB=0, armB=0.9cm]{->}{rightB}{leftA}
\ncsar{leftB}{leftA}
\ncsar{leftC}{leftA}
\ncsar{rightB}{rightA}
\ncsar{rightC}{rightA}
\ncsar{rightD}{rightA}
}

\iffalse
\newcommand{\contextualcategorynetworkdiagramreorganised}[6]
{
\pspicture(-5.6,0)(6.5,3)  % lower left is 0,0 upper right is 8,3
%\psgrid
\doublebinarynetworkdiagramnodes{#1}{#2}{#3}{#4}{#5}{#6}
\ncsar{C}{A}
\ncarr{D}{A}
\ncsar{D}{B} 
\ncsar{E}{B}
\ncsar{F}{B}
\endpspicture      
}
\fi

% Category DIAGRAMS START HERE


\newcommand{\factorisationfdiagram}{
    $
    \begin{array}{c p{1cm} c p{1.0cm} c}
    \Rnode{a}{a}&&\Rnode{Imf}{Im(f)}&&\Rnode{b}{b}
    \end{array}
    \begin{arrows}
    \ncline{->>}{a}{Imf}\alabel{f_e}
    \ncarr{Imf}{b}\alabel{f_m}\idcomp
    \end{arrows}
    $
}
\newcommand{\nakedbinarysourcediagram}[5]{
\begin{array}{c p{0.5cm} c}
             &&   \Rnode{b}{#2}\\[0.01cm]
\Rnode{a}{#1} &&               \\[0.01cm] 
             &&   \Rnode{c}{#3}
\end{array} 
\begin{arrows}
\ncarr{a}{b}
\alabel{#4}
\ncarr{a}{c}
\blabel{#5}
\end{arrows}
}

\newcommand{\binarysourcediagram}[5]{$\nakedbinarysourcediagram{#1}{#2}{#3}{#4}{#5}$}
\newcommand{\fgsourcediagram}{\binarysourcediagram{a}{b}{c}{f}{g}}

%  binary source diagram with arrows pointing SE and SW
% nakedSWSEsourcediagram{prefix}{a}{b}{c}{f}{g}
\newcommand{\nakedSWSEsourcediagram}[6]{
\begin{array}{c c c}
              & \Rnode{#1a}{#2} &               \\[1.0cm] 
\Rnode{#1b}{#3} &               &\Rnode{#1c}{#4}
\end{array} 
\begin{arrows}
\ncarr{#1a}{#1b}
\alabel{#5}
\ncarr{#1a}{#1c}
\blabel{#6}
\end{arrows}
}


%  binary sink diagram with arrows pointing SE and SW
\newcommand{\nakedNWNEsinkdiagram}[5]{
\begin{array}{c c c}
              & \Rnode{a}{#1} &               \\[0.5cm] 
\Rnode{b}{#2} &               &\Rnode{c}{#3}
\end{array} 
\begin{arrows}
\ncarr{b}{a}
\alabel{#4}
\ncarr{c}{a}
\blabel{#5}
\end{arrows}
}

\newcommand{\simpleunaryfdrepresentationdiagram}[6]{
$
\nakedbinarysourcediagram{#1}{#2}{#3}{#4}{#5}
\begin{arrows}
\ncarr{b}{c}
\alabel{#6}
\end{arrows}
$
}

\newcommand{\unaryfdrepresentationdiagram}[8]{
$
\begin{array}{c p{0.2cm} c}
\nakedbinarysourcediagram{#1}{#2}{#3}{#4}{#5}&& \Rnode{d}{#6}
\end{array}
\begin{arrows}
\ncarr{d}{b}
\idcomp
\blabel{#7}
\ncarr{d}{c}
\alabel{#8}
\end{arrows}
$
}

\newcommand{\unaryfdrepresentationmappeddiagram}[8]{
$
\begin{array}{c p{0.2cm} c}
\nakedbinarysourcediagram{D(#1)}{D(#2)}{D(#3)}{D(#4)}{D(#5)}&& \Rnode{d}{D(#6)}
\end{array}
\begin{arrows}
\ncarr{b}{d}
\alabel{D(#7)^-1}
\ncarr{d}{c}
\alabel{D(#8)}
\end{arrows}
$
}

\newcommand{\commutativetrianglediagram}[6]{
$
\begin{array}{c p{0.4cm} c p{0.4cm} c}
              && \Rnode{b}{#2}  &&                 \\[0.6cm]
\Rnode{a}{#1} &&                && \Rnode{c}{#3}  
\end{array}
\begin{arrows}
\ncarr{a}{b}
\alabel{#4}
\ncarr{b}{c}
\alabel{#5}
\ncarr{a}{c}
\blabel{#6}
\end{arrows}
$
}

\newcommand{\commutativetrianglediagrammutant}[6]{
$
\begin{array}{c  c  c}
              & \Rnode{b}{#2}  &                 \\[0.85cm]
\Rnode{a}{#1} &                & \Rnode{c}{#3}  
\end{array}
\begin{arrows}
\ncarr{a}{b}
\alabel{#4}[0.15]
\ncarr{b}{c}
\alabel{#5}[0.6]
\ncarr{a}{c}
\blabel{#6}
\end{arrows}
$
}

\newcommand{\epimonosplitdiagram}[3]{
\commutativetrianglediagram{#1}{img(#3)}{#2}{#3_e}{#3_m}{#3}   
}


\iffalse %saved
\begin{array}{c p{2.0cm} c }                
               &&  \Rnode{b1}{#3_1}    \\ [0.75cm]
               &&  \Rnode{b2}{#3_2}    \\ [0.5cm]
\Rnode{a}{#2}  &&                      \\ [-0.5cm]
               &&       \vdots         \\ [0.85cm]
               &&  \Rnode{bn}{#3_{#1}}  
\end{array}
\fi

%nakedmultisourceobjects{n}{a}{b}
\newcommand{\nakedmultisourceobjects}[3]{
\begin{array}{c p{2.0cm} c }
\Rnode{a}{#2}   &&
\begin{array}{c }                
\Rnode{b1}{#3_1}   \\ [0.75cm]
\Rnode{b2}{#3_2}   \\ [0.25cm]
\vdots             \\ [0.35cm]
\Rnode{bn}{#3_{#1}}  
\end{array}
\end{array}
}

% \nakedmultisourcediagram{n}{a}{b}{f}
\newcommand{\nakedmultisourcediagram}[4]{
\nakedmultisourceobjects{#1}{#2}{#3}
\begin{arrows}
\ncarr{a}{b1}
\alabel{#4_1}[0.5]
\ncarr{a}{b2}
\alabel{#4_2}[0.5][-1]
\ncarr{a}{bn}
\blabel{#4_{#1}}[0.5][-1]
\end{arrows}
}

% \nakedmultisourcepathdiagram{n}{a}{b}{f}
\newcommand{\nakedmultisourcepathdiagram}[4]{
\nakedmultisourceobjects{#1}{#2}{#3}{#4}
\begin{arrows}
\simplepath{a}{b1}
\alabel{#4_1}[0.5]
\simplepath{a}{b2}
\alabel{#4_2}[0.5][-1]
\simplepath{a}{bn}
\blabel{#4_{#1}}[0.5][-1]
\end{arrows}
}


\newcommand{\multisourcediagram}[4]{$\nakedmultisourcediagram{#1}{#2}{#3}{#4}$}
\newcommand{\multisourcepathdiagram}[4]{$\nakedmultisourcepathdiagram{#1}{#2}{#3}{#4}$}


% \monosourcedefinitiondiagram{x}{g}{h}{n}{a}{b}{f}
\newcommand{\monosourcedefinitiondiagram}[7]{
$
\begin{array}{c p{1.5cm} c}
\Rnode{x}{#1} && \nakedmultisourcediagram{#4}{#5}{#6}{#7}
\end{array}
\begin{arrows}
\parallelarrows{x}{a}{#2}{#3}
\end{arrows}
$
}

%\multisourcenplusonediagram{n}{a}{b}{f}{c}{g}
\newcommand{\multisourcenplusonediagram}[6]{
$
\begin{array}{c p{2.0cm} c }
\Rnode{a}{#2}   &&
\begin{array}{c }                
\Rnode{b1}{#3_1}   \\ [0.55cm]
\Rnode{b2}{#3_2}   \\ 
\vdots             \\ 
\Rnode{bn}{#3_{#1}} \\ [0.65cm] 
\Rnode{c}{#5} 
\end{array}
\end{array}
\begin{arrows}
\ncarr{a}{b1}
\alabel{#4_1}[0.6][1]
\ncarr{a}{b2}
\alabel{#4_2}[0.6][0]
\ncarr{a}{bn}
\blabel{#4_{#1}}[0.6][0]
\ncarr{a}{c}\blabel{#6}[0.6][0]
\end{arrows}
$
}

\newcommand{\fghfactordiagram}[6]
{
\binarysourcediagram{#1}{#2\roomup{0.5cm}}{#3}{#4}{#5}
\begin{arrows}
\ncarr{b}{c}
\alabel{#6}
\end{arrows}
}

\newcommand{\fghpartialfactordiagram}[6]{
\binarysourcediagram{#1}{#2\roomup{0.5cm}}{#3}{#4}{#5}
\begin{arrows}
\ncdarr{b}{c} %dashed arrow
\alabel{#6}
\end{arrows}
}

\newcommand{\fnsourceqnsource}{
$
\begin{array}{c p{0.25cm} c  p{0.25cm} c }
             &&   \Rnode{b1}{b_1} &&              \\[0.4cm]
\Rnode{a}{a} &&                   && \Rnode{c}{c} \\[0.4cm]
             &&   \Rnode{bn}{b_n} &&              
\end{array} 
\begin{arrows}
\ncarr{a}{b1}
\alabel{f_1}
\ncarr{c}{b1}
\blabel{q_1} 
\ncarr{a}{bn}
\blabel{f_n}
\ncarr{c}{bn}
\alabel{q_n}
\end{arrows}
$   
}

\newcommand{\parallelarrows}[4]{
\ncarc[nodesep=2pt,arcangle=10,offset=2pt]{->}{#1}{#2}
\alabel{#3}
\ncarc[nodesep=2pt,arcangle=-10,offset=-2pt]{->}{#1}{#2}
\blabel{#4}
}

\newcommand{\paralleldiagram}[4]{
$
\rule[-0.3cm]{0pt}{0.9cm} %to add vertical space of diagram -- based on lowering diagram 0.3cm and heght 0.9cm
                            % change thickness from 0pt to 1 pt to debug
\begin{array}{c p{0.5cm} c}
\Rnode{a}{#1}       &&   \Rnode{b}{#2}
\end{array} 
\begin{arrows}
\parallelarrows{a}{b}{#3}{#4}
\end{arrows}
$
}

\newcommand{\fgparalleldiagram}{
 $
\rule[-0.3cm]{0pt}{0.9cm} %to add vertical space of diagram -- based on lowering diagram 0.3cm and heght 0.9cm
                            % change thickness from 0pt to 1 pt to debug
\begin{array}{c p{0.5cm} c  }
 \Rnode{a}{a}            &&   \Rnode{b}{b}
\end{array} 
\begin{arrows}
\parallelarrows{a}{b}{f}{g}
\end{arrows}
$  
}

\newcommand{\fgcomposablediagram}[5]{
\mbox{
\roomup{0.45cm}
$
\begin{array}{c p{0.5cm}cp{0.5cm}c}
\Rnode{x}{#1}&&\Rnode{y}{#2}&&\Rnode{z}{#3}
\end{array}
\begin{arrows}
\ncarr{x}{y}
\alabel{#4}
\ncarr{y}{z}
\alabel{#5}
\end{arrows}
$    
}
}


%
% copied from MToD paper (preamble.tex)
\newcommand{\simplepath}[2]{
\ncline[linestyle=none,linewidth=0.1pt]{#1}{#2}   %was linestyle=dotted
\ncput[npos=0.05]{\pnode{dot#21}}
\ncput[npos=0.27]{\dotnode[dotsize=1pt]{dot#22}}
\ncput[npos=0.50]{\dotnode[dotsize=1pt]{dot#23}}
\ncput[npos=0.80]{\dotnode[dotsize=1pt]{dot#24}}
\ncput[npos=0.975]{\pnode{dot#25}}
\ncline[nodesep=2pt]{->}{dot#21}{dot#22}
\ncline[nodesep=2pt]{->}{dot#22}{dot#23}
\ncline[nodesep=2pt]{->}{dot#24}{dot#25}
\ncline[linestyle=dotted,nodesep=8pt]{dot#23}{dot#24} %was 10pt
}
%


\newcommand{\stringtype}{text}
\newcommand{\numbertype}{number}

\newcommand{\dgsrcedge}
{
\setlength{\arroffsetA}{3pt}
\setlength{\arroffsetB}{3pt}
\ncarr[5]{edge}{node} 
\arreset  
}
\newcommand{\structuraldgsrcedge}
{
\setlength{\saroffsetA}{3pt}
\setlength{\saroffsetB}{3pt}
\ncsar[5]{edge}{node} 
\sarreset 
}
\newcommand{\dgtargetedge}
{
\setlength{\arroffsetA}{-3pt}
\setlength{\arroffsetB}{-3pt}
\ncarr[-5]{edge}{node} 
\arreset   
}
\newcommand{\dgbasic}
{
\begin{array}{c}
\Rnode{node}{node}  \\[2cm]
\Rnode{edge}{edge}       
\end{array}
\begin{arrows}
\dgsrcedge
\alabel{src}
\dgtargetedge
\blabel{trg}
\end{arrows}    
}

% dgbasiclabelled{labeltype}
\newcommand{\dgbasiclabelled}[1]
{
\begin{array}{cp{0.75cm}c}
%rule [raise-height]{width}{height}
\dgbasic   &&  \Rnode{labeltypelhs}{\rule[0cm]{0cm}{0.3cm}}\Rnode{labeltype}{#1} 
\end{array}
\begin{arrows}
\ncarr{node}{labeltypelhs}
\alabel{label}
\ncarr{edge}{labeltypelhs}
\blabel{label}
\end{arrows}       
}

\newcommand{\setoflabelleddgs}
{
\begin{array}{cp{1.0cm} : p{0.5cm}c}
\setofdg   &&&
\begin{array}{l}
\Rnode{text}{}\stringtype \\[1cm]
\Rnode{number}{}\numbertype 
\end{array}
\end{array}
\begin{arrows}
\ncarr{dg}{text}
\alabel{name}[0.3]
\ncarr{node}{number}
\alabel{label}[0.3]
\ncarr{edge}{number}
\blabel{label}[0.3]   
\end{arrows}
}


\newcommand{\structuraldgbasic}
{
\begin{array}{c}
\Rnode{node}{node}  \\[1.5cm]
\Rnode{edge}{edge}       
\end{array}
\begin{arrows}
\structuraldgsrcedge
\alabel{src}
\dgtargetedge
\blabel{trg}
\end{arrows}    
}

\newcommand{\nodepartof}
{
\ncarr{node}{dg}
\alabel{p}     
}
\newcommand{\structuralnodepartof}
{
\ncsar{node}{dg}     
}

\newcommand{\setofdg}
{
\begin{array}{c}
\rnode{dg}{dg} \\[1.5cm]
\dgbasic
\begin{arrows}
\nodepartof
\end{arrows}
\end{array}
}

\newcommand{\setofdgcommutativediagram}
{
\begin{array}{c c c}
&\rnode{dg}{dg} &\\[0.75cm]
\rnode{Lnode}{node}&&\rnode{Rnode}{node}\\[0.75cm]
&\rnode{edge}{edge}&
\end{array}
\begin{arrows}
\ncarr{Lnode}{dg}
\alabel{p}
\ncarr{Rnode}{dg}
\blabel{p}
\ncarr{edge}{Lnode}
\alabel{src}
\ncarr{edge}{Rnode}
\blabel{trg}
\end{arrows}
}


\newcommand{\structuralsetofdg}
{
\begin{array}{c}
\rnode{dg}{dg} \\[2cm]
\structuraldgbasic
\begin{arrows}
\structuralnodepartof
\end{arrows}
\end{array}
}


\newcommand{\dgnumericallylabelleddetail}
{
\begin{array} {c}
\begin{array} {p{1.5cm} c}
     & \Rnode{abs}{\ \ 1\ \ }  
\end{array} \\[1.0cm]
\dgbasiclabelled{number}  
\end{array}
\begin{arrows}
%\setlength{\arroffsetA}{3pt}
\setlength{\arroffsetA}{-6pt}
\setlength{\arroffsetB}{-12pt}
\ncarr{abs}{labeltype}
\alabel{0}
\setlength{\arroffsetA}{3pt}
\setlength{\arroffsetB}{0pt}
\ncarr{abs}{labeltype}
\alabel{1}
\setlength{\arroffsetA}{9pt}
\setlength{\arroffsetB}{12pt}
\ncarr{abs}{labeltype}
\alabel{2 \hdots}
\end{arrows}
}



\newcommand{\dgabsuniquelylabelled}
{
\begin{array}{cp{0.75cm}c}
\dgbasic   &&  \Rnode{l}{l} 
\end{array}
\begin{arrows}
\ncarr{node}{l}
\alabel{label}
\idcomp
\ncarr{edge}{l}
\blabel{label}
\idcomp
\end{arrows}
}


\newcommand{\dglocallyuniquelylabelleddirectedgraph}
{
\begin{array}{cp{0.75cm}c}
\dgbasic   &&  \Rnode{l}{l} 
\end{array}
\begin{arrows}
\ncarr{node}{l}
\alabel{label}
\idcomp
\ncarr{edge}{l}
\blabel{label}
\idcomp
\dgsrcedge  % redrawn so that I can bar it with \idcomp
\idcomp
\end{arrows}
}


\newcommand{\setdgexitsuniquelylabelled}
{
\setoflabelleddgs
%
\begin{arrows}
%redraw arrows and bar using \idcomp
\ncarr{dg}{text}
\idcomp
\ncarr{node}{number}
\idcomp
\ncarr{edge}{number}
\idcomp
\dgsrcedge  % redrawn so that I can bar it with \idcomp
\idcomp
\nodepartof  % repeat so thatI can bar with \idcomp
\idcomp
\end{arrows}
}




\newcommand{\structuralsetofsgincludingabs}
{
\begin{array}{cp{0.75cm}c}
\Rnode{abs}{abs}                       \\[1cm]
\structuralsetofdg   &&  \Rnode{u}{u} 
\end{array}
\begin{arrows}
\ncsar{dg}{abs}
\ncarr{dg}{u}
\alabel{name}
\idcomp
\ncarr{node}{u}
\alabel{label}
\idcomp
\ncarr{edge}{u}
\blabel{label}
\idcomp
\structuraldgsrcedge  % redrawn so that I can bar it with \idcomp
\idcomp
\structuralnodepartof  % repeat so that I can bar with \idcomp
\idcomp
\end{arrows}
}






\renewcommand{\erpictureFolder}[0]{../../SharedPictures}
\setcounter{equation}{0}


%\usepackage{arydshln} % vertical dashed lines between columns of an array

\title[John Cartmell]{Concept-instance algebra}
%% Which is to say types as they are used in practice in software development and as represented in theory in categories and in syntactic type theories.
%% There is also a subplot concerning representation of context which certain types depend on -- again represented in practice and in theory. 
\subtitle{Retelling B-systems}
%subtitle{... and making the trivial trivially trivial}
\author{John Cartmell}
\institute{\\}
\date{Aug 15, 2022}
\bibliographystyle{plainnat}
\usepackage{framed}
\usepackage{bibentry}
\usepackage{colortbl}
\usepackage{ulem}   % for \dashuline{dashing} for seconday key
\usepackage{listings}
\lstset{%
  escapeinside={(*}{*)},%
}
\usepackage{arydshln} % vertical dashed lines between columns of an array
\usepackage{pst-arrow} %for \bigarrow
\usepackage{hhline}

%Redefine the \Fin macro to be category of sets
%\renewcommand{\Fin}{\Set

\newcommand{\CItreemode}{D}  % so that trees are top down
\newcommand{\CItreesep}{2cm}  % horizontal spearation of limbs of top down tree

\newcommand{\slidecontext}{Introduction} % to be redefined before use

\newcommand{\comingnext}[1]{
\begin{frame}{COMING NEXT}
\begin{center}
\Large #1
\end{center}
\end{frame}
}


\begin{document}
\begin{frame}
\titlepage
\nobibliography{../../SharedBibliography/temp/bibliography}
\end{frame}


\begin{frame}{Overview}
Do I want a content style overview here?
\end{frame}
\iffalse

\begin{frame}{Introduction}
I wish to show that
\begin{itemize}
\item we can genericise database normal form criteria into abstract logical terms,
\item achieve generic goodness criteria that can be applied to all data specifications,
 \item that the classic relational database normal form criteria (2NF, 3NF, BCNF, INC-NF, 4NF, 5NF)  are then consequences of these generic goodness criteria.
\end{itemize}
\end{frame}

\begin{frame}{View}
A data specification  
\begin{itemize}
\item is a  theory (of what is)
\end{itemize}
\medskip
A data specification method 
\begin{itemize}
\item is a method for expressing such a  theory
\item unequivocally it enables definitions of types and certain relationships between these types
\item types are equally types of data and types of real world entity
\end{itemize}
More precisely, 
\begin{itemize}
\item data specification are \underline{presentations} of theories of what is,
\item choice of primitives in a given presentation is choice of which data to be stored or communicated.
\end{itemize}
\end{frame}



\begin{frame}{Goodness Criteria}
I will define two types of goodness criteria
\begin{itemize}
    \item Goodness Criteria of Type 1 -- absence of redundancy in presentation.
    \item Ensures absence of redundancy in data and in data management logic.
    \item Spelt out in more detail in criteria 1A, 1B and so on
    \item  Goodness Criteria of Type 2 -- the theory be the tightest fit to the facts 
    \item Two ways of expressing this. 
    \begin{itemize}
        \item Criteria 2 is that the theory is maximally constrained.
        \item Criteria 2A, 2B etc.  that it be logically complete in some sense.
    \end{itemize}
\end{itemize}
\end{frame}

\begin{frame}{Data Specification as Category}
\begin{itemize}
 \item Not surprising that a data specification can represented as a category or as a sketch of a category
\pause \item
What types of things are there and how are they related? 
\begin{itemize}
\item Data specifications provide the answer to this question in the context of a software development. 
\item Types theories provide the answer in the context of mathematics. 
\item Category theory abstracts across both these domains.
\end{itemize}
\pause \item Nor is it surpising if data specifications can be seen in terms
of contextual categories or a sketches for a contextual categories
because as categories model types so contextual categories model types that vary
and  our instinct for types and types that vary comes from the real world not from mathematics.
\end{itemize}
\end{frame}


\begin{frame}{Overview}
\begin{center}
\begin{tabular}{p{12cm}}
\begin{itemize}
    \item data specifications 
    \begin{itemize}
        \item as sketches of structured categories of some kind
        \item data instances as certain structure preserving functors to the category of finite sets $\Fin$
    \end{itemize}
    \item database specifications
    \begin{itemize}
         \item category has designated mono sources for some of its objects 
    \end{itemize}
    \item relational database specifications charcterised by
    \begin{itemize}
         \item no use of containment or nesting
         \item no  hierarchical organisation -- said to be flat
         \item no use of pointers 
         \item instead uses foreign keys to represent relationships in the data
    \end{itemize}
\end{itemize} 
\end{tabular}
\end{center}
\end{frame}

\begin{frame}{Overview}
\shrinkbox{0.4}{
\begin{pspicture}(-6,-6)(9,9)
 \psgrid
 \psset{fillstyle=solid,opacity=0.5}
 \pscircle[fillcolor=lightyellow]{1.5}
 \psRing[fillcolor=white]{1.5}{3.5}
 \psRing[fillcolor=lightyellow]{3.5}{5}

 \end{pspicture}
}
\end{frame}

\begin{frame}{Normal Forms}
\begin{center}
\begin{tabular}{p{12cm}}
\begin{itemize}
\item data specifications 
\begin{itemize}
    \item as sketches of structured categories of some kind
    \item data instances as certain structure preserving functors to the category of finite sets $\Fin$
\end{itemize}
\item database specifications
\begin{itemize}
     \item category has designated mono sources for some of its objects 
\end{itemize}
\end{itemize} \\
\hdashline
\begin{itemize}
\item relational database normal form criteria  
\begin{itemize}
    \item first normal form (1NF)
    \item 2nd normal form (2NF)
    \item 3rd normal form (3NF)
    \item Boyce-Codd normal form (BCNF)
    \item inclusion dependency normal form (INC-NF)
    \item 4th normal form (4NF)
    \item projection-join normal form (5NF)
\end{itemize}
\end{itemize}
\end{tabular}
\end{center}
\end{frame}

\iffalse
\begin{frame}{Levels of Data Specification}
\begin{itemize}
\item logical   -- a sketch for a category of some kind
\item structural -- a sketch for some kind of category with distinguished morphisms indicating structural relationships
\item representational -- structural PLUS representational indicators for non-structural relationships
\item technological    -- IDL, SQL, XML 
\end{itemize}
\end{frame}
\fi

\newcommand{\bigdownarrow}
{
\scalebox{0.3}
{
\begin{pspicture}(3,3.5) 
%\psgrid
%\psset{doublesep=2cm} 
\psBigArrow[fillstyle=solid, fillcolor=blue!30,linecolor=blue](2.0,3)(2.0,0)
\end{pspicture}
}
}
\begin{frame}{Levels of Data Specification}
\begin{center}
\begin{tabular}{c l}
                              & \raisebox{0cm}{\parbox{5cm}{sketch of category of some kind}}\\
\cline{1-1}
\multicolumn{1}{|c|}{logical} & \\
\cline{1-1}
\multicolumn{1}{c}{\bigdownarrow} & \raisebox{0.5cm}{\parbox{5cm}{distinguish morphisms/relationships represented in data by containment}} \\
\cline{1-1}
\multicolumn{1}{|c|}{structural} &\\
\cline{1-1}
\multicolumn{1}{c}{\bigdownarrow} & \raisebox{0.5cm}{\parbox{5cm}{add edges for foreign keys representing non-containment relationships and add path equivalences that define them}} \\
\cline{1-1}
\multicolumn{1}{|c|}{representational} & \\
\cline{1-1}
\multicolumn{1}{c}{\bigdownarrow} & \raisebox{0.5cm}{choice of technology} \\
\cline{1-1}
\multicolumn{1}{|c|}{technological} & \raisebox{0cm}{\parbox{5cm}{IDL, XML, SQL}}\\
\cline{1-1}
\end{tabular}
\end{center}
\end{frame}


\begin{frame}{Classic Relational Normal Form Criteria}
\begin{itemize}
    \item database normal forms are goodness criteria (GC) based on software engineering principles
    \item relational database normal form criteria  
    \begin{itemize}
        \item first normal form (1NF)
        \item 2nd normal form (2NF)
        \item 3rd normal form (3NF)
        \item Boyce-Codd normal form (BCNF)
        \item inclusion dependency normal form (INC-NF)
        \item 4th normal form (4NF)
        \item projection-join normal form (5NF)
    \end{itemize}
\end{itemize}
\end{frame}


\iffalse
\begin{frame}{Methods of Data Specification}
\begin{itemize}
	\item schema of relational database,
	\item structure described by Carnegie-Mellon IDL,
	\item schema of nested relational database,
	\item message structure described by Google protocol buffer IDL,
	\item XML schema language,
	\item ER script.
\end{itemize}
\end{frame}
\fi

\begin{frame}{Data Specifications}
Two kinds of types in play
\begin{itemize}
\item  the \textit{definienda} -- types all of whose instances are \term{particulars}
\begin{itemize}
\item employee, department, student, account, product, order, shipment, delivery, flight, booking and so on
\item molecular structure, atom, bond, element, isotope, reaction, metabolite, mass trace, chromatogram, peak
\item table, column, primary key, foreign key
\item node and edge. 
\end{itemize}
\pause 
\item  the \textit{definiens}  -- types all of whose instances are \term{universals}
\begin{itemize}
       \item string, integer, float, boolean and so on
\end{itemize}
\end{itemize}
\pause
\begin{itemize}
\item in ER modelling 
\begin{itemize}
\item the \textit{definienda} are called \textit{entity types}
\item the \textit{definiens} are called \textit{attribute types} or \textit{domains}.
\end{itemize}
\end{itemize}
\end{frame}


\begin{frame}{Data Specifications}
A data specification is a sketch of a category with some additional structure:
\begin{itemize}
\item that it is a \term{sketch} is crucial because it is only nodes and edges of the sketch for which data is stored and/or communicated, 
\item that there are commutative diagrams is crucial to construction of physical 
specifications from logical specifications.
\item that the category had additional structure is significant:
\begin{itemize}
\item so that we can give account of database normal forms 
(BCNF, 3NF, 4NF and 5NF),
\item so that we can allow for missing data as represented by NULL values, 
\item so that we can distinguish structural from non-structural relationships to describe structure nesting and thereby hierarchical data,
\item so that types of universals can be distinguished from types of particulars.
\end{itemize}
\end{itemize}
\end{frame}



\iffalse
\begin{frame}{Database Normal Forms}
\begin{itemize}
\item presentations (sketches) should be minimal and avoid redundancy:
\item how to make this precise?
\item in case of relational data model leads to 
   \begin{itemize}
     \item third normal form (3NF)
     \item Boyce-Codd normal form (BCNF)
     \item fourth normal form (4NF)
     \item fifth normal form (5NF)
   \end{itemize}
\end{itemize}
\end{frame}
\fi

\begin{frame}{Relational Database Theory}
\begin{itemize}
\item classic relational database normal form definitions ({\scriptsize 3NF, EKNF, BCNF, 4NF,5NF, INC-NF}) can be abstracted  into a general logical framework

\item such normal forms  examine the fit of a sketch/theory (database schema) to an intended usage

\item we can assume that the intended usage is represented by a full-subcategory of the category $Fun(S,\cat{FinSet})$

\item in such a situation the classic normal forms address the question can the sketch/theory $S$ be improved by addition or removal of morphisms and/or commutative diagrams and/or limit cones.
\item normalisation has dual goal of obtaining as complete a theory as possible and of eliminating redundancy from the sketch.  
\end{itemize}
\end{frame}

\begin{frame}{Normal Forms}
\begin{itemize}
\item IN-NF -- Ling and Goh -- there are no redundant attributes except if absolutely necessary in order to specify a mono source
\end{itemize}
\end{frame}

\iffalse %moved into goodness criteria section and reworded
\begin{frame}{Normalisation}
\begin{definition}
{ \footnotesize
If $T$ is a theory and $W \subset |Mod(T,FinSet)|$ is an intended usage then an interpretation (theory morphism) $I: T \morph T'$ is an improvement of $T$ wrt $W$ iff 
$Mod(I,Finset): Mod(U',Finset) \morph Mod(U,Finset)$ is injective but not surjective
and $W \subseteq img(Mod(I,Finset))$.
i.e. for all models $U \in W$ there exists $U' \in Mod(T,Finset)$ such that $I \circ U'=U$
$
\begin{array} {c p{2cm} c}
\Rnode{T}{T} && \\ [0.25cm]
             && \Rnode{finset}{Finset} \\ [0.25cm]
\Rnode{Tp}{T'}  
\end{array}
$
\ncarr {T}{finset}
\alabel{U}
\ncarr{T}{Tp}
\blabel{I}
\ncarr{Tp}{finset}
\blabel{U'} 
}
\end{definition}

\begin{definition}
If a theory $T$ has no improvement wrt to an intended usage $W$ then $T$ is said to be \textit{optimally formulated} wrt $W$.
\end{definition}
\end{frame}


\begin{frame}{Propositions}
\begin{itemize}
\item If a relational schema $R$ can be normalised to $R'$ then the associated theory $T$ of $R$ can be improved to the associated thery $T'$ of $R'$.

\item If a relational database schema is in normal form then its associated theory is optimally formulated.
\end{itemize}
\end{frame} 
\fi

\begin{frame}{Defining Candidate Keys and/or Identifying Relationships in an EA sketch.  }
\begin{itemize}
\item concept of \textit{candidate keys} used in relational database normal form definitions {\scriptsize (3NF, EKNF, BCNF)}
\item in ER model talk about \textit{identifying} families of relationships
\item in category theory such a key or a family of relationships is a mono source i.e. a to jointly monic family of morphisms
\item mono sources and hence candidate keys can be defined as limit cones
\item more than 99.99 percent of entity modelling uses just mono sources and no other limits
\end{itemize}
\end{frame}

\begin{frame}{Additional Structure}
\resizebox{11.3cm}{!}{
\newcommand{\featurelist}{\begin{tabular}{|l|l l|}
\hline 
\multirow{11}{1.5cm}{category}
                & finitary property        & \\
\cline{2-3}
                & pu-partition             & \\
\cline{2-3}
                & \multirow{2}{3.5cm}{mono-sources}  & \multicolumn{1}{|l|}{cannonical monos}  \\
\cline{3-3}
                &                                    & \multicolumn{1}{|l|}{epi-mono factorisation}   \\
\cline{2-3}
                & finite products          & \\
\cline{2-3} 
                & finite limits            & \\
\cline{2-3}
                & restrictions             & \\
\cline{2-3}
                & \multirow{2}{3.5cm}{distinguished morphisms} & \multicolumn{1}{|l|}{hierarchical}      \\
\cline{3-3}
                &                                             &  \multicolumn{1}{|l|}{non-hierarchical} \\
\cline{3-3}
                &                                             &  \multicolumn{1}{|l|}{pullbacks} \\
\cline{2-3}
                & finite coproducts                           &                                   \\
\hline                
\end{tabular}}
\featurelist
}
\end{frame}

\begin{frame}{How to proceed}
\begin{itemize}
\pause \item I will
\begin{itemize}
   \item give an example of nested tables of data
   \item describe relational model  and other data models 
   \pause \item touch on my favourite -- ER modelling and ER script
   \pause \item describe Boyce-Codd normal form (BCNF)
   \pause \item GCs for sketches of categories as data specifications 
   \pause \item GCs for sketches of categories with designated monos and epis
   \pause \item GCs for sketches of categories with designated monos and epis and with finite products
  \pause \item with some additional assumptions prove BCNF
\end{itemize}
\end{itemize}
\end{frame}




\begin{frame}{Concepts with a dependency on context}
\begin{center}
\raisebox{-0.5cm}{
\pspicture(0,-0.1)(1.1,1)
\psline(0,0)(0,1)(1,1)(1,0)(0,0)
\psline (0,0)(1,1)
%\psline(0,0.5)(1,0.5)
%\psline(0.5,0)(0.5,1)
\endpspicture
}
\end{center}
\begin{itemize}
\item \textit{"there are two triangles and these have six sides"}, 
\item  we are understanding \textit{side} to be a concept that 
depends on \textit{triangle} for context,
\item a side, therefore, is a dependent type of thing -- it is some thing to be held in the mind
in the context of some other thing,
\item I summarise this by writing $triangle \base side$.
\end{itemize}
\end{frame}


\begin{frame}
\begin{tabular}{c p{2cm} c }
\textit{I can help.} & &\textit{I will carry the can.}
\end{tabular}
\begin{itemize}
\item a word cannot be said to be `noun', `verb', 'adjective' or such like lest it be appearing in some grammatically OK sentence so `noun' and `verb` as types of thing are dependent on sentence-like types of thing
\item what we mean by \textit{sentence} depends on a\textit{language} as context
\item we have $language \base sentence$ and $sentence \base noun$, $sentence \base verb$ and so on.
\end{itemize}
There is a tree of concepts
\begin{displaymath}
\pstree[treemode=\CItreemode,levelsep=*0.65cm,treesep=\CItreesep,nodesep=0.05]
{
    \Tr{\circ}
}
{
    \pstree [levelsep=*0.85cm]
    {
		\Tr{language} 
	}
	{		  
		\pstree [levelsep=*0.85cm]
		{
				   \Tr{sentence} 
		}
		{
					\Tr{noun}
					\Tr{verb}
					\Tr{adjective} 
		}	
	}	
}
\end{displaymath}
\end{frame}

\begin{frame}
Its the same with types that vary. 
Concepts like `angle', `edge', `boundary', `bounding line' as we learm the concepts
we learn that do not stand alone; similarly
we have dependent concepts like  face of cube,  endpoint of line,   junction between lines, citizen (of a country), tangent (to a curve), atom of a molecule,  nucleus of a cell. 
\begin{tabular}{l l}
endpoint & of line\\
junction & between lines\\
citizen & of country\\
tangent & to curve\\
%atom & of molecule\\
nucleous & of cell \\
%angle & between lines \\
edge & of something \\
%boundary & of extent\\
%bounding line \\
face & of cube\\
\end{tabular}
\end{frame}

\begin{frame}
\begin{align*}
&\tstyle \isT{Triangle} \\
&\ofT{x}{Triangle} \tstyle \isT{Side(x)} 
\end{align*}  

\begin{align*}
&\tstyle \isT{Country} \\
&x \in Country \tstyle Citizen(x) \mbox{ is a type}
\end{align*}
\noindent we might then go on to write: 
\begin{equation*}
\tstyle JohnC \in Citizen(UK)
\end{equation*}
which is well formed providing that:
\begin{equation*}
\tstyle \ofT{UK}{Country}
\end{equation*}
\end{frame}

\begin{frame}
\begin{equation*}
x \in Triangle, y \in Angle(x): oppositeSide(x,y) \in Side(x)
\end{equation*}

\noindent Having this notation then leads us to be able to make type assertions such as
\begin{equation*}
x \in Country \tstyle headOfState(x) \in Citizen(x)
\end{equation*}
\end{frame}

\begin{frame}
\noindent A second possibility is to represent a dependency between one type B and another A by a directed 
edge $B \smorph A$ then in any particular 
situation the types and their dependencies form a directed graph.
\\

\noindent Furthermore, any directed graph makes sense as, and can be interpreted as, a set of types and type dependencies 
providing (i) there are no cycles in
the graph and provided that (ii) all nodes B there are only finitely many A, such that $B \smorph A$ and (iii) there are 
no infinite sequences of the form $A_1 \smorph A_2 \smorph A_3 ....$. We will call any directed graph that 
meets these conditions a dependency graph or  d-graph and we will call the edges dependencies. 
\end{frame}

\begin{frame} The directed graph:
\begin{equation}
\begin{array}{ccc}
\Rnode{C}{C}   &            &                 \\ [0.8cm]
\Rnode{B1}{B_1}&            &\Rnode{B2}{B_2}  \\ [0.8cm]
               &\Rnode{A}{A}&                 
\end{array}
\ncsar{C}{B1}
\ncsar{B1}{A}
\ncsar{B2}{A} 
\end{equation}
\\

\noindent interpreted as types and type dependencies expresses the following:
\addtocounter{equation}{-1}
\begin{subequations}
\begin{align}
&A\mbox{ is a type} \\
&x\in A : B_1(x) \mbox{ is a type} \\
&x\in A : B_2(x) \mbox{ is a type} \\
&x\in A, y\in B_1(x): C(x,y) \mbox{ is a type}
\end{align}
\end{subequations}

\end{frame}


\begin{frame}
Similarly the directed graph:

\begin{center}
$
\begin{array}{p{1.5cm}cccp{2cm}c}
&                & \Rnode{C2}{C}&   \\ [0.8cm]
&                & \Rnode{B}{B} &  \\ [0.8cm]
&\Rnode{A1}{A_1} &              & \Rnode{A2}{A_2}\\ [0.4cm]
&                &\stepcounter{equation}(\theequation) &  \\
\end{array}
$
\ncsar{C2}{B}
\ncsar{B}{A1}
\ncsar{B}{A2}
\setlength {\saroffsetA}{-2pt}
\setlength {\saroffsetB}{-2pt}
\ncsar[-15]{C3}{B3}
\setlength {\saroffsetA}{2pt}
\setlength {\saroffsetB}{2pt}
\ncsar[15]{C3}{B3}
\sarreset
\ncsar{B3}{A3}
\end{center}

\noindent can be interpreted as representing the following type system:

\addtocounter{equation}{-1}
\begin{align}
&A_1\mbox{ is a type} && \tag*{(\theequation a)}\\
&A_2\mbox{ is a type} && \tag*{(\theequation b)}\\
&x_1\in A_1, x_2 \in A_2 : B(x_1,x_2) \mbox{ is a type} && \tag*{(\theequation c)}\\
&x_1\in A_1, x_2 \in A_2, y \in B(x_1,x_2): C(x_1,x_2,y) \mbox{ is a type} && \tag*{(\theequation d)}\\
\end{align}
\end{frame}

\begin{frame}
A third notation is an extended style of entity modelling style developed by the author and documented at \href{www.entitymodelling.org}{www.entitymodelling.org}. In this notation, type dependencies are represented by what are called composition relationships; these are distinguished from other functional relationships by being draw leaving the
lower edge of a box and entering the upper edge of the box representing the dependent type. \\

\noindent Examples are given in figures \ref{partsOfSpeech} and \ref{citizen}.

\begin{center}
\begin{figure} [H]
\hspace {1.5cm}
(a)
\begin{tabular}{>{\textit} l l}
Symbol & \itshape{Introductory Rule} \\ 
\hline 
language &$\tstyle \isT{language} $\\
sentence &$x \in language \tstyle sentence(x) \mbox{ is a type} $\\
word &$x \in language \tstyle word(x) \mbox{ is a type} $\\
noun &$x \in language ,\ y \in sentence(x)  \tstyle noun(y) \mbox{ is a type} $\\
verb &$x \in language ,\ y \in sentence(x)  \tstyle verb(y) \mbox{ is a type} $\\
adjective &$x \in language ,\ y \in sentence(x)  \tstyle adjective(y) \mbox{ is a type}$
\end{tabular} 
\vspace{0.5cm}

\hspace{0.5cm}
(b)
\setlength{\arraycolsep}{0cm}
$
\begin{array}{ c c c c}
&                            &  \Rnode{1}{1}               &             \\ [1.3cm]
&                            & \Rnode{language}{language}  &             \\ [1.3cm]
& \Rnode{sentence}{sentence} &                     & \Rnode{word}{word}\\ [1.3cm]
\Rnode{noun}{noun}         & \Rnode{verb}{verb}  & \Rnode{adjective}{adjective} &
\end{array}
$
\hspace {1.0cm}
(c)
\erinlinePicture{partsOfSpeech}{4}
\ncsar{language}{1}
\ncsar{sentence}{language}
\ncsar{word}{language}
\ncsar{noun}{sentence}
\ncsar{verb}{sentence}
\ncsar{adjective}{sentence}
\caption{Three representations of a system of types (a) rules in a formal mathematical syntax,
(b) a graph of  type dependencies (c) entity modelling notation.}
\label{partsOfSpeech}
\end{figure}
\end{center}
\end{frame}

\begin{frame}
\begin{center}
\begin{figure} [H]
(a) \erinlinePicture{performanceOfPlayAnnotated}{3.5}

(b)
\begin{minipage}[c]{0.6 \textwidth}
\begin{tabular}{l l}
Symbol & \itshape{Introductory Rule} \\ 
\hline 
$play$ &$\tstyle \isT{play} $\\
$performance$ &$\tstyle \isT{performance} $\\
$of       $& $ x \in performance \tstyle \ofT{of(x)}{play}$ \\
$character$&$x \in play \tstyle character(x) \mbox{ is a type} $\\
$castMember$&$x \in performance \tstyle castMember(x) \mbox{ is a type}$ \\
$playsPartOf$& $ x \in performance ,\ y \in castMember(x) \tstyle \ofT{playsPartOf(y)}{character} $\\
\end{tabular}
\end{minipage}
\caption{A cast member plays a part in the same play that they are pert of the performance of. This is reflected in (b) by examination of the variables in the intoductory rule for $playsPartOf$. In the entity modelling diagram, (a), this is documented in a scope constraint (~/.. = ../of)  which specifies that the square of relationships commutes. See \href{http://www.entitymodelling.org/tutorialone/scopediagrams.html}
{www.entitymodelling.org/tutorialone/scopediagrams}}.
\label{citizen}
\end{figure}
\end{center}
\end{frame}

\begin{frame}{Equivalent paths }

Consider the following two rules:
\begin{subequations}
\begin{align} 
           & x_1 \in A, y_1 \in B_1(x_1), x_2 \in A, y_2 \in B_2(x_2) : C(x_1,y_1,x_2,y_2) \mbox{ is a type}\label{multipleInstanceA}\\
\mbox{and}& \notag \\
           & x \in A, y_1 \in B_1(x), y_2 \in B_2(x) : C(x,y_1,y_2) \mbox{ is a type} \label{singleInstanceA}
\end{align}
\end{subequations}

\noindent They both give rise to a graph like this:

\begin{equation}
\begin{array}{ccc}
               &\Rnode{C}{C}   &             \\ [0.8cm]
\Rnode{B1}{B_1}&            &\Rnode{B2}{B_2}  \\ [0.8cm]
               &\Rnode{A}{A}&                 \\
\end{array}
\ncsar{C}{B1}
\ncsar{C}{B2}
\ncsar{B1}{A}
\ncsar{B2}{A} 
\end{equation}
\\

In the case of \eqref{singleInstanceA}, $C$ relies for context on instances of $B_1$ and $B_2$ which themselves are based within the context of a common instance of $A$.

In the case of \eqref{singleInstanceA}, but not in case \eqref{multipleInstanceA}, we say that the path:
$ C \smorph B_1 \smorph A$ is equivalent to the path: $ C \smorph B_2 \smorph A$.
\end{frame}

\begin{frame}
We call any graph endowed with such an equivalence relation a dependency structure. It follows from the definitions that every set of type definitions gives rise to a dependency structure and it is easy to see that the converse is true: that every dependency structure, up to isomorphism of structures, arises as the dependency structure of a set of type definitions.
\\ 

In order to specify a dependency structure it is necessary to specify the dependency graph and then for 
each subgraph of the form:

\begin{math}
\begin{array}{p{3cm}cccc}
&                      &\Rnode{C}{C}&                      & \\ [0.8cm]
& \Rnode{Ln}{L_n}      &            &\Rnode{Rm}{R_m}       & \\ [0.6cm]
& \Rnode{Ln1}{L_{n-1}} &            &\Rnode{Rm1}{R_{m-1}}  & \\
& \Rnode{Ldots}{\vdots}&            &\Rnode{Rdots}{\vdots} & \\
& \Rnode{L1}{L_1}      &            &\Rnode{R1}{R_1}       & \\ [0.8cm]
&                      &\Rnode{A}{A}&                      & \\
\end{array}
\ncsar{C}{Ln}
\ncsar{C}{Rm}
\ncsar{Ln}{Ln1}
\ncsar{Rm}{Rm1}
\ncsar{L1}{A}
\ncsar{R1}{A} 
\end{math} 
\\
whether the two paths left and right are equivalent. This is to say whether there are one or two instances of $A$ 
depended on by $C$. 
 
Hiearchical dependency structures, developed further, lead to the notion of contextual category (described  in \cite{Cartmell78} and subsequently published in \cite{Cartmell86}). 
\end{frame}










\begin{frame}{Reminder}
In a contextual category,
\begin{itemize}
\item there is a well-founded partial ordering $<$ between objects
such that there is  a rooted $\omega$-tree of objects,
\item if $y$ covers $x$ in the partial order then write $x \base y$,
\item whenever $x \base y$ in the partial order then there is a distinguished
morphism $p_y:y \morph x$ which I write as $y \smorph x$,
\item the root object is a terminal object in the category,
\item the root object represents the empty context, 
\item non-root objects are at the same time both contexts and types
because the syntactic \underline{difference beween contexts and types} is 
just that - \underline{a syntactic difference}. 
\end{itemize}
\end{frame}



\begin{frame}{Concept-Instance Algebra -  Overview}
In a concept-instance algebra there are concepts and instances and there are operations $^*$, $\crossx{}{}{}$ and $\delta$
and these must satisfy various axioms.
\begin{itemize}
\item there is a rooted $\omega$-tree of concepts,
\item for each non-root concept there is a associated set of instances (instances of that concept),
\item operation $^*$ enables  concepts and instances to be particularised (syntactically this is substitution),
\item  operation $\crossx{}{}{}$ enables the positing in-scope concepts and instances (syntactially this is sometimes called weaking),
\item operation $\delta$ is an expression of self (as for example the identity morphisms in a category).
\end{itemize}
\end{frame}

\begin{frame}{Summary of Operations -- $^*$, $\cross$ and $\delta$}
Summary: assume $x_c \base x$ and $y_c \base y$ in the tree of concepts of an algebra $A$. \\
\medskip
\scalebox{0.9}{%scope for tabcolsep
\setlength\tabcolsep{2pt}
\begin{tabular}{|l|l|p{3cm}|l|}
\hline
            &     reads as                      & defined whenever    & and is such that \\
\hhline{|=|=|=|=|}
$f^*y$      & particularise $y$ to $f$  & $f \in inst(x)$\newline \rule{0.8cm}{0pt} and $x<y$   &\begin{tabular}[t] {l}
                                                                              $x \base y$ then $x_c \base f^*y$\\
                                                                              $x \ll y$ then $f^*y_c \base f^*y$
                                                                           \end{tabular}                              \\
\hline
$f^*g$       & particularise $g$ to $f$ & if also $g\in inst(y)$      & $f^*g \in inst(f^*y)$                          \\
\hline
$x \cross y$ &posit $y$ in context $x$  & $x_c < y$                      &\begin{tabular} {l}
                                                                              $x_c \base y$ then $x \base (x \cross y)$\\
                                                                              $x_c \ll y$ then $(x \cross y_c) \base (x \cross y)$
                                                                           \end{tabular}                              \\
\hline
$x \cross g$ &posit $g$ in context $x$  & if also $g\in inst(y)$            & $x \cross g \in inst(x \cross y)$            \\
\hline
$\delta_x$   & this $x$                 &                         & $\delta_x \in inst(x \cross x)$              \\
\hline
\end{tabular}
}
\end{frame}

\begin{frame}{Concepts and their Contexts}
\begin{itemize}
\item In a concept-instance algebra $A$ concepts are equally contexts.
\item If $a \base b \base c$ in the tree of concepts of algebra $A$ 
then $a$ is  the context for concept $b$ and $b$ is to  the context for concept $c$.
\end{itemize} 
\begin{tabular}{c c c c}
\raisebox{-4cm}{\rule{0cm}{4cm}} % to claim space and stop slide wiggle in animation
 &
\onslide<2->{\only<2-3>{$\begin{displaymath}
\pstree[treemode=\CItreemode,levelsep=*0.65cm,treesep=\CItreesep,nodesep=0.05]
{
    \Tr{\circ}
}
{
    \pstree [levelsep=*0.85cm]
    {
		\Tr{language} 
	}
	{		  
		\pstree [levelsep=*0.85cm]
		{
				   \Tr{sentence} 
		}
		{
					\Tr{noun}
					\Tr{verb}
					\Tr{adjective} 
		}	
	}	
}
\end{displaymath}$}\only<4->{\def\psedge{\ncksar}$\begin{displaymath}
\pstree[treemode=\CItreemode,levelsep=*0.65cm,treesep=\CItreesep,nodesep=0.05]
{
    \Tr{\circ}
}
{
    \pstree [levelsep=*0.85cm]
    {
		\Tr{language} 
	}
	{		  
		\pstree [levelsep=*0.85cm]
		{
				   \Tr{sentence} 
		}
		{
					\Tr{noun}
					\Tr{verb}
					\Tr{adjective} 
		}	
	}	
}
\end{displaymath}$}}
& &
\onslide<3->{\only<3-4>{$\pstree[treemode=\CItreemode,levelsep=*0.65cm,treesep=\CItreesep,nodesep=0.05]
{
    \Tr{\circ}
}
{
    \pstree [levelsep=*0.85cm]
    {
		\Tr{match} 
	}
	{		  
		\pstree [levelsep=*0.85cm]
		{
				   \Tr{side} %\uppermember {homeSide}
		}
		{
					\Tr{player} %\uppermember {captain}
		}
	    \pstree [levelsep=*0.85cm]
		{
			\Tr{innings} 
		}
		{		  
		    \pstree [levelsep=*0.85cm]
			{
					   \Tr{over} 
			}
			{   
					   \Tr{delivery} 
			}			
		}		
	}	
}$}\only<5>{\def\psedge{\ncksar}$\pstree[treemode=\CItreemode,levelsep=*0.65cm,treesep=\CItreesep,nodesep=0.05]
{
    \Tr{\circ}
}
{
    \pstree [levelsep=*0.85cm]
    {
		\Tr{match} 
	}
	{		  
		\pstree [levelsep=*0.85cm]
		{
				   \Tr{side} %\uppermember {homeSide}
		}
		{
					\Tr{player} %\uppermember {captain}
		}
	    \pstree [levelsep=*0.85cm]
		{
			\Tr{innings} 
		}
		{		  
		    \pstree [levelsep=*0.85cm]
			{
					   \Tr{over} 
			}
			{   
					   \Tr{delivery} 
			}			
		}		
	}	
}$}}
\end{tabular}

\end{frame}

\def\psedge{\ncksar}  % Now redefine globally

\begin{frame}{Concepts, Contexts  and Instances}
\begin{itemize}
\item If $x_c \base x$  in algebra $A$ and if $i \in inst_A(x)$ thn $i$ is to be considered an instance of $x$ in context $x_c$.
\onslide<2-> {\item  $homeSide \in inst(side)$ and $match \base side$ implies  $match$ is context for $homeSide$}
\onslide<4-> {\item $captain \in inst(player)$ and $side \base player$ implies $side$ is context for $captain$.}
\end{itemize}
\onslide<3->{$$\vspace{0.6cm}
\begin{displaymath}
\pstree[treemode=R,levelsep=*0.65cm,treesep=1cm,nodesep=0.05]
{
    \Tr{\circ}
}
{
    \pstree [levelsep=*0.85cm]
    {
		\Tr{match} 
	}
	{		  
		\pstree [levelsep=*0.85cm]
		{
				   \Tr{side} \uppermember {homeSide}
		}
		{
					\Tr{player} \waitfor{4}{\uppermember {captain}}
		}
	    \pstree [levelsep=*0.85cm]
		{
			\Tr{innings} 
		}
		{		  
		    \pstree [levelsep=*0.85cm]
			{
					   \Tr{over} 
			}
			{   
					   \Tr{delivery} 
			}			
		}		
	}	
}
\end{displaymath}$$}
\end{frame}






\begin{frame}{Particularisation(1)}
In a concept-instance algebra $A$

\begin{itemize}
\item if $x$ is the context for $y$
\item if $f$ is an instance of $y$ and $y$ provides context to some context $z$
\item then  $f^*z$  is defined and represents $z$ particularised to instance $f$.
\item if $y$ is the entire context for $z$ then $x$ is the entire context for $f^*z$
otherwise if $z_c$ is the entire context of $z$ then $f^*z_c$ is the entire context for $f^*z$.
\end{itemize}
\medskip
\pause
\begin{tabular} {c p{0.2cm} c}
(a) if $y \base z$ then $x \base f^*z$   &&   (b) if $z_c \base z$ and $y <z_c$ then $f^*z_c \base f^*z$ \\[0.5cm]
$\displaystyle
% I coded it this way with $ and displaystyle rather than with displaymath because uit is the only way that i caould
% find that enabled me to include in a tabular display
\pstree[treemode=\CItreemode,levelsep=*0.65cm,treesep=0.8cm,nodesep=0.05]
{
	\Tr{x_c}
}
{
   \pstree
	{
	   \Tr{x}\member{f}
	}
	{
		\Tr{y} 
	}
	\onslide<4->{\Tr{f \sub y}}
}
$&&$\displaystyle
% I coded it this way with $ and displaystyle rather than with displaymath because uit is the only way that i caould
% find that enabled me to include in a tabular display
\pstree[treemode=\CItreemode,levelsep=*0.65cm,treesep=\CItreesep,nodesep=0.05]
{
	\Tr{x}
}
{
   	\pstree[levelsep=*2.0cm]
	{
	   \Tr{y}\member{f}
	}
	{
		\pstree[levelsep=*0.75cm]
	   	{
	     	\Tr[edge=\dottededge]{z_c}
	   	}
	   	{
			\Tr{z}
	   	} 
	}
	\pstree[nodesep=0, levelsep=*0.85cm] %latest added levelsep
	{
	   \Tr{\ .}
	}
	{
		\pstree[nodesep=0.05,levelsep=*0.75cm]
	   	{
	     	\Tr[edge=\dottededge]{f^*z_c}
	   	}
	   	{
			\Tr{f^*z}
	   	} 
	}
}
$ \\
\end{tabular}
\end{frame}

\begin{frame}{Particularisation cont.}
if, in addition, $g$ is a instance of $z$ then $f^*g$ is defined and is a concept at $f^*z$. \\

\medskip
\pause
\begin{tabular} {c p{0.2cm} c}
(a) if $y \base z$ then $x \base f^*z$   &&   (b) if $z_c \base z$ and $y <z_c$ then $f^*z_c \base f^*z$ \\[0.5cm]
\vspace{0.5cm}
$\displaystyle
\pstree[treemode=\CItreemode,levelsep=*0.65cm,treesep=\CItreesep,nodesep=0.05]
 {
		\Tr{x}
 }
 {
    {\pstree
		   {\Tr{y}\member{f}}
			 {
			 \Tr{z} \member{g}
			 }
		 \Tr{f \sub z} \uppermember{f \sub g}
		}
}
$&&\vspace{0.5cm}
$\displaystyle
% I coded it this way with $ and displaystyle rather than with displaymath because uit is the only way that i caould
% find that enabled me to include in a tabular display
\pstree[treemode=R,levelsep=*1.0cm,treesep=1cm,nodesep=0.05]
{
	\Tr{x}
}
{
   	\pstree[levelsep=*2.0cm]
	{
	   \Tr{y}\uppermember{f}
	}
	{
		\pstree[levelsep=*0.75cm]
	   	{
	     	\Tr[edge=\dottededge]{z_c}
	   	}
	   	{
			\Tr{z} \member{g}
	   	} 
	}
	\pstree[nodesep=0, levelsep=*0.85cm] %latest added levelsep
	{
	   \Tr{\ .}
	}
	{
		\pstree[nodesep=0.05,levelsep=*0.75cm]
	   	{
	     	\Tr[edge=\dottededge]{f^*z_c}
	   	}
	   	{
			\Tr{f^*z} \member{f^*g}
	   	} 
	}
}
$ \\
\end{tabular}
\end{frame}

\begin{frame}{Particularisation cont. (2)}
At this point it follows that if $y \base z_1 ... \base z_m \base z$ and $g$ is a instance of $z$ then we have:
\begin{displaymath}
\pstree[treemode=\CItreemode,levelsep=*0.65cm,treesep=\CItreesep,nodesep=0.05]
 {
    %\Tr{\circ}
		\Tr{x}
 }
 {%\flexbranch{Lxn}{1cm}{1cm}{x}{n}{}
    {\pstree
		   {\Tr{y}\member{f}}
			 {
			 \flexbranchplusarc{Ly}{1cm}{1cm}{z}{m}{}{g}
			 }
		 \flexbranchplusarc{Lfy}{1cm}{1.3cm}{f \sub z}{m}{} {f \sub g}
		}
}
\end{displaymath}
\vspace{0.5cm} 
\end{frame}









\begin{frame}{Positing $\cross$}
If $w,x$ and $y$ are concepts of $A$  and if $w \base x$ and $w < y$ then $\crossx{x}{y}{w}$ is a concept of $A$ then
\pause
\begin{tabular} {c p{0.2cm} c}
(a) if $w \base y$ then $x \base \crossx{x}{y}{w}$   &&   (b) f $y \base y'$ and $y <z$ then $ \crossx{x}{y}{w}  \base \crossx{x}{y'}{w}$ \\[0.5cm]
$\displaystyle
% I coded it this way with $ and displaystyle rather than with displaymath because uit is the only way that i caould
% find that enabled me to include in a tabular display
\pstree[treemode=\CItreemode,levelsep=*0.65cm,treesep=\CItreesep,nodesep=0.05]
{
	\Tr{x_c}
}
{
   \pstree
	{
	   \Tr{x}
	}
	{
		\onslide<4->{\Tr{x \cross y}}
	}
	\Tr{y}
}
$&&$\displaystyle
% I coded it this way with $ and displaystyle rather than with displaymath because uit is the only way that i caould
% find that enabled me to include in a tabular display
\pstree[treemode=\CItreemode,levelsep=*0.65cm,treesep=\CItreesep,nodesep=0.05]
{
	\Tr{x_c}
}
{
   	\pstree[levelsep=*2.0cm]
	{
	   \Tr{x}
	}
	{
		\pstree[levelsep=*0.75cm]
	   	{
	     	\Tr[edge=\dottededge]{\crossx{x}{y_c}{w}\nudgeup{0.5cm}}  
	     	              %use nudgeup to counter balances the vert space of the subscript w
	   	}
	   	{
			\Tr{\crossx{x}{y}{w}\nudgeup{0.5cm}}
	   	} 
	}
	\pstree[nodesep=0, levelsep=*0.85cm] %latest added levelsep
	{
	   \Tr{\ .}
	}
	{
		\pstree[nodesep=0.05,levelsep=*0.75cm]
	   	{
	     	\Tr[edge=\dottededge]{y_c}
	   	}
	   	{
			\Tr{y}
	   	} 
	}
}
$ \\
\end{tabular}
\end{frame}

\begin{frame} {Positing cont.}

if in addition $g$ is a instance of $y$ then $\crossx{x}{g}{w}$ is an instance of $\crossx{x}{y}{w}$.
\pause
\begin{tabular} {c p{0.2cm} c}
(a) if $w \base y$ then $x \base \crossx{x}{y}{w}$   &&   (b) f $y \base y'$ and $y <z$ then $ \crossx{x}{y}{w}  \base \crossx{x}{y'}{w}$ \\[0.5cm]
$\displaystyle
% I coded it this way with $ and displaystyle rather than with displaymath because uit is the only way that i caould
% find that enabled me to include in a tabular display
\pstree[treemode=\CItreemode,levelsep=*0.65cm,treesep=\CItreesep,nodesep=0.05]
{
	\Tr{w}
}
{
   \pstree
	{
	   \Tr{x}
	}
	{
		\Tr{\crossx{x}{y}{w}\nudgeup{0.5cm}} \member{\crossx{x}{g}{w}} 
		 %use nudgeup to counter balances the vert space of the subscript w
	}
	\Tr{y} \member{g}
}
$&&\vspace{0.5cm}
$\displaystyle
% I coded it this way with $ and displaystyle rather than with displaymath because uit is the only way that i caould
% find that enabled me to include in a tabular display
\pstree[treemode=R,levelsep=*1.0cm,treesep=1cm,nodesep=0.05]
{
	\Tr{w}
}
{
   	\pstree[levelsep=*2.0cm]
	{
	   \Tr{x}
	}
	{
		\pstree[levelsep=*0.75cm]
	   	{
	     	\Tr[edge=\dottededge]{\crossx{x}{y_c}{w}\nudgeup{0.5cm}}  
	     	              %use nudgeup to counter balances the vert space of the subscript w
	   	}
	   	{
			\Tr{\crossx{x}{y}{w}\nudgeup{0.5cm}}\member{\crossx{x}{g}{w}}
	   	} 
	}
	\pstree[nodesep=0, levelsep=*0.85cm] %latest added levelsep
	{
	   \Tr{\ .}
	}
	{
		\pstree[nodesep=0.05,levelsep=*0.75cm]
	   	{
	     	\Tr[edge=\dottededge]{y_c}
	   	}
	   	{
			\Tr{y} \member{g}
	   	} 
	}
}
$ \\
\end{tabular}
\end{frame}

\begin{frame} {Positing cont. (2)}
At this point it follows that if $w \base y_1 ... \base y_m \base y$ and $g$ is a instance of $y$ then we have

\begin{displaymath}
\pstree[treemode=R,levelsep=*0.5cm,treesep=1cm,nodesep=0.05]
 {
   \Tr{w}
 }
 {%\flexbranch{Lx}{0.9cm}{0.5cm}{x}{n}{}
   {
	  \flexbranchplusarc{Lym}{1cm}{1cm}{y}{m}{}{g}
	  \pstree[levelsep=*0.5cm,nodesep=0.05]
		{\Tr{x}}
		{
		  \flexbranchplusarc{Lxym}{1cm}{1.6cm}{x \cross_w y}{m}{}{x \cross_w g}
		}
	 }
 }
\end{displaymath}
\vspace{0.5cm}
\end{frame}
\begin{frame}{Example}
\begin{displaymath}
\pstree[treemode=\CItreemode,levelsep=*0.65cm,treesep=0.5cm,nodesep=0.05]
%{
%    \Tr{\circ}
%}
%{
    %\pstree [levelsep=*0.85cm]
    {
		\Tr{match} 
	}
	{		  
		%\Tr{side} 
	    \pstree [levelsep=*0.85cm]
		{
			\Tr{innings} 
		}
		{		  
			\pstree [levelsep=*0.85cm]
			{
				\Tr{innings \cross side} \member {fieldingSide}
			}
			{
				\Tr{innings \cross player}
			}
			%\Tr{fieldingSide^*(innings \cross player)}	
			\pstree
			{
             	\Tr{over}
			}
			{
				\Tr{over \cross (fieldingSide^*(innings \cross player))} \uppermember{bowler}
			}	
		}		
	}	
%}
\end{displaymath}
\end{frame}

\begin{frame}{Example}
A side can be posited in the context of an innnings and the batting side is an instance of such a thing:
\begin{displaymath}
\pstree[treemode=\CItreemode,levelsep=*0.65cm,treesep=\CItreesep,nodesep=0.05]
{
    \Tr{\circ}
}
{
    \pstree [levelsep=*0.85cm]
    {
		\Tr{match} 
	}
	{		  
		\Tr{side} 
	    \pstree [levelsep=*0.85cm]
		{
			\Tr{innings} 
		}
		{		  
			\Tr{innings \cross side} \uppermember {battingSide}		
		}		
	}	
}
\end{displaymath}
\end{frame}






\begin{frame}{self $delta$}
If $x$ is a concept of $A$  then $\delta_x$ is an instance of $x \cross x$
\vspace{0.5cm}
\begin{displaymath}
\pstree[treemode=\CItreemode,levelsep=*0.65cm,treesep=\CItreesep,nodesep=0.05]
{
   \Tr{x}
}
{
	\Tr{x \cross x} \member{\diag_x}
}
\end{displaymath}
 
\end{frame}


\fi
\newcommand{\objaxiom}[3]{\onslide<3->{#1&=#2&&\mbox{#3}}}
\newcommand{\instaxiom}[3]{\onslide<4->{#1&=#2&&\mbox{#3}}}
\newcommand{\otheraxiom}[3]{\onslide<5->{#1&=#2&&\mbox{#3}}}
\begin{frame}{Axioms}
6 pairs of axioms plus 2 others.
\pause \small \begin{align*}
\objaxiom{(f^*g)^*(f^*z) }{ f^*(g^*z)}{$f \in inst(x),\ g \in inst(y),\ x < y$ and  $y < z$}\\
\instaxiom{(f^*g)^*(f^*h) }{ f^*(g^*h)        
}{as above and $h \in inst(z)$ }\\
\objaxiom{(x \cross y) \cross (x \cross z) }{ x \cross (y \cross z)
}{$x_c < y$, where $x_c \base x$, and $x_c < z$} \\
\instaxiom{(x \cross y) \cross (x \cross g) }{ x \cross (y \cross g)
}{as above and $g \in inst(z)$}\\
\objaxiom{f^*(x \cross y) }{ y               
}{ $f \in inst(x)$ and $x_c < y$, where $x_c \base x$}\\
\instaxiom{f^*(x \cross g) }{ g               
}{as above and $g \in inst(y)$}\\
\objaxiom{f^*y \cross f^*z }{ f^*(y \cross z)  %problem here  
}{$f \in inst(x)$, $x < z$, $x < y$ and $y_c <z$ where $y_c \base y$}\\
\instaxiom{f^*y \cross f^*g }{ f^*(y \cross g)    
}{as above  and $g \in inst(z)$}\\
\objaxiom{(x \cross g)^*(x \cross z)}{x \cross(g^*z) 
}{$g \in inst(y)$ and $x_c < y$ and $y < z$, where $x_c \base x$}\\
\instaxiom{(x \cross g)^*(x \cross h)}{x \cross(g^*h) 
}{as above and  $h \in inst(z)$}\\
\objaxiom{\delta_x ^*(x \cross y)}{y                 
}{$x < y$}\\
\instaxiom{\delta_x ^*(x \cross g)}{g                 
}{$x < y$ and $g\in inst(y)$}\\
\otheraxiom{f \sub \diag(x) }{ f                       
}{$f \in inst(x)$}\\
\otheraxiom{f \sub \diag(y) }{ \diag(f \sub y)         
}{$f \in inst(x)$ and  $x < y$.}
\end{align*}
\end{frame}

\newcommand{\inningsCrossSide}{\crossx{innings\kern-0.3cm}{\kern-0.3cm side}{match}}
\newcommand{\inningsCrossPlayer}{\crossx{innings\kern-0.3cm}{\kern-0.3cm player}{match}}
\newcommand{\fieldingSidePlayer}{fieldingSide ^* (\inningsCrossPlayer)}
\newcommand{\battingSidePlayer}{battingSide ^* (\inningsCrossPlayer)}
\newcommand{\overCrossFieldingSidePlayer}{\crossx{over\kern-0.4cm}{\kern-0.4cm(\fieldingSidePlayer)}{innings}}
\newcommand{\deliveryCrossBattingSidePlayer}{\crossx{delivery\kern-0.4cm}{\kern-0.4cm (\battingSidePlayer)}{innings}}

\begin{frame}{Extending the $\crossx{}{}{}$ Notation}
Define $\crossx{x}{y}{w}$ whenever $w < x$ and $w<y$.

If $w \base x_1 \base ... \base x_n $
and $w \base y_1 \base ... \base y_m$ then define 
$$\crossx{x_n}{y_m}{w}=x_n \cross ( x_{n-1} \cross ... (x_1 \cross y_m)...)$$

\textit{\Large could improve this animation}
\pause
\begin{center}
$\displaystyle
% I coded it this way with $ and displaystyle rather than with displaymath because uit is the only way that i caould
% find that enabled me to include in a tabular display
\pstree[treemode=\CItreemode,levelsep=*0.65cm,treesep=\CItreesep,nodesep=0.05]
{
	\Tr{w}
}
{
   	\pstree[levelsep=*0.85cm]
	{
	   \Tr{x_1}
	}
	{   \pstree[levelsep=*0.65cm]
	    {
	    	\Tr[edge=\dottededge]{x_n}
	    }
	    {
            \onslide<3->
            {
				\pstree[nodesep=0.05,levelsep=*0.85cm]
			   	{
			     	\Tr{\crossx{x_n}{y_1}{w}} 
			   	}
			   	{
					\Tr[edge=\dottededge]{\crossx{x_n}{y_m}{w}}
			   	}
		   	}
	   	} 
	}
	\pstree[nodesep=0, levelsep=*0.95cm] %latest added levelsep
	{
	   \Tr{y_1}
	}
	{
			\Tr[edge=\dottededge]{y_m}
	}
}
$
\end{center}
\end{frame}

\begin{frame}{Extended $\delta$ Notation}
\textit{\Large could improve this animation}

If $1 \base x_1 \base x_2 ... \base x_n$
then \foreachi, define 
$\delta_{x_n,x_i} \in inst(\crossx{x_n}{x_i}{x_{i-1}})$ by:
\begin{tabular}{c c}
$\delta_{x_n,x_n}=\delta_{x_n}$ & $\delta_{x_n,x_i}=\crossx{x_n}{\delta_{x_{n-1}}}{x_{i-1}}$
\end{tabular}
\begin{center}
$\displaystyle
% I coded it this way with $ and displaystyle rather than with displaymath because uit is the only way that i caould
% find that enabled me to include in a tabular display
\pstree[treemode=\CItreemode,levelsep=*0.65cm,treesep=0.1cm,nodesep=0.05, treenodesize=0.75cm]
{
	\TR{x_{i-1}}
}
{
   	\pstree[levelsep=*0.85cm]
	{
	   \TR{x_i}
	}
	{   \Tn
	    \pstree[levelsep=*0.85cm]
	    {
	    	\TR{x_{i+1}}
	    }
	    {       
				\pstree[nodesep=0.05,levelsep=*0.85cm]
			   	{
			     	\TR[edge=\dottededge]{x_n} 
			   	}
			   	{
			   	    \onslide<2->{
			   	    \Tn   
			   	    \TR{\crossx{x_n}{x_i}{x_{i-1}}}
			   	    \onslide<2>{\uppermember{\crossx{x_n}{\delta_{x_i}}{x_i}}}
			   	    \onslide<3->{\uppermember{\delta_{x_n,x_i}=\crossx{x_n}{\delta_{x_i}}{x_i}}}
			   	               }
			   	}
	   	} 
	    \TR{\crossx{x_i}{x_i}{x_{i-1}}}\uppermember{\delta_{x_i}}
	}
}
$
\end{center}
\end{frame}



\begin{frame}{Construction of Contextual Category}
\begin{itemize}
\item From a concept-instance algebra $A$ we can construct a contextual category \catcw.
\item tree of objects of the category is the tree of concepts of the ci-algebra.
\item morphisms of \catcw n-tuples, $\tuple{\fn}:x \morph y_n$, 
      where $1 \base y_1 \base y_2 ... \base y_n$ in $A$ where $f_1 \in inst(\crossx{x}{y_1}{1})$, 
       $f_2 \in inst(\fonestar(\crossx{x}{y_2}{1}))$,...
       $f_n \in inst(\fnonestar...\ftwostar\fonestar(\crossx{x}{y_n}{1}))$
as shown here 

\scalebox{0.75}{
$
\begin{array}{ c p{0.4cm} c p{0.2cm} c p {0.2cm} c } 
\Rnode{fntarget}{\fnonestar...\ftwostar\fonestar(\crossx{x}{y_n}{1})}
&&\Rnode{f3target}{\ftwostar\fonestar(\crossx{x}{y_3}{1})}
&&\Rnode{f2target}{\fonestar(\crossx{x}{y_2}{1})}  
&& \Rnode{ab1}{\crossx{x}{y_1}{\Rnode{f1target}{1}}}     \\[2cm]
      &&     &&   \ovalnode[linestyle=none]{x}{x}     &&            
\begin{arrows}
\ncarc[arcangle=-5,nodesepA=15pt,offsetA=-2pt,nodesepB=3pt,offsetB=-5pt]{->}{x}{f1target}
\blabel{f_1}[0.6]
\ncarc[arcangle=10,nodesepA=15pt,offsetA=1pt,nodesepB=2pt,offsetB=2pt]{->}{x}{f2target}
\alabel{f_2}[0.4]
\ncarc[arcangle=10,nodesepA=15pt,offsetA=1pt,nodesepB=2pt,offsetB=2pt]{->}{x}{f3target}
\alabel{f_3}[0.65]
\ncarc[arcangle=7,nodesepA=15pt,offsetA=1pt,nodesepB=2pt,offsetB=2pt]{->}{x}{fntarget}
\alabel{f_n}[0.75][0]
\ncdotdotdot{fntarget}{f3target}
\setlength{\sarnodesepB}{10pt}
\ncsar{fntarget}{x}
\ncsar{f3target}{x}
\ncsar{f2target}{x}
\ncsar{f1target}{x}
\sarreset
\end{arrows}
\end{array}
\hspace{2cm}
\begin{array}{c}
\Rnode{bn}{y_n}             \\[1.0cm]
\Rnode{b2}{y_2}             \\[0.8cm]
\Rnode{b1}{y_1}             \\[0.8cm]
\Rnode{abs}{1}              \\
\begin{arrows}
\ncdotdotdot{bn}{b2}
\ncsar{b2}{b1}
\ncsar{b1}{abs}
\end{arrows}
\end{array}
\begin{arrows}
\nccdar{x}{abs}
\end{arrows}
$
}
\end{itemize}
\end{frame}

\iffalse
\newcommand{\ofOb}[1]{\ofT{#1}{\Ob}}
\newcommand{\ofHom}[2]{\ofT{#1}{\Hom(#2)}}

\begin{frame}{Generalised Algebraic Theories}
Generalised algebraic theories are expressed in a syntax that
\begin{itemize}
\item involves rules of these forms\\
\begin{tabular}{c p{1cm} c}
\gatdisplayrule{\xDelta{n}}{\isT{\Delta}}   & \gatdisplayrule{\xDelta{n}}{\ofT{t}{\Delta}}\\
\gatdisplayrule{\xDelta{n}}{\Delta=\Delta'} & \gatdisplayrule{\xDelta{n}}{t=\ofT{t'}{\Delta}} 
\end{tabular}
\item and meta-rules which I call principles of derivation each enabling a rule of one of the above forms to be derived from a number of other previously derived rules
\begin{itemize}
\item LI1...LI7 -- Laws of Identity,
\item T1 -- Transfer Law,
\item CF1, CF2(a) and CF2(b) -- Cut-free versions of term and type substitution laws,
\item SI1 and SI2 -- Substitution of and into Identicals.
\end{itemize} 
\end{itemize}
\end{frame}


\begin{frame}{Generalised Algebraic Theories}
... are defined to consist of
\begin{itemize}
\item A set of sort symbols each with an introductory rule. 
A sort symbol $A$ must have an introductory rule of the form \genericAintroductoryrule and this rule must be well-typed.
\item A set of operator symbols each with an introductory rule. 
An operator symbol $f$ must have an introductory rule of this form   \genericfintroductoryrule and this rule  must be well-typed,
\item a set of well-typed axioms. Each axiom is of one of these two forms
\begin{tabular}{c p{1cm} c}
\gatdisplayrule{\xDelta{n}}{\Delta=\Delta'} & \gatdisplayrule{\xDelta{n}}{t=\ofT{t'}{\Delta}} 
\end{tabular}
\end{itemize}
\end{frame}

\begin{frame}{Theory of Categories}
\highlight{Missed intro rule for composition!}
\footnotesize
\newcommand{\associativitypremisereversed}
       {\begin{array}[t]{l}\ofT{f}{Hom(z_1,z_2)},\,\ofT{g}{Hom(z_2,z_3)},\,\\
                       \hspace{1.6cm}\ofT{h}{Hom(z_3,z_4)},\,\ofT{z_1,z_2,z_3,z_4}{Ob}
        \end{array}
       }
\begin{gatrules}
\gatintros
\gatintroducing{Ob}
\isT{Ob} \\
\gatintroducing{Hom}
  \gatsingular{\ofT{x_1,x_2}{Ob}}{\isT{Hom(x_1,x_2)}} \\	
\gatintroducing{id}
  \gatsingular{\ofT{w}{Ob}}{\ofT{id_w}{Hom(w,w)}} \\	
\gataxioms
\gatintroducing{  \gataxiomno{1} \\   \gataxiomno{2}}
\begin{gatgroup}{\ofT{f}{Hom(x_1,x_2)},\ \ofT{x_1,x_2}{Ob}}
    \gatleaf{}{id_{x_1} \circ f = f} \\
    \gatleaf{}{f \circ id_{x_2} = f}
\end{gatgroup} \\
\gatintroducing{ \gataxiomno{3} }
\gatsingular{\associativitypremisereversed}{(f \circ g) \circ h = f \circ (g \circ h)} 
\end{gatrules}
\end{frame}

\begin{frame}{Instances of Generalised Algebraic Theories}
\begin{itemize}
\item in "Generalised Algebraic Theories and Contextual Categories":
\begin{itemize}
  \item notion of interpretation $I: U \morph U'$ 
  \item the  category of  generalised algebraic theories 
      is defined in my thesis (though actually a 2-category)
  \item  a $U$-algebra, $A$, is a contextual functor $A: \cat{C(U)} \morph \cat{Fam}$
  \item \cat{U-alg} -- category of $U$-algebras
  \item $\cat{I-alg}: \cat{U'-alg} \morph \cat{U-alg}$ functor induced by an interpretation of $U$ in $U'$.
\end{itemize}
\item more recently 
\begin{itemize}
    \item I have defined the notion of an instance of a generalised algebraic theory in a contextual category
         (my ResearchGate page "Instances of Generalised Algebraic Theories")
    \item a $U$-algebra is an instance of theory $U$ in the contextual category \cat{Fam}
\end{itemize}
\end{itemize}
\end{frame}


\newcommand{\USigmaA}{U_{\SigmaA}}
\newcommand{\USigmaAalg}{\USigmaA\mhyphen alg}
\newcommand{\SigmaA}{\Sigma A}

\begin{frame}{Adding a Sigma sort to a generalised algebraic theory $U$.}
If a generalised algebraic theory has a sort symbol $A$ introduced by rule
{\footnotesize \genericAintroductoryrule} then can extend $U$ by a sort symbol $\SigmaA$ and three operations: 
\begin{center}
\footnotesize
\begin{tabular}{c l}
$\SigmaA$ & \gatdisplayrule{\xDelta{n-1}} {\isT{\SigmaA(x_1,...x_{n-1})}} \\
$pr$  & \gatdisplayrule{\xDelta{n}, \ofT{y}{A(\xn)}}  {\ofT{pr(x_n,y)} {\SigmaA(x_1,...x_{n-1})}} \\
$p_1$ & \gatdisplayrule{\xDelta{n-1},\ofT{z}{\SigmaA(x_1,...x_{n-1})}}{\ofT{p1(z)} {\Delta_n}} \\
$p_2$ & \gatdisplayrule{\xDelta{n-1},\ofT{z}{\SigmaA(x_1,...x_{n-1})}}{\ofT{p2(z)} {A(x_1,...x_{n-1}, p_1(z))}} 
\end{tabular}
\end{center}
\end{frame}
\begin{frame}
subject to axioms
\begin{center}
\footnotesize 
\begin{tabular}{c}
\gatdisplayrule{\xDelta{n}, \ofT{y}{A(\xn)}}  {p_1(pr(x_n,y))=x_n} \\
\gatdisplayrule{\xDelta{n}, \ofT{y}{A(\xn)}}  {p_2(pr(x_n,y))=y} \\
\gatdisplayrule{\xDelta{n-1},\ofT{z}{\SigmaA(x_1,...x_{n-1})}}{pr(p_1(z),p_2(z))=z} \\ 
\end{tabular}
\end{center}


so that in the extended theory $\USigmaA$ there is a sort symbol representing 
$$\sum_{\ofT{x_n}{\Delta_n}}{A(\xn)}.$$


There is an obvious embedding $I: U \morph \USigmaA$ in \cat{GAT}.
It is easy to show that the functor $\Ialg : \USigmaAalg \morph \Ualg$ 
is an equivalence of categories (since \cat{Fam} has cannonical $\Sigma$ operations i.e. disjoint unions).
\end{frame}

\newcommand{\IdA}{IdA}
\begin{frame}{Adding an Id sort to a generalised algebraic theory $U$.}
If a generalised algebraic theory has a sort symbol $A$ introduced by rule
{\footnotesize \genericAintroductoryrule} then can extend $U$ by a sort symbol $\IdA$ and an operation $r$ 
\begin{center}
\footnotesize
\begin{tabular}{c l}
$\IdA$ & \gatdisplayrule{\xDelta{n}, \ofT{y_1}{A(\xn)}, \ofT{y_2}{A(\xn)}}  {\isT{\IdA(y_1,y_2)}} \\
$r$    & \gatdisplayrule{\xDelta{n}, \ofT{y}{A(\xn)}}  {\ofT{r(y)}{\IdA(y,y)}} 
\end{tabular}
\end{center}
\end{frame}
\begin{frame}
subject to the axiom
\begin{center}
\footnotesize 
\begin{tabular}{c}
\gatdisplayrule{\xDelta{n}, \ofT{y_1}{A(\xn)}, \ofT{y_2}{A(\xn)}, \ofT{z}{\IdA(y_1,y_2)}}
  {y1=y2}
\end{tabular}
\end{center}
\newcommand{\UIdA}{U_{\IdA}}
\newcommand{\UIdAalg}{\UIdA\mhyphen alg}
so that in the extended theory $\UIdA$ there is a sort symbol representing 
the identity on type $A$.

There is an obvious embedding $I: U \morph \UIdA$ in \cat{GAT}.
It is easy to show that the functor $\Ialg : \UIdAalg \morph \Ualg$ 
is an equivalence of categories (defining and using a cannonical $Id$ structure in Fam).
\end{frame}

\iffalse
\begin{frame}{Theory of Cricket}
\footnotesize

% In this example I have had to fine tune the various widths.
% I haven't been able to find a way of getting the dotfill to fill
% out the entire width of a containing column of an array.
\begin{displaymath}
\begin{array}{l}
\isT{match} \\
\begin{gatgroup}{\ofT{m}{match}}
  \gatleaf[8.0cm]{}{\isT{innings(m)}} \\
  \gatleaf[8cm]{}{\isT{side(m)}}\\
  \gatleaf[8cm]{}{\ofT{homeSide(m)}{side}}\\
  \begin{gatgroup}{\ofT{s}{side(m)}}
    \gatleaf[6cm]{} {\isT{player(s)}} \\
    \gatleaf[6cm]{} {\ofT{captain(s)}{player(s)}}
  \end{gatgroup} \\
    \begin{gatgroup}{\ofT{i}{innings(m)}}
    \gatleaf[6cm]{}{\ofT{fieldingSide(i)}{side}} \\
    \gatleaf[6cm]{}{\ofT{battingSide(i)}{side}} \\
    \gatleaf[6cm]{}{\isT{over(i)}} \\
    \begin{gatgroup}{\ofT{o} {over(i)}}
      \gatleaf[5.8cm]{}{\ofT{bowler(o)}{player(fieldingSide(i))}} \\
      \gatleaf[5.8cm]{}{\isT{delivery(o)}} \\
      \gatleaf[5.8cm]{}{\ofT{facingBatter(d)}{player(battingSide(i))}} \\
      \makebox[5.8cm][r]{\hspace{2cm} \dotfill where $\ofT{d}{delivery(o)}$  }
    \end{gatgroup} 
  \end{gatgroup} 
\end{gatgroup}
\end{array}
\end{displaymath}


\end{frame}
\fi



\newcommand{\inst}{i} 
\newcommand{\cpt}{c} 
\begin{frame}{Generalised Algebraic Theory of Concept Instance Algebras}
\begin{itemize}
\item There is a generalised algebraic theory of concept instance algebras.
\item sorts representing levels of concepts are $\cpt_1, \cpt_2 ...$.
\item sort $\cpt_{n}$ is introduced by:
\gatdisplayrule{\ofT{x_1}{\cpt_1},... \ofT{x_{n-1}}{\cpt_{n-1}(x_1,...x_{n-2})}} {\isT{\cpt_{n}(x_1,...x_{n-1})}}
\item sorts representing instances are $\inst_1, \inst_2, ....$.
\item sort $\inst_{n}$ is introduced by:
\gatdisplayrule{\context{x}{\cpt}{n}}{\isT{\inst_{n}(x_1,...x_n)}}
\item Similarly 
\begin{itemize}
\item countably many $^*$, $\cross$ and $\delta$ operations,
\item countably many axioms for each of the axioms given earlier.
\end{itemize}
\end{itemize}

\end{frame}

\begin{frame}{Relationship with Contextual Categories \& Generalised Algebraic Theories}
There are equivalences of categories between
\begin{itemize}
\item the category of generalised algebraic theories
\end{itemize}
and each of the following
\begin{itemize}
\item the category of concept instance algebras,
\item the category of contextual categories,
\item the category of contextual categories with families
\end{itemize}
but not with 
\begin{itemize}
\item the category of categories with families.
\end{itemize}
\end{frame}


\iffalse
\begin{frame}
\begin{displaymath}
\begin{array}{c p{0.1cm} c}
                     && \Rnode{root}{\circ}     \\[0.4cm]
                     && \Rnode{C0}{\cpt_0}         \\[0.4cm]
\Rnode{I0}{\inst_0}      && \Rnode{C1}{\cpt_1}         \\[0.4cm]
\Rnode{I1}{\inst_1}      && \Rnode{C2}{\cpt_2}         \\[0.25cm]
\Rnode{I2}{\inst_2}      && \vdots                  \\[0.25cm]
\vdots			     && \Rnode{Ci}{\cpt_i}         \\[0.4cm]
\Rnode{Ii}{\inst_i}      && \Rnode{Csi}{\cpt_{i+1}}    \\[0.25cm]
\Rnode{Isi}{\inst_{i+1}} && \vdots                  \\[0.25cm]
\vdots               &&
\end{array}
\begin{arrows}
\ncline[nodesep=4pt]{C0}{root}
\ncline[nodesep=4pt]{C1}{C0}
\ncline[nodesep=4pt]{C2}{C1}
\ncline[nodesep=4pt]{Csi}{Ci}
\ncline[nodesep=4pt]{I0}{C0}
\ncline[nodesep=4pt]{I1}{C1}
\ncline[nodesep=4pt]{I2}{C2}
\ncline[nodesep=4pt]{Ii}{Ci}
\ncline[nodesep=4pt]{Isi}{Csi}
\end{arrows}
\end{displaymath}
\end{frame}
\fi

\begin{frame}{Now as a Concept-Instance Algebra}
 \def\dedge{\ncline[linestyle=dotted]}
$$
\pstree[treemode=\CItreemode, treefit=loose,treenodesize=0.25cm,levelsep=0.65cm,treesep=0.7cm,nodesep=2pt]
{
  \Tr{\circ}
}
{
  \pstree
  {
     \Tr{\cpt_0}
  }
  {
    \Tr{\inst_0}
	\pstree
	{
	     \Tr{\cpt_1}
	}
	{
      \Tr{\inst_1}
  	  \pstree
	  {
	     \Tr{\cpt_2}
	  }
	  {  
		 \Tr{\inst_2}
		 \pstree%[levelsep=*0.75cm]
		 {
		    \Tr[edge=\dedge]{\cpt_i} 
		 }
		 {  
	        \Tr{\inst_i}
	        \pstree[levelsep=0.75cm,treesep=1.5cm] 
			{
			   \Tr{\cpt_{i+1}}
			}
			{
			   \Tr{\inst_{i+1}}
			   \Tr[edge=\dedge]{\ \ \ \ \ \nudgeup{0.3cm} \ \ \ \ \ \ \ \ } 
			}
		 }
	  }
	}
  }
}
$$
$$_0\cross_0$$
$$\qq{_0\cross_0}$$
$$\qq{\kern-3pt_0\cross_0\kern-3pt}$$
$$\qq{\cross_{0,0}} \in inst(\cpt_0 \cross (\cpt_0 \cross \cpt_0))$$
$$\qq{\delta_0} \in inst({\delta_{\cpt_0}}^*(\qq{\cross_0}^* (\cpt_0 \cross ( \cpt_0 \cross \inst_0))))$$
\end{frame}

\begin{frame}{As a Contextual Category}
 \def\dedge{\ncline[linestyle=dotted]}
 \def\sedge{\ncksar}
\begin{displaymath}
\pstree[edge=\sedge, treemode=U, treefit=loose,treenodesize=0.25cm,levelsep=0.8cm,treesep=0.7cm,nodesep=2pt]
{
  \Tr{1}
}
{
  \pstree
  {
     \Tr{\cpt_0}
  }
  {
    \Tr{\inst_0}
	\pstree
	{
	     \Tr{\cpt_1}
	}
	{
      \Tr{\inst_1}
  	  \pstree
	  {
	     \Tr{\cpt_2}
	  }
	  {  
		 \Tr{\inst_2}
		 \pstree%[levelsep=*0.75cm]
		 {
		    \Tr[edge=\dedge]{\cpt_i} 
		 }
		 {  
	        \Tr{\inst_i}
	        \pstree[levelsep=0.75cm,treesep=1.5cm] 
			{
			   \Tr{\cpt_{i+1}}
			}
			{
			   \Tr{\inst_{i+1}}
			   \Tr[edge=\dedge]{\ \ \ \ \ \nudgeup{0.3cm} \ \ \ \ \ \ \ \ } 
			}
		 }
	  }
	}
  }
}
\end{displaymath}
\end{frame}

\begin{frame}{Richard Garner...}
Describes a monad on the category $\Set^C$, where $C$ is this category 
 \def\dedge{\ncline[linestyle=dotted]}
 \def\backarrow{\nckarr}
$$
  \pstree[edge=\backarrow, treemode=U, treefit=loose,treenodesize=0.25cm,levelsep=0.8cm,treesep=0.7cm,nodesep=2pt]
  {
     \Tr{\cpt_0}
  }
  {
    \Tr{\inst_0}
	\pstree
	{
	     \Tr{\cpt_1}
	}
	{
      \Tr{\inst_1}
  	  \pstree
	  {
	     \Tr{\cpt_2}
	  }
	  {  
		 \Tr{\inst_2}
		 \pstree%[levelsep=*0.75cm]
		 {
		    \Tr[edge=\dedge]{\cpt_i} 
		 }
		 {  
	        \Tr{\inst_i}
	        \pstree[levelsep=0.75cm,treesep=1.5cm] 
			{
			   \Tr{\cpt_{i+1}}
			}
			{
			   \Tr{\inst_{i+1}}
			   \Tr[edge=\dedge]{\ \ \ \ \ \nudgeup{0.3cm} \ \ \ \ \ \ \ \ } 
			}
		 }
	  }
	}
  }
$$ and defines the algebras as equivalent to B-systems.
\end{frame}


\begin{frame}{Fungible Algebras}
Q. Are the following types of structure fungible
\begin{itemize}
\item concept instance algebras,
\item contextual categories,
\item contextual categories with families.
\end{itemize}
A. Depends on the matter at hand. If we are looking for Set-like instances of theories then they are, 
more generally they are not.

\begin{itemize}
\item concept-instance algebras are $\Sigma Id$-fungible
\end{itemize}
\end{frame}




\begin{frame}{Cricket as Data Specification}
\begin{tabular}{l l}
\raisebox{-4cm}{\scalebox{0.75}{\begin{erdiagram}{11.549999999999999}{9.113750000000001}

\eret{2.56}{-2}{6.56}{-1.1}{0.2}{1}\eretname{2.96}{-1.45}{l}{match}
\erattr{2.76}{-1.65}{1}{0}{id}
\eret{0.006}{-4.75}{3.634}{-2.95}{0.2}{1}\eretname{0.369}{-3.3}{l}{innings}
\onslide<3->{\erdattr{0.206}{-3.5}{1}{0}{match\textunderscore id}}
\erattr{0.206}{-3.8}{1}{0}{number}
\onslide<2->{\erdattr{0.206}{-4.1}{1}{1}{battingSide\textunderscore name}}
\onslide<2->{\erdattr{0.206}{-4.4}{1}{1}{fieldingSide\textunderscore name}}
\eret{6.634}{-4.75}{9.114}{-2.95}{0.2}{1}\eretname{6.882}{-3.3}{l}{side}
\onslide<3->{\erdattr{6.834}{-3.5}{1}{0}{match\textunderscore id}}
\erattr{6.834}{-3.8}{1}{0}{name}
\eret{0.298}{-7.75}{3.343}{-5.95}{0.2}{1}\eretname{0.602}{-6.3}{l}{over}
\onslide<3->{\erdattr{0.498}{-6.5}{1}{0}{match\textunderscore id}}
\onslide<3->{\erdattr{0.498}{-6.8}{1}{0}{innings\textunderscore number}}
\erattr{0.498}{-7.1}{1}{0}{number}
\onslide<2->{\erdattr{0.498}{-7.4}{1}{1}{bowler\textunderscore name}}
\eret{6.67}{-7.75}{9.078}{-5.95}{0.2}{1}\eretname{6.911}{-6.3}{l}{player}
\onslide<3->{\erdattr{6.87}{-6.5}{1}{0}{match\textunderscore id}}
\onslide<3->{\erdattr{6.87}{-6.8}{1}{0}{side\textunderscore name}}
\erattr{6.87}{-7.1}{1}{0}{name}
\eret{0.106}{-11.05}{3.534}{-8.95}{0.2}{1}\eretname{0.449}{-9.3}{l}{delivery}
\onslide<3->{\erdattr{0.306}{-9.5}{1}{0}{match\textunderscore id}}
\onslide<3->{\erdattr{0.306}{-9.8}{1}{0}{innings\textunderscore number}}
\onslide<3->{\erdattr{0.306}{-10.1}{1}{0}{over\textunderscore number}}
\erattr{0.306}{-10.4}{1}{0}{number}
\onslide<2->{\erdattr{0.306}{-10.7}{1}{1}{facingBatter\textunderscore name}}
\eret{0}{-0.2}{9.114}{0.3}{0.2}{1}

% relationship 
\errelname{4.71}{-0.5}{l}{}\errelarm{4.56}{-0.2}{4.56}{-0.65}{1}{0}\errelarm{4.56}{-0.65}{4.56}{-1.1}{1}{0}\ercrowfoot{4.56}{-0.95}{4.41}{-1.1}{4.56}{-1.1}{4.71}{-1.1}{0}
% relationship 
\errelname{3.743}{-2.3}{r}{}\errelarm{3.893}{-2}{3.893}{-2.075}{1}{0}\errelarm{1.82}{-2.738}{1.82}{-2.95}{1}{0}\errelname{2.707}{-2.188}{r}{D2}\errelangle{3.893}{-2.075}{3.893}{-2.15}{2.857}{-2.338}{1}{0}\errelangle{2.857}{-2.338}{1.82}{-2.525}{1.82}{-2.738}{1}{0}\eridcomprel{1.7200000000000004}{1.9200000000000006}{-2.7}\ercrowfoot{1.82}{-2.8}{1.67}{-2.95}{1.82}{-2.95}{1.97}{-2.95}{0}
% relationship 
\errelname{5.377}{-2.3}{l}{}\errelarm{5.227}{-2}{5.227}{-2.075}{1}{0}\errelarm{7.874}{-2.738}{7.874}{-2.95}{1}{0}\errelname{6.7}{-2.188}{l}{D3}\errelangle{5.227}{-2.075}{5.227}{-2.15}{6.55}{-2.338}{1}{0}\errelangle{6.55}{-2.338}{7.874}{-2.525}{7.874}{-2.738}{1}{0}\eridcomprel{7.7737500000000015}{7.973750000000001}{-2.7}\ercrowfoot{7.874}{-2.8}{7.724}{-2.95}{7.874}{-2.95}{8.024}{-2.95}{0}
% relationship 
\errelname{1.97}{-5.05}{l}{}\errelname{1.97}{-5.2}{l}{D4}\errelarm{1.82}{-4.75}{1.82}{-5.35}{1}{0}\errelarm{1.82}{-5.35}{1.82}{-5.95}{1}{0}\eridcomprel{1.7200000000000004}{1.9200000000000006}{-5.699999999999999}\ercrowfoot{1.82}{-5.8}{1.67}{-5.95}{1.82}{-5.95}{1.97}{-5.95}{0}
% relationship battingSide
\errelname{3.784}{-3.4}{l}{battingSide}\errelname{5.284}{-3.4}{l}{R1}\errelarm{3.634}{-3.55}{5.134}{-3.55}{1}{0}\errelarm{5.134}{-3.55}{6.634}{-3.55}{0}{0}\ercrowfoot{3.784}{-3.55}{3.634}{-3.4}{3.634}{-3.55}{3.634}{-3.7}{0}
% relationship fieldingSide
\errelname{3.784}{-4}{l}{fieldingSide}\errelname{5.284}{-4}{l}{R2}\errelarm{3.634}{-4.15}{5.134}{-4.15}{1}{0}\errelarm{5.134}{-4.15}{6.634}{-4.15}{0}{0}\ercrowfoot{3.784}{-4.15}{3.634}{-4}{3.634}{-4.15}{3.634}{-4.3}{0}
% relationship 
\errelname{8.024}{-5.05}{l}{}\errelname{8.024}{-5.2}{l}{D5}\errelarm{7.874}{-4.75}{7.874}{-5.35}{1}{0}\errelarm{7.874}{-5.35}{7.874}{-5.95}{1}{0}\eridcomprel{7.7737500000000015}{7.973750000000001}{-5.699999999999999}\ercrowfoot{7.874}{-5.8}{7.724}{-5.95}{7.874}{-5.95}{8.024}{-5.95}{0}
% relationship 
\errelname{1.97}{-8.05}{l}{}\errelname{1.97}{-8.2}{l}{D6}\errelarm{1.82}{-7.75}{1.82}{-8.35}{1}{0}\errelarm{1.82}{-8.35}{1.82}{-8.95}{1}{0}\eridcomprel{1.7200000000000004}{1.9200000000000006}{-8.7}\ercrowfoot{1.82}{-8.8}{1.67}{-8.95}{1.82}{-8.95}{1.97}{-8.95}{0}
% relationship bowler
\errelname{3.493}{-6.7}{l}{bowler}\errelname{5.156}{-6.7}{l}{R3}\errelarm{3.343}{-6.85}{5.006}{-6.85}{1}{0}\errelarm{5.006}{-6.85}{6.67}{-6.85}{0}{0}\ercrowfoot{3.493}{-6.85}{3.343}{-6.7}{3.343}{-6.85}{3.343}{-7}{0}
% relationship facingBatter
\errelname{3.684}{-10.3}{l}{facingBatter}\errelarm{3.534}{-10}{4.134}{-10}{1}{0}\errelarm{6.47}{-7.57}{6.67}{-7.57}{0}{0}\errelname{5.352}{-8.635}{r}{R4}\errelangle{4.134}{-10}{4.734}{-10}{5.502}{-8.785}{1}{0}\errelangle{5.502}{-8.785}{6.27}{-7.57}{6.47}{-7.57}{0}{0}\ercrowfoot{3.684}{-10}{3.534}{-9.85}{3.534}{-10}{3.534}{-10.15}{0}
\end{erdiagram}
} }
& \onslide*<1-1>{Logical or Conceptual} \onslide*<2-2>{Physical...Hierarchical} \onslide*<3-3>{Physical...Relational}
\end{tabular}
\end{frame}
\fi
\begin{frame}{Ideas next}
generalised algebraic functors - left adjoints

generalised algebraic functors - monadic

maybe refer to a contemplation of the absolute the abstract

move final Ryle quote here - is it too grand to suppose that concept-instance algebras might provide such a grammar?
\end{frame}

\end{document}