
% \mmCIalg mathmode CIalg
\newcommand{\mmCIalg}{\cat{CI}\mhyphen\cat{alg}}

\newcommand{\makespace}{\nudgedown{14pt}\nudgeup{20pt}} % vertical space for labels on arrows

\newcommand{\gatconarrows}{
\ncarr{gat}{con}
\alabel{C}
\ncarr{con}{gat}
\alabel{U}
}

\newcommand{\gatconequiv}
{
\begin{array}{c p{1.5cm} c}
\makespace\Rnode{gat}{\cat{GAT}} && \Rnode{con}{\cat{Con}}
\end{array}
\begin{arrows}
\setlength{\arroffsetA}{5pt}
\setlength{\arroffsetB}{5pt}
\gatconarrows
\end{arrows}
}
\newcommand{\gatciarrows}{
\ncarr{gat}{cialg}
\alabel{C}
\ncarr{cialg}{gat}
\alabel{U}
}

\newcommand{\gatciequiv}
{
\begin{array}{c p{1.5cm} c}
\makespace\Rnode{gat}{\cat{GAT}} && \Rnode{cialg}{\mmCIalg}
\end{array}
\begin{arrows}
\setlength{\arroffsetA}{5pt}
\setlength{\arroffsetB}{5pt}
\gatciarrows
\end{arrows}
}

\newcommand{\conciarrows}
{
\ncarr{con}{cialg}
\alabel{CI}
\ncarr{cialg}{con}
\alabel{CC}
}

\newcommand{\concialgequiv}
{\begin{array}{c p{1.5cm} c}
\makespace\Rnode{con}{\cat{Con}}  &&  \Rnode{cialg}{\mmCIalg}
\end{array}
\begin{arrows}
\setlength{\arroffsetA}{5pt}
\setlength{\arroffsetB}{5pt}
\conciarrows
\end{arrows}
}

\newcommand{\UCIequivalences}
{
\begin{array}{c p{1.5cm} c p{1.5cm} c}
\makespace\Rnode{gat}{\cat{GAT}} && \Rnode{con}{\cat{Con}}  &&  \Rnode{cialg}{\mmCIalg}
\end{array}
\begin{arrows}
\setlength{\arroffsetA}{5pt}
\setlength{\arroffsetB}{5pt}
\gatconarrows
\conciarrows
\end{arrows}
}

\begin{frame}{$\cat{GAT} \simeq \mmCIalg \simeq \cat{Con}$}
\pause
\begin{itemize}
\item The equivalence $\gatciequiv$ I proved this in 1976 when I first drafted my thesis.
\medskip
\item The equivalence $\gatconequiv$ is proved in my 1978 thesis and is summarised in my 1986 paper.
\medskip
\item The equivalence $\concialgequiv$ is an exercise in equational reasoning (passed up 1976)
-- I'm going to discuss this next.
\end{itemize}
\end{frame}