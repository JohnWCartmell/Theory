\begin{frame}{Background}
\begin{itemize}
\item generalised algebraic theories and contextual categories,
\item category of gats equivalent to category of contextual categories,
\item Voevodsky renames contextual categories as C-systems,
\item Voevodsky discovered a generalised algebraic theory of C-systems/contextual categories,
\item Voevodsky describes stripped down  C-systems as B-systems,
\item previously I worked with such algebras and I have variously named them reflexive tree categories and metaGAT algebras,
\item here I want to describe them under yet another name as \textit{concept-instance algebras},
\item \textit{concept-instance algebras} are to \textit{contextual categories} as \textit{clones} are to \textit{Lawvere's  algebraic theories}.
\item Richard Garner described these as algebras over a monad over a certain functor category.
\end{itemize}
\end{frame}

\begin{frame}{Context and Concept}
\begin{itemize}
	\item Concept-instance algebra is a grammar for describing the interplay between context and concept in the workings of ordinary language.
\pause	\item
\begin{tightquote}
[a] given word will, in different sorts of context, express ideas of an indefinite range of differing logical types and, therefore, with different logical powers. And what is true of single words is also true of complex expressions and of grammatical constructions. (Gilbert Ryle, 1945, 206)
\end{tightquote}
\medskip
\pause
\item \textit{unnoticed systematic ambiguities are a common source of type-confusions and philosophic problems}.
\end{itemize}
\end{frame}
\begin{frame}
\begin{itemize}
\item 
\begin{tightquote}
... the inflections of meaning to which most of our expressions are susceptible nonetheless have affinities: the ideas expressed by these expressions in their various uses are “intimately connected” with each other; 
they are \underline{“different inflections} \underline{of the same root”} (Gilbert Ryle, 1945, 206, my underline).
\end{tightquote}

\end{itemize}
\end{frame}


