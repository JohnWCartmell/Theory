
\begin{frame}{Gilbert Ryle, Context and Systematic Ambiguity?}
Gilbert Ryle:
\begin{tightquote}
[a] given word will, in different sorts of context, express ideas of an indefinite range of differing logical types and, therefore, with different logical powers. And what is true of single words is also true of complex expressions and of grammatical constructions. (1945, 206)
\end{tightquote}
\medskip
Ryle describes such ambiguity as systematic : \textit{unnoticed systematic ambiguities are a common source of type-confusions and philosophic problems}.
\end{frame}
\begin{frame}
and
\begin{tightquote}
... the inflections of meaning to which most of our expressions are susceptible nonetheless have affinities: the ideas expressed by these expressions in their various uses are “intimately connected” with each other; they are “different inflections of the same root” (1945, 206).
\end{tightquote}
\medskip
Ryles suggestion that such ambiguity applies to \textit{most} words and his use of words \textit{inflection} and \textit{root} suggest to me some kind of conceptual grammar. 
\end{frame}

\begin{frame}{Background}
\begin{itemize}
\item generalised algebraic theories and contextual categories,
\item category of gats equivalent to category of contextual categories,
\item Voevodsky renames contextual categories as C-systems,
\item Voevodsky discovered a generalised algebraic theory of C-systems/contextual categories,
\item Voevodsky describes stripped down  C-systems as B-systems,
\item previously I worked with such algebras and I have variously named them reflexive tree categories and metaGAT algebras,
\item here I want to describe them under yet another name as \textit{concept-instance algebras},
\item \textit{concept-instance algebras} are to \textit{contextual categories} as \textit{clones} are to \textit{Lawvere's  algebraic theories}.
\item Richard Garner described these as algebras over a monad over a certain functor category.
\end{itemize}
\end{frame}

\begin{frame}{Reminder}
In a contextual category,
\begin{itemize}
\item there is a well-founded partial ordering $<$ between objects
such that there is  a rooted $\omega$-tree of objects,
\item if $y$ covers $x$ in the partial order then write $x \base y$,
\item whenever $x \base y$ in the partial order then there is a distinguished
morphism $p_y:y \morph x$ which I write as $y \smorph x$,
\item the root object is a terminal object in the category,
\item the root object represents the empty context, 
\item non-root objects are at the same time both contexts and types
because the syntactic \underline{difference beween contexts and types} is 
just that - \underline{a syntactic difference}. 
\end{itemize}
\end{frame}

\begin{frame}{Concept-Instance Algebra -  Overview}
In a concept-instance algebra there are concepts and instances and there are operations $^*$, $\crossx{}{}{}$ and $\delta$
and these must satisfy various axioms.
\begin{itemize}
\item there is a rooted $\omega$-tree of concepts,
\item for each non-root concept there is a associated set of instances (instances of that concept),
\item operation $^*$ enables  concepts and instances to be particularised (syntactically this is substitution),
\item  operation $\crossx{}{}{}$ enables the positing in-scope concepts and instances (syntactially this is sometimes called weaking),
\item operation $\delta$ is an expression of self (as for example the identity morphisms in a category).
\end{itemize}
\underline{The concepts of a context-instance algebra are equally contexts.}
\end{frame}

\begin{frame}{Concepts, Contexts  and Instances}
\begin{itemize}
\item In a concept-instance algebra $A$ concepts are equally contexts.
\item If $a \base b \base c$ in the tree of concepts of algebra $A$ 
then $a$ is to be understood as a context for concept $b$ and $b$ is to be understood as a context for concept $c$,
\item Instances of concepts are to be understood as instances in context.
\item For example some concepts and instances in cricket
\pstree[treemode=\CItreemode,levelsep=*0.65cm,treesep=\CItreesep,nodesep=0.05]
{
    \Tr{\circ}
}
{
    \pstree [levelsep=*0.85cm]
    {
		\Tr{match} 
	}
	{		  
		\pstree [levelsep=*0.85cm]
		{
				   \Tr{side} %\uppermember {homeSide}
		}
		{
					\Tr{player} %\uppermember {captain}
		}
	    \pstree [levelsep=*0.85cm]
		{
			\Tr{innings} 
		}
		{		  
		    \pstree [levelsep=*0.85cm]
			{
					   \Tr{over} 
			}
			{   
					   \Tr{delivery} 
			}			
		}		
	}	
}
\end{itemize} 
\end{frame}

\begin{frame}{Particularisation}


\end{frame}



