\begin{frame}{Construction of CI-algebra}
\begin{itemize}
\item The tree of concepts of $CI(C)$ is defined to be the tree of objects of $C$.

\item If $A \base B$ in $C$ then $inst(B)$ in $CI(C)$ is defined to be the set of sections of
 $B$ in $C$ i.e. the set $\setsuchthat{f: A \morph B}{B \circ p_B = id_A}$

\item Operations $^*$, $\cross$ are defined in $CI(C)$ using the pullbacks and dependency morphisms of $C$.  
 There are details of suitable  operations $^*$ and $\cross$ defined in section 3 of my paper
 "Instances of Generalised Algebraic Theories in Contextual Categories" on ResearchGate.
 \item If $A$ is an object of $C$ then define $\delta_A$ in algebra $CI(C)$ to be the section
 $s(id_A)$, where $s$ is Voevodsy's `s' operator.
 \item Proving that the fourteen axioms hold is an exercise in simple equational algebra. Some of the work is done, in passing, in lemmas 5.1, 5.2, 5.11, 5.14 and 5.17 of my "Instances..." paper.
\end{itemize}
\end{frame}