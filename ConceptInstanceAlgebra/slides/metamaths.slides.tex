
\newcommand{\inst}{i} 
\newcommand{\cpt}{c} 
%\cptintro{1-2}{3-} shows formal rule on slides <1-2> and informal rule on slides <3->
\newcommand{\cptintro}[2]{
\only<#1>{ {\footnotesize \gatdisplayrule{\ofT{x_1}{\cpt_1},...\, \ofT{x_{n}}{\cpt_{n}(x_1,...x_{n-1})}} {\isT{\cpt_{n+1}(\xn)}} } }
\only<#2>{{\footnotesize \gatdisplayrule{\ofT{x_1}{\cpt_1},...\, \ofT{x_{n}}{\cpt_{n}(x_{n-1})}} {\isT{\cpt_{n+1}(x_n)}}} }
}
\newcommand{\instintro}[2]{
\only<#1>{\footnotesize \gatdisplayrule{\context{x}{\cpt}{n}}{\isT{\inst_{n}(x_1,...x_n)}}}
\only<#2>{\footnotesize \gatdisplayrule{\ofT{x_1}{\cpt_1},...\, \ofT{x_{n}}{\cpt_{n}(x_{n-1})}}{\isT{\inst_{n}(x_n)}}}
}


\begin{frame}{Generalised Algebraic Theory of Concept Instance Algebras}
\begin{itemize}
\item There is a generalised algebraic theory $tci$ of concept instance algebras.
\item sorts representing concepts are $\cpt_1, \cpt_2 ...$.
\item sort $\cpt_{n+1}$ is introduced by:
\cptintro{1}{2-}
\onslide<3->{
\item sorts representing instances are $\inst_1, \inst_2, ....$.
\item sort $\inst_{n}$ is introduced by:
\instintro{3}{4-}
}
\onslide<5->{
\item Also
\begin{itemize}
\item countably many $^*$, $\cross$ and $\delta$ operations,
\item countably many axioms for each of the axioms given earlier.
\end{itemize}
}
\end{itemize}
\end{frame}

\begin{frame}{Generalised Algebraic Theory of Contextual Categories}
\begin{itemize}
\item There is a generalised algebraic theory $tcc$ of contextual categories. This result is due to Vladimir Voevodsky.
\item I describe $tcc$ in a paper called "Generalised Algebraic Axiomatisations of Contextual Cageories"
which can be found on my ResearchGate page.
\item $tcc$ has an operator $s$ which constructs a section from a morphism.
\item Vladimir gave the axioms for $s$ and proved the necessary pullbacks (from my original formulation) follow from these axioms.
\end{itemize}
\end{frame}

\renewcommand{\obj}{obj}
\renewcommand{\hom}{hom}
\newcommand{\objintro}[2]{
	\only<#1>{{\footnotesize \gatdisplayrule{\context{x}{\obj}{n}} {\isT{\obj_{n+1}(\xn)}}}}
  \only<#2>{{\footnotesize \gatdisplayrule{\contextShortstyle{x}{\obj}{n}} {\isT{\obj_{n+1}(x_n)}}}}
}
\newcommand{\homintro}[2]{
\only<#1>{
{\footnotesize \gatdisplayrule{\context{x}{\obj}{n}, \context{y}{\obj}{m}} {\isT{\hom_{n,m}(\xn,\ym)}}}
}
\only<#2>{
{\footnotesize \gatdisplayrule{\contextShortstyle{x}{\obj}{n}, \contextShortstyle{y}{\obj}{m}} {\isT{\hom_{n,m}(x_n,y_m)}}}
}
}
\begin{frame}{Description of $tcc$ cont.}
\begin{itemize}
\item sorts representing objects are $\obj_0, \obj_1 ...$.
\item sort $\obj_{n}$ is introduced by:
\objintro{1}{2-}
\onslide<3->
{
\item sorts representing morphisms are $\hom_{n,m}$, for $n,m \geq 0$,
\item sort $\hom_{n,m}$ is introduced by:
\homintro{3}{4-}
}
\onslide<5->{
\item Also 
\begin{itemize}
\item countably many $id$, $\circ$, $p$, $^*$, $q$, and $s$ operations  
\item countably many axioms.
\end{itemize}
}
\end{itemize}
\end{frame}

\newcommand{\tccalg}{tcc\mhyphen alg}
\newcommand{\tcialg}{tci\mhyphen alg}
\newcommand{\tccplusalg}{tcc^+\mhyphen alg}
\newcommand{\tciplusalg}{tci^+\mhyphen alg}
\begin{frame}{Construction of CI-algebra from Contextual Category is Generalised Algebraic}
\begin{itemize}
\item The construction described above of a CI-algebra from Contextual Category can be described 
by an interpretation $I_{CI}:tci \morph tcc^+$ where $tcc^+$ is the theory $tcc$ extended 
by (countable many) $\Sigma$ and identity types. 
\item The construction described above of a contextual category from a CI-algebra can be described 
by an interpretation $I_{CC}:tcc \morph tci^+$ where $tci^+$ is the theory $tci$ extended by 
(countable many) $\Sigma$ types.
\item Can show that $I_{CI}$ and $I_{CC}$ extend to pseudo inverses in the category \cat{GAT}.
\item From this we have $\tcialg \simeq \tciplusalg \simeq \tccplusalg \simeq \tccalg$.
\item i.e. category of concept-instance algebras equivalent to category of contextual categories.
\end{itemize}
\end{frame}

\begin{frame}{Relationship with Contextual Categories \& Generalised Algebraic Theories}
The following categories are all equivalent
\begin{enumerate}[(1)]
\item the category of generalised algebraic theories
\item the category of concept instance algebras,
\item the category of contextual categories,
\item the category of contextual categories with families.
\end{enumerate}
but not, by the way, this one 
\begin{itemize}
\item the category of categories with families.
\end{itemize}
\end{frame}


\iffalse
\begin{frame}
\begin{displaymath}
\begin{array}{c p{0.1cm} c}
                     && \Rnode{root}{\circ}     \\[0.4cm]
                     && \Rnode{C0}{\cpt_0}         \\[0.4cm]
\Rnode{I0}{\inst_0}      && \Rnode{C1}{\cpt_1}         \\[0.4cm]
\Rnode{I1}{\inst_1}      && \Rnode{C2}{\cpt_2}         \\[0.25cm]
\Rnode{I2}{\inst_2}      && \vdots                  \\[0.25cm]
\vdots			     && \Rnode{Ci}{\cpt_i}         \\[0.4cm]
\Rnode{Ii}{\inst_i}      && \Rnode{Csi}{\cpt_{i+1}}    \\[0.25cm]
\Rnode{Isi}{\inst_{i+1}} && \vdots                  \\[0.25cm]
\vdots               &&
\end{array}
\begin{arrows}
\ncline[nodesep=4pt]{C0}{root}
\ncline[nodesep=4pt]{C1}{C0}
\ncline[nodesep=4pt]{C2}{C1}
\ncline[nodesep=4pt]{Csi}{Ci}
\ncline[nodesep=4pt]{I0}{C0}
\ncline[nodesep=4pt]{I1}{C1}
\ncline[nodesep=4pt]{I2}{C2}
\ncline[nodesep=4pt]{Ii}{Ci}
\ncline[nodesep=4pt]{Isi}{Csi}
\end{arrows}
\end{displaymath}
\end{frame}
\fi

\begin{frame}{Now as a Concept-Instance Algebra}
 \def\dedge{\ncline[linestyle=dotted]}
$$
\pstree[treemode=\CItreemode, treefit=loose,treenodesize=0.25cm,levelsep=0.65cm,treesep=0.7cm,nodesep=2pt]
{
  \Tr{\circ}
}
{
  \pstree
  {
     \Tr{\cpt_0}
  }
  {
    \Tr{\inst_0}
	\pstree
	{
	     \Tr{\cpt_1}
	}
	{
      \Tr{\inst_1}
  	  \pstree
	  {
	     \Tr{\cpt_2}
	  }
	  {  
		 \Tr{\inst_2}
		 \pstree%[levelsep=*0.75cm]
		 {
		    \Tr[edge=\dedge]{\cpt_i} 
		 }
		 {  
	        \Tr{\inst_i}
	        \pstree[levelsep=0.75cm,treesep=1.5cm] 
			{
			   \Tr{\cpt_{i+1}}
			}
			{
			   \Tr{\inst_{i+1}}
			   \Tr[edge=\dedge]{\ \ \ \ \ \nudgeup{0.3cm} \ \ \ \ \ \ \ \ } 
			}
		 }
	  }
	}
  }
}
$$
$$\qq{\cross_{0,0}} \in inst(\cpt_0 \cross (\cpt_0 \cross \cpt_0))$$
$$\qq{\delta_0} \in inst({\delta_{\cpt_0}}^*(\qq{\cross_0}^* (\cpt_0 \cross ( \cpt_0 \cross \inst_0))))$$
\end{frame}

\iffalse{
\begin{frame}{As a Contextual Category}
 \def\dedge{\ncline[linestyle=dotted]}
 \def\sedge{\ncksar}
\begin{displaymath}
\pstree[edge=\sedge, treemode=U, treefit=loose,treenodesize=0.25cm,levelsep=0.8cm,treesep=0.7cm,nodesep=2pt]
{
  \Tr{1}
}
{
  \pstree
  {
     \Tr{\cpt_0}
  }
  {
    \Tr{\inst_0}
	\pstree
	{
	     \Tr{\cpt_1}
	}
	{
      \Tr{\inst_1}
  	  \pstree
	  {
	     \Tr{\cpt_2}
	  }
	  {  
		 \Tr{\inst_2}
		 \pstree%[levelsep=*0.75cm]
		 {
		    \Tr[edge=\dedge]{\cpt_i} 
		 }
		 {  
	        \Tr{\inst_i}
	        \pstree[levelsep=0.75cm,treesep=1.5cm] 
			{
			   \Tr{\cpt_{i+1}}
			}
			{
			   \Tr{\inst_{i+1}}
			   \Tr[edge=\dedge]{\ \ \ \ \ \nudgeup{0.3cm} \ \ \ \ \ \ \ \ } 
			}
		 }
	  }
	}
  }
}
\end{displaymath}
\end{frame}
}\fi
\begin{frame}{Richard Garner...}
Describes a monad on the category $\Set^{CI_0}$, where $CI_0$ is this category 
 \def\dedge{\ncline[linestyle=dotted]}
 \def\backarrow{\nckarr}
$$
  \pstree[edge=\backarrow, treemode=\CItreemode, treefit=loose,treenodesize=0.25cm,levelsep=0.8cm,treesep=0.7cm,nodesep=2pt]
  {
     \Tr{\cpt_0}
  }
  {
    \Tr{\inst_0}
	\pstree
	{
	     \Tr{\cpt_1}
	}
	{
      \Tr{\inst_1}
  	  \pstree
	  {
	     \Tr{\cpt_2}
	  }
	  {  
		 \Tr{\inst_2}
		 \pstree%[levelsep=*0.75cm]
		 {
		    \Tr[edge=\dedge]{\cpt_i} 
		 }
		 {  
	        \Tr{\inst_i}
	        \pstree[levelsep=0.75cm,treesep=1.5cm] 
			{
			   \Tr{\cpt_{i+1}}
			}
			{
			   \Tr{\inst_{i+1}}
			   \Tr[edge=\dedge]{\ \ \ \ \ \nudgeup{0.3cm} \ \ \ \ \ \ \ \ } 
			}
		 }
	  }
	}
  }
$$ and defines the algebras as equivalent to B-systems.
\end{frame}

\begin{frame}{Open Problem}
As brought to my attention in conversation with Richard Garner:

Characterise the interpretations $I: U \morph U'$, where $U$ and $U'$ are generalised algebaric theories,
for which the functor $\Ialg: \cat{U'-alg} \morph \cat{U-alg}$ is monadic.
%                       $\Ialg: \cat{U'-alg} \morph \cat{U-alg}$

\medskip
From this characterisation it should follow that 
\begin{itemize}
\item the forgetful functor $\cat{DirectedGraphs} \morph \cat{Set}$ is NOT monadic, 
\item the forgetful functor $\cat{Cat} \morph \cat{DirectedGraphs}$ is monadic,
\item the forgetful functor $\CIalg \morph \cat{Set}^{\cat{CI}_0}$ is monadic where $\cat{CI}_0$ is the category with countably many $c$ and $i$
objects described earlier. 
\end{itemize}
\end{frame}
