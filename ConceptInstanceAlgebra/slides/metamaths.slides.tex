
\newcommand{\inst}{i} 
\newcommand{\cpt}{c} 
\begin{frame}{Generalised Algebraic Theory of Concept Instance Algebras}
\begin{itemize}
\item There is a generalised algebraic theory of concept instance algebras.
\item sorts representing levels of concepts are $\cpt_1, \cpt_2 ...$.
\item sort $\cpt_{n}$ is introduced by:
\gatdisplayrule{\ofT{x_1}{\cpt_1},... \ofT{x_{n-1}}{\cpt_{n-1}(x_1,...x_{n-2})}} {\isT{\cpt_{n}(x_1,...x_{n-1})}}
\item sorts representing instances are $\inst_1, \inst_2, ....$.
\item sort $\inst_{n}$ is introduced by:
\gatdisplayrule{\context{x}{\cpt}{n}}{\isT{\inst_{n}(x_1,...x_n)}}
\item Similarly 
\begin{itemize}
\item countably many $^*$, $\cross$ and $\delta$ operations,
\item countably many axioms for each of the axioms given earlier.
\end{itemize}
\end{itemize}

\end{frame}

\begin{frame}{Relationship with Contextual Categories \& Generalised Algebraic Theories}
There are equivalences of categories between
\begin{itemize}
\item the category of generalised algebraic theories
\end{itemize}
and each of the following
\begin{itemize}
\item the category of concept instance algebras,
\item the category of contextual categories,
\item the category of contextual categories with families
\end{itemize}
but not with 
\begin{itemize}
\item the category of categories with families.
\end{itemize}
\end{frame}


\iffalse
\begin{frame}
\begin{displaymath}
\begin{array}{c p{0.1cm} c}
                     && \Rnode{root}{\circ}     \\[0.4cm]
                     && \Rnode{C0}{\cpt_0}         \\[0.4cm]
\Rnode{I0}{\inst_0}      && \Rnode{C1}{\cpt_1}         \\[0.4cm]
\Rnode{I1}{\inst_1}      && \Rnode{C2}{\cpt_2}         \\[0.25cm]
\Rnode{I2}{\inst_2}      && \vdots                  \\[0.25cm]
\vdots			     && \Rnode{Ci}{\cpt_i}         \\[0.4cm]
\Rnode{Ii}{\inst_i}      && \Rnode{Csi}{\cpt_{i+1}}    \\[0.25cm]
\Rnode{Isi}{\inst_{i+1}} && \vdots                  \\[0.25cm]
\vdots               &&
\end{array}
\begin{arrows}
\ncline[nodesep=4pt]{C0}{root}
\ncline[nodesep=4pt]{C1}{C0}
\ncline[nodesep=4pt]{C2}{C1}
\ncline[nodesep=4pt]{Csi}{Ci}
\ncline[nodesep=4pt]{I0}{C0}
\ncline[nodesep=4pt]{I1}{C1}
\ncline[nodesep=4pt]{I2}{C2}
\ncline[nodesep=4pt]{Ii}{Ci}
\ncline[nodesep=4pt]{Isi}{Csi}
\end{arrows}
\end{displaymath}
\end{frame}
\fi

\begin{frame}{Now as a Concept-Instance Algebra}
 \def\dedge{\ncline[linestyle=dotted]}
$$
\pstree[treemode=\CItreemode, treefit=loose,treenodesize=0.25cm,levelsep=0.65cm,treesep=0.7cm,nodesep=2pt]
{
  \Tr{\circ}
}
{
  \pstree
  {
     \Tr{\cpt_0}
  }
  {
    \Tr{\inst_0}
	\pstree
	{
	     \Tr{\cpt_1}
	}
	{
      \Tr{\inst_1}
  	  \pstree
	  {
	     \Tr{\cpt_2}
	  }
	  {  
		 \Tr{\inst_2}
		 \pstree%[levelsep=*0.75cm]
		 {
		    \Tr[edge=\dedge]{\cpt_i} 
		 }
		 {  
	        \Tr{\inst_i}
	        \pstree[levelsep=0.75cm,treesep=1.5cm] 
			{
			   \Tr{\cpt_{i+1}}
			}
			{
			   \Tr{\inst_{i+1}}
			   \Tr[edge=\dedge]{\ \ \ \ \ \nudgeup{0.3cm} \ \ \ \ \ \ \ \ } 
			}
		 }
	  }
	}
  }
}
$$
$$_0\cross_0$$
$$\qq{_0\cross_0}$$
$$\qq{\kern-3pt_0\cross_0\kern-3pt}$$
$$\qq{\cross_{0,0}} \in inst(\cpt_0 \cross (\cpt_0 \cross \cpt_0))$$
$$\qq{\delta_0} \in inst({\delta_{\cpt_0}}^*(\qq{\cross_0}^* (\cpt_0 \cross ( \cpt_0 \cross \inst_0))))$$
\end{frame}

\begin{frame}{As a Contextual Category}
 \def\dedge{\ncline[linestyle=dotted]}
 \def\sedge{\ncksar}
\begin{displaymath}
\pstree[edge=\sedge, treemode=U, treefit=loose,treenodesize=0.25cm,levelsep=0.8cm,treesep=0.7cm,nodesep=2pt]
{
  \Tr{1}
}
{
  \pstree
  {
     \Tr{\cpt_0}
  }
  {
    \Tr{\inst_0}
	\pstree
	{
	     \Tr{\cpt_1}
	}
	{
      \Tr{\inst_1}
  	  \pstree
	  {
	     \Tr{\cpt_2}
	  }
	  {  
		 \Tr{\inst_2}
		 \pstree%[levelsep=*0.75cm]
		 {
		    \Tr[edge=\dedge]{\cpt_i} 
		 }
		 {  
	        \Tr{\inst_i}
	        \pstree[levelsep=0.75cm,treesep=1.5cm] 
			{
			   \Tr{\cpt_{i+1}}
			}
			{
			   \Tr{\inst_{i+1}}
			   \Tr[edge=\dedge]{\ \ \ \ \ \nudgeup{0.3cm} \ \ \ \ \ \ \ \ } 
			}
		 }
	  }
	}
  }
}
\end{displaymath}
\end{frame}

\begin{frame}{Richard Garner...}
Describes a monad on the category $\Set^C$, where $C$ is this category 
 \def\dedge{\ncline[linestyle=dotted]}
 \def\backarrow{\nckarr}
$$
  \pstree[edge=\backarrow, treemode=U, treefit=loose,treenodesize=0.25cm,levelsep=0.8cm,treesep=0.7cm,nodesep=2pt]
  {
     \Tr{\cpt_0}
  }
  {
    \Tr{\inst_0}
	\pstree
	{
	     \Tr{\cpt_1}
	}
	{
      \Tr{\inst_1}
  	  \pstree
	  {
	     \Tr{\cpt_2}
	  }
	  {  
		 \Tr{\inst_2}
		 \pstree%[levelsep=*0.75cm]
		 {
		    \Tr[edge=\dedge]{\cpt_i} 
		 }
		 {  
	        \Tr{\inst_i}
	        \pstree[levelsep=0.75cm,treesep=1.5cm] 
			{
			   \Tr{\cpt_{i+1}}
			}
			{
			   \Tr{\inst_{i+1}}
			   \Tr[edge=\dedge]{\ \ \ \ \ \nudgeup{0.3cm} \ \ \ \ \ \ \ \ } 
			}
		 }
	  }
	}
  }
$$ and defines the algebras as equivalent to B-systems.
\end{frame}


\begin{frame}{Fungible Algebras}
Q. Are the following types of structure fungible
\begin{itemize}
\item concept instance algebras,
\item contextual categories,
\item contextual categories with families.
\end{itemize}
A. Depends on the matter at hand. If we are looking for Set-like instances of theories then they are, 
more generally they are not.

\begin{itemize}
\item concept-instance algebras and contextual categories are $\Sigma Id$-fungible
\end{itemize}
\end{frame}

