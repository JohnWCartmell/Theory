\begin{frame}{Construction of Contextual Category}
\begin{itemize}
\item From a concept-instance algebra $A$ we can construct a contextual category \catcw.
\item tree of objects of the category is the tree of concepts of the ci-algebra.
\item morphisms of \catcw n-tuples, $\tuple{\fn}:x \morph y_n$, 
      where $1 \base y_1 \base y_2 ... \base y_n$ in $A$ where $f_1 \in inst(\crossx{x}{y_1}{1})$, 
       $f_2 \in inst(\fonestar(\crossx{x}{y_2}{1}))$,...
       $f_n \in inst(\fnonestar...\ftwostar\fonestar(\crossx{x}{y_n}{1}))$
as shown here 

\scalebox{0.75}{
$
\begin{array}{ c p{0.4cm} c p{0.2cm} c p {0.2cm} c } 
\Rnode{fntarget}{\fnonestar...\ftwostar\fonestar(\crossx{x}{y_n}{1})}
&&\Rnode{f3target}{\ftwostar\fonestar(\crossx{x}{y_3}{1})}
&&\Rnode{f2target}{\fonestar(\crossx{x}{y_2}{1})}  
&& \Rnode{ab1}{\crossx{x}{y_1}{\Rnode{f1target}{1}}}     \\[2cm]
      &&     &&   \ovalnode[linestyle=none]{x}{x}     &&            
\begin{arrows}
\ncarc[arcangle=-5,nodesepA=15pt,offsetA=-2pt,nodesepB=3pt,offsetB=-5pt]{->}{x}{f1target}
\blabel{f_1}[0.6]
\ncarc[arcangle=10,nodesepA=15pt,offsetA=1pt,nodesepB=2pt,offsetB=2pt]{->}{x}{f2target}
\alabel{f_2}[0.4]
\ncarc[arcangle=10,nodesepA=15pt,offsetA=1pt,nodesepB=2pt,offsetB=2pt]{->}{x}{f3target}
\alabel{f_3}[0.65]
\ncarc[arcangle=7,nodesepA=15pt,offsetA=1pt,nodesepB=2pt,offsetB=2pt]{->}{x}{fntarget}
\alabel{f_n}[0.75][0]
\ncdotdotdot{fntarget}{f3target}
\setlength{\sarnodesepB}{10pt}
\ncsar{fntarget}{x}
\ncsar{f3target}{x}
\ncsar{f2target}{x}
\ncsar{f1target}{x}
\sarreset
\end{arrows}
\end{array}
\hspace{2cm}
\begin{array}{c}
\Rnode{bn}{y_n}             \\[1.0cm]
\Rnode{b2}{y_2}             \\[0.8cm]
\Rnode{b1}{y_1}             \\[0.8cm]
\Rnode{abs}{1}              \\
\begin{arrows}
\ncdotdotdot{bn}{b2}
\ncsar{b2}{b1}
\ncsar{b1}{abs}
\end{arrows}
\end{array}
\begin{arrows}
\nccdar{x}{abs}
\end{arrows}
$
}
\end{itemize}
\end{frame}

\begin{frame}{Identity and Dependency Morphisms in $C(A)$}
If $1 \base x_1 \base x_2 ... \base x_n$ in $C(A)$
then 
\begin{itemize}
\item define the identity morphism $id_{x_n}:x_n \morph x_n$  
to be the n-tuple:
$$\tuple{\delta_{x_n,x_1},...\delta_{x_n,x_n}}$$
\item define the dependency morphism $p_{x_n}:x_n \smorph x_{n-1}$ to be the n-tuple:
$$\tuple{\delta_{x_n,x_1},...\delta_{x_n,x_{n-1}}}$$
\end{itemize}
\end{frame}

\begin{frame}{Cascade Lemma for CI-algebras }
If $x$ is an object of CI-algebra $A$, if $1 \base y_1 ... \base y_n$ and 
if $\tuple{f_1,...f_n}$ is a morphism from $x$ to $y_n$ in $C(A)$ then \foreachi, $x \base \fipvectorstar(\crossx{x}{y_i}{1})$.
Additionally if $y$ is some concept such that $y_n \base y$ in $A$ then 
 $x \base \fnvectorstar(\crossx{x}{y}{1})$  and
if $g \in inst(y)$ 
then $\fnvectorstar(\crossx{x}{g}{1}) \in inst(\fnvectorstar(\crossx{x}{y}{1}))$.
DIAGRAM HERE NEED MODULARISE "Instances" paper.
\end{frame}

\iffalse{
\begin{frame}{Composition of morphisms}
Composition of morphisms is defined as follows.

$$\tuple{f_1,...f_n}\circ \tuple{g_1,...g_m} 
=\tuple{\fnvectorstar(\crossx{x}{g_1}{1}),...\fnvectorstar(\crossx{x}{g_m}{1})}
$$
\end{frame}
}\fi

\begin{frame}{$\circ$, $f^*x$ and $q(f,x)$ and $s(f)$}
\newcommand{\pullbackobject}{\fnvectorstar(\crossx{x}{y}{1})}
\newcommand{\deltaterm}{\delta_{\pullbackobject,x}}

Define composition
\begin{equation} 
\tuple{f_1,...f_n}\circ \tuple{g_1,...g_m} 
=\tuple{\fnvectorstar(\crossx{x}{g_1}{1}),...\fnvectorstar(\crossx{x}{g_m}{1})},
\end{equation}
$^*$ 
\begin{equation} 
\tuple{\fn}^*y=\pullbackobject
\end{equation}
and $q$ 
\begin{equation} 
q(\tuple{\fn},y)=
\tuple{{\beta}^*f_1,...{\beta}^*f_n,\beta}
\end{equation}

where 
$\beta = \deltaterm$

and $s$ 
\begin{equation} 
s(\tuple{\fn})=f_n.
\end{equation}

Can prove that $C(A)$ is a contextual category by showing that as defined
$\circ$,$id$, $^*$, $q$, $s$ are all well-defined, properly typed and satisfy the axioms discovered by Vladimir Voevodsky and documented on my ResearchGate page
www.researchgate.net/profile/John-Cartmell
as 
"Generalised Algebraic Axiomatisations of Contextual Categories".
\end{frame}

\begin{frame}{xxxxxx}
\end{frame}


