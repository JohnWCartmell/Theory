\begin{frame}{CC -- Construction of Contextual Category}
\begin{itemize}
\item From a concept-instance algebra $A$ we can construct a contextual category $CC(A)$.
\item the tree of objects of the category $CC(A)$ is defined to be the tree of concepts of the CI-algebra.
\item if $x$ and $y_n$ are concepts and $1 \base y_1 \base y_2 ... \base y_n$ in $A$
then $Hom_{CC(A)}(x,y_n)$ is defined to be the set of n-tuples $\tuple{\fn}$ of instances of $A$ 
where 
\begin{align*}
f_1 \in in&st(\crossx{x}{y_1}{1}),                                 \\ 
f_2 \in in&st(\fonestar(\crossx{x}{y_2}{1})),                      \\
&\vdots                                                            \\
f_n \in in&st(\fnonestar...\ftwostar\fonestar(\crossx{x}{y_n}{1})) \\
\end{align*}
\end{itemize}
\end{frame}

\begin{frame}{Identity and Dependency Morphisms in $CC(A)$}
If $1 \base x_1 \base x_2 ... \base x_n$ in $CC(A)$
then 
\begin{itemize}
\item define the identity morphism $id_{x_n}:x_n \morph x_n$  
to be the n-tuple:
$$\tuple{\delta_{x_n,x_1},...\delta_{x_n,x_n}}$$
\item define the dependency morphism $p_{x_n}:x_n \smorph x_{n-1}$ to be the n-tuple:
$$\tuple{\delta_{x_n,x_1},...\delta_{x_n,x_{n-1}}}$$
\end{itemize}
\end{frame}


\iffalse{
\begin{frame}{Composition of morphisms}
Composition of morphisms is defined as follows.

$$\tuple{f_1,...f_n}\circ \tuple{g_1,...g_m} 
=\tuple{\fnvectorstar(\crossx{x}{g_1}{1}),...\fnvectorstar(\crossx{x}{g_m}{1})}
$$
\end{frame}
}\fi

\begin{frame}{$\circ$, $f^*x$ and $q(f,x)$ and $s(f)$}
\newcommand{\pullbackobject}{\fnvectorstar(\crossx{x}{y}{1})}
\newcommand{\deltaterm}{\delta_{\pullbackobject,x}}

Define composition by
\begin{equation} 
\tuple{f_1,...f_n}\circ \tuple{g_1,...g_m} 
=\tuple{\fnvectorstar(\crossx{x}{g_1}{1}),...\fnvectorstar(\crossx{x}{g_m}{1})},
\end{equation}
define the $^*$ operator by
\begin{equation} 
\tuple{\fn}^*y=\pullbackobject,
\end{equation}
define the $q$ operator by
\begin{equation} 
q(\tuple{\fn},y)=
\tuple{{\beta}^*f_1,...{\beta}^*f_n,\beta},
\end{equation}
where 
$\beta = \deltaterm$,

and define the $s$ operator by 
\begin{equation} 
s(\tuple{\fn})=\tuple{\delta_{x_n,x_1},...\delta_{x_n,x_n},f_n}.
\end{equation}

Can prove that $CC(A)$ is a contextual category by showing that the operators
$\circ$,$id$, $^*$, $q$, $s$ are well-defined, properly typed and satisfy the axioms discovered by Vladimir Voevodsky and documented on my ResearchGate page
www.researchgate.net/profile/John-Cartmell
as 
"Generalised Algebraic Axiomatisations of Contextual Categories".
\end{frame}


