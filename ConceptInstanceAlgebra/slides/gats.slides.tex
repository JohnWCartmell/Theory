\newcommand{\ofOb}[1]{\ofT{#1}{\Ob}}
\newcommand{\ofHom}[2]{\ofT{#1}{\Hom(#2)}}

\begin{frame}{Generalised Algebraic Theories}
Expressed in a syntax that
\begin{itemize}
\item involves rules of these forms\\
\begin{tabular}{c p{1cm} c}
\gatdisplayrule{\xDelta{n}}{\isT{\Delta}}   & \gatdisplayrule{\xDelta{n}}{\ofT{t}{\Delta}}\\
\gatdisplayrule{\xDelta{n}}{\Delta=\Delta'} & \gatdisplayrule{\xDelta{n}}{t=\ofT{t'}{\Delta}} 
\end{tabular}
\item and meta-rules which I call principles of derivation each enabling a rule of one of the above forms to be derived from a number of other previously derived rules
\begin{itemize}
\item LI1...LI7 -- Laws of Identity,
\item T1 -- Transfer Law,
\item CF1, CF2(a) and CF2(b) -- Cut-free versions of term and type substitution laws,
\item SI1 and SI2 -- Substitution of and into Identicals.
\end{itemize} 
\end{itemize}
\end{frame}


\begin{frame}{Generalised Algebraic Theories}
\begin{itemize}
\item A set of sort symbols each with an introductory rule. 
A sort symbol $A$ must have an introductory rule of the form \genericAintroductoryrule and this rule must be well-typed.
\item A set of operator symbols each with an introductory rule. 
An operator symbol $f$ must have an introductory rule of this form   \genericfintroductoryrule and this rule  must be well-typed,
\item a set of well-typed axioms. Each axiom is of one of these two forms
\begin{tabular}{c p{1cm} c}
\gatdisplayrule{\xDelta{n}}{\Delta=\Delta'} & \gatdisplayrule{\xDelta{n}}{t=\ofT{t'}{\Delta}} 
\end{tabular}
\end{itemize}
\end{frame}

\begin{frame}{Theory of Categories}
\footnotesize
\newcommand{\associativitypremisereversed}
       {\begin{array}[t]{l}\ofT{f}{Hom(z_1,z_2)},\,\ofT{g}{Hom(z_2,z_3)},\,\\
                       \hspace{1.6cm}\ofT{h}{Hom(z_3,z_4)},\,\ofT{z_1,z_2,z_3,z_4}{Ob}
        \end{array}
       }
\begin{gatrules}
\gatintros
\gatintroducing{Ob}
\isT{Ob} \\
\gatintroducing{Hom}
  \gatsingular{\ofT{x_1,x_2}{Ob}}{\isT{Hom(x_1,x_2)}} \\	
\gatintroducing{id}
  \gatsingular{\ofT{w}{Ob}}{\ofT{id_w}{Hom(w,w)}} \\	
\gataxioms
\gatintroducing{  \gataxiomno{1} \\   \gataxiomno{2}}
\begin{gatgroup}{\ofT{f}{Hom(x_1,x_2)},\ \ofT{x_1,x_2}{Ob}}
    \gatleaf{}{id_{x_1} \circ f = f} \\
    \gatleaf{}{f \circ id_{x_2} = f}
\end{gatgroup} \\
\gatintroducing{ \gataxiomno{3} }
\gatsingular{\associativitypremisereversed}{(f \circ g) \circ h = f \circ (g \circ h)} 
\end{gatrules}
\end{frame}

\begin{frame}{Theory of Cricket}
\footnotesize

% In this example I have had to fine tune the various widths.
% I haven't been able to find a way of getting the dotfill to fill
% out the entire width of a containing column of an array.
\begin{displaymath}
\begin{array}{l}
\isT{match} \\
\begin{gatgroup}{\ofT{m}{match}}
  \gatleaf[8.0cm]{}{\isT{innings(m)}} \\
  \gatleaf[8cm]{}{\isT{side(m)}}\\
  \gatleaf[8cm]{}{\ofT{homeSide(m)}{side}}\\
  \begin{gatgroup}{\ofT{s}{side(m)}}
    \gatleaf[6cm]{} {\isT{player(s)}} \\
    \gatleaf[6cm]{} {\ofT{captain(s)}{player(s)}}
  \end{gatgroup} \\
    \begin{gatgroup}{\ofT{i}{innings(m)}}
    \gatleaf[6cm]{}{\ofT{fieldingSide(i)}{side}} \\
    \gatleaf[6cm]{}{\ofT{battingSide(i)}{side}} \\
    \gatleaf[6cm]{}{\isT{over(i)}} \\
    \begin{gatgroup}{\ofT{o} {over(i)}}
      \gatleaf[5.8cm]{}{\ofT{bowler(o)}{player(fieldingSide(i))}} \\
      \gatleaf[5.8cm]{}{\isT{delivery(o)}} \\
      \gatleaf[5.8cm]{}{\ofT{facingBatter(d)}{player(battingSide(i))}} \\
      \makebox[5.8cm][r]{\hspace{2cm} \dotfill where $\ofT{d}{delivery(o)}$  }
    \end{gatgroup} 
  \end{gatgroup} 
\end{gatgroup}
\end{array}
\end{displaymath}


\end{frame}

\begin{frame}{Cricket as a Logical Data Specification}
\scalebox{0.75}{\begin{erdiagram}{11.649999999999999}{6.2958750000000006}

\eret{1.2}{-2.2}{5.2}{-1.3}{0.2}{1}\eretname{1.6}{-1.65}{l}{match}
\erattr{1.4}{-1.85}{1}{0}{id}
\eret{0.104}{-5.35}{1.836}{-4.05}{0.2}{1}\eretname{0.277}{-4.4}{l}{innings}
\erattr{0.304}{-4.6}{1}{0}{number}
\eret{4.836}{-5.35}{6.296}{-4.05}{0.2}{1}\eretname{4.982}{-4.4}{l}{side}
\erattr{5.036}{-4.6}{1}{0}{name}
\eret{0.213}{-8.2}{1.728}{-7}{0.2}{1}\eretname{0.364}{-7.35}{l}{over}
\erattr{0.413}{-7.55}{1}{0}{number}
\eret{4.87}{-8.2}{6.262}{-7}{0.2}{1}\eretname{5.009}{-7.35}{l}{player}
\erattr{5.07}{-7.55}{1}{0}{name}
\eret{0.134}{-11.15}{1.806}{-9.95}{0.2}{1}\eretname{0.301}{-10.3}{l}{delivery}
\erattr{0.334}{-10.5}{1}{0}{number}
\eret{0}{-0.2}{6.296}{0.3}{0.2}{1}

% relationship 
\errelname{3.35}{-0.5}{l}{}\errelarm{3.2}{-0.2}{3.2}{-0.75}{1}{0}\errelarm{3.2}{-0.75}{3.2}{-1.3}{1}{0}\ercrowfoot{3.2}{-1.15}{3.05}{-1.3}{3.2}{-1.3}{3.35}{-1.3}{0}
% relationship 
\errelname{2.383}{-2.5}{r}{}\errelarm{2.533}{-2.2}{2.533}{-2.275}{1}{0}\errelarm{0.97}{-3.838}{0.97}{-4.05}{1}{0}\errelangle{2.533}{-2.275}{2.533}{-2.35}{1.752}{-2.988}{1}{0}\errelangle{1.752}{-2.988}{0.97}{-3.625}{0.97}{-3.838}{1}{0}\eridcomprel{0.8700000000000003}{1.0700000000000003}{-3.8000000000000007}\ercrowfoot{0.97}{-3.9}{0.82}{-4.05}{0.97}{-4.05}{1.12}{-4.05}{0}
% relationship 
\errelname{4.017}{-2.5}{l}{}\errelarm{3.867}{-2.2}{3.867}{-2.275}{1}{0}\errelarm{5.566}{-3.838}{5.566}{-4.05}{1}{0}\errelangle{3.867}{-2.275}{3.867}{-2.35}{4.716}{-2.988}{1}{0}\errelangle{4.716}{-2.988}{5.566}{-3.625}{5.566}{-3.838}{1}{0}\eridcomprel{5.4658750000000005}{5.665875}{-3.8000000000000007}\ercrowfoot{5.566}{-3.9}{5.416}{-4.05}{5.566}{-4.05}{5.716}{-4.05}{0}
% relationship 
\errelname{1.12}{-5.65}{l}{}\errelarm{0.97}{-5.35}{0.97}{-6.175}{1}{0}\errelarm{0.97}{-6.175}{0.97}{-7}{1}{0}\eridcomprel{0.8700000000000003}{1.0700000000000003}{-6.75}\ercrowfoot{0.97}{-6.85}{0.82}{-7}{0.97}{-7}{1.12}{-7}{0}
% relationship battingSide
\errelname{1.986}{-4.333}{l}{battingSide}\errelarm{1.836}{-4.483}{3.336}{-4.483}{1}{0}\errelarm{3.336}{-4.483}{4.836}{-4.483}{0}{0}\ercrowfoot{1.986}{-4.483}{1.836}{-4.333}{1.836}{-4.483}{1.836}{-4.633}{0}
% relationship fieldingSide
\errelname{1.986}{-4.767}{l}{fieldingSide}\errelarm{1.836}{-4.917}{3.336}{-4.917}{1}{0}\errelarm{3.336}{-4.917}{4.836}{-4.917}{0}{0}\ercrowfoot{1.986}{-4.917}{1.836}{-4.767}{1.836}{-4.917}{1.836}{-5.067}{0}
% relationship 
\errelname{5.716}{-5.65}{l}{}\errelarm{5.566}{-5.35}{5.566}{-6.175}{1}{0}\errelarm{5.566}{-6.175}{5.566}{-7}{1}{0}\eridcomprel{5.4658750000000005}{5.665875}{-6.75}\ercrowfoot{5.566}{-6.85}{5.416}{-7}{5.566}{-7}{5.716}{-7}{0}
% relationship 
\errelname{1.12}{-8.5}{l}{}\errelarm{0.97}{-8.2}{0.97}{-9.075}{1}{0}\errelarm{0.97}{-9.075}{0.97}{-9.95}{1}{0}\eridcomprel{0.8700000000000003}{1.0700000000000003}{-9.7}\ercrowfoot{0.97}{-9.8}{0.82}{-9.95}{0.97}{-9.95}{1.12}{-9.95}{0}
% relationship bowler
\errelname{1.878}{-7.45}{l}{bowler}\errelarm{1.728}{-7.6}{3.299}{-7.6}{1}{0}\errelarm{3.299}{-7.6}{4.87}{-7.6}{0}{0}\ercrowfoot{1.878}{-7.6}{1.728}{-7.45}{1.728}{-7.6}{1.728}{-7.75}{0}
% relationship facingBatter
\errelname{1.956}{-10.85}{l}{facingBatter}\errelarm{1.806}{-10.55}{2.406}{-10.55}{1}{0}\errelarm{4.67}{-8.08}{4.87}{-8.08}{0}{0}\errelangle{2.406}{-10.55}{3.006}{-10.55}{3.738}{-9.315}{1}{0}\errelangle{3.738}{-9.315}{4.47}{-8.08}{4.67}{-8.08}{0}{0}\ercrowfoot{1.956}{-10.55}{1.806}{-10.4}{1.806}{-10.55}{1.806}{-10.7}{0}
\end{erdiagram}
}
\end{frame}