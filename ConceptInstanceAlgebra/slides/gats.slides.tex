\newcommand{\ofOb}[1]{\ofT{#1}{\Ob}}
\newcommand{\ofHom}[2]{\ofT{#1}{\Hom(#2)}}

\begin{frame}{Generalised Algebraic Theories}
Generalised algebraic theories are expressed in a syntax that
\begin{itemize}
\item involves rules of these forms\\
\begin{tabular}{c p{1cm} c}
\gatdisplayrule{\xDelta{n}}{\isT{\Delta}}   & \gatdisplayrule{\xDelta{n}}{\ofT{t}{\Delta}}\\
\gatdisplayrule{\xDelta{n}}{\Delta=\Delta'} & \gatdisplayrule{\xDelta{n}}{t=\ofT{t'}{\Delta}} 
\end{tabular}
\item and meta-rules which I call principles of derivation each enabling a rule of one of the above forms to be derived from a number of other previously derived rules
\begin{itemize}
\item LI1...LI7 -- Laws of Identity,
\item T1 -- Transfer Law,
\item CF1, CF2(a) and CF2(b) -- Cut-free versions of term and type substitution laws,
\item SI1 and SI2 -- Substitution of and into Identicals.
\end{itemize} 
\end{itemize}
\end{frame}


\begin{frame}{Generalised Algebraic Theories}
... are defined to consist of
\begin{itemize}
\item A set of sort symbols each with an introductory rule. 
A sort symbol $A$ must have an introductory rule of the form \genericAintroductoryrule and this rule must be well-typed.
\item A set of operator symbols each with an introductory rule. 
An operator symbol $f$ must have an introductory rule of this form   \genericfintroductoryrule and this rule  must be well-typed,
\item a set of well-typed axioms. Each axiom is of one of these two forms
\begin{tabular}{c p{1cm} c}
\gatdisplayrule{\xDelta{n}}{\Delta=\Delta'} & \gatdisplayrule{\xDelta{n}}{t=\ofT{t'}{\Delta}} 
\end{tabular}
\end{itemize}
\end{frame}

\begin{frame}{Generalised Algebraic Theory of Categories}
\footnotesize
\newcommand{\associativitypremisereversed}
       {\begin{array}[t]{l}\ofT{f}{Hom(z_1,z_2)},\,\ofT{g}{Hom(z_2,z_3)},\,\\
                       \hspace{1.6cm}\ofT{h}{Hom(z_3,z_4)},\,\ofT{z_1,z_2,z_3,z_4}{Ob}
        \end{array}
       }
\begin{gatrules}
\gatintros
\gatintroducing{Ob}
\isT{Ob} \\
\gatintroducing{Hom}
  \gatsingular{\ofT{x_1,x_2}{Ob}}{\isT{Hom(x_1,x_2)}} \\	
\gatintroducing{id}
  \gatsingular{\ofT{w}{Ob}}{\ofT{id_w}{Hom(w,w)}} \\	
\gataxioms
\gatintroducing{  \gataxiomno{1} \\   \gataxiomno{2}}
\begin{gatgroup}{\ofT{f}{Hom(x_1,x_2)},\ \ofT{x_1,x_2}{Ob}}
    \gatleaf{}{id_{x_1} \circ f = f} \\
    \gatleaf{}{f \circ id_{x_2} = f}
\end{gatgroup} \\
\gatintroducing{ \gataxiomno{3} }
\gatsingular{\associativitypremisereversed}{(f \circ g) \circ h = f \circ (g \circ h)} 
\end{gatrules}
\end{frame}

\begin{frame}{Instances of Generalised Algebraic Theories}
\begin{itemize}
\item in "Generalised Algebraic Theories and Contextual Categories":
\begin{itemize}
  \item notion of interpretation $I: U \morph U'$ 
  \item the  category of  generalised algebraic theories 
      is defined in my thesis (though actually a 2-category)
  \item  a $U$-algebra, $A$, is a contextual functor $A: \cat{C(U)} \morph \cat{Fam}$
  \item \cat{U-alg} -- category of $U$-algebras
  \item $\Ialg: \cat{U'-alg} \morph \cat{U-alg}$ functor induced by an interpretation of $U$ in $U'$.
\end{itemize}
\item more recently 
\begin{itemize}
    \item I have defined the notion of an instance of a generalised algebraic theory in a contextual category
         (my ResearchGate page "Instances of Generalised Algebraic Theories")
    \item a $U$-algebra is an instance of theory $U$ in the contextual category \cat{Fam}
\end{itemize}
\end{itemize}
\end{frame}




\newcommand{\USigmaA}{U_{\SigmaA}}
\newcommand{\USigmaAalg}{\USigmaA\mhyphen alg}
\newcommand{\SigmaA}{\Sigma A}

\begin{frame}{Adding a Sigma sort to a generalised algebraic theory $U$.}
If a generalised algebraic theory has a sort symbol $A$ introduced by rule
{\footnotesize \genericAintroductoryrule} then can extend $U$ by a sort symbol $\SigmaA$ and three operations: 
\begin{center}
\footnotesize
\begin{tabular}{c l}
$\SigmaA$ & \gatdisplayrule{\xDelta{n-1}} {\isT{\SigmaA(x_1,...x_{n-1})}} \\
$pr$  & \gatdisplayrule{\xDelta{n}, \ofT{y}{A(\xn)}}  {\ofT{pr(x_n,y)} {\SigmaA(x_1,...x_{n-1})}} \\
$p_1$ & \gatdisplayrule{\xDelta{n-1},\ofT{z}{\SigmaA(x_1,...x_{n-1})}}{\ofT{p1(z)} {\Delta_n}} \\
$p_2$ & \gatdisplayrule{\xDelta{n-1},\ofT{z}{\SigmaA(x_1,...x_{n-1})}}{\ofT{p2(z)} {A(x_1,...x_{n-1}, p_1(z))}} 
\end{tabular}
\end{center}
\end{frame}
\begin{frame}
subject to axioms
\begin{center}
\footnotesize 
\begin{tabular}{c}
\gatdisplayrule{\xDelta{n}, \ofT{y}{A(\xn)}}  {p_1(pr(x_n,y))=x_n} \\
\gatdisplayrule{\xDelta{n}, \ofT{y}{A(\xn)}}  {p_2(pr(x_n,y))=y} \\
\gatdisplayrule{\xDelta{n-1},\ofT{z}{\SigmaA(x_1,...x_{n-1})}}{pr(p_1(z),p_2(z))=z} \\ 
\end{tabular}
\end{center}

so that in the extended theory $\USigmaA$ there is a sort symbol representing 
$$\sum_{\ofT{x_n}{\Delta_n}}{A(\xn)}.$$

There is an obvious embedding $I: U \morph \USigmaA$ in \cat{GAT}.
It is easy to show that the functor $\Ialg : \USigmaAalg \morph \Ualg$ 
is an equivalence of categories (since \cat{Fam} has cannonical $\Sigma$ operations i.e. disjoint unions).
\end{frame}

\newcommand{\IdA}{IdA}
\begin{frame}{Adding an Id sort to a generalised algebraic theory $U$.}
If a generalised algebraic theory has a sort symbol $A$ introduced by rule
{\footnotesize \genericAintroductoryrule} then can extend $U$ by a sort symbol $\IdA$ and an operation $r$ 
\begin{center}
\footnotesize
\begin{tabular}{c l}
$\IdA$ & \gatdisplayrule{\xDelta{n}, \ofT{y_1}{A(\xn)}, \ofT{y_2}{A(\xn)}}  {\isT{\IdA(y_1,y_2)}} \\
$r$    & \gatdisplayrule{\xDelta{n}, \ofT{y}{A(\xn)}}  {\ofT{r(y)}{\IdA(y,y)}} 
\end{tabular}
\end{center}
\end{frame}
\begin{frame}
subject to the axiom
\begin{center}
\footnotesize 
\begin{tabular}{c}
\gatdisplayrule{\xDelta{n}, \ofT{y_1}{A(\xn)}, \ofT{y_2}{A(\xn)}, \ofT{z}{\IdA(y_1,y_2)}}
  {y1=y2}
\end{tabular}
\end{center}
\newcommand{\UIdA}{U_{\IdA}}
\newcommand{\UIdAalg}{\UIdA\mhyphen alg}
so that in the extended theory $\UIdA$ there is a sort symbol representing 
the identity on type $A$.

There is an obvious embedding $I: U \morph \UIdA$ in \cat{GAT}.
It is easy to show that the functor $\Ialg : \UIdAalg \morph \Ualg$ 
is an equivalence of categories (defining and using a cannonical $Id$ structure in Fam).
\end{frame}

\iffalse
\begin{frame}{Theory of Cricket}
\footnotesize

% In this example I have had to fine tune the various widths.
% I haven't been able to find a way of getting the dotfill to fill
% out the entire width of a containing column of an array.
\begin{displaymath}
\begin{array}{l}
\isT{match} \\
\begin{gatgroup}{\ofT{m}{match}}
  \gatleaf[8.0cm]{}{\isT{innings(m)}} \\
  \gatleaf[8cm]{}{\isT{side(m)}}\\
  \gatleaf[8cm]{}{\ofT{homeSide(m)}{side}}\\
  \begin{gatgroup}{\ofT{s}{side(m)}}
    \gatleaf[6cm]{} {\isT{player(s)}} \\
    \gatleaf[6cm]{} {\ofT{captain(s)}{player(s)}}
  \end{gatgroup} \\
    \begin{gatgroup}{\ofT{i}{innings(m)}}
    \gatleaf[6cm]{}{\ofT{fieldingSide(i)}{side}} \\
    \gatleaf[6cm]{}{\ofT{battingSide(i)}{side}} \\
    \gatleaf[6cm]{}{\isT{over(i)}} \\
    \begin{gatgroup}{\ofT{o} {over(i)}}
      \gatleaf[5.8cm]{}{\ofT{bowler(o)}{player(fieldingSide(i))}} \\
      \gatleaf[5.8cm]{}{\isT{delivery(o)}} \\
      \gatleaf[5.8cm]{}{\ofT{facingBatter(d)}{player(battingSide(i))}} \\
      \makebox[5.8cm][r]{\hspace{2cm} \dotfill where $\ofT{d}{delivery(o)}$  }
    \end{gatgroup} 
  \end{gatgroup} 
\end{gatgroup}
\end{array}
\end{displaymath}


\end{frame}
\fi
