\begin{frame}{Concept-Instance Algebra -  Overview}
In a concept-instance algebra there are concepts and instances and there are operations $^*$, $\crossx{}{}{}$ and $\delta$
and these must satisfy various axioms.
\begin{itemize}
\item there is a rooted $\omega$-tree of concepts,
\item for each non-root concept there is a associated set of instances (instances of that concept),
\item operation $^*$ enables  concepts and instances to be particularised (syntactically this is substitution),
\item  operation $\crossx{}{}{}$ enables the positing in-scope concepts and instances (syntactially this is sometimes called weaking),
\item operation $\delta$ is an expression of self (as for example the identity morphisms in a category).
\end{itemize}
\end{frame}

\iffalse
\begin{frame}{Summary of Operations -- $^*$, $\cross$ and $\delta$}
Summary: assume $x_c \base x$ and $y_c \base y$ in the tree of concepts of an algebra $A$. \\
\medskip
\scalebox{0.9}{%scope for tabcolsep
\setlength\tabcolsep{2pt}
\begin{tabular}{|l|l|p{3cm}|l|}
\hline
            &     reads as                      & defined whenever    & and is such that \\
\hhline{|=|=|=|=|}
$f^*y$      & particularise $y$ to $f$  & $f \in inst(x)$\newline \rule{0.8cm}{0pt} and $x<y$   &\begin{tabular}[t] {l}
                                                                              $x \base y$ then $x_c \base f^*y$\\
                                                                              $x \ll y$ then $f^*y_c \base f^*y$
                                                                           \end{tabular}                              \\
\hline
$f^*g$       & particularise $g$ to $f$ & if also $g\in inst(y)$      & $f^*g \in inst(f^*y)$                          \\
\hline
$x \cross y$ &posit $y$ in context $x$  & $x_c < y$                      &\begin{tabular} {l}
                                                                              $x_c \base y$ then $x \base (x \cross y)$\\
                                                                              $x_c \ll y$ then $(x \cross y_c) \base (x \cross y)$
                                                                           \end{tabular}                              \\
\hline
$x \cross g$ &posit $g$ in context $x$  & if also $g\in inst(y)$            & $x \cross g \in inst(x \cross y)$            \\
\hline
$\delta_x$   & this $x$                 &                         & $\delta_x \in inst(x \cross x)$              \\
\hline
\end{tabular}
}
\end{frame}
\fi

\begin{frame}{Concepts and their Contexts}
\begin{itemize}
\item In a concept-instance algebra $A$ concepts are equally contexts.
\medskip
\pause
\item If $x_1 \base x_2 ... \base x_n \base y$ in the tree of concepts then
\begin{itemize}
    \item $x_1\base x_2 ...\base x_n$ is a system of nested contexts,
    \item $x_n$ is the context for $y$,
    \item $x_i < y$ means that $x_i$ is part of the context for $y$.
\end{itemize}
\end{itemize} 

\end{frame}

\begin{frame}{Example context/concept trees}
\begin{tabular}{c c c c}
& \underline{From grammar}  && \onslide<3->{\underline{From the game of cricket}} \\[0.5cm]
\raisebox{-4cm}{\rule{0cm}{4cm}} % to claim space and stop slide wiggle in animation
 &
\onslide<2->{\only<2-3>{$\pstree[treemode=\CItreemode,levelsep=*0.65cm,treesep=0.2cm,nodesep=0.05]
{
    \Tr{\circ}
}
{
    \pstree [levelsep=*0.85cm]
    {
		\Tr{language} 
	}
	{		  
		\pstree [levelsep=*0.85cm]
		{
				   \Tr{sentence} 
		}
		{
					\Tr{\makebox[1.4cm][c]{noun}}
					\Tr{\makebox[1.4cm][c]{verb}}
					\Tr{\makebox[1.4cm][c]{adjective}} 
		}	
	}	
}$}\only<4->{\def\psedge{\ncksar}$\pstree[treemode=\CItreemode,levelsep=*0.65cm,treesep=0.2cm,nodesep=0.05]
{
    \Tr{\circ}
}
{
    \pstree [levelsep=*0.85cm]
    {
		\Tr{language} 
	}
	{		  
		\pstree [levelsep=*0.85cm]
		{
				   \Tr{sentence} 
		}
		{
					\Tr{\makebox[1.4cm][c]{noun}}
					\Tr{\makebox[1.4cm][c]{verb}}
					\Tr{\makebox[1.4cm][c]{adjective}} 
		}	
	}	
}$}}
& &
\onslide<3->{\only<3-4>{$\pstree[treemode=\CItreemode,levelsep=*0.65cm,treesep=\CItreesep,nodesep=0.05]
{
    \Tr{\circ}
}
{
    \pstree [levelsep=*0.85cm]
    {
		\Tr{match} 
	}
	{		  
		\pstree [levelsep=*0.85cm]
		{
				   \Tr{side} %\uppermember {homeSide}
		}
		{
					\Tr{player} %\uppermember {captain}
		}
	    \pstree [levelsep=*0.85cm]
		{
			\Tr{innings} 
		}
		{		  
		    \pstree [levelsep=*0.85cm]
			{
					   \Tr{over} 
			}
			{   
					   \Tr{delivery} 
			}			
		}		
	}	
}$}\only<5>{\def\psedge{\ncksar}$\pstree[treemode=\CItreemode,levelsep=*0.65cm,treesep=\CItreesep,nodesep=0.05]
{
    \Tr{\circ}
}
{
    \pstree [levelsep=*0.85cm]
    {
		\Tr{match} 
	}
	{		  
		\pstree [levelsep=*0.85cm]
		{
				   \Tr{side} %\uppermember {homeSide}
		}
		{
					\Tr{player} %\uppermember {captain}
		}
	    \pstree [levelsep=*0.85cm]
		{
			\Tr{innings} 
		}
		{		  
		    \pstree [levelsep=*0.85cm]
			{
					   \Tr{over} 
			}
			{   
					   \Tr{delivery} 
			}			
		}		
	}	
}$}}
\end{tabular}
\end{frame}

\def\psedge{\ncksar}  % Now redefine globally

\begin{frame}{Instances}
\begin{itemize}
\item If $i \in inst(x)$ and $x_c \base x$  so that $x_c$ is context for $x$ then
 $x_c$ is context for $i$.
 \medskip
\onslide<2-> {\item Example: $homeSide \in inst(side)$ and $match \base side$ implies  $match$ is context for $homeSide$.}
\onslide<4-> {\item Example: $captain \in inst(player)$ and $side \base player$ implies $side$ is context for $captain$.}
\end{itemize}
\onslide<3->{$$\begin{displaymath}
\pstree[treemode=\CItreemode,levelsep=*0.65cm,treesep=\CItreesep,nodesep=0.05]
{
    \Tr{\circ}
}
{
    \pstree [levelsep=*0.85cm]
    {
		\Tr{match} 
	}
	{		  
		\pstree [levelsep=*0.85cm]
		{
				   \Tr{side} \member {homeSide}
		}
		{
					\Tr{player} \waitfor{5}{\member {captain}}
		}
	    \pstree [levelsep=*0.85cm]
		{
			\Tr{innings} 
		}
		{		  
		    \pstree [levelsep=*0.85cm]
			{
					   \Tr{over} 
			}
			{   
					   \Tr{delivery} 
			}			
		}		
	}	
}
\end{displaymath}$$}
\end{frame}




