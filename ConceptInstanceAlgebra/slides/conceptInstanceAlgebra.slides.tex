\begin{frame}{Concept-Instance Algebra -  Overview}
In a concept-instance algebra there are concepts and instances and there are operations $^*$, $\crossx{}{}{}$ and $\delta$
and these must satisfy various axioms.
\begin{itemize}
\item there is a rooted $\omega$-tree of concepts,
\item for each non-root concept there is a associated set of instances (instances of that concept),
\item operation $^*$ enables  concepts and instances to be particularised (syntactically this is substitution),
\item  operation $\crossx{}{}{}$ enables the positing in-scope concepts and instances (syntactially this is sometimes called weaking),
\item operation $\delta$ is an expression of self (as for example the identity morphisms in a category).
\end{itemize}
\end{frame}

\begin{frame}{Concepts and their Contexts}
\begin{itemize}
\item In a concept-instance algebra $A$ concepts are equally contexts.
\item If $a \base b \base c$ in the tree of concepts of algebra $A$ 
then $a$ is to be understood as a context for concept $b$ and $b$ is to be understood as a context for concept $c$,
\item For example some concepts in cricket
\pstree[treemode=\CItreemode,levelsep=*0.65cm,treesep=\CItreesep,nodesep=0.05]
{
    \Tr{\circ}
}
{
    \pstree [levelsep=*0.85cm]
    {
		\Tr{match} 
	}
	{		  
		\pstree [levelsep=*0.85cm]
		{
				   \Tr{side} %\uppermember {homeSide}
		}
		{
					\Tr{player} %\uppermember {captain}
		}
	    \pstree [levelsep=*0.85cm]
		{
			\Tr{innings} 
		}
		{		  
		    \pstree [levelsep=*0.85cm]
			{
					   \Tr{over} 
			}
			{   
					   \Tr{delivery} 
			}			
		}		
	}	
}
\end{itemize} 
\end{frame}

\begin{frame}{Concepts, Contexts  and Instances}
\begin{itemize}
\item Instances of concepts are to be understood as instances in context.
\item An instance requires an context just as a concept requires a context.
\pause
\item For example 
\begin{itemize}
\onslide<3-> {\item $homeSide$ requires $match$ as a context}
\onslide<4-> {\item $captain$ requires $side$ as a context}
\end{itemize} 
\end{itemize}
\onslide<3->{\begin{displaymath}
\pstree[treemode=\CItreemode,levelsep=*0.65cm,treesep=\CItreesep,nodesep=0.05]
{
    \Tr{\circ}
}
{
    \pstree [levelsep=*0.85cm]
    {
		\Tr{match} 
	}
	{		  
		\pstree [levelsep=*0.85cm]
		{
				   \Tr{side} \member {homeSide}
		}
		{
					\Tr{player} \waitfor{5}{\member {captain}}
		}
	    \pstree [levelsep=*0.85cm]
		{
			\Tr{innings} 
		}
		{		  
		    \pstree [levelsep=*0.85cm]
			{
					   \Tr{over} 
			}
			{   
					   \Tr{delivery} 
			}			
		}		
	}	
}
\end{displaymath}}
\end{frame}




