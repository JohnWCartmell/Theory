\newcommand{\inningsCrossSide}{\crossx{innings\kern-0.3cm}{\kern-0.3cm side}{match}}
\newcommand{\inningsCrossPlayer}{\crossx{innings\kern-0.3cm}{\kern-0.3cm player}{match}}
\newcommand{\fieldingSidePlayer}{fieldingSide ^* (\inningsCrossPlayer)}
\newcommand{\battingSidePlayer}{battingSide ^* (\inningsCrossPlayer)}
\newcommand{\overCrossFieldingSidePlayer}{\crossx{over\kern-0.4cm}{\kern-0.4cm(\fieldingSidePlayer)}{innings}}
\newcommand{\deliveryCrossBattingSidePlayer}{\crossx{delivery\kern-0.4cm}{\kern-0.4cm (\battingSidePlayer)}{innings}}

\begin{frame}{Extending the $\crossx{}{}{}$ Notation}
MAYBE NOT NEEDED
For practical purposes it is useful to extend the $\crossx{}{}{}$ notation.
If $w \base x_1 \base ... \base x_n \base x$
and $w < y$ then define 
$$\crossx{x}{y}{w}=x \cross ( x_n \cross ... (x_1 \cross y)...)$$

Two [play by same playwright.]
\end{frame}

