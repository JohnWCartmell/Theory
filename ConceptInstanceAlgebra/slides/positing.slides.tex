\begin{frame}{Positing $\cross$}
In a concept-instance algebra $A$ the concept $y$ posited in the context of context $x$ is written as $x \cross y$.
If $x_c$, $x$ and $y$ are concepts of $A$ such that $x_c \base x$ and $x_c < y$ then $x \cross y$ is defined and is a concept of $A$,
\pause
\begin{tabular} {c p{0.2cm} c}
(a) if $x_c \base y$ then $x \base x \cross y$   &&   (b) if $y_c \base y$ and $x_c < y_c$then $ x \cross y_c  \base x \cross y$ \\[0.5cm]
$\displaystyle
% I coded it this way with $ and displaystyle rather than with displaymath because uit is the only way that i caould
% find that enabled me to include in a tabular display
\pstree[treemode=\CItreemode,levelsep=*0.65cm,treesep=\CItreesep,nodesep=0.05]
{
	\Tr{x_c}
}
{
   \pstree
	{
	   \Tr{x}
	}
	{
		\onslide<4->{\Tr{x \cross y}}
	}
	\Tr{y}
}
$&&$\displaystyle
% I coded it this way with $ and displaystyle rather than with displaymath because uit is the only way that i caould
% find that enabled me to include in a tabular display
\pstree[treemode=\CItreemode,levelsep=*0.65cm,treesep=\CItreesep,nodesep=0.05]
{
	\Tr{x_c}
}
{
   	\pstree[levelsep=*2.0cm]
	{
	   \Tr{x}
	}
	{
		\pstree[levelsep=*0.75cm]
	   	{
	     	\Tr[edge=\dottededge]{\crossx{x}{y_c}{w}\nudgeup{0.5cm}}  
	     	              %use nudgeup to counter balances the vert space of the subscript w
	   	}
	   	{
			\Tr{\crossx{x}{y}{w}\nudgeup{0.5cm}}
	   	} 
	}
	\pstree[nodesep=0, levelsep=*0.85cm] %latest added levelsep
	{
	   \Tr{\ .}
	}
	{
		\pstree[nodesep=0.05,levelsep=*0.75cm]
	   	{
	     	\Tr[edge=\dottededge]{y_c}
	   	}
	   	{
			\Tr{y}
	   	} 
	}
}
$ \\
\end{tabular}
\end{frame}

\begin{frame} {Positing cont.}
\begin{itemize}
\item If $g \in inst(y)$ in $A$ and $x \cross y$ is defined then $x \cross g$ is defined and $x \cross g \in inst(x \cross y)$.
\medskip
\pause
\item In the cricket example, a side can be posited in the context of an innnings and the batting side is an instance of such a thing:
\begin{displaymath}
\pstree[treemode=\CItreemode,levelsep=*0.65cm,treesep=\CItreesep,nodesep=0.05]
{
    \Tr{\circ}
}
{
    \pstree [levelsep=*0.85cm]
    {
		\Tr{match} 
	}
	{		  
		\Tr{side} 
	    \pstree [levelsep=*0.85cm]
		{
			\Tr{innings} 
		}
		{		  
			\Tr{innings \cross side} \uppermember {battingSide}		
		}		
	}	
}
\end{displaymath}
\end{itemize}
\end{frame}

\begin{frame}{Introduce the bowler}
\begin{displaymath}
\pstree[treemode=\CItreemode,levelsep=*0.65cm,treesep=0.5cm,nodesep=0.05]
%{
%    \Tr{\circ}
%}
%{
    %\pstree [levelsep=*0.85cm]
    {
		\Tr{match} 
	}
	{		  
		%\Tr{side} 
	    \pstree [levelsep=*0.85cm]
		{
			\Tr{innings} 
		}
		{		  
			\pstree [levelsep=*0.85cm]
			{
				\Tr{innings \cross side} \member {fieldingSide}
			}
			{
				\Tr{innings \cross player}
			}
			%\Tr{fieldingSide^*(innings \cross player)}	
			\pstree
			{
             	\Tr{over}
			}
			{
				\Tr{over \cross (fieldingSide^*(innings \cross player))} \uppermember{bowler}
			}	
		}		
	}	
%}
\end{displaymath}
\end{frame}




