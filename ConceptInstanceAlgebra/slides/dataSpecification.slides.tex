% I copied this whole side from cricket presentation
\begin{frame}{Cricket as Data Specification}
\erDisplayFiveSlideAnimation
%\erDisplayAllWithoutAnimation
\newcommand{\logicalAnnotation}
{\begin{tabular}[b]{p{6.5cm}}
Logical or Conceptual\\
\footnotesize  Start here 
\end{tabular}
}
\newcommand{\logicalWithScopesAnnotation}
{\begin{tabular}[b]{p{6.5cm}}
Logical or Conceptual\\
\footnotesize  Add scopes for reference relationships 
\end{tabular}
}
\newcommand{\hierarchicalAnnotation}
{\begin{tabular}[b]{p{6.5cm}}
Physical...Hierarchical (auto-generated)\\
\footnotesize Suitable for representation in XML or IDL.
 fk attrs added to represent horizontal reference (R) relationships.
 In this representation the (D) relationships are implemented by structural containment.
\end{tabular}
}
\newcommand{\relationalAnnotation}
{\begin{tabular}[b]{p{6.5cm}}
Physical...Relational (auto-generated)\\
\footnotesize  For representation in a relational schema.
Further fk attrs added to represent the dependency (D) relationships.
\end{tabular}
}
\begin{tabular}{l l}
\raisebox{-1.9cm}{\scalebox{0.75}{\begin{erdiagram}{11.549999999999999}{9.25375}

\eret{2.9}{-2}{6.9}{-1.1}{0.2}{1}\eretname{3.3}{-1.45}{l}{match}
\erCoreAttribute{3.1}{-1.65}{1}{0}{id}{}
\eret{0.146}{-4.75}{3.774}{-2.95}{0.2}{1}\eretname{0.509}{-3.3}{l}{innings}
\erRelationalAttribute{0.346}{-3.5}{1}{0}{match\textunderscore id}{(D2)}
\erCoreAttribute{0.346}{-3.8}{1}{0}{number}{}
\erHierarchicalAttribute{0.346}{-4.1}{1}{1}{battingSide\textunderscore name}{(R1)}
\erHierarchicalAttribute{0.346}{-4.4}{1}{1}{fieldingSide\textunderscore name}{(R2)}
\eret{6.774}{-4.75}{9.254}{-2.95}{0.2}{1}\eretname{7.022}{-3.3}{l}{side}
\erRelationalAttribute{6.974}{-3.5}{1}{0}{match\textunderscore id}{(D3)}
\erCoreAttribute{6.974}{-3.8}{1}{0}{name}{}
\eret{0.438}{-7.75}{3.483}{-5.95}{0.2}{1}\eretname{0.742}{-6.3}{l}{over}
\erRelationalAttribute{0.638}{-6.5}{1}{0}{match\textunderscore id}{(D4)}
\erRelationalAttribute{0.638}{-6.8}{1}{0}{innings\textunderscore number}{(D4)}
\erCoreAttribute{0.638}{-7.1}{1}{0}{number}{}
\erHierarchicalAttribute{0.638}{-7.4}{1}{1}{bowler\textunderscore number}{(R3)}
\eret{6.81}{-7.75}{9.218}{-5.95}{0.2}{1}\eretname{7.051}{-6.3}{l}{player}
\erRelationalAttribute{7.01}{-6.5}{1}{0}{match\textunderscore id}{(D5)}
\erRelationalAttribute{7.01}{-6.8}{1}{0}{side\textunderscore name}{(D5)}
\erCoreAttribute{7.01}{-7.1}{1}{0}{number}{}
\erCoreAttribute{7.01}{-7.4}{1}{1}{name}{}
\eret{0.119}{-11.05}{3.801}{-8.95}{0.2}{1}\eretname{0.487}{-9.3}{l}{delivery}
\erRelationalAttribute{0.319}{-9.5}{1}{0}{match\textunderscore id}{(D6)}
\erRelationalAttribute{0.319}{-9.8}{1}{0}{innings\textunderscore number}{(D6)}
\erRelationalAttribute{0.319}{-10.1}{1}{0}{over\textunderscore number}{(D6)}
\erCoreAttribute{0.319}{-10.4}{1}{0}{number}{}
\erHierarchicalAttribute{0.319}{-10.7}{1}{1}{facingBatter\textunderscore number}{(R4)}
\eret{0}{-0.2}{9.254}{0.3}{0.2}{1}

% relationship 
\errelname{5.05}{-0.5}{l}{}\errelarm{4.9}{-0.2}{4.9}{-0.65}{1}{0}\errelarm{4.9}{-0.65}{4.9}{-1.1}{1}{0}\ercrowfoot{4.9}{-0.95}{4.75}{-1.1}{4.9}{-1.1}{5.05}{-1.1}{0}
% relationship 
\errelname{4.083}{-2.3}{r}{}\errelarm{4.233}{-2}{4.233}{-2.075}{1}{0}\errelarm{1.96}{-2.738}{1.96}{-2.95}{1}{0}\errelid{2.947}{-2.188}{r}{D2}\errelangle{4.233}{-2.075}{4.233}{-2.15}{3.097}{-2.338}{1}{0}\errelangle{3.097}{-2.338}{1.96}{-2.525}{1.96}{-2.738}{1}{0}\eridcomprel{1.86}{2.06}{-2.7}\ercrowfoot{1.96}{-2.8}{1.81}{-2.95}{1.96}{-2.95}{2.11}{-2.95}{0}
% relationship 
\errelname{5.717}{-2.3}{l}{}\errelarm{5.567}{-2}{5.567}{-2.075}{1}{0}\errelarm{8.014}{-2.738}{8.014}{-2.95}{1}{0}\errelid{6.94}{-2.188}{l}{D3}\errelangle{5.567}{-2.075}{5.567}{-2.15}{6.79}{-2.338}{1}{0}\errelangle{6.79}{-2.338}{8.014}{-2.525}{8.014}{-2.738}{1}{0}\eridcomprel{7.91375}{8.11375}{-2.7}\ercrowfoot{8.014}{-2.8}{7.864}{-2.95}{8.014}{-2.95}{8.164}{-2.95}{0}
% relationship 
\errelname{2.11}{-5.05}{l}{}\errelid{2.11}{-5.2}{l}{D4}\errelarm{1.96}{-4.75}{1.96}{-5.35}{1}{0}\errelarm{1.96}{-5.35}{1.96}{-5.95}{1}{0}\eridcomprel{1.86}{2.06}{-5.699999999999999}\ercrowfoot{1.96}{-5.8}{1.81}{-5.95}{1.96}{-5.95}{2.11}{-5.95}{0}
% relationship battingSide
\errelname{3.924}{-3.25}{l}{battingSide}\errelid{5.424}{-3.25}{l}{R1}\erscope{5.024}{-3.7}{l}{d:..=s:..}\errelarm{3.774}{-3.4}{5.274}{-3.4}{1}{0}\errelarm{5.274}{-3.4}{6.774}{-3.4}{0}{0}\ercrowfoot{3.924}{-3.4}{3.774}{-3.25}{3.774}{-3.4}{3.774}{-3.55}{0}
% relationship fieldingSide
\errelname{3.924}{-4.15}{l}{fieldingSide}\errelid{5.424}{-4.15}{l}{R2}\erscope{5.024}{-4.6}{l}{d:..=s:..}\errelarm{3.774}{-4.3}{5.274}{-4.3}{1}{0}\errelarm{5.274}{-4.3}{6.774}{-4.3}{0}{0}\ercrowfoot{3.924}{-4.3}{3.774}{-4.15}{3.774}{-4.3}{3.774}{-4.45}{0}
% relationship 
\errelname{8.164}{-5.05}{l}{}\errelid{8.164}{-5.2}{l}{D5}\errelarm{8.014}{-4.75}{8.014}{-5.35}{1}{0}\errelarm{8.014}{-5.35}{8.014}{-5.95}{1}{0}\eridcomprel{7.91375}{8.11375}{-5.699999999999999}\ercrowfoot{8.014}{-5.8}{7.864}{-5.95}{8.014}{-5.95}{8.164}{-5.95}{0}
% relationship 
\errelname{2.11}{-8.05}{l}{}\errelid{2.11}{-8.2}{l}{D6}\errelarm{1.96}{-7.75}{1.96}{-8.35}{1}{0}\errelarm{1.96}{-8.35}{1.96}{-8.95}{1}{0}\eridcomprel{1.86}{2.06}{-8.7}\ercrowfoot{1.96}{-8.8}{1.81}{-8.95}{1.96}{-8.95}{2.11}{-8.95}{0}
% relationship bowler
\errelname{3.633}{-6.7}{l}{bowler}\errelid{5.296}{-6.7}{l}{R3}\erscope{4.096}{-7.15}{l}{d:..=s:../fieldingSide}\errelarm{3.483}{-6.85}{5.146}{-6.85}{1}{0}\errelarm{5.146}{-6.85}{6.81}{-6.85}{0}{0}\ercrowfoot{3.633}{-6.85}{3.483}{-6.7}{3.483}{-6.85}{3.483}{-7}{0}
% relationship facingBatter
\errelname{3.951}{-10.3}{l}{facingBatter}\errelarm{3.801}{-10}{4.401}{-10}{1}{0}\errelarm{6.61}{-7.57}{6.81}{-7.57}{0}{0}\errelid{5.556}{-8.635}{r}{R4}\erscope{8.256}{-8.935}{r}{d:..=s:../../battingSide}\errelangle{4.401}{-10}{5.001}{-10}{5.706}{-8.785}{1}{0}\errelangle{5.706}{-8.785}{6.41}{-7.57}{6.61}{-7.57}{0}{0}\ercrowfoot{3.951}{-10}{3.801}{-9.85}{3.801}{-10}{3.801}{-10.15}{0}
\end{erdiagram}
} }
& \kern-3cm
\onslide*<1-1>{\raisebox{-1cm}{\logicalAnnotation}}
\onslide*<2-2>{\raisebox{-1cm}{\logicalWithScopesAnnotation}}
\onslide*<3-3>{\hierarchicalAnnotation}
\onslide*<4-4>{\relationalAnnotation}
\end{tabular}
\end{frame}