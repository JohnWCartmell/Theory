
\iffalse
\begin{frame}{Concept, Context, Instance (how to think)}
All that there is...
\begin{itemize}
\item ...\textbf{is} in context.
\item ... \textbf{is} concept and instance of concept.
\end{itemize}
Only...
\begin{itemize}
\item ... concept provides context.
\end{itemize}
There is...
\begin{itemize}
\item a root concept that represents the absolute or the whole.
\end{itemize}
and to every concept $x$ there are...
\begin{itemize}
\item ...other concepts and their instances for which $x$ provides context.
\end{itemize}
\iffalse{
By the way...
\begin{itemize}
\item a concept whose context is provided by the absolute can be said to be an absolute concept.  
All of its instances can also be said to be absolute.
\end{itemize}
}\fi
\end{frame}

\begin{frame}{Concepts with a dependency on context}
\begin{center}
\raisebox{-0.5cm}{
\pspicture(0,-0.1)(1.1,1)
\psline(0,0)(0,1)(1,1)(1,0)(0,0)
\psline (0,0)(1,1)
%\psline(0,0.5)(1,0.5)
%\psline(0.5,0)(0.5,1)
\endpspicture
}
\end{center}
\begin{itemize}
\item \textit{"there are two triangles and these have six sides"}, 
\item  we are understanding \textit{side} to be a concept that 
depends on \textit{triangle} for context,
%\item a side, therefore, is a dependent type of thing
% -- it is some thing to be held in the mind
%in the context of some other thing,
\item write $$triangle \base side\ \ \ \ \ \ \ $$
\end{itemize}
\end{frame}

\begin{frame}{Opposite Side}

\textit{"in the context of an angle of a triangle, the opposite side is a side of the same triangle"}

\medskip
Write 
$$oppositeSide \in inst(angle \cross side)$$
because
\begin{itemize}
\item $angle \cross side$ represents a $side$ in the context of an $angle$ of the same $triangle$
\end{itemize}
if we assume
$$triangle \base angle$$
 and 
$$triangle \base side$$
so that
$$angle \base (angle \cross side)$$
\end{frame}
\fi
