

\begin{frame}{Concept, Context, Instance}
All that there is...
\begin{itemize}
\item ...is in context.
\item ... is concept and instance of concept.
\end{itemize}
There is...
\begin{itemize}
\item a root concept that represents the absolute or the whole.
\end{itemize}
To every concept $x$ there are...
\begin{itemize}
\item ...other concepts and their instances for which $x$ provides context.
\end{itemize}
Further...
\begin{itemize}
\item a concept whose context is provided by the absolute can be said to be an absolute concept.  
All of its instances can also be said to be absolute.
\end{itemize}
\end{frame}

\begin{frame}{Concepts with a dependency on context}
\begin{center}
\raisebox{-0.5cm}{
\pspicture(0,-0.1)(1.1,1)
\psline(0,0)(0,1)(1,1)(1,0)(0,0)
\psline (0,0)(1,1)
%\psline(0,0.5)(1,0.5)
%\psline(0.5,0)(0.5,1)
\endpspicture
}
\end{center}
\begin{itemize}
\item \textit{"there are two triangles and these have six sides"}, 
\item  we are understanding \textit{side} to be a concept that 
depends on \textit{triangle} for context,
\item a side, therefore, is a dependent type of thing -- it is some thing to be held in the mind
in the context of some other thing,
\item I summarise this by writing $triangle \base side$.
\end{itemize}
\end{frame}

\begin{frame}{Opposite Side}
\iffalse
{\begin{displaymath}
\begin{array}{c p{0.5cm} c  c} 
                &&\rnode{Bp}{} &                   \\
\rnode{A}{}&                   &                   \\[0.25cm]
                &&             & \rnode{Cp}{} \\
\end{array}
\ncline[nodesep=0pt]{A}{Bp}
\ncput{\rnode{B}{}}
\ncline[nodesep=0pt]{A}{Cp}
\ncput{\rnode{C}{}}
\ncline[nodesep=0pt]{Bp}{Cp}
\ncline[nodesep=0pt]{B}{C}
%labels
\nput[labelsep=1pt]{180}{A}{A}
\nput[labelsep=2pt]{80}{Bp}{B'}
\nput[labelsep=1pt]{290}{Cp}{C'}
\nput[labelsep=1pt]{120}{B}{B}
\nput[labelsep=2pt]{260}{C}{C}
\end{displaymath}
}
\fi

\textit{"in the context of an angle of a triangle, the opposite side is an instance of a side of the same triangle"}
\medskip
In the algebra is written as
$$oppositeSide \in inst(angle \cross side)$$
because
\begin{itemize}
\item $angle \cross side$ represents a $side$ in the context of an $angle$ of the same $triangle$
\end{itemize}
provided we assume that
\begin{itemize}
\item $triangle \base angle$ and $triangle \base side$.
\end{itemize}
With these same assumptions, in the algebra, we have that
\begin{itemize}
\item $angle \base (angle \cross side)$.
\end{itemize}
\end{frame}


\begin{frame}
\begin{equation}
\label{examplesentence}
\mbox{\textit{I \underline{can} carry the \underline{can}.}}
\end{equation}
If (\ref{examplesentence}) is given as an example of a sentence having a word that is both a verb and a noun then
\medskip
\begin{itemize}
\item \textit{noun} and \textit{verb} as types of thing are dependent on (grammatical) \textit{sentence}s for context
\item what is a (grammatical) \textit{sentence} depends on \textit{language} as context.
\end{itemize}
There is a tree of contexts and concepts which we show like this
\begin{displaymath}
\pstree[treemode=\CItreemode,levelsep=*0.65cm,treesep=\CItreesep,nodesep=0.05]
{
    \Tr{\circ}
}
{
    \pstree [levelsep=*0.85cm]
    {
		\Tr{language} 
	}
	{		  
		\pstree [levelsep=*0.85cm]
		{
				   \Tr{sentence} 
		}
		{
					\Tr{noun}
					\Tr{verb}
					\Tr{adjective} 
		}	
	}	
}
\end{displaymath}
\end{frame}
