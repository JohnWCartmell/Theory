\begin{frame}{Particularisation $f^*y$ of a concept $y$}
In a concept-instance algebra $A$
\begin{itemize}
\item concepts and their instances can be particularised. The particularisation of concept $y$ to instance $f$ is written as $f^*y$. 
\item If $f$ is an instance of $x$ and $x < y$ then $f^*y$ is defined and assuming $x_c \base x$ then 
\end{itemize}
\medskip
\pause
\begin{tabular} {c p{0.2cm} c}
\onslide<2->{(a) if $x \base y$ then $x_c \base f^*y$}   
&&   
\onslide<5->{(b) if $y_c \base y$ and $x <y_c$ then $f^*y_c \base f^*y$} \\
\onslide<3->{$\displaystyle
% I coded it this way with $ and displaystyle rather than with displaymath because uit is the only way that i caould
% find that enabled me to include in a tabular display
\pstree[treemode=\CItreemode,levelsep=*0.65cm,treesep=0.8cm,nodesep=0.05]
{
	\Tr{x_c}
}
{
   \pstree
	{
	   \Tr{x}\member{f}
	}
	{
		\Tr{y} 
	}
	\onslide<4->{\Tr{f \sub y}}
}
$}
&&
\onslide<6->{$\displaystyle
% I coded it this way with $ and displaystyle rather than with displaymath because uit is the only way that i caould
% find that enabled me to include in a tabular display
\pstree[treemode=\CItreemode,levelsep=*0.65cm,treesep=\CItreesep,nodesep=0.05]
{
	\Tr{x}
}
{
   	\pstree[levelsep=*2.0cm]
	{
	   \Tr{y}\member{f}
	}
	{
		\pstree[levelsep=*0.75cm]
	   	{
	     	\Tr[edge=\dottededge]{z_c}
	   	}
	   	{
			\Tr{z}
	   	} 
	}
	\pstree[nodesep=0, levelsep=*0.85cm] %latest added levelsep
	{
	   \Tr{\ .}
	}
	{
		\pstree[nodesep=0.05,levelsep=*0.75cm]
	   	{
	     	\Tr[edge=\dottededge]{f^*z_c}
	   	}
	   	{
			\Tr{f^*z}
	   	} 
	}
}
$} \\
\end{tabular}
\end{frame}

\begin{frame}{Particularisation $f^*g$ of an instance $g$}
\begin{itemize}
\item If $f \in inst(x)$ and  $g \in inst(y)$
then if $f^*y$ is defined, i.e. if $x < y$, then $f^*g$ is defined and $f^*g \in inst(f^*y)$.
\medskip
\pause
\item In the cricket example 
\vspace{0.6cm}
\begin{displaymath}
\pstree[treemode=R,levelsep=*0.65cm,treesep=1cm,nodesep=0.05]
{
    \Tr{\circ}
}
{
    \pstree [levelsep=*0.85cm]
    {
		\Tr{match} 
	}
	{		  
		\pstree [levelsep=*0.85cm]
		{
				   \Tr{side} \uppermember {home}
		}
		{
					\Tr{player} \uppermember {captain}
		}
	    \Tr{home^*player\kern-1cm} \member{home^*captain}		
	}	
}
\end{displaymath}
\end{itemize}
\end{frame}

\iffalse DONT USE THESE SLIDES
\begin{frame}{Particularisation cont. }
At this point it follows that if $y \base z_1 ... \base z_m \base z$ and $g$ is a instance of $z$ then we have:
\begin{displaymath}
\pstree[treemode=\CItreemode,levelsep=*0.65cm,treesep=\CItreesep,nodesep=0.05]
 {
    %\Tr{\circ}
		\Tr{x}
 }
 {%\flexbranch{Lxn}{1cm}{1cm}{x}{n}{}
    {\pstree
		   {\Tr{y}\member{f}}
			 {
			 \flexbranchplusarc{Ly}{1cm}{1cm}{z}{m}{}{g}
			 }
		 \flexbranchplusarc{Lfy}{1cm}{1.3cm}{f \sub z}{m}{} {f \sub g}
		}
}
\end{displaymath}
\vspace{0.5cm} 
\end{frame}


\begin{frame}{Example} DONT USE THIS
\vspace{0.6cm}
\begin{displaymath}
\pstree[treemode=R,levelsep=*0.65cm,treesep=1cm,nodesep=0.05]
{
    \Tr{\circ}
}
{
    \pstree [levelsep=*0.85cm]
    {
		\Tr{match} 
	}
	{		  
		\pstree [levelsep=*0.85cm]
		{
				   \Tr{side} \uppermember {home}
		}
		{
					\Tr{player} \uppermember {captain}
		}
	    \Tr{home^*player\kern-1cm} \member{home^*captain}		
	}	
}
\end{displaymath}
\end{frame}
\fi






