\documentclass[10pt,a4paper]{article}
% 08/02/2019 Maintenance -- use of \obj macro changed to use of \Ob.
\usepackage[margin=3cm]{geometry}
\usepackage{pstricks}
\usepackage{pst-node}
\usepackage{pst-tree}
\usepackage{stmaryrd}
\usepackage{amsmath}
\usepackage{amssymb}
\usepackage{verbatim}
\usepackage{enumerate}
\usepackage{calc}
\usepackage{float} % for H option for tables and figures
\usepackage{changepage} % used for adjustwidth

%\usepackage{enumitem} % to be able to [resume] an enumeration but clashes with [(i)] !!
\usepackage[margin=4.0cm]{geometry} %was 3cm
\usepackage{mathptmx}
\usepackage{amsfonts}
\usepackage{array}
\usepackage{pstricks}
\usepackage{pst-tree}
\usepackage{pst-plot}
\usepackage{pst-node}
\usepackage{stmaryrd}
\usepackage{amsmath}
\usepackage{verbatim}
\usepackage{graphicx}  
\usepackage{calc}
\usepackage{xifthen}
\usepackage{xcolor}
\usepackage{color}
\usepackage{stringstrings}
%\usepackage[small,bf,margin=3pt,format=hang, labelsep=endash,singlelinecheck=false]{caption} %prevuiously justification=justified
%\usepackage{enumerate}
%\usepackage{enumitem}
\usepackage{enumerate}
\usepackage[shortlabels]{enumitem}
\usepackage{float}
\usepackage[section]{placeins}
%\setlength{\captionmargin}{5pt}
\usepackage{environ}
\usepackage{multirow}
\usepackage{rotating}
\usepackage{longtable}
\usepackage{afterpage}
\usepackage{needspace}


%DEFINE ENVIRONMENT BLOCK
% Riddle
\newsavebox{\riddlebox}

\newenvironment{erexample}
{\newcommand\colboxcolor{F0F0F0}%was F8F8F8
\begin{lrbox}{\riddlebox}
\begin{minipage}{\dimexpr\columnwidth-2\fboxsep\relax} \textbf{} \\ \itshape}
{\end{minipage}\end{lrbox}%
%\begin{center}
\colorbox[HTML]{\colboxcolor}{\usebox{\riddlebox}}
%\end{center}
}

\newenvironment{erbox}
{\newcommand\colboxcolor{F0F0F0}%was F8F8F8
\begin{lrbox}{\riddlebox}%
\begin{minipage}{\dimexpr\columnwidth-2\fboxsep\relax} }
{\end{minipage}\end{lrbox}%
%\begin{center}
\colorbox[HTML]{\colboxcolor}{\usebox{\riddlebox}}
%\end{center}
}

%\begin{erboxedFigure}{#1 FigureParam}{#2 Label}{#3 Caption}
\NewEnviron{erboxedFigure}[3]{%
\begin{figure}[#1]
\begin{erexample}
\begin{center}
\BODY
\end{center}
\vspace{-0.5cm}
\caption{#3}
\label{#2}
\end{erexample}
\end{figure}
}

\newcommand{\erpictureFolder}[0]{../SharedPictures}

\newcommand{\ercenterPicture}[1]{
\begin{center}
\input{\erpictureFolder/#1}
\end{center}
}


\newlength{\erhalfHt}

%\erinlinePicture{#1 pictureFilename}{#2 pictureHeight}
\newcommand{\erinlinePicture}[2]{
\setlength{\erhalfHt}{#2cm * \real{0.5}}
\raisebox{-\erhalfHt}[\erhalfHt + 0.5cm][\erhalfHt + 0.5cm]{
\input{\erpictureFolder/#1}
} 
}

%\erplainFig{#1 pictureFilename}{#2 figureParam}{#3Caption}
\newcommand{\erplainFig}[3]{
\begin{figure}[#2]
\begin{center}
\input{\erpictureFolder/#1}
\end{center}
\caption{#3}
\label{#1}
\end{figure}
}

%\erboxedFigPicture{#1 pictureFilename}{#2 figureParam}{#3Caption}
\newcommand{\erboxedFigPicture}[3]{
\begin{figure}[#2]
\begin{erexample}
\vspace{-0.5cm}
\begin{center}
\input{\erpictureFolder/#1}
\end{center}
\caption{#3}
\label{#1}
\end{erexample}
\end{figure}
}

%\erLeftSideFig{#1 pictureFilename}{#2 figureParam}{#3Caption}
\newcommand{\erLeftSideFig}[3]{
\begin{figure}[#2]
\begin{erexample}
  \begin{minipage}[c]{0.4\textwidth}
    \caption{#3}
    \label{#1}
  \end{minipage}
  \begin{minipage}[c]{0.5\textwidth}
    \input{\erpictureFolder/#1}
  \end{minipage}
\end{erexample}
\end{figure}
}

%\erbulletedFig{#1 pictureFilename}{#2 figureParam}{#3Caption}
\NewEnviron{erbulletedFig}[3]{%
\begin{figure}[#2]
\begin{erexample}
\vspace{-0.5cm}
\begin{center}
$
\begin{array}{c m{0.25cm} | m{6cm}}
\raisebox{-2.0cm}{
\input{\erpictureFolder/#1}}& & \text{\parbox{6cm}{\raggedright{\footnotesize{
\begin{enumerate}[(i)]
\BODY
\end{enumerate}}}}} \\
\end{array}
$
\end{center}
\caption{#3}
\label{#1}
\end{erexample}
\end{figure} 
}


%\begin{erbulletedDimFig}{#1 pictureFilename}{#2figureParam} {#3Caption} {#4PictureHeight}{#5TextWidth}

\NewEnviron{erbulletedDimFig}[5]{%
\begin{figure}[#2]
\begin{erexample}
\vspace{-0.5cm}
\begin{center}
$
\begin{array}{c m{0.25cm} |  m{#5cm}}
\setlength{\erhalfHt}{#4cm * \real{0.5}}
\raisebox{-\erhalfHt}{
\input{\erpictureFolder/#1}}& & \text{\parbox{#5cm}{\raggedright{\footnotesize{
\begin{enumerate}[(i)]
\BODY
\end{enumerate}}}}} \\
\end{array}
$
\end{center}
\caption{#3}
\label{#1}
\end{erexample}
\end{figure} 
}

%\begin{ernotedModel}{#1 pictureFilename}{#2PictureHeight}{#3PictureWidth}{#4TextWidth}

\NewEnviron{ernotedModel}[4]{%
\begin{center}
$
\begin{array}{m{#3cm} m{1cm} | c m{#4cm}}
\setlength{\erhalfHt}{#2cm * \real{0.5}}
\raisebox{-\erhalfHt}{
\input{\erpictureFolder/#1}}& & & \text{\parbox{#4cm}{\raggedright{\footnotesize{
\BODY
}}}} \\
\end{array}
$
\end{center} 
}

%\begin{ermodelText}{#1 pictureFilename}{#2PictureHeight}{#3PictureWidth}{#4TextWidth}

\NewEnviron{ermodelText}[4]{%
\begin{center}
\begin{tabular}{m{#3cm} m{1cm}  c m{#4cm}}
\setlength{\erhalfHt}{#2cm * \real{0.5}}
\raisebox{-\erhalfHt}{
\input{\erpictureFolder/#1}}& & & \text{\parbox{#4cm}{\raggedright{\small{
\BODY
}}}} \\
\end{tabular}
\end{center} 
}


%\erbulletedModel{#1 pictureFilename}{#2PictureHeight}{#3PictureWidth}{#4TextWidth}

\NewEnviron{erbulletedModel}[4]{%
\begin{center}
$
\begin{array}{m{#3cm} m{1cm} | c m{#4cm}}
\setlength{\erhalfHt}{2cm * \real{0.5}}
\raisebox{-\erhalfHt}{
\input{\erpictureFolder/#1}}& & & \text{\parbox{#4cm}{\raggedright{\footnotesize{
\begin{enumerate}[(i)]
\BODY
\end{enumerate}}}}} \\
\end{array}
$
\end{center} 
}



%\ernotedDimFig{#1 pictureFilename}{#2 figureParam}{#3Caption}{#4PictureHeight}{#5TextWidth}
\NewEnviron{ernotedDimFig}[5]{%
\begin{figure}[#2]
\begin{erexample}
\vspace{-0.5cm}
\begin{center}
$
\begin{array}{c m{0.25cm} | c m{#5cm}}
\setlength{\erhalfHt}{#4cm * \real{0.5}}
\raisebox{-\erhalfHt}{
\input{\erpictureFolder/#1}}& & & \text{\parbox{#5cm}{\raggedright{\footnotesize{
\BODY }}}}\\
\end{array}
$
\end{center}
\caption{#3}
\label{#1}
\end{erexample}
\end{figure} 
}
%\begin{ernotedDimFigPW}{#1 pictureFilename}{#2 figureParam}{#3Caption}{#4PictureHeight}{#5PictureWidth}{#6TextWidth}
\NewEnviron{ernotedDimFigPW}[6]{%
\begin{figure}[#2]
\begin{erexample}
\vspace{-0.5cm}
\begin{center}
$
\begin{array}{>{\centering}m{#5cm} m{0.5cm} | c m{#6cm}}
\setlength{\erhalfHt}{#4cm * \real{0.5}}
\raisebox{-\erhalfHt}{
\centering \input{\erpictureFolder/#1}
}& & & \text{\parbox{#6cm - 0.5cm}{\raggedright{\footnotesize{
\BODY }}}}\\
\end{array}
$ \\
\vspace {0.2cm}
\end{center}
\caption{#3}
\label{#1}
\end{erexample}
\end{figure}
}



\newenvironment{erquote}
{\begin{quote}\itshape}
{\end{quote}}



%ccategories.macros.tex 

% Macros for diagrams in contextual categories and related categories

\usepackage{twoopt}
\usepackage{scalerel} 
\usepackage{xargs}

%\usepackage{mathabx}  %Caused font problems
%\usepackage{MnSymbol}  % caused font problems

\newcommand{\conu}
{\mathbf{C}(U)}

\newcommand{\depu}
{\mathbf{D}(U)}

\newcommand{\cat}[1]{\textbf{#1}}
\newcommand{\obj}[1]{\ensuremath{|\cat{#1}|}}
\newcommand{\ccat}[1][C]{\ensuremath{\mathbb{#1}} }
\newcommand{\ccatc}{contextual category \ccat}
\newcommand{\cobj}[2][]{\ensuremath{|\ccat[#2]|_{#1}}}
\newcommand{\cslice}[2]{\ensuremath{\ccat[#1]_{#2}}}
\newcommand{\csliceobj}[3][]{\ensuremath{|\mathbb{#2}_{#3}|_{#1} }}
\newcommand{\varset}[1][]{\ensuremath{V_{#1} }}
\newcommand{\localvarsets}{\ensuremath{\mathcal{V} }}
\newcommand{\Fam}{\ensuremath{\mathbb{F\mathrm{am}} }}
\newcommand{\Famslice}[1]{\ensuremath{\mathbb{F\mathrm{am}}_{#1} }}
\newcommand{\Famobj}[1][]{\ensuremath{|\mathbb{F\mathrm{am}}|_{#1} }}
\newcommand{\Famsliceobj}[2][]{\ensuremath{|\mathbb{F\mathrm{am}}_{#2}|_{#1} }}
\newcommand{\morph}{\rightarrow}
\newcommand{\epi}{\twoheadrightarrow}
\newcommand{\base}{\triangleleft}
\newcommand{\comp}{\circ}
\newcommand{\cross}{\otimes}
\newcommand{\pc}[2]{d^{#1}_{#2}}
\newcommand{\sub}{^*}
\newcommand{\diag}{\delta}
\newcommand{\pbase}[1]{\tilde{#1}}

\newcommand{\tuple}[1]{\langle#1\rangle}
\newcommand{\ndidly}{\ensuremath{\Join_n}}
\newcommand{\ndidlycospan}{quotiented n-cospan}

\newcommand{\crossx}[3]{#1 \underset{#3}{\cross} #2}
\newcommand{\fibrex}[3]{#1 \underset{#3}{\Join} #2}
\newcommand{\powerset}{\mathcal{P}}
\newcommand{\primeds}[1]{
\ensuremath{\mathcal{P}(#1)} }
\newcommand{\compset}{\ \dot{\circ}\, }

% darrow
%\newcommand{\darrow}{\rightarrowtriangle} %use \smorph instead
\newcommand{\smorph}{\rightarrowtriangle}

 

\newcommand\dhead{\scaleobj{0.6}{\triangleright}}
\newcommand{\dmorph}{\, \mbox{---} \! \cdot \! \raisebox{1.1pt}{\dhead}}

% projection tree
%\newcommand{\proj}[2]{proj_{#2}(#1)}

\newcommand{\proj}[2]{
\ensuremath{\mathcal{P}_{#2}(#1)} }

%pstrick supplements for arrows

\newlength{\arrnodesepA}
\newlength{\arrnodesepB}
\newlength{\arroffsetA}
\newlength{\arroffsetB}

%Modified to 2pt from 0pt on 23 July 2018
\newcommand{\arreset}{
\setlength{\arrnodesepA}{2pt}
\setlength{\arrnodesepB}{2pt}
\setlength{\arroffsetA}{0pt}
\setlength{\arroffsetB}{0pt}
}
\arreset

\newcommand{\ncarr}[3][0]{\ncarc[arcangle=#1,nodesepA=\arrnodesepA,nodesepB=\arrnodesepB,offsetA=\arroffsetA,offsetB=\arroffsetB,arrowsize=5pt,arrowinset=0.7]{->}{#2}{#3}}
\newcommand{\jcbarr}[4][0]{ % ncbarr is defined in some thridy party package so do not use!\emph{}
\ncarr[#1]{#3}{#4}
\nbput[labelsep=2pt]{\footnotesize $#2$}
}

\newcommand{\ncaarr}[4][0]{
\ncarr[#1]{#3}{#4}
\naput[labelsep=2pt]{\footnotesize $#2$}
}

% \alabel{label}[npos][labelsep_pts]
\newcommandx*\alabel[3][2=0.5,3=2,usedefault]{\naput[labelsep=#3pt,npos=#2]{\footnotesize $#1$}}
% \blabel{label}[npos][labelsep_pts]
\newcommandx*\blabel[3][2=0.5,3=2,usedefault]{\nbput[labelsep=#3pt,npos=#2]{\footnotesize $#1$}}

% \idcomp mark an arrow as one component of an identifier
\newcommand{\idcomp}{\ncput[npos=0, nrot=:U]{\psline(0.2,-0.075)(0.2,0.075)}}  %add a bar to a node connection arrow
% pstrick supplements for s-arrows (previous name for d-arrow - should convert}

\newlength{\sarnodesepA}
\newlength{\sarnodesepB}
\newlength{\saroffsetA}
\newlength{\saroffsetB}
\newlength{\sarnodesepAsav}
\newlength{\sarnodesepBsav}

\newcommand{\sarreset}{
\setlength{\sarnodesepA}{0pt}
\setlength{\sarnodesepB}{0pt}
\setlength{\saroffsetA}{0pt}
\setlength{\saroffsetB}{0pt}
}

\sarreset

% sar - S-arrow
\newcommand{\ncsar}[3][0]{
\setlength{\sarnodesepAsav}{\sarnodesepA}
\setlength{\sarnodesepBsav}{\sarnodesepB}
\addtolength{\sarnodesepA}{3pt}
\addtolength{\sarnodesepB}{7pt}
\ncarc[nodesepA=\sarnodesepA,nodesepB=\sarnodesepB,offsetA=\saroffsetA,offsetB=\saroffsetB,arcangle=#1]{-}{#2}{#3}
\ncput[nrot=:R,npos=1]{\pstriangle(0,0)(.2,.2)}
\setlength{\sarnodesepA}{\sarnodesepAsav}
\setlength{\sarnodesepB}{\sarnodesepBsav}
}


% bsar - below labelled S-arrow
\newcommand{\ncbsar}[4][0]{
\ncsar[#1]{#3}{#4}
\nbput[labelsep=2pt]{\footnotesize $#2$}
}
% asar - above labelled S-arrow
\newcommand{\ncasar}[4][0]{
\ncsar[#1]{#3}{#4}
\naput[labelsep=2pt]{\footnotesize $#2$}
}

% cdar - composite dependency arrow
\newcommand{\nccdar}[3][0]{
\setlength{\sarnodesepAsav}{\sarnodesepA}
\setlength{\sarnodesepBsav}{\sarnodesepB}
\addtolength{\sarnodesepA}{3pt}
\addtolength{\sarnodesepB}{11pt}
\ncarc[nodesepA=\sarnodesepA,nodesepB=\sarnodesepB,offsetA=\saroffsetA,offsetB=\saroffsetB,arcangle=#1]{-}{#2}{#3}
\ncput[nrot=:R,npos=1]{\pstriangle(0,0.1)(.2,.2)}
\ncput[nrot=:R,npos=1]{\psdot[dotsize=1pt](-0.0075,0.05)}   %!!
\setlength{\sarnodesepA}{\sarnodesepAsav}
\setlength{\sarnodesepB}{\sarnodesepBsav}
}


% bcdar - below labelled composite dependency arrow
\newcommand{\ncbcdar}[4][0]{
\nccdar[#1]{#3}{#4}
\nbput[labelsep=2pt]{\footnotesize $#2$}
}
% acdar - above labelled composite dependency arrow
\newcommand{\ncacdar}[4][0]{
\nccdar[#1]{#3}{#4}
\naput[labelsep=2pt]{\footnotesize $#2$}
}


% rsar - recursive S-arrow
\newcommand{\ncrsar}[2]{
\setlength{\sarnodesepAsav}{\sarnodesepA}
\setlength{\sarnodesepBsav}{\sarnodesepB}
\addtolength{\sarnodesepA}{3pt}
\addtolength{\sarnodesepB}{7pt}
\ncloop[nodesepA=\sarnodesepA,nodesepB=\sarnodesepB,
        offsetA=\saroffsetA,offsetB=\saroffsetB,
        armA=0.7cm,armB=0.6cm,angleA=90,angleB=-90,loopsize=-1,linearc=0.4
				]{-}{#1}{#2}
\ncput[nrot=:R,npos=5]{\pstriangle(0,0)(.2,.2)}
\setlength{\sarnodesepA}{\sarnodesepAsav}
\setlength{\sarnodesepB}{\sarnodesepBsav}
}

% pstrick supplements for multi-arrows

\newlength{\marnodesepA}
\newlength{\marnodesepB}
\newlength{\maroffsetB}
\newlength{\marnodesepBsav}

\newcommand{\marreset}{
\setlength{\marnodesepA}{0pt}
\setlength{\marnodesepB}{0pt}
\setlength{\maroffsetB}{0pt}
}

\marreset

%ncmarr[#1 arcangle1][#2 arcangle2]{#3 name}{#4 domain1}{#5 domain2}{#6 junction}{#7 codomain}
\newcommandtwoopt{\ncmarr}[6][8][8]{%
\ncarc[nodesepA=\marnodesepA,nodesepB=0,arcangle=#1]{-}{#3}{#5}
\ncarc[nodesepB=0,arcangle=-#1]{-}{#4}{#5}
\ncarc[arcangle=#2,nodesepB=\marnodesepB,offsetB=\maroffsetB]{->}{#5}{#6}
}%


\newcommandtwoopt{\nchmarr}[6][8][8]{%
\ncarc[nodesepA=\marnodesepA,nodesepB=0,arcangle=#1]{-}{#3}{#5}
\ncarc[nodesepB=0,arcangle=#1]{-}{#4}{#5}
\ncarc[arcangle=#2,nodesepB=\marnodesepB,offsetB=\maroffsetB]{->}{#5}{#6}
}%

\newcommandtwoopt{\ncamarr}[7][8][8]{%
\ncmarr[#1][#2]{#4}{#5}{#6}{#7}
\naput[npos=.05]{$#3$}
}%
\newcommandtwoopt{\ncbmarr}[7][8][8]{%
\ncmarr[#1][#2]{#4}{#5}{#6}{#7}
\nbput[npos=.05]{$#3$}
}%

\newcommandtwoopt{\ncbhmarr}[7][8][8]{%
\nchmarr[#1][#2]{#4}{#5}{#6}{#7}
\nbput[npos=.05]{$#3$}
}%

\newcommandtwoopt{\ncmarrr}[7][8][8]{
\ncarc[nodesepB=0,arcangle=#1]{-}{#3}{#6}
\ncline[nodesepB=0]{-}{#4}{#6}
\ncarc[nodesepB=0,arcangle=-#1]{-}{#5}{#6}
\ncarc[nodesepA=0,arcangle=#2]{->}{#6}{#7}
}

\newcommandtwoopt{\ncamarrr}[8][8][8]{
\ncmarrr[#1][#2]{#4}{#5}{#6}{#7}{#8}
\naput[npos=.05]{$#3$}
}
\newcommandtwoopt{\ncbmarrr}[8][8][8]{
\ncmarrr[#1][#2]{#4}{#5}{#6}{#7}{#8}
\nbput[npos=.05]{$#3$}
}

%gats.macros.tex

\usepackage{environ}    % also used in ermacros % here used for \NewEnvrion

\newcommand{\gat}[1][U]{
\ensuremath{\mathcal{#1}}}  % used to hav a space in here
\newcommand{\gatw}[1][U]{\gat[#1]\ }  % use this if need trailing space
\newcommand{\ingat}[1][U]{in \gat[#1]}
\newcommand{\isagat}[1][U]{\gat[#1] is a g.a.t.}
\newcommand{\inagat}{in a g.a.t. }

% macro for a generic theory
%\newcommand{\theory}
%{\textit{U}}

\newcommand{\intheory}
{is a derived rule of \gat[U]}

% Macros for GAT rules

\newcommand{\isT}[1]
{#1\mbox{ is a type}}

\newcommand{\ofT}[2]
{#1 \in #2
}

% Macros for GAT rules   <!-- new old -->
\newcommand{\istype}[1]
{#1\mbox{ is a type}}

\newcommand{\oftype}[2]
{#1 \in #2
}

%\context{x}{\Delta}{n}
\newcommand{\context}[3]
{\ofT{#1_1}{#2_1},... \ofT{#1_{#3}}{#2_{#3}(#1_1,...#1_{#3-1})}
}

%\subcontext{x}{\Delta}{i}{k}
\newcommand{\subcontext}[4]
{\ofT{#1_{#3_1}}{#2_{#3_1}},... \ofT{#1_{#3_#4}}{#2_{#3_#4}(#1_1,...#1_{#3_#4-1})}
}

% #schematic context
\newcommand{\schmcon}[3]
{\ofT{#1_1}{#2_1},... \ofT{#1_{#3}}{#2_{#3}}
}
% abbreviated to
\newcommand{\con}[3]
{\schmcon{#1}{#2}{#3}}

% schematic subcontext
%\subcon{x}{\Delta}{i}{k}
\newcommand{\subcon}[4]
{\ofT{#1_{#3_1}}{#2_{#3_1}},... \ofT{#1_{#3_#4}}{#2_{#3_#4}}
}

% permuted context
%\permcon{x}{\Delta}{n}{\sigma}
\newcommand{\permcon}[4]
{\ofT{#1_{#4(1)}}{#2_{#4(1)}},... \ofT{#1_{#4(#3)}}{#2_{#4(#3)}}
}
% permuted term
%\permterm{t}{n}{\sigma}
\newcommand{\permterm}[3]
{
#1_{#3(1)},...#1_{#3(#2)}
}


% Idioms
\newcommand{\xDelta}[1]{\con{x}{\Delta}{#1}}
\newcommand{\xDeltap}[1]{\con{x}{\Delta'}{#1}}
\newcommand{\xOmega}[1]{\con{x}{\Omega}{#1}}
\newcommand{\xOmegap}[1]{\con{x}{\Omega'}{#1}}
\newcommand{\yOmega}[1]{\con{y}{\Omega}{#1}}
\newcommand{\yOmegap}[1]{\con{y}{\Omega'}{#1}}

\newcommand{\xDeltasigma}[1]{\permcon{x}{\Delta}{#1}{\sigma}}
\newcommand{\xDeltapsigma}[1]{\permcon{x}{\Delta'}{#1}{\sigma}}
\newcommand{\xOmegasigma}[1]{\permcon{x}{\Omega}{#1}{\sigma}}
\newcommand{\xOmegapsigma}[1]{\permcon{x}{\Omega'}{#1}{\sigma}}
\newcommand{\yOmegasigma}[1]{\permcon{y}{\Omega}{#1}{\sigma}}
\newcommand{\yOmegapsigma}[1]{\permcon{y}{\Omega'}{#1}{\sigma}}

\newcommand{\xDeltainvsigma}[1]{\permcon{x}{\Delta}{#1}{\sigma^{-1}}}
\newcommand{\xDeltapinvsigma}[1]{\permcon{x}{\Delta'}{#1}{\sigma^{-1}}}
\newcommand{\xOmegainvsigma}[1]{\permcon{x}{\Omega}{#1}{\sigma^{-1}}}
\newcommand{\xOmegapinvsigma}[1]{\permcon{x}{\Omega'}{#1}{\sigma^{-1}}}
\newcommand{\yOmegainvsigma}[1]{\permcon{y}{\Omega}{#1}{\sigma^{-1}}}
\newcommand{\yOmegapinvsigma}[1]{\permcon{y}{\Omega'}{#1}{\sigma^{-1}}}

%Idioms enclosed as tuples
\newcommand{\encxDelta}[1]{\tuple{\con{x}{\Delta}{#1}}}
\newcommand{\encxDeltap}[1]{\tuple{\con{x}{\Delta'}{#1}}}
\newcommand{\encxOmega}[1]{\tuple{\con{x}{\Omega}{#1}}}
\newcommand{\encxOmegap}[1]{\tuple{\con{x}{\Omega'}{#1}}}
\newcommand{\encyOmega}[1]{\tuple{\con{y}{\Omega}{#1}}}
\newcommand{\encyOmegap}[1]{\tuple{\con{y}{\Omega'}{#1}}}

\newcommand{\encxDeltasigma}[1]{\tuple{\permcon{x}{\Delta}{#1}{\sigma}}}
\newcommand{\encxDeltapsigma}[1]{\tuple{\permcon{x}{\Delta'}{#1}{\sigma}}}
\newcommand{\encxOmegasigma}[1]{\tuple{\permcon{x}{\Omega}{#1}{\sigma}}}
\newcommand{\encxOmegapsigma}[1]{\tuple{\permcon{x}{\Omega'}{#1}{\sigma}}}
\newcommand{\encyOmegasigma}[1]{\tuple{\permcon{y}{\Omega}{#1}{\sigma}}}
\newcommand{\encyOmegapsigma}[1]{\tuple{\permcon{y}{\Omega'}{#1}{\sigma}}}

\newcommand{\encxDeltainvsigma}[1]{\tuple{\permcon{x}{\Delta}{#1}{\sigma^{-1}}}}
\newcommand{\encxDeltapinvsigma}[1]{\tuple{\permcon{x}{\Delta'}{#1}{\sigma^{-1}}}}
\newcommand{\encxOmegainvsigma}[1]{\tuple{\permcon{x}{\Omega}{#1}{\sigma^{-1}}}}
\newcommand{\encxOmegapinvsigma}[1]{\tuple{\permcon{x}{\Omega'}{#1}{\sigma^{-1}}}}
\newcommand{\encyOmegainvsigma}[1]{\tuple{\permcon{y}{\Omega}{#1}{\sigma^{-1}}}}
\newcommand{\encyOmegapinvsigma}[1]{\tuple{\permcon{y}{\Omega'}{#1}{\sigma^{-1}}}}

\newcommand{\tstyle}{\vdash}

% gat macros developed for cwf paper

% Expressing gats
\newenvironment{gatrules}
{
$$
\begin{array}{l l}
}
{
\end{array}
$$
}
\newcommand{\gatintros}
{
\textbf{Symbol} & \textbf{Introductory\ Rule}                      \\}

\newcommand{\gataxioms}
{\textbf{Axioms}\\}
\newcommand{\gatintro}[3]{\ #1 & #2 \tstyle #3 \\}
\newcommand{\gatlocalintro}[3]{\ #1 & #2 \dashv }
\newcommand{\gataxiom}[2]{\multicolumn{2}{l}{\ \ #1\mbox{,  whenever\ } #2} \\}
\newcommand{\noleft}{\left.\kern-\nulldelimiterspace} % so that no space taken by absent left brace


\newcommand{\gatmultiaxiom}[2]
{\multicolumn{2}{l}{
  \noleft
    \begin{array}{l}
		#1
    \end{array} 
  \right\} \mbox{whenever\ } 	#2 
	}\\}
	
	\newcommand{\axid}[1]{\text{#1}.\ }	

%New context sharing macros
\newcommand{\gatintroducing}[1]{
{\arraycolsep=0pt
  \begin{array}{l}
          #1
  \end{array}} &
}

%*********************************
% \begin{\gatgroup}{context}
%    rules
%  \end{\gatgroup}
%*********************************
\NewEnviron{gatgroup}[1]{%
  \noleft
  {\arraycolsep=0pt
   \begin{array}{l}
\BODY
    \end{array} 
   }
   \ \right\} 
	%\mbox{\ whenever\ } 
	#1
	\vspace{0.1cm} 
}
%*********************************

%*********************************
% \begin{\gatgroupnoshared}
%    rule
%  \end{\gatgroupnoshared}
%*********************************
\NewEnviron{gatgroupnoshared}{%
  {\arraycolsep=0pt
   \begin{array}{l}
\BODY
    \end{array} 
   }
   \ 
	\vspace{0.1cm} 
}
%*********************************

% \gatsingular[width]{context}{conclusion}
\newcommand{\gatsingular}[3][4cm]{
\begin{gatgroupnoshared}
\gatleaf[#1]{#2}{#3} 
\end{gatgroupnoshared}
}

%*********************************
% \gatleaf}[width]{context}{assertion}
%*********************************
\newcommand{\gatleaf}[3][4cm]{%
\makebox[#1]{$#3$ \dotfill} \dotfill \  #2
}
%*********************************
%*********************************
% \gatstandalonesingle}{context}{assertion}
%*********************************
\newcommand{\gatstandalonesingle}[2]{%
#2 \makebox[2.5cm]{\dotfill} \  #1
}
%*********************************

% \gataxiomno{axiomno}
\newcommand{\gataxiomno}[1]{\makebox[0.5cm]{} \axid{#1}}


% metagat.macros.tex

%Meta-theories

%\newcommand{\typ}{\triangleright}
\newcommand{\typ}{\nabla}
\newcommand{\trm}{\tau}
\newcommand{\cross}{\otimes}
\newcommand{\sub}{^*}
\newcommand{\diag}{\delta}

\newcommand{\typeseq}[2]
{\ofT{#1_1}{\typ},... \ofT{#1_{#2}}{\typ(#1_{#2-1})}}

\newcommand{\typeseqcont}[3]
{\ofT{#1_1}{\typ({#2})},... \ofT{#1_{#3}}{\typ(#1_{#3-1})}}

\newcommand{\Ob}{Ob}
\newcommand{\obj}{Ob} % <!-- new old --<
\newcommand{\Hom}{Hom}
\newcommand{\objseq}[2]
{\ofT{#1_1}{\obj},... \ofT{#1_{#2}}{\obj(#1_{#2-1})}}


\def\dottededge{\ncline[linestyle=dotted, nodesep=0.3cm]}
\def\noedge{\ncline[linestyle=none]}
\def\thinedge{\ncline[linewidth=0.4pt]}

\newcommand{\member}[1]
{\ncarc[arcangle=-30,nodesepB=0.03]{->}{\pspred}{\pssucc}
\nbput[labelsep=0.1]{#1}}

\newcommand{\loweraccutemember}[1]
{\ncarc[arcangle=-15,nodesepB=0.03]{->}{\pspred}{\pssucc}
\nbput[labelsep=0.05,npos=0.85]{#1}}

\newcommand{\uppermember}[1]
{\ncarc[arcangle=30,nodesepB=0.03]{->}{\pspred}{\pssucc}\naput{#1}}

\newcommand{\upperaccutemember}[1]
{\ncarc[arcangle=10,nodesepB=0.03]{->}{\pspred}{\pssucc}\naput[npos=0.85]{#1}}

% flexbranch 
% #1 node label
% #2 thislevelsep
% #3 next level sep
% #4 variable (eg x)
% #5 index leter (eg n)
% #6 close parenthesis
% #7 continuation branches
\newcommand{\flexbranch}[7]
{
\pstree[thislevelsep=*#2,nodesep=0.05]
		{\Rnode{#1 1}{\Tr{#4_1 #6}}}
	  {\pstree[thislevelsep=#3]  
				   {\Rnode{#1 2}{\Tr[edge=\dottededge]{#4_{#5} #6}}}
					 {#7}
		}
}

\newcommand{\flexbranchplusleaf}[6]
{
\flexbranch{#1}{#2}{#3}{#4} {#5} {#6}
  {
   %\Rnode{#1 3}{\Tr{#4 #6}}
	 \Tr{\Rnode{#1 3}{#4 #6}}
  }
}

\newcommand{\flexbranchplusarc}[7]
{
\flexbranch{#1}{#2}{#3}{#4} {#5} {#6}
  {
   %\Rnode{#1 3}{\Tr{#4 #6}\member{#7}}
	 \Tr{\Rnode{#1 3}{#4 #6}}\member{#7}
  }
}

\newcommand{\flexbranchinitialarc}[9]
{
\pstree[thislevelsep=*#2,nodesep=0.05]
		{\Rnode{#1 1}{\Tr{#4_#8 #6}}#9}
	  {\pstree[thislevelsep=#3]  
				   {\Rnode{#1 2}{\Tr[edge=\dottededge]{#4_{#5} #6}}}
					 {#7}
		}
}

\newcommand{\equality}[2]
{
\ncline [doubleline=true, nodesep=0.2cm]{#1}{#2}
}
\newcommand{\equalityarc}[2]
{
\ncarc [arcangleA=-30, arcangleB=-20, doubleline=true, nodesep=0.1cm]{#1}{#2}
}

%
%  erdiag
%
  
%\begin{erdiagram}{#1 height}{#2 width} 
% ....
% ....
%\end{erdiagram}
\newenvironment{erdiagram}[2]
{%\pspicture*(-#1,0)(#2,0)
\pspicture*(0,-#1)(#2,0)
%\psgrid
}
{\endpspicture}

\definecolor{lightyellow}{cmyk}{0,0,0.3,0}

% \eret{#1 x0} {#2 y0} {#3 x1} {#4 y1} {#5 corner radius} {#6 fill}
\newcommand {\eret}[6]
{ 
\ifthenelse{\equal{#6}{1}}
{\psframe[framearc=#5,fillstyle=solid,fillcolor=lightyellow](#1,#2)(#3,#4)}
{\psframe[framearc=#5,fillstyle=solid,fillcolor=white](#1,#2)(#3,#4)}
}

% et top 
\newcommand {\erettop}[4]
{
%\psframe[linestyle=none,linearc=2pt,cornersize=absolute,fillstyle=solid,fillcolor=lightyellow](#1,#2)(#3,#4)
\psline[linearc=2pt,fillstyle=none,fillcolor=lightyellow](#1,#4)(#1,#2)(#3,#2)(#3,#4)
}

% et bottom 
\newcommand {\eretbtm}[4]
{
%\psframe[linestyle=none,linearc=2pt,cornersize=absolute,fillstyle=solid,fillcolor=lightyellow](#1,#2)(#3,#4)
\psline[linearc=2pt,fillstyle=none,fillcolor=lightyellow](#1,#2)(#1,#4)(#3,#4)(#3,#2)
}

% et bottom left
\newcommand {\eretbl}[4]
{
%\psframe[linestyle=none,linearc=2pt,cornersize=absolute,fillstyle=solid,fillcolor=lightyellow](#1,#2)(#3,#4)
\psline[linearc=2pt,fillstyle=none,fillcolor=lightyellow](#1,#4)(#3,#4)(#3,#2)
}

% et middle left
\newcommand {\eretml}[4]
{
%\psframe[linestyle=none,linearc=2pt,cornersize=absolute,fillstyle=solid,fillcolor=lightyellow](#1,#2)(#3,#4)
\psline[linearc=2pt,fillstyle=none,fillcolor=lightyellow](#1,#2)(#3,#2)(#3,#4)(#1,#4)
}

% et top left
\newcommand {\erettl}[4]
{
%\psframe[linestyle=none,linearc=2pt,cornersize=absolute,fillstyle=solid,fillcolor=lightyellow](#1,#2)(#3,#4)
\psline[linearc=2pt,fillstyle=none,fillcolor=lightyellow](#1,#2)(#3,#2)(#3,#4)
}

% et bottom right
\newcommand {\eretbr}[4]
{
%\psframe[linestyle=none,linearc=2pt,cornersize=absolute,fillstyle=solid,fillcolor=lightyellow](#1,#2)(#3,#4)
\psline[linearc=2pt,fillstyle=none,fillcolor=lightyellow](#1,#2)(#1,#4)(#3,#4)
}

% et middle right
\newcommand {\eretmr}[4]
{
%\psframe[linestyle=none,linearc=2pt,cornersize=absolute,fillstyle=solid,fillcolor=lightyellow](#1,#2)(#3,#4)
\psline[linearc=2pt,fillstyle=none,fillcolor=lightyellow](#3,#4)(#1,#4)(#1,#2)(#3,#2)
}

% et top right
\newcommand {\erettr}[4]
{
%\psframe[linestyle=none,linearc=2pt,cornersize=absolute,fillstyle=solid,fillcolor=lightyellow](#1,#2)(#3,#4)
\psline[linearc=2pt,fillstyle=none,fillcolor=lightyellow](#1,#4)(#1,#2)(#3,#2)
}

% \ergrp{#1 x0} {#2 y0} {#3 x1} {#4 y1} {#5 corner radius} {#6 fill}
% #5 corner radius is unused!
\newcommand {\ergrp}[6]
{ 
\ifthenelse{\equal{#6}{1}}
{\psframe[fillstyle=solid,fillcolor=lightgray](#1,#2)(#3,#4)}
{\psframe[fillstyle=solid,fillcolor=white](#1,#2)(#3,#4)}
}

% \eretname {#1 x left of text} {#2 y top of text} {#3 text}
\newcommand {\eretname}[3]
{
%shift down 0.1 for height of text the anchor at baseline (B)
\rput[bl]{0}(0,-0.1){\rput[Bl]{0}(#1,#2){\footnotesize \textit{#3}}}
}

% \errelarm {#1 x0} {#2 y0} {#3 x1} {#4 y1} {#5 ismandatory} {#6 isconstructed}
\newcommand {\errelarm}[6]
{
\ifthenelse{\equal{#6}{1}}
{
%%\psline[linewidth=0.5pt,linearc=.05,linestyle=dashed,dash=6pt 6pt]{-}(#1,#2)(#3,#4)}
\ifthenelse{\equal{#5}{1}}
{\psline[linewidth=1.5pt,linearc=.05,linecolor=lightgray]{-}(#1,#2)(#3,#4)}
{\psline[linewidth=1.5pt,linearc=.05,linecolor=lightgray,linestyle=dashed,dash=2pt 2pt]{-}(#1,#2)(#3,#4)}
}
{
\ifthenelse{\equal{#5}{1}}
{\psline[linewidth=0.9pt,linearc=.05]{-}(#1,#2)(#3,#4)}
{\psline[linewidth=0.9pt,linearc=.05,linestyle=dashed,dash=2pt 2pt]{-}(#1,#2)(#3,#4)}
}
}

% \errelangle {#1 x0} {#2 y0} {#3 x1} {#4 y1} {#5 x2} {#6 y2} {#7 ismandatory} {#8 isocnstructed}
\newcommand {\errelangle}[8]
{
\ifthenelse{\equal{#8}{1}}
{
%\psline[linewidth=0.5pt,linearc=.1,linestyle=dashed,dash=6pt 6pt]{-}(#1,#2)(#3,#4)(#5,#6)}
\ifthenelse{\equal{#7}{1}}
{\psline[linewidth=1.5pt,linearc=.05,linecolor=lightgray]{-}(#1,#2)(#3,#4)(#5,#6)}
{\psline[linewidth=1.5pt,linearc=.1,linecolor=lightgray,linestyle=dashed,dash=2pt 2pt]{-}(#1,#2)(#3,#4)(#5,#6)}
}
{
\ifthenelse{\equal{#7}{1}}
{\psline[linewidth=0.9pt,linearc=.1]{-}(#1,#2)(#3,#4)(#5,#6)}
{\psline[linewidth=0.9pt,linearc=.1,linestyle=dashed,dash=2pt 2pt]{-}(#1,#2)(#3,#4)(#5,#6)}
}
}

% \ercrowfoot {#1 x0} {#2 y0} {#3 x11} {#4 y11} {#5 x12} {#6 y12} {#7 x13} {#8 y13} {#9 isconstructed}
\newcommand {\ercrowfoot}[9]
{
\ifthenelse{\equal{#9}{1}}
{
\psline[linewidth=1.5pt,linearc=.05,linecolor=lightgray]{-}(#1,#2)(#3,#4)
\psline[linewidth=1.5pt,linearc=.05,linecolor=lightgray]{-}(#1,#2)(#5,#6)
\psline[linewidth=1.5pt,linearc=.05,linecolor=lightgray]{-}(#1,#2)(#7,#8)
}{
\psline[linewidth=0.9pt,linearc=.05]{-}(#1,#2)(#3,#4)
\psline[linewidth=0.9pt,linearc=.05]{-}(#1,#2)(#5,#6)
\psline[linewidth=0.9pt,linearc=.05]{-}(#1,#2)(#7,#8)
}
}


% \eridcomprel{#1 x1}{#2 x2}{#3 y1}{#4 ymid}{#5 y2}
\newcommand {\eridcomprel}[5]
{
\psline[linewidth=0.9pt](#1,#3)(#1,#5)
\psline[linewidth=0.9pt](#2,#3)(#2,#5)
\psline[linewidth=0.9pt](#1,#4)(#2,#4)
}

% \eridrefrel{#1 x1}{#2 xmid}{#3 x2}{#4 y1}{#5 y2}
\newcommand {\eridrefrel}[5]
{
\psline[linewidth=0.9pt](#1,#4)(#3,#4)
\psline[linewidth=0.9pt](#1,#5)(#3,#5)
\psline[linewidth=0.9pt](#2,#4)(#2,#5)
}


% \errelname {#1 x} {#2 y} {#3 text}
\newcommand {\errelname}[3]
{
\rput[l]{0}(#1,#2){\textit{#3}}
}
% \errelseq {#1 x} {#2 y}
\newcommand {\erelseq}[2]
{
}
% \erattr {#1 x} {#2 y} {#3 ismandatory}{#4 idenitfying} {#5 text}
\newcommand {\erattr}[5]
{
\ifthenelse{\equal{#3}{1}}
{\rput[l]{0}(#1,#2){{\tiny $\square$} {\footnotesize \textit{\ifthenelse{\equal{#4}{0}}{\underline{#5}}{#5}}}}}
{\rput[l]{0}(#1,#2){\footnotesize $\circ$ \textit{\ifthenelse{\equal{#4}{0}}{\underline{#5}}{#5}}}}
}

%\ifthenelse{\equal{#4}{1}}
% \ertext {#1 x} {#2 y} {#3 text anchor} {#4 text}
%{\rput[l]{0}(#1,#2){\footnotesize $\circ$ \underline{\textit{#5}}}}
\newcommand {\ertext}[4]
{
\rput[B#3]{0}(#1,#2){{\footnotesize #4}}
}
% \erarc {#1 x0} {#2 y0} {#3 x1} {#4 y1} {#5 x2} {#6 y2} {#7 x3} {#8 y3}
\newcommand {\erarc}[8]
{
\psbezier[showpoints=false]{-}(#1,#2) (#3, #4)(#5,#6) (#7, #8)
}

% \erarc {#1 x0} {#2 y0} {#3 x1} {#4 y1} {#5 x2} {#6 y2} {#7 x3} {#8 y3}
\newcommand {\errelseq}[8]
{
\psbezier[showpoints=false]{-}(#1,#2) (#3, #4)(#5,#6) (#7, #8)
}
% \ertrace {#1 trace}   
\newcommand {\ertrace}[1]
{
}

\usepackage{amsthm} % added 7th April 2018
% theorems.macros.tex

\newtheorem{theorem}{Theorem}[section]
\newtheorem{observation}[theorem]{Observation}
\newtheorem{lemma}[theorem]{Lemma}

\newtheorem{proposition}[theorem]{Proposition}
\newtheorem{corollary}[theorem]{Corollary}
\newtheorem{conjecture}[theorem]{Conjecture}
\newtheorem{numbereddefinition}[theorem]{Definition}

\newenvironment{definition}[1][Definition]{\begin{trivlist}
\item[\hskip \labelsep {\bfseries #1}]}{\end{trivlist}}
\newenvironment{examples}[1][Examples]{\begin{trivlist}
\item[\hskip \labelsep {\bfseries #1}]}{\end{trivlist}}
\newenvironment{example}[1][Example]{\begin{trivlist}
\item[\hskip \labelsep {\bfseries #1}]}{\end{trivlist}}
\newenvironment{remark}[1][Remark]{\begin{trivlist}
\item[\hskip \labelsep {\bfseries #1}]}{\end{trivlist}}

\newenvironment{tageqn}[1]
{
\begin{equation}
\stepcounter{equation}
\label{#1}
\tag{\theequation --#1}
}
{
\end{equation}
}

\newenvironment{axiom}[1]
{
\begin{equation}
\label{#1}
\tag{#1}
}
{
\end{equation}
}

% when the tag is required different from the label eg when has math symbols can use:
\newenvironment{axiomtagged}[2]
{
\begin{equation}
\label{#1}
\tag{#2}
}
{
\end{equation}
}

%visible label
\newcommand{\vlabel}[2][]{\label{#2}#1(\textit{#2}):}





% 
% general.macros.v2
%
% Rename macros that conflict with beamer 31 Aug 2022
% Don't assume an index                   31 Aug 2022

\usepackage{changepage} % used for adjustwidth

\iffalse % 31 Aug 2022
\usepackage{imakeidx}
\usepackage{framed}
\makeindex[name=definitions, title=Index of Definitions]
\makeindex[name=lemmas, title=Index of Lemmas]
\fi

\definecolor{highlight}{cmyk}{0,0,0.7,0}
\newcommand{\commentary}[1]{\marginpar{\footnotesize #1}}
\newcommand{\highlight}[1]{\colorbox{highlight}{#1}}
\newcommand{\whitelight}[1]{\colorbox{white}{#1}}
\newcommand{\term}[1]{\textit{#1}\commentary{\colorbox{lightgray}{\textit{#1}}}\index[definitions]{#1}}
\newcommand{\llabel}[1]{\label{#1}\commentary{\colorbox{pink}{\scriptsize{#1}}}\index[lemmas]{#1}}
\newcommand{\lref}[1]{\ref{#1}\colorbox{pink}{\scriptsize{#1}}\index[lemmas]{#1!use of}}

\newcommand{\daynote}[1]{\commentary{See day notes #1.}}

\newcommand{\newt}[1]{\colorbox{yellow}{#1}}
\newenvironment{newtt}
{  \colorbox{yellow}{$[$ ...} 
}
{  \colorbox{yellow}{... $]$}
}
\newcommand{\oldt}[1]{\colorbox{yellow}{\sout{#1}}}
\newenvironment{oldtt}
{  \colorbox{red}{$[$ ...} 
}
{  \colorbox{red}{... $]$}
}

\newcommand{\reinstatet}[1]{\colorbox{lime}{#1}}
\newenvironment{reinstatett}
{  \colorbox{lime}{$[$ ...}
}
{  \colorbox{lime}{... $]$}
}

\newcommand{\tbd}{\highlight{TBD}}

%ithprojection function
\newcommand{\proji}[1]{\pi_#1}


\newenvironment{aside}
{\begin{framed}
\textbf{Aside}
}
{
\end{framed}
}

\newenvironment{notebox}[1][Note]
{\begin{framed}
\textbf{#1}
}
{
\end{framed}
}

\newenvironment{categoricalaside}
{\begin{framed}
\textbf{Categorical Aside}
}
{
\end{framed}
}

\newenvironment{noteforfuture}
{\begin{framed}
\textbf{Note For Future}
}
{
\end{framed}
}

\newenvironment{myproblem}       %31 Aug 2022
{\begin{framed}
\textbf{Problem}
}
{
\end{framed}
}

\newenvironment{key}
{
\begin{tabular}{c l p{4cm}}
KEY && \\
\hline
}
{
\end{tabular}
}

%  31 Aug 2022
\NewEnviron{tightquote} %italic text indented left and right hand side
{\begin{adjustwidth}{1.5cm}{1.5cm}
\textit{
\BODY
}
\end{adjustwidth}
}

\newcommand{\keyentry}[3]{#1 & #2 & #3 \\} 


%quine quote
\newcommand{\qq}[1]{
\left\ulcorner#1\right\urcorner
}

%single quote
\newcommand{\sq}[1]{
\textnormal{\textquotesingle}#1\textnormal{\textquotesingle}
}

%lower quine quote
\newcommand{\lqq}[1]{
\left\llcorner #1\right\lrcorner
}


%from berkley
\newcommand{\langl}{\begin{picture}(4.5,7)
\put(1.1,2.5){\rotatebox{60}{\line(1,0){5.5}}}
\put(1.1,2.5){\rotatebox{300}{\line(1,0){5.5}}}
\end{picture}}
\newcommand{\rangl}{\begin{picture}(4.5,7)
\put(.9,2.5){\rotatebox{120}{\line(1,0){5.5}}}
\put(.9,2.5){\rotatebox{240}{\line(1,0){5.5}}}
\end{picture}}
\newcommand{\lang}{\begin{picture}(5,7)\put(1.1,2.5){\rotatebox{45}{\line(1,0){6.0}}}\put(1.1,2.5){\rotatebox{315}{\line(1,0){6.0}}}\end{picture}}
\newcommand{\rang}{\begin{picture}(5,7)\put(.1,2.5){\rotatebox{135}{\line(1,0){6.0}}}\put(.1,2.5){\rotatebox{225}{\line(1,0){6.0}}}\end{picture}}
%Try sharper tuple brackets -- except gives errors nested in captions so comment out
%\renewcommand{\tuple}[1]{\lang #1 \rang}

\newcommand{\setsuchthat}[2]{\left\{#1 \ \middle|\ #2\right\}}
\newcommand{\set}[1]{\left\{#1\right\}} 

% one to n - wanton
\newcommand{\wanton}[1]{#1_1,...#1_n}
\newcommand{\n}{1...n}
\newcommand{\fn}{\wanton{f}}
\newcommand{\gn}{\wanton{g}}
\newcommand{\pn}{\wanton{p}}
\newcommand{\qn}{\wanton{q}}
\newcommand{\qnprime}{\wanton{q'}}
\newcommand{\tn}{\wanton{t}}
\newcommand{\xn}{\wanton{x}}
\newcommand{\xnp}{\wanton{x'}}
\newcommand{\yn}{\wanton{y}}
\newcommand{\An}{\wanton{A}}
\newcommand{\Bn}{\wanton{B}}
\newcommand{\Cn}{\wanton{C}}
\newcommand{\ntuple}[1]{\tuple{\wanton{#1}}}
\newcommand{\wantom}[2][]{#2_1,...#2_{m#1}}
\newcommand{\m}{1...m}
\newcommand{\mtuple}[1]{\tuple{#1_1,...#1_m}}
\newcommand{\gm}{\wantom{g}}
\newcommand{\qm}{\wantom{q}}
\newcommand{\sm}[1][]{\wantom[#1]{s}}
\newcommand{\smp}{\wantom{s'}}
\newcommand{\ym}{\wantom{y}}
\newcommand{\Bm}{\wantom{B}}
\newcommand {\bntuple}{\ensuremath{\ntuple{b}}}
\newcommand {\fntuple}{\ensuremath{\ntuple{f}}}
\newcommand {\fnptuple}{\ensuremath{\ntuple{f}}}
\newcommand {\pntuple}{\ensuremath{\ntuple{p}}}
\newcommand {\qntuple}{\ensuremath{\ntuple{q}}}
\newcommand {\qnptuple}{\ensuremath{\ntuple{q'}}}
\newcommand {\qmtuple}{\ensuremath{\mtuple{q}}}
\newcommand {\sntuple}{\ensuremath{\ntuple{s}}}
\newcommand {\xntuple}{\ensuremath{\ntuple{x}}}
\newcommand {\xnptuple}{\ensuremath{\ntuple{x'}}}
\newcommand {\ymtuple}{\ensuremath{\mtuple{y}}}
\newcommand{\idef}[1][n]{1 \leq i \leq #1}
\newcommand{\jdef}[1][m]{1 \leq j \leq #1}
\newcommand{\kdef}[1][l]{1 \leq k \leq #1}
\newcommand{\foreachi}[1][n]{for each $i$, $1 \leq i \leq #1$}
\newcommand{\foreachj}[1][m]{for each $j$, $1 \leq j \leq #1$}
\newcommand{\foreachk}[1][l]{for each $k$, $1 \leq k \leq #1$}
\newcommand{\Foreachi}[1][n]{For each $i$, $1 \leq i \leq #1$}
\newcommand{\Foreachj}[1][m]{For each $j$, $1 \leq j \leq #1$}
\newcommand{\Foreachk}[1][l]{For each $k$, $1 \leq k \leq #1$}
\newcommand{\forsomei}[1][n]{for some $i$, $1 \leq i \leq #1$}
\newcommand{\forsomej}[1][m]{for some $j$, $1 \leq j \leq #1$}
\newcommand{\forsomek}[1][l]{for some $k$, $1 \leq k \leq #1$}
\newcommand{\wherei}[1][n]{where $1 \leq i \leq #1$}
\newcommand{\wherej}[1][m]{where $1 \leq j \leq #1$}
\newcommand{\wherek}[1][l]{where $1 \leq k \leq #1$}


\newcommand{\fundep}[3]{#2 \xrightarrow{#1} #3}  %where does this belong? xxxx
% Following used for notes -- indented numbered paras

\newcounter{para}
\newlength{\oldparindent}
\setlength{\oldparindent}{\parindent} % Save \parindent before of change
\newcommand{\ind}{\hspace*{\oldparindent}}

\newcommand\mynote{                                                 % renamed 31 Aug 2022
%\setlength{\parskip}{0.5\baselineskip} % Definition of `parskip`
\setlength{\parindent}{0pt}
\par\ind\refstepcounter{para}\thepara.\space
\setlength{\parindent}{\oldparindent}
}



%%%%%%%%%%%%%%%%%%%%%%%%%%%%%%%%%
% alternate.beamer.macros.tex
%%%%%%%%%%%%%%%%%%%%%%%%%%%%%%%%%
% This file contains implmentation of macros that are also implmented in
% beamer.macros.tex 
% The implementations here should be used for when papers are being built
% i.e. for non-presentations i.e. for papers.
%%%%%%%%%%%%%%%%%%%%%%%%%%%%%%%%%%%%%%%%%%%%%%%%%%%%%%%%%%%%%%%%%%%%%%%%%%%%%
% Commands to control level of detail in pictures
\newcounter{levelofdetail}
\newcommand{\waitfor}[2]{\ifnum #1 > \value{levelofdetail} \else #2 \fi}
% Setting a high number for level of detail shows most detail
\setcounter{levelofdetail}{10}
% In presentations waitfor is implemented using onslide as follows
%\renewcommand{\waitfor}[2]{\onslide<#1->{#2}} 
%Because of this outside of a waitfor level of detail is 1
% frist level programmed with waitfor should be level 2
%%%%%%%%%%%%%%%%%%%%%%%%%%%%%%%%%%%%%%%%%%%%%%%%%%%%%%%%%%%%%%%%%%%%%%%%%%%%

\usepackage{hyperref}
\setcounter{equation}{0}

\renewcommand{\erpictureFolder}[0]{../../SharedPictures}


\bibliographystyle{plain} % was hplain


\newcommand{\dottree}
{ \pstree
  {\Tr[edge=\dottededge]{}}
	{
	  \Tr[edge=\dottededge]{}
		\Tr[edge=\dottededge]{}
		%\Tr[edge=\dedge]{}
	}
}

\newcommand{\flabbytree}[3]
{
\pstree[levelsep=*1.2cm]
		{\Tr{#1_1}}
		{ \Tr[edge=\dottededge]{}
		  \Tr[edge=\dottededge]{}
		  \pstree[levelsep=*1.2cm]
				{\Tr{#1_2}}
				{\Tr[edge=\dottededge]{}
         \pstree[levelsep=*0.6cm]
				    {\Tr[edge=\dottededge]{#1_{#2}}}
					  {#3}
			  }
		}
}

\newcommand{\highlightpara}[1]{\colorbox{highlight}{%
    \parbox{\dimexpr\linewidth-2\fboxsep}% a box with line-breaks that's just wide enough
        {#1}}
}


\title{MetaGAT Algebra aka Concept-Instance Algebra}
\author{John Cartmell}
\begin{document}
\maketitle

\section{Concept delineation}
\mynote 
H G Wells:
\begin{tightquote}
Finally : the Logician, intent upon perfecting the certitudes of his methods rather than upon expressing the confusing subtleties of truth, has done little to help thinking men in the perpetual difficulty that arises from the fact that the universe can be seen in many different fashions and expressed by many different systems of terms, each expression within its limits true and yet incommensurable with expression upon a differing system. There is a sort of stratification in human ideas. I have it very much in mind that various terms in our reasoning lie, as it were, in different planes, and that we accomplish a large amount of error and confusion by reasoning terms together that do not lie or nearly lie in the same plane.
\end{tightquote}
\mynote
A factor in Gilbert Ryle's analysis of the problem is this:
\begin{tightquote}
[a] given word will, in different sorts of context, express ideas of an indefinite range of differing logical types and, therefore, with different logical powers. And what is true of single words is also true of complex expressions and of grammatical constructions. (1945, 206)
\end{tightquote}
Interestingly Ryle describes such ambiguity as systematic : “[u]nnoticed systematic ambiguities are a common source of type-confusions and philosophic problems” which falls short of saying that langauge is inherently ambigous but this is suggested to me by Ryle as quoted on
by the Stanford Encyclopedia of Philosophy:
\begin{tightquote}
Unlike ambiguous words such as “bank”, “report” or “stud”, the inflections of meaning to which most of our expressions are susceptible nonetheless have affinities: the ideas expressed by these expressions in their various uses are “intimately connected” with each other; they are “different inflections of the same root” (1945, 206).
\end{tightquote}
Ryles suggestion that such ambiguity applies to \textit{most} words and his use of words \textit{inflection} and \textit{root} suggest to me some kind of conceptual grammar. What might this be and is it inherent to the mechanism  of ordinary langauge? 
\mynote 
I can't be sure of the ambiguities that Ryle or Wells have in mind. But ambiguities abound and my examples here focus on the words \textit{book},
\textit{word} and \textit{letter} and, please note, each of these three words being taken in a single dictionary sense.

\mynote 
Three words but how many concepts? Fairly quickly I can delineate nine different concepts, all incontrovertibly 
different types of things with different cardinalities and having instances which are mutually exclusive :
\begin{itemize}
\item [book(1)] This is what we mean when we ask of a publisher `how many books have you published this year?' 
or when an author tells us that they have written a new book.
\item [book(2)] A particular copy of a book. This is the type of thing that we have in mind
 when we ask of a bookshop `how many books have you sold this year?'.
\item [word(1)] This is a word of a particular language. This is what we have in mind if we ask for an estimate for the number of words 
in the English language. 
\item [word(2)] This is what I have in mind if I ask for word frequencies in the works of Shakepeare.   
\item [word(3)] These are words as appear in the printed copies of books. This is what I have in mind if I say 'that word is illegible'.
\item [letter(1)] These are the letters which are enumerated by alphabets. These are what I have in mind when I say that the English language has
twenty six letters.
\item [letter(2)] This is a letter as used in a spelling of a word. It is this type of thing that you count 
as the the number of letters in a word when you are doing a crossword puzzle. 
\item [letter(3)] This is a letter as used to spell a particular word in a particular book by an author. 
\item [letter(4)] This is a printed copy of a letter in a book. 
\end{itemize}

\mynote
These nine  concepts are in close relationship to one another, \textit{intimately related}, to use the words of Ryle.
Some of these \textit{intimate relationships} are shown in this diagram:
  
\erinlinePicture{modelBook2ToLetter1}{5.5}



\section{The Meta-Theory of Generalised Algebraic Theories}
In such and uch a place I describe a generalised algebraic theory of generalised algebraic theories  
which I call \highlight{$MetaGAT$} in view of the fact that it can said to be a meta theory of generalised algebraic theories.
Here I re-articulate in an attempt not to lean as heavily on an appeal to syntactic notions. 

\section{Background}
As described in \cite{Cartmell86}, Contextual Categories provide algebraic representations of Generalised Algebraic Theories  in the sense that to every Generalised Algebraic Theory $U$ there is a contextual category $\mathbb{C}(U)$ such that algebras of the theory $U$ are exactly structure preserving functors (contextual functors)  from $\mathbb{C}(U)$ into the contextual category $Fam$ of sets, families of sets and so on. We summarise this by saying that Contextual Categories provide the algebraic semantics of Generalised Algebraic Theories. 

\noindent
Contextual categories contain structure that is in some sense redundant; this is in the same way that a Lawvere algebraic theory contains redundant structure compared to the otherwise equivalent notion, from universal algebra, of an abstract clone (see, for example, \cite{KerkoffonClones}). In both cases the redundancy is introduced in order to achieve a definition in which the algebras are categories with some sort of additional structure. Remove the redundancy and there is no longer a category directly visible in the definitions.

Terminology: By \highlight{a section of an object $B$}
 in a contextual category \catc, I mean a morphism $f:A\rightarrow B$ in \catc, where 
$A \base B$ and satisfying  $f \comp p_B = id_A$\footnote{In my thesis the notation $Arr_{\catc}(B)$ was used for what here we refer to as the sections and denote $Sect_{\catc}(B)$.}.


A \term{concept-instance} algebra consists of an $\omega$-tree of concepts and, 
for each concept $x$, a set $i(x)$
said to be the set of instances of $x$.
It comes equipped with three operations $^*$, $\crossx{}{}{}$ and $\delta$ and   
satisfies certain axioms.

\noindent
One way of explaining what constitutes a concept-instance algebra 
is to say that it consists of the $\omega$-tree of objects of a contextual category along with, 
for each object its set of sections  and with operations and axioms that completely encapsulate the structure of the contextual category. Vladimir Voevodsy and referred to such structures as \term{B-systems}.

The concepts of a concept-instance algebra corresponds to the objects of a contextual category. 
Accordingly  each concept corresponds to  a rule of a genealised algebraic theory $U$ of the form
\gatdisplayrule{\xDelta{n}}{\isT{\Delta}}. \\
Each instance corresponds to a rule of the form:
 \gatdisplayrule{\xDelta{n}}{\ofT{t}{\Delta}}.

\vspace{1cm}
The  operations $^*$ and $\crossx{}{}{}$  
 correspond in the syntax to  substitution and  weakening but here I refer to them as \term{particulising} and 
 \term{positing}.

In brief, these operations are as follows:

\begin{enumerate}
\item In a concept-instance algebra $A$, if $y$ is a concept and if $f$ is an instance of $y$
 and if $z$ is a concept so that $y < z$ then $f^*z$ is defined and is a concept of $A$ 
      such that \commentary{we could speak of concept $z$ being particularised to instance $f$.}
			\begin{enumerate}[(a)]
			\item if $y \base z$ then $x \base f^*z$,
			\item if $z \base z'$ and $y <z$ then $f^*z \base f^*z'$,
			\end{enumerate}
\setcounter{levelofdetail}{1} % do not display  g or f^*g


\vspace{0.5cm}
\begin{displaymath}
\pstree[treemode=R,levelsep=*1.0cm,treesep=1cm,nodesep=0.05]
 {
    %\Tr{\circ}
		\Tr{x}
 }
 {%\flexbranch{Lxn}{1cm}{1cm}{x}{n}{}
    {\pstree
		   {\Tr{y}\uppermember{f}}
			 {
			 \Tr{z} \waitfor{2}{\member{g}}
			 }
		 \Tr{f \sub z} \waitfor{3}{\member{f \sub g}}
		}
}
\end{displaymath}
\vspace{0.5cm}
\item if in addition $g$ is a instance of $z$ then $f^*g$ is defined and is a concept at $f^*z$. 
\setcounter{levelofdetail}{3} % include display of g and f^*g 


\vspace{0.5cm}
\begin{displaymath}
\pstree[treemode=R,levelsep=*1.0cm,treesep=1cm,nodesep=0.05]
 {
    %\Tr{\circ}
		\Tr{x}
 }
 {%\flexbranch{Lxn}{1cm}{1cm}{x}{n}{}
    {\pstree
		   {\Tr{y}\uppermember{f}}
			 {
			 \Tr{z} \waitfor{2}{\member{g}}
			 }
		 \Tr{f \sub z} \waitfor{3}{\member{f \sub g}}
		}
}
\end{displaymath}
\vspace{0.5cm}


At this point it follows that if $y \base z_1 ... \base z_m \base z$ and $g$ is a instance of $z$ then we have:

\commentary{intrductory rules for $^*$}
\begin{displaymath}
\pstree[treemode=\CItreemode,levelsep=*0.65cm,treesep=\CItreesep,nodesep=0.05]
 {
    %\Tr{\circ}
		\Tr{x}
 }
 {%\flexbranch{Lxn}{1cm}{1cm}{x}{n}{}
    {\pstree
		   {\Tr{y}\member{f}}
			 {
			 \flexbranchplusarc{Ly}{1cm}{1cm}{z}{m}{}{g}
			 }
		 \flexbranchplusarc{Lfy}{1cm}{1.3cm}{f \sub z}{m}{} {f \sub g}
		}
}
\end{displaymath}
\vspace{0.5cm} 

\item If $w,x$ and $y$ are concepts of $A$  and if $w \base x$ and $w < y$ then $x \cross_w y$ is a concept of $A$ 
such that
    \begin{enumerate}[(a)]
		\item if $w \base y$ then $x \base x \cross_w y$,
		\item if $y \base y'$ and $y <z$ then $x \cross_w y \base x \cross_w y'$,
		\end{enumerate}
\item if in addition $g$ is a instance of $y$ then $x \cross_w g$ is an instance of $x \cross_w y$.

At this point it follows that if $w \base y_1 ... \base y_m \base y$ and $g$ is a instance of $y$ then we have
\commentary{introductory rules for $\cross$}
\begin{displaymath}
\pstree[treemode=R,levelsep=*0.5cm,treesep=1cm,nodesep=0.05]
 {
   \Tr{w}
 }
 {%\flexbranch{Lx}{0.9cm}{0.5cm}{x}{n}{}
   {
	  \flexbranchplusarc{Lym}{1cm}{1cm}{y}{m}{}{g}
	  \pstree[levelsep=*0.5cm,nodesep=0.05]
		{\Tr{x}}
		{
		  \flexbranchplusarc{Lxym}{1cm}{1.6cm}{x \cross_w y}{m}{}{x \cross_w g}
		}
	 }
 }
\end{displaymath}
\vspace{0.5cm}
\item If $x$ and $y$ are concepts of $A$ such that $x \base y$ then $\delta_y$ is a instance of $\crossx{y}{y}{x}$
\vspace{0.5cm}
\commentary{introductory rule for $\delta$}
\begin{displaymath}
\pstree[treemode=\CItreemode,levelsep=*0.65cm,treesep=\CItreesep,nodesep=0.05]
{
   \Tr{x}
}
{
	\Tr{x \cross x} \member{\diag_x}
}
\end{displaymath}
 
\end{enumerate}

Such a structure as this is a concept-instance algebra providing the following axioms hold  
\begin{equation}
(f^*g)^*(f^*v) = f^*(g^*v)
\end{equation}
\begin{equation}
(x \cross y) \cross (x \cross v) = x \cross (y \cross v), 
\end{equation}
\begin{equation}
f^*(x \cross v) = y, \mbox{ when $f$ is a instance of $x$,} 
\end{equation}
\begin{equation}
f^*y \cross f^*v = f^*(y \cross v)
\end{equation}
\begin{equation}
(x \cross g)^*(x \cross v)=x \cross(g^*v)
\end{equation}
\begin{equation}
\delta_x ^*(x \cross y)=y
\end{equation}
\highlight{axioms missing!!!!}

where $x, y, z$ are concepts, $f$ and $g$ are concepts and $v$ is either a concept or a concept. In instances of $\cross$ subscripts must be chosen appropriately
and in all cases the axiom holds whenever either side is defined. Full detail iof these axioms including choices of subscripts is given below.


We can visualise the first of these axioms like this: 
\begin{displaymath}
\vspace{0.5cm}
\pstree[treemode=R,levelsep=*0.75cm,treesep=1.5cm,nodesep=0.05]
 {
    \Tr{\circ}
 }
 { \flexbranch{Lxn}{1cm}{0.9cm}{x}{n}{}
    {\pstree[treesep=1.2cm, thislevelsep=1.0cm]
		   {\Tr{x}\uppermember{f}}
       {
         \flexbranch{Lym}{1cm}{1cm}{y}{m}{}
         {\pstree[thislevelsep=0.7cm]
		       {\Tr{y}\uppermember{g}}
			     {
			     \flexbranchplusarc{Lz}{1cm}{1cm}{z}{l} {} {h}
			     }
	  	     \flexbranchplusarc{Lgz}{1cm}{1cm}{g \sub z}{l}{} {g \sub h}
		     }
	    }
			\flexbranch{Lfy}{0.5cm}{1.5cm}{f \sub y} {m}{}
			{
			      \flexbranchplusarc{Lfgfz}{1cm}{2.3cm}{(f \sub g) \sub (f \sub z}{l}{)} 
						  {(f \sub g)\sub(f \sub h)}\nbput[labelsep=0.4cm]{\Rnode{LAB2}{}} 
		        \flexbranchplusarc{Lfgz}{1cm}{2.3cm}{f \sub (g \sub z}{l}{)} 
						  {(f \sub (g \sub h)}\ncput{\Rnode{LAB1}{}}
			}
			\equality{Lfgfz1}{Lfgz1}
			\equality{Lfgfz2}{Lfgz2}
			\equality{Lfgfz3}{Lfgz3}
			\equality{LAB1}{LAB2}
    }
}
\end{displaymath}

\vspace{0.5cm}



\noindent
\highlightpara{
Compared to the theory of contextual categories, the theory $MetaGAT$ does not introduce redundant structure; it axiomatises the types and the terms of the theory directly and these correspond to the objects and the sections of a corresponding contextual category. The theory of contextual categories, on the other hand, axiomatises n-tuples of terms for any n; it is  these that are represented by the morphisms of a contextual category.}  

\noindent
\highlightpara{
The theory MetaGAT is closely related to the notion of B-system described by Voevodsky in
\cite{Voevodsky14B}.
and to the algebras over a monad described by Richard Garner in \cite{Garner15}.  
}
\section{Summarising}

\begin{enumerate}[(i)]
\item The category of $MetaGAT$ algebras and the category of contextual categories are isomorphic. 
\item There is an isomorphism between the theory of contextual categories and the theory $MetaGAT$ in the category of existentially and identity enriched generalised algebraic theories. 
\item There is no such isomorphism in the category of generalised algebraic theories.
\end{enumerate}

\noindent $MetaGAT$ algebras may have interpretations (and, therefore, applications) where Contextual Categories do not.

\section{Concept delineation}

\section{Re-Articulation ---  Concept-Instance algebra}

\mynote
There are \term{concepts} and each concept has a set of \term{instances}. There is a concept of \term{the absolute}($Abs$)
 This concept has exactly one instance ($abs$). 

\mynote
Every concept except the absolute has a \term{context} which it \term{depends on} and which itself is a concept. 
We write $ctxt(B)$
for the  concept that concept $B$ depends on.

We write $A \base B$  when  $A$ and $B$ are contexts and $B$ depends on $A$.

\mynote
If there exists $B_1,...B_n$ such that
$A \base B_1 \base B2 ... \base B_n \base B$ then we say that $B$ \term{indirectly depends on} $A$.

\mynote Every concept is directly or indirectly dependent on the absolute.

\mynote
Two operations act on concepts and instances: \term{particularisation} denoted
$^*$ and \term{consideration}  denoted $\cross$.

\mynote
$f^*x$ is the particularisation of $x$ to $f$
\mynote
$\crossx{x}{y}{w}$ is concept or instance $y$ considered in the context $x$ constrained within the bounding context $w$
\mynote
If $y$ depends directly or indirectly on $x$ then we say that $y$ is \term{within} $x$.
We also say that an instance $g$ of a concept $y$ is within $x$ provided that concept $y$ is within $x$.

\mynote 
Therefore every concept and instance is either the absolute or is within the absolute.
Therefore the absolute  is also \term{the whole}\footnote{See also \url{www.entitymodelling.org/tutorialone/absolute.html}}. It is the concept of the whole of everything.


\subsection{Concepts, contexts and instances within a cricket match.}
\newcommand{\inningsCrossSide}{\crossx{innings\kern-0.3cm}{\kern-0.3cm side}{match}}
\newcommand{\inningsCrossPlayer}{\crossx{innings\kern-0.3cm}{\kern-0.3cm player}{match}}
\newcommand{\fieldingSidePlayer}{fieldingSide ^* (\inningsCrossPlayer)}
\newcommand{\battingSidePlayer}{battingSide ^* (\inningsCrossPlayer)}
\newcommand{\overCrossFieldingSidePlayer}{\crossx{over\kern-0.4cm}{\kern-0.4cm(\fieldingSidePlayer)}{innings}}
\newcommand{\deliveryCrossBattingSidePlayer}{\crossx{delivery\kern-0.4cm}{\kern-0.4cm (\battingSidePlayer)}{innings}}
\begin{tabular}{l p{12cm}}
%$\base\ match$           & there is such a thing as a $match$                                 \\
$match\base side$        & there is such a thing as a $match$ and within the context of a $match$ there is such a thing as a $side$  \\ 
$homeSide$               & the $homeSide$ is an instance of a $side$                             \\
$side \base player$      & within the context of a $side$ there is such a thing as a $player$   \\
$captain \in player$     & within the context of a $side$,  the $captain$ is an instance of a $player$ \\
%$wicketkeeper \in player$& within the context of a $side$ within a $match$ the $wicketkeeper$ is a(n instance of a) $player$\\
$match \base innings$    & within the context of a $match$ there is such a thing as a $innings$    \\
$innings \base over$     & within the context of an $innings$ there is such a thing as an $over$            \\
$over \base delivery$    & within the context of an $over$ there is such a thing as a $delivery$ \\
\end{tabular}

\noindent
\begin{tabular}{l p{6.5cm}}

$fieldingSide \in \inningsCrossSide$ & within the context of an $innings$ of a $match$, 
                                      the $fieldingSide$ is an instance of a $side$ of that $match$           \\
$bowler \in \overCrossFieldingSidePlayer$ & within the context of an $over$ of an innings, the $bowler$ is an instance of a $fieldingSide$
$player$ within the context of that $innings$ \\
$battingSide \in \inningsCrossSide$ & within the context of an $innings$ of a $match$, 
                                      the $battingSide$ is an instance of a $side$ of that $match$         
\end{tabular}

\vspace{1.3cm}
\begin{displaymath}
\pstree[treemode=R,levelsep=*0.65cm,treesep=1cm,nodesep=0.05]
{
    \Tr{\circ}
}
{
    \pstree [levelsep=*0.85cm]
    {
		\Tr{match} 
	}
	{		  
		\pstree [levelsep=*0.85cm]
		{
				   \Tr{side} \uppermember {homeSide}
		}
		{
					\Tr{player} \uppermember {captain}
		}
	    \pstree [levelsep=*0.85cm]
		{
			\Tr{innings} 
		}
		{		  
		    \pstree [levelsep=*0.85cm]
			{
					   \Tr{over} 
			}
			{   
					   \Tr{delivery} 
			}			
		}		
	}	
}
\end{displaymath}

\vspace{1.3cm}
\begin{displaymath}
\pstree[treemode=R,levelsep=*0.65cm,treesep=1cm,nodesep=0.05]
{
    \Tr{\circ}
}
{
    \pstree [levelsep=*0.85cm]
    {
		\Tr{match} 
	}
	{		  
		\Tr{side} 
	    \pstree [levelsep=*0.85cm]
		{
			\Tr{innings} 
		}
		{		  
			\Tr{\inningsCrossSide} \member {battingSide}		
		}		
	}	
}
\end{displaymath}


\noindent
\begin{tabular}{l | m{6.5cm}} 
$  \noleft
  {\arraycolsep=0pt
   \begin{array}{l}
facingBatter \in \deliveryCrossBattingSidePlayer \\
nonFacingBatter \in \deliveryCrossBattingSidePlayer 
    \end{array} 
   }
   \ \right\}
  $ & within the context of a $innings$ within a $match$, the $facingBatter$ and the $nonFacingBatter$ 
  are instances of $battingSide$ $player$s within the context of that $innings$ \\
\end{tabular}

\newpage

 \begin{tabular}{l p{5cm}}
$\base sport \base match$            & within the context of a $sport$ there are $match$es      \\
$cricket \in sport$                  & $cricket$ is a $sport$                                   \\
$\therefore \base cricket^*match$    & there are $cricket^*matches$                             \\
\end{tabular}

\mynote
Both concepts and instances can be particularised.
$^*$ is a binary infix operation. Its first argument is an instance its second is either concept or instance.
If $f$ is an instance of concept $A$ then $f^*x$ is defined iff $x$ is either a concept or an instance that is within $A$. 
.

\begin{itemize}
\item if $ctxt(B)=A$ then $ctxt(f^*B)=ctxt(A)$,
\item if $ctxt(B)$ indirectly depends on $A$ then $ctxt(f^*B)=f^*ctxt(B)$,
\item if $g$ is an instance of $B$ then $f^*g$ is an instance of $f^*B$.
\end{itemize}

\mynote 
The consideration operator $\cross$ is a ternary operation:
$\crossx{A}{y}{X}$ is concept or instance $y$ considered in the context $A$ constrained within some bounding context $X$.

$\crossx{A}{y}{X}$ is defined iff $X$ is a concept, $A$ is a concept  within $X$ and $y$ is either
a concept or an instance within $X$. 

\begin{itemize}
\item if $ctxt(B)=X$ then $ctxt(\crossx{A}{B}{X})=A$,
\item if $ctxt(B)$ indirectly depends on $A$ then $ctxt(\crossx{A}{B}{X})=\cross{A}{cntxt(B)}{X}$,
\item if $g$ is an instance of $B$ then $\crossx{A}{g}{X}$ is an instance of $\crossx{A}{B}{X}$.
\end{itemize}

\mynote
For every concept $A$ there is an instance $self_A$. $self_A$ is an instance of $\crossx{A}{A}{X}$ where $X$ is the context of $A$.

\mynote Axioms - as before.
\newpage
\mynote 
Analyse these two concepts: 
\begin{equation}
\label{twofromplays}
\mbox{\textit{`two characters from the plays of shakespeare'}}
\end{equation}
\begin{equation}
\label{twofromsingleplay}
\mbox{\textit{`two characters from a play by shakespeare'}}
\end{equation}

Assume this concept model:

 \vspace{0.3cm}
\begin{displaymath}
\pstree[treemode=R,levelsep=*0.65cm,treesep=1cm,nodesep=0.05]
 {
    \Tr{\circ}
 }
 {
   \pstree [levelsep=*0.85cm]
	    {
			  \Tr{w} \uppermember {s}
			}
			{		  
				\pstree [levelsep=*0.85cm]
				{
				   \Tr{p} 
				}
				{
				   \Tr{c} 
			  }			
			}
		\iffalse	
	 \Tr{f \sub y} \member {f \sub g}
	\fi
 }
\end{displaymath}
\vspace{0.2cm}

Concept (\ref{twofromplays}) is represented by $s^*(\crossx{c}{c}{w})$ because
\begin{itemize}
\item The concept of two characters in plays by the same playwright is $\crossx{c}{c}{w}$.
\item Particularised to plays by shakespeare this is $s^*(\crossx{c}{c}{w})$.
\end{itemize}

Diagramming the tree we have

\vspace{0.3cm}
\begin{displaymath}
\pstree[treemode=R,levelsep=*0.65cm,treesep=1cm,nodesep=0.05]
 {
    \Tr{\circ}
 }
 {
   \pstree [levelsep=*0.85cm]
	    {
			  \Tr{w} \uppermember {s}
			}
			{		  
				\pstree [levelsep=*0.85cm]
				{
				   \Tr{p} 
				}
				{
				   \pstree [levelsep=*0.85cm]
				   {
						   \Tr{c}
					 }
					 {
					     \pstree [levelsep=*0.85cm]
							 {
					         \Tr{c \cross_w p}
							 }
							 {
							     \Tr{c \cross_w c}
							 }
					 }
			  }			
			}
	 \pstree [levelsep=*0.85cm]
	    {
			  \Tr{s^*p} 
			}
			{		  
				\pstree [levelsep=*0.85cm]
				{
				   \Tr{s^*c} 
				}
				{   
				    \pstree [levelsep=*0.85cm]
						{
				       \Tr{s^*(c \cross_w p)} 
						}
						{
						   \Tr{s^*(c \cross_w c)}
						}
			  }			
			}			
		\iffalse	
	 \Tr{f \sub y} \member {f \sub g}
	\fi
 }
\end{displaymath}

Equally concept (\ref{twofromplays}) maybe represented by $\crossx{s^*c}{s^*c}{Abs}$.
This is an example of the distributive law at work in concept-instance algabra.


Concept (\ref{twofromsingleplay}) is represented by $s^*(\crossx{c}{c}{p})$ because
\begin{itemize}
\item The concept of two characters in the same play is $\crossx{c}{c}{p}$.
\item Particularised to plays by shakespeare this is $s^*(\crossx{c}{c}{p})$.
\end{itemize}

Diagramming the tree we have

\vspace{0.3cm}
\begin{displaymath}
\pstree[treemode=R,levelsep=*0.65cm,treesep=1cm,nodesep=0.05]
 {
    \Tr{\circ}
 }
 {
   \pstree [levelsep=*0.85cm]
	    {
			  \Tr{w} \uppermember {s}
			}
			{		  
				\pstree [levelsep=*0.85cm]
				{
				   \Tr{p} 
				}
				{
				   \pstree [levelsep=*0.85cm]
				   {
						   \Tr{c}
					 }
					 {
					     \Tr{c \cross_p c}
					 }
			  }			
			}
	 \pstree [levelsep=*0.85cm]
	    {
			  \Tr{s^*p} 
			}
			{		  
				\pstree [levelsep=*0.85cm]
				{
				   \Tr{s^*c} 
				}
				{
				   \Tr{s^*(c \cross_p c)} 
			  }			
			}			
		\iffalse	
	 \Tr{f \sub y} \member {f \sub g}
	\fi
 }
\end{displaymath}
\vspace{0.2cm}
Equally, concept (\ref{twofromsingleplay}) may be represented by $\crossx{s^*c}{s^*c}{s^*p}$. This is the distributive law at work again.

%\fi
%\iffalse
A citizen in the context of a triangle is written as:
$$
triangle \cross citizen
$$
and since a citizen needs a country for context:
$$
country \base citizen
$$
 therefore $triangle \cross citizen$ requires $triangle \cross country$:
$$ triangle \cross country \base triangle \cross citizen$$.


One citizen in the context of another, when written as:
$$citizen \cross citizen$$
is ambiguous and is therefore written instead as:
$$citizen \cross_{country} citizen$$
if we imagine a citizen of the same country or is written as as
$$citizen \cross_1 citizen$$
if we imagine absolutely any other citizen of any country whatsoever, 
where $1$ represents \textit{the absolute}. 

Particularisation of  citizen to instance france of country is written as
$france^*citizen$.



\section{Trees of Concepts and the GAT of $\omega$-trees}
\subsection {Trees of Concepts and the GAT of Trees}

\renewcommand{\highlight}[1]{#1}  % For print copy don't want highlighting.
 
Terminology: By  the generic term \term{tree} is meant a partially ordered set (poset) $(T, <)$ such that for each $t \in T$, the set $\set{s \in T : s < t}$ is well-ordered by the relation $<$.
In this discussion we restrict ourselves to \highlight{rooted $\omega$-trees} i.e. trees for which the set $\set{s \in T : s < t}$
is finite for all $t \in T$ and for which there is a unique element $t$ such that
 the set $\set{s \in T : s < t}$ is empty. We say that such a $t$ is the root of the tree.

With respect to a partial ordering $<$, we say that \highlight{an element $y$ \textit{covers}  an element $x$} in  iff $x<y$ and there does not exist $w$ such that $x < w$ and $w < y$.
If object $y$ covers object $x$ in the partial ordering 
then we write \highlight{$x \base y$} (we use this in preference to the more usual $x \lessdot y$).
For $x$ an element of the tree we define the set of elements  \highlight{$Cover(x)$} to be the set of objects covering $x$.

Such trees as these we can equivalently describe as models of the generalised algebraic theory given below table \ref{GATOFTREES} in which the nodes of height $n+1$ are represented as of a sort $Cover_{n+1}$that is dependent on the sort of nodes of height $n$.

%\newcommand{\Ft}[1]{#1 \kern -0.4em \downarrow}
\newcommand{\Ft}[1]{\downarrow \kern -0.325em #1}
If A and B are nodes of  a tree $(S,<)$ then we shall write $A \base B$ to mean that $A < B$ in S and that
there does not exist x such that $A < x < B$. For every node B of tree S other than the root node there exists a unique node A such that $A \base B$.

\newcommand{\Sz}{Base}
\newcommand{\ofS}[1]{\ofT{#1}{\Sz}}
\newcommand{\Si}[1]{C\kern-1pt over_{#1}}
\newcommand{\ofSi}[3]{\ofT{#1}{\Si{#2}(#3)}}
\vspace{0.03cm} 
\begin{table}[H]
\caption{The Generalised Algebraic Theory of $\omega$-Trees}
\label{GATOFTREES}

%
\begin{tabular}{>{\itshape}l l}
Symbol & \itshape{Introductory Rule} \\
$\Sz  $&$\isT{\Sz}$\\
$\Si{1} $&$\ofS{x_0} \tstyle \isT{\Si{1}(x_0)} $\\
$\Si{2} $&$\ofS{x_0},\ofSi{x_1}{1}{x_0} \tstyle \isT{\Si{2}(x_0,x_1)} $\\
$\vdots$  \\
$\Si{n} $&$\ofS{x_0},\ofSi{x_1}{1}{x_0}, \hdots \ofSi{x_{n-1}}{n-1}{x_0,x_1,\hdots x_{n-2}} \tstyle \isT{\Si{n}(x_0,x_1,\hdots x_{n-1})} $\\
$\vdots$   \\
\end{tabular} \\


\iffalse
% Following would need improving perhaps using pstricks
\begin{gatrules}
\gatintros
\gatintroducing{Base}
\isT{Base} \\
\gatintroducing{Cov_1\\Cov_2\\\vdots \\ \vdots\\Cov_n\\ \vdots \\ \vdots}
\begin{gatgroup}{\ofT{x_0}{Base}}
  \gatleaf[5cm]{}{\isT{Cov_1(x_0)}} \\
  \begin{gatgroup}{\ofT{x_1}{Cov_1(x_0)}}
    \gatleaf[5cm]{}{\isT{Cov_2(x_0,x_1)}} \\
    %\begin{gatgroup}{\ofT{x_2}{Cov_2(x_0,x_1)}}
    \vdots \\
    \vdots \\
    \begin{gatgroup}{\ofT{x_{n-1}} {Cov_{n-1}(x_0,...x_{n-2})}\iddots}
    \gatleaf[5cm]{}{\isT{Cov_n(x_0,...x_{n-1})}} \\
    \vdots \\
    \vdots
    \end{gatgroup}
    %\end{gatgroup}
  \end{gatgroup}
\end{gatgroup}
\end{gatrules}
\fi

\end{table} 

\subsection {Schematic Notation}
%\newcommand{\Ft}[1]{
%#1 \kern-6pt \raisebox{1.45ex}{$\leftrightline$} \kern-3pt \raisebox{.09ex}{$\downarrow$}\kern-3.4pt \raisebox{.25ex} {$|$}}
\newcommand{\ft}[1]{
#1 \kern-6pt \raisebox{1.1ex}{$\leftrightline$} \kern-3pt \raisebox{.1ex}{$\downarrow$}}
%\newcommand{\Bbar}[1]{
%#1 \kern-6pt \raisebox{1.45ex}{$\leftrightline$}
%\overline{#1}}
%\vv{#1}}
%\newcommand{\bbar}[1]{
%#1 \kern-6pt \raisebox{1.0ex}{$\leftrightline$}
%\overline{#1}}
%\vv{#1}}
\newcommand{\bbin}[1]{
\raisebox{-0.5em}{$\stackrel{\displaystyle{\in}} {\scriptstyle{#1}}$}
}
\newcommand{\ofTn}[3]{
\raisebox{0.25pt}{$\bar{#1}$} \bbin{#2} #3}

\newcommand{\genericOb}{Ob} % where we have genericOb=Base + Cover

There is a  shorthand that is convenient in the presentation  of the GAT of trees  and then subsequently in the GAT of contextual categories. We use the shorthand
$\ofTn{x}{n}{\genericOb}$ for the context $\ofS{x_0},\ofSi{x_1}{1}{x_0}, \hdots \ofSi{x_n}{n}{x_0,x_1,\hdots x_{n-1}} $. \\

\noindent Using this shorthand, for any $n \geq 0$ the sort $Cover_{n}$  in the theory of trees is introduced as follows: \\

\vspace{0.03cm} 
\begin{tabular}{>{\itshape}l l}
Symbol & \itshape{Introductory Rule} \\
$\Sz  $     & $\isT{\Sz}$\\
$\Si{n+1}, n \geq 0 $ & $\ofTn{x}{n}{\genericOb}    \tstyle \isT{\Si{n+1}(\bar{x})} $\\
\end{tabular} \\
\vspace{.1cm}  \\




\bibliography{../../SharedBibliography/temp/bibliography}

\end{document}
