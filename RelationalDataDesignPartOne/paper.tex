
%\documentclass[prodmode,acmtods]{acmsmall}
\documentclass[10pt,a4paper]{article}
\usepackage[margin=4.0cm]{geometry} %was 3cm
\usepackage{mathptmx}
\usepackage{amsfonts}
\usepackage{array}
\usepackage{pstricks}
\usepackage{pst-tree}
\usepackage{pst-plot}
\usepackage{pst-node}
\usepackage{stmaryrd}
\usepackage{amsmath}
\usepackage{verbatim}
\usepackage{graphicx}  
\usepackage{calc}
\usepackage{xifthen}
\usepackage{xcolor}
\usepackage{color}
\usepackage{stringstrings}
%\usepackage[small,bf,margin=3pt,format=hang, labelsep=endash,singlelinecheck=false]{caption} %prevuiously justification=justified
%\usepackage{enumerate}
%\usepackage{enumitem}
\usepackage{enumerate}
\usepackage[shortlabels]{enumitem}
\usepackage{float}
\usepackage[section]{placeins}
%\setlength{\captionmargin}{5pt}
\usepackage{environ}
\usepackage{multirow}
\usepackage{rotating}
\usepackage{longtable}
\usepackage{afterpage}
\usepackage{needspace}


%DEFINE ENVIRONMENT BLOCK
% Riddle
\newsavebox{\riddlebox}

\newenvironment{erexample}
{\newcommand\colboxcolor{F0F0F0}%was F8F8F8
\begin{lrbox}{\riddlebox}
\begin{minipage}{\dimexpr\columnwidth-2\fboxsep\relax} \textbf{} \\ \itshape}
{\end{minipage}\end{lrbox}%
%\begin{center}
\colorbox[HTML]{\colboxcolor}{\usebox{\riddlebox}}
%\end{center}
}

\newenvironment{erbox}
{\newcommand\colboxcolor{F0F0F0}%was F8F8F8
\begin{lrbox}{\riddlebox}%
\begin{minipage}{\dimexpr\columnwidth-2\fboxsep\relax} }
{\end{minipage}\end{lrbox}%
%\begin{center}
\colorbox[HTML]{\colboxcolor}{\usebox{\riddlebox}}
%\end{center}
}

%\begin{erboxedFigure}{#1 FigureParam}{#2 Label}{#3 Caption}
\NewEnviron{erboxedFigure}[3]{%
\begin{figure}[#1]
\begin{erexample}
\begin{center}
\BODY
\end{center}
\vspace{-0.5cm}
\caption{#3}
\label{#2}
\end{erexample}
\end{figure}
}

\newcommand{\erpictureFolder}[0]{../SharedPictures}

\newcommand{\ercenterPicture}[1]{
\begin{center}
\input{\erpictureFolder/#1}
\end{center}
}


\newlength{\erhalfHt}

%\erinlinePicture{#1 pictureFilename}{#2 pictureHeight}
\newcommand{\erinlinePicture}[2]{
\setlength{\erhalfHt}{#2cm * \real{0.5}}
\raisebox{-\erhalfHt}[\erhalfHt + 0.5cm][\erhalfHt + 0.5cm]{
\input{\erpictureFolder/#1}
} 
}

%\erplainFig{#1 pictureFilename}{#2 figureParam}{#3Caption}
\newcommand{\erplainFig}[3]{
\begin{figure}[#2]
\begin{center}
\input{\erpictureFolder/#1}
\end{center}
\caption{#3}
\label{#1}
\end{figure}
}

%\erboxedFigPicture{#1 pictureFilename}{#2 figureParam}{#3Caption}
\newcommand{\erboxedFigPicture}[3]{
\begin{figure}[#2]
\begin{erexample}
\vspace{-0.5cm}
\begin{center}
\input{\erpictureFolder/#1}
\end{center}
\caption{#3}
\label{#1}
\end{erexample}
\end{figure}
}

%\erLeftSideFig{#1 pictureFilename}{#2 figureParam}{#3Caption}
\newcommand{\erLeftSideFig}[3]{
\begin{figure}[#2]
\begin{erexample}
  \begin{minipage}[c]{0.4\textwidth}
    \caption{#3}
    \label{#1}
  \end{minipage}
  \begin{minipage}[c]{0.5\textwidth}
    \input{\erpictureFolder/#1}
  \end{minipage}
\end{erexample}
\end{figure}
}

%\erbulletedFig{#1 pictureFilename}{#2 figureParam}{#3Caption}
\NewEnviron{erbulletedFig}[3]{%
\begin{figure}[#2]
\begin{erexample}
\vspace{-0.5cm}
\begin{center}
$
\begin{array}{c m{0.25cm} | m{6cm}}
\raisebox{-2.0cm}{
\input{\erpictureFolder/#1}}& & \text{\parbox{6cm}{\raggedright{\footnotesize{
\begin{enumerate}[(i)]
\BODY
\end{enumerate}}}}} \\
\end{array}
$
\end{center}
\caption{#3}
\label{#1}
\end{erexample}
\end{figure} 
}


%\begin{erbulletedDimFig}{#1 pictureFilename}{#2figureParam} {#3Caption} {#4PictureHeight}{#5TextWidth}

\NewEnviron{erbulletedDimFig}[5]{%
\begin{figure}[#2]
\begin{erexample}
\vspace{-0.5cm}
\begin{center}
$
\begin{array}{c m{0.25cm} |  m{#5cm}}
\setlength{\erhalfHt}{#4cm * \real{0.5}}
\raisebox{-\erhalfHt}{
\input{\erpictureFolder/#1}}& & \text{\parbox{#5cm}{\raggedright{\footnotesize{
\begin{enumerate}[(i)]
\BODY
\end{enumerate}}}}} \\
\end{array}
$
\end{center}
\caption{#3}
\label{#1}
\end{erexample}
\end{figure} 
}

%\begin{ernotedModel}{#1 pictureFilename}{#2PictureHeight}{#3PictureWidth}{#4TextWidth}

\NewEnviron{ernotedModel}[4]{%
\begin{center}
$
\begin{array}{m{#3cm} m{1cm} | c m{#4cm}}
\setlength{\erhalfHt}{#2cm * \real{0.5}}
\raisebox{-\erhalfHt}{
\input{\erpictureFolder/#1}}& & & \text{\parbox{#4cm}{\raggedright{\footnotesize{
\BODY
}}}} \\
\end{array}
$
\end{center} 
}

%\begin{ermodelText}{#1 pictureFilename}{#2PictureHeight}{#3PictureWidth}{#4TextWidth}

\NewEnviron{ermodelText}[4]{%
\begin{center}
\begin{tabular}{m{#3cm} m{1cm}  c m{#4cm}}
\setlength{\erhalfHt}{#2cm * \real{0.5}}
\raisebox{-\erhalfHt}{
\input{\erpictureFolder/#1}}& & & \text{\parbox{#4cm}{\raggedright{\small{
\BODY
}}}} \\
\end{tabular}
\end{center} 
}


%\erbulletedModel{#1 pictureFilename}{#2PictureHeight}{#3PictureWidth}{#4TextWidth}

\NewEnviron{erbulletedModel}[4]{%
\begin{center}
$
\begin{array}{m{#3cm} m{1cm} | c m{#4cm}}
\setlength{\erhalfHt}{2cm * \real{0.5}}
\raisebox{-\erhalfHt}{
\input{\erpictureFolder/#1}}& & & \text{\parbox{#4cm}{\raggedright{\footnotesize{
\begin{enumerate}[(i)]
\BODY
\end{enumerate}}}}} \\
\end{array}
$
\end{center} 
}



%\ernotedDimFig{#1 pictureFilename}{#2 figureParam}{#3Caption}{#4PictureHeight}{#5TextWidth}
\NewEnviron{ernotedDimFig}[5]{%
\begin{figure}[#2]
\begin{erexample}
\vspace{-0.5cm}
\begin{center}
$
\begin{array}{c m{0.25cm} | c m{#5cm}}
\setlength{\erhalfHt}{#4cm * \real{0.5}}
\raisebox{-\erhalfHt}{
\input{\erpictureFolder/#1}}& & & \text{\parbox{#5cm}{\raggedright{\footnotesize{
\BODY }}}}\\
\end{array}
$
\end{center}
\caption{#3}
\label{#1}
\end{erexample}
\end{figure} 
}
%\begin{ernotedDimFigPW}{#1 pictureFilename}{#2 figureParam}{#3Caption}{#4PictureHeight}{#5PictureWidth}{#6TextWidth}
\NewEnviron{ernotedDimFigPW}[6]{%
\begin{figure}[#2]
\begin{erexample}
\vspace{-0.5cm}
\begin{center}
$
\begin{array}{>{\centering}m{#5cm} m{0.5cm} | c m{#6cm}}
\setlength{\erhalfHt}{#4cm * \real{0.5}}
\raisebox{-\erhalfHt}{
\centering \input{\erpictureFolder/#1}
}& & & \text{\parbox{#6cm - 0.5cm}{\raggedright{\footnotesize{
\BODY }}}}\\
\end{array}
$ \\
\vspace {0.2cm}
\end{center}
\caption{#3}
\label{#1}
\end{erexample}
\end{figure}
}



\newenvironment{erquote}
{\begin{quote}\itshape}
{\end{quote}}


%
%  erdiag
%
  
%\begin{erdiagram}{#1 height}{#2 width} 
% ....
% ....
%\end{erdiagram}
\newenvironment{erdiagram}[2]
{%\pspicture*(-#1,0)(#2,0)
\pspicture*(0,-#1)(#2,0)
%\psgrid
}
{\endpspicture}

\definecolor{lightyellow}{cmyk}{0,0,0.3,0}

% \eret{#1 x0} {#2 y0} {#3 x1} {#4 y1} {#5 corner radius} {#6 fill}
\newcommand {\eret}[6]
{ 
\ifthenelse{\equal{#6}{1}}
{\psframe[framearc=#5,fillstyle=solid,fillcolor=lightyellow](#1,#2)(#3,#4)}
{\psframe[framearc=#5,fillstyle=solid,fillcolor=white](#1,#2)(#3,#4)}
}

% et top 
\newcommand {\erettop}[4]
{
%\psframe[linestyle=none,linearc=2pt,cornersize=absolute,fillstyle=solid,fillcolor=lightyellow](#1,#2)(#3,#4)
\psline[linearc=2pt,fillstyle=none,fillcolor=lightyellow](#1,#4)(#1,#2)(#3,#2)(#3,#4)
}

% et bottom 
\newcommand {\eretbtm}[4]
{
%\psframe[linestyle=none,linearc=2pt,cornersize=absolute,fillstyle=solid,fillcolor=lightyellow](#1,#2)(#3,#4)
\psline[linearc=2pt,fillstyle=none,fillcolor=lightyellow](#1,#2)(#1,#4)(#3,#4)(#3,#2)
}

% et bottom left
\newcommand {\eretbl}[4]
{
%\psframe[linestyle=none,linearc=2pt,cornersize=absolute,fillstyle=solid,fillcolor=lightyellow](#1,#2)(#3,#4)
\psline[linearc=2pt,fillstyle=none,fillcolor=lightyellow](#1,#4)(#3,#4)(#3,#2)
}

% et middle left
\newcommand {\eretml}[4]
{
%\psframe[linestyle=none,linearc=2pt,cornersize=absolute,fillstyle=solid,fillcolor=lightyellow](#1,#2)(#3,#4)
\psline[linearc=2pt,fillstyle=none,fillcolor=lightyellow](#1,#2)(#3,#2)(#3,#4)(#1,#4)
}

% et top left
\newcommand {\erettl}[4]
{
%\psframe[linestyle=none,linearc=2pt,cornersize=absolute,fillstyle=solid,fillcolor=lightyellow](#1,#2)(#3,#4)
\psline[linearc=2pt,fillstyle=none,fillcolor=lightyellow](#1,#2)(#3,#2)(#3,#4)
}

% et bottom right
\newcommand {\eretbr}[4]
{
%\psframe[linestyle=none,linearc=2pt,cornersize=absolute,fillstyle=solid,fillcolor=lightyellow](#1,#2)(#3,#4)
\psline[linearc=2pt,fillstyle=none,fillcolor=lightyellow](#1,#2)(#1,#4)(#3,#4)
}

% et middle right
\newcommand {\eretmr}[4]
{
%\psframe[linestyle=none,linearc=2pt,cornersize=absolute,fillstyle=solid,fillcolor=lightyellow](#1,#2)(#3,#4)
\psline[linearc=2pt,fillstyle=none,fillcolor=lightyellow](#3,#4)(#1,#4)(#1,#2)(#3,#2)
}

% et top right
\newcommand {\erettr}[4]
{
%\psframe[linestyle=none,linearc=2pt,cornersize=absolute,fillstyle=solid,fillcolor=lightyellow](#1,#2)(#3,#4)
\psline[linearc=2pt,fillstyle=none,fillcolor=lightyellow](#1,#4)(#1,#2)(#3,#2)
}

% \ergrp{#1 x0} {#2 y0} {#3 x1} {#4 y1} {#5 corner radius} {#6 fill}
% #5 corner radius is unused!
\newcommand {\ergrp}[6]
{ 
\ifthenelse{\equal{#6}{1}}
{\psframe[fillstyle=solid,fillcolor=lightgray](#1,#2)(#3,#4)}
{\psframe[fillstyle=solid,fillcolor=white](#1,#2)(#3,#4)}
}

% \eretname {#1 x left of text} {#2 y top of text} {#3 text}
\newcommand {\eretname}[3]
{
%shift down 0.1 for height of text the anchor at baseline (B)
\rput[bl]{0}(0,-0.1){\rput[Bl]{0}(#1,#2){\footnotesize \textit{#3}}}
}

% \errelarm {#1 x0} {#2 y0} {#3 x1} {#4 y1} {#5 ismandatory} {#6 isconstructed}
\newcommand {\errelarm}[6]
{
\ifthenelse{\equal{#6}{1}}
{
%%\psline[linewidth=0.5pt,linearc=.05,linestyle=dashed,dash=6pt 6pt]{-}(#1,#2)(#3,#4)}
\ifthenelse{\equal{#5}{1}}
{\psline[linewidth=1.5pt,linearc=.05,linecolor=lightgray]{-}(#1,#2)(#3,#4)}
{\psline[linewidth=1.5pt,linearc=.05,linecolor=lightgray,linestyle=dashed,dash=2pt 2pt]{-}(#1,#2)(#3,#4)}
}
{
\ifthenelse{\equal{#5}{1}}
{\psline[linewidth=0.9pt,linearc=.05]{-}(#1,#2)(#3,#4)}
{\psline[linewidth=0.9pt,linearc=.05,linestyle=dashed,dash=2pt 2pt]{-}(#1,#2)(#3,#4)}
}
}

% \errelangle {#1 x0} {#2 y0} {#3 x1} {#4 y1} {#5 x2} {#6 y2} {#7 ismandatory} {#8 isocnstructed}
\newcommand {\errelangle}[8]
{
\ifthenelse{\equal{#8}{1}}
{
%\psline[linewidth=0.5pt,linearc=.1,linestyle=dashed,dash=6pt 6pt]{-}(#1,#2)(#3,#4)(#5,#6)}
\ifthenelse{\equal{#7}{1}}
{\psline[linewidth=1.5pt,linearc=.05,linecolor=lightgray]{-}(#1,#2)(#3,#4)(#5,#6)}
{\psline[linewidth=1.5pt,linearc=.1,linecolor=lightgray,linestyle=dashed,dash=2pt 2pt]{-}(#1,#2)(#3,#4)(#5,#6)}
}
{
\ifthenelse{\equal{#7}{1}}
{\psline[linewidth=0.9pt,linearc=.1]{-}(#1,#2)(#3,#4)(#5,#6)}
{\psline[linewidth=0.9pt,linearc=.1,linestyle=dashed,dash=2pt 2pt]{-}(#1,#2)(#3,#4)(#5,#6)}
}
}

% \ercrowfoot {#1 x0} {#2 y0} {#3 x11} {#4 y11} {#5 x12} {#6 y12} {#7 x13} {#8 y13} {#9 isconstructed}
\newcommand {\ercrowfoot}[9]
{
\ifthenelse{\equal{#9}{1}}
{
\psline[linewidth=1.5pt,linearc=.05,linecolor=lightgray]{-}(#1,#2)(#3,#4)
\psline[linewidth=1.5pt,linearc=.05,linecolor=lightgray]{-}(#1,#2)(#5,#6)
\psline[linewidth=1.5pt,linearc=.05,linecolor=lightgray]{-}(#1,#2)(#7,#8)
}{
\psline[linewidth=0.9pt,linearc=.05]{-}(#1,#2)(#3,#4)
\psline[linewidth=0.9pt,linearc=.05]{-}(#1,#2)(#5,#6)
\psline[linewidth=0.9pt,linearc=.05]{-}(#1,#2)(#7,#8)
}
}


% \eridcomprel{#1 x1}{#2 x2}{#3 y1}{#4 ymid}{#5 y2}
\newcommand {\eridcomprel}[5]
{
\psline[linewidth=0.9pt](#1,#3)(#1,#5)
\psline[linewidth=0.9pt](#2,#3)(#2,#5)
\psline[linewidth=0.9pt](#1,#4)(#2,#4)
}

% \eridrefrel{#1 x1}{#2 xmid}{#3 x2}{#4 y1}{#5 y2}
\newcommand {\eridrefrel}[5]
{
\psline[linewidth=0.9pt](#1,#4)(#3,#4)
\psline[linewidth=0.9pt](#1,#5)(#3,#5)
\psline[linewidth=0.9pt](#2,#4)(#2,#5)
}


% \errelname {#1 x} {#2 y} {#3 text}
\newcommand {\errelname}[3]
{
\rput[l]{0}(#1,#2){\textit{#3}}
}
% \errelseq {#1 x} {#2 y}
\newcommand {\erelseq}[2]
{
}
% \erattr {#1 x} {#2 y} {#3 ismandatory}{#4 idenitfying} {#5 text}
\newcommand {\erattr}[5]
{
\ifthenelse{\equal{#3}{1}}
{\rput[l]{0}(#1,#2){{\tiny $\square$} {\footnotesize \textit{\ifthenelse{\equal{#4}{0}}{\underline{#5}}{#5}}}}}
{\rput[l]{0}(#1,#2){\footnotesize $\circ$ \textit{\ifthenelse{\equal{#4}{0}}{\underline{#5}}{#5}}}}
}

%\ifthenelse{\equal{#4}{1}}
% \ertext {#1 x} {#2 y} {#3 text anchor} {#4 text}
%{\rput[l]{0}(#1,#2){\footnotesize $\circ$ \underline{\textit{#5}}}}
\newcommand {\ertext}[4]
{
\rput[B#3]{0}(#1,#2){{\footnotesize #4}}
}
% \erarc {#1 x0} {#2 y0} {#3 x1} {#4 y1} {#5 x2} {#6 y2} {#7 x3} {#8 y3}
\newcommand {\erarc}[8]
{
\psbezier[showpoints=false]{-}(#1,#2) (#3, #4)(#5,#6) (#7, #8)
}

% \erarc {#1 x0} {#2 y0} {#3 x1} {#4 y1} {#5 x2} {#6 y2} {#7 x3} {#8 y3}
\newcommand {\errelseq}[8]
{
\psbezier[showpoints=false]{-}(#1,#2) (#3, #4)(#5,#6) (#7, #8)
}
% \ertrace {#1 trace}   
\newcommand {\ertrace}[1]
{
}

\usepackage{mathtools}
\usepackage{alltt}
\usepackage{mnsymbol}
\usepackage{cmll}
\renewcommand{\ttdefault}{txtt}
\usepackage[font=small]{caption}
\setlength{\captionmargin}{2cm}


\begin{document}
\title{Documenting commutative diagrams of relationships to eliminate sources of redundancy in relational data design - Part One - Methodology.}

% abstract here for ams

\author{John Cartmell}

\maketitle
\begin{center}
DRAFT
\end{center}

\begin{abstract}
\noindent We illustrate a shortcoming in relational data design methodology and propose 
as a remedy the modelling
of certain relationship identities which we call relationship scope constraints.  
Part Two of this paper formulates a mathematical justification for this methodology.

This paper focuses on a practical methodology featuring (i) ER models that distinguish composition relationships from reference relationships (ii) scope constraints as as character commutative diagrams with a particular charcateristice shape. It is conjectured in this paper that from these features alone result
relational models in third normal form. The truth is a little more complicated and is described in detail in the second part of this paper which is
currently titled `A Mathematical Theory of Data'.
\end{abstract}

% comment out following for ams
%\bibliographystyle{hplain} 
\section{Introduction}

E.F.Codd's meta theory, presented as the relational model of data \cite{Codd1970}, is fully formed -- the meta concepts of table, column and primary key are defined as is that of a foreign key enabling one table to cross reference the rows of another. 
His is a theory of \textit{what data is} and this theory has come to underpin the majority of corporate databases.  
Each such database, in accord with Codd's prescriptions, holds a meta-description of its own units of storage -- the tables, columns and keys -- what their names are and how they fit together to enable navigation through the data; this description is the core of what is described as a relational schema. The development of the relational model of data was strongly influenced by the predicate calculus representation of formal logic but arguably 
this meta-mathematics that influenced Codd has been overtaken by  later 20th century meta-mathematics in the form of type theory and category theory; these are more diagrammatic in form and lead not to the relational model of data but to versions of the binary entity relationship model. It is these other meta-mathematical disciplines that influence this paper and lead to meaningful improvements in relational design methodology. Paradoxically, each such improvement in relational design methodology undermines the pre-eminence enjoyed by the relational model.\\

\noindent Codd has described various tests of goodness of a schema applicable, it must be remembered, only with cognisance to the possibilities among the data that it is designed to hold i.e. the intended usage.
In the first instance three tests were described and successively a schema is said to be is 1st normal form, 2nd normal form or 3rd normal form depending on its success in passing the tests. A process for 
fixing deficient schemas is described as normalisation of the schema. Normalisation is therefore
a method for converting or transforming one relational schema into another deemed more suitable for the purpose 
at hand. We can visualise so:
\begin{center}
\begin{tabular}{ p{1.25cm} p{1.25cm}  p{1.25cm}}
relational schema & \parbox[c]{1.25cm}{$normalise$ $\xRightarrow{\hspace*{1.3cm}}$} & relational schema
\end{tabular}
\end{center}

\noindent In formal logical terms a relational schema presents a `theory of what is' and normalisation is the process of improving a theory by (i) tightening the theory to better fit the facts and (ii) removing redundancy from the presentation so that the primitives are appropriate as units of storage.\\

%Something about Bachman, hierarchical and network databases.

\noindent Subsequently, the relations of Codd's model are more abstractly presented, as either entities or as n-ary relationships, in Chen's entity-relationship model of data described in \cite{Chen1976} ;
in the approach of Chen there is emphasis on a diagrammatic representation of the model. 
Chen describes a method for constructing 
a relational schema (in the sense of Codd) from an entity-relationship schema (ER-schema).
He states that normalisation of the relational schema might be required after construction from an ER-schema -- though why this might be is not explained. 
This yields a design process which is a combination of automatic transformation followed by 
normalisation: 

\begin{center}
\begin{tabular}{ p{1.25cm} p{1.5cm} p{1.25cm} p{1.5cm}  p{1.25cm}}
\raisebox{-0.8cm}{\parbox{1.2cm}{er schema}}& \textit{automatic transform} $\xRightarrow{\hspace*{1.5cm}}$ & \raisebox{-0.8cm}{\parbox{1.25cm}{relational schema}} & \textit{manually normalise} $ \xRightarrow{\hspace*{1.5cm}} $ & \raisebox{-0.8cm}{\parbox{1.25cm}{relational schema}} 
\end{tabular}
\end{center}
\vspace{0.2cm}
\noindent This is said by Chen to be a top-down way to develop a relational schema in contrast to the Codd approach which he describes as bottom up.
Note that some authors have mistakenly claimed that the second step within this workflow 
(the normalisation step) will be unnecessary if the ER-schema is itself in normal form. 
We will give a significant example where this is not so and in so doing illustrate the concept
of relationship scope that was  introduced in \cite{Alderson97} but is generally
absent from current day presentations of ER modelling; we show here that we can mobilise this concept of the scope of relationships to narrow the process gap from ER modelling to fully normalised relational schema.\\

\noindent Within the context of software design methodologies, the normalisation step is represented as a manual process; to perform normalisation additional information is required comprising certain meta-knowledge about the facts that are the subject matter of the data. In relational theory, this  additional required knowledge is in part represented 	as a set of integrity constraints comprising functional dependencies (FDs) and inclusion dependencies (INDs). If these integrity constraints are modelled as an enrichment of the relational schema then we can automatically normalise\footnote{this doesn't take account of 4th and 5th normal forms}:

\begin{center}
\begin{tabular}{ p{1.35cm} p{1.55cm}  p{1.25cm}}
\raisebox{-0.8cm}{\parbox{1.5cm}{relational schema  +FDs+INDs}}& \textit{automatically normalise} $\xRightarrow{\hspace*{1.5cm}}$ & \raisebox{-0.8cm}{\parbox{1.25cm}{relational schema}}  \\
\end{tabular}
\end{center}
\vspace{0.2cm}

\noindent But normalisation to defined normal forms is not exactly the endpoint - elimination of reduction is the point. In \cite{Levene2000} it is shown that  if normalisation takes into account 
both FDs and INDs then redundancy is eliminated but this result depends on a definition of redundancy given in the first place in terms of FDs and INDs. We can give a simple
 example in which FDs and INDs do not provide sufficient information about the relational 
schema for the normalisation process to eliminate elementary redundancy in the schema and  not strictly limited to redundancy as defined in \cite{Levene2000}. \\

\noindent The principal goal of this paper is to provide an answer to how the additional constraint knowledge required for normalisation can be represented in ER-terms rather than 
relationally such as in terms of FDs and INDs and to go beyond current relational theory in a methodology for removing elementary redundancy. It is for this we employ the concept of the scope of a relationship; if the notion of an ER-schema is extended by the concept of 
relationship scope then we can automatically convert from the scope enriched ER-schema to a relational schema in normal form. \\

\noindent We propose a design process which follows the left and lower edges of this square: 
\begin{center}
\begin{pspicture}(-3,-2)(7,2)
%\psgrid
\begin{tabular}{ c p{2.5cm}  c}
\Rnode{TL}{\parbox{1.65cm}{ER-schema}}& & \Rnode{TR}{\parbox{1.25cm}{relational schema}} \\[2cm]
\Rnode{BL}{\raisebox{-0.1cm}{\parbox{1.65cm}{scope \mbox{enriched} ER-schema}}} 
& &\Rnode{BR}{\parbox{1.25cm}{relational schema}}
\ncline{->}{TL}{TR}
\Aput{\parbox{1.5cm}{\textit{automatically transform}}}
\ncline{->}{TR}{BR}
\Aput{\textit{normalisation}}
\ncline{->}{TL}{BL}
\Bput{\textit{define scopes}}
\ncline{->}{BL}{BR}
\Bput{\parbox{1.5cm}{\textit{automatically transform}}}
\end{tabular}
\end{pspicture}
\end{center}
\noindent in preference to the upper and right edges. We illustrate that by following this approach sources of redundancy are eliminated. \\



%\noindent Following this approach software development software will in future offer ER modelling of data schemas  inclusive of relational scope and developers will be able to maintain a single diagrammatic data models in their favorite ER notation without needing to independently maintain relational schemas. \\

\noindent After Chen's 1976 paper, coming into and through the 1980's, came the development, concurrently, of Computer Aided Software Engineering (CASE) tools, including Meta-CASE tools, and semi-formalised and, in some instances, standardised official methodologies and notations, supporting structured systems analysis and development. Universally in the methodologies from this time the terms entity and relationship introduced in Chen's paper  were retained within a logical modelling phase and Chen's transformation step into relational database design,
inclusive of a normalisation step, is likewise retained. Though the terms and the overall shape of the process is retained the concepts behind these terms are adjusted. Most noticeably `relationships' are now `binary relationships' and at an early stage in these methodologies many-many relationships are eliminated in favour of many-one relationships. At this point there has been a conceptual \textit{volte face} for a many-one  binary relationship, implementation considerations aside, is a thinly disguised pointer between records of a file, such as in a VSAM file system,  or a link between records in the network database model and it can be conceptualised, abstractly, as a function between sets of like-typed entities - leading some authors to describe a 
functional model of data \cite{Buneman1979},\cite{Shipman1981}. The entity-relationship diagrams of these software analysis  methods and the accompanying CASE tools
that emerged in the 80's bear more resemblance to notation that preceded the work of Codd and Chen such as Bachman's data structure diagrams than to the diagrams of Chen.  
Among the many, and as summarised in  \cite{Rock-Evans1989},  there are three variants of binary entity relationship diagram that stand out, those found, respectively,  in SSADM/Barker-Ellis (now adopted by Oracle), in Clive Finkelstein and James Martin's Information Engineering,  and in IDEF. \\



\noindent 
Chen's paper introduced the idea of entities being dependent on binary relationships 
with others for both their identification and their existence:

\begin{quote}
Theoretically, any kind of relationship may be used to identify entities. For
simplicity, we shall restrict ourselves to the use of only one kind of relationship:
the binary relationships with 1:n mapping in which the existence of the n entities
on one side of the relationship depends on the existence of one entity on the other
side of the relationship. For example, one employee may have n ( = 0, 1, 2, . . .)
dependants, and the existence of the dependants depends on the existence of the
corresponding employee.
This method of identification of entities by relationships with other entities can
be applied recursively until the entities which can be identified by their own attribute
values are reached. For example, the primary key of a department in a
company may consist of the department number and the primary key of the
division, which in turn consists of the division number and the name of the company.
\end{quote}

\noindent In many cases, software methodologies and supporting CASE tools introduced an intermediate step between the ER model and the relational model naming the intermediary model the physical design model and the starting model the logical
model. This shifted the problem slightly but didn't make it go away. I shall call such an automatic transformation 
between logical and physical models the Chen transform. It is described in section \ref{ChensTransformation}.  
\begin{center}
\setlength{\tabcolsep}{2pt}
\begin{tabular}{ p{1.4cm}  p{2.2cm}  p{1.5cm} p{1.5cm} p{1.5cm} p{1.5cm}  p{1.25cm}}
\raisebox{-0.8cm}{\parbox{1.4cm}{logical er~schema}}& \textit{Chen~transform (automatic)} $\xRightarrow{\hspace*{1.75cm}}$ &
\raisebox{-0.8cm}{\parbox{1.4cm}{physical er schema}}& \textit{manually normalise} $\xRightarrow{\hspace*{1.5cm}}$ &
\raisebox{-0.8cm}{\parbox{1.4cm}{physical er schema}}& \textit{code generate} $\xRightarrow{\hspace*{1.5cm}}$ &  \raisebox{-0.8cm}{\parbox{1.25cm}{relational schema}} 
\end{tabular}
\end{center}
\vspace{0.2cm}

\noindent In the mathematical description in Part Two we shall present a general definition of ER-schema which is general enough to include both purely logical schemas that are  to the left of this diagram and the physical schemas to the right. We shall define the term ER model to mean an ER schema and all its intended usages and we shall show that by revising the definition of the Chen transform we can show that for each well-formulated purely logical schema there is a corresponding relational schema in normal form. In this paper we are more concerned with illustration of this as a methodology. \\ 

\noindent Following PCTE\footnote{ 
}\cite{Boudier1988},\cite{ECMA-149}, 
we use the term \textit{composition relationship} for Chen's \textit{binary relationships with 1:n mapping in which the existence of the n entities
on one side ... depends on the existence of one entity on the other side} and we use the term \textit{reference relationship} for binary relationships which are neither composition relationships nor their inverses. We shall also describe the inverses of composition relationships as being \textit{dependency relationships}. Earlier than this a similar distinction 
had been made by the designers of the CAIS specification \cite{Oberndorf88} but in which the two kinds of relationship were distinguished as primary and secondary - their rationale for the distinction \cite{Munck1988} was as follows: \textit{[Entities] and relationships may form a general graph or bowl of spaghetti. However, this raises various practical
problems of deletion and garbage collection, longterm
naming, and unconnected sub-graphs. CAIS
therefore designates certain relationships as primary
(and all others as secondary) and requires that all [Entities]
and primary relationships in the data base form a single
tree structure}. This distinction between composition and reference made by both CAIS and then PCTE served  
the goal of modelling computer file systems within a database framework, see figure \ref{filesystem2} for example. 
\erboxedFigPicture{filesystem2}{h}{An ER model of folder system modelling the hierarchical structure as a recursive composition relationship and shortcuts as reference relationships.} 
\noindent In this paper we shall not assume that all composition relationships are identifying nor, vice-versa, that only composition relationships may be identifying. 

\noindent In this paper to depict ER-schemas we use a version of the Barker-Ellis notation. Figure \ref{entityRelationalMetaModel1} is a meta-model of this notation -- it is an ER schema describing ER schemas.
% 
\begin{erbulletedDimFig}{scriptOfPlay}{h}{Composition and reference within the script of a play}{4}{5}
\item{a \underline{play} is composed of one or more \underline{spoken line}s}
\item{a \underline{play} is composed of one or more \underline{character}s}
\item{a \underline{spoken line} is assigned to exactly one \underline{character}}
\item{each line of a play is assigned to a character of that same play}
\end{erbulletedDimFig} 
%

\noindent In cases where we wish to distinguish composition relationships from reference relationships then we draw the diagram top down: an anonymous root entity type (the `absolute') is introduced at the top of
 the diagram, relationships leaving the lower edges of boxes are composition relationships and they always meet the top edge of the box representing the dependent type, reference relationships meet boxes from one side or the other. We note that there is a structural resemblance to diagrams drawn by Bachman\cite{Bachman1973}. 

\erboxedFigPicture{relationalMetaModel2}{h}{An ER model of Relational Schema. This is a logical ER model and so there is no indication as to how the relationships are implemented.} 

The example in figure \ref{scriptOfPlay} shows two composition relationships (which have been left unlabelled) and one reference relationship doubly labelled. See figure \ref{relationalMetaModel2} for another example of the notation. \\

\erboxedFigPicture{entityRelationalMetaModel1}{h}{The logical ER meta-model. This is a logical ER model of a logical ER model. }

\noindent It has often been noted that there is a disparity, and therefore a conflict,
between uses of the term `model' in mathematical logic and use of the same term in other disciplines including database theory, for in database theory and other disciplines,  models are `theories of what is' - they are models of conceptual situations and the term is synonymous with `theory', whereas in 
mathematical logic the term model is used
to mean  an instantiation or an interpretation of such a theory. In the database sense of the word,  models are represented in database schemas; to avoid
ambiguity, we will choose to use the term schema synonymously with database model whilst remembering that such a thing is also a theory. We will refer to different kinds of 
schema as relational schema or ER-schema. 
By ER-schema, unless stated otherwise, we will mean not the ER-schema in the 
sense of Chen but a schema of entity types, 
binary relationships and attributes as meta-modelled in figure \ref{entityRelationalMetaModel1}.\\

\erplainFig{entityRelationalMetaModelRelationshipImplementation}{h}{The Physical Entity Relational Meta Model. This is a logical ER model of a physical ER model -- in the physical model referential attributes are introduced to represent the implementation of relationships.}

\noindent 
In the terminology of Ellis\cite{ellis1982}, wherever in an entity model there is a path of single-valued relationships 
$\overset{r_1}{a \rightpitchfork \hspace{-0.35em} -  \cdot} \overset{r_2}{\rightpitchfork \hspace{-0.35em} -} \cdot ... \overset{r_n}{\rightpitchfork \hspace{-0.35em} -} b$
then the destination entity type $b$ is said to be in the \textit{logical horizon}  of the source entity type $a$. In programming, equivalently, 
we might say that it was possible to navigate from one to the other. Now if there are two such navigation paths between entity type $a$ (the source) and entity type $b$ (the destination) then a question naturally arises as to whether following one path is equivalent to
following the other i.e whether starting at any entity of type $a$ we arrive at the same destination entity of type $b$ regardless of which of the two paths we follow. In an abstract mathematical setting, diagrams showing such equivalent
paths are said to be \textit{commutative diagrams} and methods of reasoning using such diagrams is the starting point of category theory. 
\noindent Johnson and Dampney \cite{Johnson93} have emphasised the 
importance of recognising such commutative diagrams of 
relationships during entity modelling; 
in summary, there are identities between joins of derived 
relationships and these are important
and should be documented during the construction of an entity model. 
Johnson, Dampney and Wood in \cite{Johnson2002ERA} formulate a description of 
ER model that goes beyond the view of an ER schema as a directed graph 
by addition of constraints including commutative diagrams, cartesian products and 
pullbacks by defining an ER schema as a presentation of category with 
finite limits and colimits.  
A similar definition of a data model specification is given by Piessens and Steegman \cite{piessens1995}.
In a further paper, Johnson and Roseburgh \cite{Johnson2002REL} show the 
relationship between their formulation of ER models and relational models. 
These descriptions written in the style of abstract mathematics call for extensions to the notation employed by entity modellers so that ER schemas can be more expressive. 

\noindent  Shlaer and Lang in \cite{Shlaer96} describe alternative paths between two entity types as relationship loops and when they are equivalent say that there are dependencies between the relationships.  
Kolp and Zimnyi (\cite{Kolp1995}) instead use the term relationship cycle and identify them as a source of 
superfluous attributes in the transformation from ER model to relational model. They say: \textit{ER cycles can be sources of 
superfluous attributes not detected by classical normalization. Hence, the interest of enhanced ER-based design methodologies
that remove anomalies due to cycles and inclusion constraints.}

\noindent  See figure \ref{shlaerMellorDeptStudentProfessor2} for a notation proposed in\cite{Shlaer96}  for
the expression of relationship dependencies. This notation enables certain commuting diagrams to be indicated on physical entity relationship diagrams.

\erboxedFigPicture{shlaerMellorDeptStudentProfessor2}{h}{An example of a physical ER model. Physical ER models include attributes for the implementation of relationships and in \cite{Shlaer96} are annotated with the names of the relationships which they implement. If students are advised by professors within the same department then a single departmentId attribute of the student entity is used as the implementation of both R2 and R3 instead of the two shown. }

\section {Relationship Scope}

\noindent With reference to points (i) to (iv) of figure \ref{scriptOfPlay}, point (iv), unlike points (i), (ii) and (iii), expresses information not otherwise represented on the accompanying diagram. 
In the terminology that we introduce here it expresses a relationship scope constraint. This concept is significant to understanding the domain of discourse given by a model.
It is significant in this case that the reference relationships is limited in scope - we can say that the
instances of the reference relationship shown in figure \ref{scriptOfPlay} are local to the context of individual plays: to spell out what is meant by this -- a line of one play is never assigned to a character of a \textit{different} play -- we might summarise this by saying that the relationship of line assignment is intra-play not inter-play. Most significantly, whilst the ER diagram notation is able to express the types and cardinalities of reference relationships and how they are articulated, it is unable to express such a constraint as this one in point(iv) of figure \ref{scriptOfPlay} -- 
such a constraint as this we shall say is a scope constraint for the relationship.  
The thesis here is that every reference relationship has a scope, the scope is fundamentally important in data modelling,  it can be expressed in words, or, as we see later, in equations between expressible relationships or in a commutative diagram called a scope diagram\footnote{Scope diagrams, as we describe them here, are akin to the commutative diagrams used in category theory to express identities between differently composed morphisms however because relationships can be optional, they are not commutative diagrams but 2-cells in the 2-category of finite sets, partial functions and inclusions.} and present, in the terminology of 
\cite{Shlaer96}, a dependency between the reference relationship and certain other relationships. \\

\noindent Contrast the situation of figure {\ref{scriptOfPlay}} with that of figure \ref{languageBookTranslation}. There is a similarity in shape  but behind this there is a significant difference in that the reference relationship in figure \ref{languageBookTranslation} is \emph{not} limited in scope. Rather the point of the relationship \emph{translation} is to cross languages - it establishes an inter-language relationship rather than intra-language relationship. This again shows that the scope of a relationship - the extent to which it is global or local - `inter' or `intra' - cannot be deduced from the entity relationship diagram alone. Other means of expression must be used. Gaining an understanding of scope and the means of its expression is an important part of learning entity modelling. \\

\begin{erbulletedDimFig}{languageBookTranslation}{h}{Translations of a book}{4}{5}
\item{a \underline{language} has one or more \underline{native book}s}
\item{a \underline{language} has one or more \underline{translated book}s}
\item{a \underline{translated book} is a translation of exactly one \underline{native book}}
\item{a \underline{native book} is translated as  zero,one or more \underline{translated book}s}
\end{erbulletedDimFig} 


\noindent Knowledge of a relationship's scope is a significant part of understanding how a relationship is used and failure 
to respect this aspect of proper usage is what constitutes a scope violation. When talking about the cast members of a performance of King Lear it would be a scope error to suggest that a 
member of cast play the character Desdemona for Desdemona is a character within the scope of different play namely Othello.  
The type of entity is correct for `Desdemona' is indeed a  `character' of a play, but the context is not. 
It is part of our knowledge of the `plays part of'/`part played by' relationship (see figure \ref{performanceOfPlay}) that this is a 
relationship whose scope is local to the enclosing `play' context. We can say that it is intra-play rather than inter-play. \\

\noindent It would be a scope error in a conversation about antipodeans to assert that the New Zealand born physicist Ernest Rutherford could have been 
a native of Nelson in Lancashire, a place just 20 miles away from where he was Professor of Physics. 
Not only would it be a factual error -- it would be non-nonsensical and 
this is because of the grasp we have of the meta-relationship
between relationships `country of birth' and `place of birth' -- that the latter is a more detailed version of the former. 
The assertion would violate the scope of the `place of birth' relationship (see figure \ref{townOfBirth}).\\

\erboxedFigPicture{performanceOfPlay}{h}{Model of the Performance of a Play - the relationship `plays part of' is dominated by the relationship `performance of'.}

\noindent Likewise it would be a scope error to think that a local telephone call could be made between different countries or that the hydrogen of a water molecule could be covalently bonded to the oxygen of a \textit{different} molecule or that the captain of one team in a cricket match might be scheduled to bat for the opposing team. All of these errors are characterised as failures to respect the scopes of relationships. \\


\erboxedFigPicture{townOfBirth}{h}{The relationship `place of birth' is dominated by `country of birth'.}

\noindent In accord with a fundamental principle of information theory the more constrained in scope a relationship is then the less the information needed to express its individual instances. From this it follows that the definition of relationship scopes is intimately connected to specifying information requirements for representing or communicating relationship instances and, in particular,  for representing them in databases, relational or otherwise. For example if we wish to propose that, in the context of a performance of King Lear, we  give someone the role of `the fool' then we do not have to say `the fool in King Lear' we simply have to say `the fool' for our shared knowledge of the scope of the relationship `plays the part of'/`part played by' implies the rest. \\

\noindent In the final section of this paper we shall see how these observations translate into relational data design.

\subsection{Diagrams Expressing Scopes}

\noindent In hierarchical ER-schemas, relationships are classified as being either `composition' relationships or `reference' relationships. Of these, it is the reference relationships that have scopes
whereas the composition relationships enable definition of scopes by modelling nested localities i.e. possible contexts. 
In mathematical notation it is possible to include the scope constraint as a more general kind of type constraint than can be expressed in an entity model, namely a `dependent type constraint'. In entity modelling this is not possible and every relationship defined in a model should have a scope constraint specified for it. There is no standard way of doing this but an accompanying diagram, one per associative relationship is a satisfactory way of doing so. Figure \ref{scopePlaysPartOf} is an example of such a diagram. In fact it is a pullback diagram and it is a typical instance of such occurring very naturally within a database schema.   \\

\begin{ernotedDimFig}{scopePlaysPartOf}{h}{A scope constraint diagram for relationship `plays part of'. In such a diagram it is the lower (horizontal) relationship which is the subject of the constraint. 
We have given an alternative explanation of the constraint to the right of the diagram but it is wordy and difficult to follow - for this reason we prefer to draw the diagram and have this mean the very same thing.}{2.2}{4}
Whenever a cast member plays the part of a character then the performance the cast member is part of is a performance of the play the character is a part of.
\end{ernotedDimFig} 


\noindent Similarly the diagram in figure \ref{scopeLineAssignedTo}  can be interpreted as the scope constraints for relationship `assigned to' within the context of the entity model of figure \ref{scriptOfPlay}. The text on the right of the figure explains the constraint expressed by the scope diagram - what it says seems obvious but this is so only if we \emph{know} this model and this relationship i.e. providing we understand its proper usage. In this example shown in figure \ref{scopeLineAssignedTo} there is a single entity type at the top of the diagram - therefore we call the diagram a scope triangle rather than a scope square. \noindent If we allow of the use of identity relationship in a scope square and allow it to be drawn horizontally then any scope triangle can be re-expressed as a scope square as illustrated by the redrafting of figure \ref{scopeLineAssignedTo} as figure \ref{scopeSquareLineAssignedTo}. \\

\begin{ernotedDimFig}{scopeLineAssignedTo}{h}{A scope constraint diagram for relationship `assigned to'. This is a scope triangle because rooted at a single entity type.}{2.0}{4}
Whenever a line of a play is assigned to a character then the play the character is part of is the same play as the line is part of.
\end{ernotedDimFig} 


\noindent Some relationships may be unconstrained in their scope in the sense that they are global in their reach. We have given an example of such a relationship, `translation of', in figure \ref{languageBookTranslation}. The scope of this relationship, the fact that it is unconstrained, is expressed by the relationship scope diagram in figure 
\ref{scopeTranslationOf}. 
By way of explanation - what this diagram says is :
\begin {quote}
The absolute of the language of a translated book is the absolute of the language of the native book it is a translation of.
\end{quote} 

\noindent In other words it says that the relationship `translation of' is such that two absolutes are equal - which is to say nothing at all about the relationship because \emph{a priori} all absolutes are equal since absolute is unique of its type.

\erboxedFigPicture{scopeSquareLineAssignedTo}{h}{The identity relationship `same as' used to express the scope triangle of figure \ref{scopeLineAssignedTo} as a square.}

\erboxedFigPicture{scopeTranslationOf}{h}{An example of a scope diagram for a relationship which is unconstrained in scope - the scope diagram is rooted at absolute.  }

\section{Chen's Transformation}
\label{ChensTransformation}

\noindent Chen presents the transformation process from ER to relational by way of an example. 
He gives an example ER model and proceeds to say that from it that certain relations can `\textit{easily be derived}'\footnote{The verb \textit{migrate} is often used in descriptions of this process; for example I found a Wikipedia article describing a foreign key as a key that had migrated to another entity and I found a description elsewhere stating:
\begin{enumerate}
\item {Identify and define the primary key attributes for each entity}
\item {Validate primary keys and relationships}
\item {Migrate the primary keys to establish foreign keys}
\end{enumerate} The term `migrate' is inappropriate because key columns do not migrate anywhere - they stay where they are - what happens is that for each primary key column and for each relationship a corresponding foreign key column is instantiated.
}. \\


\noindent In terms of the binary ER model the transformation illustrated by Chen can be summarised thus:
\begin{enumerate} [I]
	\item For each entity type on the diagram, a table is instantiated to represent the entity type.
  \item For each attribute of each entity type, a column is instantiated within the table
	      instantiated to represent the entity type. Specifically \textit{identifying}
				attributes are instantiated as primary key columns.
  \item For all identifying relationships,
	      primary key columns of the table representing the source of the relationship
				are instantiated --
				one per primary key column of the table representing the destination entity type.
  \item For all non-identifying relationships, columns of the table representing the
	      source entity type of the
	      relationship are instantiated one per primary key column of the 
				table representing the destination entity type.
\end{enumerate}

\noindent Another way of looking at the matter, rather than speaking of cascading and migrating keys, is based simply on the observation that the columns in the physical representation on an entity type $a$ correspond to the attributes of the entity type $a$ union the set of tuples $\langle r_1,...r_n, p \rangle$ where $n \geq 1$ and where
$\overset{r_1}{a \rightpitchfork \hspace{-0.35em} -  \cdot} \overset{r_2}{\rightpitchfork \hspace{-0.35em} -} \cdot ... \overset{r_n}{\rightpitchfork \hspace{-0.35em} -} b$ is a path of single-valued relationships, where 
$r_i$ is identifying for each $i > 1$ and where $p$ is an identifying attribute of the destination entity type $b$ of the
relationship $r_n$. This observation suggests a formal mathematical definition of the Chen transform and this is the approach we follow in Part Two. \\



\section{The Shortcomings of the Simple Chen Transformation}
\subsection{The Relational Meta Schema Example}
\noindent Now let us apply this Chen transformation (I)--(V) to the meta-schema for relational databases
(figure \ref{relationalMetaModel2}).


\noindent First some background. Since the scheme is data (sometimes said to be meta-data since it is data about the structure of data)
then this schema can be held in a database (indeed, this was prescribed in articles by Codd in what became known as Codd's rules). The tables, columns and keys used to hold this data themselves have a description which itself is a schema. We shall refer to this here as the relational meta-schema. 
Different software vendors represent this relational meta-schema in different ways. One example which can be found 
online \cite{MySQLRefMan} is for the MYSQL imno database engine.\\ 

\noindent An abstract representation of the relational meta-schema using the entity relationship notation we have already seen in figure \ref{relationalMetaModel2}. \\ 

\noindent Applying the Chen transformation to the model in figure \ref{relationalMetaModel2} we get:


{\addtolength{\leftskip}{5mm}
\begin{minipage}{\dimexpr\textwidth-1.5cm}
\noindent
\texttt{TABLE(\underline{TABLE-NAME}) \\
COLUMN(\underline{TABLE-NAME},\underline{COLUMN-NAME})       \\
PRIMARY-KEY-COLUMN(\underline{TABLE-NAME},\underline{INDEX-NO},IS-TABLE-NAME,IS-COLUMN-NAME)      \\
FOREIGN-KEY(\underline{TABLE-NAME},\underline{NAME},TO-TABLE-NAME)    \\
FOREIGN-KEY-COLUMN(\underline{TABLE-NAME},\underline{FOREIGN-KEY-NAME},\underline{INDEX-NO},
          IS-TABLE-NAME, IS-COLUMN-NAME, TO-TABLE-NAME, TO-COLUMN-NAME) \\
}
\end{minipage}
}

\noindent We see that the generated table definitions have additional columns in two tables and these tables are not in normal form (for description of normal forms see \cite{Kent1983}.)
The generated definitions (relations) that we have obtained are not in normal form:

\begin{itemize}
\item {PRIMARY-KEY-COLUMN is not in 1st normal form because the column 'IS-TABLE-NAME' is a duplicate column since its value in all cases will
be identical to those of the TABLE-NAME column. We need remove the IS-TABLE-NAME column to remove the redundancy and to obtain a definition satisfying the 1st-normal form rule.}
\item {The FOREIGN-KEY-COLUMN is not in 1st-normal form for, again, the column 'IS-TABLE-NAME' is a duplicate whose values are the same as those of TABLE-NAME. Even with this removed 
the table is not in 2nd normal form for the column TO-TABLE-NAME is redundant since its value in all cases can be obtained from the parent FOREIGN-KEY entity. We can remove the TO-TABLE-NAME column. Note that in this example when we
normalise the table we pay for it the losing the possibility of expressing the referentiality constraint representing the 'TO' relationship\footnote{This is surprising - as defined by Codd and as implemented in modern relational databases not all binary relationships can have
matching foreign key constraints.}.}
\end{itemize}

\subsection{What's Gone Wrong?}
We might wonder whether something is wrong with out ER-model 
(figure \ref{relationalMetaModel2}) but that is not the case.  What is wrong is that 
the transformation has not taken account of certain commutative diagrams  among the primitive relationships of the ER model. There are identities among joins of the primitive relationships and these are the causes of the redundancies. 
There are three commutative diagrams in all that are the source of the problem. One of these is shown in  figure
\ref{relationalMetaModelIdentities}. All of the three are indicated in figure \ref{relationalMetaModel3.tightlayout} using 
scope annotations for each of the reference relationships. In each scope annotation the commutative diagram is expressed as an identity 
between two paths. The relationship which is the subject of the scope constraint is denoted by a tilde symbol (\texttildelow).

\begin{figure}[h]
\begin{center}
\begin{erdiagram}{3.7}{5.9084625}

\erettop{2.4}{-0.7}{4.3}{-0.1}\ertext{3.35}{-0.45}{}{table}
\eretml{0.792}{-2.2}{2.584}{-1.6}\ertext{1.688}{-1.95}{}{foreign key}
\eretbl{0.301}{-3.7}{3.075}{-3.1}\ertext{1.688}{-3.45}{}{foreign key column}
\eretbr{4.575}{-3.7}{5.908}{-3.1}\ertext{5.242}{-3.45}{}{column}

% relationship 
\ertext{3.183}{-1}{l}{}\ertext{1.538}{-1.45}{r}{of}\errelarm{3.033}{-0.7}{3.033}{-0.775}{0}{0}\errelarm{1.688}{-1.388}{1.688}{-1.6}{1}{0}\errelangle{3.033}{-0.775}{3.033}{-0.85}{2.361}{-1.013}{0}{0}\errelangle{2.361}{-1.013}{1.688}{-1.175}{1.688}{-1.388}{1}{0}\eridcomprel{1.5879124999999996}{1.7879124999999998}{-1.3499999999999999}\ercrowfoot{1.688}{-1.45}{1.538}{-1.6}{1.688}{-1.6}{1.838}{-1.6}{0}
% relationship 
\ertext{3.817}{-1}{l}{}\ertext{5.392}{-2.95}{l}{of}\errelarm{3.667}{-0.7}{3.667}{-0.775}{1}{0}\errelarm{5.242}{-2.35}{5.242}{-3.1}{1}{0}\errelangle{3.667}{-0.775}{3.667}{-0.85}{4.454}{-1.225}{1}{0}\errelangle{4.454}{-1.225}{5.242}{-1.6}{5.242}{-2.35}{1}{0}\eridcomprel{5.1418124999999995}{5.341812499999999}{-2.8499999999999996}\ercrowfoot{5.242}{-2.95}{5.092}{-3.1}{5.242}{-3.1}{5.392}{-3.1}{0}
% relationship 
\ertext{1.838}{-2.5}{l}{}\ertext{1.538}{-2.95}{r}{of}\ertext{1.538}{-2.65}{r}{part}\errelarm{1.688}{-2.2}{1.688}{-2.65}{1}{0}\errelarm{1.688}{-2.65}{1.688}{-3.1}{1}{0}\eridcomprel{1.5879124999999996}{1.7879124999999998}{-2.8499999999999996}\ercrowfoot{1.688}{-2.95}{1.538}{-3.1}{1.688}{-3.1}{1.838}{-3.1}{0}
% relationship is
\ertext{3.225}{-3.25}{l}{is}\errelarm{3.075}{-3.4}{3.325}{-3.4}{1}{0}\errelarm{4.325}{-3.4}{4.575}{-3.4}{0}{0}\errelarm{3.325}{-3.4}{3.825}{-3.4}{1}{0}\errelarm{3.825}{-3.4}{4.325}{-3.4}{0}{0}\ercrowfoot{3.225}{-3.4}{3.075}{-3.25}{3.075}{-3.4}{3.075}{-3.55}{0}
\end{erdiagram}
 
\end{center}
\caption{An identity among primitive relationships of the relational meta-model is given
by the scope of the `is' relationship -- the cause of a redundant copy of the IS-TABLE-NAME column. Algebraically: $\text{is}/\text{of}=\text{part of}/\text{of}$. For pragmatic diagramming reasons if we use such an equation in the context of a particular relationship as a scope
constraint on a diagram then the name of the subject relationship is replaced by a tilde symbol, to get, for example, 
$\text{\texttildelow}/\text{of}=\text{part of}/\text{of}$
}
\label{relationalMetaModelIdentities}
\end{figure}
\erplainFig{relationalMetaModel3.tightlayout}{h}{Logical ER model of the Relational Meta Model - showing relationship scope constraints in which tilde(\texttildelow)  denotes the relationship being scoped and the hat symbol ($\ \hat{ }\ $) denotes the absolute. The annotation 
$
\text{\texttildelow}/\ \hat{\ }=\hat{\ }
$
denotes a relationship of global scope.}

\noindent We see from this that the general algorithm sketched above  needs to be modified to take account of the scopes of relationships for we can see how the relational identities expressed in the scope constraints lead to the duplicate columns that have to be removed to eliminate redundancy and achieve 1st and 2nd normal form. 
\noindent We need the Chen transform to instead yield the physical ER model shown in figure \ref{relationalMetaModel3.rdb}.  It is clear now that to achieve this we need to start from a logical model that has additional scope constraints documented, as illustrated in figure \ref{relationalMetaModel3.tightlayout}.

\erplainFig{relationalMetaModel3.relational}{h}{Physical ER model of the Relational Meta Model - showing how a revised algorithm generates it from previous figure.}


\section{Conclusion}
Diagrams of relationships can be used to express integrity constraints for binary entity relationship models of data. They can be incorporated into methodologies for representing the scopes of reference relationships in top-down style Barker-Ellis ER-models. ER-models enriched in this way can be automatically transformed into relational schemas with higher degrees of normalisation (i.e lower levels of redundancy) than exhibited by previous methodologies. 

\noindent Part Two of this work is a mathematically precise description and justification for this approach.

	%\bibliographystyle{ACM-Reference-Format-Journals}
	%\bibliographystyle{hplain} 
\bibliographystyle{alpha} 
\bibliography{../SharedBibliography/temp/bibliography}
	 
\end{document}
