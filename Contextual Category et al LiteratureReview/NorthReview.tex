\documentclass[10pt,a4paper]{scrartcl}
\usepackage{framed}
\newcommand{\site}[1]{\textit{#1}}
\newenvironment{comment}[1]
{\begin{framed}
\textbf{#1}
}
{
\end{framed}
}
\begin{document}
\noindent
{\Large
Comments on Ahrens, Emmenegger, North \& Rijke, \\
`Algebraic Presentations of Dependent Type Theories',\\
\verb'arXiv:2111.09948v2 [math.CT] 20 Sep 2022'
}
 \small
\begin{comment}{Page 1, Abstract, First Sentence.}
This isn't true. Your first sentence in section 3.1  is closer to the truth.
Might you  be better paraphrasing Cartmell, 1986, para 3 and writing something along the lines of:
C-systems were defined by Cartmell as the algebraic structures that correspond exactly to generalised algebraic theories. 
\end{comment}

\begin{comment}{Page 3, Section 1.1, Para 1, First sentence. }
I disagree with this statement regarding the nature of initial semantics. I think it misleading. 
\end{comment}

\begin{comment}{Page 3. Section 1.1, Para 2.}
This para. purports to describe an example of an initial semantics and therefore in the presence of sentence 1 of the preceeding para 1 should be an example of a `rigorous specification of a syntax'. It is not. There is no syntax specified here. 
\end{comment} 

\begin{comment}{Page 4, para 4, item 5.}
Where is the notion of a convertibility relation defined?
\end{comment}

\begin{comment}{Page 4, para 6.} 
Where is the notion of a binding signature defined?
\end{comment}

\begin{comment}{Page 4, Example 1.1.}
I cannot understand this example at all. What is the syntax used here (superficially it is BNF but it is not consistently so)? In what way does this define a signature?
\end{comment} 

\begin{comment}{Page 5, first para.}
I really don't understand the definition given here of the monad $T$. 
I don't understand the right hand side of the equation that describes the mapping of a pair of sets $X$, $Y$. 
\end{comment} 

\begin{comment}{Page 6, para 1, item 4.}
I am having trouble understanding this description of the pullback operation. Can you help me?
\end{comment} 

\begin{comment}{Page 6, Section 1.2.2, penultimate para}

After the inset para that quotes Voevodsky would it be worth adding a a further para in which it is observed that B-systems are to C-systems as abstract clones are to Lawvere theories?
\end{comment} 

\begin{comment}{Page 6, section 1.3, para 1.}
Replace the text \textit{mathematical notion capturing}
by the text
\textit{appropriate mathematical structure that captures}.
Reason: types theories are themselves mathematical.
\end{comment} 


\begin{comment}{Page 6, section 1.3.1, para 2, first sentence.}
Regarding the text \textit{Voevodsky defined C-systems as a mild variant of contextual categories} I think this might be a bit confusing. 
What needs to be made clear is that V. rejected the name and formulated an alternative but equivalent definition.
(There is no variant structure defined by V. --- there is a different definition of the same structure.)
\end{comment} 

\begin{comment}{Page 6, para 2, sentence 2.}
I am not so familiar with the term \textit{stanble under pullback}
is that standard terminology? Is it any clearer to write
\textit{closed under pullback along arbitrary morphisms}?
\end{comment}

\begin{comment}{Page 6, para 2.}
Regarding Voevodsky's rationale for renaming contextual categories and appros of nothing: doesn't this mean strict categories shouldn't be called strict categories and range categories shouldn't be called range categories? Just asking!
\end{comment}

\begin{comment}{Page 6,para 2, final two sentences.}
Might this be better as a footnote?
\end{comment}  

\begin{comment}{Page 7, section 1.3.2, final para.}
Regarding these three notions you say "very similar"  which is hedging your bets a bit which I think is appropriate. Maybe add a sentence \textit{The intention is that these are all equivalent notions of structure.} 
Or maybe \textit{These are claimed to be equivalent structures.} or
such.
\end{comment} 

\begin{comment}{Page 7, remark 1.2, sentence 3.}
Much as I like Garner's paper and I might be missing something here but I don't believe that it does look like it could be upgraded to prove the equivalence of B-systems and C-systems.
\end{comment} 

\begin{comment}{Page 7, remark 1.2, sentence 4.}
Which efforts of Cartmell?
\end{comment} 

\begin{comment}{Page 7, penultimate sentence.}
I really don't get this. Syntax is syntax and structure is structure.  The two exist together and in harmony. Without true syntax how are you going to define, say, the theory of categories?  You will need to define it as a  C-system  directly without recourse to the syntax of GATs or the like.   It is maybe irrelevant here but there is another distinction that can be made  -- the distinction between a theory and a presentation of a theory.   In Cartmells work GATs are actually defined as presentations and in many cases multiple different presentations will be thought of as presentations of the same theory. It is hard to define a theory directly rather than via one or other presentations of the theory. C-systems correspond to theories themselves  rather than presentations of theories.
\end{comment} 

\begin{comment}{Page 8, section 1.4, final para, sentence 3.}
Replace the text \textit{itself a mild varinat of} by 
\textit{which as noted above is equivalent to}. 
\end{comment} 

\begin{comment}{Page 8, section 1.4, final para, penultimate sentence.}
It says in this penultimate sentence that you give a variant of Voevodsky's definition of B-system. Is it blindingly obvious that your variant definition is equivalent to Voevodsky's definition?
\end{comment} 

\begin{comment}{Page 8, section 1.4.2, para 2, penultimate sentence.}
Replace the word \textit{gadgets}
by the word \textit{structures}.
\end{comment} 

\begin{comment}{Page 8, section 1.4.2, para 2, penultimate sentence.}
Would I be right in thinking that Voevodsky wouldn't approve of the name `strict category' for the very same reason that he objected to the name `contextual category'?
\end{comment} 

\begin{comment}{Page 12, section 3, para 2, first sentence.}
Remove the text \textit{, itself a mild variant of Cartmell's contextual categories}. It is misleading (as commented above) and anyway this has been said twice already  (section 1.3.1 para 2 and section 1.4 final para) . 
\end{comment} 

\begin{comment}{Page 12, section 3.1, para 1, first sentence.}
Replace the word `gadgets' by `structures'.
They are not gadgets. 
\end{comment} 

\begin{comment}{Page 13, second para.}
This paragraph that commences with the text \textit{Intuitively an object $\Gamma$...} is a reflection on the relationship between  C-systems (structures) and  the syntax of types and terms.  It is not very well considered in my opinion and it is not expressed as clearly as it might be. For example it can equally  be said that the objects of a C-system are types and that therefore structurally there is no difference between types and contexts. Therefore types do appear "explicitly" in C-systems. It is in syntax that there is a distinction between type and context -- in pure structure this distinction disappears -- it is unnecssary. Pure structure, in my opinion, gives us a deeper understanding (an insight) into the  dual nature of type and context.

Rather than expecting the reader to be able figure  out what "explicitly given" means  I suggest to remove the observations regarding what is or is not "explicitly given" or to find a clearer way of making the observation you wish to make.
\end{comment}

\begin{comment}{Page 13, third para.}
Delete this para that commences \textit{The length function} -- it doesn't help the reader --- it throws her off the scent. It is a red herring to  say that the length function doesn't lift to a stratification. Why should it? Why would we want it to? What has it got to do with the price of fish?
\end{comment} 

\begin{comment}{Page 15, definition 3.10, the pullback square}
Why the double headed arrow here? I am not familiar with this notation.
\end{comment} 

\begin{comment}{Page 16, para 4, second sentence.}
Regarding this sentence that starts with the words \textit{To provide some intuition} --
is it common usage to speak of a category being spanned by a subset of morphisms as you do here? If not then need to reword because cannot expect the reader to know what is meant by this. 
\end{comment} 

\begin{comment}{Page 15.}
Can you give any insight into the name CE-system. Is it `C-system extended'? 
\end{comment} 

\begin{comment}{Page 16.}
The text \textit{We now give two examples of a C-system} is followed by just one single example!
\end{comment} 

\begin{comment}{Page 17.}
To show that the concept of CE-system is worthwhile need examples of CE-systems that are not stratified i.e. are not simply C-systems differently axiomatised. Otherwise it might be simpler for the reader if CE-systems are stratified. 
\end{comment} 

\begin{comment}{page 21, definition 4.1}
Can you give any insight into the choice of name``f''?
Is this short for something?
Is it read out loud as ``eff-tee'' or as ``foot''?
\end{comment} 
\begin{comment}{page 21, definition 4.1}
Can you give me any insight into the choice of name $\partial$?
Is it read out loud as ``curly-dee''?
\end{comment} 

\begin{comment}{page 21, definition 4.1}
I think it might be a good idea to add subscripts to the curly-ds and ft in the diagram above and to note around here somewhere that subsequently you omit the subscripts. 
\end{comment} 
\begin{comment}{page 21, definition 4.1}
Do you ever use the form $ft_n^m$ or do you always drop the subscript and write $f^m$? If so it would be helpful to say as much.
\end{comment} 

\begin{comment}{Page 23, final para.}
Regarding the diagram you use here: Ha ha! Very good.
\end{comment} 

\begin{comment}{Page 23, somewhere???}
Once again you need say that you are in the habit of dropping the subscript and writing $\delta$ in place of $\delta_n$. Also $\partial$ ??
\end{comment} 

\begin{comment}{Page 26, example 4.15, parenthetical remark}
Omit the parenthetical remark because it was never Garner's intention for GATs to be known as {w,p,s}-GATs rather it makes sense for him to refer to them in this way in some places in his paper to help to make the point that he is making.  
\end{comment} 

\end{document}